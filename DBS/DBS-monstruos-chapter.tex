%\begin{document}
%	\author{Andres Zanzani}
%	\title{\centering Dungeon Bell System (DBS)\\\vspace{0.5cm}Monster Manual\\\vspace{0.5cm}\includegraphics[scale=0.365] {copertina-monster}}
%	\date{\today\\v1.0.4\\}
%\maketitle
%\newpage~
%\setcounter{page}{1}
%\begin{multicols}{2}  
%{\small \tableofcontents{}}
%\end{multicols}
%\pagebreak
%\thispagestyle{empty}	
%\newpage~\newpage~
%}	
\section{Mostruario di TUS}

\textbf{Arrivano i Mostri...}

\begin{tcolorbox}[enhanced,arc=5pt,boxrule=0.3pt]{Chi lotta con i mostri deve guardarsi di non diventare, così facendo, un mostro. E se tu scruterai a lungo in un abisso, anche l'abisso scruterà dentro di te. (Friedrich Nietzsche)\\
		
I mostri possono essere sconfitti soltanto dai loro simili. (Claymore)\\

La tragedia dei mostri è di essere troppo grandi e potenti per essere accettati dal genere umano. (Ishiro Honda)
}\end{tcolorbox}\medskip

\begin{multicols}{2}

\lettrine{B}{envenuti} in un universo ricco di nemici cattivi violenti subdoli intelligenti meschini giganteschi.. e quant'altro tu vorrai.

I mostri sono il caposaldo di qualsiasi gioco di ruolo fantasy.

Vengono qui spiegati e presentati dei mostri, non certo tutti ne tanto meno esaustivi, usateli per popolare di incubi le avventure dei vostri compagni.


\begin{center}
	\includegraphics[width=0.9\linewidth]{immagini/sangiorgioedrago.png}\\
	\textit{San Giorgio e il drago è un dipinto a olio su tela (57×73 cm) di Paolo Uccello, conservata alla National Gallery di Londra e databile al 1460 circa}
\end{center}


\subsection{Introduzione}

Un avventura non e' solo un insieme di mostri ma di situazioni, di luoghi, di sorprese, insomma di tutto cio' che puo' affascinare, coinvolgere stupire, impegnare i giocatori.\\
Ma anche i mostri servono. Picchiare a un aspetto catartico, liberatorio.

Inserite nell'avventura mostri difficili e pericolosi dove serve ma ogni tanto, raramente fate sentire i giocatori potenti, fagli affrontare mostri che in pochissimi round possono risolvere. Descrivete il combattimento enfatizzando i colpi, i critici, il dolore ed il sangue dei mostri.\\
Fate capire quanto possano essere potenti i personaggi.

Altre volte fate che i mostri incutono timore non perche' sono grossi, affamati, magici o quant'altro, e' necessario che i giocatori abbiano paura per i loro personaggi, che non diano mai per scontato la vittoria.\\
La sicurezza nel muoversi, poche battute, il fissare negli occhi i personaggi..coinvolgete i giocatori innanzitutto, una volta che i giocatori avranno la vostra attenzione anche i personaggi saranno piu' attenti.\\

Cercate di mettere mostri coerenti all'ambiente, all'avventura, alla situazione. Non tirate a caso su tabelle, uno scontro ben organizzato da molta piu' soddisfazione che mostri a caso che "spawnano".\\

Non riducete tutto a un MMORG dove l'obiettivo e' solo uccidere tutto e tutti, ci possono essere sempre tante scelte se ti impegni un po'.

\subsection{Modificare le Creature}

Nonostante la versatile collezione di mostri in questo documento, potresti comunque trovarti in imbarazzo quando si tratta di trovare la creatura perfetta per una tua avventura. Sentiti libro di modificare le creature esistenti e trasformarle in qualcosa che ti sia più utile, magari prendendo in prestito uno o due tratti da un mostro diverso o usando una \textbf{variante} o \textbf{archetipo}, come quelli presentati in questo documento. Tieni a mente che modificare un mostro, anche applicando un archetipo, potrebbe cambiarne il grado di sfida. 

\subsection{Taglia}

Un mostro può essere di taglia Minuscola, Piccola, Media, Grande, Enorme o Mastodontica. La tabella Categorie di Taglia mostra quanto spazio una creatura di una specifica taglia occupi in combattimento.

\medskip

\textbf{Categorie di Taglia}

\begin{tabular}{lll}
\toprule
\textbf{Taglia}& \textbf{Spazio} & \textbf{Esempio}\\
Minuscola & 25 x 25 cm& Gatto, spiritello\\
Piccola & 0,5 x 0,5 m & Goblin, cane\\
Media & 1 x 1 m & Orco\\
Grande & 3 x 3 m& Ogre\\
Enorme & 4 x 4 m & Gigante, Ent\\
Mastodontico & 6 x 6 m o più&Kraken, Verme purpureo\\
\end{tabular}

\subsection{Tipo}

Il tipo di un mostro si riferisce alla sua natura basilare. Certi incantesimi, oggetti magici, Abilità e altri effetti del gioco interagiscono in modi speciali con le creature di un tipo specifico. Ad esempio, una \emph{freccia} \emph{ammazza draghi} infligge danni extra non solo ai draghi ma anche a tutte le altre creature del tipo drago, come i draghi tartaruga e le viverne.

Il gioco comprende i seguenti tipi di mostri, che non hanno regole specifiche.

\medskip\textbf{Aberrazioni}, creature totalmente aliene. Molte di esse possiedono innate abilità magiche che attingono alla mente aliena della creatura anziché dalle forze mistiche del mondo. Esempi classici di aberrazioni sono aboleti, osservatori, scortica mente e i batraci del caos.

\medskip\textbf{Bestie}, creature non umanoidi che sono una componente naturale di un mondo fantasy. Alcune possiedono poteri magici, ma la maggior parte è priva di Intelligenza e non ha alcuna forma di società o linguaggio. Esempi classici di bestie sono tutte le specie di animali comuni, i dinosauri e le versioni giganti degli animali. 

\medskip\textbf{Celestiali}, creature native dei Piani Superiori. Molti di loro sono servitori delle divinità, impiegati come messaggeri o agenti nel mondo dei mortali e per i piani.\\
I celestiali sono di natura buona, esempi classici di celestiali sono angeli, couatl e pegasi.

\medskip\textbf{Costrutti}, sono creati e non partoriti. Alcuni sono programmati dai loro creatori per seguire una semplice serie di istruzioni, mentre altri sono senzienti e capaci di pensare per proprio conto. I golem sono i costrutti più rappresentativi.

\medskip\textbf{Draghi}, sono grandi creature rettili di antica origine ed enorme potere. I veri draghi, compresi i buoni draghi metallici e i malvagi draghi cromatici, sono molto intelligenti e possiedono doti magiche innate. In questa categoria si collocano anche creature lontanamente imparentate con i veri draghi, ma meno potenti, meno intelligenti e meno magiche, come le viverne e gli pseudodraghi.

\medskip\textbf{Elementali}, sono creature native dei piani elementali. Alcune creature di questo tipo sono poco più che masse animate del rispettivo elemento, e includono le creature chiamate semplicemente elementali. Altre creature possiedono forme biologiche infuse di energia elementale. Le razze dei geni, compresi djinn ed efreet, formano le civiltà più importanti dei piani elementali. Altre creature elementali sono gli azer, i persecutori  invisibili e le bizzarrie d'acqua. 

\medskip\textbf{Fatati}, sono creature magiche strettamente legate alle forze della natura. Vivono in radure crepuscolari e foreste nebbiose. Esempi di fatati sono driadi, pixie e satiri.

\medskip\textbf{Giganti}, troneggiano sugli umani e i loro simili. Sono di forma umana, sebbene alcuni abbiano più teste (ettin) o deformità (fomori). Le sei varianti dei veri giganti sono gigante di collina, gigante di pietra, gigante del gelo, gigante del fuoco, gigante delle nuvole, gigante delle tempeste. Oltre questi, anche ogri e troll sono giganti. 

\medskip\textbf{Immondi}, creature perverse native dei Piani Inferi. Alcune sono al servizio di  divinità, ma molte di più operano agli ordini di arcidiavoli e principi demoni. A volte sacerdoti e maghi malvagi evocano gli immondi nel mondo materiale perché eseguano le loro volontà. Se un celestiale malvagio è una rarità, un immondo buono è praticamente inconcepibile. Gli immondi includono demoni, diavoli, segugi infernali e rakshasa. 

\begin{center}
	\includegraphics[width=0.9\linewidth]{immagini/sanmichelesatana.png}\\
	\textit{San Michele sconfigge Satana è un dipinto a olio su tavola trasportato su tela (268x160 cm) di Raffaello e aiuti, datato 1518 e conservato nel Museo del Louvre di Parigi.}
\end{center}

\medskip\textbf{Melme}, sono creature gelatinose che difficilmente hanno una forma fissa. Vivono principalmente sottoterra, stabilendosi in grotte e sotterranei, nutrendosi di rifiuti, carcasse o creature tanto sfortunate da incapparvi. I protoplasmi neri e i cubi gelatinosi sono tra le melme più riconoscibili.

\medskip\textbf{Mostruosità}, sono mostri nel senso più stretto del termine creature spaventose che non sono comuni, né davvero naturali, e quasi mai benigne. Alcune sono il risultato di esperimenti magici andati male (come l'orsogufo), mentre altri sono il prodotto di terribili maledizioni (tra cui annoveriamo il minotauro). Sfuggono a qualsiasi categorizzazione, e in qualche modo servono da categoria onnicomprensiva per quelle creature che non corrispondono a nessun altro tipo di mostro. 

\medskip\textbf{Non Morti}, sono creature un tempo vive condotte ad un orribile stato di non morte tramite la pratica della magia negromantica o qualche blasfema maledizione. Tra i non morti si annoverano cadaveri ambulanti, come vampiri e zombi, e spiriti incorporei, come fantasmi e spettri.

\medskip\textbf{Piante}, in questo contesto si tratta di creature vegetali, non della normale flora. La maggior parte di esse sono mobili, e alcune sono carnivore. L'esempio più classico di piante sono i tumuli ambulanti e gli ent. Anche le creature fungoidi come le spore gassose e i miconidi rientrano in questa categoria.

\medskip\textbf{Umanoidi}, sono la popolazione principale dei mondi di gioco, civilizzati e selvaggi, comprendono gli umani e un'ampia gamma di altre specie. Possiedono una lingua e una cultura, poche o nessuna abilità magica innata (sebbene molti umanoidi possano apprendere gli incantesimi), e una forma bipede. Le razze più comuni di umanoide sono quelle più adatte come personaggi del giocatore: umani, nani, elfi e halfling. Quasi altrettanto numerose, ma più brutali e selvagge, e quasi tutte malvagie, sono le razze goblinoidi (goblin, hobgoblin e bugbear), orchi, gnoll, lucertoloidi e coboldi.\\

\medskip

Queste categorie possono essere a loro volta raggruppate in tipologie di Creature:
\smallskip
\begin{itemize}
\item
Le \textbf{Creature Naturali}: sono Insetti, Rettili, Bestie, Umanoidi, Piante, Creature acquatiche, Monstrusita', Melme
\item
Le \textbf{Creature Magiche} sono: Immondi, Fatati, Spiriti, Non morti, Giganti, Celestiali, Costrutti, Aberrazioni (tutto ciò che e' alieno o innaturale) e Draghi.\\
Se una Creatura Naturale ha poteri magici allora si considera anche come Creatura Magica.
\end{itemize}


\medskip\textbf{Etichette}

Un mostro può presentare una o più etichette indicate tra parentesi, a seguire il suo tipo. Ad esempio un orco ha il tipo \emph{umanoide (orco)}. Le etichette tra parentesi forniscono ulteriori categorizzazioni per determinate creature. Le etichette non hanno delle proprie regole specifiche, ma alcuni elementi del gioco, come gli oggetti magici, vi possono fare riferimento. Ad esempio, una lancia particolarmente efficace contro i demoni, funzionerebbe contro qualsiasi mostro che abbia l'etichetta demone.

\subsection{Tratti}

I mostri non presentano l'elenco dettagliato dei tratti, troverete solo l'indicazione sugli assi del Chaos, Legge, Bene e Male,\\
Ricordatevi che sono indicazioni, le eccezioni possono capitare!\\

Determinate creature sono \textbf{disallineate}, ovvero non hanno una condotta morale o etica.

\subsection{Difesa}

Un mostro che indossa un'armatura o trasporta uno scudo ha una Difesa che tiene conto dell'armatura, lo scudo e della Destrezza. Altrimenti, la Difesa di un mostro è basata sul suo valore di Destrezza l'armatura naturale, se la possiede. Se un mostro possiede un'armatura naturale, indossa armature o trasporta uno scudo, viene indicato tra parentesi dopo il valore della sua Difesa.

\subsection{Punti Ferita}

Di solito quando scende a 0 punti ferita, un mostro muore o viene distrutto.

I punti ferita di un mostro sono presentati sia come un insieme di dadi che come valore medio. Ad esempio, un mostro con 2d8 punti ferita ha di media 9 punti ferita (2 x 4,5).

La taglia di un mostro determina il dado impiegato per calcolare i suoi punti ferita, come mostrato sulla tabella Dadi Vita per Taglia.

\subsection{Dadi Vita per Taglia}

\medskip
\begin{tabular}{lll}
\toprule
Taglia del Mostro & Dado Vita & PF per Dado\\
Minuscola &d4&2,5\\
Piccola &d6&3,5\\
Media&d8 &4,5\\
Grande&d10&5,5\\
Enorme&d12&6,5\\
Mastodontico&d20&10,5\\
\end{tabular}
\medskip

Anche il valore di Costituzione di un mostro influenza il numero di punti ferita che possiede. Il suo valore di Costituzione viene moltiplicato per il numero di Dadi Vita che possiede, e il risultato viene sommato ai suoi punti ferita. Ad esempio, un mostro ha Costituzione 1 e 2d8 Dadi Vita, e avrà quindi 2d8+2 punti ferita (media 11).

\subsection{Movimento}

Il Movimento di un mostro ti dice di quanto si possa muovere durante il suo round per azione di movimento

Tutte le creature possiedono un movimento di passeggio, detto semplicemente movimento del mostro. Le creature che non possiedono alcuna forma di spostamento terreno hanno velocità di passeggio 0 metri.

Alcune creature possiedono uno o più dei seguenti modi di movimento aggiuntivi.

\medskip\textbf{Nuoto}

Un mostro che possiede una velocità di nuoto non deve spendere movimento extra per nuotare (non e' terreno difficile)

\medskip\textbf{Scalata}

Un mostro che possiede una velocità di scalata può usare tutto o solo parte del suo movimento per muoversi su superfici verticali. Il mostro non deve spendere movimento extra per scalare.

\medskip\textbf{Scavo}

Un mostro che possiede una velocità di scavo può usare la sua velocità per attraversare sabbia, terra, fango, ecc. Un mostro non può scavare attraverso la roccia solida a meno che non possieda un tratto speciale che glielo permetta.

\medskip\textbf{Volo}

Un mostro che possiede una velocità di volo può usare tutto o solo parte del suo movimento per volare. Alcuni mostri hanno l'abilità di \textbf{fluttuare}, che li rende difficili da abbattere. Il mostro smette di fluttuare quando muore.

\begin{center}
	\includegraphics[width=0.7\linewidth]{immagini/roc.png}\\
	\textit{Henry Justice Ford}
\end{center}


\subsection{Punteggi di Caratteristica}

Ogni mostro possiede sei punteggi di caratteristica (Forza, Destrezza, Costituzione, Intelligenza, Saggezza, Carisma)

\subsection{Competenze}

La voce Competenze è riservata a quei mostri che sono capaci in una o più competenze. Ad esempio, un mostro che è molto attento e furtivo potrebbe avere bonus alle prove di Saggezza (Consapevolezza) e Destrezza. \\
Si possono applicare anche altri modificatori. Ad esempio, un mostro potrebbe avere un bonus più grande del previsto per tenere conto della sua grande perizia.

\subsection{Vulnerabilità, Resistenze e Immunità}

Alcune creature possiedono vulnerabilità, resistenze o immunità ad un certo tipo di danno. Creature particolari sono addirittura resistenti o immuni agli attacchi non magici (un attacco magico è un attacco sferrato tramite un incantesimo, un oggetto magico, o un'altra fonte di magia). \\
E' anche possibile che sia indicato uno specifico bonus magico minimo per poter danneggiare la creatura.\\
Inoltre, certe creature sono immuni a determinate condizioni. Se  un mostro è immune ad un effetto di gioco che non viene considerato danno o condizione, possiede invece un tratto speciale.


Nella tabella sottostante viene indicato quale incantamento magico dell'arma e' necessario per superare l'immunita' indicata.\\
E' anche indicato il livello minimo di attacco naturale nel caso si colpisca con calci e pugni.

\medskip

\textbf{Tabella: Equivalenza Armi Magiche}\index{Tabella Equivalenza Armi Magiche}

\medskip

\begin{tabularx}{0.45\textwidth}{lXX}
	\toprule
	\textbf{Immunità} & \textbf{Bonus sull'arma} & \textbf{Attacco Naturale}\\
	Incantamento +1         & +1              & Livello 3\\
	Incantamento +2         & +2              & Livello 6\\
	Ferro Freddo / Argento  & +2              & Livello 9\\
	Adamantio               & +3              & Livello 12\\
	Incantamento +3         & +3              & Livello 15\\
	Incantamento +4         & +4              & Livello 18\\
\end{tabularx}


\subsection{Sensi}

La voce Sensi elenca qualsiasi senso speciale di cui il mostro sia in possesso. I sensi speciali sono descritti di seguito. Se non e' presente la voce Sensi, la creatura ha dei sensi standard (visione...)

\subsubsection{Percezione Tellurica}

Un mostro con percezione tellurica può individuare e trovare le origini delle vibrazioni entro uno specifico raggio, purché il mostro e la fonte della vibrazione siano in contatto con lo stesso terreno o sostanza. La percezione tellurica non può essere impiegata per individuare creature volanti o incorporee. Molte creature scavatrici, come gli ankheg e i colossi di terra, possiedono questo senso speciale.

\subsubsection{Visione Crepuscolare}

Una creatura con Visione Crepuscolare può vedere nella piu' tenue delle luci, ma non nell'oscurità' completa. Molte creature che vivono sottoterra possiedono questo senso
speciale.  Vedi capitolo \hyperref[sec:sec:visione-e-luce]{Caratteristiche Speciali}

\subsubsection{Visione del Vero}

Un mostro con la visione del vero può, fino ad una specifica gittata, vedere attraverso l'oscurità normale e magica, vedere creature e oggetti invisibili, automaticamente individuare le illusioni e riuscire i Tiri Salvezza contro di loro, e percepire la forma originale di un mutaforma o di una creatura trasformata dalla magia. Inoltre, la creatura può vedere nel Piano Etereo fino alla stessa gittata.

\begin{center}
	\includegraphics[width=0.7\linewidth]{immagini/ciclope.png}\\
	\textit{Henry Justice Ford}
\end{center}


\subsubsection{Vista Cieca}

Una creatura con vista cieca può percepire l'ambiente circostante, senza fare affidamento alla vista, fino ad una specifica gittata. \\
Le creature senza occhi, come i grimlock e le melme, e le creature con ecolocazione o sensi potenziati, come i pipistrelli e i draghi puri, possiedono questo senso. \\
Se un mostro è cieco di natura, la cosa viene annotata tra parentesi, ad indicare che la gittata della sua vista cieca definisce anche la gittata massima della sua percezione.\\

\subsection{Linguaggi}

Le lingue che un mostro può parlare sono riportate in ordine alfabetico. A volte un mostro può capire una lingua ma non parlarla, e la cosa viene indicata a questa voce. Una ``-'' indica che la creatura non parla né comprende alcuna lingua.

\subsection{Telepatia}

La telepatia è un'abilità magica che permette ad un mostro di comunicare mentalmente con un'altra creatura nel raggio di azione specificato. La creatura contattata non e' necessario che parli la stessa lingua del mostro per comunicare in questo modo. Una creatura senza telepatia può ricevere e rispondere a messaggi telepatici ma non può iniziare o terminare una conversazione telepatica.\\
Un mostro telepatico non ha bisogno di vedere la creatura contattata e può terminare il contatto telepatico in qualsiasi momento. Il contatto è infranto non appena le due creature non si trovano più entro il raggio di azione o se il mostro telepatico contatta un'altra creatura a gittata. Un mostro telepatico può iniziare o terminare una conversazione  telepatica senza dover usare un'azione, ma mentre il mostro è inabile, non può dare inizio ad un contatto telepatico, e qualsiasi contatto in corso viene terminato.\\
Una creatura nell'area di un \emph{campo anti-magia} o in qualsiasi altro posto in cui la magia non funziona non può inviare o ricevere messaggi telepatici.

\subsection{Sfida}

Il \textbf{grado di sfida} (CR) di un mostro vi dice quanto sia grande la minaccia che pone. Una compagnia di quattro avventurieri equipaggiata in maniera appropriata e riposata dovrebbe essere in grado di sconfiggere un mostro dal grado di sfida pari al proprio livello medio senza subire perdite. Ad esempio, una compagnia di quattro personaggi di 3° livello dovrebbe ritenere un mostro di grado di sfida 3 una degna sfida, ma non letale.\\
I mostri che sono significativamente più deboli dei personaggi di 1° livello hanno un grado di sfida inferiore ad 1. I mostri con un grado di sfida 0 non presentano problemi eccetto in grandi numeri; quelli privi di reali attacchi non valgono punti esperienza.\\
Alcuni mostri presentano una sfida superiore a quelle che anche una compagnia di 20° livello sia in grado di gestire. Questi mostri hanno grado di sfida 21 o superiore e sono progettati proprio per mettere alla prova le capacità dei personaggi.\\

\subsection{Tratti Speciali}

I tratti speciali (che compaiono dopo il grado di sfida di un mostro ma prima di qualsiasi azione o reazione) sono peculiarità che avranno probabilmente un ruolo in un incontro di combattimento e che richiedono delle spiegazioni.

\subsection{Incantesimi}

Un mostro con il privilegio Incantesimi e' in grado di lanciare Essenze o Incantesimi.\\
Un mostro può lanciare un incantesimo dalla sua lista senza effettuare la prova di magia e senza la possibilita' di effettuare tiri critici o meno. La DC e' quella dell'incantesimo + Intelligenza o Saggezza a seconda della caratteristica primaria.

\subsection{Incantesimi Innati}

Un mostro con l'abilità innata di lanciare incantesimi ha il tratto speciale Incantesimi.
Se non indicato diversamente non e' necessario effettuare la prova di magia e l'incantesimo viene lanciato alla sua Difficoltà senza conteggiare alcun critico.\\
Un incantesimo innato può essere sottoposto a speciali regole o restrizioni. Ad esempio, un elfo oscuro mago può eseguire in maniera innata l'incantesimo \emph{levitazione}, mal'incantesimo ha la  restrizione ``solo personale'', ad indicare che ha effetto solo
sull'elfo oscuro mago. \\
Gli incantesimi innati di un mostro non possono essere scambiati con altri incantesimi. 

\subsection{Azioni}

Quando un mostro svolge le sue azioni, può scegliere tra le opzioni della sezione Azioni del suo blocco statistiche o impiegare una delle azioni disponibili a tutte le creature, come Scattare o Nascondersi.

\subsubsection{Attacchi da Mischia e a Distanza}

L'azione più comune che un mostro effettuerà in combattimento, sarà un attacco da mischia o a distanza. Possono essere attacchi con incantesimi o attacchi con armi, dove l'arma può essere un manufatto o un'arma naturale, come gli artigli o la coda chiodata.\\
\emph{\textbf{Creatura contro Bersaglio}.} Il bersaglio di un attacco da mischia o a distanza è di solito una creatura o un bersaglio, la differenza nel fatto che un ``bersaglio'' può essere una creatura o un oggetto.\\
\emph{\textbf{Colpisce.}} Qualsiasi danno inflitto o altro effetto che avviene come risultato di un attacco che colpisce il bersaglio viene descritto nell'annotazione ``\emph{Colpisce}''. Puoi scegliere se prendere il danno medio o tirare i dadi; per questo  motivo vengono presentati sia il danno medio che una formula di dadi. \\
\textbf{\emph{Manca}.} Se un attacco ha un effetto prodotto da un colpo a vuoto, quell'informazione viene fornita dall'annotazione ``\emph{Manca}''.\\
\emph{\textbf{Danni.}} Se un mostro impugna armi manufatte, infligge danni appropriati all'arma. I mostri più grossi di solito impugnano armi di dimensioni superiori che infliggono danni extra quando colpiscono. Raddoppiare i dadi dell'arma se la creatura è Grande, triplicarli se Enorme e quadruplicarli se Mastodontica.\\
Una creatura ha -1d6 ai tiri per colpire con un'arma costruita per una taglia superiore alla sua. \\
Il Narratore può decidere che le armi di due o più taglie più grandi di quella dell'attaccante sono del tutto impossibili da usare.

\subsubsection{Multiattacco}

Una creatura che può effettuare più attacchi durante il suo round ha l'abilità Multiattacco. Una creatura non può usare Multiattacco quando effettua un attacco di opportunità, il quale deve essere un singolo attacco da mischia.

\subsubsection{Regole dell'Afferrare per i Mostri}

Molti mostri possiedono un attacco speciale che gli permette di afferrare rapidamente la preda. Quando un mostro colpisce con un simile attacco, non deve effettuare un'ulteriore prova di caratteristica per determinare se l'afferrare riesce, a meno che l'attacco non dica altrimenti.\\
Una creatura afferrata dal mostro può usare un azione per tentare di sfuggirgli. Per farlo, deve riuscire una prova di Forza contro la DC di fuga nel blocco statistiche del mostro. Se non viene fornita una DC di fuga, assumere che la DC sia uguale a 10 + Forza del mostro.

\subsubsection{Munizioni}

Un mostro porta con sé munizioni sufficienti per effettuare i suoi attacchi a distanza. Puoi presumere che un mostro abbia 2d4 proiettili per un attacco con armi da lancio, e 2d10 proiettili per un'arma a proiettili come un arco o una balestra.

\begin{center}
	\includegraphics[width=0.7\linewidth]{immagini/cupido.png}\\
	\textit{Eros con il suo arco. Musei Capitolini}
\end{center}



\subsubsection{Reazioni}

Se un mostro può compiere qualcosa di speciale con le sue reazioni, è riportato qui. Se una creatura non ha reazioni speciali, questa sezione è assente.

\subsubsection{Uso Limitato}

Alcune abilità speciali hanno restrizioni sul numero di volte che
possono essere usate.

\textbf{\emph{X/Giorno}.} L'annotazione ``X/Giorno'' indica un'abilità speciale che può essere usata X volte prima che il sorga l'alba per recuperare gli usi consumati. Ad esempio, ``1/Giorno'' indica un'abilità speciale che può essere usata una volta prima che il mostro debba aspettare la nuova alba.

\emph{\textbf{Ricarica X-Y.}} L'annotazione ``Ricarica X-Y'' indica che il mostro può usare un'abilità speciale una volta e che l'abilità ha una probabilità casuale di ricaricarsi ogni round seguente di combattimento. All'inizio di ciascun round del mostro, tira un d6. Se il risultato è uno dei numeri dell'annotazione di ricarica, il mostro recupera l'uso dell'abilità speciale. L'abilità si ricarica anche all'alba di un nuovo giorno.

Ad esempio, "Ricarica 5-6" indica che un mostro può usare la sua abilità speciale una volta. Poi, all'inizio del round del mostro, recupera l'uso dell'abilità se tira 5 o 6 su di un d6.

\subsection{Equipaggiamento}

Il blocco statistiche si riferisce all'equipaggiamento, oltre le armi o le armature utilizzate dal mostro. Una creatura che normalmente indossa abiti, come un umanoide, si assume sia abbigliato in maniera appropriata.

Puoi equipaggiare i mostri con ulteriore equipaggiamento o ninnoli come preferisci, utilizzando il capitolo ``Equipaggiamento'' come fonte di ispirazione, e sei tu a decidere quanto dell'equipaggiamento del mostro è recuperabile dopo che la creatura è stata uccisa o se qualsiasi parte del suo equipaggiamento sia ancora utilizzabile. Ad esempio, un'armatura ammaccata fatta per un mostro difficilmente sarà utilizzabile da qualcun altro.  Se un mostro incantatore necessita di componenti  materiali per lanciare i suoi incantesimi, dai per  scontato che abbia le componenti materiali per lanciare  gli incantesimi nel suo blocco statistiche.  

\subsection{Azioni Aggiuntive}

Certe creature possono  possono eseguire azioni speciali al di fuori del proprio  round, e alcune possono estendere il proprio potere  all'ambiente, provocando l'avvenimento di effetti magici  straordinari nelle loro vicinanze.

Una creatura con azioni aggiuntive può effettuare un certo  numero di azioni speciali -- dette azioni aggiuntive -- al  di fuori del suo round. Solo un'azione aggiuntiva può  essere usata alla volta e solo al termine del round di  un'altra creatura. Una creatura con azioni aggiuntive recupera  all'inizio del suo round le azioni aggiuntive che ha  usato. Non è obbligata ad usare le sue azioni aggiuntive, e non può usare le azioni aggiuntive mentre è inabile o altrimenti incapace di effettuare  azioni. Se sorpresa, non può usarle fin dopo il suo  primo round di combattimento.

Se una creatura assume la forma di una creatura con azioni aggiuntive, magari tramite un incantesimo, non ne  ottiene però le azioni aggiuntive, le azioni da tana, o  gli effetti regionali.

\subsubsection{La Tana di una Creatura}

Una creatura con azioni aggiuntive può presentare una sezione che ne descrive la tana e gli effetti speciali che vi può  creare mentre si trova lì, o per propria volontà o  semplicemente grazie alla sua presenza. Questa  sezione si applica solo alle creature leggendarie che  trascorrono molto tempo nelle loro tane ed è altamente  probabile che vi vengano incontrate.

\subsubsection{Azioni da Tana}

Se una creatura con azioni aggiuntive ha un'azione da tana, può  usarla per imbrigliare la magia ambientale della sua  tana. Al conteggio di iniziativa 20, perdendo i pareggi,  la creatura può usare una delle sue opzioni di azioni da  tana. Non può farlo mentre è inabile o altrimenti  incapace di effettuare azioni. Se sorpresa, non può  farne uso fino a dopo il suo primo round di combattimento.

\subsubsection{Effetti Regionali}

La semplice presenza di una creatura con azioni aggiuntive può  avere effetti strani e meravigliosi sull'ambiente, come  indicato in questa sezione. Gli effetti regionali terminano all'istante o si dissipano col tempo una volta  morta la creatura con azioni aggiuntive.

\end{multicols}

\pagebreak
\subsection{I Mostri}

\begin{tcolorbox}[enhanced,arc=5pt,boxrule=0.3pt]{Io sono il mostro che gli uomini che respirano bramerebbero uccidere. Io sono Dracula. (Dracula di Bram Stoker)}\end{tcolorbox}\medskip

\bigskip

\begin{multicols}{2}

\medskip\index{Mostri - Aboleth}\textbf{Aboleth}

\emph{Grande aberrazione, legale malvagio}

\textbf{FORZA} +5

\textbf{DESTREZZA} -1

\textbf{COSTITUZIONE} +2

\textbf{INTELLIGENZA} +4

\textbf{SAGGEZZA} +2

\textbf{CARISMA} +4

\textbf{Iniziativa} +4 -- \textbf{Difesa} 22

\textbf{Punti Ferita} 135 (18d10 + 36)

\textbf{Movimento} 3 m, nuoto 12 m

\textbf{Tiri Salvezza} Tempra +8, Riflessi +5, Volontà +11

\textbf{Competenze} Consapevolezza +10, Storia +12

\textbf{Sensi} scurovisione 36 m

\textbf{Linguaggi} Linguaggio delle Profondità, telepatia 36 m

\textbf{Sfida} 10 (5.900 PE)

\emph{\textbf{Anfibio.}} L'aboleth può respirare aria e acqua.

\emph{\textbf{Nube di Muco.}} Mentre è sott'acqua, l'aboleth è avvolto da muco mutante. Una creatura che entri a contatto con l'aboleth, o che lo colpisca con un attacco da mischia mentre si trova entro 1,5 metri da esso, deve effettuare un tiro salvezza di Tempra CD 14. Se lo fallisce, la creatura resta ammalata per 1d4 ore. La creatura ammalata può respirare solo sott'acqua.

\emph{\textbf{Sonda Telepatica.}} Se una creatura comunica telepaticamente con l'aboleth, e l'aboleth può vederla, l'aboleth ne apprende i più grandi desideri.

\textbf{Azioni}

\emph{\textbf{Multiattacco.}} L'aboleth effettua tre attacchi con i tentacoli

\emph{\textbf{Tentacolo.} Attacco con arma da mischia}: +9 a colpire, portata 3 m, un bersaglio.

\emph{Colpisce:} 12 (2d6 + 5) danni da botta. Se il bersaglio è una creatura, deve riuscire un tiro salvezza di Tempra CD 14 o divenire ammalato. La malattia non produce alcun effetto per 1 minuto e può essere rimossa da qualsiasi magia che curi le malattie. Dopo 1 minuto, la pelle della creatura ammalata diventa trasparente e viscida, la creatura non può recuperare punti ferita a meno che non sia sott'acqua, e la malattia può essere rimossa solo da \emph{guarire} o un altro incantesimo cura malattie di Difficoltà 23 o più. Quando la creatura si trova al di fuori di un corpo d'acqua, subisce 6 (1d12) danni da acido ogni 10 minuti a meno che la sua pelle non venga bagnata prima che siano passati questi 10 minuti.

\emph{\textbf{Coda.} Attacco con arma da mischia}: +9 a colpire, portata 3 m, un bersaglio.

\emph{Colpisce:} 15 (3d6 + 5) danni da botta.

\emph{\textbf{Schiavizzare (3/Giorno).}} L'aboleth prende a bersaglio una creatura che può vedere entro 9 metri da esso. Il bersaglio deve riuscire un tiro salvezza di Volontà CD 14 o restare affascinato magicamente dall'aboleth finché l'aboleth muore o i due si trovano su piani di esistenza differenti. Il bersaglio affascinato è sotto il controllo dell'aboleth e non può effettuare reazioni. L'aboleth e il bersaglio possono comunicare telepaticamente tra di loro a qualsiasidistanza. 

Ogniqualvolta il bersaglio affascinato subisce danni, può ripetere il tiro salvezza. Se lo riesce, l'effetto termina. Non più di una volta ogni 24 ore, può ripetere il tiro salvezza quando si trova almeno a 1,5 chilometri di distanza dall'aboleth.

\textbf{Azioni Aggiuntive}

L'aboleth può effettuare 3 Azioni aggiuntive, scelte tra le opzioni seguenti. Può usare solo un'opzione leggendaria alla volta e solo al termine del turno di un'altra creatura. L'aboleth recupera le Azioni aggiuntive spese all'inizio del proprio turno.

\textbf{Individuare.} L'aboleth effettua una prova di Saggezza (Consapevolezza).

\textbf{Risucchio Psichico (Costa 2 Azioni).} Una creatura affascinata dall'aboleth subisce 10 (3d6) danni psichici, e l'aboleth recupera un numero di punti ferita pari al danno subito dalla creatura.

\textbf{Spazzata di Coda.} L'aboleth effettua un attacco di coda.

\subsection{Angeli}

\medskip\index{Mostri - Angelo Deva}\textbf{Angelo Deva}

\emph{Medio celestiale, legale buono}

\textbf{FORZA} +4

\textbf{DESTREZZA} +4

\textbf{COSTITUZIONE} +4

\textbf{INTELLIGENZA} +3

\textbf{SAGGEZZA} +5

\textbf{CARISMA} +5

\textbf{Iniziativa} +4 -- \textbf{Difesa} 22

\textbf{Punti Ferita} 136 (16d8 + 64)

\textbf{Movimento} 9 m, volo 27 m

\textbf{Tiri Salvezza} Tempra +16, Riflessi +13, Volontà +11

\textbf{Competenze} Percepire Emozioni +9, Consapevolezza +9

\textbf{Resistenze ai Danni} da Luce; da botta, perforante e tagliente di attacchi non magici

\textbf{Immunità alle Condizioni} affascinato, sfinimento, spaventato

\textbf{Sensi} scurovisione 36 m

\textbf{Linguaggi} tutte, telepatia 36 m

\textbf{Sfida} 10 (5.900 PE)

\emph{\textbf{Armi Angeliche.}} Gli attacchi con arma del deva sono magici. Quando il deva colpisce con qualsiasi arma, l'arma infligge 4d8 danni da Luce aggiuntivi (già compresi nell'attacco).

\emph{\textbf{Incantesimi Innati.}} La caratteristica da incantatore innato del deva è il Carisma (CD 17 per i tiri salvezza degli incantesimi). Il deva può lanciare in maniera innata i seguenti incantesimi, con l'uso delle sole componenti verbali: 

A volontà: \emph{individuazione del bene e del male}

1/giorno: \emph{comunione, rianimare morti}

\emph{\textbf{Resistenza alla Magia.}} Il deva ha +1d6 ai tiri salvezza contro incantesimi e altri effetti magici.

\textbf{Azioni}

\emph{\textbf{Multiattacco.}} Il deva effettua due attacchi da mischia.

\emph{\textbf{Mazza.} Attacco con arma da mischia}: +8 a colpire, portata 1 m, un bersaglio.

\emph{Colpisce:} 7 (1d6 + 4) danni da botta più 18 (4d8) danni da Luce.

\emph{\textbf{Tocco Guaritore (3/Giorno).}} Il deva entra a contatto con un'altra creatura. Il bersaglio recupera magicamente 20 (4d8 + 2) punti ferita ed è libero da qualsiasi cecità, malattia, maledizione, sordità o veleno.

\emph{\textbf{Mutare Forma.}} Il deva può trasformarsi magicamente in un umanoide o bestia il cui grado di sfida sia pari o inferiore al proprio, o tornare alla sua vera forma. Alla morte ritorna alla sua vera forma. Qualsiasi equipaggiamento stia indossando o trasportando viene assorbito o trasportato nella nuova forma (a scelta del deva).

Nella nuova forma, il deva mantiene le sue statistiche di gioco e la facoltà di parlare, ma la sua Difesa, metodi di movimento, Forza, Destrezza e sensi speciali vengono rimpiazzati da quelli della nuova forma, e ottiene qualsiasi statistica o capacità (Azioni aggiuntive e azioni da tana) possedute dalla sua nuova forma e non dalla sua originale.

\medskip\index{Mostri - Angelo Planetar}\textbf{Angelo Planetar}

\emph{Grande celestiale, legale buono}

\textbf{FORZA} +7

\textbf{DESTREZZA} +5

\textbf{COSTITUZIONE} +7

\textbf{INTELLIGENZA} +4

\textbf{SAGGEZZA} +6

\textbf{CARISMA} +7

\textbf{Iniziativa} +5 -- \textbf{Difesa} 27

\textbf{Punti Ferita} 200 (16d10 + 112)

\textbf{Movimento} 12 m, volo 36 m

\textbf{Tiri Salvezza} Tempra +19, Riflessi +11, Volontà +19

\textbf{Competenze} Consapevolezza +11

\textbf{Resistenze ai Danni} da Luce;

\textbf{Immunità alle Condizioni} affascinato, sfinimento, spaventato, armi +1

\textbf{Sensi} visione del vero 36 m

\textbf{Linguaggi} tutte, telepatia 36 m

\textbf{Sfida} 16 (15.000 PE)

\emph{\textbf{Armi Angeliche.}} Gli attacchi con arma del planetar sono magici. Quando colpisce con qualsiasi arma, l'arma infligge 5d8 danni da Luce aggiuntivi (già compresi nell'attacco).

\emph{\textbf{Consapevolezza Divina.}} Il planetar riconosce immediatamente le bugie.

\emph{\textbf{Incantesimi Innati.}} La caratteristica da incantatore innato del planetario è il Carisma (CD 20 per i tiri salvezza degli incantesimi). Il planetario può lanciare in maniera innata i seguenti incantesimi, senza bisogno di componenti materiali:

A volontà: \emph{individuazione del bene e del male}, \emph{invisibilità} (solo personale)

3/giorno: \emph{barriera di lame, colpo infuocato, dissolvi il bene e il} \emph{male, rianimare morti}

1/giorno: \emph{comunione, controllare tempo atmosferico, piaga degli insetti}

\emph{\textbf{Resistenza alla Magia.}} Il planetar ha +1d6 ai tiri salvezza contro incantesimi e altri effetti magici.

\textbf{Azioni}

\emph{\textbf{Multiattacco.}} Il planetar effettua due attacchi da mischia.

\emph{\textbf{Spadone.} Attacco con arma da mischia}: +12 a colpire, portata 1 m, un bersaglio.

\emph{Colpisce:} 21 (4d6 + 7) danni taglienti più 22 (5d8) danni da Luce.

\emph{\textbf{Tocco Guaritore (4/Giorno).}} Il planetar entra a contatto con un'altra creatura. Il bersaglio recupera magicamente 30 (6d8 + 3) punti ferita ed è libero da qualsiasi cecità, malattia, maledizione, sordità o veleno.

\medskip\index{Mostri - Angelo Solar}\textbf{Angelo Solar}

\emph{Grande celestiale, legale buono}

\textbf{FORZA} +8

\textbf{DESTREZZA} +6

\textbf{COSTITUZIONE} +8

\textbf{INTELLIGENZA} +7

\textbf{SAGGEZZA} +7

\textbf{CARISMA} +10

\textbf{Iniziativa} +7 -- \textbf{Difesa} 31

\textbf{Punti Ferita} 243 (18d10 + 144) 

\textbf{Movimento} 15 m, volo 45 m

\textbf{Tiri Salvezza} Tempra +25, Riflessi +14, Volontà +23

\textbf{Competenze} Consapevolezza +14

\textbf{Resistenze ai Danni} da Luce;

\textbf{Immunità ai Danni} da Vuoto, veleno, armi +2

\textbf{Immunità alle Condizioni} affascinato, avvelenato, sfinimento, spaventato

\textbf{Sensi} visione del vero 36 m

\textbf{Linguaggi} tutte, telepatia 36 m 

\textbf{Sfida} 21 (33.000 PE)

\emph{\textbf{Armi Angeliche.}} Gli attacchi con arma del solar sono magici. Quando colpisce con qualsiasi arma, l'arma infligge 6d8 danni da Luce aggiuntivi (già compresi nell'attacco).

\emph{\textbf{Consapevolezza Divina.}} Il solar riconosce immediatamente le bugie.

\emph{\textbf{Incantesimi Innati.}} La caratteristica da incantatore innato del solar è il Carisma (CD 25 per i tiri salvezza degli incantesimi). Il solar può lanciare in maniera innata i seguenti incantesimi, senza bisogno di componenti materiali:

A volontà: \emph{individuazione del bene e del male}, \emph{invisibilità} (solo personale)

3/giorno: \emph{barriera di lame, colpo infuocato, dissolvi il bene e il male, resurrezione}

1/giorno: \emph{comunione, controllare tempo atmosferico}

\emph{\textbf{Resistenza alla Magia.}} Il solar ha +1d6 ai tiri salvezza contro incantesimi e altri effetti magici.

\textbf{Azioni}

\emph{\textbf{Multiattacco.}} Il solar effettua due attacchi con lo spadone.

\emph{\textbf{Spadone.} Attacco con arma da mischia}: +15 a colpire, portata 1 m, un bersaglio.

\emph{Colpisce:} 22 (4d6 + 8) danni taglienti più 27 (6d8) danni da Luce.

\emph{\textbf{Arco Lungo dell'Uccisione.} Attacco con arma a distanza}: +13 a colpire, gittata 45m, un bersaglio.

\emph{Colpisce:} 15 (2d8 + 6) danni perforanti più 27 (6d8) danni da Luce. Se il bersaglio è una creatura con 100 punti ferita o meno, deve riuscire un tiro salvezza di Tempra CD 15 o morire.

\emph{\textbf{Spada Volante.}} Il solare libera il suo spadone perché fluttui magicamente in uno spazio non occupato entro 1,5 metri da lui.  Se il solare può vedere la spada, con un'azione bonus le può ordinare mentalmente di volare per un massimo di 15 metri ed effettuare un attacco contro un bersaglio o ritornare nella mano del solare. Se la spada fluttuante è bersaglio di un effetto, si considera come se fosse impugnata dal solare. Se il solare muore, la spada fluttuante cade a terra.

\emph{\textbf{Tocco Guaritore (4/Giorno).}} Il solare entra a contatto con un'altra creatura. Il bersaglio recupera magicamente 40 (8d8 + 4) punti ferita ed è libero da qualsiasi cecità, malattia, maledizione, sordità o veleno.

Il solare può effettuare 3 azioni aggiuntive, scelte tra le opzioni seguenti. Può usare solo un'Azione Aggiuntiva alla volta e solo al termine del round di un'altra creatura. Il solare recupera le azioni aggiuntive spese all'inizio del proprio round. 

\textbf{Esplosione Incandescente (Costa 2 Azioni).} Il solare emette energia magica divina. Ogni creatura di sua scelta, in un raggio di 3 metri, deve effettuare un tiro salvezza su Riflessi DC  30, subendo 14 (4d6) danni da fuoco più 14 (4d6) danni da Luce se fallisce il tiro salvezza, o la metà se lo riesce. 

\textbf{Sguardo Accecante (Costa 3 Azioni).} Il solare prende a bersaglio una creatura entro 9 metri e che possa vedere. Se il bersaglio può vedere il solare, il bersaglio deve riuscire un tiro salvezza su Tempra DC  18 o restare accecato finché un incantesimo come \emph{ristorare inferiore} non rimuoverà la cecità.

\textbf{Teletrasporto.} Il solare si teletrasporta magicamente fino a 36 metri di distanza, insieme a tutto l'equipaggiamento che sta indossando o trasportando, in uno spazio non occupato e che può vedere.

\medskip\index{Mostri - Ankheg}\textbf{Ankheg}

\emph{Grande mostruosità, disallineato}

\textbf{FORZA} +3

\textbf{DESTREZZA} +0

\textbf{COSTITUZIONE} +1

\textbf{INTELLIGENZA} -5

\textbf{SAGGEZZA} +1

\textbf{CARISMA} -2

\textbf{Iniziativa} +0 -- \textbf{Difesa} 15 , 12 mentre è prono

\textbf{Punti Ferita} 39 (6d10 + 6)

\textbf{Movimento} 9 m, scavo 3 m

\textbf{Sensi} scurovisione 18 m, percezione tellurica 18 m, 

\textbf{Linguaggi} -

\textbf{Sfida} 2 (450 PE)

\textbf{Azioni}

\emph{\textbf{Morso.} Attacco con arma da mischia}: +5 a colpire, portata 1 m, un bersaglio.

\emph{Colpisce:} 10 (2d6 + 3) danni taglienti più 3 (1d6) danni da acido. Se il bersaglio è una creatura di taglia Grande o inferiore, è afferrata (CD 13 per fuggire). Fino al termine dell'afferrare, l'ankheg può mordere solo la creatura afferrata e ha +1d6 ai tiri di attacco contro di essa.

\emph{\textbf{Spruzzo Acido (Ricarica 6).}} L'ankheg sputa acido in una linea lunga 9 metri e larga 1,5 metri, purché non stia afferrando nessuna creatura. Ogni creatura su quella linea deve effettuare un tiro salvezza di Riflessi CD 13, e subire 10 (3d6) danni da acido se fallisce il tiro salvezza, o la metà di questi danni se lo riesce.

\medskip\index{Mostri - Arpia}\textbf{Arpia}

\emph{Media mostruosità, caotico malvagio}

\textbf{FORZA} +1

\textbf{DESTREZZA} +1

\textbf{COSTITUZIONE} +1

\textbf{INTELLIGENZA} -2

\textbf{SAGGEZZA} +0

\textbf{CARISMA} +1

\textbf{Iniziativa} +1 -- \textbf{Difesa} 12

\textbf{Punti Ferita} 38 (7d8 + 7)

\textbf{Movimento} 6 m, volo 12 m

\textbf{Linguaggi} Comune

\textbf{Sfida} 1 (200 PE)

\textbf{Azioni}

\emph{\textbf{Multiattacco.}} L'armatura effettua due attacchi: uno con gli artigli e uno con il randello.

\emph{\textbf{Artigli.} Attacco con arma da mischia}: +3 a colpire, portata 1 m, un bersaglio.

\emph{Colpisce:} 6 (2d4 + 1) danni taglienti.

\emph{\textbf{Randello.} Attacco con arma da mischia}: +3 a colpire, portata 1 m, un bersaglio.

\emph{Colpisce:} 3 (1d4 + 1) danni da botta.

\emph{\textbf{Canto Ammaliatore.}} L'arpia canta una melodia magica. Ogni umanoide e gigante entro 90 metri dall'arpia e che possa udire la canzone deve riuscire un tiro salvezza di Volontà CD 11 o restare affascinato fino al termine della canzone. L'arpia deve effettuare un'azione bonus durante il suo prossimo turno per continuare a cantare. Può smettere di cantare in qualsiasi momento. Il canto ha termine se l'arpia è inabile.

Mentre è affascinato dall'arpia, un bersaglio è inabile e ignora le canzoni di altre arpie. Se il bersaglio affascinato si trova a più di 1,5 metri dall'arpia, il bersaglio deve muoversi durante il proprio turno per dirigersi verso l'arpia usando la via più diretta. Egli non eviterà attacchi di opportunità, ma prima di muoversi in un terreno pericoloso, come lava o un pozzo, e prima di subire danno da qualsiasi fonte che non sia l'arpia, il bersaglio potrà ripetere il tiro salvezza. Una creatura può ripetere il tiro salvezza al termine di ciascun proprio turno. Se il tiro salvezza ha successo, l'effetto ha termine per quel bersaglio.

Un bersaglio che riesce il tiro salvezza è immune al canto di quell'arpia per le successive 24 ore.

\medskip\index{Mostri - Azer}\textbf{Azer}

\emph{Media elementale, legale neutrale}

\textbf{FORZA} +3

\textbf{DESTREZZA} +1

\textbf{COSTITUZIONE} +2

\textbf{INTELLIGENZA} +1

\textbf{SAGGEZZA} +1

\textbf{CARISMA} +0

\textbf{Iniziativa} +1 -- \textbf{Difesa} 18 (armatura naturale, scudo)

\textbf{Punti Ferita} 39 (6d8 + 12)

\textbf{Movimento} 9 m

\textbf{Tiri Salvezza} Tempra +2, Riflessi +1, Volontà +1

\textbf{Immunità ai Danni} fuoco, veleno

\textbf{Immunità alle Condizioni} avvelenato

\textbf{Linguaggi} Ignan

\textbf{Sfida} 2 (450 PE)

\emph{\textbf{Armi Riscaldate.}} Quando l'azer colpisce con un'arma da mischia in metallo, infligge 3 (1d6) danni da fuoco aggiuntivi (già inclusi nell'attacco).

\emph{\textbf{Corpo Riscaldato.}} Una creatura che entri a contatto con l'azer o lo colpisca con un attacco da mischia mentre si trova entro 1,5 metri da lui subisce 5 (1d10) danni da fuoco.

\emph{\textbf{Fuoco Vivente.}} Un azer non ha bisogno di cibo, bevande o di dormire.

\emph{\textbf{Illuminazione.}} L'azer irradia luce intensa in un raggio di 3 metri e luce fioca per ulteriori 3 metri.

\textbf{Azioni}

\emph{\textbf{Martello da Guerra.} Attacco con arma da mischia}: +5 a colpire, portata 1 m, un bersaglio.

\emph{Colpisce:} 7 (1d8 + 3) danni da botta, o 8 (1d10 + 3) danni da botta se usato a due mani per effettuare un attacco da mischia, più 3 (1d6) danni da fuoco.

\medskip\index{Mostri - Basilisco}\textbf{Basilisco}

\emph{Media mostruosità, disallineato}

\textbf{FORZA} +3

\textbf{DESTREZZA} -1

\textbf{COSTITUZIONE} +2

\textbf{INTELLIGENZA} -4

\textbf{SAGGEZZA} -1

\textbf{CARISMA} -2

\textbf{Iniziativa} -1 -- \textbf{Difesa} 17

\textbf{Punti Ferita} 52 (8d8 + 16)

\textbf{Movimento} 6 m

\textbf{Sensi} scurovisione 18 m

\textbf{Linguaggi} -

\textbf{Sfida} 3 (700 PE)

\emph{\textbf{Sguardo Pietrificante.}} Se una creatura comincia il suo turno entro 9 metri dal basilisco e i due si possono vedere vicendevolmente, se non  inabile il basilisco può obbligare la creatura ad effettuare un tiro salvezza di Tempra CD 12. Se la creatura fallisce il tiro salvezza, inizia magicamente a trasformarsi in pietra ed è intralciata. La creatura deve ripetere il tiro salvezza al termine del suo prossimo turno. Se lo riesce, l'effetto termina. Se lo fallisce, la creatura è pietrificata finché non viene liberata dall'incantesimo \emph{ristorare} \emph{superiore} o altra magia.

Una creatura che non sia sorpresa, può distogliere lo sguardo per evitare il tiro salvezza all'inizio del suo turno. In quel caso, non potrà vedere il basilisco fino all'inizio del suo prossimo turno, quando potrà distogliere nuovamente lo sguardo. Se nel frattempo dovesse guardare il basilisco, dovrebbe immediatamente effettuare il tiro salvezza.

Se il basilisco si trova entro 9 metri dal suo riflesso a luce intensa e lo vede, lo scambia per un rivale e diventa il bersaglio del proprio sguardo.

\textbf{Azioni}

\emph{\textbf{Morso.} Attacco con arma da mischia}: +5 a colpire, portata 1 m, un bersaglio.

\emph{Colpisce:} 10 (2d6 + 3) danni perforanti più 7 (2d6) danni da veleno.


\medskip\index{Mostri - Behir}\textbf{Behir}

\emph{Enorme mostruosità, neutrale malvagio}

\textbf{FORZA} +6

\textbf{DESTREZZA} +3

\textbf{COSTITUZIONE} +4

\textbf{INTELLIGENZA} -2

\textbf{SAGGEZZA} +2

\textbf{CARISMA} +1

\textbf{Iniziativa} +3 -- \textbf{Difesa} 23

\textbf{Punti Ferita} 168 (16d12 + 64)

\textbf{Movimento} 15 m, scalata 12 m

\textbf{Competenze} Muoversi Silenziosamente / Nascondersi nelle Ombre +7, Consapevolezza +6

\textbf{Immunità al Danno} fulmine

\textbf{Sensi} scurovisione 27 m

\textbf{Linguaggi} Draconico

\textbf{Sfida} 11 (7.200 PE)

\textbf{Azioni}

\emph{\textbf{Multiattacco.}} Il behir effettua due attacchi: uno con il morso e uno per stritolare.

\emph{\textbf{Morso.} Attacco con arma da mischia}: +10 a colpire, portata 3 m, un bersaglio.

\emph{Colpisce:} 22 (3d10 + 6) danni perforanti.

\emph{\textbf{Stritolare.} Attacco con arma da mischia}: +10 a colpire, portata 1 m, una creatura di taglia Grande o inferiore.

\emph{Colpisce:} 17 (2d10 + 6) danni da botta più 17 (2d10 + 6) danni taglienti. Il bersaglio è afferrato (CD 16 per fuggire) Se il behir non sta già stritolando un'altra creatura, il bersaglio è afferrato e intralciato fino al termine dell'afferrare.

\emph{\textbf{Inghiottire.}} Il behir effettua una attacco di morso contro un bersaglio di taglia Media o inferiore che sta afferrando. Se l'attacco colpisce, il bersaglio è inghiottito, e l'afferrare ha termine. Il bersaglio inghiottito è accecato e intralciato, ha copertura totale contro gli attacchi e altri effetti all'esterno del behir, e subisce 21 (6d6) danni da acido all'inizio di ciascun turno del behir. Il behir può inghiottire solo una creatura alla volta.

Se il behir subisce 30 o più danni in un singolo turno da una creatura che ha inghiottito, deve riuscire un tiro salvezza di Tempra CD 14 al termine di quel turno o vomitare la creatura, che ricade prona in uno spazio entro 3 metri dal behir. Se il behir muore, una creatura inghiottita non è più intralciata da esso e può uscire dal cadavere utilizzando 4,5 metri di movimento, uscendo prona.

\emph{\textbf{Soffio di Fulmine (Ricarica 5-6).}} Il behir esala fulmini in una linea lunga 6 metri e larga 1,5 metri. Ogni creatura su quella linea deve effettuare un tiro salvezza di Riflessi CD 16 e subire 66 (12d10) danni da fulmine se fallisce il tiro salvezza, o la metà di questi danni se lo riesce.

\medskip\index{Mostri - Bugbear}\textbf{Bugbear}

\emph{Media umanoide (goblinoide), caotico malvagio}

\textbf{FORZA} +2

\textbf{DESTREZZA} +2

\textbf{COSTITUZIONE} +1

\textbf{INTELLIGENZA} -1

\textbf{SAGGEZZA} +0

\textbf{CARISMA} -1

\textbf{Iniziativa} +2 -- \textbf{Difesa} 17

\textbf{Punti Ferita} 27 (5d8 + 5)

\textbf{Movimento} 9 m

\textbf{Competenze} Muoversi Silenziosamente / Nascondersi nelle Ombre +6, Sopravvivenza +2

\textbf{Sensi} scurovisione 18 m

\textbf{Linguaggi} Comune, Goblin

\textbf{Sfida} 1 (200 PE)

\emph{\textbf{Attacco di Sorpresa.}} Se il bugbear sorprende una creatura e la colpisce con un attacco durante il primo round di combattimento, il bersaglio subisce 7 (2d6) danni aggiuntivi
dall'attacco.

\emph{\textbf{Bruto.}} Un'arma da mischia infligge un dado aggiuntivo di danno quando il bugbear colpisce con essa (già incluso nell'attacco).

\textbf{Azioni}

\emph{\textbf{Mazza Chiodata.} Attacco con arma da mischia}: +4 a colpire, portata 1 m, un bersaglio.

\emph{Colpisce:} 11 (2d8 + 2) danni perforanti.

\emph{\textbf{Giavellotto.} Attacco con arma da mischia o a Distanza}: +4 a colpire, portata 1 m o gittata 9m, un bersaglio.

\emph{Colpisce:} 9 (2d6 + 2) danni perforanti in mischia o 5 (1d6 + 2) danni perforanti a gittata.

\medskip\index{Mostri - Bulette}\textbf{Bulette}

\emph{Grande mostruosità, disallineato}

\textbf{FORZA} +4

\textbf{DESTREZZA} +0

\textbf{COSTITUZIONE} +5

\textbf{INTELLIGENZA} -4

\textbf{SAGGEZZA} +0

\textbf{CARISMA} -3

\textbf{Iniziativa} +0 -- \textbf{Difesa} 20

\textbf{Punti Ferita} 94 (9d10 + 45)

\textbf{Movimento} 12 m, scavo 12 m

\textbf{Competenze} Consapevolezza +6

\textbf{Sensi} scurovisione 18 m, percezione tellurica 18 m

\textbf{Linguaggi} -

\textbf{Sfida} 5 (1.800 PE)

\emph{\textbf{Salto da Fermo.}} Un bulette può saltare in lungo fino a 9  metri e in alto fino a 4,5 metri, con o senza la rincorsa.

\textbf{Azioni}

\emph{\textbf{Morso.} Attacco con arma da mischia}: +7 a colpire, portata 1 m, un bersaglio.

\emph{Colpisce:} 30 (4d12 + 4) danni perforanti.

\emph{\textbf{Salto Letale.}} Se il bulette può saltare di almeno 4,5  metri come parte del suo movimento, può usare poi questa azione per  atterrare in piedi in uno spazio che contiene una o più creature.  Ciascuna di queste creature deve riuscire un tiro salvezza di Tempra o Riflessi CD 16 (a scelta del bersaglio) o venire gettata prona e subire  14 (3d6 + 4) danni da botta più 14 (3d6 + 4) danni taglienti. Se il  tiro salvezza riesce, la creatura subisce solo la metà dei danni, non è  gettata prona, e viene spinta di 1,5 metri fuori dello spazio del  bulette in uno spazio non occupato a scelta della creatura. Se non ci  sono spazi non occupati a gittata, la creatura cade prona nello spazio  del bulette.

\medskip\index{Mostri - Centauro}\textbf{Centauro}

\emph{Grande mostruosità, neutrale buono}

\textbf{FORZA} +4

\textbf{DESTREZZA} +2

\textbf{COSTITUZIONE} +2

\textbf{INTELLIGENZA} -1

\textbf{SAGGEZZA} +1

\textbf{CARISMA} +0

\textbf{Iniziativa} +2 -- \textbf{Difesa} 13

\textbf{Punti Ferita} 45 (6d10 + 12)

\textbf{Movimento} 15 m

\textbf{Competenze} Acrobatica +6, Consapevolezza +3, Sopravvivenza +3

\textbf{Linguaggi} Elfico, Silvano

\textbf{Sfida} 2 (450 PE)

\emph{\textbf{Carica.}} Se il centauro si muove di almeno 9 metri diretto verso il bersaglio e colpisce con un attacco di picca durante lo stesso turno, il bersaglio subisce 10 (3d6) danni perforanti aggiuntivi.

\textbf{Azioni}

\emph{\textbf{Multiattacco.}} Il centauro effettua due attacchi: uno con
la picca e uno con gli zoccoli o due con l'arco lungo.

\emph{\textbf{Picca.} Attacco con arma da mischia}: +6 a colpire,
portata 3 m, un bersaglio.

\emph{Colpisce:} 9 (1d10 + 4) danni perforanti.

\emph{\textbf{Zoccoli.} Attacco con arma da mischia}: +6 a colpire,
portata 1 m, un bersaglio.

\emph{Colpisce:} 11 (2d6 + 4) danni da botta.

\emph{\textbf{Arco Lungo.} Attacco con arma a Distanza}: +4 a colpire, gittata 45m, un bersaglio.

\emph{Colpisce:} 6 (1d8 + 2) danni perforanti.

\medskip\index{Mostri - Chimera}\textbf{Chimera}

\emph{Grande mostruosità, caotico malvagio}

\textbf{FORZA} +4

\textbf{DESTREZZA} +0

\textbf{COSTITUZIONE} +4

\textbf{INTELLIGENZA} -4

\textbf{SAGGEZZA} +2

\textbf{CARISMA} +0

\textbf{Iniziativa} +0 -- \textbf{Difesa} 17

\textbf{Punti Ferita} 114 (12d10 + 48)

\textbf{Movimento} 9 m, volo 18 m

\textbf{Competenze} Consapevolezza +8

\textbf{Sensi} scurovisione 18 m

\textbf{Linguaggi} comprende il Draconico ma non può parlare

\textbf{Sfida} 6 (2.300 PE)

\textbf{Azioni}

\emph{\textbf{Multiattacco.}} La chimera effettua tre attacchi: uno con  il morso, uno con le corna e uno con gli artigli. Quando il soffio infuocato  è disponibile, può usare il soffio al posto del morso o delle corna.

\emph{\textbf{Artigli.} Attacco con arma da mischia}: +7 a colpire,  portata 1 m, un bersaglio.

\emph{Colpisce:} 11 (2d6 + 4) danni taglienti.

\emph{\textbf{Corna.} Attacco con arma da mischia}: +7 a colpire,  portata 1 m, un bersaglio.

\emph{Colpisce:} 10 (1d12 + 4) danni da botta.

\emph{\textbf{Morso.} Attacco con arma da mischia}: +7 a colpire,  portata 1 m, un bersaglio.

\emph{Colpisce:} 11 (2d6 + 4) danni perforanti.

\emph{\textbf{Soffio Infuocato (Ricarica 5-6).}} La testa di drago esala  fuoco in un cono di 4,5 metri. Ogni creatura in quell'area deve  effettuare un tiro salvezza di Riflessi CD 15 e subire 31 (7d8) danni  da fuoco se fallisce il tiro salvezza, o la metà di questi danni se lo  riesce.

\medskip\index{Mostri - Chuul}\textbf{Chuul}

\emph{Grande aberrazione, caotico malvagio}

\textbf{FORZA} +4

\textbf{DESTREZZA} +0

\textbf{COSTITUZIONE} +3

\textbf{INTELLIGENZA} -3

\textbf{SAGGEZZA} +0

\textbf{CARISMA} -3

\textbf{Iniziativa} +0 -- \textbf{Difesa} 18

\textbf{Punti Ferita} 93 (11d10 + 33)

\textbf{Movimento} 9 m, nuoto 9 m

\textbf{Competenze} Consapevolezza +4

\textbf{Immunità ai Danni} veleno

\textbf{Immunità alle Condizioni} avvelenato

\textbf{Sensi} scurovisione 18 m

\textbf{Linguaggi} comprende la Linguaggio delle Profondità ma non può
parlare

\textbf{Sfida} 4 (1.100 PE)

\emph{\textbf{Anfibio.}} Il chuul può respirare aria e acqua.

\emph{\textbf{Senso della Magia.}} Il chuul percepisce la magia entro 36  metri da sé. Questo tratto funziona come l'incantesimo  \emph{individuazione} \emph{del magico} ma di per sé non è magico.  

\textbf{Azioni}

\emph{\textbf{Multiattacco.}} Il chuul effettua due attacchi con le  chele. Se il chuul sta afferrando una creatura, può anche usare i suoi tentacoli una volta.

\emph{\textbf{Chele.} Attacco con arma da mischia}: +6 a colpire,  portata 3 m, un bersaglio.

\emph{Colpisce:} 11 (2d6 + 4) danni da botta. Un bersaglio è  afferrato (CD 14 per fuggire) se è di taglia Grande o inferiore e il  chuul non sta già afferrando altre due creature.

\emph{\textbf{Tentacoli.}} Una creatura afferrata dal chuul deve  riuscire un tiro salvezza di Tempra CD 13 o restare avvelenata per  1 minuto. Fino al termine dell'avvelenamento, il bersaglio è  paralizzato. Il bersaglio può ripetere il tiro salvezza al termine di  ciascun suo turno, terminando l'effetto per sé in caso di successo.

\medskip\index{Mostri - Coboldo}\textbf{Coboldo}

\emph{Piccola umanoide (coboldo), legale malvagio}

\textbf{FORZA} -2

\textbf{DESTREZZA} +2

\textbf{COSTITUZIONE} -1

\textbf{INTELLIGENZA} -1

\textbf{SAGGEZZA} -2

\textbf{CARISMA} -1

\textbf{Iniziativa} +2 -- \textbf{Difesa} 13

\textbf{Punti Ferita} 5 (2d6 - 2)

\textbf{Movimento} 9 m

\textbf{Sensi} scurovisione 18 m

\textbf{Linguaggi} Comune, Draconico

\textbf{Sfida} 1/8 (25 PE)

\emph{\textbf{Sensibilità alla Luce}}. Mentre è alla luce del sole, il coboldo ha -1d6 ai tiri per colpire, oltre che alle prove di Saggezza (Consapevolezza) basate sulla vista.

\emph{\textbf{Tattiche di Branco.}} Il coboldo ha +1d6 ai tiri per colpire contro una creatura se almeno uno degli alleati del coboldo si trova entro 1,5 metri dalla creatura e quell'alleato non è inabile.

\textbf{Azioni}

\emph{\textbf{Pugnale.} Attacco con arma da mischia}: +4 a colpire,
portata 1 m, un bersaglio.

\emph{Colpisce:} 4 (1d4 + 2) danni perforanti.

\emph{\textbf{Fionda.} Attacco con arma a distanza}: +4 a colpire, gittata 9m, un bersaglio.

\emph{Colpisce:} 4 (1d4 + 2) danni da botta.

\medskip\index{Mostri - Cockatrice}\textbf{Cockatrice}

\emph{Piccola mostruosità, disallineato}

\textbf{FORZA} -2

\textbf{DESTREZZA} +1

\textbf{COSTITUZIONE} +1

\textbf{INTELLIGENZA} -4

\textbf{SAGGEZZA} +1

\textbf{CARISMA} -3

\textbf{Iniziativa} +1 -- \textbf{Difesa} 12

\textbf{Punti Ferita} 27 (6d6 + 6)

\textbf{Movimento} 6 m, volo 12 m

\textbf{Sensi} scurovisione 18 m

\textbf{Linguaggi} -

\textbf{Sfida} 1/2 (100 PE)

\textbf{Azioni}

\emph{\textbf{Morso.} Attacco con arma da mischia}: +3 a colpire, portata 1 m, una creatura.

\emph{Colpisce:} 3 (1d4 + 1) danni perforanti, e il bersaglio deve riuscire un tiro salvezza di Tempra CD 11 per non essere magicamente pietrificato. Se fallisce il tiro salvezza, la creatura inizia a trasformarsi in pietra ed è intralciata. Al termine del turno successivo deve ripetere il tiro salvezza. Se lo riesce, l'effetto ha termine. Se lo fallisce, la creatura è pietrificata per 24 ore.


\medskip\index{Mostri - Couatl}\textbf{Couatl}

\emph{Media celestiale, legale buono}

\textbf{FORZA} +3

\textbf{DESTREZZA} +5

\textbf{COSTITUZIONE} +3

\textbf{INTELLIGENZA} +4

\textbf{SAGGEZZA} +5

\textbf{CARISMA} +4

\textbf{Iniziativa} +5 -- \textbf{Difesa} 21

\textbf{Punti Ferita} 97 (13d8 + 39)

\textbf{Movimento} 9 m, volo 9 m

\textbf{Tiri Salvezza} Tempra +9, Riflessi +13, Volontà +14

\textbf{Resistenze al Danno} da Luce

\textbf{Immunità al Danno} psichico; da botta, perforante e tagliente di attacchi non magici

\textbf{Sensi} visione del vero 36 m

\textbf{Linguaggi} tutte, telepatia 36 m 

\textbf{Sfida} 4 (1.100 PE)

\emph{\textbf{Armi Magiche.}} Gli attacchi con armi del couatl sono magici.

\emph{\textbf{Incantesimi Innati.}} La caratteristica da incantatore innato del couatl è il Carisma. Il couatl può lanciare questi incantesimi in maniera innata, usando solo componenti verbali:

A volontà: \emph{individuazione del bene e del male, individuazione del magico, individuazione dei pensieri}

3/giorno ciascuno: \emph{benedizione, creare cibo e acqua, cura ferite,} \emph{protezione dai veleni, ristorare inferiore, santuario, scudo} 1/giorno ciascuno: \emph{ristorare superiore, scrutare, sogno}

\emph{\textbf{Mente Protetta.}} Il couatl è immune allo scrutare e qualsiasi effetto che percepisca le sue emozioni, legga i suoi pensieri o individui la sua posizione.

\textbf{Azioni}

\emph{\textbf{Morso.} Attacco con arma da mischia}: +8 a colpire, portata 1 m, una creatura.

\emph{Colpisce:} 8 (1d6 + 5) danni perforanti, e il bersaglio deve riuscire un tiro salvezza di Tempra CD 13 o restare avvelenato per 24 ore. Fino al termine dell'avvelenamento, il bersaglio è privo di sensi. Un'altra creatura può effettuare un'azione per risvegliare il bersaglio.

\emph{\textbf{Stritolare.} Attacco con arma da mischia}: +6 a colpire, portata 3 m, una creatura di taglia Media o inferiore.

\emph{Colpisce:} 10 (2d6 + 3) danni da botta, e il bersaglio è afferrato (CD 15 per fuggire). Fino al termine dell'afferrare, il bersaglio è intralciato, e il couatl non può stritolare un altro bersaglio.

\emph{\textbf{Mutare Forma.}} Il couatl può trasformarsi magicamente in un umanoide o bestia il cui grado di sfida sia pari o inferiore al proprio, o tornare alla sua vera forma. Alla morte ritorna alla sua vera forma. Qualsiasi equipaggiamento stia indossando o trasportando viene assorbito o trasportato nella nuova forma (a scelta del couatl).

Nella nuova forma, il couatl mantiene le sue statistiche di gioco e la facoltà di parlare, ma la sua Difesa, metodi di movimento, Forza, Destrezza e altre azioni vengono rimpiazzati da quelli della nuova forma, e ottiene qualsiasi statistica o capacità (Azioni aggiuntive e azioni da tana) possedute dalla sua nuova forma e non dalla sua originale. Se la nuova forma ha un attacco di morso, il couatl può usare il proprio morso nella nuova forma.

\medskip\index{Mostri - Cumulo Strisciante}\textbf{Cumulo Strisciante}

\emph{Grande pianta, disallineato}

\textbf{FORZA} +4

\textbf{DESTREZZA} -1

\textbf{COSTITUZIONE} +3

\textbf{INTELLIGENZA} -3

\textbf{SAGGEZZA} +0

\textbf{CARISMA} -3

\textbf{Iniziativa} -1 -- \textbf{Difesa} 18

\textbf{Punti Ferita} 136 (16d10 + 48)

\textbf{Movimento} 6 m, nuoto 6 m

\textbf{Competenze} Muoversi Silenziosamente / Nascondersi nelle Ombre +2

\textbf{Resistenze al Danno} freddo, fuoco

\textbf{Immunità al Danno} fulmine

\textbf{Immunità alle Condizioni} accecato, assordato, sfinimento

\textbf{Sensi} vista cieca 18 m (cieco oltre questo raggio)

\textbf{Linguaggi} -

\textbf{Sfida} 5 (1.800 PE)

\emph{\textbf{Assorbimento dei Fulmini.}} Ogni qual volta il cumulo strisciante subisce danni da fulmine, non subisce danni e recupera un numero di punti ferita pari al danno da fulmine inferto.

\textbf{Azioni}

\emph{\textbf{Multiattacco.}} Il cumulo strisciante effettua due attacchi di schianto. Se entrambi gli attacchi colpiscono una creatura di taglia Media o inferiore, il bersaglio è afferrato (CD 14 per fuggire) e il cumulo strisciante usa Avvolgere su di esso.

\emph{\textbf{Schianto.} Attacco con arma da mischia}: +7 a colpire, portata 1 m, un bersaglio.

\emph{Colpisce:} 13 (2d8 + 4) danni da botta.

\emph{\textbf{Avvolgere.}} Il cumulo strisciante avvolge una creatura di taglia Media o inferiore che ha afferrato. Il bersaglio avvolto è accecato, intralciato e impossibilitato a respirare, e deve riuscire un tiro salvezza di Tempra CD 14 all'inizio di ciascun turno del tumulo o subire 13 (2d8 + 4) danni da botta. Se il cumulo si muove, il bersaglio avvolto si muove con esso. Il cumulo può avvolgere solo una creatura alla volta.

\subsection{Demoni}

\medskip\index{Mostri - Balor}\textbf{Balor}

\emph{Enorme immondo (demone), caotico malvagio}

\textbf{FORZA} +8

\textbf{DESTREZZA} +2

\textbf{COSTITUZIONE} +6

\textbf{INTELLIGENZA} +5

\textbf{SAGGEZZA} +3

\textbf{CARISMA} +6

\textbf{Iniziativa} +5 -- \textbf{Difesa} 29

\textbf{Punti Ferita} 262 (21d12 + 126)

\textbf{Movimento} 12 m, volo 24 m

\textbf{Tiri Salvezza} Tempra +29, Riflessi +17, Volontà +25

\textbf{Resistenze al Danno} freddo, fulmine; 

\textbf{Immunità al Danno} fuoco, veleno , armi +1

textbf{Immunità alle Condizioni} avvelenato

\textbf{Vulnerabilità al Danno} ferro freddo

\textbf{Sensi} visione del vero 36 m

\textbf{Linguaggi} Abissale, telepatia 36 m

\textbf{Sfida} 19 (22.000 PE)

\emph{\textbf{Armi Magiche.}} Gli attacchi con arma del demone sono magici.

\emph{\textbf{Aura di Fuoco.}} All'inizio di ciascun turno del demone, ciascuna creatura entro 1,5 metri da lui subisce 10 (3d6) danni da fuoco, e gli oggetti infiammabili che si trovano nell'aura e che non sono indossati o trasportati prendono fuoco. Una creatura che entri a contatto con il demone o lo colpisca con un attacco da mischia mentre si trova entro 1,5 metri da esso subisce 10 (3d6) danni da fuoco.

\emph{\textbf{Resistenza alla Magia.}} Il demone ha +1d6 ai tiri salvezza contro incantesimi e altri effetti magici.

\emph{\textbf{Spasmo Mortale.}} Quando il demone muore, esplode; ciascuna creatura entro 9 metri da esso deve effettuare un tiro salvezza di Riflessi CD 20, subendo 70 (20d6) danni da fuoco se fallisce il tiro salvezza, o la metà di questi danni se lo riesce. L'esplosione appicca il fuoco agli oggetti infiammabili che non sono indossati o trasportati, e distrugge le armi del demone.

\textbf{Azioni}

\emph{\textbf{Multiattacco.}} Il demone effettua due attacchi: uno con la spada lunga e uno con la frusta.

\emph{\textbf{Frusta.} Attacco con arma da mischia}: +14 a colpire, portata 9 m, un bersaglio.

\emph{Colpisce:} 15 (2d6 + 8) danni taglienti più 10 (3d6) danni da fuoco, e il bersaglio deve riuscire un tiro salvezza di Tempra CD 20 o venire trascinato 7,5 metri verso il demone.

\emph{\textbf{Spada Lunga.} Attacco con arma da mischia}: +14 a colpire, portata 3 m, un bersaglio.

\emph{Colpisce:} 21 (3d8 + 8) danni taglienti più 13 (3d8) danni da fulmine. Se il demone ottiene un colpo critico, tira il danno tre volte, invece che due.

\emph{\textbf{Teletrasporto.}} Il demone si teletrasporta magicamente, insieme a tutto l'equipaggiamento che indossa o trasporta, in uno spazio non occupato e che può vedere entro 36 metri.

\medskip\index{Mostri - Dretch}\textbf{Dretch}

\emph{Piccola immondo (demone), caotico malvagio}

\textbf{FORZA} +0

\textbf{DESTREZZA} +0

\textbf{COSTITUZIONE} +1

\textbf{INTELLIGENZA} -3

\textbf{SAGGEZZA} -1

\textbf{CARISMA} -4

\textbf{Iniziativa} +0 -- \textbf{Difesa} 12

\textbf{Punti Ferita} 18 (4d6 + 4)

\textbf{Movimento} 6 m

\textbf{Resistenze al Danno} freddo, fulmine, fuoco

\textbf{Immunità al Danno} veleno

\textbf{Immunità alle Condizioni} avvelenato

\textbf{Vulnerabilità al Danno} ferro freddo

\textbf{Sensi} scurovisione 18 m

\textbf{Linguaggi} Abissale, telepatia 18 m (funziona solo con le creature che comprendono l'Abissale)

\textbf{Sfida} 1/4 (50 PE)

\textbf{Azioni}

\emph{\textbf{Multiattacco.}} Il demone effettua due attacchi: uno con il morso e uno con gli artigli.

\emph{\textbf{Artigli.} Attacco con arma da mischia}: +2 a colpire, portata 1 m, un bersaglio.

\emph{Colpisce:} 5 (2d4) danni taglienti.

\emph{\textbf{Morso.} Attacco con arma da mischia}: +2 a colpire, portata 1 m, un bersaglio.

\emph{Colpisce:} 3 (1d6) danni perforanti.

\emph{\textbf{Nube Fetida (1/Giorno).}} Un disgustoso gas verde si estende in un raggio di 3 metri dal demone. Il gas si propaga intorno agli angoli, e la sua area è oscurata leggermente. Rimane per 1 minuto o finché non viene disperso da un forte vento. Qualsiasi creatura che inizi il proprio turno in quell'area deve riuscire un tiro salvezza di Tempra CD 11 o restare avvelenata fino all'inizio del suo prossimo turno. Mentre è avvelenato in questo modo, il bersaglio, durante il suo turno, può effettuare solo un'azione o un'azione bonus, ma non entrambe, e non può effettuare reazioni.

\medskip\index{Mostri - Glabrezu}\textbf{Glabrezu}

\emph{Grande immondo (demone), caotico malvagio}

\textbf{FORZA} +5

\textbf{DESTREZZA} +2

\textbf{COSTITUZIONE} +5

\textbf{INTELLIGENZA} +4

\textbf{SAGGEZZA} +3

\textbf{CARISMA} +3

\textbf{Iniziativa} +4 -- \textbf{Difesa} 22

\textbf{Punti Ferita} 157 (15d10 + 75)

\textbf{Movimento} 12 m

\textbf{Tiri Salvezza} Tempra +18, Riflessi +4, Volontà +11

\textbf{Resistenze al Danno} freddo, fulmine, fuoco; da botta, perforante e tagliente di attacchi non magici

\textbf{Immunità al Danno} veleno

\textbf{Immunità alle Condizioni} avvelenato

\textbf{Vulnerabilità al Danno} ferro freddo

\textbf{Sensi} visione del vero 36 m

\textbf{Linguaggi} Abissale, telepatia 36 m 

\textbf{Sfida} 9 (5.000 PE)

\emph{\textbf{Incantesimi Innati.}} La caratteristica da incantatore del demone è l'Intelligenza. Il demone può lanciare questi incantesimi in maniera innata, senza bisogno di componenti materiali:

A volontà: \emph{dissolvi magie, individuazione del magico, oscurità}

1/giorno ciascuno: \emph{confusione, parola del potere stordire, volare}

\emph{\textbf{Resistenza alla Magia.}} Il demone ha +1d6 ai tiri salvezza contro incantesimi e altri effetti magici.

\textbf{Azioni}

\emph{\textbf{Multiattacco.}} Il demone effettua quattro attacchi: due con le chele e due con i pugni. In alternativa, può effettuare due attacchi con le chele e lanciare un incantesimo.

\emph{\textbf{Chela.} Attacco con arma da mischia}: +9 a colpire, portata 3 m, un bersaglio.

\emph{Colpisce:} 16 (2d10 + 5) danni da botta. Se il bersaglio è una creatura di taglia Media o inferiore, è afferrato (CD 15 per fuggire). Il glabrezu possiede due chele, ciascuna delle quali può afferrare un bersaglio.

\emph{\textbf{Pugno.} Attacco in mischia con arma}: +9 a colpire, portata 1 m, un bersaglio.

\emph{Colpisce:} 7 (2d4 + 2) danni da botta.

\medskip\index{Mostri - Hezrou}\textbf{Hezrou}

\emph{Grande immondo (demone), caotico malvagio}

\textbf{FORZA} +4

\textbf{DESTREZZA} +3

\textbf{COSTITUZIONE} +5

\textbf{INTELLIGENZA} 5 (-2)

\textbf{SAGGEZZA} +1

\textbf{CARISMA} +1

\textbf{Iniziativa} +3 -- \textbf{Difesa} 20

\textbf{Punti Ferita} 136 (13d10 + 65)

\textbf{Movimento} 9 m

\textbf{Tiri Salvezza} Tempra +16, Riflessi +3, Volontà +9

\textbf{Resistenze al Danno} freddo, fulmine, fuoco; da botta, perforante e tagliente di attacchi non magici

\textbf{Immunità al Danno} veleno

\textbf{Immunità alle Condizioni} avvelenato

\textbf{Vulnerabilità al Danno} ferro freddo

\textbf{Sensi} scurovisione 36 m

\textbf{Linguaggi} Abissale, telepatia 36 m 

\textbf{Sfida} 8 (3.900 PE)

\emph{\textbf{Fetore.}} Qualsiasi creatura che inizi il suo turno entro 3 metri dal demone, deve riuscire un tiro salvezza di Tempra CD 14 o restare avvelenata fino all'inizio del proprio turno. Se riesce il tiro salvezza, la creatura è immune al fetore del demone gracidante per 24 ore.

\emph{\textbf{Resistenza alla Magia.}} Il demone ha +1d6 ai tiri salvezza contro incantesimi e altri effetti magici.

\textbf{Azioni}

\emph{\textbf{Multiattacco.}} Il demone effettua tre attacchi: uno con il morso e due con gli artigli.

\emph{\textbf{Artiglio.} Attacco con arma da mischia}: +7 a colpire, portata 1 m, un bersaglio.

\emph{Colpisce:} 11 (2d6 + 4) danni taglienti.

\emph{\textbf{Morso.} Attacco con arma da mischia}: +7 a colpire, portata 1 m, un bersaglio.

\emph{Colpisce:} 15 (2d10 + 4) danni perforanti.

\medskip\index{Mostri - Marilith}\textbf{Marilith}

\emph{Grande immondo (demone), caotico malvagio}

\textbf{FORZA} +4

\textbf{DESTREZZA} +5

\textbf{COSTITUZIONE} +5

\textbf{INTELLIGENZA} +4

\textbf{SAGGEZZA} +3

\textbf{CARISMA} +5

\textbf{Iniziativa} +5 -- \textbf{Difesa} 26

\textbf{Punti Ferita} 189 (18d10 + 90)

\textbf{Movimento} 12 m

\textbf{Tiri Salvezza} Tempra +25, Riflessi +18, Volontà +13

\textbf{Resistenze al Danno} freddo, fulmine, fuoco

\textbf{Immunità al Danno} veleno, armi +1

\textbf{Immunità alle Condizioni} avvelenato

\textbf{Vulnerabilità al Danno} ferro freddo

\textbf{Sensi} visione del vero 36 m

\textbf{Linguaggi} Abissale, telepatia 36 m 

\textbf{Sfida} 16 (15.000 PE)

\emph{\textbf{Armi Magiche.}} Gli attacchi con armi del demone sono magici.

\emph{\textbf{Reattivo.}} Il demone può effettuare una reazione durante ciascun turno di combattimento.

\emph{\textbf{Resistenza alla Magia.}} Il demone ha +1d6 ai tiri salvezza contro incantesimi e altri effetti magici.

\textbf{Azioni}

\emph{\textbf{Multiattacco.}} Il demone effettua sette attacchi: sei con le spade lunghe e uno con la coda.

\emph{\textbf{Coda.} Attacco con arma da mischia}: +9 a colpire, portata 3 m, una creatura.

\emph{Colpisce:} 15 (2d10 + 4) danni da botta. Se il bersaglio è di taglia Media o inferiore, è afferrato (CD 19 per fuggire). Fino al termine dell'afferrare, il bersaglio è intralciato, e il demone può colpire automaticamente il bersaglio con la coda, ma non può effettuare attacchi di coda contro altri bersagli.

\emph{\textbf{Spada Lunga.} Attacco con arma da mischia}: +9 a colpire, portata 1 m, un bersaglio.

\emph{Colpisce:} 13 (2d8 + 4) danni taglienti.

\textbf{Reazioni}

\emph{\textbf{Parata.}} Il demone somma 5 alla sua Difesa contro un attacco da mischia che lo colpirebbe. Per farlo, il demone deve poter vedere il suo attaccante e impugnare un'arma da mischia.

\medskip\index{Mostri - Nalfeshnee}\textbf{Nalfeshnee}

\emph{Grande immondo (demone), caotico malvagio}

\textbf{FORZA} +5

\textbf{DESTREZZA} +0

\textbf{COSTITUZIONE} +6

\textbf{INTELLIGENZA} +4

\textbf{SAGGEZZA} +1

\textbf{CARISMA} +2

\textbf{Iniziativa} +4 -- \textbf{Difesa} 25

\textbf{Punti Ferita} 184 (16d10 + 96)

\textbf{Movimento} 6 m, volo 9 m

\textbf{Tiri Salvezza} Tempra +22, Riflessi +9, Volontà +21

\textbf{Resistenze al Danno} freddo, fulmine, fuoco; da botta, perforante e tagliente di attacchi non magici

\textbf{Immunità al Danno} veleno

\textbf{Immunità alle Condizioni} avvelenato

\textbf{Vulnerabilità al Danno} ferro freddo

\textbf{Sensi} scurovisione 36 m

\textbf{Linguaggi} Abissale, telepatia 36 m 

\textbf{Sfida} 13 (10.000 PE)

\emph{\textbf{Resistenza alla Magia.}} Il demone ha +1d6 ai tiri salvezza contro incantesimi e altri effetti magici.

\textbf{Azioni}

\emph{\textbf{Multiattacco.}} Il demone usa, se possibile, Aureola di Orrore. Poi effettua tre attacchi: uno con il morso e due con gli artigli.

\emph{\textbf{Artiglio.} Attacco con arma da mischia}: +10 a colpire, portata 3 m, un bersaglio.

\emph{Colpisce:} 15 (3d6 + 5) danni taglienti.

\emph{\textbf{Morso.} Attacco con arma da mischia}: +10 a colpire, portata 1 m, un bersaglio.

\emph{Colpisce:} 32 (5d10 + 5) danni perforanti.

\emph{\textbf{Aureola di Orrore (Ricarica 5-6).}} Il demone emette una luce magica multicolore e scintillante. Ogni creatura entro 4,5 metri dal demone e che possa vedere la luce, deve riuscire un tiro salvezza su Volontà CD 15 o restare spaventata per 1 minuto. Una creatura può ripetere il tiro salvezza al termine di ciascun suo turno, terminando l'effetto per sé se lo riesce. Se il tiro salvezza della creatura riesce o l'effetto ha termine per essa, la creatura è immune all'Aureola di
Orrore del demone gemente per le successive 24 ore.

\emph{\textbf{Teletrasporto.}} Il demone si teletrasporta, insieme a tutto l'equipaggiamento che sta indossando o trasportando, in uno spazio non occupato che possa vedere fino a 36 metri di distanza.

\medskip\index{Mostri - Quasit}\textbf{Quasit}

\emph{Minuscola immondo (demone, mutaforma), caotico malvagio}

\textbf{FORZA} -3

\textbf{DESTREZZA} +3

\textbf{COSTITUZIONE} +0

\textbf{INTELLIGENZA} -2

\textbf{SAGGEZZA} +0

\textbf{CARISMA} +0

\textbf{Iniziativa} +3 -- \textbf{Difesa} 14

\textbf{Punti Ferita} 7 (3d4)

\textbf{Movimento} 12 m (3 m, volo 12 m in forma di pipistrello; 12 m, scalata 12 m in forma di centopiedi; 12 m, nuoto 12 m in forma di rospo)

\textbf{Competenze} Muoversi Silenziosamente / Nascondersi nelle Ombre +5

\textbf{Resistenze al Danno} freddo, fulmine, fuoco; da botta, perforante e tagliente di attacchi non magici

\textbf{Immunità al Danno} veleno 

\textbf{Immunità alle Condizioni}
avvelenato

\textbf{Sensi} scurovisione 36 m

\textbf{Linguaggi} Abissale, Comune

\textbf{Sfida} 1 (200 PE)

\emph{\textbf{Mutaforma.}} Il demone può usare la sua azione per trasformarsi in una forma bestiale da pipistrello, centopiedi o rospo, o per tornare alla sua vera forma. Le sue statistiche sono le stesse in tutte le forme, sebbene gli attacchi possano variare per alcune di esse. Qualsiasi equipaggiamento stia indossando o trasportando non viene trasformato. Alla morte ritorna alla sua vera forma.

\emph{\textbf{Resistenza alla Magia.}} Il demone ha +1d6 ai tiri salvezza contro incantesimi e altri effetti magici.

\textbf{Azioni}

\emph{\textbf{Artigli (Morso in Forma di Bestia).} Attacco con arma da  mischia}: +4 a colpire, portata 1 m, un bersaglio. \emph{Colpisce:} 5 (1d4 + 3) danni perforanti. Se il bersaglio è una creatura, deve riuscire un tiro salvezza di Tempra CD 10 o subire 5 (2d4) danni da veleno e restare avvelenato per 1 minuto. La creatura può ripetere il tiro salvezza al termine di ciascun suo turno, ponendo termine all'effetto se lo riesce.

\emph{\textbf{Invisibilità.}} Il demone resta invisibile finché non attacca o termina la sua concentrazione. Qualsiasi cosa che il demone stia trasportando o indossando resta invisibile finché rimane in contatto con il demone.

\emph{\textbf{Spavento (1/Giorno).}} Una creatura scelta dal demone che si trovi entro 6 metri da lui, deve riuscire un tiro salvezza su Volontà CD 10 o restare spaventata per 1 minuto. Il bersaglio può ripetere il tiro salvezza al termine di ciascun suo turno, con -1d6 se il demone è in linea di visuale, ponendo termine all'effetto prematuramente se riesce il tiro salvezza.


\medskip\index{Mostri - Vrock}\textbf{Vrock}

\emph{Grande immondo (demone), caotico malvagio}

\textbf{FORZA} +3

\textbf{DESTREZZA} +2

\textbf{COSTITUZIONE} +4

\textbf{INTELLIGENZA} -1

\textbf{SAGGEZZA} +1

\textbf{CARISMA} -1

\textbf{Iniziativa} +2 -- \textbf{Difesa} 18

\textbf{Punti Ferita} 104 (11d10 + 44)

\textbf{Movimento} 12 m, volo 18 m

\textbf{Tiri Salvezza} Tempra +13, Riflessi +10, Volontà +6

\textbf{Resistenze al Danno} freddo, fulmine, fuoco; da botta, perforante e tagliente di attacchi non magici

\textbf{Immunità al Danno} veleno 

\textbf{Immunità alle Condizioni} avvelenato

\textbf{Sensi} scurovisione 36 m

\textbf{Linguaggi} Abissale, telepatia 36 m

\textbf{Sfida} 6 (2.300 PE)

\emph{\textbf{Resistenza alla Magia.}} Il demone ha +1d6 ai tiri salvezza contro incantesimi e altri effetti magici.

\textbf{Azioni}

\emph{\textbf{Multiattacco.}} Il demone effettua due attacchi: uno con il becco e uno con gli speroni.

\emph{\textbf{Becco.} Attacco con arma da mischia}: +6 a colpire, portata 1 m, un bersaglio.

\emph{Colpisce:} 10 (2d6 + 3) danni perforanti.

\emph{\textbf{Speroni.} Attacco con arma da mischia}: +6 a colpire, portata 1 m, un bersaglio.

\emph{Colpisce:} 14 (2d10 + 3) danni taglienti.

\emph{\textbf{Spore (Ricarica 6).}} Una nube di spore tossiche si diffonde in un raggio di 4,5 metri intorno al demone. Le spore si propagano intorno agli angoli. Ogni creatura in quell'area deve riuscire un tiro salvezza di Tempra CD 14 o restare avvelenata. Mentre   avvelenato in questo modo, un bersaglio subisce 5 (1d10) danni da   veleno all'inizio di ciascun suo turno. Il bersaglio può ripetere il   tiro salvezza al termine di ciascun suo turno, ponendo termine   all'effetto se lo riesce. Anche svuotare una fiala di acqua sacra sul   bersaglio pone termine all'effetto.

\emph{\textbf{Strillo Stordente (1/Giorno).}} Il demone emette uno strillo orripilante. Ogni creatura entro 6 metri da esso e che lo possa udire, e non sia un demone, deve riuscire un tiro salvezza su Tempra CD 14 o restare stordita fino al termine del prossimo turno del demone.

\medskip\index{Mostri - Destriero da Incubo}\textbf{Destriero da Incubo}

\emph{Grande immondo, neutrale malvagio}

\textbf{FORZA} +4

\textbf{DESTREZZA} +2

\textbf{COSTITUZIONE} +3

\textbf{INTELLIGENZA} +0

\textbf{SAGGEZZA} +1

\textbf{CARISMA} +2

\textbf{Iniziativa} +2 -- \textbf{Difesa} 15

\textbf{Punti Ferita} 68 (8d10 + 24)

\textbf{Movimento} 18 m, volo 24 m

\textbf{Immunità al Danno} fuoco

\textbf{Linguaggi} comprende Abissale, Comune e Infernale ma non può parlare

\textbf{Sfida} 3 (700 PE)

\emph{\textbf{Conferire Resistenza al Fuoco.}} Il destriero da incubo può conferire resistenza al danno da fuoco a chiunque lo cavalchi.

\emph{\textbf{Illuminazione.}} Il destriero da incubo irradia luce intensa in un raggio di 3 metri e luce fioca per ulteriori 3 metri.

\textbf{Azioni}

\emph{\textbf{Zoccoli.} Attacco con arma da mischia}: +6 a colpire, portata 1 m, un bersaglio.

\emph{Colpisce:} 13 (2d8 + 4) danni da botta più 7 (2d6) danni da fuoco.

\emph{\textbf{Passo Etereo.}} Il destriero da incubo e fino a tre creature consenzienti entro 1,5 metri da esso possono entrare magicamente nel Piano Etereo dal Piano Materiale e viceversa.

\subsection{Diavoli}

\medskip\index{Mostri - Diavolo Barbuto}\textbf{Diavolo Barbuto}

\emph{Media immondo (diavolo), legale malvagio}

\textbf{FORZA} +3

\textbf{DESTREZZA} +2

\textbf{COSTITUZIONE} +2

\textbf{INTELLIGENZA} -1

\textbf{SAGGEZZA} +0

\textbf{CARISMA} +0

\textbf{Iniziativa} +2 -- \textbf{Difesa} 15

\textbf{Punti Ferita} 52 (8d8 + 16)

\textbf{Movimento} 9 m

\textbf{Tiri Salvezza} Tempra +9, Riflessi +7, Volontà +3

\textbf{Resistenze al Danno} freddo; da botta, perforante e tagliente di attacchi non magici o che non siano argentati

\textbf{Immunità al Danno} fuoco, veleno

\textbf{Immunità alle Condizioni} avvelenato

\textbf{Sensi} scurovisione 36 m

\textbf{Linguaggi} Infernale, telepatia 36 m

\textbf{Sfida} 3 (700 PE)

\emph{\textbf{Resistenza alla Magia.}} Il diavolo ha +1d6 ai tiri salvezza contro incantesimi e altri effetti magici.

\emph{\textbf{Risoluto.}} Il diavolo non può essere spaventato finché riesce a vedere una creatura alleata entro 9 metri da lui.

\emph{\textbf{Vista del Diavolo.}} La scurovisione del diavolo non è limitata dall'oscurità magica.

\textbf{Azioni}

\emph{\textbf{Multiattacco.}} Il diavolo effettua due attacchi: uno con la barba e uno con il falcione.

\emph{\textbf{Barba.} Attacco con arma da mischia}: +5 a colpire, portata 1 m, una creatura.

\emph{Colpisce:} 6 (1d8 + 2) danni perforanti, e il bersaglio deve riuscire un tiro salvezza di Tempra CD 12 o restare avvelenato per 1 minuto. Mentre è avvelenato in questo modo, il bersaglio non può recuperare punti ferita. Il bersaglio può ripetere il tiro salvezza al termine di ciascun suo turno, terminando l'effetto se riesce il tiro salvezza.

\emph{\textbf{Falcione.} Attacco con arma da mischia}: +5 a colpire, portata 3 m, un bersaglio.

\emph{Colpisce:} 8 (1d10 + 3) danni taglienti. Se il bersaglio è una creatura, ad esclusione di costrutti e non morti, deve riuscire un tiro salvezza su Tempra 12 o perdere 5 (1d10) punti ferita all'inizio di ciascun suo turno a causa della ferita infernale. Ogni volta che il diavolo colpisce il bersaglio ferito con questo attacco, il danno inflitto dalla ferita aumenta di 5 (1d10). Qualsiasi creatura può effettuare un'azione per bloccare la ferita con una prova riuscita di Saggezza (Pronto Soccorso) CD 12. La ferita si richiude anche nel caso in cui il bersaglio riceva della magia guaritrice.



\medskip\index{Mostri - Diavolo delle Catene}\textbf{Diavolo delle Catene}

\emph{Media immondo (diavolo), legale malvagio}

\textbf{FORZA} +4

\textbf{DESTREZZA} +2

\textbf{COSTITUZIONE} +4

\textbf{INTELLIGENZA} +0

\textbf{SAGGEZZA} +1

\textbf{CARISMA} +2

\textbf{Iniziativa} +2 -- \textbf{Difesa} 20

\textbf{Punti Ferita} 85 (10d8 + 40)

\textbf{Movimento} 9 m

\textbf{Tiri Salvezza} Tempra +9, Riflessi +4, Volontà +3

\textbf{Resistenze al Danno} freddo; da botta, perforante e tagliente di attacchi non magici che non siano argentati

\textbf{Immunità al Danno} fuoco, veleno 

\textbf{Immunità alle Condizioni} avvelenato

\textbf{Sensi} scurovisione 36 m

\textbf{Linguaggi} Infernale, telepatia 36 m 

\textbf{Sfida} 8 (3.900 PE)

\emph{\textbf{Resistenza alla Magia.}} Il diavolo ha +1d6 ai tiri salvezza contro incantesimi e altri effetti magici.

\emph{\textbf{Vista del Diavolo.}} La scurovisione del diavolo non è limitata dall'oscurità magica.

\textbf{Azioni}

\emph{\textbf{Multiattacco.}} Il diavolo effettua due attacchi con la catena.

\emph{\textbf{Catena.} Attacco con arma da mischia}: +8 a colpire, portata 3 m, un bersaglio.

\emph{Colpisce:} 11 (2d6 + 4) danni taglienti. Il bersaglio è afferrato (CD 14 per fuggire) se il diavolo non sta già afferrando un'altra creatura. Fino al termine dell'afferrare, il bersaglio è intralciato e subisce 7 (2d6) danni perforanti all'inizio di ciascun suo turno.

\emph{\textbf{Animare Catene (Ricarica dopo un 1 ora).}} Fino a quattro catene che il diavolo possa vedere e si trovano entro 18 metri da lui producono dei bordi affilati e si animano sotto il controllo del diavolo, purché quelle catene non siano né indossate né trasportate da qualcun altro.

Ogni catena animata è un oggetto con Difesa 20, 20 punti ferita, resistenza ai danni perforanti, e immunità ai danni psichici e da tuono. Quando il diavolo usa Multiattacco durante il suo turno, può usare ciascuna catena animata per effettuare un ulteriore attacco di catena. Una catena animata può afferrare una creatura per conto proprio ma non può effettuare attacchi mentre afferra. Una catena animata ritorna al suo stato inanimato se viene ridotta a 0 punti ferita o se il diavolo è reso inabile o muore.

\textbf{Reazioni}

\emph{\textbf{Maschera Snervante.}} Quando una creatura che il diavolo può vedere inizia il proprio turno entro 9 metri dal diavolo, il diavolo può creare un'illusione per assomigliare all'amore perduto o un acerrimo rivale di quella creatura. Se la creatura può vedere il diavolo, deve riuscire un tiro salvezza di Volontà CD 14 o rimanere spaventata fino al termine del suo turno.

\medskip\index{Mostri - Diavolo Cornuto}\textbf{Diavolo Cornuto}

\emph{Grande immondo (diavolo), legale malvagio}

\textbf{FORZA} +6

\textbf{DESTREZZA} +3

\textbf{COSTITUZIONE} +5

\textbf{INTELLIGENZA} +1

\textbf{SAGGEZZA} +3

\textbf{CARISMA} +3

\textbf{Iniziativa} +3 -- \textbf{Difesa} 23

\textbf{Punti Ferita} 178 (17d10 + 85)

\textbf{Movimento} 6 m, volo 18 m

\textbf{Tiri Salvezza} Tempra +18, Riflessi +17, Volontà +13

\textbf{Resistenze al Danno} freddo; da botta, perforante e tagliente di attacchi che non siano argentati

\textbf{Immunità al Danno} fuoco, veleno, armi +1

\textbf{Immunità alle Condizioni} avvelenato

\textbf{Sensi} scurovisione 36 m

\textbf{Linguaggi} Infernale, telepatia 36 m 

\textbf{Sfida} 11 (7.200 PE)

\emph{\textbf{Resistenza alla Magia.}} Il diavolo ha +1d6 ai tiri salvezza contro incantesimi e altri effetti magici.

\emph{\textbf{Vista del Diavolo.}} La scurovisione del diavolo non è limitata dall'oscurità magica.

\textbf{Azioni}

\emph{\textbf{Multiattacco.}} Il diavolo effettua tre attacchi da mischia: due con il forcone e uno con la coda. Può usare Scagliare Fiamma al posto di qualsiasi attacco da mischia.

\emph{\textbf{Coda.} Attacco con arma da mischia}: +10 a colpire, portata 3 m, un bersaglio.

\emph{Colpisce:} 10 (1d8 + 6) danni perforanti. Se il bersaglio è una creatura, ad esclusione di costrutti e non morti, deve riuscire un tiro salvezza su Tempra 17 o perdere 10 (3d6) punti ferita all'inizio di ciascun suo turno a causa della ferita infernale. Ogni volta che il diavolo ferisce il bersaglio con questo attacco, il danno inflitto dalla ferita aumenta di 10 (3d6). Qualsiasi creatura può effettuare un'azione per bloccare la ferita riuscendo una prova di Saggezza (Pronto Soccorso) CD 12. La ferita si richiude anche nel caso in cui il bersaglio riceva magia guaritrice.

\emph{\textbf{Forcone.} Attacco con arma da mischia}: +10 a colpire, portata 3 m, un bersaglio.

\emph{Colpisce:} 15 (2d8 + 6) danni perforanti.

\emph{\textbf{Pungiglione.} Attacco con arma da mischia}: +8 a colpire, portata 3 m, un bersaglio.

\emph{Colpisce:} 13 (2d8 + 4) danni perforanti più 17 (5d6) danni da veleno, e il bersaglio deve riuscire un tiro salvezza di Tempra CD 14, o restare avvelenato per 1 minuto. Il bersaglio può ripetere il tiro salvezza al termine di ciascun suo turno, terminando l'effetto se lo
riesce.

\emph{\textbf{Scagliare Fiamma.} Attacco con incantesimo a Distanza}: +7 a colpire, gittata 45 m, un bersaglio.

\emph{Colpisce:} 14 (4d6) danni da fuoco. Se il bersaglio è un oggetto infiammabile che non sia indossato o trasportato, prende fuoco.

\medskip\index{Mostri - Diavolo della Fossa}\textbf{Diavolo della Fossa}

\emph{Grande immondo (diavolo), legale malvagio}

\textbf{FORZA} +8

\textbf{DESTREZZA} +2

\textbf{COSTITUZIONE} +7

\textbf{INTELLIGENZA} +6

\textbf{SAGGEZZA} +4

\textbf{CARISMA} +7

\textbf{Iniziativa} +6 -- \textbf{Difesa} 29

\textbf{Punti Ferita} 300 (24d10 + 168) 

\textbf{Movimento} 9 m, volo 18 m

\textbf{Tiri Salvezza} Tempra +24, Riflessi +21, Volontà +18

\textbf{Resistenze al Danno} freddo; da botta, perforante e tagliente che non siano argentati

\textbf{Immunità al Danno} fuoco, veleno , armi +2

\textbf{Immunità alle Condizioni} avvelenato

\textbf{Sensi} visione del vero 36 m

\textbf{Linguaggi} Infernale, telepatia 36 m

\textbf{Sfida} 20 (25.000 PE)

\emph{\textbf{Arma Magica.}} Gli attacchi con arma del diavolo della fossa sono magici.

\emph{\textbf{Aura di Paura.}} Qualsiasi creatura ostile al diavolo che inizi il suo turno entro 6 metri da esso, deve effettuare un tiro salvezza su Volontà CD 21, a meno che il diavolo non sia inabile. Se fallisce il tiro salvezza, la creatura è spaventata fino all'inizio del suo prossimo turno. Se il tiro salvezza della creatura riesce, la creatura è immune all'Aura di Paura del diavolo per le successive 24 ore.

\emph{\textbf{Incantesimi Innati.}} La caratteristica da incantatore  diavolo della fossa è il Carisma. Il diavolo della fossa può lanciare questi incantesimi in maniera innata, senza bisogno di componenti materiali:

A volontà: \emph{individuazione del magico, palla di fuoco}

3/giorno ciascuno: \emph{blocca mostri, muro di fuoco}

\emph{\textbf{Resistenza alla Magia.}} Il diavolo ha +1d6 ai tiri salvezza contro incantesimi e altri effetti magici.

\textbf{Azioni}

\emph{\textbf{Multiattacco.}} Il diavolo effettua quattro attacchi: uno con il morso, uno con l'artiglio, uno con la mazza e uno con la coda.

\emph{\textbf{Artiglio.} Attacco con arma da mischia}: +14 a colpire, portata 3 m, un bersaglio.

\emph{Colpisce:} 17 (2d8 + 8) danni taglienti.

\emph{\textbf{Coda.} Attacco con arma da mischia}: +14 a colpire, portata 3 m, un bersaglio.

\emph{Colpisce:} 24 (3d10 + 8) danni da botta.

\emph{\textbf{Mazza.} Attacco con arma da mischia}: +14 a colpire, portata 3 m, un bersaglio.

\emph{Colpisce:} 15 (2d6 + 8) danni da botta più 21 (6d6) danni da fuoco.

\emph{\textbf{Morso.} Attacco con arma da mischia}: +14 a colpire, portata 1 m, un bersaglio.

\emph{Colpisce:} 22 (4d6 + 8) danni perforanti. Il bersaglio deve riuscire un tiro salvezza di Tempra CD 21 o restare avvelenato. Mentre è avvelenato in questo modo, il bersaglio non può recuperare punti ferita, e subisce 21 (6d6) danni da veleno all'inizio di ciascun suo turno. Il bersaglio avvelenato può ripetere il tiro salvezza al termine di ciascun suo turno, terminando l'effetto su di sé.

\medskip\index{Mostri - Diavolo del Ghiaccio}\textbf{Diavolo del Ghiaccio}

\emph{Grande immondo (diavolo), legale malvagio}

\textbf{FORZA} +5

\textbf{DESTREZZA} +2

\textbf{COSTITUZIONE} +4

\textbf{INTELLIGENZA} +4

\textbf{SAGGEZZA} +2

\textbf{CARISMA} +4

\textbf{Iniziativa} +4 -- \textbf{Difesa} 25

\textbf{Punti Ferita} 180 (19d10 + 76)

\textbf{Movimento} 12 m

\textbf{Tiri Salvezza} Tempra +15, Riflessi +14, Volontà +12

\textbf{Resistenze al Danno} da botta, perforante e tagliente di attacchi che non siano argentate

\textbf{Immunità al Danno} freddo, fuoco, veleno , armi +1

\textbf{Immunità alle Condizioni} avvelenato

\textbf{Sensi} vista cieca 18 m, scurovisione 36 m

\textbf{Linguaggi} Infernale, telepatia 36 m

\textbf{Sfida} 14 (11.500 PE)

\emph{\textbf{Resistenza alla Magia.}} Il diavolo ha +1d6 ai tiri salvezza contro incantesimi e altri effetti magici.

\emph{\textbf{Vista del Diavolo.}} La scurovisione del diavolo non è limitata dall'oscurità magica.

\textbf{Azioni}

\emph{\textbf{Multiattacco.}} Il diavolo effettua tre attacchi: uno con il morso, uno con gli artigli e uno con la coda. In alternativa effettua due attacchi: uno con la coda e uno con lancia.

\emph{\textbf{Artigli.} Attacco con arma da mischia}: +10 a colpire, portata 1 m, un bersaglio.

\emph{Colpisce:} 10 (2d4 + 5) danni taglienti più 10 (3d6) danni da freddo.

\emph{\textbf{Coda.} Attacco con arma da mischia}: +10 a colpire, portata 3 m, un bersaglio.

\emph{Colpisce:} 12 (2d6 + 5) danni da botta più 10 (3d6) danni da freddo.

\emph{\textbf{Lancia di Ghiaccio.} Attacco con arma da mischia}: +10 a colpire, portata 3 m, un bersaglio.

\emph{Colpisce:} 14 (2d8 + 5) danni perforanti più 10 (3d6) danni da freddo. Se il bersaglio è una creatura, deve riuscire un tiro salvezza su Tempra CD 15, o avere per 1 minuto la velocità ridotta di 3 metri; durante ciascun suo turno può effettuare solo un'azione o un'azione bonus, ma non entrambe; non può effettuare reazioni. Il bersaglio può ripetere il tiro salvezza al termine di ciascun suo turno, terminando l'effetto su di sé in caso di successo.

\emph{\textbf{Morso.} Attacco con arma da mischia}: +10 a colpire,
portata 1 m, un bersaglio.

\emph{Colpisce:} 12 (2d6 + 5) danni perforanti più 10 (3d6) danni da
freddo.

\emph{\textbf{Muro di Ghiaccio (Ricarica 6).}} Il diavolo forma magicamente un muro di ghiaccio opaco su di una superficie solida che possa vedere entro 18 metri da lui. Il muro è spesso 30 centimetri e largo fino a 9 metri per un massimo di 3 metri di altezza, oppure   una cupola semisferica di massimo 6 metri di diametro. Quando la   parete appare, ogni creatura nel suo spazio viene spinta fuori da esso   tramite la via più breve. La creatura sceglie su quale lato del muro   finire, a meno che la creatura non sia inabile. La creatura poi   effettua un tiro salvezza di Riflessi CD 17, subendo 35 (10d6) danni da freddo se lo fallisce, o la metà di questi danni se lo riesce.

Il muro rimane per 1 minuto o finché il diavolo non è reso inabile o muore. Il muro può essere danneggiato e bucato; ogni sezione di 3 metri ha Difesa 5, 30 punti ferita, vulnerabilità al danno da fuoco, e immunità al danno da acido, freddo, da Vuoto, psichico e da veleno. Se una sezione viene distrutta, lascia una patina di aria gelida nello spazio che occupava prima il muro. Ogni volta che una creatura finisce per muoversi attraverso quest'aria gelida durante un turno, consenziente o meno, deve effettuare un tiro salvezza di Tempra CD 17, subendo 17 (5d6)danni da freddo se lo  fallisce, o la metà di questi danni se lo riesce. L'aria gelida si dissipa quando il resto del muro svanisce.


\medskip\index{Mostri - Diavolo d'Ossa}\textbf{Diavolo d'Ossa}

\emph{Grande immondo (diavolo), legale malvagio}

\textbf{FORZA} +4

\textbf{DESTREZZA} +3

\textbf{COSTITUZIONE} +4

\textbf{INTELLIGENZA} +1

\textbf{SAGGEZZA} +2

\textbf{CARISMA} +3

\textbf{Iniziativa} +3 -- \textbf{Difesa} 24

\textbf{Punti Ferita} 142 (15d10 + 60)

\textbf{Movimento} 12 m, volo 12 m

\textbf{Tiri Salvezza} Tempra +12, Riflessi +12, Volontà +7

\textbf{Competenze} Ingannare +7, Percepire Emozioni +6

\textbf{Resistenze al Danno} freddo; da botta, perforante e tagliente di attacchi non magici o che non siano argentati

\textbf{Immunità al Danno} fuoco, veleno

\textbf{Immunità alle Condizioni} avvelenato

\textbf{Sensi} scurovisione 36 m

\textbf{Linguaggi} Infernale, telepatia 36 m 

\textbf{Sfida} 9 (5.000 PE)

\emph{\textbf{Resistenza alla Magia.}} Il diavolo ha +1d6 ai tiri salvezza contro incantesimi e altri effetti magici.

\emph{\textbf{Vista del Diavolo.}} La scurovisione del diavolo non è limitata dall'oscurità magica.

\textbf{Azioni}

\emph{\textbf{Multiattacco.}} Il diavolo effettua tre attacchi: due con gli artigli e uno con il pungiglione oppure uno con la sua arma inastata uncinata e uno con il pungiglione.

\emph{\textbf{Arma Inastata Uncinata.} Attacco con arma da mischia}: +8
a colpire, portata 3 m, un bersaglio.

\emph{Colpisce:} 17 (2d12 + 4) danni perforanti. Se il bersaglio è una creatura di taglia Enorme o inferiore, è afferrato (CD 14 per fuggire). Fino al termine dell'afferrare, il diavolo non può usare la sua arma inastata su di un altro bersaglio.

\emph{\textbf{Artiglio.} Attacco con arma da mischia}: +8 a colpire, portata 3 m, un bersaglio.

\emph{Colpisce:} 8 (1d8 + 4) danni taglienti.

\emph{\textbf{Pungiglione.} Attacco con arma da mischia}: +8 a colpire, portata 3 m, un bersaglio.

\emph{Colpisce:} 13 (2d8 + 4) danni perforanti più 17 (5d6) danni da veleno, e il bersaglio deve riuscire un tiro salvezza di Tempra CD 14, o restare avvelenato per 1 minuto. Il bersaglio può ripetere il tiro salvezza al termine di ciascun suo turno, terminando l'effetto se lo riesce.

\medskip\index{Mostri - Diavolo Spinoso}\textbf{Diavolo Spinoso}

\emph{Piccola immondo (diavolo), legale malvagio}

\textbf{FORZA} +0

\textbf{DESTREZZA} +2

\textbf{COSTITUZIONE} +1

\textbf{INTELLIGENZA} +0

\textbf{SAGGEZZA} +2

\textbf{CARISMA} -1

\textbf{Iniziativa} +2 -- \textbf{Difesa} 14

\textbf{Punti Ferita} 22 (5d6 + 5)

\textbf{Movimento} 6 m, volo 12 m

\textbf{Resistenze al Danno} freddo; da botta, perforante e tagliente di attacchi non magici o che non siano argentati

\textbf{Immunità al Danno} fuoco, veleno

\textbf{Immunità alle Condizioni} avvelenato

\textbf{Sensi} scurovisione 36 m

\textbf{Linguaggi} Infernale, telepatia 36 m

\textbf{Sfida} 2 (450 PE)

\emph{\textbf{Resistenza alla Magia.}} Il diavolo ha +1d6 ai tiri salvezza contro incantesimi e altri effetti magici.

\emph{\textbf{Sorvolare.}} Il diavolo non provoca attacchi di opportunità quando vola via dalla portata di un nemico.

\emph{\textbf{Spine Limitate.}} Il diavolo possiede dodici spine caudali. Le spine usate ricrescono a mezzanotte.

\emph{\textbf{Vista del Diavolo.}} La scurovisione del diavolo non è limitata dall'oscurità magica.

\textbf{Azioni}

\emph{\textbf{Multiattacco.}} Il diavolo effettua due attacchi: uno con il morso e uno con il suo forcone o due con le sue spine caudali.

\emph{\textbf{Forcone.} Attacco con arma da mischia}: +2 a colpire, portata 1 m, un bersaglio.

\emph{Colpisce:} 3 (1d6) danni perforanti.

\emph{\textbf{Morso.} Attacco con arma da mischia}: +2 a colpire, portata 1 m, un bersaglio.

\emph{Colpisce:} 5 (2d4) danni taglienti.

\emph{\textbf{Spina Caudale.} Attacco con arma a Distanza}: +4 a colpire, gittata 6m, un bersaglio.

\emph{Colpisce:} 4 (1d4 + 2) danni perforanti più 3 (1d6) danni da
fuoco.

\medskip\index{Mostri - Erinni}\textbf{Erinni}

\emph{Media immondo (diavolo), legale malvagio}

\textbf{FORZA} +4

\textbf{DESTREZZA} +3

\textbf{COSTITUZIONE} +4

\textbf{INTELLIGENZA} +2

\textbf{SAGGEZZA} +2

\textbf{CARISMA} +4

\textbf{Iniziativa} +3 -- \textbf{Difesa} 24 (armatura di piastre)

\textbf{Punti Ferita} 153 (18d8 + 72)

\textbf{Movimento} 9 m, volo 18 m

\textbf{Tiri Salvezza} Tempra +11, Riflessi +12, Volontà +7

\textbf{Resistenze al Danno} freddo; da botta, perforante e tagliente di attacchi non magici o che non siano argentati

\textbf{Immunità al Danno} fuoco, veleno

\textbf{Immunità alle Condizioni} avvelenato

\textbf{Sensi} visione del vero 36 m

\textbf{Linguaggi} Infernale, telepatia 36 m

\textbf{Sfida} 12 (8.400 PE)

\emph{\textbf{Armi Diaboliche.}} Gli attacchi con arma dell'erinni sono magici e infliggono 13 (3d8) danni da veleno aggiuntivi quando colpiscono (già incluso negli attacchi).

\emph{\textbf{Resistenza alla Magia.}} L'erinni ha +1d6 ai tiri salvezza contro incantesimi e altri effetti magici.

\textbf{Azioni}

\emph{\textbf{Multiattacco.}} L'erinni effettua tre attacchi.

\emph{\textbf{Spada Lunga.} Attacco con arma da mischia}: +8 a colpire, portata 1 m, un bersaglio.

\emph{Colpisce:} 8 (1d8 + 4) danni taglienti, o 9 (1d10 + 4) danni taglienti se usata con due mani, più 13 (3d8) danni da veleno.

\emph{\textbf{Arco Lungo.} Attacco con arma a Distanza}: +7 a colpire, gittata 45m, un bersaglio.

\emph{Colpisce:} 7 (1d8 + 4) danni perforanti più 13 (3d8) danni da veleno, e il bersaglio deve riuscire un tiro salvezza di Tempra CD 14 o restare avvelenato. Il veleno rimane finché non viene rimosso da un incantesimo \emph{ristorazione inferiore} o simile.

\textbf{Reazioni}

\emph{\textbf{Parata.}} L'erinni somma 4 alla sua Difesa contro un attacco da mischia che lo colpirebbe. Per farlo, l'erinni deve poter vedere il suo attaccante e impugnare un'arma da mischia.

\medskip\index{Mostri - Imp}\textbf{Imp}

\emph{Minuscola immondo (diavolo, mutaforma), legale malvagio}

\textbf{FORZA} -2

\textbf{DESTREZZA} +3

\textbf{COSTITUZIONE} +1

\textbf{INTELLIGENZA} +0

\textbf{SAGGEZZA} +1

\textbf{CARISMA} +2

\textbf{Iniziativa} +3 -- \textbf{Difesa} 14

\textbf{Punti Ferita} 10 (3d4 + 3)

\textbf{Movimento} 6 m, volo 12 m (6 m in forma di ratto; 6 m, volo 18 m in forma di corvo; 6 m, scalata 6 m in forma di ragno)

\textbf{Tiri Salvezza} Tempra +1, Riflessi +6, Volontà +4

\textbf{Competenze} Muoversi Silenziosamente / Nascondersi nelle Ombre +5, Ingannare +4, Percepire Emozioni +3

\textbf{Resistenze al Danno} freddo; da botta, perforante e tagliente di attacchi non magici o che non siano argentati

\textbf{Immunità al Danno} fuoco, veleno

\textbf{Immunità alle Condizioni} avvelenato

\textbf{Sensi} scurovisione 36 m 

\textbf{Linguaggi} Infernale, Comune

\textbf{Sfida} 1 (200 PE)

\emph{\textbf{Mutaforma.}} Il diavolo può usare la sua azione per trasformarsi in una forma bestiale da ratto, corvo o ragno, o per tornare alla sua vera forma. Le sue statistiche sono le stesse in tutte le forme, sebbene gli attacchi possano variare per alcune di esse. Qualsiasi equipaggiamento stia indossando o trasportando non viene trasformato. Alla morte ritorna alla sua vera forma.

\emph{\textbf{Resistenza alla Magia.}} Il diavolo ha +1d6 ai tiri salvezza contro incantesimi e altri effetti magici.

\emph{\textbf{Vista del Diavolo.}} La scurovisione del diavolo non è limitata dall'oscurità magica.

\textbf{Azioni}

\emph{\textbf{Pungiglione (Morso in Forma di Bestia).} Attacco con arma da mischia}: +5 a colpire, portata 1 m, una creatura.

\emph{Colpisce:} 5 (1d4 + 3) danni perforanti, e il bersaglio deve effettuare un tiro salvezza di Tempra CD 11, subendo 10 (3d6) danni da veleno se lo fallisce, o la metà di questi danni se lo riesce. 

\emph{\textbf{Invisibilità.}} Il diavolo resta invisibile finché non attacca o termina la sua concentrazione. Qualsiasi cosa che il diavolo stia trasportando o indossando, resta invisibile finché rimane in contatto con il diavolo.

\medskip\index{Mostri - Lemure}\textbf{Lemure}

\emph{Media immondo (diavolo), legale malvagio}

\textbf{FORZA} +0

\textbf{DESTREZZA} -3

\textbf{COSTITUZIONE} +0

\textbf{INTELLIGENZA} -5

\textbf{SAGGEZZA} +0

\textbf{CARISMA} -4

\textbf{Iniziativa} -3 -- \textbf{Difesa} 8

\textbf{Punti Ferita} 13 (3d8)

\textbf{Movimento} 4,5 m

\textbf{Tiri Salvezza} Tempra +4, Riflessi +3, Volontà +0

\textbf{Resistenze al Danno} freddo

\textbf{Immunità al Danno} fuoco, veleno

\textbf{Immunità alle Condizioni} affascinato, avvelenato, spaventato

\textbf{Sensi} scurovisione 36 m

\textbf{Linguaggi} comprende l'Infernale ma non può parlare

\textbf{Sfida} 0 (10 PE)

\emph{\textbf{Rinvigorimento Diabolico.}} Un lemure che muore nei Nove Inferi ritorna in vita con tutti i suoi punti ferita in 1d10 giorni a meno che non venga ucciso da una creatura di allineamento buono su cui sia stato eseguito l'incantesimo \emph{benedire} o i suoi resti vengano
cosparsi di acqua sacra.

\emph{\textbf{Vista del Diavolo.}} La scurovisione del diavolo non è limitata dall'oscurità magica.

\textbf{Azioni}

\emph{\textbf{Pugno.} Attacco con arma da mischia}: +3 a colpire, portata 1 m, un bersaglio.

\emph{Colpisce:} 2 (1d4) danni da botta.

\subsection{Dinosauri}

\medskip\index{Mostri - Plesiosauro}\textbf{Plesiosauro}

\emph{Grande bestia, disallineato}

\textbf{FORZA} +4

\textbf{DESTREZZA} +2

\textbf{COSTITUZIONE} +3

\textbf{INTELLIGENZA} -4

\textbf{SAGGEZZA} +1

\textbf{CARISMA} -3

\textbf{Iniziativa} +2 -- \textbf{Difesa} 14

\textbf{Punti Ferita} 68 (8d10 + 24)

\textbf{Movimento} 6 m, nuoto 12 m

\textbf{Tiri Salvezza} Tempra +18, Riflessi +11, Volontà +9

\textbf{Competenze} Muoversi Silenziosamente / Nascondersi nelle Ombre +4, Consapevolezza +3

\textbf{Linguaggi} -

\textbf{Sfida} 2 (450 PE)

\emph{\textbf{Trattenere il Fiato.}} Il plesiosauro può trattenere il fiato per 1 ora.

\textbf{Azioni}

\emph{\textbf{Morso.} Attacco con arma da mischia}: +6 a colpire, portata 3 m, un bersaglio.

\emph{Colpisce:} 14 (3d6 + 4) danni perforanti.

\medskip\index{Mostri - Tirannosauro}\textbf{Tirannosauro}

\emph{Enorme bestia, disallineato}

\textbf{FORZA} +7

\textbf{DESTREZZA} +0

\textbf{COSTITUZIONE} +4

\textbf{INTELLIGENZA} -4

\textbf{SAGGEZZA} +1

\textbf{CARISMA} -1

\textbf{Iniziativa} +0 -- \textbf{Difesa} 17

\textbf{Punti Ferita} 136 (13d12 + 52)

\textbf{Movimento} 15 m

\textbf{Tiri Salvezza} Tempra +15, Riflessi +12, Volontà +10

\textbf{Competenze} Consapevolezza +4

\textbf{Linguaggi} -

\textbf{Sfida} 8 (3.900 PE)

\textbf{Azioni}

\emph{\textbf{Multiattacco.}} Il tirannosauro effettua due attacchi: uno con il morso e uno con la coda. Non può effettuare entrambi gli attacchi contro lo stesso bersaglio.

\emph{\textbf{Coda.} Attacco con arma da mischia}: +10 a colpire, portata 3 m, un bersaglio.

\emph{Colpisce:} 20 (3d8 + 7) danni da botta.

\emph{\textbf{Morso.} Attacco con arma da mischia}: +10 a colpire, portata 3 m, un bersaglio.

\emph{Colpisce:} 33 (4d12 + 7) danni perforanti. Se il bersaglio è una creatura di taglia Media o inferiore, è afferrato (CD 17 per fuggire). Fino al termine dell'afferrare, il bersaglio è intralciato, e il tirannosauro non può usare il morso contro un altro bersaglio.

\medskip\index{Mostri - Triceratopo}\textbf{Triceratopo}

\emph{Enorme bestia, disallineato}

\textbf{FORZA} +6

\textbf{DESTREZZA} -1

\textbf{COSTITUZIONE} +3

\textbf{INTELLIGENZA} -4

\textbf{SAGGEZZA} +0

\textbf{CARISMA} -3

\textbf{Iniziativa} -1 -- \textbf{Difesa} 16

\textbf{Punti Ferita} 95 (10d12 + 30)

\textbf{Movimento} 15 m

\textbf{Tiri Salvezza} Tempra +15, Riflessi +8, Volontà +5

\textbf{Linguaggi} -

\textbf{Sfida} 5 (1.800 PE)

\emph{\textbf{Carica Travolgente.}} Se il triceratopo si muove di almeno 6 metri diretto verso una creatura e la colpisce con un attacco di incornata durante lo stesso turno, il bersaglio deve riuscire un tiro salvezza su Tempra CD 13 o cadere prono. Se il bersaglio è prono, il triceratopo può effettuare un attacco di pestone contro di lui come azione bonus.

\textbf{Azioni}

\emph{\textbf{Incornata.} Attacco con arma da mischia}: +9 a colpire,
portata 1 m, un bersaglio.

\emph{Colpisce:} 24 (3d10 + 6) danni perforanti.

\emph{\textbf{Pestone.} Attacco con arma da mischia}: +9 a colpire,
portata 1 m, una creatura prona.

\emph{Colpisce:} 22 (3d10 + 6) danni da botta.

\medskip\index{Mostri - Doppelganger}\textbf{Doppelganger}

\emph{Media mostruosità (mutaforma), neutrale}

\textbf{FORZA} +0

\textbf{DESTREZZA} +4

\textbf{COSTITUZIONE} +2

\textbf{INTELLIGENZA} +0

\textbf{SAGGEZZA} +1

\textbf{CARISMA} +2

\textbf{Iniziativa} +4 -- \textbf{Difesa} 16

\textbf{Punti Ferita} 52 (8d8 + 16)

\textbf{Movimento} 9 m

\textbf{Tiri Salvezza} Tempra +4, Riflessi +5, Volontà +6

\textbf{Competenze} Ingannare +6, Percepire Emozioni +3

\textbf{Immunità alle Condizioni} affascinato

\textbf{Sensi} scurovisione 18 m

\textbf{Linguaggi} Comune

\textbf{Sfida} 3 (700 PE)

\emph{\textbf{Mutaforma.}} Il doppelganger può usare la sua azione per cambiare la propria forma in quella di un umanoide Piccolo o Medio che abbia visto, o per tornare alla sua vera forma. Le sue statistiche, a parte la taglia, sono le stesse in tutte le forme. Qualsiasi equipaggiamento stia indossando o trasportando non viene trasformato. Alla morte ritorna alla sua vera forma.

\emph{\textbf{Appostato.}} Nel primo round di combattimento, il doppelganger ha +1d6 ai tiri di attacco contro qualsiasi creatura abbia preso di sorpresa.

\emph{\textbf{Attacco di Sorpresa.}} Se il doppelganger sorprende una creatura e la colpisce con un attacco durante il primo round di combattimento, il bersaglio subisce 10 (3d6) danni aggiuntivi dall'attacco.

\textbf{Azioni}

\emph{\textbf{Multiattacco.}} Il doppelganger effettua due attacchi da
mischia.

\emph{\textbf{Schianto.} Attacco con arma da mischia}: +6 a colpire,
portata 1 m, un bersaglio.

\emph{Colpisce:} 7 (1d6 + 4) danni da botta.

\emph{\textbf{Leggere Pensieri.}} Il doppelganger legge magicamente i pensieri di superficie di una creatura entro 18 metri da lui. L'effetto può penetrare le barriere, ma 1 metro di legno o terra, 50 centimetri di pietra, 5 centimetri di metallo, o un sottile foglio di piombo lo blocca. Mentre il bersaglio è a gittata, il doppelganger può continuare a leggerne i pensieri, purché la concentrazione del doppelganger non venga infranta (come la concentrazione di un incantesimo). Mentre legge la mente di un bersaglio, il doppelganger ha +1d6 alle prove di Saggezza e Carisma contro il bersaglio.

\subsection{Draghi Cromatici}

\medskip\index{Mostri - Drago Bianco Antico}\textbf{Drago Bianco Antico}

\emph{Mastodontica drago, caotico malvagio}

\textbf{FORZA} +8

\textbf{DESTREZZA} +0

\textbf{COSTITUZIONE} +8

\textbf{INTELLIGENZA} +0

\textbf{SAGGEZZA} +1

\textbf{CARISMA} +2

\textbf{Iniziativa} +0 -- \textbf{Difesa} 30

\textbf{Punti Ferita} 333 (18d20 + 144)

\textbf{Movimento} 12 m, nuoto 12 m, scavo 12 m, volo 24 m

\textbf{Tiri Salvezza} Tempra +19, Riflessi +14, Volontà +16

\textbf{Competenze} Muoversi Silenziosamente / Nascondersi nelle Ombre +6, Consapevolezza +13

\textbf{Immunità al Danno} freddo, arma +1

\textbf{Sensi} scurovisione 36 m, vista cieca 18 m

\textbf{Linguaggi} Comune, Draconico

\textbf{Sfida} 20 (25.000 PE)

\emph{\textbf{Camminare sul Ghiaccio.}} Il drago può muoversi e arrampicarsi su superfici ghiacciate senza bisogno di effettuare prove di caratteristica. Inoltre, il terreno difficile composto di ghiaccio o neve non gli costa movimento aggiuntivo.

\emph{\textbf{Resistenza Leggendaria (3/Giorno).}} Se il drago fallisce un tiro salvezza, può scegliere invece di riuscire.

\textbf{Azioni}

\emph{\textbf{Multiattacco.}} Il drago può usare la sua Presenza Spaventosa. Poi effettuare tre attacchi: uno con il morso e due con gli artigli.

\emph{\textbf{Artiglio.} Attacco con arma da mischia}: +14 a colpire, portata 3 m, un bersaglio.

\emph{Colpisce:} 15 (2d6 + 8) danni taglienti.

\emph{\textbf{Coda.} Attacco con arma da mischia}: +14 a colpire, portata 6 m, un bersaglio.

\emph{Colpisce:} 17 (2d8 + 8) danni da botta.

\emph{\textbf{Morso.} Attacco con arma da mischia}: +14 a colpire, portata 4,5 m, un bersaglio.

\emph{Colpisce:} 19 (2d10 + 8) danni perforanti più 9 (2d8) danni da freddo.

\emph{\textbf{Presenza Spaventosa.}} Ogni creatura scelta dal drago, che si trovi entro 36 metri da esso e consapevole della sua presenza, deve riuscire un tiro salvezza di Volontà CD 16 o restare spaventata per 1 minuto. Una creatura può ripetere il tiro salvezza al termine di ciascun suo turno, terminando l'effetto se lo riesce. Se il tiro salvezza della creatura ha successo o l'effetto ha termine per essa, la creatura è immune alla Presenza Spaventosa del drago per le successive 24 ore.

\emph{\textbf{Soffio Gelido (Ricarica 5-6).}} Il drago esala un'esplosione di ghiaccio in un cono di 27 metri. Ogni creatura in quell'area deve effettuare un tiro salvezza di Tempra CD 22 e subire 72 (16d8) danni da freddo se fallisce il tiro salvezza, o la metà di questi danni se lo riesce.

\textbf{Azioni Aggiuntive}

Il drago può effettuare 3 Azioni aggiuntive, scelte tra le opzioni seguenti. Può usare solo un'opzione leggendaria alla volta e solo al termine del turno di un'altra creatura. Il drago recupera le Azioni aggiuntive spese all'inizio del proprio turno.

\textbf{Attacco di Ala (Costa 2 Azioni).} Il drago batte le ali. Ogni creatura entro 4,5 metri dal drago deve riuscire un tiro salvezza su Riflessi CD 22 o subire 15 (2d6 + 8) danni da botta e venir gettato prono. Il drago può poi volare fino a metà del suo movimento di volo. \textbf{Attacco di Coda.} Il drago effettua un attacco di coda. \textbf{Individuare.} Il drago effettua una prova di Saggezza (Consapevolezza).

\medskip\index{Mostri - Drago Bianco Adulto}\textbf{Drago Bianco Adulto}

\emph{Enorme drago, caotico malvagio}

\textbf{FORZA} +6

\textbf{DESTREZZA} +0

\textbf{COSTITUZIONE} +6

\textbf{INTELLIGENZA} -1

\textbf{SAGGEZZA} +1

\textbf{CARISMA} +1

\textbf{Iniziativa} +0 -- \textbf{Difesa} 25

\textbf{Punti Ferita} 200 (16d12 + 96)

\textbf{Movimento} 12 m, nuoto 12 m, scavo 9 m, volo 24 m

\textbf{Tiri Salvezza} Tempra +13, Riflessi +9, Volontà +10

\textbf{Competenze} Muoversi Silenziosamente / Nascondersi nelle Ombre +5, Consapevolezza +11

\textbf{Immunità al Danno} freddo

\textbf{Sensi} scurovisione 36 m, vista cieca 18 m

\textbf{Linguaggi} Comune, Draconico

\textbf{Sfida} 13 (10.000 PE)

\emph{\textbf{Camminare sul Ghiaccio.}} Il drago può muoversi e arrampicarsi su superfici ghiacciate senza bisogno di effettuare prove di caratteristica. Inoltre, il terreno difficile composto di ghiaccio o neve non gli costa movimento aggiuntivo.

\emph{\textbf{Resistenza Leggendaria (3/Giorno).}} Se il drago fallisce un tiro salvezza, può scegliere invece di riuscire. 

\textbf{Azioni}

\emph{\textbf{Multiattacco.}} Il drago può usare la sua Presenza Spaventosa e poi effettuare tre attacchi: uno con il morso e due con gli artigli.

\emph{\textbf{Artiglio.} Attacco con arma da mischia}: +11 a colpire, portata 1 m, un bersaglio.

\emph{Colpisce:} 13 (2d6 + 6) danni taglienti.

\emph{\textbf{Coda.} Attacco con arma da mischia}: +11 a colpire, portata 4,5 m, un bersaglio.

\emph{Colpisce:} 15 (2d8 + 6) danni da botta.

\emph{\textbf{Morso.} Attacco con arma da mischia}: +11 a colpire, portata 3 m, un bersaglio.

\emph{Colpisce:} 17 (2d10 + 6) danni perforanti più 4 (1d8) danni da freddo.

\emph{\textbf{Presenza Spaventosa.}} Ogni creatura scelta dal drago, che si trovi entro 36 metri da esso e consapevole della sua presenza, deve riuscire un tiro salvezza di Volontà CD 14 o restare spaventata per 1 minuto. Una creatura può ripetere il tiro salvezza al termine di ciascun suo turno, terminando l'effetto se lo riesce. Se il tiro salvezza della creatura ha successo o l'effetto ha termine per essa, la creatura è immune alla Presenza Spaventosa del drago per le successive 24 ore.

\emph{\textbf{Soffio Gelido (Ricarica 5-6).}} Il drago esala un'esplosione di ghiaccio in un cono di 18 metri. Ogni creatura in quell'area deve effettuare un tiro salvezza di Tempra CD 19 e subire 54 (12d8) danni da freddo se fallisce il tiro salvezza, o la metà di questi danni se lo riesce. 

\textbf{Azioni Aggiuntive}

Il drago può effettuare 3 Azioni aggiuntive, scelte tra le opzioni seguenti. Può usare solo un'opzione leggendaria alla volta e solo al termine del turno di un'altra creatura. Il drago recupera le Azioni aggiuntive spese all'inizio del proprio turno.

\textbf{Attacco di Ala (Costa 2 Azioni).} Il drago batte le ali. Ogni creatura entro 3 metri dal drago deve riuscire un tiro salvezza su Riflessi CD 19 o subire 13 (2d6 + 6) danni da botta e venir gettato prono. Il drago può poi volare fino a metà del suo movimento di volo. \textbf{Attacco di Coda.} Il drago effettua un attacco di coda 
.
\textbf{Individuare.} Il drago effettua una prova di Saggezza (Consapevolezza).

\medskip\index{Mostri - Drago Bianco Giovane}\textbf{Drago Bianco Giovane}

\emph{Grande drago, caotico malvagio}

\textbf{FORZA} +4

\textbf{DESTREZZA} +0

\textbf{COSTITUZIONE} +4

\textbf{INTELLIGENZA} -2

\textbf{SAGGEZZA} +0

\textbf{CARISMA} +1

\textbf{Iniziativa} +0 -- \textbf{Difesa} 20

\textbf{Punti Ferita} 133 (14d10 + 56)

\textbf{Movimento} 12 m, nuoto 12 m, scavo 6 m, volo 24 m

\textbf{Tiri Salvezza} Tempra +8, Riflessi +7, Volontà +5

\textbf{Competenze} Muoversi Silenziosamente / Nascondersi nelle Ombre +3, Consapevolezza +6

\textbf{Immunità al Danno} freddo

\textbf{Sensi} scurovisione 36 m, vista cieca 9 m

\textbf{Linguaggi} Comune, Draconico

\textbf{Sfida} 6 (2.300 PE)

\emph{\textbf{Camminare sul Ghiaccio.}} Il drago può muoversi e arrampicarsi su superfici ghiacciate senza bisogno di effettuare prove di caratteristica. Inoltre, il terreno difficile composto di ghiaccio o neve non gli costa movimento aggiuntivo.

\textbf{Azioni}

\emph{\textbf{Multiattacco.}} Il drago può usare la sua Presenza Spaventosa. Poi effettuare tre attacchi: uno con il morso e due con gli artigli.

\emph{\textbf{Artiglio.} Attacco con arma da mischia}: +7 a colpire, portata 1 m, un bersaglio.

\emph{Colpisce:} 11 (2d6 + 4) danni taglienti.

\emph{\textbf{Morso.} Attacco con arma da mischia}: +7 a colpire, portata 3 m, un bersaglio.

\emph{Colpisce:} 15 (2d10 + 4) danni perforanti più 4 (1d8) danni da freddo.

\emph{\textbf{Soffio Gelido (Ricarica 5-6).}} Il drago esala un'esplosione di ghiaccio in un cono di 9 metri. Ogni creatura in quell'area deve effettuare un tiro salvezza di Tempra CD 15 e subire 45 (10d8) danni da freddo se fallisce il tiro salvezza, o la metà di questi danni se lo riesce.

\medskip\index{Mostri - Drago Bianco Cucciolo}\textbf{Drago Bianco Cucciolo}

\emph{Media drago, caotico malvagio}

\textbf{FORZA} +2

\textbf{DESTREZZA} +0

\textbf{COSTITUZIONE} +2

\textbf{INTELLIGENZA} -3

\textbf{SAGGEZZA} +0

\textbf{CARISMA} +0

\textbf{Iniziativa} +0 -- \textbf{Difesa} 17

\textbf{Punti Ferita} 32 (5d8 + 10)

\textbf{Movimento} 9 m, nuoto 9 m, scavo 4,5 m, volo 18 m

\textbf{Tiri Salvezza} Tempra +2, Riflessi +1, Volontà +1

\textbf{Competenze} Muoversi Silenziosamente / Nascondersi nelle Ombre +2, Consapevolezza +4

\textbf{Immunità al Danno} freddo

\textbf{Sensi} scurovisione 18 m, vista cieca 3 m

\textbf{Linguaggi} Draconico

\textbf{Sfida} 2 (450 PE)

\textbf{Azioni}

\emph{\textbf{Morso.} Attacco con arma da mischia}: +7 a colpire, portata 3 m, un bersaglio.

\emph{Colpisce:} 15 (2d10 + 4) danni perforanti più 4 (1d8) danni da freddo.

\emph{\textbf{Soffio Gelido (Ricarica 5-6).}} Il drago esala un'esplosione di ghiaccio in un cono di 4,5 metri. Ogni creatura in quell'area deve effettuare un tiro salvezza di Tempra CD 12 e subire 22 (5d8) danni da freddo se fallisce il tiro salvezza, o la metà di questi danni se lo riesce.

\medskip\index{Mostri - Drago Blu Antico}\textbf{Drago Blu Antico}

\emph{Mastodontica drago, legale malvagio}

\textbf{FORZA} +9

\textbf{DESTREZZA} +0

\textbf{COSTITUZIONE} +8

\textbf{INTELLIGENZA} +4

\textbf{SAGGEZZA} +3

\textbf{CARISMA} +5

\textbf{Iniziativa} +4 -- \textbf{Difesa} 34

\textbf{Punti Ferita} 481 (26d20 + 208)

\textbf{Movimento} 12 m, scavo 12 m, volo 24 m

\textbf{Tiri Salvezza} Tempra +21, Riflessi +13, Volontà +19

\textbf{Competenze} Muoversi Silenziosamente / Nascondersi nelle Ombre +7, Consapevolezza +17

\textbf{Immunità al Danno} fulmine, arma +1

\textbf{Sensi} scurovisione 36 m, vista cieca 18 m

\textbf{Linguaggi} Comune, Draconico

\textbf{Sfida} 23 (50.000 PE)

\emph{\textbf{Resistenza Leggendaria (3/Giorno).}} Se il drago fallisce un tiro salvezza, può scegliere invece di riuscire.

\textbf{Azioni}

\emph{\textbf{Multiattacco.}} Il drago può usare la sua Presenza Spaventosa. Poi effettuare tre attacchi: uno con il morso e due con gli artigli.

\emph{\textbf{Artiglio.} Attacco con arma da mischia}: +16 a colpire,
portata 3 m, un bersaglio.

\emph{Colpisce:} 16 (2d6 + 9) danni taglienti.

\emph{\textbf{Coda.} Attacco con arma da mischia}: +16 a colpire, portata 6 m, un bersaglio.

\emph{Colpisce:} 18 (2d8 + 9) danni da botta.

\emph{\textbf{Morso.} Attacco con arma da mischia}: +16 a colpire, portata 4,5 m, un bersaglio.

\emph{Colpisce:} 20 (2d10 + 9) danni perforanti più 11 (2d10) danni da fulmine.

\emph{\textbf{Presenza Spaventosa.}} Ogni creatura scelta dal drago, che si trovi entro 36 metri da esso e consapevole della sua presenza, deve riuscire un tiro salvezza di Volontà CD 20 o restare spaventata per 1 minuto. Una creatura può ripetere il tiro salvezza al termine di ciascun suo turno, terminando l'effetto se lo riesce. Se il tiro salvezza della creatura ha successo o l'effetto ha termine per essa, la creatura è immune alla Presenza Spaventosa del drago per le successive 24 ore.

\emph{\textbf{Soffio Fulminante (Ricarica 5-6).}} Il drago esala fulmini in una linea lunga 36 metri e larga 3 metri. Ogni creatura su quella linea deve effettuare un tiro salvezza di Riflessi CD 23 e subire 88 (16d10) danni da fulmine se fallisce il tiro salvezza, o la metà di questi danni se lo riesce.

\textbf{Azioni Aggiuntive}

Il drago può effettuare 3 Azioni aggiuntive, scelte tra le opzioni seguenti. Può usare solo un'opzione leggendaria alla volta e solo al termine del turno di un'altra creatura. Il drago recupera le Azioni aggiuntive spese all'inizio del proprio turno.

\textbf{Attacco di Ala (Costa 2 Azioni).} Il drago batte le ali. Ogni creatura entro 4,5 metri dal drago deve riuscire un tiro salvezza su Riflessi CD 24 o subire 16 (2d6 + 9) danni da botta e venir gettato prono. Il drago può poi volare fino a metà del suo movimento di volo.

\textbf{Attacco di Coda.} Il drago effettua un attacco di coda.

\textbf{Individuare.} Il drago effettua una prova di Saggezza (Consapevolezza).

\medskip\index{Mostri - Drago Blu Adulto}\textbf{Drago Blu Adulto}

\emph{Enorme drago, legale malvagio}

\textbf{FORZA} +7

\textbf{DESTREZZA} +0

\textbf{COSTITUZIONE} +6

\textbf{INTELLIGENZA} +3

\textbf{SAGGEZZA} +2

\textbf{CARISMA} +4

\textbf{Iniziativa} +3 -- \textbf{Difesa} 27

\textbf{Punti Ferita} 225 (18d12 + 108)

\textbf{Movimento} 12 m, scavo 12 m, volo 24 m

\textbf{Tiri Salvezza} Tempra +15, Riflessi +10, Volontà +13

\textbf{Competenze} Muoversi Silenziosamente / Nascondersi nelle Ombre +5, Consapevolezza +12

\textbf{Immunità al Danno} fulmine

\textbf{Sensi} scurovisione 36 m, vista cieca 18 m

\textbf{Linguaggi} Comune, Draconico

\textbf{Sfida} 16 (15.000 PE)

\emph{\textbf{Resistenza Leggendaria (3/Giorno).}} Se il drago fallisce un tiro salvezza, può scegliere invece di riuscire.

\textbf{Azioni}

\emph{\textbf{Multiattacco.}} Il drago può usare la sua Presenza Spaventosa. Poi effettuare tre attacchi: uno con il morso e due con gli artigli.

\emph{\textbf{Artiglio.} Attacco con arma da mischia}: +12 a colpire, portata 1 m, un bersaglio.

\emph{Colpisce:} 14 (2d6 + 7) danni taglienti.

\emph{\textbf{Coda.} Attacco con arma da mischia}: +12 a colpire, portata 4,5 m, un bersaglio.

\emph{Colpisce:} 16 (2d8 + 7) danni da botta.

\emph{\textbf{Morso.} Attacco con arma da mischia}: +12 a colpire, portata 3 m, un bersaglio.

\emph{Colpisce:} 18 (2d10 + 7) danni perforanti più 5 (1d10) danni da fulmine.

\emph{\textbf{Presenza Spaventosa.}} Ogni creatura scelta dal drago, che si trovi entro 36 metri da esso e consapevole della sua presenza, deve riuscire un tiro salvezza di Volontà CD 17 o restare spaventata per 1 minuto. Una creatura può ripetere il tiro salvezza al termine di ciascun suo turno, terminando l'effetto se lo riesce. Se il tiro salvezza della creatura ha successo o l'effetto ha termine per essa, la creatura è immune alla Presenza Spaventosa del drago per le successive 24 ore.

\emph{\textbf{Soffio Fulminante (Ricarica 5-6).}} Il drago esala fulmini in una linea lunga 27 metri e larga 1,5 metri. Ogni creatura su quella linea deve effettuare un tiro salvezza di Riflessi CD 19 e subire 66 (12d10) danni da fulmine se fallisce il tiro salvezza, o la metà di questi danni se lo riesce.

\textbf{Azioni Aggiuntive}

Il drago può effettuare 3 Azioni aggiuntive, scelte tra le opzioni seguenti. Può usare solo un'opzione leggendaria alla volta e solo al termine del turno di un'altra creatura. Il drago recupera le Azioni aggiuntive spese all'inizio del proprio turno.

\textbf{Attacco di Ala (Costa 2 Azioni).} Il drago batte le ali. Ogni creatura entro 3 metri dal drago deve riuscire un tiro salvezza su Riflessi CD 20 o subire 14 (2d6 + 7) danni da botta e venir gettato prono. Il drago può poi volare fino a metà della del suo movimento di  volo.

\textbf{Attacco di Coda.} Il drago effettua un attacco di coda.

\textbf{Individuare.} Il drago effettua una prova di Saggezza (Consapevolezza).

\medskip\index{Mostri - Drago Blu Giovane}\textbf{Drago Blu Giovane}

\emph{Enorme drago, legale malvagio}

\textbf{FORZA} 21(+5)

\textbf{DESTREZZA} +0

\textbf{COSTITUZIONE} +4

\textbf{INTELLIGENZA} +2

\textbf{SAGGEZZA} +1

\textbf{CARISMA} +3

\textbf{Iniziativa} +2 -- \textbf{Difesa} 23

\textbf{Punti Ferita} 152 (16d10 + 64)

\textbf{Movimento} 12 m, scavo 12 m, volo 24 m

\textbf{Tiri Salvezza} Tempra +10, Riflessi +8, Volontà +8

\textbf{Competenze} Muoversi Silenziosamente / Nascondersi nelle Ombre +4, Consapevolezza +9

\textbf{Immunità al Danno} fulmine

\textbf{Sensi} scurovisione 36 m, vista cieca 9 m 

\textbf{Linguaggi} Comune, Draconico

\textbf{Sfida} 9 (5.000 PE)

\textbf{Azioni}

\emph{\textbf{Multiattacco.}} Il drago può effettuare tre attacchi: uno con il morso e due con gli artigli.

\emph{\textbf{Artiglio.} Attacco con arma da mischia}: +9 a colpire, portata 1 m, un bersaglio.

\emph{Colpisce:} 12 (2d6 + 5) danni taglienti.

\emph{\textbf{Morso.} Attacco con arma da mischia}: +9 a colpire, portata 3 m, un bersaglio.

\emph{Colpisce:} 16 (2d10 + 5) danni perforanti più 5 (1d10) danni da fulmine.

\emph{\textbf{Soffio Fulminante (Ricarica 5-6).}} Il drago esala fulmini in una linea lunga 18 metri e larga 1,5 metri. Ogni creatura su quella linea deve effettuare un tiro salvezza di Riflessi CD 16 e subire 55 (10d10) danni da fulmine se fallisce il tiro salvezza, o la metà di questi danni se lo riesce.

\medskip\index{Mostri - Drago Blu Cucciolo}\textbf{Drago Blu Cucciolo}

\emph{Enorme drago, legale malvagio}

\textbf{FORZA} +3

\textbf{DESTREZZA} +0

\textbf{COSTITUZIONE} +2

\textbf{INTELLIGENZA} +1

\textbf{SAGGEZZA} +0

\textbf{CARISMA} +2

\textbf{Iniziativa} +1 -- \textbf{Difesa} 19

\textbf{Punti Ferita} 52 (8d8 + 16)

\textbf{Movimento} 9 m, scavo 4,5 m, volo 18 m

\textbf{Tiri Salvezza} Tempra +4, Riflessi +1, Volontà +1

\textbf{Competenze} Muoversi Silenziosamente / Nascondersi nelle Ombre +2, Consapevolezza +4

\textbf{Immunità al Danno} fulmine

\textbf{Sensi} scurovisione 18 m, vista cieca 3 m

\textbf{Linguaggi} Draconico

\textbf{Sfida} 3 (700 PE)

\textbf{Azioni}

\emph{\textbf{Morso.} Attacco con arma da mischia}: +5 a colpire, portata 1 m, un bersaglio.

\emph{Colpisce:} 8 (1d10 + 3) danni perforanti più 3 (1d6) danni da fulmine.

\emph{\textbf{Soffio Fulminante (Ricarica 5-6).}} Il drago esala fulmini in una linea lunga 9 metri e larga 1,5 metri. Ogni creatura su quella linea deve effettuare un tiro salvezza di Riflessi CD 12 e subire 22 (4d10) danni da fulmine se fallisce il tiro salvezza, o la metà di questi danni se lo riesce.

\medskip\index{Mostri - Drago Nero Antico}\textbf{Drago Nero Antico}

\emph{Mastodontica drago, caotico malvagio}

\textbf{FORZA} +8

\textbf{DESTREZZA} +2

\textbf{COSTITUZIONE} +7

\textbf{INTELLIGENZA} +3

\textbf{SAGGEZZA} +2

\textbf{CARISMA} +4

\textbf{Iniziativa} +3 -- \textbf{Difesa} 33

\textbf{Punti Ferita} 367 (21d20 + 147)

\textbf{Movimento} 12 m, scalata 12 m, volo 24 m

\textbf{Tiri Salvezza} Tempra +20, Riflessi +13, Volontà +18

\textbf{Competenze} Muoversi Silenziosamente / Nascondersi nelle Ombre +9, Consapevolezza +16

\textbf{Immunità al Danno} acido, arma +1

\textbf{Sensi} scurovisione 36 m, vista cieca 18 m

\textbf{Linguaggi} Comune, Draconico

\textbf{Sfida} 21 (33.000 PE)

\emph{\textbf{Anfibio.}} Il drago può respirare aria e acqua.

\emph{\textbf{Resistenza Leggendaria (3/Giorno).}} Se il drago fallisce un tiro salvezza, può scegliere invece di riuscire.

\textbf{Azioni}

\emph{\textbf{Multiattacco.}} Il drago può usare la sua Presenza Spaventosa. Poi effettuare tre attacchi: uno con il morso e due con gli artigli.

\emph{\textbf{Artiglio.} Attacco con arma da mischia}: +15 a colpire, portata 3 m, un bersaglio.

\emph{Colpisce:} 15 (2d6 + 8) danni taglienti.

\emph{\textbf{Coda.} Attacco con arma da mischia}: +15 a colpire, portata 6 m, un bersaglio.

\emph{Colpisce:} 17 (2d8 + 8) danni da botta.

\emph{\textbf{Morso.} Attacco con arma da mischia} : +15 a colpire, portata 4,5 m, un bersaglio.

\emph{Colpisce:} 19 (2d10 + 8) danni perforanti più 9 (4d6) danni da acido.

\emph{\textbf{Presenza Spaventosa.}} Ogni creatura scelta dal drago, che si trovi entro 36 metri da esso e consapevole della sua presenza, deve riuscire un tiro salvezza di Volontà CD 19 o restare spaventata per 1 minuto. Una creatura può ripetere il tiro salvezza al termine di ciascun suo turno, terminando l'effetto se lo riesce. Se il tiro salvezza della creatura ha successo o l'effetto ha termine per essa, la creatura è immune alla Presenza Spaventosa del drago per le successive 24 ore.

\emph{\textbf{Soffio Acido (Ricarica 5-6).}} Il drago esala acido in una linea di 27 metri larga 3 metri. Ogni creatura in quell'area deve effettuare un tiro salvezza di Riflessi CD 22 e subire 67 (15d8) danni da acido se fallisce il tiro salvezza, o la metà di questi danni se lo riesce.

\textbf{Azioni Aggiuntive}

Il drago può effettuare 3 Azioni aggiuntive, scelte tra le opzioni seguenti. Può usare solo un'opzione leggendaria alla volta e solo al termine del turno di un'altra creatura. Il drago recupera le Azioni aggiuntive spese all'inizio del proprio turno.

\textbf{Attacco di Ala (Costa 2 Azioni).} Il drago batte le ali. Ogni creatura entro 4,5 metri dal drago deve riuscire un tiro salvezza su Riflessi CD 23 o subire 15 (2d6 + 8) danni da botta e venir gettato prono. Il drago può poi volare fino a metà del suo movimento di volo.  

\textbf{Attacco di Coda.} Il drago effettua un attacco di coda.

\textbf{Individuare.} Il drago effettua una prova di Saggezza (Consapevolezza).

\medskip\index{Mostri - Drago Nero Adulto}\textbf{Drago Nero Adulto}

\emph{Enorme drago, caotico malvagio}

\textbf{FORZA} +6

\textbf{DESTREZZA} +2

\textbf{COSTITUZIONE} +5

\textbf{INTELLIGENZA} +2

\textbf{SAGGEZZA} +1

\textbf{CARISMA} +3

\textbf{Iniziativa} +2 -- \textbf{Difesa} 28

\textbf{Punti Ferita} 195 (17d12 + 85)

\textbf{Movimento} 12 m, scalata 12 m, volo 24 m

\textbf{Tiri Salvezza} Tempra +14, Riflessi +10, Volontà +12

\textbf{Competenze} Muoversi Silenziosamente / Nascondersi nelle Ombre +7, Consapevolezza +11

\textbf{Immunità al Danno} acido

\textbf{Sensi} scurovisione 36 m, vista cieca 18 m 

\textbf{Linguaggi} Comune, Draconico

\textbf{Sfida} 17 (18.000 PE)

\emph{\textbf{Anfibio.}} Il drago può respirare aria e acqua.

\emph{\textbf{Resistenza Leggendaria (3/Giorno).}} Se il drago fallisce un tiro salvezza, può scegliere invece di riuscire.

\textbf{Azioni}

\emph{\textbf{Multiattacco.}} Il drago può usare la sua Presenza Spaventosa. Poi effettuare tre attacchi: uno con il morso e due con gli artigli.

\emph{\textbf{Artiglio.} Attacco con arma da mischia}: +11 a colpire, portata 1 m, un bersaglio.

\emph{Colpisce:} 13 (2d6 + 6) danni taglienti.

\emph{\textbf{Coda.} Attacco con arma da mischia}: +11 a colpire, portata 4,5 m, un bersaglio.

\emph{Colpisce:} 15 (2d8 + 6) danni da botta.

\emph{\textbf{Morso.} Attacco con arma da mischia}: +11 a colpire, portata 3 m, un bersaglio.

\emph{Colpisce:} 17 (2d10 + 6) danni perforanti più 4 (1d8) danni da acido.

\emph{\textbf{Presenza Spaventosa.}} Ogni creatura scelta dal drago, che si trovi entro 36 metri da esso e consapevole della sua presenza, deve riuscire un tiro salvezza di Volontà CD 16 o restare spaventata per 1 minuto. Una creatura può ripetere il tiro salvezza al termine di ciascun suo turno, terminando l'effetto se lo riesce. Se il tiro salvezza della creatura ha successo o l'effetto ha termine per essa, la creatura è immune alla Presenza Spaventosa del drago per le successive 24 ore.

\emph{\textbf{Soffio Acido (Ricarica 5-6).}} Il drago esala acido in una linea di 18 metri larga 1,5 metri. Ogni creatura in quell'area deve effettuare un tiro salvezza di Riflessi CD 18 e subire 54 (12d8) danni da acido se fallisce il tiro salvezza, o la metà di questi danni se lo
riesce.

\textbf{Azioni Aggiuntive}

Il drago può effettuare 3 Azioni aggiuntive, scelte tra le opzioni seguenti. Può usare solo un'opzione leggendaria alla volta e solo al termine del turno di un'altra creatura. Il drago recupera le Azioni aggiuntive spese all'inizio del proprio turno.

\textbf{Attacco di Ala (Costa 2 Azioni).} Il drago batte le ali. Ogni creatura entro 3 metri dal drago deve riuscire un tiro salvezza su Riflessi CD 19 o subire 13 (2d6 + 6) danni da botta e venir gettato prono. Il drago può poi volare fino a metà della del suo movimento di volo.

\textbf{Attacco di Coda.} Il drago effettua un attacco di coda.

\textbf{Individuare.} Il drago effettua una prova di Saggezza (Consapevolezza).

\medskip\index{Mostri - Drago Nero Giovane}\textbf{Drago Nero Giovane}

\emph{Grande drago, caotico malvagio}

\textbf{FORZA} +4

\textbf{DESTREZZA} +2

\textbf{COSTITUZIONE} +3

\textbf{INTELLIGENZA} +1

\textbf{SAGGEZZA} +0

\textbf{CARISMA} +2

\textbf{Iniziativa} +2 -- \textbf{Difesa} 22

\textbf{Punti Ferita} 127 (15d10 + 45)

\textbf{Movimento} 12 m, scalata 12 m, volo 24 m

\textbf{Tiri Salvezza} Tempra +9, Riflessi +8, Volontà +7

\textbf{Competenze} Muoversi Silenziosamente / Nascondersi nelle Ombre +5, Consapevolezza +6

\textbf{Immunità al Danno} acido

\textbf{Sensi} scurovisione 36 m, vista cieca 9 m

\textbf{Linguaggi} Comune, Draconico

\textbf{Sfida} 7 (2.900 PE)

\emph{\textbf{Anfibio.}} Il drago può respirare aria e acqua.

\textbf{Azioni}

\emph{\textbf{Multiattacco.}} Il drago può effettuare tre attacchi: uno con il morso e due con gli artigli.

\emph{\textbf{Artiglio.} Attacco con arma da mischia}: +10 a colpire, portata 1 m, un bersaglio.

\emph{Colpisce:} 11 (2d6 + 4) danni taglienti.

\emph{\textbf{Morso.} Attacco con arma da mischia}: +7 a colpire, portata 3 m, un bersaglio.

\emph{Colpisce:} 11 (2d10 + 4) danni perforanti più 4 (1d8) danni da acido.

\emph{\textbf{Soffio Acido (Ricarica 5-6).}} Il drago esala acido in una linea di 9 metri larga 1,5 metri. Ogni creatura in quell'area deve effettuare un tiro salvezza di Riflessi CD 14 e subire 49 (11d8) danni da acido se fallisce il tiro salvezza, o la metà di questi danni se lo riesce.

\medskip\index{Mostri - Drago Nero Cucciolo}\textbf{Drago Nero Cucciolo}

\emph{Media drago, caotico malvagio}

\textbf{FORZA} +2

\textbf{DESTREZZA} +2

\textbf{COSTITUZIONE} +1

\textbf{INTELLIGENZA} +0

\textbf{SAGGEZZA} +0

\textbf{CARISMA} +1

\textbf{Iniziativa} +2 -- \textbf{Difesa} 18

\textbf{Punti Ferita} 33 (6d8 + 6)

\textbf{Movimento} 9 m, scalata 9 m, volo 18 m

\textbf{Tiri Salvezza} Tempra +2, Riflessi +2, Volontà +0

\textbf{Competenze} Muoversi Silenziosamente / Nascondersi nelle Ombre +4, Consapevolezza +4

\textbf{Immunità al Danno} acido

\textbf{Sensi} scurovisione 18 m, vista cieca 3 m

\textbf{Linguaggi} Draconico

\textbf{Sfida} 2 (450 PE)

\emph{\textbf{Anfibio.}} Il drago può respirare aria e acqua.

\textbf{Azioni}

\emph{\textbf{Morso.} Attacco con arma da mischia}: +4 a colpire, portata 1 m, un bersaglio.

\emph{Colpisce:} 7 (1d10 + 2) danni perforanti più 2 (1d4) danni da acido.

\emph{\textbf{Soffio Acido (Ricarica 5-6).}} Il drago esala acido in una linea di 4,5 metri larga 1,5 metri. Ogni creatura in quell'area deve effettuare un tiro salvezza di Riflessi CD 11 e subire 22 (5d8) danni da acido se fallisce il tiro salvezza, o la metà di questi danni se lo riesce.

\medskip\index{Mostri - Drago Rosso Antico}\textbf{Drago Rosso Antico}

\emph{Mastodontica drago, caotico malvagio}

\textbf{FORZA} +10

\textbf{DESTREZZA} +0

\textbf{COSTITUZIONE} +9

\textbf{INTELLIGENZA} +4

\textbf{SAGGEZZA} +2

\textbf{CARISMA} +6

\textbf{Iniziativa} +4 -- \textbf{Difesa} 34

\textbf{Punti Ferita} 546 (28d20 + 252)

\textbf{Movimento} 12 m, scalata 12 m, volo 24 m

\textbf{Tiri Salvezza} Tempra +22, Riflessi +13, Volontà +21

\textbf{Competenze} Muoversi Silenziosamente / Nascondersi nelle Ombre +7, Consapevolezza +16

\textbf{Immunità al Danno} fuoco, arma +1

\textbf{Sensi} scurovisione 36 m, vista cieca 18 m

\textbf{Linguaggi} Comune, Draconico

\textbf{Sfida} 24 (62.000 PE)

\emph{\textbf{Resistenza Leggendaria (3/Giorno).}} Se il drago fallisce un tiro salvezza, può scegliere invece di riuscire.

\textbf{Azioni}

\emph{\textbf{Multiattacco.}} Il drago può usare la sua Presenza Spaventosa e poi effettuare tre attacchi: uno con il morso e due con gli artigli.

\emph{\textbf{Artiglio.} Attacco con arma da mischia}: +17 a colpire, portata 3 m, un bersaglio.

\emph{Colpisce:} 17 (2d6 + 10) danni taglienti.

\emph{\textbf{Coda.} Attacco con arma da mischia}: +17 a colpire, portata 6 m, un bersaglio.

\emph{Colpisce:} 19 (2d8 + 10) danni da botta.

\emph{\textbf{Morso.} Attacco con arma da mischia}: +17 a colpire, portata 4,5 m, un bersaglio.

\emph{Colpisce:} 21 (2d10 + 10) danni perforanti più 14 (4d6) danni da fuoco.

\emph{\textbf{Presenza Spaventosa.}} Ogni creatura scelta dal drago, che si trovi entro 36 metri da esso e consapevole della sua presenza, deve riuscire un tiro salvezza di Volontà CD 21 o restare spaventata per 1 minuto. Una creatura può ripetere il tiro salvezza al termine di ciascun suo turno, terminando l'effetto se lo riesce. Se il tiro salvezza della creatura ha successo o l'effetto ha termine per essa, la creatura è immune alla Presenza Spaventosa del drago per le successive 24 ore.

\emph{\textbf{Soffio Infuocato (Ricarica 5-6).}} Il drago esala fuoco in un cono di 27 metri. Ogni creatura in quell'area deve effettuare un tiro salvezza su Riflessi CD 24 e subire 91 (26d6) danni da fuoco se fallisce il tiro salvezza, o la metà di questi danni se lo riesce.

\textbf{Azioni Aggiuntive}

Il drago può effettuare 3 Azioni aggiuntive, scelte tra le opzioni seguenti. Può usare solo un'opzione leggendaria alla volta e solo al termine del turno di un'altra creatura. Il drago recupera le Azioni aggiuntive spese all'inizio del proprio turno.

\textbf{Attacco di Ala (Costa 2 Azioni).} Il drago batte le ali. Ogni creatura entro 4,5 metri dal drago deve riuscire un tiro salvezza su Riflessi CD 25 o subire 17 (2d6 + 10) danni da botta e venir gettato prono. Il drago può poi volare fino a metà del suo movimento di volo.

\textbf{Attacco di Coda.} Il drago effettua un attacco di coda.

\textbf{Individuare.} Il drago effettua una prova di Saggezza (Consapevolezza).

\medskip\index{Mostri - Drago Rosso Adulto}\textbf{Drago Rosso Adulto}

\emph{Enorme drago, caotico malvagio}

\textbf{FORZA} +8

\textbf{DESTREZZA} +0

\textbf{COSTITUZIONE} +7

\textbf{INTELLIGENZA} +3

\textbf{SAGGEZZA} +1

\textbf{CARISMA} +5

\textbf{Iniziativa} +3 -- \textbf{Difesa} 28

\textbf{Punti Ferita} 256 (19d12 + 133) 

\textbf{Movimento} 12 m, scalata 12 m, volo 24 m

\textbf{Tiri Salvezza} Tempra +16, Riflessi +10, Volontà +15

\textbf{Competenze} Muoversi Silenziosamente / Nascondersi nelle Ombre +6, Consapevolezza +13

\textbf{Immunità al Danno} fuoco

\textbf{Sensi} scurovisione 36 m, vista cieca 18 m 

\textbf{Linguaggi} Comune, Draconico

\textbf{Sfida} 17 (18.000 PE)

\emph{\textbf{Resistenza Leggendaria (3/Giorno).}} Se il drago fallisce un tiro salvezza, può scegliere invece di riuscire.

\textbf{Azioni}

\emph{\textbf{Multiattacco.}} Il drago può usare la sua Presenza Spaventosa e poi effettuare tre attacchi: uno con il morso e due con gli artigli.

\emph{\textbf{Artiglio.} Attacco con arma da mischia}: +14 a colpire, portata 1 m, un bersaglio.

\emph{Colpisce:} 15 (2d6 + 8) danni taglienti.

\emph{\textbf{Coda.} Attacco con arma da mischia}: +14 a colpire, portata 4,5 m, un bersaglio.

\emph{Colpisce:} 17 (2d8 + 8) danni da botta.

\emph{\textbf{Morso.} Attacco con arma da mischia}: +14 a colpire, portata 3 m, un bersaglio.

\emph{Colpisce:} 19 (2d10 + 8) danni perforanti più 7 (2d6) danni da
fuoco.

\emph{\textbf{Presenza Spaventosa.}} Ogni creatura scelta dal drago, che si trovi entro 36 metri da esso e consapevole della sua presenza, deve riuscire un tiro salvezza di Volontà CD 19 o restare spaventata per 1 minuto. Una creatura può ripetere il tiro salvezza al termine di ciascun suo turno, terminando l'effetto se lo riesce. Se il tiro salvezza della creatura ha successo o l'effetto ha termine per essa, la creatura è  immune alla Presenza Spaventosa del drago per le successive 24 ore.

\emph{\textbf{Soffio Infuocato (Ricarica 5-6).}} Il drago esala fuoco in un cono di 18 metri. Ogni creatura in quell'area deve effettuare un tiro salvezza su Riflessi CD 21 e subire 63 (18d6) danni da fuoco se fallisce il tiro salvezza, o la metà di questi danni se lo riesce.

\textbf{Azioni Aggiuntive}

Il drago può effettuare 3 Azioni aggiuntive, scelte tra le opzioni seguenti. Può usare solo un'opzione leggendaria alla volta e solo al termine del turno di un'altra creatura. Il drago recupera le Azioni aggiuntive spese all'inizio del proprio turno.

\textbf{Attacco di Ala (Costa 2 Azioni).} Il drago batte le ali. Ogni creatura entro 3 metri dal drago deve riuscire un tiro salvezza su Riflessi CD 22 o subire 15 (2d6 + 8) danni da botta e venir gettato prono. Il drago può poi volare fino a metà del suo movimento di volo. 

\textbf{Attacco di Coda.} Il drago effettua un attacco di coda. 

\textbf{Individuare.} Il drago effettua una prova di Saggezza (Consapevolezza).


\medskip\index{Mostri - Drago Rosso Giovane}\textbf{Drago Rosso Giovane}

\emph{Grande drago, caotico malvagio}

\textbf{FORZA} +6

\textbf{DESTREZZA} +0

\textbf{COSTITUZIONE} +5

\textbf{INTELLIGENZA} +2

\textbf{SAGGEZZA} +0

\textbf{CARISMA} +4

\textbf{Iniziativa} +2 -- \textbf{Difesa} 23

\textbf{Punti Ferita} 178 (17d10 + 85)

\textbf{Movimento} 12 m, scalata 12 m, volo 24 m

\textbf{Tiri Salvezza} Tempra +11, Riflessi +8, Volontà +10

\textbf{Competenze} Muoversi Silenziosamente / Nascondersi nelle Ombre +4, Consapevolezza +8

\textbf{Immunità al Danno} fuoco

\textbf{Sensi} scurovisione 36 m, vista cieca 9 m

\textbf{Linguaggi} Comune, Draconico

\textbf{Sfida} 10 (5.900 PE)

\textbf{Azioni}

\emph{\textbf{Multiattacco.}} Il drago può effettuare tre attacchi: uno con il morso e due con gli artigli.

\emph{\textbf{Artiglio.} Attacco con arma da mischia}: +10 a colpire, portata 1 m, un bersaglio.

\emph{Colpisce:} 13 (2d6 + 6) danni taglienti.

\emph{\textbf{Morso.} Attacco con arma da mischia}: +10 a colpire, portata 3 m, un bersaglio.

\emph{Colpisce:} 17 (2d10 + 6) danni perforanti più 3 (1d6) danni da fuoco.

\emph{\textbf{Soffio Infuocato (Ricarica 5-6).}} Il drago esala fuoco in un cono di 9 metri. Ogni creatura in quell'area deve effettuare un tiro salvezza su Riflessi CD 17 e subire 56 (16d6) danni da fuoco se fallisce il tiro salvezza, o la metà di questi danni se lo riesce.

\medskip\index{Mostri - Drago Rosso Cucciolo}\textbf{Drago Rosso Cucciolo}

\emph{Media drago, caotico malvagio}

\textbf{FORZA} +4

\textbf{DESTREZZA} +0

\textbf{COSTITUZIONE} +3

\textbf{INTELLIGENZA} +1

\textbf{SAGGEZZA} +0

\textbf{CARISMA} +2

\textbf{Iniziativa} +1 -- \textbf{Difesa} 19

\textbf{Punti Ferita} 75 (10d8 + 30)

\textbf{Movimento} 9 m, scalata 9 m, volo 18 m

\textbf{Tiri Salvezza} Tempra +4, Riflessi +3, Volontà +1

\textbf{Competenze} Muoversi Silenziosamente / Nascondersi nelle Ombre +2, Consapevolezza +4

\textbf{Immunità al Danno} fuoco

\textbf{Sensi} scurovisione 18 m, vista cieca 3 m

\textbf{Linguaggi} Draconico

\textbf{Sfida} 4 (1.100 PE)

\textbf{Azioni}

\emph{\textbf{Morso.} Attacco con arma da mischia}: +6 a colpire, portata 1 m, un bersaglio.

\emph{Colpisce:} 9 (1d10 + 4) danni perforanti più 3 (1d6) danni da fuoco.

\emph{\textbf{Soffio Infuocato (Ricarica 5-6).}} Il drago esala fuoco in un cono di 4,5 metri. Ogni creatura in quell'area deve effettuare un tiro salvezza di Riflessi CD 13 e subire 24 (7d6) danni da fuoco se fallisce il tiro salvezza, o la metà di questi danni se lo riesce.

\medskip\index{Mostri - Drago Verde Antico}\textbf{Drago Verde Antico}

\emph{Mastodontica drago, legale malvagio}

\textbf{FORZA} +8

\textbf{DESTREZZA} +1

\textbf{COSTITUZIONE} +7

\textbf{INTELLIGENZA} +5

\textbf{SAGGEZZA} +3

\textbf{CARISMA} +4

\textbf{Iniziativa} +5 -- \textbf{Difesa} 32

\textbf{Punti Ferita} 385 (22d20 + 154) 

\textbf{Movimento} 12 m, nuoto 12 m, volo 24 m

\textbf{Tiri Salvezza} Tempra +20, Riflessi +12, Volontà +20

\textbf{Competenze} Muoversi Silenziosamente / Nascondersi nelle Ombre +8, Ingannare +11, Percepire Emozioni +10, Consapevolezza + 15

\textbf{Immunità al Danno} veleno , arma +1

\textbf{Immunità alle Condizioni}
avvelenato

\textbf{Sensi} scurovisione 36 m, vista cieca 18 m

\textbf{Linguaggi} Comune, Draconico 

\textbf{Sfida} 22 (41.000 PE)

\emph{\textbf{Anfibio.}} Il drago può respirare aria e acqua.

\emph{\textbf{Resistenza Leggendaria (3/Giorno).}} Se il drago fallisce un tiro salvezza, può scegliere invece di riuscire.

\textbf{Azioni}

\emph{\textbf{Multiattacco.}} Il drago può usare la sua Presenza Spaventosa. Poi effettuare tre attacchi: uno con il morso e due con gli artigli.

\emph{\textbf{Artiglio.} Attacco con arma da mischia}: +15 a colpire, portata 3 m, un bersaglio.

\emph{Colpisce:} 15 (2d6 + 8) danni taglienti.

\emph{\textbf{Coda.} Attacco con arma da mischia}: +15 a colpire, portata 6 m, un bersaglio.

\emph{Colpisce:} 17 (2d8 + 8) danni da botta.

\emph{\textbf{Morso.} Attacco con arma da mischia}: +15 a colpire, portata 4,5 m, un bersaglio.

\emph{Colpisce:} 19 (2d10 + 8) danni perforanti più 10 (3d6) danni da veleno.

\emph{\textbf{Presenza Spaventosa.}} Ogni creatura scelta dal drago, che si trovi entro 36 metri da esso e consapevole della sua presenza, deve riuscire un tiro salvezza di Volontà CD 19 o restare spaventata per 1 minuto. Una creatura può ripetere il tiro salvezza al termine di ciascun suo turno, terminando l'effetto se lo riesce. Se il tiro salvezza della creatura ha successo o l'effetto ha termine per essa, la creatura è immune alla Presenza Spaventosa del drago per le successive 24 ore.

\emph{\textbf{Soffio Velenoso (Ricarica 5-6).}} Il drago esala gas velenosi in un cono di 27 metri. Ogni creatura in quell'area deve effettuare un tiro salvezza di Tempra CD 22 e subire 77 (22d6) danni da veleno se fallisce il tiro salvezza, o la metà di questi danni se lo riesce.

\textbf{Azioni Aggiuntive}

Il drago può effettuare 3 Azioni aggiuntive, scelte tra le opzioni seguenti. Può usare solo un'opzione leggendaria alla volta e solo al termine del turno di un'altra creatura. Il drago recupera le Azioni aggiuntive spese all'inizio del proprio turno.

\textbf{Attacco di Ala (Costa 2 Azioni).} Il drago batte le ali. Ogni creatura entro 4,5 metri dal drago deve riuscire un tiro salvezza su Riflessi CD 23 o subire 15 (2d6 + 8) danni da botta e venire gettato prono. Il drago può poi volare fino a metà del suo movimento di volo.

\textbf{Attacco di Coda.} Il drago effettua un attacco di coda.

\textbf{Individuare.} Il drago effettua una prova di Saggezza (Consapevolezza).

\medskip\index{Mostri - Drago Verde Adulto}\textbf{Drago Verde Adulto}

\emph{Enorme drago, legale malvagio}

\textbf{FORZA} +6

\textbf{DESTREZZA} +1

\textbf{COSTITUZIONE} +5

\textbf{INTELLIGENZA} +4

\textbf{SAGGEZZA} +2

\textbf{CARISMA} +3

\textbf{Iniziativa} +4 -- \textbf{Difesa} 27

\textbf{Punti Ferita} 207 (18d12 + 90)

\textbf{Movimento} 12 m, nuoto 12 m, volo 24 m

\textbf{Tiri Salvezza} Tempra +14, Riflessi +9, Volontà +14

\textbf{Competenze} Muoversi Silenziosamente / Nascondersi nelle Ombre +6, Ingannare +8, Percepire Emozioni +7, Consapevolezza +12

\textbf{Immunità al Danno} veleno

\textbf{Immunità alle Condizioni} avvelenato

\textbf{Sensi} scurovisione 36 m, vista cieca 18 m

\textbf{Linguaggi} Comune, Draconico

\textbf{Sfida} 15 (13.000 PE)

\emph{\textbf{Anfibio.}} Il drago può respirare aria e acqua.

\emph{\textbf{Resistenza Leggendaria (3/Giorno).}} Se il drago fallisce un tiro salvezza, può scegliere invece di riuscire.

\textbf{Azioni}

\emph{\textbf{Multiattacco.}} Il drago può usare la sua Presenza Spaventosa. Poi effettuare tre attacchi: uno con il morso e due con gli artigli.

\emph{\textbf{Artiglio.} Attacco con arma da mischia}: +11 a colpire, portata 1 m, un bersaglio.

\emph{Colpisce:} 13 (2d6 + 6) danni taglienti.

\emph{\textbf{Coda.} Attacco con arma da mischia}: +11 a colpire, portata 4,5 m, un bersaglio.

\emph{Colpisce:} 15 (2d8 + 6) danni da botta.

\emph{\textbf{Morso.} Attacco con arma da mischia}: +11 a colpire, portata 3 m, un bersaglio.

\emph{Colpisce:} 17 (2d10 + 6) danni perforanti più 7 (2d6) danni da veleno.

\emph{\textbf{Presenza Spaventosa.}} Ogni creatura scelta dal drago, che si trovi entro 36 metri da esso e consapevole della sua presenza, deve riuscire un tiro salvezza di Volontà CD 16 o restare spaventata per 1 minuto. Una creatura può ripetere il tiro salvezza al termine di ciascun suo turno, terminando l'effetto se lo riesce. Se il tiro salvezza della creatura ha successo o l'effetto ha termine per essa, la creatura è immune alla Presenza Spaventosa del drago per le successive 24 ore.

\emph{\textbf{Soffio Velenoso (Ricarica 5-6).}} Il drago esala gas velenosi in un cono di 18 metri. Ogni creatura in quell'area deve effettuare un tiro salvezza di Tempra CD 18 e subire 56 (16d6) danni da veleno se fallisce il tiro salvezza, o la metà di questi danni se lo riesce.

\textbf{Azioni Aggiuntive}

Il drago può effettuare 3 Azioni aggiuntive, scelte tra le opzioni seguenti. Può usare solo un'opzione leggendaria alla volta e solo al termine del turno di un'altra creatura. Il drago recupera le Azioni aggiuntive spese all'inizio del proprio turno.

\textbf{Attacco di Ala (Costa 2 Azioni).} Il drago batte le ali. Ogni creatura entro 3 metri dal drago deve riuscire un tiro salvezza su Riflessi CD 19 o subire 13 (2d6 + 6) danni da botta e venir  gettato prono. Il drago può poi volare fino a metà del suo movimento di volo.

\textbf{Attacco di Coda.} Il drago effettua un attacco di coda.

\textbf{Individuare.} Il drago effettua una prova di Saggezza (Consapevolezza).

\medskip\index{Mostri - Drago Verde Giovane}\textbf{Drago Verde Giovane}

\emph{Grande drago, legale malvagio}

\textbf{FORZA} +4

\textbf{DESTREZZA} +1

\textbf{COSTITUZIONE} +3

\textbf{INTELLIGENZA} +3

\textbf{SAGGEZZA} +1

\textbf{CARISMA} +2

\textbf{Iniziativa} +3 -- \textbf{Difesa} 22

\textbf{Punti Ferita} 136 (16d10 + 48)

\textbf{Movimento} 12 m, nuoto 12 m, volo 24 m

\textbf{Tiri Salvezza} Tempra +9, Riflessi +7, Volontà +9

\textbf{Competenze} Muoversi Silenziosamente / Nascondersi nelle Ombre +4, Ingannare +5, Consapevolezza +7

\textbf{Immunità al Danno} veleno

\textbf{Immunità alle Condizioni} avvelenato

\textbf{Sensi} scurovisione 36 m, vista cieca 9 m
\textbf{Linguaggi} Comune, Draconico

\textbf{Sfida} 8 (3.900 PE)

\emph{\textbf{Anfibio.}} Il drago può respirare aria e acqua.

\textbf{Azioni}

\emph{\textbf{Multiattacco.}} Il drago può effettuare tre attacchi: uno con il morso e due con gli artigli. 

\emph{\textbf{Artiglio.} Attacco con arma da mischia}: +7 a colpire, portata 1 m, un bersaglio.

\emph{Colpisce:} 11 (2d6 + 4) danni taglienti.

\emph{\textbf{Morso.} Attacco con arma da mischia}: +7 a colpire, portata 3 m, un bersaglio.

\emph{Colpisce:} 15 (2d10 + 4) danni perforanti più 7 (2d6) danni da veleno.

\emph{\textbf{Soffio Velenoso (Ricarica 5-6).}} Il drago esala gas velenosi in un cono di 9 metri. Ogni creatura in quell'area deve effettuare un tiro salvezza di Tempra CD 14 e subire 42 (12d6) danni da veleno se fallisce il tiro salvezza, o la metà di questi danni se lo riesce.



\medskip\index{Mostri - Drago Verde Cucciolo}\textbf{Drago Verde Cucciolo}

\emph{Media drago, legale malvagio}

\textbf{FORZA} +2

\textbf{DESTREZZA} +1

\textbf{COSTITUZIONE} +1

\textbf{INTELLIGENZA} +2

\textbf{SAGGEZZA} +0

\textbf{CARISMA} +1

\textbf{Iniziativa} +2 -- \textbf{Difesa} 18

\textbf{Punti Ferita} 38 (7d8 + 7)

\textbf{Movimento} 9 m, nuoto 9 m, volo 18 m

\textbf{Tiri Salvezza} Tempra +3, Riflessi +1, Volontà +0

\textbf{Competenze} Muoversi Silenziosamente / Nascondersi nelle Ombre +3, Consapevolezza +4 

\textbf{Immunità al Danno} veleno 

\textbf{Immunità alle Condizioni} avvelenato

\textbf{Sensi} scurovisione 18 m, vista cieca 3 m

\textbf{Linguaggi} Draconico

\textbf{Sfida} 2 (450 PE)

\emph{\textbf{Anfibio.}} Il drago può respirare aria e acqua.

\textbf{Azioni}

\emph{\textbf{Morso.} Attacco con arma da mischia}: +4 a colpire, portata 1 m, un bersaglio.

\emph{Colpisce:} 7 (1d10 + 2) danni perforanti più 3 (1d6) danni da veleno.

\emph{\textbf{Soffio Velenoso (Ricarica 5-6).}} Il drago esala gas velenosi in un cono di 4,5 metri. Ogni creatura in quell'area deve effettuare un tiro salvezza di Tempra CD 11 e subire 21 (6d6) danni da veleno se fallisce il tiro salvezza, o la metà di questi danni se lo riesce.

\subsection{Draghi Metallici}

\medskip\index{Mostri - Drago d'Argento Antico}\textbf{Drago d'Argento Antico}

\emph{Mastodontica drago, legale buono}

\textbf{FORZA} +10

\textbf{DESTREZZA} +0

\textbf{COSTITUZIONE} +9

\textbf{INTELLIGENZA} +4

\textbf{SAGGEZZA} +2

\textbf{CARISMA} +6

\textbf{Iniziativa} +4 -- \textbf{Difesa} 34

\textbf{Punti Ferita} 487 (25d20 + 225)

\textbf{Movimento} 12 m, volo 24 m

\textbf{Tiri Salvezza} Tempra +21, Riflessi +15, Volontà +23

\textbf{Competenze} Arcano +11, Muoversi Silenziosamente / Nascondersi nelle Ombre +7, Consapevolezza +16, Storia +11

\textbf{Immunità al Danno} freddo, arma +1

\textbf{Sensi} scurovisione 36 m, vista cieca 18 m

\textbf{Linguaggi} Comune, Draconico

\textbf{Sfida} 23 (50.000 PE)

\emph{\textbf{Resistenza Leggendaria (3/Giorno).}} Se il drago fallisce un tiro salvezza, può scegliere invece di riuscire.

\textbf{Azioni}

\emph{\textbf{Multiattacco.}} Il drago può usare la sua Presenza Spaventosa. Poi effettuare tre attacchi: uno con il morso e due con gli
artigli.

\emph{\textbf{Artiglio.} Attacco con arma da mischia}: +17 a colpire, portata 3 m, un bersaglio.

\emph{Colpisce:} 17 (2d6 + 10) danni taglienti.

\emph{\textbf{Coda.} Attacco con arma da mischia}: +17 a colpire, portata 6 m, un bersaglio.

\emph{Colpisce:} 19 (2d8 + 10) danni da botta.

\emph{\textbf{Morso.} Attacco con arma da mischia}: +17 a colpire, portata 4,5 m, un bersaglio.

\emph{Colpisce:} 21 (2d10 + 10) danni perforanti.

\emph{\textbf{Presenza Spaventosa.}} Ogni creatura scelta dal drago, che si trovi entro 36 metri da esso e consapevole della sua presenza, deve riuscire un tiro salvezza di Volontà CD 21 o restare spaventata per 1 minuto. Una creatura può ripetere il tiro salvezza al termine di ciascun suo turno, terminando l'effetto se lo riesce. Se il tiro salvezza della creatura ha successo o l'effetto ha termine per essa, la creatura è immune alla Presenza Spaventosa del drago per le successive 24 ore.

\emph{\textbf{Arma a Soffio (Ricarica 5-6).}} Il drago usa una delle seguenti armi a soffio:

\emph{Soffio Gelido.} Il drago esala un'esplosione ghiacciata in un cono di 27 metri. Ogni creatura nell'area deve effettuare un tiro salvezza su Tempra CD 24, subendo 67 (15d8) danni da freddo se fallisce il tiro salvezza, o la metà di questi danni se lo riesce.

\emph{Soffio Paralizzante.} Il drago esala un gas paralizzante in un cono di 24 metri. Ogni creatura nell'area deve riuscire un tiro salvezza su Tempra 24 o restare paralizzata per 1 minuto. Una creatura può ripetere il tiro salvezza al termine di ciascun suo turno, terminando l'effetto per sé in caso di successo.

\emph{\textbf{Mutare Forma.}} Il drago può trasformarsi magicamente in un umanoide o bestia il cui grado di sfida sia pari o inferiore al proprio, o tornare alla sua vera forma. Alla morte ritorna alla sua vera forma. 

Qualsiasi equipaggiamento stia indossando o trasportando viene assorbito o trasportato nella nuova forma (a scelta del drago).

Nella nuova forma, il drago mantiene il suo allineamento, punti ferita, Dadi Vita, la facoltà di parlare, le competenze, la Resistenza Leggendaria, le azioni da tana, e i punteggi di Intelligenza, Saggezza  e Carisma, oltre a questa azione. Le sue statistiche e capacità vengono altrimenti rimpiazzate da quelle della nuova forma, eccetto Azioni aggiuntive della nuova forma.

\textbf{Azioni Aggiuntive}

Il drago può effettuare 3 Azioni aggiuntive, scelte tra le opzioni seguenti. Può usare solo un'opzione leggendaria alla volta e solo al termine del turno di un'altra creatura. Il drago recupera le  Azioni aggiuntive spese all'inizio del proprio turno.

\textbf{Attacco di Ala (Costa 2 Azioni).} Il drago batte le ali. Ogni  creatura entro 4,5 metri dal drago deve riuscire un tiro salvezza  su Riflessi CD 25 o subire 17 (2d6 + 10) danni da botta e  venir gettato prono. Il drago può poi volare fino a metà della sua  velocità di volo.

\textbf{Attacco di Coda.} Il drago effettua un attacco di coda.

\textbf{Individuare.} Il drago effettua una prova di Saggezza (Consapevolezza).


\medskip\index{Mostri - Drago d'Argento Adulto}\textbf{Drago d'Argento Adulto}

\emph{Enorme drago, legale buono}

\textbf{FORZA} +8

\textbf{DESTREZZA} +0

\textbf{COSTITUZIONE} +7

\textbf{INTELLIGENZA} +3

\textbf{SAGGEZZA} +1

\textbf{CARISMA} +5

\textbf{Iniziativa} +3 -- \textbf{Difesa} 27

\textbf{Punti Ferita} 243 (18d12 + 126)

\textbf{Movimento} 12 m, volo 24 m

\textbf{Tiri Salvezza} Tempra +15, Riflessi +12, Volontà +17

\textbf{Competenze} Arcano +8, Muoversi Silenziosamente / Nascondersi nelle Ombre +5, Consapevolezza +11, Storia +8

\textbf{Immunità al Danno} freddo

\textbf{Sensi} scurovisione 36 m, vista cieca 18 m 

\textbf{Linguaggi} Comune, Draconico

\textbf{Sfida} 16 (15.000 PE)

\emph{\textbf{Resistenza Leggendaria (3/Giorno).}} Se il drago fallisce un tiro salvezza, può scegliere invece di riuscire.

\textbf{Azioni}

\emph{\textbf{Multiattacco.}} Il drago può usare la sua Presenza Spaventosa. Poi effettuare tre attacchi: uno con il morso e due con gli artigli.

\emph{\textbf{Artiglio.} Attacco con arma da mischia}: +13 a colpire, portata 1 m, un bersaglio.

\emph{Colpisce:} 15 (2d6 + 8) danni taglienti.

\emph{\textbf{Coda.} Attacco con arma da mischia}: +13 a colpire, portata 4,5 m, un bersaglio.

\emph{Colpisce:} 17 (2d8 + 8) danni da botta.

\emph{\textbf{Morso.} Attacco con arma da mischia}: +13 a colpire, portata 3 m, un bersaglio.

\emph{Colpisce:} 19 (2d10 + 8) danni perforanti.

\emph{\textbf{Presenza Spaventosa.}} Ogni creatura scelta dal drago, che si trovi entro 36 metri da esso e consapevole della sua presenza, deve riuscire un tiro salvezza di Volontà CD 18 o restare spaventata per 1 minuto. Una creatura può ripetere il tiro salvezza al termine di ciascun suo turno, terminando l'effetto se lo riesce. Se il tiro salvezza della creatura ha successo o l'effetto ha termine per essa, la creatura è immune alla Presenza Spaventosa del drago per le successive 24 ore.

\emph{\textbf{Arma a Soffio (Ricarica 5-6).}} Il drago usa una delle seguenti armi a soffio:

\emph{Soffio Gelido.} Il drago esala un'esplosione ghiacciata in un cono di 18 metri. Ogni creatura nell'area deve effettuare un tiro salvezza su Tempra CD 20, subendo 58 (13d8) danni da freddo se fallisce il tiro salvezza, o la metà di questi danni se lo riesce.

\emph{Soffio Paralizzante.} Il drago esala un gas paralizzante in un cono di 18 metri. Ogni creatura nell'area deve riuscire un tiro salvezza su Tempra 20 o restare paralizzata per 1 minuto. Una creatura può ripetere il tiro salvezza al termine di ciascun suo turno, terminando l'effetto per sé in caso di successo.

\emph{\textbf{Mutare Forma.}} Il drago può trasformarsi magicamente in un umanoide o bestia il cui grado di sfida sia pari o inferiore al proprio, o tornare alla sua vera forma. Alla morte ritorna alla sua vera forma. Qualsiasi equipaggiamento stia indossando o trasportando viene assorbito o trasportato nella nuova forma (a scelta del drago).

Nella nuova forma, il drago mantiene il suo allineamento, punti ferita, Dadi Vita, la facoltà di parlare, le competenze, la Resistenza Leggendaria, le azioni da tana, e i punteggi di Intelligenza, Saggezza e Carisma, oltre a questa azione. Le sue statistiche e capacità vengono altrimenti rimpiazzate da quelle della nuova forma, eccetto Azioni aggiuntive della nuova forma.

\textbf{Azioni Aggiuntive}

Il drago può effettuare 3 Azioni aggiuntive, scelte tra le opzioni seguenti. Può usare solo un'opzione leggendaria alla volta e solo al termine del turno di un'altra creatura. Il drago recupera le Azioni aggiuntive spese all'inizio del proprio turno.

\textbf{Attacco di Ala (Costa 2 Azioni).} Il drago batte le ali. Ogni creatura entro 3 metri dal drago deve riuscire un tiro salvezza su Riflessi CD 21 o subire 15 (2d6 + 8) danni da botta e venir gettato prono. Il drago può poi volare fino a metà del suo movimento di volo.

\textbf{Attacco di Coda.} Il drago effettua un attacco di coda.

\textbf{Individuare.} Il drago effettua una prova di Saggezza (Consapevolezza).

\medskip\index{Mostri - Drago d'Argento Giovane}\textbf{Drago d'Argento Giovane}

\emph{Grande drago, legale buono}

\textbf{FORZA} +6

\textbf{DESTREZZA} +0

\textbf{COSTITUZIONE} +5

\textbf{INTELLIGENZA} +2

\textbf{SAGGEZZA} +0

\textbf{CARISMA} +4

\textbf{Iniziativa} +2 -- \textbf{Difesa} 23

\textbf{Punti Ferita} 168 (16d10 + 80)

\textbf{Movimento} 12 m, volo 24 m

\textbf{Tiri Salvezza} Tempra +10, Riflessi +8, Volontà +12

\textbf{Competenze} Arcano +6, Muoversi Silenziosamente / Nascondersi nelle Ombre +4, Consapevolezza +8, Storia +6

\textbf{Immunità al Danno} freddo

\textbf{Sensi} scurovisione 36 m, vista cieca 9 m

\textbf{Linguaggi} Comune, Draconico

\textbf{Sfida} 9 (5.000 PE)

\textbf{Azioni}

\emph{\textbf{Multiattacco.}} Il drago può effettuare tre attacchi: uno con il morso e due con gli artigli.

\emph{\textbf{Artiglio.} Attacco con arma da mischia}: +10 a colpire, portata 1 m, un bersaglio.

\emph{Colpisce:} 13 (2d6 + 6) danni taglienti.

\emph{\textbf{Morso.} Attacco con arma da mischia}: +10 a colpire, portata 3 m, un bersaglio.

\emph{Colpisce:} 17 (2d10 + 6) danni perforanti.

\emph{\textbf{Arma a Soffio (Ricarica 5-6).}} Il drago usa una delle seguenti armi a soffio:

\emph{Soffio Gelido.} Il drago esala un'esplosione ghiacciata in un cono di 9 metri. Ogni creatura nell'area deve effettuare un tiro salvezza su Tempra CD 17, subendo 54 (12d8) danni da freddo se fallisce il tiro salvezza, o la metà di questi danni se lo riesce.

\emph{Soffio Paralizzante.} Il drago esala un gas paralizzante in un cono di 9 metri. Ogni creatura nell'area deve riuscire un tiro salvezza su Tempra 17 o restare paralizzata per 1 minuto. Una creatura può ripetere il tiro salvezza al termine di ciascun suo turno, terminando l'effetto per sé in caso di successo.

\medskip\index{Mostri - Drago d'Argento Cucciolo}\textbf{Drago d'Argento Cucciolo}

\emph{Media drago, legale buono}

\textbf{FORZA} +4

\textbf{DESTREZZA} +0

\textbf{COSTITUZIONE} +3

\textbf{INTELLIGENZA} +1

\textbf{SAGGEZZA} +0

\textbf{CARISMA} +2

\textbf{Iniziativa} +1 -- \textbf{Difesa} 18

\textbf{Punti Ferita} 45 (6d8 + 18)

\textbf{Movimento} 9 m, volo 18 m

\textbf{Tiri Salvezza} Tempra +3, Riflessi +3, Volontà +2

\textbf{Competenze} Muoversi Silenziosamente / Nascondersi nelle Ombre +2, Consapevolezza +4

\textbf{Immunità al Danno} freddo

\textbf{Sensi} scurovisione 18 m, vista cieca 3 m

\textbf{Linguaggi} Draconico

\textbf{Sfida} 2 (450 PE)

\textbf{Azioni}

\emph{\textbf{Morso.} Attacco con arma da mischia}: +6 a colpire, portata 1 m, un bersaglio.

\emph{Colpisce:} 9 (1d10 + 4) danni perforanti.

\emph{\textbf{Arma a Soffio (Ricarica 5-6).}} Il drago usa una delle seguenti armi a soffio:

\emph{Soffio Gelido.} Il drago esala un'esplosione ghiacciata in un cono di 4,5 metri. Ogni creatura nell'area deve effettuare un tiro salvezza su Tempra 13, subendo 18 (4d8) danni da freddo se fallisce il tiro salvezza, o la metà di questi danni se lo riesce.

\emph{Soffio Paralizzante.} Il drago esala un gas paralizzante in un cono di 4,5 metri. Ogni creatura nell'area deve riuscire un tiro salvezza su Tempra 13 o restare paralizzata per 1 minuto. Una creatura può ripetere il tiro salvezza al termine di ciascun suo turno, terminando l'effetto per sé in caso di successo.


\medskip\index{Mostri - Drago di Bronzo Antico}\textbf{Drago di Bronzo Antico}

\emph{Mastodontica drago, caotico buono}

\textbf{FORZA} +9

\textbf{DESTREZZA} +0

\textbf{COSTITUZIONE} +8

\textbf{INTELLIGENZA} +4

\textbf{SAGGEZZA} +3

\textbf{CARISMA} +5

\textbf{Iniziativa} +4 -- \textbf{Difesa} 33

\textbf{Punti Ferita} 444 (24d20 + 192)

\textbf{Movimento} 12 m, nuoto 12 m, volo 24 m

\textbf{Tiri Salvezza} Tempra +21, Riflessi +13, Volontà +21

\textbf{Competenze} Muoversi Silenziosamente / Nascondersi nelle Ombre +7, Percepire Emozioni +10, Consapevolezza +17

\textbf{Immunità al Danno} fulmine, arma +1

\textbf{Sensi} scurovisione 36 m, vista cieca 18 m

\textbf{Linguaggi} Comune, Draconico

\textbf{Sfida} 22 (41.000 PE)

\emph{\textbf{Anfibio.}} Il drago può respirare aria e acqua.

\emph{\textbf{Resistenza Leggendaria (3/Giorno).}} Se il drago fallisce un tiro salvezza, può scegliere invece di riuscire.

\textbf{Azioni}

\emph{\textbf{Multiattacco.}} Il drago può usare la sua Presenza Spaventosa. Poi effettuare tre attacchi: uno con il morso e due con gli artigli.

\emph{\textbf{Artiglio.} Attacco con arma da mischia}: +16 a colpire, portata 3 m, un bersaglio.

\emph{Colpisce:} 16 (2d6 + 9) danni taglienti.

\emph{\textbf{Coda.} Attacco con arma da mischia}: +16 a colpire, portata 6 m, un bersaglio.

\emph{Colpisce:} 18 (2d8 + 9) danni da botta.

\emph{\textbf{Morso.} Attacco con arma da mischia}: +16 a colpire, portata 4,5 m, un bersaglio.

\emph{Colpisce:} 20 (2d10 + 9) danni perforanti.

\emph{\textbf{Presenza Spaventosa.}} Ogni creatura scelta dal drago, che si trovi entro 36 metri da esso e consapevole della sua presenza, deve riuscire un tiro salvezza di Volontà CD 20 o restare spaventata per 1 minuto. Una creatura può ripetere il tiro salvezza al termine di ciascun suo turno, terminando l'effetto se lo riesce. Se il tiro salvezza della creatura ha successo o l'effetto ha termine per essa, la creatura è immune alla Presenza Spaventosa del drago per le successive 24 ore.

\emph{\textbf{Arma a Soffio (Ricarica 5-6).}} Il drago usa una delle seguenti armi a soffio:

\emph{Soffio Fulminante.} Il drago esala fulmini in una linea lunga 36 metri e larga 3 metri. Ogni creatura sulla linea deve effettuare un tiro salvezza su Riflessi CD 23, subendo 88 (16d10) danni da fulmine se fallisce il tiro salvezza, o la metà di questi danni se lo riesce. \emph{Soffio Repulsivo.} Il drago esala dell'energia repulsiva in un cono di 9 metri. Ogni creatura in quell'area deve riuscire un tiro salvezza su Tempra CD 23, altrimenti viene allontana di 18 metri dal
drago.

\emph{\textbf{Mutare Forma.}} Il drago può trasformarsi magicamente in un umanoide o bestia il cui grado di sfida sia pari o inferiore al proprio, o tornare alla sua vera forma. Alla morte ritorna alla sua vera forma. Qualsiasi equipaggiamento stia indossando o trasportando viene assorbito o trasportato nella nuova forma (a scelta del drago).

Nella nuova forma, il drago mantiene il suo allineamento, punti ferita, Dadi Vita, la facoltà di parlare, le competenze, la Resistenza Leggendaria, le azioni da tana, e i punteggi di Intelligenza, Saggezza e Carisma, oltre a questa azione. Le sue statistiche e capacità vengono altrimenti rimpiazzate da quelle della nuova forma, eccetto Azioni aggiuntive della nuova forma.

\textbf{Azioni Aggiuntive}

Il drago può effettuare 3 Azioni aggiuntive, scelte tra le opzioni seguenti. Può usare solo un'opzione leggendaria alla volta e solo al termine del turno di un'altra creatura. Il drago recupera le Azioni aggiuntive spese all'inizio del proprio turno.

\textbf{Attacco di Ala (Costa 2 Azioni).} Il drago batte le ali. Ogni creatura entro 4,5 metri dal drago deve riuscire un tiro salvezza su Riflessi CD 24 o subire 16 (2d6 + 9) danni da botta e venir gettato prono. Il drago può poi volare fino a metà del suo movimento di volo.

\textbf{Attacco di Coda.} Il drago effettua un attacco di coda.

\textbf{Individuare.} Il drago effettua una prova di Saggezza (Consapevolezza).

\medskip\index{Mostri - Drago di Bronzo Adulto}\textbf{Drago di Bronzo Adulto}

\emph{Enorme drago, caotico buono}

\textbf{FORZA} +7

\textbf{DESTREZZA} +0

\textbf{COSTITUZIONE} +6

\textbf{INTELLIGENZA} +3

\textbf{SAGGEZZA} +2

\textbf{CARISMA} +4

\textbf{Iniziativa} +3 -- \textbf{Difesa} 27

\textbf{Punti Ferita} 212 (17d12 + 102)

\textbf{Movimento} 12 m, nuoto 12 m, volo 24 m

\textbf{Tiri Salvezza} Tempra +15, Riflessi +10, Volontà +15

\textbf{Competenze} Muoversi Silenziosamente / Nascondersi nelle Ombre +5, Percepire Emozioni +7, Consapevolezza +12

\textbf{Immunità al Danno} fulmine

\textbf{Sensi} scurovisione 36 m, vista cieca 18 m

\textbf{Linguaggi} Comune, Draconico

\textbf{Sfida} 15 (13.000 PE)

\emph{\textbf{Anfibio.}} Il drago può respirare aria e acqua.

\emph{\textbf{Resistenza Leggendaria (3/Giorno).}} Se il drago fallisce un tiro salvezza, può scegliere invece di riuscire.

\textbf{Azioni}

\emph{\textbf{Multiattacco.}} Il drago può usare la sua Presenza Spaventosa e poi effettuare tre attacchi: uno con il morso e due con gli
artigli.

\emph{\textbf{Artiglio.} Attacco con arma da mischia}: +12 a colpire, portata 1 m, un bersaglio.

\emph{Colpisce:} 14 (2d6 + 7) danni taglienti.

\emph{\textbf{Coda.} Attacco con arma da mischia}: +12 a colpire, portata 4,5 m, un bersaglio.

\emph{Colpisce:} 16 (2d8 + 7) danni da botta.

\emph{\textbf{Morso.} Attacco con arma da mischia}: +12 a colpire, portata 3 m, un bersaglio.

\emph{Colpisce:} 18 (2d10 + 7) danni perforanti.

\emph{\textbf{Presenza Spaventosa.}} Ogni creatura scelta dal drago, che si trovi entro 36 metri da esso e consapevole della sua presenza, deve riuscire un tiro salvezza di Volontà CD 17 o restare spaventata per 1 minuto. Una creatura può ripetere il tiro salvezza al termine di ciascun suo turno, terminando l'effetto se lo riesce. Se il tiro salvezza della creatura ha successo o l'effetto ha termine per essa, la creatura è immune alla Presenza Spaventosa del drago per le successive 24 ore. 

\emph{\textbf{Arma a Soffio (Ricarica 5-6).}} Il drago usa una delle seguenti armi a soffio:

\emph{Soffio Fulminante.} Il drago esala fulmini in una linea lunga 27 metri e larga 1,5 metri. Ogni creatura sulla linea deve effettuare un tiro salvezza di Riflessi CD 19, subendo 66 (12d10) danni da fulmine se fallisce il tiro salvezza, o la metà di questi danni se lo riesce. \emph{Soffio Repulsivo.} Il drago esala dell'energia repulsiva in un cono di 9 metri. Ogni creatura in quell'area deve riuscire un tiro salvezza su Tempra CD 19, altrimenti viene allontana di 18 metri dal drago.

\emph{\textbf{Mutare Forma.}} Il drago può trasformarsi magicamente in un umanoide o bestia il cui grado di sfida sia pari o inferiore al proprio, o tornare alla sua vera forma. Alla morte ritorna alla sua vera forma. Qualsiasi equipaggiamento stia indossando o trasportando viene assorbito o trasportato nella nuova forma (a scelta del drago).

Nella nuova forma, il drago mantiene il suo allineamento, punti ferita, Dadi Vita, la facoltà di parlare, le competenze, la Resistenza Leggendaria, le azioni da tana, e i punteggi di Intelligenza, Saggezza e Carisma, oltre a questa azione. Le sue statistiche e capacità vengono altrimenti rimpiazzate da quelle della nuova forma, eccetto Azioni aggiuntive della nuova forma.

\textbf{Azioni Aggiuntive}

Il drago può effettuare 3 Azioni aggiuntive, scelte tra le opzioni seguenti. Può usare solo un'opzione leggendaria alla volta e solo al termine del turno di un'altra creatura. Il drago recupera le Azioni aggiuntive spese all'inizio del proprio turno.

\textbf{Attacco di Ala (Costa 2 Azioni).} Il drago batte le ali. Ogni creatura entro 3 metri dal drago deve riuscire un tiro salvezza su Riflessi CD 20 o subire 14 (2d6 + 7) danni da botta e venir gettato prono. Il drago può poi volare fino a metà del suo movimento di volo.

\textbf{Attacco di Coda.} Il drago effettua un attacco di coda. 

\textbf{Individuare.} Il drago effettua una prova di Saggezza (Consapevolezza).


\medskip\index{Mostri - Drago di Bronzo Giovane}\textbf{Drago di Bronzo Giovane}

\emph{Grande drago, caotico buono}

\textbf{FORZA} +5

\textbf{DESTREZZA} +0

\textbf{COSTITUZIONE} +4

\textbf{INTELLIGENZA} +2

\textbf{SAGGEZZA} +1

\textbf{CARISMA} +3

\textbf{Iniziativa} +2 -- \textbf{Difesa} 22

\textbf{Punti Ferita} 142 (15d10 + 60)

\textbf{Movimento} 12 m, nuoto 12 m, volo 24 m

\textbf{Tiri Salvezza} Tempra +10, Riflessi +8, Volontà +10

\textbf{Competenze} Muoversi Silenziosamente / Nascondersi nelle Ombre +3, Percepire Emozioni +4, Consapevolezza +7

\textbf{Immunità al Danno} fulmine

\textbf{Sensi} scurovisione 36 m, vista cieca 9 m

\textbf{Linguaggi} Comune, Draconico

\textbf{Sfida} 8 (3.900 PE)

\emph{\textbf{Anfibio.}} Il drago può respirare aria e acqua.

\textbf{Azioni}

\emph{\textbf{Multiattacco.}} Il drago può usare effettuare tre attacchi: uno con il morso e due con gli artigli.

\emph{\textbf{Artiglio.} Attacco con arma da mischia}: +8 a colpire, portata 1 m, un bersaglio.

\emph{Colpisce:} 12 (2d6 + 5) danni taglienti.

\emph{\textbf{Morso.} Attacco con arma da mischia}: +8 a colpire, portata 3 m, un bersaglio.

\emph{Colpisce:} 16 (2d10 + 5) danni perforanti.

\emph{\textbf{Arma a Soffio (Ricarica 5-6).}} Il drago usa una delle seguenti armi a soffio:

\emph{Soffio Fulminante.} Il drago esala fulmini in una linea lunga 18 metri e larga 1,5 metri. Ogni creatura sulla linea deve effettuare un tiro salvezza di Riflessi CD 15, subendo 55 (10d10) danni da fulmine se fallisce il tiro salvezza, o la metà di questi danni se lo riesce.

\emph{Soffio Repulsivo.} Il drago esala dell'energia repulsiva in un cono di 9 metri. Ogni creatura in quell'area deve riuscire un tiro salvezza su Tempra CD 15, altrimenti viene allontana di 12 metri dal drago.

\medskip\index{Mostri - Drago di Bronzo Cucciolo}\textbf{Drago di Bronzo Cucciolo}

\emph{Media drago, caotico buono}

\textbf{FORZA} +3

\textbf{DESTREZZA} +0

\textbf{COSTITUZIONE} +2

\textbf{INTELLIGENZA} +1

\textbf{SAGGEZZA} +0

\textbf{CARISMA} +2

\textbf{Iniziativa} +1 -- \textbf{Difesa} 18

\textbf{Punti Ferita} 32 (5d8 + 10)

\textbf{Movimento} 9 m, nuoto 9 m, volo 18 m

\textbf{Tiri Salvezza} Tempra +2, Riflessi +1, Volontà +1

\textbf{Competenze} Muoversi Silenziosamente / Nascondersi nelle Ombre +2, Consapevolezza +4

\textbf{Immunità al Danno} fulmine

\textbf{Sensi} scurovisione 18 m, vista cieca 3 m

\textbf{Linguaggi} Draconico

\textbf{Sfida} 2 (450 PE)

\emph{\textbf{Anfibio.}} Il drago può respirare aria e acqua.

\textbf{Azioni}

\emph{\textbf{Morso.} Attacco con arma da mischia}: +5 a colpire,
portata 1 m, un bersaglio.

\emph{Colpisce:} 8 (1d10 + 3) danni perforanti.

\emph{\textbf{Arma a Soffio (Ricarica 5-6).}} Il drago usa una delle seguenti armi a soffio:

\emph{Soffio Fulminante.} Il drago esala fulmini in una linea lunga 12 metri e larga 1,5 metri. Ogni creatura sulla linea deve effettuare un tiro salvezza di Riflessi CD 12, subendo 16 (3d10) danni da fulmine se fallisce il tiro salvezza, o la metà di questi danni se lo riesce.

\emph{Soffio Repulsivo.} Il drago esala dell'energia repulsiva in un cono di 9 metri. Ogni creatura in quell'area deve riuscire un tiro salvezza su Tempra CD 12, altrimenti viene allontana di 9 metri dal drago.

\medskip\index{Mostri - Drago d'Oro Antico}\textbf{Drago d'Oro Antico}

\emph{Mastodontica drago, legale buono}

\textbf{FORZA} +10

\textbf{DESTREZZA} +2

\textbf{COSTITUZIONE} +9

\textbf{INTELLIGENZA} +4

\textbf{SAGGEZZA} +3

\textbf{CARISMA} +9

\textbf{Iniziativa} +4 -- \textbf{Difesa} 34

\textbf{Punti Ferita} 546 (28d20 + 252)

\textbf{Movimento} 12 m, nuoto 12 m, volo 24 m

\textbf{Tiri Salvezza} Tempra +23, Riflessi +14, Volontà +24

\textbf{Competenze} Muoversi Silenziosamente / Nascondersi nelle Ombre +9, Percepire Emozioni +10, Consapevolezza +17, Ingannare +16

\textbf{Immunità al Danno} fuoco, arma +1

\textbf{Sensi} scurovisione 36 m, vista cieca 18 m

\textbf{Linguaggi} Comune, Draconico

\textbf{Sfida} 24 (62.000 PE)

\emph{\textbf{Anfibio.}} Il drago può respirare aria e acqua.

\emph{\textbf{Resistenza Leggendaria (3/Giorno).}} Se il drago fallisce un tiro salvezza, può scegliere invece di riuscire.

\textbf{Azioni}

\emph{\textbf{Multiattacco.}} Il drago può usare la sua Presenza Spaventosa. Poi effettuare tre attacchi: uno con il morso e due con gli artigli.

\emph{\textbf{Artiglio.} Attacco con arma da mischia}: +17 a colpire, portata 3 m, un bersaglio.

\emph{Colpisce:} 17 (2d6 + 10) danni taglienti.

\emph{\textbf{Coda.} Attacco con arma da mischia}: +17 a colpire, portata 6 m, un bersaglio.

\emph{Colpisce:} 19 (2d8 + 10) danni da botta.

\emph{\textbf{Morso.} Attacco con arma da mischia}: +17 a colpire, portata 4,5 m, un bersaglio.

\emph{Colpisce:} 21 (2d10 + 10) danni perforanti.

\emph{\textbf{Presenza Spaventosa.}} Ogni creatura scelta dal drago, che si trovi entro 36 metri da esso e consapevole della sua presenza, deve riuscire un tiro salvezza di Volontà CD 24 o restare spaventata per 1 minuto. Una creatura può ripetere il tiro salvezza al termine di ciascun suo turno, terminando l'effetto se lo riesce. Se il tiro salvezza della creatura ha successo o l'effetto ha termine per essa, la creatura è immune alla Presenza Spaventosa del drago per le successive 24 ore.

\emph{\textbf{Arma a Soffio (Ricarica 5-6).}} Il drago usa una delle seguenti armi a soffio:

\emph{Soffio Infuocato.} Il drago esala fuoco in un cono di 27 metri. Ogni creatura nell'area deve effettuare un tiro salvezza di Riflessi CD 24, subendo 71(13d10) danni da fuoco se fallisce il tiro salvezza, o la metà di questi danni se lo riesce.

\emph{Soffio Indebolente.} Il drago esala del gas in un cono di 27 metri. Ogni creatura in quell'area deve riuscire un tiro salvezza su Tempra CD 24 o avere -1d6 ai tiri di attacco basati sulla Forza, prove di Forza, e tiri salvezza su Tempra per 1 minuto. Una creatura può ripetere il tiro salvezza al termine di ciascun suo turno, terminando l'effetto su di sé in caso di successo.

\emph{\textbf{Mutare Forma.}} Il drago può trasformarsi magicamente in un umanoide o bestia il cui grado di sfida sia pari o inferiore al proprio, o tornare alla sua vera forma. Alla morte ritorna alla sua vera forma. Qualsiasi equipaggiamento stia indossando o trasportando viene assorbito o trasportato nella nuova forma (a scelta del drago).

Nella nuova forma, il drago mantiene il suo allineamento, punti ferita, Dadi Vita, la facoltà di parlare, le competenze, la Resistenza Leggendaria, le azioni da tana, e i punteggi di Intelligenza, Saggezza e Carisma, oltre a questa azione. Le sue statistiche e capacità vengono altrimenti rimpiazzate da quelle della nuova forma, eccetto Azioni aggiuntive della nuova forma.

\textbf{Azioni Aggiuntive}

Il drago può effettuare 3 Azioni aggiuntive, scelte tra le opzioni seguenti. Può usare solo un'opzione leggendaria alla volta e solo al termine del turno di un'altra creatura. Il drago recupera le Azioni aggiuntive spese all'inizio del proprio turno.

\textbf{Attacco di Ala (Costa 2 Azioni).} Il drago batte le ali. Ogni creatura entro 4,5 metri dal drago deve riuscire un tiro salvezza su Riflessi CD 25 o subire 17 (2d6 + 10) danni da botta e venir gettato prono. Il drago può poi volare fino a metà del suo movimento di volo.

\textbf{Attacco di Coda.} Il drago effettua un attacco di coda.

\textbf{Individuare.} Il drago effettua una prova di Saggezza (Consapevolezza).



\medskip\index{Mostri - Drago d'Oro Adulto}\textbf{Drago d'Oro Adulto}

\emph{Enorme drago, legale buono}

\textbf{FORZA} +8

\textbf{DESTREZZA} +2

\textbf{COSTITUZIONE} +7

\textbf{INTELLIGENZA} +3

\textbf{SAGGEZZA} +2

\textbf{CARISMA} +7

\textbf{Iniziativa} +3 -- \textbf{Difesa} 28

\textbf{Punti Ferita} 256 (19d12 + 133)

\textbf{Movimento} 12 m, nuoto 12 m, volo 24 m

\textbf{Tiri Salvezza} Tempra +17, Riflessi +11, Volontà +18

\textbf{Competenze} Muoversi Silenziosamente / Nascondersi nelle Ombre +8, Percepire Emozioni +8, Consapevolezza +14, Ingannare +13 

\textbf{Immunità al Danno} fuoco

\textbf{Sensi} scurovisione 36 m, vista cieca 18 m

\textbf{Linguaggi} Comune, Draconico

\textbf{Sfida} 17 (18.000 PE)

\emph{\textbf{Anfibio.}} Il drago può respirare aria e acqua.

\emph{\textbf{Resistenza Leggendaria (3/Giorno).}} Se il drago fallisce un tiro salvezza, può scegliere invece di riuscire.

\textbf{Azioni}

\emph{\textbf{Multiattacco.}} Il drago può usare la sua Presenza Spaventosa. Poi effettuare tre attacchi: uno con il morso e due con gli artigli.

\emph{\textbf{Artiglio.} Attacco con arma da mischia}: +14 a colpire, portata 1 m, un bersaglio.

\emph{Colpisce:} 15 (2d6 + 8) danni taglienti.

\emph{\textbf{Coda.} Attacco con arma da mischia}: +14 a colpire, portata 4,5 m, un bersaglio.

\emph{Colpisce:} 17 (2d8 + 8) danni da botta.

\emph{\textbf{Morso.} Attacco con arma da mischia}: +14 a colpire, portata 3 m, un bersaglio.

\emph{Colpisce:} 19 (2d10 + 8) danni perforanti.

\emph{\textbf{Presenza Spaventosa.}} Ogni creatura scelta dal drago, che si trovi entro 36 metri da esso e consapevole della sua presenza, deve riuscire un tiro salvezza di Volontà CD 21 o restare spaventata per 1 minuto. Una creatura può ripetere il tiro salvezza al termine di ciascun suo turno, terminando l'effetto se lo riesce. Se il tiro salvezza della creatura ha successo o l'effetto ha termine per essa, la creatura è immune alla Presenza Spaventosa del drago per le successive 24 ore.

\emph{\textbf{Arma a Soffio (Ricarica 5-6).}} Il drago usa una delle seguenti armi a soffio:

\emph{Soffio Infuocato.} Il drago esala fuoco in un cono di 18 metri. Ogni creatura nell'area deve effettuare un tiro salvezza di Riflessi CD 21, subendo 66 (12d10) danni da fuoco se fallisce il tiro salvezza, o la metà di questi danni se lo riesce.

\emph{Soffio Indebolente.} Il drago esala del gas in un cono di 18 metri. Ogni creatura in quell'area deve riuscire un tiro salvezza su Tempra CD 21 o avere -1d6 ai tiri di attacco basati sulla Forza, prove di Forza, e tiri salvezza su Tempra per 1 minuto. Una creatura può ripetere il tiro salvezza al termine di ciascun suo turno, terminando l'effetto su di sé in caso di successo.

\emph{\textbf{Mutare Forma.}} Il drago può trasformarsi magicamente in un umanoide o bestia il cui grado di sfida sia pari o inferiore al proprio, o tornare alla sua vera forma. Alla morte ritorna alla sua vera forma. Qualsiasi equipaggiamento stia indossando o trasportando viene assorbito o trasportato nella nuova forma (a scelta del drago).

Nella nuova forma, il drago mantiene il suo allineamento, punti ferita, Dadi Vita, la facoltà di parlare, le competenze, la Resistenza Leggendaria, le azioni da tana, e i punteggi di Intelligenza, Saggezza e Carisma, oltre a questa azione. Le sue statistiche e capacità vengono altrimenti rimpiazzate da quelle della nuova forma, eccetto Azioni aggiuntive della nuova forma. 

\textbf{Azioni Aggiuntive}

Il drago può effettuare 3 Azioni aggiuntive, scelte tra le opzioni seguenti. Può usare solo un'opzione leggendaria alla volta e solo al termine del turno di un'altra creatura. Il drago recupera le Azioni aggiuntive spese all'inizio del proprio turno.

\textbf{Attacco di Ala (Costa 2 Azioni).} Il drago batte le ali. Ogni creatura entro 3 metri dal drago deve riuscire un tiro salvezza su Riflessi CD 22 o subire 15 (2d6 + 8) danni da botta e venir gettato prono. Il drago può poi volare fino a metà del suo movimento di volo.

\textbf{Attacco di Coda.} Il drago effettua un attacco di coda.

\textbf{Individuare.} Il drago effettua una prova di Saggezza (Consapevolezza).

\medskip\index{Mostri - Drago d'Oro Giovane}\textbf{Drago d'Oro Giovane}

\emph{Grande drago, legale buono}

\textbf{FORZA} +6

\textbf{DESTREZZA} +2

\textbf{COSTITUZIONE} +5

\textbf{INTELLIGENZA} +3

\textbf{SAGGEZZA} +1

\textbf{CARISMA} +5

\textbf{Iniziativa} +3 -- \textbf{Difesa} 23

\textbf{Punti Ferita} 178 (17d10 + 85)

\textbf{Movimento} 12 m, nuoto 12 m, volo 24 m

\textbf{Tiri Salvezza} Tempra +12, Riflessi +9, Volontà +13

\textbf{Competenze} Muoversi Silenziosamente / Nascondersi nelle Ombre +6, Percepire Emozioni +5, Consapevolezza +9, Ingannare +9

\textbf{Immunità al Danno} fuoco

\textbf{Sensi} scurovisione 36 m, vista cieca 9 m

\textbf{Linguaggi} Comune, Draconico

\textbf{Sfida} 10 (5.900 PE)

\emph{\textbf{Anfibio.}} Il drago può respirare aria e acqua.

\textbf{Azioni}

\emph{\textbf{Multiattacco.}} Il drago può effettuare tre attacchi: uno con il morso e due con gli artigli.

\emph{\textbf{Artiglio.} Attacco con arma da mischia}: +10 a colpire, portata 1 m, un bersaglio.

\emph{Colpisce:} 13 (2d6 + 6) danni taglienti.

\emph{\textbf{Morso.} Attacco con arma da mischia}: +10 a colpire, portata 3 m, un bersaglio.

\emph{Colpisce:} 17 (2d10 + 6) danni perforanti.

\emph{\textbf{Arma a Soffio (Ricarica 5-6).}} Il drago usa una delle seguenti armi a soffio:

\emph{Soffio Infuocato.} Il drago esala fuoco in un cono di 9 metri. Ogni creatura nell'area deve effettuare un tiro salvezza di Riflessi CD 17, subendo 55 (10d10) danni da fuoco se fallisce il tiro salvezza, o la metà di questi danni se lo riesce.

\emph{Soffio Indebolente.} Il drago esala del gas in un cono di 9 metri. Ogni creatura in quell'area deve riuscire un tiro salvezza di Tempra CD 17 o avere -1d6 ai tiri di attacco basati sulla Forza, prove di Forza, e tiri salvezza su Tempra per 1 minuto. Una creatura può ripetere il tiro salvezza al termine di ciascun suo turno, terminando l'effetto su di sé in caso di successo.

\medskip\index{Mostri - Drago d'Oro Cucciolo}\textbf{Drago d'Oro Cucciolo}

\emph{Media drago, legale buono}

\textbf{FORZA} +4

\textbf{DESTREZZA} +2

\textbf{COSTITUZIONE} +3

\textbf{INTELLIGENZA} +2

\textbf{SAGGEZZA} +0

\textbf{CARISMA} +3

\textbf{Iniziativa} +2 -- \textbf{Difesa} 19

\textbf{Punti Ferita} 60 (8d8 + 24)

\textbf{Movimento} 9 m, nuoto 9 m, volo 18 m

\textbf{Tiri Salvezza} Tempra +3, Riflessi +2, Volontà +1

\textbf{Competenze} Muoversi Silenziosamente / Nascondersi nelle Ombre +4, Consapevolezza +4

\textbf{Immunità al Danno} fuoco

\textbf{Sensi} scurovisione 18 m, vista cieca 3 m

\textbf{Linguaggi} Draconico

\textbf{Sfida} 3 (700 PE)

\emph{\textbf{Anfibio.}} Il drago può respirare aria e acqua.

\textbf{Azioni}

\emph{\textbf{Morso.} Attacco con arma da mischia}: +6 a colpire, portata 1 m, un bersaglio.

\emph{Colpisce:} 9 (1d10 + 4) danni perforanti.

\emph{\textbf{Arma a Soffio (Ricarica 5-6).}} Il drago usa una delle seguenti armi a soffio:

\emph{Soffio Infuocato.} Il drago esala fuoco in un cono di 4,5 metri. Ogni creatura nell'area deve effettuare un tiro salvezza di Riflessi CD 13, subendo 22 (4d10) danni da fuoco se fallisce il tiro salvezza, o la metà di questi danni se lo riesce.

\emph{Soffio Indebolente.} Il drago esala del gas in un cono di 4,5 metri. Ogni creatura in quell'area deve riuscire un tiro salvezza su Tempra CD 13 o avere -1d6 ai tiri di attacco basati sulla Forza, prove di Forza, e tiri salvezza su Tempra per 1 minuto. Una creatura può ripetere il tiro salvezza al termine di ciascun suo turno, terminando l'effetto su di sé in caso di successo.

\medskip\index{Mostri - Drago d'Ottone Antico}\textbf{Drago d'Ottone Antico}

\emph{Mastodontica drago, caotico buono}

\textbf{FORZA} +8

\textbf{DESTREZZA} +0

\textbf{COSTITUZIONE} +7

\textbf{INTELLIGENZA} +3

\textbf{SAGGEZZA} +2

\textbf{CARISMA} +4

\textbf{Iniziativa} +3 -- \textbf{Difesa} 30

\textbf{Punti Ferita} 297 (17d20 + 119)

\textbf{Movimento} 12 m, scavo 12 m, volo 24 m

\textbf{Tiri Salvezza} Tempra +20, Riflessi +13, Volontà +18

\textbf{Competenze} Muoversi Silenziosamente / Nascondersi nelle Ombre +6, Consapevolezza +14, Ingannare +10, Storia +9 

\textbf{Immunità al Danno} fuoco, arma +1

\textbf{Sensi} scurovisione 36 m, vista cieca 18 m

\textbf{Linguaggi} Comune, Draconico

\textbf{Sfida} 20 (25.000 PE)

\emph{\textbf{Resistenza Leggendaria (3/Giorno).}} Se il drago fallisce un tiro salvezza, può scegliere invece di riuscire.

\textbf{Azioni}

\emph{\textbf{Multiattacco.}} Il drago può usare la sua Presenza Spaventosa. Poi effettuare tre attacchi: uno con il morso e due con gli artigli.

\emph{\textbf{Artiglio.} Attacco con arma da mischia}: +14 a colpire, portata 3 m, un bersaglio.

\emph{Colpisce:} 15 (2d6 + 8) danni taglienti.

\emph{\textbf{Coda.} Attacco con arma da mischia}: +14 a colpire, portata 6 m, un bersaglio.

\emph{Colpisce:} 17 (2d8 + 8) danni da botta.

\emph{\textbf{Morso.} Attacco con arma da mischia}: +14 a colpire, portata 4,5 m, un bersaglio.

\emph{Colpisce:} 19 (2d10 + 8) danni perforanti.

\emph{\textbf{Presenza Spaventosa.}} Ogni creatura scelta dal drago, che si trovi entro 36 metri da esso e consapevole della sua presenza, deve riuscire un tiro salvezza di Volontà CD 18 o restare spaventata per 1 minuto. Una creatura può ripetere il tiro salvezza al termine di ciascun suo turno, terminando l'effetto se lo riesce. Se il tiro salvezza della creatura ha successo o l'effetto hatermine per essa, la creatura è immune alla Presenza Spaventosa del drago per le successive 24 ore.

\emph{\textbf{Arma a Soffio (Ricarica 5-6).}} Il drago usa una delle seguenti armi a soffio:

\emph{Soffio Infuocato.} Il drago esala fuoco in una linea lunga 27 metri e larga 3 metri. Ogni creatura sulla linea deve effettuare un tiro salvezza su Riflessi CD 21, subendo 56 (16d6) danni da fuoco se fallisce il tiro salvezza, o la metà di questi danni se lo riesce.

\emph{Soffio Soporifero.} Il drago esala del gas soporifero in un cono di 27 metri. Ogni creatura in quell'area deve riuscire un tiro salvezza su Tempra 21 o cadere svenuta per 10 minuti. Questo effetto
termina se la creatura svenuta subisce danni o qualcuno impiega un'azione per risvegliarla.

\emph{\textbf{Mutare Forma.}} Il drago può trasformarsi magicamente in un umanoide o bestia il cui grado di sfida sia pari o inferiore al proprio, o tornare alla sua vera forma. Alla morte ritorna alla sua vera forma. Qualsiasi equipaggiamento stia indossando o trasportando viene assorbito o trasportato nella nuova forma (a scelta del drago).

Nella nuova forma, il drago mantiene il suo allineamento, punti ferita, Dadi Vita, la facoltà di parlare, le competenze, la Resistenza Leggendaria, le azioni da tana, e i punteggi di Intelligenza, Saggezza e Carisma, oltre a questa azione. Le sue statistiche e capacità vengono altrimenti rimpiazzate da quelle della nuova forma, eccetto Azioni aggiuntive della nuova forma.

\textbf{Azioni Aggiuntive}

Il drago può effettuare 3 Azioni aggiuntive, scelte tra le opzioni seguenti. Può usare solo un'opzione leggendaria alla volta e solo al termine del turno di un'altra creatura. Il drago recupera le Azioni aggiuntive spese all'inizio del proprio turno.

\textbf{Attacco di Ala (Costa 2 Azioni).} Il drago batte le ali. Ogni creatura entro 4,5 metri dal drago deve riuscire un tiro salvezza su Riflessi CD 22 o subire 15 (2d6 + 8) danni da botta e venir gettato prono. Il drago può poi volare fino a metà del suo movimento di volo.

\textbf{Attacco di Coda.} Il drago effettua un attacco di coda.

\textbf{Individuare.} Il drago effettua una prova di Saggezza (Consapevolezza).

\medskip\index{Mostri - Drago d'Ottone Adulto}\textbf{Drago d'Ottone Adulto}

\emph{Enorme drago, caotico buono}

\textbf{FORZA} +6

\textbf{DESTREZZA} +0

\textbf{COSTITUZIONE} +5

\textbf{INTELLIGENZA} +2

\textbf{SAGGEZZA} +1

\textbf{CARISMA} +3

\textbf{Iniziativa} +2 -- \textbf{Difesa} 25

\textbf{Punti Ferita} 172 (15d12 + 75)

\textbf{Movimento} 12 m, scavo 9 m, volo 24 m

\textbf{Tiri Salvezza} Tempra +14, Riflessi +10, Volontà +12

\textbf{Competenze} Muoversi Silenziosamente / Nascondersi nelle Ombre +5, Consapevolezza +11, Ingannare +8, Storia +7

\textbf{Immunità al Danno} fuoco

\textbf{Sensi} scurovisione 36 m, vista cieca 18 m 

\textbf{Linguaggi} Comune, Draconico

\textbf{Sfida} 13 (10.000 PE)

\emph{\textbf{Resistenza Leggendaria (3/Giorno).}} Se il drago fallisce un tiro salvezza, può scegliere invece di riuscire.

\textbf{Azioni}

\emph{\textbf{Multiattacco.}} Il drago può usare la sua Presenza Spaventosa. Poi effettuare tre attacchi: uno con il morso e due con gli artigli.

\emph{\textbf{Artiglio.} Attacco con arma da mischia}: +11 a colpire, portata 1 m, un bersaglio.

\emph{Colpisce:} 13 (2d6 + 6) danni taglienti.

\emph{\textbf{Coda.} Attacco con arma da mischia}: +11 a colpire, portata 4,5 m, un bersaglio.

\emph{Colpisce:} 15 (2d8 + 6) danni da botta.

\emph{\textbf{Morso.} Attacco con arma da mischia}: +11 a colpire, portata 3 m, un bersaglio.

\emph{Colpisce:} 17 (2d10 + 6) danni perforanti.

\emph{\textbf{Presenza Spaventosa.}} Ogni creatura scelta dal drago, che si trovi entro 36 metri da esso e consapevole della sua presenza, deve riuscire un tiro salvezza di Volontà CD 16 o restare spaventata per 1 minuto. Una creatura può ripetere il tiro salvezza al termine di ciascun suo turno, terminando l'effetto se lo riesce. Se il tiro salvezza della creatura ha successo o l'effetto ha termine per essa, la creatura è immune alla Presenza Spaventosa del drago per le successive 24 ore.

\emph{\textbf{Arma a Soffio (Ricarica 5-6).}} Il drago usa una delle seguenti armi a soffio:

\emph{Soffio Infuocato.} Il drago esala fuoco in una linea lunga 18 metri e larga 1,5 metri. Ogni creatura sulla linea deve effettuare un tiro salvezza di Riflessi CD 18, subendo 45 (13d6) danni da fuoco se fallisce il tiro salvezza, o la metà di questi danni se lo riesce. 

\emph{Soffio Soporifero.} Il drago esala del gas soporifero in un cono di 18 metri. Ogni creatura in quell'area deve riuscire un tiro salvezza su Tempra 18 o cadere svenuta per 10 minuti. Questo effetto termina se la creatura svenuta subisce danni o qualcuno impiega un'azione per risvegliarla.

\textbf{Azioni Aggiuntive}

Il drago può effettuare 3 Azioni aggiuntive, scelte tra le opzioni seguenti. Può usare solo un'opzione leggendaria alla volta e solo al termine del turno di un'altra creatura. Il drago recupera le Azioni aggiuntive spese all'inizio del proprio turno.

\textbf{Attacco di Ala (Costa 2 Azioni).} Il drago batte le ali. Ogni creatura entro 3 metri dal drago deve riuscire un tiro salvezza su Riflessi CD 19 o subire 13 (2d6 + 6) danni da botta e venir gettato  prono. Il drago può poi volare fino a metà del suo movimento di volo.

\textbf{Attacco di Coda.} Il drago effettua un attacco di coda. 

\textbf{Individuare.} Il drago effettua una prova di Saggezza (Consapevolezza).

\medskip\index{Mostri - Drago d'Ottone Giovane}\textbf{Drago d'Ottone Giovane}

\emph{Grande drago, caotico buono}

\textbf{FORZA} +4

\textbf{DESTREZZA} +0

\textbf{COSTITUZIONE} +3

\textbf{INTELLIGENZA} +1

\textbf{SAGGEZZA} +0

\textbf{CARISMA} +2

\textbf{Iniziativa} +1 -- \textbf{Difesa} 20

\textbf{Punti Ferita} 110 (13d10 + 39)

\textbf{Movimento} 12 m, scavo 6 m, volo 24 m

\textbf{Tiri Salvezza} Tempra +9, Riflessi +8, Volontà +7

\textbf{Competenze} Muoversi Silenziosamente / Nascondersi nelle Ombre +3, Consapevolezza +6, Ingannare +5

\textbf{Immunità al Danno} fuoco

\textbf{Sensi} scurovisione 36 m, vista cieca 9 m

\textbf{Linguaggi} Comune, Draconico

\textbf{Sfida} 6 (2.300 PE)

\textbf{Azioni}

\emph{\textbf{Multiattacco.}} Il drago può effettuare tre attacchi: uno con il morso e due con gli artigli.

\emph{\textbf{Artiglio.} Attacco con arma da mischia}: +7 a colpire, portata 1 m, un bersaglio.

\emph{Colpisce:} 11 (2d6 + 4) danni taglienti.

\emph{\textbf{Morso.} Attacco con arma da mischia}: +7 a colpire, portata 3 m, un bersaglio.

\emph{Colpisce:} 15 (2d10 + 4) danni perforanti.

\emph{\textbf{Arma a Soffio (Ricarica 5-6).}} Il drago usa una delle seguenti armi a soffio:

\emph{Soffio Infuocato.} Il drago esala fuoco in una linea lunga 12 metri e larga 1,5 metri. Ogni creatura sulla linea deve effettuare un tiro salvezza di Riflessi CD 14, subendo 42 (12d6) danni da fuoco se fallisce il tiro salvezza, o la metà di questi danni se lo riesce. \emph{Soffio Soporifero.} Il drago esala del gas soporifero in un cono di 9 metri. Ogni creatura in quell'area deve riuscire un tiro salvezza su Tempra 14 o cadere svenuta per 5 minuti. Questo effetto termina se la creatura svenuta subisce danni o qualcuno impiega un'azione per risvegliarla.

\medskip\index{Mostri - Drago d'Ottone Cucciolo}\textbf{Drago d'Ottone Cucciolo}

\emph{Media drago, caotico buono}

\textbf{FORZA} +2

\textbf{DESTREZZA} +0

\textbf{COSTITUZIONE} +1

\textbf{INTELLIGENZA} +0

\textbf{SAGGEZZA} +0

\textbf{CARISMA} +1

\textbf{Iniziativa} +0 -- \textbf{Difesa} 17

\textbf{Punti Ferita} 16 (3d8 + 3)

\textbf{Movimento} 9 m, scavo 4,5 m, volo 18 m

\textbf{Tiri Salvezza} Tempra +2, Riflessi +0, Volontà +1

\textbf{Competenze} Muoversi Silenziosamente / Nascondersi nelle Ombre +2, Consapevolezza +4

\textbf{Immunità al Danno} fuoco

\textbf{Sensi} scurovisione 18 m, vista cieca 3 m

\textbf{Linguaggi} Draconico

\textbf{Sfida} 1 (200 PE)

\textbf{Azioni}

\emph{\textbf{Morso.} Attacco con arma da mischia}: +4 a colpire, portata 1 m, un bersaglio.

\emph{Colpisce:} 7 (1d10 + 2) danni perforanti.

\emph{\textbf{Arma a Soffio (Ricarica 5-6).}} Il drago usa una delle seguenti armi a soffio:

\emph{Soffio Infuocato.} Il drago esala fuoco in una linea lunga 6 metri e larga 1,5 metri. Ogni creatura sulla linea deve effettuare un tiro salvezza su Riflessi CD 11, subendo 14 (4d6) danni da fuoco se fallisce il tiro salvezza, o la metà di questi danni se lo riesce.

\emph{Soffio Soporifero.} Il drago esala del gas soporifero in un cono di 4,5 metri. Ogni creatura in quell'area deve riuscire un tiro salvezza su Tempra 11 o cadere svenuta per 1 minuto. Questo effetto termina se la creatura svenuta subisce danni o qualcuno impiega un'azione per risvegliarla.

\medskip\index{Mostri - Drago di Rame Antico}\textbf{Drago di Rame Antico}

\emph{Mastodontica drago, caotico buono}

\textbf{FORZA} +8

\textbf{DESTREZZA} +1

\textbf{COSTITUZIONE} +7

\textbf{INTELLIGENZA} +5

\textbf{SAGGEZZA} +3

\textbf{CARISMA} +4

\textbf{Iniziativa} +5 -- \textbf{Difesa} 33

\textbf{Punti Ferita} 350 (20d20 + 140) 

\textbf{Movimento} 12 m, scalata 12 m, volo 24 m

\textbf{Tiri Salvezza} Tempra +20, Riflessi +13, Volontà +19

\textbf{Competenze} Muoversi Silenziosamente / Nascondersi nelle Ombre +8, Ingannare +11, Consapevolezza +17

\textbf{Immunità al Danno} acido, arma +1

\textbf{Sensi} scurovisione 36 m, vista cieca 18 m

\textbf{Linguaggi} Comune, Draconico

\textbf{Sfida} 21 (33.000 PE)

\emph{\textbf{Resistenza Leggendaria (3/Giorno).}} Se il drago fallisce un tiro salvezza, può scegliere invece di riuscire.

\textbf{Azioni}

\emph{\textbf{Multiattacco.}} Il drago può usare la sua Presenza Spaventosa. Poi effettuare tre attacchi: uno con il morso e due con gli artigli.

\emph{\textbf{Artiglio.} Attacco con arma da mischia}: +15 a colpire, portata 3 m, un bersaglio.

\emph{Colpisce:} 15 (2d6 + 8) danni taglienti.

\emph{\textbf{Coda.} Attacco con arma da mischia}: +15 a colpire, portata 6 m, un bersaglio.

\emph{Colpisce:} 17 (2d8 + 8) danni da botta.

\emph{\textbf{Morso.} Attacco con arma da mischia}: +15 a colpire, portata 4,5 m, un bersaglio.

\emph{Colpisce:} 19 (2d10 + 8) danni perforanti.

\emph{\textbf{Presenza Spaventosa.}} Ogni creatura scelta dal drago, che si trovi entro 36 metri da esso e consapevole della sua presenza, deve riuscire un tiro salvezza di Volontà CD 19 o restare spaventata per 1 minuto. Una creatura può ripetere il tiro salvezza al termine di ciascun suo turno, terminando l'effetto se lo riesce. Se il tiro salvezza della creatura ha successo o l'effetto ha termine per essa, la creatura è immune alla Presenza Spaventosa del drago per le successive 24 ore.

\emph{\textbf{Arma a Soffio (Ricarica 5-6).}} Il drago usa una delle seguenti armi a soffio:

\emph{Soffio Acido.} Il drago esala acido in una linea lunga 27 metri e larga 3 metri. Ogni creatura sulla linea deve effettuare un tiro salvezza su Riflessi CD 22, subendo 63 (14d8) danni da acido se fallisce il tiro salvezza, o la metà di questi danni se lo riesce.

\emph{Soffio Rallentante.} Il drago esala del gas in un cono di 27 metri. Ogni creatura in quell'area deve riuscire un tiro salvezza su Tempra CD 22. Se fallisce il tiro salvezza, la creatura non può usare la sua reazione, ha la velocità dimezzata, e non può effettuare più di un attacco durante il suo turno. Inoltre, la creatura può usare un'azione o un'azione bonus, ma non entrambe. Questi effetti permangono 1 minuto. La creatura può ripetere il tiro salvezza al termine di ciascun suo turno, terminando l'effetto su di sé in caso di successo.

\emph{\textbf{Mutare Forma.}} Il drago può trasformarsi magicamente in un umanoide o bestia il cui grado di sfida sia pari o inferiore al proprio, o tornare alla sua vera forma. Alla morte ritorna alla sua vera forma. Qualsiasi equipaggiamento stia indossando o trasportando viene assorbito o trasportato nella nuova forma (a scelta del drago).

Nella nuova forma, il drago mantiene il suo allineamento, punti ferita, Dadi Vita, la facoltà di parlare, le competenze, la Resistenza Leggendaria, le azioni da tana, e i punteggi di Intelligenza, Saggezza e Carisma, oltre a questa azione. Le sue statistiche e capacità

vengono altrimenti rimpiazzate da quelle della nuova forma, eccetto Azioni aggiuntive della nuova forma.

\textbf{Azioni Aggiuntive}

Il drago può effettuare 3 Azioni aggiuntive, scelte tra le opzioni seguenti. Può usare solo un'opzione leggendaria alla volta e solo al termine del turno di un'altra creatura. Il drago recupera le Azioni aggiuntive spese all'inizio del proprio turno.

\textbf{Attacco di Ala (Costa 2 Azioni).} Il drago batte le ali. Ogni creatura entro 4,5 metri dal drago deve riuscire un tiro salvezza su Riflessi CD 23 o subire 15 (2d6 + 8) danni da botta e venir gettato prono. Il drago può poi volare fino a metà del suo movimento di volo.

\textbf{Attacco di Coda.} Il drago effettua un attacco di coda.

\textbf{Individuare.} Il drago effettua una prova di Saggezza (Consapevolezza).

\medskip\index{Mostri - Drago di Rame Adulto}\textbf{Drago di Rame Adulto}

\emph{Enorme drago, caotico buono}

\textbf{FORZA} +6

\textbf{DESTREZZA} +1

\textbf{COSTITUZIONE} +5

\textbf{INTELLIGENZA} +4

\textbf{SAGGEZZA} +2

\textbf{CARISMA} +3

\textbf{Iniziativa} +4 -- \textbf{Difesa} 25

\textbf{Punti Ferita} 184 (16d12 + 80)

\textbf{Movimento} 12 m, scalata 12 m, volo 24 m

\textbf{Tiri Salvezza} Tempra +14, Riflessi +10, Volontà +13

\textbf{Competenze} Muoversi Silenziosamente / Nascondersi nelle Ombre +6, Ingannare +8, Consapevolezza +12

\textbf{Immunità al Danno} acido

\textbf{Sensi} scurovisione 36 m, vista cieca 18 m

\textbf{Linguaggi} Comune, Draconico

\textbf{Sfida} 14 (11.500 PE)

\emph{\textbf{Resistenza Leggendaria (3/Giorno).}} Se il drago fallisce un tiro salvezza, può scegliere invece di riuscire.

\textbf{Azioni}

\emph{\textbf{Multiattacco.}} Il drago può usare la sua Presenza Spaventosa. Poi effettuare tre attacchi: uno con il morso e due con gli artigli.

\emph{\textbf{Artiglio.} Attacco con arma da mischia}: +11 a colpire, portata 1 m, un bersaglio.

\emph{Colpisce:} 13 (2d6 + 6) danni taglienti.

\emph{\textbf{Coda.} Attacco con arma da mischia}: +11 a colpire, portata 4,5 m, un bersaglio.

\emph{Colpisce:} 15 (2d8 + 6) danni da botta.

\emph{\textbf{Morso.} Attacco con arma da mischia}: +11 a colpire, portata 3 m, un bersaglio.

\emph{Colpisce:} 17 (2d10 + 6) danni perforanti.

\emph{\textbf{Presenza Spaventosa.}} Ogni creatura scelta dal drago, che si trovi entro 36 metri da esso e consapevole della sua presenza, deve riuscire un tiro salvezza di Volontà CD 16 o restare spaventata per 1 minuto. Una creatura può ripetere il tiro salvezza al termine di ciascun suo turno, terminando l'effetto se lo riesce. Se il tiro salvezza della creatura ha successo o l'effetto ha termine per essa, la creatura è immune alla Presenza Spaventosa del drago per le successive 24 ore.

\emph{\textbf{Arma a Soffio (Ricarica 5-6).}} Il drago usa una delle seguenti armi a soffio:

\emph{Soffio Acido.} Il drago esala acido in una linea lunga 18 metri e larga 1,5 metri. Ogni creatura sulla linea deve effettuare un tiro salvezza su Riflessi CD 18, subendo 54 (12d8) danni da acido se fallisce il tiro salvezza, o la metà di questi danni se lo riesce.

\emph{Soffio Rallentante.} Il drago esala del gas in un cono di 18 metri. Ogni creatura in quell'area deve riuscire un tiro salvezza su Tempra CD 18. Se fallisce il tiro salvezza, la creatura non può usare la sua reazione, ha la velocità dimezzata, e non può effettuare più di un attacco durante il suo turno. Inoltre, la creatura può usare un'azione o un'azione bonus, ma non entrambe. Questi effetti permangono 1 minuto. La creatura può ripetere il tiro salvezza al termine di ciascun suo turno, terminando l'effetto su di sé in caso di successo.

\textbf{Azioni Aggiuntive}

Il drago può effettuare 3 Azioni aggiuntive, scelte tra le opzioni seguenti. Può usare solo un'opzione leggendaria alla volta e solo al termine del turno di un'altra creatura. Il drago recupera le Azioni aggiuntive spese all'inizio del proprio turno.

\textbf{Attacco di Ala (Costa 2 Azioni).} Il drago batte le ali. Ogni creatura entro 3 metri dal drago deve riuscire un tiro salvezza su Riflessi CD 19 o subire 13 (2d6 + 6) danni da botta e venir gettato prono. Il drago può poi volare fino a metà del suo movimento di volo.

\textbf{Attacco di Coda.} Il drago effettua un attacco di coda. 

\textbf{Individuare.} Il drago effettua una prova di Saggezza (Consapevolezza).

\medskip\index{Mostri - Drago di Rame Giovane}\textbf{Drago di Rame Giovane}

\emph{Grande drago, caotico buono}

\textbf{FORZA} +4

\textbf{DESTREZZA} +1

\textbf{COSTITUZIONE} +3

\textbf{INTELLIGENZA} +3

\textbf{SAGGEZZA} +1

\textbf{CARISMA} +2

\textbf{Iniziativa} +3 -- \textbf{Difesa} 21

\textbf{Punti Ferita} 119 (14d10 + 42)

\textbf{Movimento} 12 m, scalata 12 m, volo 24 m

\textbf{Tiri Salvezza} Tempra +9, Riflessi +8, Volontà +8

\textbf{Competenze} Muoversi Silenziosamente / Nascondersi nelle Ombre +4, Ingannare +5, Consapevolezza +7

\textbf{Immunità al Danno} acido

\textbf{Sensi} scurovisione 36 m, vista cieca 9 m

\textbf{Linguaggi} Comune, Draconico

\textbf{Sfida} 7 (2.900 PE)

\textbf{Azioni}

\emph{\textbf{Multiattacco.}} Il drago può effettuare tre attacchi: uno con il morso e due con gli artigli.

\emph{\textbf{Artiglio.} Attacco con arma da mischia}: +7 a colpire, portata 1 m, un bersaglio.

\emph{Colpisce:} 11 (2d6 + 4) danni taglienti.

\emph{\textbf{Morso.} Attacco con arma da mischia}: +7 a colpire, portata 3 m, un bersaglio.

\emph{Colpisce:} 15 (2d10 + 4) danni perforanti.

\emph{\textbf{Arma a Soffio (Ricarica 5-6).}} Il drago usa una delle seguenti armi a soffio:

\emph{Soffio Acido.} Il drago esala acido in una linea lunga 12 metri e larga 1,5 metri. Ogni creatura sulla linea deve effettuare un tiro salvezza su Riflessi CD 14, subendo 40 (9d8) danni da acido se fallisce il tiro salvezza, o la metà di questi danni se lo riesce.

\emph{Soffio Rallentante.} Il drago esala del gas in un cono di 9 metri. Ogni creatura in quell'area deve riuscire un tiro salvezza su Tempra CD 14. Se fallisce il tiro salvezza, la creatura non può usare la sua reazione, ha la velocità dimezzata, e non può effettuare più di un attacco durante il suo turno. Inoltre, la creatura può usare un'azione o un'azione bonus, ma non entrambe. Questi effetti permangono 1 minuto. La creatura può ripetere il tiro salvezza al termine di ciascun suo turno, terminando l'effetto su di sé in caso di successo.

\textbf{Drago di Rame Cucciolo}

\emph{Media drago, caotico buono}

\textbf{FORZA} +2

\textbf{DESTREZZA} +1

\textbf{COSTITUZIONE} +1

\textbf{INTELLIGENZA} +2

\textbf{SAGGEZZA} +0

\textbf{CARISMA} +1

\textbf{Iniziativa} +2 -- \textbf{Difesa} 17

\textbf{Punti Ferita} 22 (4d8 + 4)

\textbf{Movimento} 9 m, scalata 9 m, volo 18 m

\textbf{Tiri Salvezza} Tempra +2, Riflessi +2, Volontà +0

\textbf{Competenze} Muoversi Silenziosamente / Nascondersi nelle Ombre +3, Consapevolezza +4

\textbf{Immunità al Danno} acido

\textbf{Sensi} scurovisione 18 m, vista cieca 3 m

\textbf{Linguaggi} Draconico

\textbf{Sfida} 1 (200 PE)

\textbf{Azioni}

\emph{\textbf{Morso.} Attacco con arma da mischia}: +4 a colpire, portata 1 m, un bersaglio.

\emph{Colpisce:} 7 (1d10 + 2) danni perforanti.

\emph{\textbf{Arma a Soffio (Ricarica 5-6).}} Il drago usa una delle seguenti armi a soffio:

\emph{Soffio Acido.} Il drago esala acido in una linea lunga 6 metri e larga 1,5 metri. Ogni creatura sulla linea deve effettuare un tiro salvezza su Riflessi CD 11, subendo 18 (4d8) danni da acido se fallisce il tiro salvezza, o la metà di questi danni se lo riesce.

\emph{Soffio Rallentante.} Il drago esala del gas in un cono di 4,5 metri. Ogni creatura in quell'area deve riuscire un tiro salvezza su Tempra CD 11. Se fallisce il tiro salvezza, la creatura non può usare la sua reazione, ha la velocità dimezzata, e non può effettuare più di un attacco durante il suo turno. Inoltre, la creatura può usare un'azione o un'azione bonus, ma non entrambe. Questi effetti permangono 1 minuto. La creatura può ripetere il tiro salvezza al termine di ciascun suo turno, terminando l'effetto su di sé in caso di successo.


\medskip\index{Mostri - Drider}\textbf{Drider}

\emph{Grande mostruosità, caotico malvagio}

\textbf{FORZA} +3

\textbf{DESTREZZA} +3

\textbf{COSTITUZIONE} +4

\textbf{INTELLIGENZA} +1

\textbf{SAGGEZZA} +2

\textbf{CARISMA} +1

\textbf{Iniziativa} +3 -- \textbf{Difesa} 22

\textbf{Punti Ferita} 123 (13d10 + 52)

\textbf{Movimento} 9 m, scalata 9 m

\textbf{Tiri Salvezza} Tempra +7, Riflessi +5, Volontà +9

\textbf{Competenze} Muoversi Silenziosamente / Nascondersi nelle Ombre +9, Consapevolezza +5

\textbf{Sensi} scurovisione 36 m

\textbf{Linguaggi} Elfico, Linguaggio delle Profondità

\textbf{Sfida} 6 (2.300 PE)

\emph{\textbf{Camminare sulla Tela.}} Il drider ignora le restrizioni al movimento provocate dalle ragnatele.

\emph{\textbf{Discendenza Fatata.}} Il drider ha +1d6 ai tiri salvezza per non restare affascinato, e la magia non può far addormentare un drider.

\emph{\textbf{Incantesimi Innati.}} La caratteristica da incantatore innato del drider è la Saggezza. Il drider può lanciare in maniera innata i seguenti incantesimi, senza bisogno di componenti materiali:

A volontà: \emph{luci danzanti}

1/Giorno: \emph{luminescenza, oscurità}

\emph{\textbf{Scalare come Ragno.}} Il drider può scalare superfici difficili, compreso lo stare a testa in giù sul soffitto, senza bisogno di effettuare una prova di abilità.

\textbf{Azioni}

\emph{\textbf{Multiattacco.}} Il drider effettua tre attacchi con la spada lunga o con l'arco lungo. Può rimpiazzare uno di questi attacchi con un attacco di morso.

\emph{\textbf{Morso.} Attacco con arma da mischia}: +6 a colpire, portata 1 m, una creatura.

\emph{Colpisce:} 2 (1d4) danni perforanti più 9 (2d8) danni da veleno.

\emph{\textbf{Spada Lunga.} Attacco con arma da mischia}: +6 a colpire, portata 1 m, un bersaglio.

\emph{Colpisce:} 7 (1d8 + 3) danni taglienti, o 8 (1d8 + 3) danni taglienti se usata con due mani.

\emph{\textbf{Arco Lungo.} Attacco con arma a Distanza}: +6 a colpire, gittata 45m, un bersaglio.

\emph{Colpisce:} 7 (1d8 + 3) danni perforanti più 4 (1d8) danni da veleno.

\medskip\index{Mostri - Driade}\textbf{Driade}

\emph{Media fatato, neutrale}

\textbf{FORZA} +0

\textbf{DESTREZZA} +1

\textbf{COSTITUZIONE} +0

\textbf{INTELLIGENZA} +2

\textbf{SAGGEZZA} +2

\textbf{CARISMA} +4

\textbf{Iniziativa} +2 -- \textbf{Difesa} 12 (17 con \emph{pelle di corteccia})

\textbf{Punti Ferita} 22 (5d8)

\textbf{Vulnerabilità al Danno} ferro freddo

\textbf{Movimento} 9 m

\textbf{Tiri Salvezza} Tempra +5, Riflessi +9, Volontà +7

\textbf{Competenze} Muoversi Silenziosamente / Nascondersi nelle Ombre +5, Consapevolezza +4

\textbf{Sensi} scurovisione 18 m

\textbf{Linguaggi} Elfico, Silvano

\textbf{Sfida} 1 (200 PE)

\emph{\textbf{Camminata Arborea.}} Uno volta durante il suo turno, la driade può usare 3 metri di movimento per entrare magicamente in un albero vivo a sua portata ed emergere da un altro albero vivo entro 18 metri dal primo albero, ricomparendo in uno spazio non occupato entro 1,5 metri dal secondo albero. Entrambi gli alberi devono essere di taglia Grande o superiore.

\emph{\textbf{Incantesimi Innati.}} La caratteristica da incantatore innato della driade è il Carisma (CD 14 per i tiri salvezza degli incantesimi). La driade può lanciare in maniera innata i seguenti incantesimi, senza aver bisogno di componenti materiali. A volontà: 

\emph{arte del druido}

3/giorno ciascuno: \emph{bacche benefiche}, \emph{intralciare} 1/giorno:
\emph{passare senza tracce, pelle coriacea, randello} \emph{incantato}

\emph{\textbf{Parlare con Animali e Piante.}} La driade può comunicare con bestie e piante come se parlassero la stessa lingua.

\emph{\textbf{Resistenza alla Magia.}} La driade ha +1d6 ai tiri salvezza contro incantesimi e altri effetti magici.

\textbf{Azioni}

\emph{\textbf{Randello.} Attacco con arma da mischia}: +2 a colpire (+6 a colpire con \emph{bastone}), portata 1 m, un bersaglio.

\emph{Colpisce:} 2 (1d4) danni da botta, o 8 (1d8 + 4) danni da botta con \emph{bastone}

\emph{\textbf{Fascino Fatato.}} La driade può prendere a bersaglio un umanoide o bestia entro 9 metri da lei e che possa vedere. Se il bersaglio può vedere la driade, deve riuscire un tiro salvezza su Volontà CD 14 o restare affascinato dalla magia. Le creature affascinate considerano la driade un'amica fidata da ascoltare e proteggere. Sebbene il bersaglio non sia sotto il controllo della driade, interpreterà le richieste o le azioni della driade nel modo più favorevole possibile.

Ogni volta che la driade o i suoi alleati arrecano danno al bersaglio, esso può ripetere il tiro salvezza, terminando l'effetto in caso di successo. Altrimenti, l'effetto permane 24 ore o finché la driade muore, si trova su di un piano di esistenza diverso rispetto al bersaglio, o termina l'effetto con un'azione bonus. Se il tiro salvezza del bersaglio riesce, il bersaglio sarà immune al Fascino Fatato della driade per le successive 24 ore.

La driade non può tenere affascinati più di un umanoide o tre bestie alla volta.

\medskip\index{Mostri - Duergar}\textbf{Duergar}

\emph{Media umanoide (nano), legale malvagio}

\textbf{FORZA} +2

\textbf{DESTREZZA} +0

\textbf{COSTITUZIONE} +2

\textbf{INTELLIGENZA} +0

\textbf{SAGGEZZA} +0

\textbf{CARISMA} -1

\textbf{Iniziativa} +2 -- \textbf{Difesa} 17 (armatura di scaglie, scudo)

\textbf{Punti Ferita} 26 (4d8 + 8)

\textbf{Movimento} 7,5 m

\textbf{Tiri Salvezza} Tempra +4, Riflessi +0, Volontà +1

\textbf{Resistenza al Danno} veleno

\textbf{Sensi} scurovisione 36 m

\textbf{Linguaggi} Nanico, Linguaggio delle Profondità 

\textbf{Sfida} 1 (200 PE)

\emph{\textbf{Resilienza Duerga.}} Il duergar ha +1d6 ai tiri salvezza contro veleni, incantesimi e illusioni, oltre al resistere al restare affascinato o paralizzato.

\emph{\textbf{Sensibilità alla Luce}}. Mentre è alla luce del sole, il duergar ha -1d6 ai tiri di attacco, oltre che alle prove di Saggezza (Consapevolezza) basate sulla vista.

\textbf{Azioni}

\emph{\textbf{Ingrandire (Ricarica dopo un 1 ora).}} Per 1 minuto, il duergar aumenta magicamente di taglia, insieme a tutto ciò che sta trasportando o indossando. Mentre è ingrandito, il duergar è di taglia Grande, raddoppia i dadi di danno degli attacchi con armi basate sulla Forza (già incluso negli attacchi), e ha +1d6 alle prove di Forza e ai tiri salvezza di Forza. Se il duergar non ha sufficiente spazio per diventare Grande, ottiene la massima taglia concessa dallo spazio a disposizione.

\emph{\textbf{Piccone da Guerra.} Attacco con arma da mischia}: +4 a colpire, portata 1 m, un bersaglio.

\emph{Colpisce:} 6 (1d8 + 2) danni perforanti, o 11 (2d8 + 2) danni perforanti quando ingrandito.

\emph{\textbf{Giavellotto.} Attacco con arma da mischia o a Distanza}: +4 a colpire, portata 1 m o gittata 9m, un bersaglio. \emph{Colpisce:} 5 (1d6 + 2) danni perforanti o 9 (2d6 + 2) danni
perforanti quando ingrandito.

\emph{\textbf{Invisibilità (Ricarica dopo un 1 ora).}} Il duergar diventa magicamente invisibile al massimo per un'ora (come se stesse mantenendo la concentrazione per un incantesimo) o finché non attacca, lancia un incantesimo, usa Ingrandire o la sua concentrazione viene spezzata. Tutto l'equipaggiamento che il duergar indossa o trasporta diventa invisibile assieme a lui.

\medskip\index{Mostri - Elementale dell'Acqua}\textbf{Elementale dell'Acqua}

\emph{Grande elementale, neutrale}

\textbf{FORZA} +4

\textbf{DESTREZZA} +2

\textbf{COSTITUZIONE} +4

\textbf{INTELLIGENZA} -3

\textbf{SAGGEZZA} +0

\textbf{CARISMA} -1

\textbf{Iniziativa} +4 -- \textbf{Difesa} 17

\textbf{Punti Ferita} 114 (12d10 + 48)

\textbf{Movimento} 9 m, nuoto 27 m

\textbf{Tiri Salvezza} Tempra +9, Riflessi +8, Volontà +2

\textbf{Resistenze al Danno} acido; da botta, perforante e tagliente di attacchi non magici

\textbf{Immunità al Danno} veleno

\textbf{Immunità alle Condizioni} afferrato, avvelenato, intralciato, paralizzato, pietrificato, privo di sensi, prono, sfinimento

\textbf{Sensi} scurovisione 18 m

\textbf{Linguaggi} Aquan

\textbf{Sfida} 5 (1.800 PE)

\emph{\textbf{Congelamento.}} Se l'elementale subisce danno da freddo, gela parzialmente; il suo movimento è ridotto di 6 metri fino al termine del suo prossimo turno.

\emph{\textbf{Forma d'Acqua.}} L'elementale può entrare nello spazio di una creatura ostile e fermarsi lì. Può muoversi attraverso uno spazio stretto fino a 2,5 centimetri senza doversi stringere.

\emph{\textbf{Natura Elementale.}} Un elementale non ha bisogno di aria,
cibo, bevande o sonno.

\textbf{Azioni}

\emph{\textbf{Multiattacco.}} L'elementale effettua due attacchi di
schianto.

\emph{\textbf{Schianto.} Attacco con arma da mischia}: +7 a colpire,
portata 1 m, un bersaglio.

\emph{Colpisce:} 13 (2d8 + 4) danni da botta.

\emph{\textbf{Sommergere (Ricarica 4-6).}} Ogni creatura nello spazio dell'elementale deve effettuare un tiro salvezza di Tempra CD 15. Se lo fallisce, il bersaglio subisce 13 (2d8 + 4) danni da botta. Se è di taglia Grande o inferiore, il bersaglio è anche afferrato (CD 14 per fuggire). Fino al termine dell'afferrare, il bersaglio è intralciato e non può respirare a meno che non sia in grado di respirare acqua. Se il tiro salvezza riesce, il bersaglio viene spinto fuori dallo spazio
dell'elementale.

L'elementale può afferrare una creatura Grande o fino a due Medie o più piccole alla volta. All'inizio di ciascun turno dell'elementale, ogni bersaglio afferrato subisce 13 (2d8 + 4) danni da botta. Una creatura entro 1,5 metri dall'elementale può trascinare fuori da esso una creatura o oggetto, impiegando un'azione per tentare di riuscire una prova di Forza CD 14.


\medskip\index{Mostri - Elementale dell'Aria}\textbf{Elementale dell'Aria}

\emph{Grande elementale, neutrale}

\textbf{FORZA} +2

\textbf{DESTREZZA} +5

\textbf{COSTITUZIONE} +2

\textbf{INTELLIGENZA} -2

\textbf{SAGGEZZA} +0

\textbf{CARISMA} -2

\textbf{Iniziativa} +5 -- \textbf{Difesa} 18

\textbf{Punti Ferita} 90 (12d10 + 24)

\textbf{Movimento} 0 m, volo 27 m (fluttua)

\textbf{Tiri Salvezza} Tempra +9, Riflessi +13, Volontà +2

\textbf{Resistenze al Danno} fulmine, tuono; da botta, perforante e tagliente di attacchi non magici

\textbf{Immunità al Danno} veleno

\textbf{Immunità alle Condizioni} afferrato, avvelenato, intralciato, paralizzato, pietrificato, privo di sensi, prono, sfinimento

\textbf{Sensi} scurovisione 18 m

\textbf{Linguaggi} Auran

\textbf{Sfida} 5 (1.800 PE)

\emph{\textbf{Forma d'Aria.}} L'elementale può entrare nello spazio di una creatura ostile e fermarsi lì. Può muoversi attraverso uno spazio stretto fino a 2,5 centimetri senza doversi stringere.

\emph{\textbf{Natura Elementale.}} Un elementale non ha bisogno di aria, cibo, bevande o sonno.

\textbf{Azioni}

\emph{\textbf{Multiattacco.}} L'elementale effettua due attacchi di schianto.

\emph{\textbf{Schianto.} Attacco con arma da mischia}: +8 a colpire, portata 1 m, un bersaglio.

\emph{Colpisce:} 14 (2d8 + 5) danni da botta.

\emph{\textbf{Turbine (Ricarica 4-6).}} Ogni creatura nello spazio dell'elementale deve effettuare un tiro salvezza di Tempra CD 13. Se lo fallisce, il bersaglio subisce 15 (3d8 + 2) danni da botta e viene scagliato a 6 metri di distanza dall'elementale in una direzione casuale e cadere prono. Se un bersaglio lanciato colpisce un oggetto, come un muro o il pavimento, subisce 3 (1d6) danni da botta per ogni 3 metri per cui è stato lanciato. Se il bersaglio viene lanciato contro un'altra creatura, quella creatura deve riuscire un tiro salvezza di Riflessi CD 13 o subire lo stesso danno e cadere prona.

Se il tiro salvezza riesce, il bersaglio subisce la metà del danno da botta e non viene scagliato via né cade prono.

\medskip\index{Mostri - Elementale del Fuoco}\textbf{Elementale del Fuoco}

\emph{Grande elementale, neutrale}

\textbf{FORZA} +0

\textbf{DESTREZZA} +3

\textbf{COSTITUZIONE} +3

\textbf{INTELLIGENZA} -2

\textbf{SAGGEZZA} +0

\textbf{CARISMA} -2

\textbf{Iniziativa} +3 -- \textbf{Difesa} 16

\textbf{Punti Ferita} 102 (12d10 + 36)

\textbf{Movimento} 15 m

\textbf{Tiri Salvezza} Tempra +8, Riflessi +11, Volontà +4

\textbf{Resistenze al Danno} da botta, perforante e tagliente di attacchi non magici

\textbf{Immunità al Danno} fuoco, veleno

\textbf{Immunità alle Condizioni} afferrato, avvelenato, intralciato, paralizzato, pietrificato, prono, privo di sensi, sfinimento

\textbf{Sensi} scurovisione 18 m

\textbf{Linguaggi} Ignan

\textbf{Sfida} 5 (1.800 PE)

\emph{\textbf{Forma di Fuoco.}} L'elementale può spostarsi attraverso uno spazio fino a 2,5 centimetri di larghezza senza stringersi. Una creatura che entri a contatto o colpisca l'elementale con un attacco da mischia mentre si trova entro 1,5 metri da esso subisce 5 (1d10) danni da fuoco. Inoltre, l'elementale può entrare nello spazio di una creatura ostile e fermarsi lì. La prima volta che entra nello spazio di una creatura in un turno, la creatura subisce 5 (1d10) danni da fuoco e prende fuoco; finché qualcuno non impiega un'azione per spegnere le fiamme, la creatura subirà 5 (1d10) danni da fuoco all'inizio di ciascun proprio turno.

\emph{\textbf{Illuminazione.}} L'elementale emette luce intensa in un raggio di 9 metri e luce fioca per ulteriori 9 metri.

\emph{\textbf{Natura Elementale.}} Un elementale non ha bisogno di aria, cibo, bevande o sonno.

\emph{\textbf{Suscettibilità all'Acqua.}} L'elementale subisce 1 danno da freddo per ogni 1,5 metri che si muove in acqua o per ogni 4 litri d'acqua che gli vengono spruzzati addosso.

\textbf{Azioni}

\emph{\textbf{Multiattacco.}} L'elementale effettua due attacchi di contatto.

\emph{\textbf{Contatto.} Attacco con arma da mischia}: +6 a colpire, portata 1 m, un bersaglio.

\emph{Colpisce:} 10 (2d8 + 5) danni da fuoco. Se il bersaglio è una creatura o un oggetto infiammabile, prende fuoco. Finché una creatura non impiega un'azione per spegnere le fiamme, la creatura subirà 5 (1d10) danni da fuoco all'inizio di ciascun proprio turno.

\medskip\index{Mostri - Elementale della Terra}\textbf{Elementale della Terra}

\emph{Grande elementale, neutrale}

\textbf{FORZA} +5

\textbf{DESTREZZA} -1

\textbf{COSTITUZIONE} +5

\textbf{INTELLIGENZA} -3

\textbf{SAGGEZZA} +0

\textbf{CARISMA} -3

\textbf{Iniziativa} -1 -- \textbf{Difesa} 20

\textbf{Punti Ferita} 126 (12d10 + 60)

\textbf{Movimento} 9 m, scavo 9 m

\textbf{Tiri Salvezza} Tempra +9, Riflessi +1, Volontà +6

\textbf{Vulnerabilità al Danno} tuono

\textbf{Resistenze al Danno} da botta, perforante e tagliente di attacchi non magici

\textbf{Immunità al Danno} veleno

\textbf{Immunità alle Condizioni} avvelenato, paralizzato, pietrificato, prono, privo di sensi, sfinimento,

\textbf{Sensi} percezione tellurica 18 m, scurovisione 18 m

\textbf{Linguaggi} Terran

\textbf{Sfida} 5 (1.800 PE)

\emph{\textbf{Mostro d'Assedio.}} L'elementale infligge danni doppi agli oggetti e le strutture.

\emph{\textbf{Natura Elementale.}} Un elementale non ha bisogno di aria, cibo, bevande o sonno.

\emph{\textbf{Planata Terrestre.}} L'elementale può scavare attraversa la terra e la pietra non magiche e non lavorate. Quando lo fa, l'elementale non disturba il materiale che sposta. 
\textbf{Azioni}

\emph{\textbf{Multiattacco.}} L'elementale effettua due attacchi di schianto.

\emph{\textbf{Schianto.} Attacco con arma da mischia}: +8 a colpire, portata 3 m, un bersaglio.

\emph{Colpisce:} 14 (2d8 + 5) danni da botta.


\medskip\index{Mostri - Ettercap}\textbf{Ettercap}

\emph{Media mostruosità, neutrale malvagio}

\textbf{FORZA} +2

\textbf{DESTREZZA} +2

\textbf{COSTITUZIONE} +1

\textbf{INTELLIGENZA} -2

\textbf{SAGGEZZA} +1

\textbf{CARISMA} 8 (-2)

\textbf{Iniziativa} +2 -- \textbf{Difesa} 14

\textbf{Punti Ferita} 44 (8d8 + 8)

\textbf{Movimento} 9 m, scalata 9 m

\textbf{Tiri Salvezza} Tempra +6, Riflessi +4, Volontà +6

\textbf{Competenze} Muoversi Silenziosamente / Nascondersi nelle Ombre +4, Consapevolezza +3, Sopravvivenza +3

\textbf{Sensi} scurovisione 18 m

\textbf{Linguaggi} -

\textbf{Sfida} 2 (450 PE)

\emph{\textbf{Camminare sulla Tela.}} L'ettercap ignora le restrizioni al movimento provocate dalle ragnatele.

\emph{\textbf{Scalare come Ragno.}} L'ettercap può scalare superfici difficili, compreso lo stare a testa in giù sul soffitto, senza bisogno di effettuare una prova di caratteristica.

\emph{\textbf{Senso della Tela.}} Mentre è in contatto con una ragnatela, l'ettercap sa l'esatta posizione di qualsiasi altra creatura in contatto con la stessa ragnatela.

\textbf{Azioni}

\emph{\textbf{Multiattacco.}} L'ettercap effettua due attacchi: uno con il morso e uno con gli artigli

\emph{\textbf{Artigli.} Attacco con arma da mischia}: +4 a colpire, portata 1 m, un bersaglio.

\emph{Colpisce:} 7 (2d4 + 2) danni taglienti.

\emph{\textbf{Morso.} Attacco con arma da mischia}: +4 a colpire, portata 1 m, un bersaglio.

\emph{Colpisce:} 6 (1d8 + 2) danni perforanti più 4 (1d8) danni da veleno. Il bersaglio deve riuscire un tiro salvezza di Tempra CD 11 o restare avvelenato per 1 minuto. La creatura può ripetere il tiro salvezza al termine di ciascun suo turno, terminando l'effetto se riesce il tiro salvezza.

\emph{\textbf{Ragnatela (Ricarica 5-6).} Attacco con arma a Distanza}: +4 a colpire, gittata 9m, una creatura di taglia Grande o minore. \emph{Colpisce:} La creatura è intralciata dalla ragnatela. Con un'azione, la creatura intralciata può effettuare una prova di Forza CD 11, liberandosi dalla tela se la riesce. L'effetto termina se la tela è distrutta. La tela ha Difesa 10, 5 punti ferita, vulnerabilità ai danni da fuoco, e immunità ai danni da botta, da veleno e psichici.

\medskip\index{Mostri - Ettin}\textbf{Ettin}

\emph{Grande gigante, caotico malvagio}

\textbf{FORZA} +5

\textbf{DESTREZZA} -1

\textbf{COSTITUZIONE} +3

\textbf{INTELLIGENZA} -2

\textbf{SAGGEZZA} +0

\textbf{CARISMA} -1

\textbf{Iniziativa} -1 -- \textbf{Difesa} 14

\textbf{Punti Ferita} 85 (10d10 + 30)

\textbf{Movimento} 12 m

\textbf{Tiri Salvezza} Tempra +9, Riflessi +2, Volontà +5

\textbf{Competenze} Consapevolezza +4

\textbf{Linguaggi} Gigante, Goblinoide

\textbf{Sfida} 4 (1.100 PE)

\emph{\textbf{Due Teste.}} L'ettin ha +1d6 alle prove di Saggezza (Consapevolezza) e sui tiri salvezza contro le condizioni accecato, affascinato, assordato, privo di sensi, spaventato e stordito.

\emph{\textbf{Veglia.}} Quando una delle due teste dell'ettin è addormentata, l'altra è sveglia.

\textbf{Azioni}

\emph{\textbf{Multiattacco.}} L'ettin effettua due attacchi: uno con l'ascia da battaglia e uno con la mazza chiodata.

\emph{\textbf{Ascia da Battaglia.} Attacco con arma da mischia}: +7 a colpire, portata 1 m, un bersaglio.

\emph{Colpisce:} 14 (2d8 + 5) danni taglienti.

\emph{\textbf{Mazza Chiodata.} Attacco con arma da mischia}: +7 a colpire, portata 1 m, un bersaglio.

\emph{Colpisce:} 14 (2d8 + 5) danni perforanti.

\medskip\index{Mostri - Fantasma}\textbf{Fantasma}

\emph{Media non morto, qualsiasi allineamento}

\textbf{FORZA} -2

\textbf{DESTREZZA} +1

\textbf{COSTITUZIONE} +0

\textbf{INTELLIGENZA} +0

\textbf{SAGGEZZA} +1

\textbf{CARISMA} +3

\textbf{Iniziativa} +1 -- \textbf{Difesa} 13

\textbf{Punti Ferita} 45 (10d8)

\textbf{Movimento} 0 m, volo 12 m (fluttua)

\textbf{Tiri Salvezza} Tempra +7, Riflessi +6, Volontà +7

\textbf{Resistenze al Danno} acido, fulmine, fuoco, tuono; da botta, perforante, tagliente di attacchi non magici

\textbf{Immunità ai Danni} freddo, da Vuoto, veleno

\textbf{Immunità alle Condizioni} affascinato, afferrato, avvelenato, intralciato, paralizzato, pietrificato, prono, sfinimento, spaventato

\textbf{Sensi} scurovisione 18 m

\textbf{Linguaggi} qualsiasi lingua conosciuta in vita

\textbf{Sfida} 4 (1.100 PE)

\emph{\textbf{Movimento Incorporeo.}} Il fantasma può attraversare altre creature e oggetti come se fossero terreno difficile. Subisce 5 (1d10) danni da forza se termina il suo turno all'interno di un oggetto. 

\emph{\textbf{Natura Non Morta.}} Il fantasma non ha bisogno di aria, cibo, bevande o di dormire.

\emph{\textbf{Vista Eterea.}} Il fantasma può vedere 18 metri nel Piano Etereo quando si trova sul Piano Materiale, e vice versa.

\textbf{Azioni}

\emph{\textbf{Tocco Avvizzente.} Attacco con arma da mischia}: +5 a colpire, portata 1 m, un bersaglio.

\emph{Colpisce:} 17 (4d6 + 3) danni da Vuoto.

\emph{\textbf{Eterealità.}} Il fantasma entra nel Piano Etereo dal Piano Materiale, o vice versa. È visibile sul Piano Materiale mentre è nel Margine Etereo, e vice versa, ma non può interagire con nulla che si trovi sull'altro piano.

\emph{\textbf{Possessione (Ricarica 6).}} Un umanoide, entro 1,5 metri e visibile al fantasma, deve riuscire un tiro salvezza di Volontà CD 13 o venire posseduto dal fantasma; il fantasma poi scompare, e il bersaglio è inabile e perde il controllo del suo corpo. Il fantasma ora controlla il corpo ma non priva il bersaglio della sua consapevolezza. Il fantasma non può essere bersaglio di attacchi, incantesimi, o altri effetti, eccetto quelli che scacciano i non morti, e mantiene il suo allineamento, Intelligenza, Saggezza, Carisma e immunità all'essere affascinato e spaventato. Per il resto usa altrimenti le statistiche del bersaglio posseduto, ma non accede al sapere e competenze del bersaglio.

La possessione dura finché il corpo scende a 0 punti ferita, il fantasma la termina con un'azione bonus, o il fantasma viene scacciato o espulso da un effetto come l'incantesimo \emph{dissolvi il bene e il male}. Quando la possessione termina, il fantasma riappare in uno spazio non occupato entro 1,5 metri dal corpo. Il bersaglio è immune alla Possessione di questo fantasma per 24 ore dopo aver riuscito il tiro salvezza o al termine della possessione.

\emph{\textbf{Viso Orripilante.}} Ogni creatura che non sia non morta, entro 18 metri dal fantasma e che lo possa vedere, deve riuscire un tiro salvezza di Volontà CD 13 o essere spaventata per 1 minuto. Se il tiro salvezza fallisce di 5 o più, il bersaglio invecchia anche di 1d4 x 10 anni. Un  bersaglio spaventato può ripetere il tiro salvezza al termine di ciascun proprio turno, terminando l'effetto per sé, qualora riuscisse il tiro  salvezza. Se il tiro salvezza del bersaglio riesce e per lui l'effetto ha fine, il bersaglio è immune al Viso Orripilante del fantasma per le successive 24  ore. L'effetto di invecchiamento può essere invertito con l'incantesimo \emph{ristorare superiore}, ma solo se eseguito entro 24 dall'effetto di  invecchiamento.

\medskip\index{Mostri - Fauci Gorgoglianti}\textbf{Fauci Gorgoglianti}

\emph{Media aberrazione, neutrale}

\textbf{FORZA} +0

\textbf{DESTREZZA} -1

\textbf{COSTITUZIONE} +3

\textbf{INTELLIGENZA} -4

\textbf{SAGGEZZA} +0

\textbf{CARISMA} -2

\textbf{Iniziativa} -1 -- \textbf{Difesa} 10

\textbf{Punti Ferita} 67 (9d8 + 27)

\textbf{Movimento} 3 m, nuoto 3 m

\textbf{Tiri Salvezza} Tempra +8, Riflessi +4, Volontà +5

\textbf{Immunità alle Condizioni} prono

\textbf{Sensi} scurovisione 18 m

\textbf{Linguaggi} -

\textbf{Sfida} 2 (450 PE)

\emph{\textbf{Gorgoglio.}} Finché la fauce è in grado di vedere una creatura e non è inabile, pronuncia frasi incoerenti. Ogni creatura che inizi il suo turno entro 6 metri dalla fauce e può udire il suo gorgoglio deve effettuare un tiro salvezza di Volontà CD 10. Se lo fallisce, la creatura non può effettuare reazioni fino all'inizio del suo prossimo turno e tira un d8 per determinare cosa farà durante il proprio turno. Da 1 a 4, la creatura non fa nulla. Con 5 o 6, la creatura non svolge nessun'azione o azione bonus e usa tutto il suo movimento per muoversi in una direzione determinata casualmente. Con 7 o 8, la creatura effettua un attacco da mischia contro una creatura determinata a caso entro la sua portata o non fa nulla se non è in grado di effettuare un simile attacco.

\emph{\textbf{Terreno Aberrante.}} Il terreno in un raggio di 3 metri intorno alla fauce è considerato terreno difficile. Ogni creatura che inizi il suo turno in quell'area deve riuscire un tiro salvezza di Tempra CD 10 o vedere il suo movimento ridotto a 0 fino all'inizio del suo turno successivo.

\textbf{Azioni}

\emph{\textbf{Multiattacco.}} La fauce gorgogliante effettua un attacco di morso e, se può, uno Sputo Accecante.

\emph{\textbf{Morso.} Attacco con arma da mischia}: +2 a colpire, portata 1 m, una creatura.

\emph{Colpisce:} 17 (5d6) danni perforanti. Se il bersaglio è di taglia Media o inferiore, deve riuscire un tiro salvezza di Tempra CD 10 o venir gettato prono. Se il bersaglio viene ucciso da questo danno, viene assorbito dalla fauce.

\emph{\textbf{Sputo Accecante (Ricarica 5-6).}} La fauce sputa un globo chimico ad un punto visibile entro 4,5 metri da essa. Il globo esplode all'impatto in un lampo accecante di luce. Ogni creatura entro 1,5 metri dal lampo deve riuscire un tiro salvezza di Riflessi CD 13 o restare accecata fino al termine del prossimo turno della fauce.



\subsection{Funghi}

\medskip\index{Mostri - Fungo Stridente}\textbf{Fungo Stridente}

\emph{Media pianta, disallineato}

\textbf{FORZA} -5

\textbf{DESTREZZA} -5

\textbf{COSTITUZIONE} +0

\textbf{INTELLIGENZA} -5

\textbf{SAGGEZZA} -4

\textbf{CARISMA} -5

\textbf{Iniziativa} -5 -- \textbf{Difesa} 6

\textbf{Punti Ferita} 13 (3d8)

\textbf{Movimento} 0 m

\textbf{Tiri Salvezza}: Fort -3, Riflessi +3, Will -4

\textbf{Immunità alle Condizioni} accecato, assordato, spaventato

\textbf{Sensi} vista cieca 9 m (cieco oltre questo raggio)

\textbf{Linguaggi} -

\textbf{Sfida} 0 (10 PE)

\emph{\textbf{Falso Aspetto.}} Mentre il fungo stridente rimane immobile, è indistinguibile da un normale fungo.

\textbf{Azioni}

\emph{\textbf{Strillo.}} Quando una luce intensa o una creatura si trova entro 9 metri dal fungo stridente, esso emette un strillo udibile fino a 90 metri di distanza. Il fungo stridente continua a strillare finché la fonte del disturbo non si è portata fuori gittata e per altri 1d4 turni successivi.

\medskip\index{Mostri - Fungo Violetto}\textbf{Fungo Violetto}

\emph{Media pianta, disallineato}

\textbf{FORZA} -4

\textbf{DESTREZZA} -5

\textbf{COSTITUZIONE} +0

\textbf{INTELLIGENZA} -5

\textbf{SAGGEZZA} -4

\textbf{CARISMA} -5

\textbf{Iniziativa} -5 -- \textbf{Difesa} 6

\textbf{Punti Ferita} 18 (4d8)

\textbf{Movimento} 1,5 m

\textbf{Tiri Salvezza}: Fort -3, Ref -3, Will -3

\textbf{Immunità alle Condizioni} accecato, assordato, spaventato 

\textbf{Sensi} vista cieca 9 m (cieco oltre questo raggio)

\textbf{Linguaggi} -

\textbf{Sfida} 1/4 (50 PE)

\emph{\textbf{Falso Aspetto.}} Mentre il fungo violetto rimane immobile, è indistinguibile da un normale fungo.

\textbf{Azioni}

\emph{\textbf{Multiattacco.}} Il fungo effettua 1d4 attacchi con Contatto Putrido.

\emph{\textbf{Contatto Putrido.} Attacco con arma da mischia}: +2 a colpire,  portata 3 m, un bersaglio.

\emph{Colpisce:} 4 (1d8) danni da Vuoto.

\medskip\index{Mostri - Fuoco Fatuo}\textbf{Fuoco Fatuo}

\emph{Minuscola non morto, caotico malvagio}

\textbf{FORZA} -5

\textbf{DESTREZZA} +9

\textbf{COSTITUZIONE} +0

\textbf{INTELLIGENZA} +1

\textbf{SAGGEZZA} +2

\textbf{CARISMA} +0

\textbf{Iniziativa} +9 -- \textbf{Difesa} 20

\textbf{Punti Ferita} 22 (9d4)

\textbf{Movimento} 0 m, volo 15 m (fluttua)

\textbf{Tiri Salvezza}: Tempra +3, Riflessi +12, Volontà +9

\textbf{Immunità ai Danni} fulmine, veleno

\textbf{Resistenze al Danno} acido, freddo, fuoco, da Vuoto, tuono; contendente, perforante e tagliente di attacchi non magici

\textbf{Immunità alle Condizioni} afferrato, avvelenato, intralciato, paralizzato, privo di sensi, prono, sfinimento

\textbf{Sensi} scurovisione 36 m

\textbf{Linguaggi} le lingue che conosceva in vita

\textbf{Sfida} 2 (450 PE)

\emph{\textbf{Consumare Vita.}} Con un'azione bonus, il fuoco fatuo può prendere a bersaglio una creatura che può vedere entro 1,5 metri da esso e che abbia 0 punti ferita e sia ancora in vita. Il bersaglio deve riuscire un tiro salvezza di Tempra CD 10 contro questa magia o morire. Se il bersaglio muore, il fuoco fatuo recupera 10 (3d6) punti ferita.

\emph{\textbf{Effimero.}} Il fuoco fatuo non può indossare né trasportare nulla.

\emph{\textbf{Illuminazione Variabile.}} Il fuoco fatuo promana luce intensa in un raggio da 1,5 a 6 metri e luce fioca per un numero di metri aggiuntivi pari al raggio scelto. Il fuoco fatuo può modificare questo raggio con un'azione bonus.

\emph{\textbf{Movimento Incorporeo.}} Il fuoco fatuo può muoversi attraverso altre creature e oggetti come se fossero terreno difficile. Subisce 5 (1d10) danni da forza se termina il suo turno all'interno di un oggetto.

\emph{\textbf{Natura Non Morta.}} Il fuoco fatuo non ha bisogno di aria, cibo o bevande.

\textbf{Azioni}

\emph{\textbf{Scossa.} Attacco con incantesimo in mischia}: +4 a colpire, portata 1 m, una creatura.

\emph{Colpisce:} 9 (2d8) danni da fulmine.

\emph{\textbf{Invisibilità.}} Il fuoco fatuo e la sua luce diventano magicamente invisibili finché non attacca o usa Consumare Vita, o finché la sua concentrazione non termina (come se si stesse concentrando su di un incantesimo).

\medskip\index{Mostri - Fustigatore}\textbf{Fustigatore}

\emph{Grande mostruosità, neutrale malvagio}

\textbf{FORZA} +4

\textbf{DESTREZZA} -1

\textbf{COSTITUZIONE} +3

\textbf{INTELLIGENZA} -2

\textbf{SAGGEZZA} +3

\textbf{CARISMA} -2

\textbf{Iniziativa} -1 -- \textbf{Difesa} 23

\textbf{Punti Ferita} 93 (11d10 + 33)

\textbf{Movimento} 3 m, scalata 3 m

\textbf{Tiri Salvezza}: Tempra +13, Riflessi +5, Volontà +13

\textbf{Competenze} Muoversi Silenziosamente / Nascondersi nelle Ombre +5, Consapevolezza +6

\textbf{Sensi} scurovisione 18 m

\textbf{Linguaggi} -

\textbf{Sfida} 5 (1.800 PE)

\emph{\textbf{Falso Aspetto.}} Quando il fustigatore rimane immobile, è indistinguibile da una normale formazione rocciosa, come una stalagmite.

\emph{\textbf{Scalare come Ragno.}} Il fustigatore può scalare superfici difficili, compreso lo stare a testa in giù sul soffitto, senza bisogno di effettuare una prova di abilità.

\emph{\textbf{Viticci Afferranti.}} Il fustigatore può avere fino a sei viticci alla volta. Ogni viticcio può essere attaccato (CA 20; 10 punti ferita; immunità ai danni psichici e da veleno). Distruggere un viticcio non infligge danni al fustigatore, che può produrre un viticcio di rimpiazzo nel suo prossimo turno. Un viticcio può essere anche rotto se una creatura effettua un'azione e riesce una prova di Forza CD 15 contro di esso.

\textbf{Azioni}

\emph{\textbf{Multiattacco.}} Il fustigatore può effettuare quattro attacchi con i suoi viticci, usare avvolgere e effettuare un attacco con il morso.

\emph{\textbf{Morso.} Attacco con arma da mischia}: +7 a colpire, portata 1 m, un bersaglio.

\emph{Colpisce:} 22 (4d8 + 4) danni perforanti.

\emph{\textbf{Viticcio.} Attacco con arma da mischia}: +7 a colpire, portata 15 m, una creatura.

\emph{Colpisce:} Il bersaglio è afferrato (CD 15 per fuggire). Fino al termine dell'afferrare, il bersaglio è intralciato e ha -1d6 alle prove di Forza e ai tiri salvezza su Tempra, mentre il fustigatore non può usare lo stesso viticcio contro un altro bersaglio.

\emph{\textbf{Avvolgere.}} Il fustigatore trascina le creature afferrate da lui di 7,5 metri verso di lui.

\medskip\index{Mostri - Gargoyle}\textbf{Gargoyle}

\emph{Media elementale, caotico malvagio}

\textbf{FORZA} +2

\textbf{DESTREZZA} +0

\textbf{COSTITUZIONE} +3

\textbf{INTELLIGENZA} -2

\textbf{SAGGEZZA} +0

\textbf{CARISMA} -2

\textbf{Iniziativa} +0 -- \textbf{Difesa} 16

\textbf{Punti Ferita} 52 (7d8 + 21)

\textbf{Movimento} 9 m, volo 18 m

\textbf{Tiri Salvezza}: Tempra +4, Riflessi +6, Volontà +4

\textbf{Resistenze al Danno} da botta, perforante e tagliente di attacchi non magici o che non siano di adamantio

\textbf{Immunità ai Danni} veleno

\textbf{Immunità alle Condizioni} avvelenato, pietrificato, sfinimento

\textbf{Sensi} scurovisione 18 m

\textbf{Linguaggi} Terran

\textbf{Sfida} 2 (450 PE)

\emph{\textbf{Falso Aspetto.}} Mentre la gargoyle rimane immobile, è indistinguibile da una statua inanimata.

\emph{\textbf{Natura Elementale.}} Una gargoyle non ha bisogno di aria, cibo, bevande o sonno.

\textbf{Azioni}

\emph{\textbf{Multiattacco.}} La gargoyle effettua due attacchi: uno con il morso e uno con gli artigli.

\emph{\textbf{Artigli.} Attacco con arma da mischia}: +4 a colpire, portata 1 m, un bersaglio.

\emph{Colpisce:} 5 (1d6 + 2) danni taglienti.

\emph{\textbf{Morso.} Attacco con arma da mischia}: +4 a colpire, portata 1 m, un bersaglio.

\emph{Colpisce:} 5 (1d6 + 2) danni perforanti.


\subsection{Geni}

\medskip\index{Mostri - Djinni}\textbf{Djinni}

\emph{Grande elementale, caotico buono}

\textbf{FORZA} +5

\textbf{DESTREZZA} +2

\textbf{COSTITUZIONE} +6

\textbf{INTELLIGENZA} +2

\textbf{SAGGEZZA} +3

\textbf{CARISMA} +5

\textbf{Iniziativa} +2 -- \textbf{Difesa} 23

\textbf{Punti Ferita} 161 (14d10 + 84)

\textbf{Movimento} 9 m, volo 27 m

\textbf{Tiri Salvezza} Tempra +4, Riflessi +9, Volontà +7

\textbf{Immunità al Danno} fulmine, tuono

\textbf{Sensi} scurovisione 36 m

\textbf{Linguaggi} Auran

\textbf{Sfida} 11 (7.200 PE)

\emph{\textbf{Decesso Elementale.}} Se il djinni muore, il suo corpo si disintegra in una brezza calda, lasciando dietro di sé solo l'equipaggiamento che il djinni stava indossando o trasportando.

\emph{\textbf{Incantesimi Innati.}} La caratteristica da incantatore innato del djinni è il Carisma 17, +9 a colpire con attacchi da incantesimo). Può lanciare in maniera innata i seguenti incantesimi, senza bisogno di componenti materiali:

A volontà: \emph{individuazione del bene e del male, individuazione del magico, onda tonante}

3/giorno ciascuno: \emph{camminare nel vento, creare cibo e acqua} (può creare vino al posto dell'acqua), \emph{linguaggi}

1/giorno ciascuno: \emph{creazione}, \emph{evoca elementali} (solo elementale dell'aria), \emph{forma gassosa, immagine maggiore}, \emph{invisibilità,} \emph{spostamento planare}

\textbf{Azioni}

\emph{\textbf{Multiattacco.}} Il djinni effettua tre attacchi di
scimitarra.

\emph{\textbf{Scimitarra.} Attacco con arma da mischia}: +9 a colpire, portata 1 m, un bersaglio.

\emph{Colpisce:} 12 (2d6 + 5) danni taglienti più 3 (1d6) danni da fulmine o tuono (a scelta del gin).

\emph{\textbf{Creare Turbine.}} Un cilindro d'aria turbinante di 1,5 metri di raggio e alto 9 metri si forma magicamente in un punto visibile al djinni entro 36 metri da esso. Il turbine resta finché il djinni mantiene la concentrazione (come se si stesse concentrando su di un incantesimo). Qualsiasi creatura salvo il djinni che entri nel turbine deve riuscire un tiro salvezza di Tempra CD 18 o restare intralciata da esso. Il djinni può muovere il turbine di massimo 18 metri con un'azione, e le creature intralciate dal turbine si muovono con esso. Il turbine termina se il djinni lo perde di vista.

Una creatura può usare la sua azione per liberare una creatura intralciata dal turbine, compresa se stessa, riuscendo una prova di Forza CD 18. Se la prova riesce, la creatura non è più intralciata e si sposta nello spazio più vicino all'esterno del turbine.

\medskip\index{Mostri - Efreeti}\textbf{Efreeti}

\emph{Grande elementale, legale malvagio}

\textbf{FORZA} +6

\textbf{DESTREZZA} +1

\textbf{COSTITUZIONE} +7

\textbf{INTELLIGENZA} +3

\textbf{SAGGEZZA} +2

\textbf{CARISMA} +3

\textbf{Iniziativa} +3 -- \textbf{Difesa} 23

\textbf{Punti Ferita} 200 (16d10 + 112)

\textbf{Movimento} 12 m, volo 18 m

\textbf{Tiri Salvezza} Tempra +7, Riflessi +10, Volontà +9

\textbf{Immunità al Danno} fuoco

\textbf{Sensi} scurovisione 36 m

\textbf{Linguaggi} Ignan

\textbf{Sfida} 11 (7.200 PE)

\emph{\textbf{Decesso Elementale.}} Se l'efreeti muore, il suo corpo si disintegra in un lampo di fuoco e uno sbuffo di fumo, lasciando dietro di sé solo l'equipaggiamento che l'efreeti stava indossando o trasportando.

\emph{\textbf{Incantesimi Innati.}} La caratteristica da incantatore innato dell'efreeti è il Carisma, +7 a colpire con attacchi da incantesimo). Può lanciare in maniera innata i seguenti incantesimi, senza bisogno di componenti materiali: 

A volontà: \emph{individuazione del magico}

3/giorno ciascuno: \emph{ingrandire/ridurre, linguaggi}

1/giorno ciascuno: \emph{evoca elementali} (solo elementale del fuoco), \emph{forma gassosa, immagine maggiore}, \emph{invisibilità, muro di fuoco, spostamento planare}

\textbf{Azioni}

\emph{\textbf{Multiattacco.}} L'efreeti effettua due attacchi di scimitarra o usa due volte Scagliare Fiamma.

\emph{\textbf{Scimitarra.} Attacco con arma da mischia}: +10 a colpire, portata 1 m, un bersaglio.

\emph{Colpisce:} 13 (2d6 + 6) danni taglienti più 7 (2d6) danni da fuoco.

\emph{\textbf{Scagliare Fiamma.} Attacco con arma a Distanza}: +7 a colpire, gittata 36 m, un bersaglio.

\emph{Colpisce:} 17 (5d6) danni da fuoco.


\subsection{Ghoul}

\medskip\index{Mostri - Ghast}\textbf{Ghast}

\emph{Media non morto, caotico malvagio}

\textbf{FORZA} +3

\textbf{DESTREZZA} +3

\textbf{COSTITUZIONE} +0

\textbf{INTELLIGENZA} +0

\textbf{SAGGEZZA} +0

\textbf{CARISMA} -1

\textbf{Iniziativa} +3 -- \textbf{Difesa} 14

\textbf{Punti Ferita} 36 (8d8)

\textbf{Movimento} 9 m

\textbf{Tiri Salvezza}: Tempra +2, Riflessi +2, Volontà +5

\textbf{Resistenze al Danno} da Vuoto

\textbf{Immunità al Danno} veleno

\textbf{Immunità alle Condizioni} affascinato, avvelenato, sfinimento

\textbf{Sensi} scurovisione 18 m

\textbf{Linguaggi} Comune

\textbf{Sfida} 2 (450 PE)

\emph{\textbf{Fetore.}} Qualsiasi creatura che inizi il suo turno entro 1,5 metri dal ghast deve riuscire un tiro salvezza di Tempra CD 10 o restare avvelenata fino all'inizio del suo prossimo turno. Se riesce il tiro salvezza, la creatura è immune al Fetore del ghast per le successive 24
ore.

\emph{\textbf{Ribellione allo Scacciare.}} Il ghast e tutti i ghoul entro 9 metri da esso hanno +1d6 ai tiri salvezza contro gli effetti che scacciano i non morti.

\textbf{Azioni}

\emph{\textbf{Artigli.} Attacco con arma da mischia}: +5 a colpire,
portata 1 m, un bersaglio.

\emph{Colpisce:} 10 (2d6 + 3) danni taglienti. Se il bersaglio è una creatura, diversa da un non morto, deve riuscire un tiro salvezza su Tempra CD 10 o restare paralizzata per 1 minuto. Il bersaglio può ripetere il tiro salvezza al termine di ciascun suo turno, terminando l'effetto se riesce il tiro salvezza. 

\emph{\textbf{Morso.} Attacco con arma da mischia}: +3 a colpire, portata 1 m, una creatura.

\emph{Colpisce:} 12 (2d8 + 3) danni perforanti.

\medskip\index{Mostri - Ghoul}\textbf{Ghoul}

\emph{Media non morto, caotico malvagio}

\textbf{FORZA} +1

\textbf{DESTREZZA} +2

\textbf{COSTITUZIONE} +0

\textbf{INTELLIGENZA} -2

\textbf{SAGGEZZA} +0

\textbf{CARISMA} -2

\textbf{Iniziativa} +2 -- \textbf{Difesa} 13

\textbf{Punti Ferita} 22 (5d8)

\textbf{Movimento} 9 m

\textbf{Tiri Salvezza}: Tempra +1, Riflessi +2, Volontà +4

\textbf{Immunità al Danno} veleno

\textbf{Immunità alle Condizioni} affascinato, avvelenato, sfinimento

\textbf{Sensi} scurovisione 18 m

\textbf{Linguaggi} Comune

\textbf{Sfida} 1 (200 PE)

\textbf{Azioni}

\emph{\textbf{Artigli.} Attacco con arma da mischia}: +4 a colpire, portata 1 m, un bersaglio.

\emph{Colpisce:} 7 (2d4 + 2) danni taglienti. Se il bersaglio è una creatura, diversa da un elfo o un non morto, deve riuscire un tiro salvezza su Tempra CD 10 o restare paralizzata per 1 minuto. Il bersaglio può ripetere il tiro salvezza al termine di ciascun suo turno, terminando l'effetto se riesce il tiro salvezza.

\emph{\textbf{Morso.} Attacco con arma da mischia}: +2 a colpire, portata 1 m, una creatura.

\emph{Colpisce:} 9 (2d6 + 2) danni perforanti.



\subsection{Giganti}

\medskip\index{Mostri - Gigante di Collina}\textbf{Gigante di Collina}

\emph{Enorme gigante, caotico malvagio}

\textbf{FORZA} +5

\textbf{DESTREZZA} -1

\textbf{COSTITUZIONE} +4

\textbf{INTELLIGENZA} -3

\textbf{SAGGEZZA} -1

\textbf{CARISMA} -2

\textbf{Iniziativa} -1 -- \textbf{Difesa} 16

\textbf{Punti Ferita} 105 (10d12 + 40)

\textbf{Movimento} 12 m

\textbf{Tiri Salvezza}: Tempra +11, Riflessi +2, Volontà +3

\textbf{Competenze} Consapevolezza +2

\textbf{Linguaggi} Gigante

\textbf{Sfida} 5 (1.800 PE)

\textbf{Azioni}

\emph{\textbf{Multiattacco.}} Il gigante effettua due attacchi con il randello pesante.

\emph{\textbf{Randello Pesante.} Attacco con arma da mischia}: +8 a colpire, portata 3 m, un bersaglio.

\emph{Colpisce:} 18 (3d8 + 5) danni da botta.

\emph{\textbf{Sasso.} Attacco con arma a Distanza}: +8 a colpire, gittata 18m, un bersaglio.

\emph{Colpisce:} 21 (3d10 + 5) danni da botta.

\medskip\index{Mostri - Gigante del Fuoco}\textbf{Gigante del Fuoco}

\emph{Enorme gigante, legale malvagio}

\textbf{FORZA} +7

\textbf{DESTREZZA} -1

\textbf{COSTITUZIONE} +6

\textbf{INTELLIGENZA} +0

\textbf{SAGGEZZA} +2

\textbf{CARISMA} +1

\textbf{Iniziativa} +0 -- \textbf{Difesa} 27 (armatura di piastre)

\textbf{Punti Ferita} 162 (13d12 + 78)

\textbf{Movimento} 9 m

\textbf{Tiri Salvezza}: Tempra +14, Riflessi +4, Volontà +9

\textbf{Competenze} Acrobatica +11, Consapevolezza +6

\textbf{Immunità ai Danni} fuoco

\textbf{Linguaggi} Gigante

\textbf{Sfida} 9 (5.000 PE)

\textbf{Azioni}

\emph{\textbf{Multiattacco.}} Il gigante effettua due attacchi con lo spadone.

\emph{\textbf{Spadone.} Attacco con arma da mischia}: +11 a colpire, portata 3 m, un bersaglio.

\emph{Colpisce:} 28 (6d6 + 7) danni taglienti.

\emph{\textbf{Sasso.} Attacco con arma a Distanza}: +11 a colpire, gittata 18m, un bersaglio.

\emph{Colpisce:} 29 (4d10 + 7) danni da botta.

\medskip\index{Mostri - Gigante del Gelo}\textbf{Gigante del Gelo}

\emph{Enorme gigante, neutrale malvagio}

\textbf{FORZA} +6

\textbf{DESTREZZA} -1

\textbf{COSTITUZIONE} +5

\textbf{INTELLIGENZA} -1

\textbf{SAGGEZZA} +0

\textbf{CARISMA} +1

\textbf{Iniziativa} -1 -- \textbf{Difesa} 19 (armatura composita)

\textbf{Punti Ferita} 138 (12d12 + 60)

\textbf{Movimento} 12 m

\textbf{Tiri Salvezza} Tempra +14, Riflessi +3, Volontà +6

\textbf{Competenze} Acrobatica +9, Consapevolezza +3

\textbf{Immunità ai Danni} freddo

\textbf{Linguaggi} Gigante

\textbf{Sfida} 8 (3.900 PE)

\textbf{Azioni}

\emph{\textbf{Multiattacco.}} Il gigante effettua due attacchi con l'ascia bipenne.

\emph{\textbf{Ascia Bipenne.} Attacco con arma da mischia}: +9 a colpire, portata 3 m, un bersaglio.

\emph{Colpisce:} 25 (3d12 + 6) danni taglienti.

\emph{\textbf{Sasso.} Attacco con arma a Distanza}: +9 a colpire, gittata 18m, un bersaglio.

\emph{Colpisce:} 28 (4d10 + 6) danni da botta.

\medskip\index{Mostri - Gigante delle Nuvole}\textbf{Gigante delle Nuvole}

\emph{Enorme gigante, neutrale buono (50\%) o neutrale malvagio (50\%)}

\textbf{FORZA} +8

\textbf{DESTREZZA} +0

\textbf{COSTITUZIONE} +6

\textbf{INTELLIGENZA} +1

\textbf{SAGGEZZA} +3

\textbf{CARISMA} +3

\textbf{Iniziativa} +1 -- \textbf{Difesa} 19

\textbf{Punti Ferita} 200 (16d12 + 96)

\textbf{Movimento} 12 m

\textbf{Tiri Salvezza} Tempra +16, Riflessi +6, Volontà +10

\textbf{Competenze} Percepire Emozioni +7, Consapevolezza +7

\textbf{Linguaggi} Comune, Gigante

\textbf{Sfida} 9 (5.000 PE)

\emph{\textbf{Incantesimi Innati.}} La caratteristica da incantatore del gigante è il Carisma. Il gigante può lanciare questi incantesimi in maniera innata, senza bisogno di componenti materiali:

A volontà: \emph{individuazione del magico, luce, nube di nebbia}

3/giorno ciascuno: \emph{caduta morbida, passo nebbioso, telecinesi}

1/giorno ciascuno: \emph{controllare tempo atmosferico, forma gassosa}

\emph{\textbf{Olfatto Affinato.}} Il gigante ha +1d6 alle prove di Saggezza (Consapevolezza) basate sull'olfatto.

\textbf{Azioni}

\emph{\textbf{Multiattacco.}} Il gigante effettua due attacchi con la morning star.

\emph{\textbf{Morning star.} Attacco con arma da mischia}: +12 a colpire, portata 3 m, un bersaglio.

\emph{Colpisce:} 21 (3d8 + 8) danni perforanti.

\emph{\textbf{Sasso.} Attacco con arma a Distanza}: +12 a colpire, gittata 18m, un bersaglio.

\emph{Colpisce:} 30 (4d10 + 8) danni da botta.

\medskip\index{Mostri - Gigante di Pietra}\textbf{Gigante di Pietra}

\emph{Enorme gigante, neutrale}

\textbf{FORZA} +6

\textbf{DESTREZZA} +2

\textbf{COSTITUZIONE} +5

\textbf{INTELLIGENZA} +0

\textbf{SAGGEZZA} +1

\textbf{CARISMA} -1

\textbf{Iniziativa} +2 -- \textbf{Difesa} 21

\textbf{Punti Ferita} 126 (11d12 + 55)

\textbf{Movimento} 12 m

\textbf{Tiri Salvezza} Tempra +12, Riflessi +6, Volontà +7

\textbf{Competenze} Acrobatica +12, Consapevolezza +4

\textbf{Sensi} scurovisione 18 m

\textbf{Linguaggi} Gigante

\textbf{Sfida} 7 (2.900 PE)

\emph{\textbf{Mimetismo di Pietra.}} Il gigante ha +1d6 alle prove di Destrezza (Nascondersi nelle ombre) effettuate per nascondersi su terreni rocciosi.

\textbf{Azioni}

\emph{\textbf{Multiattacco.}} Il gigante effettua due attacchi con il randello pesante.

\emph{\textbf{Randello Pesante.} Attacco con arma da mischia}: +9 a colpire, portata 4,5 m, un bersaglio.

\emph{Colpisce:} 19 (3d8 + 6) danni da botta.

\emph{\textbf{Sasso.} Attacco con arma a Distanza}: +9 a colpire, gittata 18m, un bersaglio.

\emph{Colpisce:} 28 (4d10 + 6) danni da botta. Se il bersaglio è una creatura, deve riuscire un tiro salvezza di Tempra CD 17 o cadere prona.

\textbf{Reazioni}

\emph{\textbf{Afferrare Sassi.}} Se un sasso o un simile oggetto viene scagliato al gigante, il gigante può, riuscendo un tiro salvezza su Riflessi CD 10, afferrare il proiettile e non subire danni da botta da esso.

\medskip\index{Mostri - Gigante delle Tempeste}\textbf{Gigante delle Tempeste}

\emph{Enorme gigante, caotico buono}

\textbf{FORZA} +9

\textbf{DESTREZZA} +2

\textbf{COSTITUZIONE} +5

\textbf{INTELLIGENZA} +3

\textbf{SAGGEZZA} +4

\textbf{CARISMA} +4

\textbf{Iniziativa} +3 -- \textbf{Difesa} 23 (armatura di scaglie)

\textbf{Punti Ferita} 230 (20d12 + 100)

\textbf{Movimento} 15 m, nuoto 15 m

\textbf{Tiri Salvezza} Tempra +17, Riflessi +8, Volontà +13

\textbf{Competenze} Arcano +8, Acrobatica +14, Consapevolezza +9, Storia +8

\textbf{Resistenze al Danno} freddo

\textbf{Immunità al Danno} fulmine, tuono

\textbf{Linguaggi} Comune, Gigante

\textbf{Sfida} 13 (10.000 PE)

\emph{\textbf{Anfibio.}} Il gigante può respirare aria e acqua.

\emph{\textbf{Incantesimi Innati.}} La caratteristica da incantatore del gigante è il Carisma. Il gigante può lanciare questi incantesimi in maniera innata, senza bisogno di componenti materiali:

A volontà: \emph{caduta controllata, individuazione del magico,} \emph{levitazione, luce}

3/giorno ciascuno: \emph{controllare tempo atmosferico, respirare} \emph{sott'acqua}

\textbf{Azioni}

\emph{\textbf{Multiattacco.}} Il gigante effettua due attacchi con lo spadone.

\emph{\textbf{Spadone.} Attacco con arma da mischia}: +14 a colpire, portata 3 m, un bersaglio.

\emph{Colpisce:} 30 (6d6 + 9) danni taglienti.

\emph{\textbf{Sasso.} Attacco con arma a Distanza}: +14 a colpire, gittata 18m, un bersaglio.

\emph{Colpisce:} 35 (4d12 + 9) danni da botta.

\emph{\textbf{Colpo Fulminante (Ricarica 5-6).}} Il gigante scaglia una folgore magica ad un punto visibile entro 150 metri da sé. Ogni creatura entro 3 metri da quel punto deve effettuare un tiro salvezza su Riflessi CD 17, subendo 54 (12d8) danni da fulmine se lo fallisce, o la metà se lo supera.


\medskip\index{Mostri - Gnoll}\textbf{Gnoll}

\emph{Media umanoide (gnoll), caotico malvagio}

\textbf{FORZA} +2

\textbf{DESTREZZA} +1

\textbf{COSTITUZIONE} +0

\textbf{INTELLIGENZA} -2

\textbf{SAGGEZZA} +0

\textbf{CARISMA} -2

\textbf{Iniziativa} +1 -- \textbf{Difesa} 16 (armatura di pelle, scudo)

\textbf{Punti Ferita} 22 (5d8)

\textbf{Movimento} 9 m

\textbf{Tiri Salvezza}: Tempra +4, Riflessi +0, Volontà +0

\textbf{Sensi} scurovisione 18 m

\textbf{Linguaggi} Gnoll

\textbf{Sfida} 1/2 (100 PE)

\emph{\textbf{Rabbia.}} Quando lo gnoll riduce una creatura a 0 punti ferita con un attacco da mischia durante il proprio turno, può svolgere un'azione bonus per muoversi fino a metà del suo movimento ed effettuare un attacco di morso.

\textbf{Azioni}

\emph{\textbf{Morso.} Attacco con arma da mischia}: +4 a colpire, portata 1 m, una creatura.

\emph{Colpisce:} 4 (1d4 + 2) danni perforanti.

\emph{\textbf{Lancia.} Attacco con arma da mischia o a Distanza}: +4 a colpire, portata 1 m o gittata 6 m, un bersaglio.

\emph{Colpisce:} 5 (1d6 + 2) danni perforanti o 6 (1d8 + 2) danni perforanti se usata con due mani per effettuare un attacco da mischia.

\emph{\textbf{Arco Lungo.} Attacco con arma a Distanza}: +3 a colpire, gittata 45m, un bersaglio.

\emph{Colpisce:} 5 (1d8 + 1) danni perforanti.

\medskip\index{Mostri - Gnomo delle Profondità (Svirfneblin)}\textbf{Gnomo delle Profondità (Svirfneblin)}

\emph{Piccola umanoide (gnomo), neutrale buono}

\textbf{FORZA} +2

\textbf{DESTREZZA} +2

\textbf{COSTITUZIONE} +2

\textbf{INTELLIGENZA} +1

\textbf{SAGGEZZA} +0

\textbf{CARISMA} -1

\textbf{Iniziativa} +2 -- \textbf{Difesa} 16 (giaco di maglia)

\textbf{Punti Ferita} 16 (3d6 + 6)

\textbf{Movimento} 6 m

\textbf{Tiri Salvezza}: Tempra +6, Riflessi +6, Volontà +2

\textbf{Competenze} Muoversi Silenziosamente / Nascondersi nelle Ombre +4, Consapevolezza +2

\textbf{Sensi} scurovisione 36 m

\textbf{Linguaggi} Gnomica, Linguaggio delle Profondità, Terran

\textbf{Sfida} 1/2 (100 PE)

\emph{\textbf{Astuzia Gnomesca.}} Lo gnomo ha +1d6 ai tiri salvezza contro la magia.

\emph{\textbf{Camuffamento di Pietra.}} Lo gnomo ha +1d6 alle prove di Destrezza (Nascondersi nelle ombre) effettuate per nascondersi su terreni rocciosi.

\emph{\textbf{Incantesimi Innati.}} La caratteristica da incantatore innato dello gnomo è l'Intelligenza. Lo gnomo può lanciare questi incantesimi in maniera innata, senza bisogno di componenti:

A volontà: \emph{anti-individuazione} (personale)

1/giorno ciascuno: \emph{camuffare sé stesso, cecità/sordità, sfocatura}

\textbf{Azioni}

\emph{\textbf{Piccone da Guerra.} Attacco con arma da mischia}: +4 a colpire, portata 1 m, un bersaglio.

\emph{Colpisce:} 6 (1d8 + 2) danni perforanti.

\emph{\textbf{Dardo Avvelenato.} Attacco con arma a Distanza}: +4 a colpire, gittata 9m, un bersaglio.

\emph{Colpisce:} 4 (1d4 + 2) danni perforanti, e il bersaglio deve riuscire un tiro salvezza di Tempra CD 12 o restare avvelenato per 1 minuto. Il bersaglio può ripetere il tiro salvezza al termine di ciascun suo turno, terminando l'effetto su di sé in caso di successo.

\subsection{Golem}

\medskip\index{Mostri - Golem di Argilla}\textbf{Golem di Argilla}

\emph{Grande costrutto, disallineato}

\textbf{FORZA} +5

\textbf{DESTREZZA} -1

\textbf{COSTITUZIONE} +4

\textbf{INTELLIGENZA} -4

\textbf{SAGGEZZA} -1

\textbf{CARISMA} -5

\textbf{Iniziativa} -1 -- \textbf{Difesa} 19

\textbf{Punti Ferita} 133 (14d10 + 56)

\textbf{Movimento} 6 m

\textbf{Tiri Salvezza}: Tempra +4, Riflessi +3, Volontà +4

\textbf{Immunità al Danno} acido, psichico, veleno; da botta, perforante e tagliente di attacchi non magici o che non siano di adamantio

\textbf{Immunità alle Condizioni} affascinato, avvelenato, paralizzato, pietrificato, sfinimento, spaventato

\textbf{Sensi} scurovisione 18 m

\textbf{Linguaggi} comprende le lingue del suo creatore ma non può parlare

\textbf{Sfida} 9 (5.000 PE)

\emph{\textbf{Berserk.}} Ogni volta che il golem inizia il suo turno con 60 punti ferita o meno, tira un d6. Se ottieni 6, il golem va in berserk. Durante ogni suo turno mentre è in berserk, il golem attacca la creatura più vicina che può vedere. Se non c'è nessuna creatura abbastanza vicina da muoversi e attaccarla, il golem attacca un oggetto, con preferenza per gli oggetti più piccoli di lui. Una volta che il golem è andato in berserk, continuerà ad esserlo finché non viene distrutto o recupera tutti i suoi punti ferita.

\emph{\textbf{Armi Magiche.}} Gli attacchi con armi del golem sono magici.

\emph{\textbf{Assorbimento dell'Acido.}} Ogni volta che il golem è vittima di danni da acido, non subisce danni ma invece recupera un pari numero di punti ferita. 

\emph{\textbf{Forma Immutabile.}} Il golem è immune a qualsiasi incantesimo o effetto che altererebbe la sua forma.

\emph{\textbf{Natura di Costrutto.}} Un golem non ha bisogno di aria, cibo, bevande o sonno.

\emph{\textbf{Resistenza alla Magia.}} Il golem ha +1d6 ai tiri salvezza contro incantesimi e altri effetti magici.

\textbf{Azioni}

\emph{\textbf{Multiattacco.}} Il golem effettua due attacchi di schianto.

\emph{\textbf{Schianto.} Attacco con arma da mischia}: +8 a colpire, portata 1 m, un bersaglio.

\emph{Colpisce:} 16 (2d10 + 5) danni da botta. Se il bersaglio è una creatura, deve riuscire un tiro salvezza di Tempra CD 15 o vedere i suoi punti ferita massimi ridotti di un ammontare pari al danno subito. Il bersaglio muore se l'attacco riduce i suoi punti ferita massimi a 0. La riduzione resta finché non viene rimossa dall'incantesimo \emph{ristorare superiore} o altra magia.

\emph{\textbf{Velocità (Ricarica 5-6).}} Fino al termine del suo prossimo turno, il golem ottiene un bonus magico di +2 alla Difesa, ha +1d6 ai tiri salvezza su Riflessi, e può usare gli attacchi di schianto come azione bonus.

\medskip\index{Mostri - Golem di Carne}\textbf{Golem di Carne}

\emph{Media costrutto, neutrale}

\textbf{FORZA} +4

\textbf{DESTREZZA} -1

\textbf{COSTITUZIONE} +4

\textbf{INTELLIGENZA} -2

\textbf{SAGGEZZA} +0

\textbf{CARISMA} -3

\textbf{Iniziativa} -1 -- \textbf{Difesa} 12

\textbf{Punti Ferita} 93 (11d8 + 44)

\textbf{Movimento} 9 m

\textbf{Tiri Salvezza}: Tempra +3, Riflessi +2, Volontà +3

\textbf{Immunità al Danno} fulmine, veleno; da botta, perforante e tagliente di attacchi non magici o che non siano di adamantio

\textbf{Immunità alle Condizioni} affascinato, avvelenato, paralizzato, pietrificato, sfinimento, spaventato

\textbf{Sensi} scurovisione 18 m

\textbf{Linguaggi} comprende le lingue del suo creatore ma non può
parlare

\textbf{Sfida} 5 (1.800 PE)

\emph{\textbf{Berserk.}} Ogni volta che il golem inizia il suo turno con 40 punti ferita o meno, tira un d6. Se ottieni 6, il golem va in berserk. Durante ogni suo turno mentre è in berserk, il golem attacca la creatura più vicina che possa vedere. Se non c'è nessuna creatura abbastanza vicina da muoversi e attaccarla, il golem attacca un oggetto, con preferenza per gli oggetti più piccoli di lui. Una volta che il golem è andato in berserk, continuerà ad esserlo finché non viene distrutto o recupera tutti i suoi punti ferita.

\emph{\textbf{Armi Magiche.}} Gli attacchi con armi del golem sono magici.

\emph{\textbf{Assorbimento dei Fulmini.}} Ogni volta che il golem sia vittima di un danno da fulmine, non subisce danni ma invece recupera un pari numero di punti ferita.

\emph{\textbf{Avversione al Fuoco.}} Se il golem subisce danni da fuoco, ha -1d6 ai tiri di attacco e le prove di abilità fino alla fine del suo prossimo turno.

\emph{\textbf{Forma Immutabile.}} Il golem è immune a qualsiasi incantesimo o effetto che altererebbe la sua forma.

\emph{\textbf{Natura di Costrutto.}} Un golem non ha bisogno di aria, cibo, bevande o sonno.

\emph{\textbf{Resistenza alla Magia.}} Il golem ha +1d6 ai tiri salvezza contro incantesimi e altri effetti magici.

\textbf{Azioni}

\emph{\textbf{Multiattacco.}} Il golem effettua due attacchi di
schianto.

\emph{\textbf{Schianto.} Attacco con arma da mischia}: +7 a colpire,
portata 1 m, un bersaglio.

\emph{Colpisce:} 13 (2d8 + 4) danni da botta.

\medskip\index{Mostri - Golem di Ferro}\textbf{Golem di Ferro}

\emph{Grande costrutto, disallineato}

\textbf{FORZA} +7

\textbf{DESTREZZA} -1

\textbf{COSTITUZIONE} +5

\textbf{INTELLIGENZA} -4

\textbf{SAGGEZZA} +0

\textbf{CARISMA} -5

\textbf{Iniziativa} -1 -- \textbf{Difesa} 28

\textbf{Punti Ferita} 210 (20d10 + 100)

\textbf{Movimento} 9 m

\textbf{Tiri Salvezza}: Tempra +6, Riflessi +5, Volontà +6

\textbf{Immunità al Danno} fuoco, psichico, veleno; da botta, perforante e tagliente di attacchi non magici o che non siano di adamantio

\textbf{Immunità alle Condizioni} affascinato, avvelenato, paralizzato, pietrificato, sfinimento, spaventato

\textbf{Sensi} scurovisione 36 m

\textbf{Linguaggi} comprende le lingue del suo creatore ma non può parlare

\textbf{Sfida} 16 (15.000 PE)

\emph{\textbf{Armi Magiche.}} Gli attacchi con armi del golem sono magici.

\emph{\textbf{Assorbimento del Fuoco.}} Ogni volta che il golem sia vittima di un danno da fuoco, non subisce danni ma invece recupera un pari numero di punti ferita.

\emph{\textbf{Forma Immutabile.}} Il golem è immune a qualsiasi incantesimo o effetto che altererebbe la sua forma.

\emph{\textbf{Natura di Costrutto.}} Un golem non ha bisogno di aria, cibo, bevande o sonno.

\emph{\textbf{Resistenza alla Magia.}} Il golem ha +1d6 ai tiri salvezza contro incantesimi e altri effetti magici.

\textbf{Azioni}

\emph{\textbf{Multiattacco.}} Il golem effettua due attacchi da mischia.

\emph{\textbf{Schianto.} Attacco con arma da mischia}: +13 a colpire, portata 1 m, un bersaglio.

\emph{Colpisce:} 20 (3d8 + 7) danni da botta.

\emph{\textbf{Spada.} Attacco con arma da mischia}: +13 a colpire, portata 3 m, un bersaglio.

\emph{Colpisce:} 23 (3d10 + 7) danni taglienti.

\emph{\textbf{Soffio Velenoso (Ricarica 6).}} Il golem esala un gas velenoso in un cono di 4,5 metri. Ogni creatura in quell'area deve effettuare un tiro salvezza di Tempra CD 19, subendo 45 (10d8) danni da veleno se fallisce il tiro salvezza, o la metà di questi danni se lo riesce.

\medskip\index{Mostri - Golem di Pietra}\textbf{Golem di Pietra}

\emph{Grande costrutto, disallineato}

\textbf{FORZA} +6

\textbf{DESTREZZA} -1

\textbf{COSTITUZIONE} +5

\textbf{INTELLIGENZA} -4

\textbf{SAGGEZZA} +0

\textbf{CARISMA} -5

\textbf{Iniziativa} -1 -- \textbf{Difesa} 22

\textbf{Punti Ferita} 178 (17d10 + 85)

\textbf{Movimento} 9 m

\textbf{Tiri Salvezza}: Tempra +4, Riflessi +3, Volontà +4

\textbf{Immunità al Danno} psichico, veleno; da botta, perforante e tagliente di attacchi non magici o che non siano di adamantio

\textbf{Immunità alle Condizioni} affascinato, avvelenato, paralizzato, pietrificato, sfinimento, spaventato

\textbf{Sensi} scurovisione 36 m

\textbf{Linguaggi} comprende le lingue del suo creatore ma non può parlare

\textbf{Sfida} 10 (5.900 PE)

\emph{\textbf{Armi Magiche.}} Gli attacchi con armi del golem sono magici.

\emph{\textbf{Forma Immutabile.}} Il golem è immune a qualsiasi incantesimo o effetto che altererebbe la sua forma.

\emph{\textbf{Natura di Costrutto.}} Un golem non ha bisogno di aria, cibo, bevande o sonno.

\emph{\textbf{Resistenza alla Magia.}} Il golem ha +1d6 ai tiri salvezza contro incantesimi e altri effetti magici.

\textbf{Azioni}

\emph{\textbf{Multiattacco.}} Il golem effettua due attacchi di schianto.

\emph{\textbf{Schianto.} Attacco con arma da mischia}: +10 a colpire, portata 1 m, un bersaglio.

\emph{Colpisce:} 19 (3d8 + 6) danni da botta.

\emph{\textbf{Lentezza (Ricarica 5-6).}} Il golem prende a bersaglio una o più creature entro 3 metri da lui e che possa vedere. Ciascun bersaglio deve effettuare un tiro salvezza di Volontà CD 17 contro questa magia. Se fallisce il tiro salvezza, il bersaglio non può usare reazioni, ha la velocità dimezzata, e durante il proprio turno non può effettuare più di un attacco. Inoltre, durante il proprio turno il bersaglio può effettuare un'azione o un'azione bonus, ma non entrambe. Questi effetti durano per 1 minuto. Il bersaglio può ripetere il tiro salvezza al termine di ciascun suo turno, terminando l'effetto per sé, in caso di successo.

\medskip\index{Mostri - Gorgone}\textbf{Gorgone}

\emph{Grande mostruosità, disallineato}

\textbf{FORZA} +5

\textbf{DESTREZZA} +0

\textbf{COSTITUZIONE} +4

\textbf{INTELLIGENZA} -4

\textbf{SAGGEZZA} +1

\textbf{CARISMA} -2

\textbf{Iniziativa} +0 -- \textbf{Difesa} 22

\textbf{Punti Ferita} 114 (12d10 + 48)

\textbf{Movimento} 12 m

\textbf{Tiri Salvezza}: Tempra +13, Riflessi +6, Volontà +7

\textbf{Competenze} Consapevolezza +4

\textbf{Immunità alle Condizioni} Pietrificato

\textbf{Sensi} scurovisione 18 m

\textbf{Linguaggi} -

\textbf{Sfida} 5 (1.800 PE)

\emph{\textbf{Carica Travolgente.}} Se la gorgone si muove di almeno 6 metri in linea retta verso il bersaglio e lo colpisce con un attacco di incornata durante lo stesso turno, il bersaglio deve riuscire un tiro salvezza su Tempra CD 16 o cadere prono. Se il bersaglio è prono, la gorgone può effettuare un attacco di zoccoli contro di lui come azione bonus.

\textbf{Azioni}

\emph{\textbf{Incornata.} Attacco con arma da mischia}: +8 a colpire, portata 1 m, un bersaglio.

\emph{Colpisce:} 18 (2d12 + 5) danni perforanti.

\emph{\textbf{Zoccoli.} Attacco con arma da mischia}: +8 a colpire, portata 1 m, un bersaglio.

\emph{Colpisce:} 16 (2d10 + 5) danni da botta.

\emph{\textbf{Soffio Pietrificante (Ricarica 5-6).}} La gorgone esala un gas pietrificante in un cono di 9 metri. Ogni creatura in quell'area deve riuscire un tiro salvezza di Tempra CD 13. Se il tiro salvezza fallisce, il bersaglio inizia a trasformarsi in pietra ed è intralciato. Il bersaglio intralciato deve ripetere il tiro salvezza al termine del suo prossimo turno. Se lo riesce, l'effetto sul bersaglio ha termine. Se lo fallisce, il bersaglio è pietrificato finché non viene liberato dall'incantesimo \emph{ripristino superiore} o simile magia.

\medskip\index{Mostri - Grick}\textbf{Grick}

\emph{Media mostruosità, neutrale}

\textbf{FORZA} +2

\textbf{DESTREZZA} +2

\textbf{COSTITUZIONE} +0

\textbf{INTELLIGENZA} -4

\textbf{SAGGEZZA} +2

\textbf{CARISMA} -3

\textbf{Iniziativa} +2 -- \textbf{Difesa} 15

\textbf{Punti Ferita} 27 (6d8)

\textbf{Movimento} 9 m, scalata 9 m

\textbf{Tiri Salvezza}: Tempra +3, Riflessi +3, Volontà +2

\textbf{Resistenza al Danno} da botta, perforante e tagliente di attacchi non magici

\textbf{Sensi} scurovisione 18 m

\textbf{Linguaggi} -

\textbf{Sfida} 2 (450 PE)

\emph{\textbf{Camuffamento di Pietra.}} Il grick ha +1d6 alle prove di Destrezza (Nascondersi nelle ombre) effettuate per nascondersi su terreno roccioso.

\textbf{Azioni}

\emph{\textbf{Multiattacco.}} Il grick effettua un attacco con i suoi tentacoli. Se l'attacco colpisce, il grick può effettuare un attacco di becco contro lo stesso bersaglio.

\emph{\textbf{Tentacoli.} Attacco con arma da mischia}: +4 a colpire, portata 1 m, un bersaglio.

\emph{Colpisce:} 9 (2d6 + 2) danni taglienti.

\emph{\textbf{Becco.} Attacco con arma da mischia}: +4 a colpire,
portata 1 m, un bersaglio.

\emph{Colpisce:} 5 (1d6 + 2) danni perforanti.

\medskip\index{Mostri - Grifone}\textbf{Grifone}

\emph{Grande mostruosità, disallineato}

\textbf{FORZA} +4

\textbf{DESTREZZA} +2

\textbf{COSTITUZIONE} +3

\textbf{INTELLIGENZA} -4

\textbf{SAGGEZZA} +1

\textbf{CARISMA} -1

\textbf{Iniziativa} +2 -- \textbf{Difesa} 13

\textbf{Punti Ferita} 59 (7d10 + 21)

\textbf{Movimento} 9 m, volo 24 m

\textbf{Tiri Salvezza}: Tempra +7, Riflessi +6, Volontà +4

\textbf{Competenze} Consapevolezza +5

\textbf{Sensi} scurovisione 18 m

\textbf{Linguaggi} -

\textbf{Sfida} 2 (450 PE)

\emph{\textbf{Vista Affinata.}} Il grifone ha +1d6 nelle prove di Saggezza (Consapevolezza) basate sulla vista.

\textbf{Azioni}

\emph{\textbf{Multiattacco.}} Il grifone effettua due attacchi: uno con il becco e uno con gli artigli.

\emph{\textbf{Artigli.} Attacco con arma da mischia}: +6 a colpire, portata 1 m, un bersaglio.

\emph{Colpisce:} 11 (2d6 + 4) danni taglienti.

\emph{\textbf{Becco.} Attacco con arma da mischia}: +6 a colpire, portata 1 m, un bersaglio.

\emph{Colpisce:} 8 (1d8 + 4) danni perforanti.

\medskip\index{Mostri - Grimlock}\textbf{Grimlock}

\emph{Media umanoide (grimlock), neutrale malvagio}

\textbf{FORZA} +3

\textbf{DESTREZZA} +1

\textbf{COSTITUZIONE} +1

\textbf{INTELLIGENZA} -1

\textbf{SAGGEZZA} -1

\textbf{CARISMA} -2

\textbf{Iniziativa} +1 -- \textbf{Difesa} 12

\textbf{Punti Ferita} 11 (2d8 + 2)

\textbf{Movimento} 9 m

\textbf{Tiri Salvezza}: Tempra +3, Riflessi +1, Volontà +0

\textbf{Competenze} Acrobatica +5, Muoversi Silenziosamente / Nascondersi nelle Ombre +3, Consapevolezza +3

\textbf{Immunità alle Condizioni} accecato

\textbf{Sensi} vista cieca 9 m o 3 m se assordato (cieco oltre questo raggio) 

\textbf{Linguaggi} Linguaggio delle Profondità

\textbf{Sfida} 1/4 (50 PE)

\emph{\textbf{Camuffamento di Pietra.}} Il grimlock ha +1d6 alle prove di Destrezza (Nascondersi nelle ombre) effettuate per nascondere su terreni rocciosi.

\emph{\textbf{Sensi Ciechi.}} Il grimlock non può usare la vista cieca mentre è assordato e non più fiutare.

\emph{\textbf{Olfatto e Udito Affinati.}} Il grimlock ha +1d6 alle prove di Saggezza (Consapevolezza) basate su udito o olfatto. 

\textbf{Azioni}

\emph{\textbf{Randello d'Osso Appuntito.} Attacco con arma da mischia}:
+5 a colpire, portata 1 m, un bersaglio.

\emph{Colpisce:} 5 (1d4 + 3) danni da botta più 2 (1d4) danni perforanti.

\emph{\textbf{Arco Lungo.} Attacco con arma a Distanza}: +3 a colpire, gittata 45m, un bersaglio.

\emph{Colpisce:} 5 (1d8 + 1) danni perforanti.

\medskip\index{Mostri - Guardiano Protettore}\textbf{Guardiano Protettore}

\emph{Grande costrutto, disallineato}

\textbf{FORZA} +4

\textbf{DESTREZZA} -1

\textbf{COSTITUZIONE} +4

\textbf{INTELLIGENZA} -2

\textbf{SAGGEZZA} +0

\textbf{CARISMA} -4

\textbf{Iniziativa} -1 -- \textbf{Difesa} 21

\textbf{Punti Ferita} 142 (15d10 + 60)

\textbf{Movimento} 9 m

\textbf{Tiri Salvezza}: Tempra +6, Riflessi +1, Volontà +2

\textbf{Immunità al Danno} veleno

\textbf{Immunità alle Condizioni} affascinato, avvelenato, paralizzato, sfinimento, spaventato

\textbf{Sensi} scurovisione 18 m, vista cieca 3 m

\textbf{Linguaggi} comprende i comandi forniti in qualsiasi lingua ma non può parlare

\textbf{Sfida} 7 (2.900 PE)

\emph{\textbf{Accumulare Incantesimi.}} Un incantatore che indossi l'amuleto del guardiano protettore può far sì che il guardiano accumuli un incantesimo di Difficoltà 18 o più basso. Per farlo, l'incantatore deve lanciare l'incantesimo sul guardiano. L'incantesimo non ha effetto ma viene accumulato all'interno del guardiano. Quando gli viene comandato di farlo da chi indossa l'amuleto o si presenta una situazione predeterminata dall'incantatore, il guardiano lancia l'incantesimo accumulato con tutti i parametri predisposti dall'incantatore originale, senza bisogno di componenti. Quando l'incantesimo viene lanciato o qualsiasi nuovo incantesimo viene accumulato, tutti gli incantesimi precedentemente accumulati vengono persi.

\emph{\textbf{Natura di Costrutto.}} Il guardiano non ha bisogno di aria, cibo, bevande o sonno.

\emph{\textbf{Rigenerazione.}} Il guardiano protettore recupera 10 punti ferita all'inizio del proprio turno se ne possiede ancora almeno 1.

\emph{\textbf{Vincolato.}} Il guardiano protettore è vincolato magicamente ad un amuleto. Finché il guardiano e l'amuleto sono sullo stesso piano di esistenza, chi indossa l'amuleto può richiamare telepaticamente il guardiano perché lo raggiunga, e il guardiano saprà la distanza e la direzione in cui si trova l'amuleto. Se il guardiano si trova entro 18 metri da chi indossa l'amuleto, metà dei danni subiti da chi lo indossa (arrotondati per difetto) vengono trasferiti al guardiano. Se l'amuleto viene distrutto, il guardiano è inabile finché non viene creato un amuleto di rimpiazzo. L'amuleto del guardiano può essere soggetto ad un attacco diretto qualora non sia indossato o trasportato da nessuno. Ha Difesa 10, 10 punti ferita e immunità ai danni psichici e da veleno. Costruire un amuleto richiede 1 settimana e costa 10.000 mo in componenti.

\textbf{Azioni}

\emph{\textbf{Multiattacco.}} Il golem effettua due attacchi di pugno.

\emph{\textbf{Pugno.} Attacco con arma da mischia}: +7 a colpire,
portata 1 m, un bersaglio.

\emph{Colpisce:} 11 (2d6 + 4) danni da botta.

\textbf{Reazioni}

\emph{\textbf{Scudo.}} Quando una creatura attacca chi indossa l'amuleto del guardiano, il guardiano conferisce un bonus di +2 alla sua Difesa, se entro 1,5 metri dal suo controllore.

\medskip\index{Mostri - Hobgoblin}\textbf{Hobgoblin}

\emph{Media umanoide (goblinoide), legale malvagio}

\textbf{FORZA} +1

\textbf{DESTREZZA} +1

\textbf{COSTITUZIONE} +1

\textbf{INTELLIGENZA} +0

\textbf{SAGGEZZA} +0

\textbf{CARISMA} -1

\textbf{Iniziativa} +1 -- \textbf{Difesa} 19 (armatura di maglia, scudo)

\textbf{Punti Ferita} 11 (2d8 + 2)

\textbf{Movimento} 9 m

\textbf{Tiri Salvezza}: Tempra +5, Riflessi +2, Volontà +1

\textbf{Sensi} scurovisione 18 m

\textbf{Linguaggi} Comune, Goblin

\textbf{Sfida} 1/2 (100 PE)

\emph{\textbf{+1d6 Marziale.}} Una volta per turno, l'hobgoblin può infliggere 7 (2d6) danni aggiuntivi ad una creatura che colpisce con un attacco con arma, se quella creatura si trova entro 1,5 metri da un alleato dell'hobgoblin che non sia inabile.

\textbf{Azioni}

\emph{\textbf{Spada Lunga.} Attacco con arma da mischia}: +3 a colpire, portata 1 m, un bersaglio.

\emph{Colpisce:} 5 (1d8 + 1) danni taglienti o 6 (1d10 + 1) danni taglienti se usata con due mani.

\emph{\textbf{Arco Lungo.} Attacco con arma a Distanza}: +3 a colpire, gittata 45m, un bersaglio.

\emph{Colpisce:} 5 (1d8 + 1) danni perforanti.

\medskip\index{Mostri - Idra}\textbf{Idra}

\emph{Enorme mostruosità, disallineato}

\textbf{FORZA} +5

\textbf{DESTREZZA} +1

\textbf{COSTITUZIONE} +5

\textbf{INTELLIGENZA} -4

\textbf{SAGGEZZA} +0

\textbf{CARISMA} -2

\textbf{Iniziativa} +1 -- \textbf{Difesa} 19

\textbf{Punti Ferita} 172 (15d12 + 75)

\textbf{Movimento} 9 m, nuoto 9 m

\textbf{Tiri Salvezza}: Tempra +8, Riflessi +7, Volontà +3

\textbf{Competenze} Consapevolezza +6

\textbf{Sensi} scurovisione 18 m

\textbf{Linguaggi} -

\textbf{Sfida} 8 (3.900 PE)

\emph{\textbf{Teste Multiple.}} L'idra ha cinque teste. Finché ha più di una testa, l'idra ha +1d6 ai tiri salvezza contro le condizioni accecata, affascinata, assordata, spaventata, stordita o svenuta.

Ogni volta che l'idra subisce 25 o più danni in un singolo turno, una delle sue teste muore. Se tutte le teste muoiono, anche l'idra muore.

Al termine del suo turno, l'idra ricresce due teste per ciascuna delle sue teste uccise dal suo ultimo turno, a meno che non abbia subito danno da fuoco dal suo ultimo turno. L'idra recupera 10 punti ferita per ogni testa ricresciuta in questo modo.

\emph{\textbf{Teste Reattive.}} Per ogni testa posseduta oltre la prima, l'idra riceve una reazione extra che può essere usata solo per compiere attacchi di opportunità.

\emph{\textbf{Trattenere il Fiato.}} L'idra può trattenere il fiato per 1 ora.

\emph{\textbf{Veglia.}} Mentre l'idra dorme, almeno una delle sue teste resta sveglia.

\textbf{Azioni}

\emph{\textbf{Multiattacco.}} L'idra effettua tanti attacchi di morso quante sono le sue teste.

\emph{\textbf{Morso.} Attacco con arma da mischia}: +8 a colpire, portata 3 m, un bersaglio.

\emph{Colpisce:} 10 (1d10 + 5) danni perforanti.

\medskip\index{Mostri - Ippogrifo}\textbf{Ippogrifo}

\emph{Grande mostruosità, disallineato}

\textbf{FORZA} +3

\textbf{DESTREZZA} +1

\textbf{COSTITUZIONE} +1

\textbf{INTELLIGENZA} -4

\textbf{SAGGEZZA} +1

\textbf{CARISMA} -1

\textbf{Iniziativa} +1 -- \textbf{Difesa} 12

\textbf{Punti Ferita} 19 (3d10 + 3)

\textbf{Movimento} 12 m, volo 18 m

\textbf{Tiri Salvezza}: Tempra +5, Riflessi +5, Volontà +2

\textbf{Competenze} Consapevolezza +5

\textbf{Linguaggi} -

\textbf{Sfida} 1 (200 PE)

\emph{\textbf{Vista Affinata.}} L'ippogrifo ha +1d6 nelle prove di Saggezza (Consapevolezza) basate sulla vista.

\textbf{Azioni}

\emph{\textbf{Multiattacco.}} L'ippogrifo effettua due attacchi: uno con il becco e uno con gli artigli.

\emph{\textbf{Artigli.} Attacco con arma da mischia}: +5 a colpire, portata 1 m, un bersaglio.

\emph{Colpisce:} 10 (2d6 + 3) danni taglienti.

\emph{\textbf{Becco.} Attacco con arma da mischia}: +5 a colpire,
portata 1 m, un bersaglio.

\emph{Colpisce:} 8 (1d10 + 3) danni perforanti.

\medskip\index{Mostri - Kraken}\textbf{Kraken}

\emph{Mastodontica mostruosità (titano), caotico malvagio}

\textbf{FORZA} +10

\textbf{DESTREZZA} +0

\textbf{COSTITUZIONE} +7

\textbf{INTELLIGENZA} +6

\textbf{SAGGEZZA} +4

\textbf{CARISMA} +5

\textbf{Iniziativa} +6 -- \textbf{Difesa} 30

\textbf{Punti Ferita} 472 (27d20 + 189)

\textbf{Movimento} 6 m, nuoto 18 m

\textbf{Tiri Salvezza}: Tempra +21, Riflessi +12, Volontà +11

\textbf{Immunità al Danno} fulmine, armi +1

\textbf{Immunità alle Condizioni} paralizzato, spaventato

\textbf{Sensi} visione del vero 36 m 

\textbf{Linguaggi} comprende Abissale, Celestiale, Infernale e Druidico ma non può parlare, telepatia 36 m 

\textbf{Sfida} 23 (50.000 PE)

\emph{\textbf{Anfibio.}} Il kraken può respirare aria e acqua.

\emph{\textbf{Libertà di Movimento.}} Il kraken ignora i terreni difficili, e gli effetti magici non possono ridurne la velocità o far sì che diventi intralciato. Può spendere 1,5 metri di movimento per liberarsi dalle restrizioni non magiche o dall'essere afferrato.

\emph{\textbf{Mostro d'Assedio.}} Il kraken infligge danni doppi agli oggetti e le strutture.

\textbf{Azioni}

\emph{\textbf{Multiattacco.}} Il kraken effettua tre attacchi di tentacolo, ciascuno dei quali può essere rimpiazzato da un uso di Fiondare.

\emph{\textbf{Morso.} Attacco con arma da mischia}: +17 a colpire, portata 1 m, un bersaglio.

\emph{Colpisce:} 23 (3d8 + 10) danni perforanti. Se il bersaglio è una creatura di taglia Grande o inferiore afferrato dal kraken, quella creatura viene inghiottita, e l'afferrare ha termine. Mentre è inghiottita, la creatura è accecata e intralciata, ha copertura totale contro gli attacchi e altri effetti provenienti dall'esterno del kraken, e subisce 42 (12d6) danni da acido all'inizio di ciascun turno del kraken.

Se il kraken subisce 50 o più danni in un singolo turno da una creatura al suo interno, il kraken deve riuscire un tiro salvezza di Tempra CD 25 o vomitare tutte le creature inghiottite, che cadono prone in uno spazio entro 3 metri dal kraken. Se il kraken muore, una creatura inghiottita non risulta più intralciata da esso e può fuggire dal cadavere usando 4,5 metri di movimento, uscendo prona.

\emph{\textbf{Tentacolo.} Attacco con arma da mischia}: +17 a colpire, portata 9 m, un bersaglio.

\emph{Colpisce:} 20 (3d6 + 10) danni da botta, e il bersaglio è afferrato (CD 18 per fuggire). Fino al termine dell'afferrare, il bersaglio è intralciato. Il kraken ha dieci tentacoli, ciascuno dei
quali può afferrare un bersaglio.

\emph{\textbf{Fiondare.}} Un oggetto impugnato o una creatura afferrata dal kraken, di taglia Grande o inferiore viene lanciato di 18 metri in una direzione casuale e gettata prona. Se il bersaglio lanciato colpisce una superficie solida, subisce 3 (1d6) danni da botta per ogni 3 metri percorsi. Se il bersaglio viene lanciato contro un'altra creatura, quella creatura deve riuscire un tiro salvezza di Riflessi CD 18 o subire lo stesso danno e cadere prona.

\emph{\textbf{Tempesta di Fulmini.}} Il kraken crea magicamente tre saette di energia, ciascuna delle quali può colpire un bersaglio entro 36 metri e che il kraken possa vedere. Il bersaglio deve effettuare un tiro salvezza di Riflessi CD 23, e subire 22 (4d10) danni da fulmine se fallisce il tiro salvezza, o la metà se lo riesce.

\textbf{Azioni Aggiuntive}

Il kraken può effettuare 3 Azioni aggiuntive, scelte tra le opzioni seguenti. Può usare solo un'opzione leggendaria alla volta e solo al termine del turno di un'altra creatura. Il kraken recupera le Azioni aggiuntive spese all'inizio del proprio turno.

\textbf{Attacco di Tentacolo o Fiondare.} Il kraken effettua un attacco di tentacolo o usa Fiondare.

\textbf{Nube di Inchiostro (Costa 3 Azioni).} Mentre si trova sott'acqua, il kraken espelle una nube di inchiostro con un raggio di 18 metri. La nube si propaga intorno agli angoli, e quell'area è oscurata pesantemente per tutte le creature tranne il kraken. Ciascuna creatura a parte il kraken che termini il suo turno nell'area deve riuscire un tiro salvezza su Tempra 23, subendo 16 (3d10) danni da veleno se fallisce il tiro salvezza, o la metà se lo riesce. Una forte corrente disperde lanube, che altrimenti svanisce al termine del prossimo turno  del kraken. \textbf{Tempesta di Fulmini (Costa 2 Azioni).} Il kraken usa Tempesta di Fulmini.


\medskip\index{Mostri - Lamia}\textbf{Lamia}

\emph{Grande mostruosità, caotico malvagio}

\textbf{FORZA} +3

\textbf{DESTREZZA} +1

\textbf{COSTITUZIONE} +2

\textbf{INTELLIGENZA} +2

\textbf{SAGGEZZA} +2

\textbf{CARISMA} +3

\textbf{Iniziativa} +2 -- \textbf{Difesa} 15

\textbf{Punti Ferita} 97 (13d10 + 26)

\textbf{Movimento} 9 m

\textbf{Tiri Salvezza}: Tempra +6, Riflessi +9, Volontà +11

\textbf{Competenze} Muoversi Silenziosamente / Nascondersi nelle Ombre +3, Ingannare +7, Percepire Emozioni +4,

\textbf{Sensi} scurovisione 18 m

\textbf{Linguaggi} Abissale, Comune

\textbf{Sfida} 4 (1.100 PE)

\emph{\textbf{Incantesimi Innati.}} La caratteristica da incantatore innato della lamia è il Carisma- La lamia può lanciare in maniera innata i seguenti incantesimi, senza bisogno di componenti materiali:

A volontà: \emph{camuffare sé stesso} (qualsiasi forma umanoide)\emph{,} \emph{immagine maggiore}

3/Giorno ciascuno: \emph{charme su persone, immagine speculare,}

\emph{scrutare, suggestione}

1/Giorno: \emph{restrizione}

\textbf{Azioni}

\emph{\textbf{Multiattacco.}} La lamia effettua due attacchi: uno con
gli artigli e uno con il pugnale o il Tocco Intossicante.

\emph{\textbf{Artigli.} Attacco con arma da mischia}: +5 a colpire, portata 1 m, un bersaglio.

\emph{Colpisce:} 14 (2d10 + 3) danni taglienti.

\emph{\textbf{Pugnale.} Attacco con arma da mischia}: +5 a colpire, portata 1 m, un bersaglio.

\emph{Colpisce:} 5 (1d4 + 3) danni perforanti.

\emph{\textbf{Tocco Intossicante.} Attacco con incantesimo in mischia}: +5 a colpire, portata 1 m, una creatura.

\emph{Colpisce:} Il bersaglio è maledetto per 1 ora da questa magia. Fino al termine della maledizione, il bersaglio ha -1d6 ai tiri salvezza su Volontà e a tutte le prove di abilità.

\medskip\index{Mostri - Lich}\textbf{Lich}

\emph{Media non morto, qualsiasi allineamento malvagio}

\textbf{FORZA} +0

\textbf{DESTREZZA} +3

\textbf{COSTITUZIONE} +3

\textbf{INTELLIGENZA} +5

\textbf{SAGGEZZA} +2

\textbf{CARISMA} +3

\textbf{Iniziativa} +5 -- \textbf{Difesa} 28

\textbf{Punti Ferita} 135 (18d8 + 54)

\textbf{Movimento} 9 m

\textbf{Tiri Salvezza}: Tempra +6, Riflessi +7, Volontà +11

\textbf{Resistenze al Danno} freddo, fulmine, da Vuoto

\textbf{Immunità al Danno} veleno; da botta, perforante e tagliente di attacchi non magici

\textbf{Immunità alle Condizioni} affascinato, avvelenato, paralizzato, sfinimento, spaventato

\textbf{Sensi} visione del vero 36 m

\textbf{Linguaggi} Comune più altre cinque lingue

\textbf{Sfida} 21 (33.000 PE)

\emph{\textbf{Incantesimi.}} Il lich ha CM 18. La sua caratteristica da incantatore è l'Intelligenza, +3 a colpire con attacchi da incantesimo). Il lich ha preparati i seguenti incantesimi:

Trucchetti (a volontà): \emph{mano magica, prestidigitazione, raggio} \emph{di gelo}

Difficoltà 10 (4 slot): \emph{dardo incantato, individuazione del magico,} \emph{onda tonante, scudo}

Difficoltà 13 (3 slot): \emph{freccia acida, immagine speculare,} \emph{individuazione dei pensieri, invisibilità}

Difficoltà 15 (3 slot): \emph{animare morti, controincantesimo, dissolvi} \emph{magie, palla di fuoco}

Difficoltà 18 (3 slot): \emph{inaridire, porta dimensionale}

Difficoltà 20 (3 slot): \emph{nube mortale, scrutare}

Difficoltà 23 (1 slot): \emph{disintegrazione, globo di invulnerabilità}

Difficoltà 25 (1 slot): \emph{dito della morte, spostamento planare}

Difficoltà 28 (1 slot): \emph{dominare mostri, parola del potere stordire}

Difficoltà 30 (1 slot): \emph{parola del potere uccidere}

\emph{\textbf{Natura Non Morta.}} Il lich non ha bisogno di aria, cibo, bevande o sonno.

\emph{\textbf{Resistenza Leggendaria (3/Giorno).}} Se il lich fallisce un tiro salvezza, può scegliere invece di riuscirvi.

\emph{\textbf{Resistenza allo Scacciare.}} Il lich ha +1d6 ai tiri salvezza contro gli effetti che scacciano i non morti.

\emph{\textbf{Rinvigorimento.}} Se possiede un filatterio, il lich distrutto ottiene un nuovo corpo in 1d10 giorni, recuperando tutti i suoi punti ferita e ritornando in attività. Il nuovo corpo compare entro 1,5 metri dal filatterio.

\emph{\textbf{Sacrifici di Anime.}} Un lich deve periodicamente nutrire di anime il suo filatterio per sostenere la magia che mantiene il suo corpo e la sua coscienza. Per farlo usa l'incantesimo \emph{imprigionare}. Invece di scegliere una delle normali opzioni dell'incantesimo, il lich lo impiega per intrappolare magicamente il corpo e l'anima del bersaglio all'interno del filatterio. Il filatterio deve trovarsi sullo stesso piano del lich, perché questo incantesimo funzioni. Il filatterio di un lich può contenere solo una creatura alla volta, e \emph{dissolvi magie} lanciato come incantesimo di Difficoltà 30 sul filatterio libera qualsiasi creatura imprigionata al suo interno. Una creatura imprigionata nel filatterio per 24 ore viene consumata e distrutta, dopodiché nulla salvo un intervento divino potrà riportarla in vita.

Un lich che dimentichi o non riesca a mantenere il suo corpo con le anime sacrificate inizia a cascare a pezzi, e potrebbe infine trasformarsi in un semilich.

\textbf{Azioni}

\emph{\textbf{Tocco Paralizzante.} Attacco con incantesimo in mischia}: +12 a colpire, portata 1 m, una creatura.

\emph{Colpisce:} 10 (3d6) danni da freddo. Il bersaglio deve riuscire un tiro salvezza di Tempra CD 18 o restare paralizzato per 1 minuto. Il bersaglio può ripetere il tiro salvezza al termine di ciascun suo turno, terminando l'effetto su di sé in caso di successo.

\textbf{Azioni Aggiuntive}

Il lich può effettuare 3 Azioni aggiuntive, scelte tra le opzioni seguenti. Può usare solo un'opzione leggendaria alla volta e solo al termine del turno di un'altra creatura. Il lich recupera le Azioni aggiuntive spese all'inizio del proprio turno.

\emph{\textbf{Distruggere Vita (Costa 3 Azioni).}} Ogni creatura ad eccezione dei non morti entro 6 metri dal lich deve effettuare un tiro salvezza su Tempra CD 18 contro questa magia, subendo 21 (6d6) danni da Vuoto se fallisce il tiro salvezza, o la metà di questi danni se lo riesce.

\emph{\textbf{Sguardo Spaventoso (Costa 2 Azioni).}} Il lich fissa il suo sguardo su di una creatura visibile entro 3 metri da esso. Il bersaglio deve riuscire un tiro salvezza di Volontà CD 18 contro questa magia o restare spaventato per 1 minuto. Il bersaglio spaventato può ripetere il tiro salvezza al termine di ciascun suo turno, terminando l'effetto su di sé in caso di successo. Se il tiro salvezza del bersaglio è riuscito o l'effetto per lui ha termine, il bersaglio è immune allo sguardo del lich per le successive 24 ore.

\emph{\textbf{Tocco Paralizzante (Costa 2 Azioni).}} Il lich usa il suo Tocco Paralizzante.

\emph{\textbf{Trucchetto.}} Il lich lancia un trucchetto.

\medskip\index{Mostri - Lucertoloide}\textbf{Lucertoloide}

\emph{Media umanoide (lucertoloide), neutrale}

\textbf{FORZA} +2

\textbf{DESTREZZA} +0

\textbf{COSTITUZIONE} +1

\textbf{INTELLIGENZA} -2

\textbf{SAGGEZZA} +1

\textbf{CARISMA} -2

\textbf{Iniziativa} +0 -- \textbf{Difesa} 16 (armatura naturale, scudo)

\textbf{Punti Ferita} 22 (4d8 + 4)

\textbf{Movimento} 9 m, nuoto 9 m

\textbf{Tiri Salvezza}: Tempra +4, Riflessi +0, Volontà +0

\textbf{Competenze} Muoversi Silenziosamente / Nascondersi nelle Ombre +4, Consapevolezza +3, Sopravvivenza +5

\textbf{Linguaggi} Draconico

\textbf{Sfida} 1/2 (100 PE)

\emph{\textbf{Trattenere il Fiato.}} Il lucertoloide può trattenere il fiato per 15 minuti.

\textbf{Azioni}

\emph{\textbf{Multiattacco.}} Il lucertoloide effettua due attacchi in mischia, ciascuno con un'arma diversa.

\emph{\textbf{Giavellotto.} Attacco con arma da mischia o a Distanza}: +4 a colpire, portata 1 m o gittata 9m, un bersaglio. \emph{Colpisce:} 5 (1d6 + 2) danni perforanti.

\emph{\textbf{Morso.} Attacco con arma da mischia}: +4 a colpire, portata 1 m, un bersaglio.

\emph{Colpisce:} 5 (1d6 + 2) danni perforanti.

\emph{\textbf{Randello Pesante.} Attacco con arma da mischia}: +4 a colpire, portata 1 m, un bersaglio.

\emph{Colpisce:} 5 (1d6 + 2) danni da botta.

\emph{\textbf{Scudo Appuntito.} Attacco con arma da mischia}: +4 a colpire, portata 1 m, un bersaglio.

\emph{Colpisce:} 5 (1d6 + 2) danni perforanti.


\subsection{Mannari}

\medskip\index{Mostri - Cinghiale Mannaro}\textbf{Cinghiale Mannaro}

\emph{Media umanoide (umano, mutaforma), neutrale malvagio}

\textbf{FORZA} +3

\textbf{DESTREZZA} +0

\textbf{COSTITUZIONE} +2

\textbf{INTELLIGENZA} +0

\textbf{SAGGEZZA} +0

\textbf{CARISMA} -1

\textbf{Iniziativa} +0 -- \textbf{Difesa} 12 in forma umanoide, 13 in forma di cinghiale o ibrida

\textbf{Punti Ferita} 78 (12d8 + 24)

\textbf{Movimento} 9 m (12 m in forma di cinghiale)

\textbf{Tiri Salvezza}: Tempra +7, Riflessi +1, Volontà +4

\textbf{Competenze} Consapevolezza +2

\textbf{Immunità al Danno} da botta, perforante e tagliente di attacchi non magici o che non siano argentati 

\textbf{Linguaggi} Comune (non può parlare in forma di cinghiale)

\textbf{Sfida} 4 (1.100 PE)

\emph{\textbf{Carica (Solo Forma di Cinghiale o Ibrida).}} Se il cinghiale mannaro si muove in linea retta di almeno 4,5 metri verso un bersaglio e poi lo colpisce con le zanne durante lo stesso turno, il bersaglio subisce 7 (2d6) danni taglienti aggiuntivi. Se il bersaglio è una creatura, deve riuscire un tiro salvezza di Tempra CD 13 o cadere prono.

\emph{\textbf{Implacabile (Ricarica dopo un 1 ora).}} Se il cinghiale mannaro subisce 14 danni o meno che lo ridurrebbero a 0 punti ferita, scende invece a 1 punto ferita.

\emph{\textbf{Mutaforma.}} Il cinghiale mannaro può usare la sua azione per trasformarsi in un ibrido cinghiale-umanoide o in un cinghiale, o per tornare alla sua vera forma, che è umanoide. Le sue statistiche, a parte la Difesa, sono le stesse in tutte le forme. Qualsiasi equipaggiamento stia indossando o trasportando non viene trasformato. Alla morte ritorna alla sua vera forma.

\textbf{Azioni}

\emph{\textbf{Multiattacco (Solo in Forma Umanoide o Ibrida).}} Il cinghiale mannaro effettua due attacchi, di cui solo uno può essere con le zanne.

\emph{\textbf{Maglio (Soltanto in Forma Umanoide o Ibrida).} Attacco con arma da mischia}: +5 a colpire, portata 1 m, un bersaglio. \emph{Colpisce:} 10 (2d6 + 3) danni da botta.

\emph{\textbf{Zanne (Soltanto in Forma di Cinghiale o Ibrida).} Attacco con arma da mischia}: +5 a colpire, portata 1 m, un bersaglio. \emph{Colpisce:} 10 (2d6 + 3) danni taglienti. Se il bersaglio è un umanoide, deve riuscire un tiro salvezza di Tempra CD 12 o venire maledetto dalla licantropia del cinghiale mannaro.

\medskip\index{Mostri - Lupo Mannaro}\textbf{Lupo Mannaro}

\emph{Media umanoide (umano, mutaforma), caotico malvagio}

\textbf{FORZA} +2

\textbf{DESTREZZA} +1

\textbf{COSTITUZIONE} +2

\textbf{INTELLIGENZA} +0

\textbf{SAGGEZZA} +0

\textbf{CARISMA} +0

\textbf{Iniziativa} +1 -- \textbf{Difesa} 13 in forma umanoide, 14 in forma di lupo o ibrida

\textbf{Punti Ferita} 58 (9d8 + 18)

\textbf{Movimento} 9 m (12 m in forma di lupo)

\textbf{Tiri Salvezza}: Tempra +5, Riflessi +1, Volontà +2

\textbf{Competenze} Muoversi Silenziosamente / Nascondersi nelle Ombre +3, Consapevolezza +4

\textbf{Immunità al Danno} da botta, perforante e tagliente di attacchi non magici o che non siano argentati 

\textbf{Linguaggi} Comune (non può parlare in forma di lupo)

\textbf{Sfida} 3 (700 PE)

\emph{\textbf{Mutaforma.}} Il lupo mannaro può usare la sua azione per trasformarsi in un ibrido lupo-umanoide o in un lupo, o per tornare alla sua vera forma, che è umanoide. Le sue statistiche, a parte la Difesa, sono le stesse in tutte le forme. Qualsiasi equipaggiamento stia indossando o trasportando non viene trasformato. Alla morte ritorna alla sua vera forma.

\emph{\textbf{Udito e Olfatto Affinato.}} Il lupo mannaro ha +1d6 nelle prove di Saggezza (Consapevolezza) basate su udito o olfatto.

\textbf{Azioni}

\emph{\textbf{Multiattacco (Soltanto in Forma Umanoide o Ibrida).}} Il lupo mannaro effettua due attacchi: uno con il morso e uno con gli artigli o la lancia.

\emph{\textbf{Artigli (Soltanto in Forma Ibrida).} Attacco con arma da mischia}: +4 a colpire, portata 1 m, una creatura. \emph{Colpisce:} 7 (2d4 + 2) danni taglienti.

\emph{\textbf{Lancia (Soltanto in Forma Umanoide).} Attacco con arma da mischia o a Distanza}: +4 a colpire, portata 1 m o gittata 6m, una creatura.

\emph{Colpisce:} 5 (1d6 + 2) danni perforanti o 6 (1d8 + 2) danni perforanti se usata con due mani in un attacco di mischia.

\emph{\textbf{Morso (Soltanto in Forma di Lupo o Ibrida).} Attacco con arma da mischia}: +4 a colpire, portata 1 m, un bersaglio.

\emph{Colpisce:} 6 (1d8 + 2) danni perforanti. Se il bersaglio è un umanoide, deve riuscire un tiro salvezza di Tempra CD 12 o venir maledetto dalla licantropia del lupo mannaro.

\medskip\index{Mostri - Orso Mannaro}\textbf{Orso Mannaro}

\emph{Media umanoide (umano, mutaforma), neutrale buono}

\textbf{FORZA} +4

\textbf{DESTREZZA} +0

\textbf{COSTITUZIONE} +3

\textbf{INTELLIGENZA} +0

\textbf{SAGGEZZA} +1

\textbf{CARISMA} +1

\textbf{Iniziativa} +0 -- \textbf{Difesa} 13 in forma umanoide, 14

in forma di orso o ibrida

\textbf{Punti Ferita} 135 (18d8 + 54)

\textbf{Movimento} 9 m (12 m, scalata 9 m in forma di orso o forma ibrida)

\textbf{Tiri Salvezza}:  Tempra +5, Riflessi +6, Volontà +2

\textbf{Competenze} Consapevolezza +7

\textbf{Immunità al Danno} da botta, perforante e tagliente di attacchi non magici o che non siano argentati

\textbf{Linguaggi} Comune (non può parlare in forma di orso)

\textbf{Sfida} 5 (1.800 PE)

\emph{\textbf{Mutaforma.}} L'orso mannaro può usare la sua azione per trasformarsi in un ibrido orso-umanoide o in un orso, o per tornare alla sua vera forma, che è umanoide. Le sue statistiche, a parte la Difesa, sono le stesse in tutte le forme. Qualsiasi equipaggiamento stia indossando o trasportando non viene trasformato. Alla morte ritorna alla sua vera forma.

\emph{\textbf{Olfatto Affinato.}} L'orso mannaro ha +1d6 nelle prove di Saggezza (Consapevolezza) basate sull'olfatto.

\textbf{Azioni}

\emph{\textbf{Multiattacco.}} In forma di orso, l'orso mannaro effettua due attacchi di artiglio. In forma umanoide, effettua due attacchi di ascia bipenne. In forma ibrida, può attaccare come un orso o un umanoide.  

\emph{\textbf{Artiglio (Soltanto in Forma di Orso o Ibrida).} Attacco con arma da mischia}: +7 a colpire, portata 1 m, un bersaglio. \emph{Colpisce:} 13 (2d8 + 2) danni taglienti.

\emph{\textbf{Ascia Bipenne (Soltanto in Forma Umanoide o Ibrida).} Attacco con arma da mischia}: +7 a colpire, portata 1 m, un bersaglio. \emph{Colpisce:} 10 (1d12 + 4) danni taglienti.

\emph{\textbf{Morso (Soltanto in Forma di Orso o Ibrida).} Attacco con arma da mischia}: +7 a colpire, portata 1 m, un bersaglio.

\emph{Colpisce:} 15 (2d10 + 4) danni perforanti. Se il bersaglio è un umanoide, deve riuscire un tiro salvezza di Tempra CD 14 o venir maledetto dalla licantropia dell'orso mannaro.

\medskip\index{Mostri - Ratto Mannar}\textbf{Ratto Mannaro}

\emph{Media umanoide (umano, mutaforma), legale malvagio}

\textbf{FORZA} +0

\textbf{DESTREZZA} +2

\textbf{COSTITUZIONE} +1

\textbf{INTELLIGENZA} +0

\textbf{SAGGEZZA} +0

\textbf{CARISMA} -1

\textbf{Iniziativa} +2 -- \textbf{Difesa} 13

\textbf{Punti Ferita} 33 (6d8 + 6)

\textbf{Movimento} 9 m

\textbf{Tiri Salvezza}: Tempra +2, Riflessi +5, Volontà +3

\textbf{Competenze} Muoversi Silenziosamente / Nascondersi nelle Ombre +4, Consapevolezza +2

\textbf{Immunità al Danno} da botta, perforante e tagliente di attacchi non magici o che non siano argentati

\textbf{Sensi} scurovisione 18 m (solo in forma di ratto)

\textbf{Linguaggi} Comune (non può parlare in forma di ratto)

\textbf{Sfida} 2 (450 PE)

\emph{\textbf{Mutaforma.}} Il ratto mannaro può usare la sua azione per trasformarsi in un ibrido ratto-umanoide o in un ratto, o per tornare alla sua vera forma, che è umanoide. Le sue statistiche, a parte la Difesa, sono le stesse in tutte le forme. Qualsiasi equipaggiamento stia indossando o trasportando non viene trasformato. Alla morte ritorna alla sua vera forma.

\emph{\textbf{Olfatto Affinato.}} Il ratto mannaro ha +1d6 nelle prove di Saggezza (Consapevolezza) basate sull'olfatto.

\textbf{Azioni}

\emph{\textbf{Multiattacco (Solo in Forma Umanoide o Ibrida).}} Il ratto mannaro effettua due attacchi, di cui solo uno può essere con il morso.

\emph{\textbf{Spada Corta (Soltanto in Forma Umanoide o Ibrida).} Attacco con arma da mischia}: +4 a colpire, portata 1 m, un bersaglio. \emph{Colpisce:} 5 (1d6 + 2) danni perforanti.

\emph{\textbf{Balestra a mano (Soltanto in Forma Umanoide o Ibrida).} Attacco con arma a Distanza}: +4 a colpire, gittata 9m, un bersaglio.

\emph{Colpisce:} 5 (1d6 + 2) danni perforanti.

\emph{\textbf{Morso (Soltanto in Forma di Ratto o Ibrida).} Attacco con arma da mischia}: +4 a colpire, portata 1 m, un bersaglio.

\emph{Colpisce:} 4 (1d4 + 2) danni perforanti. Se il bersaglio è un umanoide, deve riuscire un tiro salvezza di Tempra CD 11 o venir maledetto dalla licantropia del ratto mannaro.

\medskip\index{Mostri - Tigre Mannara}\textbf{Tigre Mannara}

\emph{Media umanoide (umano, mutaforma), neutrale}

\textbf{FORZA} +3

\textbf{DESTREZZA} +2

\textbf{COSTITUZIONE} +3

\textbf{INTELLIGENZA} +0

\textbf{SAGGEZZA} +1

\textbf{CARISMA} +0

\textbf{Iniziativa} +2 -- \textbf{Difesa} 14

\textbf{Punti Ferita} 120 (16d8 + 48)

\textbf{Movimento} 9 m (12 m in forma di tigre)

\textbf{Tiri Salvezza}: Tempra +2, Riflessi +7, Volontà +4

\textbf{Competenze} Muoversi Silenziosamente / Nascondersi nelle Ombre +4, Consapevolezza +5

\textbf{Immunità al Danno} da botta, perforante e tagliente di attacchi non magici che non siano argentati

\textbf{Sensi} scurovisione 18 m

\textbf{Linguaggi} Comune (non può parlare in forma di tigre)

\textbf{Sfida} 4 (1.1100 PE)

\emph{\textbf{Balzo.}} Se la tigre mannara si muove di almeno 4,5 metri in linea retta verso una creatura e la colpisce con un attacco di artiglio durante lo stesso turno, il bersaglio deve riuscire un tiro salvezza su Tempra CD 14 o cadere prono. Se il bersaglio è prono, la tigre mannara può effettuare un attacco di morso contro di esso come azione bonus.

\emph{\textbf{Mutaforma.}} La tigre mannara può usare la sua azione per trasformarsi in un ibrido tigre-umanoide o in una tigre, o per tornare alla sua vera forma, che è umanoide. Le sue statistiche, a parte la Difesa, sono le stesse in tutte le forme. Qualsiasi equipaggiamento stia indossando o trasportando non viene trasformato. Alla morte ritorna alla sua vera forma.

\emph{\textbf{Olfatto e Udito Affinato.}} La tigre mannara ha +1d6 nelle prove di Saggezza (Consapevolezza) basate su olfatto e udito.

\textbf{Azioni}

\emph{\textbf{Multiattacco (Solo in Forma Umanoide o Ibrida).}} In forma umanoide, la tigre mannara effettua due attacchi di scimitarra o due attacchi di arco lungo. In forma ibrida, può attaccare come un umanoide o effettuare due attacchi di artiglio.

\emph{\textbf{Artiglio (Soltanto in Forma di Tigre o Ibrida).} Attacco con arma da mischia}: +5 a colpire, portata 1 m, un bersaglio. \emph{Colpisce:} 7 (1d8 + 3) danni taglienti.

\emph{\textbf{Morso (Soltanto in Forma di Tigre o Ibrida).} Attacco con arma da mischia}: +5 a colpire, portata 1 m, un bersaglio.

\emph{Colpisce:} 8 (1d10 + 3) danni perforanti. Se il bersaglio è un umanoide, deve riuscire un tiro salvezza di Tempra CD 13 o venir maledetto dalla licantropia della tigre mannara.

\emph{\textbf{Scimitarra (Soltanto in Forma Umanoide o Ibrida).} Attacco con arma da mischia}: +5 a colpire, portata 1 m, un bersaglio. \emph{Colpisce:} 6 (1d6 + 3) danni taglienti.

\emph{\textbf{Arco Lungo (Soltanto in Forma Umanoide o Ibrida).} Attacco con arma a Distanza}: +4 a colpire, gittata 45m, un bersaglio.

\emph{Colpisce:} 6 (1d8 + 2) danni perforanti.

\medskip\index{Mostri - Manticora}\textbf{Manticora}

\emph{Grande mostruosità, legale malvagio}

\textbf{FORZA} +3

\textbf{DESTREZZA} +3

\textbf{COSTITUZIONE} +3

\textbf{INTELLIGENZA} -2

\textbf{SAGGEZZA} +1

\textbf{CARISMA} -1

\textbf{Iniziativa} +3 -- \textbf{Difesa} 16

\textbf{Punti Ferita} 68 (8d10 + 24)

\textbf{Movimento} 9 m, volo 15 m

\textbf{Tiri Salvezza}: Tempra +9, Riflessi +7, Volontà +3

\textbf{Sensi} scurovisione 18 m

\textbf{Linguaggi} Comune

\textbf{Sfida} 3 (700 PE)

\emph{\textbf{Ricrescere Spine della Coda.}} La manticora possiede ventiquattro spine nella coda. Le spine usate ricrescono all'alba.

\textbf{Azioni}

\emph{\textbf{Multiattacco.}} La manticora effettua tre attacchi: uno con il morso e due con gli artigli o tre con le spine della coda.

\emph{\textbf{Artiglio.} Attacco con arma da mischia}: +5 a colpire, portata 1 m, un bersaglio.

\emph{Colpisce:} 6 (1d6 + 3) danni taglienti.

\emph{\textbf{Morso.} Attacco con arma da mischia}: +5 a colpire, portata 1 m, un bersaglio.

\emph{Colpisce:} 7 (1d8 + 3) danni perforanti.

\emph{\textbf{Spine della Coda.} Attacco con arma a Distanza}: +5 a colpire, gittata 30m, un bersaglio.

\emph{Colpisce:} 7 (1d8 + 3) danni perforanti.

\medskip\index{Mostri - Manto Assassino}\textbf{Manto Assassino}

\emph{Grande aberrazione, caotico neutrale}

\textbf{FORZA} +3

\textbf{DESTREZZA} +2

\textbf{COSTITUZIONE} +1

\textbf{INTELLIGENZA} +1

\textbf{SAGGEZZA} +1

\textbf{CARISMA} +2

\textbf{Iniziativa} +2 -- \textbf{Difesa} 18

\textbf{Punti Ferita} 78 (12d10 + 12)

\textbf{Movimento} 3 m, volo 12 m

\textbf{Tiri Salvezza}: Tempra +6, Riflessi +5, Volontà +7

\textbf{Competenze} Muoversi Silenziosamente / Nascondersi nelle Ombre +5

\textbf{Sensi} scurovisione 18 m

\textbf{Linguaggi} Linguaggio delle Profondità

\textbf{Sfida} 8 (3.900 PE)

\emph{\textbf{Falso Aspetto.}} Mentre il manto assassino resta immobile senza esporre la parte inferiore del corpo, è indistinguibile da un manto di pelle nera.

\emph{\textbf{Sensibilità alla Luce}}. Mentre è alla luce del sole, il manto assassino ha -1d6 ai tiri per colpire, oltre che alle prove di Saggezza (Consapevolezza) basate sulla vista.

\emph{\textbf{Trasferimento di Danno.}} Mentre è appiccicato ad una creatura, il manto assassino subisce solo la metà dei danni che gli sono inferti (arrotondare per difetto), e la creatura vittima del manto assassino subisce l'altra metà.

\textbf{Azioni}

\emph{\textbf{Multiattacco.}} Il manto assassino effettua due attacchi:
uno con il morso e uno con la coda.

\emph{\textbf{Morso.} Attacco con arma da mischia}: +6 a colpire, portata 1 m, una creatura.

\emph{Colpisce:} 10 (2d6 + 3) danni perforanti, e se il bersaglio è di taglia Grande o inferiore, il manto assassino vi si appiccica. Se il manto assassino ha +1d6 contro il bersaglio, si appiccica alla sua testa e il bersaglio è accecato e impossibilitato a respirare finché il manto assassino vi rimane appiccicato. Mentre appiccicato il manto assassino può effettuare questo attacco solo   contro il bersaglio e ha +1d6 al tiro per colpire. Il manto   assassino può staccarsi spendendo 1,5 metri di movimento. Una   creatura, compreso il bersaglio, può effettuare la sua azione per   staccare il manto assassino riuscendo una prova di Forza CD 16. 


\emph{\textbf{Coda.} Attacco con arma da mischia}: +6 a colpire, portata 3 m, una creatura.

\emph{Colpisce:} 7 (1d8 + 3) danni taglienti.

\emph{\textbf{Apparizioni (Ricarica dopo un 1 ora).}} Qualora non si trovi sotto luce intensa, il manto assassino crea tre duplicati illusori di sé stesso, che si muovono assieme ad esso e ne imitano le azioni, scambiandosi di posizione per rendere impossibile capire quale sia il reale manto assassino. Se l'originale si trova in un'area di luce intensa, i duplicati svaniscono.

Ogniqualvolta una creatura prenda a bersaglio il manto assassino con un attacco o un incantesimo nocivo mentre sono ancora presenti dei duplicati, quella creatura determina casualmente se prende a bersaglio il manto assassino o uno dei duplicati. Una creatura che non possa vedere o che si affida a sensi diversi dalla vista ignora questo effetto magico.

Un duplicato possiede la Difesa e usa i tiri salvezza del manto assassino. Se un attacco colpisce un duplicato, o se un duplicato fallisce un tiro salvezza contro un effetto che infligge danni, svanisce.

\emph{\textbf{Gemito.}} Ogni creatura entro 18 metri dal manto assassino, che possa udire il suo gemito e che non sia un'aberrazione, deve riuscire un tiro salvezza su Volontà CD 13 o essere spaventata fino al termine del prossimo turno del manto assassino. Se il tiro salvezza della creatura
riesce, la creatura è immune al gemito del manto assassino per le successive 24 ore.

\medskip\index{Mostri - Mantoscuro}\textbf{Mantoscuro}

\emph{Piccola mostruosità, disallineato}

\textbf{FORZA} +3

\textbf{DESTREZZA} +1

\textbf{COSTITUZIONE} +1

\textbf{INTELLIGENZA} -4

\textbf{SAGGEZZA} +0

\textbf{CARISMA} -3

\textbf{Iniziativa} +1 -- \textbf{Difesa} 12

\textbf{Punti Ferita} 22 (5d6 + 5)

\textbf{Movimento} 3 m, volo 9 m

\textbf{Tiri Salvezza}: Tempra +5, Riflessi +3, Volontà +0

\textbf{Competenze} Muoversi Silenziosamente / Nascondersi nelle Ombre +3

\textbf{Sensi} vista cieca 18 m

\textbf{Linguaggi} -

\textbf{Sfida} 1/2 (100 PE)

\emph{\textbf{Ecolocazione.}} Il mantoscuro non può usare la vista cieca se assordato.

\emph{\textbf{Falso Aspetto.}} Mentre il mantoscuro rimane immobile, è indistinguibile da una formazione rocciosa come una stalattite o una stalagmite.

\textbf{Azioni}

\emph{\textbf{Spaccare.} Attacco con arma da mischia}: +5 a colpire, portata 1 m, una creatura.

\emph{Colpisce:} 6 (1d6 + 3) danni da botta e il mantoscuro si appiccica alla creatura. Se il bersaglio è di taglia Media o inferiore il mantoscuro ha +1d6 al tiro per colpire, si appiccica avvolgendo la testa del bersaglio, che è accecato e impossibilitato a respirare finché il mantoscuro resta appiccicato in questo modo.

Mentre è appiccicato al bersaglio, il mantoscuro non può attaccare nessun'altra creatura salvo il bersaglio, ma ha +1d6 ai suoi tiri per colpire. La velocità del mantoscuro diventa 0 e non può trarre beneficio da nessun bonus alla velocità, muovendosi assieme al bersaglio.

Una creatura può staccare il mantoscuro con un'azione e riuscendo una
prova di Forza CD 13. Durante il suo turno, il mantoscuro può staccarsi
dal bersaglio da solo usando 1,5 metri di movimento.

\emph{\textbf{Aura di Oscurità (1/Giorno).}} Un'oscurità magica con 4,5 metri di raggio si estende dal mantoscuro, muovendosi con esso, e propagandosi oltre gli angoli. L'oscurità permane finché il mantoscuro mantiene la concentrazione, massimo 10 minuti (come se si stesse concentrando su di un incantesimo). La scurovisione non può penetrare questa oscurità, né essa può essere rischiarata da alcuna luce naturale. Se qualsiasi parte dell'oscurità si sovrappone ad un'area di luce generata da un incantesimo di Difficoltà 12 o inferiore, l'incantesimo che sta creando la luce viene dissolto.

\medskip\index{Mostri - Medusa}\textbf{Medusa}

\emph{Media mostruosità, legale malvagio}

\textbf{FORZA} +0

\textbf{DESTREZZA} +2

\textbf{COSTITUZIONE} +3

\textbf{INTELLIGENZA} +1

\textbf{SAGGEZZA} +1

\textbf{CARISMA} +2

\textbf{Iniziativa} +2 -- \textbf{Difesa} 18

\textbf{Punti Ferita} 127 (17d8 + 51)

\textbf{Movimento} 9 m

\textbf{Tiri Salvezza}: Tempra +6, Riflessi +8, Volontà +7

\textbf{Competenze} Muoversi Silenziosamente / Nascondersi nelle Ombre +5, Ingannare +5, Percepire Emozioni +4, Consapevolezza +4

\textbf{Sensi} scurovisione 18 m

\textbf{Linguaggi} Comune

\textbf{Sfida} 6 (2.300 PE)

\emph{\textbf{Sguardo Pietrificante.}} Se una creatura comincia il suo turno entro 9 metri da una medusa di cui possa vedere gli occhi, la medusa, qualora la non sia inabile e possa vedere a sua volta la creatura, può obbligarla ad effettuare un tiro salvezza di Tempra CD 14. Se la creatura fallisce il tiro salvezza di 5 o più, viene pietrificata all'istante, altrimenti inizia magicamente a trasformarsi in pietra ed è intralciata. La creatura intralciata deve ripetere il tiro salvezza al termine del suo prossimo turno. Se lo riesce, l'effetto termina. Se lo fallisce, la creatura è pietrificata finché non viene liberata dall'incantesimo \emph{ristorare superiore} o altra magia.

Una creatura che non sia sorpresa può distogliere lo sguardo per evitare il tiro salvezza all'inizio del proprio turno. In quel caso, non potrà vedere la medusa fino all'inizio del suo prossimo turno, quando potrà distogliere nuovamente lo sguardo. Se nel frattempo dovesse guardare la medusa, dovrebbe immediatamente effettuare il tiro salvezza.

Se la medusa vede il suo riflesso su di una superficie riflettente entro 9 metri da lei in un'area di luce intensa, a causa della propria maledizione subirà gli effetti del suo stesso sguardo.

\textbf{Azioni}

\emph{\textbf{Multiattacco.}} La medusa effettua tre attacchi -- uno con i capelli serpentini e due con la spada corta -- o due attacchi a distanza con l'arco lungo.

\emph{\textbf{Capelli Serpentini.} Attacco con arma da mischia}: +5 a colpire, portata 1 m, un bersaglio.

\emph{Colpisce:} 4 (1d4 + 2) danni perforanti più 14 (4d6) danni da veleno.

\emph{\textbf{Spada Corta.} Attacco con arma da mischia}: +5 a colpire, portata 1 m, un bersaglio.

\emph{Colpisce:} 5 (1d6 + 2) danni perforanti.

\emph{\textbf{Arco Lungo.} Attacco con arma a Distanza}: +5 a colpire, gittata 45m, un bersaglio.

\emph{Colpisce:} 6 (1d8 + 2) danni perforanti più 7 (2d6) danni da veleno.

\subsection{Mefiti}

\medskip\index{Mostri - Mefito di Ghiaccio}\textbf{Mefito di Ghiaccio}

\emph{Piccola elementale, neutrale malvagio}

\textbf{FORZA} -2

\textbf{DESTREZZA} +1

\textbf{COSTITUZIONE} +0

\textbf{INTELLIGENZA} -1

\textbf{SAGGEZZA} +0

\textbf{CARISMA} +1

\textbf{Iniziativa} +1 -- \textbf{Difesa} 12

\textbf{Punti Ferita} 21 (6d6)

\textbf{Movimento} 9 m, volo 9 m

\textbf{Tiri Salvezza}: Tempra +2, Riflessi +5, Volontà +3

\textbf{Competenze} Muoversi Silenziosamente / Nascondersi nelle Ombre +3, Consapevolezza +2

\textbf{Vulnerabilità ai Danni} da botta, fuoco

\textbf{Immunità ai Danni} freddo, veleno

\textbf{Immunità alle Condizioni} avvelenato

\textbf{Sensi} scurovisione 18 m

\textbf{Linguaggi} Aquan, Auran

\textbf{Sfida} 1/2 (100 PE)

\emph{\textbf{Falso Aspetto.}} Mentre il mefito rimane immobile, è
indistinguibile da un ordinario frammento di ghiaccio.

\emph{\textbf{Incantesimi Innati (1/Giorno).}} Il mefito può lanciare in
maniera innata \emph{nube di nebbia}, senza bisogno di componenti
materiali. La sua caratteristica da incantatore innato è il Carisma.

\emph{\textbf{Natura Elementale.}} Un mefito non ha bisogno di cibo,
bevande o sonno.

\emph{\textbf{Scoppio Mortale.}} Quando il mefito muore, esplode in uno scoppio di frammenti di ghiaccio. Ogni creatura entro 1,5 metri da esso deve effettuare un tiro salvezza di Riflessi CD 10 o subire 4 (1d8) danni taglienti in caso di fallimento, o la metà di questi danni in caso
di successo.

\textbf{Azioni}

\emph{\textbf{Artigli.} Attacco con arma da mischia}: +3 a colpire,
portata 1 m, una creatura.

\emph{Colpisce:} 3 (1d4 + 1) danni taglienti più 2 (1d4) danni da
freddo.

\emph{\textbf{Soffio Gelido (Ricarica 6).}} Il mefito esala un cono di 4,5 metri di aria fredda. Ogni creatura nell'area deve effettuare un tiro salvezza di Riflessi CD 10, subendo 5 (2d4) danni da freddo in caso di fallimento, o la metà di questi danni in caso di successo.

\medskip\index{Mostri - Mefito di Magma}\textbf{Mefito di Magma}

\emph{Piccola elementale, neutrale malvagio}

\textbf{FORZA} -1

\textbf{DESTREZZA} +1

\textbf{COSTITUZIONE} +1

\textbf{INTELLIGENZA} -2

\textbf{SAGGEZZA} +0

\textbf{CARISMA} +0

\textbf{Iniziativa} +1 -- \textbf{Difesa} 12

\textbf{Punti Ferita} 22 (5d6 + 5)

\textbf{Movimento} 9 m, volo 9 m

\textbf{Tiri Salvezza}: Tempra +2, Riflessi +5, Volontà +3

\textbf{Competenze} Muoversi Silenziosamente / Nascondersi nelle Ombre +3

\textbf{Vulnerabilità ai Danni} freddo

\textbf{Immunità ai Danni} fuoco, veleno

\textbf{Immunità alle Condizioni} avvelenato

\textbf{Sensi} scurovisione 18 m

\textbf{Linguaggi} Ignan, Terran

\textbf{Sfida} 1/2 (100 PE)

\emph{\textbf{Falso Aspetto.}} Mentre il mefito rimane immobile, è
indistinguibile da un'ordinaria pozza di magma.

\emph{\textbf{Incantesimi Innati (1/Giorno).}} Il mefito può lanciare in
maniera innata \emph{riscaldare metallo} (CD del tiro salvezza
dell'incantesimo 10), senza bisogno di componenti materiali. La sua
caratteristica da incantatore innato è il Carisma.

\emph{\textbf{Natura Elementale.}} Un mefito non ha bisogno di cibo,
bevande o sonno.

\emph{\textbf{Scoppio Mortale.}} Quando il mefito muore, esplode in uno scoppio di lava. Ogni creatura entro 1,5 metri da esso deve effettuare un tiro salvezza di Riflessi CD 11 o subire 7 (2d6) danni da fuoco in caso di fallimento, o la metà di questi danni in caso di successo.

\textbf{Azioni}

\emph{\textbf{Artigli.} Attacco con arma da mischia}: +3 a colpire,
portata 1 m, una creatura.

\emph{Colpisce:} 3 (1d4 + 1) danni taglienti più 2 (1d4) danni da fuoco.

\emph{\textbf{Soffio Infuocato (Ricarica 6).}} Il mefito esala un cono
di 4,5 metri di fuoco. Ogni creatura nell'area deve effettuare un tiro
salvezza su Riflessi CD 11, subendo 7 (2d6) danni da fuoco in caso di
fallimento, o la metà di questi danni in caso di successo.



\medskip\index{Mostri - Mefito di Polvere}\textbf{Mefito di Polvere}

\emph{Piccola elementale, neutrale malvagio}

\textbf{FORZA} -3

\textbf{DESTREZZA} +2

\textbf{COSTITUZIONE} +0

\textbf{INTELLIGENZA} -1

\textbf{SAGGEZZA} +0

\textbf{CARISMA} +0

\textbf{Iniziativa} +2 -- \textbf{Difesa} 13

\textbf{Punti Ferita} 17 (5d6)

\textbf{Movimento} 9 m, volo 9 m

\textbf{Tiri Salvezza}: Tempra +2, Riflessi +5, Volontà +3

\textbf{Competenze} Muoversi Silenziosamente / Nascondersi nelle Ombre +4, Consapevolezza +2

\textbf{Vulnerabilità ai Danni} fuoco

\textbf{Immunità ai Danni} veleno

\textbf{Immunità alle Condizioni} avvelenato

\textbf{Sensi} scurovisione 18 m

\textbf{Linguaggi} Auran, Terran

\textbf{Sfida} 1/2 (100 PE)

\emph{\textbf{Incantesimi Innati (1/Giorno).}} Il mefito può eseguire in
maniera innata \emph{sonno} (CD del tiro salvezza dell'incantesimo 10),
senza bisogno di componenti materiali. La sua abilità da incantatore
innato è il Carisma.

\emph{\textbf{Natura Elementale.}} Un mefito non ha bisogno di cibo,
bevande o sonno.

\emph{\textbf{Scoppio Mortale.}} Quando il mefito muore, esplode in uno scoppio di polvere. Ogni creatura entro 1,5 metri da esso deve riuscire un tiro salvezza di Tempra CD 10 o restare accecata per 1 minuto. Una creatura accecata può ripetere il tiro salvezza durante ciascun suo turno, terminando l'effetto su di sé in caso di successo.

\textbf{Azioni}

\emph{\textbf{Artigli.} Attacco con arma da mischia}: +4 a colpire,
portata 1 m, una creatura.

\emph{Colpisce:} 4 (1d4 + 2) danni taglienti.

\emph{\textbf{Soffio Accecante (Ricarica 6).}} Il mefito esala un cono di 4,5 metri di polvere accecante. Ogni creatura nell'area deve riuscire un tiro salvezza di Riflessi CD 10 o restare accecata per 1 minuto. Una creatura accecata può ripetere il tiro salvezza durante ciascun suo turno, terminando l'effetto su di sé in caso di successo.

\medskip\index{Mostri - Mefito di Vapore}\textbf{Mefito di Vapore}

\emph{Piccola elementale, neutrale malvagio}

\textbf{FORZA} -3

\textbf{DESTREZZA} +0

\textbf{COSTITUZIONE} +0

\textbf{INTELLIGENZA} +0

\textbf{SAGGEZZA} +0

\textbf{CARISMA} +1

\textbf{Iniziativa} +0 -- \textbf{Difesa} 11

\textbf{Punti Ferita} 21 (6d6)

\textbf{Movimento} 9 m, volo 9 m

\textbf{Tiri Salvezza}: Tempra +2, Riflessi +5, Volontà +3

\textbf{Immunità ai Danni} fuoco, veleno

\textbf{Immunità alle Condizioni} avvelenato

\textbf{Sensi} scurovisione 18 m

\textbf{Linguaggi} Aquan, Ignan

\textbf{Sfida} 1/4 (50 PE)

\emph{\textbf{Incantesimi Innati (1/Giorno).}} Il mefito può eseguire in
maniera innata \emph{sfocatura}, senza bisogno di componenti materiali.
La sua abilità da incantatore innato è il Carisma.

\emph{\textbf{Natura Elementale.}} Un mefito non ha bisogno di cibo,
bevande o sonno.

\emph{\textbf{Scoppio Mortale.}} Quando il mefito muore, esplode in nube
di vapore. Ogni creatura entro 1,5 metri da esso deve riuscire un tiro
salvezza su Riflessi CD 10 o subire 4 (1d8) danni da fuoco.

\textbf{Azioni}

\emph{\textbf{Artigli.} Attacco con arma da mischia}: +2 a colpire,
portata 1 m, una creatura.

\emph{Colpisce:} 2 (1d4) danni taglienti più 2 (1d4) danni da fuoco.

\emph{\textbf{Soffio Vaporoso (Ricarica 6).}} Il mefito esala un cono di 4,5 metri di vapore caldo. Ogni creatura nell'area deve effettuare un tiro salvezza di Riflessi CD 10, subendo 4 (1d8) danni da fuoco in caso di fallimento, o la metà di questi danni in caso di successo.

\subsection{Megere}

\medskip\index{Mostri - Megera Marina}\textbf{Megera Marina}

\emph{Media fatato, caotico malvagio}

\textbf{FORZA} +3

\textbf{DESTREZZA} +1

\textbf{COSTITUZIONE} +3

\textbf{INTELLIGENZA} +1

\textbf{SAGGEZZA} +1

\textbf{CARISMA} +1

\textbf{Iniziativa} +1 -- \textbf{Difesa} 15

\textbf{Punti Ferita} 52 (7d8 + 21)

\textbf{Vulnerabilità al Danno} ferro freddo

\textbf{Movimento} 9 m, nuoto 12 m

\textbf{Tiri Salvezza}: Tempra +5, Riflessi +7, Volontà +5

\textbf{Sensi} scurovisione 18 m

\textbf{Linguaggi} Aquan, Comune, Gigante

\textbf{Sfida} 2 (450 PE)

\emph{\textbf{Anfibio.}} La megera può respirare aria e acqua.

\emph{\textbf{Aspetto Orripilante.}} Qualsiasi umanoide che inizi il suo turno entro 9 metri dalla megera e ne può vedere la vera forma deve effettuare un tiro salvezza di Volontà CD 11. Se fallisce il tiro salvezza, la creatura resta spaventata per 1 minuto. Una creatura può ripetere il tiro salvezza al termine di ciascun suo turno, con -1d6 se la megera è in linea di visuale, e terminando l'effetto se riesce il tiro salvezza. Se il tiro salvezza della creatura riesce o l'effetto ha termine su di essa, la creatura è immune all'Aspetto Orripilante per le successive 24 ore.

A meno che il bersaglio non sia sorpreso o la rivelazione della vera
forma della megera non sia improvvisa, il bersaglio può distogliere lo
sguardo e evitare di effettuare il tiro salvezza iniziale. Fino
all'inizio del suo prossimo turno, una creatura che distolga lo sguardo
ha -1d6 ai tiri di attacco contro la megera.

\textbf{Azioni}

\emph{\textbf{Artigli.} Attacco in mischia con arma}: +5 a colpire,
portata 1 m, un bersaglio.

\emph{Colpisce:} 10 (2d6 + 3) danni taglienti.

\emph{\textbf{Aspetto Illusorio.}} La megera ricopre se stessa e tutto
quello che sta indossando o trasportando in un'illusione magica che le
dona l'aspetto di una creatura ripugnante all'incirca della stessa
taglia e forma umanoide. L'illusione termina se la megera effettua
un'azione bonus per terminarla o se muore.

I cambiamenti apportati da questo effetto non sono in grado di superare le ispezioni fisiche. Ad esempio, la megera potrebbe apparire come una creatura priva di artigli, ma una persona in contatto con le sue mani li avvertirebbe. Altrimenti, una creatura deve effettuare un'azione per ispezionare visivamente l'illusione e riuscire una prova di Intelligenza CD 16 per comprendere che la megera si è camuffata.

\emph{\textbf{Occhiata Mortale.}} La megera prende a bersaglio una creatura spaventata visibile entro 9 metri da lei. Se il bersaglio può vedere la megera, deve riuscire un tiro salvezza di Volontà CD 11 contro questa magia o scendere a 0 punti ferita.

\medskip\index{Mostri - Megera Notturna}\textbf{Megera Notturna}

\emph{Media immondo, neutrale malvagio}

\textbf{FORZA} +4

\textbf{DESTREZZA} +2

\textbf{COSTITUZIONE} +3

\textbf{INTELLIGENZA} +3

\textbf{SAGGEZZA} +2

\textbf{CARISMA} +3

\textbf{Iniziativa} +3 -- \textbf{Difesa} 20

\textbf{Punti Ferita} 112 (15d8 + 45)

\textbf{Movimento} 9 m

\textbf{Tiri Salvezza}: Tempra +14, Riflessi +8, Volontà +11

\textbf{Competenze} Muoversi Silenziosamente / Nascondersi nelle Ombre +6, Ingannare +7, Percepire Emozioni +6, Consapevolezza +6,

\textbf{Resistenze al Danno} freddo, fuoco; da botta, perforante e tagliente di attacchi non magici o non siano argentati

\textbf{Sensi} scurovisione 36 m

\textbf{Linguaggi} Abissale, Comune, Infernale, Druidico

\textbf{Sfida} 5 (1.800 PE)

\emph{\textbf{Incantesimi Innati.}} La caratteristica da incantatore
innato della megera è il Carisma (CD 14 per i tiri salvezza degli
incantesimi, +6 a colpire con attacchi da incantesimo). La megera può
lanciare in maniera innata i seguenti incantesimi, senza aver bisogno di
componenti materiali.

A volontà: \emph{dardo incantato, individuazione del magico} 2/giorno
ciascuno: \emph{raggio di indebolimento, sonno, spostamento}
\emph{planare} (personale)

\emph{\textbf{Resistenza alla Magia.}} La megera ha +1d6 ai tiri
salvezza contro incantesimi e altri effetti magici.

\textbf{Azioni}

\emph{\textbf{Artigli (Solo in Forma di Megera).} Attacco con arma da
	mischia}: +7 a colpire, portata 1 m, un bersaglio. \emph{Colpisce:} 13
(2d8 + 4) danni taglienti.

\emph{\textbf{Forma Eeterea.}} La megera entra magicamente nel Piano
Etereo dal Piano Materiale, e viceversa. Per farlo deve essere in
possesso di un \emph{cuore di pietra}.

\emph{\textbf{Infestare Incubi (1/Giorno).}} Mentre si trova sul Piano
Etereo, la megera entra magicamente in contatto con un umanoide
addormentato che si trova sul Piano Materiale. L'incantesimo
\emph{protezione dal bene e dal male} lanciato sul bersaglio previene
questo contatto, così come \emph{cerchio magico}. Finché il contatto
persiste, il bersaglio soffre di orribili visioni. Se queste visioni
durano per almeno 1 ora, il bersaglio non ottiene benefici dal suo
riposo, e i suoi punti ferita massimi sono ridotti di 5 (1d10). Se
questo effetto riduce i punti ferita massimi del bersaglio a 0, il
bersaglio muore, e se il bersaglio era malvagio, la sua anima resta
intrappolata nella \emph{borsa} \emph{delle anime} della megera. La
riduzione dei punti ferita massimi del bersaglio rimane finché non viene
rimossa dall'incantesimo \emph{ristorare} \emph{superiore} o simile
magia.

\emph{\textbf{Mutare Forma.}} La megera può trasformarsi magicamente in
una femmina umanoide di taglia Piccola o Media, o tornare alla sua vera
forma. Le sue statistiche sono le stesse in qualsiasi forma. Tutto
l'equipaggiamento che stava trasportando o indossando non viene
trasformato. Alla morte, ritorna alla sua vera forma.



\medskip\index{Mostri - Megera Verde}\textbf{Megera Verde}

\emph{Media fatato, neutrale malvagio}

\textbf{FORZA} +4

\textbf{DESTREZZA} +1

\textbf{COSTITUZIONE} +3

\textbf{INTELLIGENZA} +1

\textbf{SAGGEZZA} +2

\textbf{CARISMA} +2

\textbf{Iniziativa} +1 -- \textbf{Difesa} 19

\textbf{Punti Ferita} 82 (11d8 + 33)

\textbf{Vulnerabilità al Danno} ferro freddo

\textbf{Movimento} 9 m

\textbf{Tiri Salvezza}: Temp. +6, Rifl. +7, Vol. +7

\textbf{Competenze} Arcano +3, Muoversi Silenziosamente / Nascondersi nelle Ombre +3, Ingannare +4, Consapevolezza +4

\textbf{Sensi} scurovisione 18 m

\textbf{Linguaggi} Comune, Draconico, Silvano

\textbf{Sfida} 3 (700 PE)

\emph{\textbf{Anfibio.}} La megera può respirare aria e acqua.

\emph{\textbf{Imitazione.}} La megera può imitare suoni animali e voci umanoidi. Una creatura che senta questi rumori può determinare che si tratti di un'imitazione riuscendo una prova di Saggezza CD 14.

\emph{\textbf{Incantesimi Innati.}} La caratteristica da incantatore innato della megera è il Carisma (CD 12 per i tiri salvezza degli incantesimi). La megera può lanciare in maniera innata i seguenti incantesimi, senza aver bisogno di componenti materiali.

A volontà: \emph{illusione minore, luci danzanti, beffa maligna}

\textbf{Azioni}

\emph{\textbf{Artigli.} Attacco con arma da mischia}: +6 a colpire,
portata 1 m, un bersaglio.

\emph{Colpisce:} 13 (2d8 + 4) danni taglienti.

\emph{\textbf{Aspetto Illusorio.}} La megera ricopre sé stessa e tutto quello che sta indossando o trasportando in un'illusione magica che le dona l'aspetto di un'altra creatura all'incirca della stessa taglia e forma umanoide. L'illusione termina se la megera effettua un'azione bonus per terminarla o se muore.

I cambiamenti apportati da questo effetto non sono in grado di superare le ispezioni fisiche. Ad esempio, la megera potrebbe apparire come una creatura dalla pelle liscia, ma il contatto rivelerebbe la sua pelle ruvida. Altrimenti, una creatura deve effettuare un'azione per ispezionare visivamente l'illusione e riuscire una prova di Intelligenza CD 20 per comprendere che si tratta di una megera camuffata.

\emph{\textbf{Passaggio Invisibile.}} La megera può rendersi invisibile finché non attacca o lancia un incantesimo, o finché non termina la concentrazione (come se si stesse concentrando su di un incantesimo). Mentre è invisibile, non lascia traccia fisica del suo passaggio, quindi le sue tracce possono essere seguite solo dalla magia. Tutto l'equipaggiamento che sta trasportando o indossando diventa invisibile assieme a lei.

\subsection{Melme}

\medskip\index{Mostri - Ameba Paglierina}\textbf{Ameba Paglierina}

\emph{Grande melma, disallineato}

\textbf{FORZA} +2

\textbf{DESTREZZA} -2

\textbf{COSTITUZIONE} +2

\textbf{INTELLIGENZA} -4

\textbf{SAGGEZZA} -2

\textbf{CARISMA} -5

\textbf{Iniziativa} +2 -- \textbf{Difesa} 9

\textbf{Punti Ferita} 45 (6d10 + 12)

\textbf{Movimento} 3 m, scalata 3 m

\textbf{Tiri Salvezza}: Tempra +8, Riflessi -3, Volontà -3

\textbf{Resistenze al Danno} acido

\textbf{Immunità al Danno} fulmine, tagliente

\textbf{Immunità alle Condizioni} accecato, affascinato, assordato,
prono, sfinimento, spaventato

\textbf{Sensi} vista cieca 18 m (cieca oltre questo raggio)

\textbf{Linguaggi} -

\textbf{Sfida} 2 (450 PE)

\emph{\textbf{Amorfo.}} L'ameba può muoversi attraverso uno spazio fino
a 2,5 centimetri di larghezza senza doversi stringere.

\emph{\textbf{Natura di Melma.}} L'ameba non necessita di dormire.

\emph{\textbf{Scalare come Ragno.}} L'ameba può scalare superfici
difficili, compreso lo stare a testa in giù sul soffitto, senza bisogno
di effettuare una prova di abilità.

\textbf{Azioni}

\emph{\textbf{Pseudopodo.} Attacco con arma da mischia}: +4 a colpire,
portata 1 m, un bersaglio.

\emph{Colpisce:} 9 (2d6 + 2) danni da botta più 3 (1d6) danni da
acido.

\textbf{Reazioni}

\emph{\textbf{Divisione.}} Quando un'ameba Media o più grande subisce danni da fulmine o taglienti, si divide in due nuove amebe che hanno almeno 10 punti ferita. Ogni nuova ameba ha un numero di punti ferita pari alla metà dell'ameba originale, arrotondati per difetto. Le nuove amebe sono di una taglia più piccola di quella originale.

\medskip\index{Mostri - Cubo Gelatinoso}\textbf{Cubo Gelatinoso}

\emph{Grande melma, disallineato}

\textbf{FORZA} +2

\textbf{DESTREZZA} -4

\textbf{COSTITUZIONE} +5

\textbf{INTELLIGENZA} -5

\textbf{SAGGEZZA} -2

\textbf{CARISMA} -5

\textbf{Iniziativa} -4 -- \textbf{Difesa} 7

\textbf{Punti Ferita} 84 (8d10 + 40)

\textbf{Movimento} 4,5 m

\textbf{Tiri Salvezza}: Tempra +9, Riflessi -4, Volontà -4

\textbf{Immunità alle Condizioni} accecato, affascinato, assordato, prono, sfinimento, spaventato

\textbf{Sensi} vista cieca 18 m (cieca oltre questo raggio)

\textbf{Linguaggi} -

\textbf{Sfida} 2 (450 PE)

\emph{\textbf{Cubo di Melma.}} Il cubo occupa il suo intero spazio. Le
altre creature possono entrare nello spazio, ma rimangono vittima del
Sommergere del cubo e hanno -1d6 al tiro salvezza.

Le creature all'interno del cubo sono visibili ma godono di copertura
totale.

Una creatura entro 1,5 metri dal cubo può effettuare un'azione per tirare una creatura od oggetto fuori dal cubo. Farlo richiede la riuscita di una prova di Forza CD 12, e la creatura che effettua il tentativo subisce 10 (3d6) danni da acido.

Il cubo può contenere solo una creatura Grande o un massimo di quattro
creature Medie o più piccole alla volta.

\emph{\textbf{Natura di Melma.}} Il cubo non necessita di dormire.

\emph{\textbf{Trasparente.}} Anche quando è in piena vista, è necessario riuscire una prova di Saggezza (Consapevolezza) CD 15 per notare un cubo che non si è mosso o non ha attaccato. Una creatura che cerchi di entrare nello spazio del cubo mentre è inconsapevole della sua presenza resta sorpresa dal cubo.

\textbf{Azioni}

\emph{\textbf{Pseudopodo.} Attacco con arma da mischia}: +4 a colpire,
portata 1 m, un bersaglio.

\emph{Colpisce:} 10 (3d6) danni da acido.

\emph{\textbf{Sommergere.}} Il cubo si muove fino al massimo del suo movimento. Nel farlo, può entrare nello spazio di una creatura di taglia Grande o più piccola. Ogni volta che il cubo entra nello spazio di una creatura, la creatura deve effettuare un tiro salvezza di Riflessi CD 12.

Se il tiro salvezza riesce, la creatura può scegliere di essere spinta indietro o di lato di 1,5 metri. Una creatura che decida di non farsi spingere subisce le conseguenze di un tiro salvezza fallito.

Se il tiro salvezza fallisce, il cubo entra nello spazio della creatura, che subisce 10 (3d6) danni da acido ed è sommersa. La creatura sommersa non può respirare, è intralciata e subisce 21 (6d6) danni da acido all'inizio del turno del cubo. Quando il cubo si muove, la creatura sommersa si muove con esso.

Una creatura sommersa può tentare di fuggire effettuando un'azione per compiere una prova di Forza CD 12. Se la riesce, la creatura sfugge e ed entra nello spazio di sua scelta entro 1,5 metri dal cubo.

\medskip\index{Mostri - Melma Grigia}\textbf{Melma Grigia}

\emph{Media melma, disallineato}

\textbf{FORZA} +1

\textbf{DESTREZZA} -2

\textbf{COSTITUZIONE} +3

\textbf{INTELLIGENZA} -5

\textbf{SAGGEZZA} -2

\textbf{CARISMA} -4

\textbf{Iniziativa} -2 -- \textbf{Difesa} 9

\textbf{Punti Ferita} 22 (3d8 + 9)

\textbf{Movimento} 3 m, scalata 3 m

\textbf{Tiri Salvezza}: Tempra +9, Riflessi -4, Volontà -4

\textbf{Resistenze al Danno} acido, freddo, fuoco

\textbf{Immunità alle Condizioni} accecato, affascinato, assordato,
prono, sfinimento, spaventato

\textbf{Sensi} vista cieca 18 m (cieca oltre questo raggio)

\textbf{Linguaggi} -

\textbf{Sfida} 1/2 (100 PE)

\emph{\textbf{Amorfo.}} La melma può muoversi attraverso uno spazio fino a 2,5 centimetri di larghezza senza doversi stringere.

\emph{\textbf{Corrodere Metallo.}} Qualsiasi arma non magica fatta di metallo che colpisca la melma si corrode. Dopo aver inflitto il danno, l'arma subisce una penalità permanente e cumulativa di -1 ai tiri di danno. Se la penalità arriva a -5, l'arma è distrutta. Le munizioni non magiche fatte di metallo che colpiscano la melma, si distruggono dopo aver inflitto il danno.

La melma può divorare metallo non magico dello spessore di 5 centimetri in un 1 round.

\emph{\textbf{Falso Aspetto.}} Quando la melma rimane immobile, è indistinguibile da una pozza d'olio o una pietra bagnata.

\emph{\textbf{Natura di Melma.}} La melma non necessita di dormire.

\textbf{Azioni}

\emph{\textbf{Pseudopodo.} Attacco con arma da mischia}: +3 a colpire, portata 1 m, un bersaglio.

\emph{Colpisce:} 4 (1d6 + 1) danni da botta più 7 (2d6) danni da acido, e se il bersaglio sta indossando un'armatura di metallo, questa viene parzialmente dissolta e subisce una penalità permanente e cumulativa di -1 alla Difesa che offre. L'armatura è distrutta se la penalità riduce la sua Difesa a 10.


\medskip\index{Mostri - Protoplasma Nero}\textbf{Protoplasma Nero}

\emph{Grande melma, disallineato}

\textbf{FORZA} +3

\textbf{DESTREZZA} -3

\textbf{COSTITUZIONE} +3

\textbf{INTELLIGENZA} -5

\textbf{SAGGEZZA} -2

\textbf{CARISMA} -5

\textbf{Iniziativa} -3 -- \textbf{Difesa} 9

\textbf{Punti Ferita} 85 (10d10 + 30)

\textbf{Movimento} 6 m, scalata 6 m

\textbf{Tiri Salvezza}: Tempra +9, Riflessi -2, Volontà -2

\textbf{Immunità al Danno} acido, freddo, fulmine, tagliente

\textbf{Immunità alle Condizioni} accecato, affascinato, assordato, prono, sfinimento, spaventato

\textbf{Sensi} vista cieca 18 m (cieco oltre questo raggio)

\textbf{Linguaggi} -

\textbf{Sfida} 4 (1.100 PE)

\emph{\textbf{Amorfo.}} Il protoplasma nero può muoversi attraverso uno spazio fino a 2,5 centimetri di larghezza senza doversi stringere.

\emph{\textbf{Forma Corrosiva.}} Una creatura che entri a contatto col protoplasma nero o lo colpisca con un attacco da mischia mentre si trova entro 1,5 metri da esso subisce 4 (1d8) danni da acido. Qualsiasi arma non magica fatta di metallo o legno che colpisca il protoplasma nero si corrode. Dopo aver inflitto il danno, l'arma subisce una penalità permanente e cumulativa di -1 ai tiri di danno. Se la penalità arriva a -5, l'arma è distrutta. Le munizioni non magiche fatte di metallo o legno che colpiscano il protoplasma nero, si distruggono dopo aver inflitto il danno.

Il protoplasma nero può divorare legno o metallo non magico dello spessore di 5 centimetri in un 1 round.

\emph{\textbf{Natura di Melma.}} Il protoplasma nero non necessita di dormire.

\emph{\textbf{Scalare come Ragno.}} Il protoplasma nero può scalare superfici difficili, compreso lo stare a testa in giù sul soffitto, senza bisogno di effettuare una prova di abilità.

\textbf{Azioni}

\emph{\textbf{Pseudopodo.} Attacco con arma da mischia}: +5 a colpire,
portata 1 m, un bersaglio.

\emph{Colpisce:} 6 (1d6 + 3) danni da botta più 18 (4d8) danni da acido. Inoltre, un'armatura non magica indossata dal bersaglio viene parzialmente dissolta e subisce una penalità permanente e cumulativa di -1 alla Difesa che offre. L'armatura è distrutta se la penalità riduce la sua Difesa a 10.

\textbf{Reazioni}

\emph{\textbf{Divisione.}} Quando un protoplasma nero di taglia Media o più grande subisce danni da fulmine o taglienti, si divide in due nuovi protoplasma neri di almeno 10 punti ferita ciascuno. Ogni nuovo protoplasma nero ha un numero di punti ferita pari alla metà del protoplasma nero originale, arrotondati per difetto. I nuovi protoplasmi neri sono di una taglia più piccola di quella originale.


\medskip\index{Mostri - Mimic}\textbf{Mimic}

\emph{Media mostruosità (mutaforma), neutrale}

\textbf{FORZA} +3

\textbf{DESTREZZA} +1

\textbf{COSTITUZIONE} +2

\textbf{INTELLIGENZA} -3

\textbf{SAGGEZZA} +1

\textbf{CARISMA} -1

\textbf{Iniziativa} +1 -- \textbf{Difesa} 13

\textbf{Punti Ferita} 58 (9d8 + 18)

\textbf{Movimento} 4,5 m

\textbf{Tiri Salvezza}: Tempra +5, Riflessi +5, Volontà +6

\textbf{Competenze} Muoversi Silenziosamente / Nascondersi nelle Ombre +5

\textbf{Immunità al Danno} acido

\textbf{Immunità alle Condizioni} prono

\textbf{Sensi} scurovisione 18 m

\textbf{Linguaggi} -

\textbf{Sfida} 2 (450 PE)

\emph{\textbf{Aderente (Solo Forma di Oggetto).}} Il mimic aderisce a qualsiasi cosa con cui entri in contatto. Una creatura di taglia Enorme o inferiore a cui il mimic aderisce è considerata afferrata da esso (CD 13 per fuggire). Le prove di caratteristica effettuare per fuggire da
questo afferrare hanno -1d6.

\emph{\textbf{Afferratore.}} Il mimic ha +1d6 ai tiri per colpire contro una creatura da esso afferrata.

\emph{\textbf{Falso Aspetto (Solo Forma di Oggetto).}} Mentre il mimic rimane immobile, è indistinguibile da un comune oggetto.

\emph{\textbf{Mutaforma.}} Il mimic può usare la sua azione per trasformarsi in un oggetto, o per tornare alla sua vera forma amorfa. Le sue statistiche sono le stesse in qualsiasi forma. Qualsiasi equipaggiamento stia indossando o trasportando non si trasforma. Alla morte ritorna al suo vero aspetto.

\textbf{Azioni}

\emph{\textbf{Morso.} Attacco con arma da mischia}: +5 a colpire, portata 1 m, un bersaglio.

\emph{Colpisce:} 7 (1d8 + 3) danni perforanti più 4 (1d8) danni da acido.

\emph{\textbf{Pseudopodo.} Attacco con arma da mischia}: +5 a colpire, portata 1 m, un bersaglio.

\emph{Colpisce:} 7 (1d8 + 3) danni da botta. Se il mimic è in forma di oggetto, il bersaglio è vittima del tratto Aderente.



\medskip\index{Mostri - Minotauro}\textbf{Minotauro}

\emph{Grande mostruosità, caotico malvagio}

\textbf{FORZA} +4

\textbf{DESTREZZA} +0

\textbf{COSTITUZIONE} +3

\textbf{INTELLIGENZA} -2

\textbf{SAGGEZZA} +3

\textbf{CARISMA} -1

\textbf{Iniziativa} +0 -- \textbf{Difesa} 16

\textbf{Punti Ferita} 76 (9d10 + 27)

\textbf{Movimento} 12 m

\textbf{Tiri Salvezza}: Tempra +6, Riflessi +5, Volontà +5

\textbf{Competenze} Consapevolezza +7

\textbf{Sensi} scurovisione 18 m

\textbf{Linguaggi} Abissale

\textbf{Sfida} 3 (700 PE)

\emph{\textbf{Carica.}} Se il minotauro si muove di almeno 3 metri diretto verso un bersaglio e lo colpisce con un attacco di incornata durante lo stesso turno, il bersaglio subisce 9 (2d8) danni perforanti aggiuntivi. Se il bersaglio è una creatura, deve riuscire un tiro salvezza su Tempra CD 14 o venire spinto via fino a 3 metri di distanza e cadere prono.

\emph{\textbf{Incauto.}} All'inizio del suo turno, il minotauro può ottenere +1d6 su tutti i tiri per colpire con armi da mischia effettuati durante quel turno, ma i tiri per colpire contro di esso hanno +1d6 fino all'inizio del suo prossimo turno.

\emph{\textbf{Ricordare Labirinto.}} Il minotauro può ricordare perfettamente qualsiasi tragitto abbia percorso.

\textbf{Azioni}

\emph{\textbf{Ascia Bipenne.} Attacco con arma da mischia}: +6 a colpire, portata 1 m, un bersaglio.

\emph{Colpisce:} 17 (2d12 + 4) danni taglienti.

\emph{\textbf{Incornata.} Attacco con arma da mischia}: +6 a colpire, portata 1 m, un bersaglio.

\emph{Colpisce:} 13 (2d8 + 4) danni perforanti.

\subsection{Mummie}

\medskip\index{Mostri - Mummia}\textbf{Mummia}

\emph{Media non morto, legale malvagio}

\textbf{FORZA} +3

\textbf{DESTREZZA} -1

\textbf{COSTITUZIONE} +2

\textbf{INTELLIGENZA} -2

\textbf{SAGGEZZA} +0

\textbf{CARISMA} +1

\textbf{Iniziativa} -1 -- \textbf{Difesa} 13

\textbf{Punti Ferita} 58 (9d8 + 18)

\textbf{Movimento} 6 m

\textbf{Tiri Salvezza}: Tempra +4, Riflessi +2, Volontà +8

\textbf{Vulnerabilità al Danno} fuoco

\textbf{Resistenze al Danno} da botta, perforante e tagliente di attacchi non magici

\textbf{Immunità al Danno} da Vuoto, veleno

\textbf{Immunità alle Condizioni} affascinato, avvelenato, paralizzato, sfinimento, spaventato

\textbf{Sensi} scurovisione 18 m 

\textbf{Linguaggi} le lingue che conosceva in vita

\textbf{Sfida} 3 (700 PE)

\emph{\textbf{Natura Non Morta.}} Un mummia non ha bisogno di aria, cibo, bevande o sonno.

\textbf{Azioni}

\emph{\textbf{Multiattacco.}} La mummia può usare la sua Occhiata Temibile ed effettuare un attacco con il pugno putrefacente.

\emph{\textbf{Pugno Putrefacente.} Attacco con arma da mischia}: +5 a colpire, portata 1 m, un bersaglio.

\emph{Colpisce:} 10 (2d6 + 3) danni da botta più 10 (3d6) danni da Vuoto. Se il bersaglio è una creatura deve riuscire un tiro salvezza su Tempra 12 o venire maledetto dalla putrefazione della mummia. Il bersaglio maledetto non può recuperare punti ferita, e i suoi punti ferita massimi diminuiscono di 10 (3d6) ogni 24 ore di durata della maledizione. Se la maledizione riduce i punti ferita massimi del bersaglio a 0, il bersaglio muore, e il suo corpo si tramuta in polvere. La maledizione dura finché non viene rimossa dall'incantesimo \emph{rimuovi maledizione} o altra magia.

\emph{\textbf{Occhiata Temibile.}} La mummia prende a bersaglio una creatura che possa vedere e si trovi entro 18 metri da lei. Se il bersaglio può vedere la mummia, deve riuscire un tiro salvezza su Volontà CD 11 contro questa magia o restare spaventato fino al termine del prossimo turno della mummia. Se il bersaglio fallisce il tiro salvezza di 5 o più, è anche paralizzato per la stessa durata. Un bersaglio che riesca il tiro salvezza è immune all'Occhiata Terribile di tutte le mummie (ma non delle mummie sovrane) per le successive 24 ore.

\medskip\index{Mostri - Mummia Sovrana}\textbf{Mummia Sovrana}

\emph{Media non morto, legale malvagio}

\textbf{FORZA} +4

\textbf{DESTREZZA} +0

\textbf{COSTITUZIONE} +3

\textbf{INTELLIGENZA} +0

\textbf{SAGGEZZA} +4

\textbf{CARISMA} +3

\textbf{Iniziativa} +0 -- \textbf{Difesa} 25

\textbf{Punti Ferita} 97 (13d8 + 39)

\textbf{Movimento} 6 m

\textbf{Tiri Salvezza}: Tempra +12, Riflessi +6, Volontà +16

\textbf{Competenze} Religione +5, Storia +5

\textbf{Vulnerabilità al Danno} fuoco

\textbf{Immunità al Danno} da Vuoto, veleno; armi +1

\textbf{Immunità alle Condizioni} affascinato, avvelenato, paralizzato,
sfinimento, spaventato

\textbf{Sensi} scurovisione 18 m

\textbf{Linguaggi} le lingue che conosceva in vita

\textbf{Sfida} 15 (13.000 PE)

\emph{\textbf{Cuore della Mummia Sovrana.}} Come parte del rituale che crea una mummia sovrana, il cuore e le viscere della creatura vengono rimossi dal cadavere e piazzati all'interno di contenitori sigillati. Questi contenitori sono di solito fatti in pietra o ceramica, incisi o dipinti con geroglifici religiosi.

Finché il suo cuore avvizzito rimane intatto, la mummia sovrana non può essere permanentemente distrutta. Quando scende a 0 punti ferita, la mummia sovrana si riduce in polvere e si riforma a piena forza 24 ore più tardi, riemergendo dalla polvere in prossimità della giara sigillata che contiene il suo cuore. Per impedire che una mummia sovrana si riformi e distruggerla una volta per tutte, bisogna ridurne il cuore in cenere. Per questo motivo, la mummia sovrana di solito tiene il cuore e le viscere nascoste all'interno di una tomba nascosta.

Il cuore della mummia sovrana ha Difesa 5, 25 punti ferita e immunità a tutti i danni eccetto il fuoco.

\emph{\textbf{Incantesimi.}} La mummia ha CM 10. La sua caratteristica da incantatore è la Saggezza, +9 a colpire con attacchi da incantesimo. La mummia ha preparati i seguenti incantesimi: Trucchetti (a volontà): \emph{fiamma sacra, taumaturgia}

Difficoltà 10 (4 slot): \emph{comando, dardo tracciante, scudo della fede}

Difficoltà 13 (3 slot): \emph{arma spirituale, blocca persone, silenzio}

Difficoltà 15 (3 slot): \emph{animare morti, dissolvi magie}

Difficoltà 18 (3 slot): \emph{divinazione, guardiano della fede}

Difficoltà 20 (2 slot): \emph{contagio, piaga degli insetti}

Difficoltà 23 (1 slot): \emph{ferire}

\emph{\textbf{Natura Non Morta.}} Un mummia non ha bisogno di aria, cibo, bevande o sonno.

\emph{\textbf{Resistenza alla Magia.}} La mummia sovrana ha +1d6 ai tiri salvezza contro incantesimi o altri effetti magici.

\emph{\textbf{Rinvigorimento.}} Una mummia sovrana forma un nuovo corpo entro 24 ore se il suo cuore resta intatto, recuperando tutti i punti ferita e potendo agire nuovamente. Il nuovo corpo compare entro 1,5 metri dal cuore della mummia sovrana.

\textbf{Azioni}

\emph{\textbf{Multiattacco.}} La mummia può usare la sua Occhiata Temibile ed effettuare un attacco con il pugno putrefacente.

\emph{\textbf{Pugno Putrefacente.} Attacco con arma da mischia}: +9 a colpire, portata 1 m, un bersaglio.

\emph{Colpisce:} 14 (3d6 + 4) danni da botta più 21 (6d6) danni da Vuoto. Se il bersaglio è una creatura deve riuscire un tiro salvezza su Tempra 16 o venire maledetto dalla putrefazione della mummia. Il bersaglio maledetto non può recuperare punti ferita, e i suoi punti ferita massimi diminuiscono di 10 (3d6) ogni 24 ore di durata della maledizione. Se la maledizione riduce i punti ferita massimi del bersaglio a 0, il bersaglio muore, e il suo corpo si tramuta in polvere. Lamaledizione dura finché non viene rimossa dall'incantesimo  \emph{rimuovere maledizione} o altra magia.

\emph{\textbf{Occhiata Temibile.}} La mummia prende a bersaglio una creatura che possa vedere e si trovi entro 18 metri da lei. Se il bersaglio può vedere la mummia, deve riuscire un tiro salvezza su Volontà CD 16 contro questa magia o restare spaventato fino al termine del prossimo turno della mummia. Se il bersaglio fallisce il tiro salvezza di 5 o più, è anche paralizzato per la stessa durata. Un bersaglio che riesca il tiro salvezza è immune all'Occhiata Terribile di tutte le mummie (ma non delle mummie sovrane) per le successive 24 ore.

\textbf{Azioni Aggiuntive}

La mummia sovrana può effettuare 3 Azioni aggiuntive, scelte tra le opzioni seguenti. Può usare solo un'opzione leggendaria alla volta e solo al termine del turno di un'altra creatura. La mummia sovrana recupera le Azioni aggiuntive spese all'inizio del proprio turno.

\emph{\textbf{Attaccare.}} La mummia sovrana effettua un attacco con il pugno putrefacente o usa la sua Occhiata Temibile.

\emph{\textbf{Incanalare Energia Negativa (Costa 2 Azioni).}} La mummia sovrana può scatenare magicamente l'energia negativa. Le creature entro 18 metri dalla mummia sovrana, comprese quelle dietro barriere o angoli, non possono recuperare punti ferita fino al termine del prossimo turno della mummia sovrana.

\emph{\textbf{Parola Blasfema (Costa 2 Azioni).}} La mummia sovrana pronuncia una parola blasfema. Ciascuna creatura, esclusi i non morti, entro 3 metri dalla mummia sovrana e che possa udire questa frase magica deve riuscire un tiro salvezza di Tempra CD 16 o restare stordita fino al termine del prossimo turno della mummia sovrana.

\emph{\textbf{Polvere Accecante.}} Polvere e sabbia accecanti turbinano magicamente intorno alla mummia sovrana. Ogni creatura entro 1,5 metri dalla mummia sovrana deve riuscire un tiro salvezza di Tempra CD 16 o restare accecata fino al termine del prossimo turno della creatura.

\emph{\textbf{Turbine di Sabbia (Costa 2 Azioni).}} La mummia sovrana può trasformarsi magicamente in un turbine di sabbia, muovendosi di massimo 18 metri, e tornando poi alla sua forma normale. Mentre è in forma di turbine, la mummia sovrana è immune a tutti i danni, e non può essere afferrata, pietrificata, gettata prona, intralciata o stordita. L'equipaggiamento indossato o trasportato dalla mummia sovrana rimane in suo possesso.

\subsection{Naga}

\medskip\index{Mostri - Naga Guardiano}\textbf{Naga Guardiano}

\emph{Grande mostruosità, legale buono}

\textbf{FORZA} +4

\textbf{DESTREZZA} +4

\textbf{COSTITUZIONE} +3

\textbf{INTELLIGENZA} +3

\textbf{SAGGEZZA} +4

\textbf{CARISMA} +4

\textbf{Iniziativa} +4 -- \textbf{Difesa} 23

\textbf{Punti Ferita} 127 (15d10 + 45)

\textbf{Movimento} 12 m

\textbf{Tiri Salvezza}: Tempra +9, Rif +12, Volontà +12

\textbf{Immunità ai Danni} veleno

\textbf{Immunità alle Condizioni} affascinato, avvelenato 

\textbf{Sensi} scurovisione 18 m 

\textbf{Linguaggi} Celestiale, Comune 

\textbf{Sfida} 10 (5.900 PE)

\emph{\textbf{Incantesimi.}} Il naga ha CM 11. La sua caratteristica da incantatore è la Saggezza (+8 a colpire con attacchi con incantesimo), e ha bisogno solo delle componenti verbali per lanciare i suoi incantesimi. Il naga prepara i seguenti incantesimi:

Trucchetti (a volontà): \emph{fiamma sacra, riparare, taumaturgia}

Difficoltà 10 (4 slot): \emph{comando, cura ferite, scudo della fede}

Difficoltà 13 (3 slot): \emph{bloccare persone, calmare emozioni}

Difficoltà 15 (3 slot): \emph{chiaroveggenza, scagliare maledizione}

Difficoltà 18 (3 slot): \emph{esilio, libertà di movimento}

Difficoltà 20 (2 slot): \emph{colpo infuocato, costrizione}

Difficoltà 23 (1 slot): \emph{visione del vero}

\emph{\textbf{Rinvigorimento.}} Se muore, il naga ritorna in vita in 1d6 giorni e recupera tutti i suoi punti ferita. Solo l'incantesimo \emph{desiderio} può impedire a questo tratto di funzionare.

\textbf{Azioni}

\emph{\textbf{Morso.} Attacco con arma da mischia}: +8 a colpire, portata 3 m, una creatura.

\emph{Colpisce:} 8 (1d8 + 4) danni perforanti, e il bersaglio deve effettuare un tiro salvezza di Tempra CD 15, subendo 45 (10d8) danni da veleno se fallisce il tiro salvezza, o la metà di questi danni se lo riesce.

\emph{\textbf{Sputare Veleno.} Attacco con arma a Distanza}: +8 a colpire, gittata 5m, una creatura.

\emph{Colpisce:} Il bersaglio deve effettuare un tiro salvezza su Tempra CD 15, subendo 45 (10d8) danni da veleno se fallisce il tiro salvezza, o la metà di questi danni se lo riesce.

\medskip\index{Mostri - Naga Spirituale}\textbf{Naga Spirituale}

\emph{Grande mostruosità, caotico malvagio}

\textbf{FORZA} +4

\textbf{DESTREZZA} +3

\textbf{COSTITUZIONE} +2

\textbf{INTELLIGENZA} +3

\textbf{SAGGEZZA} +2

\textbf{CARISMA} +3

\textbf{Iniziativa} +3 -- \textbf{Difesa} 19

\textbf{Punti Ferita} 75 (10d10 + 20)

\textbf{Movimento} 12 m

\textbf{Tiri Salvezza}: Tempra +8, Rif +10, Volontà +10

\textbf{Immunità al Danno} veleno

\textbf{Immunità alle Condizioni} affascinato, avvelenato

\textbf{Sensi} scurovisione 18 m

\textbf{Linguaggi} Abissale, Comune

\textbf{Sfida} 8 (3.900 PE)

\emph{\textbf{Incantesimi.}} Il naga ha CM 10. La sua abilità da incantatore è l'Intelligenza (+6 a colpire con attacchi con incantesimo), e ha bisogno solo delle componenti verbali per eseguire i suoi incantesimi. Il naga prepara i seguenti incantesimi:

Trucchetti (a volontà): \emph{illusione minore, mano magica, raggio di}
\emph{gelo}

Difficoltà 10 (4 slot): \emph{charme su persone, individuazione del
	magico,} \emph{sonno}

Difficoltà 13 (3 slot): \emph{blocca persone, individuazione dei pensieri}

Difficoltà 15 (3 slot): \emph{fulmine, respirare sott'acqua}

Difficoltà 18 (3 slot): \emph{inaridire, porta dimensionale}

Difficoltà 20 (2 slot): \emph{dominare persone}

\emph{\textbf{Rinvigorimento.}} Se muore, il naga ritorna in vita in 1d6 giorni e recupera tutti i suoi punti ferita. Solo l'incantesimo \emph{desiderio} può impedire a questo tratto di funzionare.

\textbf{Azioni}

\emph{\textbf{Morso.} Attacco con arma da mischia}: +7 a colpire,
portata 3 m, una creatura.

\emph{Colpisce:} 7 (1d8 + 4) danni perforanti, e il bersaglio deve effettuare un tiro salvezza di Tempra CD 13, subendo 31 (7d8) danni da veleno se fallisce il tiro salvezza, o la metà di questi danni se lo riesce.

\subsection{Oggetti Animati}

\medskip\index{Mostri - Armatura Animata}\textbf{Armatura Animata}

\emph{Media costrutto, disallineato}

\textbf{FORZA} +2

\textbf{DESTREZZA} +0

\textbf{COSTITUZIONE} +1

\textbf{INTELLIGENZA} -5

\textbf{SAGGEZZA} -4

\textbf{CARISMA} -5

\textbf{Iniziativa} +0 -- \textbf{Difesa} 19

\textbf{Punti Ferita} 33 (6d8 + 6)

\textbf{Movimento} 7,5 m

\textbf{Tiri Salvezza}: Tempra +2, Rif +0, Volontà -4

\textbf{Immunità al Danno} psichico, veleno

\textbf{Immunità alle Condizioni} accecato, affascinato, assordato, avvelenato, paralizzato, pietrificato, sfinimento, spaventato

\textbf{Sensi} vista cieca 18 m (cieco oltre questo raggio)

\textbf{Linguaggi} -

\textbf{Sfida} 1 (200 PE)

\emph{\textbf{Falso Aspetto.}} Mentre l'armatura rimane immobile, è indistinguibile da una normale armatura.

\emph{\textbf{Suscettibilità all'Anti Magia.}} L'armatura è inabile se si trova nell'area di un \emph{campo anti-magia}. Se è bersaglio di \emph{dissolvi} \emph{magie}, l'armatura deve riuscire un tiro salvezza su Tempra contro la CD del tiro salvezza dell'incantesimo o restare svenuta per 1 minuto.

\textbf{Azioni}

\emph{\textbf{Multiattacco.}} L'armatura effettua due attacchi da mischia.

\emph{\textbf{Schianto.} Attacco con arma da mischia}: +4 a colpire, portata 1 m, un bersaglio.

\emph{Colpisce:} 5 (1d6 + 2) danni da botta.

\medskip\index{Mostri - Spada Volante}\textbf{Spada Volante}

\emph{Piccola costrutto, disallineato}

\textbf{FORZA} +1

\textbf{DESTREZZA} +2

\textbf{COSTITUZIONE} +0

\textbf{INTELLIGENZA} -5

\textbf{SAGGEZZA} -3

\textbf{CARISMA} -5

\textbf{Iniziativa} +2 -- \textbf{Difesa} 18

\textbf{Punti Ferita} 17 (5d6)

\textbf{Movimento} 0 m, volo 15 m (fluttua)

\textbf{Tiri Salvezza}  Tempra +1, Riflessi +3, Volontà -4

\textbf{Immunità al Danno} psichico, veleno

\textbf{Immunità alle Condizioni} accecato, affascinato, assordato, avvelenato, paralizzato, pietrificato, spaventato

\textbf{Sensi} vista cieca 18 m (cieco oltre questo raggio)

\textbf{Linguaggi} -

\textbf{Sfida} 1/4 (50 PE)

\emph{\textbf{Falso Aspetto.}} Mentre l'arma rimane immobile e non sta volando, è indistinguibile da una normale spada.

\emph{\textbf{Suscettibilità all'Anti Magia.}} La spada è inabile se si trova nell'area di un \emph{campo anti-magia}. Se è bersaglio di \emph{dissolvi} \emph{magie}, la spada deve riuscire un tiro salvezza su Tempra contro la CD del tiro salvezza dell'incantesimo o restare svenuta per 1 minuto.

\textbf{Azioni}

\emph{\textbf{Spada Lunga.} Attacco con arma da mischia}: +3 a colpire, portata 1 m, un bersaglio.

\emph{Colpisce:} 5 (1d8 + 1) danni taglienti.


\medskip\index{Mostri - Tappeto del Soffocamento}\textbf{Tappeto del Soffocamento}

\emph{Grande costrutto, disallineato}

\textbf{FORZA} +3

\textbf{DESTREZZA} +2

\textbf{COSTITUZIONE} +0

\textbf{INTELLIGENZA} -5

\textbf{SAGGEZZA} -4

\textbf{CARISMA} -5

\textbf{Iniziativa} +2 -- \textbf{Difesa} 13

\textbf{Punti Ferita} 33 (6d10)

\textbf{Movimento} 3 m

\textbf{Tiri Salvezza}: Tempra +4, Riflessi +2, Volontà -4

\textbf{Immunità al Danno} psichico, veleno

\textbf{Immunità alle Condizioni} accecato, affascinato, assordato, avvelenato, paralizzato, pietrificato, spaventato

\textbf{Sensi} vista cieca 18 m (cieco oltre questo raggio)

\textbf{Linguaggi} -

\textbf{Sfida} 2 (450 PE)

\emph{\textbf{Falso Aspetto.}} Mentre il tappeto resta immobile, è indistinguibile da un normale tappeto.

\emph{\textbf{Suscettibilità all'Anti Magia.}} Il tappeto è inabile mentre si trova nell'area di un \emph{campo anti-magia}. Se è il bersaglio di \emph{dissolvi} \emph{magie}, il tappeto deve riuscire un tiro salvezza di Tempra contro la CD del tiro salvezza dell'incantatore o cadere privo di sensi per 1 minuto.

\emph{\textbf{Trasferimento di Danno.}} Mentre afferra una creatura, il tappeto subisce solo la metà dei danni che gli sono inferti, e la creatura afferrata dal tappeto subisce l'altra metà.

\textbf{Azioni}

\emph{\textbf{Soffocare.} Attacco con arma da mischia}: +5 a colpire, portata 1 m, una creatura di taglia Media o inferiore.

\emph{Colpisce:} La creatura è afferrata (CD 13 per fuggire). Fino al termine dell'afferrare, il bersaglio è intralciato, accecato e rischia di soffocare, ma il tappeto non può soffocare un altro bersaglio. Inoltre, all'inizio di ciascun turno del bersaglio, il bersaglio subisce 10 (2d6 + 3) danni da botta.

\medskip\index{Mostri - Ogre}\textbf{Ogre}

\emph{Grande gigante, caotico malvagio}

\textbf{FORZA} +4

\textbf{DESTREZZA} -1

\textbf{COSTITUZIONE} +3

\textbf{INTELLIGENZA} -3

\textbf{SAGGEZZA} -2

\textbf{CARISMA} -2

\textbf{Iniziativa} -1 -- \textbf{Difesa} 12 (armatura di pelle)

\textbf{Punti Ferita} 59 (7d10 + 21)

\textbf{Movimento} 12 m

\textbf{Tiri Salvezza}: Tempra +6, Riflessi +0, Volontà +1

\textbf{Sensi} scurovisione 18 m

\textbf{Linguaggi} Comune, Gigante

\textbf{Sfida} 2 (450 PE)

\textbf{Azioni}

\emph{\textbf{Randello Pesante.} Attacco con arma da mischia}: +6 a colpire, portata 1 m, un bersaglio.

\emph{Colpisce:} 13 (2d8 + 4) danni da botta. 

\emph{\textbf{Giavellotto.} Attacco con arma da mischia o a Distanza}: +6 a colpire, portata 1 m o gittata 9m, un bersaglio. 

\emph{Colpisce:} 11 (2d6 + 4) danni perforanti.

\medskip\index{Mostri - Ombra}\textbf{Ombra}

\emph{Media non morto, caotico malvagio}

\textbf{FORZA} -2

\textbf{DESTREZZA} +2

\textbf{COSTITUZIONE} +1

\textbf{INTELLIGENZA} -2

\textbf{SAGGEZZA} +0

\textbf{CARISMA} -1

\textbf{Iniziativa} +2 -- \textbf{Difesa} 13

\textbf{Punti Ferita} 16 (3d8 + 3)

\textbf{Movimento} 12 m

\textbf{Tiri Salvezza}: Tempra +3, Riflessi +3, Volontà +4

\textbf{Competenze} Muoversi Silenziosamente / Nascondersi nelle Ombre +4 (+6 a luce fioca o oscurità)

\textbf{Vulnerabilità al Danno} da Luce

\textbf{Resistenze al Danno} acido, freddo, fulmine, fuoco, tuono; da botta, perforante e tagliente di attacchi non magici

\textbf{Immunità al Danno} da Vuoto, veleno

\textbf{Immunità alle Condizioni} afferrato, avvelenato, intralciato, paralizzato, pietrificato, prono, sfinimento, spaventato

\textbf{Sensi} scurovisione 18 m

\textbf{Linguaggi} -

\textbf{Sfida} 1/2 (100 PE)

\emph{\textbf{Amorfo.}} L'ombra può muoversi attraverso uno spazio stretto fino a 2,5 centimetri senza stringersi.

\emph{\textbf{Debolezza alla Luce del Sole.}} Mentre si trova alla luce del sole, l'ombra ha -1d6 ai tiri per colpire, le prove di abilità e i tiri salvezza.

\emph{\textbf{Furtività d'Ombra.}} Quando si trova a luce fioca o all'oscurità, l'ombra può effettuare l'azione Nascondersi come azione bonus. 

\emph{\textbf{Natura Non Morta.}} Un'ombra non necessita aria, cibo, bevande o sonno.

\textbf{Azioni}

\emph{\textbf{Risucchio di Forza.} Attacco con arma da mischia}: +4 a colpire, portata 1 m, una creatura.

\emph{Colpisce:} 9 (2d6 + 2) danni da Vuoto, e il punteggio di Forza del bersaglio viene ridotto di 1d4. Il bersaglio muore se ciò riduce la sua Forza a 0. Altrimenti, la riduzione resta finché il bersaglio non riposa 8 ore.

Se un umanoide non malvagio muore a causa di questo attacco, entro 1d4 ore dal suo cadavere si animerà una nuova ombra.

\medskip\index{Mostri - Omuncolo}\textbf{Omuncolo}

\emph{Minuscola costrutto, neutrale}

\textbf{FORZA} -3

\textbf{DESTREZZA} +2

\textbf{COSTITUZIONE} +0

\textbf{INTELLIGENZA} +0

\textbf{SAGGEZZA} +0

\textbf{CARISMA} -2

\textbf{Iniziativa} +2 -- \textbf{Difesa} 14

\textbf{Punti Ferita} 5 (2d4)

\textbf{Movimento} 6 m, volo 12 m

\textbf{Tiri Salvezza}:  Tempra +0, Riflessi +4, Volontà +1

\textbf{Immunità al Danno} veleno

\textbf{Immunità alle Condizioni} affascinato, avvelenato

\textbf{Sensi} scurovisione 18 m, vista cieca 3 m

\textbf{Linguaggi} comprende le lingue del suo creatore ma non può parlare

\textbf{Sfida} 0 (10 PE)

\emph{\textbf{Legame Telepatico.}} Mentre l'omuncolo si trova sullo stesso piano di esistenza del suo padrone, può comunicare magicamente al suo padrone quello che percepisce, e i due possono comunicare telepaticamente.

\textbf{Azioni}

\emph{\textbf{Morso.} Attacco con arma da mischia}: +4 a colpire, portata 1 m, una creatura.

\emph{Colpisce:} 1 danno perforante, e il bersaglio deve riuscire un tiro salvezza di Tempra CD 10 o restare avvelenato per 1 minuto. Se il tiro salvezza viene fallito di 5 o più, il bersaglio resta invece avvelenato per 5 (1d10) minuti e mentre è avvelenato in questo modo è anche privo di sensi.

\medskip\index{Mostri - Oni}\textbf{Oni}

\emph{Grande gigante, legale malvagio}

\textbf{FORZA} +4

\textbf{DESTREZZA} +0

\textbf{COSTITUZIONE} +3

\textbf{INTELLIGENZA} +2

\textbf{SAGGEZZA} +1

\textbf{CARISMA} +2

\textbf{Iniziativa} +2 -- \textbf{Difesa} 20 (cotta di maglia)

\textbf{Punti Ferita} 110 (13d10 + 39)

\textbf{Movimento} 9 m, volo 9 m

\textbf{Tiri Salvezza}: Tempra +7, Riflessi +4, Volontà +6

\textbf{Competenze} Arcano +5, Ingannare +8, Consapevolezza +4 

\textbf{Sensi} scurovisione 18 m 

\textbf{Linguaggi} Comune, Gigante

\textbf{Sfida} 7 (2.900 PE)

\emph{\textbf{Armi Magiche.}} Gli attacchi con armi dell'oni sono magici.

\emph{\textbf{Incantesimi Innati.}} La caratteristica da incantatore dell'oni è il Carisma. L'oni può lanciare questi incantesimi in maniera innata, senza bisogno di componenti materiali:

A volontà: \emph{invisibilità, oscurità}

1/giorno: \emph{charme su persone, cono di freddo, forma gassosa,}
\emph{sonno}

\emph{\textbf{Rigenerazione.}} Se ha almeno 1 punto ferita, l'oni recupera 10 punti ferita all'inizio del suo turno.

\textbf{Azioni}

\emph{\textbf{Multiattacco.}} L'oni effettua due attacchi, con gli artigli o con il falcione.

\emph{\textbf{Artiglio (Solo Forma di Oni).} Attacco con arma da mischia}: +7 a colpire, portata 1 m, un bersaglio. \emph{Colpisce:} 8 (1d8 + 4) danni taglienti.

\emph{\textbf{Falcione.} Attacco con arma da mischia}: +7 a colpire, portata 3 m, un bersaglio.

\emph{Colpisce:} 15 (2d10 + 4) danni taglienti, o 9 (1d10 + 4) danni taglienti in forma Piccola o Media.

\emph{\textbf{Mutare Forma.}} L'oni può trasformarsi magicamente in un umanoide Piccolo o Medio, in un gigante Grande, o tornare alla sua vera forma. A parte la taglia, le sue statistiche sono le stesse in ciascuna forma. L'unico equipaggiamento che viene trasformato è il falcione, che rimpicciolisce in modo da essere impugnato anche in forma umanoide. Se l'oni muore, ritorna alla sua vera forma, e il falcione ritorna alla sua taglia originale.

\medskip\index{Mostri - Orco}\textbf{Orco}

\emph{Media umanoide (orco), caotico malvagio}

\textbf{FORZA} +3

\textbf{DESTREZZA} +1

\textbf{COSTITUZIONE} +3

\textbf{INTELLIGENZA} -2

\textbf{SAGGEZZA} +0

\textbf{CARISMA} +0

\textbf{Iniziativa} +1 -- \textbf{Difesa} 14 (armatura di pelle)

\textbf{Punti Ferita} 15 (2d8 + 6)

\textbf{Movimento} 9 m

\textbf{Tiri Salvezza}: Tempra +3, Riflessi +1, Volontà +1

\textbf{Competenze} Intimidire +2

\textbf{Sensi} scurovisione 18 m

\textbf{Linguaggi} Comune, Goblinoide

\textbf{Sfida} 1/2 (100 PE)

\emph{\textbf{Aggressivo.}} Come azione bonus, l'orco può muoversi fino a metà del suo movimento verso una creatura ostile che possa vedere.

\textbf{Azioni}

\emph{\textbf{Ascia Bipenne.} Attacco con arma da mischia}: +5 a colpire, portata 1 m, un bersaglio.

\emph{Colpisce:} 9 (1d12 + 3) danni taglienti.

\emph{\textbf{Giavellotto.} Attacco con arma da mischia o a Distanza}: +5 a colpire, portata 1 m o gittata 9m, un bersaglio. \emph{Colpisce:} 6 (1d6 + 3) danni perforanti.

\medskip\index{Mostri - Orsogufo}\textbf{Orsogufo}

\emph{Grande mostruosità, disallineato}

\textbf{FORZA} +5

\textbf{DESTREZZA} +1

\textbf{COSTITUZIONE} +3

\textbf{INTELLIGENZA} -4

\textbf{SAGGEZZA} +1

\textbf{CARISMA} -2

\textbf{Iniziativa} +1 -- \textbf{Difesa} 15

\textbf{Punti Ferita} 59 (7d10 + 21)

\textbf{Movimento} 12 m

\textbf{Tiri Salvezza}: Tempra +10, Riflessi +5, Volontà +2

\textbf{Competenze} Consapevolezza +3

\textbf{Sensi} scurovisione 18 m

\textbf{Linguaggi} -

\textbf{Sfida} 3 (700 PE)

\emph{\textbf{Olfatto e Vista Affinati.}} L'orsogufo ha +1d6 nelle prove di Saggezza (Consapevolezza) basate su olfatto o vista. 

\textbf{Azioni}

\emph{\textbf{Multiattacco.}} L'orsogufo effettua due attacchi: uno con il becco e uno con gli artigli.

\emph{\textbf{Artigli.} Attacco con arma da mischia}: +7 a colpire, portata 1 m, un bersaglio.

\emph{Colpisce:} 14 (2d8 + 5) danni taglienti.

\emph{\textbf{Becco.} Attacco con arma da mischia}: +7 a colpire, portata 1 m, una creatura.

\emph{Colpisce:} 10 (1d10 + 5) danni perforanti.

\medskip\index{Mostri - Otyugh}\textbf{Otyugh}

\emph{Grande aberrazione, neutrale}

\textbf{FORZA} +3

\textbf{DESTREZZA} +0

\textbf{COSTITUZIONE} +4

\textbf{INTELLIGENZA} -2

\textbf{SAGGEZZA} +1

\textbf{CARISMA} -2

\textbf{Iniziativa} +0 -- \textbf{Difesa} 17

\textbf{Punti Ferita} 114 (12d10 + 48)

\textbf{Movimento} 9 m

\textbf{Tiri Salvezza}: Tempra +3, Riflessi +2, Volontà +6

\textbf{Sensi} scurovisione 36 m

\textbf{Linguaggi} Otyugh

\textbf{Sfida} 5 (1.800 PE)

\emph{\textbf{Telepatia Limitata.}} L'otyugh può trasmettere magicamente dei semplici messaggi e immagini a qualsiasi creatura entro 36 metri da esso e che possa comprendere una lingua. Questa forma di telepatia non permette alla creatura ricevente di rispondere telepaticamente.

\textbf{Azioni}

\emph{\textbf{Multiattacco.}} L'otyugh effettua tre attacchi: uno con il morso e due con i tentacoli.

\emph{\textbf{Morso.} Attacco con arma da mischia}: +6 a colpire, portata 1 m, un bersaglio.

\emph{Colpisce:} 12 (2d8 + 3) danni perforanti. Se il bersaglio è una creatura, deve riuscire un tiro salvezza di Tempra CD 15 contro malattia o restare avvelenato finché la malattia non viene curata. Ogni 24 ore successive, il bersaglio deve ripetere il tiro salvezza, riducendo il suo massimo di punti ferita di 5 (1d10) se lo fallisce. Se il tiro salvezza riesce, la malattia è passata. Il bersaglio muore se la malattia riduce i suoi punti ferita massimi a 0.

Questa riduzione dei punti ferita massimi del personaggio, perdura finché la malattia non viene curata.
\
\emph{\textbf{Tentacolo.} Attacco con arma da mischia}: +6 a colpire, portata 3 m, un bersaglio.

\emph{Colpisce:} 7 (1d8 + 3) danni da botta più 4 (1d8) danni perforanti. Se il bersaglio è di taglia Media o inferiore, è afferrato (CD 13 per fuggire) e intralciato fino al termine dell'afferrare. L'otyugh ha due tentacoli, ciascun dei quali può afferrare un bersaglio diverso.

\emph{\textbf{Schianto di Tentacolo.}} L'otyugh schianta le creature afferrate dai suoi tentacoli, l'una contro l'altra o sul pavimento. Ogni creatura deve riuscire un tiro salvezza di Tempra CD 14 o subire 10 (2d6 + 3) danni da botta e restare stordita fino al termine del prossimo turno dell'otyugh. Se il tiro salvezza riesce, il bersaglio subisce la metà dei danni da botta e non è stordito.


\medskip\index{Mostri - Pegaso}\textbf{Pegaso}

\emph{Grande celestiale, caotico buono}

\textbf{FORZA} +4

\textbf{DESTREZZA} +2

\textbf{COSTITUZIONE} +3

\textbf{INTELLIGENZA} +0

\textbf{SAGGEZZA} +2

\textbf{CARISMA} +1

\textbf{Iniziativa} +2 -- \textbf{Difesa} 13

\textbf{Punti Ferita} 59 (7d10 + 21)

\textbf{Movimento} 18 m, volo 27 m

\textbf{Tiri Salvezza} Tempra +7, Riflessi +6, Volontà +4

\textbf{Competenze} Consapevolezza +6

\textbf{Linguaggi} comprende Celestiale, Comune, Elfico e Silvano ma non può parlare

\textbf{Sfida} 2 (450 PE)

\textbf{Azioni}

\emph{\textbf{Zoccoli.} Attacco con arma da mischia}: +6 a colpire, portata 1 m, un bersaglio.

\emph{Colpisce:} 11 (2d6 + 4) danni da botta.

\medskip\index{Mostri - Persecutore Invisibile}\textbf{Persecutore Invisibile}

\emph{Media elementale, neutrale}

\textbf{FORZA} +3

\textbf{DESTREZZA} +4

\textbf{COSTITUZIONE} +2

\textbf{INTELLIGENZA} +0

\textbf{SAGGEZZA} +2

\textbf{CARISMA} +0

\textbf{Iniziativa} +4 -- \textbf{Difesa} 17

\textbf{Punti Ferita} 104 (16d8 + 32)

\textbf{Movimento} 15 m, volo 15 m (fluttua)

\textbf{Tiri Salvezza}: Tempra +13, Riflessi +11, Volontà +4

\textbf{Competenze} Muoversi Silenziosamente / Nascondersi nelle Ombre +10, Consapevolezza +8

\textbf{Resistenze al Danno} da botta, perforante e tagliente di attacchi non magici

\textbf{Immunità ai Danni} veleno

\textbf{Immunità alle Condizioni} afferrato, avvelenato, intralciato, paralizzato, pietrificato, privo di sensi, prono, sfinimento

\textbf{Sensi} scurovisione 18 m

\textbf{Linguaggi} Auran, comprende il Comune ma non lo parla

\textbf{Sfida} 6 (2.300 PE)

\emph{\textbf{Cacciatore Infallibile.}} Il convocatore assegna una preda
al persecutore. Il persecutore sa la direzione e la distanza a cui si
trova la preda finché entrambi si trovano sullo stesso piano di
esistenza. Il persecutore conosce anche la posizione del suo
convocatore.

\emph{\textbf{Invisibilità.}} Il persecutore è invisibile.

\emph{\textbf{Natura Elementale.}} Un persecutore invisibile non ha bisogno di aria, cibo, bevande o sonno.

\textbf{Azioni}

\emph{\textbf{Multiattacco.}} La persecutore effettua due attacchi di schianto.

\emph{\textbf{Schianto.} Attacco con arma da mischia}: +6 a colpire, portata 1 m, un bersaglio.

\emph{Colpisce:} 10 (2d6 + 3) danni da botta.

\medskip\index{Mostri - Pseudodrago}\textbf{Pseudodrago}

\emph{Minuscola drago, neutrale buono}

\textbf{FORZA} -2

\textbf{DESTREZZA} +2

\textbf{COSTITUZIONE} +1

\textbf{INTELLIGENZA} +0

\textbf{SAGGEZZA} +1

\textbf{CARISMA} +0

\textbf{Iniziativa} +2 -- \textbf{Difesa} 14

\textbf{Punti Ferita} 7 (2d4 + 2)

\textbf{Movimento} 4,5 m, volo 18 m

\textbf{Tiri Salvezza}: Tempra +4, Riflessi +5, Volontà +4

\textbf{Competenze} Muoversi Silenziosamente / Nascondersi nelle Ombre +4, Consapevolezza +3

\textbf{Sensi} scurovisione 18 m, vista cieca 3 m

\textbf{Linguaggi} comprende il Comune e il Draconico ma non parla

\textbf{Sfida} 1/4 (50 PE)

\emph{\textbf{Resistenza alla Magia.}} Lo pseudodrago ha +1d6 ai tiri salvezza contro incantesimi e altri effetti magici.

\emph{\textbf{Sensi Affinati.}} Lo pseudodrago ha +1d6 alle prove di Saggezza (Consapevolezza) basate su vista, udito e olfatto.

\emph{\textbf{Telepatia Limitata.}} Lo pseudodrago può comunicare semplici idee, emozioni e immagini telepaticamente con qualsiasi creatura entro 30 metri da esso che può comprendere una lingua.

\textbf{Azioni}

\emph{\textbf{Morso.} Attacco con arma da mischia}: +4 a colpire,
portata 1 m, un bersaglio.

\emph{Colpisce:} 4 (1d4 + 2) danni perforanti.

\emph{\textbf{Pungiglione.} Attacco con arma da mischia}: +4 a colpire,
portata 1 m, una creatura.

\emph{Colpisce:} 4 (1d4 + 2) danni perforanti, e il bersaglio deve riuscire un tiro salvezza di Tempra CD 11 o restare avvelenato per 1 ora. Se il risultato del tiro salvezza è 6 o meno, il bersaglio cade privo di sensi per la stessa durata, o finché non subisce danni o un'altra creatura usa un'azione per risvegliarlo.

\medskip\index{Mostri - Rakshasa}\textbf{Rakshasa}

\emph{Media immondo, legale malvagio}

\textbf{FORZA} +2

\textbf{DESTREZZA} +3

\textbf{COSTITUZIONE} +4

\textbf{INTELLIGENZA} +1

\textbf{SAGGEZZA} +3

\textbf{CARISMA} +5

\textbf{Iniziativa} +3 -- \textbf{Difesa} 23

\textbf{Punti Ferita} 110 (13d8 + 52)

\textbf{Movimento} 12 m

\textbf{Tiri Salvezza}: Tempra +9, Riflessi +12, Volontà +8 

\textbf{Competenze} Ingannare +10, Percepire Emozioni +8

\textbf{Vulnerabilità al Danno} perforante di armi magiche impugnate da
creatura buone

\textbf{Immunità al Danno} da botta, armi +1

\textbf{Sensi} scurovisione 18 m

\textbf{Linguaggi} Comune, Infernale

\textbf{Sfida} 13 (10.000 PE)

\emph{\textbf{Immunità alla Magia Limitata.}} Il rakshasa è immune agli affetti o all'individuazione tramite incantesimi di Difficoltà 23 o più basso a meno che non desideri esserne soggetto. Ha +1d6 ai tiri salvezza contro tutti gli altri incantesimi ed effetti magici.

\emph{\textbf{Incantesimi Innati.}} La caratteristica da incantatore del rakshasa il Carisma (+10 a colpire   con attacchi con incantesimi). Il rakshasa può lanciare in maniera innata i seguenti incantesimi, senza aver bisogno di componenti materiali:

A volontà: \emph{camuffare sé stesso, illusione minore, individuazione} \emph{dei pensieri, mano magica}

3/Giorno ciascuno: \emph{charme su persone, immagine maggiore,} \emph{individuazione del magico, invisibilità, suggestione} 1/Giorno: \emph{dominare persone, spostamento planare, visione del} \emph{vero, volare}

\textbf{Azioni}

\emph{\textbf{Multiattacco.}} Il rakshasa può effettuare due attacchi di artiglio.

\emph{\textbf{Artiglio.} Attacco con arma da mischia}: +7 a colpire, portata 1 m, un bersaglio.

\emph{Colpisce:} 9 (2d6 + 2) danni taglienti, e se il bersaglio è una creatura rimane maledetto. La maledizione magica ha effetto ogni qualvolta il bersaglio riposa, riempiendo i pensieri del bersaglio di immagini e sogni orribili. Il bersaglio maledetto non riceve beneficio dall'aver terminato un riposo. La maledizione perdura finché non viene rimossa dall'incantesimo \emph{rimuovi maledizione} o simile magia.



\medskip\index{Mostri - Remorhaz}\textbf{Remorhaz}

\emph{Enorme mostruosità, disallineato}

\textbf{FORZA} +7

\textbf{DESTREZZA} +1

\textbf{COSTITUZIONE} +5

\textbf{INTELLIGENZA} -3

\textbf{SAGGEZZA} +0

\textbf{CARISMA} -3

\textbf{Iniziativa} +1 -- \textbf{Difesa} 23

\textbf{Punti Ferita} 195 (17d12 + 85)

\textbf{Movimento} 9 m, scavo 6 m

\textbf{Tiri Salvezza}: Tempra +11, Riflessi +7, Volontà +4

\textbf{Immunità ai Danni} freddo, fuoco

\textbf{Sensi} scurovisione 18 m, senso tellurico 18 m

\textbf{Linguaggi} -

\textbf{Sfida} 11 (7.200 PE)

\emph{\textbf{Corpo Riscaldato.}} Una creatura che entri a contatto con il remorhaz o lo colpisca con un attacco da mischia mentre si trova entro 1,5 metri da esso, subisce 10 (3d6) danni da fuoco.

\textbf{Azioni}

\emph{\textbf{Morso.} Attacco in mischia con arma}: +11 a colpire, portata 3 m, un bersaglio.

\emph{Colpisce:} 40 (6d10 + 7) danni perforanti più 10 (3d6) danni da fuoco. Se il bersaglio è una creatura, è afferrato (CD 17 per fuggire). Fino al termine dell'afferrare, il bersaglio è intralciato, e il remorhaz non può attaccare con il morso un altro bersaglio.

\emph{\textbf{Inghiottire.}} Il remorhaz effettua una attacco di morso contro un bersaglio di taglia Media o inferiore che sta afferrando. Se l'attacco colpisce, la creatura subisce il danno da morso ed è inghiottita, e l'afferrare ha termine. Il bersaglio inghiottito è accecato e intralciato, ha copertura totale contro gli attacchi e altri effetti all'esterno del remorhaz, e subisce 21 (6d6) danni da acido all'inizio di ciascun turno del remorhaz.

Se il remorhaz subisce 30 o più danni in un singolo turno da una creatura al suo interno, il remorhaz deve riuscire un tiro salvezza su Tempra CD 15 al termine di quel turno o vomitare tutte le creature inghiottite, che cadono prone in uno spazio entro 3 metri dal remorhaz. Se il remorhaz muore, una creatura inghiottita non   più intralciata da esso e può uscire dal cadavere utilizzando 4,5   metri di movimento, uscendo prona.


\medskip\index{Mostri - Rugginofago}\textbf{Rugginofago}

\emph{Media Mostruosità, disallineato}

\textbf{FORZA} +1

\textbf{DESTREZZA} +1

\textbf{COSTITUZIONE} +1

\textbf{INTELLIGENZA} -4

\textbf{SAGGEZZA} +1

\textbf{CARISMA} -2

\textbf{Iniziativa} +1 -- \textbf{Difesa} 15

\textbf{Punti Ferita} 27 (5d8 + 5)

\textbf{Movimento} 12 m

\textbf{Tiri Salvezza}: Tempra +2, Riflessi +4, Volontà +5

\textbf{Sensi} scurovisione 18 m

\textbf{Linguaggi} -

\textbf{Sfida} 1/2 (100 PE)

\emph{\textbf{Fiuto del Ferro.}} Il rugginofago può individuare, con l'olfatto, l'esatta posizione di metalli ferrosi entro 9 metri.

\emph{\textbf{Arrugginire Metallo.}} Qualsiasi arma non magica fatta di metallo che colpisca il rugginofago si corrode. Dopo aver inflitto il danno, l'arma subisce una penalità permanente e cumulativa di - 1 ai tiri di danno. Se la penalità scende fino a -5, l'arma è distrutta. Le munizioni non magiche fatte di metallo e che colpiscono il rugginofago, sono considerate distrutte dopo aver inflitto il danno.

\textbf{Azioni}

\emph{\textbf{Morso.} Attacco con arma da mischia}: +3 a colpire, portata 1 m, un bersaglio.

\emph{Colpisce:} 5 (1d8 + 1) danni perforanti.

\emph{\textbf{Antenne.}} Il rugginofago corrode gli oggetti di metallo ferroso non magici che può vedere e si trovano entro 1,5 metri. Se l'oggetto non è indossato o trasportato, il contatto col rugginofago ne distrugge un cubo di 30 centimetri di spigolo. Se l'oggetto è indossato o trasportato da una creatura, la creatura può effettuare un tiro salvezza su Riflessi CD 11 per evitare il contatto con il rugginofago.

Se l'oggetto con cui entra in contatto è un'armatura o scudo di metallo indossati o trasportati, questi subiscono una penalità permanente e cumulativa di -1 alla Difesa che forniscono. Le armature ridotte a Difesa 0 o gli scudi che scendono ad un bonus di +0 sono distrutti. Se l'oggetto con cuientra in contatto è un'arma di metallo impugnata da qualcuno, la  arrugginisce come descritto nel tratto Arrugginire Metallo.

\medskip\index{Mostri - Sahuagin}\textbf{Sahuagin}

\emph{Media umanoide (sahuagin), legale malvagio}

\textbf{FORZA} +1

\textbf{DESTREZZA} +0

\textbf{COSTITUZIONE} +1

\textbf{INTELLIGENZA} +1

\textbf{SAGGEZZA} +1

\textbf{CARISMA} -1

\textbf{Iniziativa} +1 -- \textbf{Difesa} 13

\textbf{Punti Ferita} 22 (4d8 + 4)

\textbf{Movimento} 9 m, nuoto 12 m

\textbf{Tiri Salvezza}: Tempra +4, Riflessi +4, Volontà +4

\textbf{Competenze} Consapevolezza +5

\textbf{Sensi} scurovisione 36 m

\textbf{Linguaggi} Sahuagin

\textbf{Sfida} 1/2 (100 PE)

\emph{\textbf{Anfibio Limitato.}} Il sahuagin può respirare aria e acqua, ma deve restare sommerso almeno una volta ogni 4 ore per evitare di soffocare.

\emph{\textbf{Frenesia Sanguinaria.}} Il sahuagin ha +1d6 ai tiri per colpire in mischia contro qualsiasi creatura che non sia al massimo dei suoi punti ferita.

\emph{\textbf{Telepatia con gli Squali}}. Il sahuagin può comandare magicamente qualsiasi squalo entro 36 metri da sé, usando una forma limitata di telepatia. 

\textbf{Azioni}

\emph{\textbf{Multiattacco.}} Il sahuagin può effettuare due attacchi da mischia:  uno con il morso e uno con gli artigli o la lancia.

\emph{\textbf{Artigli.} Attacco con arma da mischia}: +3 a colpire, portata 1 m, un bersaglio.

\emph{Colpisce:} 3 (1d4 + 1) danni taglienti.

\emph{\textbf{Lancia.} Attacco con arma da mischia o a Distanza}: +3 a colpire, portata 1 m o gittata 6m, un bersaglio.

\emph{Colpisce:} 4 (1d6 + 1) danni perforanti, o 5 (1d8 + 1) danni perforanti se usata con due mani per effettuare un attacco da mischia.

\emph{\textbf{Morso.} Attacco con arma da mischia}: +3 a colpire, portata 1 m, un bersaglio.

\emph{Colpisce:} 3 (1d4 + 1) danni perforanti.

\medskip\index{Mostri - Salamandra}\textbf{Salamandra}

\emph{Grande elementale, neutrale malvagio}

\textbf{FORZA} +4

\textbf{DESTREZZA} +2

\textbf{COSTITUZIONE} +2

\textbf{INTELLIGENZA} +0

\textbf{SAGGEZZA} +0

\textbf{CARISMA} +1

\textbf{Iniziativa} +2 -- \textbf{Difesa} 18

\textbf{Punti Ferita} 90 (12d10 + 24)

\textbf{Movimento} 9 m

\textbf{Tiri Salvezza}: Tempra +10, Riflessi +7, Volontà +6

\textbf{Vulnerabilità al Danno} freddo

\textbf{Resistenze al Danno} da botta, perforante e tagliente di attacchi non magici

\textbf{Immunità ai Danni} fuoco

\textbf{Sensi} scurovisione 18 m

\textbf{Linguaggi} Ignan

\textbf{Sfida} 5 (1.800 PE)

\emph{\textbf{Armi Riscaldate.}} Qualsiasi arma da mischia metallica che la salamandra impugni infligge 3 (1d6) danni da fuoco aggiuntivi per colpo (già incluso nell'attacco).

\emph{\textbf{Corpo Riscaldato.}} Una creatura che entri a contatto con la salamandra o la colpisce con un attacco da mischia mentre si trova entro 1,5 metri da essa subisce 7 (2d6) danni da fuoco.

\textbf{Azioni}

\emph{\textbf{Multiattacco.}} La salamandra effettua due attacchi: uno con la lancia e uno con la coda.

\emph{\textbf{Coda.} Attacco con arma da mischia}: +7 a colpire, portata 3 m, un bersaglio.

\emph{Colpisce:} 11 (2d6 + 4) danni da botta più 7 (2d6) danni da fuoco, e il bersaglio è afferrato (CD 14 per fuggire). Fino al termine dell'afferrare, il bersaglio è intralciato, la salamandra può colpire automaticamente il bersaglio con la coda, e la salamandra non può effettuare attacchi di coda contro altri bersagli.

\emph{\textbf{Lancia.} Attacco con arma da mischia o a Distanza}: +7 a colpire, portata 1 m, gittata 6m, un bersaglio.

\emph{Colpisce:} 11 (2d6 + 4) danni perforanti, o 13 (2d8 +4) danni perforanti se usata con due mani per effettuare un attacco da mischia, più 3 (1d6) danni da fuoco.

\medskip\index{Mostri - Satiro}\textbf{Satiro}

\emph{Media fatato, caotico neutrale}

\textbf{FORZA} +1

\textbf{DESTREZZA} +3

\textbf{COSTITUZIONE} +0

\textbf{INTELLIGENZA} +1

\textbf{SAGGEZZA} +0

\textbf{CARISMA} +2

\textbf{Iniziativa} +3 -- \textbf{Difesa} 15 (armatura di cuoio)

\textbf{Punti Ferita} 31 (7d8)

\textbf{Vulnerabilità al Danno} ferro freddo

\textbf{Movimento} 12 m

\textbf{Tiri Salvezza}: Tempra +4, Riflessi +8, Volontà +8

\textbf{Competenze} Muoversi Silenziosamente / Nascondersi nelle Ombre +5, Intrattenere +6, Consapevolezza +2

\textbf{Linguaggi} Comune, Elfico, Silvano

\textbf{Sfida} 1/2 (100 PE)

\emph{\textbf{Resistenza alla Magia.}} Il satiro ha +1d6 ai tiri salvezza contro incantesimi e altri effetti magici.

\textbf{Azioni}

\emph{\textbf{Incornata.} Attacco con arma da mischia}: +3 a colpire, portata 1 m, un bersaglio.

\emph{Colpisce:} 6 (2d4 + 1) danni da botta.

\emph{\textbf{Spada Corta.} Attacco con arma da mischia}: +5 a colpire, portata 1 m, un bersaglio.

\emph{Colpisce:} 6 (1d6 + 3) danni perforanti.

\emph{\textbf{Arco Corto.} Attacco con arma a Distanza}: +5 a colpire, gittata 24m, un bersaglio.

\emph{Colpisce:} 6 (1d6 + 3) danni perforanti.

\medskip\index{Mostri - Scheletro}\textbf{Scheletro}

\emph{Media non morto, legale malvagio}

\textbf{FORZA} +0

\textbf{DESTREZZA} +2

\textbf{COSTITUZIONE} +2

\textbf{INTELLIGENZA} -2

\textbf{SAGGEZZA} -1

\textbf{CARISMA} -3

\textbf{Iniziativa} +2 -- \textbf{Difesa} 14 (pezzi di armatura)

\textbf{Punti Ferita} 13 (2d8 + 4)

\textbf{Movimento} 9 m

\textbf{Tiri Salvezza}: Tempra +0, Riflessi +2, Volontà +2

\textbf{Vulnerabilità al Danno} da botta

\textbf{Resistenze al Danno} perforante e tagliente di attacchi non magici

\textbf{Immunità al Danno} veleno

\textbf{Immunità alle Condizioni} avvelenato, sfinimento

\textbf{Sensi} scurovisione 18 m

\textbf{Linguaggi} comprende tutte le lingue che parlava in vita ma non può parlare

\textbf{Sfida} 1/4 (50 PE)

\emph{\textbf{Natura Non Morta.}} Lo scheletro non necessita aria, cibo, bevande o sonno.

\textbf{Azioni}

\emph{\textbf{Spada Corta.} Attacco con arma da mischia}: +4 a colpire, portata 1 m, un bersaglio.

\emph{Colpisce:} 5 (1d6 + 2) danni perforanti.

\emph{\textbf{Arco Corto.} Attacco con arma a Distanza}: +4 a colpire, gittata 24m, un bersaglio.

\emph{Colpisce:} 5 (1d6 + 2) danni perforanti.

\medskip\index{Mostri - Scheletro di Cavallo da Guerra}\textbf{Scheletro di Cavallo da Guerra}

\emph{Grande non morto, legale malvagio}

\textbf{FORZA} +4

\textbf{DESTREZZA} +1

\textbf{COSTITUZIONE} +2

\textbf{INTELLIGENZA} -4

\textbf{SAGGEZZA} -1

\textbf{CARISMA} -3

\textbf{Iniziativa} +1 -- \textbf{Difesa} 14 (pezzi di bardatura)

\textbf{Punti Ferita} 22 (3d10 + 6)

\textbf{Movimento} 18 m

\textbf{Tiri Salvezza}: Tempra +4, Riflessi +3, Volontà +1

\textbf{Vulnerabilità al Danno} da botta

\textbf{Resistenze al Danno} perforante e tagliente di attacchi non magici

\textbf{Immunità al Danno} veleno

\textbf{Immunità alle Condizioni} avvelenato, sfinimento

\textbf{Sensi} scurovisione 18 m 

\textbf{Linguaggi} -

\textbf{Sfida} 1/2 (100 PE)

\emph{\textbf{Natura Non Morta.}} Lo scheletro non necessita aria, cibo, bevande o sonno.

\textbf{Azioni}

\emph{\textbf{Zoccoli.} Attacco con arma da mischia}: +6 a colpire, portata 1,5

m, un bersaglio.

\emph{Colpisce:} 11 (2d6 + 4) danni da botta.

\medskip\index{Mostri - Basilisco}\textbf{Scheletro di Minotauro}

\emph{Grande non morto, legale malvagio}

\textbf{FORZA} +4

\textbf{DESTREZZA} +0

\textbf{COSTITUZIONE} +2

\textbf{INTELLIGENZA} -2

\textbf{SAGGEZZA} -1

\textbf{CARISMA} -3

\textbf{Iniziativa} +0 -- \textbf{Difesa} 13

\textbf{Punti Ferita} 67 (9d10 + 18)

\textbf{Movimento} 12 m

\textbf{Tiri Salvezza}: Tempra +6, Riflessi +3, Volontà +2

\textbf{Vulnerabilità al Danno} da botta

\textbf{Immunità al Danno} veleno

\textbf{Resistenze al Danno} perforante e tagliente di attacchi non magici

\textbf{Immunità alle Condizioni} avvelenato, sfinimento

\textbf{Sensi} scurovisione 18 m

\textbf{Linguaggi} comprende l'Abissale ma non può parlare

\textbf{Sfida} 2 (450 PE)

\emph{\textbf{Carica.}} Se lo scheletro di minotauro si muove di almeno 3 metri in linea retta verso il bersaglio e poi lo colpisce con un attacco di incornata durante lo stesso turno, il bersaglio subisce 9 (2d8) danni perforanti aggiuntivi. Se il bersaglio è una creatura, deve riuscire un tiro salvezza di Tempra CD 14 o venire spinto di 3 metri indietro e cadere prono.

\emph{\textbf{Natura Non Morta.}} Lo scheletro non necessita aria, cibo, bevande o sonno.

\textbf{Azioni}

\emph{\textbf{Ascia Bipenne.} Attacco con arma da mischia}: +6 a
colpire, portata 1 m, un bersaglio.

\emph{Colpisce:} 17 (2d12 + 4) danni taglienti.

\emph{\textbf{Incornata.} Attacco con arma da mischia}: +6 a colpire,
portata 1 m, un bersaglio.

\emph{Colpisce:} 13 (2d8 + 4) danni perforanti.

\medskip\index{Mostri - Segugio Infernale}\textbf{Segugio Infernale}

\emph{Media immondo, legale malvagio}

\textbf{FORZA} +3

\textbf{DESTREZZA} +1

\textbf{COSTITUZIONE} +2

\textbf{INTELLIGENZA} -2

\textbf{SAGGEZZA} +1

\textbf{CARISMA} -2

\textbf{Iniziativa} +1 -- \textbf{Difesa} 17

\textbf{Punti Ferita} 45 (7d8 + 14)

\textbf{Movimento} 15 m

\textbf{Tiri Salvezza}: Tempra +6, Riflessi +5, Volontà +1

\textbf{Competenze} Consapevolezza +5

\textbf{Immunità al Danno} fuoco

\textbf{Sensi} scurovisione 18 m

\textbf{Linguaggi} comprende l'Infernale ma non può parlare

\textbf{Sfida} 3 (700 PE)

\emph{\textbf{Udito e Olfatto Affinato.}} Il segugio ha +1d6 nelle prove di Saggezza (Consapevolezza) basate su udito od olfatto.

\emph{\textbf{Tattiche di Branco.}} Il segugio ha +1d6 ai tiri per colpire contro una creatura se almeno uno degli alleati del segugio si trova entro 1,5 metri dalla creatura e quell'alleato non è inabile.

\textbf{Azioni}

\emph{\textbf{Morso.} Attacco con arma da mischia}: +5 a colpire,
portata 1 m, un bersaglio.

\emph{Colpisce:} 7 (1d6 + 3) danni perforanti più 7 (2d6) danni da
fuoco.

\emph{\textbf{Soffio Infuocato (Ricarica 5-6).}} Il segugio esala fuoco in un cono di 4,5 metri. Ogni creatura in quell'area deve effettuare un tiro salvezza di Riflessi CD 12, e subire 21 (6d6) danni da fuoco se fallisce il tiro salvezza, o la metà di questi danni se lo riesce.



\subsection{Sfingi}

\medskip\index{Mostri - Androsfinge}\textbf{Androsfinge}

\emph{Grande mostruosità, legale neutrale}

\textbf{FORZA} +6

\textbf{DESTREZZA} +0

\textbf{COSTITUZIONE} +5

\textbf{INTELLIGENZA} +3

\textbf{SAGGEZZA} +4

\textbf{CARISMA} +6

\textbf{Iniziativa} +3 -- \textbf{Difesa} 26

\textbf{Punti Ferita} 199 (19d10 + 95)

\textbf{Movimento} 12 m, volo 18 m

\textbf{Tiri Salvezza}: Tempra +12, Riflessi +8, Volontà +7

\textbf{Competenze} Arcano +9, Consapevolezza +10, Religione +15

\textbf{Immunità al Danno} psichico; da botta, perforante e tagliente di attacchi non magici

\textbf{Immunità alle Condizioni} affascinato, spaventato 

\textbf{Sensi} visione del vero 36 m 

\textbf{Linguaggi} Comune, Sfinge

\textbf{Sfida} 17 (18.000 PE)

\emph{\textbf{Armi Magiche.}} Gli attacchi con armi della sfinge sono magici.

\emph{\textbf{Imperscrutabile.}} La sfinge è immune a qualsiasi effetto in grado di percepirne le emozioni o leggerne i pensieri, oltre che a qualsiasi incantesimo di divinazione che rifiuti. Le prove di Saggezza (Percepire Emozioni) per discernere le intenzioni o la sincerità della sfinge hanno -1d6.

\emph{\textbf{Incantesimi.}} La sfinge ha CM 12.
La sua caratteristica da incantatore è la Saggezza (CD del tiro salvezza degli incantesimi 18, +10 a colpire con attacchi con incantesimo). Non ha bisogno di componenti materiali per lanciare i suoi incantesimi. La sfinge tiene preparati i seguenti incantesimi:

Trucchetti (a volontà): \emph{fiamma sacra, salvare i morenti,}
\emph{taumaturgia}

Difficoltà 10 (4 slot): \emph{comando, individuazione del magico,}
\emph{individuare male e bene}

Difficoltà 13 (3 slot): \emph{ristorare inferiore, zona di verità}

Difficoltà 15 (3 slot): \emph{dissolvi magie, linguaggi}

Difficoltà 18 (3 slot): \emph{esilio, libertà di movimento}

Difficoltà 20 (2 slot): \emph{colpo infuocato, ristorare superiore}

Difficoltà 23 (1 slot): \emph{banchetto degli eroi}

\textbf{Azioni}

\emph{\textbf{Multiattacco.}} La sfinge può effettuare due attacchi di
artiglio.

\emph{\textbf{Artiglio.} Attacco con arma da mischia}: +12 a colpire,
portata 1 m, un bersaglio.

\emph{Colpisce:} 17 (2d6 + 10) danni taglienti.

\emph{\textbf{Ruggito (3/Giorno).}} La sfinge emette un ruggito magico. Ogni volta che ruggisce prima di una nuova alba, il ruggito più forte e l'effetto è diverso, come dettagliato di seguito. Ogni creatura entro 150 metri dalla sfinge e capace di udirne il ruggito deve effettuare un tiro salvezza.

\textbf{Primo Ruggito.} Ogni creatura che fallisce un tiro salvezza su Volontà CD 18 resta spaventata per 1 minuto. Una creatura spaventata può ripetere il tiro salvezza al termine di ciascun suo turno, terminandone l'effetto per sé, se lo riesce.

\textbf{Secondo Ruggito.} Ogni creatura che fallisce un tiro salvezza su Volontà CD 18 resta assordata e spaventata per 1 minuto. Una creatura spaventata è paralizzata e può ripetere il tiro salvezza al termine di ciascun suo turno, terminandone l'effetto per sé, se lo riesce.

\textbf{Terzo Ruggito.} Ogni creatura effettua un tiro salvezza su Tempra CD 18. Chi fallisce il tiro salvezza subisce 44 (8d10) danni da tuono ed è gettato prono. Se il tiro salvezza riesce, la creatura subisce la metà di questi danni e non viene gettata prona. 

\textbf{Azioni Aggiuntive}

La sfinge può effettuare 3 Azioni aggiuntive, scelte tra le opzioni seguenti. Può usare solo un'opzione leggendaria alla volta e solo al termine del turno di un'altra creatura. La sfinge recupera le Azioni aggiuntive spese all'inizio del proprio turno.

\textbf{Attacco di Artiglio.} La sfinge effettua un attacco di artiglio. 

\textbf{Eseguire un Incantesimo (Costa 3 Azioni).} La sfinge lancia un incantesimo dalla lista degli incantesimi preparati, utilizzando uno slot incantesimo come di norma. 

\textbf{Teletrasporto (Costa 2 Azioni).} La sfinge si teletrasporta magicamente, insieme a tutto l'equipaggiamento che sta indossando o trasportando, in uno spazio non occupato che possa vedere, fino a 36 metri di distanza.


\medskip\index{Mostri - Ginosfinge}\textbf{Ginosfinge}

\emph{Grande mostruosità, legale neutrale}

\textbf{FORZA} +4

\textbf{DESTREZZA} +2

\textbf{COSTITUZIONE} +3

\textbf{INTELLIGENZA} +4

\textbf{SAGGEZZA} +4

\textbf{CARISMA} +4

\textbf{Iniziativa} +4 -- \textbf{Difesa} 23

\textbf{Punti Ferita} 136 (16d10 + 48)

\textbf{Movimento} 12 m, volo 18 m

\textbf{Tiri Salvezza}: Tempra +11, Riflessi +9, Volontà +10

\textbf{Competenze} Arcano +14, Consapevolezza +9, Religione +9, Storia +14

\textbf{Resistenze al Danno} da botta, perforante e tagliente di attacchi non magici

\textbf{Immunità al Danno} psichico

\textbf{Immunità alle Condizioni} affascinato, spaventato 

\textbf{Sensi} visione del vero 36 m

\textbf{Linguaggi} Comune, Sfinge

\textbf{Sfida} 11 (7.200 PE)

\emph{\textbf{Armi Magiche.}} Gli attacchi con armi della sfinge sono
magici.

\emph{\textbf{Imperscrutabile.}} La sfinge è immune a qualsiasi effetto in grado di percepirne le emozioni o leggerne i pensieri, oltre che a qualsiasi incantesimo di divinazione che rifiuti. Le prove di Saggezza (Percepire Inganni) per discernere le intenzioni o la sincerità della sfinge hanno -1d6.

\emph{\textbf{Incantesimi.}} La sfinge ha CM 9. La sua abilità da incantatore è l'Intelligenza (CD del tiro salvezza degli incantesimi 17, +9 a colpire con attacchi da incantesimo). Non ha bisogno di componenti materiali per eseguire i suoi incantesimi. La sfinge tiene preparati i seguenti incantesimi: Trucchetti (a volontà): \emph{illusione minore, mano magica,} \emph{prestidigitazione}

Difficoltà 10 (4 slot): \emph{identificare, individuazione del magico, scudo}

Difficoltà 13 (3 slot): \emph{localizza oggetto, oscurità, suggestione} 

Difficoltà 15 (3 slot): \emph{dissolvi magie, linguaggi, rimuovi maledizione}

Difficoltà 18 (3 slot): \emph{esilio, invisibilità superiore}

Difficoltà 20 (2 slot): \emph{conoscenza delle leggende}

\textbf{Azioni}

\emph{\textbf{Multiattacco.}} La sfinge può effettuare due attacchi di artiglio.

\emph{\textbf{Artiglio.} Attacco con arma da mischia}: +9 a colpire, portata 1 m, un bersaglio.

\emph{Colpisce:} 13 (2d8 + 4) danni taglienti.

\textbf{Azioni Aggiuntive}

La sfinge può effettuare 3 Azioni aggiuntive, scelte tra le opzioni seguenti. Può usare solo un'opzione leggendaria alla volta e solo al termine del turno di un'altra creatura. La sfinge recupera le Azioni aggiuntive spese all'inizio del proprio turno.

\textbf{Attacco di Artiglio.} La sfinge effettua un attacco di artiglio. 

\textbf{Eseguire un Incantesimo (Costa 3 Azioni).} La sfinge esegue un incantesimo dalla lista degli incantesimi preparati, utilizzando uno slot incantesimo come di norma.

\textbf{Teletrasporto (Costa 2 Azioni).} La sfinge si teletrasporta magicamente, insieme a tutto l'equipaggiamento che sta indossando o trasportando, in uno spazio non occupato che possa vedere, fino a 36 metri di distanza.

\medskip\index{Mostri - Spiritello}\textbf{Spiritello}

\emph{Minuscola fatato, neutrale buono}

\textbf{FORZA} -4

\textbf{DESTREZZA} +4

\textbf{COSTITUZIONE} +0

\textbf{INTELLIGENZA} +2

\textbf{SAGGEZZA} +1

\textbf{CARISMA} +0

\textbf{Iniziativa} +4 -- \textbf{Difesa} 16 (armatura di cuoio)

\textbf{Punti Ferita} 2 (1d4)

\textbf{Vulnerabilità al Danno} ferro freddo

\textbf{Movimento} 3 m, volo 12 m

\textbf{Tiri Salvezza}: Tempra +0, Riflessi +5, Volontà +2

\textbf{Competenze} Muoversi Silenziosamente / Nascondersi nelle Ombre +8 (la prova è fatta con -1d6 se lo spiritello sta volando), Consapevolezza +3

\textbf{Linguaggi} Comune, Elfico, Silvano

\textbf{Sfida} 1/4 (50 PE)

\textbf{Azioni}

\emph{\textbf{Spada Lunga.} Attacco con arma da mischia}: +2 a colpire,

portata 1 m, un bersaglio.

\emph{Colpisce:} 1 danno tagliente.

\emph{\textbf{Arco Corto.} Attacco con arma a Distanza}: +6 a colpire, gittata 12 m, un bersaglio.

\emph{Colpisce:} 1 danno perforante. Se il bersaglio è una creatura, deve riuscire un tiro salvezza di Tempra CD 10 o restare avvelenata per 1 minuto. Se il risultato di questo tiro salvezza è 5 o meno, il bersaglio cade privo di sensi per la stessa durata, o finché subisce danni o un'altra creatura usa un'azione per risvegliarlo. 

\emph{\textbf{Invisibilità.}} Lo spiritello resta invisibile finché non attacca o termina la sua concentrazione. Qualsiasi cosa che lo spiritello stia trasportando o indossando resta invisibile finché rimane in contatto con lo spiritello.

\emph{\textbf{Vista del Cuore.}} Lo spiritello entra in contatto con una creatura e ne apprende l'attuale stato emotivo. Se il bersaglio fallisce un tiro salvezza di Tempra CD 10, lo spiritello apprende anche l'allineamento della creatura. Celestiali, immondi e non morti falliscono automaticamente questo tiro salvezza.



\medskip\index{Mostri - Strige (Uccello Stigeo)}\textbf{Strige (Uccello Stigeo)}

\emph{Minuscola bestia, disallineato}

\textbf{FORZA} -3

\textbf{DESTREZZA} +3

\textbf{COSTITUZIONE} +0

\textbf{INTELLIGENZA} -4

\textbf{SAGGEZZA} -1

\textbf{CARISMA} -2

\textbf{Iniziativa} +3 -- \textbf{Difesa} 15

\textbf{Punti Ferita} 2 (1d4)

\textbf{Movimento} 3 m, volo 12 m

\textbf{Tiri Salvezza}: Tempra +2, Riflessi +6, Volontà +1

\textbf{Sensi} scurovisione 18 m

\textbf{Linguaggi} -

\textbf{Sfida} 1/8 (25 PE)

\textbf{Azioni}

\emph{\textbf{Risucchio di Sangue.} Attacco con arma da mischia}: +5 a
colpire, portata 1 m, una creatura.

\emph{Colpisce:} 5 (1d4 + 3) danni perforanti e lo strige si attacca al
bersaglio. Mentre è attaccato, lo strige non attacca. Invece, all'inizio
di ciascun turno dello strige, il bersaglio perde 5 (1d4 + 3) punti
ferita a causa della perdita di sangue.

Lo strige può staccarsi spendendo 1,5 metri di movimento. Lo fa
automaticamente dopo aver risucchiato 10 punti ferita dal bersaglio o
alla morte del bersaglio. Una creatura, compreso il bersaglio, può usare
la sua azione per staccare lo strige.

\medskip\index{Mostri - Succube/Incubo}\textbf{Succube/Incubo}

\emph{Media immondo (mutaforma), neutrale malvagio}

\textbf{FORZA} -1

\textbf{DESTREZZA} +3

\textbf{COSTITUZIONE} +1

\textbf{INTELLIGENZA} +2

\textbf{SAGGEZZA} +1

\textbf{CARISMA} +5

\textbf{Iniziativa} +3 -- \textbf{Difesa} 17

\textbf{Punti Ferita} 66 (12d8 + 12)

\textbf{Movimento} 9 m, volo 18 m

\textbf{Tiri Salvezza}: Tempra +7, Riflessi +9, Volontà +10

\textbf{Competenze} Furtività 5, Percepire Emozioni +5, Consapevolezza +5, Ingannare +9

\textbf{Resistenze al Danno} freddo, fulmine, fuoco, veleno; da botta, perforante e tagliente di attacchi non magici

\textbf{Sensi} scurovisione 18 m

\textbf{Linguaggi} Abissale, Comune, Infernale, telepatia 18 m

\textbf{Sfida} 4 (1.100 PE)

\emph{\textbf{Legame Telepatico.}} L'immondo ignora le restrizioni di raggio di azione della sua telepatia quando comunica con una creatura che ha affascinato. I due non sono neppure costretti a trovarsi sullo stesso piano di esistenza.

\emph{\textbf{Mutaforma.}} L'immondo può usare la sua azione per trasformarsi in un umanoide di taglia Piccola o Media, o per tornare alla sua vera forma. Senza le ali, l'immondo perde la velocità di volo. A parte la taglia e la velocità, le sue statistiche sono le stesse in tutte le forme. Qualsiasi equipaggiamento stia indossando o trasportando non viene trasformato. Alla morte ritorna alla sua vera forma.

\textbf{Azioni}

\emph{\textbf{Artiglio (Solo Forma Immonda).} Attacco con arma da
	mischia}:

+5 a colpire, portata 1 m, un bersaglio.

\emph{Colpisce:} 6 (1d6 + 3) danni taglienti.

\emph{\textbf{Affascinare.}} Un umanoide visibile all'immondo entro 9 metri da esso deve riuscire un tiro salvezza di Volontà CD 15 o restare magicamente affascinato per 1 giorno. Il bersaglio affascinato obbedisce ai comandi verbali o telepatici dell'immondo. Se il bersaglio subisce danni o riceve un comando suicida, può ripetere il tiro salvezza, terminando l'effetto se lo riesce. Se ilbersaglio riesce il tiro  salvezza contro l'effetto, o se l'effetto termina, il bersaglio è immune all'Affascinare dell'immondo per le successive 24 ore.

L'immondo può tenere affascinato solo un bersaglio alla volta. Se ne
affascina un altro, l'effetto sul bersaglio precedente termina.

\emph{\textbf{Bacio Risucchiante.}} L'immondo bacia una creatura affascinata o una creatura consenziente. Il bersaglio deve effettuare un tiro salvezza di Tempra CD 15 contro questa magia, subendo 32 (5d10 + 5) danni psichici se lo fallisce, o la metà di questi danni se lo riesce. I punti ferita massimi del bersaglio vengono ridotti di un ammontare pari ai danni subiti. Questa riduzione perdura finché non sorge l'alba. Il bersaglio muore se questo effetto riduce i suoi punti ferita massimi a 0.

\emph{\textbf{Forma Eterea.}} L'immondo entra magicamente nel Piano
Etereo dal Piano Materiale, e viceversa.

\medskip\index{Mostri - Tarrasque}\textbf{Tarrasque}

\emph{Mastodontica mostruosità (titano), disallineato}

\textbf{FORZA} +10

\textbf{DESTREZZA} +0

\textbf{COSTITUZIONE} +10

\textbf{INTELLIGENZA} -4

\textbf{SAGGEZZA} +0

\textbf{CARISMA} +0

\textbf{Iniziativa} +0 -- \textbf{Difesa} 35

\textbf{Punti Ferita} 676 (33d20 + 330)

\textbf{Movimento} 12 m

\textbf{Tiri Salvezza}: Tempra +31, Riflessi +22, Volontà +12

\textbf{Immunità al Danno} fuoco, veleno; armi +2

\textbf{Immunità alle Condizioni} affascinato, avvelenato, paralizzato, spaventato

\textbf{Sensi} vista cieca 36 m

\textbf{Linguaggi} -

\textbf{Sfida} 30 (155.000 PE)

\emph{\textbf{Carapace Riflettente.}} Ogni volta che il tarrasque è il bersaglio di un incantesimo \emph{dardo incantato}, un incantesimo a linea, o un incantesimo che richiede un tiro di attacco a gittata, tira un d6. Da 1 a 5, il tarrasque lo ignora. Con 6, il tarrasque lo ignora, e l'effetto viene riflesso contro l'incantatore come se fosse originato dal tarrasque, trasformando l'incantatore nel bersaglio. 

\emph{\textbf{Mostro d'Assedio.}} Il tarrasque infligge danni doppi agli oggetti e le strutture.

\emph{\textbf{Resistenza Leggendaria (3/Giorno).}} Se il tarrasque fallisce un tiro salvezza, può scegliere invece di riuscire.

\emph{\textbf{Resistenza alla Magia.}} Il tarrasque ha +1d6 ai tiri salvezza contro incantesimi o altri effetti magici.

\textbf{Azioni}

\emph{\textbf{Multiattacco.}} Il tarrasque può usare la sua Presenza Spaventosa. Poi effettua cinque attacchi: uno con il morso, due con gli artigli, uno con le corna, e uno con la coda. Al posto del morso può usare Inghiottire.

\emph{\textbf{Artiglio.} Attacco con arma da mischia}: +19 a colpire, portata 4,5 m, un bersaglio.

\emph{Colpisce:} 28 (4d8 + 10) danni taglienti.

\emph{\textbf{Coda.} Attacco con arma da mischia}: +19 a colpire,
portata 6 m, un bersaglio.

\emph{Colpisce:} 24 (4d6 + 10) danni da botta. Se il bersaglio è una creatura, deve riuscire un tiro salvezza di Tempra CD 20 o cadere prona.

\emph{\textbf{Corna.} Attacco con arma da mischia}: +19 a colpire, portata 3 m, un bersaglio.

\emph{Colpisce:} 32 (4d10 + 10) danni perforanti.

\emph{\textbf{Morso.} Attacco con arma da mischia}: +19 a colpire, portata 3 m, un bersaglio.

\emph{Colpisce:} 36 (4d12 + 10) danni perforanti. Se il bersaglio è una creatura, è afferrata (CD 20 per fuggire). Fino al termine dell'afferrare, il bersaglio è intralciato, e il tarrasque non può usare il morso contro un altro bersaglio.

\emph{\textbf{Inghiottire.}} Il tarrasque effettua una attacco di morso contro un bersaglio di taglia Grande o inferiore che sta afferrando. Se l'attacco colpisce, il bersaglio è inghiottito, e l'afferrare ha termine. Il bersaglio inghiottito è accecato e intralciato, ha copertura totale contro gli attacchi e altri effetti all'esterno del tarrasque, e subisce 56 (16d6) danni da acido all'inizio di ciascun turno del tarrasque.

Se il tarrasque subisce 60 o più danni in un singolo turno da una creatura al suo interno, il tarrasque deve riuscire un tiro salvezza su Tempra CD 30 al termine di quel turno o vomitare tutte le creature inghiottite, che cadono prone in uno spazio entro 3 metri dal tarrasque. Se il tarrasque muore, una creatura inghiottita non  più intralciata da esso e può uscire dal cadavere utilizzando 9 metri  di movimento, uscendo prona. 

\emph{\textbf{Presenza Spaventosa.}} Ogni creatura scelta dal tarrasque, che si trovi entro 36 metri da esso e consapevole della sua presenza, deve riuscire un tiro salvezza di Volontà CD 17 o restare spaventata per 1 minuto. Una creatura può ripetere il tiro salvezza al termine di ciascun suo turno, con -1d6 se il tarrasque è in linea di visuale, terminando l'effetto per sé, se lo riesce. Se il tiro salvezza della creatura ha successo o l'effetto ha termine per essa, la creatura è immune alla Presenza Spaventosa del tarrasque per le successive 24 ore.

\textbf{Azioni Aggiuntive}

Il tarrasque può effettuare 3 Azioni aggiuntive, scelte tra le opzioni seguenti. Può usare solo un'opzione leggendaria alla volta e solo al termine del turno di un'altra creatura. Il tarrasque recupera le azioni aggiuntive spese all'inizio del proprio turno.

\textbf{Attacco.} Il tarrasque effettua un attacco di artiglio o di coda. \textbf{Masticare (Costa 2 Azioni).} Il tarrasque effettua un attacco di morso o usa Inghiottire.

\textbf{Muoversi.} Il tarrasque si muove fino a metà del suo movimento.

\medskip\index{Mostri - Testuggine Dragona}\textbf{Testuggine Dragona}

\emph{Mastodontica drago, neutrale}

\textbf{FORZA} +7

\textbf{DESTREZZA} +0

\textbf{COSTITUZIONE} +5

\textbf{INTELLIGENZA} +0

\textbf{SAGGEZZA} +1

\textbf{CARISMA} +1

\textbf{Iniziativa} +0 -- \textbf{Difesa} 29

\textbf{Punti Ferita} 341 (22d20 + 110) 

\textbf{Movimento} 6 m, nuoto 12 m

\textbf{Tiri Salvezza} Tempra +12, Riflessi +8, Volontà +9

\textbf{Sensi} scurovisione 18 m

\textbf{Linguaggi} Aquan, Draconico

\textbf{Sfida} 17 (18.000 PE)

\emph{\textbf{Anfibio.}} La testuggine dragona può respirare aria e acqua.

\textbf{Azioni}

\emph{\textbf{Multiattacco.}} Il drago può effettuare tre attacchi: uno con il morso e due con gli artigli. Può effettuare un attacco di coda al posto di due attacchi di artiglio.

\emph{\textbf{Artiglio.} Attacco con arma da mischia}: +13 a colpire, portata 3 m, un bersaglio.

\emph{Colpisce:} 16 (2d8 + 7) danni taglienti.

\emph{\textbf{Coda.} Attacco con arma da mischia}: +13 a colpire, portata 4,5 m, un bersaglio.

\emph{Colpisce:} 26 (3d12 + 7) danni da botta. Se il bersaglio è una creatura, deve riuscire un tiro salvezza di Tempra CD 20 o venire spinta di 3 metri lontano dalla testuggine dragona e cadere prona.

\emph{\textbf{Morso.} Attacco con arma da mischia}: +13 a colpire, portata 4,5 m, un bersaglio.

\emph{Colpisce:} 26 (3d12 + 7) danni perforanti.

\emph{\textbf{Soffio di Vapore (Ricarica 5-6).}} La testuggine dragona esala un vapore caldo in un cono di 18 metri. Ogni creatura in quell'area deve effettuare un tiro salvezza di Tempra CD 18 e subire 52 (15d6) danni da fuoco se fallisce il tiro salvezza, o la metà di questi danni se lo riesce. Trovarsi sott'acqua non dà resistenza contro questo tipo di danno.

\medskip\index{Mostri - Troll}\textbf{Troll}

\emph{Grande gigante, caotico malvagio}

\textbf{FORZA} +4

\textbf{DESTREZZA} +1

\textbf{COSTITUZIONE} +5

\textbf{INTELLIGENZA} -2

\textbf{SAGGEZZA} -1

\textbf{CARISMA} -2

\textbf{Iniziativa} +1 -- \textbf{Difesa} 18

\textbf{Punti Ferita} 84 (8d10 + 40)

\textbf{Movimento} 9 m

\textbf{Tiri Salvezza}: Tempra +11, Riflessi +4, Volontà +3

\textbf{Competenze} Consapevolezza +2

\textbf{Sensi} scurovisione 18 m

\textbf{Linguaggi} Gigante

\textbf{Sfida} 5 (1.800 PE)

\emph{\textbf{Olfatto Affinato.}} Il troll ha +1d6 alle prove di Saggezza (Consapevolezza) basate sull'olfatto.

\emph{\textbf{Rigenerazione.}} Il troll recupera 10 punti ferita all'inizio del suo turno. Se il troll subisce danno da acido o da fuoco, questo tratto non funziona all'inizio del prossimo turno del troll. Il troll muore solo se inizia il suo turno a -5 punti ferita e non può rigenerarsi.

\textbf{Azioni}

\emph{\textbf{Multiattacco.}} Il troll può effettuare tre attacchi: uno con il morso e due con gli artigli.

\emph{\textbf{Artiglio.} Attacco con arma da mischia}: +7 a colpire, portata 1 m, un bersaglio.

\emph{Colpisce:} 11 (2d6 + 4) danni taglienti.

\emph{\textbf{Morso.} Attacco con arma da mischia}: +7 a colpire, portata 1 m, un bersaglio.

\emph{Colpisce:} 7 (1d6 + 4) danni perforanti.

\medskip\index{Mostri - Uomo Acquatico}\textbf{Uomo Acquatico}

\emph{Media umanoide (uomo acquatico), neutrale}

\textbf{FORZA} +0

\textbf{DESTREZZA} +1

\textbf{COSTITUZIONE} +1

\textbf{INTELLIGENZA} +0

\textbf{SAGGEZZA} +0

\textbf{CARISMA} +1

\textbf{Iniziativa} +1 -- \textbf{Difesa} 12

\textbf{Punti Ferita} 11 (2d8 + 2)

\textbf{Movimento} 3 m, nuoto 12 m

\textbf{Tiri Salvezza}:  Tempra +3, Riflessi +1, Volontà -1; +2 contro Ammaliamento

\textbf{Competenze} Consapevolezza +2

\textbf{Linguaggi} Aquan, Comune

\textbf{Sfida} 1/8 (25 PE)

\emph{\textbf{Anfibio.}} L'uomo acquatico può respirare aria e acqua.

\textbf{Azioni}

\emph{\textbf{Lancia.} Attacco con arma da mischia o a Distanza}: +2 a colpire, portata 1 m o gittata 6m, un bersaglio.

\emph{Colpisce:} 3 (1d6) danni perforanti, o 4 (1d8) danni perforanti se
usata con due mani per effettuare un attacco da mischia.

\medskip\index{Mostri - Uomo Albero (Treant)}\textbf{Uomo Albero (Treant)}

\emph{Enorme pianta, caotico buono}

\textbf{FORZA} +6

\textbf{DESTREZZA} -1

\textbf{COSTITUZIONE} +5

\textbf{INTELLIGENZA} +1

\textbf{SAGGEZZA} +3

\textbf{CARISMA} +1

\textbf{Iniziativa} +1 -- \textbf{Difesa} 21

\textbf{Punti Ferita} 138 (12d12 + 60)

\textbf{Movimento} 9 m

\textbf{Tiri Salvezza}: Tempra +13, Riflessi +3, Volontà +9

\textbf{Resistenze al Danno} da botta, perforante

\textbf{Vulnerabilità al Danno} fuoco

\textbf{Linguaggi} Comune, Druidico, Elfico, Silvano

\textbf{Sfida} 9 (5.000 PE)

\emph{\textbf{Falso Aspetto.}} Mentre l'uomo albero rimane immobile, è indistinguibile da un normale albero.

\emph{\textbf{Mostro d'Assedio.}} L'uomo albero infligge danni doppi agli oggetti e le strutture.

\textbf{Azioni}

\emph{\textbf{Multiattacco.}} L'uomo albero effettua due attacchi di schianto.

\emph{\textbf{Schianto.} Attacco con arma da mischia}: +10 a colpire, portata 1 m, un bersaglio.

\emph{Colpisce:} 16 (3d6 + 6) danni da botta.

\emph{\textbf{Sasso.} Attacco con arma a Distanza}: +10 a colpire, gittata 18m, un bersaglio.

\emph{Colpisce:} 28 (4d10 + 6) danni da botta.

\emph{\textbf{Animare Alberi (1/Giorno).}} L'uomo albero anima magicamente uno o due alberi visibili entro 18 metri da lui. Questi albeti hanno le stesse statistiche dell'ent, eccetto che hanno punteggio di Intelligenza e Carisma -3, non possono parlare, e hanno solo l'opzione di attacco Schianto. Un albero animato agisce come alleato dell'uomo albero. L'albero resta per 1 giorno o finché muore; finché l'uomo albero muore o si trova più di 36 metri lontano dall'albero, o finché l'uomo albero non effettua un'azione bonus per ritrasformarlo in un albero inanimato. Poi l'albero prenderà radici, se possibile.



\medskip\index{Mostri - Uomo Magma (Magmin)}\textbf{Uomo Magma (Magmin)}

\emph{Piccola elementale, caotico neutrale}

\textbf{FORZA} -2

\textbf{DESTREZZA} +2

\textbf{COSTITUZIONE} +1

\textbf{INTELLIGENZA} -1

\textbf{SAGGEZZA} +0

\textbf{CARISMA} +0

\textbf{Iniziativa} +2 -- \textbf{Difesa} 15

\textbf{Punti Ferita} 9 (2d6 + 2)

\textbf{Movimento} 9 m

\textbf{Tiri Salvezza}: Tempra +6, Riflessi +4, Volontà +3

\textbf{Resistenze al Danno} da botta, perforante e tagliente di attacchi non magici

\textbf{Immunità ai Danni} fuoco

\textbf{Sensi} scurovisione 18 m

\textbf{Linguaggi} Ignan

\textbf{Sfida} 1/2 (100 PE)

\emph{\textbf{Illuminazione Incendiaria.}} Come azione bonus, l'uomo magma può accendere o spegnere le sue fiamme. Mentre la fiamma è accesa, l'uomo magma irradia luce intensa in un raggio di 3 metri e luce fioca per ulteriori 3 metri.

\emph{\textbf{Scoppio Mortale.}} Quando l'uomo magma muore, esplode in uno scoppio di fuoco e magma. Ogni creatura entro 3 metri da esso deve effettuare un tiro salvezza di Riflessi CD 11, subendo 7 (2d6) danni da fuoco se fallisce il tiro salvezza, o la metà di questi danni se lo riesce. Gli oggetti infiammabili che non siano indossati o trasportati e che si trovino nell'area, prendono fuoco.

\textbf{Azioni}

\emph{\textbf{Tocco.} Attacco con arma da mischia}: +4 a colpire, portata 1 m, un bersaglio.

\emph{Colpisce:} 7 (2d6) danni da fuoco. Se il bersaglio è una creatura o un oggetto infiammabile, questi prende fuoco. Fino a che una creatura effettua un'azione per estinguere la fiamma, la creatura subisce 3 (1d6) danni da fuoco al termine di ciascun suo turno.

\medskip\index{Mostri - Unicorno}\textbf{Unicorno}

\emph{Grande celestiale, legale buono}

\textbf{FORZA} +4

\textbf{DESTREZZA} +2

\textbf{COSTITUZIONE} +2

\textbf{INTELLIGENZA} +0

\textbf{SAGGEZZA} +3

\textbf{CARISMA} +3

\textbf{Iniziativa} +2 -- \textbf{Difesa} 15

\textbf{Punti Ferita} 67 (9d10 + 18)

\textbf{Movimento} 15 m

\textbf{Tiri Salvezza}: Tempra +7, Riflessi +7, Volontà +6; +2 resistenza contro il Vuoto, Energia Negativa

\textbf{Immunità al Danno} veleno

\textbf{Immunità alle Condizioni} affascinato, avvelenato, paralizzato

\textbf{Sensi} scurovisione 18 m

\textbf{Linguaggi} Celestiale, Elfico, Silvano, telepatia 18 m

\textbf{Sfida} 5 (1.800 PE)

\emph{\textbf{Armi Magiche.}} Gli attacchi con armi dell'unicorno sono magici.

\emph{\textbf{Carica.}} Se l'unicorno si muove di almeno 6 metri in linea retta verso il bersaglio e lo colpisce con un attacco di corno durante lo stesso turno, il bersaglio subisce 9 (2d8) danni perforanti aggiuntivi. Se il bersaglio è una creatura, deve riuscire un tiro salvezza su Tempra CD 15 o cadere prono.

\emph{\textbf{Incantesimi Innati.}} La caratteristica da incantatore innato dell'unicorno è il Carisma (CD 14 per i tiri salvezza degli incantesimi). L'unicorno può lanciare in maniera innata i seguenti incantesimi, senza bisogno di componenti:

A volontà: \emph{arte del druido, individuazione del bene e male,} \emph{passare senza tracce}

1/giorno ciascuno: \emph{calmare emozioni, dissolvi il bene e il male,} \emph{intralciare}

\emph{\textbf{Resistenza alla Magia.}} L'unicorno ha +1d6 ai tiri salvezza contro incantesimi e altri effetti magici. 

\textbf{Azioni}

\emph{\textbf{Multiattacco.}} L'unicorno effettua due attacchi: uno con gli zoccoli e uno con il corno.

\emph{\textbf{Corno.} Attacco con arma da mischia}: +7 a colpire, portata 1 m, un bersaglio.

\emph{Colpisce:} 8 (1d8 + 4) danni perforanti.

\emph{\textbf{Zoccoli.} Attacco con arma da mischia}: +7 a colpire, portata 1 m, un bersaglio.

\emph{Colpisce:} 11 (2d6 + 4) danni da botta.

\emph{\textbf{Telestraporto (1/Giorno).}} L'unicorno può teletrasportare  magicamente sé stesso e fino a tre altre creature consenzienti  visibili entro 1,5 metri da esso, insieme a tutto  l'equipaggiamento che stanno indossando o trasportando, in un  luogo familiare all'unicorno, che si trova ad un massimo di 1,5  chilometri di distanza.

\emph{\textbf{Tocco Guaritore (3/Giorno).}} L'unicorno entra a contatto tramite il corno con un'altra creatura. Il bersaglio recupera magicamente 11 (2d8 + 2) punti ferita. Inoltre, il contatto rimuove tutte le malattie e neutralizza tutti i veleni che affliggono il bersaglio.

\textbf{Azioni Aggiuntive}

L'unicorno può effettuare 3 Azioni aggiuntive, scelte tra le opzioni seguenti. Può usare solo un'opzione leggendaria alla volta e solo al termine del turno di un'altra creatura. L'unicorno recupera le azioni aggiuntive spese all'inizio del proprio turno.

\textbf{Autoguarigione (Costa 3 Azioni).} L'unicorno recupera magicamente 11 (2d8 + 2) punti ferita.

\textbf{Scudo Scintillante (Costa 2 Azioni).} L'unicorno crea un campo magico scintillante che circonda lui o un'altra creatura visibile a lui entro 18 metri. Il bersaglio ottiene un bonus di +2 alla Difesa fino al termine del prossimo turno dell'unicorno.

\textbf{Zoccoli.} L'unicorno effettua un attacco con gli zoccoli.

\subsection{Vampiri}

\medskip\index{Mostri - Vampiro}\textbf{Vampiro}

\emph{Media non morto (mutaforma), legale malvagio}

\textbf{FORZA} +4

\textbf{DESTREZZA} +4

\textbf{COSTITUZIONE} +4

\textbf{INTELLIGENZA} +3

\textbf{SAGGEZZA} +2

\textbf{CARISMA} +4

\textbf{Iniziativa} +4 -- \textbf{Difesa} 23

\textbf{Punti Ferita} 144 (17d8 + 68)

\textbf{Movimento} 9 m

\textbf{Tiri Salvezza} : Tempra +13, Riflessi +11, Volontà +12

\textbf{Competenze} Muoversi Silenziosamente / Nascondersi nelle Ombre +9, Consapevolezza +17

\textbf{Immunità al Danno} da Vuoto; da botta, perforante e tagliente di attacchi non magici

\textbf{Sensi} scurovisione 36 m

\textbf{Linguaggi} le lingue che conosceva in vita

\textbf{Sfida} 13 (10.000 PE)

\emph{\textbf{Mutaforma.}} Se il vampiro non è sotto la luce del sole o immerso in acqua corrente, può usare la sua azione per trasformarsi in un Minuscolo pipistrello, una nube di foschia Media, o per tornare alla sua vera forma.

Mentre è in forma di pipistrello, il vampiro non può parlare, la sua velocità di passeggio è 1,5 metri e ha velocità di volo 9 metri. Le sue statistiche, a parte la taglia e la velocità, sono immutate. Qualsiasi equipaggiamento stia indossando si trasforma con esso, ma quello che stava trasportando viene fatto cadere a terra. Alla morte ritorna alla sua vera forma.

Mentre è in forma di foschia, il vampiro non può effettuare azioni, parlare o manipolare oggetti. È privo di peso, ha velocità di volo 6 metri, può fluttuare, e può entrare nello spazio di una creatura ostile e fermarsi lì. Inoltre, se in uno spazio vi passa dell'aria, la foschia può fare altrettanto senza stringersi, ma non può attraversare l'acqua. Ha +1d6 ai tiri salvezza su Tempra e Riflessi ed è immune a tutti i danni non magici, eccetto i danni subiti dalla luce del
sole.

\emph{\textbf{Debolezze del Vampiro.}} Il vampiro ha i seguenti difetti:

\emph{Danneggiato dall'Acqua Corrente.} Il vampiro subisce 20 danni da acido se termina il suo turno all'interno dell'acqua corrente.

\emph{Ipersensibilità alla Luce.} Il vampiro subisce 20 danni da Luce quando inizia il suo turno alla luce del sole. Mentre è alla luce del sole, ha -1d6 ai tiri di attacco e le prove di abilità. 

\emph{Paletto nel Cuore.} Se un'arma perforante fatta di legno viene conficcata nel cuore del vampiro mentre il vampiro è inabile nel suo luogo di riposo, il vampiro resta paralizzato finché il paletto non viene rimosso.

\emph{Proibizione.} Il vampiro non può entrare in un'abitazione senza invito da parte dei suoi occupanti.

\emph{\textbf{Fuga nella Foschia.}} Quando scende a 0 punti ferita al di fuori del suo luogo di riposo, il vampiro si trasforma in una nube di foschia (come per il tratto Mutaforma) invece di cadere privo di sensi, purché non sia esposto alla luce del sole o all'acqua corrente. Se non può trasformarsi, viene distrutto.

Mentre si trova a 0 punti ferita in questa forma, non può tornare alla sua forma di vampiro, e deve raggiungere il suo luogo di riposo entro 2 ore o venire distrutto. Una volta raggiunto il suo luogo di riposo, ritorna alla sua forma di vampiro. Resterà quindi paralizzato finché non avrà recuperato almeno 1 punto ferita. Dopo aver trascorso almeno 1 ora nel suo luogo di riposo a 0 punti ferita, il vampiro recupererà 1 punto ferita.

\emph{\textbf{Natura Non Morta.}} Il vampiro non ha bisogno di aria.

\emph{\textbf{Resistenza Leggendaria (3/Giorno).}} Se il vampiro fallisce un tiro salvezza, può scegliere invece di riuscire.

\emph{\textbf{Rigenerazione.}} Il vampiro recupera 20 punti ferita all'inizio del suo turno se possiede almeno 1 punto ferita e non è esposto alla luce del sole o l'acqua corrente. Se il vampiro subisce danno da Luce o danno dall'acqua sacra, questo tratto non funziona all'inizio del prossimo turno del vampiro.

\emph{\textbf{Scalare come Ragno.}} Il vampiro può scalare superfici difficili, compreso lo stare a testa in giù sul soffitto, senza bisogno di effettuare una prova di abilità.

\textbf{Azioni}

\emph{\textbf{Multiattacco.}} Il vampiro può effettuare due attacchi, ma solo uno di essi può essere un attacco con morso.

\emph{\textbf{Colpo Disarmato (Solo in Forma di Vampiro).} Attacco con arma da mischia}: +9 a colpire, portata 1 m, una creatura.

\emph{Colpisce:} 8 (1d8 + 4) danni da botta. Invece di infliggere danno, il vampiro può afferrare il bersaglio (CD per fuggire 18).

\emph{\textbf{Morso (Solo in Forma di Pipistrello o Vampiro).} Attacco con arma da mischia}: +9 a colpire, portata 1 m, una creatura consenziente o una creatura afferrata dal vampiro, inabile o intralciata.

\emph{Colpisce:} 7 (1d6 + 4) danni perforanti più 10 (3d6) danni da Vuoto. I punti ferita massimi del bersaglio sono ridotti di un ammontare pari al danno da Vuoto subito, e il vampiro recupera un numero di punti ferita pari a quell'ammontare. Questa riduzione permane fino alla nuova alba. Il bersaglio muore se questo effetto riduce i suoi punti ferita massimi a 0. Un umanoide ucciso in questo modo e poi sepolto nel terreno si rianima la notte seguente come progenie vampirica sotto il controllo del vampiro.

\emph{\textbf{Affascinare.}} Il vampiro prende a bersaglio un umanoide entro 9 metri che può vedere. Se il bersaglio può vedere il vampiro, deve effettuare un tiro salvezza di Volontà CD 17 contro questa magia o esserne affascinato. Il bersaglio affascinato considera il vampiro un amico fidato da ascoltare e proteggere. Sebbene il bersaglio non sia sotto il controllo del vampiro, prende le richieste e le azioni del vampiro nel modo più favorevole possibile, ed è un bersaglio consenziente dell'attacco con morso del vampiro.

Ogni volta che il vampiro o i compagni del vampiro fanno qualcosa di nocivo al bersaglio, questi può ripetere il tiro salvezza, terminando l'effetto su di sé in caso di successo. Altrimenti, l'effetto persiste 24 ore o finché il vampiro non viene distrutto, si trova su di un piano di esistenza diverso dal bersaglio, o effettua un'azione bonus per terminare l'effetto.

\emph{\textbf{Figli della Notte (1/Giorno).}} Il vampiro richiama magicamente 2d4 sciami di pipistrelli o ratti, purché il sole non sia sorto. Mentre è all'esterno, il vampiro può richiamare invece 3d6 lupi. Le creature richiamate arrivano in 1d4 round, agendo da alleati del vampiro e obbedendo ai suoi comandi. Le bestie restano per 1 ora, finché il vampiro non muore, o finché non le congeda con un'azione bonus.

\textbf{Azioni Aggiuntive}

Il vampiro può effettuare 3 Azioni aggiuntive, scelte tra le opzioni seguenti. Può usare solo un'opzione leggendaria alla volta e solo al termine del turno di un'altra creatura. Il vampiro recupera all'inizio del proprio turno le Azioni aggiuntive che ha speso.

\textbf{Colpo Disarmato.} Il vampiro effettua un colpo disarmato. \textbf{Morso (Costa 2 Azioni).} Il vampiro effettua un attacco con
morso.

\textbf{Muoversi.} Il vampiro si muove del suo movimento senza provocare attacchi di opportunità.

\medskip\index{Mostri - Progenie Vampirica}\textbf{Progenie Vampirica}

\emph{Media non morto, neutrale malvagio}

\textbf{Iniziativa} +0 -- \textbf{Difesa} 18

\textbf{Punti Ferita} 82 (11d8 + 33)

\textbf{Movimento} 9 m

\textbf{Tiri Salvezza} Tempra +3, Riflessi +2, Volontà +5

\textbf{FORZA} +3

\textbf{DESTREZZA} +3

\textbf{COSTITUZIONE} +3

\textbf{INTELLIGENZA} +0

\textbf{SAGGEZZA} +0

\textbf{CARISMA} +1

\textbf{Competenze} Muoversi Silenziosamente / Nascondersi nelle Ombre +6, Consapevolezza +3

\textbf{Resistenze ai Danni} da Vuoto; da botta, perforante e tagliente di attacchi non magici

\textbf{Sensi} scurovisione 18 m

\textbf{Linguaggi} le lingue che conosceva in vita 

\textbf{Sfida} 5 (1.800 PE)

\emph{\textbf{Debolezze della Progenie Vampirica.}} La Progenie Vampirica ha i seguenti difetti:

\emph{Danneggiato dall'Acqua Corrente.} La Progenie Vampirica subisce 20 danni da acido se termina il suo turno all'interno dell'acqua corrente.

\emph{Ipersensibilità alla Luce.} La Progenie Vampirica subisce 20 danni da Luce quando inizia il suo turno alla luce del sole. Mentre è alla luce del sole, ha -1d6 ai tiri di attacco e le prove di abilità.

\emph{Paletto nel Cuore.} La Progenie Vampirica è distrutto se un'arma perforante di legno gli viene conficcata nel cuore mentre è inabile all'interno del suo luogo di riposo.

\emph{Proibizione.} La Progenie Vampirica non può entrare in un'abitazione senza invito da parte dei suoi occupanti.

\emph{\textbf{Natura Non Morta.}} La Progenie Vampirica non ha bisogno di aria.

\emph{\textbf{Rigenerazione.}} La Progenie Vampirica recupera 10 punti ferita all'inizio del suo turno se possiede almeno 1 punto ferita e non è esposto alla luce del sole o l'acqua corrente. Se la Progenie Vampirica subisce danno da Luce o danno dall'acqua sacra, questo tratto non funziona all'inizio del prossimo turno del vampiro.

\emph{\textbf{Scalare come Ragno.}} La Progenie Vampirica può scalare superfici difficili, compreso lo stare a testa in giù sul soffitto, senza bisogno di effettuare una prova di abilità.

\textbf{Azioni}

\emph{\textbf{Multiattacco.}} La progenie vampirica può effettuare due attacchi, ma solo uno di essi può essere un attacco con morso.

\emph{\textbf{Artigli.} Attacco con arma da mischia}: +6 a colpire, portata 1 m, una creatura.

\emph{Colpisce:} 8 (2d4 + 3) danni taglienti. Invece di infliggere danno, il vampiro può afferrare il bersaglio (CD per fuggire 13). 

\emph{\textbf{Morso.} Attacco con arma da mischia}: +6 a colpire, portata 1 m, una creatura afferrata dal vampiro, inabile o intralciata.

\emph{Colpisce:} 6 (1d6 + 3) danni perforanti più 7 (2d6) danni da Vuoto. I punti ferita massimi del bersaglio sono ridotti di un ammontare pari al danno da Vuoto subito, e il vampiro recupera un numero di punti ferita pari a quell'ammontare. Questa riduzione permane fino alla nuova alba. Il bersaglio muore se questo effetto riduce i suoi punti ferita massimi a 0.

\medskip\index{Mostri - Verme Purpureo}\textbf{Verme Purpureo}

\emph{Mastodontica mostruosità, disallineato}

\textbf{FORZA} +9

\textbf{DESTREZZA} -2

\textbf{COSTITUZIONE} +6

\textbf{INTELLIGENZA} -5

\textbf{SAGGEZZA} -1

\textbf{CARISMA} -3

\textbf{Iniziativa} -2 -- \textbf{Difesa} 26

\textbf{Punti Ferita} 247 (15d20 + 90)

\textbf{Movimento} 15 m, scavo 9 m

\textbf{Tiri Salvezza}: Tempra +17, Riflessi +8, Volontà +4

\textbf{Sensi} vista cieca 9 m, senso tellurico 18 m

\textbf{Linguaggi} -

\textbf{Sfida} 15 (13.000 PE)

\emph{\textbf{Scavatore di Tunnel.}} Il verme può scavare attraverso la roccia solida a metà della velocità di scavare e lascia un tunnel di 3 metri di diametro dietro di sè.

\textbf{Azioni}

\emph{\textbf{Multiattacco.}} Il verme effettua due attacchi: uno con il morso e uno con il pungiglione.

\emph{\textbf{Morso.} Attacco con arma da mischia}: +9 a colpire,
portata 3 m, un bersaglio.

\emph{Colpisce:} 22 (3d8 + 9) danni perforanti. Se il bersaglio è una creatura di taglia Grande, deve riuscire un tiro salvezza di Riflessi CD 19 o venire inghiottita dal verme. Mentre è inghiottita, la creatura è accecata e intralciata, ha copertura totale contro gli attacchi e altri effetti provenienti dall'esterno del verme, e subisce 21 (6d6) danni da acido all'inizio di ciascun turno del verme.

Se il verme subisce 30 o più danni in un singolo turno da una creatura al suo interno, il verme deve riuscire un tiro salvezza di Tempra CD 21 al termine del suo turno o vomitare tutte le creature inghiottite, che cadono prone in uno spazio entro 3 metri dal verme. Se il verme muore, una creatura inghiottita non risulta più intralciata da esso e può fuggire dal cadavere usando 6 metri di movimento, uscendo prona.

\emph{\textbf{Pungiglione.} Attacco con arma da mischia}: +9 a colpire, portata 3 m, una creatura.

\emph{Colpisce:} 19 (3d6 + 9) danni perforanti, e il bersaglio deve effettuare un tiro salvezza di Tempra CD 19, subendo 42 (12d6) danni da veleno se fallisce il tiro salvezza, o la metà di questi danni se lo riesce.

\medskip\index{Mostri - Viverna}\textbf{Viverna}

\emph{Grande drago, disallineato}

\textbf{FORZA} +4

\textbf{DESTREZZA} +0

\textbf{COSTITUZIONE} +3

\textbf{INTELLIGENZA} -3

\textbf{SAGGEZZA} +1

\textbf{CARISMA} -2

\textbf{Iniziativa} +0 -- \textbf{Difesa} 16

\textbf{Punti Ferita} 110 (13d10 + 39)

\textbf{Movimento} 6 m, volo 24 m

\textbf{Tiri Salvezza}: Tempra +9, Riflessi +6, Volontà +8

\textbf{Competenze} Consapevolezza +4

\textbf{Sensi} scurovisione 18 m

\textbf{Linguaggi} -

\textbf{Sfida} 6 (2.300 PE)

\textbf{Azioni}

\emph{\textbf{Multiattacco.}} La viverna può effettuare due attacchi: uno con il morso e uno con il pungiglione. Mentre vola, può usare i suoi artigli al posto di uno degli altri attacchi.

\emph{\textbf{Artigli.} Attacco con arma da mischia}: +7 a colpire, portata 1 m, un bersaglio.

\emph{Colpisce:} 13 (2d8 + 4) danni taglienti.

\emph{\textbf{Morso.} Attacco con arma da mischia}: +7 a colpire, portata 3 m, una creatura.

\emph{Colpisce:} 11 (2d6 + 4) danni perforanti.

\emph{\textbf{Pungiglione.} Attacco con arma da mischia}: +7 a colpire, portata 3 m, una creatura.

\emph{Colpisce:} 11 (2d6 + 4) danni perforanti. Il bersaglio deve effettuare un tiro salvezza di Tempra CD 15, e subire 24 (7d6) danni da veleno se lo fallisce, o la metà di questi danni se lo riesce.

\medskip\index{Mostri - Wight}\textbf{Wight}

\emph{Media non morto, neutrale malvagio}

\textbf{FORZA} +2

\textbf{DESTREZZA} +2

\textbf{COSTITUZIONE} +3

\textbf{INTELLIGENZA} +0

\textbf{SAGGEZZA} +1

\textbf{CARISMA} +2

\textbf{Iniziativa} +2 -- \textbf{Difesa} 16 (armatura borchiata)

\textbf{Punti Ferita} 45 (6d8 + 18)

\textbf{Movimento} 9 m

\textbf{Tiri Salvezza}: Tempra +3, Riflessi +2, Volontà +5

\textbf{Competenze} Muoversi Silenziosamente / Nascondersi nelle Ombre +4, Consapevolezza +3

\textbf{Resistenze al Danno} da Vuoto; da botta, perforante e tagliente di attacchi non magici o che non siano argentati

\textbf{Immunità al Danno} veleno

\textbf{Immunità alle Condizioni} avvelenato, sfinimento

\textbf{Sensi} scurovisione 18 m

\textbf{Linguaggi} le lingue che conosceva in vita

\textbf{Sfida} 3 (700 PE)

\emph{\textbf{Natura Non Morta.}} Il wight non ha bisogno di aria, cibo, bevande o sonno.

\emph{\textbf{Sensibilità alla Luce}}. Mentre è alla luce del sole, il wight ha -1d6 ai tiri di attacco, oltre che alle prove di Saggezza (Consapevolezza) basate sulla vista.

\textbf{Azioni}

\emph{\textbf{Multiattacco.}} Il wight può effettuare due attacchi con la spada lungha o due attacchi con l'arco lungo. Può usare Risucchiare Vita al posto di uno dei suoi attacchi con la spada lungha.

\emph{\textbf{Risucchiare Vita.} Attacco con arma da mischia}: +4 a colpire, portata 1 m, una creatura.

\emph{Colpisce:} 5 (1d6 + 2) danni da Vuoto. Il bersaglio deve riuscire un tiro salvezza di Tempra CD 13 o vedere i suoi punti ferita massimi ridotti di un ammontare pari al danno subito. Questa riduzione perdura fino al sorgere della nuova alba. Il bersaglio muore se l'effetto riduce i suoi punti ferita massimi a 0.

Un umanoide ucciso da questo attacco si rianima 24 ore più tardi come zombi sotto il controllo del wight, a meno che l'umanoide non venga prima riportato in vita o il corpo sia distrutto. Il wight non può controllare più di dodici zombi alla volta.

\emph{\textbf{Spada Lunga.} Attacco con arma da mischia}: +4 a colpire, portata 1 m, un bersaglio.

\emph{Colpisce:} 6 (1d8 + 2) danni taglienti o 7 (1d10 + 2) danni taglienti se usata con due mani.

\emph{\textbf{Arco Lungo.} Attacco con arma a Distanza}: +4 a colpire, gittata 45m, un bersaglio.

\emph{Colpisce:} 6 (1d8 + 2) danni perforanti.

\medskip\index{Mostri - Wraith}\textbf{Wraith}

\emph{Media non morto, neutrale malvagio}

\textbf{FORZA} -2

\textbf{DESTREZZA} +3

\textbf{COSTITUZIONE} +3

\textbf{INTELLIGENZA} +1

\textbf{SAGGEZZA} +2

\textbf{CARISMA} +2

\textbf{Iniziativa} +3 -- \textbf{Difesa} 16

\textbf{Punti Ferita} 67 (9d8 + 27)

\textbf{Movimento} 0 m, volo 18 m (fluttua)

\textbf{Tiri Salvezza}: Tempra +6, Riflessi +4, Volontà +6

\textbf{Resistenze al Danno} acido, freddo, fulmine, fuoco, tuono; da botta, perforante e tagliente di attacchi non magici o che non siano argentati

\textbf{Immunità al Danno} da Vuoto, veleno

\textbf{Immunità alle Condizioni} affascinato, afferrato, avvelenato, intralciato, paralizzato, pietrificato, prono, sfinimento

\textbf{Sensi} scurovisione 18 m 

\textbf{Linguaggi} le lingue
che conosceva in vita 

\textbf{Sfida} 5 (1.800 PE)

\emph{\textbf{Movimento Incorporeo.}} Il wraith può attraversare creature e oggetti come fossero terreno difficile. Subisce 5 (1d10) danni da forza se termina il proprio turno all'interno di un oggetto.

\emph{\textbf{Natura Non Morta.}} Il wraith non ha bisogno di aria, cibo, bevande o sonno.

\emph{\textbf{Sensibilità alla Luce}}. Mentre è alla luce del sole, il wraith ha -1d6 ai tiri di attacco, oltre che alle prove di Saggezza (Consapevolezza) basate sulla vista.

\textbf{Azioni}

\emph{\textbf{Risucchiare Vita.} Attacco con arma da mischia}: +6 a colpire, portata 1 m, una creatura.

\emph{Colpisce:} 21 (4d8 + 3) danni da Vuoto. Il bersaglio deve riuscire un tiro salvezza di Tempra CD 14 o vedere i suoi punti ferita massimi ridotti di un ammontare pari al danno subito. Questa riduzione perdura fino al sorgere della nuova alba. Il bersaglio muore se l'effetto riduce i suoi punti ferita massimi a 0.

\emph{\textbf{Creare Spettro.}} Il wraith prende a bersaglio un umanoide entro 3 metri da esso e che sia morto da non più di 1 minuto e per cause violente. Lo spirito del bersaglio si anima come spettro nello spazio del suo cadavere e nello spazio più vicino non occupato. Lo spettro è sotto ilcontrollo del wraith. Il wraith non può tenere più di sette  spettri alla volta sotto il suo controllo.

\medskip\index{Mostri - Xorn}\textbf{Xorn}

\emph{Media elementale, neutrale}

\textbf{FORZA} +3

\textbf{DESTREZZA} +0

\textbf{COSTITUZIONE} +6

\textbf{INTELLIGENZA} +0

\textbf{SAGGEZZA} +0

\textbf{CARISMA} +0

\textbf{Iniziativa} +0 -- \textbf{Difesa} 22

\textbf{Punti Ferita} 73 (7d8 + 42)

\textbf{Movimento} 6 m, scavo 6 m

\textbf{Tiri Salvezza}: Tempra +8, Riflessi +2, Volontà +5

\textbf{Competenze} Muoversi Silenziosamente / Nascondersi nelle Ombre +3, Consapevolezza +6

\textbf{Resistenze al Danno} perforante e tagliente di attacchi non magici che non siano di adamantio

\textbf{Sensi} scurovisione 18 m, senso tellurico 18 m

\textbf{Linguaggi} Terran

\textbf{Sfida} 5 (1.800 PE)

\emph{\textbf{Mimetismo di Pietra.}} Lo xorn ha +1d6 alle prove di Destrezza (Nascondersi nelle ombre) effettuate per nascondersi su terreno roccioso.

\emph{\textbf{Scorrere sulla Terra.}} Lo xorn può scavare attraversa la terra e la pietra non magiche e non lavorate. Quando lo fa, lo xorn non disturba il materiale che sposta.

\emph{\textbf{Senso del Tesoro.}} Lo xorn può individuare precisamente, con l'olfatto, la posizione di metalli e pietre preziose, come monete e gemme, entro 18 metri da esso.

\textbf{Azioni}

\emph{\textbf{Multiattacco.}} Lo xorn effettua tre attacchi di artiglio e un attacco di morso.

\emph{\textbf{Artiglio.} Attacco con arma da mischia}: +6 a colpire, portata 1 m, un bersaglio.

\emph{Colpisce:} 6 (1d6 + 3) danni taglienti.

\emph{\textbf{Morso.} Attacco con arma da mischia}: +6 a colpire, portata 1 m, un bersaglio.

\emph{Colpisce:} 13 (3d6 + 3) danni perforanti.

\medskip\index{Mostri - Zombi}\textbf{Zombi}

\emph{Media non morto, neutrale malvagio}

\textbf{FORZA} +1

\textbf{DESTREZZA} -2

\textbf{COSTITUZIONE} +3

\textbf{INTELLIGENZA} -4

\textbf{SAGGEZZA} -2

\textbf{CARISMA} -3

\textbf{Iniziativa} -2 -- \textbf{Difesa} 9

\textbf{Punti Ferita} 22 (3d8 + 9)

\textbf{Movimento} 6 m

\textbf{Tiri Salvezza}  Tempra +0, Riflessi +0, Volontà +3

\textbf{Immunità al Danno} veleno

\textbf{Immunità alle Condizioni} avvelenato

\textbf{Sensi} scurovisione 18 m

\textbf{Linguaggi} comprende tutte le lingue che parlava in vita ma non può parlare

\textbf{Sfida} 1/4 (50 PE)

\emph{\textbf{Natura Non Morta.}} Lo zombi non ha bisogno di aria, cibo, bevande o sonno.

\emph{\textbf{Tempra dei Non Morti.}} Se il danno riduce lo zombi a 0 punti ferita, lo zombi deve effettuare un tiro salvezza di Tempra CD 5 + il danno subito, a meno che il danno non sia da Luce o un colpo critico. Se riesce, lo zombi scende invece a 1 punto ferita.

\textbf{Azioni}

\emph{\textbf{Schianto.} Attacco con arma da mischia}: +3 a colpire, portata 1 m, un bersaglio.

\emph{Colpisce:} 4 (1d6 + 1) danni da botta.

\medskip\index{Mostri - Zombi Ogre}\textbf{Zombi Ogre}

\emph{Grande non morto, neutrale malvagio}

\textbf{FORZA} +4

\textbf{DESTREZZA} -2

\textbf{COSTITUZIONE} +4

\textbf{INTELLIGENZA} -4

\textbf{SAGGEZZA} -2

\textbf{CARISMA} -3

\textbf{Iniziativa} -2 -- \textbf{Difesa} 9

\textbf{Punti Ferita} 85 (9d10 + 36)

\textbf{Movimento} 9 m

\textbf{Tiri Salvezza}: Tempra +6, Riflessi +0, Volontà +3

\textbf{Immunità al Danno} veleno

\textbf{Immunità alle Condizioni} avvelenato

\textbf{Sensi} scurovisione 18 m

\textbf{Linguaggi} comprende Comune e Gigante ma non può parlare

\textbf{Sfida} 2 (450 PE)

\emph{\textbf{Natura Non Morta.}} Lo zombi non ha bisogno di aria, cibo, bevande o sonno.

\emph{\textbf{Tempra dei Non Morti.}} Se il danno riduce lo zombi a 0 punti ferita, lo zombi deve effettuare un tiro salvezza di Tempra CD 5 + il danno subito, a meno che il danno non sia da Luce o un colpo critico. Se riesce, lo zombi scende invece a 1 punto ferita.

\textbf{Azioni}

\emph{\textbf{Mazza Chiodata.} Attacco con arma da mischia}: +6 a colpire, portata 1 m, un bersaglio.

\emph{Colpisce:} 13 (2d8 + 4) danni da botta.


\subsection{Appendice A: Creature Varie}

Questa appendice contiene le statistiche di vari animali, parassiti e
altre creature. Le statistiche sono organizzate in ordine alfabetico.

\medskip\textbf{Albero Risvegliato}\index{Mostri - Albero Risvegliato}

L'albero risvegliato è un normale albero fornito dalla magia di capacità
senziente e mobilità.

\emph{Enorme pianta, disallineato}

\textbf{FORZA} +4

\textbf{DESTREZZA} -2

\textbf{COSTITUZIONE} +2

\textbf{INTELLIGENZA} +0

\textbf{SAGGEZZA} +0

\textbf{CARISMA} -2

\textbf{Iniziativa} +0 -- \textbf{Difesa} 14

\textbf{Punti Ferita} 59 (7d12 + 14)

\textbf{Movimento} 6 m

\textbf{Tiri Salvezza}: Tempra +6, Riflessi -1, Volontà +1

\textbf{Vulnerabilità al Danno} fuoco

\textbf{Resistenze al Danno} da botta, perforante

\textbf{Lingue} una lingua conosciuta dal suo creatore

\textbf{Sfida} 2 (450 PE)

\emph{\textbf{Falso Aspetto.}} Mentre l'albero rimane immobile, è indistinguibile da un normale albero.

\textbf{Azioni}

\emph{\textbf{Schianto.} Attacco con Arma da Mischia}: +6 a colpire, portata 3 m, un bersaglio.

\emph{Colpisce:} 14 (3d6 + 4) danni da botta.

\medskip\textbf{Alce}\index{Mostri - Alce}

\emph{Grande bestia, disallineato}

\textbf{FORZA} +3

\textbf{DESTREZZA} +0

\textbf{COSTITUZIONE} +1

\textbf{INTELLIGENZA} -4

\textbf{SAGGEZZA} +0

\textbf{CARISMA} -2

\textbf{Iniziativa} +0 -- \textbf{Difesa} 11

\textbf{Punti Ferita} 13 (2d10 + 2)

\textbf{Movimento} 15 m

\textbf{Tiri Salvezza}:  Tempra +4, Riflessi +1, Volontà +0

\textbf{Lingue} -

\textbf{Sfida} 1/4 (50 PE)

\emph{\textbf{Carica.}} Se l'alce si muove di almeno 6 metri diretto verso il bersaglio e lo colpisce con un attacco di rostro durante lo stesso turno, il bersaglio subisce 7 (2d6) danni da botta aggiuntivi. Se il bersaglio è una creatura, deve riuscire un tiro salvezza di Tempra
CD 13 o cadere prono.

\textbf{Azioni}

\emph{\textbf{Rostro.} Attacco con Arma da Mischia}: +5 a colpire, portata 1 m, un bersaglio.

\emph{Colpisce:} 6 (1d6 + 3) danni da botta.

\emph{\textbf{Zoccoli.} Attacco con Arma da Mischia}: +5 a colpire, portata 1 m, una creatura prona.

\emph{Colpisce:} 8 (2d4 + 3) danni da botta.

\medskip\textbf{Alce Gigante}\index{Mostri - Alce Gigante}

\emph{Enorme bestia, disallineato}

\textbf{FORZA} +4

\textbf{DESTREZZA} +3

\textbf{COSTITUZIONE} +2

\textbf{INTELLIGENZA} -2

\textbf{SAGGEZZA} +2

\textbf{CARISMA} +0

\textbf{Iniziativa} +3 -- \textbf{Difesa} 15

\textbf{Punti Ferita} 42 (5d12 + 10)

\textbf{Movimento} 18 m

\textbf{Tiri Salvezza}:  Tempra +8, Riflessi +7, Volontà +2

\textbf{Competenze} Consapevolezza +4

\textbf{Lingue} Alce Gigante, comprende il Comune, l'Elfico e il

Silvano ma non può parlarli

\textbf{Sfida} 2 (450 PE)

\emph{\textbf{Carica.}} Se l'alce si muove di almeno 6 metri diretto verso il bersaglio e lo colpisce con un attacco di rostro durante lo stesso turno, il bersaglio subisce 7 (2d6) danni da botta aggiuntivi. Se il bersaglio è una creatura, deve riuscire un tiro salvezza di Tempra CD 14 o cadere prono.

\textbf{Azioni}

\emph{\textbf{Rostro.} Attacco con Arma da Mischia}: +6 a colpire, portata 3 m, un bersaglio.

\emph{Colpisce:} 11 (2d6 + 4) danni perforanti.

\emph{\textbf{Zoccoli.} Attacco con Arma da Mischia}: +6 a colpire, portata 1 m, una creatura prona.

\emph{Colpisce:} 22 (4d4 + 4) danni da botta.

\medskip\textbf{Aquila}\index{Mostri - Aquila}

\emph{Piccola bestia, disallineato}

\textbf{FORZA} -2

\textbf{DESTREZZA} +2

\textbf{COSTITUZIONE} +0

\textbf{INTELLIGENZA} -4

\textbf{SAGGEZZA} +2

\textbf{CARISMA} -2

\textbf{Iniziativa} +2 -- \textbf{Difesa} 13

\textbf{Punti Ferita} 3 (1d6)

\textbf{Movimento} 3 m, volo 18 m

\textbf{Tiri Salvezza}: Tempra +3, Riflessi +4, Volontà +2

\textbf{Competenze} Consapevolezza +4

\textbf{Lingue} -

\textbf{Sfida} 0 (10 PE)

\emph{\textbf{Vista Affinata.}} L'aquila ha +1d6 nelle prove di Saggezza (Consapevolezza) basate sulla vista.

\textbf{Azioni}

\emph{\textbf{Speroni.} Attacco con Arma da Mischia}: +4 a colpire, portata 1 m, un bersaglio.

\emph{Colpisce:} 4 (1d4 + 2) danni taglienti.

\medskip\textbf{Aquila Gigante}\index{Mostri - Aquila Gigante}

L'aquila gigante è una nobile creatura che parla la propria lingua e comprende quella di altre razze.

\emph{Grande bestia, neutrale buono}

\textbf{FORZA} +3

\textbf{DESTREZZA} +3

\textbf{COSTITUZIONE} +1

\textbf{INTELLIGENZA} -1

\textbf{SAGGEZZA} +2

\textbf{CARISMA} +0

\textbf{Iniziativa} +3 -- \textbf{Difesa} 14

\textbf{Punti Ferita} 26 (4d10 + 4)

\textbf{Movimento} 3 m, volo 24 m

\textbf{Tiri Salvezza}: Tempra +5, Riflessi +7, Volontà +3

\textbf{Competenze} Consapevolezza +4

\textbf{Lingue} Aquila Gigante, comprende il Comune e l'Auran ma non può parlarli

\textbf{Sfida} 1 (200 PE)

\emph{\textbf{Vista Affinata.}} L'aquila ha +1d6 nelle prove di Saggezza (Consapevolezza) basate sulla vista.

\textbf{Azioni}

\emph{\textbf{Multiattacco.}} L'aquila effettua due attacchi: uno con il becco e uno con gli speroni.

\emph{\textbf{Becco.} Attacco con Arma da Mischia}: +5 a colpire, portata 1 m, un bersaglio.

\emph{Colpisce:} 6 (1d6 + 3) danni perforanti.

\emph{\textbf{Speroni.} Attacco con Arma da Mischia}: +5 a colpire, portata 1 m, un bersaglio.

\emph{Colpisce:} 10 (2d6 + 3) danni taglienti.

\medskip\textbf{Avvoltoio}\index{Mostri - Avvoltoio}

\emph{Media bestia, disallineato}

\textbf{FORZA} -2

\textbf{DESTREZZA} +0

\textbf{COSTITUZIONE} +1

\textbf{INTELLIGENZA} -4

\textbf{SAGGEZZA} +1

\textbf{CARISMA} -3

\textbf{Iniziativa} +0 -- \textbf{Difesa} 11

\textbf{Punti Ferita} 5 (1d8 + 1)

\textbf{Movimento} 3 m, volo 15 m

\textbf{Tiri Salvezza}: Tempra +6, Riflessi +3, Volontà +1; +4 contro malattie

\textbf{Competenze} Consapevolezza +3

\textbf{Lingue} -

\textbf{Sfida} 0 (10 PE)

\emph{\textbf{Olfatto e Vista Affinati.}} L'avvoltoio ha +1d6 nelle prove di Saggezza (Consapevolezza) basate su olfatto o vista.

\emph{\textbf{Tattiche di Branco.}} L'avvoltoio ha +1d6 al tiro di attacco contro una creatura se almeno uno degli alleati dell'avvoltoio si trova entro 1,5 metri dalla creatura e quell'alleato non è inabile.

\textbf{Azioni}

\emph{\textbf{Becco.} Attacco con Arma da Mischia}: +2 a colpire, portata 1 m, un bersaglio.

\emph{Colpisce:} 2 (1d4) danni perforanti.

\medskip\textbf{Avvoltoio Gigante}\index{Mostri - Avvoltoio Gigante}

L'avvoltoio gigante possiede un'intelligenza superiore e un'attitudine maligna.

\emph{Grande bestia, neutrale malvagio}

\textbf{FORZA} +2

\textbf{DESTREZZA} +0

\textbf{COSTITUZIONE} +2

\textbf{INTELLIGENZA} -2

\textbf{SAGGEZZA} +1

\textbf{CARISMA} -2

\textbf{Iniziativa} +0 -- \textbf{Difesa} 11

\textbf{Punti Ferita} 22 (3d10 + 6)

\textbf{Movimento} 3 m, volo 18 m

\textbf{Tiri Salvezza}: Tempra +10, Riflessi +6, Volontà +3; +4 contro malattie

\textbf{Competenze} Consapevolezza +3

\textbf{Lingue} comprende il Comune ma non può parlare

\textbf{Sfida} 1 (200 PE)

\emph{\textbf{Olfatto e Vista Affinati.}} L'avvoltoio ha +1d6 nelle prove di Saggezza (Consapevolezza) basate su olfatto o vista.

\emph{\textbf{Tattiche di Branco.}} L'avvoltoio ha +1d6 al tiro di attacco contro una creatura se almeno uno degli alleati dell'avvoltoio si trova entro 1,5 metri dalla creatura e quell'alleato non è inabile.

\textbf{Azioni}

\emph{\textbf{Multiattacco.}} L'avvoltoio effettua due attacchi: uno con il becco e uno con gli speroni.

\emph{\textbf{Becco.} Attacco con Arma da Mischia}: +4 a colpire, portata 1 m, un bersaglio.

\emph{Colpisce:} 7 (2d4 + 2) danni perforanti.

\emph{\textbf{Speroni.} Attacco con Arma da Mischia}: +4 a colpire, portata 1 m, un bersaglio.

\emph{Colpisce:} 9 (2d6 + 2) danni taglienti.

\medskip\textbf{Babbuino}\index{Mostri - Babbuino}

\emph{Piccola bestia, disallineato}

\textbf{FORZA} -1

\textbf{DESTREZZA} +2

\textbf{COSTITUZIONE} +0

\textbf{INTELLIGENZA} -3

\textbf{SAGGEZZA} +1

\textbf{CARISMA} -2

\textbf{Iniziativa} +2 -- \textbf{Difesa} 13

\textbf{Punti Ferita} 3 (1d6)

\textbf{Movimento} 9 m, scalata 9 m

\textbf{Tiri Salvezza}: Tempra +3, Riflessi +4, Volontà +1

\textbf{Lingue} -

\textbf{Sfida} 0 (10 PE)

\emph{\textbf{Tattiche di Branco.}} Il babbuino ha +1d6 al tiro di attacco contro una creatura se almeno uno degli alleati del babbuino si trova entro 1,5 metri dalla creatura e quell'alleato non è inabile.

\textbf{Azioni}

\emph{\textbf{Morso.} Attacco con Arma da Mischia}: +1 a colpire, portata 1 m, un bersaglio.

\emph{Colpisce:} 1 (1d4 - 1) danni perforanti.


\medskip\textbf{Balena Assassina (Orca)}\index{Mostri - Orca}

\emph{Enorme bestia, disallineato}

\textbf{FORZA} +4

\textbf{DESTREZZA} +0

\textbf{COSTITUZIONE} +1

\textbf{INTELLIGENZA} -4

\textbf{SAGGEZZA} +1

\textbf{CARISMA} -2

\textbf{Iniziativa} +0 -- \textbf{Difesa} 14

\textbf{Punti Ferita} 90 (12d12 + 12)

\textbf{Movimento} 0 m, nuoto 18 m

\textbf{Tiri Salvezza}: Tempra +9, Riflessi +8, Volontà +5

\textbf{Competenze} Consapevolezza +3

\textbf{Sensi} vista cieca 36 m

\textbf{Lingue} -

\textbf{Sfida} 3 (700 PE)

\emph{\textbf{Ecolocazione.}} La balena non può usare la vista cieca se assordata.

\emph{\textbf{Trattenere il Fiato.}} La balena può trattenere il fiato per 30 minuti

\emph{\textbf{Udito Affinato.}} La balena ha +1d6 alle prove di Saggezza (Consapevolezza) basate sull'udito.

\textbf{Azioni}

\emph{\textbf{Morso.} Attacco con Arma da Mischia}: +6 a colpire, portata 1 m, un bersaglio.

\emph{Colpisce:} 21 (5d6 + 4) danni perforanti.

\medskip\textbf{Becco d'Ascia}\index{Mostri - Becco d'Ascia}

Il becco d'ascia è un grosso e slanciato volatile privo di ali ma con potenti gambe, un becco a cuneo, e un pessimo carattere.

\emph{Grande bestia, disallineato}

\textbf{FORZA} +2

\textbf{DESTREZZA} +1

\textbf{COSTITUZIONE} +1

\textbf{INTELLIGENZA} -4

\textbf{SAGGEZZA} +0

\textbf{CARISMA} -3

\textbf{Iniziativa} +1 -- \textbf{Difesa} 12

\textbf{Punti Ferita} 19 (3d10 + 3)

\textbf{Movimento} 15 m

\textbf{Tiri Salvezza}: Tempra +3, Riflessi +1, Volontà +1

\textbf{Lingue} -

\textbf{Sfida} 1/4 (50 PE)

\textbf{Azioni}

\emph{\textbf{Becco.} Attacco con Arma da Mischia}: +4 a colpire, portata 1 m, un bersaglio.

\emph{Colpisce:} 6 (1d8 + 2) danni taglienti.

\medskip\textbf{Cammello}\index{Mostri - Cammello}

\emph{Grande bestia, disallineato}

\textbf{FORZA} +3

\textbf{DESTREZZA} -1

\textbf{COSTITUZIONE} +2

\textbf{INTELLIGENZA} -4

\textbf{SAGGEZZA} -1

\textbf{CARISMA} -3

\textbf{Iniziativa} -1 -- \textbf{Difesa} 10

\textbf{Punti Ferita} 15 (2d10 + 4)

\textbf{Movimento} 15 m

\textbf{Tiri Salvezza}: Tempra +5, Riflessi +6, Volontà +0 

\textbf{Lingue} -

\textbf{Sfida} 1/8 (25 PE)

\textbf{Azioni}

\emph{\textbf{Morso.} Attacco con Arma da Mischia}: +5 a colpire, portata 1 m, un bersaglio.

\emph{Colpisce:} 2 (1d4) danni da botta.

\medskip\textbf{Cane della Morte}\index{Mostri - Cane della Morte}

Il cane della morte è un orribile segugio a due teste che si aggira per pianure, deserti e sotterranei.

\emph{Media mostruosità, neutrale malvagio}

\textbf{FORZA} +2

\textbf{DESTREZZA} +2

\textbf{COSTITUZIONE} +2

\textbf{INTELLIGENZA} -4

\textbf{SAGGEZZA} +1

\textbf{CARISMA} -2

\textbf{Iniziativa} +2 -- \textbf{Difesa} 13

\textbf{Punti Ferita} 39 (6d8 + 12)

\textbf{Movimento} 12 m

\textbf{Tiri Salvezza}: Tempra +4, Riflessi +5, Volontà +2

\textbf{Competenze} Muoversi Silenziosamente / Nascondersi nelle Ombre +4, Consapevolezza +5

\textbf{Sensi} visione al buio 36 m

\textbf{Lingue} -

\textbf{Sfida} 1 (200 PE)

\emph{\textbf{Bicefalo.}} Il cane ha +1d6 nelle prove di Saggezza (Consapevolezza) e nei tiri salvezza contro le condizioni accecato, affascinato, assordato, spaventato, stordito o svenuto.

\textbf{Azioni}

\emph{\textbf{Multiattacco.}} Il cane effettua due attacchi di morso. 

\emph{\textbf{Morso.} Attacco con Arma da Mischia}: +4 a colpire, portata 1 m, un bersaglio.

\emph{Colpisce:} 5 (1d6 + 2) danni perforanti. Se il bersaglio è una creatura, deve riuscire un tiro salvezza di Tempra CD 12 contro la malattia o restare avvelenato finché la malattia non viene curata. Dopo ogni 24 ore, la creatura deve ripetere il tiro salvezza, riducendo i suoi punti ferita massimi di 5 (1d10) in caso di fallimento. Questa riduzione perdura finché la malattia non viene curata. La creatura muore se la malattia riduce i suoi punti ferita massimi a 0.

\medskip\textbf{Cane Intermittente}\index{Mostri - Cane Intermittente}

Il cane intermittente deriva il nome dalla sua abilità di entrare e uscire dalla realtà, un talento che usa per attaccare ed evitare di essere attaccato.

\emph{Media fatato, legale buono}

\textbf{FORZA} +1

\textbf{DESTREZZA} +3

\textbf{COSTITUZIONE} +1

\textbf{INTELLIGENZA} +0

\textbf{SAGGEZZA} +1

\textbf{CARISMA} +0

\textbf{Iniziativa} +3 -- \textbf{Difesa} 14

\textbf{Punti Ferita} 22 (4d8 + 4)

\textbf{Vulnerabilità al Danno} ferro freddo

\textbf{Movimento} 12 m

\textbf{Tiri Salvezza}:  Tempra +5, Riflessi +5, Volontà +4

\textbf{Competenze} Muoversi Silenziosamente / Nascondersi nelle Ombre +5, Consapevolezza +3

\textbf{Lingue} Cane Intermittente, comprende il Silvano ma non può parlarlo

\textbf{Sfida} 1/4 (50 PE)

\emph{\textbf{Udito e Olfatto Affinato.}} Il cane ha +1d6 nelle prove di Saggezza (Consapevolezza) basate su udito o olfatto.

\textbf{Azioni}

\emph{\textbf{Morso.} Attacco con Arma da Mischia}: +3 a colpire, portata 1 m, un bersaglio.

\emph{Colpisce:} 4 (1d6 + 1) danni perforanti.

\emph{\textbf{Teletrasporto (Ricarica 4-6).}} Il cane si teletrasporta magicamente, insieme a qualsiasi cosa stia indossando o trasportando, fino a 12 metri in uno spazio non occupato che possa vedere. Prima o dopo il teletrasporto, il cane può effettuare un attacco di morso.

\medskip\textbf{Caprone}\index{Mostri - Caprone}

\emph{Media bestia, disallineato}

\textbf{FORZA} +1

\textbf{DESTREZZA} +0

\textbf{COSTITUZIONE} +0

\textbf{INTELLIGENZA} -4

\textbf{SAGGEZZA} +0

\textbf{CARISMA} -3

\textbf{Iniziativa} +0 -- \textbf{Difesa} 11

\textbf{Punti Ferita} 4 (1d8)

\textbf{Movimento} 12 m

\textbf{Tiri Salvezza}: Tempra +1, Riflessi +1, Volontà +0 

\textbf{Lingue} -

\textbf{Sfida} 0 (10 PE)

\emph{\textbf{Carica.}} Se il caprone si muove di almeno 6 metri diretto verso il bersaglio e colpisce con un attacco di rostro durante lo stesso turno, il bersaglio subisce 2 (1d4) danni da botta aggiuntivi. Se il bersaglio è una creatura, deve riuscire un tiro salvezza di Tempra CD 10
o cadere prona.

\emph{\textbf{Piedi Saldi.}} Il caprone ha +1d6 ai tiri salvezza su Tempra e Riflessi effettuati contro effetti che lo farebbero cadere prono.

\textbf{Azioni}

\emph{\textbf{Rostro.} Attacco con Arma da Mischia}: +3 a colpire, portata 1 m, un bersaglio.

\emph{Colpisce:} 3 (1d4 + 1) danni da botta.

\medskip\textbf{Caprone Gigante}\index{Mostri - Caprone Gigante}

\emph{Grande bestia, disallineato}

\textbf{FORZA} +3

\textbf{DESTREZZA} +0

\textbf{COSTITUZIONE} +1

\textbf{INTELLIGENZA} -4

\textbf{SAGGEZZA} +1

\textbf{CARISMA} -2

\textbf{Iniziativa} +0 -- \textbf{Difesa} 12

\textbf{Punti Ferita} 19 (3d10 + 3)

\textbf{Movimento} 12 m

\textbf{Tiri Salvezza}: Tempra +4, Riflessi +1, Volontà +1 

\textbf{Lingue} -

\textbf{Sfida} 1/2 (100 PE)

\emph{\textbf{Carica.}} Se il caprone si muove di almeno 6 metri diretto verso il bersaglio e colpisce con un attacco di rostro durante lo stesso turno, il bersaglio subisce 5 (2d4) danni da botta aggiuntivi. Se il bersaglio è una creatura, deve riuscire un tiro salvezza di Tempra CD 13 o cadere prona.

\emph{\textbf{Piedi Saldi.}} Il caprone ha +1d6 ai tiri salvezza su Tempra e Riflessi effettuati contro effetti che lo farebbero cadere prono.

\textbf{Azioni}

\emph{\textbf{Rostro.} Attacco con Arma da Mischia}: +5 a colpire, portata 1 m, un bersaglio.

\emph{Colpisce:} 8 (2d4 + 3) danni da botta.

\medskip\textbf{Cavallo da Corsa}\index{Mostri - Cavallo da Corsa}

\emph{Grande bestia, disallineato}

\textbf{FORZA} +3

\textbf{DESTREZZA} +0

\textbf{COSTITUZIONE} +1

\textbf{INTELLIGENZA} -4

\textbf{SAGGEZZA} +0

\textbf{CARISMA} -2

\textbf{Iniziativa} +0 -- \textbf{Difesa} 11

\textbf{Punti Ferita} 13 (2d10 + 2)

\textbf{Movimento} 18 m

\textbf{Tiri Salvezza}: Tempra +3, Riflessi +1, Volontà +1 

\textbf{Lingue} -

\textbf{Sfida} 1/4 (50 PE)

\textbf{Azioni}

\emph{\textbf{Zoccoli.} Attacco con Arma da Mischia}: +5 a colpire, portata 1 m, un bersaglio.

\emph{Colpisce:} 8 (2d4 + 3) danni da botta.

\medskip\textbf{Cavallo da Guerra}\index{Mostri - Cavallo da Guerra}

\emph{Grande bestia, disallineato}

\textbf{FORZA} +4

\textbf{DESTREZZA} +1

\textbf{COSTITUZIONE} +1

\textbf{INTELLIGENZA} -4

\textbf{SAGGEZZA} +1

\textbf{CARISMA} -2

\textbf{Iniziativa} +1 -- \textbf{Difesa} 12 (più possibile bardatura)

\textbf{Punti Ferita} 19 (3d10 + 3)

\textbf{Movimento} 18 m

\textbf{Tiri Salvezza}:  Tempra +4, Riflessi +2, Volontà +1 

\textbf{Lingue} -

\textbf{Sfida} 1/2 (100 PE)

\emph{\textbf{Carica Travolgente.}} Se il cavallo si muove di almeno 6 metri diretto verso il bersaglio e lo colpisce con un attacco di zoccoli durante lo stesso turno, il bersaglio deve riuscire un tiro salvezza su Tempra CD 14 o cadere prono. Se il bersaglio è prono, il cavallo può effettuare un altro attacco di zoccoli contro di lui come azione bonus.

\textbf{Azioni}

\emph{\textbf{Zoccoli.} Attacco con Arma da Mischia}: +6 a colpire, portata 1 m, un bersaglio.

\emph{Colpisce:} 11 (2d6 + 4) danni da botta.

\medskip\textbf{Cavallo da Tiro}\index{Mostri - Cavallo da Tiro}

\emph{Grande bestia, disallineato}

\textbf{FORZA} +4

\textbf{DESTREZZA} +0

\textbf{COSTITUZIONE} +1

\textbf{INTELLIGENZA} -4

\textbf{SAGGEZZA} +0

\textbf{CARISMA} -2

\textbf{Iniziativa} +0 -- \textbf{Difesa} 11

\textbf{Punti Ferita} 19 (3d10 + 3)

\textbf{Movimento} 12 m

\textbf{Tiri Salvezza}:  Tempra +5, Riflessi +1, Volontà +2 

\textbf{Lingue} -

\textbf{Sfida} 1/4 (50 PE)

\textbf{Azioni}

\emph{\textbf{Zoccoli.} Attacco con Arma da Mischia}: +6 a colpire, portata 1 m, un bersaglio.

\emph{Colpisce:} 9 (2d4 + 4) danni da botta.

\medskip\textbf{Cavallo Marino Gigante}\index{Mostri - Cavallo Marino Gigante}

Il cavallo marino gigante viene spesso impiegato come cavalcatura dagli umanoidi acquatici.

\emph{Grande bestia, disallineato}

\textbf{FORZA} +1

\textbf{DESTREZZA} +2

\textbf{COSTITUZIONE} +0

\textbf{INTELLIGENZA} -4

\textbf{SAGGEZZA} +1

\textbf{CARISMA} -3

\textbf{Iniziativa} +2 -- \textbf{Difesa} 14

\textbf{Punti Ferita} 16 (3d10)

\textbf{Movimento} 0 m, nuoto 12 m

\textbf{Tiri Salvezza}: Tempra +2, Riflessi +3, Volontà +1

\textbf{Lingue} -

\textbf{Sfida} 1/2 (100 PE)

\emph{\textbf{Carica.}} Se il cavallo marino si muove di almeno 6 metri diretto verso il bersaglio e colpisce con un attacco di rostro durante lo stesso turno, il bersaglio subisce 7 (2d6) danni da botta aggiuntivi. Se il bersaglio è una creatura, deve riuscire un tiro salvezza su Tempra CD 11 o cadere prona.

\emph{\textbf{Respirare Acqua.}} Il cavallo marino può respirare solo sottacqua.

\textbf{Azioni}

\emph{\textbf{Rostro.} Attacco con Arma da Mischia}: +3 a colpire, portata 1 m, un bersaglio.

\emph{Colpisce:} 4 (1d6 + 1) danni da botta.

\medskip\textbf{Centopiedi Gigante}\index{Centopiedi Gigante}

\emph{Piccola bestia, disallineato}

\textbf{FORZA} -3

\textbf{DESTREZZA} +2

\textbf{COSTITUZIONE} +1

\textbf{INTELLIGENZA} -5

\textbf{SAGGEZZA} -2

\textbf{CARISMA} -4

\textbf{Iniziativa} +2 -- \textbf{Difesa} 14

\textbf{Punti Ferita} 4 (1d6 + 1)

\textbf{Movimento} 9 m, scalata 9 m

\textbf{Tiri Salvezza}: Tempra -2, Riflessi +3, Volontà -2 

\textbf{Sensi} vista cieca 9 m

\textbf{Lingue} -

\textbf{Sfida} 1/4 (50 PE)

\textbf{Azioni}

\emph{\textbf{Morso.} Attacco con Arma da Mischia}: +4 a colpire, portata 1 m, una creatura.

\emph{Colpisce:} 4 (1d4 + 2) danni perforanti e il bersaglio deve riuscire un tiro salvezza di Tempra CD 11 o subire 10 (3d6) danni da veleno. Se il danno da veleno riduce il bersaglio a 0 punti ferita, il bersaglio è stabile ma resta avvelenato per 1 ora, anche dopo aver recuperato i punti ferita, e mentre è avvelenato in questo modo resta paralizzato.

\medskip\textbf{Cervo}\index{Mostri - Cervo}

\emph{Media bestia, disallineato}

\textbf{FORZA} +0

\textbf{DESTREZZA} +3

\textbf{COSTITUZIONE} +0

\textbf{INTELLIGENZA} -4

\textbf{SAGGEZZA} +2

\textbf{CARISMA} -3

\textbf{Iniziativa} +3 -- \textbf{Difesa} 14

\textbf{Punti Ferita} 4 (1d8)

\textbf{Movimento} 15 m

\textbf{Tiri Salvezza}: Tempra +2, Riflessi +3, Volontà +2 

\textbf{Lingue} -

\textbf{Sfida} 0 (10 PE)

\textbf{Azioni}

\emph{\textbf{Morso.} Attacco con Arma da Mischia}: +2 a colpire, portata 1 m, un bersaglio.

\emph{Colpisce:} 2 (1d4) danni perforanti.

\medskip\textbf{Cinghiale}\index{Mostri - Cinghiale}

\emph{Media bestia, disallineato}

\textbf{FORZA} +1

\textbf{DESTREZZA} +0

\textbf{COSTITUZIONE} +1

\textbf{INTELLIGENZA} -4

\textbf{SAGGEZZA} -1

\textbf{CARISMA} -3

\textbf{Iniziativa} +0 -- \textbf{Difesa} 12

\textbf{Punti Ferita} 11 (2d8 + 2)

\textbf{Movimento} 12 m

\textbf{Tiri Salvezza}: Tempra +2, Riflessi +1, Volontà -1 

\textbf{Lingue} -

\textbf{Sfida} 1/4 (50 PE)

\emph{\textbf{Carica.}} Se il cinghiale si muove di almeno 6 metri diretto verso il bersaglio e colpisce con un attacco di zanna durante lo stesso turno, il bersaglio subisce 3 (1d6) danni taglienti aggiuntivi. Se il bersaglio è una creatura, deve riuscire un tiro salvezza di Tempra
CD 11 o cadere prono.

\emph{\textbf{Implacabile (Ricarica dopo un 1 ora).}} Se il cinghiale subisce 7 danni o meno che lo ridurrebbero a 0 punti ferita, scende invece a 1 punto ferita.

\textbf{Azioni}

\emph{\textbf{Zanna.} Attacco con Arma da Mischia}: +3 a colpire, portata 1 m, un bersaglio.

\emph{Colpisce:} 4 (1d6 + 1) danni taglienti.

\medskip\textbf{Cinghiale Gigante}\index{Mostri - Cinghiale Gigante}

\emph{Grande bestia, disallineato}

\textbf{FORZA} +3

\textbf{DESTREZZA} +0

\textbf{COSTITUZIONE} +3

\textbf{INTELLIGENZA} -4

\textbf{SAGGEZZA} -2

\textbf{CARISMA} -3

\textbf{Iniziativa} +0 -- \textbf{Difesa} 13

\textbf{Punti Ferita} 42 (5d10 + 15)

\textbf{Movimento} 12 m

\textbf{Tiri Salvezza}: Tempra +4, Riflessi +2, Volontà +0

\textbf{Lingue} -

\textbf{Sfida} 2 (450 PE)

\emph{\textbf{Carica.}} Se il cinghiale si muove di almeno 6 metri diretto verso il bersaglio e colpisce con un attacco di zanna durante lo stesso turno, il bersaglio subisce 7 (2d6) danni taglienti aggiuntivi. Se il bersaglio è una creatura, deve riuscire un tiro salvezza di Tempra CD 13 o cadere prono.

\emph{\textbf{Implacabile (Ricarica dopo un 1 ora).}} Se il cinghiale subisce 10 danni o meno che lo ridurrebbero a 0 punti ferita, scende invece a 1 punto ferita.

\textbf{Azioni}

\emph{\textbf{Zanna.} Attacco con Arma da Mischia}: +5 a colpire, portata 1 m, un bersaglio.

\emph{Colpisce:} 10 (2d6 + 3) danni taglienti.

\medskip\textbf{Coccodrillo}\index{Mostri - Coccodrillo}

\emph{Grande bestia, disallineato}

\textbf{FORZA} +2

\textbf{DESTREZZA} +0

\textbf{COSTITUZIONE} +1

\textbf{INTELLIGENZA} -4

\textbf{SAGGEZZA} +0

\textbf{CARISMA} -3

\textbf{Iniziativa} +0 -- \textbf{Difesa} 13

\textbf{Punti Ferita} 19 (3d10 + 3)

\textbf{Movimento} 6 m, nuoto 9 m

\textbf{Tiri Salvezza}: Tempra +6, Riflessi +4, Volontà +2 

\textbf{Competenze} Muoversi Silenziosamente / Nascondersi nelle Ombre +2

\textbf{Lingue} -

\textbf{Sfida} 1/2 (100 PE)

\emph{\textbf{Trattenere il Fiato.}} Il coccodrillo può trattenere il fiato per 15 minuti.

\textbf{Azioni}

\emph{\textbf{Morso.} Attacco con Arma da Mischia}: +4 a colpire, portata 1 m, una creatura.

\emph{Colpisce:} 7 (1d10 + 2) danni perforanti, e il bersaglio è afferrato (CD 12 per fuggire). Fino al termine dell'afferrare, il bersaglio è intralciato, e il coccodrillo non può usare il morso contro un altro bersaglio.

\medskip\textbf{Coccodrillo Gigante}\index{Mostri - Coccodrillo Gigante}

\emph{Enorme bestia, disallineato}

\textbf{FORZA} +5

\textbf{DESTREZZA} -1

\textbf{COSTITUZIONE} +3

\textbf{INTELLIGENZA} -4

\textbf{SAGGEZZA} +0

\textbf{CARISMA} -2

\textbf{Iniziativa} -1 -- \textbf{Difesa} 15

\textbf{Punti Ferita} 85 (9d12 + 27)

\textbf{Movimento} 9 m, nuoto 15 m

\textbf{Tiri Salvezza}: Tempra +15, Riflessi +8, Volontà +8

\textbf{Competenze} Muoversi Silenziosamente / Nascondersi nelle Ombre +5

\textbf{Lingue} -

\textbf{Sfida} 5 (1.800 PE)

\emph{\textbf{Trattenere il Fiato.}} Il coccodrillo può trattenere il fiato per 30 minuti.

\textbf{Azioni}

\emph{\textbf{Multiattacco.}} Il coccodrillo effettua due attacchi: uno con il morso e uno con la coda.

\emph{\textbf{Coda.} Attacco con Arma da Mischia}: +8 a colpire, portata 3 m, un bersaglio non afferrato dal coccodrillo.

\emph{Colpisce:} 14 (2d8 + 5) danni da botta. Se il bersaglio è una creatura, deve riuscire un tiro salvezza di Tempra CD 16 o cadere prono.

\emph{\textbf{Morso.} Attacco con Arma da Mischia}: +8 a colpire, portata 1 m, un bersaglio.

\emph{Colpisce:} 21 (3d10 + 5) danni perforanti, e il bersaglio è afferrato (CD 16 per fuggire). Fino al termine dell'afferrare, il bersaglio è intralciato, e il coccodrillo non può usare il morso contro un altro bersaglio.

\medskip\textbf{Corvo}\index{Mostri - Corvo}

\emph{Minuscola bestia, disallineato}

\textbf{FORZA} -4

\textbf{DESTREZZA} +2

\textbf{COSTITUZIONE} -1

\textbf{INTELLIGENZA} -4

\textbf{SAGGEZZA} +1

\textbf{CARISMA} -2

\textbf{Iniziativa} +2 -- \textbf{Difesa} 13

\textbf{Punti Ferita} 1 (1d4 - 1)

\textbf{Movimento} 3 m, volo 15 m

\textbf{Tiri Salvezza}: Tempra +1, Riflessi +4, Volontà +2

\textbf{Competenze} Consapevolezza +3

\textbf{Lingue} -

\textbf{Sfida} 0 (10 PE)

\emph{\textbf{Imitazione.}} Il corvo può imitare dei semplici suoni che ha udito, come il sussurro di una persona, il pianto di un bambino o il verso di un animale. Una creatura che ode il suono può identificarlo come imitazione riuscendo una prova di Saggezza (Sopravvivenza) CD 10.

\textbf{Azioni}

\emph{\textbf{Becco.} Attacco con Arma da Mischia}: +4 a colpire, portata 1 m, un bersaglio.

\emph{Colpisce:} 1 danno perforante.

\medskip\textbf{Donnola}\index{Mostri - Donnola}

\emph{Minuscola bestia, disallineato}

\textbf{FORZA} -4

\textbf{DESTREZZA} +3

\textbf{COSTITUZIONE} -1

\textbf{INTELLIGENZA} -4

\textbf{SAGGEZZA} +1

\textbf{CARISMA} -4

\textbf{Iniziativa} +3 -- \textbf{Difesa} 14

\textbf{Punti Ferita} 1 (1d4 - 1)

\textbf{Movimento} 9 m

\textbf{Tiri Salvezza}: Tempra +2, Riflessi +4, Volontà +1

\textbf{Competenze} Muoversi Silenziosamente / Nascondersi nelle Ombre +5, Consapevolezza +3

\textbf{Lingue} -

\textbf{Sfida} 0 (10 PE)

\emph{\textbf{Udito e Olfatto Affinati.}} La donnola ha +1d6 nelle prove di Saggezza (Consapevolezza) basate su udito o olfatto.

\textbf{Azioni}

\emph{\textbf{Morso.} Attacco con Arma da Mischia}: +5 a colpire, portata 1 m, un bersaglio.

\emph{Colpisce:} 1 danno perforante.

\medskip\textbf{Donnola Gigante}\index{Mostri - Donnola Gigante}

\emph{Media bestia, disallineato}

\textbf{FORZA} +0

\textbf{DESTREZZA} +3

\textbf{COSTITUZIONE} +0

\textbf{INTELLIGENZA} -3

\textbf{SAGGEZZA} +1

\textbf{CARISMA} -3

\textbf{Iniziativa} +3 -- \textbf{Difesa} 14

\textbf{Punti Ferita} 9 (2d8)

\textbf{Movimento} 12 m

\textbf{Tiri Salvezza}:  Tempra +6, Riflessi +7, Volontà +2 

\textbf{Competenze} Muoversi Silenziosamente / Nascondersi nelle Ombre +5, Consapevolezza +3

\textbf{Sensi} visione al buio 18 m

\textbf{Lingue} -

\textbf{Sfida} 1/8 (25 PE)

\emph{\textbf{Udito e Olfatto Affinati.}} La donnola ha +1d6 nelle prove di Saggezza (Consapevolezza) basate su udito o olfatto.

\textbf{Azioni}

\emph{\textbf{Morso.} Attacco con Arma da Mischia}: +5 a colpire, portata 1 m, un bersaglio.

\emph{Colpisce:} 5 (1d4 + 3) danni perforanti.

\medskip\textbf{Elefante}\index{Mostri - Elefante}

\emph{Enorme bestia, disallineato}

\textbf{FORZA} +6

\textbf{DESTREZZA} -1

\textbf{COSTITUZIONE} +3

\textbf{INTELLIGENZA} -4

\textbf{SAGGEZZA} +0

\textbf{CARISMA} -2

\textbf{Iniziativa} -1 -- \textbf{Difesa} 14

\textbf{Punti Ferita} 76 (8d12 + 24)

\textbf{Movimento} 12 m

\textbf{Tiri Salvezza}: Tempra +13, Riflessi +7, Volontà +6 

\textbf{Lingue} -

\textbf{Sfida} 4 (1.000 PE)

\emph{\textbf{Carica Travolgente.}} Se l'elefante si muove di almeno 6 metri diretto verso una creatura e la colpisce con un attacco di incornata durante lo stesso turno, il bersaglio deve riuscire un tiro salvezza su Tempra CD 12 o cadere prono. Se il bersaglio è prono, l'elefante può effettuare un attacco di pestone contro di lui come azione bonus.

\textbf{Azioni}

\emph{\textbf{Incornata.} Attacco con Arma da Mischia}: +8 a colpire, portata 1 m, un bersaglio.

\emph{Colpisce:} 19 (3d8 + 6) danni perforanti. 

\emph{\textbf{Pestone.} Attacco con Arma da Mischia}: +8 a colpire, portata 1 m, un bersaglio prono.

\emph{Colpisce:} 22 (3d10 + 6) danni da botta.

\medskip\textbf{Falco}\index{Mostri - Falco}

\emph{Minuscola bestia, disallineato}

\textbf{FORZA} -3

\textbf{DESTREZZA} +3

\textbf{COSTITUZIONE} -1

\textbf{INTELLIGENZA} -4

\textbf{SAGGEZZA} +2

\textbf{CARISMA} -2

\textbf{Iniziativa} +3 -- \textbf{Difesa} 14

\textbf{Punti Ferita} 1 (1d4 - 1)

\textbf{Movimento} 3 m, volo 18 m

\textbf{Tiri Salvezza}: Tempra +2, Riflessi +5, Volontà +2 

\textbf{Competenze} Consapevolezza +4

\textbf{Lingue} -

\textbf{Sfida} 0 (10 PE)

\emph{\textbf{Vista Affinata.}} Il falco ha +1d6 alle prove di Saggezza (Consapevolezza) basate sulla vista.

\textbf{Azioni}

\emph{\textbf{Speroni.} Attacco con Arma da Mischia}: +5 a colpire, portata 1 m, un bersaglio.

\emph{Colpisce:} 1 danno tagliente.

\medskip\textbf{Falco di Sangue}\index{Mostri - Falco di Sangue}

Dovendo il suo nome alle sue piume cremisi e alla sua natura aggressiva, il falco di sangue attacca senza timore usando il suo becco appuntito.

\emph{Piccola bestia, disallineato}

\textbf{FORZA} -2

\textbf{DESTREZZA} +2

\textbf{COSTITUZIONE} +0

\textbf{INTELLIGENZA} -4

\textbf{SAGGEZZA} +2

\textbf{CARISMA} -3

\textbf{Iniziativa} +2 -- \textbf{Difesa} 13

\textbf{Punti Ferita} 7 (2d6)

\textbf{Movimento} 3 m, volo 18 m

\textbf{Tiri Salvezza}: Tempra +3, Riflessi +6, Volontà +3 

\textbf{Competenze} Consapevolezza +4

\textbf{Lingue} -

\textbf{Sfida} 1/8 (25 PE)

\emph{\textbf{Tattiche di Branco.}} Il falco ha +1d6 ai tiri di attacco contro una creatura se almeno uno degli alleati del falco si trova entro 1,5 metri dalla creatura e quell'alleato non è inabile.

\emph{\textbf{Vista Affinata.}} Il falco ha +1d6 alle prove di Saggezza (Consapevolezza) basate sulla vista.

\textbf{Azioni}

\emph{\textbf{Becco.} Attacco con Arma da Mischia}: +4 a colpire, portata 1 m, un bersaglio.

\emph{Colpisce:} 4 (1d4 + 2) danni perforanti.

\medskip\textbf{Pirana}\index{Mostri - Pirana}

Il pirana è un pesce carnivoro dai denti affilati.

\emph{Minuscola bestia, disallineato}

\textbf{FORZA} -4

\textbf{DESTREZZA} +3

\textbf{COSTITUZIONE} -1

\textbf{INTELLIGENZA} -5

\textbf{SAGGEZZA} -2

\textbf{CARISMA} -4

\textbf{Iniziativa} +3 -- \textbf{Difesa} 14

\textbf{Punti Ferita} 1 (1d4 - 1)

\textbf{Movimento} 0 m, nuoto 12 m

\textbf{Tiri Salvezza}: Tempra -4, Riflessi +3, Volontà -2 

\textbf{Sensi} visione al buio 18 m

\textbf{Lingue} -

\textbf{Sfida} 0 (10 PE)

\emph{\textbf{Frenesia Sanguinaria.}} Il pirana ha +1d6 ai tiri di attacco in mischia contro qualsiasi creatura che non sia al massimo dei punti ferita.

\emph{\textbf{Respirare Acqua.}} Il pirana può respirare solo sottacqua. 

\textbf{Azioni}

\emph{\textbf{Morso.} Attacco con Arma da Mischia}: +5 a colpire, portata 1 m, un bersaglio.

\emph{Colpisce:} 1 danno perforante.

\medskip\textbf{Gatto}\index{Mostri - Gatto}

\emph{Minuscola bestia, disallineato}

\textbf{FORZA} -4

\textbf{DESTREZZA} +2

\textbf{COSTITUZIONE} +0

\textbf{INTELLIGENZA} -4

\textbf{SAGGEZZA} +1

\textbf{CARISMA} -2

\textbf{Iniziativa} +2 -- \textbf{Difesa} 13

\textbf{Punti Ferita} 2 (1d4)

\textbf{Movimento} 12 m, scalata 9 m

\textbf{Tiri Salvezza}:  Tempra +1, Riflessi +4, Volontà +1

\textbf{Competenze} Muoversi Silenziosamente / Nascondersi nelle Ombre +4, Consapevolezza +3

\textbf{Lingue} -

\textbf{Sfida} 0 (10 PE)

\emph{\textbf{Olfatto Affinato.}} Il gatto ha +1d6 alle prove di Saggezza (Consapevolezza) basate sull'olfatto.

\textbf{Azioni}

\emph{\textbf{Artigli.} Attacco con Arma da Mischia}: +0 a colpire, portata 1 m, un bersaglio.

\emph{Colpisce:} 1 danno tagliente.

\medskip\textbf{Granchio Gigante}\index{Mostri - Granchio Gigante}

\emph{Media bestia, disallineato}

\textbf{FORZA} +1

\textbf{DESTREZZA} +2

\textbf{COSTITUZIONE} +0

\textbf{INTELLIGENZA} -5

\textbf{SAGGEZZA} -1

\textbf{CARISMA} -4

\textbf{Iniziativa} +2 -- \textbf{Difesa} 16

\textbf{Punti Ferita} 13 (3d8)

\textbf{Movimento} 9 m, nuoto 9 m

\textbf{Tiri Salvezza}: Tempra +5, Riflessi +2, Volontà +1

\textbf{Competenze} Muoversi Silenziosamente / Nascondersi nelle Ombre +4

\textbf{Sensi} vista cieca 9 m

\textbf{Lingue} -

\textbf{Sfida} 1/8 (25 PE)

\emph{\textbf{Anfibio.}} Il granchio può respirare aria e acqua.

\textbf{Azioni}

\emph{\textbf{Artiglio (Chela).} Attacco con Arma da Mischia}: +3 a colpire, portata 1 m, un bersaglio.

\emph{Colpisce:} 4 (1d6 + 1) danni da botta e il bersaglio è afferrato (CD 11 per fuggire). Il granchio ha due chele, ciascuna delle quali può afferrare un solo bersaglio.

\medskip\textbf{Gufo}\index{Mostri - Gufo}

\emph{Minuscola bestia, disallineato}

\textbf{FORZA} -4

\textbf{DESTREZZA} +1

\textbf{COSTITUZIONE} -1

\textbf{INTELLIGENZA} -4

\textbf{SAGGEZZA} +1

\textbf{CARISMA} -2

\textbf{Iniziativa} +1 -- \textbf{Difesa} 12

\textbf{Punti Ferita} 1 (1d4 - 1)

\textbf{Movimento} 1,5 m, volo 18 m

\textbf{Tiri Salvezza}: Tempra +2, Riflessi +5, Volontà +2 

\textbf{Competenze} Muoversi Silenziosamente / Nascondersi nelle Ombre +3, Consapevolezza +3

\textbf{Sensi} visione al buio 36 m

\textbf{Lingue} -

\textbf{Sfida} 0 (10 PE)

\emph{\textbf{Sorvolare.}} Il gufo non provoca attacchi di opportunità quando vola via dalla portata di un nemico.

\emph{\textbf{Udito e Vista Affinati.}} Il gufo ha +1d6 nelle prove di Saggezza (Consapevolezza) basate su udito o vista.

\textbf{Azioni}

\emph{\textbf{Speroni.} Attacco con Arma da Mischia}: +3 a colpire, portata 1 m, un bersaglio.

\emph{Colpisce:} 1 danno tagliente.

\medskip\textbf{Gufo Gigante}\index{Mostri - Gufo Gigante}

I gufi giganti sono creature intelligenti che proteggono i regni silvani.

\emph{Grande bestia, neutrale}

\textbf{FORZA} +1

\textbf{DESTREZZA} +2

\textbf{COSTITUZIONE} +1

\textbf{INTELLIGENZA} -1

\textbf{SAGGEZZA} +1

\textbf{CARISMA} +0

\textbf{Iniziativa} +2 -- \textbf{Difesa} 13

\textbf{Punti Ferita} 19 (3d10 + 3)

\textbf{Movimento} 1,5 m, volo 18 m

\textbf{Tiri Salvezza}: Tempra +1, Riflessi +4, Volontà +1 

\textbf{Competenze} Muoversi Silenziosamente / Nascondersi nelle Ombre +4, Consapevolezza +5

\textbf{Sensi} visione al buio 36 m

\textbf{Lingue} Gufo Gigante, comprende Comune, Elfico e Silvano ma non può parlarli

\textbf{Sfida} 1/4 (50 PE)

\emph{\textbf{Sorvolare.}} Il gufo non provoca attacchi di opportunità quando vola via dalla portata di un nemico.

\emph{\textbf{Udito e Vista Affinati.}} Il gufo ha +1d6 nelle prove di Saggezza (Consapevolezza) basate su udito o vista.

\textbf{Azioni}

\emph{\textbf{Speroni.} Attacco con Arma da Mischia}: +3 a colpire, portata 1 m, un bersaglio.

\emph{Colpisce:} 8 (2d6 + 1) danni perforanti.

\medskip\textbf{Iena}\index{Mostri - Iena}

\emph{Media bestia, disallineato}

\textbf{FORZA} +0

\textbf{DESTREZZA} +1

\textbf{COSTITUZIONE} +1

\textbf{INTELLIGENZA} -4

\textbf{SAGGEZZA} +1

\textbf{CARISMA} -3

\textbf{Iniziativa} +1 -- \textbf{Difesa} 12

\textbf{Punti Ferita} 5 (1d8 + 1)

\textbf{Movimento} 15 m

\textbf{Tiri Salvezza}: Tempra +5, Riflessi +5, Volontà +1

\textbf{Competenze} Consapevolezza +3

\textbf{Lingue} -

\textbf{Sfida} 0 (10 PE)

\emph{\textbf{Tattiche di Branco.}} La iena ha +1d6 ai tiri di attacco contro una creatura se almeno uno degli alleati della iena si trova entro 1,5 metri dalla creatura e quell'alleato non è inabile.

\textbf{Azioni}

\emph{\textbf{Morso.} Attacco con Arma da Mischia}: +2 a colpire, portata 1 m, un bersaglio.

\emph{Colpisce:} 3 (1d6) danni perforanti.

\medskip\textbf{Iena Gigante}\index{Mostri - Iena Gigante}

\emph{Grande bestia, disallineato}

\textbf{FORZA} +3

\textbf{DESTREZZA} +2

\textbf{COSTITUZIONE} +2

\textbf{INTELLIGENZA} -4

\textbf{SAGGEZZA} +1

\textbf{CARISMA} -2

\textbf{Iniziativa} +2 -- \textbf{Difesa} 13

\textbf{Punti Ferita} 45 (6d10 + 12)

\textbf{Movimento} 15 m

\textbf{Tiri Salvezza}: Tempra +6, Riflessi +6, Volontà +2 

\textbf{Competenze} Consapevolezza +3

\textbf{Lingue} -

\textbf{Sfida} 1 (200 PE)

\emph{\textbf{Rabbia.}} Quando la iena riduce una creatura a 0 punti ferita con un attacco di mischia durante il proprio turno, la iena può svolgere un'azione bonus per muoversi fino a metà del suo movimento effettuare un attacco di morso.

\textbf{Azioni}

\emph{\textbf{Morso.} Attacco con Arma da Mischia}: +5 a colpire, portata 1 m, un bersaglio.

\emph{Colpisce:} 10 (2d6 + 3) danni perforanti.

\medskip\textbf{Leone}\index{Mostri - Leone}

\emph{Grande bestia, disallineato}

\textbf{FORZA} +3

\textbf{DESTREZZA} +2

\textbf{COSTITUZIONE} +1

\textbf{INTELLIGENZA} -4

\textbf{SAGGEZZA} +1

\textbf{CARISMA} -1

\textbf{Iniziativa} +2 -- \textbf{Difesa} 13

\textbf{Punti Ferita} 26 (4d10 + 4)

\textbf{Movimento} 15 m

\textbf{Tiri Salvezza}: Tempra +6, Riflessi +7, Volontà +2 

\textbf{Competenze} Muoversi Silenziosamente / Nascondersi nelle Ombre +6, Consapevolezza +3

\textbf{Lingue} -

\textbf{Sfida} 1 (200 PE)

\emph{\textbf{Balzo.}} Se il leone si muove di almeno 6 metri diretto verso una creatura e la colpisce con un attacco di artiglio durante lo stesso turno, il bersaglio deve riuscire un tiro salvezza di Tempra CD 13 o cadere prono. Se il bersaglio è prono, il leone può effettuare un
attacco di morso come azione bonus.

\emph{\textbf{Olfatto Affinato.}} Il leone ha +1d6 alle prove di Saggezza (Consapevolezza) basate sull'olfatto.

\emph{\textbf{Salto con Rincorsa.}} Con 3 metri di rincorsa, il leone può saltare in lungo fino a 7,5 metri.

\emph{\textbf{Tattiche di Branco.}} Il leone ha +1d6 ai tiri di attacco contro una creatura se almeno uno degli alleati del leone si trova entro 1,5 metri dalla creatura e quell'alleato non è inabile.

\textbf{Azioni}

\emph{\textbf{Artiglio.} Attacco con Arma da Mischia}: +5 a colpire, portata 1 m, un bersaglio.

\emph{Colpisce:} 6 (1d6 + 3) danni taglienti. 

\emph{\textbf{Morso.} Attacco con Arma da Mischia}: +5 a colpire, portata 1 m, un bersaglio.

\emph{Colpisce:} 7 (1d8 + 3) danni perforanti.

\medskip\textbf{Lucertola}\index{Mostri - Lucertola}

\emph{Minuscola bestia, disallineato}

\textbf{FORZA} -4

\textbf{DESTREZZA} +0

\textbf{COSTITUZIONE} +0

\textbf{INTELLIGENZA} -5

\textbf{SAGGEZZA} -1

\textbf{CARISMA} -4

\textbf{Iniziativa} +0 -- \textbf{Difesa} 11

\textbf{Punti Ferita} 2 (1d4)

\textbf{Movimento} 6 m, scalata 6 m

\textbf{Tiri Salvezza}:  Tempra +1, Riflessi +4, Volontà +1 

\textbf{Sensi} visione al buio 9 m

\textbf{Lingue} -

\textbf{Sfida} 0 (10 PE)

\emph{\textbf{Scalare come Ragno.}} La lucertola può scalare superfici difficili, compreso lo stare a testa in giù sul soffitto, senza bisogno di effettuare una prova di abilità.

\textbf{Azioni}

\emph{\textbf{Morso.} Attacco con Arma da Mischia}: +0 a colpire, portata 1 m, un bersaglio.

\emph{Colpisce:} 1 danno perforante.

\medskip\textbf{Lucertola Gigante}\index{Mostri - Lucertola Gigante}

Le lucertole giganti sono temibili predatori e spesso vengono impiegate come cavalcature o animali da tiro da umanoidi rettiloidi e residenti del sottosuolo.

\emph{Grande bestia, disallineato}

\textbf{FORZA} +2

\textbf{DESTREZZA} +1

\textbf{COSTITUZIONE} +1

\textbf{INTELLIGENZA} -4

\textbf{SAGGEZZA} +0

\textbf{CARISMA} -3

\textbf{Iniziativa} +1 -- \textbf{Difesa} 13

\textbf{Punti Ferita} 19 (3d10 + 3)

\textbf{Movimento} 9 m, scalata 9 m

\textbf{Tiri Salvezza}: Tempra +11, Riflessi +8, Volontà +4 

\textbf{Sensi} visione al buio 9 m

\textbf{Lingue} -

\textbf{Sfida} 1/4 (50 PE)

\textbf{Azioni}

\emph{\textbf{Morso.} Attacco con Arma da Mischia}: +4 a colpire, portata 1 m, un bersaglio.

\emph{Colpisce:} 6 (1d8 + 2) danni perforanti.

\textbf{VARIANTE}

Alcune lucertole giganti possiedono uno o entrambi i seguenti tratti.

\emph{\textbf{Scalare come Ragno.}} La lucertola può scalare superfici difficili, compreso lo stare a testa in giù sul soffitto, senza bisogno di effettuare una prova di abilità. 

\emph{\textbf{Trattenere il Fiato.}} La lucertola può trattenere il fiato per 15 minuti. (Una lucertola con questo tratto possiede anche velocità di nuoto 9 metri).

\medskip\textbf{Lupo}\index{Mostri - Lupo}

\emph{Media bestia, disallineato}

\textbf{FORZA} +1

\textbf{DESTREZZA} +2

\textbf{COSTITUZIONE} +1

\textbf{INTELLIGENZA} -4

\textbf{SAGGEZZA} +1

\textbf{CARISMA} -2

\textbf{Iniziativa} +2 -- \textbf{Difesa} 14

\textbf{Punti Ferita} 11 (2d8 + 2)

\textbf{Movimento} 12 m

\textbf{Tiri Salvezza}: Tempra +5, Riflessi +5, Volontà +1 

\textbf{Competenze} Muoversi Silenziosamente / Nascondersi nelle Ombre +4, Consapevolezza +3

\textbf{Lingue} -

\textbf{Sfida} 1/4 (50 PE)

\emph{\textbf{Udito e Olfatto Affinato.}} Il lupo ha +1d6 nelle prove di Saggezza (Consapevolezza) basate su udito o olfatto.

\emph{\textbf{Tattiche di Branco.}} Il lupo ha +1d6 ai tiri di attacco contro una creatura se almeno uno degli alleati del lupo si trova entro 1,5 metri dalla creatura e quell'alleato non è inabile.

\textbf{Azioni}

\emph{\textbf{Morso.} Attacco con Arma da Mischia}: +4 a colpire, portata 1 m, un bersaglio.

\emph{Colpisce:} 7 (2d4 + 2) danni perforanti. Se il bersaglio è una creatura, deve riuscire un tiro salvezza di Tempra CD 11 o cadere prona.

\medskip\textbf{Dinolupo (Metalupo)}\index{Mostri - Dinolupo (Metalupo}

\emph{Grande bestia, disallineato}

\textbf{FORZA} +3

\textbf{DESTREZZA} +2

\textbf{COSTITUZIONE} +2

\textbf{INTELLIGENZA} -2

\textbf{SAGGEZZA} +1

\textbf{CARISMA} -2

\textbf{Iniziativa} +2 -- \textbf{Difesa} 15

\textbf{Punti Ferita} 37 (5d10 + 10)

\textbf{Movimento} 15 m

\textbf{Tiri Salvezza}: Tempra +7, Riflessi +6, Volontà +2 

\textbf{Competenze} Muoversi Silenziosamente / Nascondersi nelle Ombre +4, Consapevolezza +3

\textbf{Lingue} -

\textbf{Sfida} 1 (200 PE)

\emph{\textbf{Udito e Olfatto Affinato.}} Il lupo ha +1d6 nelle prove di Saggezza (Consapevolezza) basate su udito o olfatto.

\emph{\textbf{Tattiche di Branco.}} Il lupo ha +1d6 ai tiri di attacco contro una creatura se almeno uno degli alleati del lupo si trova entro 1,5 metri dalla creatura e quell'alleato non è inabile.

\textbf{Azioni}

\emph{\textbf{Morso.} Attacco con Arma da Mischia}: +5 a colpire, portata 1 m, un bersaglio.

\emph{Colpisce:} 10 (2d6 + 3) danni perforanti. Se il bersaglio è una creatura, deve riuscire un tiro salvezza di Tempra CD 13 o cadere prona.

\medskip\textbf{Lupo Invernale}\index{Mostri - Lupo Invernale}

I lupi invernali abitano nelle regioni artiche e sono creature malvagie e intelligenti dal manto bianco come la neve e gli occhi color del ghiaccio.

\emph{Grande mostruosità, neutrale malvagio}

\textbf{FORZA} +4

\textbf{DESTREZZA} +1

\textbf{COSTITUZIONE} +2

\textbf{INTELLIGENZA} -2

\textbf{SAGGEZZA} +1

\textbf{CARISMA} -1

\textbf{Iniziativa} +1 -- \textbf{Difesa} 15

\textbf{Punti Ferita} 75 (10d10 + 20)

\textbf{Movimento} 15 m

\textbf{Tiri Salvezza}: Tempra +9, Riflessi +6, Volontà +3 

\textbf{Competenze} Muoversi Silenziosamente / Nascondersi nelle Ombre +3, Consapevolezza +5

\textbf{Immunità al Danno} freddo

\textbf{Lingue} Comune, Gigante, Lupo Invernale

\textbf{Sfida} 3 (700 PE)

\emph{\textbf{Camuffamento di Neve.}} Il lupo ha +1d6 alle prove di Destrezza (Nascondersi nelle ombre) effettuate per nascondersi su terreno innevato.

\emph{\textbf{Udito e Olfatto Affinato.}} Il lupo ha +1d6 nelle prove di Saggezza (Consapevolezza) basate su udito o olfatto.

\emph{\textbf{Tattiche di Branco.}} Il lupo ha +1d6 ai tiri di attacco contro una creatura se almeno uno degli alleati del lupo si trova entro 1,5 metri dalla creatura e quell'alleato non è inabile.

\textbf{Azioni}

\emph{\textbf{Morso.} Attacco con Arma da Mischia}: +6 a colpire, portata 1 m, un bersaglio.

\emph{Colpisce:} 11 (2d6 + 4) danni perforanti. Se il bersaglio è una creatura, deve riuscire un tiro salvezza di Tempra CD 14 o cadere prona.

\emph{\textbf{Soffio Gelido (Ricarica 5-6).}} Il lupo esala un'esplosione di vento gelido in un cono di 4,5 metri. Ogni creatura in quell'area deve effettuare un tiro salvezza di Riflessi CD 12, e subire 18 (4d8) danni da freddo se fallisce il tiro salvezza, o la metà di questi danni se lo riesce.

\medskip\textbf{Mammut}\index{Mostri - Mammut}

Il mammut è una creatura simile all'elefante dalla folta pelliccia e lunghe zanne.

\emph{Enorme bestia, disallineato}

\textbf{FORZA} +7

\textbf{DESTREZZA} -1

\textbf{COSTITUZIONE} +5

\textbf{INTELLIGENZA} -4

\textbf{SAGGEZZA} +0

\textbf{CARISMA} -2

\textbf{Iniziativa} -1 -- \textbf{Difesa} 16

\textbf{Punti Ferita} 126 (11d12 + 55)

\textbf{Movimento} 12 m

\textbf{Tiri Salvezza}: Tempra +14, Riflessi +10, Volontà +7 

\textbf{Lingue} -

\textbf{Sfida} 6 (2.300 PE)

\emph{\textbf{Carica Travolgente.}} Se il mammut si muove di almeno 6 metri diretto verso una creatura e la colpisce con un attacco di incornata durante lo stesso turno, il bersaglio deve riuscire un tiro salvezza su Tempra CD 18 o cadere prono. Se il bersaglio è prono, il mammut può effettuare un attacco di pestone contro di lui come azione bonus.

\textbf{Azioni}

\emph{\textbf{Incornata.} Attacco con Arma da Mischia}: +10 a colpire, portata 3 m, un bersaglio.

\emph{Colpisce:} 25 (4d8 + 7) danni perforanti. 

\emph{\textbf{Pestone.} Attacco con Arma da Mischia}: +10 a colpire, portata 1 m, una creatura prona.

\emph{Colpisce:} 29 (4d10 + 7) danni da botta.

\medskip\textbf{Mastino}\index{Mostri - Mastino}

\textbf{I} mastini sono impressionanti segugi apprezzati dagli umanoidi per la loro realtà e sensi affinati.

\emph{Media bestia, disallineato}

\textbf{FORZA} +1

\textbf{DESTREZZA} +2

\textbf{COSTITUZIONE} +1

\textbf{INTELLIGENZA} -4

\textbf{SAGGEZZA} +1

\textbf{CARISMA} -2

\textbf{Iniziativa} +2 -- \textbf{Difesa} 13

\textbf{Punti Ferita} 5 (1d8 + 1)

\textbf{Movimento} 12 m

\textbf{Tiri Salvezza}: Tempra +3, Riflessi +3, Volontà +1 

\textbf{Competenze} Consapevolezza +3, Sopravvivenza (Seguire Tracce) +3

\textbf{Lingue} -

\textbf{Sfida} 1/8 (25 PE)

\emph{\textbf{Udito e Olfatto Affinato.}} Il mastino ha +1d6 nelle prove di Saggezza (Consapevolezza) basate su udito o olfatto.

\textbf{Azioni}

\emph{\textbf{Morso.} Attacco con Arma da Mischia}: +3 a colpire, portata 1 m, un bersaglio.

\emph{Colpisce:} 4 (1d6 + 1) danni perforanti. Se il bersaglio è una creatura, deve riuscire un tiro salvezza di Tempra CD 11 o cadere prono.

\medskip\textbf{Mulo}\index{Mostri - Mulo}

\emph{Media bestia, disallineato}

\textbf{FORZA} +2

\textbf{DESTREZZA} +0

\textbf{COSTITUZIONE} +1

\textbf{INTELLIGENZA} -4

\textbf{SAGGEZZA} +0

\textbf{CARISMA} -3

\textbf{Iniziativa} +0 -- \textbf{Difesa} 11

\textbf{Punti Ferita} 11 (2d8 + 2)

\textbf{Movimento} 12 m

\textbf{Tiri Salvezza}: Tempra +3, Riflessi +1, Volontà +1 

\textbf{Lingue} -

\textbf{Sfida} 1/8 (25 PE)

\emph{\textbf{Bestia da Soma.}} Il mulo è considerato un animale Grande al fine di determinare la sua capacità di carico.

\emph{\textbf{Piedi Saldi.}} Il mulo ha +1d6 ai tiri salvezza su Tempra e Riflessi effettuati contro effetti che lo farebbero cadere prono.

\textbf{Azioni}

\emph{\textbf{Zoccoli.} Attacco con Arma da Mischia}: +2 a colpire, portata 1 m, un bersaglio.

\emph{Colpisce:} 4 (1d4 + 2) danni da botta.

\medskip\textbf{Orso Bruno}\index{Mostri - Orso Bruno}

\emph{Grande bestia, disallineato}

\textbf{FORZA} +4

\textbf{DESTREZZA} +0

\textbf{COSTITUZIONE} +3

\textbf{INTELLIGENZA} -4

\textbf{SAGGEZZA} +1

\textbf{CARISMA} -2

\textbf{Iniziativa} +0 -- \textbf{Difesa} 12

\textbf{Punti Ferita} 34 (4d10 + 12)

\textbf{Movimento} 12 m, scalata 9 m

\textbf{Tiri Salvezza}: Tempra +6, Riflessi +2, Volontà +3 

\textbf{Competenze} Consapevolezza +3

\textbf{Lingue} -

\textbf{Sfida} 1 (200 PE)

\emph{\textbf{Olfatto Affinato.}} L'orso ha +1d6 alle prove di Saggezza (Consapevolezza) basate sull'olfatto.

\textbf{Azioni}

\emph{\textbf{Multiattacco.}} L'orso effettua due attacchi: uno con il morso e uno con gli artigli.

\emph{\textbf{Artigli.} Attacco con Arma da Mischia}: +5 a colpire, portata 1 m, un bersaglio.

\emph{Colpisce:} 11 (2d6 + 4) danni taglienti.

\emph{\textbf{Morso.} Attacco con Arma da Mischia}: +5 a colpire, portata 1 m, un bersaglio.

\emph{Colpisce:} 8 (1d8 + 4) danni perforanti.

\medskip\textbf{Orso Nero}\index{Mostri - Orso Nero}

\emph{Media bestia, disallineato}

\textbf{FORZA} +2

\textbf{DESTREZZA} +0

\textbf{COSTITUZIONE} +2

\textbf{INTELLIGENZA} -4

\textbf{SAGGEZZA} +1

\textbf{CARISMA} -2

\textbf{Iniziativa} +0 -- \textbf{Difesa} 12

\textbf{Punti Ferita} 19 (3d8 + 6)

\textbf{Movimento} 12 m, scalata 9 m

\textbf{Tiri Salvezza}: Tempra +4, Riflessi +1, Volontà +1 

\textbf{Competenze} Consapevolezza +3

\textbf{Lingue} -

\textbf{Sfida} 1/2 (100 PE)

\emph{\textbf{Olfatto Affinato.}} L'orso ha +1d6 alle prove di Saggezza (Consapevolezza) basate sull'olfatto.

\textbf{Azioni}

\emph{\textbf{Multiattacco.}} L'orso nero effettua due attacchi: uno con il morso e uno con gli artigli.

\emph{\textbf{Artigli.} Attacco con Arma da Mischia}: +3 a colpire, portata 1 m, un bersaglio.

\emph{Colpisce:} 7 (2d4 + 2) danni taglienti.

\emph{\textbf{Morso.} Attacco con Arma da Mischia}: +3 a colpire, portata 1 m, un bersaglio.

\emph{Colpisce:} 5 (1d6 + 2) danni perforanti.

\medskip\textbf{Orso Polare}\index{Mostri - Orso Polare}

\emph{Grande bestia, disallineato}

\textbf{FORZA} +5

\textbf{DESTREZZA} +0

\textbf{COSTITUZIONE} +3

\textbf{INTELLIGENZA} -4

\textbf{SAGGEZZA} +1

\textbf{CARISMA} -2

\textbf{Iniziativa} +0 -- \textbf{Difesa} 13

\textbf{Punti Ferita} 42 (5d10 + 15)

\textbf{Movimento} 12 m, nuoto 9 m

\textbf{Tiri Salvezza}: Tempra +10, Riflessi +7, Volontà +4 

\textbf{Competenze} Consapevolezza +3

\textbf{Lingue} -

\textbf{Sfida} 2 (450 PE)

\emph{\textbf{Olfatto Affinato.}} L'orso ha +1d6 alle prove di Saggezza (Consapevolezza) basate sull'olfatto.

\textbf{Azioni}

\emph{\textbf{Multiattacco.}} L'orso effettua due attacchi: uno con il morso e uno con gli artigli.

\emph{\textbf{Artigli.} Attacco con Arma da Mischia}: +7 a colpire, portata 1 m, un bersaglio.

\emph{Colpisce:} 12 (2d6 + 5) danni taglienti.

\emph{\textbf{Morso.} Attacco con Arma da Mischia}: +7 a colpire, portata 1 m, un bersaglio.

\emph{Colpisce:} 9 (1d8 + 5) danni perforanti.

\textbf{VARIANTE: ORSO DELLE CAVERNE}\index{Mostri - Orso delle Caverne}

Alcuni orsi si sono adattati alla vita sottoterra. Costoro hanno le stesse statistiche degli orsi polari, ma con visione al buio 18 m.

\medskip\textbf{Pantera}\index{Mostri - Pantera}

\emph{Media bestia, disallineato}

\textbf{FORZA} +2

\textbf{DESTREZZA} +2

\textbf{COSTITUZIONE} +0

\textbf{INTELLIGENZA} -4

\textbf{SAGGEZZA} +2

\textbf{CARISMA} -2

\textbf{Iniziativa} +2 -- \textbf{Difesa} 13

\textbf{Punti Ferita} 13 (3d8)

\textbf{Movimento} 15 m, scalata 12 m

\textbf{Tiri Salvezza}: Tempra +3, Riflessi +5, Volontà +3 

\textbf{Competenze} Muoversi Silenziosamente / Nascondersi nelle Ombre +6, Consapevolezza +4

\textbf{Lingue} -

\textbf{Sfida} 1/4 (50 PE)

\emph{\textbf{Balzo.}} Se la pantera si muove di almeno 6 metri diretta verso una creatura e la colpisce con un attacco di artiglio durante lo stesso turno, il bersaglio deve riuscire un tiro salvezza di Tempra CD 12 o cadere prono. Se il bersaglio è prono, la pantera può effettuare un  attacco di morso contro di esso come azione bonus.

\emph{\textbf{Olfatto Affinato.}} La pantera ha +1d6 alle prove di Saggezza (Consapevolezza) basate sull'olfatto.

\textbf{Azioni}

\emph{\textbf{Artiglio.} Attacco con Arma da Mischia}: +4 a colpire, portata 1 m, un bersaglio.

\emph{Colpisce:} 4 (1d4 + 2) danni taglienti.

\emph{\textbf{Morso.} Attacco con Arma da Mischia}: +4 a colpire, portata 1 m, un bersaglio.

\emph{Colpisce:} 5 (1d6 + 2) danni perforanti.


\medskip\textbf{Pony}\index{Mostri - Pony}

\emph{Media bestia, disallineato}

\textbf{FORZA} +2

\textbf{DESTREZZA} +0

\textbf{COSTITUZIONE} +1

\textbf{INTELLIGENZA} -4

\textbf{SAGGEZZA} +0

\textbf{CARISMA} -2

\textbf{Iniziativa} +0 -- \textbf{Difesa} 11

\textbf{Punti Ferita} 11 (2d8 + 2)

\textbf{Movimento} 12 m

\textbf{Tiri Salvezza}: Tempra +5, Riflessi +4, Volontà +0

\textbf{Lingue} -

\textbf{Sfida} 1/8 (25 PE)

\textbf{Azioni}

\emph{\textbf{Zoccoli.} Attacco con Arma da Mischia}: +4 a colpire, portata 1 m, un bersaglio.

\emph{Colpisce:} 7 (2d4 + 2) danni da botta.

\medskip\textbf{Ragno}\index{Mostri - Ragno}

\emph{Minuscola bestia, disallineato}

\textbf{FORZA} 2 (-5)

\textbf{DESTREZZA} +2

\textbf{COSTITUZIONE} -1

\textbf{INTELLIGENZA} -5

\textbf{SAGGEZZA} +0

\textbf{CARISMA} -4

\textbf{Iniziativa} +2 -- \textbf{Difesa} 13

\textbf{Punti Ferita} 1 (1d4 - 1)

\textbf{Movimento} 6 m, scalata 6 m

\textbf{Tiri Salvezza}: Tempra -4, Riflessi +2, Volontà -4 

\textbf{Competenze} Muoversi Silenziosamente / Nascondersi nelle Ombre +4

\textbf{Sensi} visione al buio 9 m

\textbf{Lingue} -

\textbf{Sfida} 0 (10 PE)

\emph{\textbf{Camminare sulla Tela.}} Il ragno ignora le restrizioni al movimento provocate dalle ragnatele.

\emph{\textbf{Scalare come Ragno.}} Il ragno può scalare superfici difficili, compreso lo stare a testa in giù sul soffitto, senza bisogno  di effettuare una prova di abilità.

\emph{\textbf{Senso della Tela.}} Mentre è in contatto con una ragnatela, il ragno sa l'esatta posizione di qualsiasi altra creatura in contatto con la stessa ragnatela.

\textbf{Azioni}

\emph{\textbf{Morso.} Attacco con Arma da Mischia}: +4 a colpire, portata 1 m, una creatura.

\emph{Colpisce:} 1 danno perforante e il bersaglio deve riuscire un tiro salvezza su Tempra 9 o subire 2 (1d4) danni da veleno.

\medskip\textbf{Ragno Fase}\index{Mostri - Ragno Fase}

Il ragno fase possiede l'abilità magica di entrare ed uscire dal Piano Etereo. Sembra apparire dal nulla e scompare rapidamente dopo aver attaccato.

\emph{Grande mostruosità, disallineato}

\textbf{FORZA} +2

\textbf{DESTREZZA} +2

\textbf{COSTITUZIONE} +1

\textbf{INTELLIGENZA} -2

\textbf{SAGGEZZA} +0

\textbf{CARISMA} -2

\textbf{Iniziativa} +2 -- \textbf{Difesa} 15

\textbf{Punti Ferita} 32 (5d10 + 5)

\textbf{Movimento} 9 m, scalata 9 m

\textbf{Tiri Salvezza}: Tempra +8, Riflessi +8, Volontà +3 

\textbf{Competenze} Muoversi Silenziosamente / Nascondersi nelle Ombre +6

\textbf{Sensi} visione al buio 18 m

\textbf{Lingue} -

\textbf{Sfida} 3 (700 PE)

\emph{\textbf{Camminare sulla Tela.}} Il ragno ignora le restrizioni al movimento provocate dalle ragnatele.

\emph{\textbf{Gita Eterea.}} Come azione bonus, il ragno può magicamente spostarsi dal Piano Materiale al Piano Etereo, o viceversa.

\emph{\textbf{Scalare come Ragno.}} Il ragno può scalare superfici difficili, compreso lo stare a testa in giù sul soffitto, senza bisogno di effettuare una prova di abilità.

\textbf{Azioni}

\emph{\textbf{Morso.} Attacco con Arma da Mischia}: +4 a colpire, portata 1 m, una creatura.

\emph{Colpisce:} 7 (1d10 + 2) danni perforanti e il bersaglio deve effettuare un tiro salvezza di Tempra CD 11, e subire 18 (4d8) danni da veleno se fallisce il tiro salvezza, o la metà di questo danno se lo riesce. Se il danno da veleno riduce il bersaglio a 0 punti ferita, il bersaglio è stabile ma avvelenato per 1 ora, anche dopo aver recuperato i punti ferita, e mentre è avvelenato in questo modo resta paralizzato.

\medskip\textbf{Ragno Gigante}\index{Mostri - Ragno Gigante}

\emph{Grande bestia, disallineato}

\textbf{FORZA} +2

\textbf{DESTREZZA} +3

\textbf{COSTITUZIONE} +1

\textbf{INTELLIGENZA} -4

\textbf{SAGGEZZA} +0

\textbf{CARISMA} -3

\textbf{Iniziativa} +2 -- \textbf{Difesa} 15

\textbf{Punti Ferita} 26 (4d10 + 4)

\textbf{Movimento} 9 m, scalata 9 m

\textbf{Tiri Salvezza}:  Tempra +4, Riflessi +4, Volontà +1 

\textbf{Competenze} Muoversi Silenziosamente / Nascondersi nelle Ombre +7

\textbf{Sensi} vista cieca 3 m, visione al buio 18 m

\textbf{Lingue} -

\textbf{Sfida} 1 (200 PE)

\emph{\textbf{Camminare sulla Tela.}} Il ragno ignora le restrizioni al movimento provocate dalle ragnatele.

\emph{\textbf{Scalare come Ragno.}} Il ragno può scalare superfici difficili, compreso lo stare a testa in giù sul soffitto, senza bisogno di effettuare una prova di abilità.

\emph{\textbf{Senso della Tela.}} Mentre è in contatto con una ragnatela, il ragno sa l'esatta posizione di qualsiasi altra creatura in contatto con la stessa ragnatela. 

\textbf{Azioni}

\emph{\textbf{Morso.} Attacco con Arma da Mischia}: +5 a colpire, portata 1 m, una creatura.

\emph{Colpisce:} 7 (1d8 + 3) danni perforanti e il bersaglio deve effettuare un tiro salvezza di Tempra CD 11, e subire 9

(2d8) danni da veleno se fallisce il tiro salvezza, o la metà di questi danni se lo riesce. Se il danno da veleno riduce il bersaglio a 0 punti ferita, il bersaglio è stabile ma avvelenato per 1 ora, anche dopo aver recuperato i punti ferita, e mentre è avvelenato in questo modo resta paralizzato.

\emph{\textbf{Ragnatela (Ricarica 5-6).} Attacco con Arma a Gittata}: +5 a colpire, gittata 9m, una creatura.

\emph{Colpisce:} Il bersaglio è intralciato dalla ragnatela. Con un'azione, il bersaglio intralciato può effettuare una prova di Forza CD 12 e, in caso di successo, spezzare la tela. La ragnatela può essere anche attaccata e distrutta (CA 10; pf 5; vulnerabilità al danno da fuoco; immunità ai danni da botta, psichici e da veleno). 

\medskip\textbf{Ragno Lupo Gigante}\index{Mostri - Ragno Lupo Gigante}

Un ragno lupo gigante caccia le prede su terreno aperto o si nasconde in tane o fessure del terreno per tendere imboscate.

\emph{Media bestia, disallineato}

\textbf{FORZA} +1

\textbf{DESTREZZA} +3

\textbf{COSTITUZIONE} +1

\textbf{INTELLIGENZA} -4

\textbf{SAGGEZZA} +1

\textbf{CARISMA} -3

\textbf{Iniziativa} +3 -- \textbf{Difesa} 14

\textbf{Punti Ferita} 11 (2d8 + 2)

\textbf{Movimento} 12 m, scalata 12 m

\textbf{Tiri Salvezza}:  Tempra +2, Riflessi +4, Volontà +1 

\textbf{Competenze} Muoversi Silenziosamente / Nascondersi nelle Ombre +7, Consapevolezza +3

\textbf{Sensi} vista cieca 3 m, visione al buio 18 m

\textbf{Lingue} -

\textbf{Sfida} 1/4 (50 PE)

\emph{\textbf{Camminare sulla Tela.}} Il ragno ignora le restrizioni al movimento provocate dalle ragnatele.

\emph{\textbf{Scalare come Ragno.}} Il ragno può scalare superfici difficili, compreso lo stare a testa in giù sul soffitto, senza bisogno di effettuare una prova di abilità.

\emph{\textbf{Senso della Tela.}} Mentre è in contatto con una ragnatela, il ragno sa l'esatta posizione di qualsiasi altra creatura in contatto con la stessa ragnatela.

\textbf{Azioni}

\emph{\textbf{Morso.} Attacco con Arma da Mischia}: +3 a colpire, portata 1 m, una creatura.

\emph{Colpisce:} 4 (1d6 + 1) danni perforanti e il bersaglio deve effettuare un tiro salvezza di Tempra CD 11, e subire 7 (2d6) danni da veleno se fallisce il tiro salvezza, o la metà di questi danni se lo riesce. Se il danno da veleno riduce il bersaglio a 0 punti ferita, il bersaglio è stabile ma avvelenato per 1 ora, anche dopo aver recuperato i punti ferita, e mentre è avvelenato in questo modo resta paralizzato.

\medskip\textbf{Rana}\index{Mostri - Rana}

\emph{Minuscola bestia, disallineato}

\textbf{FORZA} -5

\textbf{DESTREZZA} +1

\textbf{COSTITUZIONE} -1

\textbf{INTELLIGENZA} -5

\textbf{SAGGEZZA} -1

\textbf{CARISMA} -4

\textbf{Iniziativa} +1 -- \textbf{Difesa} 12

\textbf{Punti Ferita} 1 (1d4 - 1)

\textbf{Movimento} 6 m, nuoto 6 m

\textbf{Tiri Salvezza}:  Tempra -4, Riflessi +1, Volontà -2

\textbf{Competenze} Muoversi Silenziosamente / Nascondersi nelle Ombre +3, Consapevolezza +1

\textbf{Sensi} visione al buio 9 m

\textbf{Lingue} -

\textbf{Sfida} 0 (0 PE)

\emph{\textbf{Anfibio.}} La rana può respirare aria e acqua.

\emph{\textbf{Salto da Fermo.}} Una rana può saltare in lungo fino a 3 metri e in alto fino a 1,5 metri, con o senza la rincorsa.

Una \textbf{rana} è sprovvista di attacchi. Si nutre di piccoli insetti e di solito vive in prossimità di acquitrini, dentro gli alberi o sottoterra.

\medskip\textbf{Rana Gigante}\index{Mostri - Rana Gigante}

\emph{Media bestia, disallineato}

\textbf{FORZA} +1

\textbf{DESTREZZA} +1

\textbf{COSTITUZIONE} +0

\textbf{INTELLIGENZA} -4

\textbf{SAGGEZZA} +0

\textbf{CARISMA} -4

\textbf{Iniziativa} +1 -- \textbf{Difesa} 12

\textbf{Punti Ferita} 18 (4d8)

\textbf{Movimento} 9 m, nuoto 9 m

\textbf{Tiri Salvezza}: Tempra +2, Riflessi +2, Volontà +0 

\textbf{Competenze} Muoversi Silenziosamente / Nascondersi nelle Ombre +3, Consapevolezza +2

\textbf{Sensi} visione al buio 9 m

\textbf{Lingue} -

\textbf{Sfida} 1/4 (50 PE)

\emph{\textbf{Anfibio.}} La rana può respirare aria e acqua.

\emph{\textbf{Salto da Fermo.}} Una rana può saltare in lungo fino a 6 metri e in alto fino a 3 metri, con o senza la rincorsa.

\textbf{Azioni}

\emph{\textbf{Morso.} Attacco con Arma da Mischia}: +3 a colpire, portata 1 m, un bersaglio.

\emph{Colpisce:} 4 (1d6 + 1) danni perforanti e il bersaglio è afferrato (CD 11 per fuggire). Fino al termine dell'afferrare, il bersaglio è intralciato, e la rana non può usare il morso contro un altro bersaglio.

\emph{\textbf{Inghiottire.}} La rana effettua una attacco di morso contro un bersaglio di taglia Piccola o inferiore che sta afferrando. Se l'attacco colpisce, il bersaglio è inghiottito, e l'afferrare ha termine. Il bersaglio inghiottito è accecato e intralciato, ha copertura totale contro gli attacchi e altri effetti all'esterno della rana, e subisce 5 (2d4) danni da acido all'inizio di ciascun turno della rana. La rana può inghiottire solo un bersaglio alla volta. Se la rana muore, una creatura inghiottita non è più intralciata da essa e può uscire dal cadavere utilizzando 1,5 metri di movimento, uscendo prona.

\medskip\textbf{Ratto}\index{Mostri - Ratto}

\emph{Minuscola bestia, disallineato}

\textbf{FORZA} -4

\textbf{DESTREZZA} +0

\textbf{COSTITUZIONE} -1

\textbf{INTELLIGENZA} -4

\textbf{SAGGEZZA} +0

\textbf{CARISMA} -3

\textbf{Iniziativa} +0 -- \textbf{Difesa} 11

\textbf{Punti Ferita} 1 (1d4 - 1)

\textbf{Movimento} 6 m

\textbf{Tiri Salvezza}: Tempra -4, Riflessi +0, Volontà +0 

\textbf{Sensi} visione al buio 9 m

\textbf{Lingue} -

\textbf{Sfida} 0 (10 PE)

\emph{\textbf{Olfatto Affinato.}} Il ratto ha +1d6 alle prove di Saggezza (Consapevolezza) basate sull'olfatto.

\textbf{Azioni}

\emph{\textbf{Morso.} Attacco con Arma da Mischia}: +0 a colpire, portata 1 m, un bersaglio.

\emph{Colpisce:} 1 danno perforante.

\medskip\textbf{Ratto Gigante}\index{Mostri - Ratto Gigante}

\emph{Piccola bestia, disallineato}

\textbf{FORZA} -2

\textbf{DESTREZZA} +2

\textbf{COSTITUZIONE} +0

\textbf{INTELLIGENZA} -4

\textbf{SAGGEZZA} +0

\textbf{CARISMA} -3

\textbf{Iniziativa} +2 -- \textbf{Difesa} 13

\textbf{Punti Ferita} 7 (2d6)

\textbf{Movimento} 9 m

\textbf{Tiri Salvezza}: Tempra +3, Riflessi +5, Volontà +1 

\textbf{Sensi} visione al buio 18 m

\textbf{Lingue} -

\textbf{Sfida} 1/8 (25 PE)

\emph{\textbf{Olfatto Affinato.}} Il ratto ha +1d6 alle prove di Saggezza (Consapevolezza) basate sull'olfatto.

\emph{\textbf{Tattiche di Branco.}} Il ratto ha +1d6 al tiro di attacco contro una creatura se almeno uno degli alleati del ratto si trova entro 1,5 metri dalla creatura e quell'alleato non è inabile.

\textbf{Azioni}

\emph{\textbf{Morso.} Attacco con Arma da Mischia}: +4 a colpire, portata 1 m, un bersaglio.

\emph{Colpisce:} 4 (1d4 + 2) danni perforanti.

\textbf{VARIANTE: RATTO GIGANTE AMMALATO}\index{Mostri - Ratto Gigante ammalato}

Alcuni ratti giganti recano una terribile malattia che diffondono tramite il morso. Un ratto gigante ammalato ha grado di sfida 1/8 (25 PE) e la seguente azione invece del suo normale attacco di morso.

\emph{\textbf{Morso.} Attacco con Arma da Mischia}: +4 a colpire, portata 1 m, un bersaglio.

\emph{Colpisce:} 4 (1d4 + 2) danni perforanti. Se il bersaglio è una creatura, deve riuscire un tiro salvezza di Tempra CD 10 o contrarre una malattia. Fino a che la malattia non viene curata, il bersaglio non può recuperare punti ferita eccetto tramite metodi magici, e i punti ferita massimi del bersaglio diminuiscono di 3 (1d6) ogni 24 ore. Se i punti ferita massimi del bersaglio scendono a 0 come risultato della malattia, il bersaglio muore.

\medskip\textbf{Rinoceronte}\index{Mostri - Rinoceronte}

\emph{Grande bestia, disallineato}

\textbf{FORZA} +5

\textbf{DESTREZZA} -1

\textbf{COSTITUZIONE} +2

\textbf{INTELLIGENZA} -4

\textbf{SAGGEZZA} +1

\textbf{CARISMA} -2

\textbf{Iniziativa} -1 -- \textbf{Difesa} 12

\textbf{Punti Ferita} 45 (6d10 + 12)

\textbf{Movimento} 12 m

\textbf{Tiri Salvezza}: Tempra +10, Riflessi +4, Volontà +2

\textbf{Lingue} -

\textbf{Sfida} 2 (450 PE)

\emph{\textbf{Carica.}} Se il rinoceronte si muove di almeno 6 metri diretto verso un bersaglio e lo colpisce con un attacco di incornata durante lo stesso turno, il bersaglio subisce 9 (2d8) danni da botta aggiuntivi. Se il bersaglio è una creatura, deve riuscire un tiro salvezza su Tempra CD 15 o cadere prono.

\textbf{Azioni}

\emph{\textbf{Incornata.} Attacco con Arma da Mischia}: +7 a colpire, portata 1 m, un bersaglio.

\emph{Colpisce:} 14 (2d8 + 5) danni da botta.

\medskip\textbf{Rospo Gigante}\index{Mostri - Rospo Gigante}

\emph{Grande bestia, disallineato}

\textbf{FORZA} +2

\textbf{DESTREZZA} +1

\textbf{COSTITUZIONE} +1

\textbf{INTELLIGENZA} -4

\textbf{SAGGEZZA} +0

\textbf{CARISMA} -4

\textbf{Iniziativa} +1 -- \textbf{Difesa} 12

\textbf{Punti Ferita} 39 (6d10 + 6)

\textbf{Movimento} 6 m, nuoto 12 m

\textbf{Tiri Salvezza}: Tempra +6, Riflessi +6, Volontà +0

\textbf{Sensi} visione al buio 9 m

\textbf{Lingue} -

\textbf{Sfida} 1 (200 PE)

\emph{\textbf{Anfibio.}} Il rospo può respirare aria e acqua.

\emph{\textbf{Salto da Fermo.}} Un rospo può saltare in lungo fino a 6 metri e in alto fino a 3 metri, con o senza la rincorsa.

\textbf{Azioni}

\emph{\textbf{Morso.} Attacco con Arma da Mischia}: +4 a colpire, portata 1 m, un bersaglio.

\emph{Colpisce:} 7 (1d10 + 2) danni perforanti più 5 (1d10) danni da veleno, e il bersaglio è afferrato (CD 13 per fuggire). Fino al termine dell'afferrare, il bersaglio è intralciato, e il rospo non può usare il morso contro un altro bersaglio.

\emph{\textbf{Inghiottire.}} Il rospo effettua una attacco di morso contro un bersaglio di taglia Media o inferiore che sta afferrando. Se l'attacco colpisce, il bersaglio è inghiottito, e l'afferrare ha termine. Il bersaglio inghiottito è accecato e intralciato, ha copertura totale contro gli attacchi e altri effetti all'esterno della rana, e subisce 10 (3d6) danni da acido all'inizio di ciascun turno del rospo. Il rospo può inghiottire solo un bersaglio alla volta.

Se il rospo muore, una creatura inghiottita non è più intralciata da esso e può uscire dal cadavere utilizzando 1,5 metri di movimento, uscendo prono.

\medskip\textbf{Scarabeo di Fuoco Gigante}\index{Mostri - Scarabeo di Fuoco Gigante}

Uno scarabeo di fuoco gigante è una creatura notturna che possiede una coppia di ghiandole luminose capaci di emettere luce per 1d6 giorni dopo la morte dello scarabeo.

\emph{Piccola bestia, disallineato}

\textbf{FORZA} -1

\textbf{DESTREZZA} +0

\textbf{COSTITUZIONE} +1

\textbf{INTELLIGENZA} -5

\textbf{SAGGEZZA} -2

\textbf{CARISMA} -4

\textbf{Iniziativa} +0 -- \textbf{Difesa} 14

\textbf{Punti Ferita} 4 (1d6 + 1)

\textbf{Movimento} 9 m

\textbf{Tiri Salvezza}: Tempra +2, Riflessi +0, Volontà +0

\textbf{Sensi} vista cieca 9 m

\textbf{Lingue} -

\textbf{Sfida} 0 (10 PE)

\emph{\textbf{Illuminazione.}} Lo scarabeo irradia luce intensa in un raggio di 3 metri e luce fioca per ulteriori 3 metri.

\textbf{Azioni}

\emph{\textbf{Morso.} Attacco con Arma da Mischia}: +1 a colpire, portata 1 m, un bersaglio.

\emph{Colpisce:} 2 (1d6 - 1) danni taglienti.

\medskip\textbf{Sciacallo}\index{Mostri - Sciacallo}

\emph{Piccola bestia, disallineato}

\textbf{FORZA} -1

\textbf{DESTREZZA} +2

\textbf{COSTITUZIONE} +0

\textbf{INTELLIGENZA} -4

\textbf{SAGGEZZA} +1

\textbf{CARISMA} -2

\textbf{Iniziativa} +2 -- \textbf{Difesa} 13

\textbf{Punti Ferita} 3 (1d6)

\textbf{Movimento} 12 m

\textbf{Tiri Salvezza}: Tempra -1, Riflessi +3, Volontà +1

\textbf{Competenze} Consapevolezza +3

\textbf{Lingue} -

\textbf{Sfida} 0 (10 PE)

\emph{\textbf{Tattiche di Branco.}} Lo sciacallo ha +1d6 ai tiri di attacco contro una creatura se almeno uno degli alleati dello sciacallo si trova entro 1,5 metri dalla creatura e quell'alleato non è inabile.

\emph{\textbf{Udito e Olfatto Affinato.}} Lo sciacallo ha +1d6 nelle prove di Saggezza (Consapevolezza) basate su udito o olfatto.

\textbf{Azioni}

\emph{\textbf{Morso.} Attacco con Arma da Mischia}: +1 a colpire, portata 1 m, un bersaglio.

\emph{Colpisce:} 1 (1d4 - 1) danni perforanti.

\medskip\textbf{Sciami}\index{Mostri - Sciami}

Gli sciami presentati qui di seguito non sono dei normali o benigni raduni di piccole creature. Si formano invece come risultato di un'influenza esterna, spesso maligna. Anche i druidi non sono in grado di affascinare questi sciami, e la loro aggressività è quasi innaturale. 

\textbf{Sciame di Centopiedi}\index{Mostri - Sciame di Centopiedi}

\emph{Medio sciame di Minuscole bestie, disallineato}

\textbf{FORZA} -4

\textbf{DESTREZZA} +1

\textbf{COSTITUZIONE} +0

\textbf{INTELLIGENZA} -5

\textbf{SAGGEZZA} -2

\textbf{CARISMA} -5

\textbf{Iniziativa} +1 -- \textbf{Difesa} 13

\textbf{Punti Ferita} 22 (5d8)

\textbf{Movimento} 6 m, scalata 6 m

\textbf{Tiri Salvezza}: Tempra -1, Riflessi +3, Volontà +1

\textbf{Resistenze al Danno} da botta, perforante, tagliente

\textbf{Immunità alle Condizioni} affascinato, afferrato, intralciato,

paralizzato, pietrificato, prono, spaventato, stordito

\textbf{Sensi} vista cieca 3 m 

\textbf{Lingue} -

\textbf{Sfida} 1/2 (100 PE)

\emph{\textbf{Sciame.}} Lo sciame può occupare lo spazio di un'altra creatura e viceversa, e lo sciame può muoversi attraverso qualsiasi apertura grande abbastanza per un Minuscolo insetto. Lo sciame non può recuperare punti ferita né ottenere punti ferita temporanei.

\textbf{Azioni}

\emph{\textbf{Morsi.} Attacco con Arma da Mischia}: +3 a colpire, portata 0 m, un bersaglio nello spazio dello sciame. 

\emph{Colpisce:} 10 (4d4) danni perforanti, o 5 (2d4) danni perforanti se lo sciame è ha metà o meno dei suoi punti ferita. Una creatura ridotta a 0 punti ferita da uno sciame di centopiedi e stabile resta avvelenata per 1 ora, anche dopo aver recuperato i punti ferita, e rimane paralizzata dal veleno durante questo periodo.

\medskip\textbf{Sciame di Corvi}\index{Mostri - Sciame di Corvi}

\emph{Medio sciame di Minuscole bestie, disallineato}

\textbf{FORZA} -2

\textbf{DESTREZZA} +2

\textbf{COSTITUZIONE} -1

\textbf{INTELLIGENZA} -4

\textbf{SAGGEZZA} +1

\textbf{CARISMA} -2

\textbf{Iniziativa} +2 -- \textbf{Difesa} 13

\textbf{Punti Ferita} 24 (7d8 -- 7)

\textbf{Movimento} 3 m, volo 15 m

\textbf{Tiri Salvezza}: Tempra -1, Riflessi +3, Volontà +2

\textbf{Competenze} Consapevolezza +5

\textbf{Resistenze al Danno} da botta, perforante, tagliente \textbf{Immunità alle Condizioni} affascinato, afferrato, intralciato, paralizzato, pietrificato, prono, spaventato, stordito

\textbf{Lingue} -

\textbf{Sfida} 1/4 (50 PE)

\emph{\textbf{Sciame.}} Lo sciame può occupare lo spazio di un'altra creatura e viceversa, e lo sciame può muoversi attraverso qualsiasi apertura grande abbastanza per un Minuscolo corvo. Lo sciame non può recuperare punti ferita né ottenere punti ferita temporanei.

\textbf{Azioni}

\emph{\textbf{Becchi.} Attacco con Arma da Mischia}: +4 a colpire, portata 1 m, un bersaglio nello spazio dello sciame.

\emph{Colpisce:} 7 (2d6) danni perforanti, o 3 (1d6) danni perforanti se lo sciame è ha metà o meno dei suoi punti ferita.

\medskip\textbf{Sciame di Pirana}\index{Mostri - Sciame di Pirana}

\emph{Medio sciame di Minuscole bestie, disallineato}

\textbf{FORZA} +1

\textbf{DESTREZZA} +3

\textbf{COSTITUZIONE} -1

\textbf{INTELLIGENZA} -5

\textbf{SAGGEZZA} -2

\textbf{CARISMA} -4

\textbf{Iniziativa} +3 -- \textbf{Difesa} 14

\textbf{Punti Ferita} 28 (8d8 -- 8)

\textbf{Movimento} 0 m, nuoto 12 m

\textbf{Tiri Salvezza}: Tempra -3, Riflessi +4, Volontà -1

\textbf{Resistenze al Danno} da botta, perforante, tagliente

\textbf{Immunità alle Condizioni} affascinato, afferrato, intralciato, paralizzato, pietrificato, prono, spaventato, stordito

\textbf{Sensi} visione al buio 18 m

\textbf{Lingue} -

\textbf{Sfida} 1 (200 PE)

\emph{\textbf{Frenesia Sanguinaria.}} Lo sciame ha +1d6 ai tiri di attacco in mischia contro qualsiasi creatura che non sia al massimo dei punti ferita.

\emph{\textbf{Respirare Acqua.}} Lo sciame può respirare solo sottacqua.

\emph{\textbf{Sciame.}} Lo sciame può occupare lo spazio di un'altra creatura e viceversa, e lo sciame può muoversi attraverso qualsiasi apertura grande abbastanza per un Minuscolo pirana. Lo sciame non può recuperare punti ferita né ottenere punti ferita temporanei.

\textbf{Azioni}

\emph{\textbf{Morsi.} Attacco con Arma da Mischia}: +5 a colpire, portata 0 m, una creatura nello spazio dello sciame.

\emph{Colpisce:} 14 (4d6) danni perforanti, o 7 (2d6) danni perforanti se lo sciame è ha metà o meno dei suoi punti ferita.

\medskip\textbf{Sciame di Insetti}\index{Mostri - Sciame di Insetti}

\emph{Medio sciame di Minuscole bestie, disallineato}

\textbf{FORZA} -4

\textbf{DESTREZZA} +1

\textbf{COSTITUZIONE} +0

\textbf{INTELLIGENZA} -5

\textbf{SAGGEZZA} -2

\textbf{CARISMA} -5

\textbf{Iniziativa} +1 -- \textbf{Difesa} 13

\textbf{Punti Ferita} 22 (5d8)

\textbf{Movimento} 6 m, scalata 6 m

\textbf{Tiri Salvezza}: Tempra -3, Riflessi +2, Volontà -1

\textbf{Resistenze al Danno} da botta, perforante, tagliente

\textbf{Immunità alle Condizioni} affascinato, afferrato, intralciato, paralizzato, pietrificato, prono, spaventato, stordito

\textbf{Sensi} vista cieca 3 m

\textbf{Lingue} -

\textbf{Sfida} 1/2 (100 PE)

\emph{\textbf{Sciame.}} Lo sciame può occupare lo spazio di un'altra creatura e viceversa, e lo sciame può muoversi attraverso qualsiasi apertura grande abbastanza per un Minuscolo insetto. Lo sciame non può recuperare punti ferita né ottenere punti ferita temporanei.

\textbf{Azioni}

\emph{\textbf{Morsi.} Attacco con Arma da Mischia}: +3 a colpire, portata 0 m, un bersaglio nello spazio dello sciame.

\emph{Colpisce:} 10 (4d4) danni perforanti, o 5 (2d4) danni perforanti se lo sciame è ha metà o meno dei suoi punti ferita.

\medskip\textbf{Sciame di Pipistrelli}\index{Mostri - Sciame di Pipistrelli}

\emph{Medio sciame di Minuscole bestie, disallineato}

\textbf{FORZA} -3

\textbf{DESTREZZA} +2

\textbf{COSTITUZIONE} +0

\textbf{INTELLIGENZA} -4

\textbf{SAGGEZZA} +1

\textbf{CARISMA} -3

\textbf{Iniziativa} +2 -- \textbf{Difesa} 13

\textbf{Punti Ferita} 22 (5d8)

\textbf{Movimento} 0 m, volo 9 m

\textbf{Tiri Salvezza}: Tempra -2, Riflessi +4, Volontà +2

\textbf{Resistenze al Danno} da botta, perforante, tagliente

\textbf{Immunità alle Condizioni} affascinato, afferrato, intralciato, paralizzato, pietrificato, prono, spaventato, stordito

\textbf{Sensi} vista cieca 18 m

\textbf{Lingue} -

\textbf{Sfida} 1/4 (50 PE)

\emph{\textbf{Ecolocazione.}} Lo sciame non può usare la vista cieca se assordato.

\emph{\textbf{Sciame.}} Lo sciame può occupare lo spazio di un'altra creatura e viceversa, e lo sciame può muoversi attraverso qualsiasi apertura grande abbastanza per un Minuscolo pipistrello. Lo sciame non può recuperare punti ferita né ottenere punti ferita temporanei.

\emph{\textbf{Udito Affinato.}} Lo sciame ha +1d6 alle prove di Saggezza (Consapevolezza) basate sull'udito.

\textbf{Azioni}

\emph{\textbf{Morsi.} Attacco con Arma da Mischia}: +4 a colpire, portata 0 m, una creatura nello spazio dello sciame.

\emph{Colpisce:} 5 (2d4) danni perforanti, o 2 (1d4) danni perforanti se lo sciame è ha metà o meno dei suoi punti ferita.

\medskip\textbf{Sciame di Ragni}\index{Mostri - Sciame di Ragni}

\emph{Medio sciame di Minuscole bestie, disallineato}

\textbf{FORZA} -4

\textbf{DESTREZZA} +1

\textbf{COSTITUZIONE} +0

\textbf{INTELLIGENZA} -5

\textbf{SAGGEZZA} -2

\textbf{CARISMA} -5

\textbf{Iniziativa} +1 -- \textbf{Difesa} 13

\textbf{Punti Ferita} 22 (5d8)

\textbf{Movimento} 6 m, scalata 6 m

\textbf{Tiri Salvezza}: Tempra -3, Riflessi +2, Volontà -1

\textbf{Resistenze al Danno} da botta, perforante, tagliente

\textbf{Immunità alle Condizioni} affascinato, afferrato, intralciato, paralizzato, pietrificato, prono, spaventato, stordito

\textbf{Sensi} vista cieca 3 m

\textbf{Lingue} -

\textbf{Sfida} 1/2 (100 PE)

\emph{\textbf{Camminare sulla Tela.}} Lo sciame ignora le restrizioni al movimento provocate dalle ragnatele.

\emph{\textbf{Scalare come Ragno.}} Lo sciame può scalare superfici difficili, compreso lo stare a testa in giù sul soffitto, senza bisogno di effettuare una prova di abilità.

\emph{\textbf{Senso della Tela.}} Mentre è in contatto con una ragnatela, lo sciame sa l'esatta posizione di qualsiasi altra creatura in contatto con la stessa ragnatela.

\emph{\textbf{Sciame.}} Lo sciame può occupare lo spazio di un'altra creatura e viceversa, e lo sciame può muoversi attraverso qualsiasi apertura grande abbastanza per un Minuscolo insetto. Lo sciame non può recuperare punti ferita né ottenere punti ferita temporanei.

\textbf{Azioni}

\emph{\textbf{Morsi.} Attacco con Arma da Mischia}: +3 a colpire, portata 0 m, un bersaglio nello spazio dello sciame.

\emph{Colpisce:} 10 (4d4) danni perforanti, o 5 (2d4) danni perforanti se lo sciame è ha metà o meno dei suoi punti ferita.

\medskip\textbf{Sciame di Ratti}\index{Mostri - Sciame di Ratti}

\emph{Medio sciame di Minuscole bestie, disallineato}

\textbf{FORZA} -1

\textbf{DESTREZZA} +0

\textbf{COSTITUZIONE} -1

\textbf{INTELLIGENZA} -4

\textbf{SAGGEZZA} +0

\textbf{CARISMA} -4

\textbf{Iniziativa} +0 -- \textbf{Difesa} 11

\textbf{Punti Ferita} 24 (7d8 - 7)

\textbf{Movimento} 9 m

\textbf{Tiri Salvezza}: Tempra +0, Riflessi +1, Volontà +1

\textbf{Resistenze al Danno} da botta, perforante, tagliente

\textbf{Immunità alle Condizioni} affascinato, afferrato, intralciato, paralizzato, pietrificato, prono, spaventato, stordito

\textbf{Sensi} visione al buio 9 m
\textbf{Lingue} -

\textbf{Sfida} 1/4 (50 PE)

\emph{\textbf{Olfatto Affinato.}} Lo sciame ha +1d6 alle prove di Saggezza (Consapevolezza) basate sull'olfatto.

\emph{\textbf{Sciame.}} Lo sciame può occupare lo spazio di un'altra creatura e viceversa, e lo sciame può muoversi attraverso qualsiasi apertura grande abbastanza per un Minuscolo ratto. Lo sciame non può recuperare punti ferita né ottenere punti ferita temporanei.

\textbf{Azioni}

\emph{\textbf{Morsi.} Attacco con Arma da Mischia}: +2 a colpire, portata 0 m, un bersaglio nello spazio dello sciame.

\emph{Colpisce:} 7 (2d6) danni perforanti, o 3 (1d6) danni perforanti se lo sciame è ha metà o meno dei suoi punti ferita.

\medskip\textbf{Sciame di Scarabei}\index{Sciame di Scarabei}

\emph{Medio sciame di Minuscole bestie, disallineato}

\textbf{FORZA} -4

\textbf{DESTREZZA} +1

\textbf{COSTITUZIONE} +0

\textbf{INTELLIGENZA} -5

\textbf{SAGGEZZA} -2

\textbf{CARISMA} -5

\textbf{Iniziativa} +1 -- \textbf{Difesa} 13

\textbf{Punti Ferita} 22 (5d8)

\textbf{Movimento} 6 m, scalata 6 m, scavo 6 m

\textbf{Tiri Salvezza}: Tempra -3, Riflessi +2, Volontà -1

\textbf{Resistenze al Danno} da botta, perforante, tagliente

\textbf{Immunità alle Condizioni} affascinato, afferrato, intralciato, paralizzato, pietrificato, prono, spaventato, stordito

\textbf{Sensi} vista cieca 3 m

\textbf{Lingue} -

\textbf{Sfida} 1/2 (100 PE)

\emph{\textbf{Sciame.}} Lo sciame può occupare lo spazio di un'altra creatura e viceversa, e lo sciame può muoversi attraverso qualsiasi apertura grande abbastanza per un Minuscolo insetto. Lo sciame non può recuperare punti ferita né ottenere punti ferita temporanei.

\textbf{Azioni}

\emph{\textbf{Morsi.} Attacco con Arma da Mischia}: +3 a colpire, portata 0 m, un bersaglio nello spazio dello sciame.

\emph{Colpisce:} 10 (4d4) danni perforanti, o 5 (2d4) danni perforanti se lo sciame è ha metà o meno dei suoi punti ferita.

\medskip\textbf{Sciame di Serpenti Velenosi}\index{Sciame di Serpenti Velenosi}

\emph{Medio sciame di Minuscole bestie, disallineato}

\textbf{FORZA} -1

\textbf{DESTREZZA} +4

\textbf{COSTITUZIONE} +0

\textbf{INTELLIGENZA} -5

\textbf{SAGGEZZA} +0

\textbf{CARISMA} -4

\textbf{Iniziativa} +4 -- \textbf{Difesa} 15

\textbf{Punti Ferita} 36 (8d8)

\textbf{Movimento} 9 m, nuoto 9 m

\textbf{Tiri Salvezza}: Tempra +0, Riflessi +5, Volontà +1

\textbf{Resistenze al Danno} da botta, perforante, tagliente

\textbf{Immunità alle Condizioni} affascinato, afferrato, intralciato, paralizzato, pietrificato, prono, spaventato, stordito

\textbf{Sensi} vista cieca 3 m

\textbf{Lingue} -

\textbf{Sfida} 2 (450 PE)

\emph{\textbf{Sciame.}} Lo sciame può occupare lo spazio di un'altra creatura e viceversa, e lo sciame può muoversi attraverso qualsiasi apertura grande abbastanza per un Minuscolo serpente. Lo sciame non può recuperare punti ferita né ottenere punti ferita temporanei.

\textbf{Azioni}

\emph{\textbf{Morsi.} Attacco con Arma da Mischia}: +6 a colpire, portata 0 m, una creatura nello spazio dello sciame.

\emph{Colpisce:} 7 (2d6) danni perforanti, o 3 (1d6) danni perforanti se lo sciame è ha metà o meno dei suoi punti ferita, e il bersaglio deve effettuare un tiro salvezza di Tempra CD 10, e subire 14 (4d6) danni da veleno se fallisce il tiro salvezza, o la metà di questi danni se lo riesce.

\medskip\textbf{Sciame di Vespe}\index{Sciame di Serpenti Velenosi}

\emph{Medio sciame di Minuscole bestie, disallineato}

\textbf{FORZA} -4

\textbf{DESTREZZA} +1

\textbf{COSTITUZIONE} +0

\textbf{INTELLIGENZA} -5

\textbf{SAGGEZZA} -2

\textbf{CARISMA} -5

\textbf{Iniziativa} +1 -- \textbf{Difesa} 13

\textbf{Punti Ferita} 22 (5d8)

\textbf{Movimento} 1,5 m, volo 9 m

\textbf{Tiri Salvezza}: Tempra -3, Riflessi +2, Volontà -1

\textbf{Resistenze al Danno} da botta, perforante, tagliente

\textbf{Immunità alle Condizioni} affascinato, afferrato, intralciato, paralizzato, pietrificato, prono, spaventato, stordito

\textbf{Sensi} vista cieca 3 m

\textbf{Lingue} -

\textbf{Sfida} 1/2 (100 PE)

\emph{\textbf{Sciame.}} Lo sciame può occupare lo spazio di un'altra creatura e viceversa, e lo sciame può muoversi attraverso qualsiasi apertura grande abbastanza per un Minuscolo insetto. Lo sciame non può recuperare punti ferita né ottenere punti ferita temporanei.

\textbf{Azioni}

\emph{\textbf{Morsi.} Attacco con Arma da Mischia}: +3 a colpire, portata 0 m, un bersaglio nello spazio dello sciame.

\emph{Colpisce:} 10 (4d4) danni perforanti, o 5 (2d4) danni perforanti se lo sciame è ha metà o meno dei suoi punti ferita.

\medskip\textbf{Scimmione}\index{Mostri - Scimmione}

\emph{Media bestia, disallineato}

\textbf{FORZA} +3

\textbf{DESTREZZA} +2

\textbf{COSTITUZIONE} +2

\textbf{INTELLIGENZA} -2

\textbf{SAGGEZZA} +1

\textbf{CARISMA} -2

\textbf{Iniziativa} +2 -- \textbf{Difesa} 13

\textbf{Punti Ferita} 19 (3d8 + 6)

\textbf{Movimento} 9 m, scalata 9 m

\textbf{Tiri Salvezza}: Tempra +3, Riflessi +3, Volontà +2

\textbf{Competenze} Acrobatica +5, Consapevolezza +3

\textbf{Lingue} -

\textbf{Sfida} 1/2 (100 PE)

\textbf{Azioni}

\emph{\textbf{Multiattacco.}} Lo scimmione effettua due attacchi di pugno.

\emph{\textbf{Pugno.} Attacco con Arma da Mischia}: +5 a colpire, portata 1 m, un bersaglio.

\emph{Colpisce:} 6 (1d6 + 3) danni da botta.

\emph{\textbf{Sasso.} Attacco con Arma a Gittata}: +5 a colpire, gittata 8m, un bersaglio.

\emph{Colpisce:} 6 (1d6 + 3) danni da botta.

\medskip\textbf{Scimmione Gigante}\index{Mostri - Scimmione Gigante}

\emph{Enorme bestia, disallineato}

\textbf{FORZA} +6

\textbf{DESTREZZA} +2

\textbf{COSTITUZIONE} +4

\textbf{INTELLIGENZA} -2

\textbf{SAGGEZZA} +1

\textbf{CARISMA} -2

\textbf{Iniziativa} +2 -- \textbf{Difesa} 16

\textbf{Punti Ferita} 157 (15d12 + 60)

\textbf{Movimento} 12 m, scalata 12 m

\textbf{Tiri Salvezza}: Tempra +7, Riflessi +6, Volontà +4

\textbf{Competenze} Acrobatica +9, Consapevolezza +4

\textbf{Lingue} -

\textbf{Sfida} 7 (2.900 PE)

\textbf{Azioni}

\emph{\textbf{Multiattacco.}} Lo scimmione effettua due attacchi di pugno.

\emph{\textbf{Pugno.} Attacco con Arma da Mischia}: +9 a colpire, portata 3 m, un bersaglio.

\emph{Colpisce:} 22 (3d10 + 6) danni da botta.

\emph{\textbf{Sasso.} Attacco con Arma a Gittata}: +9 a colpire, gittata 15m, un bersaglio.

\emph{Colpisce:} 30 (7d6 + 6) danni da botta.

\medskip\textbf{Scorpione}\index{Mostri - Scorpione}

\emph{Minuscola bestia, disallineato}

\textbf{FORZA} -4

\textbf{DESTREZZA} +0

\textbf{COSTITUZIONE} -1

\textbf{INTELLIGENZA} -5

\textbf{SAGGEZZA} -1

\textbf{CARISMA} -4

\textbf{Iniziativa} +0 -- \textbf{Difesa} 12

\textbf{Punti Ferita} 1 (1d4 - 1)

\textbf{Movimento} 3 m

\textbf{Tiri Salvezza}: Tempra -3, Riflessi +2, Volontà -1

\textbf{Sensi} vista cieca 3 m

\textbf{Lingue} -

\textbf{Sfida} 0 (10 PE)

\textbf{Azioni}

\emph{\textbf{Pungiglione.} Attacco con Arma da Mischia}: +2 a colpire, portata 1 m, una creatura.

\emph{Colpisce:} 1 danno perforante e il bersaglio deve effettuare un tiro salvezza di Tempra CD 9, e subire 4 (1d8) danni da veleno se fallisce il tiro salvezza, o la metà di questi danni se lo riesce.

\medskip\textbf{Scorpione Gigante}\index{Mostri - Scorpione Gigante}

\emph{Grande bestia, disallineato}

\textbf{FORZA} +2

\textbf{DESTREZZA} +1

\textbf{COSTITUZIONE} +2

\textbf{INTELLIGENZA} -5

\textbf{SAGGEZZA} -1

\textbf{CARISMA} -4

\textbf{Iniziativa} +1 -- \textbf{Difesa} 17

\textbf{Punti Ferita} 52 (7d10 + 14)

\textbf{Movimento} 12 m

\textbf{Tiri Salvezza}: Tempra +7, Riflessi +1, Volontà +1

\textbf{Sensi} vista cieca 18 m

\textbf{Lingue} -

\textbf{Sfida} 3 (700 PE)

\textbf{Azioni}

\emph{\textbf{Multiattacco.}} Lo scorpione effettua tre attacchi: due con gli artigli e uno con il pungiglione.

\emph{\textbf{Artiglio.} Attacco con Arma da Mischia}: +4 a colpire, portata 1 m, un bersaglio.

\emph{Colpisce:} 6 (1d8 + 2) danni da botta e il bersaglio è afferrato (CD 12 per fuggire). Lo scorpione ha due artigli, ciascuno dei quali può afferrare solo un bersaglio.

\emph{\textbf{Pungiglione.} Attacco con Arma da Mischia}: +4 a colpire, portata 1 m, una creatura.

\emph{Colpisce:} 7 (1d10 + 2) danni perforanti e il bersaglio deve effettuare un tiro salvezza di Tempra CD 12, e subire 22 (4d10) danni da veleno se fallisce il tiro salvezza, o la metà di questi danni se lo riesce.

\medskip\textbf{Serpente Costrittore}\index{Mostri - Serpente Costrittore}

\emph{Grande bestia, disallineato}

\textbf{FORZA} +2

\textbf{DESTREZZA} +2

\textbf{COSTITUZIONE} +1

\textbf{INTELLIGENZA} -5

\textbf{SAGGEZZA} +0

\textbf{CARISMA} -4

\textbf{Iniziativa} +2 -- \textbf{Difesa} 13

\textbf{Punti Ferita} 13 (2d10 + 2)

\textbf{Movimento} 9 m, nuoto 9 m

\textbf{Tiri Salvezza}: Tempra +3, Riflessi +2, Volontà +0

\textbf{Sensi} vista cieca 3 m

\textbf{Lingue} -

\textbf{Sfida} 1/4 (50 PE)

\textbf{Azioni}

\emph{\textbf{Morso.} Attacco con Arma da Mischia}: +4 a colpire, portata 1 m, una creatura.

\emph{Colpisce:} 5 (1d6 + 2) danni perforanti.

\emph{\textbf{Stritolare.} Attacco con Arma da Mischia}: +4 a colpire, portata 1 m, una creatura.

\emph{Colpisce:} 6 (1d8 + 2) danni da botta, e il bersaglio è afferrato (CD 14 per fuggire). Fino al termine dell'afferrare, la creatura è intralciata, e il serpente non può stritolare un altro bersaglio.

\medskip\textbf{Serpente Costrittore Gigante}\index{Mostri - Serpente Costrittore Gigante}

\emph{Enorme bestia, disallineato}

\textbf{FORZA} +4

\textbf{DESTREZZA} +2

\textbf{COSTITUZIONE} +1

\textbf{INTELLIGENZA} -5

\textbf{SAGGEZZA} +0

\textbf{CARISMA} -4

\textbf{Iniziativa} +2 -- \textbf{Difesa} 13

\textbf{Punti Ferita} 60 (8d12 + 8)

\textbf{Movimento} 9 m, nuoto 9 m

\textbf{Tiri Salvezza}: Tempra +3, Riflessi +2, Volontà +0

\textbf{Competenze} Consapevolezza +2

\textbf{Sensi} vista cieca 3 m

\textbf{Lingue} -

\textbf{Sfida} 2 (450 PE)

\textbf{Azioni}

\emph{\textbf{Morso.} Attacco con Arma da Mischia}: +6 a colpire, portata 3 m, una creatura.

\emph{Colpisce:} 11 (2d6 + 4) danni perforanti.

\emph{\textbf{Stritolare.} Attacco con Arma da Mischia}: +6 a colpire, portata 1 m, una creatura.

\emph{Colpisce:} 13 (2d8 + 4) danni da botta, e il bersaglio è afferrato (CD 16 per fuggire). Fino al termine dell'afferrare, la creatura è intralciata, e il serpente non può stritolare un altro bersaglio.

\medskip\textbf{Serpente Velenoso}\index{Mostri - Serpente Velenoso}

\emph{Minuscola bestia, disallineato}

\textbf{FORZA} -4

\textbf{DESTREZZA} +3

\textbf{COSTITUZIONE} +0

\textbf{INTELLIGENZA} -5

\textbf{SAGGEZZA} +0

\textbf{CARISMA} -4

\textbf{Iniziativa} +3 -- \textbf{Difesa} 14

\textbf{Punti Ferita} 2 (1d4)

\textbf{Movimento} 9 m, nuoto 9 m

\textbf{Tiri Salvezza}: Tempra +1, Riflessi +4, Volontà +1

\textbf{Sensi} vista cieca 3 m

\textbf{Lingue} -

\textbf{Sfida} 1/8 (25 PE)

\textbf{Azioni}

\emph{\textbf{Morso.} Attacco con Arma da Mischia}: +5 a colpire, portata 1 m, un bersaglio.

\emph{Colpisce:} 1 danno perforante e il bersaglio deve effettuare un tiro salvezza di Tempra CD 10, e subire 5 (2d4) danni da veleno se fallisce il tiro salvezza, o la metà di questi danni se lo riesce.

\medskip\textbf{Serpente Velenoso Gigante}\index{Mostri - Serpente Velenoso Gigante}

\emph{Media bestia, disallineato}

\textbf{FORZA} +0

\textbf{DESTREZZA} +4

\textbf{COSTITUZIONE} +1

\textbf{INTELLIGENZA} -4

\textbf{SAGGEZZA} +0

\textbf{CARISMA} -4

\textbf{Iniziativa} +4 -- \textbf{Difesa} 15

\textbf{Punti Ferita} 11 (2d8 + 2)

\textbf{Movimento} 9 m, nuoto 9 m

\textbf{Tiri Salvezza}: Tempra +1, Riflessi +5, Volontà +2

\textbf{Competenze} Consapevolezza +2

\textbf{Sensi} vista cieca 3 m

\textbf{Lingue} -

\textbf{Sfida} 1/4 (50 PE)

\textbf{Azioni}

\emph{\textbf{Morso.} Attacco con Arma da Mischia}: +6 a colpire, portata 3 m, un bersaglio.

\emph{Colpisce:} 6 (1d4 + 4) danni perforanti e il bersaglio deve effettuare un tiro salvezza di Tempra CD 11, e subire 10 (3d6) danni da veleno se fallisce il tiro salvezza, o la metà di questi danni se lo riesce.

\medskip\textbf{Serpente Volante}\index{Mostri - Serpente Volante}

Un serpente volante è una serpe alata, dai colori intensi, rinvenuta in giungle remote.

\emph{Minuscola bestia, disallineato}

\textbf{FORZA} -3

\textbf{DESTREZZA} +4

\textbf{COSTITUZIONE} +0

\textbf{INTELLIGENZA} -4

\textbf{SAGGEZZA} +1

\textbf{CARISMA} -3

\textbf{Iniziativa} +4 -- \textbf{Difesa} 15

\textbf{Punti Ferita} 5 (2d4)

\textbf{Movimento} 9 m, nuoto 9 m, volo 18 m

\textbf{Tiri Salvezza}: Tempra -2, Riflessi +5, Volontà +1

\textbf{Sensi} vista cieca 3 m

\textbf{Lingue} -

\textbf{Sfida} 1/8 (25 PE)

\emph{\textbf{Sorvolare.}} Il serpente non provoca attacchi di opportunità quando vola via dalla portata di un nemico.

\textbf{Azioni}

\emph{\textbf{Morso.} Attacco con Arma da Mischia}: +6 a colpire, portata 1 m, un bersaglio.

\emph{Colpisce:} 1 danno perforante più 7 (3d4) danni da veleno.

\medskip\textbf{Squalo Cacciatore}\index{Mostri - Squalo Cacciatore}

Uno squalo cacciatore è lungo da 4,5 a 6 metri e di solito caccia in solitario nelle acque più profonde.

\emph{Grande bestia, disallineato}

\textbf{FORZA} +4

\textbf{DESTREZZA} +1

\textbf{COSTITUZIONE} +2

\textbf{INTELLIGENZA} -5

\textbf{SAGGEZZA} +0

\textbf{CARISMA} -3

\textbf{Iniziativa} +1 -- \textbf{Difesa} 13

\textbf{Punti Ferita} 45 (6d10 + 12)

\textbf{Movimento} 0 m, nuoto 12 m

\textbf{Tiri Salvezza}: Tempra +4, Riflessi +2, Volontà +0

\textbf{Competenze} Consapevolezza +2

\textbf{Sensi} vista cieca 9 m

\textbf{Lingue} -

\textbf{Sfida} 2 (450 PE)

\emph{\textbf{Frenesia Sanguinaria.}} Lo squalo ha +1d6 ai tiri di attacco in mischia contro qualsiasi creatura che non sia al massimo dei punti ferita.

\emph{\textbf{Respirare Acqua.}} Lo squalo può respirare solo sottacqua.

\textbf{Azioni}

\emph{\textbf{Morso.} Attacco con Arma da Mischia}: +6 a colpire, portata 1 m, un bersaglio.

\emph{Colpisce:} 13 (2d8 + 4) danni perforanti.

\medskip\textbf{Squalo Corallino}\index{Mostri - Squalo Corallino}

Gli squali corallini sono lunghi da 2 a 3 metri e vivono nelle acque meno profonde e lungo le barriere coralline.

\emph{Media bestia, disallineato}

\textbf{FORZA} +2

\textbf{DESTREZZA} +1

\textbf{COSTITUZIONE} +1

\textbf{INTELLIGENZA} -5

\textbf{SAGGEZZA} +0

\textbf{CARISMA} -3

\textbf{Iniziativa} +1 -- \textbf{Difesa} 13

\textbf{Punti Ferita} 22 (4d8 + 4)

\textbf{Movimento} 0 m, nuoto 12 m

\textbf{Tiri Salvezza}: Tempra +2, Riflessi +2, Volontà +1

\textbf{Competenze} Consapevolezza +2

\textbf{Sensi} vista cieca 9 m

\textbf{Lingue} -

\textbf{Sfida} 1/2 (100 PE)

\emph{\textbf{Respirare Acqua.}} Lo squalo può respirare solo sottacqua.

\emph{\textbf{Tattiche di Branco.}} Lo squalo ha +1d6 al tiro di attacco contro una creatura se almeno uno degli alleati dello squalo si trova entro 1,5 metri dalla creatura e quell'alleato non è inabile.

\textbf{Azioni}

\emph{\textbf{Morso.} Attacco con Arma da Mischia}: +4 a colpire, portata 1,5

m, un bersaglio.

\emph{Colpisce:} 6 (1d8 + 2) danni perforanti.

\medskip\textbf{Squalo Gigante}\index{Mostri - Squalo Gigante}

Lo squalo gigante è lungo 9 metri e lo si incontra

normalmente solo negli oceani più profondi.

\emph{Enorme bestia, disallineato}

\textbf{FORZA} +6

\textbf{DESTREZZA} +0

\textbf{COSTITUZIONE} +5

\textbf{INTELLIGENZA} -5

\textbf{SAGGEZZA} +0

\textbf{CARISMA} -3

\textbf{Iniziativa} +0 -- \textbf{Difesa} 16

\textbf{Punti Ferita} 126 (11d12 + 55)

\textbf{Movimento} 0 m, nuoto 15 m

\textbf{Tiri Salvezza}: Tempra +7, Riflessi +2, Volontà +1

\textbf{Competenze} Consapevolezza +3

\textbf{Sensi} vista cieca 18 m

\textbf{Lingue} -

\textbf{Sfida} 5 (1.800 PE)

\emph{\textbf{Frenesia Sanguinaria.}} Lo squalo ha +1d6 ai tiri di attacco

in mischia contro qualsiasi creatura che non sia al massimo dei punti ferita.

\emph{\textbf{Respirare Acqua.}} Lo squalo può respirare solo sottacqua. 

\textbf{Azioni}

\emph{\textbf{Morso.} Attacco con Arma da Mischia}: +9 a colpire, portata 1 m, un bersaglio.

\emph{Colpisce:} 22 (3d10 + 6) danni perforanti.

\medskip\textbf{Strige}\index{Mostri - Strige}

Questo orrendo mostro sembra un incrocio tra un grosso pipistrello e una zanzara sovradimensionata. Le sue zampe terminano in lunghe pinze, e la sua lunga proboscide, simile ad un ago, fende l'aria mentre cerca di nutrirsi del sangue delle creature viventi. 

\emph{Minuscola bestia, disallineato}

\textbf{FORZA} -3

\textbf{DESTREZZA} +3

\textbf{COSTITUZIONE} +0

\textbf{INTELLIGENZA} -4

\textbf{SAGGEZZA} -1

\textbf{CARISMA} -2

\textbf{Iniziativa} +3 -- \textbf{Difesa} 15

\textbf{Punti Ferita} 2 (1d4)

\textbf{Movimento} 3 m, volo 12 m

\textbf{Tiri Salvezza}: Tempra -3, Riflessi +4, Volontà -1

\textbf{Sensi} visione al buio 18 m

\textbf{Lingue} -

\textbf{Sfida} 1/8 (25 PE)

\textbf{Azioni}

\emph{\textbf{Risucchio di Sangue.} Attacco con Arma da Mischia}: +5 a colpire, portata 1 m, una creatura.

\emph{Colpisce:} 5 (1d4 + 3) danni perforanti e lo strige si attacca al bersaglio. Mentre è attaccato, lo strige non attacca. Invece, all'inizio di ciascun turno dello strige, il bersaglio perde 5 (1d4 + 3) punti ferita a causa della perdita di sangue.

Lo strige può staccarsi spendendo 1,5 metri di movimento. Lo fa automaticamente dopo aver risucchiato 10 punti ferita dal bersaglio o alla morte del bersaglio. Una creatura, compreso il bersaglio, può usare la sua azione per staccare lo strige.

\medskip\textbf{Tasso}\index{Mostri - Tasso}

\emph{Minuscola bestia, disallineato}

\textbf{FORZA} -3

\textbf{DESTREZZA} +0

\textbf{COSTITUZIONE} +1

\textbf{INTELLIGENZA} -4

\textbf{SAGGEZZA} +1

\textbf{CARISMA} -3

\textbf{Iniziativa} +0 -- \textbf{Difesa} 11

\textbf{Punti Ferita} 3 (1d4 + 1)

\textbf{Movimento} 6 m, scavo 1,5 m

\textbf{Tiri Salvezza}: Tempra -3, Riflessi +1, Volontà +1

\textbf{Sensi} visione al buio 9 m

\textbf{Lingue} -

\textbf{Sfida} 0 (10 PE)

\emph{\textbf{Olfatto Affinato.}} Il tasso ha +1d6 alle prove di Saggezza (Consapevolezza) basate sull'olfatto.

\textbf{Azioni}

\emph{\textbf{Morso.} Attacco con Arma da Mischia}: +2 a colpire, portata 1 m, un bersaglio.

\emph{Colpisce:} 1 danno perforante.

\medskip\textbf{Tasso Gigante}\index{Mostri - Tasso Gigante}

\emph{Media bestia, disallineato}

\textbf{FORZA} +1

\textbf{DESTREZZA} +0

\textbf{COSTITUZIONE} +2

\textbf{INTELLIGENZA} -4

\textbf{SAGGEZZA} +1

\textbf{CARISMA} -3

\textbf{Iniziativa} +0 -- \textbf{Difesa} 11

\textbf{Punti Ferita} 13 (2d8 + 4)

\textbf{Movimento} 9 m, scavo 3 m

\textbf{Tiri Salvezza}: Tempra +2, Riflessi +1, Volontà +2

\textbf{Sensi} visione al buio 9 m

\textbf{Lingue} -

\textbf{Sfida} 1/4 (50 PE)

\emph{\textbf{Olfatto Affinato.}} Il tasso ha +1d6 alle prove di Saggezza (Consapevolezza) basate sull'olfatto.

\textbf{Azioni}

\emph{\textbf{Multiattacco.}} Il tasso effettua due attacchi: uno con il morso e uno con gli artigli.

\emph{\textbf{Artigli.} Attacco con Arma da Mischia}: +3 a colpire,  portata 1 m, un bersaglio.

\emph{Colpisce:} 6 (2d4 + 1) danni taglienti.

\emph{\textbf{Morso.} Attacco con Arma da Mischia}: +3 a colpire, portata 1 m, un bersaglio.

\emph{Colpisce:} 4 (1d6 + 1) danni perforanti.

\medskip\textbf{Tigre}\index{Mostri - Tigre}

\emph{Grande bestia, disallineato}

\textbf{FORZA} +3

\textbf{DESTREZZA} +2

\textbf{COSTITUZIONE} +2

\textbf{INTELLIGENZA} -4

\textbf{SAGGEZZA} +1

\textbf{CARISMA} -1

\textbf{Iniziativa} +2 -- \textbf{Difesa} 13

\textbf{Punti Ferita} 37 (5d10 + 10)

\textbf{Movimento} 12 m

\textbf{Tiri Salvezza}: Tempra +4, Riflessi +4, Volontà +2

\textbf{Competenze} Muoversi Silenziosamente / Nascondersi nelle Ombre +6, Consapevolezza +3

\textbf{Sensi} visione al buio 18 m

\textbf{Lingue} -

\textbf{Sfida} 1 (200 PE)

\emph{\textbf{Balzo.}} Se la tigre si muove di almeno 6 metri diretta verso una creatura e la colpisce con un attacco di artiglio durante lo stesso turno, il bersaglio deve riuscire un tiro salvezza di Tempra CD 13 o cadere prono. Se il bersaglio è prono, la tigre può effettuare un attacco di morso contro di esso come azione bonus.

\emph{\textbf{Olfatto Affinato.}} La tigre ha +1d6 alle prove di Saggezza (Consapevolezza) basate sull'olfatto.

\textbf{Azioni}

\emph{\textbf{Artiglio.} Attacco con Arma da Mischia}: +5 a colpire, portata 1 m, un bersaglio.

\emph{Colpisce:} 7 (1d8 + 3) danni taglienti.

\emph{\textbf{Morso.} Attacco con Arma da Mischia}: +5 a colpire, portata 1 m, un bersaglio.

\emph{Colpisce:} 8 (1d10 + 3) danni perforanti.

\medskip\textbf{Tigre dai Denti a Sciabola}\index{Mostri - Tigre dai Denti a Sciabola}

\emph{Grande bestia, disallineato}

\textbf{FORZA} +4

\textbf{DESTREZZA} +2

\textbf{COSTITUZIONE} +2

\textbf{INTELLIGENZA} -4

\textbf{SAGGEZZA} +1

\textbf{CARISMA} -1

\textbf{Iniziativa} +2 -- \textbf{Difesa} 13

\textbf{Punti Ferita} 52 (7d10 + 14)

\textbf{Movimento} 12 m

\textbf{Tiri Salvezza}: Tempra +5, Riflessi +3, Volontà +2

\textbf{Competenze} Muoversi Silenziosamente / Nascondersi nelle Ombre +6, Consapevolezza +3

\textbf{Lingue} -

\textbf{Sfida} 2 (450 PE)

\emph{\textbf{Balzo.}} Se la tigre si muove di almeno 6 metri diretta verso una creatura e la colpisce con un attacco di artiglio durante lo stesso turno, il bersaglio deve riuscire un tiro salvezza di Tempra CD 14 o cadere prono. Se il bersaglio è prono, la tigre può effettuare un attacco di morso contro di esso come azione bonus.

\emph{\textbf{Olfatto Affinato.}} La tigre ha +1d6 alle prove di Saggezza (Consapevolezza) basate sull'olfatto.

\textbf{Azioni}

\emph{\textbf{Artiglio.} Attacco con Arma da Mischia}: +6 a colpire, portata 1 m, un bersaglio.

\emph{Colpisce:} 12 (2d6 + 5) danni taglienti.

\emph{\textbf{Morso.} Attacco con Arma da Mischia}: +6 a colpire, portata 1 m, un bersaglio.

\emph{Colpisce:} 10 (1d10 + 5) danni perforanti.

\medskip\textbf{Topi}\\\index{Mostri - Topi}
\emph{Minuscola fatata}\\
\textbf{Forza}: -1\\
\textbf{Destrezza}: +4\\
\textbf{Costituzione}: +0\\
\textbf{Intelligenza}: +6\\
\textbf{Saggezza}: +2\\
\textbf{Carisma}: +6\\
\textbf{Difesa}: 17 -- \textbf{Iniziativa}: +15\\
\textbf{Punti Ferita}: 4 (1d10 - 1)\\
\textbf{Movimento}: 6 m\\
\textbf{Tiri Salvezza}: Tempra +20, Riflessi +30, Volontà +20 \\
\textbf{Sensi}: Senso tellurico 30 , Scurovisione 9 m, Visione del Vero 30 m\\
\textbf{Lingue}: tutte\\
\textbf{Sfida}: 0 (10 PE)\smallskip\\
\textbf{Immunità}: al danno delle armi con bonus magico inferiore a +6\\
\textbf{Immunità}: a qualsiasi magia la Topi non voglia essere influenzata\\
\emph{\textbf{E' la Topi}} La Topi ha +3d6 (oppure +18) ogni volta che deve tirare dei dadi o contare un valore.
Qualsiasi attacco effettuato dalla Topi e' considerato magico +5\\
\smallskip\textbf{Azioni}\\
\emph{\textbf{Musetto}} ogni creatura a scelta di Topi, entro 30 metri, subisce un Musetto. La creatura viene allontanata di 2d6 metri e subisce 3d6 danni\\
\emph{\textbf{Morso topetto} Attacco con Arma da Mischia}: +26 al colpire, portata 1 m, un bersaglio.\\
\emph{Colpisce:} 6 danno perforante.\\
\emph{\textbf{Graffiotto} fino a 8 Attacchi con Arma da Mischia}: colpisce automaticamente, portata 1 m, fino a 4 bersagli.\\
\emph{Colpisce:} 1 danno perforante, ignora ogni Resistenza, Immunità o protezione.\\


\medskip\textbf{Vespa Gigante}\index{Mostri - Vespa Gigante}

\emph{Media bestia, disallineato}

\textbf{FORZA} +0

\textbf{DESTREZZA} +2

\textbf{COSTITUZIONE} +0

\textbf{INTELLIGENZA} -5

\textbf{SAGGEZZA} +0

\textbf{CARISMA} -4

\textbf{Iniziativa} +2 -- \textbf{Difesa} 13

\textbf{Punti Ferita} 13 (3d8)

\textbf{Movimento} 3 m, volo 15 m

\textbf{Tiri Salvezza}: Tempra +1, Riflessi +3, Volontà +0 

\textbf{Lingue} -

\textbf{Sfida} 1/2 (100 PE)

\textbf{Azioni}

\emph{\textbf{Pungiglione.} Attacco con Arma da Mischia}: +4 a colpire, portata 1 m, una creatura.

\emph{Colpisce:} 5 (1d6 + 2) danni perforanti e il bersaglio deve effettuare un tiro salvezza di Tempra CD 11, e subire 10 (3d6) danni da veleno se fallisce il tiro salvezza, o la metà di questi danni se lo riesce. Se il danno da veleno riduce il bersaglio a 0 punti ferita, il bersaglio è stabile ma avvelenato per 1 ora, anche dopo aver recuperato i punti ferita, e mentre è avvelenato in questo modo resta paralizzato.

\medskip\textbf{Worg}\index{Mostri - Worg}

I worg sono mostruosi predatori dall'aspetto simile ad un lupo che amano cacciare e divorare le creature più deboli di loro.

\emph{Grande mostruosità, neutrale malvagio}

\textbf{FORZA} +3

\textbf{DESTREZZA} +1

\textbf{COSTITUZIONE} +1

\textbf{INTELLIGENZA} -2

\textbf{SAGGEZZA} +0

\textbf{CARISMA} -1

\textbf{Iniziativa} +1 -- \textbf{Difesa} 14

\textbf{Punti Ferita} 26 (4d10 + 4)

\textbf{Movimento} 15 m

\textbf{Tiri Salvezza}: Tempra +3, Riflessi +2, Volontà +2 

\textbf{Competenze} Consapevolezza +4

\textbf{Sensi} visione al buio 18 m

\textbf{Lingue} Goblin, Worg

\textbf{Sfida} 1/2 (100 PE)

\emph{\textbf{Udito e Olfatto Affinato.}} Il worg ha +1d6 nelle prove di Saggezza (Consapevolezza) basate su udito o olfatto.

\textbf{Azioni}

\emph{\textbf{Morso.} Attacco con Arma da Mischia}: +5 a colpire, portata 1 m, un bersaglio.

\emph{Colpisce:} 10 (2d6 + 3) danni perforanti. Se il bersaglio è una creatura, deve riuscire un tiro salvezza di Tempra CD 13 o cadere prona.

\subsection{Appendice B: Personaggi Non Giocanti}\index{Mostri - Personaggi Non Giocanti}

Questa appendice contiene le statistiche di vari personaggi non giocanti (PNG) umanoidi che gli avventurieri possono incontrare nel corso di una campagna, da infimi popolani a potenti arcimaghi. Queste statistiche possono essere utilizzate per rappresentare PNG umani e non.

Personalizzare i PNG

Esistono molti semplici modi di personalizzare i PNG di questa appendice per l'uso nella tua campagna casalinga.

\emph{\textbf{Cambiare Incantesimi.}} Un modo per personalizzare un PNG incantatore è quello di rimpiazzare uno o più dei suoi incantesimi. Puoi sostituire qualsiasi incantesimo della lista di
incantesimi del PNG con un diverso incantesimo della stessa Difficoltà. Cambiare incantesimi in questo modo non modifica il grado di sfida del PNG.

\textbf{\emph{Cambiare Armi e Armatura}.} Puoi migliorare o peggiorare l'armatura del PNG o aggiungere o cambiare armi. Le modifiche alla Difesa e ai danni possono modificare il grado di sfida del PNG.

\emph{\textbf{Oggetti Magici}}. Più potente è un PNG, maggiori le probabilità che possieda uno o più
oggetti magici. Un mago, ad esempio, potrebbe avere una bacchetta o un bastone magico, oltre ad una o più pozioni e pergamene. Fornire un PNG di un potente oggetto magico capace di infliggere danni potrebbe modificarne il grado di sfida.

Alcuni oggetti magici di esempio sono descritti più avanti in questo documento.

\textbf{Combattenti}

I combattenti sono individui che si guadagnano da vivere mettendo la loro spada al servizio di un individuo o un ideale.

\medskip\textbf{Guardia}

Le guardie comprendono membri della ronda cittadina, sentinelle di una cittadella o città fortificata e le guardie del corpo di nobili e mercanti.

\emph{Media umanoide (qualsiasi razza), qualsiasi allineamento} 

\textbf{FORZA} +1

\textbf{DESTREZZA} +1

\textbf{COSTITUZIONE} +1

\textbf{INTELLIGENZA} +0

\textbf{SAGGEZZA} +0

\textbf{CARISMA} +0

\textbf{Iniziativa} +1 -- \textbf{Difesa} 17 (giaco di maglia, scudo)

\textbf{Punti Ferita} 11 (2d8 + 2)

\textbf{Movimento} 9 m

\textbf{Tiri Salvezza}: Tempra +3, Riflessi +1, Volontà +1 

\textbf{Competenze} Consapevolezza +2


\textbf{Lingue} una qualsiasi lingua (di solito il Comune)

\textbf{Sfida} 1/8 (25 PE)

\textbf{Azioni}

\emph{\textbf{Lancia.} Attacco con Arma da Mischia o a Gittata}: +3 a colpire, portata 1 m o gittata 6m, un bersaglio.

\emph{Colpisce:} 4 (1d6 + 1) danni perforanti o 5 (1d8 + 1) danni perforanti se impiegata con due mani per effettuare un attacco da mischia.

\medskip\textbf{Veterano}

Guerrieri sopravvissuti a lungo, guadagnandosi una grande fama di esperti e abili combattenti.

\emph{Media umanoide (qualsiasi razza), qualsiasi allineamento}

\textbf{FORZA} +3

\textbf{DESTREZZA} +1

\textbf{COSTITUZIONE} +2

\textbf{INTELLIGENZA} +0

\textbf{SAGGEZZA} +0

\textbf{CARISMA} +0

\textbf{Iniziativa} +1 -- \textbf{Difesa} 19 (armatura di strisce)

\textbf{Punti Ferita} 58 (9d8 + 18)

\textbf{Movimento} 9 m

\textbf{Tiri Salvezza}: Tempra +4, Riflessi +2, Volontà +3 

\textbf{Competenze} Acrobatica +5, Consapevolezza +2

\textbf{Lingue} una lingua qualsiasi (di solito il Comune)

\textbf{Sfida} 3 (700 PE)

\textbf{Azioni}

\emph{\textbf{Multiattacco.}} Il veterano effettua due attacchi con la spada lunga. Se ha estratto una spada corta, può effettuare anche un attacco con la spada corta.

\emph{\textbf{Spada Lunga.} Attacco con Arma da Mischia}: +5 a colpire, portata 1 m, un bersaglio.

\emph{Colpisce:} 7 (1d8 + 3) danni taglienti, o 8 (1d10 + 3) danni taglienti se usata con due mani.

\emph{\textbf{Spada Corta.} Attacco con Arma da Mischia}: +5 a colpire, portata 1 m, un bersaglio.

\emph{Colpisce:} 6 (1d6 + 3) danni perforanti.

\emph{\textbf{Balestra Pesante.} Attacco con Arma a Gittata}: +3 a colpire, gittata 30m, un bersaglio. \emph{Colpisce:} 6 (1d10 + 1) danni perforanti.

\medskip\textbf{Cavaliere}

I cavalieri sono combattenti che giurano fedeltà a sovrani, ordini religiosi, e nobili cause. L'allineamento del cavaliere determina fino a che punto è disposto ad onorare il suo giuramento.

\emph{Media umanoide (qualsiasi razza), qualsiasi allineamento}

\textbf{FORZA} +3

\textbf{DESTREZZA} +0

\textbf{COSTITUZIONE} +2

\textbf{INTELLIGENZA} +0

\textbf{SAGGEZZA} +0

\textbf{CARISMA} +2

\textbf{Iniziativa} +0 -- \textbf{Difesa} 20 (armatura di piastre)

\textbf{Punti Ferita} 52 (8d8 + 16)

\textbf{Movimento} 9 m

\textbf{Tiri Salvezza}: Tempra +4, Riflessi +1, Volontà +3

\textbf{Lingue} una qualsiasi lingua (di solito il Comune)

\textbf{Sfida} 3 (700 PE)

\emph{\textbf{Coraggioso.}} Il cavaliere ha +1d6 ai tiri salvezza contro l'essere spaventato.

\textbf{Azioni}

\emph{\textbf{Multiattacco.}} Il cavaliere effettua due attacchi da mischia.

\emph{\textbf{Spada Grossa.} Attacco con Arma da Mischia}: +5 a colpire, portata 1 m, un bersaglio.

\emph{Colpisce:} 10 (2d6 + 3) danni taglienti.

\emph{\textbf{Balestra Pesante.} Attacco con Arma a Gittata}: +2 a colpire, gittata 30m, un bersaglio.

\emph{Colpisce:} 5 (1d10) perforanti.

\emph{\textbf{Autorità (Ricarica dopo un 1 ora)}}. Per 1 minuto, il cavaliere può pronunciare un comando speciale o avvertimento ogni qualvolta una creatura non ostile entro 9 metri da lui, e che possa vedere, effettua un tiro di attacco o tiro salvezza. La creatura può sommare un d4 al suo tiro purchè possa udire e comprendere il cavaliere. Una creatura può beneficiare di un solo dado Autorità alla volta. Questo effetto termina se il cavaliere è inabile.

\textbf{Reazioni}

\emph{\textbf{Parata.}} Il cavaliere può aggiungere 2 alla sua Difesa contro un attacco da mischia che lo colpirebbe. Per farlo, il cavaliere deve vedere l'attaccante e star impugnando un'arma da mischia.

\medskip\textbf{Gladiatore}

Addestrati per intrattenere le folle, sono tra i combattenti più pericolosi in circolazione.

\emph{Media umanoide (qualsiasi razza), qualsiasi allineamento}
\textbf{FORZA} +4

\textbf{DESTREZZA} +2

\textbf{COSTITUZIONE} +3

\textbf{INTELLIGENZA} +0

\textbf{SAGGEZZA} +1

\textbf{CARISMA} +2

\textbf{Iniziativa} +2 -- \textbf{Difesa} 19 (armatura di cuoio borchiato, scudo)

\textbf{Punti Ferita} 112 (15d8 + 45)

\textbf{Movimento} 9 m

\textbf{Tiri Salvezza}: Tempra +5, Riflessi +5, Volontà +3 

\textbf{Competenze} Acrobatica +10, Intimidazione +5

\textbf{Lingue} una lingua qualsiasi (di solito il Comune)

\textbf{Sfida} 5 (1.800 PE)

\emph{\textbf{Bruto.}} Un'arma da mischia infligge un dado aggiuntivo di danno

quando un gladiatore colpisce con essa (già incluso nell'attacco).

\emph{\textbf{Coraggioso.}} Il gladiatore ha +1d6 ai tiri salvezza contro l'essere spaventato.

\textbf{Azioni}

\emph{\textbf{Multiattacco.}} Il gladiatore effettua tre attacchi da mischia o due attacchi a gittata.

\emph{\textbf{Lancia.} Attacco con Arma da Mischia o a Gittata}: +7 a colpire, portata 1 m o gittata 6m, un bersaglio.

\emph{Colpisce:} 11 (2d6 + 4) danni perforanti, o 13 (2d8 + 4) danni taglienti se usata con due mani.

\emph{\textbf{Botta di Scudo.} Attacco con Arma da Mischia}: +7 a colpire, portata 1 m, un bersaglio.

\emph{Colpisce:} 9 (2d4 + 4) danni da botta. Se il bersaglio è una creatura di taglia Media o inferiore, deve riuscire un tiro salvezza su Tempra CD 15 o cadere prono.

\textbf{Reazioni}

\emph{\textbf{Parata.}} Il gladiatore somma 3 alla sua Difesa contro un attacco da mischia che lo colpirebbe. Per farlo, il gladiatore deve vedere l'attaccante e impugnare un'arma da mischia.

\medskip\textbf{Cittadini}

In questa categoria rientrano quegli individui che si occupano di mandare avanti il mondo, svolgendo le mansioni necessarie affinché i campi vengano coltivati, le città amministrate, il cibo coltivato e
nuovi territori esplorati.

\medskip\textbf{Nobile}

I nobili comandano sulla popolazione, in virtù di un diritto di nascita o per le ricchezze accumulate. Tra costoro si annoverano anche i cortigiani che affollano le corti dei ricchi e dei potenti.

\emph{Media umanoide (qualsiasi razza), qualsiasi allineamento}

\textbf{FORZA} +0

\textbf{DESTREZZA} +1

\textbf{COSTITUZIONE} +0

\textbf{INTELLIGENZA} +1

\textbf{SAGGEZZA} +2

\textbf{CARISMA} +3

\textbf{Iniziativa} +1 -- \textbf{Difesa} 16 (pettorale)

\textbf{Punti Ferita} 9 (2d8)

\textbf{Movimento} 9 m

\textbf{Tiri Salvezza}: Tempra +1, Riflessi +1, Volontà +2 

\textbf{Competenze} Percepire Emozioni +4, Ingannare +5

\textbf{Lingue} due lingue qualsiasi

\textbf{Sfida} 1/8 (25 PE)

\textbf{Azioni}

\emph{\textbf{Stocco.} Attacco con Arma da Mischia}: +3 a colpire, portata 1 m, un bersaglio.

\emph{Colpisce:} 5 (1d8 + 1) danni perforanti.

\textbf{Reazioni}

\emph{\textbf{Parata.}} Il nobile somma 2 alla sua Difesa contro un attacco da mischia che lo colpirebbe. Per farlo, il nobile deve vedere

l'attaccante e impugnare un'arma da mischia.

\medskip\textbf{Popolano}

I popolani comprendono contadini, servi, schiavi, servitori, pellegrini, mercanti, artigiani ed eremiti.

\emph{Media umanoide (qualsiasi razza), qualsiasi allineamento}

\textbf{FORZA} +0

\textbf{DESTREZZA} +0

\textbf{COSTITUZIONE} +0

\textbf{INTELLIGENZA} +0

\textbf{SAGGEZZA} +0

\textbf{CARISMA} +0

\textbf{Iniziativa} +0 -- \textbf{Difesa} 11

\textbf{Punti Ferita} 4 (1d8)

\textbf{Movimento} 9 m

\textbf{Tiri Salvezza}: Tempra +0, Riflessi +0, Volontà +0 

\textbf{Lingue} una qualsiasi lingua (di solito il Comune)

\textbf{Sfida} 0 (10 PE)

\textbf{Azioni}

\emph{\textbf{Randello.} Attacco con Arma da Mischia}: +2 a colpire, portata 1 m, un bersaglio.

\emph{Colpisce:} 2 (1d4) danni da botta.

\medskip\textbf{Criminali}

I criminali sono individui che vivono al margine della legalità, procurandosi il pane svolgendo attività spesso considerate illecite e immorali.

\medskip\textbf{Picchiatore}

I picchiatori sono criminali spietati abili nell'intimidire e perpetrare atti di violenza. Lavorano per soldi e si fanno pochi scrupoli.

\emph{Media umanoide (qualsiasi razza), qualsiasi allineamento non buono}

\textbf{FORZA} +2

\textbf{DESTREZZA} +0

\textbf{COSTITUZIONE} +2

\textbf{INTELLIGENZA} +0

\textbf{SAGGEZZA} +0

\textbf{CARISMA} +0

\textbf{Iniziativa} +0 -- \textbf{Difesa} 12 (armatura di cuoio)

\textbf{Punti Ferita} 32 (5d8 + 10)

\textbf{Movimento} 9 m

\textbf{Tiri Salvezza}: Tempra +3, Riflessi +1, Volontà +0 

\textbf{Competenze} Intimidazione +2

\textbf{Lingue} una lingua qualsiasi (di solito il Comune)

\textbf{Sfida} 1/2 (100 PE)

\emph{\textbf{Tattiche di Branco.}} Il picchiatore ha +1d6 ai tiri di attacco contro una creatura se almeno uno degli alleati del picchiatore si trova entro 1,5 metri dalla creatura e quell'alleato non
è inabile.

\textbf{Azioni}

\emph{\textbf{Multiattacco.}} Il picchiatore effettua due attacchi da mischia.

\emph{\textbf{Mazza.} Attacco con Arma da Mischia}: +4 a colpire, portata 1 m, una creatura.

\emph{Colpisce:} 5 (1d6 + 2) danni da botta.

\emph{\textbf{Balestra Pesante.} Attacco con Arma a Gittata}: +2 a colpire, gittata 30m, un bersaglio. \emph{Colpisce:} 5 (1d10) danni perforanti.

\medskip\textbf{Bandito/Pirata}

Che siano uomini di strada o di mare (pirati) costoro guadagnano da vivere depredando il prossimo.

\emph{Media umanoide (qualsiasi razza), qualsiasi allineamento non legale}

\textbf{FORZA} +0

\textbf{DESTREZZA} +1

\textbf{COSTITUZIONE} +1

\textbf{INTELLIGENZA} +0

\textbf{SAGGEZZA} +0

\textbf{CARISMA} +0

\textbf{Iniziativa} +1 -- \textbf{Difesa} 13 (armatura di cuoio)

\textbf{Punti Ferita} 11 (2d8 + 2)

\textbf{Movimento} 9 m

\textbf{Tiri Salvezza}: Tempra +1, Riflessi +2, Volontà +1 

\textbf{Lingue} una qualsiasi lingua (di solito il Comune)

\textbf{Sfida} 1/8 (25 PE)

\textbf{Azioni}

\emph{\textbf{Scimitarra.} Attacco con Arma da Mischia}: +3 a colpire, portata 1 m, un bersaglio.

\emph{Colpisce:} 4 (1d6 + 1) danni taglienti.

\emph{\textbf{Balestra Leggera.} Attacco con Arma a Gittata}: +3 a colpire, gittata 24m, un bersaglio. \emph{Colpisce:} 5 (1d8 + 1) danni taglienti.

\medskip\textbf{Spia}

Una spia è un individuo addestramento nel reperire segreti per conto di qualcuno, o a volte per rivenderli al miglior offerente.

\emph{Media umanoide (qualsiasi razza), qualsiasi allineamento}

\textbf{FORZA} +0

\textbf{DESTREZZA} +2

\textbf{COSTITUZIONE} +0

\textbf{INTELLIGENZA} +1

\textbf{SAGGEZZA} +2

\textbf{CARISMA} +3

\textbf{Iniziativa} +2 -- \textbf{Difesa} 13

\textbf{Punti Ferita} 27 (6d8)

\textbf{Movimento} 9 m

\textbf{Tiri Salvezza}: Tempra +2, Riflessi +3, Volontà +3 

\textbf{Competenze} Muoversi Silenziosamente / Nascondersi nelle Ombre +4, Percepire Emozioni +4, Investigazione +5, Consapevolezza +6, Ingannare +5, Mani di fata +4

\textbf{Lingue} due lingue qualsiasi

\textbf{Sfida} 1 (200 PE)

\emph{\textbf{Attacco Furtivo (1/Turno).}} La spia infligge 7 (2d6) danni aggiuntivi quando colpisce un bersaglio con un attacco con arma e ha +1d6 al tiro di attacco, o quando il bersaglio è entro 1,5 metri da un alleato dell'assassino che non è inabile e l'assassino non ha -1d6 al tiro di attacco.

\emph{\textbf{Azione Astuta.}} Durante ciascun suo turno, la spia può usare un'azione bonus per effettuare l'azione Ritirarsi, Nascondersi o Scattare.

\textbf{Azioni}

\emph{\textbf{Multiattacco.}} La spia effettua due attacchi da mischia.

\emph{\textbf{Spada Corta.} Attacco con Arma da Mischia}: +4 a colpire, portata 1 m, un bersaglio.

\emph{Colpisce:} 5 (1d6 + 2) danni perforanti.

\emph{\textbf{Balestrino.} Attacco con Arma a Gittata}: +4 a colpire, gittata 9m, un bersaglio. \emph{Colpisce:} 5 (1d6 + 2) danni perforanti.


\medskip\textbf{Capitano dei Banditi/Pirata}

Che viva in terra o in mare, è un individuo munito di una grande personalità che riesce a tenere in riga la marmaglia che risponde ai suoi ordini.

\emph{Media umanoide (qualsiasi razza), qualsiasi allineamento non legale}

\textbf{FORZA} +2

\textbf{DESTREZZA} +3

\textbf{COSTITUZIONE} +2

\textbf{INTELLIGENZA} +2

\textbf{SAGGEZZA} +0

\textbf{CARISMA} +2

\textbf{Iniziativa} +2 -- \textbf{Difesa} 16 (armatura di cuoio borchiato)

\textbf{Punti Ferita} 65 (10d8 + 8)

\textbf{Movimento} 9 m

\textbf{Tiri Salvezza}: Tempra +5, Riflessi +5, Volontà +3 

\textbf{Competenze} Acrobatica +4, Raggiro +4 

\textbf{Lingue} due lingue qualsiasi

\textbf{Sfida} 2 (450 PE)

\textbf{Azioni}

\emph{\textbf{Multiattacco.}} Il capitano effettua tre attacchi da mischia: due con la scimitarra e uno con il pugnale. Oppure il capitano effettua due attacchi a gittata con i pugnali.

\emph{\textbf{Scimitarra.} Attacco con Arma da Mischia}: +5 a colpire, portata 1 m, un bersaglio.

\emph{Colpisce:} 6 (1d6 + 3) danni taglienti.

\emph{\textbf{Pugnale.} Attacco con Arma da Mischia o a Gittata}: +5 a colpire, portata 1 m o gittata 6m, un bersaglio. \emph{Colpisce:} 5 (1d4 + 3) danni perforanti.

\textbf{Reazioni}

\emph{\textbf{Parata.}} Il capitano somma 2 alla sua Difesa contro un attacco da mischia che lo colpirebbe. Per farlo, il capitano deve vedere l'attaccante e impugnare un'arma da mischia.

\medskip\textbf{Assassino}

Solitari o membri di una gilda, gli assassini sono pagati per eliminare, spesso in modo silenzioso e discreto, rivali e nemici dei loro datori di lavoro.

\emph{Media umanoide (qualsiasi razza), qualsiasi allineamento non buono}

\textbf{FORZA} +0

\textbf{DESTREZZA} +3

\textbf{COSTITUZIONE} +2

\textbf{INTELLIGENZA} +1

\textbf{SAGGEZZA} +0

\textbf{CARISMA} +0

\textbf{Iniziativa} +3 -- \textbf{Difesa} 19 (armatura di cuoio borchiato)

\textbf{Punti Ferita} 78 (12d8 + 24)

\textbf{Movimento} 9 m

\textbf{Tiri Salvezza}: Tempra +4, Riflessi +6, Volontà +3 

\textbf{Competenze} Acrobazia +6, Muoversi Silenziosamente / Nascondersi nelle Ombre +9, Consapevolezza +3, Raggiro +3


\textbf{Lingue} Gergo dei Ladri più due altre lingue

\textbf{Sfida} 8 (3.900 PE)

\emph{\textbf{Assassinare.}} Durante il suo primo turno, l'assassino ha +1d6 ai tiri di attacco contro le creature che non hanno ancora svolto nessun turno. Qualsiasi colpo che l'assassino mandi a segno contro una creatura sorpresa, è un colpo critico.

\emph{\textbf{Attacco Furtivo (1/Turno).}} L'assassino infligge 14 (4d6) danni aggiuntivi quando colpisce un bersaglio con un attacco con arma e ha +1d6 al tiro di attacco, o quando il bersaglio è entro 1,5 metri da un alleato dell'assassino che non è inabile e l'assassino non ha -1d6 al tiro di attacco.

\emph{\textbf{Evasione.}} Se l'assassino è vittima di un effetto che permette di effettuare un tiro salvezza di Riflessi per dimezzare i danni, l'assassino non prende danni se riesce il tiro salvezza, e solo la metà se lo fallisce.

\textbf{Azioni}

\emph{\textbf{Multiattacco.}} L'assassino effettua due attacchi con le spade corte.

\emph{\textbf{Spada Corta.} Attacco con Arma da Mischia}: +6 a colpire, portata 1 m, un bersaglio.

\emph{Colpisce:} 6 (1d6 + 3) danni perforanti, e il bersaglio deve effettuare un tiro salvezza di Tempra CD 15, subendo 24 (7d6) danni da veleno se fallisce il tiro salvezza, o la metà di questi danni se lo riesce.

\emph{\textbf{Balestra Leggera.} Attacco con Arma a Gittata}: +6 a colpire, gittata 24m, un bersaglio.

\emph{Colpisce:} 7 (1d8 + 3) danni perforanti, e il bersaglio deve effettuare un tiro salvezza di Tempra CD 15, subendo 24 (7d6) danni da veleno se fallisce il tiro salvezza, o la metà di questi danni se lo riesce.

\medskip\textbf{Mago}

Il mago trascorrono la vita nello studio e la pratica della magia.

\textbf{VARIANTE: FAMIGLI}

Qualsiasi incantatore che possa eseguire l'incantesimo \emph{trovare} \emph{famiglio} è probabile che abbia un famiglio. Il famiglio può essere una delle creature descritte nell'incantesimo (vedi le \emph{Regole Base}) o qualche altro mostro Minuscolo, come un artiglio strisciante, un diavoletto, uno pseudodrago o un demonietto.

\medskip\textbf{Mago Avventuriero}

Un Mago novizio, che ha superato con successo le sue prime avventure e ha iniziato a stabilire una reputazione come nobile o famigerato avventuriero.

\emph{Media umanoide (qualsiasi razza), qualsiasi malvagio}

\textbf{FORZA} -1

\textbf{DESTREZZA} +2

\textbf{COSTITUZIONE} +0

\textbf{INTELLIGENZA} +3

\textbf{SAGGEZZA} +1

\textbf{CARISMA} +0

\textbf{Iniziativa} +3 -- \textbf{Difesa} 13

\textbf{Punti Ferita} 22 (5d8)

\textbf{Movimento} 9 m

\textbf{Tiri Salvezza}: Tempra +0, Riflessi +3, Volontà +2 

\textbf{Competenze} Arcano +5, Storia +5

\textbf{Lingue} quattro lingue qualsiasi

\textbf{Sfida} 1 (200 PE)

\emph{\textbf{Incantesimi.}} Il mago ha CM 4. La sua abilità da incantatore è l'Intelligenza (+5 al colpire con attacchi con incantesimo). Il Mago ha preparato i seguenti incantesimi: Trucchetti (a volontà): 

\emph{luce, mano magica, stretta folgorante}

Difficoltà 10 (4 slot): \emph{charme su persone, dardo incantato}

Difficoltà 13 (3 slot): \emph{bloccare persona, passo velato}

\textbf{Azioni}

\emph{\textbf{Bastone.} Attacco con Arma da Mischia}: +1 a colpire, portata 1 m, un bersaglio.

\emph{Colpisce:} 3 (1d8 - 1) danni da botta.

\medskip\textbf{Grande Mago}

Un Mago che ha stabilito una discreta fama nel territorio e che attira intorno a sé studenti da ogni dove.

\emph{Media umanoide (qualsiasi razza), qualsiasi allineamento}

\textbf{FORZA} -1

\textbf{DESTREZZA} +2

\textbf{COSTITUZIONE} +0

\textbf{INTELLIGENZA} +3

\textbf{SAGGEZZA} +1

\textbf{CARISMA} +0

\textbf{Iniziativa} +3 -- \textbf{Difesa} 15 (18 con \emph{armatura del Mago})

\textbf{Punti Ferita} 40 (9d8)

\textbf{Movimento} 9 m

\textbf{Tiri Salvezza}: Tempra +1, Riflessi +4, Volontà +3 

\textbf{Competenze} Arcano +6, Storia +6

\textbf{Lingue} quattro lingue qualsiasi

\textbf{Sfida} 6 (2.300 PE)

\emph{\textbf{Incantesimi.}} Il mago ha CM 9. La sua abilità da incantatore è l'Intelligenza (+6 al colpire con attacchi con incantesimo). Il Mago ha preparato i seguenti incantesimi:

Trucchetti (a volontà): \emph{dardo infuocato, luce, mano magica,}
\emph{prestidigitazione}

Difficoltà 10 (4 slot): \emph{armatura del Mago, dardo incantato,}
\emph{individuare magia, scudo}

Difficoltà 13 (3 slot): \emph{passo velato, suggestione}

Difficoltà 15 (3 slot): \emph{controincantesimo, palla di fuoco, volare}

Difficoltà 18 (3 slot): \emph{invisibilità superiore, tempesta di ghiaccio}

Difficoltà 20 (1 slot): \emph{cono di freddo}

\textbf{Azioni}

\emph{\textbf{Pugnale.} Attacco con Arma da Mischia o a Gittata}: +5 a colpire, portata 1 m o gittata 6m, un bersaglio. \emph{Colpisce:} 4 (1d4 + 2) danni perforanti.

\medskip\textbf{ArciMago}

Un mago molto potente (e anche molto anziano) che studia i segreti del multiverso.

\emph{Media umanoide (qualsiasi razza), qualsiasi allineamento}

\textbf{FORZA} +0

\textbf{DESTREZZA} +2

\textbf{COSTITUZIONE} +1

\textbf{INTELLIGENZA} +5

\textbf{SAGGEZZA} +2

\textbf{CARISMA} +3

\textbf{Iniziativa} +5 -- \textbf{Difesa} 18 (21 con \emph{armatura del Mago})

\textbf{Punti Ferita} 99 (18d8 + 18)

\textbf{Movimento} 9 m

\textbf{Tiri Salvezza}: Tempra +8, Riflessi +10, Volontà +12 

\textbf{Competenze} Arcano +13, Storia +13

\textbf{Resistenze al Danno} danno degli incantesimi; da botta, perforante e tagliente non magico (da \emph{pelle di pietra})

\textbf{Lingue} sei lingue qualsiasi

\textbf{Sfida} 12 (8.400 PE)

\emph{\textbf{Incantesimi.}} Il mago ha CM 18. La sua abilità da incantatore è l'Intelligenza (+9 al colpire con attacchi con incantesimo).

L'arciMago può eseguire \emph{camuffare sé stesso} e \emph{invisibilità} a volontà e ha preparato i seguenti incantesimi: Trucchetti (a volontà): \emph{dardo infuocato, luce, mano magica,}
\emph{prestidigitazione, stretta folgorante}

Difficoltà 10 (4 slot): \emph{armatura magica*, dardo incantato,}
\emph{identificare, individuare magia}

Difficoltà 13 (3 slot): \emph{immagine speculare, individuazione dei}
\emph{pensieri, passo velato}

Difficoltà 15 (3 slot): \emph{controincantesimo, fulmine}

Difficoltà 18 (3 slot): \emph{esilio, pelle di pietra*, scudo di fuoco}

Difficoltà 20 (3 slot): \emph{cono di freddo, muro di forza, scrutare}

Difficoltà 23 (1 slot): \emph{globo di invulnerabilità}

Difficoltà 25 (1 slot): \emph{teletrasporto}

Difficoltà 28 (1 slot): \emph{vuoto mentale*}

Difficoltà 30 (1 slot): \emph{fermare il tempo}

L'arciMago esegue questi incantesimi su di sé prima del combattimento.

\textbf{Azioni}

\emph{\textbf{Pugnale.} Attacco con Arma da Mischia o a Gittata}: +6 a colpire, portata 1 m o gittata 6m, un bersaglio. \emph{Colpisce:} 4 (1d4 + 2) danni perforanti.


\medskip\textbf{Sacerdoti}

I sacerdoti sono devoti di una divinità o una fede che si prendono cura di impartire gli insegnamenti divini al loro gregge.

\medskip\textbf{Cultista}

I cultisti giurano fedeltà ai poteri oscuri, e nelle loro credenze e pratiche mostrano spesso segni di follia.

\emph{Media umanoide (qualsiasi razza), qualsiasi allineamento non buono}

\textbf{FORZA} +0

\textbf{DESTREZZA} +1

\textbf{COSTITUZIONE} +0

\textbf{INTELLIGENZA} +0

\textbf{SAGGEZZA} +0

\textbf{CARISMA} +0

\textbf{Iniziativa} +0- \textbf{Difesa} 13 (armatura di cuoio)

\textbf{Punti Ferita} 9 (2d8)

\textbf{Movimento} 9 m

\textbf{Tiri Salvezza}: Tempra +1, Riflessi +1, Volontà +2 

\textbf{Competenze} Raggiro +2, Religione +2

\textbf{Lingue} una qualsiasi lingua (di solito il Comune)

\textbf{Sfida} 1/8 (25 PE)

\emph{\textbf{Oscura Devozione.}} Il cultista ha +1d6 sui tiri salvezza contro l'essere affascinato o spaventato.

\textbf{Azioni}

\emph{\textbf{Scimitarra.} Attacco con Arma da Mischia}: +3 a colpire, portata 1 m, una creatura.

\emph{Colpisce:} 4 (1d6 + 1) danni taglienti.

\medskip\textbf{Accolito}

Gli accoliti sono membri di grado minore del clero, e di solito rispondono ad un sacerdote di rango superiore. Svolgono diverse funzioni in un tempio e gli viene conferita dalla loro divinità l'abilità di eseguire incantesimi minori.

\emph{Media umanoide (qualsiasi razza), qualsiasi allineamento}

\textbf{FORZA} +0

\textbf{DESTREZZA} +0

\textbf{COSTITUZIONE} +0

\textbf{INTELLIGENZA} +0

\textbf{SAGGEZZA} +2

\textbf{CARISMA} +0

\textbf{Iniziativa} +0 -- \textbf{Difesa} 11

\textbf{Punti Ferita} 9 (2d8)

\textbf{Movimento} 9 m

\textbf{Tiri Salvezza}: Tempra +0, Riflessi +0, Volontà +3 

\textbf{Competenze} Pronto Soccorso +4, Religione +2

\textbf{Lingue} una qualsiasi lingua (di solito il Comune)

\textbf{Sfida} 1/4 (50 PE)

\emph{\textbf{Incantesimi.}} L'accolito ha CM 1. La sua abilità da incantatore è la Saggezza (+4 al colpire con attacchi con incantesimo). L'accolito ha preparato i seguenti incantesimi: Trucchetti (a volontà): \emph{fiamma sacra, luce, taumaturgia} Difficoltà 10 (3 slot): \emph{benedizione}, \emph{cura ferite, santuario}

\medskip\textbf{Azioni}

\emph{\textbf{Randello.} Attacco con Arma da Mischia}: +2 a colpire, portata 1 m, un bersaglio.

\emph{Colpisce:} 2 (1d4) danni da botta.

\textbf{Fanatico del Culto}

Sono i capi di un culto, che usano il proprio carisma e i propri dogmi per influenzare i deboli di volontà.

\emph{Media umanoide (qualsiasi razza), qualsiasi allineamento non buono}

\textbf{FORZA} +0

\textbf{DESTREZZA} +2

\textbf{COSTITUZIONE} +1

\textbf{INTELLIGENZA} +0

\textbf{SAGGEZZA} +1

\textbf{CARISMA} +2

\textbf{Iniziativa} +2 -- \textbf{Difesa} 14 (armatura di cuoio)

\textbf{Punti Ferita} 33 (6d8 + 6)

\textbf{Movimento} 9 m

\textbf{Tiri Salvezza}: Tempra +2, Riflessi +2, Volontà +3 

\textbf{Competenze} Ingannare +4, Raggiro +4, Religione +2 

\textbf{Lingue} una qualsiasi lingua (di solito il Comune)

\textbf{Sfida} 2 (450 PE)

\emph{\textbf{Incantesimi.}} Il sacerdote ha CM 4. La sua abilità da incantatore è la Saggezza (+3 al colpire con attacchi con incantesimo). Il sacerdote ha preparato i seguenti incantesimi: Trucchetti (a volontà): \emph{fiamma sacra, luce, taumaturgia}

Difficoltà 10 (4 slot): \emph{comando, infliggi ferite, scudo della fede}

Difficoltà 13 (3 slot): \emph{arma spirituale, blocca persona}

\emph{\textbf{Oscura Devozione.}} Il cultista ha +1d6 sui tiri salvezza contro l'essere affascinato o spaventato.

\textbf{Azioni}

\emph{\textbf{Multiattacco.}} Il fanatico effettua due attacchi da mischia.

\emph{\textbf{Pugnale.} Attacco con Arma da Mischia o a Gittata}: +4 a colpire, portata 1 m o gittata 6m, una creatura. \emph{Colpisce:} 4 (1d4 + 2) danni perforanti.

\medskip\textbf{Gran Sacerdote}

Sono individui al comando di un tempio o altro luogo sacro e che hanno a loro disposizione diversi accoliti.

\emph{Media umanoide (qualsiasi razza), qualsiasi allineamento}

\textbf{FORZA} +0

\textbf{DESTREZZA} +0

\textbf{COSTITUZIONE} +1

\textbf{INTELLIGENZA} +1

\textbf{SAGGEZZA} +3

\textbf{CARISMA} +1

\textbf{Iniziativa} +1 -- \textbf{Difesa} 14 (giaco di maglia)

\textbf{Punti Ferita} 27 (5d8 + 5)

\textbf{Movimento} 7,5 m

\textbf{Tiri Salvezza}: Tempra +1, Riflessi +1, Volontà +4 

\textbf{Competenze} Pronto Soccorso +7, Ingannare +3, Religione +4

\textbf{Lingue} due lingue qualsiasi

\textbf{Sfida} 2 (450 PE)

\emph{\textbf{Eminenza Divina.}} Come azione bonus, il sacerdote può spendere uno slot incantesimo per far sì che il suo attacco con arma da mischia infligge 10 (3d6) danni da Luce aggiuntivi. Il beneficio dura fino al termine del turno.

\emph{\textbf{Incantesimi.}} Il sacerdote ha CM 5. La sua abilità da incantatore è la Saggezza (+5 al colpire con attacchi con incantesimo). Il sacerdote ha preparato i seguenti incantesimi: Trucchetti (a volontà): \emph{fiamma sacra, luce, taumaturgia}

Difficoltà 10 (4 slot): \emph{cura ferite, dardo tracciante, santuario}

Difficoltà 13 (3 slot): \emph{arma spirituale, ristorare inferiore}

Difficoltà 15 (2 slot): \emph{dissolvi magie}, \emph{guardiani spirituali}

\textbf{Azioni}

\emph{\textbf{Mazza.} Attacco con Arma da Mischia}: +2 a colpire, portata 1 m, un bersaglio.

\emph{Colpisce:} 3 (1d6) danni da botta.


\medskip\textbf{Selvaggi}

Questi individui vivono ai margini della civiltà, a volte entrandovi raramente in contatto. A disagio tra le mura e nelle terre civilizzate, si trovano nel loro ambiente quando possono muoversi tra le terre selvagge.

\medskip\textbf{Berserker}

Provenienti da terre selvagge, gli imprevedibili berserker si radunano in compagnie di guerra e sono sempre alla ricerca di conflitti in cui combattere.

\emph{Media umanoide (qualsiasi razza), qualsiasi allineamento caotico}

\textbf{FORZA} +3

\textbf{DESTREZZA} +1

\textbf{COSTITUZIONE} +3

\textbf{INTELLIGENZA} -1

\textbf{SAGGEZZA} +0

\textbf{CARISMA} -1

\textbf{Iniziativa} +1 -- \textbf{Difesa} 14 (armatura di pelle)

\textbf{Punti Ferita} 67 (9d8 + 27)

\textbf{Movimento} 9 m

\textbf{Tiri Salvezza}: Tempra +4, Riflessi +3, Volontà +2 

\textbf{Lingue} una qualsiasi lingua (di solito il Comune)

\textbf{Sfida} 2 (450 PE)

\emph{\textbf{Incauto.}} All'inizio del suo turno, il berserker può ottenere +1d6 su tutti i tiri di attacco con armi da mischia effettuati durante quel turno, ma i tiri di attacco contro di esso hanno
+1d6 fino all'inizio del suo prossimo turno.

\textbf{Azioni}

\emph{\textbf{Ascia Grossa.} Attacco con Arma da Mischia}: +5 a colpire, portata 1 m, un bersaglio.

\emph{Colpisce:} 9 (1d12 + 3) danni taglienti.

\textbf{Combattente Tribale}

Sono i difensori delle tribù che vivono ai margini della civiltà.

\emph{Media umanoide (qualsiasi razza), qualsiasi allineamento}

\textbf{FORZA} +1

\textbf{DESTREZZA} +0

\textbf{COSTITUZIONE} +1

\textbf{INTELLIGENZA} -1

\textbf{SAGGEZZA} +0

\textbf{CARISMA} -1

\textbf{Iniziativa} +0 -- \textbf{Difesa} 13 (armatura di pelle)

\textbf{Punti Ferita} 11 (2d8 + 2)

\textbf{Movimento} 9 m

\textbf{Tiri Salvezza}: Tempra +2, Riflessi +1, Volontà +1 

\textbf{Lingue} una qualsiasi lingua

\textbf{Sfida} 1/8 (25 PE)

\emph{\textbf{Tattiche di Branco.}} Il combattente tribale ha +1d6 ai tiri di attacco contro una creatura se almeno uno degli alleati del picchiatore si trova entro 1,5 metri dalla creatura e quell'alleato non è inabile.

\textbf{Azioni}

\emph{\textbf{Lancia.} Attacco con Arma da Mischia o a Gittata}: +3 a colpire, portata 1 m o gittata 6m, un bersaglio.

\emph{Colpisce:} 4 (1d6 + 1) danni perforanti, o 5 (1d8 + 1) danni perforanti se usata con due mani per effettuare un attacco da mischia.

\medskip\textbf{Druido}

I druidi proteggono il mondo naturale dai mostri e dall'avanzare della civiltà. Alcuni sono sciamani tribali che curano i malati, pregano agli spiriti animali e forniscono consigli spirituali.

\emph{Media umanoide (qualsiasi razza), qualsiasi allineamento}

\textbf{FORZA} +0

\textbf{DESTREZZA} +1

\textbf{COSTITUZIONE} +1

\textbf{INTELLIGENZA} +1

\textbf{SAGGEZZA} +2

\textbf{CARISMA} +0

\textbf{Iniziativa} +1 -- \textbf{Difesa} 12 (17 con \emph{pelle di corteccia}*)

\textbf{Punti Ferita} 27 (5d8 + 5)

\textbf{Movimento} 9 m

\textbf{Tiri Salvezza}: Tempra +1, Riflessi +2, Volontà +3 \\

\textbf{Competenze} Pronto Soccorso +4, Natura +3, Consapevolezza +4 

\textbf{Lingue} Druidico più due altre lingue

\textbf{Sfida} 2 (450 PE)

\emph{\textbf{Incantesimi.}} Il sacerdote ha CM 4. La sua abilità da incantatore è la Saggezza (+4 al colpire con attacchi con incantesimo). Il sacerdote ha preparato i seguenti incantesimi: Trucchetti (a volontà): \emph{arte druidica, bastone, produrre fiamma}

Difficoltà 10 (4 slot): \emph{intralciare, onda tonante, parlare con gli}
\emph{animali, passo veloce}

Difficoltà 13 (3 slot): \emph{animale messaggero, pelle di corteccia}

\textbf{Azioni}

\emph{\textbf{Bastone da Combattimento.} Attacco con Arma da Mischia}: +2 a colpire (+4 a colpire con \emph{bastone*}), portata 1 m o gittata 6m, un bersaglio.

\emph{Colpisce:} 3 (1d6) danni da botta, o 6 (1d8 + 2) danni da botta con \emph{bastone} o se impugnato con due mani.

\medskip\textbf{Esploratore}

Abili cacciatori e battitori di piste.

\emph{Media umanoide (qualsiasi razza), qualsiasi allineamento}

\textbf{FORZA} +0

\textbf{DESTREZZA} +2

\textbf{COSTITUZIONE} +1

\textbf{INTELLIGENZA} +0

\textbf{SAGGEZZA} +1

\textbf{CARISMA} +0

\textbf{Iniziativa} +2 -- \textbf{Difesa} 14 (armatura di cuoio)

\textbf{Punti Ferita} 16 (3d8 + 3)

\textbf{Movimento} 9 m

\textbf{Tiri Salvezza}: Tempra +1, Riflessi +2, Volontà +3

\textbf{Competenze} Muoversi Silenziosamente / Nascondersi nelle Ombre +6, Natura +4, Consapevolezza +5, Sopravvivenza +5

\textbf{Lingue} una qualsiasi lingua (di solito Comune)

\textbf{Sfida} 1/2 (100 PE)

\emph{\textbf{Olfatto e Vista Affinati.}} L'esploratore ha +1d6 nelle prove di Saggezza (Consapevolezza) basate su olfatto o vista.

\textbf{Azioni}

\emph{\textbf{Multiattacco.}} L'esploratore effettua due attacchi da mischia o due attacchi a gittata.

\emph{\textbf{Spada Corta.} Attacco con Arma da Mischia}: +4 a colpire, portata 1 m, un bersaglio.

\emph{Colpisce:} 5 (1d6 + 2) danni perforanti.

\emph{\textbf{Arco Lungo.} Attacco con Arma da Mischia}: +4 a colpire, gittata 45m, un bersaglio.

\emph{Colpisce:} 6 (1d8 + 2) danni perforanti.


\end{multicols}

%{\scriptsize 
%	\printindex}
%\end{document}