\documentclass[a4paper,10 pt,twoside,openany]{book}
%\documentclass[a4paper,10 pt,draft,twoside,openany]{book}
\usepackage{quoting}
\usepackage{tcolorbox}
\usepackage{tikz}
\usetikzlibrary{shadows}
\usepackage{multicol}
\usepackage{tocloft}
\usepackage{lmodern}
\usepackage{caption}
\usepackage[utf8]{inputenc}
%\usepackage[utf8x]{inputenc}
\usepackage[T1]{fontenc}
\usepackage{setspace}
\usepackage[a4paper]{geometry}
\geometry{verbose,tmargin=2cm,bmargin=2.5cm,lmargin=1.5cm,rmargin=2cm}
\setcounter{secnumdepth}{-1}
\usepackage{booktabs}
\usepackage{url}
\usepackage[italian]{babel}
\usepackage{setspace}
\usepackage{graphicx}

%\usepackage[allfiguresdraft]{draftfigure}  %senza figure, deve rimanere alla riga 24

\usepackage[export]{adjustbox}
\usepackage{float}
\PassOptionsToPackage{normalem}{ulem}
\usepackage{ulem}
\usepackage{amsmath}
\usepackage{makeidx}
\usepackage{multirow}
\usepackage{titlesec}
\usepackage{textcomp}
\usepackage{background}
\usepackage[unicode=true, bookmarks=true,
pdftitle={DBS - Dungeon Bell System},pdfauthor={Andres Zanzani},
breaklinks=false,pdfborder={0 0 1},backref=section,colorlinks=false]
{hyperref}
\hypersetup{colorlinks=true,linkcolor=blue,pdfcreator={LaTeX}}
\usepackage{bookmark}
%\usepackage{tgbonum}
%\usepackage{baskervald}
\usepackage{yfonts}
\usepackage{lettrine}
\usepackage{calligra}
\renewcommand{\LettrineFontHook}{\calligra}
%\newenvironment{spell}{\fontfamily{Bookinsanity}\selectfont}{\par}
%\newcommand{\spell}{\fontfamily{Bookinsanity}\selectfont}


\usepackage{accanthis}
\usepackage{palatino}
\usepackage{wrapfig}
\usepackage{fancyhdr}
\usepackage{tcolorbox}
\tcbuselibrary{skins}
\tcbset{colback=brown!10, fonttitle=\scshape}
\usepackage{imakeidx}
\usepackage{cancel}
\makeindex[columns=3, title=Indice Analitico, intoc=true]

\fancyhf{} % clear all header and footers
\renewcommand{\headrulewidth}{0pt} % remove the header rule
\pagestyle{fancy}
\fancyfoot[C]{\thepage}

%\usepackage{xltabular}
\usepackage{tabularx}
%\usepackage{longtable}
\usepackage{pdfpages}
\usepackage{hyperref}
\usepackage{tikz}

%\usepackage{tabulary}

%\usetikzlibrary{backgrounds,calc}
%\pgfdeclarelayer{background}
%\pgfdeclarelayer{foreground}
%\pgfsetlayers{background,main,foreground}  
%\pdfminorversion=5
%%\makeatletter
%%\makeatother
\raggedbottom

%textfrak

\usepackage{array}
\newcolumntype{L}[1]{>{\raggedright\let\newline\\\arraybackslash\hspace{0pt}}m{#1}}
\newcolumntype{k}[1]{>{\centering\let\newline\\\arraybackslash\hspace{0pt}}m{#1}}
\newcolumntype{R}[1]{>{\raggedleft\let\newline\\\arraybackslash\hspace{0pt}}m{#1}}
\newcolumntype{D}[1]{>{\centering}m{#1}}


%\titleformat{\section}{\huge\bfseries\accanthis}{\thesection}{1em}\textsc{}
\titleformat{\section}{\huge\bfseries\accanthis}{\thesection}{1em}\textfrak{}
\titleformat{\subsection}{\Large\bfseries\accanthis}{\thesubsection}{1em}\textsc{}
\titleformat{\subsubsection}{\normalsize\bfseries\accanthis}{\thesubsubsection}{1em}\textsc{}


%\titleformat{\subsection}{\large\bfseries\accanthis}{\thesubsection}{1em}\textsc{}
%\titleformat{\subsubsection}{\normalsize\bfseries\accanthis}{\thesubsubsection}{1em}\textsc{}

\setcounter{tocdepth}{3}

\newtcolorbox{narratore}{
	enhanced, % enable advanced settings
%	left = 3mm,
%	width=0.45\textwidth,
	left = 9mm, % pushes text away from the left edge by 10mm
	sharp corners, % disables rounded corners
	rounded corners = southeast, % "round" the bottom right corner
	arc is angular, % make the "round" corner an angle
	arc = 3mm, % controls corner cut
	boxrule=0.6pt, % sets box line thickness
	underlay={%
	\path[fill=black] ([yshift=3mm]interior.south east)--++(-0.4,-0.1)--++(0.1,-0.2); % triangle
\path[draw=black,shorten <=-0.05mm,shorten >=-0.05mm] ([yshift=3mm]interior.south east)--++(-0.4,-0.1)--++(0.1,-0.2); % triangle edge
\path[fill=gray!50!black,draw=none] (interior.south west) rectangle node[brown!10]{\Huge\bfseries ?!} ([xshift=8mm]interior.north west);
		
	},
	drop fuzzy shadow % adds drop shadow
}

\newtcolorbox{narratoresmall}{
	enhanced, % enable advanced settings
	%	left = 3mm,
	width=0.425\textwidth,
	left = 9mm, % pushes text away from the left edge by 10mm
	sharp corners, % disables rounded corners
	rounded corners = southeast, % "round" the bottom right corner
	arc is angular, % make the "round" corner an angle
	arc = 3mm, % controls corner cut
	boxrule=0.6pt, % sets box line thickness
	underlay={%
		\path[fill=black] ([yshift=3mm]interior.south east)--++(-0.4,-0.1)--++(0.1,-0.2); % triangle
		\path[draw=black,shorten <=-0.05mm,shorten >=-0.05mm] ([yshift=3mm]interior.south east)--++(-0.4,-0.1)--++(0.1,-0.2); % triangle edge
		\path[fill=gray!50!black,draw=none] (interior.south west) rectangle node[brown!10]{\Huge\bfseries ?!} ([xshift=8mm]interior.north west);
		
	},
	drop fuzzy shadow % adds drop shadow
}

\newtcolorbox{narratorewide}{
	enhanced, % enable advanced settings
	%	left = 3mm,
	width=0.95\textwidth,
	left = 9mm, % pushes text away from the left edge by 10mm
	sharp corners, % disables rounded corners
	rounded corners = southeast, % "round" the bottom right corner
	arc is angular, % make the "round" corner an angle
	arc = 3mm, % controls corner cut
	boxrule=0.6pt, % sets box line thickness
	underlay={%
		\path[fill=black] ([yshift=3mm]interior.south east)--++(-0.4,-0.1)--++(0.1,-0.2); % triangle
		\path[draw=black,shorten <=-0.05mm,shorten >=-0.05mm] ([yshift=3mm]interior.south east)--++(-0.4,-0.1)--++(0.1,-0.2); % triangle edge
		\path[fill=gray!50!black,draw=none] (interior.south west) rectangle node[brown!10]{\Huge\bfseries ?!} ([xshift=8mm]interior.north west);
		
	},
	drop fuzzy shadow % adds drop shadow
}

\newtcolorbox{enfasismall}{
	enhanced,
	arc=5pt,
	boxrule=0.3pt,
	width=0.425\textwidth
}


\newtcolorbox{enfasi}{
	enhanced,
	arc=5pt,
	boxrule=0.3pt
}

\newtcolorbox{enfasiwide}{
	enhanced,
	arc=5pt,
	boxrule=0.3pt,
	width=0.95\textwidth
}


\newtcolorbox{esempio}{
	enhanced,
	title = Example,
	before upper={\parindent15pt\noindent} % add paragraph indentation
}

%\backgroundsetup{
%	scale=1,
%	color=black,
%	opacity=0.4,
%	angle=0,
%	contents={%
%		\includegraphics[width=\paperwidth,height=\paperheight]{paper.jpg}
%	}%
%}

%\backgroundsetup{
%	scale=1,
%	angle=0,
%	opacity=1,
%	color=black,
%	contents={\begin{tikzpicture}[overlay]
%			\node at ([xshift=6cm,yshift=-2.5cm] current page.north east)
%			{\includegraphics[width = 3cm]{destra.png}};
%			node at (([xshift=6cm,yshift=-2cm] current page.north west)
%			{\includegraphics[width = 3cm]{sinistra.png}} ;
%			node at (([xshift=-12cm,yshift=-2.5cm]current page.south west)
%			{\includegraphics[width = 3cm]{sinistra2.png}} ;
%			node at ([xshift=-18cm,yshift=-2.5cm] current page.south east)
%			{\includegraphics[width = 3cm]{destra2.png}};
%	\end{tikzpicture}}
%	contents={\includegraphics[width=\paperwidth,height=\paperheight]{paper.jpg}}
%}

%\pdfcompresslevel=0
%\pdfobjcompresslevel=0

\backgroundsetup{contents={}}

%\setlength{\parindent}{1em}

\setlength{\parindent}{2em}

\begin{document}
	
	
%%\normalsize

%%\linespread{1.5}

\bigskip
Dedicato all'unica Donna mai amata, colei che ogni giorno mi accompagna nei sogni
 
Mai rinunciare ai tuoi sogni, persevera fino ad renderli reali.
\thispagestyle{empty}

\newpage~\thispagestyle{empty}%%\newpage~\thispagestyle{empty}


\pagebreak

\setcounter{page}{1}

\begin{multicols}{2}  
	{\small \tableofcontents{}}
\end{multicols}

\pagebreak{}

\section{Introduzione}

\begin{tcolorbox}[enhanced,arc=5pt,boxrule=0.3pt]{Si può scoprire di più su una persona in un'ora di gioco che in un anno di conversazione. (Platone)}\end{tcolorbox}\medskip

\begin{multicols}{2}  
\lettrine[lines=2, lhang=0.33, loversize=0.25, findent=1.5em]{D}{BS} è un gioco di ruolo (GDR o RPG, Role Play Game in inglese) cooperativo e narrativo nel quale i giocatori creano personaggi che vivranno fantastiche e strabilianti avventure. Il Narratore si preoccuperà di dipanare la storia e fare partecipare i personaggi. Come in un gioco di narrazione ogni personaggio contribuirà attivamente alla storia con le sue scelte, le sue decisioni e le sue azioni.

Se sei un giocatore ti accorgerai presto che non è un mondo facile, le cose non vengono regalate ne sono semplici. Saranno più le volte che vorranno ucciderti e derubarti di quelle che vorranno offrirti da bere.

Non affezionarti troppo al tuo personaggio, solo i forti sopravvivono e i potenti imperano. Dimostra chi sei e cosa vuoi. Vuoi essere l'eroe lucente in armatura ? fa pure, ma un ratto ha più probabilità di sopravvivere.

In ogni caso del tuo personaggio sarai tu a decidere tutto, dall'aspetto, al nome, alle sue capacità e ciò che possiede. Vorrà essere un pirata rubacuori o un cavaliere timido, un barbaro delle steppe o uno stregone ? Oro, onore, tesori, saccheggi, le avventure del tuo personaggio saranno costellate da scelte e battaglie, baldorie e.. qualsiasi cosa tu voglia!

Se sei il Narratore invece tu governi il mondo, la storia, l'avventura. Il tuo ruolo è di illustrare lo scenario in cui i giocatori si muovono e prendono decisioni. Li condurrai nelle profondità della terra alla ricerca del Tomo dimenticato di Atmos oppure a sfidare i grandi Draghi per la corona dell'Onniscienza?

Il tuo compito non è facile, usa fantasia, buon senso e la regola principale: divertiti. Quando sei in difficoltà non cercare la regola precisa, usa la tua più grande alleata: l'immaginazione, unisci un pizzico di assennatezza e cerca di stupire i giocatori. Lo scopo è sempre e solo uno, divertirsi insieme e crescere, come giocatori, come personaggi, come amici.

Oltre questo manuale avrai bisogno anche di un pò di dadi, i classici usati nei giochi di ruolo.
Chiamati solitamente d4, d6, d8, d10, d12, d20 stanno ad indicare un dado a 4 facce, il dado a 6 facce (di questi devi averne 3 o 4 almeno!), il dato a 8 facce, quello a 10 (solitamente vengono venduti in coppia, per poter ottenere il d100), il solitario dado a 12 facce e il sempiterno dado a 20 facce.
Ogni qual volta ti verrà chiesto di tirare un dado questo sarà scritto con la notazione XdZ, ovvero tiro X volte un dado con Z facce. Es. 4d6 indica di tirare 4 volte il dado a 6 facce.

Anche qualche miniatura potrebbe essere necessaria, altrimenti anche gli omaggi di merendine o degli ovetti al cioccolato possono essere sufficienti.

All'interno di questo manuale troverai tutto il necessario, come regole, per giocare, a te (voi) servirà fantasia, amicizia, dadi, qualche foglio di carta e divertimento (sorry, patatine e bibite non sono incluse nel manuale!)

Disegna ed usa una mappa ogni qual volta la descrizione e la situazione necessitano di una dettagli accurati ed un posizionamento precisa.

Crea e gioca il personaggio che più ti aggrada, che più senti tuo e ti fa divertire, non cercare le combinazioni di Abilità e capacità che ti danno più potere altrimenti prima o poi il personaggio ti verrà a noia.

Più giocherai, più il tuo personaggio guadagnerà esperienza ed anche tu lo interpreterai meglio. Il personaggio acquisirà oggetti meravigliosi, armature lucenti, armi volanti, oro, preziosi e gioielli e chissà cos'altro.

Il Narratore si preoccuperà di dirti quanta esperienza il tuo personaggio ha maturato, in base a come hai giocato, come hai collaborato nel gruppo, quanto hai aiutato il gruppo di giocatori a divertirsi. Ti terrà impegnato in scontri pericolosi, spesso mortali, metterà a dura prova il tuo personaggio e come gruppo riuscirete, forse non sempre, a risolvere le intricate situazioni che il Narratore ha preparato. Ricorda che il Narratore ha sempre l'ultima parola in ogni discussione.

Questo manuale è completo, ovvero all'interno troverai tutto (a parte i dadi!) per iniziare a giocare! 

Troverai anche molte regole, eppure molte situazioni dovranno essere gestite utilizzando la prima regola: divertirsi. Buon senso, esperienza e fiducia nel Narratore risolveranno ogni situazione.

Sia che tu decida di essere il Narratore sia che tu decida di interpretare un personaggio è necessario che tu legga con attenzione i capitoli che seguono.
E' importante che tu abbia una buona conoscenza delle regole base e che, soprattutto, sappia dove cercare qualsiasi cosa quando ti servirà!

\textit{Per gli esperti...}

Il progetto TUS e poi DBS nasce per l'insoddisfazione nel giocare la 5ed del famoso gioco di ruolo.
Il passaggio da Pathfinder alla 5ed è stato troppo traumatico, specialmente dopo anni e anni di gioco.

La 5ed appiattisce troppo i personaggi e il sistema di regole per quanto veramente efficiente non permette quella diversificazione, anche se spesso esagerata, che potevi avere in Pathfinder.

Avevo bisogno di una via di mezzo, un gioco basato comunque sul d20 e OGL ma che prendesse un pò il meglio di quanto era già stato creato e aggiungesse ciò che a me piaceva degli innumerevoli giochi di ruolo studiati e giocati.

Per questo il DBS è senza classi (l'altra opzione valutata era un sistema a professioni od ad albero), le competenze sono dettate dalla professione scelta e non arrivano a punteggi esagerati. Il combattimento non raggiunge l'epica complessità di Pathfinder ne il piattume della 5ed, cerca bensì di essere rapido e tattico, efficace e spettacolare.

Le Golden Rules, la gestione del critico, danno quel quid in più che permette ai giocatori di divertirsi ogni volta che viene tirato un dado.

La magia riprende i canoni standard della 5e ma rivisitati profondamente. Moltissimi incantesimi hanno perso la concentrazione e il concetto di potenziamento di incantesimo usando uno slot superiore non c'è piu'. Anche per gli incantesimi valgono le Golden Rules e questo permette di diversificare l'esito, aggiungendo piu' tensione, di ogni incatesimo.

Viene completamente cambiato l'approccio all'allineamento, diventando adesso un aspetto fondamentale per la costruzione del personaggio; non più due letterine striminzite (LB, CB...) ma una scelta basata sul carattere, morale ed etica.
Le divinità, pardon Patroni, hanno un ruolo più sporco e diretto leggeteli con attenzione, non sono i soliti dei.
I mostri sono quelli OGL della 5ed, modificati per essere più tosti in quanto non essendoci più il bonus competenza a scalare si hanno migliori risultati al Tiro per Colpire ed ai Tiri Salvezza, come giustamente era nella 3.5ed.

La licenza è quella affidabile della 5.1ed OGL che garantisce a chiunque di creare e produrre meravigliose avventure ed espansioni per DBS.

Un ultima nota. Il sistema vuole essere più letale della 5ed, più ferite, più sofferenza, senza il concetto di riposo breve o lungo o recupero di punti ferita in base ai dadi vita. Basta personaggi eroi sempre e comunque.

Non preoccupativi, delle buone giocate di ruolo e di gruppo e dei buoni dadi garantiranno sempre degli ottimi risultati, anche negli incantesimi!.

Col passare dei livelli diventeranno macchine dal potere unico, l'importante è che i giocatori sentano sempre che ogni combattimento è potenzialmente letale se giocato male. Stategli addosso, descrivete le azioni in modo esplicito, che sentano il rumore delle ossa spezzarsi, della palla di fuoco esplodere, i lamenti dei nemici.

Partecipazione ed immedesimazione sempre.

Infine ma non lo dico sottovoce, DBS si rifà al movimento OSR, ovvero vorrebbe essere giocato secondo quei principi. Leggete il capitolo \hyperlink{OSR}{Masterizzazione}.

\medskip

\begin{center}
	Buona lettura e Buon Divertimento!
\end{center}

\begin{flushright}
	Andres Zanzani
\end{flushright}

\end{multicols}

\medskip

\begin{tcolorbox}
"Although the masculine form of appellation is typically used when listing the level titles of the various types of characters, these names can easily be changed to the feminine if desired. This is fantasy, what's in a name? In all but a few cases sex makes no difference to ability!" 

Gary Gygax, Advanced Dungeons \& Dragons Players Handbook
\end{tcolorbox}

\bigskip

\begin{center}
	\includegraphics[width=1\linewidth]{immagini/Dragon_by_Henry_Justice_Ford_grey.jpg}
	\textit{The End of the Dragon - Henry Justice Ford}
\end{center}

%\begin{wrapfigure}{l}{0.2\textwidth}
%	\centering
%	\includegraphics[width=4cm]{Dragon_by_Henry_Justice_Ford.jpg}
%\end{wrapfigure}


%\begin{figure}[h]\centering	\includegraphics[width=18cm]{Dragon_by_Henry_Justice_Ford.jpg}\end{figure}

\pagebreak{}

\subsection{Termini Comuni}

\begin{multicols}{2}  

\lettrine[lines=2, lhang=0.33, loversize=0.25, findent=1.5em]{T}{i} elenco un pò di termini\index{Termini comuni} che troverai ripetuti più volte nel libro.

\textbf{Arrotondamenti}: \index{Arrotondamenti}sempre per difetto se non esplicitato diversamente. Es. 7/2 = 3, 9/4=2.

\textbf{Abilità}: \index{Abilità}sono le capacità particolari che il personaggio ha imparato ad usare. Sono spesso simili a capacità magiche, permettono azioni particolari ed anche di sovvertire le regole a volte. Sono rare e si prendono ai passaggi di livello.

\textbf{Azione}: \index{Azione}è ciò che si fa in un intervallo di tempo. Ogni cosa che viene fatta dal personaggio si misura in Azioni. Combattere, lanciare Incantesimi, scassinare, bere pozioni, lo spostarsi... in ogni round si possono fare 3 Azioni.

\textbf{Bonus}: \index{Bonus}qualsiasi modifica dovuta a fattori esterni, ambientali, magici, di circostanza o che decida il Narratore è un bonus o malus da applicare al tiro di dado o difficoltà nella prova.

\textbf{Classe}: In DBS non ci sono classi. Ogni personaggio è costruito in base a ciò che sa fare. Quindi non troverete la parola Classe nel manuale.

\textbf{Check/Prova}: \index{Check}\index{Prova}un check (o prova) è il tiro di 3d6 più un punteggio indicato dalla Caratteristica e Competenza coinvolta, potrebbero essere applicati modificatori dati da Abilità e circostanze.

\textbf{Prova di Magia / Prova di Competenza Magica}\index{Prova di Magia}\index{Prova di Competenza Magica}: ogni qual volta il mago vuole lanciare un incantesimo deve fare una prova di magia o competenza magica, i termini descrivono entrambe la stessa cosa. La prova di Magia o Competenza magica si effettua tirando 3d6 + Intelligenza (o altro) + Competenza Magica + bonus concesso dal Patrono + eventuali Abilità e modificatori pertinenti. Se la prova di magia supera la Difficoltà dell'incantesimo allora la magia si libera e l'incantesimo ha effetto, diversamente l'incantesimo non si manifesta ma viene contato nel numero degli incantesimi lanciati.

\textbf{Lanciare Incantesimi sotto attacco, minaccia, distrazione..}:\index{Prova di Concentrazione}\index{Lanciare Incantesimi sotto attacco, minaccia, distrazione..} quando un incantatore vuole usare una Magia ma è disturbato, attaccato, ferito o comunque distratto durante il lancio di un incantesimo allora la prova di magia deve riuscire particolarmente bene.

La prova di magia per lanciare l'incantesimo deve riuscire di almeno 5.
Se la prova fallisce l'incantesimo non è lanciato e non è contato tra gli incantesimi lanciati nel giorno.

\textbf{Classe di Difficoltà (DC)}:\index{Classe di Difficoltà} \index{DC}indica quanto è difficile riuscire in una prova. Può essere usato per verificare le competenze (nuotare..) come le conoscenze (veleni..). Negli incantesimi è la difficoltà a resistere allo stesso incantesimo. Indica a che valore arrivare per superare e riuscire nella prova.

\textbf{Competenza} \index{Competenza}(skill)\index{Skill}: la competenza ci dice ciò che sappiamo ed il suo valore indica il grado di conoscenza di una singola capacità. Possa essere lo studio di una lingua, l'arrampicarsi, il notare piccole cose.

\textbf{Competenza con le Armi (da mischia o distanza)} \index{Competenza con le Armi} è la  capacità di saper colpire l'avversario con armi da mischia (spade, mazze..) o da tiro/distanza (pugnali da lancio, archi, balestre..)

\textbf{Competenza Magica (CM)}: \index{Competenza Magica}\index{CM}è la tua capacità di usare le magie, più è alto questo valore più le magie saranno efficaci. E' il valore che si somma alla prova di competenza magica/prova di magia.

\begin{center}
	\includegraphics[width=0.45\textwidth]{immagini/spiritomagia2.png}
\end{center}

\textbf{Dadi Vita}\index{Dadi Vita}: per dadi vita si intendono i livelli di una creatura. Di base servono ad indicare quanti punti ferita e livello ha. Se non indicato una creatura ha 6 Punti Ferita per Dado Vita.

\textbf{Difesa}: \index{Difesa}per Difesa si intende il valore totale ottenuto da 10 + Scudo + Armatura + Destrezza + vari ed eventuali bonus. Rappresenta la difficoltà nell'essere colpito, più alto è più è difficile essere colpito.

\textbf{+1d6 oppure -1d6}: è un bonus o malus ad una prova. Aggiungi o sottrai un tiro di dado a 6 alla prova, oppure un +4/-4

\textbf{Distanza}:\index{Distanza} la distanza, per quando riguarda il combattimento è misurato in quadretti da 1 metro.

\textbf{Devoto}\index{Devoto}: un personaggio che si é legato ad un Patrono é ha almeno 3 Tratti in comune.
Ha il Vantaggio offerto dal Patrono ed un +4 alle Prove di Magia ad una Scuola di Magia. Vedi anche Seguace.

\textbf{Seguace}: un personaggio che si è legato ad un Patrono con 2 Tratti in comune, ottiene un +2 alla Scuola preferita del Patrono

\textbf{Esplosione del 6}:\index{Esplosione del 6} quando, esegui il Tiro per Colpire, Tiro Salvezza, prova di Magia (leggi le specifiche nel capitolo dedicato) o comunque ogni volta che viene indicato che vale l'esplosione del 6 significa che per ogni dado tirato che ha fatto 6 va segnato e ritirato il dado. Il risultato del nuovo tiro va anche lui sommato e se si fa un 6 si continua a ritirare finché si continua a fare 6.

\textbf{Iniziativa}: \index{Iniziativa}è una prova di Destrezza oppure Intelligenza. Stabilisce l'ordine delle azioni in combattimento. Chi ha il punteggio più alto agisce per primo.

\textbf{Livello}:\index{Livello} il livello indica la competenza e potere raggiunto dal personaggio. Può indicare quando è forte il nemico.

\textbf{Difficoltà}: indica quando è arduo manifestare un incantesimo. Piu' è alto il valore più è difficile la magia e potente l'incantesimo.

\textbf{Incantatore, Mago:} \index{Incantatore}indica un qualsiasi usufruitore di magia a qualsiasi titolo.

\textbf{Mischia}: \index{Mischia}con mischia si intende il combattimento di contatto, spada a spada, ovvero quando il tuo personaggio combatte con una spada o comunque con un'arma che non è da tiro (arco/balestre..) contro un avversario.
Si considera in mischia qualsiasi creatura che il personaggio possa raggiungere con la sua arma non da tiro. Un nemico di grandi dimensioni (o con un arma lunga) potrebbe essere in mischia con il personaggio ma non viceversa.

\textbf{Movimento}: \index{Movimento}il movimento rappresenta la capacità di spostarsi. Una Azione di movimento rappresenta lo spostarsi del personaggio. Più è alto il valore di movimento più una creatura può muoversi lontano.

\textbf{Narratore:}\index{Narratore} il Narratore è la persona che conduce l'avventura, stabilisce le regole e controlla gli elementi della storia. Il dovere di ogni Narratore è fare divertire ed essere corretto. Il Narratore ha l'ultima parola in ogni questione.

\textbf{Patrono}:\index{Patrono} o divinità. Il Patrono è un essere superiore che può concedere poteri e garantire vantaggi. 

\textbf{Penalità/Malus} \index{Penalità}: come il bonus le penalità o malus sono valori, numeri, che indicano le circostanze sfavorevoli, magie penalizzanti o quant'altro renda più difficile la prova. Purtroppo a differenza dei Bonus le penalità, se non specificato diversamente, si sommano sempre fra loro.

\bigskip

%\begin{wrapfigure}{c}{0.5\textwidth}
\begin{center}
	\includegraphics[width=0.45\textwidth]{immagini/Sakramentarz_tyniecki_02.png}
	\textit{Sakramentarz Tyniecki: Majuskuła "V".}
\end{center}

\bigskip

\textbf{PG, Personaggio}: \index{Personaggio}è la creatura che viene guidata, gestita, ruolata dal giocatore.

\textbf{PNG}: \index{PNG}personaggio non giocante. Sono personaggi particolari, importanti o meno che il Narratore tiene per condurre l'avventura.

\textbf{Punti Esperienza/PX}: \index{Punti Esperienza} \index{PX} ogni qual volta si risolvano difficoltà, mostri, indovinelli, si giochi bene il personaggio e ci si diverta si guadagna esperienza. Questi punti accumulati nel tempo stabiliscono il livello del personaggio.

\textbf{Punteggi caratteristica}: \index{Punteggi caratteristica} \index{Statistiche} abbreviati anche in caratteristica o statistiche. Ogni personaggio ha 6 Caratteristiche: Forza (FOR), Destrezza (DES), Intelligenza (INT), Saggezza (SAG) e Carisma (CAR). Più è alto il valore maggiore è la valenza o capacità del personaggio in quello specifico ambito.

\textbf{Punti Fato}:\index{Punti Fato} \index{Fortuna del Principiante}o Fortuna del Principiante sono dei punti a disposizione che il giocatore può trasformare in d6 da aggiungere ai Tiri Salvezza o Tiri per Colpire o Tiri Competenze. Vengono chiamati Fortuna dei Principianti perché il loro numero diminuisce all'aumentare di livello del personaggio.

\textbf{Punti ferita (Punti Ferita)}:\index{Punti Ferita} \index{Punti Ferita}indicano l'energia vitale della creatura. Finché la creatura ha 1 punto ferita combatterà al suo meglio , senza problemi (certo.. potrebbe anche decidere di scappare piuttosto che morire..).

Ad ogni passaggio di livello si guadagna un certo numero di punti ferita, stabilito dalle regole. Ogni ferita si sottrae da questa somma di energie e quando si raggiungono gli 0 (zero) punti ferita si sviene, incapaci di agire. Se si viene ulteriormente feriti o comunque i propri punti ferita scendono fino 10+valore doppio della Costituzione allora si é morti.

%\begin{center}
%	\includegraphics[width=0.2\linewidth, height=0.1\textheight]{immagini/book-3561831_1920}
%\end{center}

\textbf{Resistenza alla Magia (RM)}:\index{Resistenza alla Magia} \index{RM}Una creatura potrebbe avere una naturale resistenza alla magia, in ogni forma si presenti. Ogni qual volta la creature è influenzata direttamente da una magia deve effettuare una prova di RM, ovvero tirare 3d6 sommare il valore di RM e se è superiore alla prova di magia effettuata dall'incantatore la magia non ha effetto.

\textbf{Riduzione del Danno (DR)}: \index{Riduzione del Danno} \index{DR} alcune creature hanno una resistenza innata al danno e ferite. Questa resistenza si denota come DR.
é solitamente indicata come Valore/Particolare, dove il valore indica quanto si é resistenti e il Particolare indica da che cosa si é danneggiati. Es "10/cold iron" indica che si ha una resistenza a tutti i danni di 10 tranne quelli causati da ferro freddo.
Se il particolare é indicato da un trattino "-" allora questa resistenza non é oltrepassabile e si applica a tutti i tipi di danno.

\begin{center}
	\includegraphics[width=0.45\textwidth]{immagini/cavaliere2.png}
\end{center}

\textbf{Resistenza al Danno (RD)} \index{Resistenza al Danno}\index{RD} : una creatura potrebbe avere una resistenza ad una tipologia di danno. In questo caso si considera che dimezzi automaticamente il danno subito prima di applicare eventuali Tiri Salvezza.

\textbf{Round}:\index{Round} il combattimento o azioni sono divise in round. Un round rappresenta una unità temporale di circa 10 secondi. Durante il round ogni creatura ha la possibilità di agire in base alla sua iniziativa ed eseguire fino a 3 Azioni.

\textbf{Successo Critico/Fallimento Critico}\index{Sucesso Critico} \index{Fallimento Critico}: nel caso in cui il giocatore passi la prova con un ampio margine otterrà benefici ( o malus). Controllate nelle competenze e nella magia.

\begin{center}
	\includegraphics[width=0.45\textwidth]{immagini/Jan_Steen2.png}
	
	\textit{Jan Havicksz. Steen}
\end{center}


\textbf{Tiro per Colpire (TC)}:\index{Tiro per Colpire} \index{TC}è una prova di Attacco (Competenza Armi) contro Difesa (armatura + scudo + Abilità + magia...). Il Tiro per Colpire può essere da mischia (ovvero per le creature prossime alla tua arma, a distanza di mischia) oppure da distanza (per archi, balestre, ma anche pugnali lanciati..).. Leggi bene il capitolo del combattimento.

\textbf{Tiro Salvezza (TS)}:\index{Tiro Salvezza} \index{TS}quando una creatura è sottoposta ad un effetto particolare spesso viene concesso un Tiro Salvezza per mitigare o annullare gli effetti. Il Tiro Salvezza è un'azione che non occupa tempo o azioni.

I Tiri Salvezza riguardano i riflessi e lo schivare (Destrezza), resistere a veleni/malattie o cambiamenti del corpo (Costituzione) oppure per resistere ad attacchi mentali ed effetti che agiscano sull'arbitrio (Saggezza).

\textbf{Tratto}: \index{Tratto}indica una componente del carattere. Ogni personaggio sceglie 4 Tratti per comporre e costruire la sua personalità.

\textbf{Turno}: \index{Turno}sono 10 minuti, ovvero 60 round

\textbf{Uno porta male}: \index{Uno porta male}se tiri un 1 con il dato togli 1 dal risultato totale. Non per questo un 6 tirato diventa un 5, l'esplosione del 6 rimane.. solo che togli 1 al risultato finale. Detta diversamente 1 vale 0.

\bigskip

Nel Manuale troverete diverse tipologie di box, ognuno ha un significato preciso

\medskip


\begin{enfasi}{Esempio di box contenente una citazione o frase motivazionale}\end{enfasi}

\begin{tcolorbox}[title = Informazioni per il giocatore]Box contenente indicazioni e chiarimenti per il Giocatore.\end{tcolorbox}

\begin{narratore}Box contenente indicazioni e suggerimenti per il Narratore\end{narratore}

\end{multicols}

\pagebreak

\section{Razze}\index{Razze}

\begin{enfasi}{Il vero viaggio di scoperta non consiste nel trovare nuovi territori, ma nel possedere altri occhi, vedere l'universo attraverso gli occhi di un altro, di centinaia d'altri: di osservare il centinaio di universi che ciascuno di loro osserva, che ciascuno di loro è. (Marcel Proust)
		
\medskip

Non è la più intelligente delle specie quella che sopravvive; non è nemmeno la più forte; la specie che sopravvive è quella che è in grado di adattarsi e di adeguarsi meglio ai cambiamenti dell'ambiente in cui si trova. (Leon C. Megginson)}\end{enfasi}\medskip

\begin{multicols}{2}

\lettrine[lines=2, lhang=0.33, loversize=0.25, findent=1.5em]{Y}{eru} è un mondo sfaccettato e ricco di diversità, colturali, naturali e di persone.
Le creature rendono il pianeta vitale e ricco, ognuna nutre, apporta, arricchisce la conoscenza di tutte le altre.


\subsection{Umani}\index{Umani}

Gli umani con il loro desiderio di scoperte, potere, gloria e violenza sono la razza dominatrice.

Le caratteristiche fisiche degli umani sono varie quanto i climi del mondo. Il colore della pelle, l'abbigliamento, le tradizioni culturali ed alimentari, gli stili di vita possono essere i più disparati ed originali e tutto rende solo più umano il personaggio.

Lasciate fuori il razzismo da Yeru, ci sono già abbastanza guerre, non server crearne di nuove solo perché qualcuno usa le asce al posto delle spade.

Gli umani sono stati la razza creata da Ljust e Calicante insieme perché con la loro spinta caotica, mutevole e vitale potessero fare e disfare ricominciando continuamente da capo e migliorando di continuo.

\begin{center}
	\includegraphics[height=0.7\linewidth]{immagini/uomovitruviano2.png}
	
	\textit{Uomo vitruviano - Leonardo da Vinci}
\end{center}


\textbf{Modificatori razziali:} +1 ad una caratteristica a piacere

\textbf{Caratteristiche fisiche}: altezza 150-185 cm, 50-130 kg, aspettativa di vita 65 anni

\textbf{Dimensioni:} Medie

\textbf{velocità}: 9m

\textbf{Linguaggi}: Comune

\textbf{Vantaggio}: +1 Abilità al primo livello

\subsection{Elfi}\index{Elfi}

\label{elfi}

Gli elfi sono la razza creata direttamente da Ljust perché guidasse il mondo con l'eleganza, l'intelligenza e la lungimiranza di una razza immortale.

Dopo millenni di pace e vita nell'intero mondo, dopo che bellezze naturali ed architettoniche si erano diffuse in armonia il mondo, la creazione delle nuove razze e la loro spinta espansionistica hanno portato gli elfi a rivedere la loro missione.

Improvvisamente da portatori di bellezza, cultura, passione e stimolo per le arti sono diventati più reclusi, in loro si manifestò il timore che tutto il bello potesse andare perso.

Decisero così di isolarsi sempre più per conservare ogni forma di arte e bellezza, dalla parola scritta alle arti al riparo da chi potesse disprezzarle o imbruttirle.

Il loro desiderio di preservare il creato si concretizzo in pura lotta contro tutto e tutti, contro qualsiasi creatura volesse in sintesi vivere e creare qualcosa di nuovo. Cattalm aveva avvelenato il loro sangue in una maniera talmente profonda che neanche loro se ne accorsero.

Ne seguì un periodo di secoli estremamente nefasti e violenti dominati dalla volontà assoluta degli elfi di distruggere tutto utilizzando qualsiasi mezzo.
Tutte le nazioni e civiltà pagarono un prezzo altissimo sia in termini di vite che in termini di ritorno a tempi barbari.

Ci vollero quasi 500 anni e l'annientamento di intere nazioni e civiltà fino a ché tutte le altre creature decisero di fare fronte comune contro gli elfi in un estremo tentativo di salvezza.

In quella che viene chiamata la Settimana dell'Ombra perirono centinaia di migliaia di creature e pressoché tutti gli elfi.

Ne seguì una epurazione radicale, ogni elfo trovato venne ucciso e così per quasi un altro secolo. 

Poco meno di 100 anni fa una nuova Regina elfica, Licenea, intraprese un viaggio alla corte di ogni nazione, rischiando il linciaggio numerose volte. Riuscì a ottenere un trattato di pace che potesse salvaguardare i pochissimi elfi rimasti.

Gli elfi sopravvissuti, pur se "anziani", non hanno perso il loro odio, il loro sangue rimane tinto da Cattalm e rimuginano dietro ad un trattato che a loro dire impedisce che possano compiere il loro vero destino.

Fortunatamente i nuovi elfi, ma non tutti, non hanno questo odio viscerale e vorrebbero vivere una vita normale anche a contatto con tutte le altre creature anche se ben consci di come sono visti e trattati da tutti gli altri.

Sono i giovani elfi che non vogliono solo custodire il bello ma vivere una vita quasi infinita dove essa stessa è meraviglia.

Gli elfi sono generalmente più alti e snelli degli umani. Gli occhi sono sempre con sfumature chiare.

Gli elfi apprezzano la parola scritta e la magia. Sono una razza razionale, guidata da mente acuta e sensi eccellenti, interesse per ciò che è straordinario e per la conoscenza.

\begin{center}
	\includegraphics[height=0.5\linewidth]{immagini/elf.png}
\end{center}

\textbf{Modificatori razziali:} +1 Intelligenza, +1 Destrezza, -1 Carisma

\textbf{Caratteristiche fisiche}: altezza 165-195 cm, 50-110 kg, aspettativa di vita 1000+ anni

\textbf{Dimensioni:} Medie

\textbf{velocità}: 9m

\textbf{Linguaggi}: Elfico

\textbf{Vantaggio}: Visione crepuscolare di 18 metri


\subsection{Nani}\index{Nani}

\label{nani}

I nani sono una razza stoica e severa abituata al comunismo più puro, senza un vero concetto di proprietà ma di pura comunanza di beni secondo l'idea che ogni nano lavora per la comunità e non per se stesso.

I nani sono una razza bassa e piazzata, raggiungono un'altezza massima di circa 140 cm con una corporatura robusta e compatta che dà loro un aspetto massiccio. Sia i maschi che le femmine portano orgogliosamente i capelli lunghi e gli uomini decorano spesso le barbe con vari generi di fermagli e trecce intricate, altresì vero che nani pelati sono frequenti, ma non senza barba. Le donne nane non hanno barba ne peluria in eccesso. Il sesso è libero e socialista.

I nani sono guidati da onore e tradizione e comunismo. Sono spesso visti come burberi, ma hanno un forte sentimento di amicizia e giustizia e rispetto per chi lavora sodo e si impegna per la comunità.

I nani sono la razza creata da Erondil con l'aiuto di Atmos.

Giudicano gli Elfi con severità perché non hanno saputo portare a termine ed anzi tradito il dettato della Creazione e quindi si sentono il compito, l'onere e l'onore di forgiare il creato e nel creato la bellezza e la maestosità di Erondil.

\begin{center}
	\includegraphics[height=0.5\linewidth]{immagini/DnD_Dwarf.png}
\end{center}

\textbf{Modificatori razziali:} +1 Costituzione, +1 Saggezza, -1 Destrezza

\textbf{Caratteristiche fisiche}: altezza 100-140 cm, 45-90 kg, aspettativa di vita 450 anni

\textbf{Dimensioni:} Medie

\textbf{velocità}: 6m

\textbf{Linguaggi}: Nanico

\textbf{Speciale:} Professione: una competenza legata alla professione iniziale prende un +1

\textbf{Vantaggio}: Scurovisione di 18 metri

\textbf{Svantaggio}: Pessimo carattere


\subsection{Gnomi}\index{Gnomi}

\label{gnomi}

Gli Gnomi sono esseri dalla piccola taglia ma ricchi di energia e vita. Gli Gnomi sono una razza apparsa poco più di 1000 anni fa, non si sa esattamente da dove.
In breve tempo, grazie alla loro innata curiosità, tenacia, inventiva sono riusciti a creare popolose e ricche città, quasi sempre all'interno di foreste vergini.

Gli Gnomi sono profondamente legati alla natura, il loro rapporto è quasi simbiotico, uno Gnome non rinuncerà mai alla vista di un albero ed a costruire con ciò che la natura fornisce.

Gli Gnomi hanno un profondo rispetto per la natura, l'ambiente e gli animali, le loro città anche se perfettamente funzionali ed anzi moderne sono costruite e ricavate nelle foresta e mai distruggendola ed anzi arricchendola. 

Molti Gnomi sono inventori e costruttori capaci di prove di fantasia ed ingegno fuori dal comune. Molte delle loro invenzioni sono di aiuto e supporto a tutta la comunità e la loro vita sociale è ricca e solidale.

Gli Gnomi più curiosi spesso escono dalle loro comunità, che non sono chiuse a chi accetta lo stile di vita degli gnomi, ed intraprendono una vita di avventura alla scoperta di meraviglie e nuove opere di ingegno da poter tramandare alla comunità.

Uno gnomo costretto a stare lontano da un ambiente naturale patisce la situazione diventando triste e apatico, la sua necessità di luce e natura è qualcosa di fisico e connaturato.

Gli Gnomi vanno d'accordo con chiunque rispetti la natura e non ne abusi.

\textbf{Modificatori razziali:} +1 Intelligenza, +1 Carisma, -1 Forza

\textbf{Caratteristiche fisiche}: altezza 70-110 cm, 30-50 kg, aspettativa di vita 650 anni

\textbf{Dimensioni:} Piccolo

\textbf{velocità}: 6m

\textbf{Linguaggi}: Gnomico, Silvano

\textbf{Speciale:} Professione: una competenza legata alla professione iniziale prende un +1

\textbf{Vantaggio}: Artificio Druidico 1 al giorno.

\textbf{Svantaggio}: Depressione se lontano più di 7 giorni da un ambiente naturale.


\subsection{Mezzelfo}\index{Mezzelfo}

\label{mezzelfo}

Per un elfo non c'è nulla di più impuro di un mezz'elfo. Nessun mezz'elfo nasce per volontà di un Elfo. Ogni mezz'elfo è figlio di violenza. Questo è almeno quello che continuano a dire gli elfi.

Ci sono anche rari mezz'elfi nati da rapporti romantici. Benché solitamente di breve durata, anche per gli standard umani, questi incontri segreti portano di solito alla nascita dei mezzelfi, una razza che discende da due culture, ma non è erede di nessuna. I mezzelfi possono riprodursi tra loro, ma persino questi mezzelfi "di sangue puro" sono visti come bastardi dagli elfi.

Molti elfi vedono in un mezzelfo il tradimento della bellezza originaria, della purezza del creato.
Pochissimi vedono un gesto di amore e dono verso un mondo sempre più imbruttito. 

I mezzelfi sono più alti degli umani ma più bassi degli elfi. Ereditano la corporatura slanciata e i lineamenti attraenti del loro lignaggio elfico, ma il colore della loro pelle è normalmente dettato dalla loro parte umana. I loro occhi tendono ad essere simili a quelli degli umani nella forma, ma presentano un'esotica gamma di colori dall'ambra al viola fino al verde smeraldo e al blu scuro.

I mezzelfi comprendono la solitudine e sanno che il carattere spesso è più un prodotto dell'esperienza di vita che della razza di appartenenza.

\medskip

\begin{center}
	\includegraphics[height=0.6\linewidth]{immagini/halfelf.png}
\end{center}


\textbf{Modificatori razziali:} +1 ad una Caratteristica a propria scelta

\textbf{Caratteristiche fisiche}: altezza 160-185 cm, 50-100 kg, aspettativa di vita 210 anni

\textbf{Dimensioni:} Medie

\textbf{velocità}: 9m

\textbf{Linguaggi}: Comune oppure Elfico

\textbf{Vantaggio}: Visione crepuscolare di 18 metri

\index{Mezzorco}

\subsection{Mezzorco}

\label{mezzorco}

Agli occhi delle razze civilizzate, i mezzorchi sono delle mostruosità, il risultato di perversione e violenza e raramente sono il risultato di unioni amorose, come tali solitamente sono costretti a crescere velocemente e duramente, lottando continuamente per proteggersi o farsi un nome. Alcuni mezzorchi trascorrono le loro intere vite a dimostrare agli orchi purosangue che sono feroci quanto loro.

I mezzorchi sono alti in media 1.9 metri, con fisico potente e pelle verdastra o grigia. I loro canini crescono spesso piuttosto lunghi fino a sporgere dalle loro bocche e queste "zanne", unite ad una fronte ampia e le orecchie un pò a punta, danno loro quel noto aspetto "bestiale". A dispetto di questi ovvi tratti orcheschi, i mezzorchi sono tanto variegati quanto i loro genitori umani.

Se all'interno delle tribù orchesche devono guadagnarsi continuamente il rispetto dei "purosangue", nella società umana non va meglio. Derisi, sbeffeggiati, esclusi ed abbandonati i mezzorchi spesso trovano rifugio nella criminalità.

Gli orchi sono stati creati direttamente da Cattalm con l'aiuto di Calicante. Molto della tendenza caotica e distruttrice del loro creatore rimane nella natura dei mezzorchi.

I mezzorchi sono spesso vittime di pregiudizi.

\textbf{Modificatori razziali:} +2 Forza -1 Carisma

\textbf{Caratteristiche fisiche}: altezza 160-210 cm, 60 - 140 kg, aspettativa di vita 70 anni

\textbf{Dimensioni:} Medie

\textbf{velocità}: 9m

\textbf{Linguaggi}: Comune oppure Orchesco

\textbf{Vantaggio}: Visione crepuscolare di 18 metri

\textbf{Svantaggio}: Seguire il Chaos

\begin{center}
	\includegraphics[height=0.6\linewidth]{immagini/halforc.png}
\end{center}

\subsection{Nibali}\index{Nibali}

\label{nibali}

I Nibali sono una razza creata magicamente per essere schiava ai grandi maghi del nord.

La leggenda dice che i terribili maghi del nord, partendo da una coppia di umani (dopo che a migliaia erano morti atrocemente nei precedenti esperimenti) riuscì a creare manipolando con la magia, un razza più robusta, più forte, più intelligente ed allo stesso tempo più docile e disciplinata con pregio che ogni figlio generato sarebbe stato assolutamente identico fisicamente al padre o alla madre.

Queste cose accadevano ormai più di 2000 anni or sono ed il regno del male eterno crollò sotto la sua stessa incapacità di evolversi e comprendere le nuove sfide.

I Nibali hanno continuato a prosperare ed usufruendo di quanto il regno del ghiaccio gli aveva lasciato hanno creato una tra le civiltà più moderne, democratiche e civili del mondo.

Per molti l'estrema efficienza e dedizione dei Nibali è odiosa, un giogo che non lascia spazio alle libertà personali, per i Nibali è solo un modo naturale di progredire.

Tutti i Nibali sono uguali tra loro a parità di sesso ma il fatto che non possano avere figli con altre razze non li rende un popolo chiuso o razzista, anzi l'assorbire il meglio di ogni cultura li rende migliori ed anche ottimi diplomatici.

Ciò che veramente distingue un Nibali da un altro è l'acconciatura, i tatuaggi, il vestiario... Il rispetto per gli altri e la Legge sono legati indissolubilmente alla loro natura eppure non c'è nulla di più libero di un Nibali.

Seguire le regole, e non in maniera stupida ma efficiente e corretta è una connotazione innata nei Nibali, questo non significa che non si ribellino a leggi palesemente corrotte.

\textbf{Modificatori razziali:} +1 Costituzione, +1 Intelligenza, - 1 Saggezza

\textbf{Caratteristiche fisiche}: altezza 183cm maschi, 170cm femmine, 50 - 120 kg, aspettativa di vita 130 anni

\textbf{Dimensioni:} Medie

\textbf{velocità}: 9m

\textbf{Linguaggi}: Comune

\textbf{Svantaggio}: Seguire la Legge

\subsection{Diversi}\index{Diversi}

\label{diversi}

Benedetti o maledetti i Diversi non sono come noi. Non sono gli amici che ti aspetti. Un Diverso è frutto di una unione corrotta. Se i Patroni non possono agire direttamente nel mondo, o almeno questo è quello che cerca di evitare Gradh, sovente invece usano i loro poteri per creare una stirpe a loro fedele.

Un Diverso è fedele al suo Patrono e non può fare diversamente. Per fortuna sono sterili con gli umani, altrimenti avrebbero già dominato il mondo.

Un Diverso è più robusto e più intelligente. Purtroppo per loro la loro vita frenetica è segnata da una breve durata. Solitamente un Diverso non supera i 50 anni di vita.

Un Diverso è marchiato, da qualche parte sul suo corpo c'è il simbolo, una voglia, del suo Patrono.

\bigskip

\begin{center}
	\includegraphics[width=0.9\linewidth]{immagini/diverso.jpg}
\end{center}

\textbf{Modificatori razziali:} +1 Costituzione, +1 Intelligenza

\textbf{Caratteristiche fisiche}: altezza 155-185 cm, 50-110 kg, aspettativa di vita 45 anni (40+1d10 anni)

\textbf{Dimensioni:} Medie

\textbf{velocità}: 9m

\textbf{Linguaggi}: Comune

\textbf{Speciale}: Deve individuare un Patrono ed avere almeno 3 Tratti comuni.

\bigskip

\textbf{Nota sugli Svantaggi}
Il giocatore, in accordo con il Narratore, può scegliere uno svantaggio diverso da quello indicato purché sia coerente con la storia del personaggio.

\medskip

\index{Razze}\index{Razza}
\begin{tcolorbox}[title = Nota sulle Razze]
Nessuna descrizione di una razza potrà mai imbrigliare e sottomettere un personaggio. Ogni giocatore è libero di creare il personaggio della razza preferita (concessa dal Narratore) e descriverlo, inquadrarlo, sentirlo, renderlo vivo come più gli piace.
Non limitatevi alle descrizioni qui proposte, vogliono essere solo spunti, non sentitevi limitati nelle scelte perché la vostra razza dice questo o quello.
Fate nascere i più belli ed completi personaggi possibili.
Ogni personaggio è vivo ed è una persona e come tale sarà sempre diverso l'uno dall'altro, ognuno fantastico in maniera diversa a discapito di qualsiasi razza e pregiudizio.
\end{tcolorbox}

\begin{tcolorbox}[title = Nota sul Sesso]\index{Sesso}
Casomai foste così ottusi ribadisco che non c'è differenza di capacità o statistiche in base al sesso. Ogni giocatrice e giocatore è invitato a fare il personaggio del sesso (o non) che preferisce. 

Se l'argomento è per voi motivo di non divertimento chiaritelo con il Narratore, saprà orchestrare l'avventura in maniera idonea.
\end{tcolorbox}

\end{multicols}

\pagebreak

\section{Caratteristiche Speciali}

\label{caratteristiche-speciali}

\begin{enfasi}{Non basta avere gli occhi per vedere (anonimo)}\end{enfasi}\medskip

\begin{multicols}{2}  


\lettrine[lines=2, lhang=0.33, loversize=0.25, findent=1.5em]{O}{gni} creatura è speciale ed unica eppure ci sono essere ancora più unici e speciali per le loro caratteristiche. Queste sono le peculiarità di alcune di queste.

\subsection{Visione Crepuscolare}\index{Visione Crepuscolare}

Quello che per molti é oscurità per chi ha \hypertarget{visioneeluce}{visione crepuscolare} é vedere bene purché ci sia una fonte minima di luce.

La visione crepuscolare è una visione a colori.
Un incantatore dotato di visione crepuscolare può leggere una Pergamena fino a quando ha accanto come fonte di luce anche la più smorta della candela.

I personaggi dotati di visione crepuscolare possono vedere all'esterno nelle notti illuminate dalla luna come se si trovassero alla luce del giorno.
Nella assoluta mancanza di luce la visione crepuscolare non aiuta, rimane buio pesto impenetrabile.

\subsection{Scurovisione}\index{Scurovisione}

La Scurovisione è la capacità straordinaria di vedere senza fonti di luce, fino ad una distanza massima indicata per ogni creatura. 

La Scurovisione è solo in bianco e nero (non consente al personaggio di distinguere i colori). Non permette ai personaggi di vedere nulla che non possano altrimenti vedere: gli oggetti Invisibili sono ancora Invisibili, e le Illusioni sono ancora visibili per quello che sembrano essere.

Alla stessa maniera, la Scurovisione rende una creatura soggetta agli attacchi con lo sguardo normalmente. La presenza di luce non altera la Scurovisione.

\subsection{Fiuto}\index{Fiuto}

Questa qualità speciale permette ad una creatura di sfruttare l'olfatto per individuare i nemici nascosti o in avvicinamento e di seguire le tracce. Le creature dotate di fiuto possono identificare con l'olfatto gli odori familiari come gli umani fanno con quello che vedono.

La creatura può individuare le creature entro 6 metri di distanza con l'olfatto. Se l'avversario è sottovento, il raggio aumenta a 18 metri; se è sopravento, il raggio diminuisce a distanza di 3 metri.
Gli odori più forti, come il fumo, spazzatura o corpi in decomposizione, possono essere individuati al doppio del raggio sopra indicato.

Quando una creatura individua un odore, non viene rivelata l'esatta posizione della sua fonte, ma solo la sua presenza entro il raggio d'azione. La creatura può utilizzare un'Azione per individuare la direzione da cui proviene l'odore. Quando si trova a distanza di mischia dalla fonte, ne individua la posizione.

Una creatura dotata di fiuto può seguire tracce utilizzando l'olfatto, effettuando una prova di Seguire Tracce per trovare e seguire una traccia. La tipica DC di una traccia fresca è 10 (a prescindere dalla superficie su cui si trova la traccia). La DC aumenta o diminuisce a seconda dell'intensità della traccia, del numero di creature che la lasciano e del tempo trascorso da quando è stata lasciata. Per ogni ora trascorsa la DC aumenta di 2.

\begin{center}
	\includegraphics[width=0.9\linewidth]{immagini/mostro.png}
\end{center}

Per il resto, questa capacità segue le regole della competenza Sopravvivenza. Le creature che seguono tracce con il fiuto ignorano gli effetti delle superfici su cui si trova la traccia e della scarsa visibilità.

Una creatura con la capacità Fiuto identifica gli odori familiari così come un umano potrebbe identificare un luogo familiare. L'acqua, e in particolare l'acqua corrente, nega la capacità di seguire tracce delle creature.

Alcuni forti odori possono facilmente mascherarne altri. La presenza di un odore simile rende impossibile individuare o identificare esattamente una creatura mediante il Fiuto; la DC base della competenza Sopravvivenza per seguire tracce in presenza di odori coprenti passa da 10 a 20.


\begin{center}
	\includegraphics[width=0.9\linewidth]{immagini/argus2.png}
	
\textit{Argus Panoptes Guarding the Heifer (Io), Red Figure pitcher, c. 460 BC Museum of Fine Arts, Boston}
\end{center}


\subsection{Vista Cieca}\index{Vista Cieca}

Utilizzando sensi diversi dalla vista, come la percezione delle vibrazioni, un fiuto sensibile, un udito acuto o un sonar, una creatura dotata di vista cieca si muove e combatte bene quanto una creatura dotata della vista.

Invisibilità, buio e la maggior parte delle forme di copertura sono inutili, anche se la creatura dotata di vista cieca deve avere una linea di effetto per notare una determinata creatura o oggetto.

Il raggio della capacità è indicato nella descrizione della creatura. La creatura, in genere, non deve effettuare prove di Consapevolezza per notare creature entro il raggio della sua vista cieca.

A meno che non sia diversamente indicato, la vista cieca è sempre attiva e la creatura non deve compiere azioni per attivarla. Alcune forme di vista cieca devono essere attivate come azione immediata. In questo caso, viene indicato nella descrizione della creatura.

Se una creatura deve attivare la vista cieca, ne ottiene i benefici solo durante il proprio round.

Una creatura eterea non è visibile alla vista cieca.

\subsection{Senso Tellurico}\index{Senso Tellurico}
Una creatura dotata di Senso Tellurico è sensibile alle vibrazioni del suolo, e può automaticamente individuare qualsiasi cosa sia in contatto con il terreno entro il raggio specificato dal Senso Tellurico.

Le Creature Acquatiche dotate di Senso Tellurico (ecolocalizzazione) possono percepire la posizione di creature in contatto con l'acqua.

Il raggio della capacità è specificato nel testo descrittivo della creatura

\end{multicols}

\medskip

\begin{center}
	\includegraphics[height=0.65\linewidth]{immagini/grabroid.png}
	\medskip
	\textit{Grabroid. Conosciuti anche come Agguantatori. Tremors (Film)}
\end{center}

\pagebreak

\section{Le Caratteristiche}\index{Caratteristiche}

\label{le-caratteristiche}

\begin{enfasi}{Vivere non è respirare: è agire, è fare uso degli organi, dei sensi, delle facoltà, di tutte quelle parti di noi stessi per cui abbiamo il sentimento di esistere. (Jean-Jacques Rousseau)}\end{enfasi}\medskip

\begin{multicols}{2} 
	
\lettrine[lines=2, lhang=0.33, loversize=0.25, findent=1.5em]{O}{gni} personaggio ha 6 Caratteristiche (chiamate anche Statistiche) che rappresentano i suoi attributi base e costituiscono il suo potenziale talento e capacità innata.


Anche se non è comune che un personaggio effettui una prova usando soltanto una sua Caratteristica, i punteggi di Caratteristica influiscono praticamente su ogni aspetto delle capacità e competenze del personaggio.

Le 6 caratteristiche sono:

\textbf{Forza}\index{Forza}: indica la forza fisica, un personaggio con un punteggio di Forza pari a -5 è morto.

\textbf{Destrezza}\index{Destrezza}: indica la capacità di coordinamento, riflessi ed agilità del personaggio, Un personaggio con un punteggio di Destrezza pari a -5 è incapace di muoversi ed è completamente immobile (ma non privo di sensi).

\textbf{Costituzione}\index{Costituzione} indica la resistenza agli sforzi del personaggio. Un personaggio con Costituzione -5 non ha più il controllo del proprio corpo ed è morto.

\textbf{Intelligenza}\index{Intelligenza}: indica la componente razionale, logica, cognitiva del personaggio. Un personaggio con un punteggio di Intelligenza pari a -5 è in stato di coma.

\textbf{Saggezza}\index{Saggezza}: indica la forza di volontà, il buon senso, la perspicacia e l'intuito del personaggio. Un personaggio con un punteggio di Saggezza pari a -5 è incapace di pensiero razionale ed è privo di sensi.

\textbf{Carisma}\index{Carisma}: misura la forza della personalità, la capacità di persuasione, il magnetismo personale, la predisposizione al comando e il fascino di un personaggio. Un personaggio con un punteggio di Carisma pari a -5 è privo di sensi.

\smallskip

Ogni punteggio di Caratteristica in genere va da 0 a 3 un punteggio di Caratteristica buona è 1, 2 ottima, 0 è "normale", 3 è giudicato "eccezionale".

\textbf{I modificatori razziali non possono alzare o abbassare i punteggio oltre +3/-3}.

Un punteggio di -1 e giudicato debole, un -2 subnormale, un -3 severamente problematico, un -4 porta quasi ad un non utilizzo della caratteristica, un -5 è opportuno che stia nel letto e basta (se non è già in una bara).

\begin{center}
	\includegraphics[width=0.6\linewidth]{immagini/dice3.png}
\end{center}

\subsection{Assegnazione punteggi Caratteristiche} \hypertarget{assegnazione.punteggi.caratteristica}{} 

I punteggi delle caratteristiche hanno un ruolo importante ma non fondamentale.

Il giocatore deve capire che un punteggio "basso" non significa avere un pessimo personaggio, ma bensì si divertirà di più nel ruolarlo facendo leva sulle caratteristiche e capacità peculiari, usando ingegno e arguzia.

Sono presentati più sistemi per tirare le caratteristiche.

Personalmente suggerisco l'approccio della \textbf{Modalità Base}. In DBS i personaggi non sono eroi, non sono i prescelti che si ergono a difensori del pianeta. I personaggi sono persone normali coinvolte spesso loro malgrado in situazioni al limite se non oltre la sopravvivenza.

L'indubbio vantaggio di tirare i valori in ordine per le caratteristiche permette di scombinare gli schemi ed evitare build fatte a tavolino.

E' probabile che non vengano i risultati che speravate od addirittura siano venuti in caratteristiche che non vi interessavano. Va bene così. Cambiate idea, fatevi ispirare dai valori ottenuti! Divertitevi con il nuovo personaggio, costruite qualcosa di nuovo e diverso, lasciatevi stupire.

La \textbf{Modalità vorrei che} permette di assegnare un determinato peso (numero di dadi) alla statistica così che sia più probabile ottenere il punteggio desiderato.

\begin{center}
	\includegraphics[height=0.9\linewidth]{immagini/cavaliereprimopiano.png}
\end{center}


Infine la \textbf{Modalità opzionale} di assegnazione punti va concessa solo per avventure dove i personaggio sono gli eroi, creature che si distinguono e si elevano tra le masse e non persone piuttosto comuni che hanno la sfortuna di affrontare situazioni mortali.

\textbf{In ogni caso nessuna caratteristica tirata può avere un valore inferiore a -3 o superiore a +3 comprensivo dei modificatori razziali.}

In ultimo ricordate che DBS è un gioco di ruolo dove la morte del personaggio capita, anche più spesso che in altri GDR. Create dei validi e concreti personaggi e lasciate che sia l'avventura vissuta a forgiare i dettagli.


\subsubsection{Modalità base}\index{Modalità base}

Il giocatore tira 4d6 per ogni caratteristica ed in ordine. 

Per ogni caratteristica tirata controlla il risultato dei dadi tirati scartando il dado con il valore più basso.

\textbf{Ogni 1 e 2 tirato vale -1, 5 e 6 vale +1, 3 e 4 valgono +0}.

Si sommano i tre valori per i dadi scelti e si determina il punteggio della caratteristica.

\textbf{Eccezione}: se nel tiro di una Caratteristica e ho fatto tutti 6 allora il punteggio è +4

\subsubsection{Modalità vorrei che..}\index{Modalità che vorrei}

Il valore delle Caratteristiche del personaggio è affidato ad un pool di dadi che il giocatore assegna alle statistiche. Il giocatore può assegnare un totale di 18d6 divisi tra le 6 Caratteristiche. Potrebbe assegnare 2d6 alla Forza, 4d6 alla Destrezza, 3d6 alla Costituzione, 4d6 alla Intelligenza, 2d6 alla Saggezza, 3d6 al Carisma.

Non è possibile assegnare più di 4d6 o meno di 2d6. Tirate i dadi assegnati e sommate i risultati in base al totale ottenuto determinate il punteggio come mostrato nella tabella.


\medskip

\begin{tabular}{ll}
\textbf{Totale dei dadi}& \textbf{Valore Caratteristica}\\
	\toprule
3 (o meno)&-3\\
4-5&-2\\
6-7-8&-1\\
9-10-11-12&+0\\
13-14-15&+1\\
16-17&+2\\
18 (o più)&+3\\
\end{tabular}

\medskip

Non c'e' \textbf{Eccezione} del +4 nella Modalità che vorrei

Ripetete i tiri per tutte e 6 le caratteristiche e tenete i risultati nell'ordine tirato, quindi Forza, Destrezza, Costituzione, Intelligenza, Saggezza e Carisma.

\subsubsection{Modalità opzionale (per i codardi!)}\index{Modalità opzionale per i codardi}

Se il Narratore lo concede è possibile che ogni giocatore distribuisca 4 punti tra le 6 Caratteristiche, ogni Caratteristica deve avere come minimo un punteggio di -1 e come massimo 2 prima dei modificatori razziali.

\begin{center}
	\includegraphics[width=0.7\linewidth]{immagini/guerrieroispirato.png}
\end{center}

\subsection{Aumentare le caratteristiche}

\textbf{Ogni quattro livelli} (4, 8, 12, 16, 20..) si può aumentare di un punto una caratteristica, fino a raggiungere un massimo di valore 5. Per aumentare oltre 5 sono necessario oggetti magici o magie.

\subsection{Descrizione delle Caratteristiche}

Il punteggio delle caratteristiche non è tutto in un personaggio ne tanto meno in un mostro.

I mostri più "istintivi" ed aggressivi avranno sicuramente punteggi negativi di Intelligenza e Carisma, ma non per questo sono "stupidi", semplicemente agiscono in base ai loro schemi naturali.

Un mostro con Forza -4 non è prossimo a morire, semplicemente ha pochissima forza (immaginate di dare un valore di Forza ad un topo od uno scoiattolo se non ad un ragno..)

\subsubsection{Forza}\index{Forza}

\begin{enfasi}{
Ah, è cosa eccellente possedere la forza d'un gigante, ma usarla da gigante, è tirannia! (William Shakespeare, Isabella: da “Misura per misura”, atto II, scena II)		
}\end{enfasi}


La Forza misura la potenza fisica, l'atletismo e i limiti della forza bruta che puoi esprimere.

Una prova di Forza può essere impiegata per qualsiasi tentativo di sollevare, spingere, tirare o spaccare qualcosa, per spingere il tuo corpo all'interno di uno spazio, o una qualsiasi altra applicazione di forza bruta. 

\subsubsection{Destrezza}\index{Destrezza}

\begin{enfasi}{
Abbaiare stanca. La forza non conta niente nella vita. Saper schivare è quello che conta. (Daniel Pennac)
}\end{enfasi}

La Destrezza misura l'agilità, i riflessi e l'equilibrio. 

Una prova di Destrezza può essere impiegata per qualsiasi tentativo di muoversi agilmente, schivare un colpo o per evitare di perdere l'equilibrio o borseggiare.


\subsubsection{Costituzione}\index{Costituzione}

\begin{enfasi}{
Un po' di salute ogni tanto è il miglior rimedio per l'ammalato. (Friedrich Nietzsche)
}\end{enfasi}


La Costituzione misura la salute, la vigoria e la forza vitale.

Una prova di Costituzione può essere impiegata per i tuoi tentativi di spingerti oltre i normali limiti del tuo corpo e per prove di resistenza e durata.


\subsubsection{Intelligenza}\index{Intelligenza}

\begin{enfasi}{
La forza senza intelligenza rovina sotto il suo stesso peso. (Orazio)
}\end{enfasi}

L'Intelligenza misura l'acume mentale, l'accuratezza dei ricordi e la capacità di ragionare.
Una prova di Intelligenza entra in gioco quando hai bisogno di affidarti alla logica, l'istruzione, la memoria o le capacità deduttive. 

Le tue prove di Intelligenza (Arcana) misurano la tua capacità di ricordare informazioni su
incantesimi, oggetti magici, simboli esoterici, tradizioni magiche, i piani dell'esistenza e gli abitanti di quei piani.

Rovistare tra antiche pergamene alla ricerca di un frammento di conoscenza potrebbe richiedere una prova di Intelligenza.

\subsubsection{Saggezza}\index{Saggezza}

\begin{enfasi}{
La forza non deriva dalla capacità fisica. Deriva da una volontà indomita. (Mahatma Gandhi)}\end{enfasi}

La Saggezza riflette la tua sintonia con il mondo circostante e rappresenta la perspicacia e l'intuito. 

Una prova di Saggezza riflette uno sforzo per interpretare il linguaggio corporeo, comprendere i sentimenti di qualcuno, notare dettagli dell'ambiente o curare una persona ferita. 

\subsubsection{Carisma}\index{Carisma}

\begin{enfasi}{
Kogami, tu sai che cos'è il carisma?

– Per come la vedo io, è un'attitudine innata, come quella di un eroe o di un
leader.

– [...] Gli elementi che identificano il carisma sono tre: l'indole innata degli eroi e dei profeti, la capacità di infondere benessere agli altri con la sola presenza e una cultura che ti permetta una conversazione brillante su ogni argomento. (Psycho-Pass)
}\end{enfasi}

Il Carisma misura la tua capacità di interagire efficacemente con il prossimo. Comprende fattori come la sicurezza e l'eloquenza, e può rappresentare una personalità affascinante o autoritaria.

Una prova di Carisma può essere richiesta quando cerchi di influenzare o intrattenere altre persone, quando cerchi di fare impressione o raccontare una menzogna, o quando devi barcamenarti in una complicata situazione sociale. 

Tipiche situazioni di utilizzo del Carisma includono tentativi di raggirare una guardia, truffare un mercante, guadagnare soldi al gioco d'azzardo, farsi passare per qualcun altro grazie a un travestimento, fugare i sospetti di qualcuno con false rassicurazioni, o mantenere un volto imperturbabile mentre si racconta una lampante menzogna.

\medskip

\begin{tcolorbox}[title = Tiriamo le Caratteristiche di Tups]

* \textbf{Il primo tiro}: 1,1,4,6 la caratteristica di Forza ha valore +0.
	
* \textbf{Secondo tiro}: 5,6,6,6 la caratteristica di Destrezza ha valore +3 (nessuna Caratteristica può avere punteggio oltre +3).
	
* \textbf{Terzo tiro}: 1,1,2,1 la caratteristica di Costituzione ha valore -3 (nessuna Caratteristica può avere punteggio inferiore a -3).
	
* \textbf{Quarto tiro}: 6,6,6,6 la caratteristica di Intelligenza ha valore +4 (eccezione per i 4 sei)
	
* \textbf{Quinto tiro}: 3,4,2,5 la caratteristica di Saggezza è +1

\textbf{Sesto tiro}: 3,4,2,1 la caratteristica di Carisma è -1
\end{tcolorbox}

\end{multicols}


\pagebreak


\begin{multicols}{2} 

\section{Punti Ferita}\index{Punti Ferita}\index{Punti Ferita}

\begin{enfasi}{Chi non stima la vita, non la merita. (Leonardo da Vinci)}\end{enfasi}\medskip


\lettrine[lines=2, lhang=0.33, loversize=0.25, findent=1.5em]{I}{ Punti Ferita} rappresentano l'energia vitale del personaggio e finché il personaggio/avversario ha almeno 1 punto ferita combatterà e lotterà al meglio delle sue capacità.

Ogni personaggio parte con 4 punti ferita al primo livello + il punteggio della Costituzione.

Ad ogni livello, oltre il primo, guadagna 1d4 Punti Ferita + il punteggio della Costituzione.

Ogni punto preso in Competenza Armi aumenta i punti ferita presi di 3. Ulteriori Abilità possono alzare questo punteggio.

Segna nella scheda i Punti Ferita (Punti Ferita) totali che hai e indica il valore attuale di volta in volta che per vari motivi di gioco ne perdi o riprendi. 
Segna sulla scheda sempre qual é il totale di punti ferita attuale, dopo ogni colpo o danno.


\begin{center}
	\includegraphics[width=0.9\linewidth]{immagini/heart3.png}
\end{center}

\bigskip

I punti ferita si recuperano in diversi modi:

\begin{itemize}
	\item
	      per ogni notte di riposo (almeno 8 ore) recuperi in Punti Ferita il valore di Costituzione + CA (con un minimo di 1)
	\item
	      Tramite incantesimi curativi (magie, pozioni.. o altri oggetti magici)
	\item
	      Competenza Pronto Soccorso, tramite trattamenti più o meno lunghi
\end{itemize}

\medskip

I Punti Ferita possono essere anche temporanei\index{Punti Ferita Temporanei} ovvero aggiunti o tolti temporaneamente ai tuoi attuali. Un'arma o effetto che causa danni non letali vuole dire che causa ferite temporanee. I Punti Ferita Temporanei vanno tolti per primi quando si viene feriti

Se non esplicitato diversamente i punti ferita temporanei scompaiono dopo un ora da quando si sono aggiunti.

I punti ferita temporanei non possono essere superiori alla metà del massimo dei punti ferita reali.


\columnbreak

\section{Punti Fato - Fortuna del Principiante}\index{Punti Fato}
\begin{enfasi}{Se il destino è contro di noi, peggio per lui. (motto del 1º Reggimento Carabinieri Paracadutisti "Tuscania")}\end{enfasi}\medskip

\lettrine[lines=2, lhang=0.33, loversize=0.25, findent=1.5em]{I}{n} un mondo non facile la Fortuna del Principiante aiuta chi non ha esperienza.
Ogni personaggio ha un numero di Punti Fato pari a (20 - Livello)/5, con un minimo di 1. I Punti Fato si conteggiano per sessione di gioco.

Ad ogni sessione si azzerano e si ricalcolano, ne consegue che non si accumulano Punti Fato tra una sessione di gioco e l'altra.

Esempio di calcolo:

Un personaggio di livello 6 ha: 20-6 = 14/5 = 3 (arrotondi per eccesso) Punti Fato da usare nella sessione. I Punti Fato non usati non si cumulano per la sessione successiva.

Un Punto Fato si usa come Azione di Reazione ed ogni Punto Fato utilizzato concede un bonus di +1d6 alla prova in corso. 

Il Punto Fato può essere utilizzato per avere un dado in più nel Tiro Salvezza oppure nei Tiri per Colpire, con potenziale esplosione del dado, oppure una prova di competenza.

Non può essere usato per aumentare valori fissi (come la Difesa).



\begin{center}
	\includegraphics[width=1\linewidth]{immagini/rabbits-foot.png}
	\textit{La zampa fortunata del coniglio...}
\end{center}

I Punti Fato si devono dichiarare prima del tiro, una volta dichiarato l'ammontare di Punti Fato che si vogliono utilizzare non é possibile utilizzarne di più o di meno.

\end{multicols}

\pagebreak

\section{I Tratti}\index{Tratti}\hypertarget{tratti}{} 

\label{tratti}
\begin{enfasi}{Chi dunque sa fare il bene e non lo compie, commette peccato. (Giacomo il Giusto 4.17, Lettera di Giacomo. NdA riferendosi ai Tratti scelti)
\smallskip		
		
E' un diritto naturale saziarsi l'anima con la vendetta. (Attila)
\smallskip

Est Sularus Oth Mithas. ("Il mio onore è la mia vita", Giuramento dei Cavalieri di Solamnia.)}\end{enfasi}\medskip

\begin{multicols}{2}

\index{Tratti}
\lettrine[lines=2, lhang=0.33, loversize=0.25, findent=1.5em]{I}{n} DBS non c'è una netta distinzione tra bene e male, legge e caos, tra ciò che è giusto e ciò che è sbagliato.

In DBS esistono i Tratti, aspetti e sfumature caratteriali che contribuiscono al background del personaggio, aiutano il giocatore a ruolare meglio e gli possono fornire quelle linee guida per interpretare in maniera più corretta il personaggio che ha voluto creare.

Un Tratto è un dettaglio che aiuta meglio a inquadrare il personaggio, ne delinea le caratteristiche caratteriali principali concedendogli sfumature diverse.

\textbf{Ogni giocatore sceglie 5 Tratti per il proprio personaggio alla creazione del giocatore.} Questi saranno le "bussole morali, etiche e comportamentali" che guideranno il personaggio nell'agire e nelle scelte.

I Tratti non sono il personaggio, non lo bloccano ne lo fissano eterno nel tempo. Un personaggio è sempre in costante evoluzione e così il suo carattere, morale, comportamento e desideri. Non essere rigido ma usa i Tratti per darti delle linee guida entro cui muoverti e farti ispirare.

\textbf{Dei Tratti scelti al primo livello individuane uno, questo partirà, sempre al primo livello, con valore 1, gli altri partiranno a valore 0.}

Col passare del tempo e delle avventure potranno essere guadagnati o sostituiti (in concerto tra Narratore e giocatore in base a come giocato) da altri Tratti.

Piu' è alto un valore di Tratto più questo è presente e permeante nelle scelte del personaggio.

Potranno essere anche enfatizzati certi Tratti, ovvero il Narratore a seguito di particolari scene e ruolate potrà fare aumentare di un punto un Tratto del personaggio.

Ad esempio a seguito di una particolare scelta e climax di avventura il Narratore potrebbe concedere a tutti o qualcuno Tratto Coraggioso o dare un +1 a Coraggioso a chi ha già questo Tratto. Per i Tratti non presi si considera il valore base in punti di -1. ovvero il primo punto serve per prendere il Tratto ed i successivi per enfatizzarli.

Mentre è "relativamente" facile acquisire nuovi Tratti è estremamente difficile cambiare quelli già presenti. Parlane con il Narratore, saprà preparare situazioni ed avventure che ti aiuteranno a comprendere come evolvere il personaggio ed eventualmente a cambiare i Tratti scelti.

Ogni azione particolarmente importante dove il personaggio abbia seguito un Tratto porta il personaggio ad avvicinarsi al Patrono (o Patroni) competente per quel Tratto.

Nella scheda troverai dei check da mettere vicino ai tratti, questi vengono segnati a seguito di azioni idonee ad accrescere il valore del tratto; raggiunti i 10 punti il Tratto aumenterà di 1 punto e si ricomincerà a segnare una nuova decina.

Sarà il Narratore durante l'avventura a dirti quando segnare, o cancellare, dei punti parziali. \textbf{In linea di massima si presume che un personaggio acquisti almeno un punto Tratto a livello.}

All'aumentare del valore del Tratto il personaggio potrà acquisire dei poteri, indipendentemente sia un credente (Seguace o Devoto) di quel Patrono o meno.


- A \textbf{5} punti si può incominciare a sentire la presenza di un Patrono legato ad un Tratto

- A \textbf{10} punti si sente la vicinanza di un Patrono legato ad un Tratto

- A \textbf{15} punti si è legati ad Patrono da un Tratto

- A \textbf{20} punti si è un Campione del Patrono legato ad un Tratto.


Non è necessario credere in un Patrono per sentirne la vicinanza, semplicemente è la propria natura (i propri Tratti) che è affine al Patrono, che lo si voglia o meno.

Dato che lo scopo di un Patrono è fare che i propri Tratti siano dominanti sugli altri, avere persone di alto livello e potere che siano così affini a lui tornerà utile nel giudizio dei 1000 anni.

\smallskip

Il valore dei Tratti solitamente aumenta con il passare del livello e dell'esperienza. Il Narratore può sempre decidere in base ad azioni particolarmente ispirate o comportamenti idonei di aumentare anche nel mezzo di un livello il valore di un tratto.

Il Narratore è libero di inserire nuovi Tratti a suo piacere o richiesti dai giocatori, si suggerisce di attribuire questi nuovi Tratti anche ai Patroni.


\end{multicols}

\pagebreak

\textbf{Tabella dei Tratti}\index{Tabella dei Tratti}

\medskip
\begin{multicols}{6}
\flushleft
	Abitudinario\\
	Accumulatore\\
	Affabile\\
	Affettuoso\\
	Aggressivo\\
	Agitatore\\
	Allegro\\
	Altezzoso\\
	Altruista\\
	Anarchico\\
	Ansioso\\
	Anticonformista\\
	Antipatico\\
	Aperto\\
	Apprensivo\\
	Arrabbiato\\
	Arrogante\\
	Attento\\
	Avaro\\
	Avventato\\
	Benevolo\\
	Brutale\\
	Bugiardo\\
	Buono\\
	Burbero\\
	Burlone\\
	Calcolatore\\
	Calmo\\
	Candido\\
	Caotico\\
	Caritatevole\\
	Casinista\\
	Casto\\
	Cattivo\\
	Cauto\\
	Clemente\\
	Codardo\\
	Combattivo\\
	Competitivo\\
	Conformista\\
	Confusionario\\
	Controllato\\
	Coraggioso\\
	Cortese\\
	Creativo\\
	Crudele\\
	Curioso\\
	Debole\\
	Deciso\\
	Determinato\\
	Diffidente\\
	Diretto\\
	Disciplinato\\
	Disordinato\\
	Disponibile\\
	Distaccato\\
	Distruttivo\\
	Doppiogiochista\\
	Egoista\\
	Emotivo\\
	Empatico\\
	Equilibrato\\
	Espansivo\\
	Esuberante\\
	Falso\\
	Fiducioso\\
	Franco\\
	Freddo\\
	Frignone\\
	Geloso\\
	Generoso\\
	Gentile\\
	Giusto\\
	Imbronciato\\
	Immaturo\\
	Impacciato\\
	Imparziale\\
	Impetuoso\\
	Implacabile\\
	Imprudente\\
	Impulsivo\\
	Incoerente\\
	Incontentabile\\
	Incostante\\
	Indifferente\\
	Indipendente\\
	Indisciplinato\\
	Indomito\\
	Industrioso\\
	Infantile\\
	Ingenuo\\
	Insensibile\\
	Integerrimo\\
	Introverso\\
	Ipocrita\\
	Iracondo\\
	Irascibile\\
	Ironico\\
	Irragionevole\\
	Irritabile\\
	Istintivo\\
	Leale\\
	Libero\\
	Logorroico\\
	Lunatico\\
	Lussurioso\\
	Maleducato\\
	Malinconico\\
	Malizioso\\
	Malvagio\\
	Masochista\\
	Materno\\
	Mattacchione\\
	Meticoloso\\
	Misurato\\
	Mite\\
	Morigerato\\
	Narcisista\\
	Negligente\\
	Nervoso\\
	Neutrale\\
	Ombroso\\
	Onesto\\
	Ordinato\\
	Osservatore\\
	Ottimista\\
	Paranoico\\
	Passionale\\
	Passivo\\
	Pasticcione\\
	Perfezionista\\
	Permaloso\\
	Pessimista\\
	Piagnucolone\\
	Pianificatore\\
	Pigro\\
	Pio\\
	Possessivo\\
	Pratico\\
	Previdente\\
	Protettivo\\
	Provocatore\\
	Prudente\\
	Rabbioso\\
	Razionale\\
	Ribelle\\
	Riflessivo\\
	Rigido\\
	Riservato\\
	Risoluto\\
	Saccente\\
	Sadico\\
	Sadomasochista\\
	Sarcastico\\
	Scherzoso\\
	Schietto\\
	Scontroso\\
	Scorbutico\\
	Semplice\\
	Sensibile\\
	Serio\\
	Sicuro\\
	Silenzioso\\
	Sincero\\
	Socievole\\
	Sognatore\\
	Solitario\\
	Sospettoso\\
	Spontaneo\\
	Sprovveduto\\
	Stravagante\\
	Superbo\\
	Superficiale\\
	Suscettibile\\
	Tenace\\
	Timido\\
	Tollerante\\
	Tradizionalista\\
	Tranquillo\\
	Triste\\
	Truffatore\\
	Valoroso\\
	Vanitoso\\
	Vendicativo\\
	Violento\\
	Volubile\\
	
\end{multicols}

\medskip

Se un giocatore non gioca i Tratti del personaggio non farà acquisire punti esperienza al personaggio stesso.

\medskip

\begin{center}
	\includegraphics[height=0.53\linewidth]{immagini/troll.png}
\end{center}



\pagebreak

\section{Competenze}\index{Competenze}

\label{competenze}
\begin{enfasi}{
Chi dice che una cosa è impossibile, non dovrebbe disturbare chi la sta facendo.

\medskip
Non hai veramente capito qualcosa fino a quando non sei in grado di spiegarlo a tua nonna. (Albert Einstein)}\end{enfasi}\medskip

\begin{multicols}{2}

\lettrine[lines=2, lhang=0.33, loversize=0.25, findent=1.5em]{O}{gni} personaggio ha una Professione iniziale, un percorso di vita e lavoro che l'ha portato ad apprendere determinate competenze.
Alla creazione del personaggio il giocatore sceglie una professione iniziale ovvero ciò che faceva (e volendo continua a fare) prima di cimentarsi in pericolose avventure.

Se non specificato diversamente per tutte le prove di competenza (Base, Attive) valgono tre regole base \index{Regole Base} chiamate \textbf{Golden Rules}:\index{Golden Rules}

\begin{itemize}
\item
I \textbf{6 esplodono}, ovvero se nella prova dei 3d6 un dado fa sei, somma il risultato e ritira, e se fa 6 nuovamente sommi il risultato e ritiri ancora e ancora..
\item
Gli \textbf{1 portano male}, se fai 1 con il dado togli 1 alla somma dei dadi tirati (e quindi il dado che ha fatto uno conta zero)
\item
\textbf{Affidarsi alla sorte}. Ogni 4 punti tra Competenza e Caratteristica che rinunci a sommare nella prova tiri un dado a 6 in più (Tiro per Colpire, Tiro Salvezza, prove Competenza). Questo quattro non può essere tolto dal punteggio dato da Abilità o oggetti magici.

\end{itemize}

\subsection{Competenze di Base}\index{Competenze di Base}

\label{competenze-di-base}

\begin{enfasi}{
Anche se indubbiamente il desiderio di conoscere è naturale per tutti gli uomini, la voglia di imparare non è cosa da tutti; la maggior parte, anzi, assaggiato quanto lo studio sia fatica e provata la stanchezza sulla propria pelle, butta alla leggera la noce ancor prima di aver rotto il guscio per gustarne il gheriglio. (Richard de Bury)

\medskip

Lo studio è per i perdenti! (Lobo)}\end{enfasi}\medskip


Sono qui elencate alcune Professioni di esempio e le loro competenze relative, il personaggio acquisisce queste competenze con il punteggio indicato nella tabella.

\end{multicols}

\textbf{Tabella: Elenco delle Professioni e relative Competenze}\index{Tabella Elenco delle Professioni e relative Competenze}\index{Professioni}

\medskip

\begin{tabularx}{0.95\textwidth}{lllll}
\textbf{\textbf{Professione}}	& \textbf{1 punto} & \textbf{2 punti} & \textbf{2 punti} & \textbf{3 punti}\\
	\toprule
\textbf{Accolito}			& Occulto			& Storia o Geografia	& Arcana			& Religione\\
\textbf{Alchimista}			& Valutare			&	Natura		& Erboristeria	& Arcana\\
\textbf{Allevatore}			& Sopravvivenza		&	Seguire tracce		& Natura	& Gestire animali\\
\textbf{Apprendista mago}	& Storia e Geografia&	Occulto&	Natura magica&	Arcana\\
\textbf{Azzeccagarbugli}	& Valutare&	Ingannare&	Percepire Emozioni&	Diplomazia\\
\textbf{Bibliotecario}		& Natura			&	Tradizioni locali&	Religione e Arcana&	Storia e Geografia\\
\textbf{Boscaiolo}			& Usare corda	&Natura& Orientamento&	Sopravvivenza	\\
\textbf{Cacciatore}			& Muoversi silenz.&	Seguire tracce&	Sopravvivenza	& Natura\\
\textbf{Carovaniere}		& Storia o Geografia&	Valutare&	Cavalcare&	Orientamento\\
\textbf{Teatrante}			& Suonare	& Intrattenere&	Giocoliere&	Acrobatica\\
\textbf{Erborista}			& Natura Magica&	Geografia&	Natura&	Erboristeria\\
\textbf{Giocatore di carte}	& Travestirsi&	Valutare&	Intrattenere&	Ingannare\\
\textbf{Guardia}			& Percepire Emozioni	&	Conoscenza Legge&	Cavalcare&Intimidire\\
\textbf{Guida}				& Natura magica&	Dungeon	&Natura&	Geografia\\
\textbf{Borseggiatore}	    & Disattivare congegni&	Artista della fuga&	Muoversi silenz.&	Mani di fata\\
\textbf{Delinquente}		& Sopravvivenza&	Cavalcare&	Valutare&	Nascondersi\\
\textbf{Locandiere}			& Pronto soccorso	&Valutare&	Percepire Emozioni&	Diplomazia\\
\textbf{Mercante}			& Lingue	&Tradizioni Locali&	Valutare&	Ingannare\\
\textbf{Minatore}			& Usare corde	&Valutare&	Orientamento&	Dungeon\\
\textbf{Pescatore}			& Orientamento&	Nuotare	&Usare corde&	Natura\\
\textbf{Soldato}			& Nuotare&	Gestire animali&	Saltare&	Cavalcare\\
\textbf{Carrettiere}		& Tradizioni Locali & Orientamento & Gestire Animali & Cavalcare\\
\textbf{Uomo di medicina}	& Natura magica&	Natura&	Erboristeria&	Pronto soccorso\\
\textbf{Guardia boschi}		& Natura magica	&	Erboristeria	&	Cavalcare & Natura	\\
\textbf{Agricoltore}		& Sopravvivenza &Erboristeria& Gestire Animali &  Natura\\
\end{tabularx}

\bigskip

\begin{multicols}{2}

Nella scheda va segnata la Professione iniziale e le competenze acquisite, ovviamente in accordo con il Narratore è possibile selezionare delle competenze diverse e anche scegliere professioni diverse!

Ad ogni nuova professione che andrete a creare associate 4 competenze prese da questa lista, una competenza partirà con un punteggio di 1, due competenze partiranno con il punteggio di 2 e quella più specifica e professionale partirà con il punteggio di 3.

Ovviamente una professione non si esplicita in sole 4 competenze ma queste sono quelle che più entreranno in uso durante le avventure, il Narratore sarà aiutato dalla vostra professione a capire chi e come sa fare e come possono agire i personaggi.

Questo è l'\textbf{elenco delle competenze} da cui scegliere per eventuali nuove professioni o personalizzazioni delle stesse, tra parentesi viene indicata la Caratteristica su cui si effettua la prova.

In accordo con il Narratore è anche possibile cambiare l'ordine delle Competenze rendendo più capace il personaggio in alcune piuttosto che altre.

\subsubsection{Quinta competenza}

Un giocatore alla creazione del personaggio può anche chiedere di abbassare una competenza da 2 punti a 1 punto e quella da tre punti a 2 punti, per avere una quinta competenza ad 1 punto.

\end{multicols}

\medskip

\textbf{Tabella: Elenco Competenze e relativa Caratteristica d'uso}\index{Tabella Elenco Competenze e relativa Caratteristica d'uso}

\begin{tabularx}{0.95\textwidth}{XXXXX}
	\textbf{Forza} & \textbf{Destrezza} & \textbf{Intelligenza} & \textbf{Saggezza} & \textbf{Carisma}\\
	\toprule
	Arrampicarsi	&	Acrobatica  			& Arcana 			& Cavalcare 		& Diplomazia \\
	Intimidire 		& 	Artista della fuga  	& Conoscenza*  		& \textit{Consapevolezza} 	& Intrattenere \\
	Nuotare 		&	Giocoliere  			& Dungeon			& Gestire animali    		& Ingannare  \\
	Saltare 		&	Mani di fata 			& Erboristeria 		& Natura		& Suonare \\
	&	Muoversi silenziosamente& Disattivare congegni &Orientamento & Tradizioni locali \\
	&   Nascondersi	& Falsificare 		& Percepire Emozioni 	& \\
	&	Usare corda & Lingue 				& Pronto soccorso 	&\\
	&							& Natura magica			& Religione			&\\
	&							& Valutare				& Seguire tracce &\\
	&							&						& Sopravvivenza&\\
\end{tabularx}

\medskip

La \textbf{Conoscenza} va esplicitata su quale argomento tratto: Legge, Piani, Occulto, Architettura ed Ingegneria, Nobiltà ed Araldica, Miti e Leggende, Storia, Geografia, Lingue ...

\medskip

\begin{multicols}{2}

Ad ogni livello successivo al primo distribuisci un numero di punti pari la metà del punteggio di Intelligenza +1 (Int/2)+1 con un minimo di 1 punto, tra le competenze già conosciute o perfezionate nell'avventura o apprese ex novo. 

\textbf{Nessuna competenza di base può avere un punteggio superiore al livello del personaggio+1.}

\subsubsection{Consapevolezza}

Una competenza che hanno tutti i personaggi è \textbf{Consapevolezza}, ovvero la capacità di percepire l'ambiente intorno a loro. Questa competenza ha un punteggio fisso pari a 1/3 del livello del personaggio (arrotondato per eccesso) e può essere aumentata fino ad un massimo pari a 2/3 del livello tramite Abilità quali Percettivo.

I giocatori più che usare Consapevolezza per ricercare informazioni dovrebbero fare domande, indagare, curiosare, arguire ipotesi e confrontarsi e non limitarsi a chiedere un tiro di Consapevolezza per trovare qualcosa.

\subsubsection{Apprendere nuove competente, professioni} 

Un personaggio può apprendere una nuova competenza con uno studio/pratica di almeno 4 ore al giorno per almeno 4 mesi. Dopo questo lasso di tempo il giocatore può assegnare un punto alla competenza di base per cui si è applicato.

Può anche, utilizzando lo stesso sistema, migliorare la sua conoscenza di una competenza, ulteriore requisito è trovare un insegnate che abbia un punteggio in competenza pari o superiore a quello a cui mira il personaggio.

Per apprendere una nuova professione deve passare almeno 6 mesi per 6 ore al giorno con chi pratica quella professione. Passati i 6 mesi ed almeno 1 livello il personaggio acquisisce le 4 competenze della professione. Se ha già in parte quelle competenze aumenta il punteggio di 1 per ogni competenza già posseduta.

\subsubsection{Competenze ed i loro ambiti di utilizzo}

\textbf{Acrobatica (DES)}: Questa competenza serve per mantenere l'equilibrio su superfici strette o precarie, per tuffarsi, rotolare, fare capriole, salti mortali e superare degli ostacoli.

\textbf{Arcana (INT)}: Con questa competenza si è esperti di magia e di incantesimi, di creazione di oggetti magici, e si è in grado di identificare gli incantesimi che vengono lanciati.

\textbf{Arrampicarsi (FOR)}: Con questa competenza si possono Scalare superfici verticali, dalle mura cittadine alle pareti rocciose.

\textbf{Artista della fuga (DES)}: Con questa competenza ci si può liberare da legacci

\textbf{Cavalcare (SAG)}: Con questa competenza è possibile dare comandi alla propria cavalcatura.

\textbf{Conoscenza di Architettura ed Ingegneria (INT)}: Sei un esperto costruttore e sai valutare la struttura delle costruzioni. Sai anche riconoscere stili architettonici e creare progetti d'interno e d'arredo.

\textbf{Conoscenza dei Miti e Leggende (INT)}: Si ha una e vera propria passione per i miti e leggende tradizionali e più remoti. Conosci località, storia e creature leggendarie.

\textbf{Conoscenza di Nobiltà e Araldica (INT)}: Conosci linee nobiliari, casate, dicerie, stemmi araldici, personalità ed i maggiori possedimenti e tesori.

\textbf{Conoscenze dei Piani (INT)}: Con questa competenza si è esperti di Piani d relativi abitanti.

\textbf{Conoscenze Occulte (INT)}: Con questa competenza si è esperti di occulto, creature immondi.

\textbf{Diplomazia (CAR)}: Con questa competenza si possono risolvere diverbi, e raccogliere preziose informazioni e dicerie dalle persone. La competenza è anche usata per negoziare in modo efficace con la giusta etichetta e condotta adatta alla situazione controversa.

\textbf{Disattivare congegni (INT)}: Con questa competenza si possono disarmare Trappole e aprire serrature, sabotare congegni meccanici semplici, come le catapulte, le ruote di un carro o le porte.

\textbf{Dungeon (INT)}: Con questa competenza si hanno conoscenze di Aberrazioni, caverne, esplorazioni sotterranee, Melme

\textbf{Erboristeria (INT)}: Con questa competenza si hanno conoscenze di come preparare pozioni e veleni naturali. Il punteggio si applica alle prove per distillare pozioni.

\textbf{Falsificare (INT)}: Con questa competenza si sa falsificare oggetti d'arte, mappe, firme..

\textbf{Geografia (INT)}: Con questa competenza si hanno conoscenze sul clima, popolazione, terreni, territori, nazioni e confini.

\textbf{Gestire animali (SAG)}: Con questa competenza è possibile addestrare e ammansire animali

\textbf{Giocoliere (DES)}: Con questa competenza si è estremamente abili ad afferrare piccoli oggetti

\textbf{Intimidire (FOR)}: Intimidire si base sull'approccio fisico nel convincere l'interessato.

\textbf{Ingannare (CAR)}: La competenza Ingannare può essere usata per Raggirare (dicendo quindi fandonie) o Persuadere (adattando la verità) al fine di convincere delle proprie parole l'interessato.

\textbf{Intrattenere (CAR)}:  Con questa competenza si è esperti in diversi tipi di espressione artistica, dal canto alla recitazione.

\textbf{Lingue (INT)}: Ogni punto in questa competenza permette di apprendere un nuovo linguaggio scritto e parlato. Un buon punteggio di Lingue aiuta a comprendere lingue non note ed a farsi comprendere.

\textbf{Mani di fata (DES)}: Con questa competenza si può borseggiare, estrarre un'arma nascosta, oppure compiere altre azioni senza essere notati.

\textbf{Muoversi silenziosamente (DES)}: Con questa competenza si è in grado di muoversi senza causare rumore.

\textbf{Nascondersi (DES)}: Con questa competenza si è in grado di passare inosservati stando fermi.

\textbf{Natura magica (INT)}: Con questa competenza si hanno conoscenze di Bestie Magiche, Costrutti, Draghi e creature magiche.

\textbf{Natura (SAG)}: Con questa competenza si hanno conoscenze di Animali, Fatati, stagioni e cicli, tempo atmosferico, vegetali.

\textbf{Nuotare (FOR)}: Con questa competenza si è in grado di nuotare, anche in acque tempestose. Senza competenza si sa stare a galla in acqua placide.

\textbf{Orientamento (SAG)}: Con questa competenza si ha il senso della direzione e orientamento rendendo impossibile perdersi indipendentemente dall'ambiente in cui ci si trova.

\textbf{Percepire Emozioni (SAG)}: Con questa competenza si può capire se qualcuno sta mentendo o si possono intuire le sue vere intenzioni.

\textbf{Pronto soccorso (SAG)}: Con questa competenza si possono curare le ferite e le malattie.

\textbf{Religione (SAG)}: Con questa competenza si hanno conoscenze su Patroni, mitologia, Celestiali, Non Morti, simboli sacri, tradizione ecclesiastica, feste e ricorrenze liturgiche.

\textbf{Saltare (FOR)}: Con questa competenza si è esperti e provetti saltatori.

\textbf{Seguire tracce (SAG)}: Con questa competenza si sa seguire le tracce lasciate nell'ambiente.

\textbf{Sopravvivenza (SAG)}: Con questa competenza si può sopravvivere e orientarsi nelle terre selvagge. La competenza è usata anche per cercare attivamente trappole e fosse.

\textbf{Storia (INT)}: Con questa competenza si hanno conoscenze di Storia quali guerre, migrazioni, colonie, fondazioni di città, accadimenti importanti..

\textbf{Suonare (CAR)}: Con questa competenza si sa suonare uno strumento (specificare quale) e si è capace di leggere la musica

\textbf{Tradizioni locali (CAR)}: Con questa competenza si hanno conoscenze degli abitanti (più noti), costumi, leggende, leggi, personalità, tradizioni. E' necessario specificare una regione geografica dove è applicabile la conoscenza.

\textbf{Travestirsi (CAR)}: Con questa competenza si è capaci di truccarsi e travestirsi per mascherare e apparire diversamente.

\textbf{Usare corda (DES)}: Con questa competenza si è esperti in legacci e nodi per fissare e bloccare oggetti o persone.

\textbf{Valutare (INT)}:  Con questa competenza si sa stimare il valore monetario di un oggetto.

\medskip

\begin{narratore}{
Non sottovalutate la scelta della Professione!. Non tutto può risolversi con asciate o magia. Sapere districare nodi, seguire tracce, riconoscere erbe o malattie fanno del personaggio un esperto, creano una professione. Non dovete definire il personaggio solo in base alle Abilità che ha ma in base a cosa e quanto bene sa farlo. Un personaggio di basso livello ma esperto di sopravvivenza sarà sempre più utile di un combattente esperto se si tratta di attraversare un deserto.}\end{narratore}

\subsection{Competenze Attive}\index{Competenze Attive}

\label{competenze-attive}
\begin{enfasi}{C'è solo un modo per allenarsi: quello giusto. (Carl Lewis)
\medskip
		
		Wang Chi: Sei pronto?
		
		Jack Burton: Io sono nato pronto! (Grosso guaio a Chinatown, Film 1986)
	}\end{enfasi}\medskip

Ogni personaggio prende 1 punto da distribuire nelle Competenze Attive a livello.

Le \textbf{Competenze Attive} sono: Competenza Magica, Competenza Armi oppure Tiri Salvezza (Riflessi, Tempra, Volontà).

\textbf{Competenza Magica (CM)}: \index{CM}\index{Competenza Magica} indica la capacità e competenza nel lanciare un'incantesimo.

\textbf{Competenza Armi (CA)}: \index{CA}\index{Competenza Armi} è la capacità e bravura di combattere con un'arma da mischia o da tiro/distanza.

I \textbf{Tiri Salvezza} vengono solitamente aumentati dalla scelte delle Abilità ma e' sempre possibile attribuire il  punto preso a livello.

Il \textbf{Tiro Salvezza su Tempra} indica quanto si è in grado di sopportare le sofferenze fisiche o attacchi contro la propria vitalità e salute. Ai Tiri Salvezza su Tempra si aggiunge il valore della \textbf{Costituzione}.

Il \textbf{Tiro Salvezza su Volontà} indica la resistenza contro l'influenza mentale ed altri effetti magici, ciò che vuole modificare il tuo libero arbitrio nelle scelte e nell'agire. Ai Tiri Salvezza su Volontà si aggiunge il valore di \textbf{Saggezza}.

Il \textbf{Tiro Salvezza su Riflessi} indica quanto si è agili e pronti per evitare ostacoli o magie. Ai Tiri Salvezza su Riflessi si aggiunge il valore di \textbf{Destrezza}.

\begin{tcolorbox}[title = Tiri Salvezza non standard] 
	E' possibile che vengano richiesti dei Tiri Salvezza con modificatori diversi, ovvero un Tiro Salvezza su Tempra con modificatore Forza oppure un Tiro Salvezza su Volontà con modificatore Carisma... Sarà il Narratore a dirvi quando si applica un modificatore diverso.\end{tcolorbox}

\medskip

Ogni punto attribuito (CA, CM, TS) permette di usufruire di +1 nella prova relativa (Tiro per Colpire, Competenza Magica o Tiro Salvezza scelto).

\subsubsection{Competenza Armi}

La \textbf{Competenza Armi} (abbreviata in \textbf{CA}) indica la capacità e bravura nel colpire l'avversario con armi.

Il \textbf{Tiro per Colpire per le armi da mischia}\index{Armi da mischia} si risolve con una prova di Competenza Armi (\textbf{CA}) + Forza + eventuali Abilità e bonus magici contrapposto alla Difesa dell'avversario (Destrezza + armatura/scudi/bonus).

Il \textbf{Tiro per Colpire con armi da distanza} \index{Armi da distanza}(archi, pugnali da lancio, sassi..) si risolve con una prova di Competenza Armi (\textbf{CA}) + Destrezza + eventuali capacità e bonus magici contrapposto alla Difesa dell'avversario (Destrezza + armatura/scudi/bonus).

Quando si assegna un punto ad \textbf{CA} va sempre precisata su quale gruppo di arma si prende, se non si dichiara allora è come averlo preso nel gruppo Armi Semplici.
Controllare l'elenco  \hyperlink{lista.armi}{Armi per Tipologia Omogenea}.\index{Tipologia Omogenea}

Il personaggio può decidere di assegnare il suo punto ad una tipologia che già conosce, migliorando così la sua capacità ed abilità nell'uso od apprendere un altra tipologia di arma.

Il giocatore deve considerare che migliore è la sua capacità con una tipologia di arma più facilmente può usufruire di vantaggi nella stessa, ma conoscerà meno armi.

Se il giocatore non ha assegnato alcun punto nella \textbf{CA} può utilizzare, senza penalità al colpire, solo le armi raggruppate come Armi Semplici.

Un personaggio nelle Liste d'Armi che conosce o nelle Armi Semplici applicherà sempre il suo valore di Competenza Armi (CA) al Tiro per Colpire, solo quando usa un arma non nota avrà delle penalità.

Le \textbf{Armi Semplici} sono: Pugnale, Mazza Leggera, Randello, Morningstar, Lancia corta da fante, Bastone, Balestra (Leggera), Giavellotto\index{Armi Semplici}

Usare un'\textbf{Arma senza l'adeguata competenza} nel gruppo di appartenenza impone un -1d6 al Tiro per Colpire.\index{Arma senza competenza}

Per poter utilizzare \textbf{Armature Leggere} è necessario avere almeno un punto in Competenza Armi.\index{Armature Leggere}

Per poter utilizzare \textbf{Armature Medie} e \textbf{Scudi Leggeri} o \textbf{Medi} è necessario avere almeno 2 punto in Competenza Armi.\index{Armature Medie}

Con almeno 3 punti in CA ed 1 in Forza si possono usare senza penalità \textbf{Armature Pesanti} e \textbf{Scudi Pesanti}.\index{Armature Pesanti}\index{Scudi Pesanti}

Usare un'\textbf{Armatura senza l'adeguata competenza} impedisce di usare il valore di Destrezza in Difesa ed il bonus conferito dall'armatura alla Difesa si riduce di 1.\index{Armatura senza competenza}

Usare uno \textbf{Scudo senza l'adeguata competenza} peggiora il Tiro per Colpire di 1 e lo scudo conferisce un bonus massimo a Difesa di 1.\index{Scudo senza competenza}

\subsubsection{Competenza Magica}

La \textbf{Competenza Magica} (abbreviata in \textbf{CM}) permette al personaggio di poter conoscere più incantesimi, più potenti, più efficaci e più facilmente.

Un personaggio con alta \textbf{CM} sa manipolare più incantesimi e con risultati migliori. La difficoltà a resistere all'incantesimo è data proprio dalla tua prova di magia.

Il soggetto all'incantesimo deve, ove richiesto, fare un Tiro Salvezza, basata sulla tua prova di magia!

La tua magia del Fulmine imporrà un Tiro Salvezza su Riflessi per cercare di evitarlo. 

La Difficoltà sarà pari alla tua prova di magia confrontato contro un Tiro Salvezza su Riflessi tirato con 3d6 + valore Tiro Salvezza Riflessi + bonus Destrezza +- varie ed eventuali.

Se sei tu a dover resistere ad una magia il Narratore non ti dirà di fare un Tiro Salvezza a difficoltà 18, è lui che confronta il tuo tiro con la difficoltà, potrà dirti che la prova è complessa, difficile o facile...

La Competenza Magica viene anche usata nei Tiri per Colpire con Incantesimi!

\begin{tcolorbox}[title = Tups arriva al 4' livello!] 
	
	Tups è arrivato al 4' livello! Ecco come ha distribuito i punti delle Competenze Attive.
	
	\textbf{1 livello}: +1 Competenza Armi, Abilità: Armatura del Devoto (+2 Vol, +1 Rif), Arma Focalizzata (+1 Rif, +1 Tempra)
	
	\textbf{2 livello}: +1 Competenza Magica
	
	\textbf{3 livello}: +1 Competenza Magica, Abilità: Fedele (+2 Vol, +1 Tempra)
	
	\textbf{4 livello}: +1 Tiro Salvezza Riflessi
	
	\textbf{\textit{Totale}}: +1 CA, +2 CM, +3 Tiro Salvezza Riflessi, +2 Tiro Salvezza Tempra, +4 Tiro Salvezza Volontà
	
\end{tcolorbox}


\end{multicols}

\bigskip

\begin{center}
	\includegraphics[width=0.75\linewidth]{immagini/giavellottiragazzo4.png}
\end{center}


\pagebreak

\section{Costruiamo il Personaggio}\index{Personaggio}

\label{costruiamo-il-personaggio}
\begin{enfasi}{
"Mai dimenticare chi sei, perché di certo il mondo non lo dimenticherà. Trasforma chi sei nella tua forza, così non potrà mai essere la tua debolezza. Fanne un'armatura, e non potrà mai essere usata contro di te." (Tyrion Lannister)
}\end{enfasi}\medskip

\begin{multicols}{2}


\lettrine[lines=2, lhang=0.33, loversize=0.25, findent=1.5em]{C}{ome} prima cosa prepara davanti a te la scheda ed un foglio dove prendere note ed appunti.

Parti immaginando, visualizzando l'aspetto del tuo personaggio. Come te lo immagini ? In possente guerriero, un agile spadaccino od un mago scavezzacollo alle prime esperienze ?

Individua il nome e pensa a ciò che conosce, quali esperienze ha avuto e quali lo hanno segnato.

Come si comporta con gli altri? è un tipo ordinato, ha delle fisse, ha qualche tic o abitudine ?


\begin{center}
	\includegraphics[width=0.9\linewidth]{immagini/Leonidas_I_of_Sparta.png}
	\textit{Leonida di Sparta}
\end{center}

\medskip

E' cresciuto in famiglia, in un clan, vagabondo, per strada.. cosa l'ha portato e che scelte ha fatto per arrivare fino ad adesso ?

Quale è il suo stile di combattimento e strategia tipo ? Magia, Spada, dalle retrovie.. incitare i compagni.. scappare...

E non meno importante: quale è il suo scopo ? cosa lo ha fatto uscire di casa, dalle sue sicurezze.. da una vita "normale" e intrapreso quella di avventuriere ?

Per incominciare leggi il capitolo sulle Razze ed individua quella del tuo personaggio.

Ricorda sempre che questo è un mondo crudele, pieno di rischi, trappole e mostri, ma anche occasioni che possono renderti potente o ricchissimo. 

\textbf{Non tutti nascono \textit{Potenti} ma chi si evolve è destinato a diventarlo}.

Recupera un pò di d6 e tira!

Consulta il capitolo delle \hyperlink{assegnazione.punteggi.caratteristica}{Caratteristiche} per capire quanto sei stato fortunato.

Se hai Intelligenza pari o superiore a 2 scegli un altra \hyperlink{linguaggi}{lingua} parlata/scritta oltre al Comune, e se hai 3 puoi sceglierne 2 di lingue in più.

Competenze Attive, qui hai 1 punto da distribuire tra Competenza Armi, Competenza Magica e Tiri Salvezza.

La Competenza Armi ti aiuta nel colpire meglio. La Competenza Magica è l'unica cosa che ti permette di usare la magia.

Ricorda anche che i punti in Competenze Armi vanno dichiarati a quale \hyperlink{lista.armi}{lista di armi} sono stati applicati.

Se non hai punti in Competenza Armi puoi usare solo le \hyperlink{armi.semplici}{armi semplici} senza incorrere in penalità al Tiro per Colpire.

Puoi assegnarlo anche ad un Tiri Salvezza. Questi valori determinano la tua capacità di sopravvivenza e di resistere a traumi e magie

Le Competenze Base vengono assegnata in base alla Professione scelta. Sceglila e attenzione con cura.

I Punti Ferita sono pari a 1d4 + Costituzione + 3 se hai messo 1 punto in Competenza Armi (CA).

Ad ogni livello successivo distribuisci un numero di punti pari a metà del punteggio di Intelligenza +1 tra le competenze già conosciute e perfezionate nell'avventura o apprese ex novo.

A questo punto scegli i \hyperlink{tratti}{Tratti}, 5. Fallo con attenzione, stai costruendo il tuo personaggio ed i Tratti delineano a forti pennellate il carattere. Ricordati che saranno fondamentali per la scelta del \hyperlink{patroni}{Patrono}.

Nella scheda nello specchietto dei Tratti dove c'è Patrono segna con una X i Tratti che ti collegano al Patrono che hai scelto (se lo hai scelto).

Ricorda infine che un personaggio "solitario" e "crudele" suona bene un in racconto dove è il solo protagonista, ma qui si gioca in gruppo, non prendere Tratti in ovvia opposizione agli altri o comunque non giocare da "stronzo", altrimenti il personaggio verrà naturalmente allontanato dagli altri giocatori.

Scegli lo \hyperlink{svantaggi}{svantaggio} di ruolo e se vuoi anche svantaggi e \hyperlink{linguaggi}{vantaggi}. Ricorda di giocarlo, altrimenti non è divertente.

Se hai messo dei punti in Competenza Magica a questo punto devi scegliere quali Incantesimi conosci.
Sul tuo Tomo della Magia puoi segnare un numero di incantesimi pari a 2 + il tuo modificatore di caratteristica da incantatore, di questi incantesimi ne puo conoscere 2 + metà del valore del modificatore di caratteristica da incantatore.

Passa alle \hyperlink{abilita}{Abilità}, al primo livello ne scegli due, stai attento ai prerequisiti ed anche ad eventuali Abilità che ti concede la tua razza. Ogni livello dispari prenderai un altra Abilità.

La scelta delle Abilità modifica pesantemente i tuoi Tiri Salvezza!

Scegli l'\hyperlink{equipaggiamento}{equipaggiamento}, \hyperlink{equipaggiamento.armature.scudi}{ armatura}, \hyperlink{equipaggiamento.armi}{armi}, zaino, due torce, qualche razione di cibo.. un peluche.. quello che ti sembra indispensabile per l'avventura.

Aggiorna poi la parte di scheda relativa alla Difesa segnando che bonus ti da l'armatura e scudo indossata.

Entra nella parte, concediti di giocare questo straordinario personaggio. Se mai ti stufassi di giocarlo e volessi provare qualcosa di diverso parlane con il Narratore, saprà consigliarti e suggerirti la strada migliore.
Oltretutto hai il vantaggio che in DBS le classi non esistono, il personaggio cresce evolve ed impara in base a ciò che fai e sperimenti.

In ultimo ricordati della Legge del Premio\index{Legge del Premio}. Yeru e' feroce, spesso malvagio, ancor di piu' vorrà ucciderti, eppure per chi sopravvive c'e' la Legge del Premio, una legge che neanche i Patroni possono violare. La Legge e' piuttosto semplice nel suo concetto base "A chi sopravvive andranno i Tesori e la Gloria".

\subsection{Avanziamo di Livello}\index{Livello}\index{Avanzamento di Livello}


\begin{enfasi}{
Ma ci sono cose che non si possono capire con la riflessione, bisogna viverle. (La storia infinita | Michael Ende)
}\end{enfasi}


Ogni qual volta il Narratore vi confermi il passaggio di livello sono da compiere diverse operazioni per aggiornare il personaggio.

\flushleft Innanzitutto prendete la scheda, matita e gomma e dadi (almeno il d4).

\flushleft - Aggiornare il Livello aumentandolo di 1.

- Distribuite 3 punti tra Competenza Armi, Competenza Magica, Tiri Salvezza (Tempra, Riflessi, Volontà). Massimo 1 punto per tipo.

- Aumentate i Punti Ferita di 1d4+Costituzione ed aggiungete 3 se avete dato 1 punto in Competenza Armi

- Se avete assegnato un punto in Competenza Armi stabilite se prendete una nuova Lista d'Armi o approfondite la conoscenza di una lista già appresa.

- Aggiornate lo schema dei Tiri Salvezza in base ai punti distribuiti

- Aggiornate la parte dei Tiri per Colpire in base al nuovo valore della Competenza Armi

- Distribuite (Int/2)+1, con un minimo di 1 punto, tra le Competenze Base conosciute o apprese durante le avventure.

- Se il livello è un multiplo di 4 (4,8,12,16..) aumentate di 1 un punteggio di una Caratteristica a vostra scelta (ed aggiornate eventualmente Tiro Salvezza e TC), fino ad un massimo punteggio di 5

- Ad ogni livello dispari (3,5,7,9..) prendete una nuova Abilità o migliorate una Abilità esistente, state attenti ai prerequisiti.

- Aggiornate i Punti Fato (20-lv)/5

- Aumentate un punteggio dei Tratti come vi dirà il Narratore. Verificate se avete raggiunto un punteggio sufficiente per acquisire poteri legati ai Tratti.

- Se avete aumentato la Competenza Magica apprendete 2 nuovi incantesimi oppure sacrificandone uno potete apprendere due trucchetti (Incantesimi a Difficoltà 12) presenti nel vostro Tomo della magia. Potete anche sostituire un incantesimo appreso con un altro non appreso ma presente nel Tomo.

- Aggiornate la seconda parte della scheda in base al nuovo punteggio di Competenza Magica

\medskip

\begin{center}
	\includegraphics[width=0.9\linewidth]{immagini/Alexander_and_Bucephalus_-_Battle_of_Issus_mosaic.png}
	\textit{Alessandro Magno}
\end{center}

\medskip

Come avrete notato i punteggi delle Competenze sono ridotti, si prendono pochi punti da distribuire alla volta.
Come giocatori avete l'opportunità di prediligere un approccio specializzato, ovvero "puntare" su poche e specifiche Competenze, oppure diluire i punti su più competenze per sapere un pò di tutto e non avere malus nelle prove (la prova si fa solo con 1d6 + Caratteristica).

Un suggerimento è anche di usare le Abilità, ed in particolare Esperto, che vi concede un bonus di +2 alle prove di Competenze.

\bigskip
\subsection{Suggerimenti per divertirsi e sopravvivere nelle campagne di DBS}\index{Linee guida per i giocatori}


\begin{enfasi}{
- Ci vuole un piano.

- Da quando gli eroi hanno bisogno di piani? (Final Fantasy XIII)

\medskip

Vado matto per i piani ben riusciti! ( Colonnello John "Hannibal" Smith,  A-Team)}\end{enfasi}\medskip

Questi sono consigli derivati dai principi dell'OSR che aiutano i personaggi a rimanere in vita.

\medskip

\begin{itemize}
	
\item
Impara a scappare, non aver paura di sopravvivere.
	
\item
Ogni combattimento è potenzialmente letale. Decidi con raziocinio e approccialo con attenzione.
	
\item
Non c'e tutto nella scheda. La scheda di un personaggio è l'insieme del perimetro dello stesso ma non definisce ciò che può o non può fare. Si creativo e parla al Narratore.
	
\item
Non risolvere tutto con un tiro di dando. Fai le domande giuste, parla con i compagni e descrivi con attenzione cosa intendi fare. Ricorda che il Narratore premia le descrizioni accurate e le azioni alternative.
	
\item
Delle basse caratteristiche sono solo delle basse caratteristiche e non il personaggio.
Sfrutta le competenze, le Abilità, fai in modo di dover tirare meno dadi possibili per risolvere il problema.
	
\item
Improvvisare, adattarsi e raggiungere lo scopo! (Tom Highway - Gunny, Film). Oppure come alcune mie giocatrici preferivano "Improvvisare, \textbf{Ingannare}, raggiungere lo scopo"
	
\item
Vivi appieno il tuo personaggio. Amplifica la sua storia porta nel presente il suo passato. Aiuta i compagni a conoscerti ed il Narratore a imbastire storie migliori intorno alle vostre storie.
	
\item
Saggio è chi saggio lo dimostra non chi lo dice.
	
Una cosa che non potrà mai portarti via nessuno è l'essere eroico, intelligente, risoluto, caparbio, cocciuto ma non stupido.
Vivi l'avventura al pieno ma non avere mai paura di sopravvivere.
	
\item
Descrivi in maniera realistica ciò che fai, aiuterai il Narratore e i compagni intorno a te.
E' sicuramente meglio che dire "faccio una prova di Consapevolezza".
	
\item
Esaltati nel descrivere le azioni piu' importanti, il Narratore ne terra' conto.
	
\item
Spremiti le meningi e sii creativo, alternativo, curioso ma non suicida o avventato.
	
\item
E finché non potrai dire "\textit{Io sono cattivo, incazzato e stanco. Sono uno che mangia filo spinato, piscia napalm e riesce a mettere una palla in culo ad una pulce a 200 metri}." (Tom Highway - Gunny, Film) allora stai al tuo posto e non fare lo sbruffone, c'e' sempre qualcuno più grosso ed arrabbiato di te.
	
\item

Ricorda sempre maggiore e' il pericolo maggiore e' l'esperienza maturata. Piu' è profondo il dungeon maggiore saranno i tesori e l'esperienza guadagnata!

\item
Lo scopo è divertirsi, fare divertire ed assaporare la sfida.
	
Non creare personaggio che siano contro gli altri personaggi o diano sempre fastidio e problemi. Media la tua voglia con le necessità del gruppo, perché sempre e solo come gruppo sopravviverete e mai solo come singolo.
\end{itemize}

\end{multicols}

\bigskip

\begin{enfasi}{
	"La candela accesa da entrambe le parti dura la metà (Anonimo)"
}\end{enfasi}\medskip


\begin{figure}[htb]
	\centering
	\includegraphics[width=0.8\linewidth]{immagini/threasure2.png}
\end{figure}


\pagebreak

\section{Regole per le Competenze}\index{Regole per le Competenze}\index{Competenze}

\label{regole-per-le-competenze}
\begin{enfasi}{
		Occorre che la legge sia breve, perché più facilmente i mal pratici la ricordino. (Seneca)}\end{enfasi}\medskip

\begin{multicols}{2}

\lettrine[lines=2, lhang=0.33, loversize=0.25, findent=1.5em]{L}{\textbf{e}} \textbf{prove (i check) per le competenze si eseguono tirando 3d6, al risultato dei dadi si somma il punteggio della competenza (di base o attiva) e della caratteristica collegata ed eventuali bonus magici e di circostanza o Abilità, il risultato ottenuto deve essere comunicato al Narratore, il quale lo confronterà con la difficoltà (DC) della prova}.

Quando dovete stabilire una difficoltà partite pensando che la prova deve essere rapportata da una persona "normale". Non pensate "se la dovessi fare io allora la prova sarebbe impossibile", "se la prova la fa Arsenio Lupin la prova è facilissima". Partite dal presupposto che la difficoltà deve racchiudere in se tutti gli elementi circostanziali.

Pensate se piove, c'è poca luce, il personaggio sta correndo, è ferito, fa le cose di fretta ed anche alla complessità della cosa che deve fare, saltare un fosso di 3 metri non è come uno di 3 metri al buio, senza scarpe, sotto la pioggia ed inseguiti e con le tasche strapiene di monete...

Decifrare uno scritto antico potrà essere una passeggiata per un linguista esperto, ma per una "persona normale" che non ha idea di cosa può avere davanti la prova è semplicemente impossibile. Questo "impossibile" è la vostra DC, la difficoltà della prova.

E non spaventatevi se i personaggi falliscono le prove, renderà l'avventura più interessante, e vi permetterà a voi Narratore di introdurre fatti, indizi e nuove avventure.

\bigskip

\begin{center}
	\includegraphics[width=0.9\linewidth]{immagini/master2.png}
	
	\textit{The Master of the Gamblers (known in Italian as Maestro dei Giocatori) (fl 1620 – 1640) }
\end{center}

\bigskip

\textbf{Quando devi fare una prova per una competenza di base in cui non sei preparato, ovvero non hai punti, devi tirare solo 1d6 + punteggio della caratteristica collegata}.\index{Prova competenza senza competenze}

\medskip
Quando si scrive -1d6 significa che si tira un dado in meno (o due se è -2d6), parimente se c'è scritto +1d6 si tira un dado a 6 in più e si somma.


\bigskip

La tabella qui sotto serve a rapportare la difficoltà alla abilità minima necessaria per riuscire la prova con un tiro medio (un punteggio di 10 lanciando 3d6). Usate queste indicazione per avere una idea delle scale di difficoltà.

Il Narratore non ti dirà fammi una prova a difficoltà 10, ma dirà che la prova non presenta elementi di particolare difficoltà.

\begin{center}
	\includegraphics[width=0.9\linewidth]{immagini/difficulty.png}
	
	\textit{A City on a Rock, long attributed to Goya, is now thought to have been painted by 19th-century artist Eugenio Lucas Velázquez. Ottima prova di falsificazione}
\end{center}
\medskip

\textbf{Tabella: Classe di difficoltà}\index{Tabella Classe di difficoltà}

\medskip
\begin{tabularx}{0.45\textwidth}{lXX}
	\textbf{Difficoltà} & \textbf{Descrizione difficoltà} & \textbf{Livello Competenza}\\
	\toprule
	DC 5               & Estremamente facile        & Mediocre\\
	DC 10              & Facile           			& Normale\\
	DC 15              & Normale          			& Buona\\
	DC 20              & Difficile        			& Ottimo\\
	DC 25              & Molto difficile 	 		& Eccellente\\
	DC 30              & Eroica					    & Stupefacente\\
	DC 35              & Quasi impossibile			& Leggendaria\\
	DC 40              & Leggendaria      			& Oltre l'umano\\
\end{tabularx}

\bigskip

Se devi fare una prova su una Caratteristica devi tirare 3d6 e sommare il punteggio della Caratteristica e altri bonus inerenti. Comunica questo risultato al Narratore che la confronterà con la difficoltà (DC).

\subsection{Superare o Fallire la prova di tanto...}\index{Superare o Fallire la prova di tanto}

Ogni qual volta la prova è superata brillantemente (di 10 o più rispetto alla difficoltà necessaria) il Narratore può decidere di dare maggiori informazioni, concedere bonus alle azioni successive (+1/+2).. qualsiasi cosa possa valorizzare quanto agevolmente la prova è stata superata.

Viceversa se la prova fallisce di 10 o più rispetto al valore necessario il Narratore potrebbe descrivere come miseramente la prova è fallita e come il risultato pessimo influenzi l'azione o quelle successive.

Se la somma dei dadi è 0 (zero) perché avete tirate 3 volte 1 la prova non e' automaticamente fallita, semplicemente si andrà a contare solo il vostro punteggio di Competenza per stabilire se avete passato o meno la prova.

Questi situazioni non si applicano alla prova del Tiro per Colpire o Tiro Salvezza, troverete nel capitolo combattimento e magie le regole specifiche.

\subsection{Consapevolezza}\index{Consapevolezza}

La Consapevolezza è una di quelle competenze che entra in gioco molto spesso.

Fate in modo che siano le domande e ragionamenti dei giocatori a rivelare gli indizi, una prova di  Consapevolezza viene tirata ogni qual volta ci sia da cercare qualcosa di non ovvio, qualcosa che deve essere cercato altrimenti non risulta percettibile o intuibile.

\subsection{Prove opposte}\index{Prove opposte}

Ci sono situazioni in cui il personaggio deve effettuare una prova in contrapposizione con un avversario, ad esempio muoversi silenziosamente alle spalle di una guardia, rubare dalle tasche del mercante, intimidire l'orchetto per farsi dare indicazioni..

In questi casi il personaggio ed il Narratore effettuano una prova, chi ottiene il valore più alto vince, in caso di parità vince chi ha il valore più alto nella competenza, poi nella caratteristica ed infine l'eventuale "avversario".

\bigskip

\textbf{Alcuni esempi di prova opposte}

\flushleft - Ingannare qualcuno: Ingannare Vs Percepire Emozioni

- Travestirsi per sembrare qualcun'altro: Intrattenere Vs Consapevolezza

- Creare una mappa falsa: Falsificare Vs Valutare

- Nascondersi: Nascondersi Vs Consapevolezza

- Muoversi silenziosamente: Muoversi silenziosamente vs Consapevolezza (purche' non visto)

- Intimidire: Intimidire Vs Tiro Salvezza su Volontà (con modificatore Carisma)

- Rubare: Mani di Fata Vs Consapevolezza, o Mani di Fata se posseduta (valutare bene bonus e malus)

- Slegarsi da delle corde: Usare Corde Vs Artista della fuga

- Braccio di ferro: prova contrapposta Tiro Salvezza Tempra (con modificatore forza)

\begin{center}
	\includegraphics[width=0.7\linewidth]{immagini/Foster_Bible_Pictures.png}
	
	\textit{Bible Pictures and What They Teach Us. Prova Consapevolezza, Muoversi Silenziosamente}
\end{center}

\begin{narratore}
Ogni qual volta la prova opposta riguarda una \textbf{caratteristica correlata ai Tiri Salvezza} (Costituzione, Destrezza, Saggezza) per tutti gli interessati, fate fare un Tiro Salvezza come valore contrapposto. Es. una prova di braccio di ferro è un Tiro Salvezza su Tempra contrapposto. Chi riesce a fare più alto con il Tiro Salvezza vince.
\end{narratore}

\begin{narratore}
Non fate che siano le prove a governare il vostro gioco. \textbf{Fate giocare i giocatori}, fateli recitare, fateli partecipare ed in base a quanto dicono stabilite se la prova è passata o meno. 

Se vi dicono "convinco la guardia a farci passare" fate fare una prova di Intimidire (o Diplomazia), se invece intavolano un dialogo convincente potete considerare che la prova sia stata fatta con esito positivo (o negativo se non sono riusciti ad argomentare!)
\end{narratore}

\subsection{Vantaggi e Svantaggi} \index{Vantaggi}\index{Bonus}\index{Malus}\index{Svantaggi}

\begin{enfasi}{Audentes fortuna iuvat ("La fortuna aiuta gli audaci", Virgilio) }\end{enfasi}

\medskip

Il Narratore a seconda delle circostanze può concedervi un bonus od un svantaggio nella prova.

Il valore nelle \textbf{prove dinamiche}\index{Prove dinamiche} è da usarsi quando la prova viene fatta tirando i 3d6, in questo caso si potranno sommare bonus (+2) o addirittura tirare dadi in più (+2d6) o se in svantaggio dadi in meno, fino a tirare solo 1d6 (con 2d6 di penalità).

Se i malus accumulati portano i dadi della prova sotto zero si conta solo il valore della Competenza e Caratteristica.

Si intendono \textbf{prove a valore fisso} \index{Prove a valore fisso} quando non è necessario tirare dei dadi (es. Difesa), in questo caso il punteggio si alza/abbassa del numero indicato.

Cercate di rimanere sempre tra questi valori di vantaggio e svantaggio, altrimenti potete dire che la prova è direttamente riuscita o fallita.

Il giocatore può comunque richiedere di effettuare la prova anche se il risultato è certo.

\medskip

\textbf{Tabella: Bonus e Malus, Vantaggi e Svantaggi}:\index{Tabella Bonus e Malus, Vantaggi e Svantaggi}

\medskip

\begin{tabular}{lll}
	\multirow{2}*{Vantaggio / Svantaggio} & \multicolumn{2}{c}{\textbf{Prove}}\\
	\cmidrule(lr){2-3} & \textbf{Dinamiche} & \textbf{Fisse} \\
	\toprule
	Bonus leggero 				  & +1				& +1\\
	Bonus normale 				  & +2 				& +2\\
	Bonus forte  			      & +1d6            & +4\\
	Bonus molto forte             & +2d6            & +8\\
	Svantaggio leggero            & -1 				& -1\\
	Svantaggio normale            & -2 				& -2\\
	Svantaggio forte              & -1d6            & -4\\
	Svantaggio molto forte        & -2d6            & -8\\
\end{tabular}


\begin{narratore}
La matematica della distribuzione normalizzata (3d6) e' diversa da quella del semplice d20. I vantaggi e svantaggi pesano molto di più, fanno molta più riuscita.

Questo non significa di non dare bonus o malus, ma di essere coerente al massimo ed applicarli a tutti i contendenti ovvero sia ai giocatori che ai nemici. Cercate di rimanere sempre nell'abito di +-2 al massimo e solo in situazioni particolari di effettivo e forte vantaggio o svantaggio applicate bonus o malus più forti.	
\end{narratore}

\subsubsection{Fattore tempo}\index{Fattore tempo}

\textbf{Se un personaggio non è in difficoltà o pressione}\index{Senza problemi di tempo} nell'effettuare la prova può prendere il 10 (+ competenze + abilita..), ovvero non tirare i dadi e considerare che abbia tirato 10 con i dadi. L'azione impiega 10 round.

\textbf{Se il personaggio non ha impellenti limiti di tempo}, ovvero può dedicare almeno 10 minuti per lavorarci (60 round) può considerare di prendere 15. Ovvero come se avesse fatto la prova e tirato 15 con i 3d6.

\textbf{Se il tempo diventa un fattore da non considerare}, ovvero il personaggio ha almeno 1 ora per pensare e lavorare considerare di avere tirato 18 (ma non c'è nessuna esplosione di dadi anche se il totale è 18...)

Se vuoi prendere questi valori chiedilo al Narratore, sarà lui che ti permetterà o meno di usare questi punteggi, in base alla situazione, urgenza, pericolosità di ciò che ti circonda. 

Mettersi a scassinare una porta in un dungeon chiedendo il 10 richiede un estremo sangue freddo e incoscienza.

\textbf{Prendere il 18} è fattibile solo se il personaggio non ha penalità nell'effettuare la prova.

\subsubsection{Aiutare un Altro}

\index{Aiutare un altro} Si può aiutare un amico in una prova dandogli supporto e suggerimenti. L'aiutante deve effettuare una prova a difficoltà dimezzata (esempio se il personaggio impegnato deve fare una prova a difficoltà 25 l'aiutante la fa a difficoltà 12), se ci riesce da un +2 alla prova del compagno.

Più personaggi possono aiutare lo stesso amico; i bonus di questo tipo sono cumulabili fino ad un bonus pari alla metà della difficoltà da battere (es +12 nel caso di difficoltà 25).

Il Narratore valuterà la possibilità che più di un personaggio fornisca aiuto considerando spazi, modi e tempi (non è facile aiutare qualcuno ad infilare un filo nella cruna di un ago).

\subsection{Golden Rules}

\begin{itemize}

\item
\textbf{Esplosione del 6}: \index{Esplosione del 6}anche nelle prove di delle competenze base c'è l'esplosione del 6. Se con un dado fai 6 lo sommi e ritiri e continui così se fai ancora 6.

\item
\textbf{Tirare un 1}: \index{Tirare un 1 nelle prove}porta male anche nelle prove di competenze, ovvero non si somma. Non è un fallimento automatico è solo un tiro molto basso (zero + valore di competenza)

\item
\textbf{Tentare la sorte}\index{Tentare la sorte}: posso togliere 4 punti tra punteggio di competenza e dalla Caratteristica per aggiungere un 1d6 al tiro (Tiro per Colpire, Tiro Salvezza, prove Competenza)
\end{itemize}


\begin{tcolorbox}[title = Golden Rules]{
Sfruttate le \textbf{Golden Rules} a vostro vantaggio! Osate, tentate, rischiate quando la situazione non permette altre soluzioni!}
\end{tcolorbox}

\subsubsection{Successo Parziale* - Opzionale}\index{Successo Parziale}\index{Prova con Rischio}

Il Narratore può anche decidere di valutare una prova non riuscita come un successo parziale.

Se la prova fallisce di 1 potrà considerarsi riuscita anche se con un problema leggero, se è fallita di 2 si porta dietro un problema serio se è fallita di 4 è riuscita con un problema critico, se è fallita di oltre 4 la prova non è comunque riuscita.

Ad esempio se Tups vuole slegarsi i polsi, con un fallimento di 1 avrà fatto rumore forse svegliando la guardia, con un fallimento di 2 si è slogato il pollice con un -2 alle prove di Competenza Armi e prove con quella mano per un giorno.

Con un fallimento di 4 è si riuscito a liberarsi a costo di essersi slogato il polso!

\medskip

\begin{narratore}
Utilizzando il Successo Parziale può essere anche il giocatore a richiedere una \textbf{"Prova con Rischio}" in situazioni di particolare tensione ed urgenza in cui è più importante il risultato finale che il rischio che si corre. Questa richiesta va fatta prima di tirare i dadi.
\end{narratore}

\subsection{Esempi Prove Competenza}

\label{esempi-prove-competenza}

\begin{itemize}
	\item Una prova riuscita (base DC 15) in \textbf{Pronto Soccorso}\hypertarget{prontosoccorso}{} può fare recuperare 1d4 Punti Ferita dopo uno scontro o concedere un +2 ad un Tiro Salvezza su Tempra per resistere ad un veleno. Da fare entro 1 turno dal termine del combattimento. Costo 6 round.
	
	\item Una prova riuscita (base DC 15) in \textbf{Pronto Soccorso} riduce di 1 i danni da Sanguinamento. Per ogni valore di Sanguinamento sopra 1 la difficoltà aumenta di 3, e la prova la riduce comunque di 1 alla volta. Costo 3 Azioni

	\item Una prova riuscita (base DC 13) in \textbf{Pronto Soccorso} per prendersi cura per 8 ore di un paziente fa recuperare a questo il doppio dei punti ferita ((CA+Costituzione)*2 con un minimo di 4) e concede un nuovo Tiro Salvezza su Tempra per resistere a Malattie o Veleni già in corso.

	\item Per identificare una pozione \index{Identificare Pozione}è necessario una prova di \textbf{Erboristeria} a DC 12 + fattore di rarità della pianta.

	\item Per riconoscere un oggetto magico e le sue capacità è necessario l'incantesimo Identificare. Una prova di \textbf{Arcana} a difficoltà 25 può dare indicazioni di massima sui poteri e ambiti di utilizzo.	      
	
	\item Riconoscere un incantesimo mentre viene lanciato è una prova di \textbf{Arcana} a DC pari la Difficoltà dell'incantesimo -5 che costa una Reazione.
	
	\item \textbf{Nuotare}\index{Nuotare} \textit{Si hanno penalità dovuta all'Armatura.}
	
	In acqua calme DC 10. In acque mosse ha DC 15, in acque tempestose DC 20

	\item Per \textbf{riconoscere un mostro}, una creatura particolare si effettua una prova di
	
	\textit{Arcana}: Giganti, Costrutti, Spiriti, Mostruosità, Aberrazioni, Draghi
	
	\textit{Piani}: Elementali
	
	\textit{Occulto}: Immondi, Spiriti, Non Morti
	
	\textit{Religione}: Spiriti, Non Morti, Celestiali
	
	\textit{Dungeon}: Aberrazioni, Mostruosità, Melme, e creature sotterranee
	
	\textit{Natura}: Bestie, Piante, Fatati
	
	La DC delle prova è pari al grado di Sfida della creatura + 10, un fallimento non critico della prova concederà meno informazioni.
	
	\item \textbf{Intimidire}\index{Intimidire}
	
	Il personaggio usa 2 Azioni e fa la prova su Intimidire contrapposta al Tiro Salvezza su Volontà con bonus dato dal Carisma dell'avversario.
	
	Se il Tiro Salvezza fallisce, l'avversario fino alla fine del round successivo ha -2 al Tiro per Colpire e -2 alla Difesa contro quell'avversario soltanto.
	
	Se la prova di Intimidire fallisce in maniera critica (l'avversario riesce di 10 o più il Tiro Salvezza,  chi ha fatto la prova deve effettuare un Tiro Salvezza su Volontà con modificatore Carisma a DC 10+Grado di Sfida dell'avversario o subire le medesime penalità come se fosse stato intimidito.
	
	Se il tiro contrapposto riesce in maniera critica (l'avversario fallisce di più di 10 il Tiro Salvezza) la durata dell'effetto permane fino a fine combattimento.
	
	\item
	Un eventuale prova sulla \textbf{Professione} viene fatta con 3d6+Saggezza+metà del livello.
		
\item \textbf{Saltare}\index{Tabella Saltare} \textit{Si hanno penalità dovuta all'Armatura.}


\begin{tabular}{lc|lc}
	\textbf{Salto in Lungo} & \textbf{DC}&	\textbf{Salto in Alto} & \textbf{DC}\\
\multicolumn{2}{c}{Lunghezza} &\multicolumn{2}{c}{Altezza} \\
	\toprule
	1.5 m                              & 5 & 	0.02 m                           & 4\\
	3 m                                & 10 &	0.5 m                            & 8\\
	5 m                                & 15 & 	1 m                              & 12\\
	7 m                                & 20 & 	1.5 m                            & 16\\
	+1,5 m                             & +5 &	+0.5 m                           & +4\\
\end{tabular}


In un \textbf{salto in lungo} la punta più alta del salto è pari ad un 1/4 della lunghezza saltata. Se esegui un salto in lungo di 4 metri a metà salto sei in alto di 1 metro. 

\item \textbf{Arrampicarsi} \index{Arrampicarsi} \textit{Si hanno penalità dovuta all'Armatura.}

\begin{tabular}{ll}
	\textbf{Esempio di Superficie} & \textbf{DC}\\
	\toprule
	Parete grezza con appigli, mattoni sporgenti&10\\
	Un albero, una corda senza nodi&15\\
	Una parete liscia con appigli &20\\
	Un muro con pochissimi appigli&25\\
	Una parete naturale non liscia senza appigli&30\\
 	Ti puoi appoggiare a 2 pareti opposte&-10\\
	Ti puoi appoggiare a 2 parete angolari&-5\\
	Superficie scivolosa&+5\\
\end{tabular}


\bigskip

\item \textbf{Sopravvivenza}\index{Sopravvivenza}

\textbf{Inseguire una creatura}:

\begin{tabular}{ll}
	Difficoltà base  & DC 10\\
	\toprule
	Se il terreno è molto morbido& DC +5\\
	Se il terreno è morbido& DC +10\\
	Se il terreno è stabile& DC +15\\
	Se il terreno è duro& DC +20\\
	Ogni 3 creature inseguite& DC -1\\
	A seconda della taglia& DC +-4\\
	Ogni 24 ore passate&DC +2\\
	Ogni ora di pioggia&DC +4\\
	Visibilità scarsa&DC +2\\
	Cerca di occultare le traccie&DC +5\\
\end{tabular}


\medskip

\item \textbf{Nascondersi} \index{Nascondersi} \textit{Si hanno penalità dovuta all'Armatura.}

Usando 2 Azioni (e probabilmente una altra Azione per spostarti) puoi cercare di nasconderti alla vista degli avversari. Esegui una prova di Nascondersi, DC 15 (ed il Narratore valuterà eventuali modifiche circostanziali), se ci riesci ottieni una Copertura media.

La copertura può essere riferita a più avversari o solo qualcuno, il Narratore può valutare in caso di critici nella prova una copertura anche maggiore.

Non è possibile nascondersi se l'ambiente non lo permette, per quando la tua prova possa essere alta non puoi nasconderti se non c'è qualcosa che ti può coprire.
 

\item \textbf{Muoversi Silenziosamente} \index{Muoversi Silenziosamente} \textit{Si hanno penalità dovuta all'Armatura.}

Usi due Azioni per per Muoversi Silenziosamente, la rimanente Azione e' usata per il movimento.

La prova va confrontata con la Consapevolezza dell'avversario che potrebbe aver sentito ma che non deve aver visto.

\end{itemize}

\subsubsection{Linguaggi}\index{Linguaggi}\hypertarget{linguaggi}{} 

\label{linguaggi}

In Yeru ogni razza è custode di una propria lingua parlata e scritta. Ogni personaggio che abbia almeno Intelligenza -1 parla il linguaggio della propria razza, con 0 lo scrive anche.
Per ogni punto superiore o pari a 2 parla e scrive un altra lingua che sarà scelta alla creazione del personaggio. Per ogni punto in Conoscenza: Lingue che dedica specificatamente ai linguaggi, parla e scrive un altra lingua. Alcune lingue non possono essere parlate se non da creature appartenenti a quella razza, ad esempio le lingue degli Elementali.

Un appartenente ad una razza può benissimo avere come prima lingua non quella della propria razza se il background lo giustifica (es un Nano cresciuto presso una tribù di Goblin).

\end{multicols}

\smallskip

\textbf{Tabella delle Lingue}\index{Tabella delle Lingue}

\medskip

\begin{tabular}{lll|lll}
	\textbf{Ambito culturale}& \textbf{Parlato} & \textbf{Scritto}&	\textbf{Ambito culturale}& \textbf{Parlato} & \textbf{Scritto}\\
	\toprule
Umano 					& Comune		& Comune		& Nanico				& Nanico		& Nanico\\		Elfico					& Elfico  		& Elfico 		& Gnomico				& Gnomica  		& Gnomica \\		Gnoll  					& Gnoll 	    & Goblinoide 	& Giganti				& Gigante	    & Gigante\\ 
Orco					& Orchesco  	& Orchesco  	& Creature marine senzienti & Acquan		& Elfico\\
Uccelli senzienti		& Auran 		& Elfico		& Abitanti dei boschi*		& Silvano		& Silvano\\
Druidica* 				& Druidico 		& -				& Goblinoide			& Goblinoide	& Goblinoide\\
Celestiale				& Celestiale 	& Celestiale	& Infernale 				& Infernale  	& Infernale\\
Abissale 				& Abissale		& Abissale		& Draghi					& Draconico  	& Draconico\\
Elementali del Fuoco  		& Ignam   		& - 			& Elementali della Terra			& Terran 	    &-\\
Elementali dell'Acqua 		& Acquan  		& -  			& Elementali dell'Aria			& Auran         &-\\
Lingua dei Segni*		& dei Segni 	& dei Segni		& Varie 				&  Profondità	& Profondità\\
\end{tabular}


\medskip

I \textbf{Linguaggi Speciali}* possono essere presi sono dietro autorizzazione del Narratore a seguito di background o Abilità del personaggio.

\medskip

La \textbf{Telepatia}\index{Telepatia} non è propriamente una lingua ma un mezzo per parlare con qualsiasi creatura senziente (Intelligenza maggiore di -3). Non c'è il vincolo del linguaggio, la telepatia funge anche da traduttore universale.

\medskip

\begin{center}
	\includegraphics[width=0.35\linewidth]{immagini/babele.png}
	
	\textit{Grande Torre di Babele. Pieter Bruegel il Vecchio (1563)}
\end{center}

\pagebreak

\section{Combattimento}\index{Combattimento}

\label{combattimento}
\begin{enfasi}{
Si vis pacem, para bellum ("Se vuoi la pace, prepara la guerra", anonimo)
\medskip

Non conta come cadi, ma se e come ti rialzi (anonimo)

\medskip
Non sono un eroe. No e non lo sarò mai. Sono solo un cattivo che viene pagato per pestare tipi peggiori di lui. (Deadpool)

\medskip

Occhio per occhio... e il mondo diventa cieco (Mahatma Gandhi, NdA i suoi Tratti aborrivano la violenza!)}\end{enfasi}\medskip


\begin{multicols}{2}

\lettrine[lines=2, lhang=0.33, loversize=0.25, findent=1.5em]{I}{l} combattimento è tra le fasi principali di un avventura ed è quando i prodi o timorosi danno sfoggio della loro maestria con le armi o magie.

\bigskip

Il combattimento è diviso in 2 fasi:\index{Combattimento}
\begin{itemize}
	\item verifica dell'iniziativa
	\item risoluzione delle azioni (movimento, attacco, azione varie..)
\end{itemize}

\begin{center}
	\includegraphics[width=0.9\linewidth]{immagini/Achildbookofwarriors.png}
	
	\textit{A child's book of warriors (1907), William Canton}
\end{center}

\subsection{L'Iniziativa}\index{Iniziativa}

\label{liniziativa}

L'iniziativa è una prova (3d6) di Destrezza o Intelligenza ed Abilità inerenti che potete avere.

Il giocatore sceglie la caratteristica che preferisce. Se viene scelta Destrezza saranno i riflessi a determinare la reazione del personaggio, mentre Intelligenza guiderà la capacità di cogliere le tattiche dell'avversario ed anticiparle.

Chi ha l'iniziativa tra giocatori e nemici più alta incomincia per primo e successivamente agiscono gli altri in ordine decrescente, dichiarando le Azioni ed eseguendole.

In caso di Iniziativa di pari punteggio agisce per primo chi ha il punteggio caratteristica più alto, altrimenti lo scontro sarà in contemporanea.

L'iniziativa vale per l'intero scontro e si ritira al cambio dell'avversario.

\textbf{Nella prova di Iniziativa non valgono le Golden Rules.}


\subsubsection{Risoluzione delle Azioni}\index{Risoluzione delle Azioni}

\begin{enfasi}{
...il passato è il prologo e il futuro sta nelle vostre mani e nelle mie. (Antonio, La Tempesta, Shakespeare)}
\end{enfasi}\medskip


\label{risoluzione-delle-azioni}

Dal più veloce al più lento c'è la risoluzione delle Azioni.

Il Narratore chiederà al più veloce di dichiarare le sue Azioni ed agire, proseguirà poi a chiedere e fare agire gli altri giocatori e nemici.

In questo modo la scelta dell'azione avviene quando è il round del giocatore che potrà agire anche in base alle azioni e risoluzioni già avvenute.

\subsubsection{Variante Iniziativa - Opzionale*}

Vi propongo una variante del sistema di calcolo dell'iniziativa.

La prova di Iniziativa si calcola sempre come 3d6+Destrezza o Intelligenza, ma si applicano poi degli ulteriori modificatori che si computano round per round in base alle Azioni che si fanno a fare.

- Se si usa un arma di taglia Piccola si somma il valore di Destrezza.

- Se si usa un arma di taglia Media si tiene il valore dell'Iniziativa tirata.

- Se si usa un arma di taglia Grande si sottrae il valore di Destrezza.

- Se l'incantesimo che si va a lanciare ha solo componenti Verbali (V) si somma il valore di Intelligenza.

- Se l'incantesimo che si va a lanciare ha componenti Verbali e Somatici (VS) si tiene il valore dell'Iniziativa tirata.

- Se l'incantesimo che si va a lanciare ha componenti Verbali, Somatici, Materiali (VSM) si sottrae il valore di Intelligenza.

- Se usa una Azione di movimento prima di attaccare (o lanciare un incantesimo) l'iniziativa diminuisce di 2.

- Se si usa una Azione Immediata o Reazione prima di attaccare (o lanciare un Incantesimo) l'iniziativa diminuisce di 1.

- Se si cambia arma l'iniziativa diminuisce di 4 ma non si usano Azioni.

- Il nemico usa l'iniziativa standard (3d6+ modificatore indicato).

\bigskip

Questo sistema genera un approccio estremamente tattico alla risoluzione dell'iniziativa.

Il giocatore è invogliato a cambiare metodi di attacco, armi o incantesimi in base all'avversario che ha di fronte, se questo è veloce, lento.. se bisogna colpire per primo o basta colpire anche dopo, se l'azione della compagna è stata risolutiva o meno.

Inizialmente questo sistema rallenta il gioco ma appena i giocatori prendono consapevolezza delle loro opzioni l'iniziativa e la collaborazione tra i personaggi diventano elementi fondamentali e portanti non rallentando il flusso delle azioni.

\begin{center}
	\includegraphics[width=0.9\linewidth]{immagini/Arthur-Pyle_Two_Knights.png}
	\textit{Howard Pyle illustration from the 1903 edition of The Story of King Arthur and His Knights}
\end{center}


\subsubsection{Il Tempo (Round, Minuti e Turni)}\index{Round}

\label{il-tempo-round-minuti-e-turni}
\begin{enfasi}{
		"L'esitazione è la morte del vantaggio" (Magic, V.E. Schwab)}\end{enfasi}\medskip

Un \textbf{round} dura 10 secondi circa, è un lasso di tempo adeguato per agire, correre, parlare.. combattere. Un Minuto sono quindi 6 round, ed un Turno dura 10 Minuti (o 60 round).

I round si usano nelle azioni di combattimento o dove la tensione deve rimanere costantemente alta ed a ogni azione corrisponde un evolversi della situazione.

\begin{center}
	\includegraphics[width=0.9\linewidth]{immagini/hjford-fight.jpg}
	\textit{Henry Justice Ford -  Fairy book, Fairytale illustration}
\end{center}

\subsubsection{Tempo di attivazione Oggetti ed Abilità}\index{Tempo di attivazione Oggetti ed Abilità}

Se non specificato diversamente un oggetto o Abilità che prevede un certo numero di usi al giorno \textit{"una volta al giorno"} si "ricarica" all'alba successiva l'uso.

\subsection{Azioni nel Round}\index{Azioni nel Round}\index{Azione}

\label{azioni-nel-round}

Un personaggio può eseguire 3 Azioni per round, 1 Azione Immediata, 1 Azione di Reazione.

Queste Azioni possono essere eseguite nell'ordine preferito.

Nella tabella sottostante sono indicate le Azioni principali che un personaggio può eseguire, sono linee guida da seguire. Nel capitolo dedicato al combattimento vengono elencate altre Azioni ed i loro costi relativi.

Una Azione non può essere interrotta da un altra Azione, ma può essere seguita da una Reazione o da una Azione Immediata, se nel proprio round.

Se un personaggio vuole fare più attacchi spostandosi nel campo di battaglia può, ad esempio, usare una Azione per eseguire un attacco, usare una Azione di movimento per spostarsi fino a tutto il suo movimento a disposizione, ed usare un ultima azione di attacco per eseguire un ultimo singolo attacco, questo secondo attacco (e singolo) conta comunque come attacco multiplo con le dovute penalità.

E' possibile ritardare la propria Azione per aspettare una determinata situazione. Il personaggio che ritarda tutte le proprie Azioni agisce per primo tra i soggetti che agiscono in quel valore o successivo di iniziativa.

Se il personaggio non ha eseguito alcuna Azione nel round, come detto prima, agirà per primo in quel segmento di iniziativa, se invece ha già compiuto Azioni allora potrà agire solo tramite una Reazione, se a disposizione. In questo caso la sua Reazione arriverà dopo l'Azione scatenante.


\bigskip

\textbf{Tabella Azioni per Round}\index{Tabella delle Azioni per Round}


\begin{tabular}{lc}
	\textbf{Cosa si fa}                         & \textbf{Azioni}\\
	\toprule
	Eseguire un attacco  						& 1\\
	Eseguire due attacchi        				& 2\\
	Eseguire più di due attacchi   				& 3\\
	Lanciare un'Incantesimo                     & 2\\
	Eseguire una Azione di Movimento*           & 1\\
	Scatto   						            & 1\\
	Alzarsi da prono                            & 2\\
	Aiutare qualcuno                            & 2\\
	Scambiare un discorso con qualcuno          & 2\\
	Scambiare poche battute con qualcuno        & 0\\
	Cercare qualcosa nello zaino di pronto      & 2\\
	Usare un oggetto in mano      				& 1\\
	Bere una pozione tenuta alla cintura        & 1\\
	Estrarre l'arma 					        & 1\\
	Imbracciare lo scudo						& 1\\
	Usare un oggetto magico 					& 2\\
	Eseguire prova su una competenza        	& 2\\
	Nascondersi									& 2\\
	Concentrarsi su un Incantesimo    	 		& 1\\
	Salire o scendere dalla cavalcatura			& 1\\	
	Azione Immediata - Azione Reazione          & I - R\\
	Bere una pozione tenuta in mano     	    & I\\
	Fare cadere l'arma o lo scudo				& R\\
	Gettarsi a terra prono						& R\\	
	Riconoscere un Incantesimo					& R\\
\end{tabular}

\medskip

Per Attacco si intende sia l'uso di armi in mischia che l'uso di armi da lancio o tiro come archi, balestre o pugnali da lancio. Nel caso di armi da tiro o lancio ogni tiro o lancio conta come un attacco.

\medskip

\textbf{Azione di Movimento*}: un Azione di Movimento è una Azione dedicata a spostarsi. Ci si può spostare fino a tutto il proprio movimento (9 metri per umani, 6 metri per nani..).

Questo elenco non è completo, prendetelo come linee guida per stabilire il peso delle decisioni ed azioni dei giocatori.

\bigskip

L'ordine con cui si eseguono le Azioni non è importante se non per correlazione logica e fisica. L'Azione di Movimento può essere spezzata tra altre Azioni (movimento parziale, attacco/incantesimi, altra azione, movimento parziale).

Un personaggio potrebbe attaccare, muoversi ed ancora attaccare, questo secondo attacco avrebbe le penalità descritte negli attacchi multipli.
\smallskip

Una Azione "\textbf{Reazione (R)}" \index{Azione Reazione}può essere eseguita liberamente anche fuori dal proprio round. Questa Azione è solitamente dovuta ad Abilità o situazioni particolari. Se non indicato diversamente una Reazione costa zero Azioni e accade immediatamente dopo la causa che la scatena.

\smallskip

Una Azione "\textbf{Immediata (I)}" \index{Azione Immediata}può essere eseguita liberamente nel proprio round, primo o dopo la propria Azione. Una Azione Immediata è solitamente concessa da particolari Abilità, se non indicato diversamente una Azione Immediata costa zero Azioni.

E' possibile se non descritto specificatamente nell'Abilità eseguire solo una Azione Immediata ed una Azione di Reazione per round.


\begin{center}
	\includegraphics[width=0.9\linewidth]{immagini/Perseus_Fighting_Phineus_and_his_Companions.png}
	
	\textit{Luca Giordano: Perseus turning Phineas and his Followers to Stone}
\end{center}

\end{multicols}

\bigskip

\begin{narratore}Una creatura che ha una distanza di mischia (portata) superiore all'avversario si considera che abbia un bonus di \textbf{+4 all'iniziativa} per il primo round, ovvero come se usasse un'arma lunga. Il round successivo la sua iniziativa in corpo a corpo non avrà questo vantaggio a meno che abbia mantenuto la distanza non facendosi raggiungere.
	
Questo bonus non si applica con le armi da lancio (archi, balestre, pugnali, asce... per quelle armi che usano la \textit{gittata}). \end{narratore}

\pagebreak

\subsection{Movimento}\index{Movimento}

\label{movimento}

\begin{enfasi}{"Un mobile più lento non può essere raggiunto da uno più rapido; giacché quello che segue deve arrivare al punto che occupava quello che è seguito e dove questo non è più (quando il secondo arriva); in tal modo il primo conserva sempre un vantaggio sul secondo" (Paradosso di Zenone)}
\end{enfasi}\medskip

\begin{multicols}{2}

Il movimento di un personaggio è dato dalla sua taglia e razza e da ciò che porta, dai pesi, ingombri ma anche magie ed oggetti magici.

Il Movimento scritto nella razza del personaggio è l'indicazione di quanti metri per Azione (di Movimento) il personaggio può fare.

Una creatura o personaggio potrebbe anche decidere di spostarsi più velocemente del solito ovvero correndo (Azione di Scatto).

Se esegue una Azione di \textbf{Scatto} \index{Scatto}si raddoppiano i metri percorsi (2x9 metri per un umano) per Azione (di Movimento). Per un nano (Movimento 6m) significa fare 12 metri in una Azione.

Il personaggio che fa una Azione di Scatto \index{Azione di Scatto}corre ed ha un malus di 1d6 nel Tiro per Colpire e la Difesa diminuisce di 4 per tutto il round in cui ha usato l'Azione Scatto e si considera Distratto per il lancio di incantesimi.

Non è possibile spostarsi anche solo di 1 metro se non si spendono Azioni di Movimento.

Queste precisazioni hanno senso e vanno usate quando si tratta di combattere ed il dislocamento sul territorio, mappa, è fondamentale. Durante gli spostamenti normali, mentre si cavalca o cammina liberi senza pericoli, si usa la normale gestione del movimento orario.

Nel caso di spostamento diagonale si conta una distanza di 1,5 metri per quadretto.

Per \textbf{distanza di Tocco} \index{Distanza di Tocco} \index{Tocco}si intende una distanza che permette il toccare l'avversario, quindi non più di un metro per creature di taglia media. La distanza di tocco è anche chiamata \textbf{distanza di mischia}.

Se non indicata nell'avversario/mostro la distanza di tocco aumenta di 1 metro per ogni taglia oltre la media.

Per \textbf{distanza di Mischia} \index{Distanza di Mischia} \index{Mischia}si intende una distanza che permette il combattimento corpo a corpo (1 metro attorno al personaggio). Nei mostri questa distanza è indicata dalla portata, per le armi da lancio è chiamata gittata.

\begin{tcolorbox}[title = Esempi di Distanza in Combattimento] 
Es. per un umano armato di lancia, la distanza di mischia è 2 metri perché l'arma è lunga.

Per un nano armato di martello la distanza di mischia è 1 metro.
Per un gigante delle colline la distanza di mischia è 2 metri. Per gli attacchi a distanza si parla di Portata o Gittata.
\end{tcolorbox}

Quando si parla di "\textbf{quadretto}" \index{Quadretto}per indicare una distanza od una influenza si intende un quadretto di mappa di 1 metro x 1 metro.
Se un movimento termina in "mezzo" quadretto si arrotonda per difetto.

\medskip

\textbf{Se ci si sposta in terreno "difficile", un umano copre 4 metri per Azione di movimento (ogni quadretto conta per due).}

L'Azione (di movimento) può avvenire prima e dopo l'Azione (di Attacco).

A distanza di mischia una creatura di dimensioni medie può avere al massimo 8 creature medie.


\subsubsection{Creature Grandi e Piccole in Combattimento {*} (Opzionale)}

\label{creature-grandi-e-piccole-in-combattimento-opzionale}

Le creature molto piccole possono stare più di una ad una distanza di mischia, mentre creature grandi tenderanno ad occupare tutto lo spazio di mischia.

\medskip

\textbf{Tabella: Taglia e Scala delle Creature}\index{Tabella Taglia e Scala delle Creature}

\medskip


\begin{tabular}{ll}
	\textbf{Taglia} & \textbf{Creature}\\
\multicolumn{1}{c}{della Creatura}	&  \multicolumn{1}{c}{in distanza di mischia}\\
	\toprule
	Piccolissima                   & 100\\
	Minuta                         & 64\\
	Minuscola                      & 32\\
	Piccola                        & 16\\
	Media                          & 8\\
\end{tabular}

\smallskip
Questi sono i valori tipici delle creature per la taglia indicata. Sono frequenti eccezioni.

\end{multicols}


\begin{center}
	\includegraphics[width=0.5\linewidth]{immagini/camminata.jpg}
\end{center}

\pagebreak

\subsection{Vita e Morte}\index{Morire}

\label{vita-e-morte}
\begin{enfasi}{Chi non conosce la morte, non conosce la vita. (Grand Hotel, film 1932)

\medskip		
		
The worthy GM never purposely kills players' PCs. He presents opportunities for the rash and unthinking players to do that all on their own (Il meritevole Game Master non uccide mai volontariamente i personaggi dei giocatori. Lui presenta opportunità per i giocatori frettolosi e sbadati di fare tutto da soli) (Gary Gygax)}\end{enfasi}\medskip

\begin{multicols}{2}

Quando un personaggio raggiunge i 0 (zero) Punti Ferita si considera svenuto, ovvero inabile a fare qualsiasi cosa. Una Cura (Incantesimo, Pozione..) lo porterà cosciente ed ai punti ferita curati. Una prova di Pronto Soccorso, 3 Azioni, a DC 15 lo porterà ad 1 punto ferita. Dopo un ora se non è successo qualcosa a mutare la situazione il personaggio può fare un Tiro Salvezza su Tempra a DC 15, se riesce torna a 1 Punti Ferita, se fallisce va a -1 e diventa morente.

Un personaggio morente ha Punti Ferita negativi (-1 o meno) ed è privo di sensi e \hyperlink{morente}{prossimo alla morente morte}. Continuerà a perdere un punto ferita a round fiche il valore non raggiungerà il doppio della Costituzione+10 ed il personaggio morirà, se non viene curato.

Un incantesimo di Cura, di qualsiasi Difficoltà lo porterà a 1 Punto Ferita successive cure funzioneranno normalmente.

Una prova di \hyperlink{prontosoccorso}{Pronto Soccorso} 3 Azioni a difficoltà 11 più il valore dei punti ferita negativi porterà il personaggio a 0 punti ferita, ovvero svenuto.

\begin{center}
	\includegraphics[width=0.9\linewidth]{immagini/Nuremberg_chronicles.png}
	
	\textit{The Dance of Death (1493) by Michael Wolgemut, from the Nuremberg Chronicle of Hartmann Schedel}
\end{center}

Se un attacco o incantesimo porta il personaggio direttamente a -10-COS*2, il personaggio muore\index{Morte immediata}\index{Danno Massivo} senza possibilità di essere curato.

\begin{tcolorbox}[title = Tups sta morendo] 
Es. Tups é gravemente ferito ed ha attualmente -6 punti ferita, Jade decide di provare a curarlo (dopo averlo spostato in un posto più sicuro). Jade tenta una prova di Pronto Soccorso per almeno stabilizzare il compagno, la sua difficoltà alla prova é 11+6 ovvero deve superare con Pronto Soccorso DC 17 per riportarlo a 0 Punti Ferita (svenuto)

Una successiva prova di Pronto Soccorso a DC 15 potrà portarlo a 1 Punti Ferita e una cura magica lo curerà dell'ammontare dichiarato.
\end{tcolorbox}

Quando un personaggio torna a punti ferita positivi dopo che era andato negativo il suo numero di incantesimi lanciabile nel giorno diminuisce di uno.

Quando un personaggio arriva a punti ferita negativi pari 10+doppio del suo punteggio di Costituzione é \hyperlink{morto}{\textbf{morto}} (-10-(COS*2)).

\begin{center}
	\includegraphics[width=0.9\linewidth]{immagini/giantdeath.png}
	\textit{Henry Hustice Ford}
\end{center}

Se il danno non letale di un personaggio arriva a punti ferita negativi pari 20+4*Costituzione il personaggio é morto.

Es. Se ha Costituzione 2 morirà a -10-4=-14 Punti Ferita, se ha Costituzione 0 morirà a -10 Punti Ferita, se ha Costituzione -2 morirà a -10+4=-6 Punti Ferita. In caso di valori di Costituzione pari od inferiore a -3 il personaggio muore a -5 punti ferita.

\bigskip

Un personaggio morto non può beneficiare delle cure normali o magiche, e non può essere riportato in vita da un incantesimo. Solo un Patrono ha sufficiente potere per riportare l'anima nel corpo e riportare in vita la creatura. L'incantesimo di Animare i morti può rianimare un corpo, ma come non morto.

\subsubsection{Recupero da 0 Punti Ferita* - Opzionale} \index{Recupero} \index{Svenuto}

\textbf{Nel caso vogliate un sistema meno letale potete applicare questa regola opzionale.}

Ogni round successivo ad essere andato a 0 punti ferita o meno, quindi svenuto o morente, il personaggio deve effettuare un Tiro Salvezza Tempra a difficoltà 15, se riesce riprende coscienza e va ad 1 punto ferita.

Se fallisce la prova può effettuarne un altra a DC +1 rispetto alla precedente il round successivo. Quanto la difficoltà raggiunge 18 (ovvero 3 prove fallite di seguito) il personaggio muore.

Appena la prova riesce (entro i 3 fallimenti) il personaggio torna ad 1 punto ferita.

\subsubsection{Recupero punti Caratteristica}\index{Recupero punti caratteristica}

Eventuali punti caratteristica persi si recuperano al ritmo di 1 punto totale al giorno, se non indicati come perdita permanente.

\subsubsection{Recupero punti Ferita naturale}\index{Recupero punti naturale}

Riposare 8 ore fa recuperare il punteggio di Costituzione + Competenza Armi al giorno in Punti Ferita, con un minimo di 1.

\subsubsection{Recupero punti ferita non letali}\index{Recupero Punti Ferita non letali}\index{Punti Ferita non letali}

Ogni ora si recupera, con un minimo di 1 Punto Ferita, il proprio valore di Costituzione.

\end{multicols}

\begin{center}
	\includegraphics[width=0.95\linewidth]{immagini/caravaggioSalomeLondon.png}
	\textit{Salomè con la testa del Battista è un dipinto di Caravaggio realizzato in olio su tela (91x106 cm) tra il 1607 e il 1610. È conservato nella National Gallery di Londra.}
\end{center}

\pagebreak

\subsection{Tiro per Colpire e Difesa}\index{Tiro per Colpire}\index{Difesa}

\label{tiro-per-colpire}
\begin{enfasi}{Applica sempre la giusta forza, mai troppa mai troppo poca. (Kano Jigoro)}\end{enfasi}\medskip

\begin{multicols}{2}

Il Tiro per Colpire è una prova contrapposta alla Difesa dell'avversario.

Se l'attaccante usa:

\begin{itemize}
	\item \textbf{Armi da Mischia o Contatto}: l'attaccante deve effettuare un \textbf{Tiro per Colpire (TC)}= 3d6 + Competenza Armi + Forza ed eventuali bonus di Abilità e magici dell'arma e fattori circostanziali (ambiente, maledizioni..)
	
	\item
	\textbf{Armi da Distanza o Versatili}: l'attaccante deve effettuare un Tiro per Colpire (TC) = 3d6+ Competenza Armi + Destrezza + ed eventuali bonus di Abilità e magici dell'arma e fattori circostanziali (ambiente, maledizioni..). Vale per archi, balestre, pugnali tirati, scimitarre...
	
	\item
	\textbf{Incantesimo}: l'attaccante deve effettuare un Tiro per Colpire (TC) = 3d6+ Competenza Magica + Forza (per incantesimi in mischia) oppure + Destrezza (per incantesimi a distanza) eventuali Abilità e modificatori circostanziali.
\end{itemize}

Chi si difende ha una \textbf{Difesa} pari a: 10 + Destrezza + Scudo + Armatura + eventuali bonus magici ed Abilità e bonus circostanziali.
Il giocatore può decidere di rinunciare a del bonus dato dalla Competenza Armi per avere un migliore punteggio di Difesa. Questi punti non saranno a disposizione nell'attacco successivo (vedi Capitolo sul combattimento).

Ogni qual volta di parla di bonus alla Difesa si intende un valore da sommare al valore Difesa ottenuto con il calcolo di cui sopra.

\begin{center}
	\includegraphics[width=0.9\linewidth]{immagini/Coypel_Charles-Antoine_-_Fury_of_Achilles_-_1737.png}
	\textit{Charles Antoine Coypel - Fury of Roland - 1737}
\end{center}

\subsection{La Difesa e l'Attacco}\index{Difesa}\index{Attacco}

\label{la-difesa}
\begin{enfasi}{La difesa è sempre legittima (anonima vittima)}\end{enfasi}\medskip

Ogni Tiro per Colpire (3d6 + Competenza con Armi + Forza o Destrezza + eventuali bonus/malus) si raffronta la Difesa ovvero un valore pari a 10 + Destrezza + Scudo + Armatura + eventuali bonus/malus.

Se il Tiro per Colpire è pari o superiore al valore della Difesa l'avversario è stato colpito e si stabilirà il danno della ferita, dato dall'arma + punteggio Forza/Destrezza ed altri fattori quali bonus magici e di Abilità.

Se il TC (Tiro per Colpire) è più basso della Difesa allora l'avversario avrà parato, schivato, evitato.. La scelta la si lascia al giocatore (o Narratore), evitato l'attacco non si subiscono ferite.

Altre situazione possono avvantaggiare la Difesa quali coperture, nascondigli, come fosse, porte, compagni di taglia molto più grande della propria. Consultate i paragrafi relativi per capire il vantaggio che possono dare.

Ci sono occasioni in cui non è importante penetrare la difesa e sferrare un colpo ma semplicemente basta toccare l'avversario.

Altre volte l'avversario è sorpreso e non può difendersi completamente.

Se è \textbf{sufficiente toccare l'avversario} la Difesa sarà 10 + Destrezza + bonus magici, senza bonus Scudo e Armatura. Questa è il valore della \textbf{Difesa a tocco}.

Se \textbf{l'avversario è sorpreso} ovvero non si aspetta l'attacco la Difesa sarà 10 + bonus magici + Armatura, senza bonus di Scudo e Destrezza. Questo è il valore della \textbf{Difesa di sorpresa}.

\textbf{Anche per il Tiro per Colpire valgono le Golden Rules. I d6 esplodono in caso si tiri 6 con il dado, fare 1 porta male ed affidarsi alla sorte (ovvero togliere 4 punti tra Competenza Armi e Forza o Destrezza per aggiungere 1d6 al Tiro per Colpire).}

Se i modificatori e circostanze portano il danno inflitto ad essere negativo comunque farai 1 di danno.

Questa regola si applica ai modificatori del danno dell'arma che appunto non possono portare il danno totale ad essere inferiore a 1, se ci sono protezioni magiche o riduzioni del danno questo può diventare zero e quindi non ferirai l'avversario (ma se diventa negativo non lo curi!).

Come prima cosa, come spiegato poco prima, ricorda che per ogni 6 tirato (nei 3d6 del Tiro per Colpire) devi tirarne un altro e continuate a tirare finché continui a fare 6 con il dado.

Se colpisci, \textbf{ogni due 6 tirati} (contando quelli del Tiro per Colpire e quelli successivi scaturiti dal fatto di aver tirato 6), l'arma fa del danno in più ovvero un critico. Tira nuovamente il dado del danno dell'arma, senza magia o Forza o Abilità ogni due 6 tirati nel Tiro per Colpire.

Puoi \textbf{togliere 4} o multipli al tuo attacco per tirare un d6 in piu'. La scelta è da fare nelle situazioni più disperate dove solo la fortuna può risolvere il duello. Il valore lo togli dal valore di Competenza Armi e non di Forza o Destrezza o punteggi dati da Abilità.

In caso si tiri un 1 nel Tiro per Colpire questo abbassa di 1 (quindi 1 non conta) il valore totale ma non influisce sul fatto di aver fatto critico o meno.

\textbf{Il fatto di tirare un critico non è garanzia di aver colpito, bisogna sempre superare la Difesa}.

Anche per il Tiro per Colpire valgono le regole base delle Competenze.
La Difesa è un valore fisso e come tale usa i modificatori per le prove a valore fisso.

\subsection{Tirare 3 volte 1}\index{Tirare 3 volte 1}

Se nei primi 3 Tiri per Colpire fai tre volte 1 mancherai l'avversario (indipendentemente dal risultato finale del Tiro per Colpire) ed il Narratore potrebbe decidere brutte cose sul tuo attacco (ti cade l'arma, colpisci un amico, ti si rompe l'arma, ti ferisci, cadi, appare un demone delle fosse per deriderti...)

\subsection{Tirare 3 volte 6}\index{Tirare 3 volte 6}

Se nei primi 3 Tiri per Colpire fai tre volte 6 prenderai l'avversario indipendentemente dal risultato finale del Tiro per Colpire. Oltre ad avere la certezza di aver fatto un critico (vedi sotto) il Narratore potrebbe decidere di applicare qualche effetto descrittivo (o effettivo) ulteriore.

\begin{center}
	\includegraphics[width=0.9\linewidth]{immagini/critico.png}
	\textit{Henry Justice Ford}
\end{center}

\subsection{Tiro Critico}\index{Tiro Critico}

Ogni qual volta hai colpito, tiri un danno aggiuntivo di arma (senza bonus magici o di Abilità o Forza, solo arma) per ogni due volte che hai tirato 6 nel Tiro per Colpire.

\medskip

\begin{tcolorbox}[title = Esempio Tiro Critico] 
Es tiro 6 4 5, tiro in aggiunta 6, tiro in aggiunta un 6, tiro in aggiunta 4: come danno tiri 2 volte il danno dell'arma, una perché ho colpito una perché hai tirato tre volte 6 (se avessi tirato un ulteriore 6 sarebbero stati Arma + Forza + bonus/abilita + 2{*}Arma).
\end{tcolorbox}


\subsection{Tiro Critico Variante}\index{Tiro Critico Variante}

Il giocatore potrebbe prediligere meno il caso e gestire i critici in base alla "bravura" del personaggio di usare l'arma.
Un metodo alternativo è quello di concedere un danno di arma aggiuntivo (solo il dado del danno dell'arma, senza altri bonus) per ogni multiplo di 6 in cui il Tiro per Colpire è superiore alla Difesa.

La scelta di usare la variante del tiro critico va fatta al momento della creazione del personaggio e si può cambiare sono rinunciando ad un punto di Competenza Armi (devi reimparare a combattere in modo diverso).

\medskip


\begin{tcolorbox}[title = Tiro Critico Variante] 
	Es. Tiro per Colpire 21, la Difesa dell'avversario è 13. Lo colpisco con un margine di 8, ovvero aggiungo 1 danno di arma in piu'
	Se il Tiro per Colpire fosse stato 26 si sarebbero aggiunti 2 danni Critici ovvero due danni d'arma.
\end{tcolorbox}

In questa variante non si ha Tiro Critico (ottenere più 6 nel Tiro per Colpire) ne l'Esplosione del danno (danno massimo sul dado dell'arma), rimangono valide le Golden Rules.

\subsection{Esplosione del Danno}\index{Esplosione del Danno}

Ogni qual volta dal tiro del dado dell'arma ottieni il valore massimo (nel classico d8 per la spada ad esempio fai 8 ed è quindi il valore massimo del dado), ritiri il dado e sommi ancora il valore (del solo dado).

In caso di armi con più dadi (esempio 2d4, il valore massimo deve essere ottenuto come somma dei due dadi, ovvero 8). Non c'è esplosione del danno per le armi con danno massimo inferiore o uguale a 6.

\begin{center}
	\includegraphics[width=0.9\linewidth]{immagini/esplosionedanno.png}
\textit{Henry Justice Ford}
\end{center}

Alcune armi hanno una esplosione del danno diversa. Nella tabella delle armi dove è segnato EDX (es ED9), il valore X sta per il valore minimo sufficiente per tirare un'altra volta il danno, quindi in caso di ED9 puoi fare il critico con 9 o 10.

Questa è una caratteristica di poche armi estremamente letali.

L'esplosione del danno non esplode a sua volta, anche se fai il massimo del dado con il dado aggiunto questo non esplode nuovamente.

I tiri di dado aggiunti grazie al critico (ottenuto lanciando almeno due 6) non ha il vantaggio dell'esplosione del danno. Se il dado dell'arma in più tirato grazie al Tiro Critico fa il massimo non ritiri ed aggiungi il danno.

\subsection{Attacchi multipli in mischia}\index{Attacchi multipli}

Con una Azione il personaggio può eseguire un singolo attacco in mischia.

Con due Azioni il personaggio può effettuare fino a due Tiri per Colpire consecutivi (Azione di Multiattacco).

La prima azione di attacco non ha malus mentre la seconda azione di attacco ha -5 al colpire cumulativo per attacco.

Se il malus al colpire cumulativo diventa maggiore del Tiro per Colpire non è più possibile fare ulteriori attacchi.

Se ho CA 5 e Forza 2, il primo Tiro per Colpire sarà 3d6+7, il secondo sarà 3d6+2. Non è possibile effettuare un terzo attacco in quanto il bonus al colpire diventerebbe negativo.

Si conta solo Competenza Armi e Forza/Destrezza e bonus dati da Abilità (comprese i bonus dati dalle Liste d'Armi) e nessun bonus magico dell'arma si conta per capire quanti attacchi è possibile fare.

Nel caso il personaggio voglia eseguire attacchi multipli, deve dichiarare se fare gli attacchi su un solo avversario o su più.

Se dopo il primo attacco il target muore (in caso di azione di multiattacco) non puoi dirigere gli attacchi rimanenti su altri bersagli tranne se ha l'Abilità Proseguire e il successivo avversario è già in mischia con te.

Diversamente può dichiarare di effettuare il primo attacco su una creatura ed il secondo su altro. Questo secondo attacco può essere inframmezzato da una Azione di Movimento.

\textbf{Se invece il personaggio vuole effettuare più di due attacchi in mischia deve usare 3 Azioni e potrà attaccare solo chi è in mischia con lui se non ha altre Azioni.}

\subsection{Armi da Tiro - Archi - Balestre (Arco / Balestre / Pugnali..)}\index{Attacchi multipli armi da tiro}

Il numero di attacchi massimi con armi da lancio è 2 usando due Azioni, il secondo lancio si considera come un secondo attacco con un -5 al colpire.

Solo particolari Abilità permettono di effettuare ulteriori attacchi, se si ha la possibilità e si vuole scoccare/lanciare 3 frecce è necessario usare 3 Azioni.

Il bonus al danno dato da Forza si applica in automatico per le fionde, Pugnali..ovvero con tutte le armi che vengono lanciate "a mano", gli archi applicano questo bonus solo se sono di tipo composito, le balestre non lo applicano mai.

Non si applica bonus al danno dato da Destrezza, la Destrezza modifica solo il Tiro per Colpire.

\textbf{I proiettili lanciati da Archi, Fionde, Balestre magiche non si considerano magici.}

\begin{center}
	\includegraphics[width=0.9\linewidth]{immagini/archer.png}
\textit{Scythian archers in ancient attic vase painting}
\end{center}


\subsection{Attacchi con armi a spargimento} \index{Armi a spargimento}\index{Acqua Benedetta}\index{Olio Incediato}

Sono armi a spargimenti quelle che "spargono" il loro contenuto dove cadono, ad esempio olio incendiato/acqua benedetta..

In caso l'attacco manchi (di almeno 5) tirare un d8 e consultare questo schemino per capire dove la boccia è caduta:

1 2 3

4 \textbf{X} 5

6 7 8

\textbf{X} si considera il bersaglio dell'oggetto tirato.

Se il tiro manca di 5 o più tirare un 2d6 per determinare lungo la direzione indicata dal d8 precedente a quanti metri è caduto distante dal bersaglio, ovvero contate i metri dal target.

Ad esempio con il tiro del d8 faccio 5 e poi tirando 2d6 faccio 4, significa che la boccetta è caduta a destra del bersaglio a 4 metri.

E' anche possibile che ci si sia tirati sui piedi la boccetta (es faccio 7 e poi 6.. potrei averla tirata addosso ad un compagno o dietro di me!).

\begin{center}
	\includegraphics[width=0.9\linewidth]{immagini/fenice.png}
\textit{Henry Justice Ford ... caduta della piuma di fenice..}
\end{center}


\subsection{Impreparato -- Colti di Sorpresa}\index{Impreparato}\index{Sorpresa}

Se i personaggi vengono colti di sorpresa, ovvero non si aspettano di essere attaccati, si deve considerare questo primo round come round di sorpresa. Quando si è sorpresi non si può usare la propria Destrezza o Scudo in Difesa.

Per quel round e per gli attacchi di quel round ti difenderai solo con la Armatura (senza scudo e Destrezza), non potrai reagire, non userai Azioni o Reazioni se non esplicitamente permesse; dal round successivo potrai dichiarare l'iniziativa ed agire normalmente. Le medesime considerazioni valgono per gli avversari.

Per valutare se un personaggio è sorpreso effettuate un Tiro Salvezza su Riflessi, confrontandolo con la bravura nel nascondersi (o muoversi silenziosamente) degli avversari, se la prova è fallita il personaggio è sorpreso.

Se il personaggio è sull'attenti e si aspetta un agguato fategli fare una prova di Consapevolezza contrapposta alla prova di Nascondersi/Muoversi Silenziosamente dell'avversario, se la prova è superiore allora il personaggio non è sorpreso.

Quando personaggi e nemici sono colti entrambi di sorpresa per valutare chi effettivamente è sorpreso effettuate un Tiro Salvezza su Riflessi, chi ottiene il risultato più basso è sorpreso e per quel round non potrà agire. Ogni personaggio e nemico fa il proprio tiro.

\subsection{Modificatori di attacco o difesa per situazioni particolari} \index{Situazioni particolari}

Il migliore suggerimento che si può dare nel gestire le situazioni di combattimento più caotiche è pensare a queste come ad un film, valutate la cinematicità della situazione.

Non è una questione di miniature, spazi, quadretti.. E' una questione di divertimento e visualizzazione della scena. Soluzioni non ortodosse per situazioni non ortodosse.

Concedete un bonus o malus (+-1/2 se non esplicitato diversamente) ogni qual volta il giocatore abbia un vantaggio o svantaggio ed allo stesso modo all'avversario. Se il bonus o malus è sulla Difesa allora usate +-1/2 (o +-4 se il bonus è maggiore) di bonus al posto del d6.

\bigskip

\textbf{Esempi in situazione di Attacco (bonus o malus al Tiro per Colpire)}

\begin{itemize}
	\item Situazione con +2 bonus: più di uno ad attaccare un avversario...

	\item Situazioni con 1d6 bonus: posizione sopraelevata (a cavallo), carica, invisibile...

	\item Situazione con 1d6 di svantaggio: sei abbagliato, sei intralciato, sei prono, sei ristretto nei movimenti, sei spaventato o scosso, usare un arma da lancio contro un avversario in mischia, attaccare con arma lunga in distanza da mischia...
\end{itemize}

\textbf{Esempi in situazione di Difesa:}

\begin{itemize}
	\item Situazioni con +2/+4 bonus (bonus alla Difesa): hai copertura (vedi sotto),

	\item Situazioni con -4 di svantaggio (malus alla Difesa): sei accecato, intrappolato, sei in ginocchio o seduto, sei prono, sei ristretto in uno spazio, sei stordito, lanci un incantesimo....
\end{itemize}

\textbf{Quando si scrive -1d6 significa che si tira un dado in meno (o due se è -2d6), parimente se c'è scritto +1d6 si tira un dado a 6 in più e si somma}.

\textbf{Quando il malus è alla Difesa considerare ogni -1d6 come un -4 alla Difesa}.

Se non si vuole tirare il dado di bonus/malus considerare allora ogni d6 come valore +-4 (a seconda che sia un bonus od un malus).

\textbf{In linea di principio in combattimento un bonus leggero è un +1, medio +2, alto +1d6 (o +4), un bonus molto alto è +2d6 (o +8), viceversa per per i malus}.

\bigskip
Ricordate sempre lo scopo è divertirsi, a scapito (per il Narratore) di qualche mostro, non siate rigidi ma dinamici e adattatevi alle situazioni.


\begin{center}
	\includegraphics[width=0.9\linewidth]{immagini/vantaggio.png}
	\textit{Henry Justice Ford}
\end{center}

\subsection{Azioni particolari in combattimento} \label{sec:Azioni particolari in combattimento}

\subsubsection{Attacco a mani nude} \index{Pugno}\index{Calci} \index{Fare a botte}

due armi che non mancheranno mai a nessuno sono i propri pugni e calci.

Se non si ha preso la lista d'armi "Pugno Vuoto" un pugno o calcio farà 1d2 + Forza di danno non letale.
Solo con la lista d'armi "Pugni nudo" si diventa artisti marziali e le "armi" diventano Versatili (si può applicare al Tiro per Colpire ed al danno il valore di Forza o Destrezza) ed il danno diventa letale.

\subsubsection{Attacco di Opportunità} \index{Opportunità}\index{Attacco di Opportunità}

se un soggetto usando una Azione di movimento esce o attraversa la zona di mischia del personaggio, al personaggio è concesso un singolo attacco contro il soggetto. Questo attacco è una Reazione che costa una Azione. Stessa cosa vale per gli avversari.

\subsubsection{Alzarsi da prono}\index{Alzarsi da prono}

costa due Azioni , prendi un -4 alla Difesa ed un -4 Iniziativa. Il personaggio può eseguire una prova di Acrobatica, se è pari o superiore a 15 ed entro 20 ti permette di dimezzare questi malus e costa una Azione alzarsi, se fai 20 o più annulli i malus e costa 1 Azione alzarsi.

\subsubsection{Carica} \index{Carica}

l'avversario deve essere ad una distanza entro 2 Azione di movimento (18 o 12 metri solitamente). Si deve correre fino ad essere a distanza di mischia.

Si ottiene un +1d6 a Tiro per Colpire, -4 alla Difesa, l'attacco successivo al primo ha un -15 al colpire (questo per valutare se è possibile farlo o meno). La carica e attacco/attacchi costa 3 Azioni. Non si considerano altri malus per avere corso oltre quelli indicati.

\begin{center}
	\includegraphics[width=0.9\linewidth]{immagini/carica.png}
\textit{	Frontispiece in: A Connecticut Yankee in King Arthur's Court / Samuel Clemens. New York : Charles L. Webster \& Co., 1889}
\end{center}


\subsubsection{Controcarica}\index{Controcarica} 

un'arma con il talento controcarica se usata contro un avversario/cavalcatura in carica infligge il doppio del danno dell'arma e colpisce per prima, tranne in cui l'avversario abbia un arma lunga, in questo caso l'attacco è regolato dai tiri di iniziativa. Preparare un'arma per la controcarica è una Reazione che costa una Azione.

Se anche l'avversario ha un arma lunga senza caricare non si usa l'azione di controcarica, ma si tirano le rispettive iniziative.

\subsubsection{Carica con Arma da Controcarica} \index{Controcarica}

se usi un arma con il talento controcarica per caricare un avversario la tua arma fa il doppio del danno dell'arma, ovvero tiri due volte il dado dell'arma, entrambi i dadi di danno dell'arma possono esplodere, senza contare nel secondo tiro i bonus magici eventuali. Se il difensore non ha un arma lunga allora valgono anche le considerazioni di Arma Lunga.

\begin{center}
	\includegraphics[width=0.9\linewidth]{immagini/pilum.png}
\textit{Soldati romani armati di Pilum, pronti per una controcarica.}
\end{center}


\subsubsection{Aiutare un altro}\index{Aiutare}

Si può aiutare un amico ad attaccare o a difendersi negli scontri in mischia, distraendo o interferendo con un avversario.

Si può portare un attacco in mischia (1 Azione) contro un avversario che ha già ingaggiato battaglia con un proprio alleato.

Si effettua un Tiro per Colpire contro Difesa dell'avversario con 1d6 di bonus. Se l'attacco va a segno, non si fa danno, il compagno ottiene bonus di +1d6 al Tiro per Colpire con il prossimo attacco (entro la fine del successivo round) verso quell'avversario o un bonus di +4 alla Difesa contro il prossimo attacco di quell'avversario (a propria scelta).

Più personaggi possono aiutare lo stesso alleato; i bonus di questo tipo sono cumulabili (massimo 4 su taglia media), purché l'avversario sia circondato.

\subsubsection{Colpo di Grazia} \index{Colpo di Grazia}costa 3 Azioni, si può utilizzare un'arma da mischia per infliggere un colpo di grazia ad un target indifeso (svenuto o intrappolato). Si può anche usare un arco o una balestra, l'importante è che si sia adiacente al bersaglio.


\begin{center}
	\includegraphics[width=0.9\linewidth]{immagini/colpodigrazia.png}
	\textit{Decollazione di San Giovanni Battista. Concattedrale di San Giovanni a La Valletta (Malta). (Caravaggio). \\Colpo di Grazia}
\end{center}


L'attaccante colpisce automaticamente ed infligge due colpi critici (due volte in più il danno dell'arma e null'altro). Se il difensore sopravvive al danno, deve superare un Tiro Salvezza su Tempra DC pari ai danni subiti o muore.

La creatura è immune ai colpi critici, non subisce i danni critici, né deve superare un Tiro Salvezza su Tempra per evitare di essere uccisa da un Colpo di Grazia.

\subsubsection{Danno non letale}\index{Danno non letale} 

Il danno non letale è una forma di danno causato da armi particolari o quando volutamente lo scopo è fare svenire il nemico e non ucciderlo.

Il danno non letale si tratta come il danno normale ma va segnato a parte nella scheda.

\subsubsection{Danno non letale con arma non idonea} \index{Danno non letale con arma non idonea}

Se vuoi fare danno non letale con un'arma non predisposta al danno non letale hai un -1d6 al Tiro per Colpire.

\subsubsection{Senza Competenza}\index{Senza Competenza}

Usare un'Arma senza l'adeguata competenza impone un -1d6 al Tiro per Colpire (quindi il Tiro per Colpire diventa 2d6+Forza/Destrezza+Abilità..). Non puoi usare la capacità Versatile di un arma se non la sai usare.

\subsubsection{Armi Leggere} \index{Armi Leggere}Il giocatore può usare queste armi come armi secondarie senza subire penalità.

\subsubsection{Armi Versatili} \index{Armi Versatili}Il giocatore può liberamente usare la Destrezza invece della Forza sui Tiri per Colpire e danno con le armi versatili.

\subsubsection{Lanciare armi} \index{Lanciare armi}

Una spada o comunque un arma non fatta per essere lanciata può comunque essere scagliata contro l'avversario.

Il Tiro per Colpire prende un -1d6 e l'arma fa una categoria di danno inferiore (la spada lunga fa 1d6, una spada corta 1d4..). La gittata di lancio è 3 metri.

\subsubsection{Colpi Potenti}\index{Colpi Potenti}

Il giocatore può liberamente aggiungere un +2 al danno togliendo 1 al Tiro per Colpire (requisito Competenza Armi +1). Non si può togliere più di Competenza Armi/4 al Tiro per Colpire.

\subsubsection{Maestria del combattimento} \index{Maestria del combattimento}

Il giocatore può liberamente aggiungere +4 alla Difesa per ogni -1d6 al Tiro per Colpire. Il bonus è applicabile solo per gli attacchi in mischia. Viceversa può prendere un -4 Difesa per alzare di +1d6 il Tiro per Colpire e quindi migliorare l'attacco. 

Non è possibile assegnare in questa maniera più di +-2d6.

\begin{center}
	\includegraphics[width=0.9\linewidth]{immagini/angelospadone.png}
\end{center}

\subsubsection{Arma troppo grande}\index{Arma troppo grande} \label{sec:Arma troppo grande}

Attaccare con un'\textbf{Arma troppo grande} \index{Arma troppo grande}rispetto alla propria taglia è problematico.

\begin{itemize}
\item
Una creatura Piccola può usare armi piccole ad una mano e armi medie a due mani
\item
Una creatura Media può usare armi piccole ad una mano, armi medie ad una o due mani e armi grandi a due mani.

\item
Una creatura Grande può usare armi medie con una mano, armi grandi con una mano o due, armi enormi a due mani.
\end{itemize}

Se l'arma non è tra quelle "usabili", esempio un Alabarda (arma grande) per una creatura di taglia piccola la penalità al Tiro per Colpire è -1d6.

Nella tabella delle armi la taglia è segnata come P (piccola), M (media), G (grande), E (enorme).

\subsubsection{Fiancheggiare} \index{Fiancheggiare}se due personaggi sono attorno allo stesso bersaglio ma non sono a fianco entrambi prendono +2 al Tiro per Colpire o alla Difesa (a loro scelta quale bonus prendere).

Al massimo ci possono essere 4 personaggi attorno ad una creatura di taglia media che prendono il bonus di fiancheggiare. Il tipo di bonus si sceglie round per round, se non dichiarato vale come +2 al Tiro per Colpire.

Se tirando una ipotetica riga che collega i due personaggi questa attraversa in pieno il quadretto dell'avversario allora c'è la situazione di fiancheggiamento.

\bigskip

Esempio di fiancheggiamento\index{Esempi di Fiancheggiamento}

\medskip

\begin{tabular}{lll}
	\toprule
	Personaggio A & Personaggio G & Personaggio D\\
	Personaggio B & \textbf{Avversario}    & Personaggio E\\
	Personaggio C & Personaggio H & Personaggio F\\
\end{tabular}

\bigskip

In questo schema il fiancheggiamento è preso dalle coppie: A-F, B-E, C-D, G-H

\bigskip

Se la creatura può fronteggiare più creature contemporaneamente queste non godranno del bonus di fiancheggiamento.

\subsubsection{Arma Doppia} \index{Arma Doppia}

Un'arma doppia è un'arma che è pericolosa da entrambe le estremità. Può essere usata come arma singola, oppure, incorrendo nelle penalità del combattimento con due armi, come appunto due armi. Se non specificato un'arma doppia usata per combattimento con due armi equivale ad usare due armi medie.

\subsubsection{Arma Lunga} \index{Arma Lunga}

L'arma lunga da diritto a colpire più lontano ovvero a 2 metri. Concede un bonus all'iniziativa pari a +4. Questo bonus rimane valido finché l'avversario non entra in distanza della propria di mischia.

Nel caso in cui anche l'avversario abbia un arma lunga non considerare il bonus di 4 all'iniziativa (avendolo entrambi si annulla a vicenda).

\begin{tcolorbox}[title = Combattimento con Arma Lunga] 
Es. Tups armato di spada lunga affronta uno brigante armato di lancia lunga. Tups ha iniziativa 15, il brigante 12 ma ha un arma lunga e quindi la sua iniziativa è 16.

Il brigante attacca Tups mentre questo si avvicina, il valore di iniziativa mi "mostra" come il brigante sfruttando la sua arma lunga riesca ad agire prima di Tups. Una volta che Tups si è avvicinato in mischia sarà più veloce del brigante.

Se il brigante avesse avuto iniziativa 10 Tups avrebbe attaccato per primo, praticamente il brigante non sarebbe riuscito a sfruttare il vantaggio dato dell'arma lunga (15 contro 14 di iniziativa).

Se il brigante avesse dichiarato di attaccare e poi allontanarsi avrebbe costretto Tups ad usare due azioni di movimento per raggiungerlo.

Il brigante avrebbe attaccato da distanza di 2 metri ed usato una Azione per andare in distanza di 11 metri (2+9 di velocità). Tups avanzando normalmente non avrebbe raggiunto il brigante solo facendo una doppia Azione di Movimento riesce ad andare in mischia

Tups potrebbe fare uno Scatto, usando quindi solo un'Azione (9mx2=18 metri) e poi attaccare ma avrebbe un -1d6 al Tiro per Colpire e -4 Difesa.

Il brigante una volta raggiunto da Tups butta l'arma lunga a terra per non avere -1d6 di penalità al colpire ed estrae un pugnale.
\end{tcolorbox}

\subsubsection{Arma lunga a breve distanza} \index{Arma lunga a breve distanza}

E' possibile usare un'arma lunga a distanza di 1 metro (mischia) con un -1d6 al Tiro per Colpire, ad eccezione del Bastone.

\subsubsection{Magia in combattimento}\index{Magia in combattimento}

L'incantatore che lancia una magia mentre è in combattimento (ha un avversario in mischia) prende un -4 alla Difesa e si considera Distratto (la Difficoltà dell'incantesimo aumenta di 5).

\subsubsection{Preparare una arma lunga contro una carica} \index{Preparare una arma lunga contro una carica}

E' una Reazione che costa una Azione.

\subsubsection{Prendere la Mira (cecchino)} \index{Prendere la Mira (cecchino)}

Per ogni round in cui prendi solo la mira, 2 Azioni, guadagni un +1 al Tiro per Colpire, fino ad un massimo di +3 alla fine del terzo round, quando puoi scagliare la freccia (o dardo o pugnale..), oppure, tirato l'iniziativa, nel round 4 con un bonus di +3 al colpire.

\subsubsection{Combattimento con due armi}\index{Combattimento con due armi} ll combattimento con due armi è possibile solo se l'arma secondaria è leggera o si usa un arma doppia.

Non serve usare Azioni per attaccare con l'arma secondaria, sull'arma secondaria non applichi il danno dato dalla Forza o Destrezza.

I Tiri per Colpire di entrambe le armi hanno un -4.

Con l'arma secondaria se non si hanno Abilità specifiche si fa solo 1 attacco.

E' possibile usare l'arma secondaria come scudo concedendo un +1 alla Difesa.

\begin{center}
	\includegraphics[width=0.9\linewidth]{immagini/twoweapon.jpg}
\end{center}


\subsubsection{Usare un'arma da lancio sotto minaccia} \index{Usare un'arma da lancio sotto minaccia}

Usare un'arma da lancio come arco, balestra o pugnale (che si vuole lanciare) mentre si combatte in mischia impone la negazione del bonus della Destrezza alla Difesa ed il Tiro per Colpire ha un -4.

\subsubsection{Usare un'arma da lancio mirando ad un avversario impegnato
in combattimento}\index{Usare un'arma da lancio mirando ad un avversario impegnato in combattimento}

Non è facile prendere la mira corretta e non prendere il proprio compagno.
Hai un -1d6 al Tiro per Colpire. Il bonus si annulla se c'è una differenza di 2 o più taglie tra avversario e compagno.

\subsubsection{Usare un'arma con due mani} \index{Usare un'arma con due mani}

Un'arma non leggera se usata a due mani permette di applicare una volta e mezza il danno dovuto dalla Forza.

Questo bonus non si applica se l'arma è troppo grande e si ha la penalità al colpire.

\subsubsection{Difesa totale} \index{Difesa totale}

Costa 2 Azioni. Non puoi eseguire nessun attacco o lancio di incantesimo, puoi fare solo una Azione e guadagni un +8 in Difesa. Non causi Attacchi di Opportunità se attraversi la zona di mischia di un avversario.

\subsubsection{Disingaggiare} \index{Disingaggiare}

Costa 2 Azioni e ti sposti fino a 3 metri. Un'avversario ti può colpire se ha una iniziativa migliore della tua o ti insegue (ed è veloce quanto te). Non causi Attacchi di Opportunità se attraversi la zona di mischia di un avversario. 

Può essere usata per uscire dalla mischia e poi muoversi od effettuare un attacco utilizzando l'Azione rimasta.

Se per disingaggiare l'avversario è necessario spostarsi di più di 3 metri, a causa della grande distanza di mischia dell'avversario, è necessario usare anche un'Azione di Movimento.

\subsubsection{Mettersi sulla difensiva} \index{Mettersi sulla difensiva}

Prendi un bonus di +4 alla Difesa, il tuo Tiro per Colpire ha una penalità di -1d6

\begin{center}
	\includegraphics[width=0.9\linewidth]{immagini/teseo.png}
\textit{Henry Justice Ford - Colpo alle spalle!}
\end{center}


\subsubsection{Rete e Bolas}

Una creatura Grande o inferiore colpita da una rete resta intralciata finché non si libera.

Una rete non ha effetto sulle creature prive di forma, o le creature Enormi o superiori.

Una rete può catturare fino a 3 creature piccole o 2 creature medie (o 1 media e 2 piccole) o 1 creatura grande che siano in prossimità l'una dell'altra.

Il tuo Tiro per Colpire si confronta con ogni Difesa della creature intercettate, se si colpisce la creatura è sotto alla rete, altrimenti è riuscita ad evitare l'intralcio.

Una creatura può impiegare due Azioni per tentare una prova di Forza con DC 12 per liberarsi o liberare un'alta creatura entro la sua portata con successo.

Anche infliggere 5 danni taglienti alla rete (Difesa 11) libera la creatura senza danneggiarla, terminando l'effetto e distruggendo la rete.

Quando attacchi con una Rete puoi compiere un solo attacco, non importa quale sia il numero di attacchi che puoi normalmente effettuare.

Per le Bolas valgono le stesse considerazioni che la Rete, la differenza principale è che le Bolas hanno una gittata maggiore e possono colpire un solo bersaglio.


\subsection{Azioni Opzionali in Combattimento}

Queste Azioni di combattimento sono a discrezione del Narratore che può concederle o meno.

\medskip

\subsubsection{Disarmare*}\index{Disarmare} 

Fai una prova contrapposta Competenza Armi + Destrezza (chi disarma) contro Competenza Armi + Forza (chi viene disarmato).

Un'arma a due mani concede un bonus di +4, un'arma leggera un malus di -2 a chi deve essere disarmato. Se si fallisce di 5 o più hai disarmato te stesso e non l'avversario. Costa 1 Azione.

\begin{center}
	\includegraphics[width=0.9\linewidth]{immagini/alfieri37.jpg}
\end{center}


\subsubsection{Finta*} \index{Finta}

Fai una prova contrapposta di Competenza Armi + Ingannare (chi fa la finta) contro Competenza Armi + Consapevolezza (chi subisce la finta). Se la prova riesce l'avversario perde il bonus della Destrezza alla Difesa fino alla fine del round successivo.

Se fallisci di 5 o più perdi tu il round prossimo il bonus di Destrezza. Costa 1 Azione.

\subsubsection{Spingere un avversario*} \index{Spingere un avversario}

E' una prova contrapposta di Forza (TS Tempra contrapposto).

Se vinci spingi l'avversario fino a 0.5 metri nella direzioni che vuoi per successo nella prova (fino al massimo del tuo movimento), altrimenti l'avversario ti spinge nella direzione che vuole fino a 0.5 metri per successo ottenuto (es. se vinci la prova di 7 sposti l'avversario fino a 3.5 metri). Costa due azioni.

\subsubsection{Afferrare un avversario*}\index{Afferrare un avversario}

E' una prova contrapposta di Forza (TS Tempra contrapposto ma con bonus di Forza e non Costituzione). Chi ha una taglia maggiore guadagna un bonus di +2 per taglia di differenza.

Costa 2 Azioni fare e mantenere e liberarsi dalla presa. Si considera che chi afferra (o è afferrato) abbia almeno una mano occupata nell'afferrare.

I due contendenti perdono il bonus di Destrezza alla Difesa.

Finché l'avversario è afferrato il terreno si considera difficile.

Puoi attaccare l'avversario afferrato con un arma corta.

L'avversario afferrato può usare 1 azione per cercare di liberarsi. E' sempre un Tiro Salvezza su Tempra su Forza contrapposto.

\subsubsection{Fare cadere un avversario*} \index{Fare cadere un avversari}

E' una prova contrapposta di Forza o Destrezza, ogni contendente sceglie quella che preferisce.

Ognuno fa un Tiro Salvezza su Tempra (con modificatore Forza) o Riflessi (con modificatore Destrezza)e si confrontano i risultati, se la prova è inferiore, rispetto all'avversario, di 5 o più è chi voleva fare cadere che cade.

Per ogni gamba/zampa che ha più di te l'avversario questo ha un bonus di +2 alla prova. 

Costa 2 Azioni. L'avversario se fallisce la prova diventa prono.

\subsubsection{Modificare le proprie dimensioni*}\index{Modificare le proprie dimensioni}

Nel caso il personaggio \index{Modificare le dimensioni} la sua Difesa cambia di conseguenza

\bigskip

\begin{tabular}{ll}
	\textbf{Nuova Taglia} & \textbf{Modificatore alla Difesa}\\
	\toprule
	Piccolissima          & +8\\
	Minuta                & +4\\
	Minuscola             & +2\\
	Piccola               & +1\\
	Media                 & +0\\
	Grande                & -1\\
	Enorme                & -2\\
	Mastodontica          & -4\\
	Colossale             & -8\\
\end{tabular}

\bigskip

\subsection{Cavalcature}\index{Combattimento a cavallo}\index{Cavallo}

\begin{enfasi}{
		- E ti puoi trovare un'altra moglie!
		
		- Ah, questo sì. ma il guaio è che mi ha portato via il fucile e il cavallo! Peccato, era così bella, io mi ci ero affezionato. Le davo qualche frustata, ma lei non ci faceva caso.
		
		- Chi, tua moglie?
		
		- No, la mia cavalla. A trovare un'altra moglie si fa presto, ma una cavalla come quella non la ritrovo più. (Ombre rosse, film 1939)}\end{enfasi}\medskip


Anche una cavalcatura ha le sue 3 Azioni e di norma sono usate per spostarsi o per reagire ed ubbidire ai tuoi comandi.

Una cavalcatura agisce nel tuo round e sei tu a decidere quando esegue le sue Azioni rispetto alle tue. Non tira l'iniziativa, usa la tua.

Per spostare una cavalcatura dove vuoi del suo movimento devi usare una tua Azione, come per farlo agire.

Gli attacchi verso un personaggio a cavallo (o cavalcatura in genere) se non dichiarati diversamente mirano al cavaliere e non al cavallo.


\begin{center}
	\includegraphics[width=0.9\linewidth]{immagini/cavallo2.png}
\end{center}


\subsubsection{Situazioni e regole}

\begin{itemize}
	\item
	Ogni qual volta la cavalcatura è colpita il cavaliere deve effettuare una prova di Cavalcare a DC 15 o essere disarcionato dalla cavalcatura.
	
	Se la cavalcatura è da "guerra" (addestrata al combattimento) la prova ha difficoltà 12.
	
	\item
	Combattere da posizione sopraelevata concede un +1d6 al Tiro per Colpire se l'avversario è a piedi ( o non è alla tua altezza...).
	
	\item
	Salire o Scendere dalla cavalcatura costa 1 Azione
	
	\item
	Se una magia o situazione sposta (bruscamente) la cavalcatura contro la tua volontà devi effettuare un Tiro Salvezza su Riflessi a DC 13 o venire disarcionato
\end{itemize}

\subsubsection{Essere disarcionato}

Se vieni disarcionato una prova di Acrobatica a DC 15, serve una Azione di Reazione, eviterà di cadere prono. Se la prova fallisce di 5 o più subisci 1d6 di danno per la caduta.


\subsubsection{Controllare una Cavalcatura}

Mentre sei in sella, hai due scelte:

\flushleft{- puoi dare ordini alla tua cavalcatura}

- permettergli di agire da sola.

\medskip

Cavalcature particolarmente intelligenti tendono a privilegiare l'autonomia di azione piuttosto che essere comandati.

Puoi controllare una cavalcatura solo se questa è stata addestrata ad accettare un cavaliere. Si presume che cavalli addestrati, muli e simili creature abbiano ricevuto tale addestramento.

L'iniziativa di una cavalcatura controllata cambia per corrispondere a quella di chi la cavalca. Si muove secondo le tue indicazioni e ha solo tre opzioni di azione: Muoversi, Attaccare, Disingaggiare.
Ogni bonus e malus riportato da queste 3 Azioni vale solo per la cavalcatura.

Fare eseguire un particolare ordine ad una cavalcatura costa 1 Azione al cavaliere.

Se la cavalcatura è intelligente avere un cavaliere non restringe le azioni che la cavalcatura può effettuare e questa si muove e agisce come desidera. Potrebbe fuggire dal combattimento, lanciarsi all'attacco e divorare un nemico ferito gravemente, o agire in qualche altro modo contro la tua volontà.

In entrambi i casi, se la cavalcatura provoca un attacco di opportunità mentre le sei in sella, l'attaccante può mirare a te o alla cavalcatura.

\end{multicols}

\bigskip

\begin{center}
	\includegraphics[width=0.55\linewidth]{immagini/napoleone.png}
	
	\textit{Jacques-Louis David, Bonaparte valica il Gran San Bernardo, 1801, Castello di Malmaison}
\end{center}


\pagebreak

\section{Nascondigli e coperture}\index{Nascondigli}\index{Copertura}

\begin{enfasi}{Agiamo nell'ombra per servire la luce. (Assassin's Creed II)
		
\medskip		
		
Dove c'è molta luce, l'ombra è più nera. (Johann Wolfgang von Goethe)}
\end{enfasi}\medskip

\begin{multicols}{2}

\label{nascondigli-e-coperture}
\lettrine[lines=2, lhang=0.33, loversize=0.25, findent=1.5em]{N}{on} sempre l'avversario si palesa davanti a noi, spesso questo può essere nascosto se non addirittura invisibile.

Potrebbe essere nascosto dietro un muretto o dei barili, se non dietro un grosso e gigantesco famiglio.
E se fosse alle nostre spalle e neanche l'abbiamo visto ?

\subsection{La Copertura}\index{Copertura}

Se l'obiettivo è noto che ci sia me è occultato in qualche maniera allora si dice che ha "copertura".

\begin{itemize}
\item
Se l'obiettivo ha più della metà (ma non totale) della superficie "visibile" allora la copertura si definisce \textbf{leggera}, ovvero ha +2 alla Difesa.

Può essere il caso di un arciere in piedi dietro un muretto di 1 metro.

\item
Se l'obiettivo ha meno della metà (ma non completamente) della superficie "visibile" allora la copertura si definisce \textbf{media}, ovvero ha +4 alla Difesa.

Può essere il caso di un nemico armato di balestra che si sporge quel tanto per tenere appoggiata la balestra al muretto e sparare.

\item
Se l'obiettivo si sa dove è ma si nasconde completamente affacciandosi solo per controllare i personaggi o tirare una freccia ogni tanto, dietro ad un muro, finestra, porta, tavolo, una creatura più grande di lui (almeno 2 taglie).. allora la copertura si definisce \textbf{completa}, ovvero ha +8 alla Difesa.

Chiaramente un avversario che non si mostra non puo' essere colpito normalmente...

\end{itemize}

\begin{center}
	\includegraphics[width=0.9\linewidth]{immagini/hide.jpg}
\textit{British Soldiers Hiding From Boer Fire At The Battle Of Majuba Hill.}
\end{center}

Metà del Bonus di copertura si applica anche ai Tiri Salvezza contro Incantesimi che abbiano un effetto ad area (es. Palle di Fuoco che esplodano intorno..).

\subsection{Invisibilità}\index{Invisibilità} \hypertarget{invisibilita}{}

Se un avversario è invisibile o non si sa dove è si seguono le regole della Invisibilità.

\begin{center}
	\includegraphics[width=0.9\linewidth]{immagini/brickwall.png}
	
	\textit{C'e' qualcuno davanti a questo muro ?}
\end{center}

\label{invisibilita}

Anche se si è invisibili non è detto che non si possa essere percepiti diversamente attraverso altri sensi, come l'olfatto, l'udito o il tatto.

L'invisibilità rende una creatura non individuabile tramite la vista ma non rende di per sé una creatura impercettibile o immune ai Tiri Critici o Esplosioni del Danno.

Una creatura accecata, o che combatte contro una creatura invisibile, può effettuare una prova di Consapevolezza, 2 Azioni, a difficoltà 20 (oppure 10 + prova di Nascondersi/Muoversi Silenziosamente, dell'avversario se questo di nasconde attivamente) per individuare la creatura purché questa sia entro un raggio di 3 metri dal personaggio.

Se riesce l'osservatore ha la sensazione che "ci sia qualcosa" ma non può vederlo o prenderlo di mira in modo accurato con un attacco.

E' praticamente impossibile (DC 30) individuare la posizione esatta (il quadretto della mappa) di una creatura invisibile con una prova di Consapevolezza.

Chi attacca una creatura per lei invisibile ha un -2d6 al Tiro per Colpire, la creatura invisibile che attacca una creatura che non la vede ha +1d6 al Tiro per Colpire. 

\end{multicols}

\vspace{2cm}

Ci sono molti modificatori che possono essere applicati alla DC per individuare la creatura, ad esempio se la creatura invisibile si sta muovendo o sta compiendo un'attività rumorosa.


\medskip

\textbf{Tabella Modificatori Consapevolezza per Rilevare Creature Invisibili}\index{Tabella Modificatori Consapevolezza per Rilevare Creature Invisibili}

\medskip

\begin{tabularx}{0.9\textwidth}{Xl}
	\textbf{La Creatura Invisibile sta...} & \textbf{Consapevolezza}\\
	\toprule
	Muovendosi a velocità dimezzata        & +5\\
	Muovendosi a piena velocità            & +10\\
	Correndo o caricando                   & +20\\
	Usando Muoversi Silenziosamente        & Vs Consapevolezza -10\\
	Ferma                                  & -20\\
	A qualche metro di distanza (3 metri)  & -1, -2 per ogni 3 metri\\
	Dietro un ostacolo (porta)             & -5\\
	Dietro un ostacolo (parete di pietra)  & -20\\
\end{tabularx}

\bigskip

\begin{multicols}{2}

Se un personaggio invisibile raccoglie un oggetto visibile, l'oggetto resta visibile. Una creatura invisibile può raccogliere un piccolo oggetto visibile e nasconderselo addosso (mettendolo in una tasca o sotto il mantello, chiudendolo nel pugno) e renderlo effettivamente invisibile.

Uno potrebbe spargere su un oggetto invisibile della farina per tenere traccia almeno della sua posizione (finché la farina non cade del tutto o viene soffiata via).

Le creature invisibili lasciano impronte. Le loro tracce possono essere seguite senza problemi. Impronte su sabbia, fango o altre superfici soffici possono dare ai nemici indicazioni sulla posizione della creatura invisibile, riducendo il loro bonus di Difesa a +4 (come se fossero stati individuati).

Una creatura invisibile nell'acqua muove il liquido, rivelando la propria posizione. La creatura invisibile rimane comunque difficile da vedere e gode dei benefici di una copertura leggera (+2 alla Difesa).

Una torcia accesa invisibile emana comunque luce (così come un oggetto invisibile soggetto ad una magia di luce).

Le creature invisibili non possono utilizzare gli attacchi con lo sguardo. L'invisibilità non influisce sull'essere obiettivo di un incantesimo di Divinazione.

\end{multicols}

\pagebreak

\section{Lista Armi per Tipologia Omogenea}\index{Lista Armi}\index{Tipologia Omogenea}\hypertarget{lista.armi}{} 

\label{lista-armi-per-tipologia-omogenea}
\begin{enfasi}{La forza non risiede in una Spada, ma nelle braccia di un valoroso. (The Legend of Zelda: Twilight Princess)}\end{enfasi}\medskip

\begin{multicols}{2}

\lettrine[lines=2, lhang=0.33, loversize=0.25, findent=1.5em]{O}{gni} qual volta si assegna un punto a Competenza Armi si può decidere se continuare a perfezionarsi in una Lista di Armi già nota o apprendere una nuova, se non si dichiara l'uso questo è assegnato alla Lista delle Armi Semplici.

Nella scheda segnatevi a quale Lista d'Armi assegnate il punto di Competenza Armi.

Per riassegnare un punto di CA sono necessari almeno 4 ore di allenamento per 4 mesi.

Ricordo che usare un arma senza l'adeguata competenza impone un -1d6 al Tiro per Colpire.

I punti assegnati in una Lista d'Arma non si sommano al Tiro per Colpire, bisogna verificare il punteggio nella lista d'arma con gli eventuali bonus che la stessa lista elenca.

\subsection{Armi Leggere}\index{Armi Leggere} Spada Corta, Mazza leggera, Stocco, Scimitarra

Puoi usare Destrezza al posto di Forza con queste armi per il Tiro per Colpire ed il Danno.

\begin{itemize}
	\item 4 punti: Aumenti di un grado il dado di danno dell'arma (d4 - d6 - d8 - d10 - 2d6 - 2d8 - 2d10 - 3d6..)

	\item 8 punti: +2 Tiro per Colpire

	\item 12 punti: la tua arma acquista EDX anche con 6 di danno massimo
	
	\item 18 punti: quando effettui un Tiro per Colpire consideri anche i 5 per i Critici e l'Esplosione del Dado.
\end{itemize}

\subsection{Asce}\index{Asce} Ascia ad una mano, Ascia da battaglia, Ascia Martello, Grande Ascia Doppia

\begin{itemize}

	\item 4 punti: La furia dei tuoi attacchi è tale che guadagni un +2 al danno

	\item 8 punti: Le ferite che provochi sono cosi profonde che causi sanguinamento. Il primo attacco del round se andato a segno causa 1 punto di danno extra da sanguinamento. Il danno si applica anche il round successivo.

	\item 12 punti: Le ferite che provochi sono cosi profonde che causi sanguinamento. Il primo attacco del round se andato a segno causa 2 punto di danno extra da sanguinamento. Il danno si applica finché  la ferita non è medicata (non e' cumulativo con il vantaggio precedente).
	
	\item 18 punti:  quando effettui un Tiro per Colpire consideri anche i 5 per i Critici e l'Esplosione del Dado.	
		

\end{itemize}

\subsection{Rompi Cranio} \index{Rompi Cranio}Randello, Mazza Pesante, Morningstar, Flagello, Martello da guerra, Grosso randello, Flagello Pesante

\begin{center}
	\includegraphics[width=0.9\linewidth]{immagini/arma-mazza3.jpg}
\end{center}


\begin{itemize}
	\item 4 punti: Sei diventato cosi abile che puoi controllare la forza dei tuo colpi, puoi fare danno non letale senza malus al colpire (altrimenti -1d6 al Tiro per Colpire). 
	
	Puoi scegliere di ridurre di 4 il Tiro per Colpire per aumentare il danno di 4 (non cumulabile con Colpi Potenti).

	\item 8 punti: I tuo colpi frastornano il nemico. Se fai un critico e colpisci con il Tiro per Colpire l'avversario deve fare un Tiro Salvezza Tempra (DC pari al tuo Tiro per Colpire) se fallisce subirà -2 Iniziativa ed -2 Difesa fino alla fine del prossimo round.

	\item 12 punti: Aumenti di un grado il dado di danno dell'arma (d4 - d6 - d8 - d10 - 2d6 -2d8 - 2d10)

	\item 18 punti: quando effettui un Tiro per Colpire consideri anche i 5 per i Critici e l'Esplosione del Dado.
		
\end{itemize}

\subsection{Archi} \index{Archi}Fionda, Arco Lungo, Arco Corto, Arco Lungo Composito, Arco Corto Composito

\begin{center}
	\includegraphics[width=0.9\linewidth]{immagini/arma-arco.jpg}
\end{center}


\begin{itemize}

	\item 4 punti: Aggiungi il valore di Forza al danno, anche se l'arco non è composito. 

	\item 8 punti: La tua maestria nell'utilizzo dell'arco in combattimento è tale che non subisci nessuna penalità nel lanciare frecce a nemici in mischia o con copertura pari o minore di leggera.

	\item 12 punti: Scagli una freccia in più con un malus di -5 al Tiro per Colpire (TC, TC-5. TC-5, TC -10...)
	
	\item 18 punti: quando effettui un Tiro per Colpire consideri anche i 5 per i Critici e l'Esplosione del Dado.	

\end{itemize}

\subsection{Balestre}\index{Balestre}Balestra leggera, Balestra pesante, Balestra ad una mano

\begin{center}
	\includegraphics[width=0.9\linewidth]{immagini/arma-balestra.png}
\end{center}


\begin{itemize}

	\item 4 punti: Guadagni l'Abilità Ricarica rapida.

	\item 8 punti: La tua maestria nell'utilizzo delle balestre in combattimento è tale che non subisci nessuna penalità nel lanciare frecce a nemici in mischia o con copertura pari o minore di leggera.

	\item 12 punti: La tua mira e freddezza sono leggendari. Puoi decidere di prendere la mira (2 Azioni) su un nemico per un round, se il round immediatamente successivo lo colpisci il tuo quadrello farà il tre volte il danno dell'arma. Si applica solo al primo colpo del round.
	
	\item 18 punti: quando effettui un Tiro per Colpire consideri anche i 5 per i Critici e l'Esplosione del Dado.	

\end{itemize}

\subsection{Armi doppie} \index{Armi doppie} Grande Ascia Doppia, Flagello Doppio, Spada a due lame, Urgrosh

\begin{itemize}
	\item 4 punti: La tua competenza nell'uso di queste armi ti rende estremamente versatile dandoti la possibilità a inizio del tuo round di scegliere se essere difensivo o offensivo aumentando di 2 il Tiro per Colpire o la Difesa. Non costa Azioni.

	\item 8 punti: La tua tecnica è imprevedibile per l'avversario puoi scegliere se avere un +1d6 di danno con tutti i tuoi attacchi o +4 alla Difesa.

	\item 12 punti: La tua maestria è tale che l'avversario vede 3 armi. Puoi eseguire due attacchi con la mano secondaria senza penalita'. Non cumulabile con Abilità Combattimento a due Armi.
	
	\item 18 punti: quando effettui un Tiro per Colpire consideri anche i 5 per i Critici e l'Esplosione del Dado.	
\end{itemize}

\begin{center}
	\includegraphics[width=0.8\linewidth]{immagini/arma-mazzafrusto-mazza8.jpg}
\end{center}

\subsection{Palle rotanti} Flagello, Flagello Pesante, Flagello Doppio, Catena chiodata, Frusta

\begin{itemize}
	\item 4 punti: Dopo un'Azione di Attacco (1 o due Azioni) puoi effettuare un ulteriore attacco (consumando 1 Azione) a -5 sul Tiro per Colpire su un altro avversario in mischia (non può essere lo stesso bersaglio degli altri colpi)

	\item 8 punti: L'impatto dei tuo colpi è tale da stordire i nemici. Se effettui almeno un critico nel Tiro per Colpire l'avversario deve fare un Tiro Salvezza su Tempra (DC pari al tuo Tiro per Colpire), in caso di fallimento subisce -1 Destrezza per 1 minuto. Una creatura non può essere influenzata da questo effetto più di due volte

	\item 12 punti: La precisione ed abilità nel roteare la tua arma è tale da confondere la difesa del nemico, ignori la protezione data dallo scudo.
	
	\item 18 punti: quando effettui un Tiro per Colpire consideri anche i 5 per i Critici e l'Esplosione del Dado.		
\end{itemize}

\subsection{Armi aggraziate}\index{Armi aggraziate} Stocco, Scimitarra, Falcione

puoi decidere di usare l'Destrezza per determinare il bonus al colpire ed al danno.

\begin{center}
	\includegraphics[width=0.8\linewidth]{immagini/sciabole.png}
\end{center}


\begin{itemize}
	\item 4 punti: Puoi eseguire un critico, anche su creature normalmente immuni ai critici

	\item 8 punti: Il tuo stile assomiglia molto ad una danza. Puoi usare il valore del Carisma al Tiro per Colpire e Danno al posto della Forza o Destrezza.
	
	\item 12 punti: Per ogni -1 al danno che prendi la tua iniziativa aumenta di 2, fino ad un massimo di +6. E' una Azione di Reazione che costa 0 Azioni da dichiararsi ogni round prima della risoluzioni delle azioni.
	
	\item 18 punti: quando effettui un Tiro per Colpire consideri anche i 5 per i Critici e l'Esplosione del Dado.			
\end{itemize}

\subsection{Armi della morte}\index{Armi della morte} Picca Leggera, Picca Pesante, Falce, Falcetto

\begin{itemize}
	\item 4 punti: Puoi eseguire un Colpo di Grazia con il costo di 1 Azione

	\item 8 punti: Aumenti di un grado il dado di danno (d4 - d6 - d8 - d10 - 2d6 - 2d8 - 2d10 - 3d6...)

	\item 12 punti: Aumenti di un grado il dado di danno (d6 - d8 - d10 - 2d6 - 2d8 - 2d10 - 3d6...)
	
	\item 18 punti: quando effettui un Tiro per Colpire consideri anche i 5 per i Critici e l'Esplosione del Dado.			

\end{itemize}

\subsection{Armi da stordimento}\index{Armi da stordimento} Pugno Vuoto, Manganello, Guanto chiodato

\begin{itemize}
	\item 4 punti: Un avversario inconsapevole se colpito con queste armi (durante il round di sorpresa) deve dare un Tiro Salvezza Tempra con DC pari al danno (del round causato da Armi da Stordimento) o rimanere stordito per 1d6 round.

	\item 8 punti: Raddoppi il tuo bonus di danno dato dalla Forza.

	\item 12 punti: La tua arma da stordimento fa 1d6 di danno non letale in piu'.
	
	\item 18 punti: quando effettui un Tiro per Colpire consideri anche i 5 per i Critici e l'Esplosione del Dado.				
\end{itemize}

\subsection{Lance} \index{Lance}Alabarda, Urgrosh, Lancia da fante, Naginata, Falcione in asta, Lancia, Brandistocco, Tridente.

\begin{center}
	\includegraphics[width=0.8\linewidth]{immagini/arma-asta.png}
	
\textit{1 Spiedo dei lanzichenecchi; 2 Picca; 3 Lancia; 4 Spiedo da caccia; 6 Buttafuoco; 7 Falcione; 8 Partigiana; 9 Alabarda; 10 Alabarda; 11 Roncone; 12 Mazzapicchio}\end{center}


\begin{itemize}
	\item 4 punti: Puoi usarla anche contro avversari a distanza di 1 metro senza malus.

	\item 8 punti: Usata contro una carica fai il quadruplo del danno.

	\item 12 punti: Se non sei in mischia con un avversario puoi usare la tecnica della Colpo Perforante (questa azione richiede tutte le 3 Azioni). Carichi, puoi sacrificare 1 punto CA e guadagnare 4 al danno (massimo 10 CA/40 danno) poi esegui un attacco solo col arma. Questo colpo ti porta in mischia con l'avversario e ti lascia scoperto per quel round, hai un -4 alla Difesa.
	
	\item 18 punti: quando effettui un Tiro per Colpire consideri anche i 5 per i Critici e l'Esplosione del Dado.			

\end{itemize}

\subsection{Armi letali} Pugnale, Machete\index{Armi letali}

\begin{itemize}

	\item 4 punti: Contro avversari sorpresi aggiungi al danno il tuo CA

     \item 8 punti: La tua arma fa più danno. Guadagni una categoria di danno (d4 - d6 - d8..). Se questo porta l'arma ad avere il d8 come dado di danno acquisisce anche EDX pari a 8.

     \item 12 punti: Guadagni EDX. Lo si applica solo facendo il danno massimo con il dado, se l'arma ha già un EDX (perché con il bonus precedente è arrivata ad 1d8 di danno) questo diminuisce di 1.
     
	\item 18 punti: quando effettui un Tiro per Colpire consideri anche i 5 per i Critici e l'Esplosione del Dado.			     

\end{itemize}

\subsection{Aste} \index{Aste}Giavellotto, Lancia corta da fante, Tridente, Alabarda

\begin{itemize}

	\item 4 punti: Se fai almeno un critico con il Tiro per Colpire puoi lasciare l'arma nel corpo dell'avversario, penalizzandolo con un -1 Destrezza. L'arma quando rimossa fa il suo dado di danno (senza Forza o bonus magici).

	\item 8 punti: Puoi usare l'arma lunga in mischia entro un metro senza penalità.

	\item 12 punti: Raddoppi la gittata.
	
	\item 18 punti: quando effettui un Tiro per Colpire consideri anche i 5 per i Critici e l'Esplosione del Dado.
		
\end{itemize}

\subsection{Spade}\index{Spade} Spada Corta, Spada Lunga, Spadone a due mani, Spada bastarda, Spada a due lame, Katana

\begin{center}
	\includegraphics[width=0.9\linewidth]{immagini/arma-tipi-di-spade.png}
	
\textit{A Sciabola, B Scimitarra, C Spada ad una mano, D Spada larga, D Stocco, E Spada lunga, F Spada a una mano e mezza o bastarda, G Spadone a due mani}
\end{center}


\begin{itemize}

	\item 4 punti: La tua maestria nella tecnica della spada ti conferisce +1 a danno e Tiro per Colpire.

	\item 8 punti: La tua maestria nella tecnica della spada ti conferisce +1 a Difesa e Tiro per Colpire.

	\item 12 punti: Hai raggiunto l'apice della maestria con la spada i tuo colpi sono precisi e difficili da prevedere ottieni +1 a danno, Tiro per Colpire e Difesa.
	
	\item 18 punti: quando effettui un Tiro per Colpire consideri anche i 5 per i Critici e l'Esplosione del Dado.			     	
\end{itemize}

\subsection{Scudi}\index{Scudi} Scudi Leggeri, Medi, Pesanti

Sei un maestro nell'uso degli scudi, anche come arma.


\begin{center}
	\includegraphics[width=0.9\linewidth]{immagini/scudotorre.png}
\textit{Henry Justice Ford. Scudo Pesante}
\end{center}


Puoi usare lo scudo come arma, uno scudo piccolo fa 1d4 di danno (B/T), uno scudo medio fa 1d6 di danno (B/T), uno scudo pesante fa 1d8 di danno (B/T).
Non hai penalità al colpire con lo scudo, per te lo scudo non è un arma improvvisata.


\begin{itemize}
	\item 1 punto: Sei competente in tutte le tipologie di scudo. Non hai il vincolo del limite di Forza 1 sugli Scudi Pesanti.

	\item 2 punti: Il bonus di Difesa aumenta di 1 e ogni 4 volte che prendi la competenza.
	 Usare lo scudo come arma non ti fa perdere il bonus alla Difesa dato dallo scudo.
	
	\item 3 punti: La penalità Competenza Magica diminuisce di 1 e di 1 ogni 4 volte che prendi la competenza.

	\item 4 punti: La penalità alla Competenza Armi diminuisce di 1 e di 1 ogni 4 volte che prendi la competenza.
	
	La tua tecnica mescola efficacemente difesa e attacco. Il tuo scudo fornisce un bonus aggiuntivo di +1 alla Difesa. Puoi lanciare il tuo scudo con una gittata di 6 metri.
	
	\item 5 punti: Aumenta di 1 la categoria di danno dello scudo (1d4 - 1d6 - 1d8 - 1d10 - 2d6 - 2d8 - 2d10) ed ogni 4 punti ulteriori in lista (9,13,17..).

	\item 8 punti: Ogni alleato adiacente (entro 1 metro) a te ha un +1 Difesa. Se desideri puoi subire il danno di un attacco diretto ad un alleato entro 1 metro. Usare questa Abilità è una Reazione che non costa Azioni. Puoi lanciare il tuo scudo con una gittata di 9 metri.
	
	\item 12 punti: Puoi lanciare il tuo scudo come fosse un arma con gittata 12 metri. Se ottieni un critico lo scudo una volta lanciato torna nelle tue mani a fine round.
	
	\item 18 punti: lo scudo lanciato ha una gittata di 18 metri e torna subito e sempre nelle tue mani. Questo ti permette di effettuare attacchi multipli anche da lancio.
	
	Non è possibile applicare questi bonus se si usa più di uno scudo.

\end{itemize}

\subsection{Bloccanti} Bolas, Net\index{Bloccanti}

Una creature avvolta dalla tua rete o bolas è intralciato e non può muoversi.

\begin{itemize}

\item 4 punti: Raddoppi la gittata

\item 8 punti: Usi una rete più grande. Adesso puoi bloccare fino a 4 creature piccole, 3 creature medie, 1 creatura grande. La DC della prova di Forza per liberarsi sale a 14.

\item 12 punti: La tua rete avvolge e ferisce. La DC della prova di Forza per liberarsi sale a 15. Ogni round di permanenza nella rete causa 2 Punti Ferita di danno da taglio.

\item 18 punti: Raddoppi la gittata.

Una creatura dotata della Competenza Artista della Fuga può fare una prova per liberarsi con le difficoltà indicate.

\end{itemize}

\subsection{Armi da tiro} Lancia corta da fante, Ascia ad una mano, Giavellotto, Tridente\index{Armi da tiro}

Acquisisci la capacità del \textbf{Tiro Devastante}: puoi lanciare una delle tue armi con tale forza da triplicarne il danno (solo dado dell'arma) ma la precisione ne risente -8 al Tiro per Colpire. Costa 2 Azioni.

\begin{itemize}
	\item 4 punti: Se diventato estremamente preciso nel lancio della tua arma hai un +2 al colpire e un +1 ai danni

	\item 8 punti: La tua abilità ti permette di non avere tempi morti dopo il lancio di un arma puoi istantaneamente estrarne un altra senza consumare azioni.

	\item 12 punti: Raddoppi la Gittata dell'arma
	
	\item 18 punti: quando effettui un Tiro per Colpire consideri anche i 5 per i Critici e l'Esplosione del Dado.
\end{itemize}

\subsection{Pugno Vuoto} Pugni e Calci\index{Pugno Vuoto}\hypertarget{pugnovuoto}{} 

\begin{itemize}
	\item 1 punto: tuoi pugni fanno danno letale (1d4) e diventano Versatili
\end{itemize}

\textbf{Pugno Vuoto}: Ogni volta che prendi questa competenza il danno aumenta seguendo questa progressione: 1d6 (presa la lista 2 volte), 1d8 (3), 2d6 (4), 2d8 (5), 2d10 (6), 3d6 (7), 3d8 (8), 3d10 (9), 4d6 (10)...

Il giocatore può anche decidere di fare danno non letale non incorrendo in alcuna penalità, al danno può applicare a proprio piacere il valore di Forza o Destrezza.

* 6 punti: Saggezza della mano vuota - puoi usare il valore della Saggezza al colpire ed al Danno al posto di Forza o Destrezza

\subsection{Armi Semplici} Pugnale, Mazza Leggera, Randello, Morningstar, Lancia corta da fante, Bastone, Balestra (Leggera), Giavellotto.\index{Armi Semplici}\hypertarget{armi.semplici}{}


Questa suddivisione è sceglibile anche da chi non ha assegnato punti a Competenza Armi.

\subsection{Armi in più Lista d'Armi}\index{Armi in più Liste d'Armi}

Quando un personaggio usa un arma presente in più Liste d'Armi conosciute può applicare per avversario una sola tecnica (una Lista d'Armi) di combattimento. Il personaggio applica i bonus ottenuti da una sola lista e non cumula anche quelli delle altre liste.

Utilizzando 2 Azioni può concentrarsi e passare ad utilizzare i bonus derivanti dall'applicazione di una diversa Lista D'Armi.

\end{multicols}

\pagebreak

\section{Abilita'}\index{Abilità}\hypertarget{abilita}{} 

\label{abilita}
\begin{enfasi}{Il martirio è l'unica maniera per un uomo di diventare famoso se non ha abilità (George Bernard Shaw, The Devil's Disciple)} \end{enfasi}\medskip

\begin{multicols}{2}

\lettrine[lines=2, lhang=0.33, loversize=0.25, findent=1.5em]{L}{e} Abilità sono capacità peculiari, frutto di allenamento o doti particolari. Le Abilità hanno sempre un effetto pratico.

Le Abilità costituiscono una buona parte di ciò che può fare il personaggio, vanno scelte con attenzione e cura. E' scegliendo le Abilità che si stabilisce lo stile e capacità del personaggio, se lo si vole più guerriero o mago o curatore.. o qualsivoglia combinazione e capacità.

\textbf{Al primo livello si prendono due Abilità}. Ogni 2 livelli successivi, quindi al 3, 5, 7, 9..., si prende un'altra Abilità che può essere la stessa già presa oppure una nuova Abilità appresa durante le avventure.

E' possibile che siano indicati dei Prerequisiti sotto il nome dell'Abilità, in questo caso vanno rispettati per prendere l'Abilità in questione.
Eventuali prerequisiti successivi vengono indicati volta per volta.

Non prendete le Abilità in base al potere, forza, combinazione con altre ma perché in linea con la storia del personaggio.
Scegliere un accozzaglia di Abilità solo perché forti non rende un personaggio forte ma sbilanciato, non fate il power-player ad ogni costo.

\medskip

\textbf{Le Abilità devono essere prese in base al percorso evolutivo del personaggio, in base a quanto vissuto ed appreso durante le avventure.}

\medskip

E' possibile cambiare una Abilità scelta, rispettando comunque i requisiti, al 5, 7, 11 e 17 livello, purché ci sia una giustificazione e si sia in accordo con il Narratore.

Le capacità fornite dalle Abilità se non descritto diversamente sono cumulative o se si tratta dello stesso bonus si applica quello maggiore.

\subsection{Tiri Salvezza ed Abilità}

Ogni Abilità, tranne quelle che modificano direttamente i Tiri Salvezza, concede dei bonus ai Tiri Salvezza che si cumulano tra le Abilità, anche quando si prende piu' volte la stessa Abilità.

Quando scegliete una Abilità fate anche caso a quali Tiri Salvezza aumenta!

\subsection{Aggiungere nuove Abilità}

Questo elenco non potrà mai essere esaustivo data la fantasia dei vari giocatori! Cercate però di capire se quello che il giocatore vuole e' una Abilità o Competenza, l'avere una capacità o sapere fare qualcosa di particolare.
Valutate bene i prerequisiti e i vantaggi che concede, cercate sempre di essere bilanciati, piuttosto concedete i vantaggi a scalare, ovvero prendendo più volte l'Abilità.

Ricordatevi anche di segnare i bonus relativi ai Tiri Salvezza. Solitamente una Abilità concreta concede un bonus di +3 divisi tra 2 Tiri Salvezza, una Abilità piu' generica concede un +2 punti tra un solo Tiro Salvezza o due.


\subsection{Ali della Fenice}\index{Ali della Fenice}

\textbf{Requisito}: Lista Pugno Vuoto 4, Destrezza 3

\textbf{Tiri Salvezza}: +2 Riflessi, +1 Tempra

Il tuo stile di combattimento enfatizza i colpi portati da lontano come calci e pugni volanti.

La \textbf{prima volta} che prendi questa Abilità la tua distanza di mischia diventa di 2 metri.

Le \textbf{seconda volta} che prendi questa Abilità, Lista Pugno Vuoto 9, Gru d'Argento 3, Pugno di Ferro 1, la tua distanza di mischia diventa di 3 metri.

Le \textbf{terza volta} che prendi questa Abilità, Lista Pugno Vuoto 12, Gru d'Argento 6, Pugno di Ferro 3, la tua distanza di mischia diventa di 4 metri.


\subsection{Animalia}\index{Animalia}

\textbf{Requisito}: Seguace o Devoto di Efrem oppure Shayalia, Competenza Magica 2.

\textbf{Tiri Salvezza}: +2 Volontà, +1 Tempra

Si acquisisce la capacità di trasformarsi in un animale. Costo 2 Azioni.

I propri incantesimi di cura funzionano anche su Animali o Piante normali e magiche.

La \textbf{prima volta} che si prende questa Abilità ci si può trasformare in animali non magici di taglia piccola o media, per 10 minuti per punteggio in Competenza Magica. Ci si può trasformare 1 sola volta al giorno. Qualsiasi equipaggiamento viene lasciato a terra durante la trasformazione.

La \textbf{seconda volta}, Competenza Magica 6, che si prende questa Abilità si acquisisce la capacità di trasformarsi in animali minuscoli o grandi e ci si può trasformare in tutto 3 volte in al giorno per 20 minuti per punteggio di Competenza Magica in totale. Qualsiasi equipaggiamento viene lasciato a terra durante la trasformazione.

La \textbf{terza volta} che si prende questa Abilità, Competenza Magica 10, si acquisisce la capacità di trasformarsi in animali di taglia minuta o enorme e ci si può trasformare 5 volte al giorno. Il tempo minimo di trasformazione giornaliera è di 16 ore. Qualsiasi equipaggiamento non magico viene lasciato a terra durante la trasformazione.

Quello magico viene assorbito nella trasformazione e continua ad avere effetto se possibile. Armature e scudi anche magici non applicano alcun bonus alla Difesa ma se hanno capacità magiche possono essere usate.

\begin{center}
	\includegraphics[width=0.9\linewidth]{immagini/animalia3.png}
	\textit{Henry Justice Ford}
\end{center}

La \textbf{quarta volta}, Competenza Magica 16, si acquisisce di trasformarsi in animali di taglia piccolissima o mastodontica ed anche magici (sempre nel limite della taglia ed il Grado di Sfida massimo e' pari alla metà del punteggio di Competenza Magica). Il tempo minimo di trasformazione è di 24 ore al giorno, e può trasformarsi quante volte si vuole al giorno.

Tutto l'equipaggiamento viene assorbito nella nuova forma. Quello magico continua ad avere effetto se possibile. Armature e Scudi applicano il bonus magico alla Difesa della creatura ed eventuali capacità magiche possono essere usate.

\subsection{Animaletto / Famiglio}\index{Famiglio}

\textbf{Tiri Salvezza}: +1 Volontà, +1 Tempra

Guadagni un animale naturale. Questo animaletto ha al massimo un numero di dadi vita pari alla tua Saggezza. Puoi insegnare azioni di base al tuo animale e fargli fare dei compiti semplici.

\textbf{Requisito}: Competenza Magica 1, se prendi due volte questa Abilità guadagni un Famiglio (vedi capitolo specifico).

\subsection{Armatura del Devoto}\index{Armatura del Devoto}

\textbf{Requisito}: Tratti in comune 1 (somma dei Tratti in comune con il Patrono)

\textbf{Tiri Salvezza}: +2 Volontà, +1 Riflessi

Il costante allenamento con la tua armatura ti permette di indossare armature leggere senza rischio di sbagliare il lancio di incantesimi.

La \textbf{seconda volta} che si prende questa Abilità, Tratti in comune 6, puoi lanciare incantesimi senza rischio di fallire con armature medie.

La \textbf{quarta volta} che si prende l'Abilità, Tratti in comune con il Patrono 12, puoi portare Armature Pesanti senza penalità al lancio di incantesimi.

\subsection{Armatura della Montagna Incantata}\index{Armatura della Montagna Incantata}

\textbf{Requisito}: Lista armi Pugno Vuoto, Competenza Armi 1, Competenza Magica 1, Saggezza 1

\textbf{Tiri Salvezza}: +2 Tempra, +1 Riflessi

Il constante allenamento nello spirito e corpo di permette di indurire la tua pelle e renderla più difficile da ferire. Per usufruire di questi bonus non devi portare armature.

La \textbf{prima volta} che prendi questa Abilità la tua Difesa è 10 + Destrezza + 1/3 dei punti in Pugno Vuoto.

La \textbf{seconda volta} che prendi questa Abilità puoi aggiungere alla Difesa il tuo valore di Saggezza, nella misura massima di 1 punto (anche se hai Saggezza piu' alta).

La \textbf{terza volta} che prendi questa Abilità puoi sommare l'intero valore della Saggezza alla Difesa.

La \textbf{quarta volta} che prendi questa Abilità, Pugno Vuoto 12, la tua Difesa è 10 + Destrezza + Saggezza + 1/2 dei punti in Pugno Vuoto.

Se sei sorpreso perdi il solo bonus dovuto alla Destrezza, a tocco la tua Difesa non ha penalità.

\subsection{Arciere a cavallo}\index{Arciere a cavallo}

\textbf{Tiri Salvezza}: +1 Riflessi, +1 Tempra

Il malus di tirare frecce da cavallo diminuisce di 2 ogni volta che prendi questa Abilità.

Le penalità standard sono -4 e -6 a seconda che si trotti (movimento x2) o galoppi (movimento x3)

\begin{center}
	\includegraphics[width=0.9\linewidth]{immagini/horsearcher.png}
	\textit{Arcere Assiro}
\end{center}

\subsection{Arma Focalizzata}\index{Arma Focalizzata}

\textbf{Tiri Salvezza}: +1 Riflessi, +1 Tempra

Scegli un arma. Ottieni un +1 a Iniziativa e Tiro per Colpire quando usi questa arma di cui hai competenza.

\subsection{Attacco Turbinante}\index{Attacco turbinante}

\textbf{Requisito}: Competenza Armi 12

\textbf{Tiri Salvezza}: +2 Riflessi, +1 Tempra

Usando 3 Azioni puoi eseguire un singolo attacco (con un malus di 5 al Tiro per Colpire) contro tutti gli avversari in mischia attorno a te.

\subsection{Batteria Magica}\index{Batteria Magica}

\textbf{Requisito}: Competenza Magica 3

\textbf{Tiri Salvezza}: +2 Volontà, +1 Tempra

Hai una particolare connessione con la magia che permane Yeru.

Ogni volta che prendi questa Abilità puoi lanciare un incantesimo in più al giorno.
Il valore di CM deve essere pari al triplo delle volte che prendi quest'Abilità.

\begin{center}
	\includegraphics[width=0.9\linewidth]{immagini/Historia_Mundi_Naturalis.jpg}
	\textit{Woodcut illustration from an edition of Pliny the Elder's Naturalis Historia (1582)}
\end{center}

\subsection{Colpi poderosi}\index{Colpi poderosi}

\textbf{Requisito}: Competenza Armi 1

\textbf{Tiri Salvezza}: +2 Tempra

Il tuo stile enfatizza colpi poderosi.

Guadagni un +1 al danno con una Lista d'Arma.

\subsection{Colpo furtivo (Attacco alle spalle)}\index{Attacco alle spalle}\index{Colpo furtivo}

\textbf{Requisito}: Competenza Armi 3

\textbf{Tiri Salvezza}: +2 Riflessi, +1 Volontà

Quando l'avversario viene attaccato in mischia di sorpresa, il primo attacco andato a segno con arma di mischia, causa un danno da critico aggiuntivo da 2d6.

La \textbf{seconda volta} che si prende questa Abilità, CA 6, il danno aumenta a 4d6.

La \textbf{terza volta} che si prende questa Abilità, CA 10, il danno aumenta a 6d6.

La \textbf{quarta} che si prende questa Abilità, CA 12, il danno aumenta a 8d6.

\subsection{Colpo Indebolente}\index{Colpo Indebolente}

\textbf{Requisito}: Colpo furtivo 6d6, Competenza Armi 12

\textbf{Tiri Salvezza}: +2 Riflessi, +1 Volontà

Colpo Indebolente è una forma avanzata di colpo furtivo. Ogni Colpo Indebolente abbassa Forza o Destrezza (scelta giocatore) di 1 punto temporaneamente.

All'avversario è concesso un Tiro Salvezza Riflessi (DC pari al tuo Tiro per Colpire). Si causa il danno aggiuntivo del Colpo Furtivo o la perdita di punti caratteristica.

\subsection{Colpo Mortale}\index{Colpo Mortale}

\textbf{Requisito}: Competenza Armi 5

\textbf{Tiri Salvezza}: +2 Riflessi, +1 Volontà

Esegui il Tiro per Colpire ma ignori ogni critico ottenuto. Se colpisci, con un malus di 5, tiri una volta in piu' il dado di danno dell'arma e sommi due volte il valore della Forza per il danno.

Colpo mortale è una Azione Immediata dal costo di 0 Azioni da dichiararsi prima del Tiro per Colpire.

\subsection{Colpo Paralizzante}\index{Colpo Paralizzante}

\textbf{Requisito}: Colpo Indebolente, Colpo furtivo 8d6, Competenza Armi 18

\textbf{Tiri Salvezza}: +2 Riflessi, +1 Tempra

L'obiettivo dopo che è stato studiato per 10 round (2 Azioni a round) con il prossimo tuo colpo andato a segno in mischia, entro 6 round dal termine studio, deve effettuare un Tiro Salvezza Tempra con DC pari al danno inflitto o rimanere paralizzato per 3d6 round.

\subsection{Combattere alla Cieca}\index{Combattere alla Cieca}

E' l'abilità di attaccare gli avversari che non sono chiaramente percepibili.

\textbf{Requisito}: Consapevolezza 2

\textbf{Tiri Salvezza}: +2 Riflessi, +1 Volontà

Un avversario con copertura leggera non ottiene bonus alla Difesa, con copertura media ha un +2 alla Difesa, con copertura totale ha un +4 alla Difesa.

Un attaccante invisibile, non ottiene alcun vantaggio al colpire il personaggio in mischia. I bonus dell'attaccante Invisibile si applicano lo stesso solo per gli attacchi da distanza.

Non c'è bisogno di effettuare prove di Acrobatica per muoversi a piena velocità mentre si è Accecati.

La \textbf{seconda volta} che prendi l'Abilità (Consapevolezza a 3), riduci di ulteriori due il bonus alla Difesa da creature coperte o invisibili.

\textit{"Livello Zatoichi"}, la \textbf{terza volta} che prendi l'Abilità (Consapevolezza a 5), una creatura invisibile non ha alcun vantaggio contro di te ne tu hai malus contro di lui.

\subsection{Combattimento con due armi}\index{Combattimento con due armi}\index{Due armi}

\textbf{Requisito}: Destrezza 2, Forza 1, Competenza Armi 2

\textbf{Tiri Salvezza}: +2 Riflessi, +1 Tempra

La \textbf{prima volta} che prendi questa Abilità puoi eseguire un attacco con l'arma secondaria, che deve essere leggera (oppure usare un arma a due lame) e non applichi il danno dato dalla Forza sull'attacco secondario.

Entrambi i Tiri per Colpire hanno un -2 dato dall'attaccare a due mani.

\textbf{Requisito} Destrezza 3, Competenza Armi 12

La \textbf{seconda volta} che prendi questa Abilità puoi fare con l'arma secondaria leggera o arma doppia, fino a 2 attacchi e non applichi il danno dato dalla Forza. L'arma secondaria ha un -2 e -5 ai rispettivi Tiri per Colpire.

\textbf{Requisito} Destrezza 3, Competenza Armi 18

La \textbf{terza volta} puoi usare un arma media come arma secondaria. Applichi il danno dato dalla Forza con la mano secondaria. Non hai malus ai Tiri per Colpire con la mano secondaria.

I Tiri per Colpire dell'arma secondaria si non si considerano attacchi multipli. Il loro Tiro per Colpire viene modificato solo dalle penalità qui indicate.

\subsection{Creare Oggetti Magici}\index{Creare Oggetti Magici}

\textbf{Requisito}: Competenza Magica 6

\textbf{Tiri Salvezza}: +1 Tempra, +1 Volontà

Tramite questa Abilità l'incantatore è in grado di infondere un incantesimo fino a Difficoltà 21 in un oggetto magico.

\subsection{Creare Oggetti Magici Superiori}\index{Creare Oggetti Magici Superiori}

\textbf{Requisito}: Creare Oggetti Magici, Competenza Magica 12

\textbf{Tiri Salvezza}: +1 Tempra, +1 Volontà

Tramite questa Abilità l'incantatore è in grado di infondere un'incantesimo fino a Difficoltà 26 in un oggetto magico. 

\subsection{Creare Oggetti Magici Meravigliosi}\index{Creare Oggetti Magici Meravigliosi}

\textbf{Requisito}: Creare Oggetti Magici Superiori, Competenza Magica 16

\textbf{Tiri Salvezza}: +1 Tempra, +1 Volontà

\begin{center}
	\includegraphics[width=0.9\linewidth]{immagini/oggettimagiciuomo.png}
	\textit{Henry Purcell - King Arthur}
\end{center}

Tramite questa Abilità l'incantatore è in grado di infondere un'incantesimo fino alla Difficoltà 34.


\subsection{Creare Oggetti Magici Mitici}\index{Creare Oggetti Magici Mitici}

\textbf{Requisito}: Creare Oggetti Magici Meravigliosi, Competenza Magica 18

\textbf{Tiri Salvezza}: +1 Tempra, +1 Volontà

Tramite questa Abilità l'incantatore è in grado di infondere un'incantesimo fino alla Difficoltà 36.


\subsection{Decifrare scritti magici}\index{Decifrare scritti magici}

\textbf{Requisito}: Competenza Magica 1

\textbf{Tiri Salvezza}: +1 Tempra, +1 Volontà

Saper leggere le scritte magiche. Ha un bonus di +4 nel comprendere il contenuto di una pergamena e nel lanciare l'incantesimo contenuto.

\subsection{Difendere Cavalcatura}\index{Difendere Cavalcatura}

\textbf{Tiri Salvezza}: +1 Tempra, +1 Riflessi

Ogni qual volta la cavalcatura viene colpita, puoi effettuare una prova di Cavalcare per negare il colpo. La tua prova di Cavalcare deve essere maggiore del Tiro per Colpire dell'avversario

L'Abilità è utilizzabile solo una volta per round, per un solo attacco, costa 1 Azione.

\subsection{Distillare pozioni}\index{Distillare pozioni}

\textbf{Requisito}: Competenza Magica 1

\textbf{Tiri Salvezza}: +1 Tempra, +1 Volontà

Competenza nel distillare pozioni.

Acquisti un bonus di +4 su Conoscenze Erboristeria e nel distillare e creare pozioni e veleni naturali.

\subsection{Doppia porzione}\index{Doppia porzione}

\textbf{Requisito}: Combattimento con due armi, Competenza Armi 4

\textbf{Tiri Salvezza}: +2 Tempra, +1 Riflessi

Il costante allenamento con due armi ti permette di applicare il bonus al danno dovuto alla Forza in maniera piena anche all'arma secondaria.

\subsection{Energia Psichica}\index{Energia Psichica}

\textbf{Requisito}: Forza 1, Saggezza 2, Competenza Armi 1, Competenza Magia 1

\textbf{Tiri Salvezza}: +2 Volontà, +1 Tempra

Dopo anni di allenamento, meditazione e stage a Nanda Parbat sei in grado di raccogliere la tua Energia Chi.

Ogni giorno dopo almeno 6 ore di riposo e 2 ore di meditazione/allenamento riempi il tuo corpo di energia Chi pari a (Competenza Armi+Competenza Magica)/2+Saggezza/2

La \textbf{seconda volta} che prendi questa Abilità: \textbf{Requisito}: Forza 1, Saggezza 2, Competenza Armi 4, Competenza Magia 4

Recuperi 2 punti Chi per ogni ora di riposo.   


\begin{center}
	\includegraphics[width=0.9\linewidth]{immagini/distillare.png}
	
	\textit{The Alchemist Discovering Phosphorus . Joseph Wright of Derby (1771-1795)}
\end{center}

\subsection{Colpo Psichico}\index{Colpo Psichico}

\textbf{Requisito}: Energia Psichica, Destrezza 2

\textbf{Tiri Salvezza}: +2 Volontà, +1 Tempra

Concentri il tuo Chi nelle tue mani.
Puoi concentrare un numero di punti Chi pari alla Saggezza.
Con il primo colpo andato a segno, Difesa a tocco, nel round scarichi l'energia che causa 1d6 per punto Chi usato.

Scaricare l'energia Chi costa 1 Azione.

La \textbf{seconda volta} che prendi questa Abilità: \textbf{Requisito}: Colpo Psichico, Saggezza 3, Competenza Armi 7

Puoi utilizzare fino a doppio del tuo punteggio in Saggezza per potenziare il Colpo Psichico

\subsection{Raggio Psichico}\index{Raggio Psichico}

\textbf{Requisito}: Colpo Psichico, Saggezza 3, Competenza Armi 5

\textbf{Tiri Salvezza}: +2 Riflessi, +1 Volontà

Puoi effettuare un attacco a distanza entro 9 metri usando l'Energia Psichica.
Il colpo, Tiro per Colpire contro la Difesa a tocco, causa 1d6 di danno per punto Psichico speso focalizzato sul danno.

E' possibile focalizzare uno o più punti Psichici per aumentare la distanza ogni volta di 9 metri.
Non puoi usare un numero di punti Chi totali (per distanza e e danno) superiore alla Saggezza. Costo 1 Azione di Attacco.

La \textbf{seconda volta} che prendi questa Abilità: \textbf{Requisito}: Colpo Psichico, Saggezza 3, Competenza Armi 9

Puoi utilizzare fino a doppio del tuo punteggio in Saggezza per potenziare il Raggio Psichico

\subsection{Esperto}\index{Esperto}

\textbf{Requisito}: Caratteristica collegata almeno a 1

\textbf{Tiri Salvezza}: +1 a due Tiri Salvezza a scelta.

Sei un esperto in un argomento. Ogni qual volta prendi questa Abilità guadagni un +2 alle prove su una competenza a tua scelta.

Non si può prendere più di due volte questa Abilità sulla stessa Competenza.

Non è usabile su Consapevolezza (vedi Percettivo).

\subsection{Fare Infuriare}\index{Fare Infuriare}

Le tue abilità dialettiche sono incredibili.

\textbf{Requisito}: Competenza Armi 2 e Carisma 2

\textbf{Tiri Salvezza}: +2 Volontà, +1 Tempra

Impieghi 2 Azioni ad infamare ed inveire contro un avversario. Il target deve fare un Tiro Salvezza Volontà contro la tua prova di competenza Intrattenere oppure perdere il bonus di Destrezza (Tiri Salvezza e Tiro Colpire e Difesa) fino alla fine del round successivo.

L'avversario può non comprendere la tua lingua ma deve avere Intelligenza maggiore di -1.

\subsection{Fedele}\index{Fedele}

\textbf{Requisito}: Competenza Magica 1, Somma valore Tratti in comune 2

\textbf{Tiri Salvezza}: +2 Volontà, +1 Tempra

La tua connessione con il Patrono è forte ed energetica. Ogni volta che prendi questa Abilità puoi lanciare 1 incantesimo in più al giorno.

Ogni volta che prendi quest'Abilità il valore della somma dei Tratti in comune con il tuo Patrono deve essere il triplo delle volte che hai preso quest'Abiltà. Quest'Abilità non si somma con l'Abilità Batteria Magica.

\subsection{Ferocia}\index{Ferocia}

\textbf{Requisito}: Competenza Armi 1

\textbf{Tiri Salvezza}: +2 Tempra, +1 Volontà

La tua rabbia è tale da sconfiggere, temporaneamente, la morte.

Quando scendi sotto lo 0 punti ferita non svieni ed incominci a perdere 1 punto ferita a round.

Una creatura dotata di ferocia sviene quando ha un punteggio di punti ferita negativo pari al doppio dei punti di Forza e muore quando i suoi punti ferita scendono al punteggio negativo pari al suo triplo del punteggio di Costituzione+5 (COS*3+5)

\subsection{Finta Morte}\index{Finta Morte}

Sei in grado di simulare la morte, rallentando il cuore.

\textbf{Tiri Salvezza}: +2 Tempra, +1 Volontà

Come Reazione sei in grado di cadere a terra (stramazzare!) morto. Solo una prova di Pronto Soccorso DC 20 può rivelare che sei vivo.

L'effetto dura al massimo 2 minuti. La finta morte non è ripetibile in intervalli inferiori ai 10 minuti l'una dall'altra.

\subsection{Flagello Danzante}\index{Flagello Danzante}

\textbf{Requisito}: Competenza Armi 1

\textbf{Tiri Salvezza}: +1 Tempra, +1 Volontà

Quando usi il tuo Flagello hai un bonus di +1 al Tiro per Colpire e +1 alla Difesa

\subsection{Forgiato nella furia}\index{Forgiato nella furia}

\textbf{Requisito}: Competenza Armi 5

\textbf{Tiri Salvezza}: +1 Tempra, +1 Riflessi

Quando effettui un critico, ovvero hai tirato almeno 2 volte 6, si considera che tu abbia tirato un 6 in più per il conteggio totale del numero di critici

\subsection{Freccia chiamata, freccia consegnata}\index{Freccia chiamata, freccia consegnata}

\textbf{Requisito}: Competenza Armi 2

\textbf{Tiri Salvezza}: +2 Riflessi

Puoi tirare 1 freccia, una volta al giorno, come azione immediata, senza penalità al colpire.

\subsection{Furia}\index{Furia}

\textbf{Requisito}: Competenza Armi 1

\textbf{Tiri Salvezza}: +2 Tempra, +1 Volontà

Il tuo stile di combattimento è rappresentato dalla cieca furia omicida. Aggiungi +1d6 al danno ad ogni attacco andato a segno in mischia ed i tuoi avversare guadagna +1d6 al colpire verso di te.

\subsection{Giocoliere}\index{Giocoliere}

\textbf{Requisito}: Destrezza 2

\textbf{Tiri Salvezza}: +2 Riflessi

Hai un talento naturale per maneggiare gli oggetti.

Qualsiasi prova di Acrobatica che coinvolga il maneggiare oggetti o l'equilibrio ha un +2 di Bonus.

Puoi lanciare un secondo pugnale come azione immediata all'azione di attacco di lancio pugnale con un -3 al Tiro per Colpire. Un eventuale terzo pugnale lanciato ha il normale malus di -5 (e -10.. e così via).

\begin{center}
	\includegraphics[width=0.9\linewidth]{immagini/Early_Egyptian_juggling_art.jpg}
	
\textit{This ancient wall painting appears to depict jugglers.}
\end{center}

\subsection{Guerriero della Magia}\index{Guerriero della Magia}

\textbf{Tiri Salvezza}: +1 Volontà, +1 Riflessi

Non segui solo la via della magie e neanche quella della spada, il tuo stile abbraccia entrambi in un fendente di pura magia

La \textbf{prima volta} che prendi questa Abilità, Competenza Armi 2, Competenza Magia 2: sei in grado di scaricare un incantesimo a distanza di mischia con la tua arma. Effettui un Tiro per Colpire normale (3d6+CA+Forza+...) e se colpisci oltre al danno dell'attacco scarichi anche l'incantesimo. Consumi 3 Azioni. L'incantesimo deve avere un tempo di lancio pari o inferiore a 2 Azioni.

Se la Prova di Magia per superare la Difficoltà dell'incantesimo fallisce non scarichi l'incantesimo e comunque esegui un solo attacco e consumi 3 Azioni.

La \textbf{seconda volta} che prendi questa Abilità Competenza Armi 6, Competenza Magia 3: sei in grado di attaccare con l'arma di mischia e riattaccare scaricando l'incantesimo con l'arma. Consumi 3 Azioni, e scarichi l'incantesimo in uno degli attacchi multipli che esegui. L'incantesimo deve avere un tempo di lancio pari o inferiore a 2 Azioni.

\subsection{Gru d'Argento}\index{Gru d'argento}

\textbf{Requisito}: Lista Pugno Vuoto 3

\textbf{Tiri Salvezza}: +2 Riflessi, +1 Volontà

Il tuo stile di combattimento disarmato è basato sull'agilità ed il contrattacco.

La \textbf{prima volta} che prendi quest'Abilità la tua Difesa aumenta di 1.

La \textbf{seconda volta} che prendi quest'Abilità, \textbf{Requisito} Lista Pugno Vuoto 6, la tua Iniziativa aumenta di 1 (solo con attacchi disarmati).

La \textbf{terza volta} che prendi quest'Abilità, \textbf{Requisito} Lista Pugno Vuoto 12 e Destrezza 2, hai un bonus nei Tiri salvezza su Riflessi di 1 (cumulativo).

La \textbf{quarta volta} che prendi quest'Abilità, \textbf{Requisito} Lista Pugno Vuoto 15, la tua Difesa aumenta di 1 (cumulativo).

La \textbf{quinta volta} che prendi quest'Abilità, \textbf{Requisito} Lista Pugno Vuoto 18 e Destrezza 3, hai un bonus nei Tiri salvezza su Riflessi di 1 (cumulativo).

\subsection{Ho detto CADI!}\index{Ho detto CADI!}

\textbf{Requisito}: Competenza Armi 4

\textbf{Tiri Salvezza}: +2 Tempra, +1 Volontà

Se colpisci 3 volte consecutivamente (3 colpi consecutivi entro 3 round) un avversario questo deve fare una Tiro Salvezza su Tempra (DC pari al Tiro per Colpire dell'ultimo terzo colpo) o cadere prono.

\subsection{Il Patrono e' con me}\index{Il Patrono e' con me}

\textbf{Requisito}: Somma Tratti comuni con il Patrono 2

\textbf{Tiri Salvezza}: +1 Volontà, +1 Riflessi

La \textbf{prima volt}a che prendi questa Abilità per 1 volte in un giorno puoi ritirare gli 1 ottenuti nella prova di magia per il lancio di incantesimo. L'Abilità può essere dichiarata anche dopo il lancio dei dadi.

La \textbf{terza volta} che prendi questa Abilità per 2 volte in un giorno puoi ritirare gli 1,2 ottenuti nella prova di magia per il lancio di incantesimo. L'Abilità può essere dichiarata anche dopo il lancio dei dadi.

La \textbf{quinta volta} che prendi questa Abilità per 3 volte in un giorno puoi ritirare gli 1,2,3 ottenuti nella prova di magia per il lancio di incantesimo. L'Abilità può essere dichiarata anche dopo il lancio dei dadi.

\subsection{Il Patrono e' la mia magia}\index{Il Patrono e' la mia magia}\hypertarget{ilpatronoelamiamagia}{}  

Questa Abilità sovverte le regole di Dungeon Bell System facendo che si usi la somma Tratti in comune con il Patrono al posto della Competenza Magica in ogni tiro che coinvolge la Competenza Magica.

Per tutte le prove che sommano Competenza Magica e somma Tratti in comune si andrà a sommare due volte la somma Tratti in comune.

Questa Abilità deve essere concessa in accordo con il Narratore.

\subsection{Improvvisare}\index{Improvvisare}

\textbf{Requisito}: Competenza Armi 1

\textbf{Tiri Salvezza}: +1 Tempra, +1 Riflessi

Qualsiasi oggetto che possa definirsi un arma improvvisata per te non e' improvvisata.
Non soffri di penalita' al colpire quando usi un arma improvvisata. Ottieni un +1 al danno quando usi un arma improvvisata.

\subsection{Incantare in Combattimento}\index{Incantare in Combattimento}

La\textbf{ prima volta} che prendi questa Abilità il malus dato da Distrazione diminuisce di 2.

\textbf{Tiri Salvezza}: +1 Tempra, +1 Volontà

La \textbf{seconda volta} che prendi questa Abilità il malus dato da Distrazione diminuisce di ulteriori 2.

\subsection{Incantatore Prudente}\index{Incantatore Prudente}

\textbf{Tiri Salvezza}: +2 Riflessi, +1 Tempra

La prima volta che prendi questa Abilità il malus alla Difesa mentre lanci un'incantesimo sotto minaccia diminuisce di 2 (da -4 a -2).

La \textbf{seconda volta} che prendi questa Abilità, CA minimo 3, il malus alla Difesa diminuisce di 1 (e va a -1)

La \textbf{terza volta} che prendi questa Abilità, CA minimo 6, il malus alla Difesa diventa 0.

Questa Abilità non influisce su fatto che si è comunque Distratti nel lancio.

\subsection{Immunità ai veleni}\index{Immunità ai veleni}

 \textbf{Tiri Salvezza}: +2 Tempra, +1 Volontà

Il corpo si abitua ai veleni, il personaggio guadagna un +2 Tiro Salvezza sui veleni.

La \textbf{seconda volta} che prendi l'Abilità divieni immune ai veleni naturali. Non riesci più ad ubriacarti normalmente.

La \textbf{terza volta} hai un +4 ai Tiro Salvezza ai veleni magici e subire gli effetti di fumi tossici (ma puoi sempre soffocare).

\subsection{Imposizione delle mani} \index{Imposizione delle mani}

\textbf{Requisito}: Competenza Magica 3, Tratti comuni 3

\textbf{Tiri Salvezza}: +2 Volontà, +1 Tempra

Se i tuoi Tratti sono in comune con un Patrono positivo puoi convogliare energia positiva (effetto curativo), se sono in comune con un Patrono neutrale o malvagio puoi convogliare energia negativa (effetto dannoso). Usabile un numero di volte al giorno pari al valore di Saggezza. Effetto curativo/dannoso pari a 1d6+Saggezza

La \textbf{seconda volta}, \textbf{Requisito} punteggio Tratti in Comune 6, che prendi questa Abilità aumenti di 2d6 l'effetto e di 1 volte l'uso.

La \textbf{terza volta}, \textbf{Requisito} punteggio Tratti in Comune 12, che prendi questa Abilità aumenti di 3d6 l'effetto e di 1 volte l'uso.

La \textbf{quarta volta}, \textbf{Requisito} punteggio Tratti in Comune 18: che prendi questa Abilità aumenti di 4d6 l'effetto e di 1 volte l'uso.

L'energia proviene dalle mani (non conta se ci sono guanti) e si applica solo a tocco (o Difesa a tocco). Usa 2 Azioni.

\begin{center}
	\includegraphics[width=0.9\linewidth]{immagini/Portrait_of_V_Greatrakesv2.png}
\textit{Portrait of V. Greatrakes laying on his hands, window, in right-hand corner showing several successful cures, possibly. By W. Faithorne }
\end{center}


\subsection{Incanalare energia} \index{Incanalare energia}

\textbf{Requisito}: Competenza Magica 1, Tratti comuni 3

\textbf{Tiri Salvezza}: +2 Volontà, +1 Riflessi

Sei in grado di incanalare l'energia positiva o negativa del tuo Patrono.

Se i tuoi Tratti sono in comune con un Patrono positivo puoi convogliare energia positiva (cura), se sono in comune con un Patrono neutrale o malvagio puoi convogliare energia negativa. Usabile un numero di volte pari punteggio di Saggezza. Effetto curativo/dannoso pari a 1d6+Saggezza. Influenzi 1 creatura.

La \textbf{seconda volta}, punteggio Tratti in Comune 6, che prendi questa Abilità aumenti di 1d6 l'effetto e di 1 volta l'uso. Influenzi fino a 2 creature.

La \textbf{terza volta}, punteggio Tratti in Comune 12, che prendi questa Abilità aumenti di 2d6 l'effetto e di 1 volta l'uso. Influenzi fino a 3 creature.

La \textbf{quarta volta}. punteggio Tratti in Comune 18, che prendi questa Abilità aumenti di 3d6 l'effetto e di 1 volta l'uso. Influenzi fino a 4 creature.

L'energia proviene dalle mani (non conta se ci sono guanti) ed influenza una o più creature entro un tre metri da te, eventuale Tiro per Colpire con Difesa a tocco per colpire l'interessato. Scegli tu le creature che influenzi. Usa 2 Azioni.

\subsection{Incanalare energia a distanza}\index{Incanalare energia a distanza}

\textbf{Requisito}: Incanalare energia

Puoi lanciare l'energia, come da Incanalare Energia, fino a 9 metri, influenza un raggio di 3 metri. Usa 2 azioni.

La \textbf{seconda volta} che prendi questa Abilità l'energia arriva fino alla distanza di 18 metri.

La \textbf{terza volta} che prendi questa Abilità il l'energia arriva fino alla distanza di 36 metri.

\subsection{Incanalare energia concentrata}\index{Incanalare energia concentrata}

\textbf{Requisito}: Incanalare energia

\textbf{Tiri Salvezza}: +2 Volontà, +1 Riflessi

Puoi lanciare l'energia fino a distanza 18 metri. Singolo obiettivo. Usa 2 azioni.

Ogni volta che prendi questa competenza aggiungi un obiettivo entro 6 metri di distanza dal precedente obiettivo sul quale dividere a piacimento i dadi disponibili dell'incanalare energia.

L'Abilità non è cumulabile con "Incanalare energia a distanza".


\begin{center}
	\includegraphics[width=0.9\linewidth]{immagini/kameame.png}
	\textit{Kamehameha!}
\end{center}

\subsection{Infondere Coraggio}\index{Infondere Coraggio}

\textbf{Requisito}: Carisma 2

\textbf{Tiri Salvezza}: +2 Volontà, +1 Tempra

Tramite la tua esibizione, canora, di balletto, oratoria.. sei in grado di infondere coraggio nei compagni in grado di sentirti o vederti, nel raggio di 6 metri.

La prima volta che prendi questa Abilità i tuoi compagni hanno un bonus di +1 al Tiro per Colpire e ai Tiri Salvezza su Volontà.

La \textbf{seconda volta} che prendi questa Abilità puoi decidere di infondere fino a 2 di questi bonus. +2 TC, +2 Difesa, +2 Tiro Salvezza su Volontà. I tuoi compagni devono essere entro 9 metri di raggio.

La \textbf{terza volta} che prendi questa Abilità  puoi decidere di infondere fino a 2 di questi bonus. +3 Tiro per Colpire,  +3 Difesa, +2 Danno, +2 TS. I tuoi compagni devono essere entro 12 metri di raggio.

Attivare e mantenere l'Abilità richiede 2 Azioni. Puoi mantenere l'Abilità un numero di round, anche non consecutivi, pari a Carisma x 5 al giorno


\subsection{Infondere Paura}\index{Infondere Paura}

\textbf{Requisito}: Carisma 2

\textbf{Tiri Salvezza}: +2 Volontà, +1 Tempra

Tramite la tua esibizione, canora, di balletto, oratoria.. sei in grado di infondere paura negli avversari in grado di sentirti, nel raggio di 6 metri.

La prima volta che prendi questa Abilità i tuoi nemici hanno un malus di -1 al Tiro per Colpire e al Tiro Salvezza su Volontà.

La \textbf{seconda volta} che prendi questa Abilità la forza della tua arte aggredisce i nemici con un malus di -2 Tiro per Colpire,  -2 Difesa, -2 al Tiri Salvezza su Volontà. I tuoi nemici devono essere entro 9 metri di raggio. Agli avversari è concesso un Tiro Salvezza su Volontà a DC 15 per annullare gli effetti.

La \textbf{terza volta} che prendi questa Abilità la forza della tua arte aggredisce i nemici con un malus di -3 Tiro per Colpire,  -3 Difesa, -2 Danno, -2 TS. I tuoi nemici devono essere entro 12 metri di raggio. Agli avversari è concesso un Tiro Salvezza su Volontà a DC 17 per annullare gli effetti.

Una creatura che riesce nel Tiro Salvezza è immune per quel giorno a nuove manifestazioni di questo tuo potere.

Attivare e mantenere l'Abilità richiede 2 Azioni. Puoi mantenere l'Abilità un numero di round, anche non consecutivi, pari a Carisma x 3 al giorno

\subsection{Iniziativa migliorata}\index{Iniziativa migliorata}

\textbf{Tiri Salvezza}: +2 Riflessi

Aumenti l'iniziativa di +1. L'Abilità può essere presa fino a 2 volte ed il bonus si cumula. 

\subsection{Iaijutsu}\index{Iaijutsu}

\textbf{Tiri Salvezza}: +2 Riflessi, +1 Volontà

Per ogni -5 al Tiro per Colpire guadagni un +5 all'Iniziativa e vice versa.
Il bonus deve essere usato entro la fine del round successivo. La dichiarazione va eseguita ogni round che si intende usare al momento del controllo delle iniziative.

\subsection{La mia morte la tua morte}\index{La mia morte la tua morte}

\textbf{Tiri Salvezza}: +2 Tempra, +1 Volontà

Per ogni singolo avversario di combattimento puoi fare che il primo colpo a segno dello scontro causi un danno aggiuntivo pari al doppio di Competenza Armi. L'avversario guadagna un bonus al Tiro per Colpire ed al danno pari al valore della tua Competenza Armi.

\subsection{La mia Testa è più Dura}\index{La mia Testa è più Dura}

\textbf{Requisito}: Competenza Armi 1

\textbf{Tiri Salvezza}: +1 Tempra, +1 Volontà

La tua Arma Rompi Cranio fa +2 danni

\subsection{Lo scudo è mio amico}\index{Lo scudo è mio amico}

\textbf{Requisito}: Competenza Armi 1

\textbf{Tiri Salvezza}: +1 Tempra, +1 Riflessi

La penalità alla Competenza Magica data dall'uso dello scudo diminuisce di 1.

La \textbf{seconda volta} che si prende questa Abilità, Competenza Armi 3, la penalità alla Competenza Armi diminuisce di 1, la penalità alla Competenza Magica diminuisce ulteriormente 2.

La \textbf{terza volta} che si prende questa Abilità, Competenza Armi 5, la penalità alla Competenza Armi diminuisce di 3, la penalità alla Competenza Magica diminuisce di ulteriore 2.

\subsection{Magie potenti}\index{Magie potenti}

\textbf{Requisito}: Competenza Magica 5

\textbf{Tiri Salvezza}: +2 Volontà

Le tue magie sono straordinariamente efficaci.

Scegli una Scuola di Magia, ottieni un +1 alla prova di Competenza Magica nel lancio di incantesimi da questa scuola. L'Abilità può essere presa più volte ma il totale deve essere inferiore a CM/4 ed il bonus si somma o si applica ad altra Scuola.

\subsection{Montagna umana}\index{Montagna umana}

\textbf{Tiri Salvezza}: +2 Tempra, +1 Volontà

Forse una volta eri gracile e debole, adesso sei una montagna di muscoli.

Quando prendi questa Abilità aumenti di 1 i punti ferita presi per livello.

La \textbf{seconda volta} che prendi questa Abilità aumenti di 1 i Punti Ferita presi per livello.

La \textbf{terza volta} che prendi questa Abilità aumenti il dado per tirare i punti ferita (da d4 a d6).

I bonus sono cumulativi e retroattivi ai livelli precedenti, tranne che l'aumento di dado vita.

La \textbf{quarta volta} che prendi questa Abilità aumenti di una taglia (P > M > G > E).

\begin{center}
	\includegraphics[width=0.7\linewidth]{immagini/elcolosso.png}
	
\textit{The Colossus (also known as The Giant), is known in Spanish as El Coloso. Non è di Goya ma di un allievo.}
\end{center}


\subsection{Occhio Clinico}\index{Occhio Clinico}

\textbf{Requisito}: Competenza Armi 3

\textbf{Tiri Salvezza}: +2 Riflessi

Sei in grado di fare critici a creature normalmente immuni ai critici (tirare più volte 6).

\subsection{Occhio di Falco}\index{Occhio di Falco}

\textbf{Requisito}: Competenza Armi 3

\textbf{Tiri Salvezza}: +2 Riflessi, +1 Volontà

La penalità per i tiri tra il primo ed il secondo incremento non ha penalità

La \textbf{seconda volta} che prendi questa Abilità, la penalità per i tiri fino al terzo incremento di range è di 1d6.

La \textbf{terza volta} che prendi questa Abilità sei in grado di estendere ancora di più il tuo tiro e portarlo ad un quarto incremento con un -2d6 di penalità al colpire. Non hai penalità entro i primi 3 incrementi mentre hai -1d6 a colpire tra il terzo e quarto incremento.

\textbf{Esempio. Tups usa un Arco Corto, gittata 15 metri.}

Normalmente se deve tirare una freccia ad un orchetto entro 15 m non ha penalità, se l'orchetto è tra 15 e 30 metri ha 1-d6 al Tiro per Colpire, se è tra 30 e 45 metri ha -2d6 al colpire. Oltre non può tirare.

\begin{itemize}

\item \textbf{Tups prende l'Abilità Occhio di Falco}.

Se deve tirare una freccia ad un orchetto entro 15 m non ha penalità, se l'orchetto è tra 15 e 30 metri non ha penalità, se è tra 30 e 45 metri ha -2d6 al colpire. Oltre non può tirare.

\item \textbf{Tups prende l'Abilità Occhio di Falco una seconda volta}.

Se deve tirare una freccia ad un orchetto entro 15 m non ha penalità, se l'orchetto è tra 15 e 30 metri non ha penalità, se è tra 30 e 45 metri ha -1d6 al colpire. Oltre non può tirare.

\item \textbf{Tups prende l'Abilità Occhio di Falco una terza volta}.

Se deve tirare una freccia ad un orchetto entro 15 m non ha penalità, se l'orchetto è tra 15 e 30 metri non ha penalità, se è tra 30 e 45 metri non ha penalità colpire, se vuole può tirare la freccia tra i 45 e 60 metri con -2d6 al colpire.

\end{itemize}

\subsection{Opportunista}\index{Opportunista}

\textbf{Tiri Salvezza}: +2 Riflessi, +1 Volontà

\textbf{Requisito}: Competenza Armi 2

Puoi tentare di colpire un avversario (un attacco di opportunità) che esce da un area che tu minacci. L'Abilità è usabile una volta per round come Reazione a costo 0 Azioni.

\subsection{Passo Veloce}\index{Passo Veloce}

\textbf{Tiri Salvezza}: +2 Riflessi, +1 Tempra

Il tuo passo è naturalmente rapido. 
Se hai movimento 6m passi a movimento 7m, se hai movimento 9m passi a movimento 10m.

Ogni volta che prendi l'Abilità, massimo 3 volte, il tuo movimento aumento di 1 metro per Azione di Movimento.

\subsection{Passo sicuro}\index{Passo sicuro}

\textbf{Tiri Salvezza}: +2 Tempra, +1 Riflessi

E' la capacità di non essere rallentati in un ambiente ostile. E' necessario dichiarare su quale ambiente si prende l'Abilità. In questi ambienti il terreno non è difficile per te.

\bigskip

\begin{tabular}{l|l}
\textbf{Ambiente}     & \textbf{Ambiente}\\
\toprule
Giungla					& Acquatico \\
Montagna e colline		& Foresta    \\
Pianura					& Deserto e brullo    \\
Palude 					& Ghiacciai e tundra       \\
Urbano					& Sotterraneo   \\
\end{tabular}

\bigskip

Ogni qual volta si prende nuovamente questa Abilità si sceglie un ambiente diverso e si aggiunge al precedente.

\subsection{Percettivo}\index{Percettivo}

\textbf{Tiri Salvezza}: +1 Riflessi, +1 Volontà

La tua Consapevolezza e attenzione ai particolari è sopra la media.
Prendi un bonus di +1 alla prove di Consapevolezza. L'Abilità può essere presa massimo 3 volte.

\subsection{Persona veramente malvagia}\index{Persona veramente malvagia}

\textbf{Requisito}: Competenza Armi 1

\textbf{Tiri Salvezza}: +1 Riflessi, +1 Volontà

Due volte al giorno aggiungi il tuo valore di Competenza Armi al danno, in mischia ad un avversario che vedi. L'Abilità deve essere dichiarata prima del Tiro per Colpire. Costa una Azione.

\subsection{Più sono grossi più fanno rumore quando cadono}\index{Più sono grossi più fanno rumore quando cadono}

\textbf{Requisito}: Competenza Armi 1

\textbf{Tiri Salvezza}: +2 Tempra, +1 Volontà

Quando attacchi una creatura di almeno 2 taglie più grosse di te fai +1 danno aggiuntivo ogni 2 punti Competenza Armi. Se è solo una taglia superiore aggiungi 1 danno in più ogni 3 punti Competenza Armi.

\subsection{Proseguire}\index{Proseguire}

\textbf{Requisito}: Competenza Armi 1

\textbf{Tiri Salvezza}: +1 Tempra, +1 Volontà

Se uccidi l'avversario con il tuo ultimo colpo, in mischia, puoi effettuare un azione di attacco bonus ed attaccare l'avversario successivo entro 1m con un -2 al colpire -1 al danno, se uccidi questa creature con un colpo non puoi effettuare altri attacchi ad altre creature.

La \textbf{seconda volta} che prendi questa Abilità \textbf{Requisito}: Proseguire, Competenza Armi 6

Se uccidi la creatura con il tuo ultimo colpo, in mischia, puoi effettuare un azione di attacco bonus con l'arma ed attaccare la creatura successiva in distanza di 1 metro con un -2 al colpire -1 al danno, se la uccidi puoi proseguire con un ulteriore attacco bonus (e ti sposti entro 1 metro) con la creature successiva e così via.

Ogni volta hai un -2 al colpire ed un -1 al danno cumulativo.


\subsection{Pugno di Ferro}\index{Pugno di Ferro}

\textbf{Requisito}: Lista Pugno vuoto 3

\textbf{Tiri Salvezza}: +2 Tempra, +1 Volontà

La tua tecnica di combattimento senza armi è estremamente precisa e potente.

La \textbf{prima volta} che prendi quest'Abilità il danno causato dai tuoi pugni (e calci) aumenta di 1 per colpo andato a segno. I tuoi colpi sono considerati come armi d'argento.

La \textbf{seconda volta} che prendi quest'Abilità, \textbf{Requisito} Pugno Vuoto 6, il danno causato dai tuoi pugni (e calci) aumenta di 1 per colpo andato a segno (cumulativo). I tuoi colpi sono considerati come arma +1.

La \textbf{terza volta} che prendi quest'Abilità, \textbf{Requisito} Pugno Vuoto 12, il danno causato dai tuoi pugni (e calci) aumenta di 1 per colpo andato a segno (cumulativo). I tuoi colpi sono considerati come arma di adamantio.

La \textbf{quarta volta} che prendi quest'Abilità, \textbf{Requisito} Pugno Vuoto 15, il danno causato dai tuoi pugni (e calci) aumenta di 1 per colpo andato a segno (cumulativo). I tuoi colpi sono considerati come arma +2.

La \textbf{quinta volta} che prendi quest'Abilità, \textbf{Requisito} Pugno Vuoto 18, il danno causato dai tuoi pugni (e calci) aumenta di 1 per colpo andato a segno (cumulativo). I tuoi colpi sono considerati come arma +3.

\subsection{Questo è il mio pugnale}\index{Questo è il mio pugnale}

\textbf{Tiri Salvezza}: +2 Tempra, +1 Riflessi

\textbf{Requisito}: Competenza Armi 1

Ogni qual volta fai un critico con il tuo pugnale sommi la tua Competenza Armi al danno. L'Abilità è usabile 1 volta per avversario e si applica automaticamente al primo critico effettuato.

\subsection{Questa è la mia arma!}\index{Questa è la mia arma!}

\textbf{Requisito}: Competenza Armi 1

\textbf{Tiri Salvezza}: +2 Tempra, +1 Volontà

Ogni volta che colpisci il medesimo avversario fai un danno aggiuntivo (Max +1 per round di combattimento, anche se lo colpisci più volte nel round) fino ad un massimo +5. La prima volta che non colpisci nel round l'avversario il bonus torna a +0. Il bonus si può mantenere su un solo avversario alla volta.

\begin{center}
	\includegraphics[width=0.8\linewidth]{immagini/streghegoya.png}
	
	\textit{Sabba delle Streghe (Goya, 1798)}
\end{center}

\subsection{Radici magiche}\index{Radici magiche}

\textbf{Requisito}: Competenza Magica 1

\textbf{Tiri Salvezza}: +2 Volontà, +1 Tempra

Finché sei influenzato da un tuo incantesimo, utilizzando un'Azione la tua arma guadagna un +1 al colpire ed al danno e si considera un'arma magica. Per ogni tuo incantesimo che ti influenza nel round, oltre il primo (non da oggetti magici) il bonus aumenta di +1/+1 fino ad un massimo di +3/+3.

\subsection{Rappresaglia}\index{Rappresaglia}

\textbf{Tiri Salvezza}: +2 Volontà, +1 Tempra

Vedere i tuoi amici feriti ti riempie di rabbia.
Quanto un compagno (o te stesso) scende sotto metà dei punti ferita guadagni un +1 a Tiro per Colpire e Tiri Salvezza. La durata massima dell'effetto é 1 minuto (6 round) al giorno e deve essere consecutiva. Il giocatore sceglie se attivare o meno l'Abilità.
Puoi prendere questa Abilità fino a 3 volte, ogni volta il bonus al Tiro per Colpire e Tiro Salvezza aumentano di 1.

\subsection{Resistenza della pietra}\index{Resistenza della pietra}

Nel tempo hai allenato la tua Costituzione a reggere gli urti, trasformazioni, veleni e quant'altro volesse modificare il tuo corpo. La prima volta che prendi questa Abilità ottieni un bonus di +2 al Tiro Salvezza su Tempra. Il bonus è cumulativo, +2 la prima volta, +1 la seconda, +1 la terza ed ultima volta possibile.

\subsection{Rilevare il Magico}\index{Rilevare il Magico}

\textbf{Requisito}: Competenza Magica 1

\textbf{Tiri Salvezza}: +11 Volontà, +1 Tempra

Se lo puoi vedere sai anche se è magico. Costa una Azione attivare la vista magica e dura un round.

\subsection{Ricarica rapida (Balestra)}\index{Ricarica rapida}

\textbf{Requisito}: Destrezza 2, Tiro preciso

\textbf{Tiri Salvezza}: +2 Riflessi

Come Abilità Tiro Rapido, solo per balestre

\subsection{Riflessi fulminei}\index{Riflessi fulminei}

Nel tempo hai allenato i tuoi riflessi a schivare e prevedere qualsiasi ostacolo. La prima volta prendi questa Abilità ottieni un bonus di +2 ai Tiri Salvezza su Riflessi. Il bonus è cumulativo, +2 la prima volta, +1 la seconda, +1 la terza ed ultima volta possibile.

\subsection{Sapiente}\index{Sapiente}

\textbf{Requisito}: Competenza Magica 4

\textbf{Tiri Salvezza}: +2 Volontà

Il tuo interesse e connessione con la magia non ha eguali. Puoi conoscere un incantesimo in più (pur rispettando i vincoli di massima Difficoltà sceglibile).

L'Abilità può essere presa nuovamente purché il valore di Competenza Magica sia almeno 4 volte le volte prese questa Abilità. Quindi con valore minimo di Competenza Magica 4, 8, 12.. 

\begin{center}
	\includegraphics[width=0.7\linewidth]{immagini/turning-undead-six.jpg}
\end{center}


\subsection{Scacciare i non morti}\index{Scacciare i non morti}

\textbf{Tiri Salvezza}: +2 Volontà, +1 Tempra

Concentrandoti sulla potenza del tuo Patrono convogli l'energie positiva ed allontani o distruggi i non-morti.

Effettua una prova tirando 1d6 (valgono le Golden Rules) per ogni volta che hai preso questa Abilità + la somma dei Tratti in comune con il tuo Patrono. Confronta il risultato con questa tabella per capire gli effetti ottenuti.

\end{multicols}

%\addvspace{5cm}

\textbf{Tabella: Scacciare Non Morti}\index{Tabella Scacciare Non Morti}

\medskip

\begin{tabularx}{0.9\textwidth}{Xllllllllll}
    \textbf{Non Morto} & \multicolumn{6}{c}\textbf{\textbf{Valore Prova}}\\
 \textbf{o Grado di Sfida} 	&2-4 & 5-8 &9-11 &12-15&16-18&19-22&23-25&26-29&30-32&33-36\\
	\toprule
	\textbf{Scheletro}       	& - & T   & T   & T   & D   & D   & D   & D*  & D*  & D*  \\
	\textbf{Zombie} 			& - & -   & T   & T   & T   & D   & D   & D   & D*  & D*  \\
	\textbf{Ghoul}				& - & -   & -   & T   & T   & T   & D   & D   & D   & D*  \\
	\textbf{Ghast}				& - & -   & -   & -   & T   & T   & T   & D   & D   & D   \\
	\textbf{Wraith}				& - & -   &-    & -   & -   & T   & T   & T   & D   & D   \\
	\textbf{Mummia}				& - & -   &-    &-    & -   & -   & T   & T   & T   & D   \\
	\textbf{Spettro}			&-  & -   &-    &-    &-    & -   & -   & T   & T   & T   \\
	\textbf{Vampiro}			&-  & -   & -   &-    &-    &-    & -   & -   & T   & T   \\
	\textbf{Fantasma}			&-  & -   &-    & -   &-    &-    &-    & -   & -   & T   \\
	\textbf{Lich}				&-  & -   &-    & -   &-    &-    &-    & -   & -   & T   \\
\end{tabularx}

\bigskip

\flushleft \textit{Legenda}:
\flushleft T: 1d4 creature scappano per 1 minuto il più lontano possibile. Se aggredite rispondono all'attacco.

D: 1d4 creature vengono distrutte

D*: 2d4 creature vengono distrutte

\medskip

Le creature eseguono un Tiro Salvezza su Volontà a DC pari alla prova di Scacciare i Non Morti per resistere all'effetto.

L'Abilità è usabile un numero di volta al giorno pari alla Saggezza ma un non morto può essere influenzato solo una volta al giorno dall'effetto.

\begin{multicols}{2}

\subsection{Schivare trappole}\index{Schivare trappole}

\textbf{Tiri Salvezza}: +2 Riflessi, +1 Tempra

La \textbf{prima volta} che prendi l'Abilità requisito Destrezza 2 ottieni un bonus di +4 al Tiro Salvezza per evitare l'effetto delle trappole.

La \textbf{seconda volta} che prendi l'Abilità, requisito Competenza Armi 5, anche se la trappola non concede Tiro Salvezza la tua naturale propensione ad evitare i danni ti concede un Tiro Salvezza su Riflessi per dimezzare i danni.

E' anche possibile usare questa Abilità per evitare Attacco furtivo (TS Riflessi superiore a Tiro Colpire avversario)

La \textbf{terza volta} che prendi l'Abilità requisiti Competenza Armi 9, il Tiro Salvezza se riuscito ti permette di evitare qualsiasi effetto della trappola, se fisicamente possibile.

\subsection{Schivata prodigiosa}\index{Schivata prodigiosa}

\textbf{Tiri Salvezza}: +2 Riflessi

Come Reazione ad una Azione di attacco puoi aggiungere +2 alla tua Difesa. Puoi applicare il bonus dopo il Tiro per Colpire dell'avversario ma prima di sapere se ti ha colpito o meno.

Puoi usare l'Abilità fino a 3 volte al giorno.

\subsection{Seconda pelle}\index{Seconda pelle}

\textbf{Requisito}: Competenza Armi 1

\textbf{Tiri Salvezza}: +2 Tempra

Il costante utilizzo dell'armatura ti permette di indossarle senza grosse penalità.

Il malus alle prove di Destrezza dato dall'armatura diminuisce di 1.

La \textbf{seconda volta} che si prende questa Abilità, Competenza Armi 6, il malus alle prove di Destrezza diminuisce di ulteriori 1. Il malus alle penalità al movimento diminuisce di 1 metro.
Puoi dormire in armature medie senza essere affaticato la mattina.

La \textbf{terza volta} che si prende questa Abilità, Competenza Armi 11, il malus alle prove di Destrezza diminuisce di ulteriori 1. Il malus alle penalità al movimento diminuisce di un ulteriore 1 metro.
Puoi dormire in armature pesanti senza essere affaticato la mattina.

\subsection{Segugio}\index{Segugio}

\textbf{Requisito}: Intelligenza 1, Saggezza 1, Competenza Armi 1

\textbf{Tiri Salvezza}: +1 Riflessi, +1 Volontà

Hai un talento naturale per seguire le persone

Con due Azioni ti focalizzi su un target che puoi vedere e finché lo vedi rimani focalizzato. Tutte le tue Azioni che coinvolgono quel target hanno un +1 di bonus. Rimanere focalizzato costa 1 Azione per round.

La \textbf{seconda} volta che prendi questa Abilità, Competenza Armi 6, il bonus sale a +2.

La \textbf{terza} volta che prendi questa Abilità, Competenza Armi 12, il bonus sale a +3.

Il bonus può essere usato al Tiro per Colpire, Tiri Salvezza causati dall'avversatio, prove di competenza.. ma non al danno.

\subsection{Senso Trappola}\index{Senso Trappola}

\textbf{Requisito}: Intelligenza 2, Destrezza 3

\textbf{Tiri Salvezza}: +2 Riflessi

Hai un senso innato per trovare le trappole. Ti viene concesso una prova di Consapevolezza (costa 1 Reazione) nel passare entro 1 metro da una trappola (che farà il Narratore) .

La \textbf{seconda volta} che prendi l'Abilità il raggio aumenta fino a 3 metri e prendi un +2 alla prova.

La \textbf{terza volta} che prendi l'Abilità il raggio aumenta a 9 metri.

\subsection{Senza Traccia}\index{Senza Traccia}

\textbf{Requisito}: Passo sicuro

\textbf{Tiri Salvezza}: +2 Volontà, +1 Riflessi

La capacità di non lasciare impronte nell'ambiente scelto. Ogni volta che prendi questa Abilità puoi scegliere un ambiente diverso (vedi Abilità Passo Sicuro) di cui hai preso l'Abilità. La prova di Seguire Tracce per inseguirti ha una difficoltà aumentata di 10.

\subsection{Stai giù!}\index{Stai giù!}

\textbf{Tiri Salvezza}: +2 Tempra, +1 Volontà

Quando esegui due critici su un avversario la forza del tuo colpo è tale da metterlo prono. L'avversario deve fare un Tiro Salvezza Tempra (DC pari al danno) o cadere prono. L'Abilità funziona su creature di taglia pari o inferiore a quella del personaggio.

La \textbf{seconda volta} che prendi l'Abilità puoi influenzare anche creature di una taglia superiore.

La \textbf{terza volta} che prendi l'Abilità puoi influenzare anche creature di due taglie superiori.

\subsection{Tempesta di Furia}\index{Tempesta di Furia}

\textbf{Requisito}: Lista Pugno Vuoto 2, Destrezza 1, Forza 1

\textbf{Tiri Salvezza}: +2 Riflessi, +1 Volontà

Quando prendi questa Abilità puoi dichiarare di usare la Tempesta di Furia come tuo unico attacco (3 Azioni).

Fai un unico Tiro per Colpire e se vai a segno colpisci un numero di volte l'avversario pari a CA/4.

\begin{center}
\includegraphics[width=0.7\linewidth]{immagini/kenilguerriero.png}
\end{center}


\subsection{Tiro preciso}\index{Tiro preciso}

\textbf{Requisito}: Destrezza 3, Competenza Armi 1

\textbf{Tiri Salvezza}: +2 Riflessi

Guadagni un +1 colpire e +1 al danno ed al Tiro per Colpire, con armi da tiro o archi, entro 9 metri.

\subsection{Tiro rapido}\index{Tiro rapido}

\textbf{Requisito}: Destrezza 3, Tiro Preciso, Competenza Armi 2

\textbf{Tiri Salvezza}: +2 Riflessi

Le penalità per l'attacco multiplo sono inferiori.

Ogni arma lanciata oltre al primo prende un -4 al Tiro per Colpire cumulativo (e non il -5). La prima arma ha un Tiro per Colpire normale, la seconda ha un -4 , terza un -8 ...

\subsection{Toccata e fuga}\index{Toccata e fuga}

\textbf{Tiri Salvezza}: +2 Riflessi

Prendendo -5 al Tiro per Colpire all'Azione di Attacco, puoi effettuare un'Azione di 1 movimento in più (oltre le 3 Azioni standard). Costa 1 Azione Immediata.

\subsection{Tocco pietoso}\index{Tocco pietoso}

\textbf{Requisito}: Patrono buono, Imposizione delle mani, Tratti in comune con il Patrono 3

\textbf{Tiri Salvezza}: +2 Volontà, +1 Tempra

Il tuo tocco lenisce non solo le ferite ma anche le sofferenze e dolori. Ogni qual volta usi l'Abilità Imposizione delle mani puoi aggiungere anche questa Abilità come Azione Immediata.

Usando l'Imposizione delle mani puoi, rinunciando ad un numero di d6 curativi indicati, rimuovi le seguenti afflizioni.

In caso di afflizioni di origine magica esegui una prova di magia con bonus il punteggio dei Tratti in comune con il Patrono, il valore è da confrontare con la DC della afflizione

\textbf{2d6} Punteggio Tratti in comune 3: Affaticato - Scosso

\textbf{3d6} Punteggio Tratti in comune 6: Malato - Stordito - Confuso - Nauseato

\textbf{4d6} Punteggio Tratti in comune 9: Avvelenato - Impaurito - Con Punti Ferita Massimi ridotti

\textbf{5d6} Punteggio Tratti in comune 11: Rigenerante (arti) - Accecato - Sordo - Paralizzato - Pietrificato

\subsection{Vampiro}\index{Vampiro}

\textbf{Requisito}: Odore del sangue (Vantaggi)

\textbf{Tiri Salvezza}: +2 Tempra, +1 Volontà

La tua sete di sangue diventa cura. Il bonus di sete di sangue può aumentare fino a +5.

Se il bonus aumenta da +3 a +4 o +5 puoi, ingurgitando il sangue avversario, puoi curarti di 1d6 impiegando un 2 azioni

\begin{center}
	\includegraphics[width=0.7\linewidth]{immagini/Philip_Burne-Jones_-_The_Vampire4.jpg}
	
	\textit{Philip Burne Jones - The Vampire}
\end{center}


\subsection{Volontà Ferrea}\index{Volontà Ferrea}

Nel tempo hai allenato la tua volontà per resistere a qualsiasi debolezza e paura. La prima volta prendi questa Abilità ottieni un bonus di +2 ai Tiri Salvezza su Volontà. Il bonus è cumulativo, +2 la prima volta, +1 la seconda, +1 la terza ed ultima volta possibile


\pagebreak

\subsection{Raggruppamento Abilita' per Stile}

Per facilitare la transizione da chi viene da altri giochi di ruolo con classi sono qui suddivise le Abilità per le classi piu' canoniche.

Sono chiaramente solo indicazioni, in DBS il personaggio puo' essere costruito come meglio si preferisce e come la storia che vive lo sta istruendo.

Queste sono suggerimenti per facilitare la costruzione di un personaggio per chi e' alle prime armi con Dungeon Bell System, ovvio che sia possibile "pescare" da tutti i raggruppamenti presentati!

\end{multicols}

\addvspace{1.5cm}

\begin{multicols}{3}


\subsubsection{Guerriero}

Arma Focalizzata

Attacco turbinante

Colpi poderosi

Colpo Mortale

Combattere alla Cieca

Flagello Danzante

Ho detto CADI!

Iaijutsu

La mia Testa è più Dura

Persona veramente malvagia

Proseguire

Questa è la mia arma!

Seconda pelle

Stai giù!

\subsubsection{Barbaro}

Ferocia

Forgiato nella furia

Furia

La mia morte la tua morte

\subsubsection{Ladro}

Colpo furtivo

Colpo Indebolente

Colpo Paralizzante

Fare Infuriare

Freccia chiamata, freccia consegnata

Giocoliere

Improvvisare

Opportunista

Questo è il mio pugnale

Schivare trappole

Schivata prodigiosa

Senso Trappola

Toccata e fuga

\subsubsection{Paladino}

Imposizione delle mani (energia negativa o positiva a seconda dei Tratti)

Incanalare energia a distanza

Incanalare energia (energia negativa o positiva a seconda dei Tratti

Incanalare energia concentrata

Rappresaglia

Tocco pietoso

\subsubsection{Bardo}

Infondere Coraggio

Infondere Paura

\subsubsection{Ranger}

Arciere a cavallo

Combattimento con due armi

Difendere Cavalcatura

Doppia porzione

Occhio Clinico

Occhio di Falco

Passo sicuro

Più sono grossi più fanno rumore quando cadono

Ricarica rapida (Balestra)

Segugio

Senza Traccia

Tiro preciso

Tiro rapido

\subsubsection{Druido}

Animalia

Distillare pozioni

Il Patrono e' la mia magia

\subsubsection{Chierico}

Armatura del Devoto

Fedele

Il Patrono e' con me

Il Patrono e' la mia magia

Scacciare i non morti

\subsubsection{Mago/Stregone}

Animaletto / Famiglio

Batteria Magica

Creare Oggetti Magici

Creare Oggetti Magici Superiori

Creare Oggetti Magici Meravigliosi

Creare Oggetti Magici Mitici

Decifrare scritti magici

Il Patrono e' con me

Il Patrono e' la mia magia

Magie efficaci

Rilevare il Magico

Sapiente

\subsubsection{Monaco}

Ali della Fenice

Armatura della Montagna Incantata

Energia Psichica

Colpo Psichico

Raggio Psichico

Gru d'Argento

Montagna umana

Passo Veloce

Pugno di Ferro

Tempesta di Furia

\subsubsection{Gish (Guerriero/Mago)}

Armatura del Devoto

Guerriero della Magia

Incantare in Combattimento

Incantatore Prudente

Lo scudo è mio amico

Radici magiche

\subsubsection{Generali}

Esperto

Finta Morte

Immunità ai veleni

Iniziativa migliorata

Percettivo

Resistenza della pietra

Riflessi fulminei

Vampiro

Volontà Ferrea

\end{multicols}





\pagebreak

\section{Famiglio}\index{Famiglio}

\label{famiglio}
\begin{enfasi}{
Abbiamo imparato a volare come gli uccelli, a nuotare come i pesci, tuttavia non abbiamo imparato l'arte di vivere come fratelli. (Martin Luther King )}\end{enfasi}

\begin{multicols}{2}

\lettrine[lines=2, lhang=0.33, loversize=0.25, findent=1.5em]{I}{ famigli} sono animali scelti dal personaggio, tramite l'Abilità Famiglio, perché gli siano d'aiuto nelle avventure e per compagnia. Un famiglio ha un legame speciale con il suo padrone.

Un famiglio è un normale animale che mantiene aspetto, Dadi Vita, Competenza Armi, bonus ai Tiri Salvezza, Abilità del normale animale che era, ma viene trattato come creatura magica al fine di determinare qualsiasi effetto che dipenda dal suo tipo.

Solo un normale animale, non modificato, può diventare un famiglio.

Un famiglio conferisce delle Capacità Speciali al suo padrone. Queste Capacità Speciali si applicano solo quando il padrone e il famiglio sono entro 100 m l'uno dall'altro.

E' necessario un particolare rito di 4 ore nell'ambiente nativo dell'animale per farlo diventare un famiglio.

Se un famiglio viene congedato, perso oppure muore, può essere sostituito una settimana dopo con uno speciale rituale che costa 2 punti di Costituzione temporanea del personaggio. Per completare il rituale occorrono 8 ore.

\medskip


\textbf{Tabella: Tipi di Famiglio}\index{Tabella Tipi di Famiglio}

\medskip

\begin{tabularx}{0.53\textwidth}{lX}
	\textbf{Famiglio}    & \textbf{Capacità acquisita dal padrone}\\
		\toprule
	Lucertola   		 & bonus +2 alle prove di Sopravvivenza\\
	Rospo    	  		 & bonus +2 al Tiro Salvezza su Veleno\\
	Corvo                & bonus +2 alle prove di Intimidire\\
	Volpe 			     & bonus +1 al Tiro Salvezza su Riflessi\\
	Donnola			     & bonus +1 alle prove su Intelligenza\\
	Falco                & bonus +2 su Consapevolezza sulla vista\\
	Gatto                & bonus +2 alle prove di Muoversi Silenziosamente\\
	Gufo                 & bonus +2 su Consapevolezza sul udito\\
	Civetta              & bonus +2 alle prove di Arcana\\
	Lontra				 & bonus +2 alle prove di Nuotare\\
	Pipistrello          & bonus +2 alle prove di Acrobatica\\
	Riccio               & bonus +1 al Tiro Salvezza su Volontà\\
	Scimmia              & bonus +2 alle prove di Mani di Fata, Artista della Fuga\\
	Ratto                & bonus +1 al Tiro Salvezza su Tempra\\
	Topi			     & diventi il famiglio della Topi!!!\\
\end{tabularx}

\bigskip

Utilizzare le statistiche base di una creatura della specie del famiglio, apportando i seguenti cambiamenti.

\medskip

\textbf{Dadi Vita}: Ai fini degli effetti legati al numero dei Dadi Vita, utilizzare il punteggio di Competenza Magica del personaggio del padrone o il normale totale di DV del famiglio, quale dei due sia più alto.

\medskip

\textbf{Attacchi}: Utilizzare la Competenza Armi del padrone se più alta. Utilizzare il modificatore di Destrezza o Forza del famiglio, quale dei due sia più alto per calcolare il bonus di attacco del famiglio con gli Attacchi Naturali. Il danno è uguale a quello di una normale creatura della specie del famiglio.

\medskip

\textbf{Difesa}: il famiglio ha una Difesa pari a quello dell'animale standard più bonus dovuto alla CM del padrone. Vedi tabella Abilità del Famiglio.

\medskip

\textbf{Tiro Salvezza}: Per ogni Tiro Salvezza, utilizzare i bonus al Tiro Salvezza del famiglio (Tempra +2, Riflessi +2, Volontà +0) o quelli del padrone quali siano i migliori. Il famiglio applica i suoi valori di caratteristica come bonus ai Tiri Salvezza e non condivide nessuno dei bonus che il suo padrone potrebbe ricevere ai propri Tiri Salvezza.

\medskip

\begin{center}
	\includegraphics[width=0.8\linewidth]{immagini/donnadrago2.png}
	\textit{Henry Justice Ford}
\end{center}

\medskip

\textbf{Descrizione delle Capacità del Famiglio}

Tutti i famigli possiedono Capacità Speciali (oppure le attribuiscono ai loro padroni) a seconda del punteggio di Competenza Magica del padrone. Le capacità elencate nella tabella sono cumulative.

\end{multicols}

\bigskip
\bigskip


\textbf{Tabella: Abilità e Bonus del Famiglio}\index{Tabella Abilità del Famiglio}

\medskip

\begin{tabularx}{0.95\textwidth}{lllX}
	\textbf{CM del Padrone} & \textbf{Bonus Difesa} & \textbf{Bonus Intelligenza} & \textbf{Speciale}\\
	\toprule
	1-2      & +1       & 0    & Allerta, Condividere Incantesimi, \\
	&  		& 0    & Legame Empatico\\
	3-4      & +2       & +1   & Trasmettere Incantesimi a contatto\\
	5-6      & +3       & +1   & Parlare con gli Animali della Sua Specie\\
	7-8      & +4       & +1   & Parlare con il Padrone\\
	9-10     & +5       & +2   & \\
	11-12    & +6       & +2   & Vedere attraverso Famiglio\\
	13-14    & +7       & +2   & -\\
	15-16    & +8       & +3   & -\\
	17-18    & +9       & +3   & -\\
	19-20    & +10      & +3   & -\\
\end{tabularx}

\bigskip

\begin{multicols}{2}

\bigskip

\textbf{Competenza Magica del Padrone}: il numero indicato qui è il valore di Competenza Magica del padrone del famiglio, articolato in fasce.

\textbf{Modificatore Difesa Famiglio}: il numero indicato qui è in aggiunta alla Difesa del famiglio.

\textbf{Intelligenza}: il punteggio di Intelligenza del famiglio. Si tiene questo valore o quello del famiglio a seconda di quale sia più alto.

\textbf{Speciale}: le capacità speciali acquisite dal famiglio (e/o dal padrone).

\textbf{Allerta}: quando il famiglio è a portata di braccio dal padrone, questi guadagna +1 alle prove di Consapevolezza

\textbf{Condividere Incantesimi}: a propria discrezione, il padrone può lanciare qualsiasi Incantesimi che abbia effetto su di "sé" sul suo famiglio (come un Incantesimo a contatto), al posto di se stesso.

Il padrone può lanciare sul suo famiglio incantesimi anche se queste normalmente non hanno effetto su creature del tipo del famiglio (creature magiche).

\textbf{Legame Empatico}: il padrone ha un legame empatico con il suo famiglio fino a una distanza di 1 km. Il padrone non può vedere attraverso gli occhi del famiglio, ma può comunicare telepaticamente con esso. A causa della natura limitata del legame, si possono comunicare solo emozioni generiche.

\textbf{Trasmettere Incantesimi a Contatto}: il famiglio può trasmettere Incantesimi a contatto per lui. Se il padrone e il famiglio sono entro 9 metri quando il padrone lancia un Incantesimo a contatto, egli può designare il suo famiglio come "colui che consegna l'Incantesimo" (se a tocco).

Il famiglio può trasmettere l'Incantesimo proprio come il padrone. E' necessario che l'attacco del famiglio sia nello stesso round, ma successivamente come azione del lancio dell'Incantesimo.

\textbf{Parlare col Padrone}: il famiglio e il padrone possono comunicare verbalmente, come se utilizzassero un linguaggio comune. Le altre creature o animali non sono in grado di comprendere la loro conversazione, se non utilizzando ausili magici. La capacità funziona entro i 50m e devono sentirsi.

\textbf{Parlare con Animali della Sua Specie}: il famiglio è in grado di comunicare con animali della sua specie generica: pipistrelli con pipistrelli, topi con roditori, gatti con felini, falchi e gufi e corvi con uccelli, serpenti e lucertole con rettili, rospi con anfibi, scimmie con altri primati, donnole con ermellini e mustelidi... La comunicazione è limitata dal Intelligenza delle creature con cui il famiglio comunica.

\textbf{Vedere attraverso Famiglio}: il padrone può vedere attraverso il famiglio. Attivare questa Abilità costa 1 azione immediata. Il famiglio deve essere entro 30 metri.

\end{multicols}

\pagebreak

\section{Altre Abilita' speciali}

\label{altre-abilita-speciali}

\begin{multicols}{2}

\lettrine[lines=2, lhang=0.33, loversize=0.25, findent=1.5em]{Q}{este} Abilità non sono selezionabili da parte del giocatore, bensì possono essere innate nelle creature.

\subsection{Etereo}\index{Etereo}

\label{etereo}

Una creatura diventata Eterea è situata nel Piano Etereo che è sovrapposto a quello Materiale.

Una creatura eterea è Invisibile, senza sostanza e capace di muoversi in qualsiasi direzione, persino su e giù, ma solo a velocità dimezzata. Una creatura eterea può muoversi attraverso oggetti solidi, incluse altre creature viventi. Una creatura eterea può vedere e udire ciò che accade sul Piano Materiale, ma ogni cosa appare grigia ed effimera. La vista e l'udito di una creatura eterea che si trova sul Piano Materiale sono limitati a una distanza di 9 metri.

Gli Incantesimi se non opportunamente formulati e modificati non agiscono su creature eteree. Una creatura eterea ha Resistenza al Danno verso Luce o Vuoto, ed ignora tutte le altre forme di Energia.

Una creatura eterea non può attaccare una creatura materiale ed Incantesimi lanciati mentre ci si trova in condizione di etereo possono influenzare solo elementi eterei. Alcune creature o oggetti materiali hanno attacchi o effetti speciali che funzionano anche sul Piano Etereo. Una creatura eterea considera tutte le altre creature eteree come se tutti fossero materiali.

\subsection{Resistenza al Danno}\index{Resistenza al Danno}

Determinate creature o protezioni conferiscono la capacità di Resistere ad una tipologia di Danno.

Essere Resistenti al Danno significa automaticamente dimezzare il danno ricevuto prima di applicare qualsiasi altra protezione o Tiro Salvezza.

La Resistenza al Danno può assumere anche dei valori. Quando viene scritto Resistenza al Danno: Fulmine, il soggetto dimezza automaticamente i danni da elettricità, se scritto Resistenza al Danno: Filmine 10, significa che riduce il danno da elettricità di 10 punti prima di applicare il Tiro Salvezza o altri bonus.

Un creatura con una Resistenza al Fuoco dimezza (riduce) tutto il danno che riceve dalla fiamme, magiche o meno. 

Possono esistere abilità o incantesimi che ignorano questa Resistenza.

\subsection{Riduzione del Danno - DR}\index{Riduzione del Danno}

Determinate creature o Abilità conferiscono la capacità soprannaturale di resistere al danno di certe tipologie di armi o fino ad un certo ammontare (per attacco).

Solitamente assume il valore di XX/ZZ ovvero quanto danno (XX) è ignorato se non si è attaccati con (ZZ). Ignorare il danno significa anche che effetti connessi all'attacco non funzionano, come veleni sull'arma.

E' applicabile un'unica DR in caso ce ne siano di più di una contemporanea, la scelta va fatta ad inizio scontro e rimane la stessa finché non è finito lo scontro.

\begin{center}
	\includegraphics[width=0.8\linewidth]{immagini/morteachille.png}
	
\textit{Paris shot Achilles with an arrow - Pieter Paul Rubens - Data 1630-1632}
\end{center}


Determinate armi, particolarmente magiche possono ignorare la DR \index{Ignorare la DR}

\medskip

\textbf{Tabella: Equivalenza Armi Magiche}\index{Tabella Equivalenza Armi Magiche}


\begin{tabular}{lll}
	\textbf{DR da superare} & \textbf{Incantamento} & \textbf{Attacco}\\
	& \textbf{arma}& \textbf{Naturale}\\
	\toprule
	Incantamento +1         & +1              & Livello 3\\
	Incantamento +2         & +2              & Livello 6\\
	Ferro Freddo  			& +2              & Livello 9\\
	Argento 				& +2              & Livello 9\\
	Adamantio               & +3              & Livello 12\\
\end{tabular}

\medskip

\textbf{Proiettili (frecce, dardi, sassi) tirati da armi magiche NON sono considerati magiche.}\index{Freccie magiche}


\subsection{Vulnerabilità al Danno}\index{Vulnerabilità al Danno}

Determinate creature o magie rendono più efficaci alcuni effetti causando maggiore danno al soggetto vulnerabile.

Essere Vulnerabili ad un tipo specifico di Danno significa automaticamente raddoppiare il danno ricevuto prima di applicare qualsiasi altra protezione o Tiro Salvezza.

Un creatura con una Vulnerabilità al Fuoco raddoppia tutto il danno subito poi se possibile effettua il Tiro Salvezza indicato dall'incantesimo o effetto.

\subsection{Resistenza alla Magia}\index{Resistenza alla Magia}

Una creatura potrebbe avere una naturale resistenza alla magia.

Il valore di RM (Resistenza Magia) indica tale resistenza e più è alta più la creatura è immune alla magia, che lo voglia o meno.

Ogni qual volta la creatura è influenzata direttamente da un incantesimo o effetto magico deve effettuare una prova di RM, ovvero tirare 3d6 sommare il valore di RM e se è superiore alla prova di magia effettuata dall'incantatore l'incantesimo non ha effetto.

In caso di magie scaturite da oggetti (anelli, bastoni, pozioni) la prova di RM deve superare la Difficoltà dell'incantesimo generato per annullarne gli effetti.

\subsection{Paura}\index{Paura}

\begin{center}
\includegraphics[width=0.8\linewidth]{immagini/the-scream.png}

\textit{L'urlo (titolo originale: Skrik)\\Edvard Munch - Data 1893–1910}
\end{center}


\label{paura}

Incantesimi, Oggetti Magici e certe creature possono influenzare i personaggi con paura. In molti casi, il personaggio deve effettuare un Tiro Salvezza su Volontà per resistere agli effetti, e un tiro fallito indica che il personaggio è scosso, spaventato o in preda al panico.

\textbf{Scosso}\index{Scosso}

I personaggi che sono scossi subiscono penalità di -2 ai Tiri per Colpire, ai Tiri Salvezza e alle prove.

\textbf{Spaventato}\index{Spaventato}

I personaggi spaventati sono anche scossi, e inoltre fuggono dalla fonte della loro paura il più velocemente possibile, anche se possono scegliere la direzione di fuga. A parte ciò, una volta che sono fuori vista (o udito) dalla fonte della loro paura, possono agire normalmente. Se la durata della paura non è ancora arrivata al termine, qualora dovessero incontrare di nuovo la fonte della loro paura, cercherebbero nuovamente di fuggire. I personaggi che non sono in grado di fuggire possono combattere (anche se continuano ad essere scossi).

\textbf{In Preda al Panico}\index{In Preda al Panico}

I personaggi in preda al panico sono scossi e, inoltre, hanno una probabilità del 50\% di far cadere a terra qualsiasi cosa stanno tenendo in mano, e di fuggire dalla fonte del loro terrore il più in fretta possibile seguendo un percorso di fuga completamente casuale. I personaggi in preda al panico fuggono davanti a qualsiasi altro pericolo che possano trovarsi di fronte.

A parte ciò, una volta che sono fuori vista (o udito) dalla fonte della loro paura, possono agire normalmente. I personaggi in preda al panico prendono anche la condizione Accovacciato se non possono fuggire.

\textbf{Terrore Crescente}\index{Terrore Crescente}

Gli effetti della paura sono cumulativi. Un personaggio scosso che viene nuovamente scosso diventa spaventato, mentre invece un personaggio scosso che viene spaventato cade in preda al panico. Un personaggio spaventato che viene scosso o spaventato cade in preda al panico.


\subsection{Paralizzato}\index{Paralizzato}

\label{paralizzato}

Un personaggio paralizzato è bloccato sul posto ed è incapace di muoversi od agire

Ha punteggi effettivi di Forza e Destrezza pari a -4 (-4 alla Difesa oltre ad non avere bonus di Destrezza), è Indifeso e può compiere azioni esclusivamente mentali. Una creatura alata in volo, nel momento in cui viene paralizzata non può più battere le ali e precipita. Un nuotatore paralizzato non può più Nuotare e potrebbe annegare.

\end{multicols}

\pagebreak

\section{La Magia}\index{Magia}\index{Essenza}


\label{la-magia}
\begin{enfasi}{
Le parole sono, nella mia NON modesta opinione, la nostra massima ed inesauribile fonte di magia. In grado sia di infliggere dolore che di alleviarlo (Albus Silente)

\medskip

Non lascerai vivere colei che pratica la magia. (Libro dell'Esodo)(Sempre a seconda dei propri Tratti...)} \end{enfasi} \medskip


\begin{multicols}{2}

\lettrine[lines=2, lhang=0.33, loversize=0.25, findent=1.5em]{L}{a} magia permea i mondi di gioco e la sua forma più comune è quella di un incantesimo. Questo capitolo fornisce le regole per lanciare incantesimi. 

\medskip

\subsection{Cos'è un Incantesimo?}

Un incantesimo è un preciso effetto magico, una singola alterazione delle energie magiche che permeano il multiverso in una specifica, limitata, espressione. Nel lanciare un incantesimo, un personaggio tira attentamente i fili invisibili della magia pura che i Patroni hanno concesso e li ricuce in una trama particolare, li fa vibrare in un modo specifico, e poi li libera per scatenare l'effetto desiderato: tutto ciò, nella maggior parte dei casi, in pochi secondi.

Gli incantesimi possono essere strumenti versatili, armi o barriere protettive. Possono infliggere danni o ripararli, imporre o rimuovere condizioni, risucchiare l'energia vitale e ridare vita ai morti (se permesso!).

Nel corso della storia del multiverso sono stati creati innumerevoli migliaia di incantesimi, molti dei quali sono andati dimenticati. Alcuni possono ancora essere nascosti tra le pagine di libri degli incantesimi impolverati all'interno di antiche rovine o segregati nella mente di divinità morte. Oppure potrebbero un giorno essere reinventati da un personaggio che abbia ammassato sufficiente potere e capacità per farlo.

\subsection{Le caratteristiche degli incantesimi}

La descrizione di ciascun incantesimo inizia con un blocco di informazioni che comprende la Difficoltà, scuola di magia, tempo di lancio, gittata, componenti e durata dell'incantesimo. Il resto della descrizione ci informa dell'effetto dell'incantesimo. 

Quando un personaggio lancia qualsiasi incantesimo, si usano le seguenti regole base indipendentemente dall'effetto dell'incantesimo.

\medskip

\textbf{Tempo di Lancio}

La maggior parte degli incantesimi possono essere lanciati con due Azioni. Alcuni incantesimi richiedono un'Azione Immediata, una Reazione o molto più tempo per essere lanciati.

\textbf{Azione Immediata}

Un incantesimo lanciato con un'Azione Immediata è particolarmente rapido. Puoi usare un'Azione Immediata durante il tuo round per lanciare l'incantesimo che sia Immediato, purché tu non abbia già effettuato un'Azione Immediata durante il tuo round. Durante lo stesso round non puoi lanciare un altro incantesimo, a meno che non si tratti di un incantesimo a Difficoltà 12 (chiamati trucchetti).


\includegraphics[width=0.7\linewidth]{immagini/Hex32.jpg}
	
\textit{The Witchcraft Art of Jacques de Gheyn II}


\textbf{Reazioni}

Alcuni incantesimi possono essere lanciati come Reazioni. Questi incantesimi richiedono una frazione di secondo per essere creati e possono essere lanciati in risposta a un evento. Se un incantesimo può essere lanciato come reazione, la descrizione dell'incantesimo ti dice esattamente quando puoi farlo. Devi avere a disposizione una Reazione e non averla già usata.

\textbf{Tempo di Lancio Più Lungo}

Certi incantesimi richiedono più tempo per essere lanciati: minuti o addirittura ore. Quando lanci un incantesimo con tempo di lancio più lungo di due Azioni, devi spendere una azione ogni round per lanciare l'incantesimo, e nel farlo devi mantenere anche la concentrazione (vedi "Concentrazione" di seguito). Se la tua concentrazione viene infranta, l'incantesimo fallisce, ma non avrai speso lo slot incantesimo. Se vuoi tentare di lanciare l'incantesimo di nuovo, dovrai ricominciare da capo.

\begin{center}
	\includegraphics[width=0.8\linewidth]{immagini/infanticidalwitch.jpg}
	
	\textit{The Witchcraft Art of Jacques de Gheyn II}
\end{center}

\medskip

\textbf{Le Scuole di Magia}\index{Le Scuole di Magia}

Le accademie di magia raggruppano gli incantesimi in nove categorie dette scuole di magia. Gli studiosi applicano queste categorie a tutti gli incantesimi, credendo che tutta la magia funzioni essenzialmente allo stesso modo, che derivi da uno studio rigoroso e venga conferita da un Patrono.

Le scuole di magia aiutano a descrivere gli incantesimi; non hanno delle proprie regole, sebbene alcune regole possano fare riferimento a queste scuole.

\begin{itemize}
\item
\textit{Abiurazione} riguarda incantesimi di natura protettiva, sebbene ne contenga anche alcuni dall'uso aggressivo. Questi incantesimi creano barriere magiche, negano effetti dannosi, danneggiano i violatori, o bandiscono le creature in altri piani di esistenza.

\item
\textit{Ammaliamento} riguarda incantesimi che agiscono sulla mente altrui, influenzandone o controllandone il comportamento. Questi incantesimi possono far sì che i nemici considerino l'incantatore un amico, forzare creature a effettuare determinate azioni, o addirittura controllare un'altra creatura come fosse una marionetta.

\item
\textit{Cura} riguarda gli incantesimi che permettono di recuperare le energie fisiche, mentali ed annullare debolezze e veleni.

\item
\textit{Divinazione} riguarda incantesimi che rivelano informazioni nella forma di segreti da tempo dimenticati, visioni del futuro, la posizione di oggetti nascosti, la verità dietro le illusioni o immagini di persone e luoghi lontani.

\item
\textit{Evocazione} riguarda incantesimi che trasportano oggetti e creature da un luogo all'altro. Alcuni incantesimi richiamano creature o oggetti al fianco dell'incantatore, mentre altri permettono all'incantatore di teletrasportarsi da un luogo a un altro. Alcune evocazioni creano oggetti o effetti dal nulla. 

\begin{center}
	\includegraphics[width=0.8\linewidth]{immagini/Leonids-1833.jpg}
	
	\textit{The most famous depiction of the famous 1833 Leonids \textbf{Meteor Storm}.}
\end{center}

\textit{Illusione} riguarda incantesimi che ingannano i sensi e la mente altrui. Fanno vedere alle persone cose che non esistono, non gli fanno notare le cose che esistono, fanno udire rumori fasulli o ricordare cose che non sono mai accadute. Alcune illusioni creano immagini spettrali che chiunque può vedere, ma le illusioni più insidiose impiantano un'immagine direttamente nella mente di una creatura.

\item
\textit{Invocazione} riguarda incantesimi che manipolano l'energia magica per produrre un effetto desiderato. Molti incantesimi creano esplosioni di fuoco o fulmini.

\item
\textit{Necromanzia} riguarda incantesimi che manipolano le energie della vita e della morte. Questi incantesimi possono conferire una riserva aggiuntiva di forza vitale, risucchiare l'energia vitale da un'altra creatura, creare non morti o addirittura riportare in vita i morti (se concesso).

Creare non morti tramite l'uso di incantesimi di necromanzia come animare morti non è un'azione buona, e solo gli incantatori malvagi fanno frequentemente uso di questo incantesimo.

In DBS solo un Patrono ha sufficiente potere per poter riportare in vita un morto.

\item
\textit{Trasmutazione} riguarda incantesimi che cambiano le proprietà di una creatura, oggetto o ambiente. Possono trasformare un nemico in una creatura innocua, aumentare la forza di un alleato, far spostare un oggetto agli ordini dell'incantatore o potenziare le capacità di guarigione innate di una creatura per farle recuperare più in fretta da una ferita.

\item
\textit{Universale} alcuni incantesimi sono capisaldi della magia in se e come tali accessibili a tutte le scuole e maghi.

\end{itemize}

\begin{narratoresmall}{In DBS i giocatori hanno libero accesso a tutti gli incantesimi con quindi la possibilità di avere una lista estremamente variegata e potente. Per fare in modo che le scelte siano diverse tra i vari giocatori e tra le avventure, insistete sulle componenti, fate in modo che siano indispensabili al personaggio e siano usate. Fate anche in modo che gli incantesimi nuovi siano non facili da trovare come alla stregua di un tesoro.
}\end{narratoresmall}


\medskip

\textbf{Gittata}\index{Gittata Incantesimi}

Il bersaglio di un incantesimo deve essere nella gittata dell'incantesimo. Per un incantesimo come dardo incantato, il bersaglio è una creatura. Per un incantesimo come palla di fuoco, il bersaglio è il punto nello spazio da cui la sfera di fuoco esplode. La maggior parte degli incantesimi hanno una gittata espressa in metri. Alcuni incantesimi possono prendere a bersaglio solo una creatura (te compreso) con cui sei in contatto fisico. Altri incantesimi, come l'incantesimo scudo, agiscono solo su di te: questi incantesimi hanno come gittata personale.

Gli incantesimi che creano coni o linee di effetto che originano da te, hanno anch'essi gittata personale, a indicare che sei tu il punto di origine dell'effetto dell'incantesimo (vedi "Aree di Effetto" più avanti in questo capitolo).

Una volta lanciato l'incantesimo, i suoi effetti non sono più limitati dalla sua gittata, a meno che la descrizione dell'incantesimo non dica altrimenti.

\medskip

\textbf{Componenti}\index{Componenti Incantesimi}

Le componenti di un incantesimo sono i requisiti fisici che devi soddisfare per lanciarlo. La descrizione di ciascun incantesimo indica se richieda componenti verbali (V), somatiche (S) o materiali (M). Se non sei in grado di fornire una o più delle componenti dell'incantesimo, non potrai lanciarlo.

La maggior parte degli incantesimi richiede di intonare parole mistiche. Le parole, il ritmo, la cadenza e risonanza che mettono in moto i filamenti della magia. Di conseguenza, un personaggio imbavagliato o in un'area di silenzio, come quella creata dall'incantesimo silenzio, non può lanciare incantesimi con componenti verbali.

\medskip

\begin{center}
	\includegraphics[width=0.8\linewidth]{immagini/merlin.png}
	
	\textit{Merlin dictating his prophecies to his scribe. Robert de Boron's Merlin en prose (written ca 1200)}
\end{center}

\medskip

\textbf{Lanciare Incantesimi in Armatura}

Data la concentrazione mentale e i gesti precisi richiesti l'armatura distrae e sbilancia i flussi. La Difficoltà di lancio dell'incantesimo aumenta come indicato nella tabella delle armature.\\

\medskip

\textbf{Durata}\index{Durata Incantesimi}

La durata di un incantesimo è la lunghezza di tempo per cui esso persiste. La durata può essere espressa in round, minuti, ore o addirittura anni. Alcuni incantesimi specificano che i loro effetti durano finché l'incantesimo non viene dissolto o distrutto.

\begin{itemize}
	
\item
\textit{Istantanea}

Molti incantesimi sono istantanei. L'incantesimo ferisce, cura, crea o altera una creatura o un oggetto in modo che non possa essere dissolto, dato che la sua magia esiste solo per un istante.

\item

\textit{Concentrazione}

Alcuni incantesimi richiedono che tu mantenga la concentrazione per tenerne la magia attiva. Se perdi la concentrazione, l'incantesimo avrà fine. Se un incantesimo deve essere mantenuto tramite concentrazione, la cosa è indicata alla voce Durata, e l'incantesimo specifica quanto a lungo vi potrai mantenere la concentrazione. Puoi terminare la concentrazione in qualsiasi momento (senza usare un'azione).

Normali attività, come muoversi e attaccare, non interferiscono con la concentrazione. Mantenere la concentrazione costa 1 Azione a round.
\end{itemize}

\medskip

\textbf{Somatica (S)}

La gestualità del lancio di un incantesimo può includere un gesticolare forzato o intricate serie di gesti. Se un incantesimo richiede una componente somatica, l'incantatore deve essere libero di usare almeno una mano per eseguire questi gesti.

\medskip

\textbf{Materiale (M)}

Lanciare taluni incantesimi richiede oggetti particolari,specificati tra parentesi alla voce componenti. Un personaggio può usare una borsa dei componenti o un focus di incantamento al posto delle componenti specificate da un incantesimo. Se la componente ha un costo indicato, il personaggio deve procurarsi quella specifica componente prima di poter lanciare l'incantesimo.

Se un incantesimo indica che la componente materiale viene consumata dall'incantesimo, l'incantatore deve fornire questa componente per ogni lancio dell'incantesimo.
Un incantatore deve avere una mano libera per avere accesso a queste componenti, ma può essere la stessa mano utilizzata per eseguire le componenti somatiche.

\medskip

\textbf{Lanciare un altro incantesimo che richiede concentrazione}

Perdi la concentrazione su di un incantesimo se lanci un altro incantesimo che richieda concentrazione. Non ti puoi concentrare su due incantesimi alla volta. 


\medskip

\textbf{Essere inabile o ucciso}

Perdi la concentrazione su di un incantesimo se svieni o muori. Se scendi sotto gli zero PF perdi uno slot di incantesimi da lanciare per quel giorno.

\medskip

Il Narratore può anche decidere che certi fenomeni ambientali, come un'onda che si abbatte su di te mentre sei sul ponte di una nave in mezzo a una tempesta, una forte grandinata... siano situazioni di disturbo per lanciare un incantesimo.

\medskip

\textbf{Bersagli}\index{Bersagli Incantesimi}

Un normale incantesimo richiede che tu scelga uno o più bersagli che siano affetti dalla sua magia. La descrizione dell'incantesimo ti dice se l'incantesimo prende a bersaglio creature, oggetti o un punto di origine per generare un'area di effetto (descritta di seguito).A meno che l'incantesimo non abbia un effetto percepibile, una creatura potrebbe non capire mai di essere stata bersaglio di un incantesimo. Un effetto come un fulmine crepitante è palese, ma un effetto più subdolo, come il tentativo di leggere i pensieri di una creatura, di solito non viene notato, a meno che l'incantesimo non dica altrimenti.

Lanciare un incantesimo è una azione che non passa inosservata. Una prova di Nascondersi a difficoltà 13 oppure lanciare l'incantesimo come se si fosse Distratto (+5 alla Difficoltà) permettono di celare il lancio, se non avviene proprio davanti al bersaglio.

\textbf{Traiettoria Sgombra Verso il Bersaglio}

Per prendere come bersaglio una cosa, devi avere la traiettoria sgombera verso di essa, e quindi questa non può trovarsi dietro una copertura totale. Se piazzi un'area di effetto in un punto che non puoi vedere e un'ostruzione, come un muro, si trova tra di te e quel punto, il punto di origine si crea dal lato più vicino dell'ostruzione.

\textbf{Prendere Te Stesso Come Bersaglio} 

Se un incantesimo prende come bersaglio una creatura a tua scelta, puoi scegliere anche te stesso, a meno che la creatura non debba essere ostile o sia specificato che non possa essere tu. Se ti trovi nell'area di effetto di un incantesimo lanciato da te, anche tu ne sarai influenzato.

\medskip

\textbf{Aree di Effetto}\index{Aree di Effetto Incantesimi}

Incantesimi come mani brucianti e cono di freddo coprono un'area, permettendogli di colpire più creature alla volta.

La descrizione di un incantesimo specifica la sua area di effetto, che di solito rientra in una di queste cinque forme: cilindro, cono, cubo, linea o sfera. Ogni area di effetto ha un punto di origine, un luogo da cui erutta l'energia dell'incantesimo. Le regole per ciascuna forma specificano come posizionare il suo punto di origine. Di solito il punto di origine è un punto nello spazio, ma alcuni incantesimi hanno un'area la cui origine è una creatura o un oggetto.

\begin{itemize}
\item
\textit{Cilindro}

Il punto di origine di un cilindro è il centro di un cerchio di specifico raggio, come indicato nella descrizione dell'incantesimo. Il cerchio deve essere sul pavimento o all'altezza dell'effetto dell'incantesimo. L'energia in un cilindro si espande in linee dritte dal punto di origine al perimetro del cerchio, formando la base del cilindro. L'effetto dell'incantesimo parte poi dal basso verso l'alto o dall'alto verso il basso, fino a una distanza uguale all'altezza del cilindro.Il punto di origine del cilindro è incluso nella sua area di effetto.

\item

\textit{Cono}

Un cono si estende in una direzione a tua scelta dal suo punto di origine. L'ampiezza di un cono in un dato punto della sua lunghezza è uguale alla distanza di quel punto dal punto di origine. L'area di effetto di un cono specifica la sua lunghezza massima. Il punto di origine del cono non è incluso nella sua area di effetto, a meno che tu non decida altrimenti.

\item
\textit{Cubo}

Selezioni il punto di origine di un cubo, su cui si piazzerà una delle facce dell'effetto cubico. La taglia del cubo viene espressa come lunghezza di ciascun suo spigolo. Il punto di origine del cubo non è incluso nella sua area di effetto, a meno che tu non decida altrimenti.

\item
\textit{Linea}

Una linea si estende dal suo punto di origine in un percorso dritto per tutta la sua lunghezza e copre un'area definita dalla sua larghezza. Il punto di origine della linea non è incluso nella sua area di effetto, a meno che tu non decida altrimenti.

\item
\textit{Sfera}

Selezioni il punto di origine di una sfera, e la sfera si estenderà da quel punto. La taglia della sfera è espressa come raggio in metri che si estende da quel punto.
Il punto di origine della sfera è incluso nella sua area di effetto.

\end{itemize}


\begin{center}
	\includegraphics[width=0.6\linewidth]{immagini/3dforme.png}
	\textit{Cono, Sfera, Cilindro, Cubo}
\end{center}

\textbf{Combinare Effetti Magici}\\
Gli effetti di incantesimi diversi si sommano fino a che la loro durata si sovrappone. Tuttavia, gli effetti dello stesso incantesimo lanciato più volte sullo stesso bersaglio non si combinano. Sarà invece l'incantesimo più potente fra quelli lanciati (per esempio, quello con il bonus più alto) ad applicarsi finché le durate si sovrappongono.\\

\subsection{Regole di base...}

\begin{itemize}

\item
Il mago al lancio del suo primo incantesimo sceglie se utilizzare come modificatore alla prova di Competenza Magica l'Intelligenza oppure se e' un Devoto puo' scegliere la Caratteristica indicata dal Patrono.

Una volta fatta la scelta non e' più possibile cambiarla. Questo modificatore viene chiamato \emph{modificatore di caratteristica da incantatore}.
\item 
Ogni volta che il mago acquisisce un punto in Competenza Magica puo' apprendere due nuovi incantesimi che abbia a disposizione nel suo Tomo. Questi possono avere Difficoltà massima pari a 12 + Competenza Magica + modificatore di caratteristica da incantatore + eventuale bonus di scuola preferita dato dal Patrono (se Seguace o Devoto).
\item
Il personaggio quando assegna il primo punto di Competenza Magica apprende due incantesimi + metà del modificatore di caratteristica/2 (arrotondato per difetto)
\item
Ogni volta che il mago acquisisce un punto in Competenza Magica puo' rinunciare ad apprendere un incantesimo con Difficoltà pari o superiore a 16 per apprendere due Trucchetti (Difficoltà 12).
\item 
Ogni volta che il mago acquisisce un punto in Competenza Magica è possibile dimenticare un incantesimo appreso e sostituirlo con un altro a disposizione nel Tomo.
\item 
Per poter lanciare un incantesimo il mago deve superare la Difficoltà dell'incantesimo con la prova di Competenza Magica.

La prova si effettua con 3d6+Competenza Magica+modificatore di caratteristica da incantatore ed eventuali bonus dati da Abilità e dal Patrono.
\item 
Un incantatore può formulare nel giorno un numero di magie pari a Competenza Magica/2 (arrotondato per difetto, con un minimo di 1)+modificatore di caratteristica da incantatore. Questo numero viene anche chiamato slot incantesimo.
\item
Un Devoto aggiunge +4 alla Prova di Magia nelle scuole preferite dal Patrono e puo' sostituire le forme di energia degli incantesimi con quelle preferite dal Patrono
\item
Gli incantesimi con Difficoltà 12 (detti anche trucchetti) non contano nel numero delle magie lanciate nel giorno. Va comunque fatta la prova di competenza finché Competenza Magica non e' maggiore di 4.
\item
Rispettivamente a Competenza Magica 9, 12, 14 al giocatore non e' piu' richiesto fare una prova di magia per lanciare incantesimi con Difficoltà 16, 19 e 21.
\item
Se nel lancio di un incantesimo ottiene almeno un critico (due volte 6 nel lancio dei dadi) non si computa questa magia nel numero di incantesimi lanciabili al giorno.
\item
Se sei un Seguace ottieni +2 alle prove di magia nella scuola preferita dal Patrono
\end{itemize}


\subsubsection{Il Tomo della Magia}\index{Il Tomo della Magia}

Se i Patroni garantiscono l'accesso alla fonte della magia è solo l'applicazione di antichi riti e formule che permette di manifestare questa energia grezza in una forma ed espressione che chiamiamo incantesimo.

Ogni usufruitore di magia ha un proprio \textbf{Tomo} degli incantesimi, non pensate solo a un grosso tomo antico rilegato in pelle, le diverse culture hanno sviluppato nel tempo la capacità di iscrivere le rune degli incantesimi in carte, bastoni, lastre di pietra, tatuaggi... fate la vostra scelta quando create il personaggio.
Questa scelta non vi impedirà di copiare incantesimi da \textbf{Tomi} fatti diversamente (foglie di tabacco, liquidi della conoscenza..) per voi sarà sempre facile (Arcana DC 12) capire se si e' di fronte ad un Tomo di qualche tipo, semplicemente il vostro "metodo" di studio (o tradizione) vi ha insegnato un sistema diverso.

Il personaggio con Competenza Magica 1 partirà con 2 + modificatore da caratteristica per incantesimi di magie a sua scelta (oppure dei trucchetti) trascritti nel suo "tomo" e ogni altro incantesimi che vorrà imparare dovrà trovarlo e trascriverlo sul suo libro.

\includegraphics[width=0.8\linewidth]{immagini/spellbook.jpg}

Ogni incantesimi occupa un numero di pagine nel tomo pari alla propria (Difficoltà-10)/2 (arrotondando per eccesso), copiare una pagina di incantesimo porta via 1 ora di lavoro e 10 mo di preziosi inchiostri.

Un Tomo (libro) di incantesimi costa 10 mo per pagina che puo' custodire.

Un incantatore puo' copiare sul suo tomo incantesimi fino ad una Difficoltà pari a 14 + Competenza Magica + modificatore di caratteristica da incantatore + eventuale bonus di scuola preferita dato dal Patrono (se Seguace o Devoto). Se l'incantesimo e' di piu' alta Difficoltà il mago deve fare una prova di magia a quel grado di Difficoltà e solo se passa la prova potrà copiare l'incantesimo nel tomo.

Se la prova fallisce non riuscirà a copiare quell'incantesimo fino al prossimo punto di Competenza Magica acquisito. Se la prova fallisce con un risultato inferiore alla metà della Difficoltà dell'incantesimo, accadranno brutte cose al Tomo e 1d4 incantesimi verranno cancellati dal tomo stesso.

La sorgente di nuovi incantesimi puo' essere un altro tomo, bastone, pergamena.. insomma qualsiasi cosa che il precedente mago usasse per custodire gli incantesimi. Un oggetto magico (bastone magico, anello, verga..bacchetta..) non e' idoneo quale fonte da cui copiare l'incantesimo che contiene, si deve copiare dall'equivalente tomo di un altro mago.

Durante le avventure il vostro mago potrà copiare tanti e numerosi incantesimi sul suo tomo ma non potrà apprenderli immediatamente. Il personaggio quando acquisirà un nuovo punto di Competenza Magica potrà dimenticare un incantesimo appreso per sostituirlo con un incantesimo presente nel suo tomo.

\medskip

\begin{tcolorbox}[width=0.43\textwidth,title = Scegliere gli Incantesimi]
Avete letto bene, gli incantesimi non si apprendono da soli, non si scelgono da una lista. Ogni incantesimo e' un'arte preziosa che si impara e si deve trovare.	
	
I personaggi dovranno intraprendere perigliose avventure, pagare mercenari, cercare i tomi antichi e svelare i segreti piu' oscuri e dimenticati per poter imparare nuovi incantesimi.
	
Ogni incantesimo e' alla stregua di un oggetto magico, un vero tesoro da cercare!
\end{tcolorbox}

\medskip

Tramite un rito magico difficile e costoso potrà per un solo giorno sostituire 1 incantesimo con altro presente nel tomo. Questo rito, della durata di 1 ora per Difficoltà e dal costo di 10 mo sempre Difficoltà dell'incantesimo, permetterà per le 24 ore successive di sostituire un incantesimo appreso con uno presente nel proprio tomo e apprendibile data la Difficoltà.

\medskip

\begin{narratoresmall}
	Gli incantesimi diventano oggetti e premi magici a tutti gi effetti. Sfruttate la sete di conoscenza e potere dei personaggi per costruire avventure interessanti che possano ruotare attorno tomi antichi e leggendari incantesimi perduti.
\end{narratoresmall}

\subsubsection{Tiro per Colpire con le Magie}\index{Tiro per Colpire con Incantesimi}

Quando l'incantesimo ti dice di fare un Tiro per Colpire se e' un "\textbf{attacco a distanza con incantesimo}" devi effettuare un Tiro per Colpire contro la Difesa dell'avversario. Questo Tiro per Colpire e' effettuato con 3d6+\textbf{Competenza Magica}+Destrezza più Abilità e modificatori vari.\\

\medskip

Se il Tiro \textbf{per Colpire con incantesimo e' in mischia} eseguirai un Tiro per Colpire con il bonus di Forza (3d6+\textbf{Competenza Magica}+Forza) ed Abilità e modificatore vari contro la Difesa dell'avversario.

\medskip

Quando la magia e' ad area non e' necessario effettuare un Tiro per Colpire se non per difficili e specificate aree, ovvero si mira in una area ben circoscritta o si vuole evitare di colpire qualcuno con un incantesimo ad area.

\medskip

\begin{tcolorbox}[width=0.85\linewidth,title=Personaggi Gish (Guerrieri Maghi)]
Usando un unica Competenza Attiva per i Tiri per Colpire e usare la Magia e' piu' facile costruire personaggi che usino la magia per combattere, ma non che combattano anche con la magia...

Per quanto il mago sarà bravo a colpire ed esperto in incantesimi, con un arma in mano i suoi colpi non avranno mai la precisione che con la magia.

Distribuite con attenzione i punti delle Competenze Attive, un punticino in Competenza Armi si rileverà sempre molto utile\end{tcolorbox}


\subsubsection{L'esplosione del 6 nella Magia}\index{Esplosione del 6 nella Magia}

Anche nella prova di Competenza Magica i 6 esplodono, i 6 tirati nella prova di magia vengono ritirati, e ritirati ancora nel caso, e sommati alla prova di magia. Per le prove di Competenza Magica l'uno non viene conteggiato, conta 0.

Tenete traccia di quanti critici (due 6 tirati) fate, potrebbero servire per ottenere effetti "speciali" nell'incantesimo! Ricordate che se effettuate un critico l'incantesimo non si conta tra gli incantesimi lanciati nel giorno.


\begin{tcolorbox}[width=0.85\linewidth,title = La Prova di Magia] DBS e' strutturato perché sia difficile lanciare gli incantesimi, specialmente quelli di massima Difficoltà, mediamente un tiro di 12 o piu' sui dadi e' necessario per passare la prova a lanciare la magia. Si presume però che il personaggio che intraprenda la strada della magia continui a perseguirla, detta diversamente con valori di Competenza Magica bassa non aspettatevi di poter lanciare facilmente incantesimi di Difficoltà elevata!	
\end{tcolorbox}

\subsubsection{Tentare la sorte con la Magia}\index{Tentare la sorte con la Magia}

Anche nella prova di competenza magica puoi Tentare la Sorte, ovvero rinunci ad un +4 di bonus (solo tra Competenza Magica, da modificatore di caratteristica da incantatore e da bonus concesso da Patrono) e aggiungi un d6 in più nel tiro della prova.

\subsubsection{Riuscire e Fallire nella prova di Magia}\index{Riuscire e Fallire nella prova di Magia}

Per capire se si riesce nella formulazione dell'incantesimo si deve innanzitutto superare, con una prova di Competenza Magica (3d6 + Competenza Magica + caratteristica da incantatore + bonus concesso dal Patrono + Abilità...) la Difficoltà indicata nell'incantesimo stesso.

Se non si riesce a superare la Difficoltà l'incantesimo non si attiva e non sortisce effetto (anche se il Narratore potrebbe descrivere l'eventuale incantesimo scaturito tirando tre volte uno con i dadi...).

Anche se la prova fallisce l'incantesimo si conta tra quelli lanciati e i componenti eventuali sono consumati.

\begin{narratoresmall}
Concedete un +1 alla prova di magia quando il giocatore declama con perizia e trasporto il lancio dell'incatesimo. Se dice "\textit{Lancio una palla di fuoco}" non otterrà vantaggi ma se con trasporto declama "\textit{Per tutti i fuochi infernali possa Nedraf trascinarvi all'inferno con le sue sacre fiamme. Bruciate indegni. Palla di Fuoco!}" allora un bonus di +1 e' piu' che doveroso.	
\end{narratoresmall}	


\subsubsection{Riuscire molto bene nella prova di Magia* - Opzionale}\index{Riuscire molto bene nella prova di Magia - Opzionale}

Questa opzione puo' essere applicata se non si vuole lasciare al fato dei dadi la possibilità o meno di fare dei critici, bensì all'abilità del mago.

Se il Narratore lo permette, da stabilire ad inizio campagna, si ottiene un critico per ogni 50\% in più che si ottiene nella prova di magia.

Es. 6,5,6. Ritiro i sei e ottengo 5,5. Sommo 9 di Competenza Magica e 3 di caratteristica da incantatore per un totale di 39. Non conto quindi 1 critico perché ho fatto due volte 6, ma dato che volevo lanciare l'incantesimo Cura Ferite Leggere che ha Difficoltà 19 si avrà questo effetto:

1d8 + Saggezza + (39-16)/(16/2) = 1d8 + Saggezza + 2d6 aggiuntivi (ovvero due critici)


\subsubsection{Riuscire nel fallimento della prova di Magia* - Opzionale}\index{Riuscire nel fallimento della prova di Magia - Opzionale}

Può essere molto frustrante mancare magari di un solo punto una prova di magia e vedere vanificato l'incantesimo, ancor di piu' se si erano anche ottenuti critici.

Queste regole opzionali, sceglietene una sola quando si parte con la campagna, permettono al Narratore di stabilire se una prova di magia fallita può comunque essere sufficiente per ottenere un qualche risultato.

\begin{enumerate}

\item
\textbf{I 6 comandano}\index{I 6 comandano}: con questa regola se nella prova di magia fallita si sono ottenuti almeno 2 volte sei (quindi successo critico) il giocatore potrà lanciare un incantesimo di Difficoltà inferiore purché la Difficoltà sia superata.

Es: Tups fa 6,2,6. Ritira i sei ed ottiene 2,1, somma 2 come modificatore di caratteristica e 3 per la Competenza Magica per un totale di 22. Non potrà quindi lanciare Blocca Persona Avanzato  come aveva sperato (Difficoltà 23) ma potrà comunque lanciare ad esempio l'incantesimo Mura di Vento (Difficoltà 21). L'incantesimo così formulato consumerà tutte e 3 le Azioni.

\item
Fare che ogni personaggio con Competenza Magica 1 e che sia Seguace o Devoto abbia gratuitamente l'Abilità  \hyperlink{ilpatronoelamiamagia}{Il Patrono e' la mia magia}. 

\end{enumerate}

\begin{tcolorbox}[width=0.85\linewidth,title = Fallire gli Incantesimi]{
Può capitare ed anche spesso di fallire nella prova di magia quando si lanciano incantesimi della massima Difficoltà conosciuta. Provate con incantesimi a minore Difficoltà. Ricordate anche le possibilità offerta da Alterare la Magia ovvero Magia Efficace, Aumentare il tempo di lancio, Magie Collaborative e Circolo di Potere possono concedere un grosso vantaggio.

Infine un Seguace ottiene un +2 alla prova di Competenza Magia per la scuola preferita dal Patrono ed il Devono ottiene un +4 alla stessa prova. 

Questi modificatori, compreso l'Abilità Magie Potenti, sono tutti cumulativi.
}\end{tcolorbox}

\subsubsection{Resistere all'incantesimo (Tiro Salvezza)}\index{Resistere all'Incantesimo}\index{Tiro Salvezza Incantesimi}

\label{resistere-allessenza-tiro-salvezza}

Una volta che la prova di magia è superata e quindi la magia liberata, anche in base alla descrizione e note dell'incantesimo, è possibile resistere all'effetto della magia.

Il Tiro Salvezza in base a quanto richiesto dalla magia e' pari alla prova di magia effettuata dal mago. Una prova di magia particolarmente efficace renderà altrettanto difficile resistergli e in base all'incantesimo usato ed eventuali critici ottenuti nella prova potranno manifestarsi risultati ancora migliori. Nella descrizione dell'incantesimo e' descritto cosa succede in caso di critico nella prova di magia.

Se il Tiro Salvezza riesce o fallisce di più di 10 (\textbf{successo critico}\index{Successo Critico Incantesimi} o \textbf{fallimento critico}\index{Fallimento Critico Incantesimi}) il Narratore potrà decidere di applicare svantaggi o vantaggi al risultato finale.\index{Più di 10}.
E' anche possibile che nella descrizione dell'incantesimo sia riportato cosa succede in casa di successo o fallimento critico. 

Per i mostri o comunque per un lancio di incantesimi dato da abilità naturali o oggetti, se non specificato la DC del Tiro Salvezza e' pari alla Difficoltà dell'incantesimo+Intelligenza.

\medskip

\includegraphics[width=0.85\linewidth]{immagini/donnaserpente.png}
\textit{Henry Justice Ford}

\subsubsection{Resistenza alla Magia}\index{Resistenza alla Magia}

Una creatura potrebbe avere una naturale resistenza alle magie.

Il valore di RM (Resistenza Magia) indica tale resistenza e più è alta più la creatura è immune alla magia, che lo voglia o meno.

Ogni qual volta la creatura è influenzata direttamente da un incantesimo o effetto magico deve effettuare una prova di RM, ovvero tirare 3d6+Carisma sommare il valore di RM e se è superiore alla prova di magia effettuata dall'incantatore l'incantesimo non ha effetto.

In caso di essenze scaturite da oggetti (anelli, bastoni, pozioni) la prova di RM deve superare la Difficoltà dell'incantesimo.

\subsubsection{Distrazioni - Problemi nel lancio dell'incantesimo}\index{Distrazioni - Problemi nel lancio dell'incantesimo}

Se l'incantatore viene è severamente distratto, impedito, disturbato, è sotto attacco mentre cerca di lanciare un incantesimo la prova di magia questa deve riuscire di almeno +5 rispetto alla Difficoltà dell'incantesimo, altrimenti la "Distrazione" e' stata tale da impedire il buon esito della magia.
Un incantesimo fallito per distrazione non si conta nel novero degli incantesimi lanciati.

Un mago che lancia incantesimi mentre e' in combattimento di mischia ha un -4 alla Difesa.

\subsubsection{Conservare la magia}\index{Conservare la magia}

Il mago può lanciare l'incantesimo (solitamente 2 Azioni) e trattenerlo nel suo pugno. Per farlo deve riuscire a lanciare l'incantesimo dopo di che lo puo' trattenere fino ad 1 round per punto Competenza Magica posseduto.

Trattenere un incantesimo equivale a rimanere concentrato (costo 1 Azione per round).
Per lanciare l'incantesimo conservato e' sufficiente tirare l'iniziativa ed usare 1 Azione. Non e' possibile lanciare ulteriori incantesimi con difficoltà 10 o piu' finché si conserva un'incantesimo e nel round in cui si manifesta l'incantesimo trattenuto non e' possibile lanciare incantesimi con Difficoltà 10 o piu'.

\subsubsection{Influenzati da più Magie}\index{Influenzati da più Magie}

Quando un personaggio è influenzato da \textbf{due o più effetti magici} che danno lo stesso tipo di bonus, malus o danno nello stesso segmento di iniziativa (protezione verso fuoco, bonus alla Difesa o TS... , multiple palle di acido), si tiene conto solo di quella che ha la Difficoltà maggiore.

Un personaggio che prende 2 Palle di Fuoco nel medesimo segmento di Iniziativa fara' il Tiro Salvezza solo per quella con la difficoltà più alta.

Se prende una Palla di Fuoco in due tempi diversi del medesimo round fara' due Tiri Salvezza distinti prendendo il danno relativo.

\subsubsection{Alterare le Magie}\index{Alterare le Magie}

Il mago può modificare la sua prova di magia tramite diverse possibilità.

\begin{itemize}
	\item
	\textbf{Magia efficace}\index{Magia efficace}: sacrificando Punti Ferita puo' aumentare il successo nel lancio dell'incantesimo. Ottiene un +1 alla prova di Competenza Magica ogni 4 punti ferita sacrificati
	\item
	\textbf{Magia eterea}\index{Magia eterea}: aumentando di 3 la Difficoltà dell'incantesimo le proprie magie hanno pieno effetto su creature eteree o incorporee
	\item
	\textbf{Magia pietosa}: aumentando di 3 la Difficoltà di lancio le magie infliggono danni temporanei. 
	Le magie che infliggono danni di un tipo particolare (come da fuoco) infliggono danni temporanei dello stesso tipo.
	\item
	\textbf{Aumentare il tempo}\index{Aumentare il tempo} di lancio da 2 Azioni a 1 round diminuisce la Difficoltà dell'incantesimo di 1.
	\item
	\textbf{Magie collaborative}\index{Magie collaborative}: un altro mago, sacrificando uno slot incantesimi e impiegando le stesso tempo di lancio di incantesimo del compagno, puo' concedere +4 alla prova di magia del compagno.
	\item
	\textbf{Circolo del Potere}\index{Circolo del Potere}: piu' maghi che siano tutti Devoti o Seguaci dello stesso Patrono possono collaborare affinché uno di loro riesca meglio nel lancio di un incantesimo.
	Ogni mago sacrifica uno slot di incantesimo per concedere +1d6 nella prova di magia ad un collega nel lancio del suo incantesimo (massimo 3 maghi ovvero +3d6). Il tempo di lancio di un incantesimo nel Circolo di Potere diviene almeno 1 turno.
	\item
	\textbf{Modifiche lievi}\index{Modifiche lievi agli incantesimi} alla manifestazione dell'incantesimo possono essere concordati con il Narratore per una Difficoltà aggiuntiva. 
\end{itemize}

\subsubsection{Tentare Incantesimi con impedimenti}\index{Tentare Incantesimi con impedimenti} \index{Impedimenti}

Il lancio di un incantesimo e' vincolato a gesti e parole particolari e unici. Quando il personaggio si trova in una situazione in cui non puo' gesticolare o parlare allora può tentare di lanciare l'incantesimo comunque anche se diventa molto piu' difficile.

La Difficoltà di lancio aumenta di 5 se non puo' gesticolare e di 7 se non puo' parlare. Se non puo' ne parlare ne gesticolare la Difficoltà aumenta di 15 e lanciare un incantesimo richiede 1 round minimo.

\subsubsection{Definizioni obiettivi degli incantesimi}\index{Obiettivi degli incantesimi}

Negli incantesimi sotto elencati troverete spesso i riferimenti alle tipologie di soggetti ed obiettivi influenzabili nonché a diverse tipologie di energia ed elementi. 

\item Le \textbf{Creature} \textbf{Naturali} sono Insetti, Rettili, Bestie, Umanoidi, Piante, Creature acquatiche, Monstrusità, Melme.

\item Le \textbf{Creature} \textbf{Magiche} sono: Immondi, Fatati, Spiriti, Non morti, Giganti, Celestiali, Elementali, Costrutti, Aberrazioni (tutto ciò che e' alieno o innaturale) e Draghi.
Se una Creatura Naturale ha poteri magici allora si considera anche come Creatura Magica. Una descrizione piu' completa di questi "categorie" la trovate nel Capitolo delle Mostruosità.

\item \textbf{Energia} comprende: Fuoco, Luce, Suono, Elettricità, Energia Positiva, Energia Negativa, Freddo, Vuoto.

Il danno causato da \textbf{Luce}\index{Luce} e' per metà da fuoco e per metà da energia positiva, ovvero una resistenza al fuoco od all'energia positiva si applica solo su metà del danno causato dall'attacco.

Il danno causato da \textbf{Vuoto}\index{Vuoto} e' per metà da freddo e per metà da energia negativa, eventuali protezioni si applicano alle rispettive metà del danno.

L'energia negativa danneggia\index{Energia Negativa} i viventi e cura i non morti, l'energia positiva\index{Energia Positiva} danneggia i non morti ma non cura i viventi (a discrezione del Narratore l'esposizione potrebbe equivalere ad un incantesimo di Ristorare Inferiore), vedi anche descrizioni dei Piani.

\end{multicols}

\pagebreak

\begin{multicols}{2}
	
\begin{narratore}
Volete un sistema piu' tradizionale e piu' facile ?

\medskip

Opzione 1) Lasciate tutto come e'. Semplicemente non fate eseguire le prove di magia per verificare il successo nel lancio di incantesimi. L'eventuale esplosione dell'incantesimo la avrete usando 2 slot di incantesimo invece che 1.


Opzione 2) Fate comunque tirare 3d6 non modificati, ma solo per vedere se si genera un critico
\end{narratore}	

\medskip

\begin{tcolorbox}[title = Più effetti speciali!]
Gli incantesimi elencati sono quelli della 5ed piu' alcune mie proposte ed altre rivisitazioni. Se avete suggerimenti per il Narratore per gestire critici non previsti parlatene con lui! Lo spirito di collaborazione deve essere sempre costruttivo.	
\end{tcolorbox}


\subsection{La lista degli incantesimi}

\medskip\textbf{Aiuto}\index{Incantesimi - Aiuto}\\
\textbf{Scuola}: Abiurazione\\
\textbf{Difficoltà}: 19\\
\textbf{Tempo di Lancio}: 2 Azioni\\
\textbf{Gittata}: 9 metri\\
\textbf{Componenti}: V, S, M (una sottile striscia di tessuto bianco)\\
\textbf{Durata}: 8 ore\\
Il tuo incantesimo aumenta la robustezza e risolutezza dei tuoi alleati. Scegli fino a tre creature a gittata. Per la durata, i punti ferita massimi e i punti ferita attuali di ciascun bersaglio aumentano di 5.\\
\textbf{Per ogni Critico ottenuto} nella prova di magia i punti ferita del bersaglio aumentano di ulteriori 5 punti

\medskip\textbf{Allarme}\index{Incantesimi - Allarme}\\
\textbf{Scuola}: Abiurazione\\
\textbf{Difficoltà}: 16\\
\textbf{Tempo di Lancio}: 1 minuto\\
\textbf{Gittata}: 9 metri\\
\textbf{Componenti}: V, S, M (una campanella e un pezzo di pregiato filo d'argento)\\
\textbf{Durata}: 8 ore\\
Predisponi un allarme contro intrusioni indesiderate. Scegli una porta, una finestra o un'area a gittata che non sia più grande di un cubo di 6 metri di spigolo. Fino al termine dell'incantesimo, sarai avvertito da un allarme ogni volta che una creatura di taglia Minuscola o superiore entri in contatto o acceda all'area protetta. Quando lanci l'incantesimo, puoi indicare delle creature che non faranno scattare l'allarme Scegli anche se l'allarme è udibile o solo mentale. Un allarme mentale, qualora ti trovi entro 1,5 chilometri dall'area protetta, ti avverte con un rumore nella tua mente. Il rumore è in grado di svegliarti se stai dormendo. Un allarme udibile produce il suono di una campanella per 10 secondi, udibile entro 18 metri.

\medskip\textbf{Allucinazione Mortale}\index{Incantesimi - Allucinazione Mortale}\\
\textbf{Scuola}: Illusione\\
\textbf{Difficoltà}: 23\\
\textbf{Tempo di Lancio}: 2 Azioni\\
\textbf{Gittata}: 36 metri\\
\textbf{Componenti}: V, S\\
\textbf{Durata}: Istantanea\\
Attingi agli incubi di una creatura a gittata e che puoi vedere, e crei una manifestazione illusoria delle sue più insite paure, visibile solo per quella creatura. Il bersaglio deve effettuare un Tiro Salvezza su Volontà.\\
Se fallisce il Tiro Salvezza, il bersaglio è spaventato per 1 minuto e subisce 4d10 di danno psichico. \\
\textbf{Per ogni Critico ottenuto} nella prova di magia il danno aumenta di 1d10

\medskip\textbf{Alterare Sé Stesso}\index{Incantesimi - Alterare Sé Stesso}\\
\textbf{Scuola}: Trasmutazione\\
\textbf{Difficoltà}: 19\\
\textbf{Tempo di Lancio}: 2 Azioni\\
\textbf{Gittata}: Personale\\
\textbf{Componenti}: V, S\\
\textbf{Durata}: 1 ora\\
Assumi una forma diversa. Quando lanci questo incantesimo, scegli una della seguenti opzioni, l'effetto della quale permane per la durata dell'incantesimo. Per la durata dell'incantesimo puoi terminare un'opzione per ottenere i benefici di un'altra.\\
Adattamento Acquatico. Adatti il tuo corpo a un ambiente acquatico, sviluppando branchie e dita palmate. Puoi respirare sott'acqua e ottieni velocità di nuoto pari alla tua velocità di passeggio.\\
\textit{Armi Naturali}. Sviluppi degli artigli, zanne, spuntoni, corna o una diversa arma naturale a tua scelta. I tuoi colpi senz'armi infliggono 1d6 danni da botta, perforanti o taglienti, come appropriato all'arma naturale scelta, con la quale sei competente. Infine, l'arma naturale è magica e ricevi un bonus di +1 ai Tiri per Colpire e danno effettuati quando la usi.\\
\textit{Cambio di Aspetto}. Trasformi il tuo aspetto. Decidi il tuo aspetto esteriore, compresa l'altezza, il peso, i lineamenti facciali, il suono della tua voce, la lunghezza dei capelli, il colorito e qualsiasi peculiarità tu desideri. Puoi apparire come membro di un'altra razza, sebbene nessuna delle tue statistiche cambi. Inoltre non puoi apparire come una creatura di taglia diversa dalla tua, e la tua forma base resta la medesima; se sei bipede, non puoi usare questo incantesimo per diventare quadrupede, per esempio.\\
In qualsiasi momento della durata dell'incantesimo, puoi usare due Azioni per cambiare nuovamente di aspetto in questo modo.\\

\medskip\textbf{Amicizia con gli Animali}\index{Incantesimi - Amicizia con gli Animali}\\
\textbf{Scuola}: Ammaliamento\\
\textbf{Difficoltà}: 16\\
\textbf{Tempo di Lancio}: 2 Azioni\\
\textbf{Gittata}: 9 metri\\
\textbf{Componenti}: V, S, M (un po' di cibo)\\
\textbf{Durata}: 24 ore\\
Questo incantesimo ti permette di convincere una bestia naturale che non vuoi arrecargli danno. Scegli una bestia a gittata che puoi vedere. Questa deve vederti e udirti. Se l'Intelligenza della bestia è 4 o più, l'incantesimo fallisce. Altrimenti, la bestia deve superare un Tiro Salvezza su Volontà o restare affascinata da te per la durata dell'incantesimo. Se tu o uno dei tuoi compagni danneggiate il bersaglio, l'incantesimo ha termine.\\
\textbf{Per ogni critico ottenuto} nella prova di magia puoi agire su di una bestia aggiuntiva. 

\medskip\textbf{Anatema}\index{Incantesimi - Anatema}\\
\textbf{Scuola}: Ammaliamento\\
\textbf{Difficoltà}: 16\\
\textbf{Tempo di Lancio}: 1 minuto\\
\textbf{Gittata}: 9 metri\\
\textbf{Componenti}: V, S, M (un goccio di sangue)\\
\textbf{Durata}: 1 minuto\\
Fino a tre creature di tua scelta che puoi vedere, e che sono a gittata, devono effettuare un Tiro Salvezza su Volontà. Ogni bersaglio che fallisca questo Tiro Salvezza ed effettua un tiro per colpire o un Tiro Salvezza prima del termine dell'incantesimo, deve tirare un d4 e sottrarre il numero così ottenuto dal tiro per colpire o Tiro Salvezza.\\
\textbf{Per ogni Critico ottenuto} nella prova di magia puoi prendere come bersaglio una creatura aggiuntiva.

\medskip\textbf{Animale Messaggero}\index{Incantesimi - Animale Messaggero}\\
\textbf{Scuola}: Ammaliamento\\
\textbf{Difficoltà}: 19\\
\textbf{Tempo di Lancio}: 2 Azioni\\
\textbf{Gittata}: 9 metri\\
\textbf{Componenti}: V, S, M (un poco di cibo)\\
\textbf{Durata}: 24 ore\\
Tramite questo incantesimo, usi un animale per consegnare un messaggio. Scegli una bestia Minuscola a gittata e che puoi vedere, come uno scoiattolo, una ghiandaia o un pipistrello. Specifichi un luogo, che devi aver visitato in passato, e un destinatario che corrisponda a una descrizione generica, come "un uomo o una donna che vesta l'uniforme della guardia cittadina" o "un nano dai capelli rossi che indossa una fedora". Pronuncia anche un messaggio di massimo venticinque parole. La bestia bersaglio viaggia per la durata dell'incantesimo verso il luogo specificato, coprendo circa 75 chilometri in 24 ore per un messaggero volante, o 40 chilometri per gli altri animali. Quando il messaggero arriva a destinazione, consegna il messaggio alla creatura da te descritta, replicando il suono della tua voce. Il messaggero parla solo a una creatura corrispondente alla descrizione da te fornita. Se il messaggero non riesce a raggiungere la destinazione prima del termine dell'incantesimo, il messaggio è perduto, e la bestia ritorna verso il punto in cui hai lanciato l'incantesimo.\\
\textbf{Per ogni Critico ottenuto} nella prova di magia la durata dell'incantesimo aumenta di 8 ore

\medskip\textbf{Animare Morti}\index{Incantesimi - Animare Morti}\\
\textbf{Scuola}: Necromanzia\\
\textbf{Difficoltà}: 21\\
\textbf{Tempo di Lancio}: 1 minuto\\
\textbf{Gittata}: 3 metri\\
\textbf{Componenti}: V, S, M (una goccia di sangue, un pezzo di carne e un pizzico di polvere d'ossa)\\
\textbf{Durata}: Istantanea\\
Questo incantesimo crea un servitore non morto. Scegli una pila di ossa o un cadavere di un umanoide Medio o Piccolo a gittata. Il tuo incantesimo imbeve il bersaglio di una nefanda parvenza di vita, rianimandolo come creatura non morta. Il bersaglio diventa uno scheletro se scegli le ossa o uno zombi se scegli un cadavere. Durante ciascun tuo round, puoi usare un'Azione per comandare mentalmente qualsiasi creatura da te creata con questo incantesimo che si trovi entro 18 metri da te (se controlli più creature, puoi comandarle tutte o solo alcune di loro allo stesso momento, inviando lo stesso comando a tutte). Decidi quale azione la creatura svolgerà e dove si muoverà durante il suo prossimo round, oppure inviale un comando generale, come quello di stare di guardia a una particolare stanza o corridoio. Se non invii alcun comando, la creatura si limita a difendersi dalle creature ostili. Una volta ricevuto un ordine, la creatura continuerà a svolgerlo fino al suo compimento. La creatura è sotto il tuo controllo per 24 ore, dopodiché smetterà di eseguire i comandi che le impartirai. Per mantenere il controllo sulla creatura per altre 24 ore, devi lanciare di nuovo questo incantesimo su di essa prima del termine dell'attuale periodo di 24 ore. Questo impiego dell'incantesimo riafferma il tuo controllo su di un massimo di quattro creature che hai animato con questo incantesimo, piuttosto che animarne una nuova.\\
\textbf{Per ogni Critico ottenuto} nella prova di magia animi o riaffermi il controllo su due creature non morte. Ciascuna di queste creature deve provenire da un cadavere o pila di ossa differenti.

\medskip\textbf{Animare Oggetti}\index{Incantesimi - Animare Oggetti}\\
\textbf{Scuola}: Trasmutazione\\
\textbf{Difficoltà}: 26\\
\textbf{Tempo di Lancio}: 1 minuto\\
\textbf{Gittata}: 36 metri\\
\textbf{Componenti}: V, S\\
\textbf{Durata}: Concentrazione, massimo 1 minuto\\
Gli oggetti prendono vita al tuo comando. Scegli fino a dieci oggetti non magici a gittata e che non siano indossati o trasportati. I bersagli Medi contano come due oggetti, i bersagli Grandi contano come quattro oggetti, i bersagli Enormi contano come otto oggetti. Non puoi animare oggetti di taglia più grossa di Enorme. Ogni bersaglio si anima e diventa una creatura sotto il tuo controllo fino al termine dell'incantesimo o finché non viene ridotto a 0 punti ferita.\\
Con un'Azione puoi comandare mentalmente qualsiasi creatura che hai generato con questo incantesimo e che si trovi entro 150 metri da te (se controlli più creature, puoi comandarne solo alcune o tutte allo stesso tempo, impartendo lo stesso comando a ciascuna). Decidi tu quale azione intraprenderà la creatura e dove si muoverà durante il suo round successivo, o puoi emettere un comando generico,come quello di fare la guardia a una particolare stanza o corridoio. Se non impartisci comandi, la creatura si limiterà a difendersi dalle creature ostili. Una volta dato un ordine, la creatura continuerà a seguirlo finché non avrà completato il suo compito.
\bigskip

\end{multicols}

\textbf{Statistiche degli Oggetti Animati}
\bigskip

\begin{tabular}{llllll}
Taglia		&Punti Ferita	&Difesa	&CA, danni					&Forza	&Destrezza\\ 
\toprule
Minuscola 	&20 			&18		&8, {1d4+4} 	&-3 		&4\\
Piccola 	&25 			&16 	&6, {1d8+2} 	&-2 		&2\\
Media 		&40 			&13 	&5, {2d6+1} 	&0 			&1\\
Grande 		&50 			&10 	&6, {2d10+2}	&2 			&0\\
Enorme 		&80 			&10 	&8, {2d12+4}	&4 			&-2\\
\end{tabular}

\bigskip

\begin{multicols}{2}

Un oggetto animato è un costrutto con Difesa, punti ferita, attacchi, Forza e Destrezza in base alla sua taglia. Il suo punteggio di Intelligenza e Saggezza è -3, mentre Carisma e' -4.\\
Ha movimento 9 metri; se l'oggetto è privo di gambe o altre appendici che può usare per muoversi ha movimento 0, ha invece movimento volare 9 metri e può fluttuare. 
\\Se l'oggetto è ancorato a una superficie o un oggetto più grosso, come una catena attaccata al muro, la sua velocità è 0.\\
Possiede vista cieca con un raggio di 9 metri ed è cieco oltre questa distanza.\\
Quando l'oggetto animato scende a 0 punti ferita, ritorna alla sua normale forma di oggetto, e tutti i danni in eccesso vengono inflitti alla sua forma originale.\\
Se comandi a un oggetto di attaccare, questo può effettuare un singolo attacco da mischia contro una creatura entro 1 metri da esso. Effettua un attacco on CA e danni determinati dalla taglia (vedi tabella). Il Narratore potrebbe determinare che a seconda della sua forma, un oggetto potrebbe invece infliggere danni taglienti o perforanti.\\
\textbf{Per ogni Critico ottenuto} nella prova di magia puoi animare due oggetti aggiuntivi.

\medskip\textbf{Anti-Individuazione}\index{Incantesimi - Anti-Individuazione}\\
\textbf{Scuola}: Abiurazione\\
\textbf{Difficoltà}: 21\\
\textbf{Tempo di Lancio}: 2 Azioni\\
\textbf{Gittata}: Contatto\\
\textbf{Componenti}: V, S, M (un pizzico di polvere di diamante del valore di 25 mo sparsa sul bersaglio, che l'incantesimo consuma)\\
\textbf{Durata}: 8 ore\\
Per la durata, nascondi il bersaglio con cui sei stato in contatto dalla magia di divinazione. Il bersaglio può essere una creatura consenziente o un luogo o un oggetto che occupi uno spazio equivalente a un cubo non superiore ai 3 metri di spigolo. Il bersaglio non può divenire bersaglio di alcuna magia di divinazione o essere percepito tramite sensi di scrutamento magici.

\medskip\textbf{Antipatia/Simpatia}\index{Incantesimi - Antipatia/Simpatia}\\
\textbf{Scuola}: Ammaliamento\\
\textbf{Difficoltà}: 34\\
\textbf{Tempo di Lancio}: 1 ora\\
\textbf{Gittata}: 18 metri\\
\textbf{Componenti}: V, S, M (o un pezzo di allume immerso nell'aceto per l'effetto antipatia o un goccio di miele per l'effetto simpatia)\\
\textbf{Durata}: 10 giorni\\
Questo incantesimo attrae o repelle delle creature di tua scelta. Prendi un bersaglio a gittata, che sia un oggetto Enorme o più piccolo o una creatura o un'area non più grande di un cubo di 60 metri di spigolo. Poi specifica una specie di creature intelligenti, come i draghi rossi, i goblin o i vampiri. Investi il bersaglio di un'aura che attrae o respinge le creature specificate per la durata. Scegli antipatia o simpatia come effetto dell'aura.\\
Antipatia. L'ammaliamento fa sì che le creature del tipo da te indicato provino un forte impulso a lasciare l'area ed evitare il bersaglio. Quando una creatura del genere può vedere il bersaglio o si avvicina entro 18 metri da esso, la creatura deve superare un Tiro Salvezza su Volontà o diventare spaventata. La creatura rimane spaventata finché può vedere il bersaglio o resta entro 18 metri da esso. Mentre è spaventata dal bersaglio, la creatura deve impiegare il suo movimento per muoversi verso il posto sicuro più vicino dal quale non possa più vedere il bersaglio. Se la creatura si muove più di 18 metri lontano dal bersaglio e non può vederlo, la creatura non è più spaventata, ma torna a essere spaventata se torna a vedere il bersaglio o si muove entro 18 metri da esso.\\
Simpatia. L'ammaliamento fa sì che le creature specificate provino un forte impulso ad avvicinarsi al bersaglio se si trovano entro 18 metri da esso o possono vederlo. Quando una simile creatura può vedere il bersaglio o si avvicina entro 18 metri da esso, la creatura deve superare un Tiro Salvezza su Volontà o usare il suo movimento durante ciascun round per entrare nell'area, o muoversi a portata del bersaglio. Quando la creatura l'avrà fatto, non potrà più volontariamente muoversi lontano dal bersaglio. Se il bersaglio danneggia o altrimenti nuoce alla creatura soggetta, questa può effettuare un Tiri Salvezza su Volontà per terminare l'effetto, come descritto di seguito.\\
Terminare l'Effetto. Se una creatura soggetta termina il suo round mentre si trova più lontana di 18 metri dal bersaglio o non può vederlo, la creatura effettua un Tiro Salvezza su Volontà. Se supera il Tiro Salvezza, la creatura non è più soggetta al bersaglio e riconosce la sensazione di ripugnanza o attrazione come magica. Inoltre, una creatura soggetta all'incantesimo, ha diritto a un altro Tiro Salvezza su Volontà ogni 24 ore di durata dell'incantesimo. Una creatura che supera il Tiro Salvezza contro questo effetto è immune a esso per 1 minuto, dopodiché può subirlo nuovamente.

\medskip\textbf{Arma Magica}\index{Incantesimi - Arma Magica}\\
\textbf{Scuola}: Trasmutazione\\
\textbf{Difficoltà}: 19\\
\textbf{Tempo di Lancio}: 1 Azione Immediata\\
\textbf{Gittata}: Contatto\\
\textbf{Componenti}: V, S\\
\textbf{Durata}: 10 minuti\\
Lanci l'incantesimo a contatto di un'arma non magica. Fino al termine dell'incantesimo, l'arma diventa un'arma magica con un bonus di +1 ai Tiri per Colpire e di danno.\\
\textbf{Per ogni Critico ottenuto} nella prova di magia puoi il bonus aumenta a +1.

\medskip\textbf{Arma Spirituale}\index{Incantesimi - Arma Spirituale}\\
\textbf{Scuola}: Invocazione\\
\textbf{Difficoltà}: 19\\
\textbf{Tempo di Lancio}: 2 Azioni\\
\textbf{Gittata}: 18 metri\\
\textbf{Componenti}: V, S\\
\textbf{Durata}: 1 minuto\\
In un punto nella gittata, crei un'arma spettrale fluttuante, che resta per la durata o finché non lanci di nuovo questo incantesimo. Quando lanci l'incantesimo, puoi effettuare un attacco da incantesimo in mischia contro una creatura entro 1 metro dall'arma. Se colpisci, il bersaglio subisce danni da forza pari a 1d8 + il tuo valore di caratteristica da incantatore. Durante il tuo round, con due Azioni, puoi spostare l'arma di 6 metri e ripetere l'attacco contro una creatura entro 1 metro dall'arma. L'arma può assumere qualsiasi forma tu voglia, magari affine al Patrono.\\
\textbf{Per ogni Critico ottenuto} nella prova di magia il danno aumenta di 2.

\medskip\textbf{Armatura Magica}\index{Incantesimi - Arma Magica}\\
\textbf{Scuola}: Abiurazione\\
\textbf{Difficoltà}: 16\\
\textbf{Tempo di Lancio}: 2 Azioni\\
\textbf{Gittata}: Contatto\\
\textbf{Componenti}: V, S, M (un pezzo di cuoio lavorato)\\
\textbf{Durata}: 8 ore
Lanci l'incantesimo a contatto di una creatura consenziente che non indossa un'armatura. Una forza magica protettiva circonda il bersaglio fino al termine dell'incantesimo. La Difesa del bersaglio diventa 13 + Destrezza. L'incantesimo termina se il bersaglio indossa un'armatura o interrompe l'incantesimo con un'azione.


\medskip\textbf{Artificio Druidico}\index{Trucchetto - Artificio Druidico}\\
\textbf{Scuola}: Universale\\
\textbf{Difficoltà}: 12\\
\textbf{Tempo di Lancio}: 2 Azioni\\
\textbf{Gittata}: 9 metri\\
\textbf{Componenti}: V, S\\
\textbf{Durata}: Istantanea\\
Sussurrando agli spiriti della natura, crei, a gittata, uno dei seguenti effetti:
\begin{itemize}
\item
Crei un minuscolo e innocuo effetto sensoriale che predice quale clima ci sarà nel luogo in cui ti trovi per le prossime 24 ore. L'effetto potrebbe manifestarsi come una sfera dorata per i cieli limpidi, una nube per la pioggia, fiocchi di neve per la neve, e così via. L'effetto persiste per 1 round.
\item 
Fai immediatamente sbocciare un fiore, un seme o simile pianta.
\item 
Crei un istantaneo e innocuo effetto sensoriale, come foglie che cadono, uno sbuffo di vento, il suono di un piccolo animale, o il lieve tanfo di una puzzola. L'effetto deve entrare in un cubo di 1 metro.
\item Accendi o spegni istantaneamente una candela, una torcia o un piccolo falò.
\end{itemize}

\medskip\textbf{Aura Magica dell'Arcanista}\index{Incantesimi - Aura Magica dell'Arcanista}\\
\textbf{Scuola}: Illusione\\
\textbf{Difficoltà}: 19\\
\textbf{Tempo di Lancio}: 2 Azioni\\
\textbf{Gittata}: Contatto\\
\textbf{Componenti}: V, S, M (un piccolo quadretto di seta)\\
\textbf{Durata}: 24 ore\\
Poni un'illusione su di una creatura od oggetto con cui sei in contatto, così che gli incantesimi di divinazione rivelino false informazioni su di esso. Il bersaglio può essere una creatura consenziente o un oggetto che non sia trasportato o indossato da un'altra creatura. Quando lanci questo incantesimo, scegli uno o entrambi i seguenti effetti. L'effetto permane per la durata. Se esegui questo incantesimo sulla stessa creatura od oggetto ogni giorno per 30 giorni, piazzando ogni volta lo stesso effetto, l'illusione permarrà finché non viene dissolta.\\
\textit{Aura Falsa}. Cambi il modo in cui il bersaglio risulta a incantesimi ed effetti magici, come individuazione del magico, che individuano le aure magiche. Puoi far apparire magico un oggetto normale, non magico un oggetto magico, o cambiare l'aura magica dell'oggetto così che sembri appartenere a una scuola di magia di tua scelta. Quando impieghi questo effetto su di un oggetto, puoi far sì che la falsa magia sia apparente a qualsiasi creatura che lo manipoli.\\ \textit{Mascherare}. Cambi il modo in cui il bersaglio risulta a incantesimi ed effetti magici che individuano il tipo di creatura o Tratti, come l'attivazione dell'incantesimo simbolo. Scegli un tipo di creatura o Tratto, e gli altri incantesimi ed effetti magici considereranno il bersaglio come fosse una creatura di quel tipo o di quel Tratto, e non più di quello originale.

\medskip\textbf{Aura Sacra}\index{Incantesimi - Aura Sacra}\\
\textbf{Scuola}: Abiurazione\\
\textbf{Difficoltà}: 34\\
\textbf{Tempo di Lancio}: 2 Azioni\\
\textbf{Gittata}: Personale\\
\textbf{Componenti}: V, S, M (un minuscolo reliquario del valore di almeno 1.000 mo contenente una reliquia sacra, come un pezzo di tessuto dell'abito di un santo o un frammento di pergamena di un testo religioso)\\
\textbf{Durata}: Concentrazione, 1 minuto\\
Irradi da te luce divina che si raccoglie in una debole luminosità con raggio di 9 metri intorno a te. Quando lanci l'incantesimo, le creature da te scelte in questo raggio emanano luce fioca con un raggio di 1 metro e hanno {+2d6} a tutti i Tiri Salvezza, mentre le altre creature hanno {-2d6} sui Tiri per Colpire contro di loro fino al termine dell'incantesimo. Inoltre, quando un demone o non morto colpisce una creatura bersaglio con un attacco in mischia, l'aura risplende di una luce intensa e deve superare un Tiro Salvezza su Tempra o restare accecato fino al termine dell'incantesimo.

\medskip\textbf{Bacche Benefiche}\index{Incantesimi - Bacche Benefiche}\\
\textbf{Scuola}: Trasmutazione\\
\textbf{Difficoltà}: 19\\
\textbf{Tempo di Lancio}: 2 Azioni\\
\textbf{Gittata}: Contatto\\
\textbf{Componenti}: V, S, M (un rametto di vischio)\\
\textbf{Durata}: Istantanea\\
Incanti fino a 2d4 bacche nella tua mano che vengono infuse di magia per la durata. Una creatura può usare 1 Azione Immediata per mangiare una bacca. Mangiare una bacca ripristina 1 punto ferita, e la bacca inoltre provvede nutrimento sufficiente per alimentare una creatura per un giorno. Solo la prima bacca e' efficace nel giorno.\\
Le bacche perdono la loro efficacia se non vengono consumate entro 72 ore dal lancio dell'incantesimo. \\
\textbf{Per ogni Critico ottenuto} nella prova di magia le bacche durano un giorno in più oppure incanti una bacca in più (fino ad un massimo totale di 8).

\medskip\textbf{Bagliore Solare}\index{Incantesimi - Bagliore Solare}\\
\textbf{Scuola}: Invocazione\\
\textbf{Difficoltà}: 29\\
\textbf{Tempo di Lancio}: 2 Azioni\\
\textbf{Gittata}: Personale (linea di 18 metri)\\
\textbf{Componenti}: V, S, M (una lente di ingrandimento)\\
\textbf{Durata}: Concentrazione, massimo 1 minuto\\
Una fascio di luce brillante esplode dalla tua mano in una linea larga 1 metri e lunga 18 metri. Ogni creatura sulla linea deve effettuare un Tiro Salvezza su Tempra. Se fallisce il Tiro Salvezza, la creatura subisce 6d8 danni da Luce e rimane accecata fino al tuo prossimo round. Se supera il Tiro Salvezza, subisce la metà dei danni e non è accecata. I non morti e le melme hanno -1d6 su questo Tiro Salvezza. Puoi creare una nuova linea di luminosità con un'azione durante qualsiasi tuo round fino al termine dell'incantesimo.\\
Per la durata, una particella di luce brillante risplende nella tua mano. Produce luce in un raggio di 9 metri e penombra per ulteriori 9 metri. Questa luce è considerata luce solare.

\medskip\textbf{Banchetto degli Eroi}\index{Incantesimi - Banchetto degli Eroi}\\
\textbf{Scuola}: Evocazione\\
\textbf{Difficoltà}: 29\\
\textbf{Tempo di Lancio}: 10 minuti\\
\textbf{Gittata}: 9 metri\\
\textbf{Componenti}: V, S, M (una ciotola incrostata di gemme del valore di almeno 1.000 mo, che l'incantesimo consuma)\\
\textbf{Durata}: Istantanea\\
Crei un magnifico banchetto, comprensivo di cibi e bevande prelibate. Il banchetto viene consumato in 1 ora e scompare al termine di questo periodo, ma gli effetti benefici non si faranno sentire fino al termine dell'ora. Fino ad altre dodici creature possono
partecipare al banchetto. Una creatura che partecipi al banchetto ottiene diversi benefici. La creatura viene guarita da tutte le malattie e i veleni, diventa immune al veleno e all'essere
spaventata, e ha +2d6 su tutti i Tiri Salvezza su Volontà. I suoi punti ferita massimi aumentano di 2d10, e guarisce lo stesso quantitativo di punti ferita attuali. Questi benefici durano 24 ore. 

\medskip\textbf{Barriera di Lame}\index{Incantesimi - Barriera di Lame}\\
\textbf{Scuola}: Invocazione\\
\textbf{Difficoltà}: 29\\
\textbf{Tempo di Lancio}: 2 Azioni\\
\textbf{Gittata}: 18 metri\\
\textbf{Componenti}: V, S\\
\textbf{Durata}: 10 minuti \\
Crei un muro verticale di lame rotanti fatte di energia magica, affilate come rasoi. Il muro compare a gittata e resta per la durata. Puoi creare un muro diritto lungo fino a 30 metri, alto 6 metri e spesso 1 metro, o un muro circolare di 18 metri massimo di diametro, alto 6 metri e spesso 1 metro. Il muro fornisce tre quarti di copertura alle creature dietro di esso, e il suo spazio è terreno difficile. \\
Quando una creatura entra per la prima volta in un round nell'area del muro o comincia il suo round lì, la creatura deve effettuare un Tiro Salvezza su Riflessi. Se la creatura fallisce il Tiro Salvezza subisce 6d10 danni taglienti, o la metà se lo supera.\\
Un incantatore che è ad una distanza di un metro dalla Barriera di Lame si considera distratto.

\medskip\textbf{Beffa Crudele}\index{Trucchetto - Beffa Crudele}\\
\textbf{Scuola}: Ammaliamento\\
\textbf{Difficoltà}: 12\\
\textbf{Tempo di Lancio}: 2 Azioni\\
\textbf{Gittata}: 18 metri\\
\textbf{Componenti}: V\\
\textbf{Durata}: Istantanea\\
Scateni una serie di insulti avvolti da una subdola malia contro una creatura a gittata e che puoi vedere. Se il bersaglio ti può udire (sebbene non è necessario che ti comprenda), deve superare un Tiro Salvezza su Volontà o subire 1d4 danni e avere -1d6 al prossimo tiro per colpire che effettuerà prima del termine del suo prossimo round.\\
Il danno dell'incantesimo aumenta di 1d4 quando raggiungi CM 5, CM 11 e CM 17.

\medskip\textbf{Benedici Acqua}\index{Incantesimi - Benedici Acqua}\\
\textbf{Scuola}: Invocazione\\
\textbf{Difficoltà}: 19\\
\textbf{Tempo di Lancio}: 10 Minuti\\
\textbf{Gittata}: Tocco\\
\textbf{Componenti}: V, S, M (25 monete d'oro in offerta alla chiesa)\\
\textbf{Durata}: Istantanea\\
Benedici fino ad un litro di liquido, sufficiente a creare 5 boccette di acqua benedetta.\\
\textbf{Per ogni Critico ottenuto} nella prova benedici un litro di liquido in piu'.\\

\medskip\textbf{Benedizione}\index{Incantesimi - Benedizione}\\
\textbf{Scuola}: Invocazione\\
\textbf{Difficoltà}: 16\\
\textbf{Tempo di Lancio}: 2 Azioni\\
\textbf{Gittata}: 9 metri\\
\textbf{Componenti}: V, S, M (uno spruzzo di acqua benedetta)\\
\textbf{Durata}: 1 minuto\\
Benedici fino a tre creature a gittata, scelte da te. I bersagli guadagnano +1 ai Tiri Salvezza e Tiro per Colpire.\\
Piu' benedizioni, anche da Patroni diversi non si sommano.\\
\textbf{Per ogni Critico ottenuto} nella prova puoi aggiungere una creatura come bersaglio.

\medskip\textbf{Benedizione Superiore}\index{Incantesimi - Benedizione Superiore}\\
\textbf{Scuola}: Invocazione\\
\textbf{Difficoltà}: 19\\
\textbf{Tempo di Lancio}: 1 Minuto\\
\textbf{Gittata}: 18 metri\\
\textbf{Componenti}: V, S, M (uno spruzzo di acqua benedetta, 10 monete d'oro)\\
\textbf{Durata}: 1 ora\\
Benedici una creatura a tua scelta. La creatura entro la durata puo' aggiungere 1d6 ad un tiro prima di sapere se la prova (TC/TS/Check..) ha avuto successo o meno. Questo bonus può essere usato 2 volte nell'ora.\\
\textbf{Per ogni Critico ottenuto} nella prova puoi aggiungere una creatura come bersaglio o aggiungere un ora alla durata.\\

\medskip\textbf{Benedizione Suprema}\index{Incantesimi - Benedizione Suprema}\\
\textbf{Scuola}: Invocazione\\
\textbf{Difficoltà}: 21\\
\textbf{Tempo di Lancio}: 1 Reazione\\
\textbf{Gittata}: 27 metri\\
\textbf{Componenti}: V, S, M (uno spruzzo di acqua benedetta, 25 monete d'oro)\\
\textbf{Durata}: Istantanea\\
Benedici una creatura a tua scelta. La creatura puo' ritirare due dadi di una singola prova prima di sapere se la prova ha avuto successo o meno. La creatura sceglie se prendere i nuovi tiri ottenuti o tenere i vecchi.\\
\textbf{Per ogni Critico ottenuto} nella prova la creatura prende un +1 di bonus alla prova.\\

\medskip\textbf{Blocca Mostri}\index{Incantesimi - Blocca Mostri}\\
\textbf{Scuola}: Ammaliamento\\
\textbf{Difficoltà}: 26\\
\textbf{Tempo di Lancio}: 2 Azioni\\
\textbf{Gittata}: 27 metri\\
\textbf{Componenti}: V, S, M (un piccolo pezzo dritto di ferro)\\
\textbf{Durata}: 1 minuto\\
Scegli una creatura a gittata e che puoi vedere. Il bersaglio deve superare un Tiro Salvezza su Volontà, o restare paralizzato per la durata. Questo incantesimo non ha effetto su non morti o costrutti. Al termine di ciascun suo round, il bersaglio può effettuare un altro Tiro Salvezza su Volontà. Se lo supera, per quel bersaglio l'incantesimo ha termine.\\
\textbf{Per ogni Critico ottenuto} nella prova di magia puoi aggiungere una creatura come bersaglio purché siano entro 9 metri l'una dall'altra.

\medskip\textbf{Blocca Persona}\index{Incantesimi - Blocca Persona}\\
\textbf{Scuola}: Ammaliamento\\
\textbf{Difficoltà}: 19\\
\textbf{Tempo di Lancio}: 2 Azioni\\
\textbf{Gittata}: 18 metri\\
\textbf{Componenti}: V, S, M (un piccolo pezzo dritto di ferro)\\
\textbf{Durata}: 1 minuto\\
Scegli un umanoide a gittata e che puoi vedere. L'incantesimo non ha effetto su creature con CR 4 o piu'. Il bersaglio deve superare un Tiro Salvezza su Volontà o restare paralizzato per la durata.\\
\textbf{Per ogni Critico ottenuto} nella prova di magia puoi aggiungere una creatura come bersaglio purché siano entro 9 metri l'una dall'altra.

\medskip\textbf{Blocca Persona Avanzato}\index{Incantesimi - Blocca Persona Avanzato}\\
\textbf{Scuola}: Ammaliamento\\
\textbf{Difficoltà}: 23\\
\textbf{Tempo di Lancio}: 2 Azioni\\
\textbf{Gittata}: 18 metri, raggio 6 metri\\
\textbf{Componenti}: V, S, M (un piccolo pezzo dritto di argento)\\
\textbf{Durata}: 1 minuto\\
Blocchi fino a 2d4 CR (o livelli) di creature entro 18 metri da te in un raggio di 6 metri. Si incomincia bloccando le creature dal CR piu' basso e sottraendo ai 2d4 tirati il CR, procedi finche' non hai piu' punti per bloccare le creature. I bersagli deve superare un Tiro Salvezza su Volontà o restare paralizzati per la durata.\\
\textbf{Per ogni Critico ottenuto} nella prova di magia puoi aggiungere 2 punti ai 2d4 tirati.

\medskip\textbf{Bocca Magica}\index{Incantesimi - Bocca Magica}\\
\textbf{Scuola}: Illusione\\
\textbf{Difficoltà}: 19\\
\textbf{Tempo di Lancio}: 1 minuto\\
\textbf{Gittata}: 9 metri\\
\textbf{Componenti}: V, S, M (un piccolo pezzo di favo e polvere di giada del valore di almeno 10 mo, che l'incantesimo consuma)\\
\textbf{Durata}: Fino a che dissolto\\
Impianti un messaggio in un oggetto a gittata, messaggio che viene pronunciato quando si soddisfa la condizione di attivazione. Scegli un oggetto che puoi vedere e che non sia indossato o trasportato da un'altra creatura. Poi pronuncia il messaggio, che deve essere di 25 parole o meno, ma può essere distribuito in un periodo di massimo 10 minuti. Infine, determina la circostanza che attiverà l'incantesimo, affinché questo trasmetta il tuo messaggio.\\
Quando la circostanza si manifesta, una bocca magica appare sull'oggetto e recita il messaggio con la tua voce e allo stesso volume con cui l'hai pronunciato. Se l'oggetto da te scelto ha una bocca o qualcosa che assomiglia a una bocca (per esempio, la bocca di una statua), la bocca magica appare così che le parole sembrino provenire dalla bocca dell'oggetto. Quando lanci questo incantesimo, puoi far sì che l'incantesimo termini dopo aver trasmesso il suo messaggio, o che perduri e ripeta il messaggio ogni volta che la condizione si attiva.\\
La circostanza di attivazione può essere generica o dettagliata quanto desideri, ma deve essere basata su condizioni visibili o udibili che avvengono entro 9 metri dall'oggetto. Per esempio, potresti istruire la bocca di parlare quando una qualsiasi creatura si avvicina entro 9 metri dall'oggetto o quando una campanella d'argento suona entro 9 metri da esso.

\medskip\textbf{Caduta Morbida}\index{Incantesimi - Caduta Morbida}\\
\textbf{Scuola}: Trasmutazione\\
\textbf{Difficoltà}: 16\\
\textbf{Tempo di Lancio}: 1 reazione, che effettui quando tu o una creatura entro 18 metri da te cadete\\
\textbf{Gittata}: 18 metri\\
\textbf{Componenti}: V, M (una piccola piuma o un pezzo di piuma)\\
\textbf{Durata}: 1 minuto\\
Scegli fino a cinque creature a gittata. La velocità di discesa di una creatura che cade diminuisce a 18 metri per round fino al termine dell'incantesimo. Se la creatura atterra prima del termine dell'incantesimo, non subisce danni da caduta e può atterrare sui suoi piedi; per quella creatura l'incantesimo ha termine.\\

\medskip\textbf{Calmare Emozioni}\index{Incantesimi - Calmare Emozioni}\\
\textbf{Scuola}: Ammaliamento\\
\textbf{Difficoltà}: 19\\
\textbf{Tempo di Lancio}: 2 Azioni\\
\textbf{Gittata}: 18 metri\\
\textbf{Componenti}: V, S\\
\textbf{Durata}: Concentrazione, massimo 1 minuto\\
Tenti di sopprimere le forti emozioni in un gruppo di persone. Ogni umanoide in una sfera di 6 metri di raggio centrata su di un punto a gittata da te scelto, deve effettuare un Tiro Salvezza su Volontà; se lo desidera, una creatura può scegliere di fallire questo Tiro Salvezza. Se una creatura fallisce il Tiro Salvezza, scegli uno di questi due effetti. \\
\textit{Placare}. Puoi sopprimere qualsiasi effetto che renda il bersaglio affascinato o spaventato. Quando questo incantesimo termina, gli effetti soppressi riprendono, purché la loro durata non sia nel frattempo esaurita.\\
\textit{Indifferenza}. Puoi rendere un bersaglio indifferente nei confronti di una creatura di tua scelta, verso la quale è ostile. Questa indifferenza termina se il bersaglio viene attaccato o danneggiato da un incantesimo o se vede uno dei suoi amici venir danneggiato. Quando l'incantesimo termina, la creatura diventa di nuovo ostile, a meno che il Narratore non determini diversamente.

\medskip\textbf{Camminare sull'Acqua}\index{Incantesimi - Camminare sull'Acqua}\\
\textbf{Scuola}: Trasmutazione\\
\textbf{Difficoltà}: 21\\
\textbf{Tempo di Lancio}: 2 Azioni\\
\textbf{Gittata}: 9 metri\\
\textbf{Componenti}: V, S, M (un pezzo di sughero)\\
\textbf{Durata}: 1 ora\\
Questo incantesimo conferisce la capacità di muoversi attraverso superfici liquide (come acqua, acido, fango, neve, sabbie mobili o lava) come se fossero innocuo terreno solido (le creature che attraversano la lava fusa possono comunque subire danni dal calore). Fino a dieci creature consenzienti a gittata, e che puoi vedere, ricevono questa capacità per tutta la durata. Se il tuo bersaglio è immerso in un liquido, l'incantesimo riporta il bersaglio in superficie del liquido a una velocità di 9 metri per round. 

\medskip\textbf{Camminare nel Vento}\index{Incantesimi - Camminare nel Vento}\\
\textbf{Scuola}: Trasmutazione\\
\textbf{Difficoltà}: 29\\
\textbf{Tempo di Lancio}: 1 minuto\\
\textbf{Gittata}: 9 metri\\
\textbf{Componenti}: V, S, M (fuoco e Acqua Benedetta)\\
\textbf{Durata}: 8 ore\\
Per la durata, tu e fino ad altre dieci creature consenzienti a gittata, che puoi vedere, assumete forma gassosa, diventando nubi. Mentre è in forma di nube, una creatura ha velocità di volo 90 metri e ha resistenza ai danni dalle armi non magiche. Ritornare alla forma normale richiede 1 minuto, durante il quale la creatura è inabile e non può muoversi. Fino al termine dell'incantesimo, una creatura può tornare alla forma di nube, che richiede una trasformazione di un minuto. Se una creatura è in forma di nube e sta volando quando l'effetto ha termine, la creatura scende 18 metri per round al minuto finché non atterra, al sicuro. Se non riesce ad atterrare dopo 1 minuto, la creatura cadrà per la distanza rimanente.

\medskip\textbf{Charme su Persone}\index{Incantesimi - Charme su Persone}\\
\textbf{Scuola}: Ammaliamento\\
\textbf{Difficoltà}: 16\\
\textbf{Tempo di Lancio}: 2 Azioni\\
\textbf{Gittata}: 9 metri\\
\textbf{Componenti}: V, S\\
\textbf{Durata}: 1 ora\\
Cerchi di affascinare un umanoide a gittata e che puoi vedere. Egli deve effettuare un Tiro Salvezza su Volontà, e avrà +1d6 se sta combattendo contro di te o i tuoi alleati. Se fallisce il Tiro Salvezza, è affascinato da te fino al termine dell'incantesimo o finché tu o i tuoi alleati non gli facciate qualcosa di nocivo. La creatura affascinata ti considera un amichevole conoscente. Quando l'incantesimo termina, la creatura è consapevole di essere stata affascinata da te. \\
\textbf{Per ogni Critico ottenuto} nella prova di magia puoi puoi aggiungere una creatura come bersaglio. Quando lanci l'incantesimo, le creature bersaglio devono trovarsi entro 9 metri l'una dall'altra.

\medskip\textbf{Campo Anti-Magia}\index{Incantesimi - Campo Anti-Magia}\\
\textbf{Scuola}: Abiurazione\\
\textbf{Difficoltà}: 34\\
\textbf{Tempo di Lancio}: 2 Azioni\\
\textbf{Gittata}: Personale (sfera di 3 metri di raggio)\\
\textbf{Componenti}: V, S, M (un pizzico di ferro in polvere o lime di ferro)\\
\textbf{Durata}: Concentrazione, massimo 1 ora\\
Vieni circondato da una sfera invisibile di anti-magia di 3 metri di raggio. Quest'area è separata dall'energia magica che permea il multiverso. All'interno della sfera non si possono lanciare incantesimi, le creature richiamate scompaiono e anche gli oggetti magici diventano normali. Fino al termine dell'incantesimo, la sfera si muove con te, centrata su di te. Gli incantesimi e altri effetti magici, eccetto quelli creati da un artefatto o Patrono, sono soppressi all'interno della sfera né vi possono penetrare. Uno slot speso per lanciare un incantesimo soppresso è consumato. Mentre un effetto è soppresso, non funziona, ma il tempo che trascorre soppresso è conteggiato per la sua durata. 
\\\textit{Effetti con Bersaglio}. Incantesimi e altri effetti magici, come dardo incantato e charme su persone, che prendono come bersaglio una creatura o un oggetto all'interno della sfera non hanno effetto su quel bersaglio.
\\\textit{Aree di Magia}. L'area di un altro incantesimo o effetto magico, come palla di fuoco, non può estendersi all'interno della sfera. Se la sfera si sovrappone a un'area di magia, la parte di quell'area coperta dalla sfera viene soppressa. Per esempio, le fiamme generate da un muro di fuoco vengono soppresse all'interno della sfera, creando un buco nel muro se la sovrapposizione è sufficientemente grande. Incantesimi. Qualsiasi incantesimo o altro effetto magico attivo su di una creatura od oggetto all'interno della sfera viene soppresso finché la creatura o l'oggetto si trovano all'interno della sfera.\\
\textit{Oggetti Magici}. Le proprietà e poteri degli oggetti magici vengono soppressi dalla sfera. Per esempio, una spada lunga +1 all'interno della sfera funziona come una spada lunga non magica. Le proprietà e i poteri delle armi magiche vengono soppressi se sono usati contro un bersaglio all'interno della sfera o impugnate da un attaccante dentro la sfera. Se un'arma magica o munizione magica lascia interamente la sfera (per esempio, se tiri una freccia magica o scagli una lancia magica a un bersaglio al di fuori della sfera), la magia dell'oggetto non è più soppressa non appena esce dalla sfera.
\\\textit{Magia di Viaggio}. Il teletrasporto e il viaggio planare non funzionano all'interno della sfera, che la sfera sia il punto di destinazione o di partenza di questo viaggio magico. All'interno della sfera, un portale verso un altro luogo, mondo, o piano di esistenza, così come uno spazio extradimensionale come quello creato dall'incantesimo trucco della corda, resta chiuso.
\\\textit{Creature e Oggetti}. All'interno della sfera, una creatura o oggetto evocati o creati dalla magia svaniscono temporaneamente dall'esistenza. La creatura od oggetto riappare istantaneamente una volta che lo spazio occupato da essa non si trova più all'interno della sfera.
\\\textit{Dissolvi magie}. Gli incantesimi e gli effetti magici come dissolvi magie non hanno effetto sulla sfera. Allo stesso modo, le sfere create da altri incantesimi campo antimagia non si annullano vicendevolmente. 

\medskip\textbf{Camuffare Sé Stesso}\index{Incantesimi - Camuffare Sé Stesso}\\
\textbf{Scuola}: Illusione\\
\textbf{Difficoltà}: 16\\
\textbf{Tempo di Lancio}: 2 Azioni\\
\textbf{Gittata}: Personale\\
\textbf{Componenti}: V, S\\
\textbf{Durata}: 1 ora\\
Cambi il tuo aspetto, assieme a quello dei tuoi abiti, armatura, armi e altri oggetti che indossi, fino al termine dell'incantesimo o finché non impieghi un'azione per interrompere l'incantesimo. Puoi apparire 30 centimetri più basso o più alto, magro, grasso o una via di mezzo. Non puoi modificare la tua conformazione fisica, quindi devi adottare una forma che abbia la medesima distribuzione di arti. Per tutto il resto, l'illusione è limitata solo dalla tua fantasia.\\
I cambi apportati da questo incantesimo non sono in grado di sostenere un'ispezione fisica. Per esempio, se usi questo incantesimo per aggiungere un cappello al tuo abbigliamento, gli oggetti attraversano il cappello, e chiunque lo tocchi non avvertirebbe nulla e finirebbe per toccarti la testa e i capelli. Se usi questo incantesimo per apparire più magro di quello che sei, la mano di una persona che provasse a toccarti rimbalzerebbe su di te, mentre alla vista sembrerebbe fermarsi a mezz'aria. Per distinguere il tuo camuffamento, una creatura può usare 2 Azioni per ispezionare il tuo aspetto e deve superare una prova di Consapevolezza+4 contro la DC del Tiro Salvezza dell'incantesimo. 

\medskip\textbf{Capanna}\index{Incantesimi - Capanna}\\
\textbf{Scuola}: Invocazione\\
\textbf{Difficoltà}: 21\\
\textbf{Tempo di Lancio}: 1 minuto\\
\textbf{Gittata}: Personale (semisfera di 3 metri di raggio)\\
\textbf{Componenti}: V, S, M (una piccola biglia di cristallo)\\
\textbf{Durata}: 8 ore\\
Una cupola di forza immobile del raggio di 3 metri si forma intorno e sopra di te, restando stazionaria per la durata. L'incantesimo termina se lasci l'area. Nove creature di taglia Media o inferiore possono entrare nella cupola insieme a te. L'incantesimo fallisce se l'area include una creatura più grande o più di dieci creature. Le creature e gli oggetti all'interno della cupola, quando lanci questo incantesimo, la possono attraversare liberamente. Tutte le altre creature e oggetti sono proibiti dall'attraversarla. Incantesimi e altri effetti magici non possono estendersi oltre la cupola o attraversarla (non puoi lanciare una palla di fuoco all'esterno della Capanna). L'atmosfera all'interno dello spazio è confortevole e asciutta, quale che sia il clima all'esterno.\\
Fino al termine dell'incantesimo, puoi comandare che l'interno diventi illuminato fioco o buio. La cupola è opaca dall'esterno, di qualsiasi colore tu scelga, ma è trasparente dall'interno. 

\medskip\textbf{Caratteristica Potenziata}\index{Incantesimi - Caratteristica Potenziata}\\
\textbf{Scuola}: Trasmutazione\\
\textbf{Difficoltà}: 19\\
\textbf{Tempo di Lancio}: 2 Azioni\\
\textbf{Gittata}: Contatto\\
\textbf{Componenti}: V, S, M (pelo o piuma di una bestia)\\
\textbf{Durata}: massimo 10 minuti\\
Conferisci un potenziamento magico a una creatura con cui sei in contatto. Scegli uno degli effetti seguenti; il bersaglio ottiene quell'effetto fino al termine dell'incantesimo.\\
\textit{Astuzia della Volpe}. Il bersaglio ha +1d6 alle prove di Intelligenza e Forza\\
\textit{Forza del Toro}. Il bersaglio ha +1d6 alle prove di Forza, e la sua capacità di Ingombro raddoppia.\\
\textit{Grazia del Energia Luminosa}. Il bersaglio ha +1d6 alle prove di Destrezza. Inoltre, qualora non sia inabile, non subisce danni dalle cadute di 6 metri o meno.\\
\textit{Resistenza dell'Orso}. Il bersaglio ha +1d6 alle prove di Costituzione. Ottiene anche 2d6 punti ferita temporanei, che vengono persi alla fine dell'incantesimo.\\
\textit{Saggezza del Gufo}. Il bersaglio ha +1d6 alle prove di Saggezza. \\
\textit{Splendore dell'Aquila}. Il bersaglio ha +1d6 alle prove di Carisma.\\
\textbf{Per ogni Critico ottenuto} nella prova di magia puoi prendere come bersaglio un'ulteriore creatura

\medskip\textbf{Carne in Pietra - Pietra in Carne}\index{Incantesimi - Carne in Pietra}\\
\textbf{Scuola}: Trasmutazione\\
\textbf{Difficoltà}: 29\\
\textbf{Tempo di Lancio}: 2 Azioni\\
\textbf{Gittata}: 18 metri\\
\textbf{Componenti}: V, S, M (un pizzico di lime, acqua e terra)\\
\textbf{Durata}: Permanente\\
Cerchi di trasformare in pietra una creatura a gittata che puoi vedere. Se il corpo del bersaglio è fatto di carne, la creatura deve effettuare un Tiro Salvezza su Tempra. Se fallisce il Tiro Salvezza, è intralciata e la sua carne comincia a indurirsi. Se supera il Tiro Salvezza, la creatura non subisce l'incantesimo. Una creatura intralciata da questo incantesimo deve effettuare un altro Tiro Salvezza su Tempra al termine di ciascun suo round. Se supera il Tiro Salvezza con successo per tre volte, l'incantesimo termina. Se fallisce il Tiro Salvezza per tre volte, viene trasformata in pietra e resta vittima della condizione pietrificato per la durata. I successi e i fallimenti non devono essere continuativi; tenere traccia di entrambi finché il bersaglio non ne ottiene tre di un tipo.\\
Se la creatura viene danneggiata fisicamente mentre è pietrificata, soffre di deformità simili ai danni arrecati alla pietra, se ritorna al suo stato originale. Se mantieni la tua concentrazione su questo incantesimo per la sua intera possibile durata, la creatura è trasformata in pietra finché l'effetto non viene rimosso.\\
L'incantesimo \emph{Pietra in Carne} fa tornare una creatura di carne purché non sia stata trasformata da piu' di un anno.\\


\medskip\textbf{Catena di Fulmini}\index{Incantesimi - Catena di Fulmini}\\
\textbf{Scuola}: Invocazione\\
\textbf{Difficoltà}: 29\\
\textbf{Tempo di Lancio}: 2 Azioni\\
\textbf{Gittata}: 45 metri\\
\textbf{Componenti}: V, S, M (un po' di pelliccia; un pezzo d'ambra, vetro o una verga di cristallo; e tre spille d'argento)\\
\textbf{Durata}: Istantanea\\
Crei una saetta di luce che colpisce un bersaglio a gittata che puoi vedere, scelto da te. Da questo si genera una ulteriore saetta che colpisce il piu' vicino bersaglio entro 6 metri. Il processo continua finche' non sono state colpite 7 bersagli o non c'e' piu' nessun nuovo avversario a distanza. Un bersaglio può essere una creatura o oggetto almeno di taglia media e può essere bersaglio di una sola saetta. Un bersaglio deve effettuare un Tiro Salvezza su Riflessi. Il bersaglio subisce 10d8 danni da fulmine se fallisce il Tiro Salvezza, o la metà di questi danni se lo supera.\\
\textbf{Per ogni Critico ottenuto} nella prova di magia la saetta si protende su un ulteriore bersaglio.

\medskip\textbf{Cecità/Sordità}\index{Incantesimi - Cecità/Sordità}\\
\textbf{Scuola}: Necromanzia\\
\textbf{Difficoltà}: 19\\
\textbf{Tempo di Lancio}: 2 Azioni\\
\textbf{Gittata}: 9 metri\\
\textbf{Componenti}: V\\
\textbf{Durata}: 1 minuto, Concentrazione\\
Puoi accecare o assordare un nemico. Scegli una creatura a gittata e che puoi vedere. Il bersaglio deve effettuare un Tiro Salvezza su Tempra. Se lo fallisce, il bersaglio è accecato o assordato (a tua scelta) per la durata.\\

\medskip\textbf{Cecità/Sordità Avanzata}\index{Incantesimi - Cecità/Sordità}\\
\textbf{Scuola}: Necromanzia\\
\textbf{Difficoltà}: 21\\
\textbf{Tempo di Lancio}: 2 Azioni\\
\textbf{Gittata}: 9 metri\\
\textbf{Componenti}: V,S,M (del cerume oppure un pezzo di stoffa nera)\\
\textbf{Durata}: 1 minuto\\
Puoi accecare o assordare un nemico. Scegli una creatura a gittata e che puoi vedere. Il bersaglio deve effettuare un Tiro Salvezza su Tempra. Se lo fallisce, il bersaglio è accecato o assordato (a tua scelta) per la durata.\\
\textbf{Per ogni Critico ottenuto} nella prova di magia puoi prendere come bersaglio una creatura aggiuntiva.

\medskip\textbf{Celare}\index{Incantesimi - Celare}\\
\textbf{Scuola}: Trasmutazione\\
\textbf{Difficoltà}: 31\\
\textbf{Tempo di Lancio}: 2 Azioni\\
\textbf{Gittata}: Contatto\\
\textbf{Componenti}: V, S, M (una polvere composta da polvere di diamante, smeraldo, rubino e zaffiro del valore di almeno 50.000 mo, che l'incantesimo consuma)\\
\textbf{Durata}: Fino a che dissolto \\
Tramite questo incantesimo, una creatura consenziente o un oggetto può essere nascosto, impossibile da individuare per la durata. Eseguendo questo incantesimo ed entrando in contatto con un bersaglio, questo diventa invisibile e non può essere preso come bersaglio dagli incantesimi di divinazione, né percepito da sensori di scrutamento creati da incantesimi di divinazione.\\
Se il bersaglio è una creatura, cade in uno stato di animazione sospesa. Per lui il tempo cessa di scorrere, e non invecchia. \\
Puoi predisporre una condizione per cui l'incantesimo termini anticipatamente. La condizione può essere qualsiasi cosa tu voglia, ma deve avvenire o essere visibile entro 1,5 chilometri dal bersaglio. Esempi includono "al prossimo giudizio dei Patroni" o "quando il tarrasque si risveglia". Questo incantesimo termina anche qualora il bersaglio subisca danni.


\medskip\textbf{Cerchio Magico}\index{Incantesimi - Cerchio Magico}\\
\textbf{Scuola}: Abiurazione\\
\textbf{Difficoltà}: 21\\
\textbf{Tempo di Lancio}: 1 minuto\\
\textbf{Gittata}: 3 metri\\
\textbf{Componenti}: V, S, M (Acqua Benedetta o argento e ferro in polvere del valore di almeno 100 mo, che l'incantesimo consuma)\\
\textbf{Durata}: 1 ora\\
Crei un cilindro di energia magica di 3 metri di raggio e alto 6 metri, centrato su di un punto del terreno a gittata e che puoi vedere. Rune luminose compaiono dovunque il cilindro si intersechi con il pavimento o altra superficie.\\
Scegli uno o più dei seguenti tipi di creature: celestiali, elementali, fatati, demoni o non morti. Il circolo influisce su di una creatura del tipo scelto nei modi seguenti:\\
\begin{itemize}
	\item 
La creatura non può entrare consapevolmente nel cilindro tramite alcun mezzo non magico. Se la creatura prova a usare il teletrasporto o il viaggio tra i piani per farlo, deve prima superare un Tiro Salvezza su Volontà.
	\item 
La creatura ha -1d6 ai Tiri per Colpire contro i bersagli all'interno del cilindro.
	\item 
I bersagli all'interno del cilindro non possono essere affascinati, spaventati o posseduti dalla creatura. Quando lanci questo incantesimo, puoi decidere che la magia operi in direzione opposta, impedendo a una creatura del tipo specificato di lasciare il cilindro e proteggendo i bersagli all'esterno.
\end{itemize}
\textbf{Per ogni Critico ottenuto} nella prova di magia puoi aumentare la durata di 1 ora.

\medskip\textbf{Cerchio di Morte}\index{Incantesimi - Cerchio di Morte}\\
\textbf{Scuola}: Invocazione\\
\textbf{Difficoltà}: 29\\
\textbf{Tempo di Lancio}: 2 Azioni\\
\textbf{Gittata}: 45 metri\\
\textbf{Componenti}: V, S, M (una perla nera ridotta in polvere del valore di almeno 500 mo)\\
\textbf{Durata}: Istantanea\\
Una sfera di energia negativa del raggio di 18 metri, erutta in un punto a gittata. Ogni creatura in quell'area deve effettuare un Tiro Salvezza su Tempra. Un bersaglio subisce 8d6 danni da Vuoto se fallisce il Tiro Salvezza, o la metà di questi danni se lo supera. \\
\textbf{Per ogni Critico ottenuto} nella prova il danno aumenta di 2d6.

\medskip\textbf{Cerchio di Teletrasporto}\index{Incantesimi - Cerchio di Teletrasporto}\\
\textbf{Scuola}: Evocazione\\
\textbf{Difficoltà}: 26\\
\textbf{Tempo di Lancio}: 1 minuto\\
\textbf{Gittata}: 3 metri\\
\textbf{Componenti}: V, M (gessi e inchiostri rari infusi di gemme preziose del valore di almeno 50 mo, che l'incantesimo consuma)\\
\textbf{Durata}: 1 round\\
Mentre lanci l'incantesimo, tracci un cerchio di 3 metri di diametro sul pavimento, inscritto con sigilli che collegano il posto in cui ti trovi a un cerchio di teletrasporto permanente di tua scelta, di cui conosci la sequenza dei sigilli e che si trovi sullo stesso piano di esistenza in cui ti trovi tu. Un portale luminoso si apre all'interno del cerchio tracciato da te e resta aperto fino al termine del tuo prossimo round. Qualsiasi creatura che entri nel portale, riappare istantaneamente entro 1 metro dal cerchio di destinazione o nello spazio non
occupato più vicino, se non può comparire entro 1 metro da esso.\\
Molti grandi templi, gilde, e altri luoghi importanti possiedono dei cerchi di teletrasporto permanenti, incisi da qualche parte nelle loro prossimità. Ciascuno di questi cerchi possiede una sequenza di sigilli unica: una serie di rune magiche disposte seguendo una trama precisa.\\ Quando ottieni la capacità di lanciare questo incantesimo, apprendi le sequenze di sigilli di
due destinazioni sul Piano Materiale, determinate dal Narratore. Nel corso delle tue avventure puoi imparare nuove sequenze di sigilli. Puoi mandare a memoria una sequenza di sigilli dopo averla studiata per almeno 1 minuto.\\
Puoi creare un cerchio di teletrasporto permanente eseguendo questo incantesimo nello stesso luogo ogni giorno per un anno. Non devi usare il cerchio di teletrasporto quando lanci l'incantesimo in questo modo.

\medskip\textbf{Chiaroveggenza}\index{Incantesimi - Chiaroveggenza}\\
\textbf{Scuola}: Divinazione\\
\textbf{Difficoltà}: 21\\
\textbf{Tempo di Lancio}: 10 minuti\\
\textbf{Gittata}: 1,5 chilometri\\
\textbf{Componenti}: V, S, M (un focus del valore di almeno 100 mo, che sia un corno ingioiellato per udire o un occhio di vetro per guardare)\\
\textbf{Durata}: Concentrazione, massimo 10 minuti\\
Crei un sensore invisibile in un luogo a te familiare e che sia a gittata (un luogo che hai già visitato o visto precedentemente) o in un luogo ovvio ma che non ti è familiare (come dietro una porta o un angolo, o in mezzo un boschetto di alberi). Il sensore rimane sul posto per la durata, e non può essere attaccato né altrimenti vi si può interagire. Quando lanci questo incantesimo, scegli se vedere o udire. Puoi usare il senso scelto tramite il sensore, come ti trovassi nel suo spazio. Con due azioni, puoi passare da udire a sentire e viceversa. Una creatura che può vedere il sensore (una creatura munita di vedere invisibilità o di visione del vero) lo percepisce come un orbe intangibile e luminoso delle dimensioni del tuo pugno.

\medskip\textbf{Clone}\index{Incantesimi - Clone}\\
\textbf{Scuola}: Necromanzia\\
\textbf{Difficoltà}: 34\\
\textbf{Gittata}: Contatto
\textbf{Componenti}: V, S, M (un diamante del valore di almeno 1.000 mo e almeno 16 centimetri cubi di carne della creatura che deve essere clonata, che l'incantesimo consuma, e un recipiente da almeno 2.000 mo di valore che abbia un coperchio sigillabile e sia grande a sufficienza da contenere una creatura Media, come una grossa urna, una bara, una fossa piena di fango nel terreno o un contenitore di cristallo pieno di acqua salata)\\
\textbf{Durata}: Istantanea\\
Questo incantesimo produce il duplicato inerte di una creatura vivente come salvaguardia dalla morte. Questo clone si forma all'interno di un recipiente sigillato e raggiunge la massima dimensione e maturità dopo 120 giorni; puoi anche decidere che il clone sia una versione più giovane della stessa creatura. Rimane inerte e sopravvive all'infinito, purché il recipiente resti indisturbato.\\
In qualsiasi momento dopo che il clone è maturato, se la creatura originale muore, la sua anima si trasferisce nel clone, purché l'anima sia libera e consenziente a tornare. Il clone è fisicamente identico all'originale e ha la stessa personalità, ricordi e caratteristiche, ma nulla dell'equipaggiamento dell'originale. I resti fisici della creatura originale, se esistono ancora, divengono inerti e non possono essere riportati alla vita, dato che l'anima della creatura è altrove. \\
\textbf{Questo incantesimo non e' selezionabile se i Patroni sono attivi}

\medskip\textbf{Colpo Accurato}\index{Trucchetto - Colpo Accurato}\\
\textbf{Scuola}: Divinazione\\
\textbf{Difficoltà}: 12\\
\textbf{Tempo di Lancio}: 2 Azioni\\
\textbf{Gittata}: 9 metri\\
\textbf{Componenti}: S\\
\textbf{Durata}: 1 round\\
Allunghi la mano e punti il dito verso un bersaglio a gittata. La tua magia ti conferisce una breve comprensione delle difese del bersaglio. Durante il tuo prossimo round, purché questo incantesimo non sia terminato, ottieni +1d6 al primo tiro per colpire contro quel bersaglio.

\medskip\textbf{Colpo Infuocato}\index{Incantesimi - Colpo Infuocato}\\
\textbf{Scuola}: Invocazione\\
\textbf{Difficoltà}: 26\\
\textbf{Tempo di Lancio}: 2 Azioni\\
\textbf{Gittata}: 18 metri\\
\textbf{Componenti}: V, S, M (pizzico di zolfo)\\
\textbf{Durata}: Istantanea\\
Una colonna verticale di fuoco divino scende dal cielo e si abbatte sul luogo da te specificato. Ogni creatura in un cilindro di 3 metri di raggio e alto 12 metri centrato su di un punto a gittata deve effettuare un Tiro Salvezza su Riflessi. Una creatura subisce 8d6 danni da Luce se fallisce il Tiro Salvezza, o la metà di questi danni se lo supera.\\
\textbf{Per ogni Critico ottenuto} nella prova di magia il danno Luce aumenta di 2d6.

\medskip\textbf{Comando}\index{Incantesimi - Comando}\\
\textbf{Scuola}: Ammaliamento\\
\textbf{Difficoltà}: 16\\
\textbf{Tempo di Lancio}: 2 Azioni\\
\textbf{Gittata}: 18 metri\\
\textbf{Componenti}: V\\
\textbf{Durata}: 1 round\\
Pronunci un comando di una parola verso una creatura a gittata e che puoi vedere. Il bersaglio deve superare un Tiro Salvezza su Volontà o eseguire il comando entro il suo prossimo round. L'incantesimo non ha effetto se il bersaglio è non morto, se non capisce la tua lingua, o se il tuo comando gli recherebbe danni. Seguono alcuni tipici comandi e i loro effetti. Puoi dare comandi diversi da quelli descritti qui, e in quel caso il Narratore determinerà il comportamento del bersaglio. Se il bersaglio non può eseguire il tuo comando, l'incantesimo ha fine.
\begin{itemize}
	\item 
\textit{Avvicinati}. Il bersaglio si muove verso di te per il tragitto più breve e diretto, terminando il suo round se si avvicina a 1 metro da te.
	\item 
\textit{Fermo}. Il bersaglio non si muove e poi termina il suo round. Una creatura volante resta sul posto, purché le sia possibile. Se deve muoversi per restare in aria, vola la distanza minima necessaria per farlo.
	\item 
	\textit{Getta}. Il bersaglio getta qualsiasi cosa stia tenendo in mano e poi termina il suo round. 	
	\item 
	\textit{Scappa}. Il bersaglio spende il suo round a muoversi lontano da te con il mezzo più veloce a sua disposizione.
	\item \textit{Striscia}. Il bersaglio si getta prono e poi termina il suo round.
\end{itemize}

\textbf{Per ogni Critico ottenuto} nella prova di magia puoi agire su di un'ulteriore creatura. Nel momento in cui lanci l'incantesimo, le creature bersaglio devono trovarsi entro 9 metri l'una da l'altra ed eseguono il medesimo comando.

\medskip\textbf{Comprensione dei Linguaggi}\index{Incantesimi - Comprensione dei Linguaggi}\\
\textbf{Scuola}: Divinazione\\
\textbf{Difficoltà}: 16\\
\textbf{Tempo di Lancio}: 2 Azioni\\
\textbf{Gittata}: Personale\\
\textbf{Componenti}: V, S, M (un pizzico di sale e fuliggine)\\
\textbf{Durata}: 1 ora\\
Per la durata, capisci il significato letterale di qualsiasi linguaggio parlato che ascolti.

\medskip\textbf{Comprensione degli Scritti}\index{Incantesimi - Comprensione dei Scritti}\\
\textbf{Scuola}: Divinazione\\
\textbf{Difficoltà}: 19\\
\textbf{Tempo di Lancio}: 2 Azioni\\
\textbf{Gittata}: Personale\\
\textbf{Componenti}: V, S, M (un pizzico di argento e inchiostro secco)\\
\textbf{Durata}: 1 ora\\
Per la durata comprendi qualsiasi linguaggio scritto non magico che vedi. Devi essere a contatto con la superficie su cui le parole sono scritte. Per leggere una pagina di testo impieghi 1 minuto. Questo incantesimo non decodifica i messaggi segreti in un testo o glifo, come un sigillo arcano, che non faccia parte di un linguaggio scritto.

\medskip\textbf{Compulsione}\index{Incantesimi - Compulsione}\\
\textbf{Scuola}: Ammaliamento\\
\textbf{Difficoltà}: 23\\
\textbf{Tempo di Lancio}: 2 Azioni\\
\textbf{Gittata}: 9 metri\\
\textbf{Componenti}: V, S\\
\textbf{Durata}: Concentrazione, massimo 1 minuto\\
Le creature di tua scelta entro la gittata, che puoi vedere e che ti possono sentire, devono effettuare un Tiro Salvezza su Volontà. Un bersaglio supera automaticamente il Tiro Salvezza se non può essere affascinato. Fino al termine dell'incantesimo, puoi usare un'Azione durante ciascun tuo round per indicare una direzione orizzontale rispetto a te. Ogni bersaglio soggetto all'incantesimo deve usare quanto più possibile del suo movimento, durante il suo prossimo round, per muoversi in quella direzione. Il bersaglio non può effettuare nessuna azione prima di muoversi. Dopo essersi mosso in questo modo, il bersaglio può effettuare un altro Tiro Salvezza su Volontà per tentare di terminare l'effetto.\\
Un bersaglio non può essere obbligato a muoversi dentro un pericolo palesemente letale, come fiamme o pozzi, ma per muoversi nella direzione indicata potrà provocare attacchi di opportunità.

\medskip\textbf{Comunione}\index{Incantesimi - Comunione}\\
\textbf{Scuola}: Divinazione\\
\textbf{Difficoltà}: 26\\
\textbf{Tempo di Lancio}: 1 minuto\\
\textbf{Gittata}: Personale\\
\textbf{Componenti}: V, S, M (incenso e una fiala di Acqua Benedetta o blasfema)\\
\textbf{Durata}: 1 minuto\\
Comunichi con il tuo Patrono e gli poni fino a tre domande a cui si può dare risposta con un sì o un no. Devi porre le domande prima della fine dell'incantesimo. Riceverai la risposta corretta a ciascuna domanda. Le creature divine non sono necessariamente onniscienti, quindi potresti ricevere "non è chiaro" come risposta a una domanda che riguarda informazioni non pertinenti alle conoscenze del Patrono. Nel caso in cui una risposta di una parola potrebbe essere fuorviante o contraria agli interessi del Patrono, il Narratore potrebbe invece dare una breve frase come risposta.\\
Se lanci l'incantesimo due o più volte prima che sia sorta la nuova alba c'è una probabilità cumulativa del 25\% che per ogni lancio dopo il primo tu non ottenga alcuna risposta. Il Narratore effettua questo tiro in segreto.\\
\textbf{Questo incantesimo non e' selezionabile se i Patroni non sono attivi}

\medskip\textbf{Comunione con la Natura}\index{Incantesimi - Comunione con la Natura}\\
\textbf{Scuola}: Divinazione\\
\textbf{Difficoltà}: 26\\
\textbf{Tempo di Lancio}: 1 minuto\\
\textbf{Gittata}: Personale\\
\textbf{Componenti}: V, S\\
\textbf{Durata}: Istantanea\\
Per un istante diventi tutt'uno con la natura e ottieni informazioni sul territorio circostante. In ambienti esterni, l'incantesimo ti fornisce informazioni sul territorio entro 5 chilometri da te. In grotte e altri ambienti naturali sotterranei, il raggio è limitato a 100 metri. L'incantesimo non funziona nei luoghi in cui la natura è stata soppiantata da costruzioni, come in sotterranei e paesi.\\
Apprendi immediatamente informazioni su un massimo di tre argomenti a tua scelta su uno dei seguenti soggetti, in relazione all'area:
\begin{itemize}
	\item 
	terreno e corpi d'acqua
	\item 
	piante, minerali, animali e popolazioni prevalenti
	\item 
  potenti celestiali, elementali, fatati, demoni o non morti
	\item 
  influenze da altri piani di esistenza
	\item
  edifici
\end{itemize}

\medskip\textbf{Confusione}\index{Incantesimi - Confusione}\\
\textbf{Scuola}: Ammaliamento\\
\textbf{Difficoltà}: 23\\
\textbf{Tempo di Lancio}: 2 Azioni\\
\textbf{Gittata}: 27 metri\\
\textbf{Componenti}: V, S, M (tre gusci di noce)\\
\textbf{Durata}: 1 minuto\\
Questo incantesimo assale e piega la mente delle creature, generando illusioni e provocando azioni incontrollate. Quando lanci questo incantesimo ogni creatura, in una sfera di 3 metri di raggio centrata su di un punto da te scelto entro la gittata, deve superare un Tiro Salvezza su Volontà o subirne gli effetti. Un bersaglio soggetto all'incantesimo non può effettuare reazioni e deve tirare un d10 all'inizio di ciascun suo round per determinare il proprio comportamento per quel round. 

\medskip

\begin{tabularx}{0.45\textwidth}{lX}
	\hline 
d10 & Comportamento\\ 
1 & La creatura usa tutto il suo movimento per muoversi in una direzione casuale. Per determinare la direzione, tira un d8 assegnando a ciascuna faccia un punto cardinale. La
creatura non effettuerà nessuna azione in questo round. \\
2-6 & La creatura non può muoversi né attaccare in questo round.\\
7-8 & La creatura usa la sua 2 Azioni (e nessun'altra) per effettuare un attacco da mischia contro una creatura determinata a caso entro la sua portata. Se non c'è nessuna creatura a portata, per questo round la creatura non farà nulla.\\
9-10 & La creatura può agire e muoversi normalmente.\\
\end{tabularx} 

\medskip

Al termine di ciascun suo round, un bersaglio soggetto all'incantesimo può effettuare un Tiro Salvezza su Volontà. Se lo supera, per lui l'effetto ha termine. \\
\textbf{Per ogni Critico ottenuto} nella prova di magia il raggio della sfera aumenta di 1 metro.

\medskip\textbf{Cono di Freddo}\index{Incantesimi - Cono di Freddo}\\
\textbf{Scuola}: Invocazione\\
\textbf{Difficoltà}: 26\\
\textbf{Tempo di Lancio}: 2 Azioni\\
\textbf{Gittata}: Personale (cono di 18 metri)\\
\textbf{Componenti}: V, S, M (un piccolo cristallo o cono di vetro)\\
\textbf{Durata}: Istantanea\\
Un'esplosione di aria fredda erutta dalle tue mani. Ogni creatura in un cono di 18 metri deve effettuare un Tiro Salvezza su Tempra. Una creatura subisce 8d8 danni da freddo se fallisce il Tiro Salvezza, o la metà di questi danni se lo supera. Una creatura uccisa da questo incantesimo diventa una statua di ghiaccio fino a quando disgela.\\
\textbf{Per ogni Critico ottenuto} nella prova di magia il danno aumenta di 1d8

\medskip\textbf{Conoscenza delle Leggende}\index{Incantesimi - Conoscenza delle Leggende}\\
\textbf{Scuola}: Divinazione\\
\textbf{Difficoltà}: 26\\
\textbf{Tempo di Lancio}: 10 minuti\\
\textbf{Gittata}: Personale\\
\textbf{Componenti}: V, S, M (incenso del valore di almeno 250 mo, che l'incantesimo consuma, e quattro strisce d'avorio del valore di almeno 50 mo)\\
\textbf{Durata}: Istantanea\\
Nomina o descrivi una persona, luogo od oggetto. L'incantesimo ti porta alla mente un breve riassunto delle conoscenze più importanti sull'argomento da te nominato. Se la cosa da te nominata non ha alcuna rilevanza leggendaria, non ottieni alcuna informazione. Maggiori informazioni hai sull'argomento, più precise e dettagliate saranno le informazioni che riceverai. L'informazione che riceverai sarà accurata, ma celata magari in linguaggio metaforico.

\medskip\textbf{Contagio}\index{Incantesimi - Contagio}\\
\textbf{Scuola}: Necromanzia\\
\textbf{Difficoltà}: 26\\
\textbf{Tempo di Lancio}: 2 Azioni\\
\textbf{Gittata}: Contatto\\
\textbf{Componenti}: V, S\\
\textbf{Durata}: 7 giorni\\
Tramite il contatto puoi infliggere malattie. Effettua un attacco da mischia contro una creatura a portata. Se colpisci, infetti la creatura con una malattia a tua scelta tra quelle descritte di seguito. Al termine di ciascun round del bersaglio, esso deve effettuare un Tiro Salvezza su Tempra. Dopo aver fallito tre di questi Tiri Salvezza, gli effetti della malattia permangono per la durata, e la creatura non effettua più Tiri Salvezza. Dopo aver superato tre di questi Tiri Salvezza, la creatura recupera dalla malattia, e l'incantesimo ha termine. \\
Dato che questo incantesimo induce nel suo bersaglio una malattia naturale, qualsiasi effetto che rimuova le malattie o migliori gli effetti delle malattie si applica a essa.\\
\begin{itemize}
	\item 
	\textit{Carne Putrida}. La pelle della creatura marcisce. La creatura ha -1d6 alle prove di Carisma e ogni danno e' raddoppiato.
\item 
	\textit{Debolezza Accecante}. Il dolore attanaglia la mente della creatura mentre i suoi occhi diventano bianco latte. La creatura ha -1d6 alle prove di Saggezza e ai Tiri Salvezza su Volontà, ed è accecata.
\item 
  \textit{Febbre Lurida}. Una febbre devastante sconvolge il corpo della creatura. La creatura ha -1d6 alle prove di Forza e ai Tiri Salvezza su Tempra, e ai Tiri per Colpire che usano la Forza.
\item 
\textit{Fitte}. La creatura è sopraffatta dai tremiti. La creatura ha -1d6 alle prove di Destrezza e ai Tiri Salvezza su Destrezza, e ai Tiri per Colpire che usano l'Destrezza.
\item 
\textit{Fuoco Mentale}. La mente della creatura è preda della febbre. La creatura ha -1d6 alle prove di Intelligenza e ai Tiri Salvezza su Intelligenza, e si comporta come se in combattimento fosse sotto l'effetto dell'incantesimo confusione.
\item 
\textit{Morte Melmosa}. La creatura inizia a sanguinare incessantemente. La creatura ha -1d6 alle prove di Costituzione e ai Tiri Salvezza su Tempra. Inoltre, ogni qualvolta la creatura subisce danni, è stordita fino alla fine del suo prossimo round.
\end{itemize}

\medskip\textbf{Contingenza}\index{Incantesimi - Contingenza}\\
\textbf{Scuola}: Invocazione\\
\textbf{Difficoltà}: 29\\
\textbf{Tempo di Lancio}: 10 minuti\\
\textbf{Gittata}: Personale\\
\textbf{Componenti}: V, S, M (una statuetta raffigurante te stesso scolpita in avorio e decorata con gemme del valore di almeno 1.500 mo)\\
\textbf{Durata}: 10 giorni\\
Scegli un incantesimo di Difficoltà 23 o più basso che puoi lanciare, che abbia il tempo di lancio di 2 Azioni e che può avere te come bersaglio. Lanci quell'incantesimo (detto incantesimo contingente) come parte del lancio di contingenza, spendendo gli slot incantesimo di entrambi, ma senza che l'incantesimo contingente abbia effetto. Avrà invece effetto quando si avvererà una determinata circostanza. Descrivi questa circostanza mentre lanci i due incantesimi. Per esempio, contingenza lanciato assieme a respirare sott'acqua potrebbe stipulare che respirare sott'acqua entra in azione quando sei immerso nell'acqua o simile liquido.\\
l'incantesimo contingente ha effetto immediatamente dopo che la circostanza si verifica per la prima volta, che tu lo voglia o no, e poi contingenza termina. L'incantesimo contingente agisce solo su di te, anche se normalmente può prendere come bersaglio anche altri. Puoi usare un solo incantesimo contingenza alla volta. Se lanci di nuovo questo incantesimo, l'effetto di un altro incantesimo contingenza su di te avrà termine. Inoltre, contingenza per te ha termine se la componente materiale non dovesse più trovarsi sulla tua persona.


\medskip\textbf{Controincantesimo}\index{Incantesimi - Controincantesimo}\\
\textbf{Scuola}: Abiurazione\\
\textbf{Difficoltà}: 21\\
\textbf{Tempo di Lancio}: 1 reazione, che effettui quando vedi una creatura entro 18 metri da te lanciare un incantesimo\\
\textbf{Gittata}: 18 metri\\
\textbf{Componenti}: S \\
\textbf{Durata}: Istantanea\\
Cerchi di interrompere una creatura nell'atto di lanciare un incantesimo. Se la creatura sta lanciando un incantesimo di Difficoltà 21 o più basso, l'incantesimo fallisce e non ha effetto. 


\medskip\textbf{Controllare Acqua}\index{Incantesimi - Controllare Acqua}\\
\textbf{Scuola}: Trasmutazione\\
\textbf{Difficoltà}: 23\\
\textbf{Tempo di Lancio}: 2 Azioni\\
\textbf{Gittata}: 90 metri\\
\textbf{Componenti}: V, S, M (un goccio d'acqua e un pizzico di polvere)\\
\textbf{Durata}: Concentrazione, massimo 10 minuti\\
Fino al termine dell'incantesimo, controlli qualsiasi acqua libera all'interno dell'area che hai scelto fino a un cubo di 30 metri di spigolo. Quando lanci questo incantesimo puoi scegliere qualsiasi tra i seguenti effetti. Come azione, durante il tuo round, puoi ripetere lo stesso effetto o sceglierne uno diverso.\\
\begin{itemize}
\item 
\textit{Allagamento}. Fai sì che il livello di tutta l'acqua nell'area aumenti fino a 6 metri. Se l'area include una costa, l'acqua inonda la terraferma. Se scegli un'area all'interno di un grosso corpo d'acqua, crei invece un'onda alta 6 metri che viaggia da un lato all'altro dell'area prima di infrangersi. Qualsiasi veicolo di taglia Enorme o inferiore sul percorso dell'onda viene trasportato dall'altro lato. Qualsiasi veicolo di taglia Enorme o inferiore colpito dall'acqua ha una percentuale del 25\% di cappottarsi.\\
Il livello dell'acqua resta elevato fino al termine dell'incantesimo o finché non scegli un effetto diverso. Se questo effetto ha prodotto un'onda, l'onda si ripete all'inizio del tuo round successivo, finché perdura l'effetto di allagamento.\\
\item 
\textit{Dividere le Acque}. Fai sì che l'acqua nell'area si sposti a lato per creare un varco. Il varco si estende per l'area dell'incantesimo, e l'acqua divisa forma un muro su entrambi i lati del varco. Il varco resta fino al termine dell'incantesimo o finché non scegli un effetto diverso. L'acqua tornerà poi lentamente a riempire il varco nel corso del round successivo, fino a che non sarà risalita al suo normale livello.
\item 
\textit{Ridirigere il Flusso}. Fai sì che l'acqua corrente nell'area si muova in una direzione a tua scelta, anche se l'acqua deve superare degli ostacoli, risalire muri o dirigersi verso altre direzioni improbabili. L'acqua nell'area si muove secondo le tue indicazioni, ma una volta giunta oltre l'area dell'incantesimo, riprende il suo flusso in base alle condizioni del terreno. L'acqua continua a muoversi nella direzione da te scelta fino al termine dell'incantesimo o finché non scegli un effetto diverso.
\item 
\textit{Turbine}. Questo effetto richiede un corpo d'acqua che copra un quadrato di 15 metri di lato e abbia una profondità di 7 metri. Fai sì che si formi un turbine al centro dell'area. Il turbine produce un vortice largo 1 metro alla base, largo fino a 15 metri in cima e alto 7
metri. Qualsiasi creatura od oggetto nell'acqua e che si trovi entro 7 metri dal vortice viene trascinato 3 metri verso di esso. Una creatura può nuotare per allontanarsi dal vortice effettuando una prova di Destrezza (Atletica) contro la DC del Tiro Salvezza dell'incantesimo.
Quando una creatura entra nel vortice per la prima volta durante un round o inizia lì il suo round, deve effettuare un Tiro Salvezza su Tempra. Se lo fallisce, la creatura subisce 2d8 danni da botta e viene catturata dal vortice fino al termine dell'incantesimo. Se supera il Tiro Salvezza, la creatura subisce la metà di questi danni, e non è catturata dal vortice. Una creatura catturata dal vortice può usare 3 Azioni per cercare di nuotare via dal vortice come descritto sopra, ma ha -1d6 alle prove di Destrezza (Atletica) per farlo. La prima volta durante ciascun round in un cui un oggetto entra nel vortice, l'oggetto subisce 2d8 danni da botta; questo danno viene subito ogni round in cui l'oggetto rimane nel vortice.
\end{itemize}


\medskip\textbf{Controllare Tempo Atmosferico}\index{Incantesimi - Controllare Tempo Atmosferico}\\
\textbf{Scuola}: Trasmutazione\\
\textbf{Difficoltà}: 34\\
\textbf{Tempo di Lancio}: 10 minuti\\
\textbf{Gittata}: Personale (raggio di 1,5 chilometri)\\
\textbf{Componenti}: V, S, M (incenso bruciato e po' di terra e legno mescolati nell'acqua)\\
\textbf{Durata}: Concentrazione, massimo 8 ore \\
Per la durata, assumi il controllo del clima entro 7,5 chilometri da te. Per lanciare questo incantesimo devi essere all'esterno. Muoversi in un posto dove non hai la visuale aperta verso il cielo, termina l'incantesimo anticipatamente. Quando lanci questo incantesimo, cambia le attuali condizioni climatiche, determinate dal Narratore in base alla stagione e la latitudine. Puoi modificare le precipitazioni, la temperatura e il vento. Ci vogliono 1d4 x 10 minuti perché la nuova condizione prenda effetto. Una volta che la condizione avrà preso effetto, potrai cambiarla di nuovo. Quando l'incantesimo termina, il clima tornerà gradualmente alla norma.\\
Quando cambi le condizioni climatiche, trova l'attuale condizione sulla seguente tabella e cambiala di uno stadio, verso l'alto o il basso. Quando cambi il vento, puoi cambiarne anche la direzione. \\
\medskip
\textit{Precipitazione}
\begin{itemize}
	\item 
1 Limpido
	\item 
2 Qualche nuvola
	\item 
3 Coperto o foschia a terra
	\item 
4 Pioggia, grandine o neve
	\item 
5 Pioggia torrenziale, grandinata pesante, tormenta
\end{itemize}

\textit{Temperatura}

\begin{itemize}
 \item 
1 Caldo insopportabile
	\item 
2 Caldo
	\item 
3 Tiepido
	\item 
4 Fresco
	\item 
5 Freddo
	\item 
6 Freddo polare
\end{itemize}

\textit{Vento}
\begin{itemize}
	\item 
1 Calmo
	\item 
2 Vento moderato
	\item 
3 Vento moderato
	\item 
4 Fortunale
	\item 
5 Tempesta
\end{itemize}

\medskip\textbf{Costrizione}\index{Incantesimi - Costrizione}\\
\textbf{Scuola}: Ammaliamento\\
\textbf{Difficoltà}: 26\\
\textbf{Tempo di Lancio}: 1 minuto\\
\textbf{Gittata}: 18 metri\\
\textbf{Componenti}: V\\
\textbf{Durata}: 30 giorni\\
Imponi un comando magico a una creatura a gittata che puoi vedere, obbligandolo ad adempiere un determinato compito o vietandole di svolgere un'azione o corso d'attività deciso da te. Se la creatura ti può capire, deve superare un Tiro Salvezza su Volontà o restare affascinata da te per la durata. Mentre la creatura è affascinata da te, subisce 3d10 danni ogni volta che agisce in maniera direttamente contraria alle tue istruzioni, ma non più di una volta al giorno. Una creatura che non ti può capire ignora gli effetti di questo incantesimo. Puoi dare qualsiasi comando di tua scelta, tranne un'attività che provocherebbe morte certa. Dovessi tu pronunciare un comando suicida, l'incantesimo avrebbe termine.\\
Puoi terminare l'incantesimo usando un'azione. Anche rimuovi maledizione, ristorare superiore o desiderio vi pongono termine.\\
\textbf{Se ottini almeno due Critici} nella prova di magia la durata è 1 anno. Se ottieni 3 Critici l'incantesimo dura finché non viene terminato da uno degli incantesimi sopra menzionati.

\medskip\textbf{Creare Cibo e Acqua}\index{Incantesimi - Creare Cibo e Acqua}\\
\textbf{Scuola}: Evocazione\\
\textbf{Difficoltà}: 21\\
\textbf{Tempo di Lancio}: 2 Azioni\\
\textbf{Gittata}: 9 metri\\
\textbf{Componenti}: V, S\\
\textbf{Durata}: Istantanea\\
Crei 22,5 chili di cibo e 120 litri d'acqua sul terreno o in contenitori a gittata, sufficienti a sostenere fino a quindici umanoidi o cinque cavalcature per 24 ore. Il cibo è blando ma nutriente, e marcisce dopo 24 ore se non viene consumato, come anche l'acqua.

\medskip\textbf{Creare Birra}\index{Incantesimi - Creare Birra}\\
\textbf{Scuola}: Evocazione\\
\textbf{Difficoltà}: 12\\
\textbf{Tempo di Lancio}: 2 Azioni o piu'\\
\textbf{Gittata}: 9 metri\\
\textbf{Componenti}: V, S, M (lievito di birra, malto, acqua)\\
\textbf{Durata}: 1 ora\\
Crei 1 litro di birra. La qualita' e tipologia di birra dipende dal lievito, malto e acqua usata.
Maggiore e' il tempo di lancio dell'incantesimo piu' viene alta la gradazione alcolica, con un tempo di lancio di due azioni la gradazione e' di 4.3, se 1 azione e' analcolica, ogni azione spesa aumenta la gradazione di 0.3 vol fino ad un massimo di 12.5 vol.
Dopo un ora la birra svanisce, se consumata dopo un ora terminano anche eventuali effetti alcolici della stessa sulle persone che l'hanno bevuta.\\
\textbf{Per ogni Critico ottenuto} nella prova di magia aumenti di un litro o di un ora la durata.

\medskip\textbf{Creare o Distruggere Acqua}\index{Incantesimi - Creare o Distruggere Acqua}\\
\textbf{Scuola}: Trasmutazione\\
\textbf{Difficoltà}: 16\\
\textbf{Tempo di Lancio}: 2 Azioni\\
\textbf{Gittata}: 9 metri\\
\textbf{Componenti}: V, S, M (un goccio d'acqua per creare acqua o qualche granello di sale per distruggerla)\\
\textbf{Durata}: Istantanea\\
Crei o distruggi l'acqua.\\
\textit{Creare Acqua}. Crei fino a 40 litri di acqua limpida dalle tue mani che spruzzano fino a 9 metri. In alternativa l'acqua cade come pioggia in un cubo di 9 metri di spigolo che si trovi entro la gittata, estinguendo le fiamme esposte nell'area.\\
L'incantesimo non puo' essere usato su fiamme magiche.\\
\textit{Distruggere Acqua}. Distruggi fino a 40 litri di acqua in un contenitore aperto a gittata. In alternativa, puoi distruggere la nebbia in un cubo di 9 metri di spigolo entro la gittata.\\
\textbf{Per ogni Critico ottenuto} nella prova di magia crei o distruggi ulteriori 40 litri d'acqua, o le dimensioni del cubo aumentano di 1 metro di spigolo in caso di nebbia.\\
L'acqua e' potabile e disseta se bevuta entro un round dalla creazione.

\medskip\textbf{Creare Non Morti}\index{Incantesimi - Creare Non Morti}\\
\textbf{Scuola}: Necromanzia\\
\textbf{Difficoltà}: 29\\
\textbf{Tempo di Lancio}: 2 Azioni\\
\textbf{Gittata}: 3 metri\\
\textbf{Componenti}: V, S, M (un vaso di terracotta pieno di terra di cimitero, un vaso di terracotta pieno di acqua salmastra, e un onice nero del valore di 50 mo per ogni cadavere)\\
\textbf{Durata}: Istantanea\\
Puoi lanciare questo incantesimo solo di notte. Scegli fino a tre cadaveri di umanoidi Medi o Piccoli a gittata. Ogni cadavere diventa un ghoul sotto il tuo controllo (il Narratore possiede le statistiche di gioco di queste creature). Durante il tuo round, con due Azioni, puoi comandare mentalmente una qualsiasi creatura da te animata con questo incantesimo, se la creatura si trova entro 36 metri da te (se controlli più creature, puoi comandarle tutte o solo una nello stesso momento impartendo lo stesso comando). Decidi tu quale azione effettuerà la creatura e dove si muoverà durante il suo prossimo round, oppure puoi impartire un comando generico, come quello di fare la guardia a una specifica stanza o corridoio. Se non impartisci comandi, le creature si limiteranno a difendersi dalle creature ostili. Una volta ricevuto un comando, la creatura continuerà a eseguirlo finche il compito sarà completo. La creatura è sotto il tuo controllo per 24 ore, dopodiché smetterà di rispondere ai comandi che gli impartisci. Per mantenere il controllo della creatura per altre 24 ore, devi lanciare questo incantesimo sulla creatura prima che l'attuale periodo di 24 ore abbia termine. Questo impiego dell'incantesimo riasserisce il tuo controllo su di un massimo di tre creature che hai animato con questo incantesimo, anziché animarne di nuove.\\
\textbf{Se ottieni un Critico} nella prova di magia puoi rianimare o riasserire il controllo su quattro ghoul. Con due Critici puoi animare o riasserire il controllo su cinque
ghoul o due ghast o wight. Con tre Critici puoi animare o riasserire il controllo su sei ghoul, tre ghast o wight, o due mummie. 

\medskip\textbf{Creazione}\index{Incantesimi - Creazione}\\
\textbf{Scuola}: Illusione\\
\textbf{Difficoltà}: 26\\
\textbf{Tempo di Lancio}: 1 minuto\\
\textbf{Gittata}: 9 metri\\
\textbf{Componenti}: V, S, M (un minuscolo pezzo di materiale dello stesso tipo di oggetto che intendi creare) \\
\textbf{Durata}: Speciale\\
Afferri pezzi di materia d'ombra dal piano delle Ombre per creare, a gittata, oggetti non viventi di materia vegetale: beni morbidi, corda, legno o qualcosa di simile. Puoi usare questo incantesimo anche per creare oggetti minerali come pietra, cristallo o metallo. L'oggetto creato non può essere più grosso di un cubo di 1 metro di spigolo, e l'oggetto deve essere di una forma e materiale che hai già visto in passato.\\
La durata dipende dal materiale dell'oggetto. Se l'oggetto è composto da più materiali, usa la durata più breve.
\medskip
Tabella Materiale - Durato
\medskip

\begin{tabularx}{0.45\textwidth}{lX}
	\hline 
Materia vegetale &1 giorno\\
Pietra o cristallo &12 ore\\
Metalli preziosi &1 ora\\
Gemme &10 minuti\\
Adamantio o mithril &1 minuto\\
\end{tabularx} 
\medskip

Usare qualsiasi materiale creato da questo incantesimo come componente materiale di un altro incantesimo farà fallire il nuovo incantesimo.\\
\textbf{Per ogni Critico ottenuto} nella prova di magia il cubo aumenta di 1 metro di spigolo

\medskip\textbf{Crescita di Spuntoni}\index{Incantesimi - Crescita di Spuntoni}\\
\textbf{Scuola}: Trasmutazione\\
\textbf{Difficoltà}: 19\\
\textbf{Tempo di Lancio}: 2 Azioni\\
\textbf{Gittata}: 45 metri\\
\textbf{Componenti}: V, S, M (sette spine affilate o sette ramoscelli, ciascuno di esse appuntito ad un'estremità)\\
\textbf{Durata}: 10 minuti\\
Il terreno in un raggio di 6 metri centrato su di un punto a gittata si contorce e genera spuntoni e spine molto acuminate. Per la durata, l'area diventa terreno difficile. Quando una creatura entra o si muove all'interno dell'area, subisce 2d4 danni per ogni 1 metro percorsi.
La trasformazione del terreno è talmente ben camuffata da sembrare naturale. Qualsiasi creatura che non abbia visto l'area al momento del lancio dell'incantesimo deve effettuare una prova di Consapevolezza contro la DC del Tiro Salvezza dell'incantesimo, per riconoscere il pericolo rappresentato dal terreno prima di entrarvi. 

\medskip\textbf{Crescita Vegetale}\index{Incantesimi - Crescita Vegetale}\\
\textbf{Scuola}: Trasmutazione\\
\textbf{Difficoltà}: 21\\
\textbf{Tempo di Lancio}: 2 Azioni o 8 ore\\
\textbf{Gittata}: 45 metri\\
\textbf{Componenti}: V, S\\
\textbf{Durata}: Istantanea\\
Questo incantesimo incanala vitalità nei vegetali entro una specifica area. Esistono due usi possibili per questo incantesimo, che conferiscono benefici immediati o a lungo termine. Se lanci questo incantesimo impiegando 1 azione, scegli un punto a gittata. Tutte i vegetali normali in un raggio di 30 metri centrato su quel punto diventano densi e folti. Una creatura che attraversa l'area quadruplica il costo del suo movimento.\\
Puoi escludere dai suoi effetti una o più aree di qualsiasi dimensione all'interno dell'area dell'incantesimo.\\
Se lanci questo incantesimo nel corso di 8 ore, nutri la terra. Tutti i vegetali in un raggio di 750 metri centrato su di un punto a gittata diventano super produttivi per 1 anno. I vegetali producono il doppio del normale ammontare di cibo al momento del raccolto.

\medskip\textbf{Cuoco Invisibile}\index{Incantesimi - Cuoco Invisibile}\\
\textbf{Scuola}: Evocazione\\
\textbf{Difficoltà}: 16\\
\textbf{Tempo di Lancio}: 2 Azioni\\
\textbf{Gittata}: 18 metri\\
\textbf{Componenti}: V, S, M (un mestolo di legno e qualche goccia di olio di oliva, il cibo che si vuole cucinato)\\
\textbf{Durata}: 2 ore\\
Questo incantesimo crea una forza quasi invisibile solo delimitata da una leggera aura (di colore a tua scelta) capace e competente nel cucinare. Assieme al cuoco si manifesta anche un set di pentole e padelle nonché stoviglie ed un piccolo fornello da campo.\\
In base agli ingredienti a disposizione o erbe e verdure nel raggio di 100 metri (il cuoco non va a caccia) il cuoco cucinera' al meglio degli ingredienti preparando delle ottime vivande fino a 4 persone. L'incantesimo non crea cibo o acqua, questo deve essere a disposizione al momento del lancio dell'incantesimo. \\
Una volta a disposizione gli ingredienti entro le due ore il cuoco invisibile preparera' da mangiare. E' possibile anche affrettare l'esecuzione ma a discapito della qualita'.\\
Nessuna delle pentole, stoviglie o fuoco puo' essere usato fuorché dal cuoco invisibile.

\medskip\textbf{Cura Ferite Leggere}\index{Incantesimi - Cura Ferite Leggere}\\
\textbf{Scuola}: Cura\\
\textbf{Difficoltà}: 16\\
\textbf{Tempo di Lancio}: 2 Azioni\\
\textbf{Gittata}: Contatto\\
\textbf{Componenti}: V, S\\
\textbf{Durata}: Istantanea\\
Una creatura in mischia con te recupera un numero di punti ferita uguale a 1d8 + Saggezza. Questo incantesimo se usato su un non morto, tiro per colpire con incantesimo, lo danneggia dello stesso ammontare.\\
Questo incantesimo se non esplicitato diversamente non puo' essere usato su animali o piante.\\
\textbf{Per ogni Critico ottenuto} nella prova di magia curi 1d6 PF in più.\\
Se incantatore e creatura curata sono entrambi Seguaci dello stesso Patrono l'incantesimo cura 1d8 in piu'.\\
Se incantatore e creatura curata sono entrambi Devoti dello stesso Patrono ogni valore sul dado pari a 1,2,3 sara' considerato 4.\\

\medskip\textbf{Cura Ferite Serie}\index{Incantesimi - Cura Ferite Serie}\\
\textbf{Scuola}: Cura\\
\textbf{Difficoltà}: 21\\
\textbf{Tempo di Lancio}: 2 Azioni\\
\textbf{Gittata}: Contatto\\
\textbf{Componenti}: V, S\\
\textbf{Durata}: Istantanea\\
Una creatura in mischia con te recupera un numero di punti ferita uguale a 3d8 + 2*Saggezza. Questo incantesimo se usato su un non morto, tiro per colpire con incantesimo, lo danneggia dello stesso ammontare.\\
Questo incantesimo se non esplicitato diversamente non puo' essere usato su animali o piante.\\
\textbf{Per ogni Critico ottenuto} nella prova di magia curi 1d6 PF in più.\\
Se incantatore e creatura curata sono entrambi Seguaci dello stesso Patrono l'incantesimo cura 1d8 in piu'.\\
Se incantatore e creatura curata sono entrambi Devoti dello stesso Patrono ogni valore sul dado pari a 1,2,3 sara' considerato 4.\\

\medskip\textbf{Cura Ferite Critiche}\index{Incantesimi - Cura Ferite Critiche }\\
\textbf{Scuola}: Cura\\
\textbf{Difficoltà}: 20 \\
\textbf{Tempo di Lancio}: 2 Azioni\\
\textbf{Gittata}: Contatto\\
\textbf{Componenti}: V, S\\
\textbf{Durata}: Istantanea\\
Una creatura in mischia con te recupera un numero di punti ferita uguale a 5d8 + 3*Saggezza. Questo incantesimo se usato su un non morto, tiro per colpire con incantesimo, lo danneggia dello stesso ammontare.\\
Questo incantesimo se non esplicitato diversamente non puo' essere usato su animali o piante.\\
\textbf{Per ogni Critico ottenuto} nella prova di magia curi 1d6 PF in più.\\
Se incantatore e creatura curata sono entrambi Seguaci dello stesso Patrono l'incantesimo cura 1d8 in piu'.\\
Se incantatore e creatura curata sono entrambi Devoti dello stesso Patrono ogni valore sul dado pari a 1,2,3 sara' considerato 4.\\

\medskip\textbf{Cura Ferite di Massa}\index{Incantesimi - Cura Ferite di Massa}\\
\textbf{Scuola}: Cura\\
Come le Cura Ferite ma curi fino a 4 creature.\\
La Difficoltà aumenta di 6 rispetto al Cura Ferite selezionato.\\
\textbf{Per ogni Critico ottenuto} nella prova curi una creatura in piu'.\\
Se incantatore e creatura curata sono entrambi Seguaci dello stesso Patrono l'incantesimo cura 1d8 in piu'.\\
Questo incantesimo se non esplicitato diversamente non puo' essere usato su animali o piante.\\
Se incantatore e creatura curata sono entrambi Devoti dello stesso Patrono ogni valore sul dado pari a 1,2,3 sara' considerato 4.\\

\medskip\textbf{Dardo di Fuoco}\index{Incantesimi - Dardo di Fuoco}\\
\textbf{Scuola}: Invocazione\\
\textbf{Difficoltà}: 16\\
\textbf{Tempo di Lancio}: 2 Azioni\\
\textbf{Gittata}: 36 metri\\
\textbf{Componenti}: V, S\\
\textbf{Durata}: Istantanea\\
Scagli una scintilla infuocata a una creatura od oggetto a gittata. Effettua un attacco a distanza con incantesimo contro il bersaglio. Se colpisci, il bersaglio subisce 1d10 danni da fuoco. Un oggetto infiammabile colpito da questo incantesimo prende fuoco, se non è indossato o trasportato.\\
Il danno dell'incantesimo aumenta di 1d8 quando raggiungi CM 5, CM 11 e CM 17.

\medskip\textbf{Dardo Tracciante}\index{Incantesimi - Dardo Tracciante}\\
\textbf{Scuola}: Invocazione\\
\textbf{Difficoltà}: 16\\
\textbf{Tempo di Lancio}: 2 Azioni\\
\textbf{Gittata}: 36 metri\\
\textbf{Componenti}: V, S\\
\textbf{Durata}: 1 round\\
Un lampo di luce viaggia verso una creatura a gittata, scelta da te. Effettua un attacco a distanza con incantesimo contro il bersaglio. Se colpisci, il bersaglio subisce 4d6 danni da Luce, e il prossimo tiro per colpire effettuato contro di lui prima del termine del tuo
prossimo round ha +1d6 al TC, grazie alla mistica luce fioca che continuerà a brillare intorno al bersaglio fino ad allora.\\
\textbf{Per ogni Critico ottenuto} nella prova di magia il danno aumenta di 1d6.

\medskip\textbf{Danza Irresistibile}\index{Incantesimi - Danza Irresistibile}\\
\textbf{Scuola}: Ammaliamento\\
\textbf{Difficoltà}: 34\\
\textbf{Tempo di Lancio}: 2 Azioni\\
\textbf{Gittata}: 9 metri\\
\textbf{Componenti}: V\\
\textbf{Durata}: 1 minuto\\
Scegli una creatura a gittata e che puoi vedere. Il bersaglio comincia un comico balletto sul posto: agitando le gambe, battendo i piedi e saltellando per la durata. Le creature che non possono essere affascinate sono immuni a questo incantesimo.\\
Una creatura che balla deve usare 2 Azioni di Movimento per ballare senza lasciare il suo spazio e ha -1d6 ai Tiri Salvezza su Destrezza e i Tiri per Colpire. Mentre il bersaglio è soggetto a questo incantesimo, le altre creature hanno +1d6 ai Tiri per Colpire contro di lui. Spendendo 2 Azioni la creatura che balla puo' affettuare un nuovo Tiro Salvezza su Volontà per recuperare il controllo di se stessa. Se lo supera, l'incantesimo ha fine.

\medskip\textbf{Dardo Incantato}\index{Incantesimi - Dardo Incantato}\\
\textbf{Scuola}: Invocazione\\
\textbf{Difficoltà}: 16\\
\textbf{Tempo di Lancio}: 2 Azioni\\
\textbf{Gittata}: 36 metri\\
\textbf{Componenti}: V, S\\
\textbf{Durata}: 1 Turno\\
Crei un dardo luminoso di forza magica. Lanciare uno o più dardi gia' evocati costa 1 Azione e puo' cumularsi con il lancio di incantesimi. Il dardo colpisce una creatura a gittata che puoi vedere, scelta da te. Un dardo infligge 1d4 + 1 danni da forza al suo bersaglio e li puoi dirigere perché colpiscano una o più creature.\\
Crei un dardo aggiuntivo quando raggiungi CM 3, CM 5 e CM 7\\.
\textbf{Per ogni Critico ottenuto} nella prova di magia l'incantesimo crea un dardo aggiuntivo

\medskip\textbf{Deflagrazione Occulta}\index{Trucchetto - Deflagrazione Occulta}\\
\textbf{Scuola}: Invocazione\\
\textbf{Difficoltà}: 12\\
\textbf{Tempo di Lancio}: 2 Azioni\\
\textbf{Gittata}: 36 metri\\
\textbf{Componenti}: V, S\\
\textbf{Durata}: Istantanea\\
Un fascio di energia crepitante si dirige verso una creatura a gittata. Effettua un attacco a distanza con incantesimo contro il bersaglio. Se colpisci, il bersaglio subisce 1d10 danni da forza.\\
Il danno dell'incantesimo aumenta di 1d8 quando raggiungi CM 5, CM 11 e CM 17.

\medskip\textbf{Desiderio}\index{Incantesimi - Desiderio}\\
\textbf{Scuola}: Evocazione\\
\textbf{Difficoltà}: 36\\
\textbf{Tempo di Lancio}: 2 Azioni\\
\textbf{Gittata}: Personale\\
\textbf{Componenti}: V\\
\textbf{Durata}: Istantanea\\
Desiderio è il più potente incantesimo che una creatura mortale possa lanciare. Semplicemente parlando ad alta voce, puoi modificare le stesse fondamenta della realtà a seconda dei tuoi bisogni. \\
L'uso basilare di questo incantesimo è quello di riprodurre l'effetto di qualsiasi altro incantesimo con Difficoltà 28 o meno. Non devi soddisfare nessuno dei requisiti dell'incantesimo, comprese le componenti materiali costose. L'incantesimo ha semplicemente effetto.\\
In alternativa, puoi creare uno dei seguenti effetti a tua scelta:
\begin{itemize}
	\item 
Crei un oggetto del valore massimo di 25.000 mo, che non sia un oggetto magico. L'oggetto non può avere dimensioni superiori ai 90 metri in qualsiasi dimensione, e compare in uno spazio non occupato sul terreno.
	\item 
Permetti fino a venti creature che puoi vedere di recuperare tutti i punti ferita, e termini tutti gli effetti su di loro descritti dall'incantesimo ristorare superiore. 
	\item 
Conferisci a un massimo di dieci creature che puoi vedere la resistenza a un tipo di danno a tua scelta.
	\item 
Conferisci a un massimo di dieci creature che puoi vedere l'immunità a un singolo incantesimo o altro effetto magico per 8 ore. Per esempio, potresti rendere te e tutti tuoi compagni immuni all'attacco risucchia vita del lich.
	\item 
Annulli un qualsiasi evento recente obbligando a ritirare qualsiasi tiro effettuato nell'ultimo round (compreso il tuo ultimo round). La realtà si rimodella per assecondare il nuovo risultato. Puoi far sì che il nuovo tiro abbia +2d6 o -2d6, puoi scegliere se usare il tiro originale o il nuovo tiro. Potresti anche riuscire a ottenere altro, oltre gli obiettivi negli esempi di cui sopra.\\
\end{itemize}
\medskip
Definisci i tuoi desideri quanto più possibile al Narratore. Il Narratore ha grande spazio di
manovra nel decidere cosa accada in questi casi; maggiore il desiderio, più grosse le probabilità che qualcosa vada storto. L'incantesimo potrebbe semplicemente fallire, l'effetto desiderato manifestarsi solo in parte, oppure potresti subire delle conseguenze impreviste, in base a come hai proferito il desiderio. Lo stress del lanciare questo incantesimo per creare qualsiasi effetto che non sia riprodurre un altro incantesimo ti indebolisce.\\
Dopo averne retto lo stress, ogni volta che lancerai un incantesimo, fino a che non avrai terminato una notte di riposo, subirai 1d10 danni da Vuoto per Difficoltà/3 dell'incantesimo. Questo danno non può essere ridotto o diminuito in alcun modo. Inoltre, la tua Costituzione scende a -3, se non è già a -3 o meno, per 2d4 giorni.\\
Per ciascun giorno che trascorri a riposare e non svolgere altro che un'attività leggera, il tuo tempo di recupero rimanente diminuisce di 2 giorni. Infine, c'è una probabilità del 33 percento che tu non sia mai più in grado di lanciare desiderio a causa dello stress sofferto per il lancio dell'incantesimo.

\medskip\textbf{Destriero Fantasma}\index{Incantesimi - Destriero Fantasma}\\
\textbf{Scuola}: Illusione\\
\textbf{Difficoltà}: 21\\
\textbf{Tempo di Lancio}: 1 minuto\\
\textbf{Gittata}: 9 metri\\
\textbf{Componenti}: V, S\\
\textbf{Durata}: 1 ora\\
Una creatura quasi reale simile a un cavallo di taglia Grande, appare sul terreno in uno spazio non occupato di tua scelta e a gittata. Decidi tu l'aspetto della creatura, e questa compare equipaggiata di sella, morso e briglia. Qualsiasi equipaggiamento creato dall'incantesimo svanisce in una nuvola di fumo se viene portato a più di 3 metri di distanza dal destriero. Per la durata, tu o una creatura di tua scelta potete cavalcare il destriero. La creatura usa le statistiche del cavallo da corsa, eccetto che ha velocità 30 metri e può percorrere 15 chilometri in un'ora, o 20 chilometri ad andatura veloce. Quando l'incantesimo termina, il destriero inizia gradualmente a svanire, dando al cavallerizzo 1 minuto per smontare di sella. L'incantesimo termina se usi un'azione per interromperlo o se il destriero subisce danni.

\medskip\textbf{Disco Fluttuante}\index{Incantesimi - Disco Fluttuante}\\
\textbf{Scuola}: Evocazione\\
\textbf{Difficoltà}: 16\\
\textbf{Tempo di Lancio}: 2 Azioni\\
\textbf{Gittata}: 9 metri\\
\textbf{Componenti}: V, S, M (una goccia di mercurio)\\
\textbf{Durata}: 1 ora\\
Questo incantesimo crea un piano di forza orizzontale, perfettamente circolare, di 1 metro di diametro e 2,5 centimetri di spessore che fluttua a 1 metro da terra, in uno spazio non occupato di tua scelta a gittata e che puoi vedere. Il disco rimane attivo per la durata, e può sostenere 250 chili. Se gli viene poggiato sopra un peso superiore, l'incantesimo termina e tutto quello che vi si trova sopra cade a terra. Finché ti trovi entro 6 metri da esso, il disco è immobile. Se ti muovi più di 6 metri lontano da esso, il disco ti segue in modo da rimanere sempre a 6 metri da te. Può muoversi attraverso terreno disomogeneo, su e giù per le scale, pendenze e simili, ma non può superare cambi di altitudine di 3 o più metri. Per esempio, il disco non può attraversare un fossato profondo 3 metri, né potrebbe lasciare il fossato se fosse creato in fondo a esso. Il disco puo' essere afferrato dall'incantatore e spostato manualmente. Se ti allontani più di 30 metri dal disco (di solito perché non riesce ad aggirare un ostacolo nel seguirti) l'incantesimo termina.

\medskip\textbf{Disintegrazione}\index{Incantesimi - Disintegrazione}\\
\textbf{Scuola}: Trasmutazione\\
\textbf{Difficoltà}: 29\\
\textbf{Tempo di Lancio}: 2 Azioni\\
\textbf{Gittata}: 18 metri\\
\textbf{Componenti}: V, S, M (una calamita e un pizzico di polvere)\\
\textbf{Durata}: Istantanea\\
Un sottile raggio verde parte dal tuo dito puntato verso un bersaglio a gittata e che puoi vedere. Il bersaglio può essere una creatura, un oggetto o una creazione di forza magica, come un muro creato da muro di forza. Una creatura bersaglio di questo incantesimo deve effettuare un Tiro Salvezza su Riflessi. Il bersaglio subisce 10d6 + 40 danni da forza se fallisce il Tiro Salvezza. Se questo danno riduce il bersaglio a 0 punti ferita, questi è disintegrato. Una creatura disintegrata e tutto quello che indossa e trasporta, eccetto gli oggetti magici, viene ridotta a una pila di sottile polvere grigia. La creatura può essere riportata in vita solo tramite l'intervento di un Patrono\\
Questo incantesimo disintegra automaticamente gli oggetti non magici o una creazione di forza magica di taglia Grande o più piccola. Se il bersaglio è un oggetto non magico o una creazione di forza di taglia Enorme o più grossa, questo incantesimo disintegra una porzione di essa pari a un cubo di 3 metri di spigolo. Gli oggetti magici ignorano questo incantesimo.\\
\textbf{Per ogni Critico ottenuto} nella prova di magia danno aumenta di 3d6.

\medskip\textbf{Dissolvi il Bene e il Male}\index{Incantesimi - Dissolvi il Bene e il Male}\\
\textbf{Scuola}: Abiurazione\\
\textbf{Difficoltà}: 26\\
\textbf{Tempo di Lancio}: 2 Azioni\\
\textbf{Gittata}: Personale\\
\textbf{Componenti}: V, S, M (Acqua Benedetta o argento e ferro in polvere)\\
\textbf{Durata}: Concentrazione, 1 minuto \\
Un'energia luminosa ti circonda e ti protegge da fatati, non morti e creature originarie di luoghi al di là del Piano Materiale. Per la durata, i celestiali, elementali, fatati, demoni e non morti hanno -1d6 ai Tiri per Colpire contro di te. Puoi terminare l'incantesimo anticipatamente usando una delle seguenti funzioni speciali.\\
\textit{Spezzare Ammaliamento}. Con un'azione, puoi entrare in contatto con una creatura affascinata, spaventata o posseduta da un celestiale, elementale, fatato, demoni o non morto. La creatura con cui sei in contatto non è più affascinata, spaventata o posseduta da queste creature.\\
\textit{Congedo}. Con un'azione, effettua un attacco da mischia contro un celestiale, elementale, fatato, demone o non morto nella tua portata. Se lo colpisci, puoi cercare di rimandare la creatura al suo piano di origine. La creatura deve superare un Tiro Salvezza su Volontà o venire rispedita sul suo piano nativo (se non vi si trova già). Se non si trovano sul loro piano nativo, i non morti vengono rispediti nel Mondo delle Ombre e i fatati nel Primo Mondo.

\medskip\textbf{Dissolvi Magie}\index{Incantesimi - Dissolvi Magie}\\
\textbf{Scuola}: Abiurazione\\
\textbf{Difficoltà}: 21\\
\textbf{Tempo di Lancio}: 2 Azioni\\
\textbf{Gittata}: 36 metri\\
\textbf{Componenti}: V, S\\
\textbf{Durata}: Istantanea\\
Scegli una creatura, oggetto o effetto magico a gittata. Qualsiasi incantesimo di Difficoltà 18 o più basso sul bersaglio ha fine. \\
\textbf{Per ogni Critico ottenuto} nella prova di magia la Difficoltà dispellabile aumenta di 2.

\medskip\textbf{Dito}\index{Incantesimi - Dito}\\
\textbf{Scuola}: Ammaliamento\\
\textbf{Difficoltà}: 12\\
\textbf{Tempo di Lancio}: 1 Azione Immediata\\
\textbf{Gittata}: 18 metri\\
\textbf{Componenti}: S\\
\textbf{Durata}: 3 round\\
Fai il dito (o pernacchia o gesto dell'ombrello) all'avversario che deve poterlo vedere (o sentire)\\
Questo deve fare un Tiro Salvezza su Volontà, se riesce non succede nulla.
Se fallisce il TS di 5 o piu' viene umiliato, per i prossimi 3 round ha una penalità di un 1d6 ai Tiri per Colpire, TS ed alle prove di Competenza.\\
Se fallisce il TS di 3 o 4, viene mortificato, per i prossimi 3 round ha una penalità di 1d6 ai Tiri per Colpire e Competenza.\\
Se fallisce il TS di 2 o 1, e' punito, per i prossimi 3 round ha una penalità di 2 ai Tiri per Colpire.\\
\textbf{Per ogni Critico ottenuto} nella prova di magia puoi influenzare una altra creatura che possa vedere il dito.\\

\medskip\textbf{Dito della Morte}\index{Incantesimi - Dito della Morte}\\
\textbf{Scuola}: Necromanzia\\
\textbf{Difficoltà}: 29\\
\textbf{Tempo di Lancio}: 2 Azioni\\
\textbf{Gittata}: 18 metri\\
\textbf{Componenti}: V, S\\
\textbf{Durata}: Istantanea\\
Invii una scarica di energia negativa a una creatura a gittata e che puoi vedere, provocandole profondo dolore. Il bersaglio deve effettuare un Tiro Salvezza su Tempra. Il bersaglio subisce 7d8 + 30 danni da Vuoto se fallisce il Tiro Salvezza, o la metà di questi danni se lo supera.\\
Un umanoide ucciso da questo incantesimo si rianima come zombi sotto il tuo comando permanente all'inizio del tuo prossimo round, e seguirà i tuoi ordini verbali al meglio delle sue capacità.

\medskip\textbf{Divinazione}\index{Incantesimi - Divinazione}\\
\textbf{Scuola}: Divinazione\\
\textbf{Difficoltà}: 29\\
\textbf{Tempo di Lancio}: 2 Azioni\\
\textbf{Gittata}: Personale\\
\textbf{Componenti}: V, S, M (incenso e un'offerta sacrificale appropriata alla tua religione, il cui valore complessivo sia di 25 mo, che saranno consumati dall'incantesimo)\\
\textbf{Durata}: Istantanea\\
La tua magia e un'offerta votiva ti mettono in comunicazione con un Patrono o il servitore di un Patrono. Gli puoi porre una singola domanda in merito a uno specifico obiettivo, evento o attività che debba verificarsi entro 7 giorni. Il Narratore dà una risposta veritiera. La replica potrebbe essere una breve frase, una rima criptica o un presagio. \\
L'incantesimo non tiene conto di ogni possibile circostanza che possa modificare il risultato, come il lancio di ulteriori incantesimi o la perdita o l'arrivo di un alleato.\\
Se lanci l'incantesimo due o più volte prima di aver terminato il giorno lungo, c'è una probabilità cumulativa del 25\% che per ogni lancio dopo il primo tu ottenga una lettura erronea. Il Narratore effettua questo tiro in segreto. 

\medskip\textbf{Dominare Bestie}\index{Incantesimi - Dominare Bestie}\\
\textbf{Scuola}: Ammaliamento\\
\textbf{Difficoltà}: 23\\
\textbf{Tempo di Lancio}: 2 Azioni\\
\textbf{Gittata}: 18 metri\\
\textbf{Componenti}: V, S\\
\textbf{Durata}: Concentrazione, massimo 1 minuto\\
Cerchi di affascinare una bestia a gittata che puoi vedere. Essa deve superare un Tiro Salvezza su Volontà o restare affascinata per la durata, ricevendo +1d6 al tiro se tu o i tuoi alleati la state combattendo.\\
Mentre la bestia è affascinata, finché voi due vi trovate sullo stesso piano di esistenza mantieni un collegamento telepatico con essa. Puoi usare questo collegamento telepatico per inviare comandi alla creatura mentre sei cosciente (richiede 1 azione), a cui essa obbedirà al suo meglio. Puoi specificare un corso d'azione semplice e generico, come "Attacca quella creatura", "Corri laggiù", o "Prendi quell'oggetto". Se la creatura completa l'ordine e non riceve ulteriori indicazioni da te, si difenderà e preserverà al meglio delle sue capacità.\\
Puoi impiegare 2 tue azioni per assumere il totale e preciso controllo del bersaglio. Fino al termine del tuo prossimo round, il bersaglio effettuerà solo le azioni decise da te, e non farà nulla che tu non gli permetta di fare. Durante questo periodo, puoi anche far usare una reazione al bersaglio, ma ciò richiede l'uso della tua reazione.\\
Ogni volta che il bersaglio subisce danni, effettua un nuovo Tiro Salvezza su Volontà contro l'incantesimo. Se supera il Tiro Salvezza, l'incantesimo termina.\\
\textbf{Per ogni Critico ottenuto} nella prova di magia la durata raddoppia fino ad un massimo di 8 ore.

\medskip\textbf{Dominare Mostri}\index{Incantesimi - Dominare Mostri}\\
\textbf{Scuola}: Ammaliamento\\
\textbf{Difficoltà}: 34\\
\textbf{Tempo di Lancio}: 2 Azioni\\
\textbf{Gittata}: 18 metri\\
\textbf{Componenti}: V, S\\
\textbf{Durata}: Concentrazione, massimo 1 ora\\
Cerchi di affascinare una creatura a gittata che puoi vedere. Essa deve superare un Tiro Salvezza su Volontà o restare affascinata per la durata, ricevendo +1d6 al tiro se tu o i tuoi alleati la state combattendo.\\
Mentre la creatura è affascinata, finché voi due vi trovate sullo stesso piano di esistenza mantieni un collegamento telepatico con essa. Puoi usare questo collegamento telepatico per inviare comandi alla creatura mentre sei cosciente (richiede 1 azione), a cui essa obbedirà al suo meglio. Puoi specificare un corso d'azione semplice e generico, come "Attacca quella creatura", "Corri laggiù", o "Prendi quell'oggetto". Se la creatura completa l'ordine e non riceve ulteriori indicazioni da te, si difenderà e preserverà al meglio delle sue capacità.\\
Puoi impiegare due tua Azioni per assumere il totale e preciso controllo del bersaglio. Fino al termine del tuo prossimo round la creatura effettuerà solo le azioni decise da te, e non farà nulla che tu non le permetta di fare. Durante questo periodo, puoi anche far usare una reazione alla creatura, ma ciò richiede l'uso della tua reazione. Ogni volta che il bersaglio subisce danni, effettua un nuovo Tiro Salvezza su Volontà contro l'incantesimo. Se supera il Tiro Salvezza, l'incantesimo termina.\\
\textbf{Per ogni Critico ottenuto} nella prova di magia la durata raddoppia fino ad un massimo di 8 ore.

\medskip\textbf{Dominare Persone}\index{Incantesimi - Dominare Persone}\\
\textbf{Scuola}: Ammaliamento\\
\textbf{Difficoltà}: 26\\
\textbf{Tempo di Lancio}: 2 Azioni\\
\textbf{Gittata}: 18 metri\\
\textbf{Componenti}: V, S\\
\textbf{Durata}: Concentrazione, massimo 1 minuto\\
Cerchi di affascinare un umanoide a gittata che puoi vedere. Esso deve superare un Tiro Salvezza su Volontà o restare affascinato per la durata, ricevendo +1d6 al tiro se tu o i tuoi alleati lo state combattendo.\\
Mentre il bersaglio è affascinato, finché voi due vi trovate sullo stesso piano di esistenza mantieni un collegamento telepatico con esso. Puoi usare questo collegamento telepatico per inviare comandi al bersaglio mentre sei cosciente (richiede 1 azione), a cui esso obbedirà al suo meglio. Puoi specificare un corso d'azione semplice e generico, come "Attacca quella creatura", "Corri laggiù", o "Prendi quell'oggetto". Se il bersaglio completa l'ordine e non riceve ulteriori indicazioni da te, si difenderà e preserverà al meglio delle sue capacità.\\
Puoi impiegare 2 Azioni per assumere il totale e preciso controllo del bersaglio. Fino al termine del tuo prossimo round, il bersaglio effettuerà solo le azioni decise da te, e non farà nulla che tu non gli permetta di fare. Durante questo periodo, puoi anche far usare una reazione al bersaglio, ma ciò richiede l'uso della tua reazione. Ogni volta che il bersaglio subisce danni, effettua un nuovo Tiro Salvezza su Volontà contro l'incantesimo. Se supera il Tiro Salvezza, l'incantesimo termina.\\
\textbf{Per ogni Critico ottenuto} nella prova di magia la durata raddoppia fino ad un massimo di 8 ore.

\medskip\textbf{Eroismo}\index{Incantesimi - Eroismo}\\
\textbf{Scuola}: Ammaliamento\\
\textbf{Difficoltà}: 16\\
\textbf{Tempo di Lancio}: 2 Azioni\\
\textbf{Gittata}: Contatto\\
\textbf{Componenti}: V, S\\
\textbf{Durata}: 1 minuto\\
Una creatura consenziente con cui sei in contatto vene infusa di coraggio. Fino al termine dell'incantesimo, la creatura è immune all'essere spaventata e, all'inizio di ciascun suo round, ottiene punti ferita temporanei pari al tuo valore di Intelligenza o modificatore da incantesimo. Quando l'incantesimo ha termine, il bersaglio perde tutti i punti ferita temporanei rimanenti derivati da questo incantesimo.

\medskip\textbf{Esilio}\index{Incantesimi - Esilio}\\
\textbf{Scuola}: Abiurazione\\
\textbf{Difficoltà}: 23\\
\textbf{Tempo di Lancio}: 2 Azioni\\
\textbf{Gittata}: 18 metri\\
\textbf{Componenti}: V, S, M (un oggetto disprezzato dal bersaglio)\\
\textbf{Durata}: 1 minuto\\
Cerchi di spedire una creatura a gittata e che puoi vedere in un altro piano di esistenza. Il bersaglio deve superare un Tiro Salvezza su Volontà o venire esiliato. Se il bersaglio è natio del piano di esistenza in cui ti trovi, esili il bersaglio in un semipiano innocuo. Mentre è lì, il bersaglio è inabile. Il bersaglio rimane lì fino al termine dell'incantesimo, quando riapparirà nello spazio che aveva lasciato o nello spazio non occupato più vicino, se il suo spazio originale adesso è occupato. Se il bersaglio è natio di un diverso piano di esistenza da quello in cui ti trovi, il bersaglio svanisce emettendo un lieve scoppio, tornando al suo piano natio. Se l'incantesimo termina prima che sia trascorso 1 minuto, il bersaglio riappare nello spazio che aveva lasciato o nello spazio non occupato più vicino, se il suo spazio originale è occupato.\\
\textbf{Per ogni Critico ottenuto} nella prova di magia puoi influenzare un altra creatura

\medskip\textbf{Esplosione Solare}\index{Incantesimi - Esplosione Solare}\\
\textbf{Scuola}: Invocazione\\
\textbf{Difficoltà}: 34\\
\textbf{Tempo di Lancio}: 2 Azioni\\
\textbf{Gittata}: 45 metri\\
\textbf{Componenti}: V, S, M (fuoco e un pezzo di pietra di sole)\\
\textbf{Durata}: Istantanea\\
Un'intensa luce solare illumina in un raggio di 18 metri centrato su di un punto a gittata, scelto da te. Tutte le creature all'interno della luce devono effettuare un Tiro Salvezza su Tempra. Se fallisce il Tiro Salvezza, una creatura subisce 12d6 danni da Luce e resta accecata per 1 minuto. Se lo supera, subisce la metà dei danni e non resta accecata dall'incantesimo. Non morti e melme hanno -2d6 a questo Tiro Salvezza. Una creatura accecata da questo incantesimo effettua un altro Tiro Salvezza su Tempra alla fine di ciascun suo round. Se supera il Tiro Salvezza, non è più accecata.\\
Nella sua area, questo incantesimo dissolve qualsiasi oscurità generata da un incantesimo. 

\medskip\textbf{Estasiare}\index{Incantesimi - Estasiare}\\
\textbf{Scuola}: Ammaliamento\\
\textbf{Difficoltà}: 19\\
\textbf{Tempo di Lancio}: 2 Azioni\\
\textbf{Gittata}: Personale\\
\textbf{Componenti}: V, S\\
\textbf{Durata}: 1 minuto\\
Intessi una serie di parole svianti, facendo sì che delle creature di tua scelta entro la gittata, che puoi vedere e possano sentirti, effettuino un Tiro Salvezza su Volontà. Qualsiasi creatura che non può restare affascinata supera il Tiro Salvezza automaticamente, e se tu o i tuoi compagni state combattendo una creatura, questa ha +1d6 al Tiro Salvezza. Se fallisce il Tiro Salvezza, il bersaglio ha -1d6 sulle prove di Consapevolezza effettuate per percepire una qualsiasi creatura diversa da te fino al termine dell'incantesimo o finché il bersaglio non può più sentirti.
L'incantesimo termina se sei reso inabile o non puoi più parlare.

\medskip\textbf{Evoca Animali}\index{Incantesimi - Evoca Animali}\\
\textbf{Scuola}: Evocazione\\
\textbf{Difficoltà}: 21\\
\textbf{Tempo di Lancio}: 2 Azioni\\
\textbf{Gittata}: 18 metri\\
\textbf{Componenti}: V, S\\
\textbf{Durata}: 1 ora\\
Evochi spiriti fatati che assumono l'aspetto di bestie e compaiono in spazi non occupati a gittata e che puoi vedere. Scegli una delle seguenti opzioni per determinare ciò che appare:
\begin{itemize}
\item
Una bestia di grado di sfida 2 o inferiore
\item
Due bestie di grado di sfida 1 o inferiore
\item
Quattro bestie di grado di sfida 1/2 o inferiore
\item
Otto bestie di grado di sfida 1/4 o inferiore
\end{itemize}
\medskip
Ogni bestia è considerata anche un fatato, e sparisce quando scende a 0 punti ferita o quando l'incantesimo termina. \\
Le creature evocate sono amichevoli verso di te e i tuoi compagni. Tirare l'iniziativa per le creature evocate come gruppo, che agisce durante il proprio round. Esse obbediscono a qualsiasi comando verbale che gli viene dato (senza bisogno che tu compia azioni). Se non dai comandi alle bestie, si difenderanno dalle creature ostili, ma non compiranno altre azioni.\\
\textbf{Per ogni Critico ottenuto} nella prova di magia appariranno due bestie in piu'

\medskip\textbf{Evoca Creature Boschive}\index{Incantesimi - Evoca Creature Boschive}\\
\textbf{Scuola}: Evocazione\\
\textbf{Difficoltà}: 23\\
\textbf{Tempo di Lancio}: 2 Azioni\\
\textbf{Gittata}: 18 metri\\
\textbf{Componenti}: V, S, M (una bacca di agrifoglio per creature convocata)\\
\textbf{Durata}: 1 ora \\
Evochi spiriti fatati che compaiono in spazi non occupati a gittata e che puoi vedere. Scegli una delle seguenti opzioni per determinare ciò che appare:
\begin{itemize}
\item Un fatato di grado di sfida 2 o inferiore
\item Due fatati di grado di sfida 1 o inferiore
\item Quattro fatati di grado di sfida 1/2 o inferiore
\item Otto fatati di grado di sfida 1/4 o inferiore
\end{itemize}
\medskip
Una creatura evocata sparisce quando scende a 0 punti ferita o quando l'incantesimo termina. Le creature evocate sono amichevoli verso di te e i tuoi compagni. Tirare l'iniziativa per le creature evocate come gruppo, che agisce durante il proprio round. Esse obbediscono a qualsiasi comando verbale che gli viene dato (senza bisogno che tu compia azioni). Se non dai comandi ai fatati, si difenderanno dalle creature ostili, ma non compiranno altre azioni.\\
\textbf{Per ogni Critico ottenuto} nella prova di magia appariranno due creature in piu'

\medskip\textbf{Evoca Elementale}\index{Incantesimi - Evoca Elementale}\\
\textbf{Scuola}: Evocazione\\
\textbf{Difficoltà}: 26\\
\textbf{Tempo di Lancio}: 1 minuto\\
\textbf{Gittata}: 27 metri\\
\textbf{Componenti}: V, S, M (incenso bruciato per l'aria, argilla malleabile per la terra, zolfo e fosforo per il fuoco, o acqua e sabbia per l'acqua) \\
\textbf{Durata}: 1 ora\\
Evochi un servitore elementale. Scegli un'area a gittata composta di acqua, aria, fuoco o terra e che riempia un cubo di 3 metri di spigolo. Un elementale di grado di sfida 5 o minore appropriato all'area da te scelta compare in uno spazio non occupato entro 3 metri da essa. L'elementale sparisce quando scende a 0 punti ferita o l'incantesimo termina.\\
L'elementale è amichevole verso di te e i tuoi compagni per la durata dell'incantesimo. Tira l'iniziativa per l'elementale, che agisce durante il proprio round. Obbedisce a qualsiasi comando verbale che gli viene dato (se il comando e' complesso consumi delle azioni). Se non dai comandi all'elementale, si difenderà dalle creature ostili, ma non compirà altre azioni.\\
\textbf{Per ogni Critico ottenuto} nella prova di magia il grado di sfida dell'elementale evocato aumenta di 1

\medskip\textbf{Evoca Elementali Minori}\index{Incantesimi - Evoca Elementali Minori}\\
\textbf{Scuola}: Evocazione\\
\textbf{Difficoltà}: 23\\
\textbf{Tempo di Lancio}: 1 minuto\\
\textbf{Gittata}: 27 metri\\
\textbf{Componenti}: V, S\\
\textbf{Durata}: 1 ora\\
Evochi degli elementali che compariranno in spazi non occupati a gittata e che puoi vedere. Scegli una della seguenti opzioni per decidere cosa appare:
\begin{itemize}
\item Un elementale di grado di sfida 2 o meno
\item Due elementali di grado di sfida 1 o meno
\item Quattro elementali di grado di sfida 1/2 o meno
\item Otto elementali di grado di sfida 1/4 o meno
\end{itemize}
\medskip
Un elementale evocato sparisce quando scende a 0 punti ferita o l'incantesimo termina. Un elementale evocato è amichevole verso di te e i tuoi compagni. Tirare l'iniziativa per gli elementali evocati come gruppo, che agisce durante il proprio round. Essi obbediscono a qualsiasi comando verbale che gli viene dato (se il comando e' complesso consumi delle azioni). Se non dai comandi agli elementali, si difenderanno dalle creature ostili, ma non compiranno altre azioni.\\
\textbf{Per ogni Critico ottenuto} nella prova di magia appariranno due Elementali in piu'.

\medskip\textbf{Evocazioni Istantanee}\index{Incantesimi - Evocazioni Istantanee}\\
\textbf{Scuola}: Evocazione\\
\textbf{Difficoltà}: 29\\
\textbf{Tempo di Lancio}: 1 minuto\\
\textbf{Gittata}: Contatto\\
\textbf{Componenti}: V, S, M (uno zaffiro del valore di 1.000 mo)\\
\textbf{Durata}: Fino a che dissolto \\
Entri a contatto con un oggetto del peso di 5 chili o meno e la cui dimensione più grossa non superi i 180 centimetri. L'incantesimo lascia un marchio sulla superficie dell'oggetto e ne incide invisibilmente il nome sullo zaffiro usato come componente materiale. Ogni volta che lanci questo incantesimo, devi usare uno zaffiro diverso.\\
In qualsiasi momento successivo, puoi usare 2 Azioni per pronunciare il nome dell'oggetto e frantumare lo zaffiro. L'oggetto appare istantaneamente nella tua mano quale che sia la distanza fisica o planare che vi separa, e l'incantesimo ha termine.\\
Se un'altra creatura sta impugnando o trasportando l'oggetto, frantumare lo zaffiro non trasporterà l'oggetto da te, ma invece apprenderai chi sia la creatura che ne è in possesso e indicativamente dove si trovi in questo momento.\\
Dissolvi magie, o un effetto simile applicato con successo allo zaffiro, termina l'effetto dell'incantesimo. 

\medskip\textbf{Fabbricare}\index{Incantesimi - Fabbricare}\\
\textbf{Scuola}: Trasmutazione\\
\textbf{Difficoltà}: 23\\
\textbf{Tempo di Lancio}: 10 minuti\\
\textbf{Gittata}: 36 metri\\
\textbf{Componenti}: V, S\\
\textbf{Durata}: Istantanea\\
Converti le materie prime in prodotti finiti dello stesso materiale. Per esempio, puoi fabbricare un piccolo ponte di legno da un cumulo di alberi, una corda da un mucchio di canapa, e abiti dal lino o la lana. Scegli le materie prima che puoi vedere a gittata. Puoi fabbricare un oggetto di taglia Grande o inferiore (contenuto in un cubo di 3 metri di spigolo, o otto cubi connessi di 1 metro di spigolo) data una sufficiente quantità di materie prime. Se stai lavorando con il metallo, la pietra o altre sostanze minerali, l'oggetto fabbricato non può essere più grande di taglia Media (contenuto in un singolo cubo di 1 metro di spigolo). La qualità degli oggetti creati da questo incantesimo è commisurata alla qualità delle materie prime.\\
Tramite questo incantesimo non si possono creare o trasmutare creature od oggetti magici. Inoltre non puoi usarlo per creare oggetti che normalmente richiedono un alto livello di lavorazione, come i gioielli, le armi, il vetro o le armature, a meno che tu non abbia la competenza con il tipo di strumenti da artigiano utilizzati per costruire questi oggetti.

\medskip\textbf{Faro di Speranza}\index{Incantesimi - Faro di Speranza}\\
\textbf{Scuola}: Abiurazione\\
\textbf{Difficoltà}: 21\\
\textbf{Tempo di Lancio}: 2 Azioni\\
\textbf{Gittata}: 9 metri\\
\textbf{Componenti}: V, S\\
\textbf{Durata}: 1 minuto, Concentrazione\\
Questo incantesimo conferisce speranza e vitalità. Scegli fino a 6 creature a gittata. Per la durata, ciascun bersaglio ha +1d6 ai Tiri Salvezza su Volontà e da ogni dado di cura ottiene +1 PF curato.

\medskip\textbf{Fatale}\index{Incantesimi - Fatale}\\
\textbf{Scuola}: Illusione\\
\textbf{Difficoltà}: 36\\
\textbf{Tempo di Lancio}: 2 Azioni\\
\textbf{Gittata}: 36 metri\\
\textbf{Componenti}: V, S\\
\textbf{Durata}: Concentrazione, massimo 1 minuto\\
Attingendo alle paure più intime di un gruppo di creature, crei delle creature illusorie nella loro mente, visibili solo a loro. Ogni creatura in una sfera di 9 metri di raggio centrata su di un punto a tua scelta nella gittata, deve effettuare un Tiro Salvezza su Volontà. Se fallisce il Tiro Salvezza, la creatura diventa spaventata per la durata. L'illusione affonda nelle paure più intime della creatura, manifestando i suoi incubi peggiori come una implacabile minaccia. Alla fine di ciascun round della creatura spaventata, questa deve superare un Tiro Salvezza su Volontà o subire 4d10 danni. Se supera il Tiro Salvezza, per quella creatura l'incantesimo ha termine.

\medskip\textbf{Favore Divino}\index{Incantesimi - Favore Divino}\\
\textbf{Scuola}: Invocazione\\
\textbf{Difficoltà}: 16\\
\textbf{Tempo di Lancio}: 1 Azione Immediata\\
\textbf{Gittata}: Personale\\
\textbf{Componenti}: V, S\\
\textbf{Durata}: 1 minuto\\
Le tue preghiere potenziano te e la tua arma. Fino al termine dell'incantesimo, quando colpisce, la tua arma infligge 1d4 danni da Luce aggiuntivi.

\medskip\textbf{Ferire}\index{Incantesimi - Ferire}\\
\textbf{Scuola}: Necromanzia\\
\textbf{Difficoltà}: 29\\
\textbf{Tempo di Lancio}: 2 Azioni\\
\textbf{Gittata}: 18 metri\\
\textbf{Componenti}: V, S\\
\textbf{Durata}: Istantanea\\
Scateni una malattia virulenta su di una creatura a gittata che puoi vedere. Il bersaglio deve effettuare un Tiro Salvezza su Tempra. Il bersaglio subisce 14d6 danni da Vuoto se fallisce il Tiro Salvezza, o la metà di questi danni se lo supera. il danno non può ridurre i punti ferita del bersaglio sotto l'1. Se il bersaglio fallisce il Tiro Salvezza, i suoi punti ferita massimi sono ridotti per 1 ora di un ammontare uguale al danno da Vuoto subito. Qualsiasi effetto che rimuova una malattia permette ai punti ferita massimi del personaggio di poter tornare al valore normale prima che trascorra quel tempo.

\medskip\textbf{Fermare il Tempo}\index{Incantesimi - Fermare il Tempo}\\
\textbf{Scuola}: Trasmutazione\\
\textbf{Difficoltà}: 36\\
\textbf{Tempo di Lancio}: 2 Azioni\\
\textbf{Gittata}: Personale\\
\textbf{Componenti}: V\\
\textbf{Durata}: Istantanea\\
Fermi brevemente il flusso del tempo per tutti, tranne che per te. Il tempo non scorre per le altre creature, mentre tu effettui 1d4 + 1 round di fila, durante i quali puoi effettuare azioni e muoverti come sempre. Questo incantesimo termina se una delle azioni che usi durante questo periodo, o qualsiasi effetto che crei durante questo periodo, ha effetto su di una creatura diversa da te o su di un oggetto indossato o trasportato da qualcuno che non sia tu. Inoltre, l'incantesimo termina se ti muovi in un posto lontano più di 300 metri da quello in cui lo hai lanciato.

\medskip\textbf{Fiamma Perenne}\index{Incantesimi - Fiamma Perenne}\\
\textbf{Scuola}: Invocazione\\
\textbf{Difficoltà}: 19\\
\textbf{Tempo di Lancio}: 2 Azioni\\
\textbf{Gittata}: Contatto\\
\textbf{Componenti}: V, S, M (polvere di rubino del valore di 50 mo, che l'incantesimo consuma)\\ \textbf{Durata}: Fino a che dissolto\\
Una luminosità simile a quella prodotta da una fiaccola si sprigiona da un oggetto con cui sei in contatto. L'effetto sembra quello di una normale fiamma, ma non produce calore né necessita ossigeno. Una fiamma perpetua può essere celata o nascosta ma non può essere smorzata né spenta.

\medskip\textbf{Fiamma Sacra}\index{Trucchetto - Fiamma Sacra}\\
\textbf{Scuola}: Invocazione\\
\textbf{Difficoltà}: 12\\
\textbf{Tempo di Lancio}: 2 Azioni\\
\textbf{Gittata}: 18 metri\\
\textbf{Componenti}: V, S\\
\textbf{Durata}: Istantanea\\
Una luminosità simile a quella prodotta da una fiaccola discende su di una creatura a gittata che puoi vedere. Il bersaglio deve superare un Tiro Salvezza su Riflessi o subire 1d8 danni da Luce. Il bersaglio non riceve il beneficio della copertura per questo Tiro Salvezza.\\
Il danno dell'incantesimo aumenta di 1d8 quando raggiungi CM 5, CM 11 e CM 17.

\medskip\textbf{Fiotto Acido}\index{Trucchetto - Fiotto Acido}\\
\textbf{Scuola}: Evocazione\\
\textbf{Difficoltà}: 12\\
\textbf{Tempo di Lancio}: 2 Azioni\\
\textbf{Gittata}: 18 metri\\
\textbf{Componenti}: V, S\\
\textbf{Durata}: Istantanea\\
Scagli una bolla di acido. Scegli una creatura a gittata o due creature a gittata che siano entro 1 metro l'una dall'altra. Il bersaglio deve superare un Tiro Salvezza su Riflessi o subire 1d6 danni da acido.\\
Il danno dell'incantesimo aumenta di 1d8 quando raggiungi CM 5, CM 11 e CM 17.

\medskip\textbf{Folata di Vento}\index{Incantesimi - Folata di Vento}\\
\textbf{Scuola}: Invocazione\\
\textbf{Difficoltà}: 19\\
\textbf{Tempo di Lancio}: 2 Azioni\\
\textbf{Gittata}: Personale (linea di 18 metri)\\
\textbf{Componenti}: V, S, M (un seme di legume)\\
\textbf{Durata}: Concentrazione, massimo 1 minuto\\
Una linea di forte vento lunga 18 metri e larga 3 metri esplode partendo da te in una direzione a tua scelta, per la durata dell'incantesimo. Ogni creatura che inizia il suo round dentro la linea deve superare un Tiro Salvezza su Tempra o venire spinta lontano da te di 4 metri, seguendo la direzione della linea.\\
Qualsiasi creatura sulla linea deve spendere il doppio del movimento per avvicinarsi a te.\\
La folata disperde gas o vapori, estingue candele, torce e simili fiamme non protette nell'area. Le fiamme protette, come quelle della lanterne, si agitano, e hanno una probabilità del 50\% di estinguersi. Come 1 Azione durante ciascun tuo round, prima del termine dell'incantesimo, puoi cambiare la direzione in cui la linea si proietta da te.\\
Un arma da lancio che attraversa una folata di vento ha il 50\% di mancare il bersaglio.

\medskip\textbf{Fondersi nella Pietra}\index{Incantesimi - Fondersi nella Pietra}\\
\textbf{Scuola}: Trasmutazione\\
\textbf{Difficoltà}: 21\\
\textbf{Tempo di Lancio}: 2 Azioni\\
\textbf{Gittata}: Contatto\\
\textbf{Componenti}: V, S\\
\textbf{Durata}: 8 ore\\
Entri in un oggetto o superficie di pietra grossi abbastanza da contenere tutto il tuo corpo, fondendoti con la pietra assieme a tutto l'equipaggiamento che trasporti per la durata. Usando il tuo movimento, entri nella pietra in un punto con cui sei in contatto. Non resta nulla della tua presenza che rimanga visibile o altrimenti possa essere individuato da sensi non magici. Mentre sei fuso con la pietra, non puoi vedere ciò che avviene all'esterno, e qualsiasi prova di Consapevolezza che effettui per ascoltare i suoni prodotti fuori da essa è fatta con -1d6. Resti consapevole del passare del tempo e puoi lanciare incantesimi su di te mentre sei fuso con la pietra. Puoi usare il tuo movimento per lasciare la pietra e ricomparire nel punto in cui vi sei entrato, terminando così l'incantesimo. Altrimenti non puoi muoverti.\\
I danni minori alla pietra non ti danneggiano, ma la sua parziale distruzione o cambio di forma (di modo che tu non vi entri più) ti espellono da essa e ti infliggono 6d6 danni da botta. La completa distruzione della pietra (o la sua trasmutazione in un'altra sostanza) ti fa espellere e ti infligge 50 danni da botta. Se vieni espulso, cadi prono in uno spazio non occupato, nel punto più vicino a quello in cui sei entrato nella pietra.

\medskip\textbf{Forma Eterea}\index{Incantesimi - Forma Eterea}\\
\textbf{Scuola}: Trasmutazione\\
\textbf{Difficoltà}: 31\\
\textbf{Tempo di Lancio}: 2 Azioni\\
\textbf{Gittata}: Personale\\
\textbf{Componenti}: V, S\\
\textbf{Durata}: Massimo 8 ore\\
Entri nelle regioni di confine del Piano Etereo, nell'area che si sovrappone al tuo piano attuale. Resti sul Confine Etereo per la durata o finché non usi un'azione per interrompere l'incantesimo. Se ti muovi verso l'alto o il basso, il costo del movimento è raddoppiato, se ti muovi invece orizzontalmente il movimento e' raddoppiato per azione di movimento. Puoi vedere e udire il piano da cui provieni, ma tutto quello che si trova lì ti appare grigio, e non puoi vedere a più di 18 metri di distanza.\\
Mentre sei sul Piano Etereo, può interagire solo con altre creature su quel piano. Le creature che non sono sul Piano Etereo non ti possono percepire né interagire con te, a meno che una capacità speciale o la magia gli fornisca la possibilità di farlo.\\
Ignori tutti gli oggetti e gli effetti che non sono sul Piano etereo, potendo così attraversare gli oggetti che percepisci sul piano da cui provieni. Quando l'incantesimo termina, ritorni immediatamente al piano da cui provieni nel punto che occupi attualmente. Se quando accade occupi lo stesso spazio di un oggetto solido o di una creatura, vieni immediatamente spostato nel più vicino spazio non occupato che puoi occupare e subisci 6 danni da forza per ogni metro di cui vieni spostato (o sua frazione). Questo incantesimo non ha effetto se lo esegui mentre sei già nel Piano Etereo o su di un piano che non vi confina, come uno dei Piani Esterni.\\
\textbf{Per ogni Critico ottenuto} nella prova di magia puoi portare con te un altra creatura.

\medskip\textbf{Forma Gassosa}\index{Incantesimi - Forma Gassosa}\\
\textbf{Scuola}: Trasmutazione\\
\textbf{Difficoltà}: 21\\
\textbf{Tempo di Lancio}: 2 Azioni\\
\textbf{Gittata}: Contatto\\
\textbf{Componenti}: V, S, M (un pezzo di garza e un filo di fumo)\\
\textbf{Durata}: Concentrazione, massimo 1 ora\\
Trasformi una creatura consenziente insieme a tutto ciò che sta indossando e trasportando, in una nube vaporosa per la durata. L'incantesimo termina se la creatura scende a 0 punti ferita. Le creature incorporee ignorano questo effetto. Mentre è in questa forma, l'unico metodo di movimento del bersaglio è una velocità di volo 3 metri. Il bersaglio può entrare e occupare lo spazio di un'altra creatura. Il bersaglio ha resistenza ai danni non magici, e ha +1d6 ai Tiri Salvezza su Tempra e Riflessi. Il bersaglio può attraversare piccoli buchi, strettoie, e anche semplici fori, sebbene consideri i liquidi come superfici solide. Il bersaglio non può cadere e resta fluttuante nell'aria anche se stordito o altrimenti reso inabile.\\
Mentre è nella forma di una nube vaporosa, il bersaglio non può parlare né manipolare oggetti, e qualsiasi oggetto stesse indossando o trasportando non può essere gettato, usato o altrimenti impiegato. Il bersaglio non può attaccare né lanciare incantesimi. 

\medskip\textbf{Forme Animali}\index{Incantesimi - Forme Animali}\\
\textbf{Scuola}: Trasmutazione\\
\textbf{Difficoltà}: 34\\
\textbf{Tempo di Lancio}: 2 Azioni\\
\textbf{Gittata}: 9 metri\\
\textbf{Componenti}: V, S\\
\textbf{Durata}: 24 ore\\
Trasformi magicamente altre creature in bestie. Scegli un qualsiasi numero di creature consenzienti a gittata e che puoi vedere. Trasformi ciascun bersaglio nella forma di una bestia di taglia Grande o minore con un grado di sfida 4 o inferiore. Nei turni successivi, puoi usare 2 Azioni per trasformare le creature soggette in nuove forme.\\
La trasformazione permane per ciascun bersaglio per la durata dell'incantesimo, o finché quel bersaglio scende a 0 punti ferita o muore. Puoi scegliere una forma diversa per ciascun bersaglio. Le statistiche di gioco del bersaglio sono rimpiazzate dalle statistiche della bestia scelta, a eccezione dei Tratti e dei punteggi di Intelligenza, Saggezza e Carisma che restano quelli del
bersaglio. Il bersaglio assume i punti ferita della sua nuova forma e, quando ritorna alla sua forma normale, ritorna al numero di punti ferita che aveva prima di trasformarsi. Se si ritrasforma perché è sceso a 0 punti ferita, il danno in eccesso viene applicato alla forma originale. Purché il danno in eccesso non riduca la forma normale della creatura a 0 punti ferita, essa non è priva di sensi. La creatura è limitata nelle azioni che può svolgere dalla natura della sua nuova forma, e non può parlare né lanciare incantesimi.\\
L'equipaggiamento del bersaglio si fonde nella nuova forma. Il bersaglio non può attivare, impugnare o in altro modo beneficiare del suo equipaggiamento.

\medskip\textbf{Frantumare}\index{Incantesimi - Frantumare}\\
\textbf{Scuola}: Invocazione\\
\textbf{Difficoltà}: 19\\
\textbf{Tempo di Lancio}: 2 Azioni\\
\textbf{Gittata}: 18 metri\\
\textbf{Componenti}: V, S, M (un frammento di metallo)\\
\textbf{Durata}: Istantanea\\
Un forte rombo, molto intenso, erutta da un punto a gittata di tua scelta. Ogni creatura in una sfera di 3 metri di raggio centrata su quel punto deve effettuare un Tiro Salvezza su Tempra. Una creatura subisce 3d8 danni da tuono se fallisce il Tiro Salvezza, o la metà di questi danni se lo supera. Una creatura composta di materiale inorganico, come pietra, cristallo o metallo, ha -1d6 sul Tiro Salvezza. Un oggetto non magico che non è indossato né trasportato subisce anch'esso danni se si trova nell'area dell'incantesimo.\\
\textbf{Per ogni Critico ottenuto} nella prova di magia il danno aumenta di 1d8.

\medskip\textbf{Freccia Acida}\index{Incantesimi - Freccia Acida}\\
\textbf{Scuola}: Invocazione\\
\textbf{Difficoltà}: 19\\
\textbf{Tempo di Lancio}: 2 Azioni\\
\textbf{Gittata}: 27 metri\\
\textbf{Componenti}: V, S, M (una foglia di rabarbaro in polvere e uno stomaco di pitone)\\
\textbf{Durata}: Istantanea\\
Una freccia verde luminosa saetta verso un bersaglio a gittata ed esplode con uno spruzzo d'acido. Effettua un attacco a distanza con incantesimo contro il bersaglio. Se colpisci, il bersaglio subisce immediatamente 4d4 danni da acido e 2d4 danni da acido al termine del suo prossimo round. Se manchi, la freccia spruzza il bersaglio di acido infliggendo la metà dei danni iniziali e non arrecando danni al termine del prossimo round del bersaglio.\\
\textbf{Per ogni Critico ottenuto} nella prova di magia il danno aumenta di 1d4.

\medskip\textbf{Fulmine}\index{Incantesimi - Fulmine}\\
\textbf{Scuola}: Invocazione\\
\textbf{Difficoltà}: 21\\
\textbf{Tempo di Lancio}: 2 Azioni\\
\textbf{Gittata}: Personale (linea di 30 metri)\\
\textbf{Componenti}: V, S, M (un pezzo di pelliccia e una verga d'ambra, cristallo o vetro)\\
\textbf{Durata}: Istantanea\\
Esplodi un fulmine che forma una linea lunga 30 metri e larga 1 metro che parte da dove ti trovi in una direzione scelta da te. Ogni creatura sulla linea deve superare un Tiro Salvezza su Riflessi. La creatura subisce 8d6 danni da fulmine se fallisce il Tiro Salvezza, o la metà di questi danni se lo supera.\\
Il fulmine incendia gli oggetti infiammabili nell'area che non sono indossati o trasportati.\\
Il fulmine se lanciato contro della pietra lavorata rimbalza con un angolo di 90 gradi rispetto all'incidenza di origine. Un fulmine lanciato in acqua crea una sfera di 3 metri di raggio di elettricità nel punto in cui entra.\\
\textbf{Per ogni Critico ottenuto} nella prova di magia il danno aumenta di 1d6.\\
\textbf{Successo/Fallimento Critico}: In caso si fallimento critico il danno raddoppia, in caso di successo critico il danno viene ulteriormente dimezzato

\medskip\textbf{Fuorviare}\index{Incantesimi - Fuorviare}\\
\textbf{Scuola}: Illusione\\
\textbf{Difficoltà}: 26\\
\textbf{Tempo di Lancio}: 2 Azioni\\
\textbf{Gittata}: Personale\\
\textbf{Componenti}: S\\
\textbf{Durata}: 1 ora\\
Diventi invisibile nello stesso momento in cui un tuo doppione illusorio compare nel posto in cui ti trovi. Il doppione resta per la durata dell'incantesimo, ma l'invisibilità termina se attacchi o lanci un incantesimo. Puoi usare 2 Azioni per far muovere il doppione illusorio fino al doppio della tua velocità e fargli compiere un gesto, parlare e comportarsi in qualsiasi maniera tu voglia.\\
Puoi vedere attraverso i suoi occhi e udire tramite le sue orecchie come se fossi nello spazio in cui si trova lui. Durante ciascun tuo round, con un'Azione, puoi passare dall'usare i suoi sensi all'usare i tuoi, o viceversa. Mentre stai usando i suoi sensi, sei accecato e assordato riguardo i tuoi dintorni. 

\medskip\textbf{Gabbia di Forza}\index{Incantesimi - Gabbia di Forza}\\
\textbf{Scuola}: Invocazione\\
\textbf{Difficoltà}: 28\
\textbf{Tempo di Lancio}: 2 Azioni\\
\textbf{Gittata}: 30 metri\\
\textbf{Componenti}: V, S, M (polvere di rubino del valore di 1.500 mo)\\
\textbf{Durata}: 1 ora\\
Una prigione cubica, immobile e invisibile, composta di forza magica compare intorno a un'area a gittata da te scelta. La prigione può essere una gabbia o una scatola solida, a tua scelta. Una prigione nella forma di una gabbia può avere 6 metri di lato ed essere composta da sbarre di 1,5 centimetri separate di 1,5 centimetri tra di loro. Una prigione a forma di scatola può avere 3 metri di lato, creando una barriera solida che impedisce a qualsiasi materia di attraversarla e bloccando qualsiasi incantesimo lanciato dall'interno o l'esterno dell'area. Quando lanci questo incantesimo, qualsiasi creatura che è completamente all'interno della gabbia, è intrappolata. Le creature solo parzialmente nell'area della gabbia, o quelle troppo grosse per entrarvi, vengono spinte via dal centro dell'area finché non ne sono completamente fuori.\\
Una creatura all'interno della gabbia non può lasciarla tramite mezzi non magici. Se la creatura prova a usare il teletrasporto o il viaggio interplanare per lasciare la gabbia, deve prima effettuare un Tiro Salvezza su Volontà. Se lo supera, la creatura può usare quella magia per sfuggire alla gabbia. Se lo fallisce, la creatura non può uscire dalla gabbia e spreca l'uso dell'incantesimo o dell'effetto. La gabbia si estende anche sul Piano Etereo, bloccando così il viaggio etereo.\\
Questo incantesimo non può essere dissolto da dissolvi magie.

\medskip\textbf{Giara Magica}\index{Incantesimi - Giara Magica}\\
\textbf{Scuola}: Necromanzia\\
\textbf{Difficoltà}: 29\\
\textbf{Tempo di Lancio}: 1 minuto\\
\textbf{Gittata}: Personale\\
\textbf{Componenti}: V, S, M (una gemma, cristallo, reliquario o qualche altro contenitore ornamentale del valore di almeno 500 mo)\\
\textbf{Durata}: Finché a che dissolto\\
Il tuo corpo entra in uno stato catatonico mentre la tua anima lo abbandona ed entra nel contenitore da te usato come componente materiale. Mentre la tua anima occupa il contenitore, sei consapevole dei tuoi dintorni come se fossi nello spazio del contenitore. Non puoi muoverti né usare reazioni. L'unica azione che puoi effettuare è quella di proiettare la tua anima fino a 30 metri di distanza, fuori dal contenitore, ritornando al tuo corpo vivente (e terminando l'incantesimo) o cercando di possedere un corpo umanoide.\\
Puoi tentare di possedere qualsiasi umanoide entro 30 metri da te e che tu possa vedere (le creature protette dagli incantesimi protezione dal bene e dal male o cerchio magico non possono essere possedute). Il bersaglio deve effettuare un Tiro Salvezza su Volontà e, se lo fallisce, la tua anima entra nel corpo del bersaglio, mentre l'anima del bersaglio resta intrappolata nel contenitore. Se lo supera, il bersaglio resiste ai tuoi tentativi di possederlo, e non puoi tentare di possederlo nuovamente prima che siano trascorse 24 ore.\\
Una volta che possiedi il corpo di una creatura, lo puoi controllare. Le tue statistiche di gioco sono rimpiazzate dalle statistiche della creatura, a eccezione dei tuoi Tratti e dei tuoi punteggi di Intelligenza, Saggezza e Carisma. Mantieni i benefici forniti dalle Abilità. Se il bersaglio possiede delle Abilità non puoi usarne nessuna.\\
Nel frattempo, l'anima della creatura posseduta può percepire i dintorni del contenitore usando i propri sensi, ma non può muoversi né effettuare alcuna azione.\\
Mentre possiedi un corpo, puoi usare 2 Azioni per ritornare dal corpo ospite al contenitore, se ti trovi entro 30 metri da esso, riportando l'anima della creatura ospite nel suo corpo. Se il corpo ospite muore mentre sei al suo interno, la creatura muore, e tu devi effettuare un Tiro Salvezza su Volontà contro la tua DC dei Tiri Salvezza degli incantesimi. Se lo superi, ritorni al contenitore, se si trova entro 30 metri da te. Altrimenti, morirai.\\
Se il contenitore viene distrutto o l'incantesimo termina, la tua anima ritorna immediatamente al tuo corpo. Se il tuo corpo è più di 30 metri lontano o se è morto mentre cerchi di farvi ritorno, morirà anche la tua anima. Se l'anima di un'altra creatura è nel contenitore quando viene distrutto, l'anima della creatura ritorna al suo corpo, se il corpo è vivo e si trova entro 30 metri. Altrimenti, la creatura muore. Quando l'incantesimo termina, il contenitore viene distrutto.

\medskip\textbf{Glifo di Interdizione}\index{Incantesimi - Glifo di Interdizione}\\
\textbf{Scuola}: Abiurazione\\
\textbf{Difficoltà}: 21\\
\textbf{Tempo di Lancio}: 2 Azioni\\
\textbf{Gittata}: Contatto\\
\textbf{Componenti}: V. S, M (incenso e diamante in polvere del valore di almeno 200 mo, che l'incantesimo consuma)\\
\textbf{Durata}: Fino a che dissolto o attivato \\
Quando lanci questo incantesimo, inscrivi un glifo che danneggia altre creature su di una superficie (come un tavolo o una sezione di pavimento o muro) o all'internodi un oggetto che può essere chiuso (come un libro, una pergamena o un forziere) per celare il glifo. Se scegli una superficie, il glifo può coprire un'area di superficie non maggiore di 3 metri di diametro. Se scegli un oggetto, quell'oggetto deve restare al suo posto; se l'oggetto viene spostato più di 3 metri dal punto in cui è stato lanciato l'incantesimo, il glifo è spezzato, e l'incantesimo termina senza essere stato attivato.\\
Il glifo è quasi invisibile e può essere trovato con una prova di Intelligenza (Indagare) contro la DC del Tiro Salvezza dei tuoi incantesimi. Decidi tu cosa attivi il glifo al momento del lancio dell'incantesimo.\\
Per i glifi inscritti su di una superficie, l'attivazione tipica comprende entrare in contatto o stare sopra il glifo, rimuovere un altro oggetto che copra il glifo, avvicinarsi a una certa distanza dal glifo, o manipolare l'oggetto su cui è inscritto il glifo. Per i glifi inscritti su di un oggetto, l'attivazione tipica comprende aprire l'oggetto, avvicinarsi a una certa distanza dall'oggetto, o vedere o leggere il glifo. Una volta che il glifo è stato attivato, l'incantesimo ha termine.\\
Puoi definire meglio l'attivazione così che l'incantesimo si attivi solo in determinate circostanze o secondo certe peculiarità fisiche (come l'altezza o il peso), specie di creatura (per esempio, l'interdizione potrebbe agire contro le aberrazioni o gli elfi oscuri), o specifici Tratti. Puoi anche predisporre condizioni per evitare che il glifo venga attivato, come la pronuncia di una parola d'ordine.\\
Quando inscrivi il glifo scegli rune esplosive o glifo incantesimo.
\medskip
\begin{itemize}
\item
\textit{Glifo Incantesimo}. Puoi inserire un incantesimo preparato di Difficoltà 18 o inferiore nel glifo lanciandolo come parte della creazione del glifo. L'incantesimo deve prendere come bersaglio una singola creatura o un'area. L'incantesimo che viene inserito non ha effetto immediato se lanciato in questo modo. Quando il glifo è attivato, l'incantesimo inserito viene lanciato. Se l'incantesimo ha un bersaglio, prende come bersaglio la creatura che ha attivato il glifo. Se l'incantesimo agisce su di un'area, l'area è incentrata su quella creatura. Se l'incantesimo evoca creature ostili o crea oggetti o trappole nocive, questi appaiono quanto più vicino possibile all'intruso e lo attaccano. Se l'incantesimo richiede concentrazione, questa è mantenuta fino al termine della sua normale durata.
\item
\textit{Rune Esplosive}. Quando attivato, il glifo erutta energia magica in una sfera di raggio 6 metri centrata sul glifo. La sfera si propaga intorno agli angoli. Ogni creatura nell'area deve effettuare un Tiro Salvezza su Riflessi. Una creatura subisce 5d8 danni da acido, fulmine, fuoco, freddo o tuono se fallisce il Tiro Salvezza (a tua scelta quando crei il glifo), o la metà di questi danni se supera il Tiro Salvezza.
\end{itemize}
\medskip
\textbf{Per ogni Critico ottenuto} nella prova di il danno del glifo rune esplosive aumenta di 1d8.

\medskip\textbf{Globo di Invulnerabilità}\index{Incantesimi - Globo di Invulnerabilità}\\
\textbf{Scuola}: Abiurazione\\
\textbf{Difficoltà}: 29\\
\textbf{Tempo di Lancio}: 2 Azioni\\
\textbf{Gittata}: Personale (raggio di 3 metri)\\
\textbf{Componenti}: V. S, M (una pallina di vetro o di cristallo che si frantuma quando l'incantesimo termina) \\
\textbf{Durata}: Concentrazione, massimo 1 minuto\\
Una barriera immobile e lievemente scintillante si erge in un raggio di 3 metri intorno a te e vi rimane per la durata.\\
Qualsiasi incantesimo di Difficoltà 23 (ad esclusione di risultati superiori grazie a critici) o più basso lanciato dall'esterno della barriera non può agire sulle creature o gli oggetti al suo interno. Questi incantesimi possono prendere come bersaglio creature e oggetti all'interno della barriera, ma non avranno effetto su di essi. Allo stesso modo, l'area all'interno della barriera viene esclusa dalle aree di effetto di questi incantesimi.\\
\textbf{Per ogni Critico ottenuto} nella prova di magia puoi bloccare un livello superiore di Difficoltà.

\medskip\textbf{Guarigione}\index{Incantesimi - Guarigione}\\
\textbf{Scuola}: Cura\\
\textbf{Difficoltà}: 29\\
\textbf{Tempo di Lancio}: 2 Azioni\\
\textbf{Gittata}: 18 metri\\
\textbf{Componenti}: V, S\\
\textbf{Durata}: Istantanea\\
Scegli una creatura a gittata e che puoi vedere. un'ondata di energia positiva travolge la creatura, facendole recuperare 70 punti ferita. L'incantesimo pone anche termine a qualsiasi cecità, sordità e malattia (anche magica) che affligga il bersaglio. Questo incantesimo causa 50 PF di danno ad un non morto.\\
\textbf{Per ogni Critico ottenuto} nella prova di magia l'ammontare guarito aumenta di 10.
Se incantatore e creatura curata sono entrambi Seguaci dello stesso Patrono l'incantesimo cura 90 PF.\\
Se incantatore e creatura curata sono entrambi Devoti dello stesso Patrono l'incantesimo riporta a pieno di PF.\\

\medskip\textbf{Guarigione di Massa}\index{Incantesimi - Guarigione di Massa}\\
\textbf{Scuola}: Cura\\
\textbf{Difficoltà}: 36\\
\textbf{Tempo di Lancio}: 2 Azioni\\
\textbf{Gittata}: 18 metri\\
\textbf{Componenti}: V, S\\
\textbf{Durata}: Istantanea\\
Un effluvio di energia guaritrice scorre da te verso le creature ferite che ti circondano. Ripristini fino a 700 punti ferita, divisi come preferisci tra qualsiasi creatura a gittata e che puoi vedere (con un massimo di 70 pf a creatura). Le creature guarite da questo incantesimo sono curate anche di tutte le malattie e da qualsiasi effetto che le renda accecate o assordate. Questo incantesimo puo' infliggere fino a 120 PF di danno ad un non morto. TS su Tempra per annullare l'effetto.
Se l'incantatore e creatura curata sono entrambi Seguaci dello stesso Patrono la cura assegnata aumenta di 20\%\\
Se l'incantatore e creatura curata sono entrambi Devoti dello stesso Patrono la cura assegnata aumenta di 50\%\\

\medskip\textbf{Guida}\index{Incantesimi - Guida}\\
\textbf{Scuola}: Divinazione\\
\textbf{Difficoltà}: 10\
\textbf{Tempo di Lancio}: 2 Azioni\\
\textbf{Gittata}: Contatto\\
\textbf{Componenti}: V, S\\
\textbf{Durata}: Concentrazione, massimo 1 minuto\\
Lanci l'incantesimo a contatto di una creatura consenziente. Una volta, prima che l'incantesimo termini, il bersaglio può tirare un d4 e sommare il risultato tirato a una prova di caratteristica a sua scelta. Può tirare il dado prima o dopo aver effettuato la prova di caratteristica. L'incantesimo ha poi termine. 

\medskip\textbf{Guscio Anti-Vita}\index{Incantesimi - Guscio Anti-Vita}\\
\textbf{Scuola}: Abiurazione\\
\textbf{Difficoltà}: 26\\
\textbf{Tempo di Lancio}: 2 Azioni\\
\textbf{Gittata}: Personale (raggio di 3 metri)\\
\textbf{Componenti}: V, S\\
\textbf{Durata}: Concentrazione, massimo 1 ora\\
Una barriera luminosa si estende fino a un raggio di 3 metri intorno a te, muovendosi con te e rimanendo centrata su di te, tenendo distanti le creature che non siano non morti o costrutti. La barriera permane per la durata. \\
La barriera impedisce a una creatura soggetta di attraversarla in alcun modo. Una creatura soggetta può lanciare incantesimi o effettuare attacchi con armi a distanza o con portata attraverso la barriera. Se ti muovi in modo che una creatura soggetta venga forzata ad attraversare la barriera, l'incantesimo termina.

\medskip\textbf{Identificare}\index{Incantesimi - Identificare}\\
\textbf{Scuola}: Divinazione\\
\textbf{Difficoltà}: 16\\
\textbf{Tempo di Lancio}: 1 minuto\\
\textbf{Gittata}: Contatto\\
\textbf{Componenti}: V, S, M (una perla del valore di almeno 100 mo e una piuma di gufo)\\ \textbf{Durata}: Istantanea\\
Scegli un oggetto con cui devi restare a contatto per tutto il lancio dell'incantesimo. Se si tratta di un oggetto magico o altro oggetto imbevuto di magia effettua una prova di Arcana a DC 25 con un +10 di bonus, se riesci ne apprendi le proprietà e come usarle e quante cariche abbia, se ne ha. \\
Apprendi se degli incantesimi stiano agendo sull'oggetto e cosa siano. Se l'oggetto è stato creato da un incantesimo, apprendi quale incantesimo lo abbia creato. Se invece durante l'esecuzione resti a contatto con una creatura, apprendi se degli incantesimi stiano agendo su di essa e quali siano.

\medskip\textbf{Illusione Minore}\index{Incantesimi - Illusione Minore}\\
\textbf{Scuola}: Illusione\\
\textbf{Difficoltà}: 10\
\textbf{Tempo di Lancio}: 2 Azioni\\
\textbf{Gittata}: 9 metri\\
\textbf{Componenti}: S, M (un pezzo di vello)\\
\textbf{Durata}: 1 minuto\\
Crei l'immagine di un oggetto o un suono a gittata per la durata dell'incantesimo. L'illusione ha termine se la interrompi con un'azione o lanci di nuovo questo incantesimo.\\
Se crei un suono, il suo volume può variare da quello di un bisbiglio a un urlo. Può essere la tua voce, la voce di qualcun altro, il ruggito di un leone, un battito di tamburi, o qualsiasi altro suono tu scelga. Il suono continua incessante per tutta la durata, oppure puoi produrre suoni diversi in momenti diversi prima del termine dell'incantesimo.\\
Se crei l'immagine di un oggetto (come una sedia, un'impronta fangosa o un piccolo forziere) non può essere più grande di un cubo di 1 metro di spigolo. L'immagine non può produrre suoni, luci, odori o qualsiasi altro effetto sensoriale. L'interazione fisica con l'oggetto lo rivela come illusione, perché le cose lo possono attraversare.\\
Una creatura che usa 3 Azioni per esaminare il suono o l'immagine può determinare che si tratta di un'illusione con una prova riuscita di Intelligenza (Indagare) contro la DC del Tiro Salvezza del tuo incantesimo. Se una creatura riconosce l'illusione per quello che è, per lei l'illusione sbiadisce. 

\medskip\textbf{Illusione Programmata}\index{Incantesimi - Illusione Programmata}\\
\textbf{Scuola}: Illusione\\
\textbf{Difficoltà}: 29\\
\textbf{Tempo di Lancio}: 2 Azioni\\
\textbf{Gittata}: 36 metri\\
\textbf{Componenti}: V, S, M (un pezzo di vello e polvere di giada del valore di almeno 25 mo)\\
\textbf{Durata}: Fino a che dissolto\\
Crei, a gittata, l'illusione di un oggetto, creatura o qualche altro fenomeno visibile che si attiva quando viene soddisfatta una specifica condizione. Fino ad allora l'illusione è impercettibile. Non può essere più grande di un cubo di 9 metri di spigolo, e decidi tu quando lanci l'incantesimo, come si comporti l'illusione e che suoni produca. L'esibizione programmata può durare fino a 5 minuti. Quando occorrono le condizioni da te specificate, l'illusione si manifesta e si comporta nel modo da te descritto. Una volta che l'illusione ha terminato la sua esibizione, scompare e rimane dormiente per 10 minuti. Dopo questo periodo, l'illusione può essere attivata di nuovo.\\
La condizione di attivazione può essere generica o dettagliata quanto vuoi, sebbene debba essere basata su condizioni visibili o udibili che avvengano entro 9 metri dall'area. Per esempio, potresti creare un'illusione di te stesso che appare e avverta chi tenti di aprire una porta munita di trappola, oppure potresti predisporre l'illusione perché si attivi solo quando una creatura pronunci la parola o la frase giusta.\\
L'interazione fisica con l'immagine la rivela come illusione, dato che le cose le passano attraverso. Una creatura che usi 3 Azioni per esaminare l'immagine può determinare che è un'illusione con una prova riuscita di Intelligenza (Indagare) contro la DC del Tiro Salvezza dell'incantesimo. Se una creatura riconosce l'illusione per quello che è, essa può vedere attraverso l'immagine, e qualsiasi suono prodotto dall'immagine le suona artefatto.

\medskip\textbf{Immagine Maggiore}\index{Incantesimi - Immagine Maggiore}\\
\textbf{Scuola}: Illusione\\
\textbf{Difficoltà}: 21\\
\textbf{Tempo di Lancio}: 2 Azioni\\
\textbf{Gittata}: 36 metri\\
\textbf{Componenti}: V, S, M (un pezzo di vello)\\
\textbf{Durata}: Concentrazione, massimo 10 minuti\\
Crei l'immagine di un oggetto, una creatura o qualche altro fenomeno visibile non più grande di un cubo di 6 metri di spigolo. L'immagine appare in un punto a gittata che puoi vedere e vi rimane per la durata dell'incantesimo. L'immagine sembra completamente reale, e comprende suoni, odori e la temperatura appropriata alla cosa raffigurata. Non puoi generare calore o freddo sufficiente a provocare danni, né un suono abbastanza forte da infliggere danno da tuono o assordare una creatura, o un odore che possa far star male una creatura (come il fetore di un troglodita). Finché resti a gittata dell'illusione, puoi usare un'azione per far muovere l'immagine in qualsiasi altro punto a gittata.\\
Quando l'immagine cambia posizione, puoi alterarne l'aspetto così che i suoi movimenti appaiano naturali. Per esempio, se crei l'immagine di una creatura e la muovi, puoi alterare l'immagine in modo che sembri camminare. Allo stesso modo, puoi impiegare l'illusione per produrre suoni diversi in momenti diversi, fino a farle portare avanti una conversazione.\\
L'interazione fisica con l'immagine la rivela come illusione, dato che le cose vi passano attraverso. Una creatura che usa 3 Azioni per esaminare l'immagine può determinare che si tratta di un'illusione con una prova riuscita di Intelligenza (Indagare) contro la DC del Tiro Salvezza del tuo incantesimo. Se una creatura riconosce l'illusione per quello che è, la creatura può vedervi attraverso, e per quella creatura tutte le altre qualità sensoriali svaniscono.\\
\textbf{Se ottieni un critico} l'incantesimo dura finché non viene dissolto, senza richiedere la tua concentrazione.
	
\medskip\textbf{Immagine Proiettata}\index{Incantesimi - Immagine Proiettata}\\
\textbf{Scuola}: Illusione\\
\textbf{Difficoltà}: 31\\
\textbf{Tempo di Lancio}: 2 Azioni\\
\textbf{Gittata}: 750 chilometri\\
\textbf{Componenti}: V, S, M (una tua piccola riproduzione fatta di materiali del valore almeno di 5 mo)\\
\textbf{Durata}: 1 giorno\\
Crei una copia illusoria di te stesso che permane per la durata. La copia può apparire in qualsiasi luogo entro la gittata che tu abbia già visto, ignorando qualsiasi ostacolo frapposto. L'illusione riproduce il tuo aspetto e i tuoi rumori ma è intangibile. Se l'illusione subisce danni, scompare, e l'incantesimo ha termine.\\
Puoi usare 2 Azioni per far muovere questa illusione fino al doppio della tua velocità e farle compiere un gesto, parlare e comportarsi in qualsiasi maniera tu voglia. Imita alla perfezione i tuoi comportamenti.\\
Puoi vedere attraverso i suoi occhi e udire tramite le sue orecchie come se fossi nello spazio in cui essa si trova. Durante ciascun tuo round, con un'Azione, puoi passare dall'usare i suoi sensi all'usare i tuoi, o viceversa. Mentre stai usando i suoi sensi, sei accecato e assordato riguardo i tuoi dintorni.\\
L'interazione fisica con l'immagine la rivela come illusione, dato che le cose le passano attraverso. Una creatura che usi 3 Azioni per esaminare l'immagine può determinare che è un'illusione con una prova riuscita di Consapevolezza contro la DC del Tiro Salvezza dell'incantesimo. Se una creatura riconosce l'illusione per quello che è, essa può vedere attraverso l'immagine, e qualsiasi suono prodotto dall'immagine le suona artefatto.

\medskip\textbf{Immagine Silenziosa}\index{Incantesimi - Immagine Silenziosa}\\
\textbf{Scuola}: Illusione\\
\textbf{Difficoltà}: 16\\
\textbf{Tempo di Lancio}: 2 Azioni\\
\textbf{Gittata}: 36 metri\\
\textbf{Componenti}: V, S, M (un pezzo di vello)\\
\textbf{Durata}: Concentrazione, massimo 10 minuti\\
Crei l'immagine di un oggetto, una creatura o qualche altro fenomeno visibile non più grande di un cubo di 4 metri di spigolo. L'immagine appare in un punto a gittata che puoi vedere e resta per la durata dell'incantesimo. L'immagine è puramente visiva; non è accompagnata da suoni, odori o altri effetti sensoriali. Puoi usare un'azione per far muovere l'immagine in qualsiasi altro punto a gittata. Quando l'immagine cambia posizione, puoi alterarne l'aspetto così che i suoi movimenti appaiano naturali. Per esempio, se crei l'immagine di una creatura e la muovi, puoi alterare l'immagine in modo che sembri camminare.\\
L'interazione fisica con l'immagine la rivela come illusione, dato che le cose vi passano attraverso. Una creatura che usa 3 Azioni per esaminare l'immagine può determinare che si tratta di un'illusione con una prova di Consapevolezza contro la DC del Tiro Salvezza del tuo incantesimo. Se una creatura riconosce l'illusione per quello che è, la creatura può vedervi attraverso.

\medskip\textbf{Immagine Speculare}\index{Incantesimi - Immagine Speculare}\\
\textbf{Scuola}: Illusione\\
\textbf{Difficoltà}: 19\\
\textbf{Tempo di Lancio}: 2 Azioni\\
\textbf{Gittata}: Personale\\
\textbf{Componenti}: V, S\\
\textbf{Durata}: 1 minuto\\
Nel tuo spazio compaiono 2d4 duplicati illusori di te stesso. Fino al termine dell'incantesimo, i duplicati si muovono con te e imitano le tue azioni, scambiandosi di posto in modo da rendere impossibile determinare quale sia l'immagine reale. Puoi usare 2 Azioni per congedare i duplicati illusori.\\
Ogni volta che una creatura ti prende in realtà ha colpito una immagine illusoria.
Se una creatura fa piu' attacchi a turno può disperdere una immagine per ogni attacco andato a buon fine. Se vieni colpito da un incantesimo ad area tutte le immagini svaniscono.\\
Una creatura che non può vedere, o si affida a sensi diversi dalla vista (come la vista cieca), o che può distinguere le illusioni come false (come la visione del vero), ignora gli effetti di questo incantesimo. 

\medskip\textbf{Imprigionare}\index{Incantesimi - Imprigionare}\\
\textbf{Scuola}: Abiurazione\\
\textbf{Difficoltà}: 36\\
\textbf{Tempo di Lancio}: 2 Azioni\\
\textbf{Gittata}: 9 metri\\
\textbf{Componenti}: V, S, M (una raffigurazione su vello o una statuetta incisa con le fattezze del bersaglio, e una componente speciale che varia a seconda della versione che scegli dell'incantesimo, del valore di almeno 500 mo per Dado Ferita del bersaglio)\\
\textbf{Durata}: Fino a dissolvimento\\
Crei dei vincoli magici per bloccare una creatura a gittata e che puoi vedere. Il bersaglio deve superare un Tiro Salvezza su Volontà o essere avvinto dall'incantesimo; se lo supera, è immune all'incantesimo qualora lo lanci di nuovo. Mentre è soggetta a questo incantesimo, la creatura non ha bisogno di respirare, mangiare o bere e non invecchia. Gli incantesimi di divinazione non possono localizzare né percepire il bersaglio.\\
Quando lanci questo incantesimo, scegli una delle seguenti forme di prigionia.
\medskip
\begin{itemize}
\item
\textit{Incatenamento}. Catene pesanti, ben saldate al terreno, tengono il bersaglio ancorato. Il bersaglio è intralciato fino al termine dell'incantesimo, e non può muoversi né essere mosso in alcun modo fino ad allora. La componente speciale per questa versione dell'incantesimo è una catenella di metallo prezioso. 
\item
\textit{Isolamento Minimo}. Il bersaglio rimpicciolisce fino a 2,5 centimetri di altezza ed è imprigionato in una gemma o simile oggetto. La luce può attraversare normalmente la gemma (permettendo al bersaglio di vedere all'esterno e ad altre creature di vedere dentro), ma null'altro può attraversarla, neppure tramite teletrasporto o viaggio planare. La gemma non può essere tagliata né infranta finché l'incantesimo rimane in atto. La componente speciale per questa versione dell'incantesimo è una grande gemma trasparente, come il corindone, il diamante o il rubino.
\item
\textit{Prigione Confinata}. L'incantesimo trasporta il bersaglio in un minuscolo semipiano interdetto al teletrasporto e al viaggio planare. Il semipiano può essere un labirinto, una gabbia, una torre, o qualsiasi altra struttura chiusa scelta da te. La componente speciale per questa versione dell'incantesimo è una rappresentazione in miniatura della prigione fatta di giada.
\item
\textit{Sepoltura}. Il bersaglio viene sepolto nelle profondità della terra in una sfera di forza magica grande a sufficienza da contenere il bersaglio. Nulla può attraversare la sfera, né alcuna creatura può teletrasportarsi o usare il viaggio planare per entrarvi o uscire. La componente speciale per questa versione dell'incantesimo è una piccola sfera di mithril. 
\item
\textit{Sopore}. Il bersaglio cade addormentato e non può essere risvegliato. La componente speciale per questa versione dell'incantesimo consiste di rare erbe soporifere.
\end{itemize}
\medskip
\textit{Terminare l'incantesimo}. Durante il lancio dell'incantesimo, in qualsiasi delle sue versioni, puoi specificare una condizione che possa porre fine all'incantesimo e liberare il bersaglio. La condizione può essere tanto specifica o elaborata quanto desideri, ma il Narratore deve concordare che la condizione sia ragionevole e possa avverarsi. Le condizioni possono essere basate sul nome, l'identità o il Patrono di una creatura, ma comunque basate su azioni o qualità percepibili e non su cose intangibili come il livello, le Abilità o i punti ferita.\\
Un incantesimo dissolvi magie può porre fine all'incantesimo solo se lanciato come incantesimo a Difficolà 30, che prenda come bersaglio la prigione o la componente materiale usata per crearla.\\
Puoi usare una particolare componente speciale per creare solo una prigione alla volta. Se lanci l'incantesimo di nuovo usando la stessa componente, il bersaglio del primo lancio dell'incantesimo viene immediatamente liberato dal suo vincolo.

\medskip\textbf{Inaridire}\index{Incantesimi - Inaridire}\\
\textbf{Scuola}: Necromanzia\\
\textbf{Difficoltà}: 23\\
\textbf{Tempo di Lancio}: 2 Azioni\\
\textbf{Gittata}: 9 metri\\
\textbf{Componenti}: V, S\\
\textbf{Durata}: Istantanea\\
Energia necromantica avvolge una creatura di tua scelta a gittata e che puoi vedere, deprivandola di linfa e vitalità. Il bersaglio deve effettuare un Tiro Salvezza su Tempra. Se fallisce il Tiro Salvezza, il bersaglio subisce 8d8 danni da Vuoto, o la metà di questi danni se supera il Tiro Salvezza. L'incantesimo non ha effetto su non morti o costrutti.\\
Se il bersaglio è un vegetale non magico che non sia anche una creatura, come un albero o un cespuglio, non effettua alcun Tiro Salvezza, avvizzisce e muore all'istante.\\
\textbf{Per ogni Critico ottenuto} nella prova di magia il danno aumenta di 1d8.

\medskip\textbf{Individuazione del Bene e del Male}\index{Incantesimi - Individuazione del Bene e del Male}\\
\textbf{Scuola}: Divinazione\\
\textbf{Difficoltà}: 16\\
\textbf{Tempo di Lancio}: 2 Azioni\\
\textbf{Gittata}: Personale\\
\textbf{Componenti}: V, S\\
\textbf{Durata}: 1 minuto\\
Per la durata, apprendi se entro 9 metri da te si trova un'aberrazione, celestiale, elementale, fatato, demone o non morto, e la sua posizione. Allo stesso modo, apprendi se entro 9 metri da te si trovi un luogo o oggetto che sia stato consacrato o dissacrato magicamente.\\
l'incantesimo può penetrare la maggior parte delle barriere, ma è bloccato da 30 centimetri di pietra, 2,5 centimetri di metallo comune, un sottile foglio di piombo o 1 metro di legno o terra. 
\textbf{Nota}: questo incantesimo non ha effetto sulle creature che seguono i Tratti.

\medskip\textbf{Individuazione del Magico}\index{Incantesimi - Individuazione del Magico}\\
\textbf{Scuola}: Divinazione\\
\textbf{Difficoltà}: 16\\
\textbf{Tempo di Lancio}: 2 Azioni\\
\textbf{Gittata}: Personale\\
\textbf{Componenti}: V, S\\
\textbf{Durata}: 1 minuto\\
Per la durata, percepisci la presenza della magia entro 9 metri da te. Puoi usare 1 Azione per vedere una flebile aura che si estende intorno a qualsiasi creatura o oggetto visibile nell'area che rechi magia. Con due Azioni ne apprendi anche la scuola di magia, se ce l'ha.\\
L'incantesimo può penetrare la maggior parte delle barriere, ma è bloccato da 30 centimetri di pietra, 2,5 centimetri di metallo comune, un sottile foglio di piombo o 1 metro di legno o terra.

\medskip\textbf{Individuazione delle Malattie e dei Veleni}\index{Incantesimi - Individuazione delle Malattie e dei Veleni}\\
\textbf{Scuola}: Divinazione\\
\textbf{Difficoltà}: 13\
\textbf{Tempo di Lancio}: 2 Azioni\\
\textbf{Gittata}: Personale\\
\textbf{Componenti}: V, S, M (una foglia di tasso)\\
\textbf{Durata}: 1 minuto\\
Per la durata, percepisci la presenza e posizione di veleni, creature velenose e malattie entro 9 metri da te. Inoltre riesci a identificare il tipo di veleno, creatura velenosa o malattia. L'incantesimo può penetrare la maggior parte delle barriere, ma è bloccato da 30 centimetri di pietra, 2,5 centimetri di metallo comune, un sottile foglio di piombo o 1 metro di legno o terra.

\medskip\textbf{Individuazione dei Pensieri}\index{Incantesimi - Individuazione dei Pensieri}\\
\textbf{Scuola}: Divinazione\\
\textbf{Difficoltà}: 19\\
\textbf{Tempo di Lancio}: 2 Azioni\\
\textbf{Gittata}: Personale\\
\textbf{Componenti}: V, S, M (un pezzo di rame)\\
\textbf{Durata}: 1 minuto\\
Per la durata, puoi leggere i pensieri di certe creature. Quando lanci questo incantesimo e con altre due Azioni in ciascun round successivo sino al termine dell'incantesimo, puoi concentrare la tua mente su qualsiasi creatura che tu possa vedere e si trovi entro 9 metri da te. Se la creatura che hai scelto ha un punteggio di Intelligenza -3 o meno o non parla nessun linguaggio, la creatura ignora l'effetto.\\
Inizialmente, apprendi solo i pensieri di superficie della creatura: quelli più ricorrenti. Con un'azione, puoi o spostare la tua attenzione sui pensieri di un'altra creatura o tentare di sondare più a fondo la mente della stessa creatura. Se sondi più a fondo, il bersaglio deve effettuare un Tiro Salvezza su Volontà. Se lo fallisce, ottieni una percezione dei suoi ragionamenti (se ve ne sono), del suo stato emotivo, e di ogni cosa abbia prevalenza nei suoi pensieri (come una preoccupazione, l'amore, o l'odio). Se supera il Tiro Salvezza, l'incantesimo termina. A ogni modo, il bersaglio sa che stai sondando la sua mente e, a meno che non sposti la tua attenzione verso la mente di un'altra creatura, nel suo round la creatura può usare la 2 Azioni per effettuare una prova di Intelligenza contesa dalla tua prova di Intelligenza; se la vince, l'incantesimo termina.\\
Le domande poste verbalmente alla creatura bersaglio, ovviamente, modellano il corso dei suoi pensieri, cosicché questo incantesimo risulta particolarmente efficace negli interrogatori.\\
Puoi anche usare questo incantesimo per individuare la presenza di creature pensanti che non puoi vedere. Quando lanci questo incantesimo o con 2 Azioni nella sua durata, puoi cercare pensieri entro 9 metri da te. L'incantesimo può penetrare le barriere, ma è bloccato da 60 centimetri di pietra, 5 centimetri di metallo che non sia il piombo, o un sottile foglio di piombo. Non puoi individuare una creatura con Intelligenza -3 o meno, o una creatura che non parla alcun linguaggio. Una volta individuata in questo modo la presenza di una creatura, puoi leggerne i pensieri per la durata dell'incantesimo finché resta nella gittata, come descritto sopra, anche se non puoi vederla.
Mentre hai attivo questo incantesimo per il lancio di altri incantesimo risulterai Distratto.

\medskip\textbf{Infliggi Ferite}\index{Incantesimi - Infliggi Ferite}\\
\textbf{Scuola}: Necromanzia\\
\textbf{Difficoltà}: 13 \\
\textbf{Tempo di Lancio}: 2 Azioni\\
\textbf{Gittata}: Contatto\\
\textbf{Componenti}: V, S\\
\textbf{Durata}: Istantanea\\
Effettua un attacco in mischia con incantesimo contro una creatura a portata. Se colpisci, il bersaglio subisce 3d10 danni da Vuoto.\\
\textbf{Per ogni Critico ottenuto} nella prova di magia il danno aumenta di 1d8.

\medskip\textbf{Ingrandire/Ridurre}\index{Incantesimi - Ingrandire/Ridurre}\\
\textbf{Scuola}: Trasmutazione\\
\textbf{Difficoltà}: 19\\
\textbf{Tempo di Lancio}: 2 Azioni\\
\textbf{Gittata}: 9 metri\\
\textbf{Componenti}: V, S, M (un pizzico di ferro in polvere)\\
\textbf{Durata}: 1 minuto\\
Fai sì che una creatura od oggetto a gittata e che puoi vedere ingrandisca o rimpicciolisca per la durata dell'incantesimo. Scegli una creatura o un oggetto che non sia né indossato né trasportato. Se il bersaglio non è consenziente, può effettuare un Tiro Salvezza su Tempra, se lo supera, l'incantesimo non ha effetto. Se il bersaglio è una creatura, tutto ciò che sta indossando e trasportando cambia taglia assieme a essa. Qualsiasi oggetto lasciato cadere da una creatura soggetta a questo incantesimo ritorna subito alla sua taglia normale.\\
\medskip
\begin{itemize}
\item
\textit{Ingrandire}. La taglia del bersaglio raddoppia in tutte le dimensioni, e il suo peso è moltiplicato per otto. Questa crescita aumenta la sua taglia di una categoria: da Media a Grande, per esempio. Se non c'è spazio sufficiente perché il bersaglio raddoppi la sua taglia, la creatura od oggetto assume la taglia più grossa possibile permessagli dallo spazio disponibile. Fino al termine dell'incantesimo, il bersaglio ha +1d6 alle prove di Forza e ai Tiri Salvezza su Tempra. Le armi del bersaglio crescono per raggiungere la nuova taglia. Mentre queste armi sono ingrandite, gli attacchi del bersaglio con esse faranno una categoria di danno ulteriore. 
\item
\textit{Ridurre}. La taglia del bersaglio si dimezza in tutte le dimensioni, e il suo peso è ridotto a un ottavo. Questa crescita diminuisce la sua taglia di una categoria: da Media a Piccola, per esempio. Fino al termine dell'incantesimo, il bersaglio ha -1d6 alle prove di Forza e ai Tiri Salvezza su Tempra. Le armi del bersaglio rimpiccioliscono per raggiungere la nuova taglia. Mentre queste armi sono rimpicciolite, gli attacchi del bersaglio con esse faranno una categoria di danno inferiore (ma senza ridurre il danno dell'arma a meno di 1).\\
\textbf{Per ogni due Critici ottenuti} nella prova di magia la creatura aumenta di un altra taglia.

\end{itemize}

\medskip\textbf{Insetto Gigante}\index{Incantesimi - Insetto Gigante}\\
\textbf{Scuola}: Trasmutazione\\
\textbf{Difficoltà}: 23\\
\textbf{Tempo di Lancio}: 2 Azioni\\
\textbf{Gittata}: 9 metri\\
\textbf{Componenti}: V, S\\
\textbf{Durata}: 10 minuti\\
Per la durata dell'incantesimo, trasformi fino a dieci centopiedi, tre ragni, cinque vespe o uno scorpione a gittata, in versioni giganti della loro forma naturale. Un centopiedi diventa un centopiedi gigante, un ragno diventa un ragno gigante, una vespa diventa una vespa gigante e uno scorpione diventa uno scorpione gigante. Ogni creatura obbedisce ai tuoi comandi vocali e, in combattimento, agisce in ciascun round durante il tuo round. Il Narratore possiede le statistiche di queste creature, e sarà sempre Il Narratore a risolvere le loro azioni e i loro movimenti. Una creatura resta nella sua forma gigante per la durata, finché non scende a 0 punti ferita, o finché non usi un'azione per interrompere l'effetto su di essa.\\
Il Narratore può permetterti di scegliere bersagli differenti. Per esempio, se trasformi un'ape, la sua versione gigante potrebbe avere le stesse statistiche della vespa gigante.

\medskip\textbf{Interdizione alla Morte}\index{Incantesimi - Interdizione alla Morte}\\
\textbf{Scuola}: Abiurazione\\
\textbf{Difficoltà}: 23\\
\textbf{Tempo di Lancio}: 2 Azioni\\
\textbf{Gittata}: Contatto\\
\textbf{Componenti}: V, S\\
\textbf{Durata}: 8 ore\\
Lanci l'incantesimo a contatto con una creatura. Conferisci al bersaglio protezione dalla morte. La prima volta che il bersaglio dovesse scendere a 0 punti ferita in seguito al danno subito, il bersaglio scende invece a 1 punto ferita e l'incantesimo ha fine. Se l'incantesimo è ancora attivo quando il bersaglio è vittima di un effetto che lo ucciderebbe all'istante senza infliggere danni,quell'effetto viene invece negato sul bersaglio e l'incantesimo ha fine.

\medskip\textbf{Intermittenza}\index{Incantesimi - Intermittenza}\\
\textbf{Scuola}: Trasmutazione\\
\textbf{Difficoltà}: 21\\
\textbf{Tempo di Lancio}: 2 Azioni\\
\textbf{Gittata}: Personale\\
\textbf{Componenti}: V, S\\
\textbf{Durata}: 1 minuto\\
Tira un 1d6 alla fine di ciascun tuo round per la durata di questo incantesimo. Se ottieni un numero dispari svanisci dal tuo attuale piano di esistenza e riappari sul Piano Etereo (l'incantesimo fallisce e il lancio è sprecato qualora tu fossi già su quel piano). All'inizio del tuo prossimo round, e quando l'incantesimo termina, qualora tu fossi sul Piano Etereo, ritorni in uno spazio non occupato di tua scelta e che puoi vedere, entro 3 metri dallo spazio da cui sei svanito. Se nessuno spazio non occupato è disponibile entro questa gittata, compari nello spazio non occupato più vicino (determinato casualmente se è disponibile più di uno spazio). Puoi interrompere l'incantesimo con un'azione.\\
Mentre sei sul Piano Etereo, puoi vedere e udire il piano da cui provieni, che percepisci in sfumature di grigio, ma non puoi comunque percepire nulla che si trovi a più di 18 metri di distanza. Puoi interagire solo con creature che si trovano sul Piano Etereo. Le creature che non si trovano lì non possono né percepirti né interagire con te, a meno che non siano provviste della capacità di farlo.

\medskip\textbf{Intimorire Infernale}\index{Incantesimi - Intimorire Infernale}\\
\textbf{Scuola}: Evocazione\\
\textbf{Difficoltà}: 16\\
\textbf{Tempo di Lancio}: 1 reazione, che puoi effettuare in risposta al danno arrecatoti da una creatura entro 18 metri da te che puoi vedere\\
\textbf{Gittata}: 18 metri\\
\textbf{Componenti}: V, S\\
\textbf{Durata}: Istantanea\\
Punti il dito, e la creatura che ti ha danneggiato viene momentaneamente avvolta da fiamme diaboliche. La creatura deve effettuare un Tiro Salvezza su Riflessi. Subisce 2d10 danni da fuoco se fallisce il Tiro Salvezza, o la metà di questi danni se lo supera.\\
\textbf{Per ogni Critico ottenuto} nella prova di magia i danno aumenta di 1d8

\medskip\textbf{Intralciare}\index{Incantesimi - Intralciare}\\
\textbf{Scuola}: Invocazione\\
\textbf{Difficoltà}: 16\\
\textbf{Tempo di Lancio}: 2 Azioni\\
\textbf{Gittata}: 27 metri\\
\textbf{Componenti}: V, S\\
\textbf{Durata}: 1 minuto\\
Rampicanti e rami stritolanti spuntano dal terreno in un quadrato di 6 metri di lato a partire da un punto a gittata. Per la durata, questi vegetali trasformano il terreno nell'area in terreno difficile.\\
Una creatura nell'area nel momento in cui lanci questo incantesimo deve superare un Tiro Salvezza su Tempra o restare intralciata da questi vegetali fino al termine dell'incantesimo. Una creatura intralciata dai vegetali può usare le sue azioni per effettuare una prova di Forza contro la DC del Tiro Salvezza dell'incantesimo. Se la supera, si libera. Quando l'incantesimo ha termine, i vegetali evocati svaniscono.

\medskip\textbf{Inversione della Gravità}\index{Incantesimi - Inversione della Gravità}\\
\textbf{Scuola}: Trasmutazione\\
\textbf{Difficoltà}: 31\\
\textbf{Tempo di Lancio}: 2 Azioni\\
\textbf{Gittata}: 30 metri\\
\textbf{Componenti}: V, S, M (una calamita e un fil di ferro)\\
\textbf{Durata}: Concentrazione, massimo 1 minuto 
Questo incantesimo inverte la gravità in un cilindro di raggio 15 metri, alto 30 metri, centrato in un punto a gittata. Quando lanci questo incantesimo, tutte le creature e gli oggetti che non sono in qualche modo ancorati al terreno cadono verso l'alto e raggiungono la cima dell'area. Una creatura può tentare un Tiro Salvezza su Riflessi per afferrare un oggetto fisso a portata, per evitare di cadere in questo modo, in caso lo superi.\\
Se lungo questa caduta si incontra un oggetto solido (il soffitto), gli oggetti e le creature che cadono vi impattano come accadrebbe durante una normale caduta. Se un oggetto o creatura raggiunge la cima dell'area senza colpire nulla, rimane lì, oscillando lievemente, per la durata.\\
Al termine della durata, gli oggetti e le creature colpite ricadono verso il basso.

\medskip\textbf{Inviare}\index{Incantesimi - Inviare}\\
\textbf{Scuola}: Invocazione\\
\textbf{Difficoltà}: 21\\
\textbf{Tempo di Lancio}: 2 Azioni\\
\textbf{Gittata}: Illimitata\\
\textbf{Componenti}: V, S, M (un piccolo pezzo di cavo di rame)\\
\textbf{Durata}: 1 round\\
Invii un breve messaggio di 25 parole o meno a una creatura con cui sei familiare. La creatura sente il messaggio nella sua mente, ti riconosce come mittente, e può risponderti in modo simile. L'incantesimo permette a creature con un punteggio di Intelligenza almeno di 1 di comprendere il significato del tuo messaggio anche se non comprende la tua lingua.\\
Puoi inviare il messaggio attraverso qualsiasi distanza e anche su altri piani di esistenza, ma se il bersaglio è su di un piano diverso dal tuo, c'è una probabilità del 5\% che il messaggio non arrivi.

\medskip\textbf{Invisibilità}\index{Incantesimi - Invisibilità}\\
\textbf{Scuola}: Illusione\\
\textbf{Difficoltà}: 19\\
\textbf{Tempo di Lancio}: 2 Azioni\\
\textbf{Gittata}: Contatto\\
\textbf{Componenti}: V, S, M (un ciglio avvolto nella gomma arabica)\\
\textbf{Durata}: 1 ora \\
Lanci l'incantesimo a contatto di una creatura. Il bersaglio diventa invisibile fino alla fine dell'incantesimo. Qualsiasi cosa il bersaglio stia indossando o trasportando diventa invisibile finché resta sul bersaglio. L'incantesimo ha fine per il bersaglio che attacca o esegue un incantesimo.\\
\textbf{Per ogni Critico ottenuto} nella prova di magia puoi scegliere un'ulteriore creatura bersaglio.

\medskip\textbf{Invisibilità Superiore}\index{Incantesimi - Invisibilità Superiore}\\
\textbf{Scuola}: Illusione\\
\textbf{Difficoltà}: 23\\
\textbf{Tempo di Lancio}: 2 Azioni\\
\textbf{Gittata}: Contatto\\
\textbf{Componenti}: V, S\\
\textbf{Durata}: 1 minuto\\
Lanci l'incantesimo a contatto di una creatura. Il bersaglio diventa invisibile fino alla fine dell'incantesimo. Qualsiasi cosa indossata o trasportata dal bersaglio diventa invisibile finché resta addosso al bersaglio.\\
Eseguire incantesimi o azioni di attacco non ti fa diventare visibile.

\medskip\textbf{Invocare il Fulmine}\index{Incantesimi - Invocare il Fulmine}\\
\textbf{Scuola}: Evocazione\\
\textbf{Difficoltà}: 21\\
\textbf{Tempo di Lancio}: 1 round\\
\textbf{Gittata}: 36 metri\\
\textbf{Componenti}: V, S\\
\textbf{Durata}: Concentrazione, massimo 10 minuti\\
Una nube di tempesta compare nella forma di un cilindro alto 3 metri con un raggio di 18 metri, centrato su di un punto che puoi vedere, 30 metri sopra di te. L'incantesimo fallisce automaticamente se non puoi vedere il punto nell'aria dove apparirà la nube di tempesta (per esempio, se sei in una stanza che non può accogliere la nube). Quando lanci l'incantesimo, scegli un punto che puoi vedere entro la gittata. Un fulmine si abbatterà dalla nuvola su quel punto. Ogni creatura entro 1 metro da quel punto deve effettuare un Tiro Salvezza su Riflessi. Una creatura subisce 3d10 danni da fulmine se fallisce il Tiro Salvezza, o la metà di questi danni se lo supera. Durante ciascun tuo round fino al termine dell'incantesimo, puoi usare due Azioni per richiamare un altro fulmine in questo modo, prendendo come bersaglio lo stesso punto o uno diverso.\\
Se quando lanci questo incantesimo ti trovi all'esterno in condizioni di tempesta, l'incantesimo ti fornisce il controllo della tempesta esistente invece di crearne una nuova. Sotto queste condizioni, il danno dell'incantesimo aumenta di 1d10. \\
\textbf{Per ogni Critico ottenuto} nella prova di magia il danno aumenta di 1d8

\medskip\textbf{Labirinto}\index{Incantesimi - Labirinto}\\
\textbf{Scuola}: Evocazione\\
\textbf{Difficoltà}: 34\\
\textbf{Tempo di Lancio}: 2 Azioni\\
\textbf{Gittata}: 18 metri\\
\textbf{Componenti}: V, S\\
\textbf{Durata}: Concentrazione, massimo 10 minuti\\
Bandisci una creatura a gittata e che puoi vedere in un semipiano labirintico. Il bersaglio rimane lì per la durata dell'incantesimo o finché non fugge dal labirinto. Il bersaglio può impiegare 3 Azioni per tentare di fuggire. Quando lo fa, effettua una prova di Intelligenza DC 25. Se la supera, fugge, e l'incantesimo termina (un minotauro o un demone goristro riescono automaticamente).\\
Quando l'incantesimo termina, il bersaglio riappare nello spazio che aveva lasciato o, se quello spazio è occupato, nel più vicino spazio non occupato. 

\medskip\textbf{Lama Infuocata}\index{Incantesimi - Lama Infuocata}\\
\textbf{Scuola}: Invocazione\\
\textbf{Difficoltà}: 19\\
\textbf{Tempo di Lancio}: 1 Azione Immediata\\
\textbf{Gittata}: Personale\\
\textbf{Componenti}: V, S, M (una foglia di sommacco)\\
\textbf{Durata}: Concentrazione, massimo 10 minuti \\
Crei nella tua mano una lama infuocata. La lama è simile in dimensioni e forma a una scimitarra, e rimane per la durata. Se lasci andare la lama, questa sparisce, ma ne puoi creare un'altra con un'Azione. Puoi usare 2 Azioni per effettuare un attacco in mischia con la lama infuocata. Se colpisci, il bersaglio subisce 3d6 danni da fuoco. La lama infuocata emana luce intensa in un raggio di 3 metri e luce fioca per ulteriori 3 metri.\\
\textbf{Per ogni due Critici ottenuti} nella prova di magia il danno aumenta di 1d6.

\medskip\textbf{Legame Telepatico}\index{Incantesimi - Legame Telepatico}\\
\textbf{Scuola}: Divinazione\\
\textbf{Difficoltà}: 26\\
\textbf{Tempo di Lancio}: 2 Azioni\\
\textbf{Gittata}: 9 metri\\
\textbf{Componenti}: V, S, M (pezzi di gusci d'uovo da due differenti specie di creature)\\
\textbf{Durata}: 1 ora\\
Stabilisci un collegamento telepatico tra un massimo di otto creature consenzienti a gittata di tua scelta, collegando psichicamente ciascuna creatura alle altre per la durata dell'incantesimo. Le creature con punteggio di Intelligenza -3 o meno ignorano questo incantesimo. Fino al termine dell'incantesimo, i bersagli possono comunicare telepaticamente tramite questo legame, che condividano o meno un linguaggio comune. La comunicazione è possibile a qualsiasi distanza, ma non può estendersi su differenti piani di esistenza.

\medskip\hypertarget{lentezza}{\textbf{Lentezza}}\index{Incantesimi - Lentezza}\\
\textbf{Scuola}: Trasmutazione\\
\textbf{Difficoltà}: 21\\
\textbf{Tempo di Lancio}: 2 Azioni\\
\textbf{Gittata}: 36 metri\\
\textbf{Componenti}: V, S, M (una goccia di melassa) \\
\textbf{Durata}: 1 minuto, Concentrazione\\
Modifichi lo scorrere del tempo intorno a un massimo di sei creature di tua scelta in un cubo di 12 metri di spigolo a gittata. Ciascun bersaglio deve superare un Tiro Salvezza su Volontà o subire gli effetti dell'incantesimo per la sua durata.\\
La velocità di un bersaglio soggetto all'incantesimo è dimezzata, questi subisce una penalità di -2 alla Difesa e ai Tiri Salvezza su Destrezza, e non può usare reazioni. Durante il suo round, può usare un'Azione o un'Azione Immediata, ma non entrambe. Quali che siano le capacità o gli oggetti magici della creatura, durante il suo round questa non può effettuare più di un attacco in mischia o a distanza.\\
Se la creatura tenta di lanciare un incantesimo con tempo di lancio di 2 azioni, tira un 1d6. Con 4 o più, l'incantesimo non avrà effetto fino al prossimo round della creatura, e la creatura dovrà usare 3 Azioni in quel round per completare l'incantesimo. Se non potrà farlo, l'incantesimo viene sprecato.\\
Una creatura sotto l'effetto di questo incantesimo effettua un altro Tiro Salvezza su Volontà al termine del suo round. Se supera questo Tiro Salvezza, l'effetto ha termine.\\

\medskip\textbf{Levitazione}\index{Incantesimi - Levitazione}\\
\textbf{Scuola}: Trasmutazione\\
\textbf{Difficoltà}: 19\\
\textbf{Tempo di Lancio}: 2 Azioni\\
\textbf{Gittata}: 18 metri\\
\textbf{Componenti}: V, S, M (o un piccolo laccio di cuoio oppure un pezzo di cavo d'oro piegato a forma di tazza con un lungo stelo alla fine)\\
\textbf{Durata}: 10 minuti \\
Una creatura o oggetto a gittata che puoi vedere, scelto da te, si alza verticalmente fino a 6 metri e rimane sospeso per la durata dell'incantesimo. L'incantesimo può levitare un bersaglio pesante fino a 250 chili. Una creatura non consenziente che superi un Tiro Salvezza su Tempra ignora l'effetto.\\
Il bersaglio può muoversi solo spingendo o tirando verso un oggetto fisso o superficie a portata (per esempio un muro o un soffitto). Durante il tuo round puoi cambiare l'altitudine del bersaglio fino a 6 metri in entrambe le direzioni. Se sei tu il bersaglio, ti puoi muovere verso l'alto o il basso come parte del tuo movimento. Altrimenti puoi usare 1 Azione per muovere il bersaglio, che deve rimanere nella gittata dell'incantesimo. Quando l'incantesimo termina, qualora sia ancora in aria, il bersaglio fluttua dolcemente a terra.\\
Mentre sei sotto l'influenza di questo incantesimo sei considerato Distratto nel lancio di incantesimi.

\medskip\textbf{Libertà di Movimento}\index{Incantesimi - Libertà di Movimento}\\
\textbf{Scuola}: Abiurazione\\
\textbf{Difficoltà}: 23\\
\textbf{Tempo di Lancio}: 2 Azioni\\
\textbf{Gittata}: Contatto\\
\textbf{Componenti}: V, S, M (una striscia di cuoio, avvolta intorno a un braccio o simile appendice)\\
\textbf{Durata}: 1 ora\\
Lanci l'incantesimo a contatto di una creatura consenziente. Per la sua durata, il movimento del bersaglio ignora il terreno difficile, mentre gli incantesimi o altri effetti magici non possono ridurre la sua velocità né far sì che il bersaglio sia paralizzato o intralciato.\\
Il bersaglio può usare due Azioni per liberarsi automaticamente da qualsiasi restrizione non magica, come manette o una creatura da cui è afferrato. Infine, trovarsi sott'acqua non comporta penalità al movimento o gli attacchi del bersaglio. 

\medskip\textbf{Lingue}\index{Incantesimi - Lingue}\\
\textbf{Scuola}: Divinazione\\
\textbf{Difficoltà}: 21\\
\textbf{Tempo di Lancio}: 2 Azioni\\
\textbf{Gittata}: Contatto\\
\textbf{Componenti}: V, M (un piccolo modello di argilla di una ziggurat)\\
\textbf{Durata}: 1 ora\\
Questo incantesimo conferisce alla creatura con cui sei stato in contatto al momento del lancio dell'incantesimo la capacità di comprendere qualsiasi linguaggio parlata che ode. Inoltre, quando il bersaglio parla, qualsiasi creatura che conosca almeno un linguaggio e può udire il bersaglio, comprende ciò che dice.

\medskip\textbf{Localizza Animali e Piante}\index{Incantesimi - Localizza Animali e Piante}\\
\textbf{Scuola}: Divinazione\\
\textbf{Difficoltà}: 19\\
\textbf{Tempo di Lancio}: 2 Azioni\\
\textbf{Gittata}: Personale\\
\textbf{Componenti}: V, S, M (un pezzo di pelo di un segugio) \\
\textbf{Durata}: Istantanea\\
Descrivi o nomina uno specifico tipo di bestia o vegetale. Concentrandoti sulla voce della natura nei tuoi dintorni, apprendi la direzione e la distanza dalla più vicina creatura o vegetale di quella specie, se ce ne sono entro 7,5 chilometri.

\medskip\textbf{Localizza Creatura}\index{Incantesimi - Localizza Creatura}\\
\textbf{Scuola}: Divinazione\\
\textbf{Difficoltà}: 23\\
\textbf{Tempo di Lancio}: 2 Azioni\\
\textbf{Gittata}: Personale\\
\textbf{Componenti}: V, S, M (un pezzo di pelliccia di segugio)\\
\textbf{Durata}: Concentrazione, massimo 1 ora\\
Descrivi o nomina una creatura che ti è familiare. Percepisci la direzione della posizione della creatura, purché quella creatura si trovi entro 300 metri da te. Se la creatura si muove, conosci anche la direzione del suo movimento.\\
L'incantesimo può localizzare una specifica creatura a te nota, o la più vicina creatura di una specie (come umano o unicorno), purché tu abbia visto una simile creatura da vicino (entro 9 metri) almeno una volta. Se la creatura che descrivi o nomini ha una forma diversa, per esempio è sotto gli effetti dell'incantesimo metamorfosi, questo incantesimo non sarà in grado di localizzare la creatura.\\
Questo incantesimo non può localizzare una creatura se un flusso di acqua corrente largo almeno 3 metri blocca un percorso diretto tra te e la creatura.

\medskip\textbf{Localizza Oggetto}\index{Incantesimi - Localizza Oggetto}\\
\textbf{Scuola}: Divinazione\\
\textbf{Difficoltà}: 19\\
\textbf{Tempo di Lancio}: 2 Azioni\\
\textbf{Gittata}: Personale\\
\textbf{Componenti}: V, S, M (un ramoscello biforcuto)\\
\textbf{Durata}: Concentrazione, massimo 10 minuti \\
Descrivi o nomina un oggetto che ti è familiare. Percepisci la direzione della posizione dell'oggetto, purché quell'oggetto si trovi entro 300 metri da te. Se l'oggetto si muove, conosci anche la direzione del suo movimento.\\
l'incantesimo può localizzare uno specifico oggetto a te noto, purché tu lo abbia visto da vicino (entro 9 metri) almeno una volta. In alternativa, l'incantesimo può localizzare l'oggetto più vicino di un particolare tipo, come certi tipi di abbigliamento, gioielleria, mobili, attrezzi o armi.\\
Questo incantesimo non può localizzare un oggetto se qualsiasi spessore di piombo, anche un foglio sottile, blocca un percorso diretto tra di te e l'oggetto. 

\medskip\textbf{Loquacità}\index{Incantesimi - Loquacità}\\
\textbf{Scuola}: Trasmutazione\\
\textbf{Difficoltà}: 34\\
\textbf{Tempo di Lancio}: 2 Azioni\\
\textbf{Gittata}: Personale\\
\textbf{Componenti}: V\\
\textbf{Durata}: 1 ora\\
Fino al termine dell'incantesimo, quando effettui una prova di Carisma puoi rimpiazzare il numero tirato con 15. Inoltre, non importa quello che dici, la magia o l'analisi che determina se stai dicendo la verità indicherà sempre che sei onesto.

\medskip\textbf{Luce}\index{Incantesimi - Luce}\\
\textbf{Scuola}: Invocazione\\
\textbf{Difficoltà}: 16\\
\textbf{Tempo di Lancio}: 2 Azioni\\
\textbf{Gittata}: Contatto\\
\textbf{Componenti}: V, M (una lucciola o del muschio fosforescente)\\
\textbf{Durata}: 1 ora +1 turno per Competenza Magica\\
Lanci l'incantesimo a contatto di un oggetto che non sia più grosso di 3 metri in qualsiasi direzione. Fino al termine dell'incantesimo, l'oggetto irradia una luce intensa in un raggio di 6 metri e penombra per ulteriori 6 metri. La luce può essere di qualsiasi colore tu voglia. Coprire completamente l'oggetto con qualcosa di opaco blocca la luce. Se un oggetto bersaglio è tenuto o indossato da una creatura ostile, quella creatura deve superare un Tiro Salvezza su Riflessi per evitare l'incantesimo. Una creatura colpita dall'incantesimo non si considera accecata.\\
\textbf{Per ogni due Critici ottenuti} nella prova di magia la durata aumenta di 1 ora.

\medskip\textbf{Luce Diurna}\index{Incantesimi - Luce Diurna}\\
\textbf{Scuola}: Invocazione\\
\textbf{Difficoltà}: 21\\
\textbf{Tempo di Lancio}: 2 Azioni\\
\textbf{Gittata}: 18 metri\\
\textbf{Componenti}: V, S\\
\textbf{Durata}: 1 ora\\
Una sfera di luce con raggio 18 metri si espande da un punto a tua scelta entro la gittata. La sfera irradia luce intensa e luca fioca per ulteriori 18 metri. Se scegli un punto su di un oggetto che stai reggendo o che non è indossato o trasportato, la luce si irradia dall'oggetto e si muove con esso. Coprire completamente un oggetto con qualcosa di opaco, come un vaso o un elmo, blocca la luce. Se qualsiasi parte dell'area di questo incantesimo si sovrappone con l'area di oscurità creata da un incantesimo di Difficoltà 18 o più basso, l'incantesimo che ha creato l'oscurità viene dissolto.\\
La luce creata si considera luce solare.

\medskip\textbf{Luci Danzanti}\index{Trucchetto - Luci Danzanti}\\
\textbf{Scuola}: Invocazione\\
\textbf{Difficoltà}: 12\\
\textbf{Tempo di Lancio}: 2 Azioni\\
\textbf{Gittata}: 36 metri\\
\textbf{Componenti}: V, S, M (un pezzo di fosforo o legno stregato, o un lombrico)\\
\textbf{Durata}: 1 minuto\\
Crei, a gittata, fino a quattro luci delle dimensioni di una torcia, facendole apparire come torce, lanterne o sfere luminose che fluttuano nell'aria per la durata dell'incantesimo. Puoi anche combinare le quattro luci in un'unica forma luminosa vagamente umanoide di taglia Media. Qualsiasi forma scegli, ciascuna luce emette una luce fioca in un raggio di 3 metri. Come 1 Azione di movimento durante il tuo round, puoi spostare le luci fino a 18 metri in un nuovo punto a gittata.\\
Una luce deve trovarsi entro 6 metri da un'altra luce creata con questo incantesimo, e le luci svaniscono se eccedono la gittata dell'incantesimo. 

\medskip\textbf{Luminescenza}\index{Incantesimi - Luminescenza}\\
\textbf{Scuola}: Invocazione\\
\textbf{Difficoltà}: 16\\
\textbf{Tempo di Lancio}: 2 Azioni\\
\textbf{Gittata}: 18 metri\\
\textbf{Componenti}: V\\
\textbf{Durata}: 1 minuto \\
Tutti gli oggetti in un cubo di 6 metri di spigolo a gittata vengono circondati da una luce blu, verde o viola (a tua scelta). Qualsiasi creatura nell'area quando l'incantesimo viene lanciato, viene anch'essa circondata dalla luce se fallisce un Tiro Salvezza su Riflessi. Per la durata dell'incantesimo, gli oggetti e le creature soggette emettono una luce fioca con raggio di 3 metri. Qualsiasi tiro per colpire contro una creatura od oggetto soggetto ha +1d6 se l'attaccante può vederlo, e la creatura od oggetto non può beneficiare dell'invisibilità.

\medskip\textbf{Mani Brucianti}\index{Incantesimi - Mani Brucianti}\\
\textbf{Scuola}: Invocazione\\
\textbf{Difficoltà}: 16\\
\textbf{Tempo di Lancio}: 2 Azioni\\
\textbf{Gittata}: Personale (cono di 4 metri)\\
\textbf{Componenti}: V, S\\
\textbf{Durata}: Istantanea\\
Mentre tieni le mani con i pollici che si toccano e le dita tese, un sottile fiotto di fiamme parte da ciascuna delle punta delle tue dita. Ogni creatura in un cono di 4 metri deve effettuare un Tiro Salvezza su Riflessi. Una creatura subisce 3d6 danni da fuoco se fallisce il Tiro Salvezza, o la metà se lo supera. Il fuoco incendia gli oggetti infiammabili nell'area che non siano indossati o trasportati.\\
\textbf{Per ogni Critico ottenuto} nella prova di magia il danno aumenta di 1d6

\medskip\textbf{Mano Arcana}\index{Incantesimi - Mano Arcana}\\
\textbf{Scuola}: Invocazione\\
\textbf{Difficoltà}: 26\\
\textbf{Tempo di Lancio}: 2 Azioni\\
\textbf{Gittata}: 36 metri\\
\textbf{Componenti}: V, S, M (un guscio d'uovo e un guanto di pelle di serpente)\\
\textbf{Durata}: Concentrazione, 1 minuto\\
Crei una mano Grande, composta di energia trasparente e luminosa, in uno spazio non occupato a gittata e che puoi vedere. La mano permane per la durata dell'incantesimo, e si muove al tuo comando, imitando i movimenti della tua mano.\\
La mano è un oggetto che ha Difesa 25 e punti ferita uguali ai tuoi punti ferita massimi. Ha Forza 8 e Destrezza 0 . La mano non riempie il suo spazio.
Quando lanci l'incantesimo e come 2 Azioni durante i tuoi round successivi, puoi muovere la mano fino a 18 metri e poi generare uno dei seguenti effetti. 
\medskip
\begin{itemize}
\item
\textit{Mano Afferrante}. La mano cerca di afferrare una creatura di taglia Enorme o più piccola che si trovi entro 1 metro da essa. Per risolvere l'azione di lottare usi la Forza della mano. Se il bersaglio è di taglia Media o inferiore, hai +1d6 alla prova. Mentre la mano tiene afferrato il bersaglio, puoi usare un'Azione per fare stritolare il bersaglio dalla mano. Quando lo fai, il bersaglio subisce danni da botta pari a 2d6 + il tuo valore di Intelligenza o Saggezza
\item
\textit{Mano di Forza}. La mano cerca di spingere una creatura di 1 metro in una direzione a tua scelta. Effettua una prova di Forza della mano contesa dalla prova di Forza del bersaglio. Se il bersaglio è di taglia Media o inferiore, hai +1d6 alla prova. Se vinci la contesa, la mano spinge il bersaglio di 1 metro più 1 metro moltiplicato per il valore di Intelligenza o Saggezza (minimo 1 metro). La mano si muove assieme al bersaglio per restare entro 1 metro da lui.\\
\item
\textit{Mano Frapposta}. La mano si frappone tra di te e una creatura di tua scelta finché non le dai un comando diverso. La mano si muove di modo da restare tra di te e il bersaglio, fornendoti metà copertura contro il bersaglio. Il bersaglio non può muoversi attraverso lo spazio della mano se il suo punteggio di Forza è uguale o inferiore al punteggio di Forza della mano. Se il suo punteggio di Forza è superiore al punteggio di Forza della mano, il bersaglio può muoversi attraverso lo spazio della mano, ma considera quello spazio come fosse terreno difficile. \\
\item
\textit{Pugno Serrato}. La mano colpisce una creatura o un oggetto entro 1 metro da essa. Effettua un attacco in mischia con incantesimo usando la mano, il TC e' basato sul CM e Destrezza. Se colpisci, il bersaglio subisce 4d8 danni da forza.
\end{itemize}
\medskip
\textbf{Per ogni Critico ottenuto} nella prova di magia il danno dell'opzione pugno serrato aumenta di 1d8 e il danno dell'opzione mano afferrante aumenta di 1d6.

\medskip\textbf{Mano Magica}\index{Trucchetto - Mano Magica}\\
\textbf{Scuola}: Evocazione\\
\textbf{Difficoltà}: 12\\
\textbf{Tempo di Lancio}: 2 Azioni\\
\textbf{Gittata}: 9 metri\\
\textbf{Componenti}: V, S\\
\textbf{Durata}: 1 minuto\\
Una mano spettrale fluttuante compare in un punto a gittata, scelto da te. La mano resta per la durata dell'incantesimo o finché non viene interrotta con un'azione. La mano svanisce se si dovesse trovare a più di 9 metri da te o se lanci nuovamente l'incantesimo.\\
Puoi usare 2 Azioni per controllare la mano. Puoi usare la mano per manipolare un oggetto, aprire una porta o un contenitore non chiusi a chiave, inserire o recuperare un oggetto da un contenitore aperto, o versare fuori i contenuti di una fiala. Puoi muovere la mano di 9 metri ogni volta che la usi.\\
La mano non può attaccare, attivare oggetti magici o trasportare più di 5 chili.

\medskip\textbf{Messaggio}\index{Trucchetto - Messaggio}\\
\textbf{Scuola}: Trasmutazione\\
\textbf{Difficoltà}: 12\\
\textbf{Tempo di Lancio}: 2 Azioni\\
\textbf{Gittata}: 36 metri\\
\textbf{Componenti}: V, S, M (un piccolo pezzo di cavo di rame)\\
\textbf{Durata}: 1 round\\
Punti il dito verso una creatura a gittata e sussurri un messaggio breve. Il bersaglio (e solo il bersaglio) ode il messaggio e può replicare con un sussurro che solo tu puoi udire.\\
Puoi lanciare questo incantesimo anche attraverso oggetti solidi, se sei familiare col bersaglio e sai che questi si trova dietro la barriera. Il silenzio magico, 30 centimetri di pietra, 2,5 centimetri di metallo normale, un sottile foglio di piombo o 1 metro di legno bloccano l'incantesimo. L'incantesimo non deve seguire una linea retta, e può liberamente aggirare gli angoli o attraversare gli spiragli.

\medskip\textbf{Metamorfosi}\index{Incantesimi - Metamorfosi}\\
\textbf{Scuola}: Trasmutazione\\
\textbf{Difficoltà}: 23\\
\textbf{Tempo di Lancio}: 2 Azioni\\
\textbf{Gittata}: 18 metri\\
\textbf{Componenti}: V, S, M (un bozzolo di bruco)\\
\textbf{Durata}: 1 ora \\
Questo incantesimo trasforma una creatura a gittata, che puoi vedere, in una nuova forma. Una creatura non consenziente deve superare un Tiro Salvezza su Volontà per evitare l'effetto. I mutaforma superano automaticamente il Tiro Salvezza. L'incantesimo non ha effetto su di un bersaglio con 0 punti ferita. \\
La trasformazione permane per la durata dell'incantesimo o finché il bersaglio non scende a 0 punti ferita o muore. La nuova forma può essere quella di qualsiasi bestia il cui grado di sfida sia uguale o più basso di quello del bersaglio (o del livello del bersaglio, se questi non ha un grado di sfida). Le statistiche di gioco del bersaglio, compresi i punteggi delle caratteristiche mentali, vengono rimpiazzate dalle statistiche della bestia scelta. Egli mantiene però i suoi Tratti e personalità.\\
Il bersaglio mantiene i medesimi punti ferita e ne recupera 1d12 punti ferita nella sua nuova forma. Quando ritorna alla sua forma normale, la creatura ritorna al numero di punti ferita che aveva prima di trasformarsi. Se però si ritrasforma perché ridotto a 0 punti ferita, qualsiasi danno in eccesso si ripercuote sulla sua normale forma. Purché il danno in eccesso non riduca la forma normale della creatura a 0 punti ferita, ella non cade priva di sensi.\\
La creatura è limitata nelle azioni che può svolgere dalla natura della sua nuova forma, e non può dialogare, lanciare incantesimi, o effettuare qualsiasi altra azione che richieda mani o di parlare. L'equipaggiamento del bersaglio si fonde nella nuova forma. La creatura non può attivare, usare, impugnare o beneficiare in alcun modo del suo equipaggiamento. 

\medskip\textbf{Metamorfosi Pura}\index{Incantesimi - Metamorfosi Pura}\\
\textbf{Scuola}: Trasmutazione\\
\textbf{Difficoltà}: 36\\
\textbf{Tempo di Lancio}: 2 Azioni\\
\textbf{Gittata}: 9 metri\\
\textbf{Componenti}: V, S, M (un goccio di mercurio, un mucchietto di gomma arabica, e uno sbuffo di fumo) \\
\textbf{Durata}: 1 ora \\
Scegli una creatura od oggetto non magico a gittata e che puoi vedere. L'incantesimo non ha effetto su di un bersaglio con 0 punti ferita. Trasformi la creatura in una creatura diversa, la creatura in un oggetto, o l'oggetto in una creatura (l'oggetto non deve essere indossato né trasportato da un'altra creatura). La trasformazione permane per la durata dell'incantesimo o finché il bersaglio non scende a 0 punti ferita o muore. Se ti concentri su questo incantesimo per l'intera durata, la trasformazione diventa permanente.\\
I mutaforma ignorano questo incantesimo. Una creatura non consenziente può effettuare un Tiro Salvezza su Volontà e, se lo supera, ignora l'effetto di questo incantesimo.\\
\medskip\begin{itemize}
\item
\textit{Creatura in Creatura}. Se trasformi una creatura in un'altra specie di creatura, la nuova forma può essere quella di qualsiasi specie tu voglia, il cui grado di sfida sia pari o inferiore a quello del bersaglio (o del suo livello, se il bersaglio non ha un grado di sfida). Le statistiche di gioco del bersaglio, compresi i punteggi delle caratteristiche mentali, vengono rimpiazzate dalle statistiche della nuova forma. Egli mantiene però il suoi Tratti e personalità. Il bersaglio mantiene i medesimi punti ferita e ne recupera 1d12 punti ferita nella sua nuova forma.\\
Quando ritorna alla sua forma normale, la creatura ritorna al numero di punti ferita che aveva prima di trasformarsi. Se però si ritrasforma perché ridotta a 0 punti ferita, qualsiasi danno in eccesso si ripercuote sulla sua normale forma. Purché il danno in eccesso non riduca la forma normale della creatura a 0 punti ferita, ella non cade priva di sensi. La creatura è limitata nelle azioni che può svolgere dalla natura della sua nuova forma, e non può dialogare, lanciare incantesimi, o effettuare qualsiasi altra azione che richieda mani o di parlare, a meno che la nuova forma non sia capace di svolgere queste azioni. L'equipaggiamento del bersaglio si fonde nella nuova forma. La creatura non può attivare, usare, impugnare o beneficiare in alcun modo del suo equipaggiamento. Oggetto in Creatura. Puoi trasformare un oggetto in un qualsiasi tipo di creatura, purché la taglia della creatura non sia maggiore della taglia dell'oggetto e il grado di sfida della creatura sia 9 o meno. La creatura è amichevole verso di te e i tuoi compagni. Essa agisce durante i tuoi turni. Decidi tu quali azioni essa compirà e come si muove. Il Narratore possiede le statistiche della creatura e risolverà tutte le sue azioni e i suoi movimenti.\\
Se l'incantesimo diventa permanente, perdi il controllo della creatura. A seconda di come l'hai trattata, potrebbe restare amichevole nei tuoi confronti.\\
\item
\textit{Creatura in Oggetto}. Se trasformi una creatura in un oggetto, essa si trasforma assieme a qualsiasi cosa stia indossando o trasportando. Le statistiche della creatura diventano quelle dell'oggetto, e, dopo che l'incantesimo termina e la creatura ritorna alla sua forma normale, questa non ha più ricordi del tempo trascorso in forma di oggetto.
\end{itemize}

\medskip\textbf{Miraggio Arcano}\index{Incantesimi - Miraggio Arcano}\\
\textbf{Scuola}: Illusione\\
\textbf{Difficoltà}: 31\\
\textbf{Tempo di Lancio}: 10 minuti\\
\textbf{Gittata}: Vista\\
\textbf{Componenti}: V, S\\
\textbf{Durata}: 10 giorni\\
Fai sì che un pezzo di terreno a gittata, in un'area quadrata fino a 1,5 chilometri, appaia, risuoni e odori come qualche altro tipo di terreno. La conformazione generale del terreno rimane tuttavia la stessa. Campi aperti o una strada possono essere trasformati in un acquitrino, colline, un crepaccio o qualche altro tipo di terreno difficile o invalicabile. Un laghetto può essere trasformato in una radura erbosa, un precipizio in una lieve pendenza, un burrone cosparso di rocce in una strada ampia e liscia.\\
Allo stesso modo, puoi modificare l'aspetto delle strutture, o aggiungerne dove non ve ne sono. L'incantesimo non camuffa, occulta né aggiunge creature.\\
L'illusione comprende elementi uditivi, visivi, tattili e olfattivi, così da poter trasformare un terreno sgombro in terreno difficile (o viceversa) o impedire altrimenti il movimento nell'area. Qualsiasi pezzo di terreno illusorio (come una pietra o un bastone), che venga rimosso dall'area dell'incantesimo, svanisce immediatamente. Le creature con visione del vero possono vedere oltre l'illusione e distinguere la vera forma del terreno; tuttavia, gli altri elementi dell'illusione rimangono, così, sebbene la creatura sia consapevole della presenza dell'illusione, vi può comunque interagire fisicamente. 

\medskip\textbf{Modificare Memoria}\index{Incantesimi - Modificare Memoria}\\
\textbf{Scuola}: Ammaliamento\\
\textbf{Difficoltà}: 26\\
\textbf{Tempo di Lancio}: 2 Azioni\\
\textbf{Gittata}: 9 metri\\
\textbf{Componenti}: V, S\\
\textbf{Durata}: Concentrazione, massimo 1 minuto\\
Tenti di rimodellare i ricordi di un'altra creatura. Una creatura che puoi vedere deve effettuare un Tiro Salvezza su Volontà. Se la stai combattendo, la creatura ha +1d6 sul Tiro Salvezza. Se fallisce il Tiro Salvezza, il bersaglio diventa affascinato da te per la durata dell'incantesimo. Il bersaglio affascinato è inabile e inconsapevole dell'ambiente circostante, sebbene sia ancora in grado di udirti. Se subisce danni o diviene bersaglio di un altro incantesimo, questo incantesimo termina, e nessuno dei ricordi del bersaglio viene modificato.\\
Mentre il bersaglio resta affascinato da questo incantesimo, puoi agire sui ricordi del bersaglio in merito a un evento che abbia vissuto nelle ultime 24 ore e che non sia durato più di 10 minuti. Puoi eliminare permanentemente tutti i ricordi dell'evento, permettere al bersaglio di ricordare l'evento con perfetta chiarezza e dettagli particolareggiati, modificare il ricordo dei dettagli dell'evento, o creare il ricordo di un altro evento. Devi poter parlare al bersaglio per descrivere il modo in cui i suoi ricordi saranno colpiti, e questi deve essere in grado di comprendere il tuo linguaggio, affinché i ricordi modificati si instaurino nella sua memoria. Se l'incantesimo termina prima che tu abbia finito di descrivere i ricordi modificati, la memoria della creatura non viene alterata. Altrimenti, i ricordi modificati si instaurano al termine dell'incantesimo.\\
Una memoria modificata non influisce necessariamente sul comportamento della creatura, in particolare se i suoi ricordi contraddicono le inclinazioni naturali, i Tratti o la fede della creatura. Una memoria modificata in modo illogico, come impiantare il ricordo di quanto la creatura ami immergersi nell'acido, viene rimossa, come fosse un brutto sogno. Il Narratore può giudicare un ricordo modificato troppo insensato perché abbia alcun effetto su di una creatura. Un incantesimo rimuovi maledizione o ristorare superiore lanciato sul bersaglio ne ripristina i veri ricordi.\\
\textbf{Per ogni critico ottenuto} nella prova di magia puoi alterare i ricordi di un bersaglio riguardo un evento svoltosi fino a 7 giorni prima, 30 giorni prima, 1 anno prima o qualsiasi punto nel passato della creatura.

\medskip\textbf{Movimenti del Ragno}\index{Incantesimi - Movimenti del Ragno}\\
\textbf{Scuola}: Trasmutazione\\
\textbf{Difficoltà}: 19\\
\textbf{Tempo di Lancio}: 2 Azioni\\
\textbf{Gittata}: Contatto\\
\textbf{Componenti}: V, S, M (una goccia di bitume e un ragno)\\
\textbf{Durata}: 10 minuti \\
Lanci l'incantesimo a contatto di una creatura consenziente. Fino al termine dell'incantesimo, la creatura ottiene la capacità di spostarsi verso l'alto, il basso e lungo superfici verticali o stando a testa in giù sul soffitto, tenendo le mani libere. Il bersaglio ottiene anche velocità di scalata pari alla sua velocità di passeggio.\\
La creatura soggetta all'incantesimo si considera Distratta nel lancio di altri incantesimi.\\

\medskip\textbf{Muovere il Terreno}\index{Incantesimi - Muovere il Terreno}\\
\textbf{Scuola}: Trasmutazione\\
\textbf{Difficoltà}: 29\\
\textbf{Tempo di Lancio}: 2 Azioni\\
\textbf{Gittata}: 36 metri\\
\textbf{Componenti}: V, S, M (un badile di ferro e un piccola borsa contenente un misto di tipi di terreno - argilla, concime e sabbia)\\
\textbf{Durata}: Concentrazione, massimo 2 ore\\
Scegli un'area sul terreno a gittata, non più grande di 12 metri di lato. Per la durata, puoi rimodellare terriccio, sabbia o argilla nell'area in qualsiasi modo tu voglia. Puoi innalzare o abbassare l'altitudine dell'area, creare o riempire un fossato, erigere o abbassare un muro, o formare un pilastro. La portata di questi cambiamenti non può eccedere metà della dimensione più grossa dell'area. Così, se operi su di un quadrato di 12 metri di lato, puoi creare un pilastro alto 6 metri, innalzare o abbassare l'altitudine del terreno di 6 metri, scavare un fossato profondo 6 metri, e così via. Ci vogliono 10 minuti per completare questi mutamenti. Al termine di ogni 10minuti trascorsi a concentrarsi sull'incantesimo, puoi scegliere una nuova area di terreno su cui operare.\\
Dato che la trasformazione del terreno avviene lentamente, le creature nell'area di solito non possono restare intrappolate o ferite dal movimento del terreno. L'incantesimo non può manipolare la pietra naturale o le costruzioni in pietra. Le rocce e le strutture si muovono per adattarsi al nuovo terreno. Se il modo in cui modelli il terreno renderebbe una struttura instabile, questa potrebbe crollare. Allo stesso modo, questo incantesimo non influenza direttamente la crescita dei vegetali. La terra smossa trasporta con sé qualsiasi vegetale presente.

\medskip\textbf{Muro di Forza}\index{Incantesimi - Muro di Forza}\\
\textbf{Scuola}: Invocazione\\
\textbf{Difficoltà}: 26\\
\textbf{Tempo di Lancio}: 2 Azioni\\
\textbf{Gittata}: 36 metri\\
\textbf{Componenti}: V, S, M (un pizzico di polvere prodotta frantumando una gemma trasparente)\\
\textbf{Durata}: 10 minuti\\
Un invisibile muro di forza si forma in un punto a gittata scelto da te. Il muro appare in qualsiasi orientamento da te desiderato, come una barriera orizzontale o verticale oppure angolata. Può fluttuare nell'aria o appoggiarsi su di una superficie solida. Puoi darle la forma di una cupola semisferica o di una sfera con un raggio massimo di 3 metri, oppure darle l'aspetto di una superficie piana composta da un massimo di dieci pannelli di 3 metri per 3 metri. Ogni pannello deve essere contiguo a un altro pannello. In qualsiasi forma, il muro ha uno spessore di 75 centimetri e resta per tutta la durata dell'incantesimo. Se il muro taglia uno spazio di una creatura, quando compare, la creatura viene spinta da un lato del muro (a tua discrezione). Nulla può attraversare fisicamente il muro. È immune a tutti i danni e non può essere dissolto da dissolvi magie. Tuttavia, il muro è distrutto all'istante dall'incantesimo disintegrazione. Il muro si estende anche sul Piano Etereo, impedendo ai viaggiatori eterei di attraversarlo.

\medskip\textbf{Muro di Fuoco}\index{Incantesimi - Muro di Fuoco}\\
\textbf{Scuola}: Invocazione\\
\textbf{Difficoltà}: 23\\
\textbf{Tempo di Lancio}: 2 Azioni\\
\textbf{Gittata}: 36 metri\\
\textbf{Componenti}: V, S, M (un piccolo pezzo di fosforo)\\
\textbf{Durata}: 1 minuto\\
Crei un muro di fuoco su di una superficie solida a gittata. Puoi creare un muro lungo fino a 18 metri, alto fino a 6 metri e spesso 30 centimetri, o un muro circolare di 6 metri di diametro, 6 metri di altezza e 30 centimetri di spessore. Il muro è opaco e rimane per la durata dell'incantesimo. \\
Quando il muro appare, ogni creatura nella sua area deve effettuare un Tiro Salvezza su Riflessi. Una creatura subisce 5d8 danni da fuoco se fallisce il Tiro Salvezza, o la metà se lo supera. Un lato del muro, selezionato da te quando lanci questo incantesimo, infligge 5d8 danni da fuoco a ciascuna creatura che termini il suo round entro 3 metri da quel lato o all'interno del muro. Una creatura subisce lo stesso danno quando entra nel muro per la prima volta durante un round. L'altro lato del muro non infligge danni.\\
\textbf{Per ogni critico ottenuto} nella prova di magia il danno aumenta di 1d8.

\medskip\textbf{Muro di Ghiaccio}\index{Incantesimi - Muro di Ghiaccio}\\
\textbf{Scuola}: Invocazione\\
\textbf{Difficoltà}: 29\\
\textbf{Tempo di Lancio}: 2 Azioni\\
\textbf{Gittata}: 36 metri\\
\textbf{Componenti}: V, S, M (un piccolo pezzo di quarzo)\\
\textbf{Durata}: 10 minuti\\
Crei un muro di ghiaccio su di una superficie solida a gittata. Puoi creare una cupola semisferica o una sfera con un raggio massimo di 3 metri, o puoi creare una superficie piana composta di un massimo di dieci panelli quadrati di 3 metri di lato. Ogni pannello deve essere contiguo ad almeno un altro pannello. In ogni forma, il muro è spesso 30 centimetri e rimane per la durata dell'incantesimo. \\
Se, quando compare, il muro attraversa lo spazio di una creatura, la creatura viene spinta da una parte del muro (a tua scelta) e deve effettuare un Tiro Salvezza su Riflessi. Se fallisce il Tiro Salvezza, la creatura subisce 10d6 danni da freddo, o la metà di questi danni se lo supera.\\
Il muro è un oggetto che può essere danneggiato e sfondato. Ogni sezione di 3 metri ha CA 12 e 30 punti ferita, ed è vulnerabile al danno da fuoco. Ridurre una sezione di 3 metri a 0 punti ferita la distrugge e lascia nello spazio che era occupato dal muro una brezza di vento gelido. Una creatura che si muova attraverso questa brezza di vento gelido per la prima volta in un round, deve effettuare un Tiro Salvezza su Tempra. Se lo fallisce, la creatura subisce 5d6 danni da freddo, o la metà di questi danni se lo supera.\\
\textbf{Per ogni critico ottenuto} nella prova di magia il danno aumentano di 1d8.

\medskip\textbf{Muro di Pietra}\index{Incantesimi - Muro di Pietra}\\
\textbf{Scuola}: Invocazione\\
\textbf{Difficoltà}: 26\\
\textbf{Tempo di Lancio}: 2 Azioni\\
\textbf{Gittata}: 36 metri\\
\textbf{Componenti}: V, S, M (un piccolo blocco di granito)\\
\textbf{Durata}: 10 minuti\\
Un muro di pietra solida non magico si forma in un punto a gittata, scelto da te. Il muro è spesso 15 centimetri ed è composto da 10 pannelli di 3 per 3 metri. Ogni pannello deve essere contiguo ad almeno un altro pannello. In alternativa, puoi creare pannelli 3 x 6 metri di soli 7,5 centimetri di spessore.\\
Se, quando compare, il muro attraversa lo spazio di una creatura, la creatura viene spinta da una parte del muro (a tua scelta). Se la creatura fosse circondata da tutte le parti dal muro (o dal muro e un'altra superficie solida), la creatura può effettuare un Tiro Salvezza su Riflessi. Se lo supera, può usare la sua reazione per muoversi della sua velocità in modo da non essere più intrappolata nel muro.\\
Il muro può aver qualsiasi forma tu desideri, sebbene non possa occupare lo stesso spazio di una creatura od oggetto. Il muro può anche non essere verticale o poggiare su di un piano. Deve, tuttavia, fondersi con ed essere sostenuto da pietra già esistente. Quindi, puoi usare questo incantesimo per creare un ponte su di un baratro o creare un rampa.\\
Se crei un muro non verticale del genere, più lungo di 6 metri, devi dimezzare le dimensioni di ciascun pannello per creare dei supporti. Puoi modellare rozzamente la pietra per creare merlature, spalti e così via. Il muro è un oggetto fatto di pietra che può essere danneggiato e sfondato. Ogni pannello ha Difesa CA 15, Durezza 15 e 15 punti ferita ogni 2,5 centimetri di spessore. Ridurre un pannello a 0 punti ferita lo distrugge e potrebbe far crollare i pannelli connessi, a discrezione del Narratore. Se mantieni la concentrazione su questo incantesimo per la sua intera durata, il muro diventa permanente e non può essere dissolto. Altrimenti, il muro sparisce al termine dell'incantesimo.

\medskip\textbf{Muro Prismatico}\index{Incantesimi - Muro Prismatico}\\
\textbf{Scuola}: Abiurazione\\
\textbf{Difficoltà}: 36\\
\textbf{Tempo di Lancio}: 2 Azioni\\
\textbf{Gittata}: 18 metri\\
\textbf{Componenti}: V, S\\
\textbf{Durata}: 10 minuti\\
Un piano di luci brillanti e multicolore forma un muro verticale opaco, largo fino a 27 metri, alto 9 metri e spesso 2,5 centimetri, centrato su di un punto a gittata e che puoi vedere. In alternativa, puoi modellare il muro in una sfera, fino a 9 metri di diametro, centrata su di un punto a gittata di tua scelta. Il muro resta fisso sul posto per la durata dell'incantesimo. Se posizioni il muro in modo che attraversi lo spazio occupato da una creatura, l'incantesimo fallisce e lo slot incantesimo sono sprecati. Il muro irradia luce intensa fino a una gittata di 30 metri e luce fioca per ulteriori 30 metri. Tu e le creature indicate da te al momento del lancio dell'incantesimo potete attraversare e restare vicini al muro senza pericolo. Se un'altra creatura che può vedere il muro si muove entro 6 metri da esso o inizia lì il suo round, deve superare un Tiro Salvezza su Tempra o restare accecata per 1 minuto. Il muro consiste di sette strati, ognuno di un diverso colore. Quando una creatura cerca di immergersi o attraversare il muro, lo fa uno strato alla volta, attraverso tutti gli strati del muro. Mentre si immerge oattraversa ciascuno strato, la creatura deve superare un Tiro Salvezza su Riflessi o subire le proprietà di ciascuno strato, uno alla volta, come descritto di seguito.\\
Il muro può essere distrutto, uno strato alla volta, in ordine dal rosso al violetto, in un modo specifico per ogni strato. Una volta che uno strato è distrutto, lo sarà per la durata dell'incantesimo. Una verga di cancellazione distrugge un muro prismatico, ma un campo anti-magia non ha effetto su di esso.
\medskip
\begin{itemize}
\item
\textit{1. Rosso}. Il bersaglio subisce 10d6 danni da fuoco se fallisce il Tiro Salvezza, o la metà di questi danni se lo supera. Finché questo strato esiste, gli attacchi a distanza non magici non possono attraversare il muro. Lo strato può essere distrutto infliggendogli 25 danni da freddo.
\item
\textit{2. Arancio}. Il bersaglio subisce 10d6 danni da acido se fallisce il Tiro Salvezza, o la metà di questi danni se lo supera. Finché questo strato esiste, gli attacchi a distanza magici non possono attraversare il muro. Lo strato può essere distrutto da un forte vento. 3. Giallo. Il bersaglio subisce 10d6 danni da fulmine se fallisce il Tiro Salvezza, o la metà di questi danni se lo supera. Questo strato può essere distrutto infliggendogli 60 danni di forza.
\item
\textit{4. Verde}. Il bersaglio subisce 10d6 danni da veleno se fallisce il Tiro Salvezza, o la metà di questi danni se lo supera. Un incantesimo passapareti, o un altro incantesimo di pari Difficoltà o più alto che può aprire un portale su di una superficie solida, distrugge questo strato.
\item
\textit{5. Blu}. Il bersaglio subisce 10d6 danni da freddo se fallisce il Tiro Salvezza, o la metà di questi danni se lo supera. Lo strato può essere distrutto infliggendogli almeno 25 danni da fuoco.
\item
\textit{6. Indaco}. Se fallisce il Tiro Salvezza, il bersaglio è intralciato. Deve poi effettuare un Tiro Salvezza su Tempra all'inizio di ciascun suo round. Se supera il Tiro Salvezza tre volte,l'incantesimo termina. Se fallisce il Tiro Salvezza tre volte, viene permanentemente trasformato in pietra e diventa vittima della condizione pietrificato. I successi e i fallimenti non devono essere consecutivi; tieni traccia di entrambi finché il bersaglio non ne ha ottenuti tre dello stesso tipo. Finché questo strato esiste, non si possono lanciare incantesimi attraverso il muro. Lo strato viene distrutto dalla luce intensa emanata dall'incantesimo luce diurna o da un simile incantesimo di Difficoltà più alta.
\item
\textit{7. Violetto}. Se fallisce il Tiro Salvezza, il bersaglio è accecato. Deve poi effettuare un Tiro Salvezza su Volontà all'inizio del tuo prossimo round. Se supera il Tiro Salvezza, la cecità termina. Se fallisce il Tiro Salvezza, la creatura viene trasportata su di un altro piano di esistenza a scelta del Narratore e non è più accecata (di solito, una creatura che non è sul suo piano natio, viene esiliata su di esso, mentre le altre creature sono di solito gettate nei pianiAstrale o Etereo). Questo strato è distrutto dall'incantesimo dissolvi magie o da un incantesimo simile di pari Difficoltà o più alto che possa porre fine a incantesimi ed effetti magici.
\end{itemize}

\medskip\textbf{Muro di Spine}\index{Incantesimi - Muro di Spine}\\
\textbf{Scuola}: Evocazione\\
\textbf{Difficoltà}: 29\\
\textbf{Tempo di Lancio}: 2 Azioni\\
\textbf{Gittata}: 36 metri\\
\textbf{Componenti}: V, S, M (una manciata di spine)\\
\textbf{Durata}: massimo 10 minuti\\
Crei un muro di cespugli robusti, malleabili e impigliati, ricolmi di spine appuntite. Il muro compare a gittata su di una superficie solida e rimane per la durata dell'incantesimo. Il muro che puoi creare può essere lungo fino a 18 metri, alto fino a 3 metri, e spesso fino a 1 metro o un circolo che abbia un diametro di 6 metri e sia alto fino a 6 metri e spesso 1 metro. Il muro blocca la linea di visuale.\\
Quando il muro compare, ogni creatura nella sua area deve effettuare un Tiro Salvezza su Riflessi. Se fallisce il Tiro Salvezza, una creatura subisce 7d8 danni perforanti, o la metà di questi danni se lo supera. Una creatura può muoversi attraverso il muro, seppure in maniera lenta e dolorosa. Per ogni 1 metro che la creatura si muove attraverso il muro, deve spendere 6 metri di movimento. Inoltre, la prima volta che una creatura entra nel muro durante un round o vi termina il suo round dentro, la creatura deve effettuare un Tiro Salvezza su Riflessi. Subisce 7d8 danni taglienti se fallisce il Tiro Salvezza, o la metà di questi danni se lo supera.\\
\textbf{Per ogni critico ottenuto} nella prova di magia il danno aumenta di 1d8.

\medskip\textbf{Muro di Vento}\index{Incantesimi - Muro di Vento}\\
\textbf{Scuola}: Invocazione\\
\textbf{Difficoltà}: 21\\
\textbf{Tempo di Lancio}: 2 Azioni\\
\textbf{Gittata}: 36 metri\\
\textbf{Componenti}: V, S, M (un minuscolo ventaglio e una piuma di origini esotiche)\\
\textbf{Durata}: 1 minuto\\
Un muro di forte vento si leva dal terreno in un punto a gittata di tua scelta. Puoi creare un muro lungo fino a 15 metri, alto 4 metri e spesso 30 centimetri. Puoi modellare il muro in qualsiasi maniera desideri purché componga un percorso continuo sul terreno. Il muro rimane per la durata dell'incantesimo. Quando il muro appare, ogni creatura all'interno della sua area deve effettuare un Tiro Salvezza su Tempra. Una creatura subisce 3d8 danni da botta se fallisce il Tiro Salvezza, o la metà di questi danni se lo supera. Il forte vento tiene lontana foschia, fumo e altri gas. Le creature volanti di taglia Piccola o minore non possono attraversare il muro. I materiali leggeri trascinati nel muro volano verso l'alto. Frecce, quadrelli e altre munizioni normali vengono deviati e mancano automaticamente il bersaglio (i macigni scagliati dai giganti e dalle macchine d'assedio, e munizioni simili, ne ignorano invece gli effetti). Le creature in forma gassosa non possono attraversarlo.

\medskip\textbf{Nube Incendiaria}\index{Incantesimi - Nube Incendiaria}\\
\textbf{Scuola}: Evocazione\\
\textbf{Difficoltà}: 34\\
\textbf{Tempo di Lancio}: 2 Azioni\\
\textbf{Gittata}: 45 metri\\
\textbf{Componenti}: V, S\\
\textbf{Durata}: 1 minuto\\
Una nube di fumo turbinante attraversata da lapilli incandescenti si forma in una sfera di 6 metri di raggio centrata su di un punto a gittata. La nube si propaga intorno agli angoli ed è in penombra. Rimane per la durata dell'incantesimo o finché un vento di velocità moderata o superiore (almeno 15 chilometri all'ora) non la disperde.\\
Quando la nube appare, ogni creatura al suo interno deve effettuare un Tiro Salvezza su Riflessi. Una creatura subisce 10d8 danni da fuoco se fallisce il Tiro Salvezza, e la metà di questi danni se lo supera. Una creatura deve effettuare il Tiro Salvezza anche quando entra per la prima volta nell'area o termina lì il suo round.\\
All'inizio di ciascun tuo round, la nube si muove di 3 metri lontano da te in una direzione a tua scelta. \\

\medskip\textbf{Nube Maleodorante}\index{Incantesimi - Nube Maleodorante}\\
\textbf{Scuola}: Evocazione\\
\textbf{Difficoltà}: 21\\
\textbf{Tempo di Lancio}: 2 Azioni\\
\textbf{Gittata}: 27 metri\\
\textbf{Componenti}: V, S, M (un uovo marcio o foglie di cavolo puzzolente)\\
\textbf{Durata}: 1 ora\\
Crei, in un punto a gittata, una sfera di 6 metri di raggio composta di un gas giallo e nauseabondo. La nube si propaga dietro gli angoli e la sua area è in penombra. La nube permane nell'aria per la durata. Ogni creatura che si trovi completamente all'interno della nube all'inizio del proprio round, deve effettuare un Tiro Salvezza su Tempra contro il veleno. Se il Tiro Salvezza fallisce, la creatura spende la 2 Azioni di quel round a vomitare e barcollare. Le creature che non hanno bisogno di respirare o che sono immuni al veleno superano automaticamente il Tiro Salvezza.\\
Un vento moderato (almeno 15 chilometri all'ora) disperde la nube dopo 4 round. Un vento forte (almeno 30 chilometri all'ora) lo disperde dopo 1 round.

\medskip\textbf{Nube Mortale}\index{Incantesimi - Nube Mortale}\\
\textbf{Scuola}: Evocazione\\
\textbf{Difficoltà}: 26\\
\textbf{Tempo di Lancio}: 2 Azioni\\
\textbf{Gittata}: 36 metri\\
\textbf{Componenti}: V, S\\
\textbf{Durata}: 10 minuti \\
Crei una sfera di 6 metri di raggio formata da una nebbia velenosa giallo-verde centrata in un punto a gittata di tua scelta. La nebbia si propaga dietro gli angoli. Rimane per la durata dell'incantesimo o finché un forte vento non disperde la nebbia, terminando l'incantesimo. La sua area è in penombra. Quando una creatura entra nell'area dell'incantesimo per la prima volta in un round o inizia lì il suo round, quella creatura deve effettuare un Tiro Salvezza su Tempra. La creatura subisce 5d8 danni da veleno se fallisce il Tiro Salvezza, o la metà di questi danni se lo supera. Le creature ne sono soggette anche se trattengono il respiro o non hanno bisogno di respirare. La nebbia si allontana di 3 metri da te all'inizio di ogni tuo round, spostandosi lungo la superficie del terreno. I vapori, essendo più pesanti dell'aria, tendono a scendere verso il basso, arrivando addirittura a insinuarsi nelle aperture.\\
\textbf{Per ogni critico ottenuto} nella prova di magia il danno aumenta di 1d8.

\medskip\textbf{Nube di Nebbia}\index{Incantesimi - Nube di Nebbia}\\
\textbf{Scuola}: Evocazione\\
\textbf{Difficoltà}: 16\\
\textbf{Tempo di Lancio}: 2 Azioni\\
\textbf{Gittata}: 36 metri\\
\textbf{Componenti}: V, S\\
\textbf{Durata}: 1 ora\\
Crei una sfera di foschia del raggio di 6 metri centrata su di un punto a gittata. La sfera si propaga intorno agli angoli, e la sua area è in penombra. Rimane per la durata dell'incantesimo o finché un vento di velocità moderata o superiore (almeno 15 chilometri all'ora) non la disperde.\\
\textbf{Per ogni critico ottenuto} nella prova di magia il raggio della foschia aumenta di 6 metri.

\medskip\textbf{Occhio Arcano}\index{Incantesimi - Occhio Arcano}\\
\textbf{Scuola}: Divinazione\\
\textbf{Difficoltà}: 23\\
\textbf{Tempo di Lancio}: 2 Azioni\\
\textbf{Gittata}: 9 metri\\
\textbf{Componenti}: V, S, M (un pezzo di manto di pipistrello)\\
\textbf{Durata}: Concentrazione, massimo 1 ora\\
Crei a gittata un occhio magico e invisibile, che fluttua nell'aria per la durata dell'incantesimo.\\
Ricevi mentalmente le informazioni visive dall'occhio, che ha vista normale e scurovisione fino a 9 metri. L'occhio può guardare in tutte le direzioni. Con un'azione di movimento, puoi spostare l'occhio di 9 metri in qualsiasi direzione. Non c'è limite a quanto lontano possa spostarsi l'occhio, ma non può entrare in un altro piano di esistenza. Una barriera solida blocca il movimento dell'occhio, ma questo può attraversare un'apertura di una grandezza minima di 2,5 centimetri di diametro.

\medskip\textbf{Onda Tonante}\index{Incantesimi - Onda Tonante}\\
\textbf{Scuola}: Invocazione\\
\textbf{Difficoltà}: 16\\
\textbf{Tempo di Lancio}: 2 Azioni\\
\textbf{Gittata}: Personale (cubo di 4 metri di spigolo)\\
\textbf{Componenti}: V, S\\
\textbf{Durata}: Istantanea\\
un'onda di forza tonante si proietta da te. Ogni creatura in un cubo di 4 metri di spigolo che origina da te deve effettuare un Tiro Salvezza su Tempra. Se fallisce il Tiro Salvezza, una creatura subisce 2d8 danni da tuono e viene allontana 3 metri da te. Se supera il Tiro Salvezza, la creatura subisce la metà dei danni e non viene allontanata. Inoltre, gli oggetti non ancorati che sono totalmente all'interno dell'area vengono spinti 3 metri lontano da te dall'effetto dell'incantesimo, e l'incantesimo produce un rimbombo tonante udibile fino a 90 metri.\\
\textbf{Per ogni critico ottenuto} nella prova di magia il danno aumenta di 1d8.

\medskip\textbf{Oscurità}\index{Incantesimi - Oscurità}\\
\textbf{Scuola}: Invocazione\\
\textbf{Difficoltà}: 16\\
\textbf{Tempo di Lancio}: 2 Azioni\\
\textbf{Gittata}: 18 metri\\
\textbf{Componenti}: V, M (pelo di pipistrello e un pizzico di bitume o un pezzo di carbone)\\
\textbf{Durata}: 10 minuti\\
L'oscurità magica si propaga da un punto a gittata, scelto da te, per riempire una sfera di 4 metri di raggio per la durata dell'incantesimo. L'oscurità si propaga intorno agli angoli. Una creatura con scurovisione non può vedere in questa oscurità, e la luce non magica non può illuminarla.\\
Se il punto che hai scelto è su di un oggetto che stai trasportando o uno che non è indossato o trasportato, l'oscurità emana dall'oggetto e si muove con esso. Coprire completamente la fonte dell'oscurità con un oggetto opaco, come un vaso o un elmo, blocca l'oscurità.\\
Se qualsiasi parte dell'area di questo incantesimo si sovrappone con l'area di luce creata da un incantesimo con difficoltà 13 o più basso, l'incantesimo che ha creato la luce viene dissolto.

\medskip

\begin{enfasi}{
		Mi sparpaglio in giro per evitare incantesimi ad area (detta da un giocatore per evitare una Palla di Fuoco)
}\end{enfasi}

\medskip\textbf{Palla di Fuoco}\index{Incantesimi - Palla di Fuoco}\\
\textbf{Scuola}: Invocazione\\
\textbf{Difficoltà}: 21\\
\textbf{Tempo di Lancio}: 2 Azioni\\
\textbf{Gittata}: 45 metri\\
\textbf{Componenti}: V, S, M (una minuscola palla di guano di pipistrello e zolfo)\\
\textbf{Durata}: Istantanea\\
Un fascio di luce gialla parte dal tuo dito puntato verso un punto a gittata scelto da te, e poi esplode con un boato roboante e si trasforma in lingua di fiamme.\\
Ogni creatura in una sfera di 6 metri di raggio centrata in quel punto deve effettuare un Tiro Salvezza su Riflessi. Una creatura subisce 8d6 danni da fuoco se fallisce il Tiro Salvezza, o la metà di questi danni se lo supera.\\
Il fuoco si propaga ed occupa tutto il volume disponibile entro i 6 metri dal punto di esplosione. Il fuoco incendia gli oggetti infiammabili nell'area che non sono indossati o trasportati.\\
\textbf{Per ogni critico ottenuto} nella prova di magia il danno base aumenta di 1d6.\\
\textbf{Successo/Fallimento Critico}: In caso si fallimento critico il danno raddoppia, in caso di successo critico il danno viene ulteriormente dimezzato

\medskip\textbf{Palla di Fuoco Ritardata}\index{Incantesimi - Palla di Fuoco Ritardata}\\
\textbf{Scuola}: Invocazione\\
\textbf{Difficoltà}: 31\\
\textbf{Tempo di Lancio}: 2 Azioni\\
\textbf{Gittata}: 45 metri\\
\textbf{Componenti}: V, S, M (una grossa palla di guano di pipistrello e zolfo)\\
\textbf{Durata}: Concentrazione, 1 minuto\\
Un fascio di luce gialla parte dal tuo dito puntato, per condensarsi per la durata dell'incantesimo nella forma di una pallina luminosa in un punto a gittata, scelto da te. Quando l'incantesimo termina, o perché la tua concentrazione è spezzata o perché decidi tu di porgli fine, la pallina esplode con un boato sommesso e si trasforma in un getto di fiamme che si propaga dietro gli angoli. Ogni creatura in una sfera di 6 metri di raggio centrata in quel punto deve effettuare un Tiro Salvezza su Riflessi. Una creatura subisce danni da fuoco pari al danno totale accumulato se fallisce il Tiro Salvezza, o la metà di questi danni se lo supera. Il danno base dell'incantesimo è 12d6. Se al termine del tuo round la pallina non è ancora detonata, il danno aumenta di 1d6.\\
Se la pallina luminosa viene toccata prima che l'incantesimo abbia avuto fine, la creatura che la tocca deve effettuare un Tiro Salvezza su Riflessi. Se fallisce il Tiro Salvezza, l'incantesimo termina immediatamente, facendo eruttare fiamme dalla pallina. Se supera il Tiro Salvezza, la creatura può lanciare la pallina fino a 12 metri di distanza. Quando colpisce una creatura od oggetto solido, l'incantesimo ha fine e la pallina esplode.\\
Il fuoco danneggia gli oggetti nell'area e incendia gli oggetti infiammabili che non sono indossati o trasportati.\\
\textbf{Per ogni critico ottenuto} nella prova di magia il danno aumento di 1d6.\\
\textbf{Successo/Fallimento Critico}: In caso si fallimento critico il danno raddoppia, in caso di successo critico il danno viene ulteriormente dimezzato.

\medskip\textbf{Parlare con gli Animali}\index{Incantesimi - Parlare con gli Animali}\\
\textbf{Scuola}: Divinazione\\
\textbf{Difficoltà}: 16\\
\textbf{Tempo di Lancio}: 2 Azioni\\
\textbf{Gittata}: Personale\\
\textbf{Componenti}: V, S\\
\textbf{Durata}: 10 minuti\\
Per la durata dell'incantesimo, ottieni la capacità di comprendere e comunicare verbalmente con le bestie. Il sapere e la consapevolezza di molte bestie sono limitati dal loro intelletto ma, come minimo, le bestie possono fornirti informazioni riguardo luoghi e mostri nelle vicinanze, compresi quelli che possono percepire o hanno percepito nei giorni passati. A discrezione del Narratore potresti riuscire a convincere una bestia a farti un piccolo favore.

\medskip\textbf{Parlare con i Morti}\index{Incantesimi - Parlare con i Morti}\\
\textbf{Scuola}: Necromanzia\\
\textbf{Difficoltà}: 21\\
\textbf{Tempo di Lancio}: 2 Azioni\\
\textbf{Gittata}: 3 metri\\
\textbf{Componenti}: V, S, M (incenso acceso)\\
\textbf{Durata}: 10 minuti\\
Conferisci un'apparenza di vita e Intelligenza a un cadavere a gittata, scelto da te, permettendogli di rispondere alle domande che gli poni. Il cadavere deve avere ancora una bocca e non può essere non morto. L'incantesimo fallisce se il cadavere è già stato bersaglio di questo incantesimo negli ultimi 10 giorni. Fino al termine dell'incantesimo, puoi porre al cadavere fino a cinque domande. Il cadavere conosce solo quello che già sapeva in vita, compresi i linguaggi parlati. Le risposte sono di solito brevi, criptiche o ripetitive, e il cadavere non è sotto nessun obbligo a darti risposte veritiere se gli sei ostile o ti riconosce come suo nemico. Questo incantesimo non riporta l'anima della creatura nel corpo, ma solo lo spirito che lo muove. Di conseguenza, il cadavere non può apprendere nuove informazioni, non capisce nulla di quello che è successo da quando è morto, e non può fare valutazioni su eventi futuri.

\medskip\textbf{Parlare con le Piante}\index{Incantesimi - Parlare con le Piante}\\
\textbf{Scuola}: Trasmutazione\\
\textbf{Difficoltà}: 21\\
\textbf{Tempo di Lancio}: 2 Azioni\\
\textbf{Gittata}: Personale (raggio di 9 metri)\\
\textbf{Componenti}: V, S\\
\textbf{Durata}: 10 minuti\\
Infondi i vegetali entro 9 metri da te di capacità senziente e di limitata mobilità, dandole la capacità di comunicare con te ed eseguire dei semplici comandi. Puoi interrogare i vegetali in merito a eventi avvenuti nell'ultimo giorno nell'area dell'incantesimo, ottenendo informazioni sulle creature di passaggio, il clima e altro. Puoi anche trasformare il terreno difficile prodotto dalla crescita dei vegetali (come cespugli e fitto sottobosco) in terreno ordinario per la durata dell'incantesimo.\\
Oppure puoi trasformare del terreno normale in cui siano presenti dei vegetali in terreno difficile, che rimane per la durata dell'incantesimo facendo sì, per esempio, che rampicanti e rami rallentino gli inseguitori. \\
A discrezione del Narratore i vegetali potrebbero svolgere anche altri compiti per tuo conto. L'incantesimo non permette ai vegetali di sradicarsi e muoversi, ma possono muovere liberamente rami, steli e gambi. Se una creatura vegetale si trova nell'area, puoi comunicare con essa come se parlaste lo stesso linguaggio, ma non ottieni alcuna capacità magica per influenzarla.\\
Questo incantesimo può far sì che i vegetali creati dall'incantesimo intralciare rilascino una creatura intralciata. 

\medskip\textbf{Parola Divina}\index{Incantesimi - Parola Divina}\\
\textbf{Scuola}: Invocazione\\
\textbf{Difficoltà}: 31\\
\textbf{Tempo di Lancio}: 1 Azione Immediata\\
\textbf{Gittata}: 9 metri\\
\textbf{Componenti}: V\\
\textbf{Durata}: Istantanea\\
Pronunci una parola divina, infusa del potere del tuo Patrono. Scegli un qualsiasi numero di creature a gittata e che puoi vedere. Ogni creatura che può udirti deve effettuare un Tiro Salvezza su Volontà. Se fallisce il Tiro Salvezza, la creatura subisce un effetto in base ai suoi attuali punti ferita:
\medskip
\begin{itemize}
\item	
0 punti ferita o meno: assordata per 1 minuto
\item	
40 punti ferita o meno: assordata e accecata per 10 minuti
\item	
30 punti ferita o meno: accecata, assordata e stordita per 1 ora
\item	
20 punti ferita o meno: uccisa all'istante
\end{itemize}
\medskip
Quali che siano i suoi attuali punti ferita, un celestiale, elementale, fatato o demone che fallisca il Tiro Salvezza è obbligato a tornare al suo piano di origine (se non vi si trova già) e non può tornare sul tuo attuale piano prima che siano passate 24 ore, a meno dell'uso dell'incantesimo desiderio.

\medskip\textbf{Parola Guaritrice}\index{Incantesimi - Parola Guaritrice}\\
\textbf{Scuola}: Cura\\
\textbf{Difficoltà}: 16\\
\textbf{Tempo di Lancio}: 1 Azione Immediata\\
\textbf{Gittata}: 18 metri\\
\textbf{Componenti}: V\\
\textbf{Durata}: Istantanea\\
Una creatura a gittata che puoi vedere, scelta da te, recupera punti ferita pari a 1d4 + il tuo valore di caratteristica da incantatore. Questo incantesimo causa lo stesso ammontare di danno su un non morto.\\
\textbf{Per ogni critico ottenuto} nella prova di magia la cura aumenta di 1d4.\\
Se incantatore e creatura curata sono entrambi Seguaci dello stesso Patrono l'incantesimo cura 1d4 in piu'.\\
Se incantatore e creatura curata sono entrambi Devoti dello stesso Patrono l'incantesimo cura 2d4 in piu'.\\

\medskip\textbf{Parola Guaritrice di Massa}\index{Incantesimi - Parola Guaritrice di Massa}\\
\textbf{Scuola}: Cura\\
\textbf{Difficoltà}: 21\\
\textbf{Tempo di Lancio}: 1 Azione Immediata\\
\textbf{Gittata}: 18 metri\\
\textbf{Componenti}: V\\
\textbf{Durata}: Istantanea\\
Mentre pronunci parole di cura, fino a sei creature a gittata che puoi vedere, scelte da te, recuperano punti ferita pari a 1d4 + il tuo modificatore di caratteristica da incantatore. Questo incantesimo causa lo stesso ammontare di danno sui non morti.\\
\textbf{Per ogni critico ottenuto} nella prova di magia la cura aumenta di 1d4.\\
Se incantatore e creatura curata sono entrambi Seguaci dello stesso Patrono l'incantesimo cura 1d4 in piu'.\\
Se incantatore e creatura curata sono entrambi Devoti dello stesso Patrono l'incantesimo cura 2d4 in piu'.\\
 
\medskip\textbf{Parola del Potere Stordire}\index{Incantesimi - Parola del Potere Stordire}\\
\textbf{Scuola}: Ammaliamento\\
\textbf{Difficoltà}: 34\\
\textbf{Tempo di Lancio}: 1 Azione Immediata\\
\textbf{Gittata}: 18 metri\\
\textbf{Componenti}: V\\
\textbf{Durata}: 1 minuti\\
Pronunci una parola di potere che può travolgere la mente di una creatura a gittata e che puoi vedere, lasciandola confusa. Se il bersaglio ha 150 punti ferita o meno, è stordito. Altrimenti, l'incantesimo non ha effetto.\\

\medskip\textbf{Parola del Potere Uccidere}\index{Incantesimi - Parola del Potere Uccidere}\\
\textbf{Scuola}: Ammaliamento\\
\textbf{Difficoltà}: 36\\
\textbf{Tempo di Lancio}: 1 Azione Immediata\\
\textbf{Gittata}: 18 metri\\
\textbf{Componenti}: V\\
\textbf{Durata}: Istantanea\\
Pronunci una parola di potere che costringe a morire all'istante una creatura a gittata che puoi vedere. Se la creatura che scegli ha 100 punti ferita o meno, muore. Altrimenti, l'incantesimo non ha effetto. 

\medskip\textbf{Parola del Ritiro}\index{Incantesimi - Parola del Ritiro}\\
\textbf{Scuola}: Evocazione\\
\textbf{Difficoltà}: 29\\
\textbf{Tempo di Lancio}: 2 Azioni\\
\textbf{Gittata}: 1 metro\\
\textbf{Componenti}: V\\
\textbf{Durata}: Istantanea\\
Te e fino a cinque creature consenzienti entro 1 metro da te vi teletrasportate istantaneamente in un luogo sicuro indicato precedentemente, detto santuario. Tu e tutte le creature teletrasportate con te, riapparite nello spazio non occupato più vicino al punto indicato quando hai preparato questo santuario (vedi sotto). Se lanci questo incantesimo senza aver prima preparato un santuario, l'incantesimo non ha effetto.\\
Devi indicare un santuario, che sia dedicato o fortemente collegato al tuo Patrono. Se tenti di lanciare l'incantesimo perche' ti porti in un'area che non sia dedicata dal tuo Patrono, l'incantesimo non ha effetto.

\medskip\textbf{Passapareti}\index{Incantesimi - Passapareti}\\
\textbf{Scuola}: Trasmutazione\\
\textbf{Difficoltà}: 26\\
\textbf{Tempo di Lancio}: 2 Azioni\\
\textbf{Gittata}: 9 metri\\
\textbf{Componenti}: V, S, M (un pizzico di semi di sesamo)\\
\textbf{Durata}: 1 ora\\
Per la durata dell'incantesimo, compare un passaggio in un punto a gittata che puoi vedere, su di una superficie di legno, muro o pietra (come una parete, un soffitto o un pavimento) scelta da te. Scegli le dimensioni dell'apertura: al massimo larga 1 metro, alta 2,4 metri e profonda 6 metri. Il passaggio non crea instabilità nella struttura che lo circonda.\\
Quando l'apertura sparisce, qualsiasi creatura od oggetto ancora nel passaggio creato dall'incantesimo viene espulso al sicuro nello spazio non occupato più vicino alla superficie su cui hai lanciato l'incantesimo.

\medskip\textbf{Passare Senza Tracce}\index{Incantesimi - Passare Senza Tracce}\\
\textbf{Scuola}: Abiurazione\\
\textbf{Difficoltà}: 19\\
\textbf{Tempo di Lancio}: 2 Azioni\\
\textbf{Gittata}: Personale\\
\textbf{Componenti}: V, S, M (ceneri di una foglia di vischio bruciata e un ramoscello di abete rosso)\\
\textbf{Durata}: Concentrazione, 1 ora
Per la durata dell'incantesimo le tue tracce non possono essere seguite eccetto che da mezzi magici. La creatura che riceve questo bonus non lascia tracce né altri segni del suo passaggio.
\textbf{Per ogni critico ottenuto} nella prova di magia puoi includere un altra creature nei benefici dell'incantesimo.\\


\medskip\textbf{Passo Velato}\index{Incantesimi - Passo Velato}\\
\textbf{Scuola}: Evocazione\\
\textbf{Difficoltà}: 19\\
\textbf{Tempo di Lancio}: 1 Azione Immediata\\
\textbf{Gittata}: Personale\\
\textbf{Componenti}: V\\
\textbf{Durata}: Istantanea\\
Avvolto rapidamente da una foschia argentata, ti teletrasporti di massimo 9 metri in uno spazio non occupato che puoi vedere.

\medskip\textbf{Passo Veloce}\index{Incantesimi - Passo Veloce}\\
\textbf{Scuola}: Trasmutazione\\
\textbf{Difficoltà}: 16\\
\textbf{Tempo di Lancio}: 2 Azioni\\
\textbf{Gittata}: Contatto\\
\textbf{Componenti}: V, S, M (un pizzico di terra)\\
\textbf{Durata}: 1 ora\\
La velocità di una creatura aumenta di 3 metri fino al termine dell'incantesimo. \\
\textbf{Per ogni critico ottenuto} nella prova di magia puoi prendere come bersaglio un'ulteriore creatura.

\medskip\textbf{Paura}\index{Incantesimi - Paura}\\
\textbf{Scuola}: Illusione\\
\textbf{Difficoltà}: 21\\
\textbf{Tempo di Lancio}: 2 Azioni\\
\textbf{Gittata}: Personale (cono di 9 metri)\\
\textbf{Componenti}: V, S, M (una piuma bianca o il cuore di una gallina)\\
\textbf{Durata}: 1 minuto\\
Proietti un'immagine illusoria delle peggiori paure di una creatura. Ogni creatura in un cono di 9 metri deve superare un Tiro Salvezza su Volontà o far cadere qualsiasi cosa stia impugnando e restare spaventata per la durata dell'incantesimo.\\
Mentre è spaventata da questo incantesimo, una creatura deve, durante ciascun suo round, effettuare l'azione Scattare e muoversi lontano da te tramite il tragitto più sicuro, a meno che non abbia spazio per muoversi. Se la creatura termina il suo round in un posto dove non ha linea di visuale su di te, può effettuare un Tiro Salvezza su Volontà. Se lo supera, l'incantesimo, per quella creatura, ha termine. 

\medskip\textbf{Pelle di Corteccia}\index{Incantesimi - Pelle di Corteccia}\\
\textbf{Scuola}: Trasmutazione\\
\textbf{Difficoltà}: 19\\
\textbf{Tempo di Lancio}: 2 Azioni\\
\textbf{Gittata}: Contatto\\
\textbf{Componenti}: V, S, M (una manciata di corteccia di quercia)\\
\textbf{Durata}: 1 ora\\
La pelle del bersaglio con cui sei in contatto al momento del lancio dell'incantesimo diventa ruvida e dall'aspetto simile alla corteccia fino al termine dell'incantesimo, e la Difesa del bersaglio non può essere inferiore a 16, quale che sia l'armatura che stia indossando.

\medskip\textbf{Pelle di Pietra}\index{Incantesimi - Pelle di Pietra}\\
\textbf{Scuola}: Abiurazione\\
\textbf{Difficoltà}: 23\\
\textbf{Tempo di Lancio}: 2 Azioni\\
\textbf{Gittata}: Contatto\\
\textbf{Componenti}: V, S, M (polvere di diamante del valore di 100 mo, che l'incantesimo consuma)\\
\textbf{Durata}: 1 ora\\
Lanci l'incantesimo a contatto di una creatura consenziente, la cui pelle si tramuta in una sostanza dura come la pietra. Tira 1d4+metà del valore di CA, la somma risultante e' le volte che un attacco con arma di mischia o distanza viene annullato (indipendentemente che di colpisca o meno).\\

\medskip\textbf{Piaga degli Insetti}\index{Incantesimi - Piaga degli Insetti}\\
\textbf{Scuola}: Evocazione\\
\textbf{Difficoltà}: 26\\
\textbf{Tempo di Lancio}: 2 Azioni\\
\textbf{Gittata}: 90 metri\\
\textbf{Componenti}: V, S, M (qualche granello di zucchero, qualche chicco di grano, e una passata di lardo)\\
\textbf{Durata}: 10 minuti\\
Uno sciame di locuste affamate riempie una sfera di 6 metri di raggio centrata in un punto a gittata scelto da te. La sfera si propaga intorno agli angoli. La sfera rimane per la durata dell'incantesimo, e la sua area è in penombra. L'area della sfera è terreno difficile.\\
Quando l'area appare, ogni creatura al suo interno deve effettuare un Tiro Salvezza su Tempra. Una creatura subisce 4d10 danni se fallisce il Tiro Salvezza, o la metà di questi danni se lo supera. Una creatura deve effettuare questo Tiro Salvezza anche quando entra per la prima volta nell'area dell'incantesimo durante un round o se termina il proprio round al suo interno.\\
\textbf{Per ogni critico ottenuto} nella prova di magia il danno aumenta di 1d8.

\medskip\textbf{Porta Dimensionale}\index{Incantesimi - Porta Dimensionale}\\
\textbf{Scuola}: Evocazione\\
\textbf{Difficoltà}: 23\\
\textbf{Tempo di Lancio}: 2 Azioni\\
\textbf{Gittata}: 150 metri\\
\textbf{Componenti}: V\\
\textbf{Durata}: Istantanea\\
Ti teletrasporti dalla tua attuale posizione in qualsiasi altro posto a gittata. Arrivi esattamente nel posto desiderato. Può essere un luogo che puoi vedere, uno che puoi visualizzare, o uno che puoi descrivere indicando distanza e direzione, come "30 metri verso il basso" o "90 metri in alto a nordovest con un angolo di 45 gradi."\\
Puoi portare con te oggetti il cui peso non ecceda la tua capacità di ingombro. Puoi portare con te anche una creatura consenziente della tua taglia o più piccola con equipaggiamento fino al limite della sua capacità di carico. La creatura deve essere entro 1 metro da te quando lanci questo incantesimo. \\
Se dovessi arrivare in un posto già occupato da un oggetto o creatura, tu e la creatura che viaggia con te subite ciascuno 4d6 danni da forza, e l'incantesimo non riesce a teletrasportarvi.


\medskip\textbf{Preghiera di Guarigione}\index{Incantesimi - Preghiera di Guarigione}\\
\textbf{Scuola}: Cura\\
\textbf{Difficoltà}: 19\\
\textbf{Tempo di Lancio}: 10 minuti\\
\textbf{Gittata}: 9 metri\\
\textbf{Componenti}: V\\
\textbf{Durata}: Istantanea\\
Fino a sei creature a gittata che puoi vedere, scelte da te, recuperano ciascuna punti ferita pari a 2d8 + il tuo modificatore di caratteristica da incantatore. Questo incantesimo causa lo stesso ammontare di danno sui non morti.\\
\textbf{Per ogni critico ottenuto} nella prova di magia la cura aumento di aumenta di 1d8.

\medskip\textbf{Presagio}\index{Incantesimi - Presagio}\\
\textbf{Scuola}: Divinazione\\
\textbf{Difficoltà}: 19\\
\textbf{Tempo di Lancio}: 1 minuto\\
\textbf{Gittata}: Personale\\
\textbf{Componenti}: V, S, M (dei bastoncini, ossa o simili oggetti marchiati appositamente e del valore di almeno 25 mo)\\
\textbf{Durata}: Istantanea\\
Gettando bastoncini intarsiati con gemme, facendo rotolare ossa di drago, impilando carte elaborate o impiegando qualche altro strumento di divinazione, ricevi un presagio da un'entità ultraterrena riguardo il risultato di uno specifico corso di azione che intendi intraprendere nei prossimi 30 minuti. Il Narratore sceglie tra i seguenti presagi:
\medskip
\begin{itemize}
\item 
Prosperità, per i risultati positivi
\item 
Calamità, per i risultati negativi
\item 
Prosperità e calamità, per i risultati sia positivi che negativi
\item 
Nulla, per i risultati che non sono né particolarmente positivi né negativi
\end{itemize}
\medskip
L'incantesimo non tiene conto di ogni possibile circostanza che possa modificare il risultato, come il lancio di ulteriori incantesimi o la perdita o l'arrivo di un alleato. Se lanci l'incantesimo due o più volte prima che sia sorto il nuovo sole, c'è una probabilità cumulativa del 25\% che per ogni lancio dopo il primo tu ottenga una lettura erronea. Il Narratore effettua questo tiro in segreto.

\medskip\textbf{Prestidigitazione}\index{Trucchetto - Prestidigitazione}\\
\textbf{Scuola}: Universale\\
\textbf{Difficoltà}: 12\\
\textbf{Tempo di Lancio}: 2 Azioni\\
\textbf{Gittata}: 3 metri\\
\textbf{Componenti}: V, S\\
\textbf{Durata}: Massimo 1 ora\\
Questo incantesimo è un trucco magico minore che gli incantatori novizi impiegano per fare pratica. Crei a gittata uno dei seguenti effetti magici:
\medskip
\begin{itemize}
\item
Crei un effetto sensoriale innocuo e istantaneo come una pioggia di scintille, un soffio di vento, una debole nota musicale o uno strano odore.
\item
Illumini o spegni istantaneamente una candela, una torcia o piccolo fuoco da campo.
\item
Ripulisci o insozzi istantaneamente un oggetto non più grosso di 0,03 metri cubi.
\item
Raffreddi, riscaldi o insapori per 1 ora fino a 0,03 metri cubi di materiale non vivente.
\item
Fai comparire per 1 ora un colore, un piccolo segno o un simbolo su di un oggetto o una superficie.
\item
Crei un ninnolo non magico o un'immagine illusoria che entri nella tua mano e che resta fino al termine del tuo prossimo round.
\end{itemize}
\medskip
Se lanci questo incantesimo più volte, puoi tenere attivi fino a tre effetti non istantanei alla volta, e puoi interrompere uno di questi effetti con un'azione.

\medskip\textbf{Previsione}\index{Incantesimi - Previsione}\\
\textbf{Scuola}: Divinazione\\
\textbf{Difficoltà}: 36\\
\textbf{Tempo di Lancio}: 1 minuto\\
\textbf{Gittata}: Contatto\\
\textbf{Componenti}: V, S, M (una piuma di colibrì)\\
\textbf{Durata}: 8 ore\\
Lanci l'incantesimo a contatto di una creatura consenziente per conferirle una limitata capacità di vedere nell'immediato futuro. Per la durata, il bersaglio non può essere sorpreso e ha +1d6 sui Tiri per Colpire, prove di caratteristica e Tiri Salvezza. Inoltre, sempre per la durata, le altre creature hanno -1d6 sui Tiri per Colpire contro il bersaglio. L'incantesimo ha immediatamente termine se lo lanci di nuovo prima che la sua durata abbia fine.

\medskip\textbf{Produrre Fiamma}\index{Trucchetto - Produrre Fiamma}\\
\textbf{Scuola}: Evocazione\\
\textbf{Difficoltà}: 12\\
\textbf{Tempo di Lancio}: 2 Azioni\\
\textbf{Gittata}: Personale\\
\textbf{Componenti}: V, S\\
\textbf{Durata}: 10 minuti\\
Una fiammella compare nella tua mano. La fiammella resta lì per la durata dell'incantesimo e non danneggia né te né il tuo equipaggiamento. La fiamma produce luce intensa nel raggio di 3 metri e luce fioca per ulteriori 3 metri. L'incantesimo termina se lo interrompi con un'azione o se lo lanci di nuovo.\\
Puoi usare la fiamma anche per attaccare, sebbene farlo ponga termine all'incantesimo. Quando lanci questo incantesimo, o con un'azione in un round successivo, puoi scagliare la fiamma a una creatura entro 9 metri da te. Effettua un attacco a distanza con incantesimo. Se colpisci, il bersaglio subisce 1d8 danni da fuoco.\\
Il danno dell'incantesimo aumenta di 1d8 quando raggiungi CM 5, CM 11 e CM 17.

\medskip\textbf{Proibizione}\index{Incantesimi - Proibizione}\\
\textbf{Scuola}: Abiurazione\\
\textbf{Difficoltà}: 29\\
\textbf{Tempo di Lancio}: 10 minuti\\
\textbf{Gittata}: Contatto\\
\textbf{Componenti}: V, S, M (uno spruzzo di Acqua Benedetta, incensi rari, e un rubino in polvere del valore di 1.000 mo)\\
\textbf{Durata}: 1 giorno\\
Crei una interdizione al viaggio magico che protegge fino a 4.000 metri quadri di pavimento, fino a un'altezza di 9 metri dal suolo. Per la durata dell'incantesimo, le creature non possono teletrasportarsi nell'area o usare passaggi, come quello creato dall'incantesimo portale, per entrare nell'area. L'incantesimo protegge l'area dal viaggio planare, e quindi impedisce alle creature di accedere all'area tramite il Piano Astrale, il Piano Etereo, le Lande Fatate o il Mondo delle Ombre, o l'incantesimo spostamento planare.\\
Inoltre, l'incantesimo danneggia i tipi di creatura scelti da te durante il lancio. Scegli uno o più dei seguenti: celestiali, elementali, fatati, demoni e non morti. Quando una creatura selezionata entra nell'area dell'incantesimo per la prima volta in un round o inizia qui il suo round, la creatura subisce 5d10 danni da Luce o da Vuoto (a tua scelta, quando lanci l'incantesimo). \\
Quando lanci questo incantesimo, puoi stabilire una parola d'ordine. Una creatura che pronuncia la parola d'ordine mentre entra nell'area dell'incantesimo, non subisce danni da esso.\\
L'area dell'incantesimo non può sovrapporsi all'area di un altro incantesimo proibizione. Se esegui proibizione ogni giorno per 30 giorni nello stesso posto, l'incantesimo durerà finché non viene dissolto, e le componenti materiali saranno consumate durante l'ultimo lancio.

\medskip\textbf{Proiezione Astrale}\index{Incantesimi - Proiezione Astrale}\\
\textbf{Scuola}: Necromanzia\\
\textbf{Difficoltà}: 36\\
\textbf{Tempo di Lancio}: 2 Azioni\\
\textbf{Gittata}: 3 metri\\
\textbf{Componenti}: V, S, M (per ogni creatura soggetta a questo incantesimo, devi fornire un giacinto del valore di almeno 1.000 mo e un lingotto d'argento elegantemente scolpito del valore di almeno 100 mo, tutti i quali sono consumati dall'incantesimo)\\
\textbf{Durata}: Speciale\\
Tu e fino ad altre otto creature consenzienti a gittata proiettate i vostri corpi astrali nel Piano Astrale (l'incantesimo fallisce e il lancio è sprecato qualora vi trovaste già in quel piano). Il corpo materiale che ti lasci alle spalle è privo di sensi e in uno stato di animazione sospesa; non ha bisogno di cibo né di acqua e non invecchia.\\
Il tuo corpo astrale assomiglia in tutto e per tutto alla tua forma mortale, replicando le tue statistiche di gioco e i tuoi oggetti. La principale differenza è l'aggiunta di un cordone argenteo che si estende dalle scapole per 30 centimetri dietro di te, divenendo poi invisibile. Il cordone è la tua connessione al tuo corpo materiale. Finché questa connessione resterà intatta, potrai tornare a casa. Se il cordone viene tagliato (un avvenimento che accade solo quando uno specifico effetto lo indica) la tua anima e corpo vengono separati, uccidendoti all'istante.\\
La tua forma astrale può viaggiare liberamente per il Piano Astrale e attraversare i portali che da lì conducono ad altri piani. Se entri in un nuovo piano o ritorni al piano su cui eri al momento del lancio dell'incantesimo, il tuo corpo e i tuoi oggetti vengono trasportati lungo il cordone argenteo, permettendoti di rientrare nel tuo corpo al momento dell'ingresso nel nuovo piano. La tua forma astrale è una incarnazione separata. Qualsiasi danno o altro effetto che si applica a essa, non ha effetto sul tuo corpo fisico, né vi compare al tuo ritorno.\\
L'incantesimo ha termine per te e i tuoi compagni quando userai un'azione per interromperlo. Quando l'incantesimo termina, la creatura su cui agisce torna al proprio corpo fisico, e si risveglia. L'incantesimo potrebbe anche avere una fine anticipata per te o uno dei tuoi compagni. Un incantesimo dissolvi magie usato con successo sul corpo astrale o fisico termina l'incantesimo per quella creatura. Se il corpo originale della creatura o la sua forma astrale scende a 0 punti ferita, per quella creatura l'incantesimo ha termine. Se l'incantesimo ha termine e il cordone argenteo è intatto, il cordone trascina indietro al suo corpo la forma astrale della creatura, ponendo fine al suo stato di animazione sospesa.\\
Se vieni riportato al tuo corpo prematuramente, i tuoi compagni devono restare nella loro forma astrale e trovare per proprio conto la via di ritorno ai loro corpi, di solito scendendo a 0 punti ferita.

\medskip\textbf{Protezione dal Bene e dal Male}\index{Incantesimi - Protezione dal Bene e dal Male}\\
\textbf{Scuola}: Abiurazione\\
\textbf{Difficoltà}: 16\\
\textbf{Tempo di Lancio}: 2 Azioni\\
\textbf{Gittata}: Contatto\\
\textbf{Componenti}: V, S, M (Acqua Benedetta o argento e ferro in polvere, che l'incantesimo consuma)\\
\textbf{Durata}: 10 minuti\\
Fino al termine dell'incantesimo, una creatura consenziente in contatto con te al momento dell'esecuzione è protetta da certi tipi di creature: aberrazioni, celestiali, elementali, fatati, demoni e non morti.\\
La protezione conferisce diversi benefici. Le creature di quei tipi hanno -1d6 ai Tiri per Colpire contro il bersaglio. Il bersaglio non può essere affascinato, spaventato o posseduto da loro. Se il bersaglio è già affascinato, spaventato o posseduto da una simile creatura, il bersaglio ha +1d6 su qualsiasi nuovo Tiro Salvezza contro l'effetto in questione.\\
\textbf{Questo incantesimo non e' usabile se si usano Tratti}

\medskip\textbf{Protezione dall'Energia}\index{Incantesimi - Protezione dall'Energia}\\
\textbf{Scuola}: Abiurazione\\
\textbf{Difficoltà}: 21\\
\textbf{Tempo di Lancio}: 2 Azioni\\
\textbf{Gittata}: Contatto\\
\textbf{Componenti}: V, S\\
\textbf{Durata}: 10 minuti\\
Lanci l'incantesimo a contatto di una creatura consenziente. Per la durata dell'incantesimo, il bersaglio ha resistenza a un tipo di danno scelto da te: acido, freddo, fuoco, fulmine o tuono. Puoi sacrificare tutta la durata dell'incantesimo, terminandolo, per annullare completamente il danno subito da una fonte di energia.\\
\textbf{Per ogni critico ottenuto} nella prova di magia puoi influenzare un altra persona o raddoppiare la durata.\

\medskip\textbf{Protezione dai Veleni}\index{Incantesimi - Protezione dai Veleni}\\
\textbf{Scuola}: Abiurazione\\
\textbf{Difficoltà}: 19\\
\textbf{Tempo di Lancio}: 2 Azioni\\
\textbf{Gittata}: Contatto\\
\textbf{Componenti}: V, S\\
\textbf{Durata}: 1 ora\\
Per la durata dell'incantesimo, il bersaglio ha +1d6 ai Tiri Salvezza contro l'essere avvelenato, e ha resistenza al danno da veleno.

\medskip\textbf{Punizione Marchiante}\index{Incantesimi - Punizione Marchiante}\\
\textbf{Scuola}: Invocazione\\
\textbf{Difficoltà}: 19\\
\textbf{Tempo di Lancio}: 1 Azione Immediata\\
\textbf{Gittata}: Personale\\
\textbf{Componenti}: V\\
\textbf{Durata}: 1 minuto\\
La prossima volta che colpisci una creatura con un attacco in mischia con arma nella durata dell'incantesimo, l'arma riluce di un bagliore astrale mentre colpisci. L'attacco infligge 1d6 danni da Luce aggiuntivi al bersaglio, che diventa visibile qualora sia invisibile ed emette luce fioca in un raggio di 1 metro. Inoltre il bersaglio non può diventare invisibile fino al termine dell'incantesimo. \\
\textbf{Per ogni critico ottenuto} nella prova di magia il danno aggiuntivo aumenta di 1d6.

\medskip\textbf{Purificare Cibo e Bevande}\index{Incantesimi - Purificare Cibo e Bevande}\\
\textbf{Scuola}: Trasmutazione\\
\textbf{Difficoltà}: 16\\
\textbf{Tempo di Lancio}: 2 Azioni\\
\textbf{Gittata}: 3 metri\\
\textbf{Componenti}: V, S\\
\textbf{Durata}: Istantanea\\
Tutti i cibi e le bevande non magiche in una sfera di 1 metri di raggio, centrata in un punto a gittata di tua scelta, vengono purificati e liberati da veleni e malattie. 

\medskip\textbf{Raggio di Gelo}\index{Trucchetto - Raggio di Gelo}\\
\textbf{Scuola}: Invocazione\\
\textbf{Difficoltà}: 12\\
\textbf{Tempo di Lancio}: 2 Azioni\\
\textbf{Gittata}: 18 metri\\
\textbf{Componenti}: V, S\\
\textbf{Durata}: Istantanea\\
Un fascio gelato di luce azzurra colpisce una creatura a gittata. Effettua un attacco a distanza con incantesimo contro il bersaglio. Se colpisci, egli subisce 1d8 danni da freddo, e la sua velocità è ridotta di 3 metri fino all'inizio del tuo prossimo round. \\
Il danno dell'incantesimo aumenta di 1d8 quando raggiungi CM 5, CM 11 e CM 17.

\medskip\textbf{Raggio di Affaticamento}\index{Incantesimi - Raggio di Affaticamento}\\
\textbf{Scuola}: Necromanzia\\
\textbf{Difficoltà}: 19\\
\textbf{Tempo di Lancio}: 2 Azioni\\
\textbf{Gittata}: 18 metri\\
\textbf{Componenti}: V, S\\
\textbf{Durata}: 1 minuto\\
Un fascio nero di energia debilitante parte dal tuo dito diretto contro una creatura a gittata. Effettua un attacco a distanza con incantesimo contro il bersaglio. Se colpisci, il bersaglio infliggerà la metà dei danni con gli attacchi con arma che usano la Forza fino al termine dell'incantesimo.\\ 

\medskip\textbf{Raggio Rovente}\index{Incantesimi - Raggio Rovente}\\
\textbf{Scuola}: Invocazione\\
\textbf{Difficoltà}: 19\\
\textbf{Tempo di Lancio}: 2 Azioni\\
\textbf{Gittata}: 36 metri\\
\textbf{Componenti}: V, S\\
\textbf{Durata}: Istantanea\\
Crei tre raggi di fuoco e li proietti verso tre bersagli a gittata. Puoi proiettarli contro lo stesso bersaglio o bersagli diversi.\\
Effettua un attacco a distanza con incantesimo per ciascun raggio. Se colpisci, il bersaglio subisce 2d6 danni da fuoco.\\
\textbf{Per ogni critico ottenuto} nella prova di magia crei un raggio aggiuntivo.

\medskip\textbf{Ragnatela}\index{Incantesimi - Ragnatela}\\
\textbf{Scuola}: Evocazione\\
\textbf{Difficoltà}: 19\\
\textbf{Tempo di Lancio}: 2 Azioni\\
\textbf{Gittata}: 18 metri\\
\textbf{Componenti}: V, S, M (un pezzo di tela di ragno)\\
\textbf{Durata}: 1 ora\\
Evochi una spessa massa di tela densa e appiccicosa in un punto a gittata, scelto da te. Per la durata, la ragnatela riempie un cubo di 6 metri di spigolo da quel punto. La ragnatela è terreno difficile e rende quell'area oscurata leggermente.\\
Se la tela non è ancorate tra due masse solide (come pareti o alberi) o stesa lungo un pavimento, parete o soffitto, la ragnatela evocata crolla su se stessa, e l'incantesimo termina all'inizio del tuo prossimo round. Le tele distese su di una superficie piatta hanno una profondità di 1 metro.\\
Ogni creatura che inizia il suo round nella ragnatela o che vi entra durante il proprio round deve effettuare un Tiro Salvezza su Riflessi. Se lo fallisce, la creatura è intralciata finché rimane nella ragnatela o finché non si libera.\\
Una creatura intralciata dalle ragnatele può usare 2 Azioni per effettuare una prova di Forza contro la DC del Tiro Salvezza dell'incantesimo. Se la supera, non è più intralciata.\\
Le ragnatele sono infiammabili. Qualsiasi cubo di 1 metro di spigolo di ragnatela che venga esposto al fuoco, brucia in 1 round, infliggendo 2d4 danni da fuoco a qualsiasi creatura che inizi il suo round in mezzo al fuoco.\\

\medskip\textbf{Randello Incantato}\index{Trucchetto - Randello Incantato}\\
\textbf{Scuola}: Trasmutazione\\
\textbf{Difficoltà}: 12\\
\textbf{Tempo di Lancio}: 1 Azione Immediata\\
\textbf{Gittata}: Contatto\\
\textbf{Componenti}: V, S, M (vischio, una foglia di quadrifoglio, e una randello o bastone da combattimento)\\
\textbf{Durata}: 1 minuto\\
Il legno di un randello o bastone da combattimento che stai impugnando viene infuso del potere della natura. Per la durata dell'incantesimo, usando quell'arma puoi usare la tua caratteristica da incantatore al posto della Forza per i Tiri per Colpire e danno da mischia, e il dado di danno dell'arma diventa un d8. L'arma diventa anche magica, se già non lo è. L'incantesimo ha termine se lo lanci di nuovo o se lasci l'arma.

\medskip\textbf{Reggia Meravigliosa}\index{Incantesimi - Reggia Meravigliosa}\\
\textbf{Scuola}: Evocazione\\
\textbf{Difficoltà}: 31\\
\textbf{Tempo di Lancio}: 1 minuto\\
\textbf{Gittata}: 90 metri\\
\textbf{Componenti}: V, S, M (un portale in miniatura scolpito in avorio, un piccolo pezzo di marmo lucido, e un minuscolo cucchiaio d'argento, ciascuno di questi oggetti deve essere almeno del valore di 5 mo)\\
\textbf{Durata}: 24 ore\\
Entro la gittata, evochi un'abitazione extradimensionale che rimane per la durata dell'incantesimo. Scegli dove è posizionato il suo portone d'ingresso. Il portone d'ingresso emette una lieve luminosità ed è largo 1 metri per 3 metri di altezza. Tu e tutte le creature da te indicate quando hai lanciato l'incantesimo potete entrare nell'abitazione extradimensionale, fino a quando il portone resta aperto. Puoi aprire o chiudere il portone se ti trovi entro 9 metri da esso. Mentre è chiuso, il portone è invisibile.\\
Oltre il portone si trova un magnifico ingresso, oltre il quale si dipanano numerose stanze. L'atmosfera è pulita, fresca e accogliente. Puoi creare quanti piani desideri, ma lo spazio non può eccedere 50 cubi ognuno di 3 metri di spigolo. Il luogo è ammobiliato e decorato come preferisci. Contiene cibo sufficiente a soddisfare un banchetto di 9 portate per 100 persone. Uno staff di 100 servitori quasi trasparenti è al servizio di chiunque vi faccia ingresso. Sta a te decidere l'aspetto visivo di questi servitori e il loro abbigliamento. Essi obbediscono assolutamente ai tuoi ordini. Ogni servitore può svolgere qualsiasi compito un normale servitore umano possa svolgere, ma non possono attaccare o effettuare alcuna azione che potrebbe arrecare direttamente danno a un'altra creatura. I servitori possono quindi raccogliere oggetti, pulire, riparare, ripiegare vestiti, accendere fuochi, servire cibi, versare vini e così via. I servitori possono recarsi in qualsiasi punto della dimora, ma non possono uscirne. I mobili e gli altri oggetti creati da questo incantesimo diventano fumo quando vengono portati fuori dalla dimora. Quando l'incantesimo termina, qualsiasi creatura all'interno dello spazio extradimensionale viene espulsa nello spazio aperto più vicino all'uscita.\\
\textbf{Nota}: l'incantesimo lanciato per un anno tutti i giorni sempre nello stesso luogo diventa permanente.

\medskip\textbf{Regressione Mentale}\index{Incantesimi - Regressione Mentale}\\
\textbf{Scuola}: Ammaliamento\\
\textbf{Difficoltà}: 34\\
\textbf{Tempo di Lancio}: 2 Azioni\\
\textbf{Gittata}: 45 metri\\
\textbf{Componenti}: V, S, M (una manciata di sfere di argilla, cristallo, vetro o minerali)\\
\textbf{Durata}: Istantanea\\
Assalti la mente di una creatura a gittata e che puoi vedere, cercando di frammentarne l'intelletto e la personalità. Il bersaglio subisce 4d6 danni e deve effettuare un Tiro Salvezza su Volontà. Se fallisce il Tiro Salvezza, i punteggi di Intelligenza e Carisma della creatura scendono a -4. La creatura non può lanciare incantesimi, attivare oggetti magici, comprendere linguaggi, o comunicare in alcun modo comprensibile. La creatura può, tuttavia, identificare i suoi amici, seguirli e anche proteggerli. Dopo 30 giorni, la creatura può ripetere il Tiro Salvezza contro l'incantesimo. Se lo supera, l'incantesimo ha termine se fallisce l'effetto e' permanete.\\ 
l'incantesimo può essere terminato entro i 30 giorni da ristorare superiore, guarigione o desiderio.

\medskip\textbf{Reincarnazione}\index{Incantesimi - Reincarnazione}\\
\textbf{Scuola}: Trasmutazione\\
\textbf{Difficoltà}: 26\\
\textbf{Tempo di Lancio}: 1 ora\\
\textbf{Gittata}: Contatto\\
\textbf{Componenti}: V, S, M (oli e unguenti rari del valore di almeno 1.000 mo, che l'incantesimo consuma)\\
\textbf{Durata}: Istantanea\\
Entri a contatto con un umanoide morto o un frammento di umanoide morto. Purché la creatura non sia morta da più di 10 giorni, l'incantesimo gli forma un nuovo corpo adulto e poi ne richiama l'anima affinché entri nel corpo. Se l'anima del bersaglio non è libera o consenziente a farlo, l'incantesimo fallisce.\\
La magia modella un nuovo corpo, che probabilmente provocherà un cambio di razza alla creatura. Il Narratore tira un d10 e consulta la seguente tabella per determinare quale forma assuma la creatura una volta riportata in vita, oppure sarà Il Narratore a scegliere la forma.\\

\medskip
\begin{tabular}{ll}
\textbf{d100} &\textbf{Razza}\\
\toprule
0 & Lupo/Aquila/Volpe/Lince (tirate 1d4)\\
1&Nano\\
2&Elfo\\
3&Mezzelfo\\
4&Mezzorco\\
5&Cinghiale/Tasso/Cane/Ratto (tirate 1d4)\\
6&Nibali\\
7&Diversi\\
8&Orso/Gufo/Procione/Gatto (tirate 1d4)\\
9&Umano\\
10&Stessa razza precedente\\
\end{tabular}

La creatura reincarnata ricorda la sua vita e le sue esperienze passate. Mantiene le capacità che aveva nella sua forma originale se e' in grado di applicarle.\\
\textbf{Questo incantesimo non non e' disponibile se non ai Devoti e Seguaci di Shayalia od Efrem}\\
\textit{Nota}: un Devoto o Seguace di Shayalia od Efrem reincanerà la creatura sempre in un animale, pero' potendo scegliere il tipo.\\
Non e' possibile reincarnarsi in uno gnomo se non si era prima uno gnomo.

\medskip\textbf{Resistenza}\index{Trucchetto - Resistenza}\\
\textbf{Scuola}: Abiurazione\\
\textbf{Difficoltà}: 12\\
\textbf{Tempo di Lancio}: 2 Azioni\\
\textbf{Gittata}: Contatto\\
\textbf{Componenti}: V, S, M (un mantello in miniatura)\\
\textbf{Durata}: Concentrazione, 1 minuto\\
Lanci l'incantesimo a contatto con una creatura consenziente. Una volta prima del termine dell'incantesimo, il bersaglio può tirare un d4 e sommare il risultato ottenuto a un Tiro Salvezza a sua scelta. Può tirare il dado prima o dopo aver effettuato il Tiro Salvezza. Poi l'incantesimo termina.

\medskip\textbf{Respirare Sott'Acqua}\index{Incantesimi - Respirare Sott'Acqua}\\
\textbf{Scuola}: Trasmutazione\\
\textbf{Difficoltà}: 21\\
\textbf{Tempo di Lancio}: 2 Azioni\\
\textbf{Gittata}: 9 metri\\
\textbf{Componenti}: V, S, M (una cannuccia o una pagliuzza)\\
\textbf{Durata}: 24 ore\\
Questo incantesimo consente a un massimo di dieci creature consenzienti a gittata e che puoi vedere, di respirare sott'acqua fino al termine dell'incantesimo. Le creature soggette mantengono anche il loro normale metodo di respirazione.\\
\textbf{Per ogni critico ottenuto} nella prova di magia puoi scegliere una creatura aggiuntiva.

\medskip\textbf{Resurrezione}\index{Incantesimi - Resurrezione}\\
\textbf{Scuola}: Necromanzia\\
\textbf{Difficoltà}: 31\\
\textbf{Tempo di Lancio}: 1 ora\\
\textbf{Gittata}: Contatto\\
\textbf{Componenti}: V, S, M (un diamante del valore di almeno 1.000 mo, che l'incantesimo consuma)\\
\textbf{Durata}: Istantanea\\
Lanci l'incantesimo a contatto di una creatura morta da non più di un secolo, che non è morta di vecchiaia e che non sia non morta. Se la sua anima è libera e consenziente, il bersaglio ritornerà in vita con tutti i suoi punti ferita.\\
Questo incantesimo neutralizza tutti i veleni e cura le normali malattie che affliggevano la creatura quando è morta. Tuttavia non rimuove malattie magiche, maledizioni e simili; se questi effetti non sono rimossi prima del lancio dell'incantesimo, affliggeranno il bersaglio al suo ritorno in vita.\\
Questo incantesimo chiude tutte le ferite mortali e ripristina qualsiasi parte del corpo mancante. Tornare dalla morte è un'ordalia. Il bersaglio subisce una penalità di -4 a tutti i Tiri per Colpire, Tiri Salvezza e prove di caratteristica. Ogni volta che il bersaglio termina una notte di riposo la penalità viene ridotta di 1 finché non scompare.\\
Lanciare questo incantesimo per riportare in vita una creatura che è morta da un anno o più ti sfianca. Fino al termine di una notte di riposo, non potrai più lanciare incantesimi e avrai -1d6 su tutti i Tiri per Colpire, prove di caratteristica e Tiri Salvezza.\\
\textbf{Questo incantesimo non dovrebbe essere disponibile. Solo un Patrono puo' riportare in vita.}

\medskip\textbf{Resurrezione Pura}\index{Incantesimi - Resurrezione Pura}\\
\textbf{Scuola}: Trasmutazione\\
\textbf{Difficoltà}: 36\\
\textbf{Tempo di Lancio}: 1 ora\\
\textbf{Gittata}: Contatto\\
\textbf{Componenti}: V, S, M (un po' di Acqua Benedetta e diamanti del valore di 25.000 mo, che l'incantesimo consuma)\\
\textbf{Durata}: Istantanea\\
Lanci l'incantesimo a contatto di una creatura morta da non più di 200 anni e che sia morta per qualsiasi motivo ma non di vecchiaia. Se la sua anima è libera e consenziente, la creatura ritornerà in vita con tutti i suoi punti ferita. \\
Questo incantesimo chiude tutte le ferite, neutralizza qualsiasi veleno, cura tutte le malattie e rimuove qualsiasi maledizione che affliggeva la creatura quando è morta. L'incantesimo rimpiazza gli organi e gli arti danneggiati.\\
l'incantesimo può fornire anche un nuovo corpo se l'originale non esiste più, in qual caso devi pronunciare il nome della creatura. La creatura riapparirà poi in uno spazio non occupato di tua scelta, entro 3 metri da te. \\
\textbf{Questo incantesimo non dovrebbe essere disponibile. Solo un Patrono puo' riportare in vita.}

\medskip\textbf{Rianimare Morti}\index{Incantesimi - Rianimare Morti}\\
\textbf{Scuola}: Necromanzia\\
\textbf{Difficoltà}: 26\\
\textbf{Tempo di Lancio}: 1 ora
\textbf{Gittata}: Contatto
\textbf{Componenti}: V, S, M (una diamante del valore di almeno 500 mo, che l'incantesimo consuma)\\
\textbf{Durata}: Istantanea\\
Riporti in vita una creatura morta, purché questa non sia morta da più di 10 giorni. Se l'anima della creatura è sia consenziente che libera di riunirsi al corpo, la creatura torna in vita con 1 punto ferita.\\
Questo incantesimo neutralizza anche qualsiasi veleno e cura le malattie non magiche che affliggevano la creatura al momento della morte. Questo incantesimo, tuttavia, non rimuove le malattie magiche, maledizioni o simili effetti; se questi non vengono rimossi prima del lancio dell'incantesimo, riprenderanno a manifestarsi quando la creatura torna in vita. L'incantesimo non può riportare in vita una creatura non morta.\\
Questo incantesimo richiude tutte le ferite mortali, ma non ripristina le parti del corpo mancanti. Se la creatura è priva di parti del corpo o organi fondamentali per la sopravvivenza (la testa, per esempio) l'incantesimo fallisce automaticamente.\\
Tornare dalla morte è un'ordalia. Il bersaglio subisce una penalità di -4 a tutti i Tiri per Colpire, Tiri Salvezza e prove di caratteristica. Ogni volta che il bersaglio termina una notte di riposo la penalità viene ridotta di 1 finché non scompare.\\
\textbf{Questo incantesimo non dovrebbe essere disponibile. Solo un Patrono puo' riportare in vita.}

\medskip\textbf{Rigenerazione}\index{Incantesimi - Rigenerazione}\\
\textbf{Scuola}: Trasmutazione\\
\textbf{Difficoltà}: 31\\
\textbf{Tempo di Lancio}: 1 minuto\\
\textbf{Gittata}: Contatto\\
\textbf{Componenti}: V, S, M (un rosario e Acqua Benedetta)\\
\textbf{Durata}: 1 ora\\
Lanci l'incantesimo a contatto di una creatura per stimolare la sua capacità di guarigione naturale. Il bersaglio recupera 4d8 + 15 punti ferita. Per la durata dell'incantesimo, il bersaglio recupera 1 punto ferita all'inizio di ciascun suo round (6 punti ferita al minuto). Le membra recise del corpo del bersaglio (dita, gambe, code e così via), se ne ha, vengono ripristinate in 2 minuti. Se hai la parte recisa e la tieni appoggiata al moncherino, l'incantesimo fa sì che l'arto si ricucia istantaneamente col moncherino.\\
\textbf{Per ogni critico ottenuto} nella prova di magia raddoppi i punti ferita recuperati per round.

\medskip\textbf{Rimuovi Malattia}\index{Incantesimi - Rimuovi Malattia}\\
\textbf{Scuola}: Cura\\
\textbf{Difficoltà}: 23\\
\textbf{Tempo di Lancio}: 1 turno\\
\textbf{Gittata}: Contatto\\
\textbf{Componenti}: V, S\\
\textbf{Durata}: Istantanea\\
Puoi porre fine a una malattia anche magica la cui Difficolta' (se magica o DC se naturale) sia inferiore alla prova di magia ottenuta lanciando questo incantesimo.\\

\medskip\textbf{Rimuovi Maledizione}\index{Incantesimi - Rimuovi Maledizione}\\
\textbf{Scuola}: Abiurazione\\
\textbf{Difficoltà}: 21\\
\textbf{Tempo di Lancio}: 2 Azioni\\
\textbf{Gittata}: Contatto\\
\textbf{Componenti}: V, S\\
\textbf{Durata}: Istantanea\\
Puoi terminare una maledizione la cui Difficolta' sia inferiore alla tua prova di magia ottenuta con il lancio di questo incantesimo.\\
Se l'oggetto è un oggetto magico maledetto, la maledizione resta, ma l'incantesimo permette di rimuovere l'oggetto e gettarlo.

\medskip\textbf{Rimuovi Veleno}\index{Incantesimi - Rimuovi Veleno}\\
\textbf{Scuola}: Cura\\
\textbf{Difficoltà}: 21\\
\textbf{Tempo di Lancio}: 1 round\\
\textbf{Gittata}: Contatto\\
\textbf{Componenti}: V, S\\
\textbf{Durata}: Istantanea\\
L'obiettivo oggetto dell'incantesimo non e' piu' avvelenato. Puoi terminare un avvelenamento la cui Difficolta' sia inferiore alla tua prova di magia ottenuta con il lancio di questo incantesimo.

\medskip\textbf{Rinascita}\index{Incantesimi - Rinascita}\\
\textbf{Scuola}: Cura\\
\textbf{Difficoltà}: 21\\
\textbf{Tempo di Lancio}: 2 Azioni\\
\textbf{Gittata}: Contatto\\
\textbf{Componenti}: V, S, M (diamante del valore di 300 mo, che l'incantesimo consuma)\\
\textbf{Durata}: Istantanea\\
Una creatura morta nell'ultimo minuto e con cui sei in contatto, ritorna in vita con 1 punto ferita. Questo incantesimo non può riportare in vita le persone morte di vecchiaia, né può ripristinare le parti del corpo mancanti.\\
\textbf{Nota}: a discrezione del Narratore questo potrebbe essere l'unico incantesimo concesso per riportare in vita una creatura, altrimenti vale la regola che solo un Patrono puo' riportare in vita.

\medskip\textbf{Riparare}\index{Trucchetto - Riparare}\\
\textbf{Scuola}: Trasmutazione\\
\textbf{Difficoltà}: 12\\
\textbf{Tempo di Lancio}: 1 minuto\\
\textbf{Gittata}: Contatto\\
\textbf{Componenti}: V, S, M (due calamite)\\
\textbf{Durata}: Istantanea\\
Questo incantesimo ripara una singola rottura o spaccatura in un oggetto con cui sei a contatto, come una catenella spezzata, due metà di una chiave rotta, un mantello lacerato, o un otre che perde. Purché la rottura o la spaccatura non sia più grande di 30 centimetri in qualsiasi dimensione, sei in grado di ripararle, senza lasciare traccia dei danni subiti. Questo incantesimo può riparare fisicamente un oggetto magico o un costrutto, ma non è in grado di ripristinare le funzioni magiche di questi oggetti.

\medskip\textbf{Riposo Inviolato}\index{Incantesimi - Riposo Inviolato}\\
\textbf{Scuola}: Necromanzia\\
\textbf{Difficoltà}: 19\\
\textbf{Tempo di Lancio}: 2 Azioni\\
\textbf{Gittata}: Contatto\\
\textbf{Componenti}: V, S, M (un pizzico di sale e un pezzo di rame posto su ciascun occhio del cadavere, che devono restare lì per la durata)\\
\textbf{Durata}: 10 giorni\\
Entri a contatto con un cadavere o altri resti. Per la durata, il bersaglio è protetto dalla putrefazione e non può diventare non morto. \\
\textbf{Per ogni critico ottenuto} nella prova di magia raddoppi la durata fino ad un massimo di un anno.

\medskip\textbf{Risata Incontenibile}\index{Incantesimi - Risata Incontenibile}\\
\textbf{Scuola}: Ammaliamento\\
\textbf{Difficoltà}: 16\\
\textbf{Tempo di Lancio}: 2 Azioni\\
\textbf{Gittata}: 9 metri\\
\textbf{Componenti}: V, S, M (piccole torte e una piuma che viene agitata nell'aria)\\
\textbf{Durata}: 1 minuto 
Una creatura a gittata di tua scelta e che puoi vedere percepisce tutto come tremendamente ilare e divertente, scoppiando in fragorose risate finché è soggetta a questo incantesimo. Il bersaglio deve superare un Tiro Salvezza su Volontà o cadere prono, restando inabile e incapace di rialzarsi per la durata. Le creature con un punteggio di Intelligenza -2 o meno, ignorano l'effetto.\\
Al termine di ciascun suo round, e ogni volta che subisce danni, il bersaglio può effettuare un altro Tiro Salvezza su Volontà. Il bersaglio ha +1d6 al Tiro Salvezza se questo è stato provocato dai danni. Se lo supera, l'incantesimo termina.

\medskip\textbf{Riscaldare il Metallo}\index{Incantesimi - Riscaldare il Metallo}\\
\textbf{Scuola}: Trasmutazione\\
\textbf{Difficoltà}: 19\\
\textbf{Tempo di Lancio}: 2 Azioni\\
\textbf{Gittata}: 18 metri\\
\textbf{Componenti}: V, S, M (un pezzo di ferro e una fiamma)\\
\textbf{Durata}: 1 minuto\\
Scegli un manufatto di metallo, come un'arma di metallo o un'armatura di metallo media o pesante, a gittata e che puoi vedere. Fai sì che l'oggetto risplenda di rosso per il calore. Qualsiasi creatura in contatto fisico con l'oggetto subisce 2d8 danni da fuoco quando lanci questo incantesimo. Fino al termine dell'incantesimo, puoi usare 2 Azioni per infliggere di nuovo questo danno nei tuoi turni successivi.\\
Se una creatura sta impugnando o indossando l'oggetto e subisce danno da esso, la creatura deve superare un Tiro Salvezza su Tempra o gettare l'oggetto se ne è in grado. Se non getta l'oggetto, ha -1d6 ai Tiri per Colpire e le prove di caratteristica fino all'inizio del suo prossimo round.\\
\textbf{Per ogni critico ottenuto} nella prova di magia il danno aumenta di 1d8.

\medskip\textbf{Ristorare Inferiore}\index{Incantesimi - Ristorare Inferiore}\\
\textbf{Scuola}: Cura\\
\textbf{Difficoltà}: 19\\
\textbf{Tempo di Lancio}: 2 Azioni\\
\textbf{Gittata}: Contatto\\
\textbf{Componenti}: V, S\\
\textbf{Durata}: Istantanea\\
Puoi porre fine a una malattia non magica o condizione che affligge una creatura con cui sei a contatto. La condizione può essere accecato, assordato o paralizzato. Da Esausto puo' portare ad Affaticato. Se la condizione e' stata causata da una magia la prova di magia di lancio di questo incantesimo deve superare la DC/Difficoltà che ha originato l'effetto.\\
Puoi recuperare 1 punto di Caratteristica perso non permanentemente.

\medskip\textbf{Ristorare Superiore}\index{Incantesimi - Ristorare Superiore}\\
\textbf{Scuola}: Cura\\
\textbf{Difficoltà}: 26\\
\textbf{Tempo di Lancio}: 2 Azioni\\
\textbf{Gittata}: Contatto\\
\textbf{Componenti}: V, S, M (polvere di diamante del valore di almeno 100 mo, che l'incantesimo consuma)\\
\textbf{Durata}: Istantanea\\
Imbevi una creatura a contatto di energia positiva per annullare un effetto debilitante. 
Se la condizione e' stata causata da una magia la prova di magia di lancio di questo incantesimo deve superare la DC/Difficoltà che ha originato l'effetto.\\
Rimuovi la condizione di Esausto o terminare uno dei seguenti effetti che affliggono il bersaglio: 
\medskip
\begin{itemize}
\item
Un effetto che ha affascinato il bersaglio.
\item
Fai recuperare 1d2 punti ad una statistica al bersaglio. Anche punti permanenti.
\item
Un effetto che riduce i punti ferita massimi del bersaglio.
\end{itemize}

\medskip\textbf{Risveglio}\index{Incantesimi - Risveglio}\\
\textbf{Scuola}: Trasmutazione\\
\textbf{Difficoltà}: 26\\
\textbf{Tempo di Lancio}: 8 ore\\
\textbf{Gittata}: Contatto\\
\textbf{Componenti}: V, S, M (un'agata del valore di almeno 1.000 mo, che l'incantesimo consuma)\\
\textbf{Durata}: Istantanea\\
Dopo aver trascorso il tempo di lancio a disegnare tracciati magici con una gemma preziosa, entri a contatto con una bestia o vegetale Enorme o di taglia inferiore. Il bersaglio deve essere privo di punteggio di Intelligenza o avere Intelligenza -2 o meno. Il bersaglio ottiene Intelligenza 0. Il bersaglio ottiene anche la capacità di parlare un linguaggio che conosci. Se il bersaglio è un vegetale, ottiene la capacità di muovere i suoi arti, radici, liane, rampicanti e così via, e ottiene sensi simili a quelli di un umano. Il Narratore sceglierà le statistiche appropriate al tipo di vegetale risvegliato, come le statistiche per il cespuglio risvegliato o l'albero risvegliato.\\
La bestia o vegetale risvegliato è affascinato da te per 30 giorni o finché tu o i tuoi compagni non gli arrecherete danno. Quando la condizione affascinato termina, la creatura risvegliata sceglie se rimanerti amichevole, in base a come l'hai trattata mentre era affascinata.\\
\textbf{Per ogni critico ottenuto} nella prova di magia raddoppi la durata della fascinazione fino ad un massimo di 1 anno.

\medskip\textbf{Ritirata Rapida}\index{Incantesimi - Ritirata Rapida}\\
\textbf{Scuola}: Trasmutazione\\
\textbf{Difficoltà}: 16\\
\textbf{Tempo di Lancio}: 1 Azione Immediata\\
\textbf{Gittata}: Personale\\
\textbf{Componenti}: V, S\\
\textbf{Durata}: Concentrazione, 1 minuto\\
Questo incantesimo ti permette di muoverti a un'andatura incredibile. Quando lanci questo incantesimo guadagni un Azione di Movimento bonus.\\
\textbf{Per ogni critico ottenuto} nella prova di magia la durata aumenta di 1 minuto.

\medskip\textbf{Saltare}\index{Incantesimi - Saltare}\\
\textbf{Scuola}: Trasmutazione\\
\textbf{Difficoltà}: 16\\
\textbf{Tempo di Lancio}: 2 Azioni\\
\textbf{Gittata}: Contatto\\
\textbf{Componenti}: V, S, M (la zampa posteriore di una cavalletta)\\
\textbf{Durata}: 1 minuto\\
La distanza di salto della creatura con cui sei in contatto al momento del lancio è triplicata fino al termine dell'incantesimo.\\

\medskip\textbf{Santificare}\index{Incantesimi - Santificare}\\
\textbf{Scuola}: Invocazione\\
\textbf{Difficoltà}: 26\\
\textbf{Tempo di Lancio}: 24 ore\\
\textbf{Gittata}: Contatto\\
\textbf{Componenti}: V, S, M (erbe, oli e incensi del valore di almeno 1.000 mo, che l'incantesimo consuma)\\
\textbf{Durata}: Fino a che dissolto\\
Infondi l'area circostante a un punto con cui sei in contatto del potere del tuo Patrono. L'area può avere un raggio massimo di 18 metri, e l'incantesimo fallisce se include un'area già sotto l'effetto di un incantesimo santificare. L'area soggetta all'incantesimo genera i seguenti effetti.
\textit{Per prima cosa}, celestiali, elementali, fatati, demoni e non morti non possono entrare nell'area, né una simile creatura può affascinare, spaventare o possederne altre al suo interno. Qualsiasi creatura affascinata, spaventata o posseduta da una creatura simile non è più affascinata,spaventata o posseduta dal momento in cui entra in quest'area. Puoi escludere uno o più tipi di queste creature da questo effetto.\\
\textit{Seconda cosa}, puoi vincolare un effetto ulteriore all'area. Scegli l'effetto dalla lista seguente, o scegline uno presentatoti dal Narratore. Alcuni di questi effetti si applicano alle creature nell'area; puoi decidere se gli effetti si applichino a tutte le creature, le creature Devote o Seguaci di specifica Patrono, o le creature di un tipo specifico, come orchi o troll. Quando una creatura soggetta all'incantesimo entra in quest'area per la prima volta durante un round o inizia il suo round qui, deve effettuare un Tiro Salvezza su Volontà. Se lo supera, la creatura ignora l'effetto aggiuntivo finché non lascia l'area.\\
\medskip
\begin{itemize}
\item
\textit{Coraggio}. Le creature soggette non possono essere spaventate mentre restano in quest'area. Interferenza Extradimensionale. Le creature soggette non possono muoversi o viaggiare usando il teletrasporto o altri mezzi extradimensionali o interplanari.
\item
\textit{Lingue}. Le creature soggette possono comunicare con qualsiasi altra creatura nell'area, anche se non condividono un linguaggio comune. 
\item
\textit{Luce Diurna}. Luce intensa riempie l'area. L'oscurità magica creata da incantesimi di piu' bassa Difficoltà di quella usata per lanciare questo incantesimo non possono estinguere la luce.
\item
\textit{Oscurità}. L'oscurità riempie l'area. La luce normale, e anche la luce magica creata da incantesimi di Difficoltà più bassa di quella usata per lanciare questo incantesimo, non possono illuminare l'area. 
\item
\textit{Paura}. Le creature soggette sono spaventate mentre restano in quest'area.
\item
\textit{Protezione dall'Energia}. Le creature soggette ricevono resistenza a un tipo di danno a tua scelta (a eccezione dei danni da botta, perforanti o taglienti), finché restano nell'area.
\item
\textit{Riposo Inviolato}. I corpi morti seppelliti nell'area non possono essere trasformati in non morti. 
\item
\textit{Silenzio}. Nessun suono può emanare dall'interno dell'area, e nessun suono può entrarvi.
\item
\textit{Vulnerabilità all'Energia}. Le creature soggette ricevono vulnerabilità a un tipo di danno a tua scelta (a eccezione dei danni da botta, perforanti o taglienti), finché restano nell'area.
\end{itemize}

\medskip\textbf{Santuario}\index{Incantesimi - Santuario}\\
\textbf{Scuola}: Abiurazione\\
\textbf{Difficoltà}: 16\\
\textbf{Tempo di Lancio}: 1 Azione Immediata\\
\textbf{Gittata}: 9 metri\\
\textbf{Componenti}: V, S, M (un piccolo specchio d'argento)\\
\textbf{Durata}: 1 minuto\\
Proteggi una creatura a gittata dagli attacchi. Fino al termine dell'incantesimo, qualsiasi creatura che prenda come bersaglio la creatura protetta con un attacco o incantesimo dannoso deve prima effettuare un Tiro Salvezza su Volontà. Se fallisce il Tiro Salvezza, l'attaccante deve scegliere un nuovo bersaglio o perdere l'attacco o l'incantesimo. Questo incantesimo non protegge la creatura protetta dagli effetti ad area, come lo scoppio di una palla di fuoco. Se la creatura protetta effettua un attacco o lancia un incantesimo che agisce su creature nemiche, l'incantesimo termina.

\medskip\textbf{Santuario Privato}\index{Incantesimi - Santuario Privato}\\
\textbf{Scuola}: Abiurazione\\
\textbf{Difficoltà}: 23\\
\textbf{Tempo di Lancio}: 10 minuti\\
\textbf{Gittata}: 36 metri\\
\textbf{Componenti}: V, S, M (un sottile foglio di piombo, un pezzo di vetro opaco, un batuffolo di cotone o tessuto, e crisolito in polvere)\\
\textbf{Durata}: 24 ore \\
Proteggi con la magia un'area. L'area è un cubo che può essere piccolo fino a 1 metro di spigolo o grande fino a 30 metri di spigolo. L'incantesimo agisce fino al termine della durata o finché non usi un'azione per interromperlo.\\
Quando lanci l'incantesimo, decidi che tipo di protezione questo fornisce, scegliendo una o più delle seguenti proprietà:\\
\medskip
\begin{itemize}
\item
Il suono non può attraversare il perimetro dell'area protetta.
\item
Il perimetro dell'area protetta appare buio e nebbioso, impedendo di vedervi attraverso (anche
alla scurovisione).
\item
Sensori creati da incantesimi di divinazione non possono apparire all'interno dell'area protetta o attraversare la sua barriera perimetrale.
\item
Le creature nell'area non possono essere bersaglio di incantesimi di divinazione.
\item
Nulla può teletrasportarsi dentro o fuori dell'area protetta.
\item
All'interno dell'area protetta, il viaggio planare è interdetto.
\end{itemize}
Lanciare questo incantesimo sullo stesso punto ogni giorno per un anno, rende l'effetto permanente.\\
\textbf{Per ogni critico ottenuto} nella prova di magia puoi aumentare le dimensioni del cubo di 10 metri di spigolo.

\medskip\textbf{Scagliare Maledizione}\index{Incantesimi - Scagliare Maledizione}\\
\textbf{Scuola}: Necromanzia\\
\textbf{Difficoltà}: 21\\
\textbf{Tempo di Lancio}: 2 Azioni\\
\textbf{Gittata}: Contatto\\
\textbf{Componenti}: V, S\\
\textbf{Durata}: 1 minuto\\
Una creatura con cui sei a contatto deve superare un Tiro Salvezza su Volontà o restare maledetta per la durata dell'incantesimo. Quando lanci questo incantesimo, scegli la natura della maledizione tra le seguenti opzioni:
\medskip
\begin{itemize}
\item
Scegli un punteggio di caratteristica. Mentre è maledetto, il bersaglio ha -1d6 alle prove di
caratteristica e i Tiri Salvezza basati eventualmente su quel punteggio di caratteristica.
\item
Mentre è maledetto, il bersaglio ha -1d6 ai Tiri per Colpire contro di te.
\item
Mentre è maledetto, il bersaglio deve effettuare un Tiro Salvezza su Volontà all'inizio di ciascun suo round. Se lo fallisce, spreca l'azione di quel suo round senza fare nulla.
\item
Mentre il bersaglio è maledetto, i tuoi attacchi e incantesimi infliggono 1d8 danni da Vuoto aggiuntivi contro di lui.
\end{itemize}
\medskip
L'incantesimo rimuovi maledizione (vedi descrizione) termina questo effetto. A discrezione del Narratore, puoi scegliere una maledizione dall'effetto diverso, ma non dovrebbe essere comunque più potente di quelle descritte qui sopra. Il Narratore detiene il giudizio finale sull'effetto di una maledizione.\\
\textbf{Se ottieni un critico} la durata della maledizione e' un giorno. Se ottieni 3 critici la durata e' permanente.

\medskip\textbf{Scassinare}\index{Incantesimi - Scassinare}\\
\textbf{Scuola}: Trasmutazione\\
\textbf{Difficoltà}: 19\\
\textbf{Tempo di Lancio}: 2 Azioni\\
\textbf{Gittata}: 18 metri\\
\textbf{Componenti}: V\\
\textbf{Durata}: Istantanea\\
Scegli un oggetto a gittata e che puoi vedere. L'oggetto può essere una porta, scatola, delle manette, una serratura o un altro oggetto che possieda un metodo comune o magico per prevenirne l'accesso.\\
Un bersaglio che è chiuso da una serratura comune o che è bloccato o sbarrato viene aperto, sbloccato o liberato. Se l'oggetto ha più serrature, solo una di queste viene aperta.\\
Se scegli un bersaglio che è tenuto chiuso con serratura arcana, quell'incantesimo resta soppresso per 10 minuti, durante i quali il bersaglio può essere aperto come di norma. Quando lanci questo incantesimo, un sonoro bussare, udibile fino a 90 metri di distanza, emana dall'oggetto bersaglio.\\
\textbf{Per ogni critico ottenuto} nella prova di magia puoi aprire un altro lucchetto/serratura.

\medskip\textbf{Sciame di Meteore}\index{Incantesimi - Sciame di Meteore}\\
\textbf{Scuola}: Invocazione\\
\textbf{Difficoltà}: 36\\
\textbf{Tempo di Lancio}: 2 Azioni\\
\textbf{Gittata}: 1,5 chilometri\\
\textbf{Componenti}: V, S\\
\textbf{Durata}: Istantanea\\
Sfere incandescenti di fuoco si schiantano a terra in quattro punti differenti a gittata e che puoi vedere. Ogni creatura, in una sfera di 2 metri di raggio centrata sul punto scelto da te, deve effettuare un Tiro Salvezza su Riflessi. La sfera si propaga intorno agli angoli. Una creatura subisce 20d6 danni da fuoco e 20d6 danni da botta se fallisce il Tiro Salvezza, o la metà di
questi danni se lo supera. Una creatura nell'area di più di uno scoppio infuocato ne subisce gli effetti solo una volta.\\
\textbf{Successo/Fallimento Critico}: In caso si fallimento critico il danno raddoppia, in caso di successo critico il danno viene ulteriormente dimezzato\\
\textbf{Ogni 3 critici ottenuti} nella prova di magia scegli un altro punto di impatto.

\medskip\textbf{Scolpire Pietra}\index{Incantesimi - Scolpire Pietra}\\
\textbf{Scuola}: Trasmutazione\\
\textbf{Difficoltà}: 23\\
\textbf{Tempo di Lancio}: 2 Azioni\\
\textbf{Gittata}: Contatto\\
\textbf{Componenti}: V, S, M (argilla malleabile, che deve essere lavorata per ottenere una vaga forma dell'oggetto di pietra)\\
\textbf{Durata}: Istantanea\\
Scolpisci in qualsiasi forma che si presti ai tuoi scopi un oggetto di pietra di taglia Media o inferiore o una sezione di pietra non più grossa di 1 metro in qualsiasi direzione, con cui sei in contatto.\\
Così, per esempio, potresti scolpire una grossa pietra in un'arma, idolo o feretro, o creare un piccolo passaggio attraverso il muro, purché il muro sia spesso meno di 1 metro. Potresti anche modellare una porta di pietra o la sua cornice per sigillare la porta. L'oggetto che crei può avere fino a due cardini e un chiavistello, ma è impossibile creare meccanismi più complessi.

\medskip\textbf{Scopri il Percorso}\index{Incantesimi - Scopri il Percorso}\\
\textbf{Scuola}: Divinazione\\
\textbf{Difficoltà}: 29\\
\textbf{Tempo di Lancio}: 1 minuto\\
\textbf{Gittata}: Personale\\
\textbf{Componenti}: V, S, M (degli attrezzi da divinazione - dei bastoncini d'avorio, ossa, carte, denti o rune incise - del valore di almeno 100 mo e un oggetto dal luogo che desideri trovare)\\
\textbf{Durata}: 1 giorno\\
Questo incantesimo ti permette di trovare la rotta fisica più breve e diretta verso uno specifico luogo fisso con cui hai familiarità ed è sullo stesso piano di esistenza. Se indichi una destinazione su di un altro piano di esistenza, una destinazione che si muove (come una fortezza mobile) o una destinazione non specifica (come "la tana di un drago verde"), l'incantesimo fallisce.\\
Per la durata dell'incantesimo, finché sei nello stesso piano di esistenza della destinazione, saprai quanto è distante e in che direzione si trovi. Mentre sei in viaggio verso di essa, ogni volta che ti si presenterà la possibilità di scegliere tra percorsi diversi, determinerai automaticamente qual è la via più breve e la rotta più diretta (ma non necessariamente la più sicura) per raggiungere la destinazione.\\
\textbf{Per ogni critico} ottenuto nella prova di magia l'incantesimo dura 8 ore in piu'.

\medskip\textbf{Scopri Trappole}\index{Incantesimi - Scopri Trappole}\\
\textbf{Scuola}: Divinazione\\
\textbf{Difficoltà}: 19\\
\textbf{Tempo di Lancio}: 2 Azioni\\
\textbf{Gittata}: 36 metri\\
\textbf{Componenti}: V, S\\
\textbf{Durata}: 1 ora\\
Per la durata dell'incantesimo avverti la presenza di qualsiasi trappola a gittata che sia nella tua linea di visuale. Una trappola, ai fini di questo incantesimo, comprende qualsiasi cosa che sia in grado di infliggere un effetto improvviso o inaspettato che tu possa considerare dannoso o indesiderabile, e che è stato espressamente inteso come tale dal suo creatore. Di conseguenza, l'incantesimo percepirebbe un'area sotto l'incantesimo allarme, un glifo di interdizione, o una botola meccanica, ma non rivelerebbe una debolezza naturale del pavimento, un soffitto instabile o una buca nascosta.\\
La trappola viene evidenziata alla tua vista con un segnale viola.

\medskip\textbf{Scrigno Segreto}\index{Incantesimi - Scrigno Segreto}\\
\textbf{Scuola}: Evocazione\\
\textbf{Difficoltà}: 23\\
\textbf{Tempo di Lancio}: 2 Azioni\\
\textbf{Gittata}: Contatto\\
\textbf{Componenti}: V, S, M (un forziere lavorato, 1 metro x 50 cm x 50 cm, costruito con rari materiali del valore di almeno 5.000 mo, e una sua replica Minuscola fatta degli stessi materiali e del valore di almeno 50) \\
\textbf{Durata}: Istantanea\\
Nascondi un forziere e tutti i suoi contenuti sul Piano Etereo. Quando lanci questo incantesimo devi essere in contatto con il forziere e la replica in miniatura che serve da componente materiale. Il forziere può contenere fino a 0,25 metri cubi di materiale non vivente (1 x metro x 50 centimetri x 50 centimetri). Mentre il forziere rimane sul Piano Etereo, puoi usare un'azione per entrare in contatto con la replica e richiamare il forziere. Esso riapparirà in uno spazio non occupato sul terreno entro 1 metro da te. Puoi rispedire il forziere nel Piano Etereo, usando un'azione ed entrando in contatto sia col forziere che con la replica.\\
Dopo 60 giorni, c'è una percentuale cumulativa del 5\% al giorno che l'effetto dell'incantesimo abbia termine. \\
L'effetto termina se l'incantesimo viene lanciato nuovamente, se la replica del forziere viene distrutta, o se decidi di terminare l'incantesimo con un'azione. Se l'incantesimo termina e il forziere si trova sul Piano Etereo, viene irrimediabilmente perduto.

\medskip\textbf{Scritto Illusorio}\index{Incantesimi - Scritto Illusorio}\\
\textbf{Scuola}: Illusione\\
\textbf{Difficoltà}: 16\\
\textbf{Tempo di Lancio}: 1 minuto\\
\textbf{Gittata}: Contatto\\
\textbf{Componenti}: S, M (un inchiostro a base di piombo del valore di almeno 10 mo, che l'incantesimo consuma)\\
\textbf{Durata}: 10 giorni\\
Scrivi su di una pergamena, un pezzo di carta o qualche altro materiale adatto a scrivere e lo infondi di una potente illusione che permane per la durata dell'incantesimo.\\
Per te e qualsiasi creatura da te indicata al lancio dell'incantesimo, la scritta appare normale, con la tua grafia, e trasmette qualsiasi significato volevi comunicare quando hai vergato il testo. Per tutti gli altri, la scritta appare come se fosse redatta in una scrittura ignota o magica, che risulta incomprensibile. In alternativa, puoi far sì che la scritta sembri un messaggio totalmente diverso, in una grafia e linguaggio differente, sebbene debba essere un linguaggio a te conosciuto.\\
In caso l'incantesimo venisse dissolto, sia la scritta originale che l'illusione svaniscono. Una creatura con visione del vero può leggere il messaggio nascosto.

\medskip\textbf{Scrutare}\index{Incantesimi - Scrutare}\\
\textbf{Scuola}: Divinazione\\
\textbf{Difficoltà}: 26\\
\textbf{Tempo di Lancio}: 10 minuti\\
\textbf{Gittata}: Personale\\
\textbf{Componenti}: V, S, M (un focus del valore di almeno 1.000 mo, come una sfera di cristallo, un specchio d'argento o una fonte ricolma di Acqua Benedetta)\\
\textbf{Durata}: Concentrazione, massimo 10 minuti\\
Puoi vedere e udire una particolare creatura a tua scelta che si trovi sul tuo stesso piano di esistenza. Il bersaglio deve effettuare un Tiro Salvezza su Volontà, modificato da quanto bene conosci il bersaglio e la tua connessione fisica a esso. Se il bersaglio sa che stai lanciando l'incantesimo, può fallire volontariamente il Tiro Salvezza, in caso desiderasse essere osservato da
te.
\medskip
\begin{tabular}{ll}
\toprule
\textbf{Conoscenza} & \textbf{Mod. al Tiro Salvezza}\\
Ne hai sentito parlare &+5\\
Hai incontrato il bersaglio &+0\\
Conosci bene il bersaglio &-5\\
\end{tabular}

\medskip

\begin{tabular}{ll}
	\toprule
\textbf{Connessione} & \textbf{Mod. Tiro Salvezza}\\
Descrizione o immagine &-2\\
Proprietà o indumento & -4\\
Parte del corpo (capelli...)&-10\\
\end{tabular}
\medskip

Se supera il Tiro Salvezza, il bersaglio ignora gli effetti dell'incantesimo, e non potrai usare di nuovo questo incantesimo contro di lui prima che siano passate 24 ore.\\
Se il Tiro Salvezza fallisce, l'incantesimo crea un sensore invisibile entro 3 metri dal bersaglio. Tramite il sensore puoi udire e vedere come se fossi sul posto. Il sensore si muove assieme al bersaglio, rimanendo entro 3 metri da lui per la durata dell'incantesimo. Una creatura che può vedere oggetti invisibili vede il sensore come una sfera luminosa delle dimensioni all'incirca di un pugno.\\
Invece di prendere come bersaglio una creatura, puoi scegliere come bersaglio dell'incantesimo un luogo che hai già visto in passato. Quando scegli questa opzione, il sensore compare in quel luogo ma non si muove. 

\medskip\textbf{Scudo}\index{Incantesimi - Scudo}\\
\textbf{Scuola}: Abiurazione\\
\textbf{Difficoltà}: 16\\
\textbf{Tempo di Lancio}: 1 reazione, che effettui quando sei colpito da un attacco o bersaglio dell'incantesimo dardo incantato\\
\textbf{Gittata}: Personale\\
\textbf{Componenti}: V, S\\
\textbf{Durata}: 1 round\\
Compare una barriera di forza magica invisibile a proteggerti. Fino all'inizio del tuo prossimo round hai un bonus di +5 alla Difesa compreso l'attacco innescante, e non subisci danni da dardo incantato.

\medskip\textbf{Scudo della Fede}\index{Incantesimi - Scudo della Fede}\\
\textbf{Scuola}: Abiurazione\\
\textbf{Difficoltà}: 16\\
\textbf{Tempo di Lancio}: 1 Azione Immediata\\
\textbf{Gittata}: 18 metri\\
\textbf{Componenti}: V, S, M (una piccola pergamena con su scritto un frammento di testo sacro)\\
\textbf{Durata}: 10 minuti\\
Compare un campo scintillante che circonda una creatura a gittata, scelta da te, conferendole un bonus di +2 alla Difesa per la durata dell'incantesimo.

\medskip\textbf{Scudo di Fuoco}\index{Incantesimi - Scudo di Fuoco}\\
\textbf{Scuola}: Invocazione\\
\textbf{Difficoltà}: 23\\
\textbf{Tempo di Lancio}: 2 Azioni\\
\textbf{Gittata}: Personale\\
\textbf{Componenti}: V, S, M (un po' di fosforo o una lucciola) \\
\textbf{Durata}: 10 minuti\\
Fiamme sottili e vaporose avvolgono il tuo corpo per la durata dell'incantesimo, emettendo luce intensa in un raggio di 3 metri e luce fioca per ulteriori 3 metri. Puoi terminare l'incantesimo in anticipo, usando un'azione per interromperlo.\\
Le fiamme ti forniscono uno scudo caldo o uno scudo freddo, a tua scelta. Lo scudo caldo ti conferisce resistenza al danno da freddo, mentre lo scudo freddo ti fornisce resistenza al danno da caldo.\\
Inoltre, ogni qualvolta una creatura entro 1 metro da te ti colpisce con un attacco in mischia, lo scudo erutta fiamme. L'attaccante subisce 2d8 danni da fuoco da uno scudo caldo, o 2d8 danni da freddo da uno scudo freddo.

\medskip\textbf{Scurovisione}\index{Incantesimi - Scurovisione}\\
\textbf{Scuola}: Trasmutazione\\
\textbf{Difficoltà}: 19\\
\textbf{Tempo di Lancio}: 2 Azioni\\
\textbf{Gittata}: Contatto\\
\textbf{Componenti}: V, S, M (o un pizzico di carota o di agata secca)\\
\textbf{Durata}: 8 ore\\
Una creatura consenziente con cui sei in contatto ottiene la capacità di vedere al buio. Per la durata dell'incantesimo, quella creatura ha scurovisione fino a una gittata di 18 metri.

\medskip\textbf{Segugio Fedele}\index{Incantesimi - Segugio Fedele}\\
\textbf{Scuola}: Evocazione\\
\textbf{Difficoltà}: 23\\
\textbf{Tempo di Lancio}: 2 Azioni\\
\textbf{Gittata}: 9 metri\\
\textbf{Componenti}: V, S, M (un minuscolo fischietto d'argento, e un pezzo d'osso, e un filo)\\
\textbf{Durata}: 8 ore\\
Puoi evocare un cane da guardia fantasma in uno spazio non occupato a gittata e che puoi vedere, dove rimarrà per la durata dell'incantesimo, finché non viene congedato con un'azione, o finché non si allontanerà più di 30 metri da te.\\
Il segugio è invisibile a tutte le creature eccetto che a te e non può essere danneggiato. Quando una creatura di taglia Piccola o superiore si avvicina entro 9 metri da esso senza aver prima pronunciato la parola d'ordine da te specificata quando hai lanciato l'incantesimo, il segugio inizia ad abbaiare a grande volume. Il segugio vede le creature invisibili e può vedere nel Piano Etereo. Esso ignora le illusioni. All'inizio di ciascun tuo round, il segugio tenta di mordere una creatura entro 1 metro da esso e che ti sia ostile. Il bonus di attacco del segugio è uguale al tuo modificatore di caratteristica da incantatore + CM. Se colpisce, infligge 2d8 danni perforanti.

\medskip\textbf{Sembrare}\index{Incantesimi - Sembrare}\\
\textbf{Scuola}: Illusione\\
\textbf{Difficoltà}: 26\\
\textbf{Tempo di Lancio}: 2 Azioni\\
\textbf{Gittata}: 9 metri\\
\textbf{Componenti}: V, S\\
\textbf{Durata}: 8 ore\\
Questo incantesimo ti permette di cambiare l'aspetto di un qualsiasi numero di creature a gittata e che puoi vedere. Fornisci a ciascun bersaglio un nuovo aspetto illusorio. Una creatura non consenziente può effettuare un Tiro Salvezza su Volontà e, se lo supera, ignora l'incantesimo.\\
L'incantesimo camuffa l'aspetto fisico oltre che gli abiti, le armature, le armi e l'equipaggiamento. Puoi far sembrare ciascuna creatura 30 centimetri più bassa o più alta, sembrare magra, grassa o una via di mezzo. Non puoi cambiare la conformazione del corpo del bersaglio, e quindi devi scegliere una forma che abbia la stessa distribuzione basilare di arti. \\
Per tutto il resto, l'illusione è limitata solo dalla tua fantasia. L'incantesimo permane per la sua durata, a meno che tu non usi una azione per interromperlo prima. I cambi apportati da questo incantesimo non sono in grado di sostenere un'ispezione fisica. Per esempio, se usi questo incantesimo per aggiungere un cappello all'abbigliamento di una creatura, gli oggetti attraversano il cappello, e chiunque lo tocchi non avvertirebbe nulla e finirebbe per toccare la testa e i capelli della creatura.
Se usi questo incantesimo per apparire più magro di quello che sei, la mano di una persona che provasse a toccarti rimbalzerebbe su di te, mentre alla vista sembrerebbe fermarsi a mezz'aria. Una creatura può usare 2 Azioni per ispezionare un bersaglio ed effettuare una prova di Consapevolezza contro la DC del Tiro Salvezza dell'incantesimo, se impiega 3 Azione ha +1d6 di bonus. Se la riesce, capisce che il bersaglio è camuffato.

\medskip\textbf{Semipiano}\index{Incantesimi - Semipiano}\\
\textbf{Scuola}: Evocazione\\
\textbf{Difficoltà}: 34\\
\textbf{Tempo di Lancio}: 2 Azioni\\
\textbf{Gittata}: 18 metri\\
\textbf{Componenti}: S\\
\textbf{Durata}: 1 ora\\
Crei una porta d'ombra su di una superficie piana a gittata e che puoi vedere. La porta è grande abbastanza da permettere il passaggio senza problemi a una creatura Media. Quando viene aperta, la porta conduce a un semipiano che appare come una stanza vuota di 9 metri in ciascuna dimensione, fatta di legno e pietra. Quando l'incantesimo termina, la porta scompare, e qualsiasi creatura od oggetto all'interno del semipiano rimane intrappolato lì, mentre la porta scompare anche dall'altro lato.\\
Ogni volta che esegui questo incantesimo, crei un nuovo semipiano, oppure permetti alla porta d'ombra di connettersi a un semipiano creato da un precedente lancio dell'incantesimo oppure aumenti di altri 9 metri in ciascuna dimensione un semipiano conosciuto creato da te precedentemente. \\
Inoltre, se conosci la natura e i contenuti di un semipiano creato dal lancio di questo incantesimo da parte di un'altra creatura, puoi far sì che la porta d'ombra si colleghi invece a quel semipiano.

\medskip\textbf{Serratura Arcana}\index{Incantesimi - Serratura Arcana}\\
\textbf{Scuola}: Abiurazione\\
\textbf{Difficoltà}: 19\\
\textbf{Tempo di Lancio}: 2 Azioni\\
\textbf{Gittata}: Contatto\\
\textbf{Componenti}: V, S, M (polvere d'oro del valore di almeno 25 mo, che viene consumata dall'incantesimo) \\
\textbf{Durata}: Fino a che dissolto\\
Lanci l'incantesimo a contatto di una porta, finestra, portale, forziere o altro ingresso chiuso, e questo diventa chiuso a chiave per la durata. Tu e le creature che hai indicato, quando hai lanciato questo incantesimo, potete aprire l'oggetto normalmente. Puoi anche predisporre una parola d'ordine che, quando pronunciata entro 1 metro dall'oggetto, sopprime l'incantesimo per 1 minuto. Altrimenti l'apertura è invalicabile fino a che non viene distrutta ol'incantesimo è dissolto o soppresso. Lanciare scassinare sull'oggetto sopprime serratura arcana per 10 minuti.\\
Mentre è soggetto a questo incantesimo, l'oggetto è più difficile da distruggere o aprire a forza; la DC per romperlo o scassinare una serratura su di esso aumenta di 10.

\medskip\textbf{Servitore Invisibile}\index{Incantesimi - Servitore Invisibile}\\
\textbf{Scuola}: Evocazione\\
\textbf{Difficoltà}: 16\\
\textbf{Tempo di Lancio}: 2 Azioni\\
\textbf{Gittata}: 18 metri\\
\textbf{Componenti}: V, S, M (un pezzo di corda e un pezzo di legno)\\
\textbf{Durata}: 1 ora\\
Questo incantesimo crea una forza quasi invisibile solo delimitata da una leggera aura (di colore a tua scelta) che svolge dei semplici compiti al tuo comando, fino al termine dell'incantesimo. Il servitore si forma in uno spazio sul terreno non occupato, entro la gittata. Ha Difesa 10, 1 punto ferita, Forza 0 e non può attaccare. Se scende a 0 punti ferita, l'incantesimo ha termine.\\
Come Azione Immediata, durante ciascun tuo round, puoi comandare mentalmente il servitore di muoversi fino a 4 metri e interagire con un oggetto. Il servitore può svolgere dei semplici compiti alla stregua di un servitore umano, come raccogliere cose, pulire, riparare, piegare abiti, accendere fuochi, servire il cibo e versare il vino. Una volta impartito il comando, il servitore svolgerà il compito al meglio delle sue capacità finché non l'avrà completato, e poi aspetterà il tuo prossimo comando. \\
Se comandi al servitore di svolgere un compito che lo farà muovere a più di 18 metri da te, l'incantesimo termina.

\medskip\textbf{Sfera Congelante}\index{Incantesimi - Sfera Congelante}\\
\textbf{Scuola}: Invocazione\\
\textbf{Difficoltà}: 29\\
\textbf{Tempo di Lancio}: 2 Azioni\\
\textbf{Gittata}: 90 metri\\
\textbf{Componenti}: V, S, M (una piccola sfera di cristallo)\\
\textbf{Durata}: Istantanea\\
Un globo gelido di energia fredda parte dalla punta delle tue dita verso un punto di tua scelta a gittata, dove esplode in una sfera di 18 metri di raggio. Ogni creatura nell'area deve effettuare un Tiro Salvezza su Tempra. Se fallisce il Tiro Salvezza, una creatura subisce 10d6 danni da freddo. Se lo supera, subisce la metà di questi danni.\\
Se il globo colpisce un corpo d'acqua o un liquido composto principalmente d'acqua (escluse però le creature a base d'acqua), congela il liquido fino a una profondità di 15 centimetri in un'area quadrata di 9 metri di lato. Il ghiaccio dura 1 minuto. Le creature che stavano nuotando sulla superficie dell'acqua congelata restano intrappolate nel ghiaccio. Una creatura intrappolata può usare due azioni per effettuare una prova di Forza contro la DC del Tiro Salvezza dell'incantesimo, al fine di liberarsi.\\
Se lo desideri, dopo aver completato l'incantesimo, puoi trattenerti dallo sparare il globo. Un piccolo globo, circa delle dimensioni di una pietra da fionda, freddo al contatto, appare nella tua mano. In qualsiasi momento, tu, o una creatura a cui hai dato il globo, potete lanciare il globo (fino a una gittata di 12 metri). Questo si frantumerà all'impatto, con lo stesso effetto del normale lancio dell'incantesimo. Puoi anche poggiare il globo a terra senza che si frantumi. Dopo 1 minuto, se il globo non è già stato frantumato, esploderà.\\
\textbf{Per ogni critico ottenuto} nella prova di magia il danno aumenta di 1d6

\medskip\textbf{Sfera Elastica}\index{Incantesimi - Sfera Elastica}\\
\textbf{Scuola}: Invocazione\\
\textbf{Difficoltà}: 23\\
\textbf{Tempo di Lancio}: 2 Azioni\\
\textbf{Gittata}: 90 metri\\
\textbf{Componenti}: V, S, M (un pezzo semisferico di cristallo trasparente e un pezzo semisferico corrispondente di gomma arabica)\\
\textbf{Durata}: Concentrazione, massimo 1 minuto\\
Una sfera di energia luminosa avvolge una creatura od oggetto di taglia Grande o inferiore a gittata. Una creatura non consenziente deve effettuare un Tiro Salvezza su Riflessi. Se lo fallisce, la creatura è avvolta dall'incantesimo per la sua durata.\\
Nulla (né oggetti fisici, né energia, né altri effetti di incantesimi) può attraversare questa barriera, in entrata o uscita, sebbene una creatura all'interno della sfera possa respirare senza problemi. La sfera è immune a tutti i danni, e una creatura al suo interno non può essere danneggiata da attacchi o effetti originanti dall'esterno, né una creatura all'interno della sfera può danneggiare nulla che si trovi all'esterno. La sfera è priva di peso e grande giusto a sufficienza per contenere la creatura o l'oggetto al suo interno. Una creatura avvolta può usare 1 Azione per spingere contro le pareti della sfera e quindi farla rotolare fino alla metà della velocità della creatura. Allo stesso modo, il globo può essere raccolto e mosso da altre creature.\\
Un incantesimo disintegrazione che prenda come bersaglio il globo lo distrugge senza danneggiare nulla al suo interno.

\medskip\textbf{Sfera Infuocata}\index{Incantesimi - Sfera Infuocata}\\
\textbf{Scuola}: Evocazione\\
\textbf{Difficoltà}: 19\\
\textbf{Tempo di Lancio}: 2 Azioni\\
\textbf{Gittata}: 18 metri\\
\textbf{Componenti}: V, S, M (un po' di sego, un pizzico di zolfo, e una manciata di ferro in polvere)\\
\textbf{Durata}: 1 minuto\\
Per la durata dell'incantesimo compare una sfera di 1 metro di diametro in uno spazio a gittata, scelto da te. Qualsiasi creatura che termini il suo round entro 1 metro dalla sfera deve effettuare un Tiro Salvezza su Riflessi. La creatura subisce 2d6 danni da fuoco se fallisce il Tiro Salvezza, o la metà di questi danni se lo supera.\\
Con un'azione puoi spostare la sfera di 9 metri. Se fai schiantare la sfera contro una creatura, la creatura deve effettuare un Tiro Salvezza contro il danno della sfera, e la sfera smetterà di muoversi per quel round.
Quando muovi la sfera, la puoi spostare oltre barriere alte fino a 1 metro, e farle saltare spazi larghi fino a 3 metri. La sfera incendia gli oggetti infiammabili non indossati o trasportati, e irradia una luce intensa in un raggio di 6 metri e una luce fioca per ulteriori 6 metri.\\
Mentre hai questo incantesimo attivo sei Distratto nel lancio di altri incantesimi.\\
\textbf{Per ogni critico ottenuto} nella prova di magia il danno aumenta di 1d6.\\

\medskip\textbf{Sfocatura}\index{Incantesimi - Sfocatura}\\
\textbf{Scuola}: Illusione\\
\textbf{Difficoltà}: 19\\
\textbf{Tempo di Lancio}: 2 Azioni\\
\textbf{Gittata}: Personale\\
\textbf{Componenti}: V\\
\textbf{Durata}: 1 minuto \\
Il tuo corpo diventa sfocato, indistinto e tremolante a chiunque ti veda. Per la durata dell'incantesimo, tutte le creature hanno ha -1d6 ai Tiri per Colpire contro di te. Gli attaccanti che non si affidano alla vista sono immuni a questo effetto, per esempio se hanno vista cieca o sono in grado di distinguere le illusioni, come per visione del vero.

\medskip\textbf{Sguardo Penetrante}\index{Incantesimi - Sguardo Penetrante}\\
\textbf{Scuola}: Necromanzia\\
\textbf{Difficoltà}: 29\\
\textbf{Tempo di Lancio}: 2 Azioni\\
\textbf{Gittata}: Personale\\
\textbf{Componenti}: V, S\\
\textbf{Durata}: Concentrazione, massimo 1 minuto\\
Per la durata dell'incantesimo, i tuoi occhi si tramutano in un vuoto nero infuso di terribile potere. Una creatura a tua scelta entro 18 metri da te e che puoi vedere, deve superare un Tiro Salvezza su Volontà o, per la durata, subire uno dei seguenti effetti di tua scelta. Durante ciascun tuo round, fino al termine dell'incantesimo, puoi usare due Azioni per prendere come bersaglio un'altra creatura, ma non puoi prendere di nuovo come bersaglio una creatura che abbia superato un Tiro Salvezza contro questo lancio di sguardo penetrante.\\
\medskip
\begin{itemize}
\item
\textit{Addormentato}. Il bersaglio cade privo di sensi. Si risveglia qualora subisca qualsiasi ammontare di danno o se un'altra creatura usa 2 Azioni per scuoterlo dal sonno.
\item
\textit{Ammalato}. Il bersaglio ha -1d6 ai Tiri per Colpire e le prove di caratteristica. Al termine di ciascun suo round, può effettuare un altro Tiro Salvezza su Volontà. Se lo supera, l'effetto ha termine.
\item
\textit{Impanicato}. Il bersaglio è spaventato da te. Durante ciascun suo round, la creatura spaventata deve effettuare usare due Azioni di Movimento e muoversi lontano da te tramite il tragitto più breve e sicuro possibile, a meno che non abbia spazio per muoversi. Se il bersaglio si muove in un luogo lontano almeno 18 metri da te, dove non ti possa vedere, questo effetto ha termine.
\end{itemize}

\medskip\textbf{Silenzio}\index{Incantesimi - Silenzio}\\
\textbf{Scuola}: Illusione\\
\textbf{Difficoltà}: 19\\
\textbf{Tempo di Lancio}: 2 Azioni\\
\textbf{Gittata}: 36 metri\\
\textbf{Componenti}: V, S\\
\textbf{Durata}: 10 minuti\\
Per la durata dell'incantesimo, nessun suono può essere creato all'interno o attraversare una sfera di 6 metri di raggio centrata su di un punto a gittata, scelto da te. Qualsiasi creatura o oggetto che si trovi completamente all'interno della sfera è immune al danno da tuono, e le creature che sono completamente al suo interno sono assordate. È impossibile lanciare un incantesimo che comprende una componente verbale mentre si è al suo interno.

\medskip\textbf{Simbolo}\index{Incantesimi - Simbolo}
\textbf{Scuola}: Abiurazione\\
\textbf{Difficoltà}: 31\\
\textbf{Tempo di Lancio}: 2 Azioni\\
\textbf{Gittata}: Contatto\\
\textbf{Componenti}: V, S, M (mercurio, fosforo e diamante e opale in polvere con un valore totale di almeno 1.000 mo, che l'incantesimo consuma)\\
\textbf{Durata}: Fino a che dissolto o attivato\\
Quando lanci questo incantesimo, inscrivi un glifo dannoso su di una superficie (come una sezione di pavimento, muro o un tavolo) o all'interno di un oggetto che può essere chiuso per nascondere il glifo (come un libro, una pergamena o un forziere). Se scegli una superficie, il glifo può coprire un'area di superficie non maggiore di 3 metri di diametro. Se scegli un oggetto,quell'oggetto deve restare al suo posto; se l'oggetto viene spostato più di 3 metri dal punto in cui è stato lanciato l'incantesimo, il glifo è spezzato, e l'incantesimo termina senza essere stato attivato.\\
Il glifo è quasi invisibile e può essere trovato con una prova di Sopravvivenza contro la DC del Tiro Salvezza dei tuoi incantesimi.\\
Decidi tu cosa attivi il glifo al momento del lancio dell'incantesimo.\\
Per i glifi inscritti su di una superficie, l'attivazione tipica comprende entrare in contatto o stare sopra il glifo, rimuovere un altro oggetto che copra il glifo, avvicinarsi a una certa distanza dal glifo, o manipolare l'oggetto su cui è inscritto il glifo.\\
Per i glifi inscritti su di un oggetto, l'attivazione tipica comprende aprire l'oggetto, avvicinarsi a una certa distanza dall'oggetto, o vedere o leggere il glifo.\\
Puoi definire meglio l'attivazione così che l'incantesimo si attivi solo in determinate circostanze o secondo certe peculiarità fisiche (come l'altezza o il peso) o specie di creatura (per esempio, la protezione potrebbe agire contro le megere o i mutaforma). Puoi anche predisporre condizioni per evitare che il glifo venga attivato, come la pronuncia di una parola d'ordine.\\
Quando inscrivi il glifo scegli una delle opzioni seguenti come suo effetto. Una volta attivato, il glifo riluce, riempiendo una sfera di 18 metri di raggio di luce fioca per 10 minuti, dopo i quali l'incantesimo termina. Ogni creatura nella sfera quando il glifo si attiva diventabersaglio del suo effetto, così come una creatura cheentri per la prima volta nella sfera durante un round o termine lì il suo round.\\
\medskip
\begin{itemize}
\item
\textit{Demenza}. Ogni bersaglio deve effettuare un Tiro Salvezza su Volontà. Se fallisce il Tiro Salvezza, il bersaglio diventa demente per 1 minuto. Una creatura demente non può effettuare azioni, non comprende quello che gli altri le dicono, non può leggere, e parla solo farfugliando. Il Narratore ne controlla i movimenti, che risultano erratici.\\
\item
\textit{Discordia}. Ogni bersaglio deve effettuare un Tiro Salvezza su Tempra. Se lo fallisce, il bersaglio inizia a bisticciare e discutere con un'altra creatura per 1 minuto. In questo periodo, è incapace di effettuare qualsiasi comunicazione significativa e ha -1d6 ai Tiri per Colpire e le prove di caratteristica. Dolore. Ogni bersaglio deve effettuare un Tiro Salvezza su Tempra. Se lo fallisce, il bersaglio diventa inabile a causa del dolore lacerante.
\item
\textit{Morte}. Ogni bersaglio deve effettuare un Tiro Salvezza su Tempra, subendo 10d10 danni da Vuoto se lo fallisce, o la metà di questi danni se lo supera. Paura. Ogni bersaglio deve effettuare un Tiro Salvezza su Volontà e, se lo fallisce, restare spaventato per 1 minuto. Mentre è spaventato, il bersaglio getta qualsiasi cosa stesse tenendo e deve muoversi almeno 9 metri lontano dal glifo durante ciascuno suo round, se in grado.
\item\
\textit{Sfiducia}. Ogni bersaglio deve effettuare un Tiro Salvezza su Volontà. Se fallisce il Tiro Salvezza, il bersaglio è sopraffatto dalla disperazione per 1 minuto. Durante questo periodo, non può attaccare o prendere come bersaglio nessuna creatura con capacità, incantesimi o altri effetti magici nocivi.
\item\
\textit{Sonno}. Ogni bersaglio deve effettuare un Tiro Salvezza su Volontà, e cadere privo di sensi per 10 minuti se lo fallisce. Una creatura si risveglia se subisce danni o se qualcuno usa un'azione per risvegliarla. 
\item
\textit{Stordimento}. Ogni bersaglio deve effettuare un Tiro Salvezza su Volontà, e restare stordito per 1 minuto se lo fallisce.
\end{itemize}

\medskip\textbf{Simulacro}\index{Incantesimi - Simulacro}\\
\textbf{Scuola}: Illusione\\
\textbf{Difficoltà}: 31\\
\textbf{Tempo di Lancio}: 12 ore\\
\textbf{Gittata}: Contatto\\
\textbf{Componenti}: V, S, M (neve o ghiaccio in quantità per creare una copia a dimensioni reali della creatura duplicata; un po' di capelli, unghie o altro pezzo del corpo di quella creatura da piazzare in mezzo alla neve o al ghiaccio; e un rubino in polvere del valore di 1.500 mo, sparso sopra il duplicato e consumato dall'incantesimo)\\
\textbf{Durata}: Fino a che dissolto\\
Modelli un duplicato illusorio di una bestia o umanoide che resti a gittata per l'intero tempo di lancio dell'incantesimo. Il duplicato è una creatura, in parte reale e formata di ghiaccio o neve, che può effettuare azioni e interagire come una normale creatura. Sembra essere identica all'originale, ma ha la metà dei punti ferita massimi di quella creatura e si presenta priva di equipaggiamento. Altrimenti, l'illusione usa tutte le statistiche della creatura che duplica.\\
Il simulacro è amichevole verso di te e le creature da te indicate. Obbedisce ai comandi da te pronunciati, muovendosi e agendo in accordo ai tuoi desideri e agendo durante il tuo round in combattimento. Il simulacro è privo della capacità di apprendere o diventare più potente, e quindi non accresce mai di livello o nelle caratteristiche, né può recuperare gli slot incantesimi spesi.\\
Se il simulacro è danneggiato, puoi ripararlo in un laboratorio alchemico, usando erbe rare e minerali del valore di 100 mo per punto ferita recuperato. Il simulacro rimane finché non scende a 0 punti ferita, a quel punto si ritrasforma in neve e si scioglie all'istante. Se lanci di nuovo questo incantesimo, qualsiasi duplicato da te creato con questo incantesimo attualmente attivo viene immediatamente distrutto.

\medskip\textbf{Sogno}\index{Incantesimi - Sogno}\\
\textbf{Scuola}: Illusione\\
\textbf{Difficoltà}: 26\\
\textbf{Tempo di Lancio}: 2 Azioni\\
\textbf{Gittata}: Speciale\\
\textbf{Componenti}: V, S, M (una manciata di sabbia, una punta di inchiostro, e una penna per scrivere presa da un volatile addormentato)\\
\textbf{Durata}: 8 ore\\
Questo incantesimo modella i sogni di una creatura. Scegli una creatura a te nota come bersaglio dell'incantesimo. Il bersaglio deve trovarsi sul tuo stesso piano di esistenza. Le creature che non dormono non possono essere soggette a questo incantesimo. Tu o una creatura consenziente con cui sei a contatto entrate in uno stato di trance, agendo da messaggero. Mentre è in trance, il messaggero è consapevole di ciò che lo circonda, ma non può effettuare azioni o muoversi.\\
Per la durata dell'incantesimo, se il bersaglio è addormentato, il messaggero appare nei sogni del bersaglio e può conversare con lui finché questi rimane addormentato. Il messaggero può anche modellare l'ambiente del sogno, creando terreni, oggetti e altre immagini. Il messaggero può emergere dalla trance in qualsiasi momento, terminando anticipatamente l'effetto dell'incantesimo. Al risveglio, il bersaglio ricorda perfettamente il suo sogno. Se il bersaglio è sveglio quando lanci l'incantesimo, il messaggero ne viene a conoscenza e può porre fine alla trance (e all'incantesimo) o aspettare che il bersaglio si addormenti. A quel punto il messaggero potrà comparire nei sogni del bersaglio.\\
Puoi fare apparire il messaggero al bersaglio con un aspetto mostruoso e terrificante. Se lo fai, il messaggero può consegnare un messaggio di al massimo dieci parole e poi il bersaglio deve effettuare un Tiro Salvezza su Volontà. Se fallisce il Tiro Salvezza, gli echi della spaventosa mostruosità generano un incubo per la durata del sonno del bersaglio che gli impedisce di ottenere qualsiasi beneficio da quel riposo. Inoltre, quando il bersaglio si sveglia, subisce 3d6 danni.\\
Se possiedi una ciocca di capelli, delle unghie tagliate, o simile porzione del corpo del bersaglio, egli effettuerà il suo Tiro Salvezza con -1d6.

\medskip\textbf{Sonno}\index{Incantesimi - Sonno}\\
\textbf{Scuola}: Ammaliamento\\
\textbf{Difficoltà}: 16\\
\textbf{Tempo di Lancio}: 2 Azioni\\
\textbf{Gittata}: 27 metri\\
\textbf{Componenti}: V, S, M (un pizzico di sabbia, petali di rosa o un grillo)\\
\textbf{Durata}: 1 minuto\\
Questo incantesimo pone le creature in un torpore magico. Tira 5d8; il totale è il numero di punti ferita di creature su cui l'incantesimo può agire. Le creature, entro 6 metri dal punto a gittata scelto da te, sono influenzate in ordine ascendente di punti ferita (ignorando le creature svenute).\\
A partire dalla creatura con il numero più basso di punti ferita attuali, ogni creatura soggetta a questo incantesimo perde i sensi fino al termine dell'incantesimo, chi dorme subisce danni, o qualcuno usa un'azione per scuotere o prendere a schiaffi l'addormentato. Sottrarre i punti ferita di ciascuna creatura dal totale prima di considerare la creatura con il valore di punti ferita più basso successiva. I punti ferita di una creatura devono essere uguali o inferiori al totale rimanente perché l'effetto agisca su di essa. Non morti e creature che non possono essere affascinate non sono influenzate da questo incantesimo.\\
\textbf{Per ogni critico ottenuto} nella prova di magia tira 2d8 pf aggiuntivi.

\medskip\textbf{Spada Arcana}\index{Incantesimi - Spada Arcana}\\
\textbf{Scuola}: Invocazione\\
\textbf{Difficoltà}: 31\\
\textbf{Tempo di Lancio}: 2 Azioni\\
\textbf{Gittata}: 18 metri\\
\textbf{Componenti}: V, S, M (una spada di platino in miniatura con l'impugnatura e il pomello di rame e zinco, del valore di 250 mo)\\
\textbf{Durata}: Concentrazione, massimo 1 minuto \\
Per la durata dell'incantesimo, crei a gittata un piano di forza a forma di spada fluttuante. Quando la spada appare, effettui un attacco in mischia con modificatore CM + modificatore da incantesimo contro un bersaglio scelto da te entro 1 metro dalla spada. Se colpisci, il bersaglio subisce 3d10 danni da forza. Fino al termine dell'incantesimo,puoi usare un'azione ogni tuo round per muovere la spada di 6 metri in un punto che puoi vedere e ripetere questo attacco contro lo stesso bersaglio o uno differente.

\medskip\textbf{Spostamento Planare}\index{Incantesimi - Spostamento Planare}\\
\textbf{Scuola}: Evocazione\\
\textbf{Difficoltà}: 31\\
\textbf{Tempo di Lancio}: 2 Azioni\\
\textbf{Gittata}: Contatto\\
\textbf{Componenti}: V, S, M (una verga di metallo biforcuta del valore di almeno 250 mo, sintonizzata verso uno specifico piano di esistenza)\\
\textbf{Durata}: Istantanea\\
Tu e un massimo di altre otto creature consenzienti, che si stringono le mani per formare un cerchio, venite trasportati su di un diverso piano di esistenza. Puoi specificare una destinazione bersaglio in termini generici, e riapparirai all'interno o in prossimità di quella destinazione, a discrezione del Narratore.\\
In alternativa, se conosci la sequenza di sigilli di un cerchio di teletrasporto verso un altro piano di esistenza, l'incantesimo può condurti a quel cerchio. Se il cerchio di teletrasporto è troppo piccolo per contenere tutte le creature che trasporti con te, esse riappariranno nello spazio non occupato più vicino possibile al cerchio.\\
Puoi usare questo incantesimo per bandire una creatura non consenziente in un altro piano. Scegli una creatura a portata ed effettua un attacco in mischia con incantesimo contro di essa. Se colpisci, la creatura deve effettuare un Tiro Salvezza su Volontà. Se la creatura fallisce il Tiro Salvezza, viene trasportata in un luogo casuale sul piano di esistenza da te specificato. Una creatura così trasportata dovrà trovare per proprio conto il modo di tornare sul tuo attuale piano di esistenza.

\medskip\textbf{Spruzzo Colorato}\index{Incantesimi - Spruzzo Colorato}\\
\textbf{Scuola}: Illusione\\
\textbf{Difficoltà}: 16\\
\textbf{Tempo di Lancio}: 2 Azioni\\
\textbf{Gittata}: Personale (cono di 4 metri)\\
\textbf{Componenti}: V, S, M (un pizzico di polvere o sabbia che sia colorata di rosso, giallo e blu)\\
\textbf{Durata}: 1 round\\
Dalla tua mano si sprigiona una raffica di luci colorate e abbaglianti. Tira 6d10; il totale è l'ammontare di punti ferita di creature su cui questo incantesimo agisce. Le creature, in un cono di 4 metri che origina da te, sono soggette in ordine ascendente dei loro attuali punti ferita (ignorando le creature prive di sensi e le creature che non possono vedere).\\
A partire dalla creatura che ha il minor numero di punti ferita attuali, ciascuna creatura soggetta a questo incantesimo resta accecata fino al termine dell'incantesimo. Sottrarre i punti ferita di ciascuna creatura dal totale prima di passare alla creatura col totale più basso di punti ferita successiva. I punti ferita di una creatura devono essere uguali o minori del totale rimanente perché l'incantesimo agisca su di essa. \\
\textbf{Per ogni critico ottenuto} nella prova di magia tira 2d10 aggiuntivi.

\medskip\textbf{Spruzzo Prismatico}\index{Incantesimi - Spruzzo Prismatico}\\
\textbf{Scuola}: Invocazione\\
\textbf{Difficoltà}: 31\\
\textbf{Tempo di Lancio}: 2 Azioni\\
\textbf{Gittata}: Personale (cono di 18 metri)\\
\textbf{Componenti}: V, S\\
\textbf{Durata}: Istantanea\\
Otto raggi di luce multicolore partono dalla tua mano. Ogni raggio è di un diverso colore e ha un potere e uno scopo diverso. Ogni creatura in un cono di 18 metri deve effettuare un Tiro Salvezza su Riflessi. Per ogni bersaglio, tirare un d8 per determinare quale sia il colore del raggio che lo ha colpito.\\
\medskip
\begin{itemize}
\item
\textit{1. Rosso}. Il bersaglio subisce 10d6 danni da fuoco se fallisce il Tiro Salvezza, o la metà di questi danni se lo supera.
\item
\textit{2. Arancio}. Il bersaglio subisce 10d6 danni da acido se fallisce il Tiro Salvezza, o la metà di questi danni se lo supera.
\item
\textit{3. Giallo}. Il bersaglio subisce 10d6 danni da fulmine se fallisce il Tiro Salvezza, o la metà di questi danni se lo supera.
\item
\textit{4. Verde}. Il bersaglio subisce 10d6 danni da veleno se fallisce il Tiro Salvezza, o la metà di questi danni se lo supera.
\item
\textit{5. Blu}. Il bersaglio subisce 10d6 danni da freddo se fallisce il Tiro Salvezza, o la metà di questi danni se lo supera.
\item
\textit{6. Indaco}. Se fallisce il Tiro Salvezza, il bersaglio è intralciato. Deve poi effettuare un Tiro Salvezza su Tempra all'inizio di ciascun suo round. Se supera il Tiro Salvezza tre volte, l'incantesimo termina. Se fallisce il Tiro Salvezza tre volte, viene permanentemente trasformato in pietra e diventa vittima della condizione pietrificato. I successi e i fallimenti non devono essere consecutivi; tieni traccia di entrambi finché il bersaglio non ne ha ottenuti tre dello stesso tipo.
\item
\textit{7. Violetto}. Se fallisce il Tiro Salvezza, il bersaglio è accecato. Deve poi effettuare un Tiro Salvezza su Volontà all'inizio del tuo prossimo round. Se supera il Tiro Salvezza, la cecità termina. Se fallisce il Tiro Salvezza, la creatura viene trasportata su di un altro piano di esistenza a scelta del Narratore e non è più accecata (di solito, una creatura che non è sul suo piano natio, viene esiliata su di esso, mentre le altre creature sono di solito portate nei piani Astrale o Etereo).
\item
\textit{8. Speciale}. Il bersaglio è colpito da due raggi. Tira altre due volte, ritirando gli 8.
\end{itemize}

\medskip\textbf{Spruzzo Velenoso}\index{Trucchetto - Spruzzo Velenoso}\\
\textbf{Scuola}: Evocazione\\
\textbf{Difficoltà}: 12\\
\textbf{Tempo di Lancio}: 2 Azioni\\
\textbf{Gittata}: 3 metri\\
\textbf{Componenti}: V, S\\
\textbf{Durata}: Istantanea\\
Stendi la mano verso una creatura a gittata e che puoi vedere, e proietti una nube di gas velenoso dal tuo palmo. La creatura deve superare un Tiro Salvezza su Tempra o subire 1d12 danni da veleno. \\
Il danno dell'incantesimo aumenta di 1d8 quando raggiungi CM 5, CM 11 e CM 17.

\medskip\textbf{Stretta Folgorante}\index{Trucchetto - Stretta Folgorante}\\
\textbf{Scuola}: Invocazione\\
\textbf{Difficoltà}: 12\\
\textbf{Tempo di Lancio}: 2 Azioni\\
\textbf{Gittata}: Contatto\\
\textbf{Componenti}: V, S\\
\textbf{Durata}: Istantanea\\
Dalle tue mani saettano lampi che infliggono una scossa a una creatura con cui provi a entrare in contatto. Effettua un attacco in mischia con incantesimo contro il bersaglio. Hai +1d6 sul tiro per colpire se il bersaglio sta indossando un'armatura fatta di metallo.Se colpisci, il bersaglio subisce 1d8 danni da fulmine, e non può effettuare reazioni fino all'inizio del suo prossimo round.\\
Il danno dell'incantesimo aumenta di 1d8 quando raggiungi CM 5, CM 11 e CM 17.

\medskip\textbf{Suggestione}\index{Incantesimi - Suggestione}\\
\textbf{Scuola}: Ammaliamento\\
\textbf{Difficoltà}: 19\\
\textbf{Tempo di Lancio}: 2 Azioni\\
\textbf{Gittata}: 9 metri\\
\textbf{Componenti}: V, M (la lingua di un serpente e un pezzo di favo o un goccio di olio dolce)\\
\textbf{Durata}: 8 ore \\
Suggerisci un corso di attività (limitato a una o due frasi) e influenzi magicamente una creatura a gittata e che puoi vedere e udire e ti possa capire, scelta da te. Le creature che non possono essere affascinate sono immuni a questo effetto. La suggestione deve essere pronunciata in modo che il corso d'azione suoni ragionevole. Chiedere a una creatura di pugnalarsi,gettarsi su una lancia, darsi fuoco, o fare qualche altro atto palesemente dannoso nega automaticamente gli effetti dell'incantesimo.\\
Il bersaglio deve effettuare un Tiro Salvezza su Volontà. Se fallisce il Tiro Salvezza, esso segue il corso d'azione da te descritto al meglio delle sue capacità. Il corso d'azione suggerito può proseguire per l'intera durata dell'incantesimo. Se l'attività suggerita può essere completata in un tempo più breve,l'incantesimo ha termine quando il soggetto termina di fare ciò che gli è stato chiesto.\\
Puoi anche specificare condizioni che attiveranno un'attività speciale per la durata dell'incantesimo. Per esempio, potresti suggerire a un cavaliere di cedere il suo cavallo da guerra al primo mendicante che incontri. Se la condizione non viene soddisfatta prima del termine dell'incantesimo, l'attività non verrà svolta. Se tu o uno qualsiasi dei tuoi compagni danneggia il bersaglio, l'incantesimo ha termine.\\

\medskip\textbf{Suggestione di Massa}\index{Incantesimi - Suggestione di Massa}\\
\textbf{Scuola}: Ammaliamento\\
\textbf{Difficoltà}: 29\\
\textbf{Tempo di Lancio}: 2 Azioni\\
\textbf{Gittata}: 18 metri\\
\textbf{Componenti}: V, M (la lingua di un serpente e un pezzo di favo o un goccio di olio dolce)\\
\textbf{Durata}: 24 ore\\
Suggerisci un corso di attività (limitato a una o due frasi) e influenzi magicamente fino a dodici creature a gittata che puoi vedere e udire e ti possano capire, scelte da te. Le creature che non possono essere affascinate sono immuni a questo effetto. La suggestione deve essere pronunciata in modo che il corso d'azione suoni ragionevole. Chiedere a una creatura di pugnalarsi, gettarsi su di una lancia, darsi fuoco, o fare qualche altro atto palesemente dannoso nega automaticamente gli effetti dell'incantesimo.\\
Ogni bersaglio deve effettuare un Tiro Salvezza su Volontà. Se fallisce il Tiro Salvezza, esso segue il corso d'azione da te descritto al meglio delle sue capacità. Il corso d'azione suggerito può proseguire per l'intera durata dell'incantesimo. Se l'attività suggerita può essere completata in un tempo più breve, l'incantesimo ha termine quando il soggetto termina di fare ciò che gli è stato chiesto.\\
Puoi anche specificare condizioni che attiveranno un'attività speciale per la durata dell'incantesimo. Per esempio, potresti suggerire a un gruppo di soldati di cedere tutti i loro soldi al primo mendicante che incontrino. Se la condizione non viene soddisfatta prima del termine dell'incantesimo, l'attività non verrà svolta. Se tu o uno qualsiasi dei tuoi compagni danneggia una creatura soggetta a questo incantesimo, per quella creatura l'incantesimo ha termine.\\
\textbf{Per ogni critico ottenuto} nella prova di magia aggiungi un giorno alla durata.

\medskip\textbf{Taumaturgia}\index{Trucchetto - Taumaturgia}\\
\textbf{Scuola}: Universale\\
\textbf{Difficoltà}: 12\\
\textbf{Tempo di Lancio}: 2 Azioni\\
\textbf{Gittata}: 9 metri\\
\textbf{Componenti}: V\\
\textbf{Durata}: Massimo 1 minuto\\
Manifesti a gittata una trucco minore, un segno di potere soprannaturale. Crei a gittata uno dei seguenti effetti magici:
\medskip
\begin{itemize}
\item
La tua voce risuona tre volte più forte del normale per 1 minuto.
\item
Fai sì che le fiamme tremolino, si intensifichino, affievoliscano o cambino colore per 1 minuto.
\item
Provochi innocui tremori sul terreno per 1 minuto. 
\item
Crei un rumore istantaneo, come un rombo di tuono, il verso di un corvo, o un sussurro inquietante, che origina da un punto a gittata scelto da te.
\item
Fai sì che una porta o una finestra non chiusa a chiave si spalanchi o si chiuda di colpo.
\item
Modifichi l'aspetto dei tuoi occhi per 1 minuto.
\end{itemize}
\medskip
Se lanci questo incantesimo più volte, puoi tenere attivi fino a tre effetti da un minuto alla volta, e puoi interrompere questi effetti con un'azione.

\medskip\textbf{Telecinesi}\index{Incantesimi - Telecinesi}\\
\textbf{Scuola}: Trasmutazione\\
\textbf{Difficoltà}: 26\\
\textbf{Tempo di Lancio}: 2 Azioni\\
\textbf{Gittata}: 18 metri\\
\textbf{Componenti}: V, S\\
\textbf{Durata}: Concentrazione, massimo 10 minuti \\
Ottieni la capacità di muovere o manipolare creature o oggetti tramite il pensiero. Quando lanci questo incantesimo, e come 2 Azioni durante ciascun round, puoi esercitare la tua volontà su di una creatura od oggetto a gittata e che puoi vedere, provocando l'effetto appropriato tra quelli seguenti. Puoi agire round dopo round sempre sullo stesso bersaglio, o sceglierne uno nuovo ogni volta. Se cambi bersaglio, il bersaglio precedente non è più soggetto all'incantesimo.
\textit{Creatura}. Puoi tentare di muovere una creatura di taglia Enorme o più piccola. Effettua una prova di caratteristica usando la tua caratteristica da incantatore contesa da una prova di Forza della creatura. Se vinci la contesa, muovi la creatura di 9 metri in qualsiasi direzione, compreso verso l'alto, ma senza eccedere la gittata dell'incantesimo. Fino al termine del tuo prossimo round, la creatura è intralciata dalla tua presa telecinetica. Una creatura sollevata in alta, resta sospesa a mezz'aria.\\
Nei round successivi, puoi usare 2 Azioni per tentare di mantenere la tua presa telecinetica sulla creatura ripetendo la contesa.
\textit{Oggetto}. Puoi tentare di muovere un oggetto che pesa fino a 500 chili. Se l'oggetto non è indossato o trasportato, lo sposti automaticamente di 9 metri in qualsiasi direzione, ma senza superare la gittata dell'incantesimo.\\
Se l'oggetto è indossato o trasportato da una creatura, devi effettuare una prova di caratteristica con la tua caratteristica da incantatore contesa dalla prova di Forza della creatura. Se vinci la contesa, trascini via l'oggetto da quella creatura e lo muovi di 9 metri in una qualsiasi direzione, senza però superare la gittata dell'incantesimo.\\
Puoi esercitare un controllo preciso sugli oggetti tramite la tua presa telecinetica, riuscendo così a manipolare un attrezzo semplice, aprire una porta o un contenitore,inserire o recuperare un oggetto da un contenitore aperto, o versare del materiale in una fiala.

\medskip\textbf{Teletrasporto}\index{Incantesimi - Teletrasporto}\\
\textbf{Scuola}: Evocazione\\
\textbf{Difficoltà}: 31\\
\textbf{Tempo di Lancio}: 2 Azioni\\
\textbf{Gittata}: 3 metri\\
\textbf{Componenti}: V\\
\textbf{Durata}: Istantanea\\
Questo incantesimo teletrasporta istantaneamente te e altre otto creature consenzienti (oppure un singolo oggetto) a gittata e che puoi vedere, scelte da te, in una destinazione di tua scelta. Se il bersaglio è un oggetto, deve poter entrare in un cubo di 3 metri di spigolo, e non può essere tenuto o trasportato da una creatura non consenziente.\\
La destinazione che scegli ti deve essere nota, e deve essere sullo stesso piano di esistenza in cui ti trovi. La tua familiarità con la destinazione determina se vi riesce ad arrivare.\\
Il DM tira un d100 e consulta la tabella.
\end{multicols}
\medskip
\begin{tabular}{lllll}
\toprule
d100 				&	Errore		&	Area Simile	&Fuori Bersaglio&Sul Bersaglio\\
Cerchio permanente	&-	&-	&-	&01-100\\
Oggetto Associato	&-	&-	&-	&01-100\\
Molto Familiare		&01-05	&06-13	&14-24	&25-100\\
Visto per caso		&01-33	&34-43	&44-53	&54-100\\
Visto una volta		&01-43	&44-53	&54-73	&74-100\\
Descrizione			&01-43	&44-53	&54-73	&74-100\\
Falsa Destinazione	&01-50	&51-100	&-	&-\\
\end{tabular}
\medskip
\begin{multicols}{2}

\textit{Cerchio permanente} indica un cerchio di teletrasporto permanente di cui conosci la sequenza dei sigilli.\\
\textit{Oggetto associato} indica che possiedi uno ggetto preso negli ultimi sei mesi dalla destinazione desiderata, come il libro della biblioteca di un mago, biancheria della suite reale, o un pezzo di marmo della tomba segreta di un lich.\\
\textit{Molto familiare} è un luogo in cui sei stato molto spesso, un posto che hai studiato attentamente, o un posto che puoi vedere quando lanci l'incantesimo.\\
\textit{Visto casualmente} è un posto che hai visto più di una volta ma con cui non sei molto familiare. \\
\textit{Visto una volta} è un posto che hai visto una volta sola, magari tramite la magia.\\ \textit{Descrizione} è un luogo la cui posizione e aspetto conosci solo tramite la descrizione di qualcun altro, magari una mappa.\\
\textit{Falsa destinazione} è un posto che non esiste. Magari hai cercato di scrutare il nascondiglio di un nemico ma hai invece visto un'illusione, oppure stai cercando di teletrasportarti in un posto familiare che non esiste più. \\
\textit{Sul Bersaglio}. Tu e il tuo gruppo (o l'oggetto bersaglio) apparite dove desideri.\\
\textit{Fuori Bersaglio}. Tu e il tuo gruppo (o l'oggetto bersaglio) apparite a una distanza casuale dalla destinazione in una direzione casuale. La distanza fuori bersaglio è 1d10 x 1d10 percento della distanza viaggiata. Per esempio, se hai provato a viaggiare per 180 chilometri, atterri fuori bersaglio e tiri 5 e 3 su due d10, allora saresti fuori bersaglio del 15\%, ovvero 27 chilometri. Il Narratore determina la direzione fuori bersaglio casualmente, tirando un d8 e indicando l'1 come nord, il 2 come nordest, il 3 come est e così via seguendo le direzioni della bussola. Se ti stai teletrasportando in una città costiera e finisci 27 chilometri al largo in mare, potresti essere nei guai!\\
\textit{Area Simile}. Tu e il tuo gruppo (o l'oggetto bersaglio) finite in un'area diversa che è visualmente o tematicamente simile all'area bersaglio. Per esempio, se sei diretto al tuo laboratorio personale, potresti finire nel laboratorio di un altro mago o in un negozio di oggetti alchemici che possiede molti degli attrezzi e strumenti del tuo laboratorio. In genere, compari nel luogo simile più vicino, ma dato che l'incantesimo non ha limiti di gittata, potresti finire praticamente dovunque sullo stesso piano.\\
\textit{Errore}. L'imprevedibile magia dell'incantesimo provoca un viaggio difficile. Ogni creatura teletrasportata (o l'oggetto bersaglio) subisce 3d10 danni da forza, e il Narratore ritira sulla tabella per vedere dove finiscano (possono capitare più errori, che infliggono danni ogni
volta).

\medskip\textbf{Tempesta di Fuoco}\index{Incantesimi - Tempesta di Fuoco}\\
\textbf{Scuola}: Invocazione\\
\textbf{Difficoltà}: 31\\
\textbf{Tempo di Lancio}: 2 Azioni\\
\textbf{Gittata}: 45 metri\\
\textbf{Componenti}: V, S\\
\textbf{Durata}: Istantanea\\
Una tempesta composta di fiamme ruggenti compare in un punto a gittata, scelto da te. L'area della tempesta consiste di un massimo di dieci cubi di 3 metri di spigolo, che puoi disporre come preferisci. Ogni cubo deve avere almeno una faccia adiacente a quella di un altro cubo. Ogni creatura nell'area deve effettuare un Tiro Salvezza su Riflessi. Se lo fallisce subisce 7d10 danni da fuoco, o la metà di questi danni se lo supera. Il fuoco danneggia gli oggetti nell'area e incendia gli oggetti infiammabili che non sono indossati o trasportati. Se lo desideri, la vita vegetale nell'area resta illesa dagli effetti di questo incantesimo. \\
\textbf{Successo/Fallimento Critico}: In caso si fallimento critico il danno raddoppia, in caso di successo critico il danno viene ulteriormente dimezzato

\medskip\textbf{Tempesta di Ghiaccio}\index{Incantesimi - Tempesta di Ghiaccio}\\
\textbf{Scuola}: Invocazione\\
\textbf{Difficoltà}: 23\\
\textbf{Tempo di Lancio}: 2 Azioni\\
\textbf{Gittata}: 90 metri\\
\textbf{Componenti}: V, S, M (un pizzico di polvere e alcune gocce d'acqua)\\
\textbf{Durata}: Istantanea\\
Una grandinata di ghiaccio si abbatte a terra in un cilindro di 6 metri di raggio e 12 metri di altezza centrato su di un punto a gittata. Ogni creatura nel cilindro deve effettuare un Tiro Salvezza su Riflessi. La creatura subisce 2d8 danni da botta e 4d6 danni da freddo se fallisce il Tiro Salvezza, o la metà se lo supera. La grandine trasforma l'area di effetto della tempesta in terreno difficile fino al termine del tuo prossimo round.
\textbf{Per ogni critico ottenuto} nella prova di magia il danno aumento di 1d8.\\
\textbf{Successo/Fallimento Critico}: In caso si fallimento critico il danno raddoppia, in caso di successo critico il danno viene ulteriormente dimezzato

\medskip\textbf{Tempesta di Nevischio}\index{Incantesimi - Tempesta di Nevischio}\\
\textbf{Scuola}: Evocazione\\
\textbf{Difficoltà}: 21\\
\textbf{Tempo di Lancio}: 2 Azioni\\
\textbf{Gittata}: 45 metri\\
\textbf{Componenti}: V, S, M (un pizzico di polvere e qualche goccia d'acqua)\\
\textbf{Durata}: 1 minuto\\
Fino al termine dell'incantesimo, pioggia gelida e nevischio si abbattono in un cilindro alto 6 metri e del raggio di 12 metri centrato in un punto da te scelto a gittata. L'area è in penombra, mentre le fiamme esposte vengono spente. Il terreno nell'area è coperto di ghiaccio scivoloso, rendendolo terreno difficile. Quando una creatura entra nell'area dell'incantesimo per la prima volta durante un round o inizia il suo round lì, deve effettuare un Tiro Salvezza su Riflessi. Se lo fallisce, cade prona. Se una creatura nell'area dell'incantesimo si sta concentrando, deve superare un Tiro Salvezza su Tempra contro la DC del Tiro Salvezza dell'incantesimo o perdere la concentrazione. 

\medskip\textbf{Tentacoli Neri}\index{Incantesimi - Tentacoli Neri}\\
\textbf{Scuola}: Evocazione\\
\textbf{Difficoltà}: 23\\
\textbf{Tempo di Lancio}: 2 Azioni\\
\textbf{Gittata}: 27 metri\\
\textbf{Componenti}: V, S, M (un pezzo di tentacolo di una piovra gigante o di un calamaro gigante)\\
\textbf{Durata}: 1 minuto\\
Viscidi tentacoli d'ebano riempiono un quadrato di 6 metri di lato sul terreno, a gittata e che puoi vedere. Per la durata dell'incantesimo, questi tentacoli trasformano l'area in terreno difficile.\\
Quando una creatura entra nell'area soggetta per la prima volta in un round o comincia qui il suo round, la creatura deve superare un Tiro Salvezza su Riflessi o subire 3d6 danni da botta e rimanere intralciata dai tentacoli fino al termine dell'incantesimo. Una creatura che inizia il suo round nell'area ed è già intralciata dai tentacoli, subisce 3d6 danni da botta. Una creatura intralciata dai tentacoli può usare 2 Azioni per effettuare una prova di Forza o Destrezza (a sua scelta) contro la DC del Tiro Salvezza dell'incantesimo, se la supera, si libera.

\medskip\textbf{Terremoto}\index{Incantesimi - Terremoto}\\
\textbf{Scuola}: Invocazione\\
\textbf{Difficoltà}: 34\\
\textbf{Tempo di Lancio}: 2 Azioni\\
\textbf{Gittata}: 150 metri\\
\textbf{Componenti}: V, S, M (un pizzico di terriccio, un pezzo di pietra e un grumo di argilla)\\
\textbf{Durata}: Concentrazione, massimo 1 minuto\\
Provochi un disturbo sismico in un punto sul terreno a gittata e che puoi vedere. Per la durata, un intenso tremore scuote il terreno in un cerchio di 30 metri di raggio centrato su quel punto e scuote le creature e le strutture in quell'area che sono a contatto del terreno.Il terreno nell'area diventa terreno difficile. Ogni creatura a terra che si sta concentrando deve effettuare un Tiro Salvezza su Tempra. Se lo fallisce, la sua concentrazione è infranta.\\
Quando lanci questo incantesimo e alla fine di ogni round che hai speso a concentrarti su di esso, ogni creatura nell'area che si trovi a terra deve effettuare un Tiro Salvezza su Riflessi. Se lo fallisce, la creatura cade prona.\\
Questo incantesimo ha effetti aggiuntivi a seconda del tipo di terreno nell'area, a discrezione del Narratore. Fenditure. All'inizio del round successivo a quello in cui hai lanciato l'incantesimo si aprono delle fenditure per tutta l'area dell'incantesimo. Un totale di 1d6 fenditure si aprono in punti scelti dal Narratore. Ognuna di esse è profonda 1d10 x 3 metri, larga 3 metri e si estende da un lato dell'area dell'incantesimo all'altro. Una creatura che si trova sul punto in cui si apre una fenditura deve superare un Tiro Salvezza su Riflessi o cadervi dentro. Una creatura che riesca il Tiro Salvezza si sposta sul bordo della fenditura, nel momento in cui questa si apre.\\
Una fenditura che si apre sotto una struttura la fa crollare immediatamente (vedi sotto). Strutture. Il tremore infligge 50 danni da botta a qualsiasi struttura in contatto col terreno nell'area quando lanci l'incantesimo e alla fine di ciascuno dei tuoi turni fino al termine dell'incantesimo. Se una struttura scende a 0 punti ferita, crolla e potrebbe danneggia le creature vicine. Una creatura distante dalla struttura metà della altezza o meno della struttura, deve effettuare un Tiro Salvezza su Riflessi. Se lo fallisce, la creatura subisce 5d6 danni da botta, cade prona ed è sommersa dalle macerie. Dovrà poi impiegare un'azione riuscendo una prova di Destrezza (Atletica) DC 20 per liberarsi. Il Narratore può modificare verso l'alto o il basso la DC, a seconda della natura delle macerie. Se supera il Tiro Salvezza, la creatura subisce solo la metà dei danni e non cade né resta sepolta.

\medskip\textbf{Terreno Illusorio}\index{Incantesimi - Terreno Illusorio}\\
\textbf{Scuola}: Illusione\\
\textbf{Difficoltà}: 23\\
\textbf{Tempo di Lancio}: 10 minuti\\
\textbf{Gittata}: 90 metri\\
\textbf{Componenti}: V, S, M (una pietra, un rametto e un pezzo di pianta verde)\\
\textbf{Durata}: 24 ore \\
Fai sì che un pezzo di terreno naturale a gittata, in un cubo di 45 metri di spigolo, appaia, risuoni e odori come qualche altro tipo di terreno naturale. Di conseguenza, campi aperti o una strada possono essere trasformati in un acquitrino, colline, un crepaccio o qualche altro tipo di terreno difficile o invalicabile. Un laghetto può essere trasformato in una radura erbosa, un precipizio in una lieve pendenza, un burrone cosparso di rocce in una strada ampia e liscia. Le strutture edificate, l'equipaggiamento e le creature all'interno dell'area non mutano d'aspetto.\\
Le peculiarità tattili del terreno sono immutate, così che le creature che entrano nell'area è probabile che svelino l'illusione. Se al contatto la differenza non è ovvia, una creatura che esamina con cautela l'illusione può tentare una prova di Intelligenza (Indagare) contro la DC del Tiro Salvezza dei tuoi incantesimi per dubitare di essa. Una creatura che riconosca l'illusione per quello che è, la percepisce come una vaga immagine sovrapposta al terreno.

\medskip\textbf{Tocco Gelido}\index{Trucchetto - Tocco Gelido}\\
\textbf{Scuola}: Necromanzia\\
\textbf{Difficoltà}: 12\\
\textbf{Tempo di Lancio}: 2 Azioni\\
\textbf{Gittata}: 36 metri\\
\textbf{Componenti}: V, S\\
\textbf{Durata}: 1 round\\
Crei una scheletrica mano spettrale nello spazio di una creatura a gittata. Effettua un attacco a distanza con incantesimo contro la creatura, per aggredirla con il gelo della morte. Se colpisci, il bersaglio subisce 1d8 danni da Vuoto, e non può recuperare punti ferita fino all'inizio del tuo prossimo round. Fino ad allora, la mano resterà serrata sul bersaglio. Se colpisci un bersaglio non morto, esso avrà anche -1d6 ai Tiri per Colpire contro di te fino alla fine del suo prossimo round.\\
Il danno dell'incantesimo aumenta di 1d8 quando raggiungi CM 5, CM 11 e CM 17.

\medskip\textbf{Tocco Vampirico}\index{Incantesimi - Tocco Vampirico}\\
\textbf{Scuola}: Necromanzia\\
\textbf{Difficoltà}: 21\\
\textbf{Tempo di Lancio}: 2 Azioni\\
\textbf{Gittata}: Personale\\
\textbf{Componenti}: V, S\\
\textbf{Durata}: 1 minuto \\
Il contatto con la tua mano avvolta dall'ombra può risucchiare la forza vitale altrui per curare le tue ferite. Effettua un attacco in mischia con incantesimo contro una creatura a portata. Se colpisci, il bersaglio subisce 3d6 danni da Vuoto, e tu recuperi un numero di punti ferita pari alla metà del danno da Vuoto che hai inflitto. Fino al termine dell'incantesimo, puoi effettuare ogni round di nuovo questo attacco come tua azione di attacco.\\
Mentre hai questo incantesimo attivo sei considerato Distratto per il lancio di altri incantesimi.\\
\textbf{Per ogni critico ottenuto} nella prova di magia il danno aumento di 1d6.\\

\medskip\textbf{Trama Ipnotica}\index{Incantesimi - Trama Ipnotica}\\
\textbf{Scuola}: Illusione\\
\textbf{Difficoltà}: 21\\
\textbf{Tempo di Lancio}: 2 Azioni\\
\textbf{Gittata}: 36 metri\\
\textbf{Componenti}: S, M (un bastoncino luminoso di incenso o una fiala di cristallo piena di materiale fosforescente)\\
\textbf{Durata}: 1 minuto\\
Crei a gittata una trama contorta di colori che si muove nell'aria all'interno di un cubo di 9 metri di spigolo. La trama appare per un momento e poi svanisce. Ogni creatura nell'area che veda la trama deve effettuare un Tiro Salvezza su Volontà. Se fallisce il Tiro Salvezza, una creatura rimane affascinata per la durata. Mentre è affascinata da questo incantesimo, la creatura è inabile e ha velocità 0. L'incantesimo termina per la creatura soggetta, qualora questa subisca danni o se qualcuno usa un'azione per scuoterla dal suo stato confusionale.

\medskip\textbf{Trasformazione}\index{Incantesimi - Trasformazione}\\
\textbf{Scuola}: Trasmutazione\\
\textbf{Difficoltà}: 36\\
\textbf{Tempo di Lancio}: 2 Azioni\\
\textbf{Gittata}: Personale\\
\textbf{Componenti}: V, S, M (un cerchietto di giada del valore di almeno 1.500 mo, che devi poggiare sulla tua testa prima di lanciare l'incantesimo)\\
\textbf{Durata}: 1 ora\\
Per la durata assumi la forma di una creatura differente. La nuova forma può essere quella di qualsiasi creatura il cui grado di sfida sia pari o inferiore alla tua CM. La creatura non può essere un costrutto o un non morto, e devi averla vista almeno una volta. Ti trasformi in un esemplare medio di quella creatura, uno senza Abilità specifiche. Puoi restare nella forma assunta fino al termine dell'incantesimo. Ti ritrasformi automaticamente se cadi privo di sensi, scendi a 0 punti ferita o muori. Le tue statistiche di gioco sono rimpiazzate dalle statistiche della creatura scelta, fatta accezione per i tuoi Tratti, e dei tuoi punteggi di Intelligenza, Saggezza e Carisma. Mantieni tutte le tue competenze nelle abilità e i Tiri Salvezza, oltre a ottenere quelle della creatura. Se la creatura possiede le tue stesse competenze e il bonus indicato nelle sue statistiche è più alto del tuo, usa il bonus della creatura al posto del tuo. Non puoi usare nessuna azione aggiuntiva o azione da tana della nuova forma.\\
Quando ti trasformi, assumi i punti ferita e i Dadi Vita della creatura. Quando ritorni alla tua forma normale, ritorni al numero di punti ferita che avevi prima di trasformarti. Tuttavia, se ti ritrasformi perché sei stato ridotto a 0 punti ferita, tutto il danno in eccesso viene riportato alla tua forma originale. A meno che il danno in eccesso non riduca la tua forma normale a 0 punti ferita, non cadrai privo di sensi. \\
Mantieni tutti i benefici di qualsiasi Abilità possedessi, razza, o altra fonte e puoi usarli se la nuova forma è fisicamente capace di farne uso. Tuttavia, non puoi usare nessuno dei tuoi sensi speciali, come la scurovisione, a meno che la nuova forma non possieda anch'essa lo stesso senso. Puoi parlare solo se la creatura è normalmente in grado di parlare.\\
Quando ti trasformi scegli se il tuo equipaggiamento cade a terra nel tuo spazio, si fonde con la nuova forma o sia indossato da essa. L'equipaggiamento indossato funziona come di norma, ma sta al Narratore decidere se sia comodo per la nuova forma indossare un simile pezzo di equipaggiamento, in base alla taglia e le dimensioni della creatura. Il tuo equipaggiamento non cambia dimensioni né si adatta alla nuova forma, e qualsiasi equipaggiamento che la nuova forma non può indossare deve essere fatto cadere a terra o fondersi con la nuova forma. L'equipaggiamento che si fonde è inefficace.\\
Nella durata dell'incantesimo, puoi usare due azioni per assumere una forma diversa seguendo le stesse restrizioni e regole della forma originale, con una eccezione: se la tua nuova forma ha più punti ferita della forma attuale, i tuoi punti ferita restano al livello attuale.

\medskip\textbf{Traslazione Arborea}\index{Incantesimi - Traslazione Arborea}\\
\textbf{Scuola}: Evocazione\\
\textbf{Difficoltà}: 26\\
\textbf{Tempo di Lancio}: 2 Azioni\\
\textbf{Gittata}: Personale\\
\textbf{Componenti}: V, S\\
\textbf{Durata}: massimo 1 minuto\\
Ottieni la capacità di entrare in un albero e muoverti dal suo interno all'interno di un altro albero della stessa specie entro 150 metri. Entrambi gli alberi devono essere vivi e almeno della tua stessa taglia. Devi usare 1 metro di movimento per entrare nell'albero. Apprendi istantaneamente la posizione di tutti gli altri alberi della stessa specie entro 150 metri e, come parte del movimento impiegato per entrare nell'albero, puoi passare in uno degli altri alberi o uscire dall'albero in cui sei entrato. Riappari in un punto a tua scelta entro 1 metri dall'albero di destinazione, utilizzando altri 1 Azione di movimento. Se non ti rimane movimento da usare, riappari entro 1 metro dall'albero in cui sei entrato.\\
Per la durata dell'incantesimo puoi usare questa capacità di trasporto una volta per round. Devi terminare ogni round al di fuori di un albero.

\medskip\textbf{Trasporto Vegetale}\index{Incantesimi - Trasporto Vegetale}\\
\textbf{Scuola}: Evocazione\\
\textbf{Difficoltà}: 29\\
\textbf{Tempo di Lancio}: 2 Azioni\\
\textbf{Gittata}: 3 metri\\
\textbf{Componenti}: V, S\\
\textbf{Durata}: 1 round\\
Questo incantesimo crea un legame magico tra un vegetale inanimato di taglia Grande o maggiore a gittata e un altro vegetale, a qualsiasi distanza, sullo stesso piano di esistenza. Devi aver visto o essere entrato in contatto almeno una volta con il vegetale di destinazione. Per la durata dell'incantesimo, qualsiasi creatura può entrare nel vegetale bersaglio e uscire dal vegetale di destinazione usando 1 azione di movimento.

\medskip\textbf{Trova Cavalcatura}\index{Incantesimi - Trova Cavalcatura}\\
\textbf{Scuola}: Evocazione\\
\textbf{Difficoltà}: 19\\
\textbf{Tempo di Lancio}: 10 minuti\\
\textbf{Gittata}: 9 metri\\
\textbf{Componenti}: V, S\\
\textbf{Durata}: Istantanea\\
Evochi uno spirito che assume la forma di una cavalcatura insolitamente intelligente, forte e leale, stabilendo un legame duraturo con esso. Apparendo in uno spazio a gittata, non occupato, il destriero assume la forma di tua scelta, come quella di un cavallo da guerra, un pony, un cammello, un alce o un mastino (il Narratore potrebbe darti la possibilità di evocare come destrieri anche altri tipi di animali). Il destriero ha le statistiche della forma scelta, sebbene sia di tipo celestiale, fatato o demone (a tua scelta) invece del suo normale tipo. Inoltre, se il tuo destriero ha Intelligenza -3 o meno, la sua Intelligenza diventa -2, e ottiene la capacità di comprendere un linguaggio a tua scelta tra quelli che sei in grado di parlare. Il tuo destriero serve da cavalcatura, sia in combattimento che fuori da esso, e possiedi un legame istintivo con esso, che vi permette di combattere come foste un unico insieme. Mentre sei in groppa alla tua cavalcatura, puoi far sì che qualsiasi incantesimo che lanci e che prenda come bersaglio solo te, prenda come bersaglio anche il tuo destriero.\\
Quando il destriero scende a 0 punti ferita, scompare, non lasciandosi dietro alcuna forma fisica. puoi congedare il destriero in qualsiasi momento con un'azione, facendolo sparire. In entrambi i casi, lanciare di nuovo questo incantesimo evoca lo stesso destriero, ripristinato al massimo dei suoi punti ferita. Mentre il tuo destriero si trova entro 1,5 chilometri da te, puoi comunicare telepaticamente con esso.\\
Non puoi avere più di un destriero legato da questo incantesimo alla volta. Con un'azione, puoi liberare il destriero da questo legame in qualsiasi momento, facendolo sparire.

\medskip\textbf{Trucco della Corda}\index{Incantesimi - Trucco della Corda}\\
\textbf{Scuola}: Trasmutazione\\
\textbf{Difficoltà}: 19\\
\textbf{Tempo di Lancio}: 1 minuto\\
\textbf{Gittata}: Contatto\\
\textbf{Componenti}: V, S, M (estratto di grano in polvere e un laccio di pergamena)\\
\textbf{Durata}: 1 ora\\
Entri a contatto con un pezzo di corda lungo fino a 18 metri. Un'estremità della corda si leva nell'aria finché la corda non pende perpendicolare al terreno. All'estremità opposta della corda, un'entrata invisibile si apre su di uno spazio extradimensionale che resta fino al termine dell'incantesimo \\
Lo spazio extradimensionale può essere raggiunto arrampicandosi fino alla cima della corda. Lo spazio può contenere fino a otto creature di taglia Media o inferiore. La corda può essere trascinata nello spazio, facendola sparire dalla vista di chi è fuori di esso.\\
Attacchi e incantesimi non possono attraversare l'ingresso in entrata o uscita dallo spazio extradimensionale, ma chi si trova al suo interno può vedere fuori come se vedesse attraverso una finestra di 1 x 1 metro centrata sulla corda.\\
Qualsiasi cosa si trovi nello spazio extradimensionale ne cade fuori al termine dell'incantesimo

\medskip\textbf{Unto}\index{Incantesimi - Unto}\\
\textbf{Scuola}: Evocazione\\
\textbf{Difficoltà}: 16\\
\textbf{Tempo di Lancio}: 2 Azioni\\
\textbf{Gittata}: 18 metri\\
\textbf{Componenti}: V, S, M (un pezzo di cotenna di maiale o burro o unto topetto)\\
\textbf{Durata}: 1 minuto\\
Grasso scivoloso ricopre il terreno in un quadrato di 3 metri di lato, centrato su di un punto a gittata, e lo trasforma in terreno difficile per la durata dell'incantesimo\\
Quando compare il grasso, ciascun bersaglio che si trova in piedi nell'area deve superare un Tiro Salvezza su Riflessi o cadere prono. Una creatura che entra nell'area o termina il suo round lì, deve superare un Tiro Salvezza su Riflessi o cadere prona.\\

\medskip\textbf{Vedere Invisibilità}\index{Incantesimi - Vedere Invisibilità}\\
\textbf{Scuola}: Divinazione\\
\textbf{Difficoltà}: 19\\
\textbf{Tempo di Lancio}: 2 Azioni\\
\textbf{Gittata}: Personale\\
\textbf{Componenti}: V, S, M (un pizzico di talco e una manciata di polvere d'argento)\\
\textbf{Durata}: 1 ora\\
Per la durata dell'incantesimo, vedi le creature e gli oggetti invisibili come se fossero visibili, e inoltre puoi vedere nel Piano Etereo. Le creature e gli oggetti eterei ti appaiono spettrali e trasparenti.

\medskip\textbf{Velocità}\index{Incantesimi - Velocità}\\
\textbf{Scuola}: Trasmutazione\\
\textbf{Difficoltà}: 21\\
\textbf{Tempo di Lancio}: 2 Azioni\\
\textbf{Gittata}: 9 metri\\
\textbf{Componenti}: V, S, M (una grattata di radice di liquirizia)\\
\textbf{Durata}: 1 minuto\\
Scegli una creatura consenziente a gittata e che puoi vedere. Fino al termine dell'incantesimo, la velocità del bersaglio è raddoppiata, ottiene un bonus di +2 alla Difesa, ha +1d6 ai Tiri Salvezza su Destrezza, e ottiene un'Azione aggiuntiva durante ciascun suo round.\\
Quest'azione può essere impiegata solo per effettuare un Azione di Attacco, di Movimento oppure o Usare un Oggetto.\\
Questo incantesimo contrasta ed e' contrastato da \hyperlink{lentezza}{Lentezza}.\\
Quando l'incantesimo termina, il bersaglio non può muoversi o effettuare azioni fino al suo prossimo round, mentre è pervaso da un'improvvisa sonnolenza.

\medskip\textbf{Vigilanza e Interdizione}\index{Incantesimi - Vigilanza e Interdizione}\\
\textbf{Scuola}: Abiurazione\\
\textbf{Difficoltà}: 29\\
\textbf{Tempo di Lancio}: 10 minuti\\
\textbf{Gittata}: Contatto\\
\textbf{Componenti}: V, S, M (incenso bruciato, un piccolo misurino di zolfo e olio, un laccio legato, un piccolo ammontare di sangue di colosso di terra, e una piccola verga d'argento del valore di almeno 10 mo)\\
\textbf{Durata}: 24 ore\\
Crei una interdizione che protegge fino a 225 metri quadri di pavimento (un'area quadrata di 15 metri di lato, o cento quadrati di 1 metro di lato o venticinque quadrati di 3 metri di lato). L'area interdetta può essere alta fino a 6 metri, e modellata come preferisci. Puoi interdire diversi piani di una roccaforte dividendo l'area tra di essi, purché tu possa camminare ininterrottamente in ogni area adiacente, mentre lanci l'incantesimo\\
Quando lanci questo incantesimo, puoi specificare gli individui che ignorano qualcuno o tutti gli effetti di questo incantesimo. Puoi anche specificare una parola d'ordine che, pronunciata ad alta voce, rende chi la proferisce immune a questi effetti.\\
Vigilanza e interdizione crea i seguenti effetti all'interno dell'area interdetta.\\
\textit{Corridoi}. La nebbia riempie tutti i corridoi interdetti, rendendoli oscurati pesantemente. Inoltre, a ogni intersezione o biforcazione del passaggio che offre una scelta di direzione, c'è una probabilità del 50\% che una creatura, escluso te, creda di stare andando nella direzione opposta a quella che ha scelto.\\
\textit{Porte}. Tutte le porte nell'area interdetta sono chiuse magicamente, come se fossero sigillate dall'incantesimo serratura arcana. Inoltre, puoi coprire fino a dieci porte con un'illusione (equivalente della funzione oggetto illusorio dell'incantesimo illusione minore) per farle sembrare delle semplici sezioni di muro.\\
\textit{Scale}. Ragnatele ricoprono da cima a fondo tutte le scale nell'area interdetta, come per l'incantesimo ragnatela. Questi fili ricrescono in 10 minuti se vengono bruciati o strappati mentre vigilanza e interdizione resta attivo.\\
Altri Incantesimi in Effetto. Puoi piazzare uno dei seguenti effetti magici di tua scelta all'interno dell'area interdetta dell'edificio
\medskip
\begin{itemize}
\item
Piazza luci danzanti in quattro corridoi. Puoi indicare un semplice programma che le luci ripeteranno per la durata di vigilanza e interdizione.
\item
Piazza bocca magica in due posti.
\item
Piazza nube maleodorante in due posti. I vapori appaiono nel posto da te indicato; ritornano entro 10 minuti se dispersi dal vento mentre vigilanza e interdizione è ancora attivo.
\item
Piazza una folata di vento costante in un corridoio o stanza.
\item
Piazza una suggestione in un luogo. Seleziona un'area quadrata di 1 metro di lato, e qualsiasi creatura che entra o passa attraverso quell'area riceve mentalmente la suggestione.
\end{itemize}
\medskip
L'intera area interdetta irradia magia. Un incantesimo dissolvi magie lanciato contro uno specifico effetto, se riesce, rimuove solo quell'effetto Puoi creare una struttura perennemente vigilata e interdetta lanciandovi questo incantesimo ogni giorno per un anno.\\
\textbf{Se effettui tre critici} la durata e' permanente.

\medskip\textbf{Vincolo di Interdizione}\index{Incantesimi - Vincolo di Interdizione}\\
\textbf{Scuola}: Abiurazione\\
\textbf{Difficoltà}: 19\\
\textbf{Tempo di Lancio}: 2 Azioni\\
\textbf{Gittata}: Contatto\\
\textbf{Componenti}: V, S, M (una coppia di anelli di platino del valore di 50 mo l'uno, che tu e il bersaglio dovete indossare per la durata)\\
\textbf{Durata}: 1 ora\\
Lanci l'incantesimo a contatto di una creatura che vuoi proteggere. Crei una connessione mistica tra di te e il bersaglio fino al termine dell'incantesimo. Finché il bersaglio è entro 18 metri da te, ottiene un bonus di +1 alla Difesa e ai Tiri Salvezza e ha resistenza a tutti i danni. Inoltre, ogni volta che il bersaglio subisce danni, tu ne subisci la stessa quantità. L'incantesimo ha fine se scendi a 0 punti ferita o tu e il bersaglio vi allontanate più di 18 metri. Ha fine anche se lo lanci di nuovo sulla stessa creatura su cui è già in atto. Puoi interrompere l'incantesimo con un'azione.

\medskip\textbf{Visione del Vero}\index{Incantesimi - Visione del Vero}\\
\textbf{Scuola}: Divinazione\\
\textbf{Difficoltà}: 29\\
\textbf{Tempo di Lancio}: 2 Azioni\\
\textbf{Gittata}: Contatto\\
\textbf{Componenti}: V, S, M (un unguento per gli occhi che costa 25 mo; fatto di funghi in polvere, zafferano e grasso; viene consumato dall'incantesimo)\\
\textbf{Durata}: 1 ora\\
Lanci l'incantesimo a contatto di una creatura consenziente. Il bersaglio riceve la capacità di vedere le cose come sono realmente. Per la durata dell'incantesimo, la creatura ha visione del vero, nota porte segrete nascoste dalla magia, e può vedere nel Piano Etereo, fino a una gittata di 36 metri.

\medskip\textbf{Vita Falsata}\index{Incantesimi - Vita Falsata}\\
\textbf{Scuola}: Necromanzia\\
\textbf{Difficoltà}: 16\\
\textbf{Tempo di Lancio}: 2 Azioni\\
\textbf{Gittata}: Personale\\
\textbf{Componenti}: V, S, M (un piccolo ammontare di alcool o spirito distillato)\\
\textbf{Durata}: 1 ora\\
Potenziandoti con una parvenza necromantica di vitalità, ottieni 1d4 + 4 punti ferita temporanei per la durata.\\
\textbf{Per ogni critico ottenuto} nella prova di magia ottieni 5 punti ferita temporanei.

\medskip\textbf{Volare}\index{Incantesimi - Volare}\\
\textbf{Scuola}: Trasmutazione\\
\textbf{Difficoltà}: 21\\
\textbf{Tempo di Lancio}: 2 Azioni\\
\textbf{Gittata}: Contatto\\
\textbf{Componenti}: V, S, M (una piuma dell'ala di qualsiasi volatile)\\
\textbf{Durata}: 10 minuti \\
Lanci l'incantesimo a contatto di una creatura consenziente. Per la durata dell'incantesimo, il bersaglio ottiene velocità di volo 18 metri. Quando l'incantesimo ha fine, qualora sia ancora in aria, il bersaglio cade, a meno che non riesca a frenare la discesa.\\
Lanciare un incantesimo mentre si vola e' piu' complesso, si e' Distratti. La Difficoltà aumenta di 5.\\
\textbf{Per ogni critico ottenuto} nella prova di magia puoi prendere come bersaglio un'ulteriore creatura oppure raddoppiare la durata.

\medskip\textbf{Vuoto Mentale}\index{Incantesimi - Vuoto Mentale}\\
\textbf{Scuola}: Abiurazione\\
\textbf{Difficoltà}: 34\\
\textbf{Tempo di Lancio}: 2 Azioni\\
\textbf{Gittata}: Contatto\\
\textbf{Componenti}: V, S\\
\textbf{Durata}: 24 ore\\
Fino al termine dell'incantesimo, una creatura consenziente con cui sei in contatto durante il lancio è immune al danno psichico, qualsiasi effetto che ne percepirebbe le emozioni o leggerebbe i pensieri, incantesimi di divinazione e la condizione affascinato. l'incantesimo nega anche gli incantesimi desiderio e altri incantesimi o effetti di simili potenza impiegati per
influenzare la mente del bersaglio o per ottenere informazioni su di esso.\\
\textbf{Per ogni critico ottenuto} nella prova di magia la durata raddoppia. Se ottieni tre critici la durata e' permanente.

\medskip\textbf{Zona di Verità}\index{Incantesimi - Zona di Verità}\\
\textbf{Scuola}: Ammaliamento\\
\textbf{Difficoltà}: 19\\
\textbf{Tempo di Lancio}: 2 Azioni\\
\textbf{Gittata}: 18 metri\\
\textbf{Componenti}: V, S\\
\textbf{Durata}: 10 minuti\\
Crei una zona magica che protegge contro i raggiri in una sfera di 4 metri di raggio centrata su di un punto a gittata di tua scelta. Fino al termine dell'incantesimo, una creatura che entra nell'area dell'incantesimo per la prima volta durante un round, o inizia il suo round al suo interno, deve effettuare un Tiro Salvezza su Volontà. Se fallisce il Tiro Salvezza, la creatura non può pronunciare bugie deliberatamente mentre è nel raggio dell'incantesimo. Sei a conoscenza se una creatura ha superato o fallito il Tiro Salvezza. Una creatura soggetta all'incantesimo ne è consapevole e può quindi evitare di rispondere a domande a cui risponderebbe normalmente con una bugia. Questa creatura può dare risposte elusive purché rimanga entro i confini della verità.

\end{multicols}

\pagebreak

\begin{multicols}{3}
	
	
	\subsection{Lista degli Incantesimi divisi per Difficoltà}
	
	\textbf{Legenda}: Nome Incantesimo (Scuola di Magia di appartenenza)\\
	
	\subsubsection{Trucchetti - Difficoltà 12}
	Artificio Druidico (Universale)\\
	Beffa Crudele (Ammaliamento)\\
	Colpo Accurato (Divinazione)\\
	Creare Birra (Evocazione)\\
	Deflagrazione Occulta (Invocazione)\\
	Dito (Ammaliamento)\\
	Fiamma Sacra (Invocazione)\\
	Fiotto Acido (Evocazione)\\
	Luci Danzanti (Invocazione)\\
	Mano Magica (Evocazione)\\
	Messaggio (Trasmutazione)\\
	Prestidigitazione (Universale)\\
	Produrre Fiamma (Evocazione)\\
	Raggio di Gelo (Invocazione)\\
	Randello Incantato (Trasmutazione)\\
	Resistenza (Abiurazione)\\
	Riparare (Trasmutazione)\\
	Spruzzo Velenoso (Evocazione)\\
	Stretta Folgorante (Invocazione)\\
	Taumaturgia (Universale)\\
	Tocco Gelido (Necromanzia)\\
	
	\subsubsection{Difficoltà 16}
	Allarme (Abiurazione)\\
	Amicizia con gli Animali (Ammaliamento)\\
	Anatema (Ammaliamento)\\
	Armatura Magica (Abiurazione)\\
	Benedizione (Invocazione)\\
	Caduta Morbida (Trasmutazione)\\
	Charme su Persone (Ammaliamento)\\
	Camuffare Sé Stesso (Illusione)\\
	Comando (Ammaliamento)\\
	Comprensione dei Linguaggi (Divinazione)\\
	Creare o Distruggere Acqua (Trasmutazione)\\
	Cuoco Invisibile (Evocazione)\\
	Cura Ferite Leggere (Cura)\\
	Dardo di Fuoco (Invocazione)\\
	Dardo Tracciante (Invocazione)\\
	Dardo Incantato (Invocazione)\\
	Disco Fluttuante (Evocazione)\\
	Eroismo (Ammaliamento)\\
	Favore Divino (Invocazione)\\
	Guida (Divinazione)\\
	Identificare (Divinazione)\\
	Illusione Minore (Illusione)\\
	Immagine Silenziosa (Illusione)\\
	Individuazione del Bene e del Male (Divinazione)\\
	Individuazione del Magico (Divinazione)\\
	Intimorire Infernale (Evocazione)\\
	Intralciare (Invocazione)\\
	Luce (Invocazione)\\
	Luminescenza (Invocazione)\\
	Mani Brucianti (Invocazione)\\
	Nube di Nebbia (Evocazione)\\
	Onda Tonante (Invocazione)\\
	Oscurità (Invocazione)\\
	Parlare con gli Animali (Divinazione)\\
	Parola Guaritrice (Cura)\\
	Passo Veloce (Trasmutazione)\\
	Protezione dal Bene e dal Male (Abiurazione)\\
	Purificare Cibo e Bevande (Trasmutazione)\\
	Risata Incontenibile (Ammaliamento)\\
	Ritirata Rapida (Trasmutazione)\\
	Saltare (Trasmutazione)\\
	Santuario (Abiurazione)\\
	Scritto Illusorio (Illusione)\\
	Scudo (Abiurazione)\\
	Scudo della Fede (Abiurazione)\\
	Servitore Invisibile (Evocazione)\\
	Sonno (Ammaliamento)\\
	Spruzzo Colorato (Illusione)\\
	Trova Famiglio (Evocazione)\\
	Unto (Evocazione)\\
	Vita Falsata (Necromanzia)\\
	
	\subsubsection{Difficoltà 19}
	Aiuto (Abiurazione)\\
	Alterare Sé Stesso (Trasmutazione)\\
	Animale Messaggero (Ammaliamento)\\
	Arma Magica (Trasmutazione)\\
	Arma Spirituale (Invocazione)\\
	Aura Magica dell'Arcanista (Illusione)\\
	Bacche Benefiche (Trasmutazione)\\
	Benedici Acqua (Invocazione)\\
	Benedizione Superiore (Invocazione)\\
	Blocca Persona (Ammaliamento)\\
	Bocca Magica (Illusione)\\
	Calmare Emozioni (Ammaliamento)\\
	Caratteristica Potenziata (Trasmutazione)\\
	Cecità/Sordità (Necromanzia)\\
	Comprensione degli Scritti (Divinazione)\\
	Crescita di Spuntoni (Trasmutazione)\\
	Estasiare (Ammaliamento)\\
	Fiamma Perenne (Invocazione)\\
	Folata di Vento (Invocazione)\\
	Frantumare (Invocazione)\\
	Freccia Acida (Invocazione)\\
	Immagine Speculare (Illusione)\\
	Individuazione delle Malattie e dei Veleni (Divinazione)\\
	Individuazione dei Pensieri (Divinazione)\\
	Infliggi Ferite (Necromanzia)\\
	Ingrandire/Ridurre (Trasmutazione)\\
	Invisibilità (Illusione)\\
	Lama Infuocata (Invocazione)\\
	Levitazione (Trasmutazione)\\
	Localizza Animali e Piante (Divinazione)\\
	Localizza Oggetto (Divinazione)\\
	Movimenti del Ragno (Trasmutazione)\\
	Passare Senza Tracce (Abiurazione)\\
	Passo Velato (Evocazione)\\
	Pelle di Corteccia (Trasmutazione)\\
	Preghiera di Guarigione (Cura)\\
	Presagio (Divinazione)\\
	Protezione dai Veleni (Abiurazione)\\
	Punizione Marchiante (Invocazione)\\
	Raggio di Affaticamento (Necromanzia)\\
	Raggio Rovente (Invocazione)\\
	Ragnatela (Evocazione)\\
	Riposo Inviolato (Necromanzia)\\
	Riscaldare il Metallo (Trasmutazione)\\
	Ristorare Inferiore (Cura)\\
	Scassinare (Trasmutazione)\\
	Scopri Trappole (Divinazione)\\
	Scurovisione (Trasmutazione)\\
	Serratura Arcana (Abiurazione)\\
	Sfera Infuocata (Evocazione)\\
	Sfocatura (Illusione)\\
	Silenzio (Illusione)\\
	Suggestione (Ammaliamento)\\
	Trova Cavalcatura (Evocazione)\\
	Trucco della Corda (Trasmutazione)\\
	Vedere Invisibilità (Divinazione)\\
	Vincolo di Interdizione (Abiurazione)\\
	Zona di Verità (Ammaliamento)\\
	
	\subsubsection{Difficoltà 21}
	Animare Morti (Necromanzia)\\
	Anti-Individuazione (Abiurazione)\\
	Benedizione Suprema (Invocazione)\\
	Camminare sull'Acqua (Trasmutazione)\\
	Capanna (Invocazione)\\
	Cecità/Sordità Avanzata (Necromanzia)\\
	Cerchio Magico (Abiurazione)\\
	Chiaroveggenza (Divinazione)\\
	Controincantesimo (Abiurazione)\\
	Creare Cibo e Acqua (Evocazione)\\
	Crescita Vegetale (Trasmutazione)\\
	Cura Ferite Serie (Cura)\\
	Cura Ferite di Massa (Cura)\\
	Destriero Fantasma (Illusione)\\
	Dissolvi Magie (Abiurazione)\\
	Evoca Animali (Evocazione)\\
	Faro di Speranza (Abiurazione)\\
	Fondersi nella Pietra (Trasmutazione)\\
	Forma Gassosa (Trasmutazione)\\
	Fulmine (Invocazione)\\
	Glifo di Interdizione (Abiurazione)\\
	Immagine Maggiore (Illusione)\\
	Intermittenza (Trasmutazione)\\
	Inviare (Invocazione)\\
	Invocare il Fulmine (Evocazione)\\
	Lentezza (Trasmutazione)\\
	Lingue (Divinazione)\\
	Luce Diurna (Invocazione)\\
	Muro di Vento (Invocazione)\\
	Nube Maleodorante (Evocazione)\\
	Palla di Fuoco (Invocazione)\\
	Parlare con i Morti (Necromanzia)\\
	Parlare con le Piante (Trasmutazione)\\
	Parola Guaritrice di Massa (Cura)\\
	Paura (Illusione)\\
	Protezione dall'Energia (Abiurazione)\\
	Respirare Sott'Acqua (Trasmutazione)\\
	Rimuovi Maledizione (Abiurazione)\\
	Rimuovi Veleno (Cura)\\
	Rinascita (Cura)\\
	Scagliare Maledizione (Necromanzia)\\
	Tempesta di Nevischio (Evocazione)\\
	Tocco Vampirico (Necromanzia)\\
	Trama Ipnotica (Illusione)\\
	Velocità (Trasmutazione)\\
	Volare (Trasmutazione)\\
	
	\subsubsection{Difficoltà 23}
	Allucinazione Mortale (Illusione)\\
	Blocca Persona Avanzato (Ammaliamento)\\
	Compulsione (Ammaliamento)\\
	Confusione (Ammaliamento)\\
	Controllare Acqua (Trasmutazione)\\
	Dominare Bestie (Ammaliamento)\\
	Esilio (Abiurazione)\\
	Evoca Creature Boschive (Evocazione)\\
	Evoca Elementali Minori (Evocazione)\\
	Fabbricare (Trasmutazione)\\
	Inaridire (Necromanzia)\\
	Insetto Gigante (Trasmutazione)\\
	Interdizione alla Morte (Abiurazione)\\
	Invisibilità Superiore (Illusione)\\
	Libertà di Movimento (Abiurazione)\\
	Localizza Creatura (Divinazione)\\
	Metamorfosi (Trasmutazione)\\
	Muro di Fuoco (Invocazione)\\
	Occhio Arcano (Divinazione)\\
	Pelle di Pietra (Abiurazione)\\
	Porta Dimensionale (Evocazione)\\
	Rimuovi Malattia (Cura)\\
	Santuario Privato (Abiurazione)\\
	Scolpire Pietra (Trasmutazione)\\
	Scrigno Segreto (Evocazione)\\
	Scudo di Fuoco (Invocazione)\\
	Segugio Fedele (Evocazione)\\
	Sfera Elastica (Invocazione)\\
	Tempesta di Ghiaccio (Invocazione)\\
	Tentacoli Neri (Evocazione)\\
	Terreno Illusorio (Illusione)\\
	
	\subsubsection{Difficoltà 26}
	Animare Oggetti (Trasmutazione)\\
	Blocca Mostri (Ammaliamento)\\
	Cerchio di Teletrasporto (Evocazione)\\
	Colpo Infuocato (Invocazione)\\
	Comunione (Divinazione)\\
	Comunione con la Natura (Divinazione)\\
	Cono di Freddo (Invocazione)\\
	Conoscenza delle Leggende (Divinazione)\\
	Contagio (Necromanzia)\\
	Costrizione (Ammaliamento)\\
	Creazione (Illusione)\\
	Cura Ferite Critiche (Cura)\\
	Dissolvi il Bene e il Male (Abiurazione)\\
	Dominare Persone (Ammaliamento)\\
	Evoca Elementale (Evocazione)\\
	Fuorviare (Illusione)\\
	Guscio Anti-Vita (Abiurazione)\\
	Legame Telepatico (Divinazione)\\
	Mano Arcana (Invocazione)\\
	Modificare Memoria (Ammaliamento)\\
	Muro di Forza (Invocazione)\\
	Muro di Pietra (Invocazione)\\
	Nube Mortale (Evocazione)\\
	Passapareti (Trasmutazione)\\
	Piaga degli Insetti (Evocazione)\\
	Reincarnazione (Trasmutazione)\\
	Rianimare Morti (Necromanzia)\\
	Ristorare Superiore (Cura)\\
	Risveglio (Trasmutazione)\\
	Santificare (Invocazione)\\
	Scrutare (Divinazione)\\
	Sembrare (Illusione)\\
	Sogno (Illusione)\\
	Telecinesi (Trasmutazione)\\
	Traslazione Arborea (Evocazione)\\
	
	\subsubsection{Difficoltà 29}
	Alleato Planare (Evocazione)\\
	Bagliore Solare (Invocazione)\\
	Banchetto degli Eroi (Evocazione)\\
	Barriera di Lame (Invocazione)\\
	Camminare nel Vento (Trasmutazione)\\
	Carne in Pietra - Pietra in Carne (Trasmutazione)\\
	Catena di Fulmini (Invocazione)\\
	Cerchio di Morte (Invocazione)\\
	Contingenza (Invocazione)\\
	Creare Non Morti (Necromanzia)\\
	Disintegrazione (Trasmutazione)\\
	Dito della Morte (Necromanzia)\\
	Divinazione (Divinazione)\\
	Evocazioni Istantanee (Evocazione)\\
	Ferire (Necromanzia)\\
	Giara Magica (Necromanzia)\\
	Globo di Invulnerabilità (Abiurazione)\\
	Guarigione (Cura)\\
	Illusione Programmata (Illusione)\\
	Muovere il Terreno (Trasmutazione)\\
	Muro di Ghiaccio (Invocazione)\\
	Muro di Spine (Evocazione)\\
	Parola del Ritiro (Evocazione)\\
	Proibizione (Abiurazione)\\
	Scopri il Percorso (Divinazione)\\
	Sfera Congelante (Invocazione)\\
	Sguardo Penetrante (Necromanzia)\\
	Suggestione di Massa (Ammaliamento)\\
	Trasporto Vegetale (Evocazione)\\
	Vigilanza e Interdizione (Abiurazione)\\
	Visione del Vero (Divinazione)\\
	
	\subsubsection{Difficoltà 31}
	Celare (Trasmutazione)\\
	Evoca Celestiali (Evocazione)\\
	Forma Eterea (Trasmutazione)\\
	Immagine Proiettata (Illusione)\\
	Inversione della Gravità (Trasmutazione)\\
	Miraggio Arcano (Illusione)\\
	Palla di Fuoco Ritardata (Invocazione)\\
	Parola Divina (Invocazione)\\
	Reggia Meravigliosa (Evocazione)\\
	Resurrezione (Necromanzia)\\
	Rigenerazione (Trasmutazione)\\
	Simbolo (Abiurazione)\\
	Simulacro (Illusione)\\
	Spada Arcana (Invocazione)\\
	Spostamento Planare (Evocazione)\\
	Spruzzo Prismatico (Invocazione)\\
	Teletrasporto (Evocazione)\\
	Tempesta di Fuoco (Invocazione)\\
	
	\subsubsection{Difficoltà 34}
	Antipatia/Simpatia (Ammaliamento)\\
	Aura Sacra (Abiurazione)\\
	Campo Anti-Magia (Abiurazione)\\
	Clone (Necromanzia)\\
	Controllare Tempo Atmosferico (Trasmutazione)\\
	Danza Irresistibile (Ammaliamento)\\
	Dominare Mostri (Ammaliamento)\\
	Esplosione Solare (Invocazione)\\
	Forme Animali (Trasmutazione)\\
	Gabbia di Forza (Invocazione)\\
	Labirinto (Evocazione)\\
	Loquacità (Trasmutazione)\\
	Nube Incendiaria (Evocazione)\\
	Parola del Potere Stordire (Ammaliamento)\\
	Regressione Mentale (Ammaliamento)\\
	Semipiano (Evocazione)\\
	Terremoto (Invocazione)\\
	Vuoto Mentale (Abiurazione)\\
	
	\subsubsection{Difficoltà 36}
	Desiderio (Evocazione)\\
	Fatale (Illusione)\\
	Fermare il Tempo (Trasmutazione)\\
	Guarigione di Massa (Cura)\\
	Imprigionare (Abiurazione)\\
	Metamorfosi Pura (Trasmutazione)\\
	Muro Prismatico (Abiurazione)\\
	Parola del Potere Uccidere (Ammaliamento)\\
	Portale (Evocazione)\\
	Previsione (Divinazione)\\
	Proiezione Astrale (Necromanzia)\\
	Resurrezione Pura (Trasmutazione)\\
	Sciame di Meteore (Invocazione)\\
	Trasformazione (Trasmutazione)\\
	
	\subsection{Lista degli Incantesimi divisi per Scuola}
	
	\textbf{Legenda}: Nome Incantesimo (Difficoltà di lancio)\\
	
	\subsubsection{Abiurazione}
	Aiuto (19)\\
	Allarme	(10)\\
	Anti-Individuazione (21)\\
	Armatura Magica (16)\\
	Aura Sacra (34)\\
	Campo Anti-Magia (34)\\
	Cerchio Magico (21)\\
	Controincantesimo (21)\\
	Dissolvi il Bene e il Male (26)\\
	Dissolvi Magie (21)\\
	Esilio (23)\\
	Faro di Speranza (21)\\
	Glifo di Interdizione (21)\\
	Globo di Invulnerabilità (29)\\
	Guscio Anti-Vita (26)\\
	Imprigionare (36)\\
	Interdizione alla Morte (23)\\
	Libertà di Movimento (23)\\
	Muro Prismatico (36)\\
	Passare Senza Tracce (19)\\
	Pelle di Pietra (23)\\
	Proibizione (29)\\
	Protezione dal Bene e dal Male (16)\\
	Protezione dall'Energia (21)\\
	Protezione dai Veleni (19)\\
	Resistenza (12)\\
	Rimuovi Maledizione (21)\\
	Santuario (16)\\
	Santuario Privato (23)\\
	Scudo (16)\\
	Scudo della Fede (16)\\
	Serratura Arcana (19)\\
	Simbolo (31)\\
	Vigilanza e Interdizione (29)\\
	Vincolo di Interdizione (19)\\
	Vuoto Mentale (34)\\
	
	\subsubsection{Ammaliamento}
	Amicizia con gli Animali (10)	\\
	Anatema (16)\\
	Animale Messaggero (19)\\
	Antipatia/Simpatia (34)\\
	Beffa Crudele (12)\\
	Blocca Mostri (26)\\
	Blocca Persona (19)\\
	Blocca Persona Avanzato (23)\\
	Calmare Emozioni (19)\\
	Charme su Persone (16)\\
	Comando (16)\\
	Compulsione (23)\\
	Confusione (23)\\
	Costrizione (26)\\
	Danza Irresistibile (34)\\
	Dito (12)\\
	Dominare Bestie (23)\\
	Dominare Mostri (34)\\
	Dominare Persone (26)\\
	Eroismo (16)\\
	Estasiare (19)\\
	Modificare Memoria (26)\\
	Parola del Potere Stordire (34)\\
	Parola del Potere Uccidere (36)\\
	Regressione Mentale (34)\\
	Risata Incontenibile (16)\\
	Sonno (16)\\
	Suggestione (19)\\
	Suggestione di Massa (29)\\
	Zona di Verità (19)\\
	
	\subsubsection{Cura}
Cura Ferite Critiche (26)\\
Cura Ferite di Massa (variabile)\\
Cura Ferite Leggere (16)\\
Cura Ferite Serie (21)\\
Guarigione (29)\\
Guarigione di Massa (36)\\
Parola Guaritrice (16)\\
Parola Guaritrice di Massa (21)\\
Preghiera di Guarigione (19)\\
Rimuovi Malattia (23)\\
Rimuovi Veleno (21)\\
Rinascita (21)\\
Ristorare Inferiore (19)\\
Ristorare Superiore (26)\\	
	
	\subsubsection{Divinazione}
	Chiaroveggenza (21)\\
	Colpo Accurato (12)\\
	Comprensione dei Linguaggi (16)\\
	Comprensione degli Scritti (19)\\
	Comunione (26)\\
	Comunione con la Natura (26)\\
	Conoscenza delle Leggende (26)\\
	Divinazione (29)\\
	Guida (16)\\
	Identificare (16)\\
	Individuazione del Bene e del Male (16)\\
	Individuazione del Magico (16)\\
	Individuazione delle Malattie e dei Veleni (19)\\
	Individuazione dei Pensieri (19)\\
	Legame Telepatico (26)\\
	Lingue (21)\\
	Localizza Animali e Piante (19)\\
	Localizza Creatura (23)\\
	Localizza Oggetto (19)\\
	Occhio Arcano (23)\\
	Parlare con gli Animali (16)\\
	Presagio (19)\\
	Previsione (36)\\
	Scopri il Percorso (29)\\
	Scopri Trappole (19)\\
	Scrutare (26)\\
	Vedere Invisibilità (19)\\
	Visione del Vero (29)\\
	
	\subsubsection{Evocazione}
	Alleato Planare (29)\\
	Cerchio di Teletrasporto (26)\\
	Creare Cibo e Acqua (21)\\
	Creare Birra (12)\\
	Cuoco Invisibile (16)\\
	Desiderio (36)\\
	Disco Fluttuante (16)\\
	Evoca Animali (21)\\
	Evoca Celestiali (31)\\
	Evoca Creature Boschive (23)\\
	Evoca Elementale (26)\\
	Evoca Elementali Minori (23)\\
	Evocazioni Istantanee (29)\\
	Fiotto Acido (12)\\
	Intimorire Infernale (16)\\
	Invocare il Fulmine (21)\\
	Labirinto (34)\\
	Mano Magica (12)\\
	Muro di Spine (29)\\
	Nube Incendiaria (34)\\
	Nube Maleodorante (21)\\
	Nube Mortale (26)\\
	Nube di Nebbia (16)\\
	Parola del Ritiro (29)\\
	Passo Velato (19)\\
	Piaga degli Insetti (26)\\
	Porta Dimensionale (23)\\
	Portale (36)\\
	Produrre Fiamma (12)\\
	Ragnatela (19)\\
	Reggia Meravigliosa (31)\\
	Scrigno Segreto (23)\\
	Segugio Fedele (23)\\
	Semipiano (34)\\
	Servitore Invisibile (16)\\
	Sfera Infuocata (19)\\
	Spostamento Planare (31)\\
	Spruzzo Velenoso (12)\\
	Teletrasporto (31)\\
	Tempesta di Nevischio (21)\\
	Tentacoli Neri (23)\\
	Traslazione Arborea (26)\\
	Trasporto Vegetale (29)\\
	Trova Cavalcatura (19)\\
	Trova Famiglio (16)\\
	Unto (16)\\
	
	\subsubsection{Illusione}	
	Allucinazione Mortale (23)\\
	Aura Magica dell'Arcanista (19)\\
	Bocca Magica (19)\\
	Camuffare Sé Stesso (16)\\
	Creazione (26)\\
	Destriero Fantasma (21)\\
	Fatale (36)\\
	Fuorviare (26)\\
	Illusione Minore (16)\\
	Illusione Programmata (29)\\
	Immagine Maggiore (21)\\
	Immagine Proiettata (31)\\
	Immagine Silenziosa (16)\\
	Immagine Speculare (19)\\
	Invisibilità (19)\\
	Invisibilità Superiore (23)\\
	Miraggio Arcano (31)\\
	Paura (21)\\
	Scritto Illusorio (16)\\
	Sembrare (26)\\
	Sfocatura (19)\\
	Silenzio (19)\\
	Simulacro (31)\\
	Sogno (26)\\
	Spruzzo Colorato (16)\\
	Terreno Illusorio (23)\\
	Trama Ipnotica (21)\\
	
	\subsubsection{Invocazione}	
	Arma Spirituale (19)\\
	Bagliore Solare (29)\\
	Banchetto degli Eroi (29)\\
	Barriera di Lame (29)\\
	Benedici Acqua (19)\\
	Benedizione (16)\\
	Benedizione Superiore (19)\\
	Benedizione Suprema (21)\\
	Capanna (21)\\
	Catena di Fulmini (29)\\
	Cerchio di Morte (29)\\
	Colpo Infuocato (26)\\
	Cono di Freddo (26)\\
	Contingenza (29)\\
	Dardo di Fuoco (16)\\
	Dardo Tracciante (16)\\
	Dardo Incantato (16)\\
	Deflagrazione Occulta (12)\\
	Esplosione Solare (34)\\
	Favore Divino (16)\\
	Fiamma Perenne (19)\\
	Fiamma Sacra (12)\\
	Folata di Vento (19)\\
	Frantumare (19)\\
	Freccia Acida (19)\\
	Fulmine (21)\\
	Gabbia di Forza (34)\\
	Intralciare (16)\\
	Inviare (21)\\
	Lama Infuocata (19)\\
	Luce (16)\\
	Luce Diurna (21)\\
	Luci Danzanti (12)\\
	Luminescenza (16)\\
	Mani Brucianti (16)\\
	Mano Arcana (26)\\
	Muro di Forza (26)\\
	Muro di Fuoco (23)\\
	Muro di Ghiaccio (29)\\
	Muro di Pietra (26)\\
	Muro di Vento (21)\\
	Onda Tonante (16)\\
	Oscurità (16)\\
	Palla di Fuoco (21)\\
	Palla di Fuoco Ritardata (31)\\
	Parola Divina (31)\\
	Punizione Marchiante (19)\\
	Raggio di Gelo (12)\\
	Raggio Rovente (19)\\
	Santificare (26)\\
	Sciame di Meteore (36)\\
	Scudo di Fuoco (23)\\
	Sfera Congelante (29)\\
	Sfera Elastica (23)\\
	Spada Arcana (31)\\
	Spruzzo Prismatico (31)\\
	Stretta Folgorante (12)\\
	Tempesta di Fuoco (31)\\
	Tempesta di Ghiaccio (23)\\
	Terremoto (34)\\
	
	\subsubsection{Necromanzia}
	Animare Morti (21)\\
	Cecità/Sordità (19)\\
	Cecità/Sordità Avanzata (21)\\
	Clone (34)\\
	Contagio (26)\\
	Creare Non Morti (29)\\
	Dito della Morte (29)\\
	Ferire (29)\\
	Giara Magica (29)\\
	Inaridire (23)\\
	Infliggi Ferite (19)\\
	Parlare con i Morti (21)\\
	Proiezione Astrale (36)\\
	Raggio di Affaticamento (19)\\
	Resurrezione (31)\\
	Rianimare Morti (26)\\
	Rinascita (21)\\
	Riposo Inviolato (19)\\
	Scagliare Maledizione (21)\\
	Sguardo Penetrante (29)\\
	Tocco Gelido (12)\\
	Tocco Vampirico (21)\\
	Vita Falsata (16)\\
	
	\subsubsection{Trasmutazione}
	Alterare Sé Stesso (13)		\\
	Animare Oggetti (26)\\
	Arma Magica (19)\\
	Bacche Benefiche (19)\\
	Caduta Morbida (16)\\
	Camminare sull'Acqua (21)\\
	Camminare nel Vento (29)\\
	Caratteristica Potenziata (19)\\
	Carne in Pietra - Pietra in Carne (29)\\
	Celare (31)\\
	Controllare Acqua (23)\\
	Controllare Tempo Atmosferico (34)\\
	Creare o Distruggere Acqua (16)\\
	Crescita di Spuntoni (19)\\
	Crescita Vegetale (21)\\
	Disintegrazione (29)\\
	Fabbricare (23)\\
	Fermare il Tempo (36)\\
	Fondersi nella Pietra (21)\\
	Forma Eterea (31)\\
	Forma Gassosa (21)\\
	Forme Animali (34)\\
	Ingrandire/Ridurre (19)\\
	Insetto Gigante (23)\\
	Intermittenza (21)\\
	Inversione della Gravità (31)\\
	Lentezza (21)\\
	Levitazione (19)\\
	Loquacità (34)\\
	Messaggio (12)\\
	Metamorfosi (23)\\
	Metamorfosi Pura (36)\\
	Movimenti del Ragno (19)\\
	Muovere il Terreno (29)\\
	Parlare con le Piante (21)\\
	Passapareti (26)\\
	Passo Veloce (16)\\
	Pelle di Corteccia (19)\\
	Purificare Cibo e Bevande (16)\\
	Randello Incantato (12)\\
	Reincarnazione (26)\\
	Respirare Sott'Acqua (21)\\
	Resurrezione Pura (36)\\
	Rigenerazione (31)\\
	Riparare (12)\\
	Riscaldare il Metallo (19)\\
	Risveglio (26)\\
	Ritirata Rapida (16)\\
	Saltare (16)\\
	Scassinare (19)\\
	Scolpire Pietra (23)\\
	Scurovisione (19)\\
	Telecinesi (26)\\
	Trasformazione (36)\\
	Trucco della Corda (19)\\
	Velocità (21)\\
	Volare (21)\\
	
	\subsubsection{Universale}
	Artificio Druidico (12)\\
	Prestidigitazione (12)\\
	Taumaturgia (12)\\
	
	
	
\end{multicols}

\pagebreak

\subsection{Incantesimi antichi e perduti}

Gli incantesimi qui presenti sono stati persi nella storia e solo leggende rimandano alla loro esistenza.\\
Questi incantesimi sono i più rari da poter trovare e non possono essere scelti alla creazione del personaggio.

\begin{multicols}{2}

\medskip\textbf{Alleato Planare}\index{Incantesimi - Alleato Planare}\\
\textbf{Scuola}: Evocazione\\
\textbf{Difficoltà}: 29\\
\textbf{Tempo di Lancio}: 10 minuti\\
\textbf{Gittata}: 18 metri\\
\textbf{Componenti}: V, S\\
\textbf{Durata}: Istantanea\\
Supplichi un'entità ultraterrena perché ti conceda aiuto. L'essere ti deve essere noto: un dio, un primordiale, un principe dei demoni, o qualche altra creatura di grande potere. Quell'entità invia un celestiale, elementale o demone a essa leale perché ti aiuti, facendo comparire la creatura in uno spazio non occupato a gittata. Se conosci il nome di una specifica creatura, puoi pronunciarne il nome quando lanci questo incantesimo per richiedere l'aiuto di quella creatura, sebbene tu possa comunque riceverne un'altra (a discrezione del Narratore).\\
Quando la creatura appare, non è sotto l'obbligo di agire in alcun modo particolare. Puoi chiedere alla creatura di svolgere un servizio in cambio di una ricompensa, ma essa non è obbligata a soddisfarti. Il compito richiesto potrebbe essere facile ("portaci in volo oltre il baratro" o "aiutaci a combattere questa battaglia") o complesso ("spia i nostri nemici" o "proteggici durante la nostra esplorazione del sotterraneo"). Devi essere in grado di comunicare con la creatura per patteggiare i suoi servigi. La ricompensa può assumere diverse forme. Un celestiale potrebbe chiedere una considerevole donazione di oro od oggetti magici a un tempio alleato, mentre un demone potrebbe richiedere un sacrificio umano o il dono di un tesoro. Alcune creature potrebbero scambiare i loro servigi per una missione che dovrai intraprendere per conto loro. Come regola generale, un compito che può essere misurato in minuti richiede una ricompensa di 100 mo al minuto. Un compito misurato in ore, richiede 1.000 mo all'ora. Un compito misurato in giorni (massimo 10 giorni) richiede 10.000 mo al giorno. Il Narratore può modificare queste ricompense in base alle circostanze nelle quali si è lanciato l'incantesimo Se il compito è allineato alla morale della creatura, la richiesta di pagamento potrebbe essere dimezzata o addirittura annullata. I compiti non pericolosi di solito chiedono solo la metà di quanto suggerito come pagamento, mentre i compiti molto pericolosi possono richiedere donazioni superiori. È raro che queste creature accettino compiti che sembrino suicida.\\
Dopo che la creatura ha completato il compito, o quando il periodo di servizio concordato è terminato, la creatura tornerà al suo piano natio dopo averti fatto rapporto, se appropriato al compito svolto e se possibile. Se non sei in grado di concordare un prezzo per i servigi della creatura, la creatura tornerà immediatamente al suo piano natio. Una creatura arruolata per unirsi al tuo gruppo è considerata come un suo membro, e riceve una quota piena delle ricompense in punti esperienza.

\medskip\textbf{Bagliore Lunare}\index{Incantesimi - Bagliore Lunare}\\
\textbf{Scuola}: Invocazione\\
\textbf{Difficoltà}: 19\\
\textbf{Tempo di Lancio}: 2 Azioni\\
\textbf{Gittata}: 36 metri\\
\textbf{Componenti}: V, S, M (diversi semi di bella di notte e un pezzo di felpato opalescente)\\
\textbf{Durata}: Concentrazione, massimo 1 minuto\\
Un fascio argenteo di luce pallida risplende in un cilindro di raggio 1 metro, alto 12 metri centrato in un punto a gittata. Fino al termine dell'incantesimo, una luce fioca riempie il cilindro. \\
Quando una creatura entra nell'area dell'incantesimo per la prima volta durante un round o inizia qui il suo round, è avvolta da fiamme spettrali che provocano un dolore terribile, e deve effettuare un Tiro Salvezza su Tempra. Se fallisce il Tiro Salvezza subisce 2d10 danni da Luce, o la metà di questi danni se lo supera. Un mutaforma effettua il Tiro Salvezza con -1d6. Se lo fallisce ritorna immediatamente alla sua forma originale e non può assumere una forma diversa finché non esce dalla luce dell'incantesimo.\\
Durante ciascun tuo round dopo aver lanciato l'incantesimo, puoi usare un'azione per muovere il
fascio di 18 metri in qualsiasi direzione. \\
\textbf{Per ogni Critico ottenuto} nella prova di magia il danno aumenta di 1d10

\medskip\textbf{Contattare Altri Piani}\index{Incantesimi - Contattare Altri Piani}\\
\textbf{Scuola}: Divinazione\\
\textbf{Difficoltà}: 26\\
\textbf{Tempo di Lancio}: 1 minuto\\
\textbf{Gittata}: Personale\\
\textbf{Componenti}: V\\
\textbf{Durata}: 1 minuto\\
Contatti mentalmente un semidio, lo spirito di un saggio da tempo defunto, o qualche altra misteriosa entità di un altro piano. Contattare l'Intelligenza extraplanare può affaticare o addirittura spezzare la tua mente. Quando lanci questo incantesimo, effettua un Tiro Salvezza su Volontà con DC 15. Se lo fallisci, subisci 6d6 danni e resti demente fino all'alba del giorno dopo. Mentre sei demente, non puoi effettuare azioni, non puoi capire quello che dicono le altre creature, non puoi leggere, e parli solo farneticando. L'incantesimo ristorare superiore può porre fine a questo effetto. Se superi il Tiro Salvezza, puoi porre all'entità fino a cinque domande. Devi porre le domande prima del termine dell'incantesimo. Il Narratore risponderà a ciascuna domanda con una parola: "sì", "no", "forse", "mai", "irrilevante" o "confuso" (se l'entità non conosce la risposta alla domanda). Se una risposta di una parola potrebbe risultare fuorviante, il Narratore potrebbe invece dare come risposta una breve frase.

\medskip\textbf{Evoca Celestiali}\index{Incantesimi - Evoca Celestiali}\\
\textbf{Scuola}: Evocazione\\
\textbf{Difficoltà}: 31\\
\textbf{Tempo di Lancio}: 1 minuto\\
\textbf{Gittata}: 27 metri\\
\textbf{Componenti}: V, S\\
\textbf{Durata}: 10 minuti\\
Evochi un celestiale di grado di sfida 4 o inferiore, che appare in uno spazio non occupato a gittata e che puoi vedere. Il celestiale sparisce quando scende a 0 punti ferita o l'incantesimo termina. Il celestiale è amichevole verso di te e i tuoi compagni per la durata dell'incantesimo. Tira l'iniziativa per il celestiale, che agisce durante il proprio round. Obbedisce a qualsiasi comando verbale che gli viene dato (senza bisogno che tu compia azioni), purché non violi i suoi Tratti. Se non dai comandi al celestiale, si difenderà dalle creature ostili, ma non compirà altre azioni.\\
\textbf{Per ogni Critico ottenuto} nella prova di magia aumenti di uno il CR della creatura evocata.
	
\medskip\textbf{Evoca Folletto}\index{Incantesimi - Evoca Folletto}\\
\textbf{Scuola}: Evocazione\\
\textbf{Difficoltà}: 29\\
\textbf{Tempo di Lancio}: 1 minuto\\
\textbf{Gittata}: 27 metri\\
\textbf{Componenti}: V, S\\
\textbf{Durata}: 1 ora \\
Evochi uno spirito fatato di grado di sfida 6 o inferiore, o uno spirito fatato che assuma la forma di una bestia di grado di sfida 6 o inferiore. Esso compare in uno spazio non occupato a gittata e che puoi vedere. La creatura fatata sparisce quando scende a 0 punti ferita o quando l'incantesimo termina.\\
La creatura fatata è amichevole verso di te e i tuoi compagni. Tirare l'iniziativa per la creatura fatata, che agisce durante i propri turni. Essa obbedisce a qualsiasi comando verbale che gli viene dato (senza bisogno che tu compia azioni), purché non violi i suoi Tratti. Se non dai comandi, si difenderà dalle creature ostili, ma non compirà altre azioni.\\
\textbf{Per ogni Critico ottenuto} nella prova di magia aumenti di 1 il CR della creatura evocata.

\medskip\textbf{Guardiano della Fede}\index{Incantesimi - Guardiano della Fede}\\
\textbf{Scuola}: Evocazione\\
\textbf{Difficoltà}: 23\\
\textbf{Tempo di Lancio}: 2 Azioni\\
\textbf{Gittata}: 9 metri\\
\textbf{Componenti}: V\\
\textbf{Durata}: 8 ore\\
Un guardiano spettrale Grande compare per la durata e fluttua in uno spazio non occupato a gittata e che puoi vedere, scelto da te. Il guardiano occupa quello spazio ed è indistinguibile eccetto per una spada luminosa e uno scudo con il simbolo del tuo Patrono.\\
Qualsiasi creatura a te ostile che entri in uno spazio entro 3 metri dal guardiano per la prima volta durante un round, deve effettuare un Tiro Salvezza su Riflessi. La creatura subisce 20 danni da Luce/Vuoto se fallisce il Tiro Salvezza, o la metà di questi danni se lo supera. Il guardiano svanisce dopo aver inflitto un totale di 60 danni.

\medskip\textbf{Guardiani Spirituali}\index{Incantesimi - Guardiani Spirituali}\\
\textbf{Scuola}: Evocazione\\
\textbf{Difficoltà}: 21\\
\textbf{Tempo di Lancio}: 2 Azioni\\
\textbf{Gittata}: Personale (raggio di 4 metri)\\
\textbf{Componenti}: V, S, M (un simbolo sacro)\\
\textbf{Durata}: Concentrazione, massimo 10 minuti\\
Richiami degli spiriti che ti proteggano. Per la durata dell'incantesimo, essi fluttueranno intorno a te a una distanza di 4 metri. Se sei buono o neutrale, la forma spettrale sarà angelica o fatata (a tua scelta). Se sei malvagio, avranno un aspetto demone. Quando lanci questo incantesimo, puoi designare un qualsiasi numero di creature che ne siano immuni. La velocità di una creatura soggetta viene dimezzata all'interno dell'area, e quando una creatura entra nell'area per la prima volta durante un round o inizia il suo round lì, deve effettuare un Tiro Salvezza su Volontà. Se fallisce il Tiro Salvezza subisce 3d8 danni da Luce (se sei buono o neutrale) o 3d8 danni da Vuoto (se sei malvagio), o la metà di questi danni se lo supera.\\
\textbf{Per ogni Critico ottenuto} nella prova di magia il danno aumenta di 1d8 \\

\medskip\textbf{Legame Planare}\index{Incantesimi - Legame Planare}\\
\textbf{Scuola}: Abiurazione\\
\textbf{Difficoltà}: 26\\
\textbf{Tempo di Lancio}: 1 ora\\
\textbf{Gittata}: 18 metri\\
\textbf{Componenti}: V, S, M (un gioiello del valore di almeno 1.000 mo, che l'incantesimo consuma)\\
\textbf{Durata}: 24 ore\\
Con questo incantesimo, cerchi di vincolare un celestiale, elementale, fatato o demone al tuo servizio. La creatura deve restare nella gittata per l'intero lancio dell'incantesimo. (Di solito, la creatura viene prima evocata al centro di un cerchio magico invertito per tenerla intrappolata mentre questo incantesimo viene lanciato). Al completamento del lancio, il bersaglio deve effettuare un Tiro Salvezza su Volontà. Se fallisce il Tiro Salvezza, è vincolato al tuo servizio per la durata. Se la creatura è stata evocata o creata da un altro incantesimo, la durata di quell'incantesimo viene estesa per corrispondere alla durata di questo incantesimo. Una creatura vincolata deve eseguire le tue istruzioni al meglio delle sue capacità. Potresti comandare la creatura di accompagnarti nel corso di un'avventura, di proteggere un luogo o di consegnare un messaggio. La creatura obbedisce le tue istruzioni alla lettera, ma se ti è ostile, cercherà di distorcere le tue parole ai suoi fini. Se la creatura adempie completamente alle tue istruzioni prima del termine dell'incantesimo, qualora vi troviate sullo stesso piano di esistenza ritornerà da te per comunicarti l'avvenuto. Se vi trovate su piani di esistenza diversi, ritornerà nel luogo dove l'hai vincolata e rimarrà lì fino al termine dell'incantesimo.\\
\textbf{Per ogni Critico ottenuto} nella prova di magia raddoppi la permanenza della creatura

\medskip\textbf{Marchio del Cacciatore}\index{Incantesimi - Marchio del Cacciatore}\\
\textbf{Scuola}: Divinazione\\
\textbf{Difficoltà}: 16\\
\textbf{Tempo di Lancio}: 2 Azioni\\
\textbf{Gittata}: 27 metri\\
\textbf{Componenti}: V \\
\textbf{Durata}: Concentrazione, massimo 1 ora\\
Scegli una creatura a gittata che puoi vedere. La creatura è misticamente marchiata come tua preda. Fino al termine dell’incantesimo, infliggi 1d6 danni aggiuntivi al bersaglio ogni volta che lo colpisci con un attacco con arma, e hai +1d6  alle prove di Consapevolezza o Sopravvivenza per trovarlo.\\
Se il bersaglio scende a 0 punti ferita prima del termine dell’incantesimo, puoi usare un’azione immediata durante il tuo prossimo turno per marchiare una nuova creatura.\\
\textbf{Per ogni Critico ottenuto} nella prova di magia puoi mantenere la concentrazione sull’incantesimo un altra ora\\

\medskip\textbf{Portale}\index{Incantesimi - Portale}\\
\textbf{Scuola}: Evocazione\\
\textbf{Difficoltà}: 36\\
\textbf{Tempo di Lancio}: 2 Azioni\\
\textbf{Gittata}: 18 metri\\
\textbf{Componenti}: V, S, M (un diamante del valore di almeno 5.000 mo)\\
\textbf{Durata}: Concentrazione, massimo 1 minuto\\
Evochi in uno spazio non occupato a gittata che puoi vedere un portale collegato a un posto preciso su di un diverso piano di esistenza. Il portale è un'apertura circolare creata da te, da 1 a 6 metri di diametro. Puoi orientare il portale in qualsiasi direzione desideri. Il portale resta per la durata.\\
Il portale ha un fronte e un dietro su entrambi i piani in cui compare. Il viaggio attraverso il portale è possibile solo muovendosi dal fronte. Qualsiasi cosa lo faccia viene istantaneamente trasportata nell'altro piano, comparendo nello spazio non occupato più vicino al portale.\\
Divinità e altri sovrani planari possono impedire ai portali creati da incantesimi di aprirsi in loro presenza o in qualsiasi punto dei loro domini. Quando lanci questo incantesimo, puoi pronunciare il nome di una specifica creatura (lo pseudonimo, titolo o soprannome non funzionano). Se quella creatura si trova su di un piano diverso dal tuo, il portale si apre in prossimità della creatura nominata e attira la creatura attraverso di sé, verso lo spazio non occupato più vicino dal tuo lato del portale. Non detieni alcun potere speciale sulla creatura, ed essa è libera di agire come il Narratore ritiene appropriato. Potrebbe andarsene, attaccarti o aiutarti.

\medskip\textbf{Salvare i Morenti}\index{Trucchetto - Salvare i Morenti}\\
\textbf{Scuola}: Necromanzia\\
\textbf{Difficoltà}: 12\\
\textbf{Tempo di Lancio}: 2 Azioni\\
\textbf{Gittata}: Contatto\\
\textbf{Componenti}: V, S, M (un offerta al tuo Patrono di almeno 5 mo, che l'incantesimo consuma)\\	
\textbf{Durata}: Istantanea\\
Una creatura a 0 punti ferita, con cui sei a contatto, torna a 1 PF. L'incantesimo non ha effetto su non morti o costrutti.

\medskip\textbf{Tempesta di Vendetta}\index{Incantesimi - Tempesta di Vendetta}\\
\textbf{Scuola}: Evocazione\\
\textbf{Difficoltà}: 36\\
\textbf{Tempo di Lancio}: 2 Azioni\\
\textbf{Gittata}: Vista\\
\textbf{Componenti}: V, S\\
\textbf{Durata}: Concentrazione, massimo 1 minuto\\
Si forma una ribollente nube di tempesta, centrata in un punto che puoi vedere e che si propaga in un raggio di 110 metri. L'area è illuminata da fulmini, vi riecheggiano tuoni e venti forti la spazzano. Quando la nube compare, ogni creatura sotto di essa (ovvero non più di 1.500 metri sotto la nube) deve effettuare un Tiro Salvezza su Tempra. Se fallisce il Tiro Salvezza, la creatura subisce 2d6 danni da tuono e resta assordata per 5 minuti.\\
Ogni round in cui mantieni la concentrazione su questo incantesimo, la tempesta, durante il tuo round, produce ulteriori effetti.\\
\textit{Round 2}. Pioggia acida cade dalla nube. Ogni creatura e oggetto sotto la nube subiscono 1d6 danni da acido.\\
\textit{Round 3}. Richiami sei fulmini dalla nube per colpire sei creature o oggetti di tua scelta, che si trovino sotto la nube. Una specifica creatura od oggetto non può essere colpita da più di un fulmine. Una creatura colpita deve effettuare un Tiro Salvezza su Riflessi. La creatura subisce 10d6 danni da fulmine se fallisce il Tiro Salvezza, o la metà di questi danni se lo supera. \\
\textit{Round 4}. La nube produce una fitta grandinata. Ogni creatura sotto la nube subisce 2d6 danni da botta.\\
\textit{Round 5-10}. Folate di vento e pioggia gelida si abbattono sull'area sotto la nube. L'area diventa terreno difficile ed è in penombra. Ogni creatura nell'area subisce 1d6 danni da freddo. Nell'area diventa impossibile effettuare attacchi con armi a distanza. Il vento e la pioggia sono considerati una distrazione grave ai fini del mantenere la concentrazione sugli incantesimi.\\ Infine, folate di forte vento (che va dai 30 ai 75 chilometri all'ora) disperdono automaticamente nebbia, foschia e simili fenomeni nell'area, che siano naturali o magici.

\medskip\textbf{Trova Famiglio}\index{Incantesimi - Trova Famiglio}\\
\textbf{Scuola}: Evocazione\\
\textbf{Difficoltà}: 16\\
\textbf{Tempo di Lancio}: 1 ora\\
\textbf{Gittata}: 3 metri\\
\textbf{Componenti}: V, S, M (10 mo di carbone, incenso e erbe che devono essere consumate dal fuoco in un braciere d'ottone)\\
\textbf{Durata}: Istantanea\\
Ottieni il servizio di un famiglio, uno spirito che assume una forma animale di tua scelta: cavalluccio marino, corvo, donnola, falco, gatto, granchio, gufo, lucertola, pesce (frizzo), piovra, pipistrello, ragno, rana (rospo), ratto o serpente velenoso. Apparendo in uno spazio a gittata, non occupato, il famiglio ha le statistiche della forma scelta, sebbene sia di tipo celestiale, fatato o demone (a tua scelta) invece di una bestia. Il tuo famiglio agisce in maniera indipendente da te, ma ubbidisce sempre ai tuoi comandi. In combattimento, tira la propria iniziativa e agisce durante il proprio round. Un famiglio non può attaccare, ma può svolgere le altre azioni come di norma. 
Non puoi avere più di un famiglio alla volta. \\
\textbf{Verifica Abilità Famiglio} per le capacità del famiglio. Devi avere l'Abilita' Famiglio.

\subsection{Nuovi incantesimi}

In accordo con il Narratore il giocatore e' invitato a creare o proporre nuovi incantesimi.\\
Per praticità includo la tabella di conversione del Livello dell'Incantesimo, se usate Pathfinder o la 5e del Gioco di Ruolo Fantasy.\\

\medskip

\textbf{Tabella conversione Livello Incantesimo - Difficoltà}

\medskip

\begin{tabular}{ll}
\textbf{Livello} & \textbf{Difficoltà}\\
0	& 12\\
1	& 16\\
2	& 19\\
3	& 21\\
4	& 23\\
5	& 26\\
6	& 29\\
7	& 31\\
8	& 34\\
9	& 36\\
\end{tabular}
\end{multicols}


\pagebreak

\section{Vantaggi}\index{Vantaggi}\hypertarget{vantaggi}{} 

\label{vantaggi}
\begin{enfasi}{Adoro fare il supereroe! L'orario di lavoro è pessimo, la paga è inesistente... ma almeno non corro il rischio di venire licenziato! (PK)
	}\end{enfasi}\medskip

\begin{multicols}{2}

\lettrine[lines=2, lhang=0.33, loversize=0.25, findent=1.5em]{O}{gni} personaggio può avere, e non è obbligatorio averne, dei Vantaggi. Questi devono essere interessanti, piacevoli, divertenti e soprattutto giocabili.

Ogni Vantaggio ha un costo, da pagare ad ogni livello. Come detto non deve essere obbligatorio prendere un Vantaggio, né tanto meno si devono prendere vantaggi solo perché fanno essere forti. Lo scopo di un Vantaggio è stupire e divertirsi.

Avere un Vantaggio significa essere diverso, essere un freak, avere quel particolare che ti rende diverso ed unico, ma non per questo sempre il più forte, potente o invincibile. Un vantaggio non è solo una capacità, è un'occasione di gioco di ruolo. Il giocatore è invitato ad essere creativo nella scelta dei vantaggi ed anche nella creazione di nuovi, il costo poi si decide con il Narratore.

Ed è sempre e comunque il Narratore ad avere l'ultima parola sui Vantaggi scelti.

Diversi vantaggi non hanno un effetto pratico concreto ed immediato ma sono di arricchimento al background, alla storia del personaggio, introducono occasioni di gioco e divertimento. Quando si scelgono i vantaggi, e di conseguenza gli svantaggi, non è come andare a fare scorta di poteri super e straordinarie abilità, ma di peculiarità, manie, specialità che il personaggio possiede e che ancora una volta lo rendono diverso, unico, solo tuo.

Pertanto vantaggi e svantaggi vanno anche e soprattutto giocati ed interpretati.

\begin{itemize}

\item
I Vantaggi con {*} e tutti quelli con costo 20 o superiore sono a discrezione del Narratore nell'essere ammessi alla scelta.

\item
I vantaggi si scelgono al primo livello, ogni vantaggio preso a livelli successivi va concordato con il Narratore.

\item
I punti di costo di un Vantaggio si pagano con i punti guadagnati dagli Svantaggi.

\item
I bonus dati alle competenze si intendono specifiche sulla prova quando indicato tra parentesi.

\item Se non indicato diversamente costa una Azione attivare un Vantaggio (se l'effetto non è permanente).

\end{itemize}


\begin{enfasi}{
Da un grande Vantaggio deriva un grosso Svantaggio ! (cit. "Da un grande potere derivano grandi responsabilità", Amazing Fantasy 15, Stan Lee)
}\end{enfasi}\medskip


\subsection{Elenco Vantaggi}\index{Elenco Vantaggi}

\textbf{Ali della provvidenza} \index{Ali della provvidenza}20 : hai delle ali, a te la scelta di forma e colore, solitamente stanno sulle scapole e ti fanno volare. Se non concordato diversamente il movimento in volo è pari a quello razziale di terra.

\textbf{Ambidestro}\label{Ambidestro}\index{Ambidestro} 10: puoi usare indifferentemente entrambe le mani. I malus alle prove dove si usano due mani diminuiscono di 2

\textbf{Amico degli animali}\index{Amico degli animali} 5: +2 alle prove per gestire gli animali (anche selvaggi)

\textbf{Anfibio}\index{Anfibio} 20: puoi respirare sia sott'acqua che l'aria

\textbf{Arcobaleno}\index{Arcobaleno} 10: sei un artista. Le tue dita spontaneamente producono colore

\textbf{Aura di coraggio}\index{Aura di coraggio} 15: intorno a te, in distanza entro 3 metri infondi coraggio. +2 Tiro Salvezza vs effetti naturali o magici di paura.

\textbf{Artigli}\index{Artigli} 5: ogni tanto ricordati di spuntare gli unghiotti. 1d4 di danno per attacco. Gli attacchi naturali con la seconda mano prendono il bonus al danno dato da Forza.

\begin{center}
	\includegraphics[width=0.5\linewidth]{immagini/claw.png}
\end{center}


\textbf{Bere fa bene}\index{Bere fa bene} 5: Prerequisito: Il fegato non conta. Il tuo corpo metabolizza l'alcool in maniera molto efficace. Un litro di birra ti fa recuperare 1d4 Punti Ferita, un bottiglia di liquore 1d8 Punti Ferita. Se di pessima qualità no.. Puoi comunque ubriacarti.

\textbf{Bere fa bene}\index{Bere fa molto bene} 10: Prerequisito: Il fegato non conta. Il tuo corpo metabolizza l'alcool in maniera molto efficace. Un litro di birra ti fa recuperare 2d4 Punti Ferita, un bottiglia di liquore 12d8 Punti Ferita. Se di pessima qualità no. Non puoi ubriacarti con liquidi naturali.

\textbf{Caduta gatto}\index{Caduta gatto} 5: +2 alle prove di Destrezza sulle cadute e +2 a Muoversi Silenziosamente.

\textbf{Camaleonte}\index{Camaleonte} 10-20: la tua pelle può cambiare colore. Tempo necessario 1 minuto/1 round.

\textbf{Cambiaforma}\index{Cambiaforma} 40: come incantesimo Alterare Se Stesso. Può essere usato ogni 10 minuti.

\textbf{Camminare sull'aria} \index{Camminare sull'aria}30: non troppo controllato. Qualsiasi cosa che non sia camminare richiede una prova di Destrezza o cadere prono (ma non per terra).

\textbf{Camminare sulle acque} \index{Camminare sulle acque} 30: ma non darti delle arie..

\textbf{Magnetico} \index{Magnetico}5-10: sprigioni luce quando vuoi. per fortuna non letteralmente. +2 alle prove al Carisma.

\textbf{Consumi ridotti} \index{Consumi ridotti}5: bevi e mangi la metà di un uomo normale. Sei sotto peso.

\textbf{Controllo del metabolismo} \index{Controllo del metabolismo} 10: solo il nome è fantastico! Annulli il danno da Sanguinamento.

Recuperi i punti ferita come se avessi il doppio del punteggio di Costituzione.

\textbf{Cure efficaci} \index{Cure efficaci}10: +1d6 Punti Ferita curati ogni volta che tu usi un incantesimo di Cura su te stesso o altri.

\textbf{Daredevil} \index{Daredevil}10: ti piace buttarti nelle mischia, specialmente se si corrono pericoli. +2 Tiri per Colpire / Difesa finché sei circondato da tre o più avversari.

\textbf{Denti} \index{Denti}5: il tuo morso fa male, 1d. Lavati i denti ogni tanto..

\textbf{Digestione universale} \index{Digestione universale}5: purché non faccia male si mangia, +2 Tiro Salvezza su Tempra vs Veleni. Immune ai disturbi di stomaco naturali.

\textbf{Direzione Assoluta} \index{Direzione Assoluta}5: sai sempre dove è il nord magnetico. Hai un +4 alle prove di orientamento.

\textbf{Duro da soggiogare} \index{Duro da soggiogare}10: +2 Tiro Salvezza su Volontà su incantesimi della scuola Charme

\textbf{Duro da uccidere} \index{Duro da uccidere}5: non svieni a 0 Punti Ferita, ma a -LV/2 in Punti Ferita. Muori a 15+Costituzione x 3 Punti Ferita.

\textbf{Empatia con le piante} \index{Empatia con le piante}10: io comprendo la sofferenza dell'erba pestata.

\textbf{Empatia} 5: +2 alle prove di Percepire Emozioni.

\textbf{Empatia Animale} \index{Empatia Animale}10: +4 alle prove per gestire gli animali (anche selvaggi).

\textbf{Empatia spirituale} \index{Empatia spirituale}5: non parli con gli spiriti, ma ne senti le emozioni.

\textbf{Ermafrodito} \index{Ermafrodito}10: lgbtE!.

\textbf{Forgiato nell'acciaio} \index{Forgiato nell'acciaio}5: Tramite dolorose operazioni la tua pelle è stata rivestita con placche di metallo. La tua Difesa base e' 13.

\textbf{Forma d'ombra} \index{Forma d'ombra}30: considera il poterti trasformare in un ombra per 1 ora per livello. Puoi spostarti solo su zone ombreggiate.

\textbf{Fortunato} \index{Fortunato}5: 2 volte al giorno puoi ritirare un 1 sul dado a 6, da dichiarare prima del tiro di dado.

\textbf{Guarigione accelerata}\index{Guarigione accelerata}: 5 ogni mattina recuperi il doppio dei Punti Ferita che normalmente recupereresti. Si cumula con Controllo Metabolismo. \index{Controllo Metabolismo}.

\textbf{Guaritore}\index{Guaritore} 5: sai dove mettere le mani. +4 alle prove di Pronto soccorso

\textbf{Il fegato non si conta} \index{Il fegato non si conta}10: puoi bere tanto e non ti ubriachi

\textbf{Illuminato} \index{Illuminato}10-20: fai luce.. letteralmente. Emetti luce in un raggio di 3/6 metri. Puoi controllare (20) l'emissione o meno (10).

\textbf{Immune}\index{Immune} 5-20: a cosa ?

\textbf{Invisibile} \index{Invisibile}40: il tuo corpo è invisibile. Sempre. E non è magia...

\textbf{Ira} \index{Ira}5: sei capace di infuriarti. +2 al danno in mischia e -1 Tiro per Colpire e Difesa. Ogni altri 5 punti +2 danno -1 Tiro per Colpire e Difesa, max 20 punti. Durata 4 (anche non consecutivi)round ogni 5 punti. Si attiva come una Azione.

\textbf{La mia ombra è mia amica} \index{La mia ombra è mia amica}10: Riesci a posizionare la tua ombra dove vuoi entro 3 metri. La tua ombra può manipolare gli oggetti come fosse un servitore invisibile. Si considera tu possa lanciare Incantesimi a tocco tramite la tua ombra (che deve essere presente) entro 3 metri.

\begin{center}
	\includegraphics[width=0.6\linewidth]{immagini/shadow.png}
\end{center}


\textbf{Legami di furia} 15\index{Legami di furia} : Puoi evocare lacci eterei che minacciano i tuoi nemici. Per 3 volte al giorno con il costo di 1 Azione tutti gli avversari in raggio entro 9 metri attorno a te sono intralciati per un round. Tiro Salvezza vs Riflessi DC 10 + 1/2 LV + Carisma) per liberarsi.

\textbf{Lento e Fermo} 5: \index{Lento e Fermo}Sei eccezionalmente stabile sui tuoi piedi. Non puoi essere mosso o sollevato se non da una creatura di 2 taglie superiori.

\textbf{Lingua universale} \index{Lingua universale}10. Le tue capacità linguistiche sono impressionanti. Dopo due giorni a contatto con una nuova lingua sei in grado di parlarla correttamente. Dopo 3 giorni di lontananza dall'ambiente dimentichi la lingua. Guadagni un +2 alle prove basate sui linguaggi.

\textbf{Magia esplosiva} \index{Magia esplosiva}10: I tuoi incantesimi di Invocazione che causano danno hanno un dado in più di danno (quando c'è da tirare un dado..).

\textbf{Mani di Fata} \index{Mani di Fata}10: +4 prova di Mani di Fata e Artista della Fuga che coinvolgano le mani. Puoi prendere 16 come prendessi un 10 nelle prove relative.

\textbf{Mano Piede palmata} \index{Mano Piede palmata}5: +4 alle prove di nuotare.

\textbf{Mattiniero} \index{Mattiniero}5-10-15: ti basta dormire 6/5/4 ore per notte per essere riposato completamente.

\textbf{Medium} \index{Medium}10-20: alcune volte lo vuoi tu, altre volte ti cercano loro.

\textbf{Memoria fotografica}\index{Memoria fotografica} 20-50: per fortuna non è permanente (50). +8 alle prove per ricordare dettagli (Conoscenza e Consapevolezza).

\textbf{Naso peloso} \index{Naso peloso}5: le tue narici filtrano le tossine presenti nell'aria che respiri. +2 alle prove relative. Il tuo naso è di dimensioni.. non piccole.

\textbf{Non dormi}\index{Non dormi} 20{*}: e non so come fai..

\textbf{Non invecchi}\index{Non invecchi} 20{*}: non invecchi (ma possono ucciderti lo stesso).

\textbf{Non mangi bevi} \index{Non mangi bevi}20: e non so come fai..

\textbf{Non respiri} \index{Non respiri}20: e non so come fai..

\textbf{L'Odore del sangue} \index{L'Odore del sangue}10: L'odore di sangue è una droga potente
Prerequisiti: non puoi avere "Il fegato non conta" . Guadagni un +1 a Tiro per Colpire ed un +1 al danno per ogni nemico che hai ucciso con la tua arma nel round. Questo bonus non può superare il +4/+4. Il bonus rimane attivo fino al round successivo all'ultima uccisione fatta. Creature con meno di 3 LV di te non contano.

\textbf{Oracolo} \index{Oracolo}20: per qualcuno è una maledizione. L'uso va sempre concordato con il Narratore.

\textbf{Ottima vista} \index{Ottima vista}5: hai un ottima vista (12/10). +2 alle prove relative di Consapevolezza che usano la vista.

\textbf{Ottimo olfatto e gusto} \index{Ottimo olfatto e gusto}5: hai un ottimo gusto ed olfatto. +2 alle prove di Consapevolezza che usano olfatto o gusto.

Con un prova su Intelligenza (oppure Arcana od Erboristeria a DC 13) a DC 16 puoi capire cosa è una pozione.

\textbf{Ottimo tatto} \index{Ottimo tatto}5: Hai un ottimo tatto. sai leggere con le dita. Sei in grado di trovare una porta nascosta toccando la parete.

\begin{center}
	\includegraphics[width=0.8\linewidth]{immagini/braille.png}
\end{center}


\textbf{Ottimo udito}\index{Ottimo udito} 5: Hai un ottimo udito. +2 alle prove di Consapevolezza che coinvolgono l'udito.

\textbf{Parlare con gli animali}\index{Parlare con gli animali} 20: scegli una famiglia (ovini, marsupiali, caviette..).

\textbf{Parlare con le piante} \index{Parlare con le piante}20: ho sempre voluto parlare con le zucchine..

\textbf{Percezione Cieca}(vista cieca):\index{Percezione Cieca} \index{Vista Cieca}30: riesci a percepire qualsiasi cosa con i tuoi sensi entro 18 metri, dall'odore, al calore. Riesci a "vedere" attraverso e fino 18 metri, 5 cm di pietra, 10 cm di legno, 0.5 cm di metallo.

\textbf{Perfetto equilibrio} 5:\index{Perfetto equilibrio} +2 alle prove relative di Acrobatica.

\textbf{Piedi veloci}\index{Piedi veloci} 10/20: il tuo movimento aumenta di 1/3 metri.

\textbf{Pollice verde}\index{Pollice verde} 5: +4 alle prove di Professione (Erboristeria, Giardiniere..).

\textbf{Polmoni di ferro}\index{Polmoni di ferro} 5: puoi trattenere il respiro 2*Costituzione minuti (minimo 2 minuti).

\textbf{Precognizione} 30{*}\index{Precognizione}: Come l'incantesimo Previsione.

\textbf{Recupero}\index{Recupero} 10: Il tuo corpo produce spontaneamente caffeina. Ignori la condizione affaticato.

\textbf{Resistenza}\index{Resistenza} 5-10: +1/+2 Tiro Salvezza a Riflessi o Tempra o Volontà.

\textbf{Resistenza al danno}\index{Resistenza al danno} 10: -1 danno. -1 danno aggiuntivo ogni 5 punti aggiuntivi. Stabilisce il tipo di resistenza (taglio, botta, perforazione, energia..).

\textbf{Resistenza al magico}\index{Resistenza al magico} 20: Hai una RM 10.

\textbf{Resistenza al fuoco/freddo/elettricità}\index{Resistenza al...} 5-10: ignori i primi 3/6 punti di danno per round.

\textbf{Ricostruzione}\index{Ricostruzione} 30: perdere una mano non è mai stato un problema..

\textbf{Rigenerazione}\index{Rigenerazione} 30: +1Punti Ferita per Turno (non rigeneri arti). 

\textbf{Rigenerazione veloce} 40: +1Punti Ferita per round (non rigeneri arti). Muori se distruggono il tuo corpo (o non rimane che cenere).

\textbf{Rimpicciolimento} 30: puoi diminuire fino a due taglie. Durata fino a 8 ore.

\textbf{Rinoceronte} 10 : La tua carica è distruttiva. Si considera che niente sotto la robustezza di sbarre di ferro (durezza 15) possa fermare la tua carica. Dietro di te lasci una scia di distruzione. +2 ai Tiri per Colpire in Carica e +1d6 di danno.

\textbf{Scudo Mentale}\index{Scudo Mentale} 5: +2 Tiro Salvezza su controlli ed influenze mentali.

\textbf{Sensi protetti}\index{Sensi protetti} 5: +2 Tiro Salvezza contro suoni/luci/vapori o incantesimi che agiscano sui tuoi sensi.

\textbf{Senso comune}\index{Senso comune} 5: se stai per fare una brutta figura un campanellino ti avvisa.

\textbf{Senso della moda}\index{Senso della moda} 5: sai sempre come vestirti bene, anche solo con uno straccetto.

\textbf{Senso delle vibrazioni} \index{Senso delle vibrazioni} \index{Senso Tellurico} (Senso Tellurico) 30: tutto fa tremare un poco la terra, o quasi, raggio di 18 metri intorno a te.

\textbf{Senso del tempo} \index{Senso del tempo}5: sai sempre che ore sono, giorno o notte.

\textbf{Senso ragno}\index{Senso ragno} 15: no non ti ha morso un uomo radioattivo, ma sei estremamente sensibile ai pericoli. +2 iniziativa, non puoi essere sorpreso.

\textbf{Senza paura} \index{Senza paura}10: sei immune alla paura, magica o meno.

\textbf{Silenzioso} \index{Silenzioso}5: +4 alle prove di Consapevolezza (muoversi silenziosamente).

\textbf{Spine} \index{Spine}5: e sei pure brutto. 1d4 di danno.

\textbf{Super piastrine} \index{Super piastrine}5: Riduci il danno da Sanguinamento di 1 a fine di ogni round.

\textbf{Talento per le lingue}\index{Talento per le lingue} 5: Impari due lingue investendo 1 punto in Conoscenza Linguistica.

\textbf{Talento selvaggio}: \index{Talento selvaggio}Parliamone.

\textbf{Tocco gelido} \index{Tocco gelido}10: Toccando un morto (entro 1 giorno per livello) puoi vedere e sentire cosa è successo nel suo ultimo round di vita.

\textbf{Troll} \index{Troll}60: Rigeneri 5 Punti Ferita a round anche se i Punti Ferita sono negativi. Rigeneri anche arti. Puoi essere "ucciso" solo da fuoco o acido. Una condizione potrebbe comunque tenerti a punti ferita negativi (es. immerso sott'acqua).

\textbf{Udito subsonico}\index{Udito subsonico} 10: Senti le frequenze inudibili per gli umani (come un cane)

\textbf{Vedere l'invisibile} \index{Vedere l'invisibile}15: Meglio la vista a raggi X.. sbav..

\textbf{Comprensione del vero}\index{Comprensione del vero} 10: La verità ha un suono tutto suo. +4 alla prove di Percepire Emozioni.

\textbf{Vista demoniaca} \index{Vista demoniaca}15: Vedi nell'oscurità più totale, anche magica, fino a 18 metri.

\textbf{Visione Perimetrale} \index{Visione Perimetrale}5: Sogliola ? +2 alle prove di Consapevolezza da lato.

\textbf{Visione Telescopica}\index{Visione Telescopica} 10: +4 alle prove di Consapevolezza e visione ma da lontano.

\textbf{Voce suadente} \index{Voce suadente}5: +2 alle prove di Carisma che usano la voce,

\textbf{Voce subsonica}\index{Voce subsonica} 10: Emetti suoni non udibili dagli umani. I cani ti odiano.

\end{multicols}

\pagebreak

\section{Svantaggi}\index{Svantaggi}

\label{svantaggi}
\begin{enfasi}{Se devi essere storpio, meglio essere uno storpio ricco. (Tyrion Lannister)}\end{enfasi}\medskip

\begin{multicols}{2}

\lettrine[lines=2, lhang=0.33, loversize=0.25, findent=1.5em]{U}{no} svantaggio caratterizza il personaggio, ne definisce limiti e paure. Ogni personaggio deve avere almeno 1 svantaggio di ruolo e questo non gli da punti bonus.

I punti presi con gli Svantaggi psico/fisici servono a coprire i punti spesi con i Vantaggi. Ovviamente l'Evil Narratore gradisce anche più svantaggi...

\textbf{Ogni giocatore deve giocare i suoi svantaggi altrimenti non acquisisce punti esperienza e gli sarà negato l'uso dei Vantaggi.}

Uno svantaggio può essere "annullato" nel corso della storia del personaggio e deve esserci una avventura che giustifichi il tutto. Come sempre il Narratore ha l'ultima parola su ogni scelta di vantaggi e svantaggi.

\bigskip

Suggerimenti
\begin{itemize}
	\item
	      Prendi degli svantaggi che siano divertenti da giocare, anche se ti metteranno nei guai.
	\item
	      Prendi degli svantaggi che siano interessanti da giocare con gli altri giocatori anche se metteranno loro nei guai
	\item
	      Prendi degli svantaggi che c'entrino con il personaggio
	\item
	      Prendi degli svantaggi di cui non andrai a pentirti
\end{itemize}

\textbf{Fai attenzione}:

\begin{itemize}
	\item
	      Evita gli svantaggi che sono difficili da giocare o perché completamente avulsi dal sistema o totalmente inutili o severamente dannosi per gli altri. Se vuoi essere un pacifista estremo, valuta bene il personaggio ed il gruppo..
	\item
	      Non prendere svantaggi che ti possa vergognare a recitare
	\item
	      Non prendere svantaggi che non c'entrano con il personaggio (in perfetta contraddizione con quanto già detto...)
	\item
	      Non prendere svantaggi insulsi (tipo la paura di girare a destra, degli ascensori..)
	\item
	      Se prendi uno svantaggio severo, recitalo bene, il Narratore saprà ricompensarti
\end{itemize}

\subsection{Svantaggi di Ruolo e Svantaggi Psico/Fisici}\hypertarget{svantaggi}{} 

Gli svantaggi si dividono in due categorie, \textbf{Svantaggi di Ruolo} e \textbf{Svantaggi psico/fisici}.

Gli \textbf{Svantaggi di Ruolo} sono dei piccoli difetti, tic, problemi grandi e piccoli che servono a dare uno spessore più "umano" al personaggio. Hanno una descrizione volutamente ambigua e scherzosa, sceglili con attenzione e discuti con il Narratore come intendi interpretare questo svantaggio.

Il giocatore è invitato a creare nuovi svantaggi di ruolo. Questi svantaggi non concedono un bonus o malus ne danno punti per prendere vantaggi. \emph{Però sono divertenti!}

\bigskip

Gli \textbf{Svantaggi psico/fisici} sono invece più impattanti nel gioco, nella quotidianità dando concreti svantaggi. Questi svantaggi forniscono i punti con i quali "pagare" i vantaggi. In fondo trovate un elenco di Fobie.

\end{multicols}

\pagebreak

\subsubsection{Svantaggi di Ruolo}\index{Svantaggi di Ruolo}


\begin{multicols}{2}


\textbf{Alcolismo}:\index{Alcolismo} ti piace bere, e tanto.. ma quando smetti ?

\textbf{Alla moda}\index{Alla moda}: tua probabilmente, anche con vestiti nuovi non ti vesti mai bene. L'accostamento di colori è sempre un pugno nell'occhio.

\textbf{Amico degli animali}:\index{Amico degli animali} intesi come pulci, zecche, pidocchi, cimici.. mosche. Hai uno zoo su di te.

\textbf{Attira animali}: \index{Attira animali}non sai il perché ma sei sempre circondata da gatti, cani, coniglietti, coccatrici..

\textbf{Attira guai}\index{Attira guai}: non è colpa mia se il drago ha deviato per venire a fare la popò qui..

\textbf{Banana}: \index{Banana}quella che provi a farti nei capelli, ma non riesci. I tuoi capelli non vanno d'accordo con te.

\textbf{Bassa soglia del dolore}: \index{Bassa soglia del dolore}mi ha graffiato, aiuto! sto morendo!!!

\textbf{Brufoli}: \index{Brufoli}pieno, hai la faccia butterata e continuano a formarsi questi disgustosi brufoli gialli.

\textbf{Ciuccione:} \index{Ciuccione}non lo fai spesso, ma nei momenti in cui sei più nervoso tiri fuori il vecchio ciuccio di legno.. (o in mancanza va sempre bene il proprio pollice).

\textbf{Codardo}:\index{Codardo} è meglio scappare, pardon, raccogliamo prima di tutte le informazioni prima di attaccare.

\textbf{Cogito ergo sum}: \index{Cogito ergo sum}hai la tendenza a parlare tra te e te, ma ad alta voce anche se ci sono persone intorno e pure se non sono amichevoli.

\textbf{Credulone}: \index{Credulone}ma dai ? davvero ? e a quale altezza volava l'asino ?

\textbf{Criceto}: \index{Criceto}intesa come memoria. Non riesci ad associare nomi a volti.

\textbf{Denti marci}: \index{Denti marci}probabilmente lo spazzolino che usi non ha setole di vero cinghiale...

\textbf{Dita nel naso}:\index{Dita nel naso} spero che siano almeno buone.

\textbf{Diva}: \index{Diva}o almeno tu credi di esserlo. Non perdi occasione per dare sfoggio delle tue inesistenti capacità canore, comiche, estetiche... con grosse risate di tutti.

\textbf{Faccia comune}: \index{Faccia comune}come ti chiami ? mi sembra di averti già visto...

\textbf{Galante}: \index{Galante}al limite del maniacale, in ogni tuo gesto sei formale, appropriato e cordiale.

\textbf{Killer}:\index{Killer} no, non sei un assassino. Hai però sempre le mani ed i piedi freddi.

\textbf{Impaurisci animali}: \index{Impaurisci animali}può essere anche comodo, se non fosse per i cavalli che scappano e gli orsi che attaccano...

\textbf{Incapace di divertirsi}: \index{Incapace di divertirsi}quindi ? è un problema tuo, non mio.

\textbf{Inglese:} \index{Inglese}inteso come umorismo. Nessuno mai capisce le tue battute.

\textbf{Mangione}: \index{Mangione}CIOMP!. Mai lesinare, potrebbe essere l'ultimo pasto!

\textbf{Meteora}:\index{Meteora} soffri di meteorismo compulsivo e rumoroso, per non parlare dell'odore sgradevole.

\textbf{Megalomane}:\index{Megalomane} coinvolgiamo gli eserciti dei sette regni e penetriamo nel dungeon!

\textbf{Mentina:} \index{Mentina}se mangiassi solo aglio e cipolla il tuo alito sarebbe meno puzzolente

\textbf{Musichiere}: \index{Musichiere}con la bocca. Fischi di continuo, in ogni occasione che sei sovrappensiero o molto teso.. ti metti a fischiettare.

\textbf{Non empatico}: \index{Non empatico}perché piange il bambino a cui ho appena dato a fuoco l'orsetto ?

\textbf{Ossessione}:\index{Ossessione} ancora, ancora, ancora. Un'altra tubetto di crema per la pelle!!

\textbf{Pacco}: \index{Pacco}il tuo. Hai sempre una mano laggiù. Forse i pantaloni sono stretti ? e no, non ti stringo la mano.

\textbf{Pessimo carattere}: \index{Pessimo carattere}va bene essere burbero.. ma devi sempre renderlo palese ?

\textbf{Pezzata}: \index{Pezzata}no, non la mucca o la tua cavalla ma la tua ascella. Sudi copiosamente, che sia per caldo o freddo.. o nervoso.

\textbf{Rigidezza mentale}:\index{Rigidezza mentale} no, non capisco, la mappa dice di andare a destra. Non mi importa se non c'è una destra.

\textbf{Saccente}\index{Saccente}: la risposta giusta è solo la tua. Non c'è dubbio.. per te.

\textbf{Sangue dal naso}: \index{Sangue dal naso}capita, e sempre appena vedi una donna/uomo (a seconda dei gusti) che ti piace.

\textbf{Sciarpina}: \index{Sciarpina}devi sempre avere addosso e visibile un capo di un certo tipo, altrimenti non esci di caverna.

\textbf{Segreto}: \index{Segreto}ho un segreto, talmente tanto segreto che non so se lo so neanche io...

\textbf{Seguire il Chaos}: \index{Seguire il Chaos}è più forte di te, non riesci mai ad ubbidire a qualsiasi legge o autorità preposta.

\textbf{Seguire la Legge}: \index{Seguire la Legge}è più forte di te, non importa che legge sia, tu non la violi.

\textbf{Tatuato:} \index{Tatuato}il tatuaggio è il modo di vivere. Hai almeno il 30\% del corpo già tatuato e non perdi occasioni per farti nuovi tatuaggi.

\textbf{Topi}:\index{Topi} sei una TOPI!

\textbf{Unghie}:\index{Unghie} sei un divoratore compulsivo di unghie, la punta delle dita ti sanguina a volte.

\textbf{Ultima parola}\index{Ultima parola}: è più forte di te, devi avere l'ultima parola in ogni discorso.

\end{multicols}

\pagebreak

\subsection{Svantaggi psico/fisici}\index{Svantaggi psico/fisici}

\label{svantaggi-psicofisici}

\begin{multicols}{2}

\textbf{Albino}\index{Albino}

Sei Bianco, quasi fosse latte. Non ti abbronzi e non sopporti la luce, la tua pelle è delicata.

\textbf{13}: Oltre ad essere estremamente riconoscibile hai i seguenti svantaggi: Miopia e Fotosensibilità e Pelle Sensibile.

\textbf{Allergia}\index{Allergia}

Hai una qualche forma allergica. Spero non grave. Assicurati di avere sempre con te una pozione di di rimuovi veleno.

\textbf{5:} In presenza di un allergene specifico il personaggio starnutisce sonoramente finché l'allergene non viene allontanato, -1 a tutte le prove. (es. Allergico alla Birra)

\textbf{10}: Il personaggio soffre di attacchi di tosse, iperlacrimazione, giramenti di testa, -2 a tutte le prove. Tiro Salvezza su Tempra DC 10 per non soffocare. Il tiro va ripetuto ogni 20 round finché non ti sei allontanato dall'allergene.

\textbf{15:} Il personaggio soffre di violenti attacchi di tosse, nausea, sudori freddi, palpitazione. -5 a tutte le prove, è necessario un Tiro Salvezza su Tempra DC 15 o perdere i sensi. I tiri vanno ripetuti ogni 5 round finché l'allergene non è allontanato.

\textbf{20}: Il personaggio cade in preda di una crisi respiratoria, ed è incapace di compiere qualsiasi azione che non sia vomitare, annaspare e tossire sangue. Fallendo un Tiro Salvezza su Tempra DC 25 il personaggio muore annegando nel suo stesso vomito. Il tiro va ripetuto ogni round fino a che l'allergene non è allontanato.

Nota: allergeni troppo rari non valgono.

\textbf{Allucinazioni}\index{Allucinazioni}

c'è qualcosa che non va nella tua testa, ogni tanto si innesca una scintilla.

\textbf{10}: Il personaggio vede e sente cose che non ci sono. Ogni giorno tiri un dado a sei facce.
Se escono 1 o 2, non succede nulla.
Con 3,4 o 5 si verificheranno uno o due episodi allucinatori con modalità e tempi a discrezione del Narratore.
Con 6 il personaggio sarà vittima di visioni orrende e disgustose (o il contrario) con durata di 1d4 ore.

\textbf{Amnesia}\index{Amnesia}

\textbf{10}: Hai dimenticato il tuo passato e con quello il ricordo di amici, nemici, obiettivi. Non c'è modo di recuperare i ricordi perduti.

\textbf{Asceta}\index{Asceta}

10, lo dice la regola. Non porterai con te più di 10 oggetti.

\textbf{20}: non puoi avere più di 10 oggetti con te, magici o normali o monete o armi. Per fortuna i vestiti non contano.

\textbf{Balbuziente}\index{Balbuziente}

Sai parlare, ma male.

\textbf{5:} Hai una fastidiosa tendenza a balbettare proprio quando hai qualcosa da dire di importante. In queste situazioni critiche dalle tue labbra escono solo suoni abbozzati.

\textbf{Pessimo Carattere}\index{Pessimo Carattere}

Le buone maniere non sono mai una opzione.

\textbf{5}: Non hai mai imparato l'arte della diplomazia, e detesti essere contraddetto o insultato. Questo non significa che passi alle vie di fatto, ma che di fronte ad un insulto o ad una critica schietta tendi a zittire il proprio interlocutore con espressioni davvero poco simpatiche. Hai un -2 alle prove basate sul Carisma

\textbf{Spendaccione}\index{Spendaccione}

\textbf{10}: devi spendere metà dei tuoi guadagni di missione in piaceri futili (mangiare cibi costosi, bere vino e liquori pregiati, vestiti lussuosi, no armi od oggetti magici)

\textbf{15}: devi spendere tutti i tuoi guadagni di missione in piaceri futili (mangiare cibi costosi, bere vino e liquori pregiati, vestiti lussuosi, no armi od oggetti magici)

\textbf{Caritatevole}\index{Caritatevole}

\textbf{10}: devi donare metà dei tuoi guadagni di missione in beneficenza

\textbf{15}: non può tenere più di 10 mo in contanti

\textbf{Cecità}\index{Cecità}

\textbf{10}: Sei orbo, visione laterale compromessa, problemi nel capire la distanza delle cose.
Le competenze quali Consapevolezza e i Tiro per Colpire per colpire con armi da lancio hanno un -4. La Difesa peggiora di 2.

\textbf{20}: sei cieco. Non vedi. tutti i nemici sono Invisibili.

\textbf{Cleptomania}\index{Cleptomania}

\textbf{5}: Senti il bisogno irresistibile di appropriarti di oggetti "interessanti", di tanto in tanto. Se in un giorno non hai rubato almeno un oggetto non potrai usare Punti Fato per quel giorno.

\textbf{Codice Etico/Voto}\index{Codice Etico}\index{Voto}

Hai fatto un voto, una promessa, un giuramento che condiziona il tuo agire.

5-10 : stabilisci bene le regole, nero su bianco, e sii chiaro con il Narratore

\textbf{Compulsivo}\index{Compulsivo}

Ci sono certi comportamenti, per te necessari, dei quali non puoi fare assolutamente a meno (es: camminare evitando le macchie sul terreno o passando solo su quelle, sfilare l'arma solo in un certo modo, ecc)
Questi comportamenti vanno dichiarati ed esplicitati al momento della scelta dello svantaggio.

\textbf{5-10}: quando sei preda del comportamento compulsivo hai un -2 alle prove di Consapevolezza / sei sempre l'ultimo ad agire indipendentemente dall'iniziativa tirata o dall'ordine di marcia.

\textbf{Daltonismo}\index{Daltonismo}

Sei cieco ai colori, un tramonto sarà qualcosa di triste visto in grigio

\textbf{5}: non hai la consapevolezza dei colori (acromatopsia). Vedi tutto in scala di grigi.

\textbf{Deformità}\index{Deformità}

Non tutti nascono belli o dritti. c'è anche chi nasce storto e brutto.

\textbf{5}: Malformazione minore, incide a scelta tra Forza o Destrezza o Costituzione. Togli 1 punto a questa statistica.

\textbf{10}: Due caratteristiche a tua scelta non possono superare i 2 punti se non magicamente. Hai movimento dimezzato.

\textbf{20}: Malformazione grave. Tre caratteristiche a tua scelta non possono superare 1 punto. Hai movimento dimezzato

\textbf{Depressione}\index{Depressione}

Ogni giorno è un pessimo giorno e nulla lo farà migliorare

\textbf{8}: Adori il Blues ma purtroppo hai perso la gioia di vivere, l'entusiasmo, la speranza.

Nulla sembra avere importanza, non fai che trascinarti stancamente da un giorno all'altro. -2 ad ogni prova di competenza

\textbf{Dipendenza}\index{Dipendenza}

\textbf{10}: Hai una dipendenza, possa essere alcool, droga, donne...Se non ne consumi ogni giorno una congrua dose (il Narratore ti saprà dire quanto basta) prendi un -2 a tutti i Tiri Salvezza. Dopo 3 giorni di astinenza divieni anche Depresso

\textbf{Dislessia}\index{Dislessia}

jk j0j zo mdbbdfd

\textbf{10}: Non sei in grado di leggere e scrivere. Non sei capace di associare correttamente suoni a lettere e forme a suoni

\textbf{Disonestà Compulsiva}\index{Disonestà Compulsiva}

Menti, è più forte di te.

\textbf{5}: Il personaggio è portato dalla propria insicurezza a mentire sempre e comunque. Ogni volta che il personaggio è costretto ad ammettere le proprie responsabilità o comunque a parlare contro il proprio interesse, o in qualunque situazione in cui si senta "esaminato", egli si inventerà storielle piuttosto fantasiose anche mettendo in pericolo amici e parenti.

\textbf{Dolore Cronico}\index{Dolore Cronico}

oh che male. Incantatore usi una cura su di me anche oggi ?

\textbf{10}: non recuperi Punti Ferita se non magicamente

\textbf{Emofilia}\index{Emofilia}

tendi a sanguinare sempre, anche nei momenti meno opportuni

\textbf{8}: CEROTTO!!! (ogni attacco che subisci automaticamente cumula Sanguinamento +1)

\textbf{Epilessia}\index{Epilessia}

sempre e solo nei momenti meno opportuni

\textbf{15}: ogni qual volta fai un 3 con un Tiro Salvezza o un Tiro per Colpire, cadi a terra per 1d6 round in preda alle convulsioni, si considera che il Tiro per Colpire o Tiro Salvezza sia fallito. Sei considerato indifeso.

\textbf{Feticismo}\index{Feticismo}

Se non annusi un piede di donna diventi depresso.

\textbf{5}: Il personaggio è irresistibilmente attratto da un oggetto, corpo, categoria... Ogni giorno in cui egli si trova lontano dalla sua fonte di piacere, si consideri caduto in Depressione.

\textbf{Ricordi}\index{Ricordi}

ehi.. ci sei? perché ti sei paralizzato ? e queste cose quando le hai imparate ?

\textbf{5}: ad ogni prova di competenza tira un d4. Con 1-2 fai la prova normale, con 3 fai la prova con un -2, con 4 fai la prova con un +2

\textbf{Fobie}\index{Fobie}

\textbf{Varie}: Il personaggio è terrorizzato da un oggetto, da una categoria di persone o di esseri viventi, da una situazione. In presenza della causa scatenante, il personaggio cade in preda ad un attacco di panico: l'unico suo desiderio è quello di fuggire il più possibile lontano dalla fonte del suo terrore, con ogni mezzo; chiunque gli sbarri il cammino è da considerarsi un nemico. Se il personaggio si trova nell'impossibilità di fuggire, egli cade in uno stato catatonico finché la causa scatenante non viene eliminata. Vedere in fondo tabella possibili fobie

\textbf{Fotosensibilità}\index{Fotosensibilita}

La luce anche se leggera ti da fastidio.

\textbf{5}: Il personaggio ha un -1 in ogni tiro in cui la luminosità è almeno quella diurna

\textbf{10}: Il personaggio ha un -2 in ogni tiro in cui la luminosità è almeno quella di una lanterna

\textbf{20}: Il personaggio ha un -3 in ogni tiro in cui la luminosità è almeno quella di una torcia.

Il personaggio è così sensibile che è per lui impossibile muoversi liberamente in luoghi direttamente o meno illuminati, preferirà muoversi e viaggiare di notte.

\textbf{Ghiro}\index{Ghiro}

ti piace dormire e tanto. Ronf

\textbf{5}: +2 per ogni 2 ore oltre le 8, altrimenti sei affaticato.

\textbf{Goffaggine}\index{Goffaggine}

\textbf{10}:Il punteggio della Destrezza non può superare 2. Hai un -2 a tutte le prove che richiedano Destrezza (disattivare congegni, svuotare tasche, arrampicarsi, iniziativa....)

\textbf{Igienista}\index{Igenista}

ho finito il sapone. HO FINITO IL SAPONE! .. non tocco quella spada, anche se brilla di luce sacra e vola a mezz'aria finché non sarà disinfettata!

\textbf{5}: hai l'impulso a pulirti di continuo e pulire tutto ciò che dovrai toccare.

\textbf{Incoscienza}\index{Incoscienza}

\textbf{5}: Non hai paura di nulla. Letteralmente. Se devi fare una cosa il piano più diretto ed immediato è la scelta migliore. Non riesci a studiare piani che durino più di un minuto. Prendi un +1 all'Iniziativa ed un -1 al Tiro per Colpire.

\textbf{Indeciso}\index{Indeciso}

Non facciamolo, aspettiamo domani..magari è meglio!

\textbf{10}: non agisci mai per primo. -4 alle prove di iniziativa

\textbf{Incubi Ricorrenti}\index{Incubi Ricorrenti}

\textbf{8}: Il personaggio non riesce a dormire bene. Ogni notte tira un 1d4. Con 1 il personaggio dorme normalmente, 2 o 3 il personaggio dorme un sonno agitato e si sveglia affaticato, con 4 ti svegli in piena notte urlando, la mattina sei esausto.

\textbf{Libro Aperto}\index{Libro Aperto}

si, lo so, posso stare zitto, tanto avete già capito tutto.

\textbf{5}: non è che non sei in grado di mentire è che hai un -4 alle prove di Ingannare

\textbf{Emicrania}\index{Emicrania}

Non è mai un buon giorno. Soffri di continui e feroci mal di testa.

\textbf{15}: Il personaggio soffre di violenti mal di testa. Ogni giorno il personaggio tira un d4: con 1 il personaggio non lamenta alcun effetto, con 2 o 3 subisce una penalità di -1 a tutti lle prove, con 4 la penalità diventa -2.

\textbf{Maledetto}\index{Maledetto}

Sei Maledetto. Un oscuro destino ha macchiato la tua anima

\textbf{5-10}: porti una maledizione. Discutine con il Narratore

\textbf{Miopia}\index{Miopia}

Spera di trovare degli occhiali

\textbf{5}: Ci vedi poco. Hai un -2 a tutte le prove di competenza con armi da colpire da lontano e prove di Consapevolezza oltre i 12 metri.

\textbf{15}: Ci vedi molto poco. Hai una prova -4 competenza con armi da colpire da lontano e prove di Consapevolezza oltre i 9 metri.

\textbf{Muto}\index{Muto}

Non puoi parlare e cosa peggiore non riesci neanche ad infamare il tizio che ti sta pestando il piede

\textbf{10}: Non sei in grado di emettere suoni. Non parli o meglio nessuno ti sente. Prendi un -4 alle prove basate su Carisma e competenze orali

\textbf{Discalculia}\index{Discalculia}

1+1= ?

\textbf{10}: il personaggio ha un disturbo che gli impedisce di padroneggiare il concetto di numerazione. Non solo non è in grado di svolgere le operazioni più semplici, non è neanche in grado di comprendere i concetti di maggiore/minore, o informazioni quantitative di qualunque tipo.
Attenzione al resto che ti danno...

\textbf{Obesità}\index{Obesità}

Sei decisamente fuori forma, e di tanto.

\textbf{10}: Destrezza non può essere sopra 2. Hai un -4 alla prove di Destrezza ed ai Tiri Salvezza su riflessi. Guadagni un +2 ai Tiri Salvezza su Tempra

\textbf{Olfatto/Gusto Difettoso}\index{Olfatto/Gusto Difettoso}

Naso, palato, lingua bruciata, abuso di peperoncino o wasabi.. possono essere tante le cause

\textbf{5}: -2 due alle prove che usano gusto od olfatto. Non senti sapori e odori se non estremi.

\textbf{Onestà Compulsiva}\index{Onestà Compulsiva}

\textbf{7}: Non sai mentire, la sola idea di dire una menzogna ti rende nervoso. Prendi un -4 a Ingannare. Se messo alle strette, il personaggio confesserà tutto a prescindere dall'importanza delle informazioni in suo possesso.

\textbf{Ossa di Cristallo}\index{Ossa di Cristallo}

Si chiamerebbe osteogenesi imperfetta ma per te sono solo dolori continui.

\textbf{5}: Il personaggio ha le ossa fragili. Ogni danno causato da arma da botta causa 2 Punti Ferita in più di danno

\textbf{10}: Il personaggio ha le ossa fragili. Ogni danno causato da arma da botta causa 5 Punti Ferita in più di danno

\textbf{Monco}\index{Monco}

Sei monco, a te la scelta quale sia la mano.

\textbf{7}: ti manca la mano secondaria

\textbf{13}: ti manca la mano primaria. -2 a tutti i tiri che coinvolgono l'uso della mano.

\textbf{Paranoioso}\index{Paranoioso}

Sei paranoico e noioso.

\textbf{5}: Ti comporti sempre in modo furtivo, anche senza che ce ne sia effettivo bisogno, destando così sospetti nelle persone che hai attorno.

Ogni prova di Consapevolezza contrapposta ha una difficoltà di -5 aggiuntiva ed un fallimento critico indica che il target ha qualcosa di vitale da nascondere.

\textbf{Pelle Sensibile}\index{Pelle Sensibile}

Non ami il Sole, o almeno la tua pelle non lo ama.

\textbf{5}: Il tuo personaggio si scotta facilmente, un'esposizione prolungata senza le adeguate protezioni comporta dolorose e antiestetiche bruciature e disagi.

\textbf{10}: Sei oltremodo sensibile agli ultravioletti. Ogni danno da fuoco o luce causa 2 danni aggiunti.

\textbf{Pigro}\index{Pigro}

sei lento e svogliato

\textbf{5}: -2 all'iniziativa

\textbf{Rumoroso}\index{Rumoroso}

Non lo fai apposta, ma c'è sempre un qualche rumore intorno a te. Una spada che sbattocchia, uno sbadiglio, un rutto, una scarpa rumorosa..

\textbf{5}: hai un -4 alle prove di muoversi silenziosamente

\textbf{Sangue Debole}\index{Sangue Debole}

\textbf{10}: Il sistema immunitario del personaggio fa decisamente pena. -2 ai Tiri Salvezza su Tempra

\textbf{Sbadataggine}\index{Sbadataggine}

Ops..non me ne ero accorta!

\textbf{7}: Tendi a non fare caso a quello che succede intorno a te, meno che tu non abbia ottimi motivi per stare all'erta, o non stia cercando attivamente qualcosa prendi un -4 a Consapevolezza

\textbf{Schizofrenia}\index{Schizofrenia}

Non sono stato io, ma l'altro!

\textbf{4}: Hai più personalità, o forse ne è convinto l'altro.

Il personaggio ha almeno una seconda personalità (max 6).
Ogni Personalità in più da gestire, oltre la prima, concede un +1 al costo.
Quindi avere 3 personalità porta lo svantaggio a 6 punti

Ogni giorno viene tirato 1d6. Con 6, durante il giorno la seconda (o terza) personalità viene alla luce.

\textbf{Sfortunato}\index{Sfortunato}

le cose non capitano e basta, bisogna saperle anche cercare

\textbf{5}: ignori il primo critico che fai (TC o TS) nella giornata

\textbf{7}: ignori i primi tre critici che fai (TC o TS) nella giornata

\textbf{Sindrome Maniaco Depressiva}\index{Sindrome Maniaco Depressiva}\index{Depressione}

Oggi è venerdì !!! è Venerdì!!!

\textbf{7}: Il personaggio alterna stati di euforia a momenti di cupa disperazione. Ogni giorno viene tirato 1d4. Con 1 il personaggio ha un umore "normale". Con 2 o 3 si consideri in Depressione, con 4 è in uno stato di gioiosa esaltazione (vedi Incoscienza ) e spavalderia.

\textbf{Soggezione}\index{Soggezione}

chiedo scusa

\textbf{10}: Il personaggio è molto insicuro e tende a fidarsi ciecamente degli altri, specie se carismatici . Prendi un -2 alle prove di Intimidire e Intrattenere
Prendi un -2 ai Tiri Salvezza su Charme e Dominazione

\textbf{Sonno Leggero}\index{Sonno Leggero}\index{affaticato}

Ogni rumore di disturba, non riesci mai a dormire bene

\textbf{5}: Se dormi in una zona con rumori naturali / umani (bosco/città) non riesci a riposare bene. La mattina sei Affaticato. Puoi evitare il problema usando tappi per le orecchie, che ti impongono un -4 alle prove di Consapevolezza su udito per svegliarti.

\textbf{Sordità}\index{Sordità}

Il silenzio ha un suono tutto suo dice chi ci sente, per te è solo uno straziante urlo muto.

\textbf{10}: Non ci senti. Non puoi fare prova di Consapevolezza che richiedano l'uso dell'udito. Non puoi ascoltare le persone che parlano. Ma puoi leggere le labbra se sai farlo.

\textbf{Vertigini}\index{Vertigini}

I disagi si manifestano nel momento in cui il personaggio è conscio dell'altezza. Solo per il fatto di camminare in montagna non ha penalità

\textbf{5}: Ad altezze superiori i 20 metri tendi a bloccarti. Prendi un -2 a tutti i tiri

\textbf{7}: Ad altezze superiori i 10 metri tendi a bloccarti. Prendi un -2 a tutti i tiri

\textbf{10}: Ad altezze superiori i 6 metri tendi a bloccarti. Prendi un -2 a tutti i tiri

\textbf{Visione notturna ridotta}\index{Visione notturna ridotta}

I tuoi occhi non lavorano bene con luminosità ridotta.

\textbf{5}: Quando la luminosità è pari o inferiore a quella di una torcia (o fioca) il personaggio ha un -2 ai Tiro per Colpire.

\textbf{Timidezza}\index{Timidezza}

\textbf{5}: Sei timido e riservato.

Hai un -2 alle prove basate su Carisma

\textbf{Zoppo}\index{Zoppo}

sei claudicante

\textbf{5}: Hai movimento dimezzato

\textbf{10}: sei significativamente storpio. -2 alle prove che richiedono Destrezza, il tuo movimento è dimezzato

\bigskip

\end{multicols}

\textbf{Tabella Fobie (5-15 punti)}\index{Fobie}\index{Tabella delle Fobie}

\begin{tabular}{ll}
	\textbf{Nome Fobia} & \textbf{Descrizione}\\
	\toprule
	Blennofobia         & Paura Delle Cose Viscide\\
	Keraunofobia        & Paura Dei Tuoni\\
	Ipocondria          & Paura Delle Malattie\\
	Claustrofobia       & Paura Dei Luoghi Chiusi\\
	Coimetrofobia       & Paura Del Cimitero\\
	Edonofobia          & Paura Di Poter Provare Piacere Fisico\\
	Eisoptrofobia       & Paura Degli Specchi\\
	Glossofobia         & Paura Di Parlare In Pubblico\\
	Monofobia           & Paura Di Rimanere Solo\\
	Necrofobia          & Paura Dei Corpi Morti\\
	Nictofobia          & Paura Del Buio\\
	Acrofobia           & Paura Delle Altezze\\
	Agorafobia          & Paura Degli Spazi Aperti\\
	Rupofobia           & Paura Dello Sporco E Non Igienico. Senti il bisogno di pulire\\
	Afefobia            & Paura Del Contatto E Di Essere Toccati\\
	Asimmetrofobia      & Paura Delle Cose Non Simmetriche\\
	Gimnofobia          & Paura Della Nudità\\
	Emofobico           & Paura Del Sangue\\
	Traumatofobia       & Paura Di Ferirsi\\
	Sciofobia           & Paura Delle Ombre\\
\end{tabular}


\pagebreak

\section{Cosmologia}\index{Cosmologia}

\label{cosmologia}
\begin{enfasi}{
E' più facile dominare su chi non crede in niente (La Storia Infinita, Kmorf)

\medskip

Tu credi che c'è un Dio solo? Fai bene; anche i demoni lo credono e tremano! (Giacomo Il Giusto 2, 19. NdA Riferendosi al proprio Patrono...)}\end{enfasi}\medskip

\begin{narratore}
In DBS le divinità sono diverse dalle tradizionali divinità dei giochi di ruolo.

In DBS le divinità amano sporcarsi le mani, partecipare nelle faccende delle creature che le adorano, per loro è una sfida continua ad avere più credenti, adepti e persone più simili, per Tratti, a loro.

I Patroni sono stati creati come parossismo dell'animo umano, dove tutto è un eccesso. Come spiriti liberati dal vaso di Pandora hanno il solo scopo di portare i loro Tratti al dominio rendendoli i più comuni e presenti tra le creature.
\end{narratore}

\begin{multicols}{2}

\bigskip

\lettrine[lines=2, lhang=0.33, loversize=0.25, findent=1.5em]{I}{n} principio era il nulla che in sé conteneva il tutto.

Energia derivante dalle più primordiali pulsioni esplodeva in tutta la sua potenza e senza alcun controllo.

Amore, odio, paura, dolore, gioia, serenità...tutto era aggrovigliato in una fitta ed infinita matassa il cui bandolo era nascosto, attorcigliato, introvabile o...probabilmente ancora inesistente.

Nei millenni a seguire dette energie e pulsioni hanno iniziato a scindersi, trovando una propria personale connotazione e si sono venute a creare tre entità: Atmos, colui che per suo principio controllava l'andamento del tempo e dello spazio, il testimone, lo scriba; Ljust l'energia positiva, il calore, la luce e la vita; Calicante, energia negativa, gelido odio, distruzione e morte.

Atmos è una sorta di spettatore imparziale mentre Ljust e Calicante rappresentano le due lingua di fiamma di un unica energia creatrice.

\textbf{Ljust} \index{Ljust}è la rappresentazione di ciò che luce ed amore portano sempre con sè. Rappresenta la purezza del sentimento d'amore, la protezione della vita, il rispetto per l'altro, la curiosità per il nuovo, la voglia di migliorarsi sempre, la forza di combattere con coraggio e valore per il proprio credo. E' la spinta vitale al cambiamento, il chaos che trasforma ma non distrugge.

\textbf{Calicante}\index{Calicante} è la rappresentazione del buio, dell'odio e della rabbia. Lui è vendetta e fredda distruzione. Lui non protegge alcuna forma di vita, le usa, le sfrutta e solo in tali casi ne subisce la presenza. Lui ama sadicamente la sofferenza. E' l'entropia che annienta e annichilisce.

\textbf{Atmos} \index{Atmos}è lo storico, colui che segna il passaggio del tempo e trascrive ogni accadimento di Yeru. 

E' il testimone del groviglio divino che sono Ljust e Calicante, due creature unite da un unica energia.

Insieme i due Patroni della Genesi hanno dato vita a tutto ciò che conosciamo. Calicante ha creato Tiya\index{Tiya} ed Ljust ha generato Curyan\index{Curyan}, i due regni che compongono il nostro mondo, Yeru. 

Hanno giocato con forme ed energie creando due regni fra loro speculari ma contrapposti e distinti. Tiya e Curyan, come Calicante ed Ljust, sono parte di un tutto ma, esattamente come i Patroni delle Genesi, sono anche profondamente diversi e magicamente divisi. Esiste infatti una barriera sia fisica formata da tempeste marine perenni e mortali ma anche magiche, che ne delimitano i confini e che li tiene nettamente divisi.

Ma proprio come i loro due creatori che divisi e distanti totalmente non possono stare, non possono esistere, così Tiya e Curyan sono sì divisi ma anche in contatto fra loro tramite i Portali. Portali che si generano autonomamente, senza alcun controllo e previsione, a causa dell'entropia magica che preme, spinge e si autoalimenta nel "non luogo" al confine dei due regni e che è generata dalle continue sfide fra i due Patroni.

Sono queste magiche vie che consentono di spostarsi tra Tiya e Curyan e viaggiare nel "non luogo" ovvero ciò che è al di fuori di Yeru.

Ljust e Calicante decisero, stranamente di comune accordo, di generare un Patrono che sovrintendesse a queste fratture, che fosse capace di percepire, aprire e bloccare questi Portali. Così venne creato \textbf{Lynx}, il Guardiano dei Portali.

In molti cercano di passare da Tiya a Curyan per cercare la pace, la serenità...altri cercano di valicare il confine inverso alla ricerca di avventura, potere\ldots .alcuni ci provano per vie normali, altri attraversando i Portali, molti si sono persi per sempre nel "non luogo".

Lynx \index{Lynx}sovrintende allo spazio, ai portali che con l'avvicendarsi di caos e ordine, di bene e male, di luce e tenebra stanno sempre più creando fratture al confine esistente fra i due regni. Lynx li percepisce, li "sente", sa dove si stanno generando o riassorbendo, con il passare del tempo infatti alcuni di questi Portali sono divenuti stabili e definitivi mentre altri sono stati riassorbiti dalla naturale entropia, altri invece si generano casualmente e sempre in modo totalmente casuale rimangono attivi o si esauriscono. Viaggiando di continuo nel non luogo Lynx chiude i portali più grandi ma per uno che ne chiude un altro si apre.

Proprio nello svolgimento di questo suo importante ruolo Lynx si scontrò con una strana creatura, rettiloide, gigantesca, alata, potente, forte, sapiente e magica.

Un Drago rosso,\index{Tàhil} Tàhil. Quest'ultimo si muoveva nel "non luogo" con la massima libertà, senza alcuna difficoltà e si avvicinò a Lynx. I diari di Atmos narrano di come Lynx cercò di fermarlo e di parlarci, di come venne ferocemente attaccato, delle urla del Patrono Custode che si sentirono echeggiare in entrambi regni, del suono quasi simile ad un gutturale ruggito che squarciò il silenzio nei regni di Tiya e Curyan. Dell'intervento di Ljust e Calicante. La prima a salvare Lynx ed il secondo a scoprire, conoscere questa nuova affascinante "arma".

Lynx si salvò. Ljust gli infuse le sue magie di cura e lo aiutò a rigenerarsi. Si lasciò però sfregiato a memoria dell'incontro.

I Draghi avendo scoperto il nostro mondo e mossi dalla loro sete di conoscenza e di potere si sono avventurati nel "non luogo" e sono usciti dai Portali presenti su Tiya e Curyan.

Un'orda di draghi di tutti i colori ha oscurato i cieli. Da quel momento saccheggi, razzie e violenza furono perpetrati indifferentemente nei due regni. Erano molto intelligenti e furbe. Potenti oltre l'immaginabile, manipolavano una magia per alcuni versi diversa e slegata dalle magie tradizionali.

Avevano una robustezza fuori dall'ordinario. Ma soprattutto, non temevano i Patroni. Non si sottomisero a loro.

Atmos incanalò le energie primordiali e divine dei Patroni della Genesi andando a creare delle divinità che potessero rivaleggiare con i draghi e potessero difendere Yeru.

Il primo creato da Atmos, con l'aiuto di Ljust fu \textbf{Gradh}\index{Gradh}, Patrono dell'Umanità (e di tutte le razze senzienti), colui che avrebbe difeso il creato dai draghi e dagli altri Patroni.

Gradh racchiude in sé il dualismo dei due Patroni della Genesi, l'istinto innato alla protezione, alla difesa ed alla cura propri di Ljust e l'istinto di vendetta, violenza e furia di Calicante.

Si getta con coraggio nelle battaglie, attacca il nemico senza paura, protegge il più debole, difende la vita ma non teme di percorrere la strada della vendetta più distruttiva verso chi sfrutta e distrugge vite senza motivo.

Gradh ama "calarsi" fra la gente e vivere con loro, come loro.
Non si sente totalmente a suo agio nè nel Pantheon con gli altri Patroni nè fra la gente comune, lui è Umano tra i Patroni e Patrono tra gli Umani. Passionale e gentile è il Patrono che maggiormente ha a cuore le sorti di Yeru e delle sue razze.

Le lingue di energie divine erano troppo intense, chaotiche e pure perché Atmos potesse governarle per plasmare da solo gli ulteriori Patroni con la stessa naturalezza con cui aveva creato Gradh. Ed ecco che questi altri Patroni risultano meno perfetti e divini, più imperfetti ed "umani" in quanto originati da pulsioni violente e incontrollabili, vere e non filtrate in alcun modo.

I Patroni plasmano le volontà, fondano regni, comandano nell'ombra come pedine le creature che osano chiedere i loro favori.

Gradh percepì sin da subito che i Draghi rappresentavano un elemento di ulteriore chaos, di ulteriore sofferenze e guerra. Come Patrono di Yeru e delle sue creature sentiva i Draghi come creature aliene, non originarie, non facenti parte del piano della Genesi.

Diffidente per natura Gradh decise di proporre ai Patroni della Genesi di fare un patto con i Draghi.

Ecco che poco più di 300 anni fa, il 15 Prineva del 65 del sesto ciclo, sull'isola di Alantia che divide Tiya e Curyan si trovarono Atmos, la fiamma di Ljust e Calicante e Gradh da una parte mentre Tàhil, il drago rosso malvagio e immortale e Dyenos\index{Dyenos}, il drago d'argento sapiente e buono dall'altra.

Gradh cercò di imporre la cacciata dei Draghi e la chiusura dei Portali, Atmos rimase in silenzio a trascrivere la discussione. Ljust cercò di mediare capendo che non tutti i Draghi erano malvagi e che avrebbero potuto dare tanto a Yeru.

Calicante finse di dare ragione a Ljust con il solo scopo di portare maggior caos e distruzione attraverso i Draghi.

Capito che l'esito dell'incontro era già deciso Gradh abbandonò la Piana della Solitudine lasciando ai Draghi ed ai Patroni della Genesi di formalizzare la spartizione di Yeru. Per lui era stata una sonora sconfitta e da allora fu ancora di più è diffidente, se non prevenuto, verso tutti i draghi.

Calicante e Tàhil rimasero strettamente in contatto e si stanziarono a Tiya mentre Dyenos giurò fedeltà e fiducia a Ljust e decisero di governare insieme Curyan, grazie allo loro volontà di calarsi tra gli umani.

Lynx da spettatore esterno non rimase senza fare niente, temendo il peggio creò un Portale che portasse ad un nuovo Yeru, un pianeta diverso e fuori dall'influenza di Draghi e Patroni.

Le sue ricerche lo portano in un mondo quasi idilliaco ricco di natura, vita animale ed senza Patroni, come a lui piace.
Lo chiamò Ker\index{Ker} in memoria di una antica amata.

Sono pochissime le informazioni su questo mondo, solo pochi alti Devoti di Lynx lo conoscono e ancora meno lo anno visitato.

Rimane il fatto che il portale di congiunzione esiste ed è stabile, dove sia esattamente non è noto ma il suo dono a Yeru lo ha già fatto, gli Gnomi\index{Gnomi}.

Se un Patrono agisce in prima persona o in modo indiscriminato sa che scatenerà la reazione di Gradh o l'intervento di Atmos che gli impediranno un uso incontrollato e massivo dei suoi poteri direttamente sul mondo. Questo però non sempre li ferma e la stessa natura creature e piante, vengono spesso influenzate dal volere dei Patroni.

A Tiya, ma a volte anche a Curyan, nascono sempre più spesso aberrazioni, malattie sempre nuove, terre maledette dove non può crescere nulla, per non parlare di pazzie che spesso coinvolgono chi invece dovrebbe proteggere i comuni cittadini.

E' una dura vita quella dell'uomo comune che continuamente deve affrontare siccità o alluvioni, morie di animali ed un meteo irregolare se non assurdo. Ad ogni passo deve guardarsi intorno perché non puoi mai sapere chi ha venduto l'anima per vivere un giorno in più.

A Curyan si vede svilupparsi l'armonia e la quasi perfetta convivenza fra natura e razze superiori. Esiste il dolore, esiste la malattia e la morte ma il tutto come naturale ciclo della vita come parte della stessa che viene protetta, guidata, aiutata.

I nemici principali sono i Draghi che spesso fanno incursioni per portare distruzione e morte e seminare paura e pazzia.

Non è sempre tutto idilliaco, vaste regioni di Curyan stanno diventando incubatrici di razze oscure e malvagie, legioni di non morti guidate da potenti negromanti si ammassano sui confini, i Draghi addestrano i loro adepti corrotti, e scure spire nere nel cielo promettono tempesta.

\subsection{Patroni}\index{Patroni}\hypertarget{patroni}{} 

\label{patroni-dei}
\begin{enfasismall}{
Conan: A quali dei preghi?

Subotai: Io prego ai quattro venti e tu?

Conan: Io prego Crom, ma solo raramente... lui non ascolta. (Conan il Barbaro, film 1982)

\medskip

Infatti, come il corpo senza lo spirito è morto, così anche la fede senza le opere è morta. (Giacomo Il Giusto 2, 26. NdA Riferendosi ai punteggi dei Tratti collegati al Patrono...)} \end{enfasismall}\medskip

Le creature tutte, anche chi non usa la magia possono sentire l'influenza di questi Poteri, di questi Patroni.

Se un personaggio per il suo modo di essere (giocare) e comportarsi ha almeno un Tratto in comune con un Patrono ed anzi matura e potenzia queste convinzioni, anche se non ha giurato fedeltà ad un Patrono potrebbe comunque sentire l'influenza del Patrono e ricevere dei doni da lui.

Un Patrono è ben contento se qualcuno segue i suoi dettami, Tratti, senza che sia un usufruitore di magia e dona a coloro che lo fanno dei piccoli poteri come riconoscimento per la fedeltà a lui riservata, volutamente o meno. I poteri indicati sotto "Tratti in Comune" sono cumulativi. Se non indicato diversamente i poteri sono usabili 1 volta al giorno e costano 2 Azioni utilizzarli se non specificato diversamente.

Ogni \textbf{Patrono predilige uno o più Elementi}, se sei un Devoto o Seguace puoi usare quell'energia nelle tue magie o meno.

Le forme di Energia vengono distinte tra fonti positive, neutrali e negative, vi servono anche per inquadrare meglio il vostro Padrone pardon il Patrono che servite.

Fate la somma degli elementi, se positiva il Patrono si può considerare buono, se a valore zero il Patrono è neutrale, se a valore negativo il Patrono è malvagio.

Nella descrizione del Patrono troverete anche la sua manifestazione, ovvero cosa accade quando un personaggio agisce in maniera particolarmente e significativamente consona ai Tratti seguiti dal Patrono. L'effetto è puramente ambientale e di circostanza ma lascia sempre colpito chiunque lo possa osservare.

Un incantatore che si affida ad un Patrono, almeno 3 Tratti in comune, diventa un Devoto.

Se ha almeno 2 Tratti in comune e si affida ad un Patrono allora si dice che è un Seguace.

Potrebbe anche non seguire alcun Patrono pur avendo più Tratti in comune.

\begin{narratoresmall}
Il Narratore potrebbe comunque concedere l'essere Seguace o Devoto pur se i Tratti non collimano perfettamente. A sua discrezione e su richiesta del giocatore potrebbe valutare la somiglianza di alcuni Tratti del personaggio a quelli del Patrono e valutarli idonei per essere un Seguace o Devoto.
\end{narratoresmall}

Le capacità acquisite legate ai Tratti in comune sono indipendenti dall'essere un Devoto, Seguace o semplicemente "ateo". Il \textbf{Vantaggio} indicato è solo per il Devoto. \index{Vantaggi}

Infine ogni Patrono concede l'energia magica per lanciare gli incantesimi e predilige una forma di magia (scuola) tra tutte, su questa, indicata nella descrizione del Patrono. Il Devoto ha un +4 alle prove di magia mentre il Seguace +2.

\bigskip

\textbf{Tabella Elementi - Energia}\index{Tabella degli Elementi}

\medskip

\begin{tabular}{lll}
	\textbf{Positivi} (+1) & \textbf{Neutrali} (0) & \textbf{Negativi} (-1)\\
	\toprule
	En. Positiva       & Fuoco                 & En. Negativa\\
	Luce                   & Freddo                & Vuoto\\
	& Suono                 & \\
	& Elettricità  & \\
\end{tabular}

\medskip

\textbf{I Patroni sono}:

\subsubsection{Ljust}

\label{ljust}\index{Ljust}

La Dama della Luce, colei che irradia calore e amore. Generatrice delle pulsioni d'amore, protezione, gentilezza, gioia e perdono. Racchiude in sé l'aspetto protettivo di una madre, la forza e l'audacia di una combattente, la passionalità di una giovane amante e l'allegria, la ricerca del nuovo, la fantasia di una bambina. Ljust incarna la bellezza della vita ed ogni creatura che la contempla vede quella che per lei è la massima armonia e cade prona al suo fascino.

Ljust può essere scelta solo da un personaggio con 4 Tratti in comune con lei, fondamentalmente si nasce per essere Devoti di Ljust. Nel corso delle ere Ljust decise di selezionare, scegliere e premiare le donne che più mostravano in modo innato e profondo amore per la vita, curiosità per il nuovo, forza incrollabile, dedizione, fiducia, rispetto e cura degli altri donando loro i poteri e la possibilità di studiare e crescere come Allieve della Luce. Queste Allieve devono seguire la regola degli 8 Passi.

\begin{itemize}

\item \textbf{Simbolo:} Una stella circondata da raggi solari
\item \textbf{Caratteristica}(Devoto): Saggezza o Carisma
\item \textbf{Tratti}: Coraggioso, Generoso, Empatico, Protettivo, Istintivo, Anticonformista
Il Devoto di Ljust ha 4 Tratti in comune con il Patrono.
\item \textbf{Manifestazione}: luce dorata inonda l'incantatore.
\item \textbf{Somma dei Tratti in comune a 5 punti}: puoi lanciare l'incantesimo Luce solo come atto di volontà
\item \textbf{\textbf{Somma dei Tratti in comune a 10 punti}}: guadagni un +2 ai Tiri Salvezza su Tempra
\item \textbf{Somma dei Tratti in comune a 15 punti}: una armatura di luce ti protegge, guadagni un +2 a tutti i Tiri Salvezza
\item \textbf{Somma dei Tratti in comune a 20 punti}: puoi emettere un globo di luce che causa 10d6 di danno. Distanza 54 metri, Tiro Salvezza Riflessi DC 25 per dimezzare, un target
\item \textbf{Elementi} (Seguace/Devoto): Energia Positiva, Luce
\item \textbf{Vantaggio} (Devoto): Cure efficaci
\item \textbf{Scuole Privilegiate}(Seguace/Devoto): Cura, Abiurazione (+2 alle prove di CM)

\end{itemize}

\textbf{Gli 8 Passi delle Allieve}\index{8 Passi delle Allieve}\index{Allieve}

\label{gli-8-passi-delle-allieve}

Le Allieve della Luce sono una gruppo segreto di Devote che per totale affinità con Ljust hanno intrapreso il duro percorso del bene e dell'amore. E' tra i gruppi più antichi fondati a Yeru.

Le Allieve, 99 come numero massimo, ma purtroppo spesso meno numerose, sono Devote di Ljust e devono seguire gli 8 Passi della Luce

\begin{enumerate}
	\item Ama e proteggi con tutta te stessa, con totale e sincera dedizione chi hai attorno a te.

	\item Non lasciare che la tua inazione generi sofferenza.

	\item Si un punto di paragone. Fai che la tua Luce elevi le persone che hai intorno e possano vedere in Tu sei speranza, serenità, calma, protezione e sicurezza.

	\item Usa l'intelligenza, la furbizia e l'arguzia. Si lungimirante e risoluta nell'azione.

	\item La tua opera è per il bene comune. Fa che la tua Luce sia sempre alta ed intensa.

	\item Non cercare altra Luce se non la tua e quella delle tue sorelle.

	\item Sii luminosa ma non accecare chi è intorno a te.

	\item Sii la differenza tra la disperazione e la speranza.
\end{enumerate}

Le Allieve hanno costruito un ballo armonioso trasformando in danza i passi della loro Regola.

Esistono anche Allieve di genere incerto, rari ma storicamente accertati.

\subsubsection{Calicante}\index{Calicante}


\begin{enfasismall}{
La superstizione è la religione degli spiriti deboli. (Edmund Burke)
}\end{enfasismall}

\label{calicante}

E' oscuro, gelido e arrabbiato. Racchiude in sé odio, violenza, distruzione, vendetta e perenne insoddisfazione. Raccoglie la personalità capricciosa e scontenta di un bambino, la noia violenta e sadica di un giovane uomo, la forza distruttiva di un uragano e la rabbia di un combattente che non ha più nulla da perdere. Calicante solo con la presenza mette a disagio, ti fa sentire in pericolo, affascina ma con le armi della paura e della incostanza.

Calicante può essere scelto solo dai personaggi che hanno 4 Tratti in comune con lui. I suoi Devoti sono i migliori assassini, sua professione più affine. Coloro che mostrano il maggiore sprezzo del pericolo e della vita altrui. I suoi prediletti sono coloro che sono temuti, odiati, coloro che sono violenti e crudeli ma mortalmente efficienti e decisivi in ogni situazione di combattimento.

\begin{itemize} 
\item \textbf{Simbolo}: Un turbine nero
\item \textbf{Caratteristica}: Forza o Destrezza
\item \textbf{Tratti}: Egoista, Vendicativo, Superbo, Iracondo, Passionale, Diretto
Il Devoto di Calicante ha 4 Tratti in comune con il Patrono.
\item \textbf{Manifestazione}: spada grondante di sangue nero
\item \textbf{Somma dei Tratti in comune a 5 punti} punti: Puoi creare una zona di oscurità. Raggio 1 metro, entro 9 metri, durata 10 minuti. Una volta al giorno
\item \textbf{Somma dei Tratti in comune a 10 punti}: La tua lama si ammanta di ombra. Guadagni un +2 al Tiro per Colpire e +1d4 di danno per 2d6 round, Una volta al giorno.
\item \textbf{Somma dei Tratti in comune a 15 punti}: Crei 4 dardi di energia negativa. Ogni dardo fa 2d6 di danno, colpisce automaticamente entro 18 metri. Una volta al giorno.
\item \textbf{Somma dei Tratti in comune a 20 punti}: Crei una zona di energia protettiva attorno a te nel raggio di 3 metri, dimezzi tutto il danno che ricevi, non è possibile recuperare Punti Ferita nell'area. Durata 10 minuti consecutivi, 1 volta al giorno.
\item \textbf{Elementi}: Energia Negativa, Vuoto
\item \textbf{Vantaggi}: Scudo Mentale
\item \textbf{Scuole Privilegiate}: Invocazione, Necromanzia (+2 alle prove di CM)
\end{itemize}

\subsubsection{Atmos}\index{Atmos}

\label{atmos}

il custode del Tempo e della Torre dell'Orologio, come ha avviato il tempo e la creazione dei nuovi Patroni così fermerà la sfida fra loro, i Patroni sopravvissuti saranno giudicati, le loro opere valutate e Ljust o Calicante ne trarranno giovamento. Come una sfida da una singola moneta di rame nuovi Patroni, nuovi ideali saranno creati e noi, piccole creature vedremo nascere nuove civiltà e regni fiorenti. La storia è poco nota, solo i pochi Devoti di Atmos, scribi e studiosi della biblioteca del Tempo, conoscono il segreto e lo scorrere del tempo e della gara, gli altri, ignoranti, vivranno il loro tempo con un padrone sicuramente guidato da un Patrono.

Atmos, il Patrono del Tempo è il custode della storia e del tempo, è colui che tiene traccia dei mille e più mondi che sono stati creati.

Ha il compito di avviare ed interrompere il tempo. Atmos ha il potere unico e riservato solo a lui di poter bandire dal creato un Patrono qualora questo diventasse troppo forte e minacciasse Calicante e Ljust. Atmos ha già usato questo potere in passato. Atmos sia per la sua natura totalmente neutrale sia per il suo ruolo non si è mai schierato. 

Tutti i Patroni temono Atmos per il suo potere, il più terribile per loro, ovvero il loro alienamento, l'oblio, la dimenticanza l'essere distolti dal tempo e dalla sfida.

Per essere un Devoto di Atmos al momento del rito è necessario che il futuro Devoto possieda almeno quattro Tratti in comune con lui, amare la storia e la conoscenza.

Vestito di un morbido saio marrone e calzari di cuoio si muove tra gli infiniti scaffali della Biblioteca del Sapere con sempre uno strano misuratore del tempo appeso alla vita.

\begin{itemize} 


\item \textbf{Simbolo:} Un libro bianco con un orologio da taschino appoggiato sopra
\item \textbf{Caratteristica}: Intelligenza
\item \textbf{Tratti}: Osservatore, Distaccato, Prudente, Riflessivo, Integerrimo, Crudele
\item \textbf{Manifestazione}: l'incantesimo si sviluppa come a rallentatore, è solo un effetto illusorio
\item \textbf{Somma dei Tratti in comune a 5 punti}: Conosci sempre la data esatta e l'ora.
\item \textbf{Somma dei Tratti in comune a 10 punti}: Hai una intuizione innata per la conoscenza. Hai +1d6 alle prove di Conoscenza ed Arcana prende un bonus +2
\item \textbf{Somma dei Tratti in comune a 15 punti}: Puoi creare 8 tue immagini speculari per trarre in inganno i tuoi avversari. Una volta al giorno, durata 1 ora o finché colpite.
\item \textbf{Somma dei Tratti in comune a 20 punti}: Ogni qual volta che devi fare una prova di Arcana puoi prendere il 18 come prendessi 10
\item \textbf{Elementi}: Suono, Freddo
\item \textbf{Vantaggio}: Senso del tempo
\item \textbf{Scuole Privilegiate}: Divinazione (+2 alle prove di CM)

\end{itemize}

\subsubsection{Lynx}\index{Lynx}

\label{lynx}

Patrono dei Portali, è sceglibile solo da personaggi che abbiano almeno 3 Tratti in comune. E' il primo Patrono generato da Ljust e Calicante, creato per proteggere Yeru dagli attacchi esterni.

Serio, occhi gelidi di un azzurro chiarissimo è il Custode dei Portali e di ciò che è Oltre. Letale guardiano per chi cerca di passarli senza permesso, guida attenta per chi chiede il suo aiuto ed il suo permesso. Si fa scudo delle sue cicatrici per allontanare tutti. E' il solitario controllore del mondo.

I suoi Devoti sono i viaggiatori per eccellenza, coloro che presidiano e proteggono Yeru da ciò che è alieno, da ciò che potrebbe disturbare la creazione.

\begin{itemize} 
\item \textbf{Simbolo}: Un portale
\item \textbf{Caratteristica}: Destrezza
\item \textbf{Tratti}: Solitario, Serio, Rigido, Controllato, Coraggioso, Imbronciato
\item \textbf{Manifestazione}: come se il panorama non avesse più orizzonte
\item \textbf{Somma dei Tratti in comune a 5 punti} punti: Una volta al giorno puoi eseguire una Azione di movimento in piu'
\item \textbf{Somma dei Tratti in comune a 10 punti}: Acquisisci una Azione di movimento in più a round
\item \textbf{Somma dei Tratti in comune a 15 punti}: Puoi toccare una creatura extraplanare e costringerla a tornare sul suo piano. Tiro Salvezza su Volontà DC 30. Una volta al giorno
\item \textbf{Somma dei Tratti in comune a 20 punti}: Puoi teletrasportarti per 500km al giorno (anche più teletrasporti purché la somma totale non superi 500km)
\item \textbf{Elementi}: Fuoco, Elettricità
\item \textbf{Vantaggio}: Lento e Fermo
\item \textbf{Scuole Privilegiate}: Evocazione (+2 alle prove di CM)
\end{itemize}

\subsubsection{Gradh}\index{Gradh}

\label{gradh}

Il primo Patrono creato da Atmos sotto la guida di Ljust e l'influenza di Calicante.

Gradh racchiude in sé l'istinto innato alla protezione, alla difesa ed alla cura propri di Ljust. Gradh è quanto di più simile e profondamente legato a Ljust sia stato generato. Lui è equilibrio, razionalità ed empatia.

Dove vi è difesa, cura e protezione vi è Gradh.

Gradh non ama sfidare apertamente Cattalm perché sa che farebbe esattamente il suo gioco, ecco che con astuzia cerca di attirarlo nel suo terreno di gioco, dove nessuna vita sarà in pericolo e lì da sfoggio a della sua superiorità strategica e di combattimento.

Ma Calicante non poteva permettere la creazione di un Patrono totalmente votato ad Ljust e così infuse in Gradh la freddezza della vendetta e la furia della rabbia. Ecco che allora Gradh nell'atto di difendere l'umanità, spesso la deve in primis proteggere da sé stesso.

Passionale e freddo è forse il Patrono più umano del Pantheon attuale. Il suo sguardo caldo e carismatico che quando ama e protegge è di un rassicurante color cioccolato, può divenire freddo e tagliente con le sfumature della fredda terra ghiacciata quando è preda della furia della battaglia o della vendetta. Gradh ama studiare il mondo attorno a sé e passare inosservato. Spesso si nasconde fra la gente e "vive" la sua vita umana. Ma non si lascia avvicinare veramente da nessuno.
Gradh attira a sé con la stessa facilità con cui allontana da sè.

Il Devoto di Gradh è fiero ed orgoglioso, indomito e protettivo, ed addolorato, perché per quanto si sforzi di punire il male questo continua sempre a prosperare.

\begin{itemize}
	
\item \textbf{Simbolo}: Uno scudo con incise sopra due spirali intrecciate.
\item \textbf{Caratteristica}: Forza
\item \textbf{Tratti}: Indomito, Protettivo, Vendicativo, Coraggioso, Freddo, Diffidente
\item \textbf{Manifestazione}: due spire una nera come ombra ed una lucente come scintilla circondano la sua arma intrecciandosi
\item \textbf{Somma dei Tratti in comune a 5 punti} punti: Il tuo tocco cura 3d6 Punti Ferita, ma ti causa 1d6 di danno. 2 volte al giorno
\item \textbf{Somma dei Tratti in comune a 10 punti}: Per 10 minuti consecutivi hai un bonus di +4 Tiro Salvezza su Riflessi e Tempra. Una volta al giorno
\item \textbf{Somma dei Tratti in comune a 15 punti}: Emani un aura che concede a tutti i tuoi compagni entro raggio 3 metri un +2 TS. Una volta al giorno, per 30 minuti consecutivi
\item \textbf{Somma dei Tratti in comune a 20 punti}: Esplodi la tua ira in una palla di energia negativa. 10d6 di danno, raggio 6 metri entro 36 metri. DC 25 Riflessi. Una volta al giorno
\item \textbf{Elementi}: Energia Positiva - Energia Negativa
\item \textbf{Vantaggio}: Sensi protetti
\item \textbf{Scuole Privilegiate}: Abiurazione, (+2 alle prove di CM)
\end{itemize}

\subsubsection{Atherim}\index{Atherim}

\label{atherim}

Il Patrono custode. Molti vedono nel seno generoso di Atherim un segno di voluttà e passione. Si lasciano incantare dalla sua procace bellezza e non vedono gli occhi di cristallo che incutono timore a chi osa anche solo pensare di avvicinarla.

Atherim è la custode dei sogni e delle speranze, colei alla quale affidare, come ad una madre, i desideri. E' il Patrono dei Bambini, dei Segreti e delle Levatrici.

Dal sorriso allegro e dall'animo buono sarà sempre pronta ad aiutarti a realizzare i tuoi sogni. E come una madre Atherim protegge e custodisce i segreti e le passioni. Atherim è muta. E' colei che custodisce per sempre, dentro il suo animo i segreti di Yeru.

Il Devoto di Atherim si prende a cuore coloro che hanno fatto una promessa, punisce chi le infrange e chi svela i segreti. Molti Devoti di Atherim sono diplomatici, notai e levatrici.

\begin{itemize} 
\item \textbf{Simbolo:} Una mano di donna guantata che tiene un'ampolla ricca di flussi
\item \textbf{Caratteristica}: Saggezza
\item \textbf{Tratti}: Allegro, Calmo, Industrioso, Buono, Silenzioso, Clemente
\item \textbf{Manifestazione}: un silenzio sereno e tranquillizzante cala attorno all'incantatore
\item \textbf{Somma dei Tratti in comune a 5 punti}: Puoi aggiungere 1d6 ad un Tiro salvezza dopo averlo tirato ma prima di sapere se ha avuto successo o meno. Una volta al giorno, come Reazione.
\item \textbf{Somma dei Tratti in comune a 10 punti}: Guadagni 30 Punti Ferita temporanei. Durata 1 ora, una volta al giorno, come azione immediata.
\item \textbf{Somma dei Tratti in comune a 15 punti}: Puoi lanciare l'incantesimo Zona di Verità 3 volte al giorno
\item \textbf{Somma dei Tratti in comune a 20 punti}: Ogni pozione che bevi ha il doppio come durata o effetto se immediata.
\item \textbf{Elementi}: Energia Positiva, Elettricità
\item \textbf{Vantaggio}: Controllo del metabolismo 
\item \textbf{Scuole Privilegiate}: Ammaliamento (+2 alle prove di CM)
\end{itemize}

\subsubsection{Belevon}\index{Belevon}

\label{belevon}

E' il Patrono che meglio incarna la bugia e la finzione al fine di un proprio tornaconto. Lui ama solo se stesso. E' un narcisista che si circonda solo di persone che lo assecondano e lo adulano. Aborrisce la solitudine ma allo stesso tempo odia essere toccato da qualcuno.

E' sempre alla ricerca di nuove cose, di oggetti meravigliosi che scambia e ricambia con altri oggetti. Gli piace discutere e mercanteggiare, controbattere e portare fino allo stremo la vendita.

Dall'aspetto di un giovane ragazzo incarna perfettamente una pericolosa canaglia.

Il Devoto di Belevon è ben descritto dal mercante ricco e curioso che mai si lascia perdere una occasione di trattare merci nuove. Non è spinto dalla cupidigia o dall'accumulo bensì dall'Arte del commercio e dello scambio.

\begin{itemize} 
\item \textbf{Simbolo}: Una gabbia dorata
\item \textbf{Caratteristica}: Intelligenza
\item \textbf{Tratti}: Bugiardo, Narcisista, Casto, Doppiogiochista, Curioso, Confusionario
\item \textbf{Manifestazione}: come se le sbarre dorate di una gabbia si intrecciassero attorno all'incantatore
\item \textbf{Somma dei Tratti in comune a 5 punti} punti: Puoi creare un suono immaginario. Durata 10 secondi, entro 9 metri, 3 volte al giorno. Azione Reazione. 
\item \textbf{Somma dei Tratti in comune a 10 punti}: Acquisisci la capacità di respirare sott'acqua per 10 minuti. Una volta al giorno. Azione Immediata.
\item \textbf{Somma dei Tratti in comune a 15 punti}: La creatura che tocchi si placa e diventa indifferente a quello che succedo. Tiro Salvezza su Volontà DC 25. 3 volte al giorno. Costa 2 Azioni.
\item \textbf{Somma dei Tratti in comune a 20 punti}: Toccando un oggetto vieni a conoscenza per sommi capi della storia di chi l'ha creato. Una volta al giorno. Costa 3 Azioni.
\item \textbf{Elementi}: Fuoco, Suono
\item \textbf{Vantaggio}: Fortunato
\item \textbf{Scuole Privilegiate}: Illusione (+2 alle prove di CM)
\end{itemize}

\subsubsection{Cattalm}\index{Cattalm}

\begin{enfasismall}{
		Nessuno dà nulla per nulla. (anonimo)
}\end{enfasismall}

\label{cattalm}

Generato direttamente da Calicante, come risposta alla creazione di Gradh da parte di Ljust, è pura distruzione, chaos ed entropia. Cattalm si prefigura il solo scopo di distruggere, portare chaos e malattie, terremoti ed alluvioni.

Cattalm è tra i pochi Patroni che osa sfidare apertamente Gradh e lo fa con gioia perché sa che la loro battaglia altro non farà che portare ulteriore distruzione. Cattalm accetta ed invita ad essere suo Devoto ogni creatura capace di odio, capace di distruggere e ferire. Molti suoi Devoti sono creature mostruose o aberrazioni.

Cattalm invece è tra i Patroni più meravigliosi, con una candida pelle lucente, ali di piuma soffice ed una leggera armatura argentata. Per quanto i lineamenti delicati ne facciano un essere bellissimo per quanto ambisca alla distruzione.

Cattalm adora il chaos che manifesta nei modi più violenti con terremoti, alluvioni, maremoti, malattie se non direttamente piogge infuocate. Non agisce quasi mai uccidendo le persone piuttosto infettandole e spargendo piaghe, carestie per ottenere il massimo risultato.

Ljust non poteva non intervenire nella creazione di un Patrono così esplicitamente malvagio e, di nascosto da Calicante, instillò in Cattalm l'amore e protezione per i bambini. Cattalm distrugge, avvelena, indebolisce ma non i bambini, neanche indirettamente, piuttosto si attiva lui stesso per annullare i malefici causati dalla sua natura.

E' già capitato che interi villaggi venissero inondati e fossero trovati sui tetti in legno a modo di chiatte tutti i piccoli.

Ogni qual volta succede una calamità si suole dire che "Cattalm ha battuto il piede".

\begin{itemize} 
\item \textbf{Simbolo}: Un'onda gigante che sovrasta la costa
\item \textbf{Caratteristica}: Forza
\item \textbf{Tratti}: Distruttivo, Anarchico, Meticoloso, Sadico, Provocatore, Brutale
\item \textbf{Manifestazione}: il rombo del tuono
\item \textbf{Somma dei Tratti in comune a 5 punti} punti: Attraverso le tue armi indebolisci l'avversario designato. -2 Forza per 1 minuto dopo un attacco andato a segno. Una volta al giorno.
\item \textbf{Somma dei Tratti in comune a 10 punti}: Il tuo tocco imputridisce cibo (fino a 50kg) e acqua (un cubo con uno spigolo di 10 m). Una volta al giorno
\item \textbf{Somma dei Tratti in comune a 15 punti}: Il tuo sguardo riempie di collera. Tiro Salvezza su Volontà DC 30 o il target attacca un soggetto a caso entro la fine del round successivo. Due volta al giorno
\item \textbf{Somma dei Tratti in comune a 20 punti}: Generi un cono di Vuoto dal palmo della tua mano, largo 6 metri e lungo 27 metri, 10d6 di danno, DC 25 Riflessi per dimezzare. Una volta al giorno
\item \textbf{Elementi}: Energia Negativa - Vuoto
\item \textbf{Vantaggio}: Duro da uccidere
\item \textbf{Scuole Privilegiate}: Invocazione (+2 alle prove di CM)
\end{itemize}

\subsubsection{Efrem}\index{Efrem}

\label{efrem}

E' il Patrono di chi fa della natura la propria casa. Incarna in sé gli aspetti più puri della natura stessa, aggressivo come solo i felini più letali sanno essere; ma anche selvaggio come le radure più nascoste e rigorosa come solo la natura può essere.

Efrem si prefigge di difendere la Natura dalla contaminazione dell'uomo, da questa specie infestante che distrugge tutto ciò che incontra.

I Devoti di Efrem sono legati maggiormente all'elemento naturale. Manipolano la magia principalmente elementale e si difendono o attaccano usando anche animali e creature naturali. In rari casi costringendo anche i Draghi alla ubbidienza.

I Devoti di Efrem hanno l'obiettivo supremo di proteggere gli animali e le piante, i luoghi e tutto ciò che è naturale e non artificiale. Solitamente solitario e scontroso non riesce a capire il perché dell'odio che, dal suo punto di vista, l'uomo scarica su Yeru.

Un Devoto di Efrem rispetta la vita come la morte, nel processo naturale che è l'evoluzione ed il ciclo vitale. A volte decide di stabilirsi in un certo ambiente e lo elegge come suo territorio e come la sua casa lo protegge. Altre volte decide di essere ramingo ed intervenire in tutta Yeru per proteggere i suoi amati fiori e cuccioli.

\begin{itemize} 
\item \textbf{Simbolo}: Una staffa con un rampicante attorcigliato attorno
\item \textbf{Caratteristica}: Costituzione
\item \textbf{Tratti}: Indifferente, Leale, Ironico, Pratico, Misurato, Incostante 
\item \textbf{Manifestazione}: spire di foglie avvolgono la spada
\item \textbf{Somma dei Tratti in comune a 5 punti} punti: Il tuo tocco rende docili gli animali non magici. Tiro Salvezza su Volontà 20 per resistere. 3 volte al giorno. Costo 2 Azioni.
\item \textbf{Somma dei Tratti in comune a 10 punti}: Guadagni un +4 a tutte le prove di Sopravvivenza che si effettuano in un ambiente naturale. 
\item \textbf{Somma dei Tratti in comune a 15 punti}: Puoi benedire delle bacche affinché queste siano nutrienti e curative. Puoi incantare 1d4 bacche al giorno. Ogni bacca, max 1 al giorno, cura 1d6 Punti Ferita e rimuove le malattie o veleni non magici. Costa 3 Azioni.
\item \textbf{Somma dei Tratti in comune a 20 punti}: Il tuo tocco è quello del padrone. Puoi ammansire creature anche magiche che tocchi. Tiro Salvezza su Volontà DC 30. Una volta al giorno. Costo 2 Azioni
\item \textbf{Elementi}: Elettricità, Suono
\item \textbf{Vantaggio}: Empatia con le piante
\item \textbf{Scuole Privilegiate}: Evocazione (+2 alle prove di CM)
\end{itemize}

\subsubsection{Erondil}\index{Erondil}

\label{erondil}

Patrono di Terra e Aria, Erondil è il Signore degli elementi più concreti e razionali. Colui che dotato di infinito potere e razionalità dona ai suoi Devoti il potere della manipolazione della terra. Il dono di creare con semplice "fango" costruzioni gigantesche e di millenaria forza. Conclude le sue opere con attenzione e precisione.
Pur con fatica perché se il risultato finale non lo soddisfa scatena i suoi fulmini per distruggerlo all'istante. Perfezionista ed incontentabile, difficilmente qualcosa è esattamente come lui se la immaginava.

Ordinato ed esuberante è il signore delle tempeste, dei tuoni e dei fulmini,dei terremoti e distruzioni. Ama circondarsi del fragore del tuono, del rombo della terra che si sgretola. Sa essere distruttivo verso coloro che non rispettano Yeru.
Ha braccia e petto ricoperti da tatuaggi quasi argentei che narrano le leggende di Terra e Aria. Erondil, il signore di tuoni e terremoti.

I Devoti di Erondil sono gli ingegneri dell'impossibile, ogni qual volta si deve sfidare la materia, la gravità e la stessa ragione un Devoto di Erondil troverà pane per i suoi denti, troverà la sfida adatta ad un Costruttore dell'Impossibile.

\begin{itemize} 
\item \textbf{Simbolo:} un castello di sabbia con un fulmine sopra
\item \textbf{Caratteristica}: Saggezza
\item \textbf{Tratti}: Perfezionista, Incontentabile, Sognatore, Esuberante, Geloso, Imprudente
\item \textbf{Manifestazione}: suono di tempesta e rombo di frana
\item \textbf{Somma dei Tratti in comune 5 punti}: Non temi più le cadute. Ogni volta che cadi da più di 1 metro un soffio d'aria ti sostiene facendoti atterrare dolcemente.
\item \textbf{Somma dei Tratti in comune a 10 punti}: Il tuo tocco plasma la pietra, Puoi aprire un passaggio (un cubo di 3 metri di spigolo) in una parete di pietra. Una volta al giorno. Costo 2 Azioni.
\item \textbf{Somma dei Tratti in comune a 15 punti}: Puoi scagliare un fulmine dalle tue mani. 10D6 di danno, fino a 3 target. Tiro Salvezza Riflessi DC 30 per dimezzare. Costo 2 Azioni.
\item \textbf{Somma dei Tratti in comune a 20 punti}: Sei in grado di creare una fossa profondissima (1km) sotto il tuo avversario (taglia fino a grande). Tiro Salvezza Riflessi 30 o cadere. Una volta al giorno. Dopo 1 minuto la fossa si chiude con chi c'è dentro. Costo 2 Azioni.
\item \textbf{Elementi}: Fuoco, Elettricità
\item \textbf{Vantaggio}: Digestione universale
\item \textbf{Scuole Privilegiate}: Trasmutazione (+2 alle prove di CM)
\end{itemize}

\subsubsection{Gaya}\index{Gaya}

\label{gaya}

Patrono di Acqua e Fuoco, nelle profondità della terra, della nebbia più fitta, Gaya si diverte a dipingere. Adora circondarsi dei flussi di fuoco e acqua quasi a creare una danza in mezzo a loro. Adora i suoni della natura, l'infrangersi delle onde sugli scogli, il cadere delle gocce di pioggia sull'acciottolato, il borbottare di un fuoco scoppiettante.

Dipinge mescolando il caldo ed il freddo. L'acqua cristallina ed impetuosa al fuoco intrigante ed ardente. Gelosa del bello e delle arti tiene tutte le sue opere al sicuro in un ordine quasi maniacale e protette. Da vera artista utilizza gli elementi per far risplendere le meraviglie della natura. Gaya è la pittrice di tramonti e tempeste.

I Devoti di Gaya sono artisti volubili e sopra le righe. Sono coloro che ricreano la magia dell'alba o del tramonto o del mare in tempesta nelle loro opere, sono coloro che mettono poesia e follia nella normalità.

\begin{itemize} 
\item \textbf{Simbolo:} un pennello sul cielo
\item \textbf{Caratteristica}: Intelligenza
\item \textbf{Tratti}: Anarchico, Istintivo, Impetuoso, Emotivo, Permaloso, Lunatico
\item \textbf{Manifestazione}: spire di fuoco e acqua avvolgono all'incantatore
\item \textbf{Somma dei Tratti in comune a 5 punti} punti: Puoi creare fino a 5 litri di acqua o 1 litro di liquore di buona qualità. Una volta al giorno. Costo 2 Azioni.
\item \textbf{Somma dei Tratti in comune a 10 punti}: Il tuo metabolismo non teme il freddo. Resisti al Danno magico da freddo e sei immune a quello naturale.
\item \textbf{Somma dei Tratti in comune a 15 punti}: Puoi respirare sott'acqua come respiri l'aria.
\item \textbf{Somma dei Tratti in comune a 20 punti}: Generi una pioggia di fuoco. In 6 metri di circonferenza, 10d6 di danno, intorno a te, costo 2 Azioni, DC 25 Riflessi per dimezzare. Resisti il danno da fuoco, anche magico.
\item \textbf{Elementi}: Elettricità, Fuoco
\item \textbf{Vantaggio}: Arcobaleno
\item \textbf{Scuole Privilegiate}: Trasmutazione (+2 alle prove di CM)
\end{itemize}

\bigskip

\textbf{Gaia} ed \textbf{Erondil} sono come le due facce della stessa medaglia e sovraintendono agli elementi, Gaia acqua e fuoco e Erondil Aria e Terra; agiscono come espressione diretta dei Patroni maggiori, sono piccole manifestazione del loro immane potere.

\subsubsection{Krondal}\index{Krondal}

\label{krondal}

E' un Patrono potente ma schivo e riservato. Si tiene in disparte, fuori dai giochi finché non percepisce la privazione della libertà.

Non può vedere nel futuro, non può conoscere le persone ma il suo formidabile istinto lo fa diventare il combattente più temibile che si possa incontrare. Coraggioso fin quasi all'avventato, agisce in battaglia senza paura. Dallo spirito buono, Krondal entra in campo nei momenti più importanti, quando non è una situazione a decidersi ma il futuro della vita, della propria libertà personale.

Krondal, la furia cieca, nutre un profondo rispetto per la libertà ed è profondamente contrario ad ogni schiavismo, razzismo o dittatura.

Un Devoto di Krondal è tipicamente una guardia del corpo, un protettore, lo sceriffo che sa e deve decidere per il bene del suo paese, costi quello che costi.
Un Devoto di Krondal non giudica le persone o i fatti bensì si attiene alla sua etica di protezione e libertà.

Sotto vestiti dimessi e lisi, ma sempre puliti nasconde un fisico da combattente.

\begin{itemize} 
\item \textbf{Simbolo}: Una spada tenuta verticalmente davanti a se
\item \textbf{Caratteristica}: Carisma
\item \textbf{Tratti}: Attento, Pio, Corretto, Libero, Conformista, Incontentabile
\item \textbf{Manifestazione}: il mantello o veste del Devoto diventa sporco di terra e sangue
\item \textbf{Somma dei Tratti in comune a 5 punti} punti: Maledici il tuo avversario, dandogli un -2 Tiro per Colpire e Difesa, per 1 minuto. Tiro Salvezza su Volontà DC 20 per resistere. Costo 2 Azione.
\item \textbf{Somma dei Tratti in comune a 10 punti}: Non puoi essere legato o ammanettato. Ad un tuo gesto i nodi si sciolgono e le manette si aprono. Costo 1 Azione Immediata. 3 volte al giorno.
\item \textbf{Somma dei Tratti in comune a 15 punti}: La tua presenza toglie la vista agli avversari. Designa fino a 6 creature entro 9 metri, queste devono fare un Tiro Salvezza Tempra a DC 30 o essere ciechi nei tuoi confronti fino alla fine del prossimo round.
\item \textbf{Somma dei Tratti in comune a 20 punti}: La tua arma è più efficace contro i nemici. Ogni creatura colpita deve fare un Tiro Salvezza Volontà DC 20 o rimanere paralizzata per 3 round. Una volta che la creatura riesce nel Tiro Salvezza non può essere più influenzata per le successive 24 ore.
\item \textbf{Elementi}: Energia Positiva, Fuoco
\item \textbf{Vantaggio}: Magnetico
\item \textbf{Scuole Privilegiate}: Abiurazione (+2 alle prove di CM)
\end{itemize}

\subsubsection{Ledyal}\index{Ledyal}

\label{ledyal}

E' il Patrono senza un volto preciso, senza una voce se non un canto. Mutevole di corpo e senza una definizione chiara del suo essere. Si manifesta con un lungo mantello color rosso fuoco dal tessuto fatto da mille farfalle. Il suo tocco è vita e pace, protegge chi necessita dei suoi favori indipendentemente dal fatto che li chieda o meno. Desidera un mondo senza sofferenza, con solo felicità ed armonia. Sospettoso/a e profondamente introverso non crede a coloro che gli danno ragione. Ha il cuore pieno di vita e di bontà ma non un corpo da amare.

Ledyal ha anche una sorella gemella, o forse un'altra personalità. O forse sono lo stesso Patrono, nessuno le ha mai viste insieme. La "gemella" \textbf{Laydel}\index{Laydel} non tollera la sofferenza, disprezza chi causa dolore, uccide senza timore qualunque creatura abbia peccato contro un innocente, chiunque abbia causato sofferenza.

\begin{itemize} 
\item \textbf{Simbolo}: Una farfalla che gronda sangue mentre vola
\item \textbf{Caratteristica}: Saggezza (Ledyal) - Forza (Laydel)
\item \textbf{Tratti}: Caritatevole, Sospettoso. Introverso/Integerrimo, Clemente/Implacabile, Leale/Passionale, Timido/Suscettibile
\item \textbf{Manifestazione}: come se un mantello di farfalle avvolgesse il Devoto
\item \textbf{Somma dei Tratti in comune a 5 punti} punti: Il tuo tocco è vita/attacco. 3 volte al giorno puoi toccare una creatura vivente e curarla/causa di 1d6 Punti Ferita. Costo 2 Azioni.
\item \textbf{Somma dei Tratti in comune a 10 punti}: Il tuo tocco è pace. La creatura toccata deve riuscire in un Tiro Salvezza Volontà DC 25 o cadere addormentata. Non puoi attaccare/danneggiare questa creatura e se viene attaccata si spezza la malia e cadi tu addormentato. Una volta al giorno, durata 1 ora. Costo 2 Azioni (comprende anche l'Azione di tocco)
\item \textbf{Somma dei Tratti in comune a 15 punti}: La tua aura protegge i tuoi compagni. Entro raggio 6 metri i tuoi compagni hanno un +4 alla Difesa ed un +2 ai Tiri Salvezza. Durata 10 minuti consecutivi, una volta al giorno. Costo 2 Azioni.
\item \textbf{Somma dei Tratti in comune a 20 punti}: Irradi una sfera curativa intorno a te. Ogni creatura nel raggio di 6 metri viene curata di 60Punti Ferita. Una volta al giorno. In caso di Laydel l'effetto è opposto, DC 25 Riflessi per dimezzare. Costo 2 Azioni
\item \textbf{Elementi}: Energia Positiva, Elettricità
\item \textbf{Vantaggio}: Guaritore (Ledyal) oppure Senza paura (Laydel) 
\item \textbf{Scuole Privilegiate}: Invocazione o Cura (+2 alle prove di CM)
\end{itemize}

\subsubsection{Nethergal}\index{Nethergal}

\label{nethergal}

Il Patrono Messaggero. Sulla piuma di un'oca vola la lettera di Nethergal. Rapida, impetuosa, diretta, Nethergal è la messaggera, colei alla quale affidare pensieri e scritti. Sarcastica e logorroica curioserà sui tuoi scopi, ti chiederà informazioni sugli scritti affidatole con esplicita franchezza ed avrà sempre qualcosa da ridire sul messaggio da portare ma sarà anche altrettanto diretta e precisa nel consegnarlo.

Nethergal non è solo chiacchiere e pettegolezzi, qualsiasi testo venga scritto lei lo conosce, non esiste codice o segreto scritto che lei non conosca.

Il Devoto di Nethergal è un fine linguista, un esperto di indovinelli e rebus, un Devoto che a differenza di Atmos non si limita a custodire gli scritti ma ne diffonde la conoscenza.

Un Devoto di Nethergal è un maestro, una professoressa di lingue di un Collegio, un dotto esperto di mille argomenti.

\begin{itemize} 
\item \textbf{Simbolo:} una piuma bianca cangiante
\item \textbf{Caratteristica}: Destrezza
\item \textbf{Tratti}: Sarcastico, Impetuoso, Immaturo, Logorroico, Competitivo, Avventato
\item \textbf{Manifestazione}: cascata di piume, un oca in volo
\item \textbf{Somma dei Tratti in comune a 5 punti} punti: Puoi inviare un messaggio di massimo 144 caratteri ad un soggetto entro 50 metri senza essere udito/visto. Una volta all'ora. Costo 1 Azione. Il soggetto deve comprendere la lingua usata.
\item \textbf{Somma dei Tratti in comune a 10 punti}: Mettendo la mano su un libro ne apprendi il contenuto come se lo avessi letto. Un libro a settimana. Perdi le conoscenze così acquisite dopo una settimana. Costo 3 Azioni. La lingua scritta del tomo deve essere nota.
\item \textbf{Somma dei Tratti in comune a 15 punti}: Puoi volare, come omonimo incantesimo, 1 ora al giorno. Costo 1 Reazione.
\item \textbf{Somma dei Tratti in comune a 20 punti}: Puoi costringere una creatura a rivelarti le informazioni che ha, 3 domande precise. Il bersaglio deve fare un unico Tiro Salvezza su Volontà DC 30 per resistere. Una volta al giorno. Costo 2 Azioni.
\item \textbf{Elementi}: Elettricità, Suono
\item \textbf{Vantaggio}: Direzione Assoluta
\item \textbf{Scuole Privilegiate}: Trasmutazione (+2 alle prove di CM)
\end{itemize}

\subsubsection{Nedraf}\index{Nedraf}

\label{nedraf}

Il Patrono Sopravvissuto, il vecchio lupo ormai stanco che ha attraversato e combattuto mille battaglie. La sua carne è ferita, il suo corpo ricoperto di cicatrici di guerra e lividi ma nulla lo farà crollare. Tenacia, passione, esperienza e tanta rabbia rendono Nedraf non solo un combattente eccellente in qualsiasi occasione ma un conoscitore dell'ambiente attorno a sè. Grazie al suo impeccabile allenamento sa sfruttare al meglio le risorse a disposizione. Sa spronare con passione gli uomini a suoi ordini.
Nedraf rappresenta colui che vorresti sempre accanto in ogni battaglia.

Molti capitani di ventura e ufficiali al comando sono Devoti di Nedraf. Il Devoto di Nedraf non si arrende, non rinuncia, non abbandona i compagni ma non per questo è avventato o irrazionale nelle scelte.

\begin{itemize} 
\item \textbf{Simbolo:} una mano forte, avvolta in una benda sporca di sangue che brandisce una spada
\item \textbf{Caratteristica}: Costituzione
\item \textbf{Tratti}: Disciplinato, Combattivo, Tenace, Aggressivo, Pianificatore, Malizioso
\item \textbf{Manifestazione}: si spande nell'aria odore di sangue e metallo
\item \textbf{Somma dei Tratti in comune a 5 punti} punti: Puoi portare armature leggere senza penalità alla CM
\item \textbf{Somma dei Tratti in comune a 10 punti}: Acquisisci un punto bonus su una Lista armi. Può essere nota o meno
\item \textbf{Somma dei Tratti in comune a 15 punti}: Puoi portare armature medie senza penalità alla CM ed Destrezza
\item \textbf{Somma dei Tratti in comune a 20 punti}: Acquisisci un punto bonus su una Lista armi. Può essere nota o meno
\item \textbf{Elementi}: Energia positiva, Suono
\item \textbf{Vantaggio}: Guarigione accelerata
\item \textbf{Scuole Privilegiate}: Ammaliamento, (+2 alle prove di CM)
\end{itemize}

\subsubsection{Nihar}\index{Nihar}

\label{nihar}

E' il Patrono degli eroi per caso. Ponderato e tranquillo è anche amante del buon vino e del gozzovigliare. E' colui che non sceglieresti mai come compagno d'armi a causa del suo aspetto "comune" e del suo atteggiamento goliardico. Ma poi al momento di esserci, di combattere, di far la differenza ecco che strabilia tutti e "risolve" la partita.

Ha le sembianze di un piccolo uomo, dai vestiti sfarzosi e ricercati e dall'espressione guardinga ed allegra. Si protegge sempre e a qualunque costo, mostrando al mondo esattamente ciò che l mondo vuole vedere. Controlla attentamente la realtà attorno a sé e anche se è sempre più facile vederlo con un calice in mano, se non ci si lascia ingannare dalle apparenze si noterà come i suoi occhi non perdano mai di vista il pericolo, il problema. Sta attento, non si fida di nulla e di nessuno. Ha fatto dei suoi difetti i suoi punti di forza.

\begin{itemize} 
\item \textbf{Simbolo}: Una daga appoggiata vicino ad un calice di vino
\item \textbf{Caratteristica}: Intelligenza
\item \textbf{Tratti}: Altruista, Determinato, Cortese, Attento, Diffidente, Caotico
\item \textbf{Manifestazione}: il suono di un brindisi
\item \textbf{Somma dei Tratti in comune a 5 punti} punti: Puoi trasformare l'acqua in vino. Un litro al giorno. Costo 2 Azioni. 2 volte al giorno.
\item \textbf{Somma dei Tratti in comune a 10 punti}: Costo una Azione immediata, ottieni un bonus di +2d6 ad una azione in quel round. 1 volta al giorno.
\item \textbf{Somma dei Tratti in comune a 15 punti}: Il tuo pugnale causa 1d4 di danno aggiuntivo. Il bonus è sempre attivo.
\item \textbf{Somma dei Tratti in comune a 20 punti}: I manicaretti che prepari sono buonissimi. Chiunque si sazi con una pietanza da te preparata recupera 2d6 Punti Ferita e viene curato dai veleni anche magici. Max 6 persone al giorno. 0.5 ore di preparazione per persona.
\item \textbf{Elementi}: Energia Positiva, Fuoco
\item \textbf{Vantaggio}: Lingua universale
\item \textbf{Scuole Privilegiate}: Ammaliamento (+2 alle prove di CM)
\end{itemize}

\subsubsection{Orudjs}\index{Orudjs}

\label{orudjs}

Ovvero il Patrono della illusione e della finzione. Colui che solo con il dono della parola, il gesticolare delle mani, la voce carismatica e lo sguardo intrigante riesce a vendere ogni sua parola come verità assoluta. Adora il teatro per ciò che per lui è, la rappresentazione della falsità umana, l'essere tante persone ed in realtà nessuna. Adora la politica ed i suoi intrighi. Finge di ascoltare chi gli sta vicino ma in realtà non è interessato alle storie altrui perché le sue sono sempre le migliori.

E' un codardo senza limiti e le poche verità che dice, e sono veramente rare, sono da lui dette solo per salvarsi.

Dall'aspetto piuttosto ordinario e quasi scontato appena apre bocca ed inizia i suoi racconti riesce a calamitare l'attenzione dell'intera sala. Possiede infatti una voce calda e suadente che accompagnata alla buonissima dialettica da lui posseduta, incanta ogni orecchio in ascolto.

I suoi Devoti sono abili attori ed intrattenitori, spie sotto copertura, diplomatici o politicanti.

\begin{itemize} 
\item \textbf{Simbolo}: Una maschera teatrale con solo la bocca aperta e gli occhi
\item \textbf{Caratteristica}: Carisma
\item \textbf{Tratti}: Ironico, Codardo, Saccente, Socievole, Incostante, Creativo
\item \textbf{Manifestazione}: il suono di una risata profonda e contagiosa
\item \textbf{Somma dei Tratti in comune a 5 punti} punti: Il tuo eloquio è già leggendario. +2 alle prove di Intrattenere.
\item \textbf{Somma dei Tratti in comune a 10 punti}: Sei in grado di creare fino a 4 suoni/rumori distanti 6 metri l'uno dall'altro. Durata 1 minuto. Tre volte al giorno. Costo 1 Azione Reazione
\item \textbf{Somma dei Tratti in comune a 15 punti}: Il tuo eloquio è già leggendario. +4 aggiuntivo alle prove di Intrattenere.
\item \textbf{Somma dei Tratti in comune a 20 punti}: La tua voce è suadente. Una creatura, da te individuata, che ti ascolti per più di un minuto deve fare un Tiro Salvezza Volontà DC 30 oppure considerarsi charmato. Una volta al giorno
\item \textbf{Elementi}: Elettricità, Fuoco
\item \textbf{Vantaggio}: Voce suadente
\item \textbf{Scuole Privilegiate}: Ammaliamento (+2 alle prove di CM)
\end{itemize}

\subsubsection{Orlaith}\index{Orlaith}

\label{orlaith}

Ovvero il Patrono della Giustizia e della Vendetta. Lui segue pedissequamente le leggi e pretende che i suoi sottoposti eseguano senza alcuna discussione gli ordini impartiti. E' mosso da uno spirito gentile e buono che però tiene ben nascosto dietro le sue azioni dirette ed incisive, spudorate e mortali. Orlaith è vendetta che si fa legge. Agisce per senso di giustizia con i suoi metodi. Di lui attirano il portamento e lo sguardo fiero.

I Devoti di Orlaith spesso sono giudici e giustizieri, persone che hanno deciso di portare la giustizia ovunque, perché Orlaith non può stare fermo, c'è sempre qualcuno da giudicare e punire.

\begin{itemize} 
\item \textbf{Simbolo}: La bilancia
\item \textbf{Caratteristica}: Forza
\item \textbf{Tratti}: Imparziale, Giusto, Vendicativo, Valoroso, Schietto, Espansivo
\item \textbf{Manifestazione}: l'immagine di una stadera, sbilanciata.
\item \textbf{Somma dei Tratti in comune a 5 punti} punti: Richiami a te 1 mastino (normale) che obbedisce ai tuoi comandi. Durata 1 minuto. Una volta al giorno. Costo 2 Azioni.
\item \textbf{Somma dei Tratti in comune a 10 punti}: Un paio di manette si manifesta attorno ai polsi della creatura (massimo taglia grande) entro 27 metri. Tiro Salvezza Riflessi DC 25 per annullare. Costo 2 Azioni. Una volta al giorno.
\item \textbf{Somma dei Tratti in comune a 15 punti}: La creatura toccata deve dire la verità alle tue domande. Durata 10 minuti (DC 25 Volontà per resistere). Una volta al giorno. Costo 2 Azioni.
\item \textbf{Somma dei Tratti in comune a 20 punti}: Crei un raggio di luce lungo 27 metri e largo pochi centimetri. Ogni creatura attraversata subisce 5d6 di danno, DC 25 Riflessi per dimezzare. Una volta al giorno. Costo 2 Azioni.
\item \textbf{Elementi}: Luce, Suono
\item \textbf{Vantaggio}: Senso comune
\item \textbf{Scuole Privilegiate}: Illusione (+2 alle prove di CM)
\end{itemize}

\subsubsection{Rezh}\index{Rezh}

\label{rezh}

Il Patrono che disprezza tutto. Rezh ama, vuole, tocca, rimira solo le sue monete lucide e brillanti. Non sono mai abbastanza, nessuna ricchezza è mai abbastanza per lei. Rezh, l'avara tiene tutto per sè, non conosce compassione, non conosce carità, non conosce condivisione. La sua fame di denaro, di ricchezze la rende prona a qualsiasi bassezza. Disprezza tutto e tutti e giudica tutto e tutti seguendo solo il suo personale metro di giudizio. In ogni moneta c'è un pò di Rezh. Nella ossidatura di ogni moneta si può vedere l'impronta di Rezh.

Se il denaro compra la libertà Rezh deve accumularne ancora e ancora se mai sarà abbastanza.

I Devoti di Rezh solitamente sono scelti da lei tra le fila dei più avidi e ricchi. Il loro scopo è di accumulare ricchezze, sempre di più.
Spesso i Devoti di Rezh diventano esploratori, tombaroli, persone sempre alla ricerca di un tesoro e di una moneta in più.

\begin{itemize} 
\item \textbf{Simbolo:} una pila di monete con un ratto vicino
\item \textbf{Caratteristica}: Intelligenza
\item \textbf{Tratti}: Avaro, Arrogante, Cattivo, Freddo, Geloso, Abitudinario
\item \textbf{Manifestazione}: un rumore di monete che cadono avvolge l'incantatore
\item \textbf{Somma dei Tratti in comune a 5 punti} punti: Sei un esperto di monete e gemme, nessun falsario può ingannarti. +4 alle prove di Consapevolezza e Conoscenza relative.
\item \textbf{Somma dei Tratti in comune a 10 punti}: Puoi incantare una gemma (valore minimo 10mo) e usarla per proiettare una illusione fino a 20m{*}10m{*}10m. La gemma viene poi distrutta. Durata illusione 1 ora ogni 10 mo di valore della gemma. Costo 2 Azioni.
\item \textbf{Somma dei Tratti in comune a 15 punti}: Puoi tirare fuori dalle tasche 1 moneta d'oro ogni volta che vuoi. Max 10 mo al giorno. Costo 1 Azione.
\item \textbf{Somma dei Tratti in comune a 20 punti}: La tua armatura viene coperta da scintillio dorato e di gemme. Guadagni +4 alla Difesa e +4 Tiro Salvezza Tempra per 1 ora. Costo 1 Reazione

\item \textbf{Elementi}: Vuoto, Elettricità
\item \textbf{Vantaggio}: Mani di Fata
\item \textbf{Scuole Privilegiate}: Abiurazione (+2 alle prove di CM)
\end{itemize}

\subsubsection{Sumkjr}\index{Sumkjr}

\begin{enfasismall}{
Tutto ciò che non viene donato va perduto. (Dominique Lapierre)
}\end{enfasismall}

\label{sumkjr}

Patrono dell'Arcano di Luce. Sumkjr è bontà, correttezza, lealtà, giustizia, protezione.

Sumkjr è il cavaliere che protegge gli innocenti, è la spada "di Ljust" nella battaglia finale. Difende i deboli e lenisce le ferite.

Sumkjr porta la Luce di Ljust ovunque, nessun pericolo potrà mai fermare Sumkjr dalla sua continua, infinita, cerca del bene.

Un Devoto di Sumkjr agisce lealmente e con onore, sempre perseguendo il bene ultimo, il suo essere non può essere piegato al male, all'ingiustizia, al disonore.

Con coraggio e determinazione il Devoto affronta ogni sfida ma non solo per senso del dovere, ma perché profondamente votato al suo destino. Sumkjr sa che poche persone reggono tale standard perché a differenza dei Devoti della Patrona delle Genesi i suoi Devoti non nascono per essere tali, ma lo diventano grazie alla loro profonda e determinata forza di volontà.

Per questo motivo Ljust interviene in loro favore con l'elaborato Rito del Rinnovo, grazie al quale ogni anno al Devoto meritevole e pentito di aver perso anche solo per poco la giusta direzione, la Luce, viene fatto recuperare ogni punto Tratto perso perché agito fuori dalle 7 Regole Luminose.

Sumkjr è un soldato valoroso, il migliore amico del giusto.

Calicante, preso dall'orrore alla vista di un Patrono così fatto, lo privò della capacità di amare e provare veri sentimenti d'affetto. Portare il bene per un Devoto di Sumkjr è un qualcosa di normale come è normale non riuscire ad essere empatico con chi soffre. Il Devoto sa cosa deve fare e perché, ma non riesce a commuoversi od amare di fronte alle sofferenze od alle carezze di una donna/uomo.

\begin{itemize} 
\item \textbf{Simbolo:} tre gocce di sangue che cadono una dietro l'altra
\item \textbf{Caratteristica}: Carisma
\item \textbf{Tratti}: Giusto, Curioso, Buono, Valoroso, Candido, Disordinato
\item \textbf{Manifestazione}: il Devoto è avvolto da un mantello di broccato dorato
\item \textbf{Somma dei Tratti in comune a 5 punti} punti: Il tocco della tua spada é vita. Una creatura toccata con la tua arma recupera 3d6 punti ferita. Una volta al giorno. Costo 2 Azioni.
\item \textbf{Somma dei Tratti in comune a 10 punti}: La tua Volontà é più forte del metallo. Guadagni un +2 ai Tiri Salvezza su Volontà
\item \textbf{Somma dei Tratti in comune a 15 punti}: Concentri l'energia del tuo Patrono in un cono di assordante. Il cono é lungo 18 metri e largo al termine 3 metri, chiunque sia preso nell'area subisce 10d6 di danno. Una volta al giorno. Costo 2 Azioni.
\item \textbf{Somma dei Tratti in comune a 20 punti}: Sacrifichi la tua vita per portare in vita una creatura morta da non più di 1 settimana. Una volta. Costo 3 Azioni.
\item \textbf{Elementi}: Energia Positiva, Elettricità
\item \textbf{Vantaggio}: Aura di coraggio
\item \textbf{Scuole Privilegiate}: Cura (+2 alle prove di CM)
\end{itemize}

\paragraph{Le 7 Regole Luminose}\index{7 Regole Luminose}

\label{le-7-regole-luminose}

Le Sette regole Luminose sono un insieme di norme e comportamenti tenuti, a vario titolo, dai Devoti che vogliono seguire la strada della Luce di Ljust.

I Devoti di Sumkjr devono seguirle tutte e 7 pena la perdita di potere (punti Tratto), altri Devoti di altri Patroni, sempre positivi od almeno neutrali, seguono solo alcune di questi dettami, come regola per non cadere nelle braccia di Calicante

\medskip

\begin{enumerate}
	\item Proteggi i deboli e chi non sa difendersi dai soprusi
	\item Ama la vita e proteggila.
	\item Combatti contro le ingiustizie e chi porta sofferenze e dolore
	\item Lenisci le ferite ed i dolori. Placa gli animi e favorisci la pace ed armonia
	\item Onestà e Lealtà sono le tua fondamenta
	\item Sei un maestro di virtù. Fa che gli altri possano prendere ispirazione dalle tue gesta
	\item Sii luminoso ma non accecare gli altri
\end{enumerate}

\subsubsection{Shayalia}\index{Shayalia}

\label{shayalia}

Patrono dell'Arcano di Tenebra. Shayalia è l'anima oscura della perdizione, del tradimento, della lussuria più sordida e peccaminosa. Adora i bordelli. Le piace l'odore acre del sudore, la pelle lucida di oli e profumi. Le passioni, le vendette che li si consumano, la distruzione fisica e morale che in quei luoghi viene perpetrata è la sua vita.

Shayalia è una donna che gode di se stessa, che vive dei piaceri più fisici. Vive di vendette lungamente e ben dettagliatamente progettate. Vendicativa ed amorale, non giudica con metro di giudizio umano, il suo godere non è neppure lontanamente comprensibile. Shayalia è quanto di più vicino a Calicante sia stato creato. Sono le passioni, le pulsioni, i liquidi umorali che la fanno inebriare.

Shayalia è la concubina che ti ammalia e ti distrugge, goccia dopo goccia. I veleni sono le sue armi, le debolezze umane il suo campo.

I Devoti di Shayalia sono spie, figli bastardi, amanti di potenti signori che agiscono all'ombra.

Ljust disgustata dalla visione di un Patrono del genere instillò in Shayalia l'amore per la natura, piante ed animali. E così molti dei più famosi botanici, erboristi e zoologi sono Devoti di Shayalia, forse le uniche cose che Shayalia veramente può amare.

\begin{itemize} 
\item \textbf{Simbolo:} un cuscino stropicciato e sporco di sangue
\item \textbf{Caratteristica}: Carisma
\item \textbf{Tratti}: Lussurioso, Volubile, Pessimista, Sadomasochista, Arrogante, Falso
\item \textbf{Manifestazione}: il Devoto è avvolto da un mantello di velluto nero
\item \textbf{Somma dei Tratti in comune a 5 punti} punti: I tempi per preparare una pozione sono dimezzati.
\item \textbf{Somma dei Tratti in comune a 10 punti}: Il tuo tocco è vita per la natura. I tuoi incantesimi di cura agiscono su animali e piante naturali.
\item \textbf{Somma dei Tratti in comune a 15 punti}: Dal tuo palmo secerni veleno. Il tuo tocco, o tramite arma in mischia veicola il veleno. Tiro Salvezza Tempra DC 25 o -2 a Saggezza e Destrezza per 10 minuti. Costo 1 Azione.
\item \textbf{Somma dei Tratti in comune a 20 punti}: Il tuo tocco è vita per la natura. Puoi curare animali e piante magiche. +4 alle prove di Lavoro (erboristeria), +4 Conoscenza Natura.
\item \textbf{Elementi}: Vuoto, Elettricità
\item \textbf{Vantaggio}: Empatia Animale
\item \textbf{Scuole Privilegiate}: Illusione (+2 alle prove di CM)
\end{itemize}

\bigskip

\textbf{Sumkjr} e \textbf{Shayalia} sono complementari nel tenere in mano le file sfuggenti delle creature. Agiscono come espressione diretta dei Patroni della Genesi.

\subsubsection{Sixiser}\index{Sixiser}

\begin{enfasismall}{
La forza che si oppone al destino è in realtà una debolezza. (Franz Kafka)
}\end{enfasismall}


Il Patrono che è indifferente al presente in quanto totalmente, compulsivamente ossessionato dal futuro e dal suo destino. Negli angoli più remoti dei mondi conosciuti si narra che Sixiser accumuli di tutto, indifferente a tutto e tutti.

Terrorizzato dal futuro che vede, da una ipotetica fine di sé e del tutto vive una vita di ritiro, spirituale e fisico. Si priva volontariamente di tutto il necessario. Ma allo stesso accumula qualunque oggetto incroci la sua strada nella speranza di un ritorno.

E' paranoico e non si fida di nessuno. Usa i suoi poteri di divinazione per conoscere e scrutare tutti.

I Devoti di Sixiser sono spesso negromanti circondati da non morti ed altre creature silenziose ed ubbidienti. Chi si rifugia alla ricerca della solitudine e dello studio, chi invece mira ad espandere e governare intere città e nazioni al fine di sentirsi più sicuro, è Devoto a Sixiser.

\begin{itemize} 
\item \textbf{Simbolo}: Un forziere straripante di ogni cosa che non si può chiudere
\item \textbf{Caratteristica}: Saggezza
\item \textbf{Tratti}: Riservato, Indifferente, Accumulatore, Paranoico, Bugiardo, Insensibile
\item \textbf{Manifestazione}: due mani che circondano, come a nascondere, la testa dell'incantatore
\item \textbf{Somma dei Tratti in comune a 5 punti} punti: acquisisci la visione crepuscolare fino 18 metri, o 36 metri se già presente.
\item \textbf{Somma dei Tratti in comune a 10 punti}: vedi nell'oscurità anche magica entro 18 metri. Vedi le trappole nel raggio di mischia intorno a te.
\item \textbf{Somma dei Tratti in comune a 15 punti}: Toccando un oggetto sei in grado di capirne tutte le proprietà magiche e non. 3 volte al giorno.
\item \textbf{Somma dei Tratti in comune a 20 punti}: Sei in grado di animare una creatura morta da non più di un giorno come non morto da 1 grado di Sfida (tipo zombi/scheletro a secondo dello stato). Una volta al giorno. Costo 2 Azioni.
\item \textbf{Elementi}: Elettricità, Energia Negativa
\item \textbf{Vantaggio}: Consumi ridotti
\item \textbf{Scuole Privilegiate}: Divinazione (+2 alle prove di CM)
\end{itemize}

\subsubsection{Tazher}\index{Tazher}

\label{tazher}

il Patrono delle Ombre; colui che silenzioso, ti uccide. Non saprai mai il perché. Non conoscerai mai il suo aspetto ma, se improvvisamente hai una sensazione di gelo, Tazher è dietro di te pronto a prendere la tua vita.
Doppiogiochista dall'animo cattivo, chiedi il suo aiuto solo se sei disposto a pagarne il prezzo che lui e lui solo deciderà.

Vive di notte, vive la notte. Le ombre sono le sue amiche e la tenebra il suo mantello. Profondamente individualista con un carattere scontroso e permaloso, non ha amici, non intrattiene relazioni di alcun tipo.

Il Devoto di Tazher è il ladro, l'assassino, il bandito, chiunque viva per l'oscurità ed il proprio tornaconto. Un Devoto di Tazher è estremamente pericoloso in combattimento.

\begin{itemize} 
\item \textbf{Simbolo}: Lo scintillio della lama nel buio
\item \textbf{Caratteristica}: Destrezza
\item \textbf{Tratti}: Scontroso, Calcolatore, Perfezionista, Cattivo, Insensibile, Burbero
\item \textbf{Manifestazione}: l'ombra del Devoto prende vita muovendo l'arma
\item \textbf{Somma dei Tratti in comune a 5 punti} punti: Guadagni +2 alle prove di Muoversi Silenziosamente e Nascondersi.
\item \textbf{Somma dei Tratti in comune a 10 punti}: Una volta al giorno fai un attacco (senza malus) in più. Una Azione Immediata.
\item \textbf{Somma dei Tratti in comune a 15 punti}: finché cammini sopra delle ombre o al buio sei invisibile. Puoi essere comunque rilevato con la luce o incantesimi di divinazione.
\item \textbf{Somma dei Tratti in comune a 20 punti}: Una volta al giorno su tutti gli attacchi andati a segno in quel round fai il doppio del danno. Costo 1 Azione di Reazione da dichiararsi anche dopo il Tiro per Colpire ma prima di sapere se i tiri sono andati a buon segno.
\item \textbf{Elementi}: Vuoto, Fuoco
\item \textbf{Vantaggio}: La mia ombra è mia amica
\item \textbf{Scuole Privilegiate}: Necromanzia (+2 alle prove di CM)
\end{itemize}

\subsubsection{Thaft}\index{Thaft}

\label{thaft}

Il Patrono che accompagna nella nascita e nella morte. Silenzioso, resta in disparte e osserva lo scorrere della vita degli uomini. Quasi umile nella sua semplicità, Thaft è ovunque. Testimone silenzioso della vita umana; nel momento in cui una vita scivola via, Thaft assiste, nell'attimo in cui una vita nasce, Thaft è presente.

Thaft sa anche che non si può essere sempre e solo osservatori. Attraverso il suo taccuino sacro e magico può decidere e giudicare della vita degli uomini, perché se una spada ferisce, è solo Thaft che ne decide la morte.

I Devoti di Thaft sono i sacerdoti dell'ultimo viaggio, coloro che proteggono e vegliano sulle anime e corpi dei morti. Profondamente contrari all'utilizzo dei non-morti ne perseguono la distruzione.

Un Devoto di Thaft rispetta la vita come la morte e non teme di arrecare distruzione per un equilibrio maggiore.

Thaft è stato plasmato da Atmos.

\begin{itemize} 
\item \textbf{Simbolo}: Un libro aperto con un teschio sopra
\item \textbf{Caratteristica}: Saggezza
\item \textbf{Tratti}: Semplice, Silenzioso, Mite, Sicuro, Disciplinato, Ottimista
\item \textbf{Manifestazione}: si sente il pianto di un bambino appena nato o il sospiro della morte
\item \textbf{Somma dei Tratti in comune a 5 punti} punti: Il tuo tocco è letale per i non morti. Un tuo tocco infligge 2d6 di danno ad un non morto. Costo 2 Azioni. Fino a 3 volte al giorno.
\item \textbf{Somma dei Tratti in comune a 10 punti}: Il tuo tocco lenisce. Una volta al giorno puoi rimuovere Cecità o Sordità. Costo 2 Azioni.
\item \textbf{Somma dei Tratti in comune a 15 punti}: Un non morto deve effettuare un Tiro Salvezza Tempra DC 30 o essere distrutto se toccato dalla tua mano. Costo 2 Azioni.
\item \textbf{Somma dei Tratti in comune a 20 punti}: Uccidi la creatura toccata. Tiro Salvezza su Volontà DC 30 o morte. Una volta alla settimana. Costo 2 Azioni.
\item \textbf{Elementi}: Suono, Elettricità
\item \textbf{Vantaggio}: Tocco gelido
\item \textbf{Scuole Privilegiate}: Necromanzia (+2 alle prove di CM)
\end{itemize}

\subsubsection{Torbiorn}\index{Torbiorn}

\label{torbiorn}

Il Patrono che meglio incarna il concetto "non è mai abbastanza". Alto, bello come un quadro ma, proprio come quest'ultimo, senza calore e vita, Torbiorn rasenta la perfezione maniacale nel vestirsi, nell'atteggiarsi.

Nulla è mai abbastanza per lui. Nessuno è mai alla sua altezza. Ed eccolo che con arroganza e ironia va a modificare tutto il modificabile per poter placare questa profonda insoddisfazione. Qualora il risultato finale raggiunto non lo soddisfi, e accade molto spesso, ecco che prende il sopravvento il suo cinismo e distrugge tutto senza curarsi della sofferenza che sta arrecando a chi gli sta attorno.

Il Devoto di Torbiorn è il tipico aristocratico ricco e svogliato,colui che cerca sempre la strada più facile e meno rischiosa.

Incurante degli altri si diverte nello sfruttare i lavori altrui e trarne giovamento.

\begin{itemize} 
\item \textbf{Simbolo}: Uno specchio opaco
\item \textbf{Caratteristica}: Carisma
\item \textbf{Tratti}: Altezzoso, Ansioso, Vanitoso, Permaloso, Cauto, Irascibile
\item \textbf{Manifestazione}: schegge di specchio rotto tutto intorno al Devoto come un turbine
\item \textbf{Somma dei Tratti in comune a 5 punti} punti: Con un gesto puoi rinfrescare i tuoi vestiti rendendoli puliti e profumati. Costo 1 Azione. 3 volte al giorno.
\item \textbf{Somma dei Tratti in comune a 10 punti}: Il tuo sputo è velenoso. Tiro Salvezza Tempra DC 20 oppure -2 Forza. Durata 1 minuto. Tre volte al giorno. Costo 1 Azione.
\item \textbf{Somma dei Tratti in comune a 15 punti}: Fissando l'obiettivo negli occhi lo costringi a fermarsi. Il target non può più muovere le gambe. Tiro Salvezza su Volontà DC 30. Una volta al giorno. Costo 2 Azioni.
\item \textbf{Somma dei Tratti in comune a 20 punti}: Dalle tue dita partono dei viticci che pungono fino a 10 avversari. Ogni viticcio causa 2d6 di danno, Tiro Salvezza Riflessi DC 25 per dimezzare. Costo 2 Azioni.
\item \textbf{Elementi}: Fuoco, Suono
\item \textbf{Vantaggio}: Duro da soggiogare
\item \textbf{Scuole Privilegiate}: Trasmutazione, (+2 alle prove di CM)
\end{itemize}

\end{multicols}

\pagebreak

\subsubsection{Tabella collegamento Patrono - Tratto}\index{Tabella collegamento Patrono - Tratto}

\medskip
%\begin{xltabular}{1\textwidth}{lllllll}
\begin{tabular}{lllllll}
	\textbf{Patrono} & \textbf{Tratto} & \textbf{Tratto} & \textbf{Tratto} & \textbf{Tratto}  & 	\textbf{Tratto}& \textbf{Tratto}\\
		\toprule
	\textbf{Ljust}& Coraggioso& Generoso& Empatico& Protettivo& Istintivo& Anticonformista\\
	\textbf{Calicante}& Egoista& Vendicativo& Superbo& Iracondo& Passionale& Diretto\\
	\textbf{Atmos}& Osservatore& Distaccato& Prudente& Riflessivo& Integerrimo& Crudele\\
	\textbf{Lynx}& Solitario& Serio& Rigido& Controllato& Coraggioso& Imbronciato\\
	\textbf{Gradh}& Indomito& Protettivo& Vendicativo& Coraggioso& Freddo& Diffidente\\
	\textbf{Atherim}& Allegro& Calmo& Industrioso& Buono& Silenzioso& Clemente\\
	\textbf{Belevon}& Confusionario& Narcisista& Casto& Bugiardo& Curioso& Doppiogiochista\\
	\textbf{Cattalm}& Distruttivo& Anarchico& Meticoloso& Sadico& Provocatore& Brutale\\
	\textbf{Efrem}& Indifferente& Leale& Ironico& Pratico& Misurato& Incostante\\
	\textbf{Erondil}& Incontentabile&Perfezionista & Sognatore& Esuberante& Geloso& Imprudente\\
	\textbf{Gaya}& Anarchico& Istintivo& Impetuoso& Emotivo& Permaloso& Lunatico\\
	\textbf{Krondal}& Attento& Pio& Corretto& Libero& Conformista& Incontentabile\\
	\textbf{Ledyal}& Introverso & Sospettoso& Caritatevole& Clemente& Timido & Leale \\
	\textbf{Laydel}&  Integerrimo& Sospettoso& Caritatevole&  Implacabile&  Suscettibile& Passionale\\
	\textbf{Nethergal}& Sarcastico& Impetuoso& Immaturo& Logorroico& Competitivo& Avventato\\
	\textbf{Nedraf}& Disciplinato& Combattivo& Tenace& Aggressivo& Pianificatore& Malizioso\\
	\textbf{Nihar}& Altruista& Determinato& Cortese& Attento& Diffidente& Caotico\\
	\textbf{Orudjs}&  Ironico& Codardo& Saccente& Socievole& Incostante& Creativo\\
	\textbf{Orlaith}& Imparziale& Giusto& Vendicativo& Valoroso& Schietto& Espansivo\\
	\textbf{Rezh}& Avaro& Arrogante& Cattivo& Freddo& Geloso& Abitudinario\\
	\textbf{Sumkjr}& Giusto& Curioso& Buono& Valoroso& Candido& Disordinato\\
	\textbf{Shayalia}& Lussurioso& Volubile& Pessimista& Sadomasochista& Arrogante& Falso\\
	\textbf{Sixiser}& Riservato& Indifferente& Accumulatore& Paranoico& Bugiardo& Insensibile\\
	\textbf{Tazher}& Scontroso& Calcolatore& Perfezionista& Cattivo& Insensibile& Burbero\\
	\textbf{Thaft}& Semplice& Silenzioso& Mite& Sicuro& Disciplinato& Ottimista\\
	\textbf{Torbiorn}&  Altezzoso& Ansioso& Vanitoso& Permaloso& Cauto& Irascibile\\
\end{tabular}


\bigskip

\label{tabella-collegamento-patrono---tratto}

\bigskip

In accordo con il Narratore, ed adeguatamente motivato, è possibile cambiare Abilità e Scuole di Magia.

\pagebreak

\section{Equipaggiamento}\hypertarget{equipaggiamento}{} 

\label{equipaggiamento}

\subsection{Ricchezza e Denaro}\index{Ricchezza e Denaro}
\begin{enfasismall}{Sono pronto ad andare. Ho uno zaino! (Chuck, Serie TV)
}\end{enfasismall}

\begin{multicols}{2}

\label{ricchezza-e-denaro}

\lettrine[lines=2, lhang=0.33, loversize=0.25, findent=1.5em]{L}{e} monete comuni si presentano in diverse denominazioni in base al valore relativo del metallo di cui sono composte. I tre tipi più comuni di monete sono la moneta d’oro (mo), la moneta d’argento (ma) e la moneta di rame (mr).

Un artigiano capace (ma non eccezionale) può guadagnare una moneta d’oro al giorno. La moneta d’oro è l’unità di misura standard della ricchezza, sebbene la moneta in sé per sé non sia molto impiegata. Quando i mercanti discutono accordi che riguardano merci o servizi del valore di centinaia o migliaia di monete d’oro, le transazioni non comprendono normalmente alcuno scambio di moneta contante.

La moneta d’oro è invece una misura di valore, e lo scambio viene effettuato in lingotti d’oro, lettere di credito o merci di valore. Una moneta d’oro vale dieci monete d’argento, il tipo di moneta più usato tra la popolazione. Una moneta d’argento può coprire il salario giornaliero di un manovale e comperare un’ampolla d’olio per lampada o un giaciglio per una notte in una locanda scadente.

Una moneta d’argento vale dieci monete di rame, che vengono normalmente usate dagli operai e i mendicanti. Una singola moneta di rame può comperare una candela, una torcia o un pezzo di gesso.

A volte, però, in mezzo ai tesori compaiono monete insolite, composte di altri metalli preziosi. La moneta di electrum (me) e la monete di platino (mp) provengono da imperi dimenticati e regni perduti, e quando usate nelle transazioni a volte suscitano sospetti e scetticismo. Una moneta di electrum vale cinque monete d’argento mentre una moneta di platino vale dieci monete d’oro.

Una moneta comune pesa circa dieci grammi, cosicché cinquanta monete pesano mezzo chilo. 

Un personaggio che inizia a giocare generalmente ha monete d'oro sufficienti per acquistare gli elementi di base: qualche arma, un'armatura di seconda mano (quella meno costosa) ed un pò di attrezzatura varia. Man mano che il personaggio intraprende avventure e accumula bottino può permettersi un equipaggiamento migliore ed oggetti magici. Al primo livello i personaggi hanno monete ed equipaggiamento per un totale di circa 100 mo.

\subsubsection{Monete}\index{Monete}

La moneta più comune è la moneta d'oro (mo). Una moneta d'oro vale 10 monete d'argento (ma). Ogni moneta d'argento vale 10 monete di rame (mr). Oltre a monete di rame, argento e oro ci sono anche le monete di platino (mp), che valgono ognuna 10 monete d'oro.

\medskip

\textbf{Tabella: Equivalenza delle Monete}\index{Tabella Equivalenza delle Monete}

\medskip

\begin{tabular}{lllll}

\textbf{Tipo di} & \textbf{Rame} & \textbf{Argento} & \textbf{Oro} & \textbf{Platino}\\
\textbf{Moneta} & (\textbf{mr})&(\textbf{ma})&(\textbf{mo})&(\textbf{mp})\\
	\toprule
	Rame& 1& 1/10& 1/100& 1/1000\\
	Argento  & 10    & 1   & 1/10 & 1/100\\
	Oro & 100   & 10  & 1    & 1/10\\
	Platino  & 1000  & 100 & 10   & 1\\
\end{tabular}

\medskip

Di solito pagamenti oltre le 100 monete d'oro avvengono in lingotti da 1,2,5 kilogrammi d'oro, equivalenti a 100, 200 e 500 monete d'oro. In caso di somme ancora piu' cospicue e' possibile che sia rilasciata una lettera di credito di qualche istituto bancario (ma valido in pochissime ed importanti città).

\subsubsection{Altre Ricchezze - Merci di scambio}\index{Altre Ricchezze}

I mercanti di solito scambiano merci anche senza l'uso di monete.
Per farsi un'idea delle transazioni commerciali, alcune merci di scambio sono descritte nella tabella.

\medskip

\textbf{Tabella: Esempi altre ricchezze}\index{Tabella Esempi altre ricchezze}

\medskip


\begin{tabular}{ll}
	\textbf{Costo} & \textbf{Oggetto}\\
	\toprule
	1 mr & Frumento (0.5 kg)\\
	2 mr & Farina (0.5 kg) o pollo (1)\\
	1 ma & Ferro (0.5 kg)\\
	5 ma & Tabacco o rame (0.5 kg)\\
	1 mo & Cannella (0.5 kg) o capra \\
	2 mo & Zenzero o pepe (0.5 kg) o pecora (1)\\
	3 mo & Maiale (1) \\
	4 mo & Lino (1 m\textsuperscript{2}\\
	5 mo & Sale o argento (0.5 kg) \\
	10 mo& Seta (1 m) o mucca (1)\\
	15 mo& Zafferano(0.5 kg)/bue (1)\\
	30 mo&Chiodi di garofano (1kg)\\
\end{tabular}

\medskip

Consultate anche il capitolo sull'Ingombro in Movimento e Trasporto.

\end{multicols}

\pagebreak

\section{Equipaggiamento - Armi}\index{Equipaggiamento}\index{Armi} 
\hypertarget{equipaggiamento.armi}{} 

\label{equipaggiamento---armi}
\begin{enfasi}{
Questo è il mio fucile. Ce ne sono tanti come lui, ma questo è il mio. Il mio fucile è il mio migliore amico, è la mia vita. Io debbo dominarlo come domino la mia vita. Senza di me il mio fucile non è niente; senza il mio fucile io sono niente. Debbo saper colpire il bersaglio, debbo sparare meglio del mio nemico che cerca di ammazzare me, debbo sparare io prima che lui spari a me e lo farò. Al cospetto di Dio giuro su questo credo: il mio fucile e me stesso siamo i difensori della patria, siamo i dominatori dei nostri nemici, siamo i salvatori della nostra vita e così sia, finché non ci sarà più nemico ma solo pace, amen. (Full Metal Jacket, Film 1987)
	
\medskip
		
La spada davvero buona è quella che rimane nel suo fodero. (Sanjuro)}\end{enfasi}

\medskip

Ricordo che usare un'Arma senza l'adeguata competenza impone un -1d6 al colpire

La tabella presenta il nome dell'arma, il suo costo in monete d'oro, il danno ed il tipo di danno (se da Taglio, Botta o Punta), la gittata, la Lista d'Arma appartenente e le caratteristiche speciali che può avere. \hyperref[sec:Azioni particolari in combattimento]{Vedi sezione Azioni Particolari in Combattimento}, vedi anche \hyperref[sec:capacita-di-carico-e-trasporto-ingombro]{Capacità di Carico e Trasporto.}

\medskip

\textbf{Tabella: Lista della Armi}\index{Tabella Lista della Armi}

\begin{tabular}{lllll}
\textbf{Arma}&\textbf{Costo}&\textbf{Taglia/Danno} & \textbf{Gittata, Lista, Speciale} & Peso (kg)\\
	\toprule
Alabarda& 10 & G/1d10 P/T& \textbf{Lance}, \textbf{Aste}, Controcarica, Arma lunga, ED9 & 4\\
Arco Corto& 30 & M/1d6 P& 15 metri, \textbf{Arco}, da tiro& 1\\
Arco Corto Composito& note*& M/Frecce& 20 metri, \textbf{Arco}, da tiro& 1.5\\
Arco Lungo& 75 & G/Frecce& 20 metri, \textbf{Arco}, da tiro& 2\\
Arco Lungo Composito& note*& G/Frecce& 36 metri, \textbf{Arco}, da tiro& 2.5\\
Ascia ad una mano& 6  & M/1d6 T& 6 metri, \textbf{Asce}, \textbf{Armi da Tiro}, Versatile& 1\\
Ascia da battaglia& 10 & M/1d10 T&\textbf{Asce}& 3\\
Ascia Martello& 16 & M/1d6 T/B& \textbf{Asce}& 3\\
Balestra ad una mano& 100& M/Dardi& 6 metri, \textbf{Balestre}, da tiro& 1\\
Balestra leggera& 35 & P/Dardi& 15 metri, \textbf{Balestre}, \textbf{Armi Semplici}, da tiro& 0.5\\
Balestra pesante& 50 & G/Dardi& 20 metri \textbf{Balestre}, da tiro& 3\\
Bastone& 3& M/1d6 B& \textbf{Armi Semplici}, Arma lunga, Versatile& 2\\
Bolas& 4& P/1d3 B&6 metri, \textbf{Bloccanti}, intralciato& 0.5\\
Brandistocco& 10 & M/2d4 P/T& \textbf{Lance}, Controcarica, Arma lunga& 3\\
Catena chiodata& 25 & G/2d4 P& \textbf{Palle rotanti}, Arma lunga& 4\\
Falce& 18 & G/2d4 P/T& \textbf{Armi della Morte}, Arma lunga& 3\\
Falcetto& 6& P/1d6 T& \textbf{Armi della Morte} & 1\\
Falcione& 75 & M/2d4 T& \textbf{Armi Aggraziate}, \textbf{Lance}, ED7& 2\\
Falcione in asta& 12 & G/1d10 P/T& \textbf{Lance}, Controcarica, Arma lunga, ED9& 3\\
Fionda& -& P/1d4 B& 10 metri, \textbf{Archi}, da tiro& 0.5\\
Flagello& 8& M/1d8 B& \textbf{Palle Rotanti}, \textbf{Rompi Cranio}& 3\\
Flagello Doppio& 90 & M/1d10 B& \textbf{Palle Rotanti}, \textbf{Armi doppie} & 4\\
Flagello Pesante& 15 & M/1d10 B& \textbf{Palle Rotanti}, \textbf{Armi doppie}& 3\\
Frusta& 1& M/1d3 T& \textbf{Palle Rotanti}, Arma lunga& 2\\
Giavellotto& 1& P/1d6P& 12 metri, \textbf{Aste}, \textbf{Armi da tiro} \textbf{Armi Semplici}& 1.5\\
Grande Ascia Doppia& 25 & G/1d12 T& \textbf{Asce}, \textbf{Armi doppie}, Arma lunga& 4\\
Grosso randello& 2& M/1d8 B&\textbf{Rompi Cranio}& 2\\
Guanto chiodato& 5& P/1d4 P&\textbf{Armi da Stordimento}& 1\\
Katana& 300& M/1d10 T& \textbf{Spade}, ED9, Versatile& 1.5\\
Lancia& 10 & G/1d8 P&\textbf{Lance}, Arma lunga, Controcarica& 3\\
Lancia corta da fante& 1& M/1d6 P& 6 metri, \textbf{Armi da tiro}, \textbf{Armi Semplici},\textbf{Aste} & 1.5\\
Lancia da fante& 2& M/1d8 P&6 metri, \textbf{Lance}, Arma lunga, Controcarica& 2 \\
Machete& 10 & M/1d6 T&\textbf{Armi letali} & 1\\
Manganello& 1& P/1d6 B& \textbf{Armi da stordimento}, non letale& 0.5\\
Martello da guerra& 5& M/1d8 B/P& 6 metri, \textbf{Rompi Cranio}& 1.5\\
Mazza Leggera& 3& P/1d6 B/T& \textbf{Armi Leggere}, \textbf{Rompi Cranio}, \textbf{Armi Semplici}&1\\
Mazza Pesante& 5& M/1d8 B/T& \textbf{Rompi Cranio}& 2\\
Morningstar& 6& M 1d8 B/P&\textbf{Rompi Cranio},\textbf{ Armi Semplici}& 1\\
Naginata& 8& G/1d10 T&\textbf{Lance}, Arma lunga, ED9& 2\\
\end{tabular}


\begin{tabular}{lllll}
\textbf{Arma}&\textbf{Costo}&\textbf{Taglia/Danno} & \textbf{Gittata, Lista, Speciale} & Peso (kg)\\
	\toprule
Picca Leggera& 4& M/1d4 P&\textbf{Armi della morte}& 1\\
Picca Pesante& 8& G/1d6 P&\textbf{Armi della morte}, Arma lunga& 3\\
Pugnale& 2& P/1d4 P& 6 metri, \textbf{Armi leggere}, \textbf{Armi da tiro}\\
&&& \textbf{Armi letali}, \textbf{Armi Semplici}, Versatile& 0.5\\
Pugno/Calcio nudo& note*& P/1d4 B&Versatile& -\\
Randello& 1& P/1d6 B&\textbf{Rompi Cranio}, \textbf{Armi Semplici}& 0.5\\
Rete& 8& M-&3 metri, \textbf{Bloccanti}, intralciato& 1\\
Scimitarra& 15 & M/1d6 T&\textbf{Armi Leggere}, \textbf{Armi Aggraziate}, Versatile& 1.5\\
Spada a due lame& 100& G/1d8 T& \textbf{Armi doppie}, \textbf{Spade}, Arma lunga& 3\\
Spada bastarda& 35 & M/1d10 T&\textbf{Spade}& 2\\
Spada Corta& 10 & P/1d6 P&\textbf{Armi Leggere}, \textbf{Spade}, Versatile& 1\\
Spada Lunga& 15 & M/1d8 T&\textbf{Spade}& 1.5\\
Spadone a due mani& 50 & G/2d6 T&\textbf{Spade}& 3\\
Stocco& 20 & P/1d6 P& \textbf{Armi Leggere}, \textbf{Armi Aggraziate}, Versatile& 1\\
Tridente& 15 & M/1d6 P/T& 3 metri, \textbf{Aste}, \textbf{Armi da tiro}, Arma Lunga, Controcarica& 2\\
Urgrosh& 18 & M/1d6 T/P& \textbf{Lance}, \textbf{Armi doppie}, Arma lunga & 3\\
\end{tabular}


\bigskip

\textbf{Tabella: Lista dei proiettili - Archi - Balestre - Fionde}\index{Tabella Lista dei proiettili - Archi - Balestre - Fionde}

\begin{tabular}{lccc}	
	\textbf{Nome Proiettile}& \textbf{Numero di colpi/Costo (mo)} & \textbf{Danno / Tipo} &  Peso(kg)\\
	\toprule
	Dardi da balestra (una mano, leggera) & 10/1 mo & 1d6 P  & 0.1\\
	Dardi per balestra (pesante) & 3/1 mo & 1d10 P  & 0.3\\
	Frecce da caccia (Arco Corto, Arco Lungo)& 20/1 mo & 1d6 P& 0.1\\
	Frecce da guerra (Arco Lungo)& 10/1 mo & 1d8 P &0.2\\
	Biglie di Marmo (fionde)& 15/1 mo & 1d4 B &  0.2\\
	Sasso (fionde)& -& 1d2 B & 0.2\\
\end{tabular}

\medskip

\begin{multicols}{2}
	
Una Freccia/Dardo/Sasso magico con un bonus +1 costa 25 mo, se +2 costa 100 mo, se +3 costa 400 mo.
	
\textbf{Un proiettile non acquisisce proprietà magiche perché il suo lanciatore è magico.}
	
\medskip
	
\textbf{Pugno Vuoto}: \hyperlink{pugnovuoto}{vedi Lista d'Armi}
	
\bigskip
	
\textbf{Arco Composito}\index{Arco Composito}
Un arco composito è un arco particolarmente robusto e rigido che richiede un certo minimo di Forza per essere usato.
Un arco composito ha un modificatore fisso, da +1 a +5. Se chi lo indossa ha Forza superiore a questo modificatore può applicare al danno della freccia un bonus pari al modificatore dell'arco composito.
Un arco composito +3 usato da un personaggio con Forza 2 può tirare l'arco non completamente e quindi la freccia che parte avrà un modificatore al danno di +2.
Un arco composito +1 usato da un personaggio con Forza 4 verrà tirato completamente ma il modificatore al danno potrà essere al massimo +1.
	
Il costo di un arco composito dipende dal sul modificatore.
Un arco composito con modificatore di +1 costa 75 mo, +2 150 mo, +3 300 mo, +4 600 mo, +5 1500 mo.
	
Un arco corto composito ha come massimo modificatore +4.

\textbf{Gittata}\index{Gittata}
La distanza indicata è quello a pieno Tiro per Colpire. Ogni arma a distanza può colpire entro tre volte la distanza indicata. 
	
Se il target è entro la distanza indicata non si hanno malus al colpire, se il target è tra il primo e secondo incremento il malus al colpire è -1d6. Se il target è tra il secondo è terzo incremento il malus al colpire è di -2d6.
	
Un giavellotto tirato entro 12 metri non ha malus, ma tirato entro 24 metri ha un -1d6 al colpire, a distanza tra 24 e 36 metri un -2d6 al colpire.
	
\medskip

\begin{center}
	\includegraphics[width=0.7\linewidth]{immagini/bow2.png}
\end{center}
	
\medskip
	
Una \textbf{Freccia o Dardo che colpisce si considera distrutta}, se manca si considera che abbia un 50\% (4-5-6 su un d6) di probabilità che sia ancora integra.
	
Una Freccia/Dardo/Sasso magico somma i suoi bonus a quelli del lanciatore per determinare il Tiro per Colpire ed il Danno.
	
Ricordate che un proiettile normale lanciato da un lanciatore magico non diviene magico.
	
\medskip
	
La \textbf{Taglia dell'Arma} è indicata come P (piccola), M (media), G (grande). \hyperref[sec:Arma troppo grande]{Vedi sezione Arma troppo grande}
	
Una \textbf{arma di taglia superiore} \index{Arma di taglia superiore} come ad esempio una Spada Lunga forgiata per un Ogre aumenta di una categoria il suo dado di danno (1d4-1d6-1d8-1d10-2d6-2d8-2d10..)
	

Le Armi hanno indicato una \textbf{Tipologia di danno}\index{Tipologia di danno}, ovvero T/B/P.
Queste lettere stanno ad indicare se il danno è di tipo Taglio, Botta o da Perforazione. Questa caratteristica può essere importante perché	determinate creature possono essere immuni o subire meno danno da un particolare tipo di ferita (es uno scheletro contro un'arma da penetrazione o un cubo gelatinoso contro un arma da taglio..)
	
\medskip
	
	
\textbf{Armi Perfette}\index{Armi Perfette}
	
Un arma perfetta è un arma creata da un abilissimo armaiolo che pur non essendo magica, grazie al suo perfetto bilanciamento ed affilatura, ha un +1 al Tiro per Colpire.

Un armaiolo per creare un arma perfetta deve superare di 10 la DC impostata per la creazione dell'arma.

Un arma perfetta costa il doppio di un arma normale.
	
\textbf{Armi Improvvisate}\index{Armi Improvvisate}
	
Talvolta oggetti che non sono stati creati per essere armi possono avere una certa efficacia in combattimento. Dal momento che non si tratta di oggetti pensati per questo utilizzo, la creatura che attacca con uno di essi subisce una penalità -1d6 al Tiro per Colpire. Un'arma improvvisata di piccole dimensioni (bottiglia) fa 1d3 di danno, di medie dimensioni (la gamba di una sedia) da 1d6, di grandi dimensioni (la gamba di un tavolo) fa 1d8 di danno.
	
	Un'arma da lancio improvvisata ha una gittata 3 metri.
	
	\medskip
	
	\textbf{Lanciare armi}\index{Lanciare armi}
	
	Una spada o comunque un arma non fatta per essere lanciata può comunque essere scagliata contro l'avversario. Il Tiro per Colpire prende un -1d6 e l'arma fa una categoria di danno inferiore (la spada lunga fa 1d6, una spada corta 1d4..). La gittata è 3 metri.
	
	\medskip
	
	\textbf{Usare un'Arma senza l'adeguata competenza se non è un Arma Semplice} Impone un -2d6 al Tiro per Colpire.
	
	\textbf{Esempio}: Una creatura piccola che usa un alabarda in combattimento ravvicinato ha -1d6 perché l'arma è grande, -1d6 perché non è competente, -1d6 perché usa l'arma in mischia.
	
	In questo caso essendo i malus superiori ai 3d6 il personaggio non tira dadi ma usa solo la sua Competenza Armi come valore per colpire.
	
\end{multicols}

\addvspace{4cm}

\begin{center}
	\includegraphics[width=0.8\linewidth]{immagini/armiriempitivo3.png}
\end{center}


\pagebreak

\section{Equipaggiamento - Armature e Scudi} \index{Armature}\index{Scudi}\hypertarget{equipaggiamento.armature.scudi}{} 

\label{equipaggiamento---armature-e-scudi}

\begin{enfasi}{
Armatura (s.f.). Abito che si indossa se il proprio sarto è un fabbro. (Ambrose Bierce)
	
\medskip
	
Armatura Fantozzi: banderuola 4 venti in funzione di pennacchio, pauroso elmo vichingo con visibilità azzerata, sospensorio in bronzo sottratto alla statua di Pipino il Breve e, ai piedi, ferroni da stiro a carbonella di piombo fuso. Peso complessivo armatura Fantozzi: 4 quintali, 32 chili e 7 etti e mezzo. (Superfantozzi, Film)} \end{enfasi}\medskip

\begin{multicols}{2}
	
\lettrine[lines=2, lhang=0.33, loversize=0.25, findent=1.5em]{L}{e} armature aiutano ad essere non colpiti (alzano la Difesa) e penalizzano la prova di magia e le prove di competenza basate su Destrezza. 

La Prove Destrezza è la penalità che si applica alle prove di competenze di Destrezza mentre si indossa un quel tipo di armatura.

Armature diverse, specifiche o magiche hanno punteggio diversi, questa tabella serve come linea guida per il Narratore.
	
	
\end{multicols}

\subsubsection{Tabella Armature}\index{Tabella Armature}

\medskip

\label{tabella-armature}
\begin{tabular}{llllllll}
	%\begin{xltabular}{0.95\textwidth}{lXXXXXXX}
	\textbf{Armatura} & \textbf{Costo (mo)} & \textbf{Difesa} & \textbf{Prove Destrezza} & \textbf{Prove di Magia} & \textbf{Tipo} & \textbf{Mov.} & \textbf{Peso (kg)}\\
	\toprule
	Imbottita   		& 5    & 1   & 0  	 & 0  & L   & 0   & 4\\
	Cuoio   			& 10   & 2   & 0   & -1 & L   & 0   & 5\\
	Cuoio rinforzato   	& 25  &3  & 0   & -2 & L   & 0   & 6.5\\
	Giaco di Maglia   	& 15   & 4  & -1  & -3 & M   & 0   & 10\\
	Scaglie				& 50   & 5  & -1  & -4 & M   & 0   & 22.5\\
	Anelli 				& 150  & 6  & -1  & -5 & M   & 0   & 20\\
	Pettorale    		& 200  & 6  & -2  & -5 & M   & 0   & 10\\
	Bande   			& 250  & 7  & -2  & -6 & P   & 0   & 30\\
	Mezza armatura   	& 1200 & 8  & -2  & -7 & P   & 1   & 20\\
	da Campo			& 1400 & 9 & -3  & -7 & P   & 2   & 25\\
	Completa			& 1500 & 10  & -4  & -8 & P   & 3   & 32.5\\
\end{tabular}

\medskip

\textbf{Costo}: è per un armatura di taglia media.

\textbf{Difesa}: è il bonus data alla Difesa

\textbf{Prove Destrezza}: è il malus dato alle prove di Destrezza dato dal peso ed ingombro dell'armatura.

\textbf{Prove di Magia}: è il malus dato alle prove di Competenza Magica per lanciare un incantesimo

\textbf{Tipo}: indica se l'armature è \textbf{L}eggera, \textbf{M}edia oppure \textbf{P}esante

\textbf{Mov. (movimento)}: è la riduzione in metri di movimento da applicare per Azione di Movimento.

\textbf{Peso}: è il peso dell'armatura da conteggiare.

\medskip

\begin{narratore} Quando conteggiate il peso dato dall'armatura e scudo \textbf{indossato} dovete dividerlo per due (arrotondate per eccesso).
	
Il peso segnato per armatura e scudi è da intendersi quando è "caricata nello zaino", ovvero trasportata ma non indossata.\end{narratore}

\medskip

\begin{multicols}{2}
	
\subsubsection{Descrizione delle Armature}

\textbf{Armature Leggere}

Fatte di materiali leggeri e flessibili, le armature leggere favoriscono gli avventurieri agili dato che offrono protezione senza sacrificare la mobilità.

\textit{Imbottita}. Le armature imbottite consistono di strati di tessuto e imbottitura cuciti insieme. 

\textit{Cuoio}. Il corpetto e le protezioni delle spalle di questa armatura sono fatte di cuoio indurito dopo essere stato bollito nell'olio. Il resto dell'armatura è composto di
materiali più morbidi e flessibili.

\textit{Cuoio Rinforzato}. Fatta di cuoio duro ma flessibile, l'armatura di cuoio rinforzato è arricchita da rivetti o spuntoni.
	
\medskip
	
\textbf{Armature Medie}

Le armature medie offrono più protezione di quelle leggere, ma limitano i movimenti. 

\textit{Giaco di Maglia}. Composto di anelli metallici intrecciati tra di loro, un giaco di maglia viene indossato sopra strati di abiti o cuoio. Questo tipo di armatura offre una protezione modesta alla parte superiore del corpo, mentre il rumore degli anelli che strusciano fra di loro viene attutito dagli altri strati.

\textit{Scaglie}. Quest'armatura consiste in una cotta e gambali (a volte anche di una gonna separata) di cuoio coperti da pezzi di metallo sovrapposti, in maniera simile alle scaglie di un pesce. L'armatura è completa di guanti.

\textit{Anelli}. Quest'armatura è un'armatura di cuoio con dei pesanti anelli cuciti sopra. Gli anelli servono a rinforzare l'armatura contro i colpi di spada e d'ascia. L'armatura è completa di guanti.

\textit{Pettorale}. Questa armatura consiste di un corpetto di metallo indossato su di uno strato di cuoio. Sebbene lasci braccia e gambe relativamente scoperte, l'armatura fornisce una buona protezione agli organi vitali del personaggio, senza procurargli grande ingombro.
	
\medskip
		
\textbf{Armature Pesanti}

\textit{Bande}. Questa armatura e fatta di strisce di metallo cucite a un robusto schienale di cuoio e maglia di ferro. Le dimensioni delle piastre metalliche, interconnesse alle bande di metallo, e gli strati di armatura sottostanti la rendono una delle più protettive tra le armature.

\textit{Mezza Armatura}. La mezza armatura di piastre consiste di piastre di metallo sagomate che coprono gran parte del corpo del personaggio. Non comprende protezioni per le gambe oltre a dei semplici schinieri legati con lacci di cuoio.

\textit{da Campo}. Molto simile all'armatura completa ma più leggera in costruzione sacrificando un poco di protezione a favore di una maggiore flessibilità e mobilità.

\textit{Completa}. Quest'armatura consiste di piastre di metallo sagomate a incastro che coprono l'intero corpo. Un'armatura di piastre comprende guanti, stivali di cuoio pesanti, un elmo con visiera, e uno spesso strato di imbottitura sotto l'armatura. Fibbie e lacci distribuiscono il peso dell'armatura su tutto il corpo.
	

\subsubsection{Regole base per l'utilizzo dell'armatura}
	
\textbf{Usare un'Armatura senza l'adeguata competenza} impedisce di usare il bonus di Destrezza e diminuisce il bonus alla Difesa fornito di 1.
	
\textbf{Usare uno Scudo senza l'adeguata competenza} peggiora il Tiro per Colpire di 1 e diminuisce di 1 il Bonus Difesa concesso.
	
\textbf{Dormire in Armatura}: se si dorme in un'armatura media o pesante, il giorno seguente si è automaticamente Affaticati (-1 Cos, Des non si può correre).
	
Dormire in un'armatura leggera non provoca Affaticamento.
	
La \textbf{capacità di movimento} del personaggio rimarrà la medesima fino all'armatura a bande poi calerà progressivamente. Il valore indicato nella colonna Mov. sono i metri in meno che il personaggio fa per Azione di Movimento.
	
Ad Esempio un umano in armatura completa ha movimento 6 metri, un nano 3 metri.
	
\textbf{Peso}: il peso indicato si riferisce alla versione per personaggi di taglia Media. Le armature adattate per personaggi di taglia Piccola pesano la metà, mentre per quelli di taglia Grande pesano il doppio.
	
\textbf{Armature Perfette}\index{Armature Perfette}
	
Un'armatura perfetta è un armatura creata da un abilissimo fabbro che pur non essendo magica, grazie al suo perfetto bilanciamento ed affilatura, ha un +1 alla Difesa.

Un fabbro per creare un arma perfetta deve superare di 10 la DC impostata per la creazione dell'armatura.

Un Armatura perfetta costa il doppio di un armatura normale.
	
	
\begin{center}
	\includegraphics[width=0.9\linewidth]{immagini/donnacavalierecavallo.png}
\end{center}
	
\subsubsection{Gli Scudi}
	
Gli \textbf{Scudi} \index{Scudi}permettono di aumentare la propria Difesa, più lo scudo è imponente e pesante più protegge, più aumentano le penalità alle prove di competenza magica e meno rende facile combattere (penalità Tiro per Colpire).
	
Gli Scudi possono essere di tipo Leggero, Medio, Pesante.
	
\end{multicols}

\subsubsection{Tabella Scudi}\index{Tabella Scudi}

\label{tabella-scudi}
\medskip

\begin{tabular}{lcccccc}
\textbf{Scudi} & \textbf{Costo} & \textbf{Bonus Difesa} & \textbf{Malus TC} & \textbf{Malus prova di Magia} & \textbf{Peso (kg)} & \textbf{Tipo}\\
	\toprule
Brocchiero					& 5 mo  	& 1 & 0		& 1& 1  & L\\
Scudo leggero di legno   	& 3 mo  	& 1 & 0		& 2& 2  & L\\
Scudo leggero di metallo 	& 9  mo  	& 1 & 0		& 3& 3  & L\\
Scudo medio legno   		& 5 mo   	& 2 & -1	& 4& 3  & M\\
Scudo medio metallo 		& 12 mo  	& 2 & -1  	& 5& 5  & M\\
Scudo pesante di legno   	& 7  mo  	& 3 & -2    & 6& 5  & P\\
Scudo pesante di metallo 	& 20 mo  	& 3 & -2    & 7& 7  & P\\
\end{tabular}

\medskip

\begin{multicols}{2}
	
\textbf{Bonus Difesa}: e' il bonus che si applica alla Difesa quando si imbraccia lo scudo.

\textbf{Penalità TC}: e' la penalita' al Tiro per Colpire che si ha quando si imbraccia lo scudo.

\textbf{Penalità prova di Magia}: e' la penalita' alla prova di magia che si ha quando si tenta di lanciare un incantesimo con uno scudo in mano.

\textbf{Peso (kg)}: e' il peso espresso in kilogrammi. Per il computo dell'ingombro uno scudo indossato pesa la metà.

\textbf{Tipo}: indica la taglia dello scudo. \textbf{L}eggero, \textbf{M}edio, \textbf{P}esante.

Uno scudo può essere usato come \textbf{arma improvvisata}. Il Tiro per Colpire è penalizzato di -1d6 e uno scudo piccolo fa 2 di danno (B/T), uno scudo medio fa 1d4 di danno (B/T), uno scudo pesante fa 1d6 di danno (B/T).

Usare lo scudo come arma improvvisata non lo fa applicare il suo bonus alla Difesa.

Imbracciare uno scudo occupa una mano/braccio.
	
\end{multicols}

\subsubsection{Indossare e Togliere Armature}\index{Indossare e Togliere Armature}

\begin{multicols}{2}
	
Indossare e togliere armature è una operazione che richiede tempo ed attenzione, farlo in fretta non aiuta ed anzi tende a peggiorare la protezione data dall'armatura.

\end{multicols}


\textbf{Tabella: Tempi per indossare e togliere l'armatura}\index{Tabella Tempi per indossare e togliere l'armatura}


\medskip

\begin{tabular}{llll}
\textbf{Tipo di Armatura}& \textbf{Indossare} & \textbf{Indossare in fretta} & \textbf{Togliere}\\
	\toprule
Scudo								& 1 azione 	& -     	& 1 azione\\
Imbottita, Cuoio, Cuoio rinforzata  & 1 minuto	& 3 round  	& - \\
Giaco di Maglia						& 1 minuto	& 5 round  & 5 round\\
Scaglie, Anelli, Pettorale, Bande   & 4 minuti 	& 1 minuto{*}  & 1 minuto\\
Mezza armatura, da Campo, Completa  & 4 minuti{*}{*}& 4 minuti{*}& 1d4+1 minuti\\
\end{tabular}

\medskip

\begin{multicols}{2}
	
{*} Se qualcuno aiuta, il tempo si dimezza. Un singolo personaggio che non sta facendo altro può aiutare uno o due personaggi adiacenti a lui. Due personaggi non possono aiutarsi l'un l'altro a indossare un'armatura contemporaneamente.

	
{*}{*} Bisogna essere aiutati per indossare questa armatura. Senza aiuto è possibile indossarla solo in fretta.
	
\textbf{Indossare un'armatura in fretta} implica un malus di -1 alla Difesa ed un malus aggiuntivo di +1 alle prove di Destrezza.
	
\end{multicols}

\medskip

\begin{center}
	\includegraphics[width=0.30\linewidth]{immagini/armaturacorpetto.png}
\end{center}

\pagebreak


\section{Merci e Servizi}\index{Merci}\index{Servizi}


\subsection{Ricchezza e Denaro}\index{Ricchezza e Denaro}


\begin{enfasi}{
- Doc... c'è soltanto bisogno di un pochino di plutonio.

\medskip

- Ah, sono certo che nell'85 il plutonio si compra nella drogheria sotto casa, ma nel '55 la faccenda è molto più complicata! (Ritorno al futuro, Film 1985)} 
\end{enfasi}\medskip

\begin{multicols}{2}
	
\subsubsection{Vendere Tesori}
	
\lettrine[lines=2, lhang=0.33, loversize=0.25, findent=1.5em]{N}{ei} sotterranei che esplorerai avrai ampie opportunità di trovare tesori, equipaggiamento, armi, armature e altro ancora. Di solito, potrai vendere tesori e ninnoli quando raggiungerai un paese o altro insediamento, purché tu riesca a trovare acquirenti e mercanti interessati al tuo bottino.
	
\medskip
	
\textbf{Armi, Armature e Altro Equipaggiamento }
	
Come regola generale, le armi, le armature e il resto dell’equipaggiamento illeso quando viene venduto costa la metà. È difficile che le armi e le armature utilizzate dai mostri siano in condizioni ottimali per la vendita.
	
\medskip
	
\textbf{Oggetti Magici}
	
La vendita di oggetti magici è un problema. Trovare qualcuno che voglia comprare una pozione o pergamena non comporta grandi difficoltà, ma la maggior parte degli oggetti sono fuori della portata delle tasche di chiunque salvo dei nobili più ricchi. Inoltre, a parte alcuni oggetti magici comuni, è difficile riuscire a trovare oggetti magici o incantesimi in vendita. Il valore della magia trasale la vile moneta e dovrebbe essere sempre trattato con riguardo.
	
\medskip
	
\textbf{Gemme, Gioielli e Oggetti d’Arte}
	
Questi oggetti mantengono il loro pieno valore sul mercato, e li puoi decidere di scambiare per soldi o usarli come moneta corrente nelle transazioni. Nel caso di tesori di eccezionale valore, il Narratore potrebbe richiedere che tu riesca prima a trovare un acquirente in un grosso paese o addirittura una comunità più grande.
	
\medskip
	
\textbf{Merci}
	
Sulle terre di confine, la maggior parte delle transazioni avvengono tramite baratto. Come le gemme e gli oggetti d’arte, le merci - lingotti di ferro, sacchi di sale, bestiame e così via - possono essere scambiate come moneta corrente al loro pieno valore.
	
\medskip
	
\end{multicols}

\addvspace{4cm}

\begin{center}
	\includegraphics[width=0.7\linewidth]{immagini/jewelry-box-2931784_1280.png}
\end{center}

\pagebreak

\begin{multicols}{2}
	
\subsubsection{Equipaggiamento da Avventura}

Questo è un breve e non esaustivo elenco di equipaggiamento che i vostri personaggi potrebbero essere interessati a comprare.  L'elenco non è certo esaustivo o completo ma potrà fornirvi linee guida sui prezzi.

Come Narratore usate sempre il buon senso nelle richieste valutate bene la tipologia di richiesta, la necessità dell'oggetto, il luogo dove si compra e come lo si compra.

In base alla tipologia di compagna potrebbero essere disponibili ulteriori oggetti quali armi da fuoco o alchemici.

\end{multicols}

\begin{tabularx}{\linewidth}{Xcc|Xcc}	
	\textbf{Oggetto}	&\textbf{Costo}&	\textbf{Peso}&\textbf{Oggetto}	&\textbf{Costo}	&\textbf{Peso}\\
&&&&&\\
	\toprule
	Abaco&	2 mo&	1 kg&Abito, Comune&	5 ma&	1,5 kg\\
	Abito, Costume	&5 mo&	2 kg&Abito, da Viaggiatore	&2 mo&	2 kg\\
	Abito, Pregiato	&15 mo&	3 kg&Acciarino e Pietra Focaia&	5 ma&	0,5 kg\\
	Acido (fiala)	&25 mo&	0,5 kg&Acqua Santa (ampolla)&	25 mo&	0,5 kg\\
	Ampolla o Boccale	&2 mr&	0,5 kg&Anello con Sigillo&	2 mo&	-\\
	Antitossina (fiala)&	50 mo&-&Ariete Portatile	&	4 mo&	17,5 kg\\
	Asta (3 metri)	&5 mr&	3,5 kg&Attrezzi da Scalatore	&25 mo&	6 kg\\
	Barile&	2 mo&	35 kg&Bilancia da Mercante&	5 mo&	1,5 kg\\
	Borsa&	5 ma&	0,5 kg&Borsa del Guaritore&	5 mo&	1,5 kg\\
	Borsa per Componenti	&25 mo&	1 kg&Bottiglia di Vetro&	2 mo	&1 kg\\
	Brocca o Caraffa&	2 mr	&2 kg&Campanella	&1 mo&-\\
	Candela	&1 mr&-&Canestro&	4 ma	&1 kg\\
	Canna da Pesca&	1 mo&	2 kg&Cannocchiale&	1.000 mo	&0,5 kg\\
	Carrucola e Paranco&	1 mo&	2,5 kg&Carta (un foglio)&	2 ma&-\\
	Catena (3 metri)&5 mo&5 kg&Chiodo da Rocciatore&5 mr&0,125 kg\\
	Clessidra&25 mo&0,5 kg&Coperta&5 ma&1,5 kg\\
	Corda di Canapa (15 metri)&1 mo&5 kg&Corda di Seta (15 metri)&10 mo&2,5 kg\\
	Cote per Affilare&1 mr&0,5 kg&Custodia per Mappe/Pergamene&1 mo&0,5 kg\\
	Custodia per Dardi da Balestra&1 mo&0,5 kg&Faretra&1 mo&0,5 kg\\
	Fiala&1 mo&-&Fischietto da Richiamo&5 mr&-\\
	Bacchetta&10 mo&0,5 kg&Bastone&5 mo&2 kg\\
	Cristallo&10 mo&0,5 kg&Globo&20 mo&1,5 kg\\
	Verga&10 mo&1 kg&Bacchetta in Legno di Tasso&10 mo&0,5 kg\\
	Bastone di Legno&5 mo&2 kg&Rametto di Vischio&1 mo&-\\
	Totem&1 mo&-&Forziere&5 mo&12,5 kg\\
	Fuoco dell'Alchimista (ampolla)&50 mo&0,5 kg&Gavetta&2 ma&0,5 kg\\
	Gessetto (1 pezzo)&1 mr&-&Giaciglio&1 mo&3,5 kg\\
	Inchiostro (30 grammi)&10 mo&-&Lampada&5 ma&0,5 kg\\
	Lanterna a Lente Sporgente&10 mo&1 kg&Lanterna Schermabile&5 mo&1 kg\\
	Lente d'Ingrandimento&100 mo&-&Libro&25 mo&2,5 kg\\
	Libro degli Incantesimi&50 mo&1,5 kg&Manette&2 mo&3 kg\\
	Martello&1 mo&1,5 kg&Martello da Demolizione&2 mo&5 kg\\
	Aghi da Cerbottana (50)&1 mo&0,5 kg&Frecce (20)&1 mo&0,5 kg\\
	Olio (ampolla)&1 ma&0,5 kg&Otre&2 ma&2,5 kg\\
	Pennino&2 mr&-&Pergamena (un foglio)&1 ma&-\\
	Piccone da Minatore&2 mo&5 kg&Piede di Porco&2 mo&2,5 kg\\
	Pozione di Guarigione&50 mo&0,25 kg&Profumo (fiala)&5 mo&-\\
	Rampino&2 mo&2 kg&Razioni (1 giornata)&5 ma&1 kg\\
	Sacco&1 mr&0,25 kg&Sapone&2 mr&-\\
	Scala a Pioli (3 metri)&1 ma&12,5 kg&Secchio&5 mr&1 kg\\
	Serratura&10 mo&0,5 kg&Sfere Metalliche (sacchetto da 100)&1 mo&1 kg\\
	Amuleto&5 mo&0,5 kg&Emblema&5 mo&-\\
	Reliquiario&5 mo&1 kg&Specchio Piccolo di Metallo&5 mo&0,25 kg\\
	Spuntoni, Ferro (10)&1 mo&2,5 kg&Tagliola&5 mo&12,5 kg\\
	Tenda, per Due Persone&2 mo&10 kg&Torcia&1 mr&0,5 kg\\
	Triboli (sacchetto da 20)&1 mo&1 kg&Vanga o Badile&2 mo&2,5 kg\\
	Vaso di Ferro&2 mo&5 kg&Veleno, base (fiala)&100 mo&-\\
	Veste&1 mo&2 kg&Zaino&2 mo&2,5 kg\\
\end{tabularx}


\bigskip

\begin{multicols}{2}

\begin{enfasismall}{
		Ogni tecnologia sufficientemente avanzata è indistinguibile dalla magia. (Arthur C. Clarke, da Profiles of the Future)
}\end{enfasismall}

\textbf{Acido}. Con un’azione, puoi spargere il contenuto di questa fiala su di una creatura entro 1 metri da te o lanciare la fiala fino a 6 metri, fracassandola all’impatto. In entrambi i casi, effettua un Tiro per Colpire a distanza contro la creatura o l’oggetto, trattando l’acido come un’arma improvvisata (-1d6 Tiro per Colpire). Se colpisci, il bersaglio subisce 2d6 danni da acido.

\textbf{Acqua Sacra}. Con un’azione, puoi spargere il contenuto di questa ampolla su di una creatura entro 1 metri da te o lanciare l’ampolla fino a 6 metri, fracassandola all’impatto. In entrambi i casi, effettua un Tiro per Colpire a distanza contro la creatura o l’oggetto, trattando l’acqua sacra come un’arma improvvisata. Se colpisci, e il bersaglio è un immondo o un non morto, subisce 2d6 danni da energia positiva.

\textbf{Antitossina}. Una creatura che beve da questa fiala di liquido ottiene +1d6 sui tiri salvezza contro il veleno per 1 ora. Non conferisce alcun bonus ai non morti e ai costrutti.

\textbf{Ariete Portatile}. Puoi usare un ariete portatile per abbattere le porte. Nel farlo, ottieni un bonus di +4 alle prove di Forza. Un altro personaggio può aiutarti con l’uso dell’ariete, dandoti +1d6 sulla prova.

\textbf{Attrezzatura da Pesca}. Questo kit comprende un’asta di legno, filo di seta, taglierino di legno, ami d’acciaio, peso di piombo, esche di velluto e un retino. 

\textbf{Biglie di Metallo}. Con un’azione, puoi spargere una singola borsa di queste minuscole biglie di metallo per coprire un’area piana quadrata di 3 metri di lato. Una creatura che attraversa l’area coperta deve superare un Tiro Salvezza su Riflessi con DC 12 o cadere prona. Una creatura che attraversa l’area a metà velocità non deve effettuare il Tiro Salvezza.

\textbf{Bilancia da Mercante}. Una bilancia da mercante include un piccolo bilanciere, un piatto, e un assortimento di pesi fino a 1 chilo. Con essa, puoi misurare il peso esatto di piccoli oggetti, come metalli preziosi o merci, per aiutarti a determinarne il valore. 

\begin{center}
	\includegraphics[height=0.5\linewidth]{immagini/stadera.png}
\end{center}

\textbf{Borsa dei Componenti}. Una borsa dei componenti è un piccolo borsello da cinta di cuoio impermeabile munito di compartimenti contenenti tutte le componenti materiali e altri oggetti speciali di cui hai bisogno per lanciare i tuoi incantesimi, eccetto per quelle componenti che hanno un costo specifico o sono materiali non comuni (come indicato nella descrizione dell’incantesimo).

\textbf{Borsello}. Un borsello di tessuto o cuoio può contenere, tra le altre cose, fino a 20 proiettili da fionda o 50 aghi da cerbottana. Un borsello diviso in compartimenti per contenere componenti per incantesimi viene detto borsa dei componenti.

\textbf{Candela}. Per 1 ora, una candela proietta luce in un raggio di 1,5 metri e luce fioca per ulteriori 1,5 metri.

\textbf{Cannocchiale}. Gli oggetti osservati tramite un cannocchiale sono ingranditi al doppio delle loro dimensioni.

\textbf{Carrucola e Paranco}. Una serie di leve collegate da un cavo e un gancio per attaccarsi ad oggetti, carrucola e paranco ti permettono di tirare su fino a quattro volte il
peso che puoi normalmente sollevare.

\textbf{Catena}. Una catena ha 15 punti ferita e durezza 6. Può essere spezzata superando una prova di Forza con DC 24.

\textbf{Corda}. Una corda, che sia fatta di canapa o seta, ha 2 punti ferita e può essere spezzata superando una prova di Forza con DC 19.

\textbf{Faretra}. Una faretra può contenere fino a 12 frecce. 

	
\begin{center}
	\includegraphics[width=0.5\linewidth]{immagini/forziere.jpg}
\end{center}


\textbf{Fuoco dell’Alchimista}. Questo fluido appiccicoso si incendia quando entra a contatto con l’aria. Con due azioni, puoi lanciare questa ampolla fino a 6 metri, fracassandola all’impatto. Effettua un Tiro per Colpire a distanza contro la creatura o l’oggetto, trattando il fuoco dell’alchimista come un’arma improvvisata. Se colpisci, il bersaglio subisce 1d6 danni da fuoco all’inizio di ciascun suo turno. Una creatura può porre fine a questi danni spendendo due Aziono e superando una prova di Destrezza con DC 12. Se la prova riesce le fiamme si estinguono.

\textbf{Kit da Guaritore}. Questo kit è una borsa di cuoio contenente bende, unguenti e stecche. Il kit può essere usato dieci volte. Concede un +2 alle prove di pronto soccorso.

\textbf{Kit da Pranzo}. Questa piccola scatola di latta contiene una ciotola e delle semplici posate. Le due parti della scatola possono essere staccate, e un lato impiegato come pentola per cucinare e l’altro come piatto o contenitore

\textbf{Kit da Scalatore}. Un kit da scalatore comprende chiodi speciali, punte per stivali, guanti e un’imbracatura. Puoi ancorarti usando il kit da scalatore con un’azione; quando lo fai, non puoi cadere per più di 7 metri dal punto in cui ti sei ancorato, e non puoi arrampicarti a più di 7 metri di distanza dal punto a cui ti sei ancorato senza prima disfare l’ancora.

\textbf{Lanterna}. Una lampada proietta luce intensa in un raggio di 4 metri e luce fioca per ulteriori 9 metri. Una volta accesa, brucia per 6 ore con un’ampolla (0,5 litri) d’olio.

\begin{center}
	\includegraphics[width=0.6\linewidth]{immagini/lanterna.png}
\end{center}

\textbf{Lanterna a Lente Sporgente}. Una lanterna a lente sporgente proietta luce in un cono di 18 metri e luce fioca per ulteriori 18 metri. Una volta accesa, brucia per 6 ore con un’ampolla (0,5 litri) d’olio.

\textbf{Lanterna Schermabile}. Una lanterna schermabile proietta luce in un raggio di 9 metri e luce fioca per ulteriori 9 metri. Una volta accesa, brucia per 6 ore con un’ampolla (0,5 litri) d’olio. Con un’azione, puoi abbassare la schermatura, riducendo la luce a fioca con un raggio di 1 metri.

\textbf{Lente Ingranditrice}. Questa lente permette di dare un’occhiata più ravvicinata agli oggetti piccoli. È anche un utile sostituto per pietra focaia e acciarino nell’accendere un fuoco. Appiccare un fuoco con la lente ingranditrice richiede almeno una luce di intensità pari a quella solare, legna da accendere, e circa 5 minuti di tempo perché il legno prenda fuoco. Una lente ingranditrice fornisce aiuto (+1d6) in qualsiasi prova effettuata per valutare o analizzare un oggetto piccolo o molto dettagliato.

\textbf{Libro}. Un libro può contenere poesie, resoconti storici, informazioni pertinenti a uno specifico campo del sapere, diagrammi e note su congegni gnomeschi, o qualsiasi altra cosa possa essere rappresentata usando testo e disegni. Un libro contenente incantesimi viene detto Tomo della Magia.

\textbf{Tomo degli Incantesimi}. Essenziale per i maghi, un libro degli incantesimi è un tomo rilegato in pelle con 100 pagine di vello bianche adatte a registrare gli incantesimi.

\textbf{Manette}. Questi strumenti di metallo possono imprigionare una creatura Piccola o Media. Per liberarsi dalle manette bisogna superare una prova di Destrezza con DC 24. Per romperle bisogna superare una prova di Forza con DC 24. Ogni set di manette è fornito di una chiave. Senza la chiave, una creatura può usare Artista della Fuga per aprire la serratura superando una prova di Destrezza con DC 18.
Le manette hanno 15 punti ferita e Durezza 2

\textbf{Olio}. Di solito si compra in un’ampolla d’argilla che contiene 0,5 litri. Con un’azione, puoi spargere l’olio in questa ampolla su di una creatura entro 1 metri da te o lanciarla fino a 6 metri, fracassandola all’impatto. In entrambi i casi, effettua un Tiro per Colpire a distanza contro la creatura o l’oggetto, trattando l’olio come un’arma improvvisata. Se colpisci, il bersaglio è ricoperto d’olio. Se il bersaglio subisse qualsiasi entità di danno da fuoco prima che l’olio si asciughi (dopo 1 minuto), il bersaglio subisce altri 5 danni da fuoco dall’olio infiammato. Puoi anche versare un’ampolla d’olio sul pavimento per coprire un quadrato di 1 metro di lato, purché la superficie sia piana. Se infiammato, l’olio brucia per 2 round e infligge 5 danni da fuoco a qualsiasi creatura che entri nell’area o termini il suo turno dentro di essa. Una creatura può subire questo danno solo una volta per turno.

\textbf{Piede di Porco}. Utilizzare un piede di porco dà +1d6 alle prove di Forza ogni volta si possa applicare la leva del piede di porco. 

\textbf{Razioni}. Le razioni consistono di cibo secco adatto a lunghi viaggi, e includono carne secca, frutta secca, gallette e noci.

\textbf{Scatola con l’Esca}. Questo piccolo contenitore contiene pietra, acciarino ed esca (di solito uno straccio secco imbevuto d’olio) impiegati per appiccare un fuoco. Utilizzarlo per accendere una torcia (o qualsiasi altro oggetto facilmente incendiabile) richiede due azioni. Accendere qualsiasi altro fuoco richiede 1 minuto.

\textbf{Scatola per Mappe o Pergamene}. Questa scatola cilindrica di cuoio può contenere, arrotolati, fino a dieci pezzi di carta o cinque fogli di pergamena.

\textbf{Scatola per Quadrelli da Balestra}. Questa scatola di legno contiene fino a 12 quadrelli per balestra.

\textbf{Serratura}. Insieme alla serratura viene fornita una chiave. Senza la chiave, una creatura può scassinare questa serratura superando una prova di Disattivare Congegni con DC 17. Il Narratore può decidere che per cifre maggiori sono disponibili serrature di qualità migliore.

\begin{center}
	\includegraphics[width=0.8\linewidth]{immagini/serratura.jpg}
\end{center}

\textbf{Simbolo Sacro}. Un simbolo sacro è la raffigurazione di un Patrono. Potrebbe essere un amuleto che raffigura il simbolo di un Patrono, lo stesso simbolo accuratamente inciso o intrecciato su di un emblema o scudo, o una minuscola scatola contenente una reliquia sacra.

\textbf{Tenda}. Un semplice riparo portabile di tela, una tenda può contenere due persone. 

\textbf{Torcia}. Una torcia brucia per 1 ora, fornendo luce in un raggio di 6 metri e luce fioca per ulteriori 6 metri. Se effettui un Tiro per Colpire con una torcia accesa e colpisci, infliggi 1 danno da fuoco aggiuntivo.

\textbf{Trappola da Caccia}. Usi due azioni per disporre questa trappola, formata da un anello d’acciaio seghettato, che scatta quando una creatura calpesta la piastra metallica al centro di essa. La trappola è fissata tramite una catena pesante a un oggetto immobile, come un albero o uno spuntone conficcato nel terreno. Una creatura che calpesti la piastra deve superare un Tiro Salvezza su Riflessi con DC 15 o subire 1d4 danni perforanti e interrompere il movimento. Dopodiché, finché la creatura non si libera dalla trappola, il suo movimento è limitato dalla lunghezza della catena (lunga di solito 90 centimetri). Una creatura può usare 2 azioni per superare una prova di Forza con DC 15, e se la riesce si libera o libera un’altra creatura a portata. Ogni tentativo fallito infligge 1 danno perforante alla creatura intrappolata.

\textbf{Tribolo}. Con un’azione, puoi spargere una singola borsa di questi minuscoli triboli per coprire un’area quadrata di 1 metri di lato. Una creatura che attraversa l’area coperta deve superare un Tiro Salvezza su Riflessi con DC 17 o subire 1 danno perforante. Finché la creatura non recupera almeno 1 punto ferita, la sua velocità a piedi è diminuita di 3 metri. Una creatura che attraversa l’area a metà velocità non deve effettuare il Tiro Salvezza.

\begin{center}
	\includegraphics[width=0.6\linewidth]{immagini/tribolo.png}
\end{center}

\textbf{Veleno Base}. Puoi usare il veleno in questa fiala per coprire un’arma tagliente o perforante o fino a tre pezzi di munizioni. Applicare il veleno necessita un’azione. Una creatura colpita da un’arma o munizione avvelenata deve superare un Tiro Salvezza su tempra con DC 12 o subire 1d4 danni da veleno.
Una volta applicato, il veleno mantiene la sua efficacia per 1 minuto prima di seccarsi.

\subsubsection{Dotazioni}
L'equipaggiamento di partenza che un personaggio riceve al primo livello dipende dalla professione e include anche un insieme di oggetti da avventuriero. Il contenuto di ogni dotazione è elencato di seguito. Se il personaggio sceglie di acquistare il suo equipaggiamento di partenza, può acquistare una dotazione al prezzo indicato, che generalmente è più conveniente rispetto all'acquisto dei singoli oggetti separati.

\textbf{Dotazione da Avventuriero (12 mo)}. Include uno zaino, un piede di porco, un martello, 10 chiodi da rocciatore, 10 torce, un acciarino e pietra focaia, 10 razioni giornaliere e un otre. La dotazione include anche 15 metri di corda di canapa legata allo zaino.

\begin{center}
	\includegraphics[width=0.7\linewidth]{immagini/zaino.png}
\end{center}

\textbf{Dotazione da Diplomatico (39 mo)}. Include un forziere, 2 custodie per mappe e pergamene, un abito pregiato, una boccetta di inchiostro, un pennino, una lampada, 2 ampolle di olio, 5 fogli di carta, una fiala di profumo, cera per sigillo e sapone.

\textbf{Dotazione da Esploratore (10 mo)}. Include uno zaino, un giaciglio, una gavetta, un acciarino e pietra focaia, 10 torce, 10 razioni giornaliere e un otre. La dotazione include anche 15 metri di corda di canapa legata allo zaino.

\textbf{Dotazione da Intrattenitore (40 mo}). Include uno zaino, un giaciglio, 2 costumi, 5 candele, 5 razioni giornaliere, un otre e trucchi per il camuffamento.

\textbf{Dotazione da Sacerdote (19 mo)}. Include uno zaino, una coperta, 10 candele, un acciarino e pietra focaia, una cassetta per le offerte, 2 cubetti di incenso, un incensiere, una veste, 2 razioni giornaliere e un otre.


\textbf{Dotazione da Scassinatore (16 mo)}. Include uno zaino, un sacchetto con 1.000 sfere metalliche, 3 metri di spago, una campanella, 5 candele, un piede di porco, un martello, 10 chiodi da rocciatore, una lanterna schermabile, 2 ampolle di olio, 5 razioni giornaliere, un acciarino e pietra focaia e un otre. La dotazione include anche 15 metri di corda di canapa legata allo zaino.

\textbf{Dotazione da Studioso (40 mo)}. Include uno zaino, un libro di studio, una boccetta d'inchiostro, un pennino, 10 fogli di pergamena, un sacchetto di sabbia e un coltellino.


\end{multicols}

\subsubsection{Capienza dei Contenitori}

\begin{tabularx}{0.9\textwidth}{lX}
	\textbf{Contenitore}&\textbf{Capienza}\\
	\toprule
	Ampolla o Boccale&0,5 litri\\
	Barile&160 litri liquidi, 4 cubi con spigolo di 30 cm\\
	Borsa&1 cubo con spigolo di 10 cm/3 kg di equipaggiamento\\
	Bottiglia&1 litro di liquido\\
	Brocca o Caraffa&4 litri liquidi\\
	Canestro&2 cubi con spigolo di 30 cm/20 kg di equipaggiamento\\
	Fiala&120 ml di liquidi\\
	Forziere&12 cubi con spigolo di 30 cm/150 kg di equipaggiamento\\
	Otre&2 litri liquidi\\
	Sacco&1 cubo con spigolo di 30 cm/15 kg di equipaggiamento\\
	Secchio&12 litri liquidi, 1 cubo con spigolo di 25 cm\\
	Vaso di Ferro&4 litri liquidi\\
	Zaino*&1 cubo con spigolo di 30 cm/15 kg di equipaggiamento\\
\end{tabularx}

\medskip

\begin{multicols}{2}
	
\subsubsection{Strumenti}
Uno strumento consente a un personaggio di esercitare una Professione.

L'utilizzo di uno strumento non è collegato a una singola caratteristica, dal momento che la competenza in uno strumento rappresenta una conoscenza più vasta dal suo utilizzo basilare. Per esempio, il Narratore potrebbe chiedere a un personaggio di effettuare una prova di Destrezza per incidere qualche minuscolo dettaglio con i suoi strumenti da intagliatore, o una prova di Forza per ricavare un oggetto da un pezzo di legno particolarmente duro.

Se la prova riguarda la Professione del personaggio questa viene effettuata con un bonus pari a metà del livello del personaggio (3d6+Saggezza+1/2LV).

\end{multicols}

\medskip

\begin{tabularx}{0.9\textwidth}{llX|llX}
	\textbf{Oggetto}&\textbf{Costo}&\textbf{Peso}&	Arnesi da Falsario&15 mo&2,5 kg\\
		\toprule
	Arnesi da Scasso&25 mo&0,5 kg&	Borsa da Erborista&5 mo&1,5 kg\\
	Dadi&1 ma&-&	Mazzo di Carte&5 ma&-\\
	Scacchi dei Draghi&1 mo&0,25 kg&	Tre Draghi al Buio&1 mo&\\
	Sostanze da Avvelenatore&50 mo&1 kg&	Scorte da Alchimista&50 mo&4 kg\\
	Scorte da Calligrafo&10 mo&2,5 kg&	Scorte da Mescitore&20 mo&4,5 kg\\
	Strumenti da Calzolaio&5 mo&2,5 kg&	Strumenti da Cartografo&15 mo&3 kg\\
	Strumenti da Conciatore&5 mo&2,5 kg&	Strumenti da Costruttore&10 mo&4 kg\\
	Strumenti da Fabbro&20 mo&4 kg&	Strumenti da Falegname&8 mo&3 kg\\
	Strumenti da Gioielliere&25 mo&1 kg&	Strumenti da Intagliatore&1 mo&2,5 kg\\
	Strumenti da Inventore&50 mo&5 kg&	Strumenti da Pittore&10 mo&2,5 kg \\
	Strumenti da Soffiatore&30 mo&2,5 kg&	Strumenti da Tessitore&1 mo&2,5 kg\\
	Strumenti da Vasaio&10 mo&1,5 kg&	Utensili da Cuoco&1 mo&4 kg \\
	Strumenti da Navigatore&25 mo&1 kg&	Ciaramella&2 mo&0,5 kg\\
	Cornamusa&30 mo&3 kg&	Corno&3 mo&1 kg\\
	Dulcimer&25 mo&5 kg&	Flauto&2 mo&0,5 kg\\
	Flauto di Pan&12 mo&1 kg&	Lira&30 mo&1 kg\\
	Liuto&35 mo&1 kg&	Tamburo&6 mo&1,5 kg\\
	Viola&30 mo&0,5 kg&	Trucchi per il Camuffamento&25 mo&1,5 kg\\
\end{tabularx}

\begin{multicols}{2}

\medskip

\subsubsection{Cavalcature e Veicoli}

Una buona cavalcatura può consentire a un personaggio di attraversare rapidamente un territorio selvaggio, ma il suo scopo primario è trasportare l'equipaggiamento che altrimenti rallenterebbe il suo padrone. 

La tabella "Cavalcature e Altri Animali" indica la velocità e la capacità di trasporto base di ogni animale. Un animale che tira una biga, un carretto, un carro, una carrozza o una slitta può spostare un peso pari a cinque volte la sua capacità di trasporto, incluso il peso del veicolo. Se più animali tirano lo stesso veicolo, possono sommare assieme la loro capacità di trasporto. 

Nei mondi di fantasy esistono altre cavalcature oltre a quelle elencate in questa sezione, ma si tratta di cavalcature rare che normalmente non sono disponibili per l'acquisto, come certe cavalcature volanti (pegasi, grifoni, ippogrifi e altri animali simili) o perfino alcune cavalcature acquatiche (come per esempio i cavallucci marini giganti).

Per entrare in possesso di una cavalcatura del genere spesso è necessario rubare un uovo e crescere la creatura di persona, stipulare un patto con una potente entità o negoziare con la cavalcatura stessa.

\textbf{Bardatura}. Una bardatura è un'armatura concepita per proteggere la testa, il collo, il petto e il corpo di un animale. Ogni tipo di armatura elencato nella tabella "Armature" di questo capitolo può essere acquistata come bardatura. Il costo è pari al quadruplo dell'armatura equivalente fabbricata per gli umanoidi, mentre il peso è pari al doppio.

\begin{center}
	\includegraphics[width=0.7\linewidth]{immagini/bardatura.png}
	
\textit{Bardatura completa}
\end{center}


\textbf{Sella}. Un cavalcatore può agganciarsi a una sella militare per rimanere al suo posto su una cavalcatura attiva, nel corso di una battaglia. Una sella militare conferisce vantaggio alle prove che il personaggio effettua per rimanere in sella. E' necessaria una sella esotica per cavalcare una creatura acquatica o volante.


\begin{center}
	\includegraphics[height=0.7\linewidth]{immagini/sella2.png}
\end{center}


\textbf{Imbarcazioni a Remi}. I barconi e le barche a remi sono solitamente usati sui laghi e sui fiumi. Se un'imbarcazione segue la corrente, si aggiunge la velocità della corrente (solitamente 4,5 km all'ora) alla sua velocità. In genere non è possibile remare controcorrente se la corrente ha un'intensità rilevante, ma è possibile far risalire un corso d'acqua a queste imbarcazioni portandole a riva e facendole trainare da una o più bestie da soma. Una barca a remi pesa 50 kg qualora gli avventurieri debbano trasportarla via terra.

\subsubsection{Cavalcature e Altri Animali}


\begin{tabular}{llll}
		\toprule
	\textbf{Cavalcatura}&\textbf{Costo}&\textbf{Mov.}&\textbf{Trasporto}\\
	&(\textbf{mo})&&\\
	Asino o Mulo&8&12 m&210 kg\\
	Cammello&50&15 m&240 kg\\
	Cavallo da Galoppo&75&18 m&240 kg\\
	Cavallo da Guerra&400&18 m&270 kg\\
	Cavallo da Tiro&50&12 m&270 kg\\
	Elefante&200&12 m&660 kg\\
	Mastino&25&12 m&97,5 kg\\
	Pony&30&12 m&112,5 kg\\
\end{tabular}


\bigskip

\textbf{Finimenti e Veicoli da Tiro}\\
\begin{tabularx}{0.45\textwidth}{llX}
		\toprule
	\textbf{Oggetto}&\textbf{Costo}&\textbf{Peso}\\
	Bardatura&x4&x2\\
	Biga&250 mo&50 kg\\
	Bisacce&4 mo&4 kg\\
	Carretto&15 mo&100 kg\\
	Carro&35 mo&200 kg\\
	Carrozza&100 mo&300 kg\\
	Morso e Briglie&2 mo&0,5 kg\\
	Nutrimento (al giorno)&5 mr&5 kg\\
\end{tabularx}

\bigskip

\textbf{Sella}\\
\begin{tabularx}{0.45\textwidth}{llX}
		\toprule
\textbf{Oggetto}&\textbf{Costo}&\textbf{Peso}\\
	Da Carico&5 mo&7,5 kg\\
	Da Galoppo&10 mo&12,5 kg\\
	Esotica&60 mo&20 kg\\
	Militare&20 mo&15 kg\\
	Slitta&20 mo&150 kg\\
	Stallaggio (al giorno)&5 ma&\\
\end{tabularx}

\bigskip

\textbf{Imbarcazioni}\\
\begin{tabularx}{0.45\textwidth}{llX}
	\toprule
	\textbf{Oggetto}&\textbf{Costo}&\textbf{Velocità}\\
	Barca a Remi&50 mo&2,25 km orari\\
	Barcone&3.000 mo&1,5 km orari\\
	Galea&30.000 mo&6 km orari\\
	Nave a Vela&10.000 mo&3 km orari\\
	Nave da Guerra&25.000 mo&3,75 km orari\\
	Nave Lunga&10.000 mo&4,5 km orari\\
\end{tabularx}


\subsubsection{Spese}
Quando non si calano nelle viscere della terra, non esplorano rovine in cerca di tesori perduti o non muovono guerra alle forze dell'oscurità incombente, anche gli avventurieri devono pensare ai bisogni più comuni. Anche in un modo fantastico, la gente deve soddisfare bisogni basilari come un vitto, alloggio e vestiario. Tutto questo ha un costo, anche se certi stili di vita costano più di altri.

\textbf{Spese dello Stile di Vita}


Le spese dello stile di vita sono un modo semplice per tenere conto dei costi della vita in un mondo fantasy. Coprono l'alloggio, il cibo, le bevande e tutte le altre necessità essenziali di un personaggio. Queste spese coprono inoltre il costo di manutenzione dell'equipaggiamento del personaggio, per consentirgli di essere pronto quando giungerà la prossima chiamata all'avventura. All'inizio di ogni settimana o mese (a scelta del giocatore), ogni personaggio sceglie uno stile di vita dalla tabella "Spese dello Stile di Vita" e paga il prezzo richiesto per mantenere quello stile di vita. I prezzi elencati sono giornalieri, quindi chi desidera calcolare il suo costo di sostentamento per un periodo di trenta giorni dovrà moltiplicare il prezzo indicato per 30. Un personaggio può cambiare il suo stile di vita da un periodo all'altro, in base ai fondi a sua disposizione, oppure può mantenere lo stesso stile di vita nel corso di tutta la propria carriera. 

La scelta dello stile di vita può avere delle conseguenze. Un personaggio che mantiene uno stile di vita ricco può stringere più facilmente contatti con i ricchi e i potenti, ma corre il rischio di attirare qualche ladro. Analogamente, uno stile di vita povero può aiutarlo a evitare i criminali, ma difficilmente gli consentirà di stringere contatti importanti.

\bigskip

\textbf{Spese dello Stile di Vita}

\medskip

\begin{tabular}{ll}
	Stile di Vita&Prezzo al Giorno\\
			\toprule
	Miserabile&-\\
	Squallido&1 ma\\
	Povero&2 ma\\
	Modesto&1 mo\\
	Agiato&2 mo\\
	Ricco&4 mo\\
	Aristocratico&Minimo di 10 mo\\
\end{tabular}

\bigskip

\textbf{Miserabile}. Il personaggio vive in condizioni disumane. Non ha un luogo che può chiamare casa e si ripara dove può, intrufolandosi in un granaio, rannicchiandosi in una vecchia cassa o affidandosi al buon cuore di chi è più fortunato di lui. Uno stile di vita miserabile presenta pericoli in abbondanza. La violenza, le malattie e la fame seguono il personaggio ovunque egli si rechi. Gli altri miserabili potrebbero mettere gli occhi sulla sua armatura, le sue armi e la sua attrezzatura da avventuriero, che rappresentano una fortuna per i loro parametri. La maggior parte della gente non prende minimamente in considerazione il personaggio.

\textbf{Squallido}. Il personaggio vive in una stalla piena di spifferi, una capanna dal pavimento di fango situata appena fuori dal paese o in un ostello pieno di pulci nel quartiere peggiore della città. Beneficia di un minimo riparo dagli elementi, ma vive in un ambiente disperato e spesso violento, in luoghi afflitti dalle malattie, dalla fame e dalle disgrazie. La maggior parte della gente non lo prende minimamente in considerazione e la legge lo protegge poco o nulla. La maggior parte delle persone che conduce questo stile di vita è segnata da qualche terribile disgrazia: bollati come esuli, affetti da un disturbo mentale o da una malattia di qualche tipo.

\textbf{Povero}. Uno stile di vita povero significa dover tirare avanti senza le comodità disponibili in una comunità stabile. Vitto e alloggio essenziali, abiti di scarsa qualità e condizioni di vita imprevedibili generano come risultato uno stile di vita forse sufficiente a sopravvivere, ma di sicuro poco piacevole. Il personaggio dorme in un ostello o in una sala comune al primo piano di una taverna. Beneficia di un minimo di protezione legale, ma deve comunque vedersela con atti di violenza, crimini e malattie. I manovali privi di una specializzazione, robivechi, mendicanti, ladri, mercenari e altre figure poco raccomandabili tendono ad adottare questo stile di vita.

\begin{center}
	\includegraphics[width=0.7\linewidth]{immagini/mendicante.png}
	
	\textit{Mendicante - Francesco Londonio}
\end{center}

\textbf{Modesto}. Uno stile di vita modesto tiene un personaggio fuori dai bassifondi e gli consente di prendersi cura del suo equipaggiamento. Il personaggio vive in una parte vecchia della città, ha una camera in affitto in una pensione, una locanda o un tempio. Non patisce la fame o la sete e vive in un ambiente pulito, anche se spartano. Gli individui comuni che conducono uno stile di vita modesto includono soldati con una famiglia, manovali, studenti, sacerdoti, maghi dilettanti e così via.

\begin{center}
	\includegraphics[width=0.9\linewidth]{immagini/mercante.png}
\end{center}


\textbf{Agiato}. Un personaggio in grado di adottare uno stile di vita agiato può permettersi abiti di qualità e prendersi cura del proprio equipaggiamento senza difficoltà. Vive in un'abitazione in un isolato di buona fama o dispone di una stanza privata presso una locanda di qualità. Frequenta mercanti, abili artigiani e ufficiali militari.

\textbf{Ricco}. Un personaggio che adotta uno stile di vita ricco vive nel lusso, anche se forse non ha raggiunto il prestigio sociale associato ai vecchi valori della nobiltà e del sangue reale. Conduce uno stile di vita paragonabile a quello di un mercante di grande successo, uno stimato servitore di un casato reale o un proprietario di alcune piccole attività commerciali. Alloggia in una dimora rispettabile, solitamente una casa spaziosa in una parte rispettabile della città o un comodo appartamento presso una locanda rinomata. Probabilmente è assistito da un piccolo gruppo di servitori.


\begin{center}
	\includegraphics[width=0.7\linewidth]{immagini/lucullo.png}
	
	\textit{Lucio Licinio Lucullo. Roma, 117 aC, Napoli 56 aC). Militare e politico romano}
\end{center}


\textbf{Aristocratico}. Il personaggio vive comodamente e nell'abbondanza e frequenta gli ambienti popolati dalle figure più potenti della comunità. Dispone di una dimora eccellente, forse una casa nel quartiere più elegante della città o forse una serie di camere nella locanda più rinomata. Pranza ai ristoranti migliori, si serve presso i sarti più abili e alla moda e può contare su vari servitori che si occupano di ogni suo bisogno. Riceve inviti agli eventi di società dei ricchi e dei potenti e trascorre le sue serate in compagnia di politici, capi di gilda, sommi sacerdoti e nobili. Deve anche vedersela con gli inganni e i tradimenti perpetrati ai livelli più alti. Più grande è la sua ricchezza, maggiori sono le possibilità che sia trascinato in qualche intrigo politico, a volte come pedina, a volte come partecipante attivo.


\subsubsection{Vitto e Alloggio}


La tabella "Vitto e Alloggio" indica i prezzi delle singole pietanze e di un singolo pernottamento. Questi prezzi sono inclusi nelle spese di stile di vita totale di un personaggio.

\textbf{Vitto e Alloggio}

\bigskip

\begin{tabular}{ll}
	\textbf{Oggetto}&\textbf{Costo}\\
	\toprule
	\textbf{Birra}&\\
	Boccale&4 mr\\
	Caraffa (4 litri)&2 ma\\
	\textbf{Pietanze} &\\
	Banchetto (a persona)&10 mo\\
	Carne, 1 pezzo&3 ma\\
	Formaggio, 1 pezzo&1 ma\\
	Pane (a pagnotta)&2 mr\\
	\textbf{Locanda (al giorno})&\\
	Squallida&7 mr\\
	Povera&1 ma\\
	Modesta&5 ma\\
	Agiata&8 ma\\
	Ricca&2 mo\\
	Aristocratica&4 mo\\
	\textbf{Pasto (al giorno)}&\\
	Squallido&3 mr\\
	Povero&6 mr\\
	Modesto&3 ma\\
	Agiato&5 ma\\
	Ricco&8 ma\\
	Aristocratico&2 mo\\
	\textbf{Vino}&\\
	Buono (bottiglia)&10 mo\\
	Comune (caraffa)&2 ma\\
\end{tabular}

\begin{center}
	\includegraphics[width=0.9\linewidth]{immagini/riempitivocavalieriapranzo.png}
\end{center}


\subsubsection{Servizi}


Gli avventurieri possono pagare i personaggi non giocanti affinché li aiutino o agiscano in loro vece nelle circostanze più disparate. La maggior parte di questi gregari è dotata di abilità pressoché ordinarie, mentre altri hanno padroneggiato un'arte o un mestiere e alcuni si sono specializzati in qualche abilità da avventuriero. 

Altri gregari comuni includono i numerosi abitanti di un tipico paese o città che gli avventurieri possono ingaggiare per svolgere un compito specifico. Per esempio, un mago potrebbe pagare un falegname per farsi costruire un pregiato scrigno (e la sua replica in miniatura) da usare per un incantesimo.
Un guerriero potrebbe commissionare a un fabbro la forgiatura di una spada speciale. 

\medskip

\textbf{Servizi}

\bigskip

\begin{tabular}{ll}
	Servizio&Costo\\
	\toprule
	Carrozza&\\
	All'interno di una città&1 mr\\
	Tra due paesi&2 mr per 1 km\\
	Gregario&\\
	Abile&2 mo al giorno\\
	Inesperto&2 ma al giorno\\
	Messaggero&2 mr per 1,5 km\\
	Passaggio in nave&1 ma per 1,5 km\\
	Pedaggio stradale o di ingresso&1 mr\\
\end{tabular}


\subsubsection{Servizi Magici}

\textbf{Difficoltà x Difficoltà×2 mo}

Questo è il costo per avere un incantatore che manipola la magia. Questo costo presuppone che si possa andare dall’incantatore e chiedergli di manipolare una certa magia a proprio piacimento (solitamente gli servono almeno 8 ore per prepararsi). Se si vuole portare da qualche parte l’incantatore per fargli usare la magia è necessario negoziare con lui, e la risposta di base è "no".

Se l’incantesimo a ha conseguenze pericolose, l’incantatore deve ricevere delle prove certe che il personaggio ha la possibilità di pagare e che non mancherà di farlo nel caso queste conseguenze si verifichino (sempre che accetti di lanciare l’incantesimo richiesto, cosa nient’affatto sicura). Quando si tratta di incantesimi che trasportano il personaggio e l’incantatore lungo una distanza, è necessario pagare l’incantesimo due volte anche se il personaggio non desidera tornare indietro con l’incantatore.

Non tutti i villaggi e i paesi hanno un incantatore abbastanza capace a manipolare la magia. Come regola generale, è necessario spostarsi almeno in un piccolo paese per essere abbastanza sicuri di trovare un incantatore. In un piccolo paese si potrebbe trovare un incantatore in grado di lanciare incantesimi a Difficoltà 19, in un grande paese quelli a Difficoltà 21, una piccola città per quelli a Difficoltà 26, in una grande città per quelli di Difficoltà 29-31, in una metropoli per quelli di Difficoltà 34. Nemmeno in una metropoli si è certi di trovare un incantatore capaci di lanciare magie con Difficoltà 36 o piu’.

\end{multicols}

\bigskip

\begin{center}
	\includegraphics[width=0.775\linewidth]{immagini/carrozza.png}
\end{center}


\pagebreak

\subsection{Materiali Speciali}\index{Materiali Speciali}

\begin{enfasi}{
All'uopo il capitano De Medici ha fatto brunire tutte le armature, per sorprendere il nemico anche col buio. (Il mestiere delle armi, Ermanno Olmi, film 2001)}\end{enfasi}\medskip

\begin{multicols}{2}

Le armature e le armi si possono costruire con materiali che possiedono delle innate qualità speciali. Se si costruisce un'armatura o arma con più di un materiale speciale, si ricevono i benefici solo del materiale prevalente. Si può però costruire un'arma doppia con ogni testa fatta di un materiale speciale diverso.

\subsubsection{Acciaio Vivente}\index{Acciaio Vivente}\index{Tabella Costo Acciaio Vivente}

\label{acciaio-vivente}

\begin{tabularx}{0.45\textwidth}{Xl}
	\textbf{Tipo di oggetto in Acciaio Vivente} & \textbf{Modificatore al costo}\\
	\toprule
 	Munizione                         & +10 mo per munizione\\
	Arma                              & +600 mo\\
	Armatura leggera                  & +600 mo\\
	Armatura media                    & +1000 mo\\
	Armatura pesante                  & +1500 mo\\
	Scudo                             & +100 mo\\
	Altri oggetti                     & 500 mo/kg\\
\end{tabularx}


Alcuni alberi succhiano potenti minerali attraverso le loro radici nello stesso modo in cui altri attingono acqua dal terreno. Anche se questi alberi smussano le seghe e le asce usate per abbatterli e ignorano il fuoco, alla fine soccombono al tempo o agli elementi.

Se correttamente raccolti, questi alberi caduti producono pepite di un metallo chiamato acciaio vivente.


Le armature e gli scudi di acciaio vivente possono danneggiare le armi di metallo che li colpiscono. Ogni volta che chi impugna un'arma di metallo ottiene un 3 naturale a un Tiro per Colpire contro una creatura che indossa un'armatura di acciaio vivente o brandisce uno scudo di questo metallo, l'oggetto deve superare un Tiro Salvezza su Tempra con DC 20 od ottiene la condizione Rotto. L'acciaio vivente non può danneggiare in questo modo le armi di Adamantio.

L'acciaio vivente ha 35 Punti Ferita per 2,5 cm di spessore e Durezza 15. 

\subsubsection{Adamantio}\index{Adamantio}\index{Tabella Costo Adamantio}

\label{adamantio}

\begin{tabularx}{0.45\textwidth}{Xl}
	\textbf{Tipo di oggetto in Adamantio} & \textbf{Modificatore al costo}\\
	\toprule
	Munizione                   & +60 mo per munizione\\
	Arma                        & +1500 mo\\
	Armatura leggera            & +5000 mo\\
	Armatura media              & +10000 mo\\
	Armatura pesante            & +15000 mo\\
	Scudo                       & +1000 mo\\
	Altri oggetti               & 5000 mo/kg\\
\end{tabularx}

Questo metallo durissimo si trova solo nei meteoriti e contribuisce alla qualità di un'arma o di un'armatura.

Quindi le armi e le munizioni in adamantio hanno Bonus di +1 ai Tiri per Colpire, e la penalità date dall'armatura (Destrezza e CM) viene diminuita di 1 rispetto ad una normale armatura del suo stesso tipo. Gli oggetti senza parti metalliche non possono essere costruiti con l'adamantio. Una freccia può essere in adamantio, ma un bastone ferrato no.

Armi e armature fatte normalmente d'acciaio e costruite con l'adamantio hanno un terzo dei Punti Ferita in più del normale. L'adamantio ha 40 Punti Ferita per 2,5 cm di spessore e Durezza 20. 

\subsubsection{Argento Alchemico}\index{Argento Alchemico}\index{Tabella Costo Armi Argentate}

\label{argento-alchemico}

\begin{tabularx}{0.45\textwidth}{Xl}
	\textbf{Tipo di Oggetto in Argento Alchemico} & \textbf{Modificatore al costo}\\
	\toprule
	Munizione                      & +2 mo per munizione\\
	Arma leggera                   & +20 mo\\
	Arma media                     & +90 mo\\
	Arma pesante                   & +180 mo\\
	Scudo                         & +100 mo\\
\end{tabularx}

Il processo di argentatura alchemica può essere applicato solo alle armi metalliche e non funziona sui metalli speciali come ad esempio l'adamantio, il ferro freddo ed il mithral.

Un complesso processo che coinvolge la metallurgia e l'alchimia può legare l'argento a un'arma fatta d'acciaio in modo che oltrepassi la Riduzione del Danno di creature come i Licantropi.

L'argento alchemico ha 10 Punti Ferita per ogni 2,5 cm di spessore e Durezza 8. 

\subsubsection{Ferro Freddo}\index{Ferro Freddo}

\label{ferro-freddo}

Questo ferro viene estratto nelle profondità del sottosuolo ed è noto per la sua efficacia contro demoni e folletti. Viene forgiato ad una temperatura inferiore per conservare le sue delicate proprietà. Costruire armi fatte di ferro freddo costa il doppio rispetto alle loro normali controparti. Inoltre qualsiasi potenziamento magico costa 2.000 mo addizionali. Questo aumento viene applicato la prima volta che l'oggetto viene potenziato, non una volta per qualità aggiunta.

Gli oggetti senza parti di metallo non possono essere costruiti in ferro freddo. Una freccia potrebbe essere fatta di ferro freddo ma un randello no (tranne se tutto di metallo). Un'arma doppia che è fatta solo per metà di ferro freddo aumenta il suo costo del 50\%.

Il ferro freddo ha 30 Punti Ferita per 2,5 cm di spessore e Durezza 10. 


\subsubsection{Mithral}\index{Mithral}\index{Tabella Costo Armi Mithral}

\label{mithral}

\begin{tabularx}{0.45\textwidth}{Xl}
	\textbf{Tipo di Oggetto in Mithral} & \textbf{Modificatore al costo}\\
	\toprule
	Armatura leggera                    & +1000 mo\\
	Armatura media                      & +4000 mo\\
	Armatura pesante                    & +9000 mo\\
	Scudo                               & +1000 mo\\
	Altri oggetti                       & +1000 mo/kg\\
\end{tabularx}

\bigskip

\begin{center}
	\includegraphics[width=0.9\linewidth]{immagini/mithral.png}
\end{center}


Il mithral è un metallo molto raro, luccicante, simile all'argento, più leggero del ferro ma altrettanto duro. Quando viene lavorato come l'acciaio, diventa un meraviglioso materiale con cui creare armature, e occasionalmente viene usato anche per altri oggetti. La maggior parte delle armature in mithral è più leggera di una categoria del normale, ed è più agevole per il movimento e le altre limitazioni. Le armature pesanti sono trattate come armature medie, e le armature medie sono trattate come leggere, ma le armature leggere restano leggere.

Questa diminuzione non si applica alla competenza necessaria per indossare l'armatura in questione (per indossare un'armatura pesante di mithral occorre avere Competenza Armi 3, anche se questa viene considerata come media per altri fattori). Occorre essere competenti nel tipo di armatura appropriato, altrimenti si incorre nelle relative penalità come di norma.

Le probabilità di fallimento di un incantesimo per armature e scudi in mithral diminuiscono di 3 punti e la penalità alla Destrezza diminuiscono di 2 (fino a un minimo di 0), le penalità al movimento diminuiscono di 1 metro.

Il mithral ha 30 Punti Ferita per ogni 2,5 cm di spessore e Durezza 15. 

\subsubsection{Pelle di Drago}\index{Pelle di Drago}

\label{pelle-di-drago}

I fabbricanti di armature possono lavorare le pelli dei draghi per produrre armature o scudi.
Un drago fornisce pelle sufficiente per una singola armatura di pelle per una creatura di una taglia più piccola del drago. Selezionando solo le scaglie e le parti di pelle migliori, un fabbricante di armature può produrre una corazza di bande per una creatura di due taglie più piccola, una mezza armatura per una creatura tre taglie più piccola e una corazza di piastre o un'armatura completa per una creatura di quattro taglie più piccola.

\begin{center}
	\includegraphics[width=0.9\linewidth]{immagini/dragonhide.png}
\end{center}


In ogni caso, c'è sempre pelle sufficiente per produrre uno scudo leggero o pesante in aggiunta all'armatura, purché il drago sia Grande o maggiore.
Se la pelle di drago proviene da un Drago che ha immunità ad un tipo di energia, anche l'armatura è immune a quel tipo di energia, sebbene non conferisca alcuna protezione a chi la indossa. Se allo scudo o all'armatura viene conferita in seguito la capacità di proteggere chi la indossa da un tipo di energia specifico, il costo di questo potenziamento viene ridotto del 25\%.

Le probabilità di fallimento di un incantesimo per armature in pelle di drago va a zero mentre la penalità alla Destrezza diminuiscono di 1 (fino a un minimo di 0), le penalità al movimento diminuiscono di 1 metro.

Le armature di pelle di drago costano il doppio di un'armatura di quel tipo, ma non richiedono più tempo per essere costruite.

La pelle di drago ha 10 Punti Ferita per 2,5 cm di spessore e Durezza 10. Solitamente la pelle di drago è spessa da 1,25 a 2,5 cm. 

\end{multicols}

\pagebreak

\section{Sfondare ed Entrare}\index{Sfondare}\index{Entrare}

\begin{enfasi}{
Nella vita di un uomo prima o poi arriva un giorno in cui, per andare dove deve andare, se non ci sono porte né finestre, gli tocca sfondare la parete. (Bernard Malamud)}\end{enfasi}\medskip

\begin{multicols}{2}

\label{sfondare-ed-entrare}

\lettrine[lines=2, lhang=0.33, loversize=0.25, findent=1.5em]{Q}{ando} si tenta di spaccare un oggetto le scelte sono due: colpirlo con un'oggetto (arma?) o romperlo con la forza bruta.

\medskip

\textbf{Le dimensioni contano...}

A seconda delle dimensioni dell'oggetto questo può essere più o meno facile da colpire.

\medskip

\textbf{Tabella: Taglia e Difesa degli Oggetti - Colpire un Oggetto}\index{Tabella Taglia e Difesa degli Oggetti - Colpire un Oggetto}

\medskip

\begin{tabularx}{0.43\textwidth}{XXX}
	\textbf{Taglia degli Oggetti} & \textbf{Modificatore Difesa} & \textbf{Dimensioni}\\
	\toprule
	Colossale                              & -8 &18m+\\
	Mastodontica                           & -6 &9-18m\\
	Enorme                                 & -4 &4-9m\\
	Grande                                 & -2 &2.4-4m\\
	Media                                  & +0 &1.2-2.4m\\
	Piccola                                & +2 &60-120cm\\
	Minuscola                              & +4 &30-60cm\\
	Minuta                                 & +6 &15-30cm\\
	Piccolissima                           & +8 &5-20cm\\
\end{tabularx}

\bigskip

\textbf{Modificatore Difesa}

Gli oggetti sono più facili da colpire delle creature poiché di solito non si muovono, ma molti sono abbastanza resistenti da scrollarsi di dosso qualche danno ad ogni colpo. La Difesa di un oggetto è pari a 10 + il suo modificatore di Taglia (vedi Tabella: Colpire un Oggetto) + il suo modificatore di Destrezza (caso mai ne avesse uno).

Se si usano 3 Azioni per prendere la mira, si colpisce automaticamente con un'arma da mischia e si ottiene bonus +2d6 al colpire con un'arma a distanza.

\subsection{Durezza}

\textbf{Tabella: Durezza e Punti Ferita oggetti}\index{Tabella Durezza e Punti Ferita oggetti}

\medskip

\begin{tabularx}{0.43\textwidth}{llX}
	\textbf{Sostanza} & \textbf{Durezza} & \textbf{Punti Ferita per spessore} \\
	\toprule
	Vetro             & 1                & 1 ogni 2,5 cm\\
	Carta o stoffa    & 0                & 2 ogni 2,5 cm\\
	Corda             & 0                & 2 ogni 2,5 cm\\
	Ghiaccio          & 0                & 3 ogni 2,5 cm\\
	Cuoio o pelle     & 2                & 5 ogni 2,5 cm\\
	Legno             & 5                & 10 ogni 2,5 cm\\
	Pietra            & 15               & 15 ogni 2,5 cm\\
	Ferro o acciaio   & 10               & 10 ogni 2,5 cm\\
	Mithral           & 15               & 30 ogni 2,5 cm\\
	Adamantio         & 20               & 40 ogni 2,5 cm\\
\end{tabularx}

\subsection{Danneggiare gli oggetti}

\textbf{Attacchi di Energia}: Gli attacchi di energia (fuoco, elettricità..) infliggono metà danno alla maggior parte degli oggetti; dividere per 2 i danni prima di applicare la Durezza. Alcuni tipi di energia possono essere particolarmente efficaci contro certi oggetti, a discrezione del Narratore.

Per esempio, il fuoco potrebbe infliggere danno pieno a pergamene, stoffa e altri oggetti che bruciano facilmente. Un attacco sonoro potrebbe causare danno pieno (massimo valore dei dadi) ad oggetti di vetro e cristallo o ceramiche.

Ricordatevi che un danno da Luce è per metà Energia Positiva \index{Luce}e per metà Fuoco, mentre il Vuoto\index{Vuoto} è per metà Freddo e per metà Energia Negativa.

Energia Negativa o Positiva non danneggiano gli oggetti, solo le creature viventi o meno.



\textbf{Danni da Armi a Distanza}: Gli oggetti subiscono la metà dei danni da un'arma a distanza (tranne che per le Macchine d'Assedio e simili). Dividere per 2 i danni prima di applicare la Durezza dell'oggetto.

\textbf{Armi Inefficaci}: Certe armi semplicemente non possono infliggere danni a certi oggetti. Per esempio, un'arma contundente non è in grado di tagliare una corda.
Allo stesso modo è decisamente difficile abbattere una porta o un muro di pietra con la maggior parte delle armi da mischia, a meno che non siano specificamente ideate per farlo, come picconi e martelli.

\textbf{Immunità}: Gli oggetti inanimati sono immuni ai Danni Non Letali e ai Colpi Critici (ma non all'esplosione del danno). Anche gli oggetti animati, se non considerati come delle creature, hanno queste immunità.

\textbf{Vulnerabilità a Certi Attacchi}: Certi attacchi possono essere particolarmente efficaci contro alcuni oggetti. In questi casi gli attacchi infliggono danni raddoppiati e possono ignorare la Durezza dell'oggetto.

\textbf{Oggetti Danneggiati}: Un oggetto danneggiato rimane pienamente funzionale con la condizione Rotto fino a quando i Punti Ferita non arrivano a 0, e a quel punto è considerato distrutto. Gli oggetti danneggiati (ma non quelli distrutti) possono essere riparati da una Professione Artigiano e alcuni Incantesimi (vedi la condizione Rotto per maggiori dettagli).

\textbf{Tiro Salvezza}: Gli oggetti non magici incustoditi non effettuano mai Tiro Salvezza. Si considera che abbiano fallito i loro Tiro Salvezza, e siano quindi sempre soggetti all'incantesimo ed altri attacchi che ammettono un Tiro Salvezza per resistere o negare l'effetto.

Un oggetto custodito da un personaggio (che lo tenga in mano, lo tocchi o lo indossi) ottiene un Tiro Salvezza proprio come se lo stesse effettuando il personaggio (cioè usando il suo bonus al Tiro Salvezza).

\textbf{Gli Oggetti Magici hanno sempre Tiro Salvezza}. Il bonus ai Tiri Salvezza su Tempra, Riflessi o Volontà di un Oggetti Magico sono pari a 2 + 1/2 della Difficoltà dell’incantesimo piu’ potente che ospitano. Se non l'oggetto non ha un incantesimo si considera un bonus di +4 per ogni +1 di bonus posseduto. Gli Oggetti Magici custoditi (indossati) effettuano i Tiri Salvezza come il loro possessore oppure usano i loro Tiri Salvezza, quali che siano i migliori tranne se un effetto influenza specificatamente l'oggetto magico e non chi lo indossa.

Oggetti animati: Gli oggetti animati contano come creature per determinarne la Difesa (non sono considerati oggetti inanimati).

\medskip

\begin{center}
	\includegraphics[width=0.60\linewidth]{immagini/portarinforzata2.png}
	
	\textit{Porta rinforzata}
\end{center}

\subsection{Rompere Oggetti}\index{Rompere Oggetti}

\label{rompere-oggetti}

Quando si tenta di rompere qualcosa con forza improvvisa piuttosto che infliggendo danni regolari, bisogna effettuare una prova di Forza per capire se ci si riesce.

Poiché la Durezza non influisce sulla DC per rompere l'oggetto, questo valore dipende più dal modo in cui è costruito l'oggetto che non dal materiale. Vedi Tabella: DC per Rompere o forzare Oggetti per una lista delle DC più comuni relative al rompere gli oggetti.

\textbf{Tabella: Durezza, Punti Ferita e DC su Forza per Rompere Oggetti}\index{Tabella Durezza, Punti Ferita e DC Forza per Rompere Oggetti}

\medskip

\begin{tabular}{llll}
	\textbf{Oggetto}& \textbf{Dur.} & \textbf{Punti Ferita} & \textbf{DC}\\
	\toprule
	Corda (2,5 cm di diametro)     & 0      & 2           & 23\\
	Porta di legno semplice        & 5      & 10          & 13\\
	Cassa piccola                  & 5      & 1           & 17\\
	Porta di legno buona           & 5      & 15          & 15\\
	Forzare corde legate           & 0      & 2			  & 15\\
	Porta di legno robusta         & 5      & 20          & 18\\
	Cassa del tesoro               & 5      & 15          & 23\\
	Piegare sbarre di ferro        & 10     &10           & 24\\
    Porta rinforzata               & 10     &10           & 25\\
	Muro di pietra (spesso 30 cm)  & 8      & 90          & 35\\
	Pietra tagliata (spessa 90 cm) & 8      & 540         & 50\\
	Catena                         & 10     & 5           & 26\\
	Manette                        & 10     & 10          & 26\\
	Manette perfette               & 10     & 10          & 28\\
	Porta di ferro (spessa 5 cm)   & 10     & 60          & 28\\
\end{tabular}


\bigskip

Creature di Taglia superiore o inferiore a quella Media hanno bonus o penalità dati dalla taglia sulla prova di Forza per sfondare una porta:

\medskip

\textbf{Tabella: Modificatori prova di Forza in base alla propria taglia}\index{Tabella Modificatori prova di Forza per Sfondare porta}

\medskip

\begin{tabular}{ll}
	\textbf{Taglia} & \textbf{Modificatore}\\
		\toprule
	Piccolissima    & -16\\
	Minuta          & -12\\
	Minuscola       & -8\\
	Piccola         & -4\\
	Normale         & +0\\
	Grande          & +4\\
	Enorme          & +8\\
	Mastodontica    & +12\\
	Colossale       & +16\\
\end{tabular}



\medskip

Un piede di porco o un ariete portatile aumentano la probabilità del personaggio di sfondare una porta (+1d6)

\end{multicols}


\pagebreak

\section{Ambiente}\index{Ambiente}

\label{ambiente}
\begin{enfasi}{
La natura non è crudele, è solo spietatamente indifferente. Questa è una delle più dure lezioni che un essere umano debba imparare. (Richard Dawkins)
		
\medskip

L'antidoto principale contro un cattivo ambiente consiste, naturalmente, nel sostituirlo con uno buono. (Robert Baden-Powell)}\end{enfasi}\medskip

\begin{multicols}{2}

\lettrine[lines=2, lhang=0.33, loversize=0.25, findent=1.5em]{D}{ai} deserti senza vita ai dungeon zeppi di trappole, l'ambiente aiuta a definire il mondo, renderlo vivo, dinamico e ricco. Consente di creare un'esperienza di gioco emozionante e coinvolgente. Questo capitolo contiene le regole per aiutare il Narratore a definire il mondo di gioco, come dungeon, trappole, terreni e pericoli ambientali.

\subsection{Regole Ambientali}

\label{regole-ambientali}

\subsubsection{Visione e Luce}\index{Visione}\index{Luce}

\label{sec:visione-e-luce}

In un ambiente naturale l'illuminazione può assumere diverse gradazioni e queste gradazioni aiutano a capire fino a che distanza una creatura può vedere.

Le gradazione di luce possono essere:
\begin{itemize}
	\item
	      \textbf{Oscurita}': buio pesto, può essere naturale o magico
	\item
	      \textbf{Luce Fioca/Oscurata leggermente}: poca illuminazione, permette di riconoscere le sagome
	\item
	      \textbf{Luce}: luce intensa, una luce brillante, coprente, assolata
\end{itemize}

Saranno le fonti di illuminazione, o la loro assenza, a stabilire quanta luce c'e' e fino a che distanza. La Tabella Fonti di Luce indica per le più comuni fonti di luce il raggio illuminato a pieno, quello illuminato in maniera inferiore (Luce Fioca) e la durata.

\medskip

\begin{center}
	\includegraphics[width=0.8\linewidth]{immagini/oscurita.png}
	
	\textit{Henry Justice Ford}
\end{center}

\bigskip

\textbf{Tabella: Fonti di luce}\index{Tabella delle fonti di luce}

\medskip

\index{Luce Fioca}

\begin{tabular}{l|cc|c}
	\textbf{Fonte di} &	\multicolumn{2}{c}{\textbf{Raggio in metri}}& \textbf{Durata}  \\
	\textbf{Luce}& \textbf{Luce} & \textbf{Luce Fioca} &\\
	\toprule
	Candela          & 1 metro         & -              & 1 ora\\
	Torcia           & 3 metri         & 3 metri        & 1 ora\\
	Lanterna         & 6 metri         & 6 metri        & 6 ore \\
	\multicolumn{4}{c}{\textbf{Incatesimi}}\\
	Luce	      	 & 6 metri         & 6 metri        & 1 ora* \\
	Luce Diurna      & 18 metri        & 18 metri       & 1 ora \\
\end{tabular}

\medskip

La \textbf{Luce fioca}\index{Luce Fioca} è la luce oltre una fonte di luce. E' il passare in un corridoio di 3 metri se e illuminato solo da fioche candele. E' una notte di luna piena.
In linea di massima una fonte di luce crea luce fioca in un raggio doppio rispetto al raggio di luce normale, nella tabella viene indicata come distanza oltre la Luce normale.

Per una creatura con visione normale la luce fioca fornisce una Copertura leggera, ovvero +2 alla Difesa e rende piu difficili (+2 alla DC) le prove di Consapevolezza. Se tutti i contendenti (personaggi e nemici) condividono questa penalità potete semplicemente ignorarla per tutti.

\medskip

\textbf{Oscurità}\index{Oscurità}: è il buio più completo senza alcuna fonte di luce. Per creature con visione normale l'oscurità è ciò che c'è oltre la Luce fioca.
Il \textbf{personaggio cieco}\index{Cieco} o che combatte nell'oscurità (e non puo' vedere nell'oscurità) ha -4 alla Consapevolezza visiva e tutti gli avversari sono invisibili.

\medskip

La \textbf{Luce}\index{Luce} è la luce all'aperto sotto il sole, ma anche se si tiene una torcia in mano o in un corridoio illuminato da lanterne. Nel suo piccolo anche una candela fornisce luce, ma solo quanto basta ad avvolgere noi stessi.

\subsubsection{Tipi di Visione e Creature}

\begin{itemize}
	\item
	Una creatura con \textbf{Visione Normale} \index{Visione Normale}vede fino alla distanza, come raggio circolare intorno alla fonte di luce, indicato in Luce. Oltre è Luce Fioca e oltre ancora Oscurità.
	
	\item
	Una creatura con \textbf{Visione Crepuscolare} \index{Visione Crepuscolare}vede fino alla distanza, come raggio circolare intorno alla fonte di luce, indicato in Luce fioca, o indicato dalla razza se minore, oltre è oscurità.
	
	\item
	Una creatura con \textbf{Scurovisione} \index{Scurovisione} vede fino alla distanza indicata dalla sua capacità di Scurovisione, indipendentemente che ci sia luce o meno, oltre non può vedere.
	La Scurovisione è una visione in bianco e nero.
\end{itemize}


\subsubsection{Buio}\index{Buio}

\label{buio}

Le torce e le lanterne possono essere spente all'improvviso da una folata di vento, le fonti di luce magiche possono essere dissolte o contrastate ed alcune trappole magiche possono creare aree di buio impenetrabile.

In certi casi, alcuni personaggi o mostri potrebbero essere in grado di vedere mentre gli altri sono Accecati. Ai fini delle regole che seguono, una creatura Accecata è semplicemente una creatura che non è in grado di vedere ciò che la circonda.

\subsubsection{Accecato}\index{Accecato}\index{Invisibile}

\label{accecato}

Le creature Accecate perdono la loro Abilità di infliggere danni extra causati ad esempio dall'Abilità di pugnalare alle spalle (ma non da esplosione del danno o critico al colpire).

Le creature accecate si muovono a velocità dimezzata. Devono effettuare una prova di Acrobatica con DC 12 per azione di movimento per muoversi a velocità normale. Se la prova fallisce cadono a terra prone. Le creature accecate non possono correre o caricare.

Una creatura accecata, o che combatte contro una creatura invisibile,\index{Invisibile} può effettuare una prova di Consapevolezza a difficoltà 20 (oppure 10+Muoversi silenziosamente dell'avversario se questo non vuole farsi trovare) per individuare la creatura purché questa sia entro 3 metri dal personaggio.

Una creatura Accecata \index{Accecata}subisce penalità -4 alle prove di Consapevolezza e -2 alle prove basate su Forza e Destrezza e fallisce automaticamente qualsiasi prova di Consapevolezza dipendenti dalla vista.

Inoltre, una creatura accecata dal buio non può usare incantesimi che prevedano l'uso dello sguardo ed è immune agli incantesimi che prevedono lo sguardo.

Vedi dettagli modificatori di attacco in \hyperlink{invisibilita}{Invisibilità}.

\subsubsection{Cadute}\index{Cadute}

\label{cadute}

Le creature che cadono si fanno male. Dividi l'altezza di caduta (in metri) per 3, arrotonda per difetto, il numero che risulta sono i d6 di danno subiti. Es 16 metri di caduta sono 16/3=5d6 di danno. I danni da caduta non possono superare i 20d6 di danno (anche se la caduta è da kilometri di altezza).

Le creature che subiscono danni da una caduta, atterrano in posizione prona.

Una prova di Acrobatica riuscita con DC 15 permette al personaggio di dimezzare il danno quando cade da meno di 20 metri.

Cadute su superfici morbide (terreno morbido, fango ecc.) riducono di 1d6 i danni. Questa riduzione si applica prima della diminuzione del danno per l'uso della competenza Acrobatica.


\begin{center}
	\includegraphics[width=0.8\linewidth]{immagini/oggetticadenti.png}
	
	\textit{Henry Justice Ford}
\end{center}


Un personaggio non può utilizzare incantesimi mentre cade, a meno che la caduta non sia superiore a 150 metri o l'incantesimo sia lanciabile come Reazione, Azione Immediata o 1 Azione. Lanciare un'incantesimo mentre si cade aumenta la Difficoltà di lancio di 10.

\medskip

\textbf{Cadere in Acqua}\index{Cadere in Acqua}

Le cadute in acqua sono gestite in modo leggermente diverso. Fino a quando l'acqua ha una profondità di almeno di 3 metri ed il tuffo è da una altezza entro 12 metri non si subiscono danni.

Si subiscono 2d6 di danni da una caduta oltre i 15 metri e 5d6 per cadute oltre i 15 metri.

I personaggi che si tuffano volontariamente in acqua non subiscono danni se superano una prova di Acrobatica o di Nuotare con DC 15 se l'acqua è profonda almeno 6 metri. La DC della prova aumenta di 5 ogni 5 metri oltre i 15 metri.

\subsubsection{Effetti dell'Acido}\index{Acido}

\label{effetti-dellacido}

Gli acidi corrosivi infliggono 1d6 danni per round di esposizione, tranne nel caso di totale immersione (come in una vasca d'acido), che infligge 10d6 danni per round. Un attacco con l'acido, come quello di una boccetta lanciata o la saliva/soffio di un mostro, deve essere considerato come un round di esposizione.

I vapori prodotti dalla maggior parte degli acidi sono equivalenti a veleni inalati. Coloro che si avvicinano molto ad un grosso ammasso di acido devono effettuare un Tiro Salvezza su Tempra con DC 13 o subiranno 1 danno alla Costituzione a round. Questo veleno non ha frequenza, pertanto una creatura è salva se si allontana dall'acido.

Le creature immuni alle proprietà caustiche dell'acido potrebbero comunque annegare se vi vengono totalmente immerse (vedi Annegamento).

\subsubsection{Effetti del Fumo}\index{Fumo}

\label{effetti-del-fumo}

Un personaggio costretto a respirare del fumo denso deve superare un Tiro Salvezza su Tempra ogni round (DC 15, +1 per ogni prova precedente) oppure passa il round a tossire e soffocare. Un personaggio che continua a soffocare per 2 round consecutivi subisce 1d6 Danni Non Letali per ulteriore round di esposizione. Il fumo oscura la vista, fornendo Copertura leggera (+2 Difesa) ai personaggi che si trovano al suo interno.

\subsubsection{Fame e Sete}\index{Fame}\index{Sete}

\label{fame-e-sete}

I personaggi potrebbero trovarsi senz'acqua o cibo e privi dei mezzi per procurarsene. Nei climi normali, i personaggi di taglia Media hanno bisogno di almeno 2 litri di liquidi e 0.5 kg di cibo decente al giorno per evitare la fame. (I personaggi di taglia Piccola necessitano della metà). Nei climi molto caldi, i personaggi possono aver bisogno di due o tre volte quella quantità d'acqua per evitare la disidratazione.

Ogni giorno senza cibo è necessario effettuare un Tiro Salvezza su Tempra a difficoltà 11 +1 per giorno senza cibo, se non si ha neanche da bere la difficoltà aumenta a +2.

Se si fallisce il Tiro Salvezza si subiscono 1d4 di danno e si diventa Affaticati. La penalità alla Costituzione dovuta all'affaticamento è cumulativa finché non si mangia e beve abbastanza.

\subsubsection{Oggetti Cadenti}\index{Oggetti Cadenti}

\label{oggetti-cadenti}

Proprio come i personaggi subiscono danni dalle cadute superiori a distanze di mischia, allo stesso modo subiscono danni se vengono colpiti da oggetti cadenti.

Gli oggetti che cadono addosso ai personaggi infliggono danni a seconda del loro peso e della distanza da cui sono caduti.

La \textbf{Tabella: Danno da Oggetti Cadenti} determina la quantità di danni inflitti da un oggetto in base alla sua taglia. Si presume che l'oggetto sia fatto di un materiale denso e pesante, come la pietra.
Gli oggetti fatti di materiali più leggeri potrebbero infliggere la metà o meno del danno indicato, a discrezione del Narratore. Per esempio un masso Enorme che colpisce un personaggio infligge 6d6 danni, mentre un carro di legno potrebbe infliggerne solo 3d6.

Inoltre, se l'oggetto cade da una distanza inferiore ai 3 metri, infligge la metà dei danni indicati. Se un oggetto cade da una distanza superiore ai 20 metri, infligge danni raddoppiati. L'oggetto che cade subisce la stessa quantità di danni che infligge.

\bigskip

\textbf{Tabella: Danno da Oggetti Cadenti}\index{Tabella Danno da Oggetti Cadenti}

\medskip

\begin{tabular}{ll}
	\textbf{Taglia dell'Oggetto} & \textbf{Danno}\\
	\toprule
	Minuscola o Più Piccola      & 1d6\\
	Piccola                      & 2d6\\
	Media                        & 3d6\\
	Grande                       & 4d6\\
	Enorme                       & 6d6\\
	Mastodontica                 & 8d6\\
	Colossale                    & 10d6\\
\end{tabular}

\bigskip

Lasciar cadere addosso ad una creatura un oggetto richiede un attacco di contatto a distanza (Difesa a tocco contro attacco basato su Destrezza). Questi attacchi hanno di solito una gittata di 3 metri. Se un oggetto cade su una creatura (invece di venire lanciato), quella creatura deve effettuare un Tiro Salvezza su Riflessi con DC 15 per dimezzare il danno se è consapevole dell'oggetto che sta cadendo. Gli oggetti cadenti che sono parte di una trappola usano le regole relative alle trappole invece che quelle qui descritte.

\subsubsection{Pericoli dell'Acqua}\index{Pericoli dell'Acqua}\index{Acqua}

\label{pericoli-dellacqua}

Qualsiasi personaggio può attraversare acque relativamente calme che non abbiano una profondità maggiore alla sua altezza, senza bisogno di prove. Allo stesso modo, Nuotare in acque calme richiede una prova con DC 10. I nuotatori addestrati (almeno 1 punto in Nuotare) possono prendere 10. Si ricordi che l'armatura o l'equipaggiamento pesante rendono qualsiasi tentativo di nuotare più difficile

D'altra parte, le acque rapide sono molto più pericolose.

Con una prova riuscita di Nuotare o una prova di Forza con DC 15, i personaggi non rischiano di finire sott'acqua. Se falliscono, subiscono 1d3 Danni Non Letali per round (1d6 danni letali se le acque scorrono sopra rocce e avvallamenti.

L'acqua molto profonda non è solo nera come la pece, rendendo molto pericolosa la navigazione, ma infligge danni ancora peggiori a causa della pressione dell'acqua nell'ordine di 1d6 danni al minuto ogni 30 metri che separano il personaggio dalla superficie. Un Tiro Salvezza su Tempra superato con successo (DC 15, +1 per ogni prova precedente) indica che il personaggio immerso non subisce danni in quel minuto. L'acqua molto fredda infligge 1d6 Danni Non Letali per minuto di esposizione a causa dell'ipotermia.

\textbf{Annegamento}\index{Annegamento}

Qualsiasi personaggio può trattenere il fiato per un numero di round pari 6 round per il suo punteggio di Costituzione, con un minimo di 3 round. Per ogni Azione compiuta la durata restante diminuisce di 1 round. Trascorso questo periodo di tempo, il personaggio deve effettuare un Tiro Salvezza su Tempra con DC 12 ogni round per continuare a trattenere il fiato. Ogni round, la DC aumenta di 1.

\medskip

\includegraphics[width=0.8\linewidth]{immagini/affogare.png}	
\begin{center}\textit{Henry Justice Ford}\end{center}
	
\medskip

Se il Tiro Salvezza fallisce il personaggio va immediatamente a 0 punti ferita e sviene. Dal round successivo incomincia a perdere 1 punto ferita a round fino alla morte (o alla rianimazione!) 

Si può annegare in sostanze diverse dall'acqua, come la sabbia, le sabbie mobili, la polvere molto fine o un silos pieno di farro o semplicemente trattenendo il respiro.

\subsubsection{Pericoli del Caldo}\index{Caldo}

\label{pericoli-del-caldo}

Una creatura sottoposta a temperature molto elevate (sopra i 40° C) deve superare un Tiro Salvezza su Tempra ogni ora (DC 15, +1 per ogni prova precedente) oppure subisce 1d4 danni Non Letali. Se indossa abiti pesanti o qualsiasi tipo di armatura, subisce penalità -4 a questi Tiri Salvezza. Un personaggio somma il suo valore di competenza di Sopravvivenza e può dare un bonus ai compagni pari alla metà del valore per lo stesso Tiro Salvezza. I personaggi Privi di Sensi iniziano a subire danni letali (1d4 danni all'ora).

Un personaggio che subisce Danni Non Letali a causa dell'esposizione al caldo, è soggetto ad un colpo di calore ed è Affaticato. Queste penalità terminano quando il personaggio recupera i Danni Non Letali subiti a causa del caldo.

\begin{center}
	\includegraphics[height=0.5\linewidth]{immagini/desert.png}
\end{center}

Il caldo infernale (temperatura dell'aria sopra i 60° C, fuoco, acqua che bolle, lava) infligge danni letali. Respirare l'aria con queste temperature infligge 1d6 danni da fuoco al minuto (senza Tiro Salvezza).
Inoltre, il personaggio deve superare un Tiro Salvezza su Tempra (con i modificatori di cui sopra) ogni 5 minuti (DC 15, +1 per ogni prova precedente) oppure subisce 1d4 Danni Non Letali. Coloro che indossano abiti pesanti o qualsiasi tipo di armatura, subiscono penalità -4 a questi Tiro Salvezza.

L'acqua bollente infligge 1d6 danni da scottatura, a meno che il personaggio non vi venga completamente immerso, nel qual caso subirebbe 10d6 danni per round di esposizione.

\subsubsection{Prendere Fuoco}\index{Prendere Fuoco}\index{Fuoco}

\label{prendere-fuoco}

I personaggi esposti ad olio bollente, fuochi da campo, e fuochi magici non istantanei possono vedere i loro abiti, capelli o equipaggiamento prendere fuoco. Gli incantesimi specificano se sono in grado di appiccare il fuoco.

I personaggi che rischiano di prendere fuoco possono effettuare un Tiro Salvezza su Riflessi con DC 15 per evitare questo destino. Se i vestiti o i capelli di un personaggio prendono fuoco, egli subisce immediatamente 1d6 danni. Per ogni round successivo il personaggio in fiamme deve effettuare un altro Tiro Salvezza su Riflessi. Il fallimento indica che subisce altri 1d6 danni in quel round. Il successo indica che il fuoco si è estinto (ovvero, una volta che supera il Tiro Salvezza, non sta più andando a fuoco).

Un personaggio che va a fuoco può estinguere automaticamente le fiamme saltando dentro a dell'acqua sufficiente a spegnerle. Se non ci sono grosse quantità d'acqua a disposizione, rotolarsi sul terreno o smorzare la fiamma con mantelli o simili può concedere al personaggio un altro Tiro Salvezza con bonus +4.

Coloro che sono talmente sfortunati dal vedere il loro Equipaggiamento o vestiti prendere fuoco devono superare un Tiro Salvezza su Riflessi (DC 15) per ogni oggetto. Gli oggetti infiammabili che falliscono il tiro, subiscono la stessa quantità di danni del personaggio.

\includegraphics[width=0.8\linewidth]{immagini/fuocopericolo.png}



\medskip

\textbf{Effetti della Lava}\index{Lava}

La lava o il magma infliggono 2d6 danni per round di esposizione, tranne in caso di totale immersione (come quando un personaggio cade nel cratere di un vulcano attivo), che infligge 20d6 danni per round (più eventuali danni da caduta).

I danni provocati dal magma continuano per 1d3 round dopo il termine dell'esposizione, ma questi danni addizionali sono solo la metà di quelli inflitti durante l'ultimo round di effettivo contatto. Un'Immunità o una Resistenza al fuoco servono anche come resistenza o resistenza alla lava o al magma. Tuttavia, le creature Immuni o Resistenti al Fuoco potrebbero annegare se immerse nella lava (vedi Annegamento).


\subsubsection{Pericoli del Freddo}\index{Freddo}

\label{pericoli-del-freddo}

I personaggi non ben vestiti in climi freddi (sotto i 5° C) devono superare un Tiro Salvezza su Tempra ogni ora (DC 15, +1 per ogni prova precedente) oppure subiscono 1d6 Danni Non Letali. 
In condizioni di freddo estremo o di esposizione sotto i -17° C, un personaggio scoperto deve effettuare un Tiro Salvezza su Tempra ogni 10 minuti (DC 15, +1 per ogni prova precedente), subendo 1d6 Danni Letali per ogni Tiro Salvezza fallito. I personaggi che indossano abiti invernali hanno bisogno di effettuare la prova per il freddo e l'esposizione solo una volta all'ora.

Un personaggio somma il suo valore di competenza di Sopravvivenza ai Tiri Salvezza e può dare un bonus ai compagni pari alla metà del valore per lo stesso Tiro Salvezza.

Un personaggio che subisce Danni Non Letali a causa del freddo o dell'esposizione, è soggetto ai geloni o all'ipotermia (considerarlo come Affaticato). Queste penalità terminano quando il personaggio recupera dai Danni Non Letali subiti a causa del freddo e dell'esposizione.

Le condizioni di freddo intollerabile o di esposizione (sotto i -28° C) infliggono ai personaggi 1d6 danni letali per minuto (senza alcun Tiro Salvezza) se non specificatamente protetti.


\subsubsection{Effetti del Ghiaccio}\index{Ghiaccio}

I personaggi che camminano sul ghiaccio è come se camminassero su terreno difficile. Il movimento è dimezzato, eventuali prove di Acrobatica hanno un aumento di difficoltà +5. I personaggi che sono per lungo tempo a contatto con il ghiaccio potrebbero subire dei danni da freddo estremo.

\begin{center}
	\includegraphics[height=0.6\linewidth]{immagini/snowfall.png}
\end{center}

\subsubsection{Soffocamento Lento}\index{Soffocamento}

Un personaggio di taglia Media può respirare tranquillamente per circa 6 ore in una camera sigillata che misura 3 metri di lato. Dopo questo tempo, subisce 1d6 Danni Non Letali ogni 15 minuti. Ogni ulteriore personaggio di taglia Media oppure ogni fuoco significativo (una torcia, per esempio) riducono proporzionalmente la durata dell'aria respirabile. Una volta privi di sensi per l'accumulo di Danni Non Letali, i personaggi iniziano a subire Danni Letali allo stesso ritmo. I personaggi di taglia Piccola consumano metà dell'aria dei personaggi di taglia Media.

\subsection{Tempo Atmosferico - Meteo}\index{Meteo}

\label{tempo-atmosferico---meteo}

A volte il tempo atmosferico può giocare un ruolo importante nel corso di un'avventura. La Tabella: Tempo Atmosferico Casuale è una tabella generica che può essere utilizzata per stabilire le condizioni atmosferiche locali. I termini della tabella sono definiti qui di seguito:

\end{multicols}

\medskip

\textbf{Tabella: Tempo Atmosferico Casuale}\index{Tabella Tempo Atmosferico Casuale}

\medskip

\begin{tabular}{lllll}
	\textbf{d\%} & \textbf{Meteo} & \textbf{Clima Freddo}& \textbf{Clima Temperato {*}}   & \textbf{Deserto}\\
		\toprule
	01-70   & Normale& Freddo, calmo   & Normale per la stagione {*}{*} & Torrido,calmo\\
	71-80   & Anormale    & Ondata di Caldo (01-30)&&   \\
	   &   & Ondata di Freddo (31-100) & Ondata di Caldo (01-50)&\\
	   &   &  & Ondata di Freddo (51-100) & Torrido,ventilato\\
	81-90   & Inclemente  & Precipitazioni (neve)& Precipitazioni&    \\
	   &   &  & (normali per la stagione) & Torrido, ventilato\\
	91-99   & Tempesta    & Tempesta di neve& Tempesta di fulmini &   \\
	   &   &  & tempesta di neve& Tempesta di polvere\\
	100& Tempesta violenta& Tormenta   & Bufera,tormenta  & \\
	   &   &  & uragano,tornado & Acquazzone\\
\end{tabular}

\medskip

* Temperato comprende foreste, colline, paludi, montagne, pianure e zone marine calde.

** L'inverno è freddo, l'estate è calda, l'autunno e la primavera sono moderati. Le paludi sono sempre leggermente più calde d'inverno.

\begin{multicols}{2}

\textbf{Acquazzone}: Considerarlo come pioggia (vedi Precipitazioni sotto), ma offre copertura come la nebbia. Può provocare inondazioni e dura di solito 2d4 ore.

\textbf{Caldo}: La temperatura è tra 15° e 30° C di giorno, e tra 6 e 11 gradi in meno di notte.

\textbf{Calmo}: Vento leggero (tra 0 e 15 km/h).

\textbf{Freddo}: Temperatura tra -17° e 5° C durante il giorno, e tra 6 a 11 gradi in meno di notte.

\textbf{Moderato}: Temperatura tra i 5° e i 15° C durante il giorno, e tra 6 e 11 gradi in meno di notte.

\textbf{Ondata Caldo}: Fa aumentare la temperatura di 6° C.

\textbf{Ondata Freddo}: Abbassa la temperatura di 6° C.

\textbf{Precipitazioni}: Tirare un d100 per determinare se la precipitazione è nebbia (01-30), pioggia/neve (31-90), o nevischio/ grandine (91-00). La neve e il nevischio si verificano solo quando la temperatura è di 0° C o inferiore. La maggior parte delle precipitazioni dura 2d4 ore. La grandine, invece, dura solo 3d6 minuti ma di solito è accompagnata da 1d4 ore di pioggia.

\textbf{Tempesta} (di Fulmini/di Neve/di Polvere): Il vento è molto forte (da 45 a 75 km/h) e la visibilità è ridotta di tre quarti. Le tempeste durano 2d4-1 ore. Vedi Tempeste, sotto, per ulteriori dettagli.

\textbf{Tempesta} (Bufera/Tormenta/Uragano/Tornado): La velocità del vento è superiore ai 75 km/h (vedi Tabella: Effetti del Vento). Inoltre, le tormente sono accompagnate da pesanti nevicate (1d3 \texttimes{} 30 cm), e gli uragani sono accompagnati da acquazzoni. Le bufere durano 1d6 ore, le tormente 1d3 giorni. Gli uragani possono durare fino a una settimana, ma l'impatto maggiore per i personaggi avverrà in un periodo di tempo tra le 24 e le 48 ore, mentre il centro della tempesta si sposta nella loro zona. I tornado durano molto poco (1d6 \texttimes{} 10 minuti), e di solito si formano come parte di una tempesta di fulmini.

\textbf{Torrido}: Temperatura tra i 30° e i 43° C durante il giorno e tra 6 e 11 gradi in meno di notte.

\textbf{Ventilato}: La velocità del vento va da moderata a forte (da 15 a 45 km/h); vedi Tabella: Effetti del Vento.

\textbf{Pioggia, Neve, Nevischio e Grandine}

La brutta stagione frequentemente rallenta o blocca i trasporti via terra e rende praticamente impossibile la navigazione. acquazzoni torrenziali e bufere oscurano la visuale tanto quanto lo farebbe una nebbia densa.

La maggior parte delle precipitazioni si manifesta come pioggia, ma nei climi freddi possono manifestarsi anche come neve, nevischio o grandine. Le precipitazioni di qualsiasi tipo, seguite da un calo della temperatura da sopra a sotto gli 0° C possono produrre ghiaccio.

\begin{center}
	\includegraphics[width=0.9\linewidth]{immagini/Paesaggio-pioggia-Auvers.png}
	
\textit{Vincent van Gogh, Paesaggio sotto la pioggia ad Auvers, 1890, olio su tela, cm 50 x 100. Cardiff, Amgueddfa Cymru – National Museum Wales / The Davies Sisters Collection}
\end{center}


\textbf{Pioggia intensa}\index{Pioggia intensa}

La pioggia dimezza la visibilità, e impone penalità -4 alle prove di Consapevolezza. Ha lo stesso effetto di un vento molto forte sulle fiamme, sugli attacchi con armi a distanza e sulle prove di Consapevolezza come vento molto forte.

\textbf{Neve}\index{Neve}

Mentre cade, la neve ha gli stessi effetti della pioggia su visibilità, attacchi con armi a distanza e prove di Consapevolezza ed il terreno e' considerato difficile. Una nevicata della durata di un giorno lascia al suolo 3d6*2.5 centimetri di neve.

\textbf{Neve Fitta}

Una fitta nevicata ha gli stessi effetti di una nevicata normale, ma oscura la visibilità come la nebbia (vedi Nebbia). Un giorno di neve fitta lascia sul terreno 2d4 x 30 centimetri di neve ed il terreno viene considerato doppiamente difficile (movimento/4). Una fitta nevicata accompagnata da venti forti o molto forti può dare origine a cumuli di neve profondi 1d4 x 1,5 metri, specialmente sopra e intorno ad oggetti abbastanza grandi da deflettere il vento (una capanna o una grande tenda, per esempio).
C'è una probabilità del 10\% che una nevicata fitta sia accompagnata da fulmini (vedi Tempesta di Fulmini). La neve ha gli stessi effetti del vento moderato sulle fiamme.

\textbf{Nevischio}

Si tratta fondamentalmente di pioggia congelata, che ha gli stessi effetti della pioggia quando cade (eccetto che la probabilità di estinguere fiamme protette è del 75\%) e quelli della neve una volta depositatasi.

\textbf{Grandine}

La grandine non riduce la visibilità, ma il suono della grandine che cade rende più difficili le prove di Consapevolezza basate sull'udito (penalità -4). A volte (probabilità del 5\%) la grandine può essere talmente grossa da infliggere 1 danno letale (per tempesta) a qualsiasi cosa si trovi all'aperto. Una volta depositata, la grandine ha lo stesso effetto della neve sul movimento.

\subsubsection{Tempeste}\index{Tempeste}

\label{tempeste}

Gli effetti combinati delle precipitazioni (o della polvere) e del vento, che accompagnano tutte le tempeste, riducono la visibilità di tre quarti, imponendo penalità -8 a tutte le prove di Consapevolezza. Le tempeste rendono impossibili gli attacchi con le armi a distanza, tranne che con le armi da assedio, che subiscono penalità -4 i Tiri per Colpire.
Estinguono automaticamente le candele, le torce o simili fiamme non protette. Le fiamme protette, come quelle delle lanterne, vengono agitate violentemente e hanno una probabilità del 50\% di estinguersi. Vedi Tabella: Effetti del Vento per le possibili conseguenze sulle creature sorprese all'esterno senza ripari.

Le tempeste sono di tre tipi.

\textbf{Tempesta di Polvere (grado di Sfida 3)}

queste tempeste desertiche si differenziano dalle altre tempeste in quanto non hanno precipitazioni. Al contrario, le tempeste di polvere trasportano granelli di sabbia che oscurano la vista, soffocano le fiamme non protette e possono addirittura spegnere quelle protette (probabilità del 50\%). Molte tempeste di polvere sono accompagnate da venti molto forti e si lasciano alle spalle un deposito di 1d6 \texttimes{} 2.5 centimetri di sabbia.
Esiste anche una probabilità del 10\% di incontrare grandi tempeste di polvere con bufere di vento (vedi Tabella: Effetti del Vento). Queste violente tempeste di polvere infliggono 1d3 danni non letali per round a chiunque venga sorpreso all'aperto senza riparo e pongono anche il rischio del soffocamento (vedi Annegamento, eccetto che un personaggio con una sciarpa o simile protezione sulla bocca e il naso, non inizia a soffocare se non dopo un numero di round pari 10 \texttimes{} il suo punteggio di Costituzione). Le grandi tempeste di polvere si depositano alle spalle (2d3-1) x 30 centimetri di sabbia.

\textbf{Tempesta di Neve}

oltre ai venti e alle precipitazioni comuni alle altre tempeste, le tempeste di neve depositano 1d6 \texttimes{} 2.5 centimetri di neve sul terreno.

\textbf{Tempesta di Fulmini}

oltre ai venti e alle precipitazioni (di solito pioggia, ma a volte anche grandine), le tempeste di fulmini sono accompagnate da scariche elettriche che rappresentano un pericolo per i personaggi che si trovano all'aperto senza riparo (specialmente se indossano armature metalliche). Come regola generale, si può considerare un fulmine al minuto per un periodo di un'ora nel cuore della tempesta. Ogni fulmine infligge danni da elettricità tra 4d8 e 10d8. Una tempesta di fulmini su dieci viene accompagnata da un tornado.

\textbf{Tempeste Violente}

Venti molto forti e precipitazioni torrenziali riducono la visibilità a zero, e rendono impossibile effettuare prove di Consapevolezza e compiere attacchi con armi a distanza. Le fiamme non protette vengono automaticamente spente, e c'è una probabilità del 75\% che ciò si verifichi anche per quelle protette. Le creature sorprese in queste zone devono effettuare un Tiro Salvezza su Tempra o devono affrontare effetti a seconda della propria taglia (vedi Tabella: Effetti del Vento). Le tempeste violente sono suddivise nei seguenti quattro tipi.

\begin{center}
	\includegraphics[width=0.9\linewidth]{immagini/Vincent_van_Gogh_tempesta.png}
	
	\textit{Vincent van Gogh, Campo di grano sotto un cielo tempestoso (Auvers-sur-Oise, luglio 1890); olio su tela, 50×100,5 cm, Van Gogh Museum, Amsterdam. F 778, JH 2097.}
	
\end{center}


\textbf{Bufera}: Sebbene abbiano poche o nessuna precipitazione, le bufere possono provocare danni ingenti a causa della forza del vento.

\textbf{Tormenta}: La combinazione di forti venti, neve fitta (di solito 1d3 \texttimes{} 30 cm) e freddo intenso rende le tormente letali per chiunque non vi sia preparato.

\textbf{Uragano}: Oltre ai venti molto forti e alla pioggia intensa, gli uragani sono seguiti da inondazioni. Molte attività in un'avventura sono impossibili in queste condizioni.

\textbf{Tornado}: Oltre ai venti molto forti, i tornado possono ferire gravemente ed uccidere quelli che vengono catturati al suo interno.

\subsubsection{Nebbia}\index{Nebbia}

\label{nebbia}

Sia nella forma di una nube a bassa altitudine che di una foschia che sale dal terreno, la nebbia ostacola la visuale oltre la distanza di 3 metri. Le creature più lontane di 3 metri godono di Copertura leggera (+2 Difesa).

La nebbia rende il terreno difficile.

La nebbia potrebbe essere anche molto fitta in quel caso le creature piu' lontane di 3 metri godono di Copertura media (+4 alla Difesa) e quelle entro 1 metro hanno comunque copertura leggera.

\subsubsection{Venti}\index{Venti}

\label{venti}

I venti possono creare turbini di sabbia o polvere, alimentare grossi incendi, rovesciare piccole imbarcazioni e disperdere gas o vapori. Se sono forti a sufficienza possono addirittura buttare a terra i personaggi (vedi Tabella: Effetti del Vento), interferire con gli attacchi a distanza, o imporre penalità ad alcune Prove di Competenze.

\medskip

\textbf{Tabella: Effetti del Vento Forza del Vento}\index{Tabella Effetti del Vento Forza del Vento}

\medskip

\begin{tabularx}{0.45\textwidth}{XXX}
	\textbf{Forza del Vento} & \textbf{Velocità del Vento}   & \textbf{Attacchi a Distanza} \\
	\toprule
	Leggero     & 0-15km &\\
	Moderato    & 16,5-30 km/h&  \\
	Forte       & 31.5-45        & -2 \\
	Molto forte & 45.5-75km/h    & -4 \\
	Bufera      & 76.5-111km/h   & impossibile  \\
	Uragano     & 12-261km/h     & impossibile \\
	Tornado     & 262-450km/h    & impossibile\\
\end{tabularx}

\bigskip

\textbf{Vento Leggero}

Una brezza gentile, che non ha effetti pratici sul gioco.

\textbf{Vento Moderato}

Un vento sostenuto, che ha una probabilità del 50\% di estinguere qualsiasi piccola fiamma non protetta, come quella di una candela.

\textbf{Vento forte:} Folate che spengono automaticamente le fiamme non protette (candele, torce e simili). Queste folate impongono penalità -2 ai Tiri per Colpire a distanza ed alle prove di Consapevolezza.

\textbf{Vento Molto Forte}

Oltre a spegnere automaticamente le fiamme non protette, i venti di questa intensità agitano violentemente le fiamme protette (come quelle di una lanterna) e hanno una probabilità del 50\% di estinguerle. Gli attacchi con le armi a distanza e le prove di Consapevolezza subiscono penalità -4. 

\textbf{Bufera}\index{Bufera}

Abbastanza forti da abbattere i rami o addirittura interi alberi, le bufere estinguono automaticamente le fiamme non protette e hanno una probabilità del 75\% di estinguere quelle protette, come quelle delle lanterne. Gli attacchi con le armi a distanza sono impossibili, e anche le armi da assedio subiscono penalità -4 ai Tiri per Colpire. Le prove di Consapevolezza basate sull'udito subiscono penalità -8 per l'ululare del vento.

\textbf{Uragano}\index{Uragano}

Estingue tutte le fiamme. Gli attacchi a distanza sono impossibili (eccetto con le armi da assedio che subiscono penalità -8 ai Tiri per Colpire). Anche le prove di Consapevolezza basate sull'udito sono impossibili e tutto ciò che i personaggi possono udire è l'ululare del vento. Gli uragani spesso sono in grado di abbattere gli alberi.

\textbf{Tornado (grado di Sfida 10)}\index{Tornado}

Estingue tutte le fiamme. Tutti gli attacchi a distanza sono impossibili (compresi quelli con le armi da assedio), così come le prove di Consapevolezza basate sull'udito. Invece di essere portati via (vedi Tabella: Effetti del Vento), i personaggi che si trovano nelle immediate vicinanze di un tornado e che falliscono un Tiro Salvezza su Tempra vengono risucchiati dentro il tornado.

Coloro che entrano in contatto con la nube conica vengono sollevati da terra e sbatacchiati per 1d10 round, subendo 6d6 danni per round, prima di venirne espulsi violentemente (con l'applicazione dei danni da caduta).

Sebbene la velocità rotatoria di un tornado possa raggiungere i 450 km/h, il cono stesso si muove in avanti ad una media di 45 km/h (circa 75 metri per ogni round). Un tornado è in grado di sradicare alberi, distruggere edifici e provocare altre forme di simile devastazione.

\end{multicols}

\pagebreak

\section{Avventure in Acqua}\index{Avventure in Acqua}

\label{avventure-in-acqua}
\begin{enfasi}{
Guardò il mare e capì fino a che punto era solo, adesso. (Il vecchio e il mare, Ernest Hemingway)}\end{enfasi}\medskip

\begin{multicols}{2}

\lettrine[lines=2, lhang=0.33, loversize=0.25, findent=1.5em]{L}{'acqua} permette alle società di esistere, ma può anche distruggerle. La vita non potrebbe esistere senza di essa. Il commercio ed il viaggio sono agevolati dalla sua presenza. Eppure, l'acqua può anche uccidere, sia annegando le persone, sia generando alluvioni e tsunami su larga scala. La vita terrestre è dipendente dall'acqua ma allo stesso tempo la teme.

\textbf{Avventure Acquatiche}

Un'avventura acquatica può aver luogo ovunque l'acqua rappresenti l'elemento principale del territorio: come paludi, fiumi, laghi, stagni, oceani, il Piano dell'Acqua e simili. Le avventure Acquatiche, comunque, non richiedono che i personaggi abbiano la capacità di respirare sott'acqua; l'introduzione di sfide Acquatiche per avventurieri di basso livello apportano ad un'avventura un bel pò di tensione e sensazione di pericolo.

\textbf{Adattarsi agli Ambienti Acquatici}

Le regole per il combattimento sott'acqua si applicano alle creature che non sono native di questo pericoloso ambiente, come la maggior parte dei PG. Per avventure Acquatiche prolungate ed esplorazioni particolarmente in profondità, i personaggi necessiteranno dell'uso della magia per proseguire le proprie avventure. incantesimi di Trasformazione o Abiurazione sono di ovvia utilità.

Il danno da pressione può essere totalmente evitato tramite incantesimi che offrano una resistenza. 

\subsection{Combattimento sott'acqua}
Le creature che vivono sulla terra hanno considerevoli difficoltà a combattere sott'acqua. L'acqua influenza la Difesa di una creatura, i suoi Tiri per Colpire, i danni e il movimento. 

\begin{itemize}
\item
Una creatura sott'acqua perde il bonus di Destrezza alla Difesa.
\item
Una creatura che non sia sotto l'incantesimo \emph{Libertà di movimento} o una velocità di Nuotare effettua i Tiri per Colpire con un -1d6 e l'avversario si considera che abbia Resistenza al danno da Taglio e da Botta.

Armi come Tridente, Lancia corta, Spada Corta, Giavellotto non hanno penalità al colpire.
\item
Se non si ha una movimento di tipo Nuotare ci si può muovere a metà del movimento per Azione di Movimento (terreno difficile)
\end{itemize}
\medskip

\subsubsection{Attacchi a distanza sott'acqua}\index{Attacchi sott'acqua}
Le armi da lancio sono inefficaci sott'acqua, anche quando vengono lanciate da terra. Gli attacchi con le armi a distanza subiscono penalità -2 ai Tiri per Colpire per ogni 1,5 metri d'acqua che attraversano e -1 al danno.

\subsubsection{Attacchi dalla terraferma}
Quei personaggi che nuotano, galleggiano o attraversano l'acqua in superficie, o guadano un tratto in cui l'acqua è alta almeno fino al petto, godono di copertura media.

Una creatura completamente sommersa dispone di copertura totale contro gli avversari sulla terraferma.


\subsubsection{Effetti magici in acqua}
Gli effetti magici non sono influenzati, tranne quelli che richiedono un Tiro per Colpire (che vengono trattati come tutti gli altri effetti) e gli effetti di fuoco.

\textbf{Fuoco}: Il fuoco non magico (incluso il fuoco dell'alchimista) non brucia sott'acqua. Gli incantesimi o gli effetti magici di fuoco sono inefficaci sott'acqua. Una creatura parzialmente sommerso ha resistenza al fuoco.

\textbf{Lanciare incantesimi sott'acqua}\\\index{Lanciare incantesimi sott'acqua}
Lanciare incantesimi mentre ci si trova sott'acqua può essere difficile per chi non ha la capacità di respirare sott'acqua.

Una creatura che non è in grado di respirare sott'acqua consuma tre round di trattenere il fiato per lanciare un incantesimo sott'acqua.

Alcuni incantesimi potrebbero funzionare diversamente sott'acqua, a discrezione del Narratore

Ricordo che un personaggio può trattenere il fiato per COS*6, minimo 3 round, round prima di incominciare ad affogare ed ogni Azione consuma un round aggiuntivo.


\subsection{Avventure Nautiche}

L'acqua può fornire l'ambientazione per un'esperienza di gioco diversa ed unica: l'avventura nautica. In un simile scenario, gli effetti e i pericoli delle avventure subacquee sono sostituiti dalle sfide di superficie, dal momento che i personaggi e i loro avversari utilizzano navi e barche per spostarsi in tale ambiente. Di solito, le avventure nautiche si risolvono normalmente, con un combattimento a bordo di una nave simile ad uno terrestre. Se il combattimento avviene durante una tempesta o in mari agitati, considerate il ponte della nave come terreno difficile. Ricordatevi di considerare gli effetti sulle prove di Concentrazione per il tempo atmosferico o il rollio.

\begin{center}
\flushleft
	\includegraphics[width=0.8\linewidth]{immagini/avventure_acqua_grey.jpg}
	
	\textit{The Mermaid and the Boy" 1904 by H.J.Ford}
\end{center}


\subsubsection{Combattimento Rapido in Mare}

Quando sono le navi a combattere, le cose cambiano un pò. Le regole seguenti non hanno lo scopo di simulare accuratamente tutti gli aspetti di un combattimento navale, ma solo fornirvi rapide e semplici regole per sbrogliare tali situazioni quando si tramutano in un'avventura nautica, che sia una battaglia tra due navi o tra una nave ed un mostro marino.

{Preparazione:} Stabilite quali tipi di navi sono coinvolte nel combattimento (vedi Tabella: Statistiche delle Navi). Utilizzate una griglia da battaglia ampia e vuota per rappresentare le acque in cui ha luogo la battaglia. Un singolo quadretto corrisponde a 9 metri di distanza. Raffigurate ogni nave piazzando dei segnalini che occupino l'appropriato numero di quadretti (le navi giocattolo sono ottimi segnalini e potete reperirle nei negozi di modellismo).

{Cominciare il Combattimento:} Quando il combattimento inizia, lasciate che i personaggi (ed importanti PNG alleati) tirino l'Iniziativa normalmente; la nave si muove ed attacca sulla base del risultato di iniziativa del capitano. Se una delle navi in battaglia usa le vele per spostarsi, determinate casualmente in quale direzione sta soffiando il vento tirando 1d8 e seguendo le linee guida per le Armi a Spargimento che mancano il bersaglio.

\begin{center}
\flushleft
\includegraphics[width=0.8\linewidth]{immagini/acquapericoli.png}
\end{center}


{Movimento:} Sulla base del punteggio di Iniziativa del capitano, la nave può muoversi alla propria velocità base in un singolo round come se l'Azione corrispondesse a quella del capitano stesso (o al doppio della sua velocità come unica azione del round), finché ha il proprio equipaggio minimo al completo. La nave può incrementare o diminuire la propria velocità di 9 metri per round, fino al raggiungimento della velocità massima. In alternativa, il capitano può cambiare direzione (al massimo un lato di un quadretto alla volta) (2 Azioni). Una nave può cambiare direzione solo all'inizio del round.

{Attacchi:} I membri in eccesso rispetto al requisito minimo di equipaggio di una nave possono essere collocati a manovrare le Macchine d'Assedio. Le Macchine d'Assedio attaccano sulla base del punteggio di Iniziativa del capitano.

Una nave può anche tentare di speronare un bersaglio se ospita l'equipaggio minimo. Per speronare un bersaglio, la nave deve muoversi di almeno 9 metri e finire con la prua in un quadretto adiacente ad esso.
Quindi, il capitano della nave effettua una prova di Professione (marinaio): se il risultato è pari o superiore alla Difesa del bersaglio, la nave colpisce il suo obiettivo, infliggendogli danni come indicato nella Tabella: Statistiche delle Navi e allo stesso tempo subendo il danno minimo. Una nave equipaggiata con uno sperone infligge al bersaglio 3d6 danni addizionali (l'imbarcazione attaccante non subisce danno addizionale).

\textbf{Affondamento}\index{Affondamento}

Una nave ottiene la condizione in affondamento quando i suoi Punti Ferita scendono a 0 o meno. Una nave in affondamento non può muoversi o attaccare e dopo 10 round si considera affondata. Ogni 25 danni subiti da una nave che affonda si riduce l'affondamento di 1 round. L'incantesimo di Fabbricare consente di riparare una nave che affonda se i Punti Ferita della stessa sono riportati sopra lo 0, caso in cui la nave perde la condizione in affondamento. In genere, le riparazioni non magiche richiedono troppo tempo per salvare una nave dall'affondamento una volta che questa inizia a sprofondare.

\textbf{Statistiche di una Nave}

Nel mondo reale esiste una grande varietà di barche e navi, dalle piccole zattere agli imponenti galeoni. A rappresentanza di ciò, la Tabella: Statistiche delle Navi classifica sette dimensioni standard di nave e le rispettive statistiche. Così come le culture del mondo reale hanno creato ed adattato differenti tipi di imbarcazioni, così le razze di mondi fantasy potrebbero creare le proprie bizzarre navi.
I Narratore potrebbero utilizzare o modificare queste statistiche per soddisfare le esigenze delle loro creazioni e, comunque, descrivere tali mezzo di trasporto a proprio piacimento. Tutte le navi presentano i seguenti tratti.

\textbf{Tipo}: Si tratta di una categoria generale che elenca la tipologia base di nave.

\textbf{Difesa}: La Difesa della nave. Per calcolare la Difesa effettiva di una nave, aggiungete il punteggio di Professione (marinaio) del capitano alla Difesa base della stessa. Gli attacchi di contatto contro una nave ignorano il modificatore del capitano. Una nave non è mai Impreparata.

\textbf{TS Base}: Il modificatore dei Tiri Salvezza Base di una nave (Tempra, Riflessi e Saggezza) hanno lo stesso valore. Per determinare gli effettivi modificatori dei Tiro Salvezza di una nave, aggiungete il modificatore di Professione (marinaio) del capitano a questo valore.

Velocità Massima: La velocità massima di una nave in combattimento. Un asterisco indica che la nave ha delle vele e può spostarsi a velocità raddoppiata se si muove nella stessa direzione del vento. Una nave che abbia solo delle vele può spostarsi solo in presenza di vento.

\textbf{Armamenti}: Il numero di Macchine d'Assedio che possono essere equipaggiate sulla nave. Uno sperone utilizza uno di questi slot e una nave può essere equipaggiata soltanto con uno sperone.

\begin{center}
	\includegraphics[width=0.8\linewidth]{immagini/navenotte.png}
\end{center}


\textbf{Speronamento}: L'ammontare di danni che infligge una nave con un attacco di speronamento riuscito (senza uno sperone).

\textbf{Quadretti}: Il numero di quadretti che la nave occupa sulla griglia di combattimento. Una nave si considera sempre della larghezza di un quadretto.

\textbf{Equipaggio}: Il primo numero indica l'equipaggio minimo di cui la nave ha bisogno per funzionare normalmente, ad esclusione degli addetti alle armi. Il secondo indica il numero massimo della ciurma più i soldati o passeggeri aggiuntivi. Una nave senza il suo equipaggio minimo può soltanto muoversi, cambiare velocità, cambiare direzione, o speronare se il suo capitano supera una prova di Professione (marinaio) con DC 20.
Un equipaggio che eccede il numero minimo non influenza il movimento, ma i suoi componenti possono sostituire i membri caduti o manovrare armi aggiuntive.

\bigskip

\end{multicols}

\textbf{Tabella: Statistiche delle Navi}\index{Tabella Statistiche delle Navi}

\medskip

\begin{tabular}{lllllllll}
	\textbf{Tipo}  & \textbf{Difesa} & \textbf{Punti Ferita} & \textbf{TS base} & \textbf{Vel. (m/s)} & \textbf{Arma} & \textbf{Speronamento} & \textbf{Quad}. & \textbf{Equipaggio}\\
\toprule
	Zattera   & 9     & 10& +0& 4.5  & 0   & 1d6    & 1    & 1/4\\
	Barca a Remi   & 9& 20& +2& 9    & 0   & 2d6+6  & 1    & 1/3\\
	Battello  & 8& 60& +4& 9    & 1   & 2d6+6  & 2    & 4/15+100\\
	Nave Lunga& 6& 75& +5& 18   & 1   & 4d6+18 & 3    & 50/75+100\\
	Barca a Vela   & 6& 125    & +6& 18   & 2   & 3d6+12 & 3    & 20/50+120\\
	Nave da Guerra & 2& 175    & +7& 18   & 3   & 3d6+12 & 4    & 60/80+160\\
	Galea& 2& 200    & +8& 27   & +4  & 6d6+24 & 4    & 200/250+200\\
\end{tabular}

\pagebreak

\section{Avventure in Citta'}\index{Città}

\label{avventure-in-citta}
\begin{enfasi}{
Dio creò la campagna, e l'uomo creò la città. (William Cowper)}\end{enfasi}\medskip

\begin{multicols}{2}

\lettrine[lines=2, lhang=0.33, loversize=0.25, findent=1.5em]{A}{ prima} vista, una città è molto simile a un dungeon, in quanto è composta da pareti, porte, stanze e corridoi. Le avventure ambientate in città differiscono da quelle ambientate nei dungeon per due motivi principali. I personaggi hanno accesso a un maggior numero di risorse e devono tenere conto della presenza delle forze dell'ordine.

\textbf{Accesso alle Risorse}: A differenza dei dungeon e delle terre selvagge, i personaggi possono comprare e vendere Equipaggiamento molto rapidamente in città. Una città grande o una metropoli probabilmente dispongono di PNG ed esperti di alto livello specializzati nei settori più oscuri della conoscenza, in grado di offrire aiuto e di interpretare gli indizi. E quando i personaggi sono malconci e contusi, possono sempre fare ritorno alle comodità delle loro camere nella locanda.

La libertà di effettuare una ritirata e l'accesso alle merci del mercato significa che i giocatori dispongono di un maggior controllo sui ritmi di gioco di un'avventura in città.

\textbf{Forze dell'Ordine}: L'altro elemento di distinzione tra andare all'avventura in una città ed esplorare un dungeon sta nel fatto che il dungeon è, quasi per definizione, un luogo senza regole dove la sola legge è quella della giungla: uccidere o essere uccisi.

Una città, d'altro canto, è sorretta da un codice di leggi, molte delle quali sono state ideate esplicitamente per prevenire quel genere di comportamento nel quale gli avventurieri indulgono spesso e volentieri: uccidere e saccheggiare. Tuttavia, le leggi cittadine riconoscono la gravità della minaccia che i mostri costituiscono alla stabilità cittadina, ed è assai raro che la proibizione di uccidere valga anche per mostri come le aberrazioni o gli Immondi.

La maggior parte degli umanoidi malvagi, tuttavia, solitamente gode della stessa protezione riservata a tutti gli altri cittadini. Avere un insieme di Tratti malvagi non è un crimine (tranne forse in quelle città dove vige una severa teocrazia, col potere magico necessario per far valere la legge); soltanto gli atti malvagi vengono considerati un'infrazione alla legge.

\begin{center}
	\includegraphics[width=0.8\linewidth]{immagini/cavalieri.png}
\end{center}


Anche quando gli avventurieri incontrano un malfattore impegnato a commettere i crimini più orribili nei confronti della popolazione cittadina, la legge vede comunque di cattivo occhio chi si fa giustizia da solo uccidendo il malfattore o impedendo in altri modi che venga condotto davanti a un tribunale per essere processato.

\textbf{Limitazioni alle Armi e agli Incantesimi}

Ogni città ha le sue leggi riguardo alle armi che è possibile portare con sé circolando in pubblico e alle limitazioni agli incantesimi.

Le leggi cittadine potrebbero non influenzare tutti i personaggi in egual modo. Un uomo di fede che si muove con un'arma al seguito non viene ostacolato in alcun modo dalla legge che impone di legare con un laccio le armi, ma un incantatore subisce una riduzione considerevole del suo potere se il suo Scrigno viene confiscato alle porte della città.

\textbf{Elementi Urbani}

Pareti, porte, illuminazione scarsa e terreno sconnesso: sotto molti aspetti, una città è simile a un dungeon. Di seguito vengono descritti nuovi elementi adatti a un'ambientazione cittadina.

\textbf{Mura e Cancelli}

Molte città sono difese da un cerchio di mura. Delle normali mura cittadine sono in pietra rinforzata, spesse 1,5 metri e alte 6 metri. Un muro simile è piuttosto liscio ed è necessaria una prova di Arrampicarsi con DC 30 per potervisi arrampicare. Le mura dispongono di piccoli merli su un lato per fornire un parapetto alle guardie in cima, e lo spazio per camminare sulle mura è a malapena sufficiente per una guardia.

\textbf{Le mura}

A differenza delle città più piccole, le metropoli spesso sono dotate anche di mura interne, a volte delle vecchie mura erette quando la città era più piccola, oppure mura che separano i vari quartieri gli uni dagli altri. A volte queste mura sono alte e larghe come quelle esterne, ma molto più spesso hanno le dimensioni di quelle di una città grande o piccola.

\textbf{Torri di Guardia}: Alcune mura cittadine sono dotate di torri di guardia che spuntano a intervalli regolari. Sono poche le città che hanno guardie a sufficienza da collocare su ogni torre di guardia, a meno che la città non si aspetti un attacco dall'esterno. Le torri offrono una visuale elevata della campagna circostante oltre a un baluardo di difesa contro gli invasori nemici.

\medskip

\begin{center}
\includegraphics[width=0.85\linewidth]{immagini/muraparigi.png}

	\textit{Cronache, Jean Froissart. La Regina Isabella di Francia arriva a Parigi, 15 secolo}
\end{center}

\medskip


Le torri di guardia solitamente sono più alte di 3 metri rispetto al muro di cui fanno parte, e il loro diametro è pari a 5 volte lo spessore delle mura. Delle feritoie per gli arcieri si aprono ai piani alti della torre, e la cima è merlata allo stesso modo delle mura circostanti. Nelle torri più piccole (del diametro di circa 7,5 metri, lungo un muro spesso 1,5 metri) una semplice scala a pioli collega l'interno della torre al tetto. Nelle torri più grandi si trovano vere e proprie scale.

L'accesso alla torre è protetto da pesanti porte in legno, con rinforzi in ferro e serrature buone (Disattivare Congegni DC 30). Normalmente è il capitano delle guardie a custodire la chiave d'accesso alla torre, e una seconda copia viene conservata nella fortezza interna o nella caserma cittadina.

\textbf{Cancelli}: Un tipico cancello d'accesso alla città è composto da una guardiola con due saracinesche e delle feritoie nello spazio tra di esse. Nei paesi e nelle città piccole, l'entrata principale è protetta da doppie porte di ferro incastrate nelle mura cittadine.

I cancelli rimangono solitamente aperti durante il giorno e chiusi a chiave o sbarrati di notte. Generalmente, soltanto un cancello lascia entrare i viaggiatori dopo il tramonto, ed è sorvegliato da guardie che apriranno le porte solo per qualcuno che abbia un aspetto onesto, presenti i documenti appropriati, o le corrompa con una cifra sufficiente (in base al tipo di città e di guardie).

\textbf{Guardie e Soldati}

Una città solitamente è dotata di personale militare di servizio a tempo pieno pari all'1\% della sua popolazione adulta, in aggiunta ai soldati di turno o di leva pari al 5\% della popolazione. I soldati a tempo pieno sono guardie cittadine responsabili del mantenimento dell'ordine in città, con un ruolo simile a quello della polizia moderna, e (in misura assai minore) della difesa della città dagli assalti esterni. I soldati in leva forzata vengono chiamati alle armi in caso di un attacco in città.

Un tipico schieramento di guardie cittadine si distribuisce in tre turni di servizio da otto ore ciascuno, col 30\% delle sue forze in servizio di giorno (dalle 8 alle 16), il 35\% in servizio di sera (dalla 16 alle 24) e il 35\% di servizio nel turno di notte (dalle 24 alle 8). In qualsiasi momento, l'80\% delle guardie in servizio è di pattuglia per le strade, mentre il 20\% rimanente è assegnato a varie postazioni per la città, pronti a reagire ad eventuali allarmi. Una postazione di guardia simile è presente almeno in ogni vicinato cittadino (un vicinato è composto da vari quartieri).

La maggioranza delle guardie cittadine è composta da combattenti, quasi tutti di 1° livello. Gli ufficiali sono combattenti di livello più alto, e forse anche qualche incantatore.

\textbf{Macchine d'Assedio}\index{Macchine d'Assedio}

Le macchine d'assedio sono grosse armi, strutture temporanee o meccanismi tradizionalmente usati per assediare un castello o una fortezza.

\begin{center}
	\includegraphics[width=0.8\linewidth]{immagini/armidaassedio.png}
\end{center}


\textbf{Catapulta Pesante}: \index{Catapulta}Una catapulta pesante è una gigantesca macchina d'assedio in grado di scagliare macigni o altri oggetti pesanti con grande forza.Dal momento che l'arco di lancio della catapulta è molto alto, il marchingegno è in grado di colpire anche aree al di fuori della sua linea di visuale. Per fare fuoco con una catapulta pesante, il capo degli operatori del macchinario effettua una prova speciale con DC 15 usando solo il suo valore di Competenza di Difesa, il suo modificatore di Intelligenza, la penalità per la gittata e il modificatore relativo alla sezione inferiore della Tabella: Macchine d'Assedio.

Se la prova ha successo, il macigno della catapulta colpisce la zona di mischia a cui la catapulta aveva mirato, infliggendo i danni indicati a qualsiasi oggetto o personaggio nella zona. I personaggi che superano con successo un Tiro Salvezza su Riflessi con DC 15 subiscono danni dimezzati. Una volta che il macigno ha colpito la zona, i tiri successivi colpiranno la stessa zona, a meno che la catapulta non venga ridirezione o il vento non cambi direzione o velocità.

Se il macigno di una catapulta manca il bersaglio, si tira 1d8 per determinare dove atterra. Il risultato indica la direzione in cui il colpo devia, dove 1 indica verso la catapulta stessa e i valori da 2 a 8 le direzioni successive in senso orario attorno alla zona bersaglio. La distanza coperta è pari a 1d4x10 metri.

Per caricare una catapulta è necessaria una serie di azioni che portano via tutto il round. Occorre una prova di Forza con DC 15 per abbassare il braccio della catapulta; la maggior parte delle catapulte hanno ruote che permettono fino a due operatori di usare l'azione di Aiutare un Altro per assistere l'operatore principale della carrucola.

Una prova di Professione (ingegnere d'assedio) con DC 15 consente di agganciare il braccio in posizione, e poi un'altra prova di Professione (ingegnere d'assedio) con DC 15 servirà per caricare il proiettile sulla catapulta. Sono necessarie quattro round per ricaricare una catapulta pesante (vari operatori della catapulta possono compiere queste azioni nello stesso round, quindi quattro persone possono ricaricare una catapulta nel giro di 1 solo round). Una catapulta pesante occupa uno spazio di 4,5 metri.


\textbf{Catapulta Leggera}: Questa è una versione più piccola e più leggera della catapulta pesante. Funziona essenzialmente come una catapulta pesante, con la differenza che è necessaria una prova di Forza con DC 10 per agganciare il braccio al suo posto, e soltanto 2 round per ridirezionare la catapulta.

Una catapulta leggera occupa uno spazio di 3 metri.

\textbf{Balista}: \index{Balista}Una balista è in pratica una balestra pesante enorme fissa. La sua taglia rende difficile il suo utilizzo per la maggior parte delle creature.Quindi, una creatura media subisce penalità --4 ai Tiri per Colpire quando usa una balista, e una creatura piccola subisce penalità --6. Per una creatura di taglia inferiore alla grande sono necessari 2 round per ricaricare la balista dopo aver fatto fuoco.

Una balista occupa uno spazio di 1,5 metri.

\textbf{Ariete}:\index{Ariete} Questo tronco massiccio a volte è legato e sospeso a un traliccio mobile che consente a coloro che lo manovrano di farlo oscillare con forza sempre crescente contro un bersaglio. Come unica azione del round, il personaggio più vicino alla punta dell'ariete effettua un Tiro per Colpire contro la Difesa della costruzione, applicando penalità --4 per la mancanza di competenza (non è possibile avere competenza nell'uso di questo macchinario). Oltre ai danni indicati nella Tabella: Macchine d'Assedio, fino a nove altri personaggi possono spingere l'ariete e aggiungere i loro modificatori di Forza al danno dell'ariete, se riservano un'azione di attacco per farlo. E' necessaria almeno una creatura Enorme o di taglia superiore, 2 creature Grandi, 4 creature Medie oppure 8 creature Piccole per manovrare un ariete (le creature Minuscole o di taglia inferiore non possono usare un ariete).

Un ariete solitamente è lungo 9 metri. In una battaglia, le creature che manovrano un ariete devono disporsi in due file adiacenti di eguale lunghezza con l'ariete sorretto tra le due file.

\textbf{Torre da Assedio}\index{Torre da Assedio}: Questo macchinario è un'enorme torre di legno montata su ruote o cilindri che può essere spinta contro un muro per consentire agli assedianti di scalare la torre e quindi arrivare in cima alle mura beneficiando di Copertura. Le pareti in legno della torre di solito sono spesse circa 30 cm.

Una torre da assedio tipica occupa uno spazio di 4,5 metri. Le creature al suo interno la spingono a una velocità di 3 metri (una torre da assedio non può correre). Le otto creature che spingono la torre al pian terreno godono di Copertura totale, quelle ai piani superiori godono di Copertura migliorata e possono tirare attraverso le feritoie per gli arcieri.


\end{multicols}

\bigskip

\textbf{Tabella: Modificatori di Attacco delle Catapulte}\index{Tabella Modificatori di Attacco delle Catapulte}

\begin{tabularx}{0.95\textwidth}{XX}
	\textbf{Circostanza}      & \textbf{Modificatore}\\
\toprule
	La linea di visuale non giunge fino alla zona bersaglio & -6\\
	Tiro consecutivo (gli operatori riescono a vedere dove sono caduti i colpi andati a vuoto più recenti )  & +2 cumulativo per per colpo mancato precedente (max +10)\\
	Tiro consecutivo (gli operatori non riescono a vedere dove sono caduti i colpi andati a vuoto più recenti ma un osservatore fornisce indicazioni) & +1 cumulativo per per colpo mancato precedente (max +5))\\
\end{tabularx}

\bigskip

\textbf{Tabella: Macchine d'Assedio}\index{Tabella Macchine d'Assedio}

\bigskip

\begin{multicols}{2}

\begin{tabularx}{0.45\textwidth}{lXXXX}
	\textbf{Macchina} & \textbf{Costo (mo)} & \textbf{Danno} & \textbf{Gittata} & \textbf{Soldati}\\
\toprule
	Catapulta pesante & 800  & 6d6  & 60m   & 4\\
	Catapulta leggera & 5500 & 4d6  & 45m   & 2\\
	Ballista& 500  & 3d8  & 36m   & 1\\
	Ariete  & 1000 & 3d6  & -    & 10\\
	Torri da Assedio  & 2000 & -    & -    & 20\\
\end{tabularx}

\bigskip

\textbf{Strade Cittadine}\index{Strade Cittadine}

Le tipiche strade di una città sono strette e tortuose. La maggior parte delle vie cittadine è larga dai 3 ai 6 metri, mentre i vicoli vanno da una larghezza di 3 metri a una di soltanto 1,5 metri. Se il pavimento lastricato è in buone condizioni, è possibile muoversi normalmente, mentre le strade in brutte condizioni e gravemente rovinate vengono considerate equivalenti a detriti sparsi, e aumentano la DC delle prove di Acrobatica di 2.

Alcune città non hanno grandi viali d'accesso, specialmente quelle che sono cresciute gradualmente partendo come piccoli insediamenti. Le città che sono state progettate a tavolino, o che forse sono state consumate da un grave incendio che ha consentito alle autorità di costruire nuove strade su quelle che un tempo erano aree abitate, potrebbero disporre di alcune strade più grandi che le attraversano. Queste strade principali sono ampie 7,5 metri, e consentono ai carri di passare l'uno di fianco all'altro, con marciapiedi di 1,5 metri su entrambi i lati.

\textbf{Folla}: Le strade cittadine sono gremite di gente che va e viene, impegnata nelle varie faccende giornaliere. Nella maggior parte dei casi non è necessario includere ogni popolano di 1° livello sulla mappa quando si giunge a un combattimento sul viale principale della città.

E' sufficiente invece indicare quali zone sulla mappa sono occupati dalla folla. Se la folla vede qualcosa di pericoloso, si allontanerà alla velocità di 9 metri per round a conteggio di Iniziativa 0. Per entrare in contatto con la folla bisogna avere una distanza di mischia. La folla fornisce Copertura a chiunque riesca a entrarvi, consentendo una prova di Nascondersi (tra la folla..) e fornendo un bonus alla Difesa (copertura) e ai Tiri Salvezza su Riflessi.

\textbf{Dirigere la Folla}: è necessaria una prova di Diplomazia con DC 15 o di Intimidire con DC 20 per convincere una folla a spostarsi in una certa direzione, e la folla deve essere in grado di sentire o vedere il personaggio che effettua il tentativo. E' necessaria tutto un round per effettuare la prova di Diplomazia, mentre serve solo un'Azione per effettuare la prova di Intimidire.

Se due o più personaggi tentano di spingere la folla in due direzioni diverse, effettuano prove di Diplomazia o di Intimidire contrapposte per determinare a chi la folla darà ascolto. La folla ignorerà entrambi, se tutti e due i risultati delle prove dovessero essere inferiori alle DC sopra indicate.

\textbf{Tetti}: Per arrampicarsi su un tetto di solito è necessario scalare un muro, a meno che un personaggio non possa raggiungere un tetto saltando giù da una finestra, un balcone o un ponte più alto. I tetti piatti sono comuni solo nelle zone a clima caldo (la neve, accumulandosi, può far crollare un tetto piatto) e sono facili da percorrere correndo. Spostarsi sulla cima di un tetto richiede una prova di Acrobatica con DC 20. Spostarsi orizzontalmente su un tetto inclinato (muovendosi in parallelo alla sua cima, in pratica) richiede una prova di Acrobatica con DC 15. Spostarsi su e giù lungo un tetto inclinato richiede una prova di Acrobatica con DC 10.

Prima o poi un personaggio giungerà alla fine del tetto, e dovrà effettuare un lungo salto per passare al tetto successivo o per scendere a terra. La distanza che separa un tetto dal successivo di solito è di 3 metri, ma il tetto dall'altra parte potrebbe essere più alto o più basso di 1,5 metri, o alla stessa altezza. Si usano le indicazioni date per Acrobatica (il picco d'altezza in un salto in lungo è pari ad un quarto della distanza orizzontale) per determinare se il personaggio è in grado di effettuare un salto.

\textbf{Fognature}: Per entrare nelle fognature, i personaggi solitamente devono aprire una grata (1 round) e saltare in basso per 3 metri. Le fognature sono costruite esattamente come dei dungeon, con la differenza che il pavimento è scivoloso o ricoperto d'acqua. Le fognature sono anche simili ai dungeon per quello che riguarda le creature che è possibile incontrare al loro interno. Alcune città sono state costruite sulle rovine di civiltà più antiche, quindi le fognature potrebbero anche condurre a tesori e pericoli appartenenti a un'era passata.

\textbf{Edifici Cittadini}
La maggior parte degli edifici cittadini è divisa in tre categorie. Molti edifici in una città sono alti da due a cinque piani e sono costruiti l'uno di fianco all'altro per formare lunghe file, interrotte soltanto dalle vie principali o secondarie. Questi edifici a schiera solitamente ospitano un negozio a pianterreno, con uffici o appartamenti ai piani superiori.
Le locande, le imprese commerciali più ricche e i magazzini più grandi (oltre a eventuali mulini, concerie e altre attività che richiedano molto spazio) in genere sono grossi edifici indipendenti alti fino a cinque piani.
Infine, le abitazioni, i negozi, i magazzini e i depositi più piccoli sono dei semplici edifici di legno a un piano, specialmente nei quartieri più poveri.

\textbf{Illuminazione Cittadina} 
Se una città possiede grandi viali d'accesso, questi saranno illuminati da lanterne appese a un'altezza di circa 2 metri sui lati degli edifici. Queste lanterne sono poste a una distanza di 9 metri l'una dall'altra, quindi l'illuminazione in queste strade è praticamente continua. Le strade secondarie e i vicoli non sono illuminati; è consuetudine per i cittadini pagare un lanternaio che li accompagni, se devono uscire di notte.
I vicoli possono essere luoghi bui anche di giorno, grazie alle ombre degli edifici più alti circostanti. Un vicolo buio di giorno non è buio a sufficienza da poter conferire copertura completa ma leggera.

\end{multicols}

\pagebreak

\section{Avventure e Disastri}\index{Avventure}\index{Disastri}

\label{avventure-e-disastri}
\begin{enfasi}{
Per prima cosa, nessuno rimane indietro. (anonimo)}\end{enfasi}\medskip

\begin{multicols}{2}


\lettrine[lines=2, lhang=0.33, loversize=0.25, findent=1.5em]{I}{ disastri} naturali sono pericoli ambientali terrificanti che portano morte e devastazione. Quelli soprannaturali possono essere anche più distruttivi, poiché possono sfigurare per sempre un mondo. Un disastro è più simile ad un'avventura che ad un incontro, e non ha uno specifico Grado di Sfida. Piuttosto, ogni parte del disastro dovrebbe essere trattata come un incontro separato ideato con un grado di Sfida adeguato ai PG.

Sotto vengono presentate le regole per gestire gli effetti di tre diversi tipi di disastri, sia naturali che soprannaturali. Alcuni disastri si verificano rapidamente, come terremoti e tsunami, mentre altri procedono attraverso numerose fasi, come gli incendi forestali, i vulcani e le sollevazioni di non morti. Aggiustate lo schema dell'avventura per adattarlo al disastro, per permettere agli eventi di svolgersi nel corso di pochi minuti o molti giorni a seconda di quello che vi serve.

\textbf{Vulcani}\index{Vulcani}

Quando la crosta terrestre si rompe ed espelle il suo cuore fuso ha luogo uno dei disastri naturali più drammatici: l'eruzione di un vulcano. Le eruzioni vulcaniche offrono una vasta gamma di opzioni al Narratore, inclusi lava, bombe laviche, gas venefici e colate piroclastiche. I Narratore potrebbero anche considerare l'idea di far presagire una drammatica eruzione vulcanica (o draghi vulcanici) con pericoli preesistenti, come valanghe e terremoti minori.

\textbf{Lava}\index{Lava}

I flussi lavici generalmente sono associati alle eruzioni non esplosive e possono essere un elemento permanente dei vulcani attivi. Le colate laviche sono per lo più lente e si muovono a 4,5 metri per round (penalità azione movimento 1), ma quelle più calde sono rapide, e raggiungono i 12 metri per round (nessuna penalità azione movimento). La lava incanalata, come in un tubo lavico, è molto pericolosa, poiché si muove alla velocità di 36 metri per round (4 azioni di movimento a round) (un pericolo con grado di Sfida 6). Le creature raggiunte da una colata lavica devono superare un Tiro Salvezza su Riflessi con DC 20 o sono sommerse dalla lava. Il successo indica che sono a contatto con la Lava ma non Immerse.

\textbf{Bombe Laviche} (grado di Sfida 2 o 8)\index{Bombe Laviche}

Agglomerati di pietra fusa possono essere scagliati a molti chilometri da un vulcano che erutta, raffreddandosi in solida pietra prima di raggiungere il terreno. Una tipica bomba lavica colpisce un punto designato dal Narratore ed esplode in un raggio medio. Tutte le creature nell'area devono superare un Tiro Salvezza su Riflessi con DC 15 o subiscono 4d6 danni. Le creature che hanno Copertura o sono in grado di coprirsi (come con uno scudo) ottengono bonus +2 a questo tiro. A volte si formano bombe laviche molto grandi che infliggono 12d6 danni. Le bombe laviche normali hanno grado di Sfida 2, quelle grandi grado di Sfida 5.

\textbf{Gas Venefici} (grado di Sfida 5)\index{Gas Venefici}

Una delle minacce più insidiose di un vulcano è il gas tossico, spesso non notato tra il fuoco e la distruzione. Diversi tipi di vapori venefici scaturiscono da un'eruzione vulcanica, alcuni visibili, altri no. I gas venefici infliggono 1d3 danni alla Costituzione per round se inalati (Tempra DC 15 nega, la DC aumenta di 1 per ogni Tiro Salvezza precedente), e quelli visibili funzionano anche come Fumo Denso. Le nubi di gas venefici fluiscono verso il basso, e generalmente arrivano ad una altezza di 6 metri. Forti venti possono deviare le nubi di gas, così come alte barriere, a condizione che il gas abbia un altro posto dove andare.

\textbf{Colate Piroplastiche} (grado di Sfida 10)

Alcune eruzioni vulcaniche creano una devastante ondata di cenere ardente, gas bollenti e detriti vulcanici chiamata colata piroclastica che può viaggiare per chilometri. Una colata piroclastica viene trattata come una Valanga che viaggia a 150 metri per round, combinata con gli effetti dei gas venefici indicati sopra. Il contatto con i detriti roventi della colata infligge 2d6 danni da fuoco per round, mentre qualsiasi creatura seppellita dalla colata subisce 10d6 danni per round.

\textbf{Tsunami}\index{Tsunami}

Gli tsunami, talvolta attribuiti ad onde di marea, sono tremende ondate d'acqua causate da terremoti sottomarini, esplosioni vulcaniche, smottamenti o impatti di asteroidi. Gli tsunami non si possono individuare finché non raggiungono l'acqua poco profonda, quando la massa d'acqua forma una grande onda. A seconda dalle dimensioni dello tsunami e della pendenza della costa, l'onda può coprire qualsiasi distanza, dal centinaio di metri fino ad oltre un chilometro sulla terra ferma, lasciandosi dietro una scia di distruzione. L'acqua poi si ritira, trascinando via ogni sorta di detriti e creature fino in alto mare.

L'esatta devastazione causata è soggetta alla discrezione del Narratore, ma un tipico tsunami abbatte o sradica tutte le strutture temporanee o mal costruite sul suo percorso, distrugge circa il 25\% degli edifici ben costruiti (causando danni significativi a quelli che restano) e lascia le fortificazioni solide leggermente danneggiate. Almeno 1/4 della popolazione che vive nell'area (inclusi animali e mostri) muore nel disastro, trascinato in mare, affogato sulla spiaggia o seppellito sotto le macerie.

Una creatura può evitare di essere portata via dal mare con una prova di Nuotare con DC 25; altrimenti viene trascinata a 6d6 x 3 metri dalla riva. Le acque dopo uno tsunami sono sempre considerate agitate o tempestose, salvo influenze magiche. Una creatura coinvolta nel cedimento di un edificio subisce 6d6 danni (TS su Riflessi DC 15 dimezza), o la metà se la struttura è particolarmente piccola. c'è una probabilità del 50\% che la creatura venga sepolta (come per un Crollo), o che lo tsunami possa distruggere l'edificio, liberando la creatura dalle macerie.

\textbf{Sollevazione di Non Morti}\index{Non Morti}

Frutto di un'antica maledizione o di atti necromantici, uno dei disastri soprannaturali più terrificanti è la sollevazione di Non Morti: il morto che emerge dalla tomba per reclamare il vivo. Questo disastro può colpire qualsiasi area dove sono stati sepolti dei morti, non solo paesi e città. Più di un campo di battaglia ha visto sorgere una legione di rinsecchiti combattenti Non Morti. Le sollevazioni di Non Morti si svolgono ad ondate, con la tempistica che varia secondo le forze principali in gioco. Gli eventi possono succedersi nel corso di pochi giorni, con la devastazione di una città, o protrarsi per settimane con la popolazione terrorizzata che si rannicchia dietro porte sprangate e lotta per sopravvivere. Durante il giorno, spesso la vita ritorna ad una parvenza di normalità, poiché la luce del giorno sopprime temporaneamente il potere della non morte.

\textbf{I Morti Inquieti}

Nelle prime notti di una sollevazione di Non Morti, i morti recenti si rianimano come zombi. Quelli sepolti in terra consacrata non si rianimano, ma i corpi lasciati insepolti o in fosse comuni barcollano fuori per le strade, portando scompiglio. Inizialmente, solo alcuni cadaveri sono capaci di liberarsi dalle loro bare e tombe, ma ogni sera, il numero di cadaveri vivi aumenta. Quando giunge l'alba, i morti cercano la sicurezza nelle loro tombe o di altri luoghi nascosti. Chiunque venga colto dalla luce del giorno si agita Confuso finché non viene distrutto o raggiunge un rifugio. A discrezione del Narratore, cadaveri di non umanoidi possono risorgere come Non Morti nelle notti seguenti.

\textbf{Il Risveglio degli Scheletri}

Con l'avanzare della sollevazione, cadaveri sempre più vecchi si uniscono alle schiere dei Non Morti. Scheletri che recano tracce di vesti funebri marcite da tempo scavano con gli artigli una via d'uscita da cimiteri e cripte, ed agiscono con una malevolenza ed organizzazione raramente riscontrate tra i loro simili. I Non Morti rimangono privi di Intelligenza, ma il potere magico dietro all'incursione dona loro l'efficienza e l'acume tattico di un esercito di viventi. Gli Scheletri scovano armi e corazze con cui equipaggiarsi per la battaglia. L'élite degli Scheletri campioni guida le truppe, utilizzando Oggetti Magici trafugati da tombe abbandonate. Infine, anche Ghoul e Wight vagano in cerca di preda per le strade durante il buio, insieme ad altri Non Morti minori dotati di libero arbitrio

\textbf{Anime Perse}

Mentre la sollevazione raduna le forze, si risvegliano anche le anime inquiete di cadaveri da tempo ridotti in polvere. Fantasmi, Ombre, Wraith e persino Spettri sorgono per dare la caccia ai vivi. Alcuni Fantasmi potrebbero liberarsi dalla malevola influenza della sollevazione e dei personaggi intraprendenti potrebbero raccogliere preziose informazioni da questi spiriti inquieti.

L'infusione di energia negativa fortifica i Non Morti all'interno dell'area dell'incursione, concedendo i benefici di una Benedizione. Le aree una volta consacrate sono ora trattate come terreno normale, e possono fungere da nuove fonti di cadaveri per le armate Non Morte; il terreno santificato rimane inviolato.

Quando i Non Morti diventano più forti, l'ondata crescente di energia negativa avvicina il Piano delle Ombre, stingendo o ingrigendo i colori tranne durante le ore più brillanti del giorno. Anche i Non Morti più vulnerabili alla luce possono muoversi impunemente dal tardo pomeriggio alla mezza mattinata.

\textbf{Necropoli}

Il flusso di energia negativa è irreversibile, l'oscurità infine reclama l'area, coprendola con un'ombra perpetua.Il terreno santificato resta un raro santuario, ma solo finché non viene distrutto dalle forze malevoli forze esterne.

Gli eroi morti negli scontri ritornano come spaventosi generali Non Morti. I pochi superstiti viventi vengono assoggettati come schiavi. L'area diviene una città della morte o ne viene cominciata la costruzione se non esisteva o non è sopravvissuta alcuna città. I Non Morti dotati di libero arbitrio si radunano in questo nuovo santuario e solo gli eroi più grandi riescono a tornare da quest'area ormai avvizzita al mondo dei vivi.

\end{multicols}

\pagebreak

\section{Avventure nei Dungeon}\index{Dungeon}

\begin{enfasi}{
Linux is user friendly. It's just very picky about who its friends are (anonimo)

\medskip

Il dungeon è inclinato. Le creature sono infuriate perché non riescono a giocare a biglie (Dungeon Keeper 2,Videogioco, 1999)}\end{enfasi}\medskip

\label{avventure-nei-dungeon}
\begin{multicols}{2}

\lettrine[lines=2, lhang=0.33, loversize=0.25, findent=1.5em]{D}{i}  tutti i luoghi strani che un avventuriero può esplorare, nessuno è più letale di un dungeon. Questi labirinti, pieni di trappole mortali, mostri affamati e tesori meravigliosi, provano ogni Abilità dei personaggi. Queste regole si possono applicare a qualsiasi tipo di dungeon, dal relitto di una nave ad un vasto complesso di grotte sotterranee.

\textbf{Tipi di Dungeon}

I quattro tipi base di dungeon sono definiti dal loro stato attuale. Molti dungeon sono varianti di questi tipi base o combinazioni di più tipi. Occasionalmente, antichi dungeon vengono usati ripetutamente da nuovi abitanti per scopi diversi.

\textbf{Struttura in Rovina}: Un tempo abitato, questo luogo è ora abbandonato (completamente o in parte) dai suoi creatori originari ed è occupato da altre creature. Molte creature sotterranee vanno alla ricerca di costruzioni sotterrane ed abbandonate in cui stabilire le loro tane. Qualsiasi trappola che possa essere esistita è stata probabilmente già rimossa o attivata, ma è possibile trovare bestie erranti.

\textbf{Struttura Occupata}: Questo dungeon viene ancora utilizzato. Delle creature (di solito intelligenti) ancora lo abitano, anche se potrebbero non essere i creatori del dungeon. Una struttura occupata potrebbe essere una casa, una fortezza, un tempio, una miniera attiva, una prigione, un quartier generale.

Questo tipo di dungeon è meno probabile che abbia trappole o bestie erranti, e più probabilmente dispone di guardie organizzate, sia stazionarie che di pattuglia. Le trappole e le bestie erranti che si possono incontrare sono spesso sotto il controllo degli occupanti. Le strutture occupate dispongono di arredo adatto agli abitanti, così come decorazioni, riserve di cibo, e la possibilità per gli abitanti di muoversi.

Gli abitanti possono disporre anche di un sistema di comunicazione, e quasi sempre controllano almeno un accesso verso l'esterno.

Alcuni dungeon sono parzialmente occupati e parzialmente vuoti o in rovina. In questi casi, gli occupanti di solito non sono gli originari costruttori del luogo, ma bensì un gruppo di creature intelligenti che hanno stabilito la loro base, tana o fortificazione all'interno del dungeon abbandonato.


\begin{center}
	\includegraphics[width=0.8\linewidth]{immagini/dungeon.png}
\end{center}

\textbf{Riparo Sicuro}: Quando qualcuno vuole proteggere una cosa, spesso la seppellisce sottoterra. Che l'oggetto che vuole proteggere sia un favoloso tesoro, un artefatto proibito o il cadavere di un uomo importante, questi oggetti di valore vengono posti all'interno di un dungeon e circondati da barriere, trappole e guardiani.

Il dungeon del tipo riparo sicuro è quello che avrà più trappole e meno bestie erranti. E' normalmente costruito in base alla funzionalità piuttosto che all'aspetto, anche se a volte viene decorato con statue e pareti dipinte, specie per le tombe di personaggi importanti.

A volte, però, una sala del tesoro o una cripta vengono costruite in modo da ospitare guardiani viventi. Il problema con questa strategia è che occorre tenere in vita le creature tra un tentativo di intrusione e un altro. La magia è di solito la soluzione migliore per rifornire di cibo e acqua queste creature. I costruttori di tombe e sepolcri, di solito, pongono non morti e costrutti, che non hanno bisogno di sostentamento o di riposo, a protezione dei loro dungeon. Le trappole magiche possono attaccare gli intrusi convocando mostri nel dungeon che scompaiono quando terminano il loro compito.

\begin{center}
	\includegraphics[width=0.8\linewidth]{immagini/avventure_dungeon.png}
\end{center}
\begin{center}
	\textit{The Red Romance Book, Illustration by H.J. Ford}
\end{center}

\textbf{Complesso di Caverne Naturali}: Le caverne sotterranee offrono riparo a qualsiasi tipo di creatura delle profondità. Create naturalmente e collegate da un sistema di passaggi labirintici, queste caverne mancano di qualsiasi parvenza di ordine, logica o decorazioni. Senza alcuna potenza intelligente che lo abbia costruito, questo tipo di dungeon è quello che ha minori probabilità di presentare trappole o porte.

Molteplici varietà di funghi vivono nelle caverne, a volte crescendo fino a formare enormi foreste di funghi e vesce, dove si aggirano predatori sotterranei si aggirano a caccia di chi si nutre di questi vegetali. Alcune varietà di funghi producono un bagliore fosforescente in grado di fornire al complesso di caverne naturali una propria limitata fonte di illuminazione. In altre zone, l'uso di incantesimi di Luce Diurna può garantire luce sufficiente per la crescita di piante verdi.

Spesso, un complesso di caverne naturali è collegato ad altri tipi di dungeon, essendo stato scoperto quando è stato costruito il dungeon artificiale. Un complesso di caverne può collegare due dungeon indipendenti, producendo a volte uno strano ambiente misto. Un complesso di caverne naturali unito a un altro dungeon, spesso, offre un percorso che le creature sotterranee possono usare per raggiungere un dungeon artificiale e popolarlo.

\textbf{Terreno del Dungeon}

Le regole seguenti riguardano i terreni di base che si possono trovare in un dungeon.

\textbf{Pareti}

A volte, pareti in mattoni (pietre accatastate una sopra l'altra solitamente, ma non sempre, tenute insieme con la calce) dividono i dungeon in corridoi e stanze. Le pareti dei dungeon possono anche essere scolpite nella nuda roccia, ottenendo così un aspetto scalpellato, oppure possono essere composte di pietra liscia e semplice come si trova nelle caverne naturali. Le pareti dei dungeon sono difficili da danneggiare o da sfondare, ma di solito sono facilmente scalabili.

\end{multicols}
\bigskip

\textbf{Tabella: Pareti}\index{Tabella Pareti}
\medskip

\begin{tabularx}{0.95\textwidth}{XllllX}
	\textbf{Tipo di Parete} & \textbf{Spessore} & \textbf{Sfondare} & \textbf{Durezza} & \textbf{Punti Ferita} & \textbf{DC per Scalare}\\
\toprule
	Mattoni  & 30 cm& 35   & 8 & 90& 20\\
	Mattoni superiori  & 30 cm& 35   & 8 & 120    & 25\\
	Mattoni rinforzati & 30   & 45   & 8 & 180    & 20\\
	Pietra Scolpita    & 90   & 50   & 8 & 540    & 25\\
	Pietra grezza & 150 cm    & 65   & 8 & 900    & 25\\
	Ferro    & 7.5 cm    & 30   & 10& 90& 25\\
	Carta    & variabile & 1    & --& 1 & 30\\
	Legno    & 15 cm& 20   & 5 & 60& 21\\
\end{tabularx}

\bigskip

\begin{multicols}{2}

\textbf{Pareti in Mattoni}: Il tipo più comune di parete per un dungeon, le pareti in mattoni di solito sono spesse almeno 30 centimetri. Spesso queste antiche pareti presentano fori e fessure, all'interno dei quali possono annidarsi fanghiglie e piccole creature, che aspettano lì le loro prede. Le pareti di mattone sono in grado di bloccare tutti i rumori, tranne quelli più forti. E' necessaria una prova di Arrampicarsi con DC 20 per muoversi lungo una parete in mattoni.

\textbf{Pareti in Mattoni di Qualità Superiore}: A volte le pareti in mattoni sono costruite meglio (più lisce, con pietre meglio incastrate e meno danneggiate) e occasionalmente queste pareti di qualità superiore sono coperte da calcina o stucco. Queste pareti sono spesso abbellite da dipinti, bassorilievi o altre decorazioni. Le pareti in mattoni di qualità superiore non sono più difficili da danneggiare delle normali pareti in mattoni, ma sono più difficili da Arrampicarsi (DC 25).

\textbf{Pareti rinforzate} Queste sono pareti in mattoni con sbarre di ferro su uno o entrambi i lati, o inserite all'interno della parete stessa per rinforzarla. La Durezza della parete rinforzata resta la stessa, ma i Punti Ferita vengono raddoppiati e la DC per la prova di Forza per sfondarla viene incrementata di 10.

\textbf{Pareti di Pietra Scolpita}: Queste pareti generalmente si trovano in stanze o passaggi scavati nella nuda roccia. La ruvida superficie di una parete scolpita presenta minuscole sporgenze su cui possono crescere funghi e crepe all'interno delle quali possono vivere parassiti, pipistrelli o serpi sotterranee.

Quando una parete di questo tipo ha un "altro lato" (la parete separa due stanze in un dungeon), la parete è spessa almeno 90 centimetri; se fosse più sottile rischierebbe di far crollare tutto perché non sarebbe in grado di sostenere il peso della volta di pietra. E' necessaria una prova di Arrampicarsi con DC 25 per scalare una parete di pietra scolpita.

\textbf{Pareti di Pietra Grezza}: Queste superfici sono irregolari e raramente piatte. Sono lisce al tocco ma piene di minuscoli buchi, alcove nascoste e sporgenze a varie altezze. Di solito sono bagnate o perlomeno umide, in quanto le caverne naturali sono in genere il prodotto di infiltrazioni d'acqua. Quando una parete di questo tipo da un "altro lato", la parete è di solito spessa almeno 150 centimetri.

è necessaria una prova di Arrampicarsi con DC 15 per muoversi lungo una parete di pietra grezza.

\textbf{Pareti di Ferro}: Queste pareti sono poste all'interno dei dungeon intorno a luoghi importanti come le sale del tesoro.

\textbf{Pareti di Carta}: Le pareti di carta sono l'opposto di quelle di ferro, utilizzate come schermi per impedire la vista ma nulla più.

\textbf{Pareti di Legno}: Le pareti di legno si trovano spesso come recenti aggiunte a dungeon più antichi, utilizzate per creare recinti per animali, depositi, o anche solo per dividere in una serie di stanze più piccole una più grande.

\textbf{Pareti Trattate Magicamente}: Queste pareti sono più forti della media, con una Durezza maggiore, con più Punti Ferita e per sfondarle bisogna superare una DC maggiore. La magia può di solito raddoppiare la Durezza e i Punti Ferita della parete e aggiungere fino a +20 alla sua DC per sfondarla. Una parete trattata magicamente ottiene anche un Tiro Salvezza contro Incantesimi che potrebbero avere effetto su di essa, con il bonus al Tiro Salvezza pari a 2 + metà del livello dell'incantatore della magia che rinforza la parete. Creare una parete magica richiede il talento Creare Oggetti Meravigliosi e la spesa di 1.500 mo per ogni sezione di 3 per 3 metri.

\textbf{Pareti con Feritoie}: Le pareti con feritoie possono essere costruite con qualsiasi materiale resistente, ma sono di solito fatte in mattoni, pietra scolpita o legno. Permettono ai difensori di scagliare frecce o quadrelli da balestra contro gli intrusi restando dietro la relativa protezione di un muro. Gli arcieri dietro alle feritoie godono di una Copertura superiore che fornisce loro bonus +8 alla Difesa, bonus +4 ai Tiri Salvezza su Riflessi.

\textbf{Pavimenti}

Così come per le pareti, esistono molti tipi di pavimenti per dungeon.

\textbf{Lastricato}: Come le pareti in mattoni, i pavimenti possono essere composti da pietre incastrate tra loro. Sono di solito piene di fessure e solitamente appena livellate. Fanghiglie e muffe crescono all'interno di queste fessure. In certi casi l'acqua scorre in piccoli scoli attraverso le pietre o forma pozze stagnanti. Il lastricato è il tipo di pavimento più comune nei dungeon.

\textbf{Lastricato Irregolare}: Col passare del tempo, alcuni pavimenti possono diventare talmente irregolari da richiedere una prova di Acrobatica con DC 10 per correre o Caricare sulla loro superficie. Coloro che falliscono la prova non possono muoversi durante quel round. Pavimenti così pericolosi dovrebbero essere in realtà l'eccezione e non la regola.

\textbf{Pavimento di Pietra Scolpita}: Ruvidi e irregolari, i pavimenti scolpiti nella pietra sono di solito coperti da pietre smosse, ghiaia, polvere e altri detriti. Una prova di Acrobatica con DC 10 è necessaria per correre o Caricare su un simile pavimento. Un fallimento significa che il personaggio può ancora agire, ma non può correre o Caricare in quel round.


\begin{center}
	\includegraphics[width=1\linewidth]{immagini/pavimento_grey.jpg}
\end{center}


\textbf{Pietrisco Scarso}: Piccoli e sparuti detriti sono presenti a terra. Un pavimento su cui sia presente del pietrisco scarso aggiunge 2 alla DC delle prove di Acrobatica.

\textbf{Pietrisco Denso}: Il terreno è ricoperto di detriti di tutte le dimensioni. Entrare in una zona di mischia ricoperto di pietrisco denso costa 2 azioni di movimento. Un pavimento cosparso di pietrisco denso aggiunge 5 alla DC delle prove di Acrobatica, e aggiunge 2 alla DC delle prove di Consapevolezza (Muoversi Silenziosamente)

\textbf{Pavimento di Pietra Liscia}: Pavimenti lisci, perfetti e a volte anche levigati si trovano solo nei dungeon creati da costruttori capaci e attenti.

\textbf{Pavimento di Pietra Naturale}: Il pavimento di una caverna naturale è irregolare quanto le pareti. E' difficile che queste caverne presentino ampie superfici piane; è più probabile che i loro pavimenti siano disposti su più livelli.

Alcune superfici adiacenti potrebbero variare in elevazione di appena 30 centimetri, cosicché lo spostamento da un punto all'altro non sia più difficile del salire un gradino di una scala, ma in certi punti il pavimento potrebbe scendere o salire di diverse decine di centimetri, obbligando il personaggio a una prova di Arrampicarsi per spostarsi da una superficie a un'altra.

A meno che non ci sia un percorso scavato dal tempo o ben battuto il terreno è considerato difficile e quindi il movimento è dimezzato, la DC delle prove di Acrobatica è aumentata di 5. La Carica e la corsa in questi ambienti sono impossibili, tranne che sui percorsi in questione.

\textbf{Scivoloso}: Acqua, ghiaccio, melma o sangue possono rendere qualunque pavimento descritto in questa sezione più insidioso. I pavimenti scivolosi aumentano la DC delle prove di Acrobatica di 5.

\textbf{Grata}: Una grata spesso copre una fossa o una zona al di sotto del pavimento principale. Le grate sono di solito costruite in ferro, ma quelle più grosse potrebbero essere anche fatte di tronchi d'albero rinforzati. Molte grate hanno cardini che permettono l'accesso alla zona sottostante (queste grate possono essere chiuse a chiave come una porta), mentre altre sono fisse e create per non poter essere spostate. Una tipica grata di ferro spessa 2,5 centimetri ha 25 Punti Ferita, Durezza 10, e DC 27 per sfondarla o smuoverla.

\textbf{Sporgenze}: Le sporgenze permettono alle creature di camminare al di sopra di un'area sottostante. Spesso sono disposte intorno a fosse, lungo il corso di fiumi sotterranei, come balconate che circondano un'ampia stanza oppure forniscono una posizione dalla quale gli arcieri possono appostarsi per attaccare i nemici dall'alto.

Le sporgenze strette (di ampiezza inferiore a 30 centimetri) richiedono a coloro che vi si muovono sopra delle prove di Acrobatica. Un fallimento implica che il personaggio che si stava muovendo cade dalla sporgenza.

A volte le sporgenze hanno una ringhiera. In questi casi i personaggi ottengono Bonus +5 alle prove di Acrobatica per muoversi lungo la sporgenza. Un personaggio vicino alla ringhiera ha Bonus +2 alla propria prova contrapposta di Forza per evitare di essere spinto giù dalla sporgenza.

Le sporgenze a volte possono anche essere delimitate da balaustre alte 60-90 centimetri. Simili muri forniscono Copertura da aggressori entro distanza 3 metri dall'altro lato del muro, ammesso che il bersaglio sia più vicino alla balaustra di chi attacca.


Pavimenti Trasparenti: I pavimenti trasparenti, fatti di vetro rinforzato o di materiali magici permettono di osservare un ambiente pericoloso dall'alto. I pavimenti trasparenti sono di solito posti al di sopra di pozze di lava, arene, tane di mostri e stanze di tortura.Possono essere usati dai difensori
per sorvegliare un'area.

\textbf{Pavimenti Scorrevoli}: Un pavimento scorrevole è un tipo di botola, creato per essere spostato e rivelare qualcosa che si trova al di sotto. In genere un pavimento scorrevole si muove tanto lentamente che chiunque vi si trovi sopra può evitare di cadere nell'apertura, purché abbia spazio per spostarsi. Se un pavimento di questo tipo scorre tanto velocemente che c'è la possibilità che un personaggio cada in quello che si trova sotto di esso (lance acuminate, una vasca con olio bollente, o una pozza infestata da squali) allora è una trappola.

\textbf{Pavimenti Trappola}: Questi pavimenti sono stati progettati per diventare di colpo pericolosi. Con l'applicazione della giusta quantità di peso o l'azionamento di una leva nelle vicinanze, spuntoni sbucano dal pavimento, fiammate o sbuffi di vapore partono da fori nascosti, o l'intero pavimento si muove. Questi strani pavimenti si trovano di solito dentro alle arene, progettati per rendere i combattimenti più appassionanti e letali. Questo tipo di pavimento è costruito nello stesso modo di una trappola.

\textbf{Porte} \index{Porte}Le porte all'interno dei dungeon sono ben più che semplici entrate o uscite. Spesso possono essere dei veri e propri incontri. Le porte dei dungeon si presentano in tre tipi basilari: di legno, di pietra e di ferro.


\end{multicols}

\bigskip

\textbf{Tabella: Porte}\index{Tabella Porte}

\bigskip

\begin{tabular}{llllll}
	\textbf{Tipo di porta} & \textbf{Spessore tipico (cm)} & \textbf{Durezza} & \textbf{Punti Ferita} & \multicolumn{2}{c}{\textbf{DC per sfondare}} \\

    &&   &   & Bloccata  & Chiusa a chiave\\
    \toprule
	Legno semplice    & 2.5  & 5 & 10& 13   & 15\\
	Legno buono  & 3.75 & 5 & 15& 16   & 18\\
	Legno robusto& 5    & 5 & 20& 23   & 25\\
	Pietra  & 10   & 8 & 60& 28   & 28\\
	Ferro   & 5    & 10& 60& 28   & 28\\
	Saracinesca di legno   & 7.5  & 5 & 30& 25{*}& 25{*}\\
	Saracinesca di ferro   & 5    & 10& 60& 25{*}& 25{*}\\
	Serratura    & -    & 15& 30& -    & -\\
	Cardini & -    & 10& 30& -    & -\\
\end{tabular}

\medskip

{*} DC per sollevare. Usate la voce appropriata di porta per sfondare.

\begin{multicols}{2}

\bigskip

\textbf{Porte di Legno}: Costruite con spesse assi inchiodate, a volte rinforzate con sbarre di ferro (poste anche per impedire le deformazioni prodotte dall'umidità dei dungeon), quelle di legno sono il tipo più comune di porta. Le porte di legno variano per durezza: possono essere semplici, buone o robuste. Le porte semplici (DC 15 per sfondarle) non sono progettate per tenere alla larga assalitori motivati.

Le porte di buona fattura (DC 18 per sfondarle), sebbene forti e resistenti, non sono comunque progettate per subire una grande quantità di danni. Le porte robuste (DC 25 per sfondarle) sono rivestite in ferro e sono delle barriere discretamente resistenti contro coloro che cerchino di oltrepassarle. Cardini di ferro sorreggono la porta, e di solito un anello circolare posto al centro serve ad aprirla.A volte, al posto di un anello, una porta dispone di una sbarra di ferro su uno o entrambi i lati che funziona come maniglia.

Nei dungeon abitati queste porte sono di solito ben tenute (non bloccate) e non chiuse a chiave, anche se le zone importanti probabilmente saranno chiuse a chiave.

\textbf{Porte di Pietra}: Costruite da blocchi di pietra solida, queste porte pesanti e poco maneggevoli sono spesso pensate in modo da ruotare su se stesse quando vengono aperte, anche se i nani e altri abili artigiani sono in grado di costruire cardini forti abbastanza da sostenere il peso di una porta di pietra.

Le porte segrete nascoste lungo una parete di pietra sono solitamente di pietra. Altrimenti, le porte di questo tipo sono studiate per diventare resistenti barriere che proteggono qualsiasi cosa si trovi al di là di esse. Di conseguenza si trovano spesso chiuse a chiave o sbarrate.

\textbf{Porte di Ferro}: Arrugginite ma resistenti, le porte di ferro in un dungeon sono dotate di cardini come quelle di legno. Queste porte sono le porte più resistenti del tipo non magico. Sono di solito chiuse a chiave o sbarrate.

\begin{center}
	\includegraphics[width=0.85\linewidth]{immagini/porta_grey.png}
\end{center}

\textbf{Sfondare}: Le porte dei dungeon possono essere chiuse a chiave, munite di trappole, rinforzate, sbarrate, sigillate magicamente o, a volte, semplicemente bloccate.

Tutti, ad eccezione dei personaggi più deboli, riusciranno a buttar giù una porta con un pesante attrezzo come un maglio, e numerosi incantesimi ed oggetti magici possono offrire ai personaggi un modo facile per superare una porta chiusa.

\textbf{DC 10 o inferiore}: Una porta che chiunque può sfondare.

\textbf{DC 11--15}: Una porta che una persona forte dovrebbe sfondare con un solo tentativo, e che una persona di forza media potrebbe avere qualche speranza di abbattere in un solo colpo.

\textbf{DC 16--20}: Una porta che praticamente chiunque potrebbe sfondare, avendo a disposizione il tempo necessario.

\textbf{DC 21--25}: Una porta che solo una persona forte o molto forte ha una speranza di sfondare, e probabilmente non al primo tentativo.

\textbf{DC 26 o superiore}: Una porta che solo una persona dotata di una forza eccezionale può avere una qualche speranza di sfondare.

\textbf{Serrature}: Le porte dei dungeon sono spesso chiuse a chiave e così torna utile l'Abilità Disattivare Congegni. Le serrature sono di solito costruite sulle porte, sul bordo opposto ai cardini o dritte nel centro della porta. Le serrature costruite dentro le porte di solito controllano una sbarra di ferro che si estende dalla porta dentro il muro che la sostiene, o una sbarra di ferro o di legno massiccio scorrevole che si prolunga dietro tutta la porta.

Al contrario, i lucchetti non sono costruiti dentro la porta ma di solito scorrono tra due anelli, uno sulla porta e uno sul muro. Serrature più complesse, come quelle a combinazione o quelle ad enigma, sono di solito costruite dentro la porta stessa.

Siccome queste serrature senza chiave sono più grandi e complesse, di solito si trovano solo sulle porte resistenti (robuste porte di legno, di pietra o di ferro).
La DC per scassinare una serratura con una prova di Disattivare Congegni spesso ricade tra 20 e 30, anche se esistono serrature con DC maggiori o inferiori. Una porta può disporre di più di una serratura, ognuna delle quali da aprire separatamente.

Le serrature sono spesso dotate di trappole, di solito aghi avvelenati che scattano all'infuori per pungere le dita del ladro.

\textbf{Spaccare una serratura}

Una porta speciale potrebbe avere una serratura senza chiave, ma che richiede che venga indovinata la giusta combinazione delle leve vicine o vengano premuti nell'ordine corretto i simboli su un pannello per riuscire ad aprirla.

\textbf{Porte Bloccate}: I dungeon sono spesso luoghi umidi, e in alcuni casi le porte rimangono bloccate, in modo particolare se sono fatte di legno. Di solito si suppone che all'incirca il 10\% delle porte di legno e il 5\% delle altre porte siano bloccate. Questi valori possono essere raddoppiati (al 20\% e 10\% rispettivamente) nel caso di dungeon da tempo abbandonati o trascurati.

\textbf{Porte Sbarrate}: Quando un personaggio cerca di sfondare una porta sbarrata, è la qualità della sbarra che fa la differenza, non il materiale della porta in sé. Sfondare una porta chiusa da una sbarra di legno richiede una prova di Forza con DC 25, e la DC sale a 30 nel caso di una sbarra metallica.

I personaggi possono attaccare la porta e distruggerla, lasciando la sbarra appesa nel passaggio sgombro.

\textbf{Sigilli Magici}: Incantesimi messi su una porta possono rendere ostico l'attraversamento di una porta.

Una porta su cui è stato lanciato un blocco magico si considera chiusa anche se non ha fisicamente una serratura. E' necessario un incantesimo che scassina o Distruggi magie oppure una prova riuscita di Forza per oltrepassare una porta chiusa in questo modo.

\textbf{Cardini}: La maggior parte delle porte è dotata di cardini. Ovviamente le porte scorrevoli non lo sono (queste sono piuttosto dotate di solchi sul pavimento, che permettono loro di scorrere a lato con facilità).

\textbf{Cardini Standard}: Questi cardini sono di metallo e tengono unita la porta al suo sostegno o alla parete. Ricordarsi che la porta si apre verso il lato dove si trovano i cardini (quindi se i cardini sono dal lato dei PG, la porta si aprirà verso di loro; altrimenti si aprirà verso l'altra direzione).

Gli avventurieri possono rimuovere i cardini uno alla volta superando varie prove di Disattivare Congegni (solo se, naturalmente, sono davanti al lato della porta su cui si trovano i cardini). Una simile azione ha una DC di 20, in quanto molti dei cardini sono arrugginiti o bloccati.

Spaccare un cardine è difficile. La maggior parte ha Durezza 10 e 30 Punti Ferita. La DC per spaccare un cardine è la stessa che serve per abbattere la porta

\textbf{Cardini a Inserimento}: Questi cardini sono molto più complessi e si trovano solo in zone di eccellente costruzione. Questi cardini sono costruiti dentro la parete e permettono alla porta di aprirsi in entrambe le direzioni. I personaggi non possono raggiungere i cardini per rimuoverli a meno che non sfondino il sostegno della porta o la parete. I cardini a inserimento si trovano di solito sulle porte di pietra, ma a volte si vedono anche su porte di legno o di ferro.

\begin{center}
	\includegraphics[width=0.9\linewidth]{immagini/cardini.png}
\end{center}


\textbf{Perni}: I perni non sono veri cardini, ma semplici pioli che si protendono dal lato superiore e inferiore della porta e si infilano dentro i buchi nel suo sostegno, permettendole di girare. I vantaggi dei perni è che non possono essere rimossi come i cardini e che sono facili da realizzare. Lo svantaggio è che siccome la porta gira sul suo centro di gravità (di solito nel mezzo), nulla più grosso di metà dell'ampiezza della porta vi può passare attraverso.

Le porte dotate di perni sono di solito di pietra e spesso anche abbastanza larghe per ovviare allo svantaggio. Un'altra soluzione è quella di piazzare il perno verso un'estremità e fare la porta più spessa da quella parte e più sottile dall'altra, in modo che si apra più o meno come una porta normale.

Le porte segrete all'interno di muri spesso ruotano, in quanto la mancanza di cardini rende più facile occultare la presenza della porta. I perni permettono anche a oggetti come una libreria di essere usati come porte segrete.

\textbf{Porte Segrete}: Camuffata da comune porzione di muro (o di pavimento o di soffitto), da libreria, da focolare, da fontana, una porta segreta porta ad un passaggio segreto oppure ad una stanza.

Qualcuno che stia esaminando la zona può trovare una porta segreta (se ne esiste una) con una prova riuscita di Consapevolezza (con DC 20 per una porta segreta comune e DC 30 per una porta molto ben nascosta).

Molte porte segrete richiedono un metodo speciale per essere aperte, come un bottone nascosto o una piastra a pressione. Le porte segrete possono aprirsi come porte comuni, girare su un perno, scorrere, sprofondare, sollevarsi o anche calare come un ponte levatoio.

Un costruttore potrebbe piazzare una porta segreta molto bassa vicino al pavimento oppure molto in alto su un muro, in modo da rendere più difficile sia il rinvenimento che l'utilizzo della porta.

\begin{center}
	\includegraphics[width=0.9\linewidth]{immagini/arcoserpenti.png}
	
	\textit{Henry Justice Ford}
\end{center}


\textbf{Porte Magiche} Incantata dal costruttore originario, una porta può apostrofare gli esploratori invitandoli a non proseguire. Potrebbe essere protetta dai danni, con una Durezza maggiore o un numero maggiore di Punti Ferita, oltre che un bonus al Tiro Salvezza migliorato. Una porta magica potrebbe non condurre allo spazio che si trova dietro di essa, ma essere in realtà un portale verso un luogo molto distante o addirittura verso un altro piano di esistenza. Altre porte magiche potrebbero aver bisogno di una parola d'ordine o di chiavi speciali per aprirsi.

\textbf{Saracinesche}: Queste porte speciali sono fatte con aste di ferro o di spesso legno rinforzato che calano da un recesso nella parte superiore di un arco. A volte una saracinesca dispone di barre orizzontali a formare una griglia, altre volte no. Sollevate di solito con un argano o simile macchinario, le saracinesche possono esser fatte scendere in fretta, e le sbarre terminano in punte per scoraggiare chiunque dal passarci sotto (o dal tentare di attraversarle in corsa mentre calano). Una volta scesa, una saracinesca si chiude, a meno che non sia così grande che nessuna persona normale sarebbe in grado di sollevarla. In ogni caso, sollevare una tipica saracinesca richiede una prova di Forza con DC 25.

\textbf{Pareti, Porte ed azioni di Individuazione}

Le pareti di pietra, di ferro e le porte di ferro sono generalmente sufficientemente spessi da bloccare la maggior parte delle Divinazioni. Le pareti di legno, le porte di legno e di pietra in genere non sono sufficientemente spesse da fare altrettanto. Tuttavia, una porta segreta di pietra costruita in un muro e spessa come il muro stesso (almeno 30 centimetri) bloccherà la maggior parte di queste Azioni.

\textbf{Scale} Il metodo più tradizionale per collegare differenti livelli di un dungeon è attraverso le scale. Un personaggio può salire o scendere una scala come parte del suo movimento senza penalità ma non può correre. Aumentate la DC di qualsiasi prova di Acrobatica effettuate su una scala di 4. Alcune scale, particolarmente ripide, vengono trattate come terreno difficile.

\subsubsection{Pericoli nei Dungeon}

Nei dungeon e nelle caverne oltre ai mostri ci sono anche altri pericoli tra crolli, muffe, funghi e altro.

\textbf{Crolli e Cedimenti (grado di Sfida 8)}

I crolli e i cedimenti nei tunnel sono estremamente pericolosi. Non solo gli esploratori di dungeon corrono il rischio di essere schiacciati da tonnellate di pietra, ma anche, qualora dovessero sopravvivere, di rimanere bloccati sotto un mucchio di detriti o di essere impossibilitati a raggiungere un'uscita.

Un crollo seppellisce chiunque si trovi nel mezzo della zona sepolta, e quindi i detriti che rotolano via infliggeranno danni a tutti coloro che si trovano nelle zone periferiche alla zona sepolta. Un tipico corridoio soggetto a un crollo potrebbe disporre di una zona sepolta con raggio 3 metri e una zona di scorrimento con raggio di mischia all'estremità di quella sepolta.

Un soffitto pericolante può essere identificato con una prova di Ingegneria con DC 20 o Professione Muratore con DC 20. Un Nano può effettuare questa prova semplicemente passando entro 3 metri di distanza da un soffitto pericolante.

Un soffitto pericolante può crollare sotto l'impatto di una grossa forza. Un personaggio può provocare un crollo distruggendo la metà dei pilastri che reggono il soffitto.

I personaggi che si trovano nella zona sepolta subiscono 8d6 danni, o danni dimezzati se superano un Tiro Salvezza su Riflessi con DC 15. A quel punto sono sepolti. I personaggi nella zona di scorrimento subiscono 3d6 danni, o nessun danno se superano un Tiro Salvezza su Riflessi con DC 15. I personaggi che si trovano nella zona di scorrimento, sono anch'essi sepolti, se falliscono il Tiro Salvezza.

I personaggi sepolti subiscono 1d6 danni non letali per ogni minuto che rimangono sotto le macerie. Se un personaggio in queste condizioni cade privo di sensi, deve effettuare una prova di Costituzione con DC 15. Se il personaggio fallisce la prova, inizia a subire 1d6 danni letali al minuto fino a quando non viene liberato o muore.

I personaggi che non sono stati sepolti possono estrarre i loro compagni da sotto le macerie. In 1 minuto, usando solo le mani, un personaggio può spostare una quantità di roccia e detriti pari a cinque volte il proprio limite di carico pesante. La quantità di roccia smossa che riempie un'area di mischia pesa all'incirca 1 tonnellata (1.000 kg). Equipaggiato con gli strumenti adatti, come un piccone, un piede di porco, o una pala, uno scavatore può impiegare la metà del tempo che impiegherebbe facendolo a mano. Si potrebbe anche concedere a un personaggio sepolto di liberarsi da solo superando una prova di Forza con DC 25.

\textbf{Fanghiglie, Muffe e Funghi}

Negli umidi e oscuri recessi dei dungeon, le muffe e i funghi prosperano, temete le colonne di muffa! Per quanto riguarda gli incantesimi ed altri effetti speciali, tutte le fanghiglie, le muffe e i funghi sono considerati vegetali. Come le trappole, le fanghiglie e le muffe pericolose sono dotate di un grado di Sfida, e i personaggi guadagnano Punti Esperienza per averle incontrate.

Una lucida melma organica ricopre qualsiasi cosa che rimanga per troppo tempo immersa nell'oscurità e nell'umidità dei dungeon. Questo tipo di fanghiglia, benché possa essere repellente, non è pericoloso. Le muffe e i funghi abbondano nei luoghi bui, freddi e umidi. Sebbene alcuni siano innocui quanto le normali fanghiglie dei dungeon, altri sono alquanto pericolosi. Funghi commestibili, vesce, lieviti, muffe e altri tipi di funghi fibrosi, bulbosi o intere distese di spore fungine possono essere rinvenuti nella maggior parte dei dungeon. Di solito sono innocui e spesso sono anche commestibili (anche se la maggior parte è poco invitante o ha uno strano sapore).

\begin{center}
	\includegraphics[width=0.9\linewidth]{immagini/funghi.png}
	
	\textit{Sono luminosi al buio, credetemi..}
\end{center}


\textbf{Boleto Stridente}\index{Boleto Stridente}: Questi funghi viola di grandezza umana emettono un suono penetrante che dura 1d3 round ogni volta che c'è un movimento o una sorgente di luce entro raggio 3 metri. Questo grido rende impossibile sentire altri suoni o rumori entro raggio di mischia. Il suono attira le creature nelle vicinanze che sono disposte ad investigare. Alcune creature che vivono vicino ai boleti stridenti hanno imparato che il rumore significa molto spesso cibo.

\textbf{Fanghiglia Verde}\index{Fanghiglia Verde} (grado di Sfida 4): Questo pericolo dei dungeon è una varietà insidiosa della normale fanghiglia. La fanghiglia verde divora la carne e i materiali organici che vi entrano in contatto, ed è addirittura capace di dissolvere i metalli. Di un verde splendente, bagnata e appiccicosa, si distribuisce a chiazze su pareti, pavimenti e soffitti e si riproduce consumando materiale organico. Si lascia cadere dalle pareti e dai soffitti quando individua del movimento (e possibile nutrimento) sotto di sé.

La fanghiglia verde infligge 1 danno alla Costituzione per ogni round in cui divora la carne. Al primo round di contatto, la fanghiglia può essere asportata da una creatura (con la probabile distruzione dell'oggetto utilizzato per asportarla), ma dopo il primo round deve essere congelata, bruciata o tagliata (infliggendo danni anche alla sua vittima) per essere rimossa. Tutto ciò che infligge danni da fuoco o da freddo, la luce solare o un incantesimo di rimuovi malattia distruggono una chiazza di fanghiglia verde. Nel caso di legno o metallo, la fanghiglia verde infligge 2d6 danni per round, ignorando la Durezza del metallo ma non quella del legno. Non danneggia la pietra.

\textbf{Fungo Fosforescente}\index{Fungo Fosforescente}: Questo strano fungo sotterraneo emana una debole luminescenza violacea che illumina le caverne e i passaggi sotterranei come una candela. Rare macchie di questo fungo illuminano come una torcia.

\textbf{Muffa Gialla} \index{Muffa Gialla}(grado di Sfida 6): Se disturbata, nel raggio di 3 metri rilascia una nube di spore velenose. Tutti coloro entro raggio di 3 metri dalla muffa devono superare un Tiro Salvezza su Tempra con DC 15 o subiscono 1d3 danni a Costituzione. Un altro Tiro Salvezza su Tempra con DC 15 è necessario una volta per round per i successivi 5 round o per evitare di subire altri 1d3 danni a Costituzione. Un Tiro Salvezza riuscito blocca questo effetto. Il fuoco distrugge la muffa gialla, mentre la luce solare la rende inerte.

\textbf{Muffa Marrone} \index{Muffa Marrone}(grado di Sfida 2): La muffa marrone si nutre di calore, estraendolo da tutto ciò che la circonda. Di solito si presenta in chiazze con diametro di dimensione mischia e la temperatura attorno alla muffa risulta sempre fredda in un raggio di 3 metri. Le creature viventi entro una distanza di mischia da essa subiscono 3d6 danni non letali da freddo. Se viene portata una fonte di fuoco entro mischia dalla muffa questa raddoppia immediatamente le proprie dimensioni. I danni da freddo, come quelli inflitti da un cono di freddo, la distruggono all'istante.

\end{multicols}

\pagebreak

\section{Pericoli in Avventura}\index{Pericoli in Avventura}


\begin{enfasi}{
Un'avventura è un risultato ragionevole. Due sono meglio, tre meritano di essere tramandate, e quattro... nessuno potrà mai contestare quattro avventure. (John Steinbeck)
		
\medskip
		

Corre meno pericoli colui che, anche se è al sicuro, sta in guardia. (Publilio Siro)} \end{enfasi}\medskip

\label{pericoli-in-avventura}

\begin{multicols}{2}

\lettrine[lines=2, lhang=0.33, loversize=0.25, findent=1.5em]{I}{l} mondo è pieno dì pericoli oltre che di draghi ed immondi famelici. I pericoli sono minacce basate sulle peculiarità della zona che hanno molto in comune con le trappole, ma che di solito fanno parte del posto anziché venir costruite. I pericoli si dividono in tre categorie principali: ambientali, viventi e magici.

I pericoli ambientali includono frane, incendi e simili. I pericoli viventi includono creature che pur non essendo considerate mostri, rappresentano una minaccia per gli avventurieri incauti, come fanghiglie, funghi e muschi. I pericoli magici sono i più imprevedibili e possono essere residui di esperimenti arcani, strane radiazioni sotterraneo o antichi incantesimi falliti.

\includegraphics[width=0.8\linewidth]{immagini/boscopericoli.png}


\textbf{Antidweomer (grado di Sfida 6)}\index{Antidweomer}

Zona di entropia magica che distruggono le magie, gli antidweomer si formano sui siti di grandi duelli magici, attraverso la distruzione di potenti artefatti o da vortici di energia mistica ai margini delle zone di antimagia. Le dimensioni variano da piccole bolle di appena pochi metri fino a grandi aree delle dimensioni di una città.

Una prova riuscita di Arcana con DC 20 rivela la vicinanza di un antidweomer con un formicolio nell'aria. Una magia attiva portata in un antidweomer potrebbe venir dissolta, e qualsiasi incantesimo lanciato al suo interno è soggetta ad un contro incantesimo immediato. Se la prova di magia riesce di 10 o più l'incantesimo riesce a passare l'antidweomer, altrimenti l'incantesimo svanisce in un guizzo di colore.

Se l'incantesimo fallisce il rilascio di energia magica infligge 1d6 danni in un'esplosione in un raggio di 3 metri centrata su chi ha tentato l'incantesimo; un Tiro Salvezza su Riflessi a DC 15 permetti di dimezzare questo danno.

Una magia manifestata da un oggetto fallisce sempre.

Se più scoppi sovrapposti colpiscono lo stesso bersaglio, si applica solo quello più dannoso. Una magia che ha resistito ad un tentativo di dissoluzione, non viene influenzato nuovamente a meno che non esca e rientri nell'antidweomer.

Gli antidweomer più potenti sono ancora più distruttivi. Ogni +1 di incremento del grado di Sfida aumenta la difficoltà di magia di 1 e la DC del Tiro Salvezza per i danni dell'esplosione di 1.

\medskip
\textbf{Aria Viziata (grado di Sfida 1 o 4)}\index{Aria Viziata}

Un pericolo invisibile, le sacche di gas sono un rischio per minatori, speleologi e avventurieri che investigano nelle caverne. I gas ininfiammabili come il diossido di carbonio o l'azoto hanno grado di Sfida 1 e richiedono una prova di Sopravvivenza con DC 25 per essere notati.

Le creature che respirano quell'aria devono superare un Tiro Salvezza su Tempra (DC 15 +1 per ogni tiro precedente) ogni ora o diventano Affaticate. Una volta Affaticate, iniziano a Soffocare Lentamente. Le creature che trattengono il fiato possono evitare questi effetti.

I vapori infiammabili come il gas di carbone sono molto più pericolosi (grado di Sfida 4). Questo gas sostituisce l'aria respirabile nei polmoni, provocando affaticamento: inoltre, qualsiasi fiamma aperta o scintilla causa un'esplosione che infligge 6d6 danni (TS su Riflessi con DC 15 dimezza) a chi è nella caverna o entro 3 metri da un ingresso. Il fuoco brucia l'ossigeno nell'aria, rendendola irrespirabile per 2d4 minuti. Dopo un'esplosione, il gas infiammabile generalmente impiega molti giorni per ritornare a livelli pericolosi.

\medskip
\textbf{Parassiti}\index{Parassiti}

Parassiti come cercaorecchie o larve necrofaghe provocano parassitosi, un tipo di Afflizione simile alle Malattie. Le parassitosi possono essere guarite solo attraverso trattamenti specifici; indipendentemente da quanti Tiri Salvezza si effettuano, la parassitosi continua ad affliggere il bersaglio. Anche se un Rimuovi Malattia (o un effetto simile) uccide immediatamente una parassitosi, l'immunità alle Malattie non offre protezione, dato che è causata da parassiti.

\medskip
\textbf{Cercaorecchie (grado di Sfida 5)}\index{Cercaorecchie}

I cercaorecchie sono minuscoli vermi bianchi che vivono nel legno marcio o altri detriti organici. Si possono notare con una prova di Consapevolezza (DC 15). Altrimenti, una creatura vivente che frughi nella loro tana si trasferisce inavvertitamente addosso uno o più cercaorecchie, i quali poi cercano una zona calda sul corpo della creatura, prediligendo il condotto uditivo, e li depongono 2d8 uova prima di morire.

Le uova si schiudono 4d6 ore dopo e le larve divorano la carne intorno. Alla morte del loro ospite, i vermetti strisciano fuori e ne cercano uno nuovo.

Rimuovi Malattia uccide tutti i cercaorecchie o le uova non ancora schiuse su un ospite. Alcuni cercaorecchie preferiscono vivere nel legno corrotto, spesso nascondendosi nelle porte dei sotterranei. I piccoli fori lasciati da questa variante sono molto difficili da notare (Consapevolezza DC 20).

\medskip
\textbf{Cercaorecchie}

Tipo: Parassitosi

TS: Tempra DC 15

Insorgenza: 4d6 ore

Frequenza Tiro Salvezza: 1 ogni ora

Effetti: 1d3 a Costituzione se fallisce il Tiro Salvezza

\medskip
\textbf{Cristalli Mnemonici (grado di Sfida 3)}\index{Cristalli Mnemonici}

I cristalli mnemonici sono grandi (3-12 metri d'altezza) grappoli di cristalli di quarzo viola che irradiano un'aura di alterazione forte. Per identificarli occorre una prova di Arcana con DC 25.

I cristalli mnemonici cumulano energia magica per crescere e difendersi, risucchiando gli incantesimi preparati degli incantatori che devono effettuare un Tiro Salvezza su Volontà con DC 22 ogni round mentre sono entro 3 metri dai cristalli.

Se il tiro fallisce, perdono 1 slot di incantesimo (un incantesimo in meno lanciabile) a disposizione. Danneggiando o rompendo i cristalli, le magie assorbite vengono espulsi con un'esplosione di energia mentale che infligge 2d3 danni alla Saggezza a tutti coloro che si trovano entro 6 metri di raggio.

I cristalli mnemonici sono molto fragili (Durezza 0, 1 Punto Ferita).
In aree ricche di cristalli, le creature che vi passano attraverso devono superare una prova di Acrobatica con DC 10 per evitare di camminarci sopra o sfiorarli rompendoli.

\medskip
\textbf{Larve Necrofaghe (grado di Sfida 4)}\index{Larve Necrofaghe}

Una volta occupato un corpo vivente, le larve scavano verso il cuore, il cervello e altri organi interni chiave dell'ospite, provocandone infine la morte.

Nel primo round di parassitosi, applicando del fuoco nel foro di ingresso si possono uccidere le larve e salvare l'ospite, ma questo subisce 1d6 danni da fuoco.

Anche estrarle funziona, ma più a lungo le larve restano nell'ospite, più danni provoca questo metodo. Per estrarre le larve occorre un'arma tagliente ed una prova di Pronto Soccorso con DC 20, infliggendo 1d6 danni per ogni round che l'ospite è stato afflitto da parassitosi. Se la prova di Pronto Soccorso riesce una larva viene rimossa. Rimuovi Malattia uccide tutte le larve necrofaghe presenti in un ospite.

\medskip
\textbf{Larve Necrofaghe}

Tipo: Parassitosi

TS: Tempra DC 17

Insorgenza: immediata

Frequenza: 1/round

Effetti: 1d2 danni a Costituzione per larva

\medskip
\textbf{Minerale Magnetizzato (grado di Sfida 2)}\index{Minerale Magnetizzato}

Le strane energie del mondo sotterraneo possono caricare pietre e vene di minerali con potenti campi magnetici, creando un pericolo per chi porta o indossa metalli ferrosi. Tutte le cose di ferro o acciaio portate entro raggio di 3 metri dal minerale sono trascinate verso di esso.

\begin{center}
	\includegraphics[width=0.8\linewidth]{immagini/neodimio.png}
	
	\textit{Neodimio}
\end{center}


Le creature Piccole vengono trascinate anche con 7,5 kg di metallo, quelle Grandi solo con 30 kg. Per creature di altre taglie, il peso cambia in base alle regole della Capacità di Trasporto. Le creature con indosso armature metalliche subiscono una penalità, chi è colpito è trascinato fino a 9 metri, subisce 2d6 danni per l'impatto con la roccia ed è considerato afferrato. Liberare un oggetto colpito richiede una prova di Forza DC 20-25

\medskip
\textbf{Polla Maledetta (grado di Sfida 3)}\index{Polla Maledetta}

I prolungati effetti di antiche maledizioni o l'energia nociva che si propaga da un oggetto magico maledetto sommerso possono trasformare una semplice polla d'acqua in un rischioso pericolo magico. Una polla maledetta attira i passanti nelle sue profondità attraverso l'illusione (TS su Volontà con DC 16 per dubitare) di uno sfavillante tesoro sul fondo profondo 3 metri. Qualsiasi creatura che giunga al tesoro attiva la maledizione.

Una creatura all'interno della polla deve superare un Tiro Salvezza su Volontà con DC 16 o è colpita dalla maledizione, che distorce la sua percezione della polla. L'acqua sembra addensarsi in un viscoso sapropelite (NdA: anche sapropel o melma fetida, usato in geologia per indicare una fanghiglia di colore nerastro, pastosa e più o meno costipata, originatasi per il depositarsi in acque stagnanti o poco mosse di resti di organismi mescolati a gusci calcarei o silicei di microrganismi e a sostanze argillose), mentre la polla sembra raggiungere una profondità di 12 metri.

Le prove di Nuotare nella polla subiscono penalità -10, la velocità viene ridotta alla metà del normale a causa di questi effetti e la difficoltà degli incantesimi aumenta di 5.

Una polla maledetta irradia una forte magia, e può essere distrutta da Distruzione Magie o da Rimuovi Maledizione.

\medskip
\textbf{Quercia Velenosa (grado di Sfida 1 o 3)}\index{Quercia Velenosa}

Il contatto con una quercia velenosa (grado di Sfida 1) causa una dolorosa eruzione cutanea pruriginosa che rende la vittima Inferma finché i danni non guariscono. Un pieno contatto col corpo o l'inalazione del fumo di una quercia velenosa che brucia potrebbero essere fatali (grado di Sfida 3). Una prova di Natura (o Erboristeria) con DC 15 rivela i pericoli insiti nella pianta all'apparenza innocua. Questo pericolo può essere usato anche per piante nocive simili (edera velenosa, sommaco velenoso od ortiche pungenti, ma quest'ultime non sono pericolose quando bruciano).

\begin{center}
	\includegraphics[width=0.8\linewidth]{immagini/ederavelenosa.png}
\end{center}


\textbf{Quercia Velenosa}

Tipo: Veleno, contatto

TS: Tempra DC 13

Insorgenza: 1 ora

Effetti: 1d4 danni a Destrezza, la creatura è Inferma finché i danni non guariscono

Cura: 1 TS

\end{multicols}

\pagebreak

\subsection{Avventure e Trappole}\index{Trappole}

\begin{enfasi}{
Chi pone la trappola sempre allo stesso posto non prenderà alcun'iguana. (proverbio africano)}\end{enfasi}\medskip


\begin{multicols}{2}

Quasi ovunque si può incontrare una trappola. Le trappole possono essere di natura magica o meccanica. Le trappole meccaniche comprendono fosse, frecce, massi che cadono, stanze piene d'acqua, lame rotanti e qualsiasi altra cosa che dipenda da un meccanismo per operare. Le trappole magiche sono congegni magici trappola o incantesimi trappola. I congegni magici trappola quando attivati generano gli effetti di un incantesimo. Gli incantesimi trappola sono incantesimi come glifo di interdizione e simbolo che funzionano come trappole.

\textbf{Le Trappole nel Gioco}
Quando gli avventurieri si imbattono in una trappola, dovreste sapere come la trappola si attiva e cosa faccia, oltre ad avere un'idea di come i personaggi possano individuare la trappola e riuscire a disarmarla o evitarla.

\subsubsection{Attivare una Trappola}
La maggior parte delle trappole si attivano quando una creatura giunge in un punto o tocca qualcosa che il creatore della trappola voleva proteggere. Normali sistemi di attivazione sono pedane a pressione o false sezioni di pavimento, tirare un cavo, girare una maniglia e usare la chiave sbagliata nella serratura. Le trappole magiche spesso si attivano quando una creatura entra in un'area o tocca un oggetto. Alcune trappole magiche (come l'incantesimo glifo di interdizione) possiedono delle condizioni di attivazione più complesse, tra cui l'impiego di parole d'ordine per impedire l'attivazione della trappola.

\subsubsection{Individuare e Disabilitare una Trappola}
Di solito, alcuni elementi di una trappola sono ben visibili a un'attenta ispezione.

La descrizione della trappola specifica le prove e le DC necessarie per individuarla, disabilitarla o entrambe. Un personaggio che cerchi attivamente una trappola può tentare una prova di \textbf{Sopravvivenza} contro la DC della trappola. 

Il Narratore può anche comparare la DC per individuare la trappola contro il punteggio di Sopravvivenza (a tiro 8) dei personaggi al fine di determinare se un membro della compagnia noti la trappola. Se gli avventurieri notano la trappola prima di attivarla, potrebbero tentare di disarmarla, in maniera permanente o abbastanza a lungo da permettergli il passaggio.

Il Narratore potrebbe richiedere una prova di Disattivare Congegni usando gli attrezzi da scasso, per svolgere l'intervento necessario.

Qualsiasi personaggio può tentare una prova di Intelligenza (oppure Arcana, a punteggio almeno 1) per individuare o disarmare una trappola magica, in aggiunta a qualsiasi altra prova indicata nella descrizione della trappola. Le DC sono sempre le stesse, quale che sia la prova effettuata. Inoltre, l'incantesimo dissolvi magie ha una probabilità di disabilitare la maggior parte delle trappole magiche.

La descrizione di una trappola magica fornisce la DC per la prova di caratteristica da effettuare quando si usa dissolvi magie.

Nella maggior parte dei casi, la descrizione della trappola è abbastanza chiara da far sì che il Narratore possa giudicare se le azioni di un personaggio localizzino o sventino la trappola.

Usate il buon senso, attingendo alla descrizione della trappola per determinare ciò che accade. Nessun progetto di trappola potrebbe mai essere in grado di anticipare ogni possibile azione che i personaggi possano tentare di effettuare.

Il Narratore dovrebbe permettere a un personaggio di scoprire una trappola senza dover effettuare prove di competenza se una sua azione riuscirebbe palesemente a rivelare la presenza della trappola.

Sventare le trappole può essere un pò più complicato. Prendiamo il caso di un forziere difeso da una trappola. Se il forziere viene aperto senza tirare sulle due maniglie messe ai lati, un meccanismo posto al suo interno spara una raffica di aghi avvelenati verso chiunque vi si trovi di fronte.

Dopo aver ispezionato il forziere e aver effettuato qualche prova, i personaggi non sono ancora certi che sia munito di trappole. Piuttosto che aprire il forziere, essi puntano uno scudo di fronte a esso e lo aprono a distanza usando un'asta di ferro. In questo caso, la trappola si attiva, ma la raffica di aghi viene sparata contro lo scudo senza nuocere a nessuno.

Le trappole sono spesso progettate con meccanismi che permettono di disarmarle o oltrepassarle.

\subsubsection{Effetti delle Trappole}
Gli effetti delle trappole possono essere da semplici inconvenienti a letali. La descrizione di una trappola specifica cosa accade quando viene attivata.
Il bonus di attacco di una trappola, la DC del Tiro Salvezza per resistere ai suoi effetti, e il danno che infligge possono variare in base alla pericolosità della trappola.

Usare la tabella DC dei Tiri Salvezza e Bonus di Attacco delle Trappole e la tabella Gravità del Danno per Livello come suggerimenti sui tre livelli di gravità delle trappole.

\medskip

\begin{center}
	\includegraphics[width=0.9\linewidth]{immagini/medusa.png}
\end{center}


\textbf{Tabella: DC dei Tiri Salvezza e Bonus di Attacco delle Trappole}\index{Tabella DC dei Tiri Salvezza e Bonus di Attacco delle Trappole}

\medskip

\begin{tabularx}{0.45\textwidth}{XXX}
Pericolosità della Trappola	&DC Tiro Salvezza	& Bonus di Attacco\\
\toprule
Contrattempo	&13-14&	+4 a +6\\
Pericolosa	&16-20&	+8 a +10\\
Mortale	&21-26	&+12 a +15\\
\end{tabularx}

\medskip

\textbf{Tabella: Gravità del Danno per Livello}\index{Tabella Gravità del Danno per Livello}

\medskip

\begin{tabularx}{0.45\textwidth}{XlXX}
Livello Personaggio	&Contrattempo	&Pericolosa	&Mortale\\
\toprule
1°-4°	&1d10&	2d10	&4d10\\
5°-10°	&2d10&	4d10	&10d10\\
11°-16°	&4d10&	10d10	&18d10\\
17°-20°	&10d10	&18d10	&24d10\\
\end{tabularx}

\medskip

\subsubsection{Trappole Complesse}
Le trappole complesse funzionano come le trappole normali, eccetto che una volta attivate eseguono una serie di azioni ogni round. 

Una trappola complessa trasforma il processo dell'affrontare una trappola in qualcosa di più simile a un incontro di combattimento. Quando si attiva una trappola complessa, tirare la sua iniziativa.

La descrizione della trappola comprende un bonus di iniziativa. Durante il suo turno, la trappola si attiva di nuovo, effettuando spesso un'azione, che sia un attacco, un effetto che muta nel tempo, una sfida dinamica. Altrimenti, la trappola complessa può essere individuata e disabilitata nei soliti modi.

\subsubsection{Trappole di Esempio}
\textbf{Ago Avvelenato}

Trappola meccanica

Un ago avvelenato è nascosto all'interno della serratura di un forziere, o altro oggetto che si possa aprire. Aprire il forziere senza la chiave adeguata farebbe scattare l'ago, che dispensa una dose di veleno.

Quando la trappola viene attivata, l'ago si estende per 7,5 centimetri dalla serratura. Una creatura a gittata subisce 1 danno perforante e 11 (2d10) danni da veleno, e deve superare un Tiro Salvezza su Tempra con DC 20 o prendere -4 al Tiro per Colpire e -4 alle prove di Competenza per 1 ora.

Il personaggio che superi una prova di Sopravvivenza con DC 22, può dedurre la presenza della trappola dalle modifiche apportate alla serratura per ospitare l'ago. Una prova superata di Disattivare Congegni facendo uso di attrezzi da scasso, disarma la trappola rimuovendo l'ago dalla serratura. Una prova fallita per scassinare la serratura fa scattare la trappola.

\medskip

\textbf{Dardi Avvelenati}

Trappola meccanica

Quando una creatura calpesta una pedana a pressione nascosta, dei dardi avvelenati vengono sparati da un meccanismo a molla o da tubi pressurizzati astutamente nascosti all'interno delle pareti circostanti. Un'area potrebbe presentare più pedane a pressione, ciascuna collegata alla propria serie di dardi.


I minuscoli fori nelle pareti sono celati da polvere e ragnatele, oppure astutamente celati tra i bassorilievi, murali o affreschi che adornano la stanza. La DC della prova per notarli (Sopravvivenza) è 18. 

Il personaggio che superi una prova di Sopravvivenza con DC 18, può dedurre la presenza della pedana a pressione nascosta dalle differenze nella pavimentazione di cui è composta rispetto al resto del pavimento.

Incuneare una punta di ferro o altro oggetto sotto la pedana a pressione previene l'attivazione della trappola. Riempire i fori di tessuto o cera impedisce la fuoriuscita dei dardi contenuti all'interno.

La trappola si attiva quando più di 10 chili di peso vengono posti sulla pedana a pressione, facendo così sparare quattro dardi. Ogni dardo effettua un attacco a distanza con un bonus di attacco +10 contro un bersaglio casuale entro 3 metri dalla pedana a pressione (la visuale non ha alcun impatto su questo tiro di attacco).

Se non ci sono bersagli nell'area, il dardo non colpisce nulla. Un bersaglio colpito subisce 2 (1d4) danni perforanti e deve effettuare un Tiro Salvezza su Tempra con DC 18 e subire 11 (2d10) danni da veleno se lo fallisce, o la metà di questi danni se lo supera.


\medskip

\textbf{Fosse}

Trappola meccanica

Presentiamo di seguito quattro tipi base di fosse.


\medskip

\textit{Fossa Semplice}

La fossa semplice è un buco scavato nel terreno. Il buco è coperto da un grosso tessuto ancorato ai margini della fossa e mimetizzato con terra e detriti.
La DC per notare la fossa è 12. Chiunque metta piede sul tessuto cade all'interno della buca e si tira dietro il tessuto, subendo danni in base alla profondità della fossa (di solito 3 metri, ma alcune fosse sono più profonde).


\medskip

\textit{Fossa Nascosta}

Questa fossa possiede una copertura fatta di materiale identico a quello del pavimento circostante.
Superando una prova di Consapevolezza con DC 18 si nota l'assenza di tracce nella sezione di pavimento che forma la copertura della fossa. 

È necessario superare una prova di Sopravvivenza con DC 18 per confermare che quella sezione di pavimento copra in realtà una fossa.

Quando una creatura mette piede sulla copertura, questa si spalanca come una botola, facendo precipitare l'intruso nella fossa sottostante. La fossa è profonda solitamente tra i 3 e i 6 metri, ma può esserlo anche di più.

Una volta che la fossa è stata individuata, uno spuntone di ferro o simile oggetto può essere conficcato tra la copertura della fossa e il terreno circostante per impedire che la copertura si apra, rendendo sicuro il passaggio. La copertura può anche essere tenuta chiusa magicamente tramite l'incantesimo serratura arcana o magie simili.

\medskip
\textit{Fossa a Scatto}

Questa fossa è identica alla trappola fossa nascosta, con un'eccezione fondamentale: la botola che copre la fossa nasconde un meccanismo a molla. Dopo che una creatura è caduta nella fossa, la copertura si richiude di scatto per intrappolare la vittima al suo interno.

È necessario superare una prova di Forza con DC 20 per aprire a forza la copertura. La copertura può essere anche distrutta. Un personaggio all'interno della fossa può anche tentare di disabilitare il meccanismo a molla dall'interno superando una prova di Disattivare Congegni on DC 18 e utilizzando attrezzi da scasso, purché possa raggiungere e vedere il meccanismo in questione. In alcuni casi, un altro meccanismo fa riaprire la fossa. 

\medskip

\textit{Fossa con Spuntoni}

La fossa è una fossa semplice, nascosta o a scatto, sul cui fondo si trovano delle punte di legno o degli spuntoni di ferro. Una creatura che caschi nella fossa subisce 11 (2d10) danni perforanti dagli spuntoni, oltre al danno da caduta.

Versioni più crudeli di questa trappola sono munite di veleno cosparso sulle punte collocate in fondo alla fossa. In quel caso, chiunque subisca danni perforanti dagli spuntoni deve anche effettuare un Tiro Salvezza su Tempra con DC 16 e subire 22 (4d10) danni da veleno se lo fallisce, o la metà di questi danni se lo supera.
 

\medskip

\textbf{Rete che Casca}

Trappola meccanica

Questa trappola usa un cavetto per liberare una rete appesa al soffitto.

Il cavetto è collocato a 7 centimetri dal terreno e si estende tra due colonne o alberi. La rete è nascosta da ragnatele o fogliame. La DC (Sopravvivenza) per notare il cavetto e la rete è 15. Una prova superata di Disattivare Congegni con DC 20 usando gli attrezzi da scasso disabilita il cavetto.

Un personaggio privo degli attrezzi da scasso può tentare comunque la prova con -1d6 usando un'arma o un attrezzo affilati. Se la prova fallisce, la trappola si attiva.

Quando la trappola viene attivata, la rete viene rilasciata coprendo un'area quadrata di 3 metri di lato. Tutte le creature nell'area vengono intrappolate dalla rete e sono intralciate, mentre quelle che falliscono un Tiro Salvezza su Tempra, con modificatore Forza, con DC 13 cadono anche prone.

Una creatura può usare 2 Azioni per effettuare una prova di Forza DC 13, liberando se stessa o un'altra creatura a portata se la supera.

La rete ha Difesa 10 e 20 punti ferita. infliggere 5 danni taglienti alla rete ne distrugge una sezione quadrata di 1,5 metri di lato, liberando qualsiasi creatura intrappolata in quella sezione.


\medskip

\textbf{Sfera Rotolante}

Trappola meccanica

Quando 10 o più chili vengono posti sulla pedana a pressione della trappola, una botola nascosta nel soffitto si apre, rilasciando una sfera di 3 metri di diametro interamente fatta di pietra.


Superando una prova di Sopravvivenza con DC 20 un personaggio può notare la botola e la pedana a pressione. Se un esame del pavimento è accompagnato da una prova superata di Sopravvivenza con DC 20, rivelerà la presenza della pedana a pressione tramite la differenza di struttura della pavimentazione che la accomoda. La stessa prova effettuata mentre si controlla il soffitto, rivelerà la presenza di una botola. Incuneare uno spuntone di ferro o un altro oggetto sotto la pedana a pressione impedirà l'attivazione della trappola.

L'attivazione della sfera fa sì che tutte le creature presenti tirino per l'iniziativa. La sfera tira l'iniziativa con un bonus di +8. 

Durante il suo turno, la sfera si muove di 18 metri in linea retta. La sfera può muoversi attraverso lo spazio di una creatura, e le creature possono muoversi attraverso lo spazio che occupa, considerandolo terreno difficile.

Ogni qualvolta la sfera entri nello spazio di una creatura o una creatura entri nel suo spazio mentre la sfera sta rotolando, la creatura deve superare un Tiro Salvezza su Riflessi con DC 15 o subire 55 (10d10) danni contundenti e cadere prona.

La sfera si ferma quando colpisce un muro o una barriera simile. Non può girare gli angoli, ma gli abili costruttori di sotterranei incorporano lievi curve e svolte curvilinee nei passaggi limitrofi che permettono alla sfera di continuare a muoversi.

Con 2 Azioni, una creatura entro 1 metro dalla sfera può tentare di rallentarla superando una prova di Forza con DC 20. Se la prova viene superata, la velocità della sfera viene ridotta di 4,5 metri. Se la velocità della sfera scende a 0, arresta il movimento e non è più una minaccia.


\medskip

\textbf{Soffitto che Crolla}

Trappola meccanica

Questa trappola usa un cavetto per fare crollare i sostegni che sorreggono una sezione instabile di soffitto.

Il cavetto è collocato a 7 centimetri dal terreno e si estende tra i due sostegni. La DC (Sopravvivenza) per notare il cavetto è 13. Una prova superata di Disattivare Congegni con DC 20 usando gli attrezzi da scasso disabilita il cavetto. 

Un personaggio privo degli attrezzi da scasso può tentare comunque la prova con -1d6 usando un'arma o un attrezzo affilati. Se la prova fallisce, la trappola si attiva.

Chiunque ispezioni i sostegni può facilmente dedurre che sono solo appoggiati. Con un'azione, il personaggio può far cadere un sostegno e attivare la trappola.

Il soffitto sopra il cavetto è in cattivo stato, e chiunque possa vederlo può capire che rischia di crollare. Quando la trappola viene attivata, il soffitto instabile crolla. Tutte le creature nell'area sotto la sezione instabile devono effettuare un Tiro Salvezza su Riflessi con DC 20, subendo 22 (4d10) danni contundenti se lo falliscono o la metà di questi danni se lo superano. Una volta attivata la trappola, il pavimento dell'area è pieno di macerie e diventa terreno difficile.


\medskip

\textbf{Statua Soffia Fuoco}

Trappola magica

Questa trappola si attiva quando un intruso calpesta una pedana a pressione nascosta, liberando una vampata di fiamme magiche da una statua vicina.

La DC (Sopravvivenza) per notare la pedana a pressione o segni di bruciature sul pavimento e le pareti è 20. Un incantesimo o altro effetto che può percepire la presenza di magia, come individuazione del magico, rivela un'aura magica di invocazione intorno alla statua. 

La trappola si attiva quando più di 10 chili di peso vengono posti sulla pedana a pressione, facendo sì che dalla statua scaturisca un cono di fuoco di 9 metri. Tutte le creature nel cono devono effettuare un Tiro Salvezza su Riflessi con DC 17, subendo 22 (4d10) danni da fuoco se lo falliscono o la metà di questi danni se lo superano.

Infilare uno spuntone di ferro o altro oggetto sotto la pedana a pressione impedisce alla trappola di attivarsi. Un dissolvi magie (DC 17) lanciato sulla statua distrugge la trappola.


\medskip

\textbf{Trappola ad incantesimo}

Trappola magica

Le trappole di cui sopra possono essere dotate di un incantesimo che si attiva con la trappola.
I Tiri Salvezza per resistere all'incantesimo sono i medesimi dell'incantesimo lanciato.

\bigskip

\subsubsection{Altri esempi di trappole}

Sono qui presentata ulteriori trappole per la vostra gioia.


\medskip

\textbf{Piccola legenda}:

Grado di Sfida: indica quale è il grado di sfida della trappola

Tipo: se la trappola è di tipo meccanico o magico

DC Sopravvivenza: quale è la prova e difficoltà per rivelare la trappola

DC Disattivare Congegni: quale è la prova e difficoltà per disattivare la trappola. Se la prova riesce di 5 o più il personaggio può decidere se riattivare la trappola una volta evitata, altrimenti è disinnescata.

Attivatore: se si attiva a contatto o distanza

Ripristino: se è possibile ripristinare la trappola una volta scattata

Effetto: quale è l'effetto della trappola


\medskip


	\textbf{Dardo Avvelenato}
	
	grado di Sfida: 1 
	
	Tipo: meccanico 
	
	DC Sopravvivenza: 20 
	
	DC Disattivare Congegni: 20 
	
	Attivatore: contatto 
	
	Ripristino: nessuno 
	
	Effetto: Attacco a distanza 12 metri +10 (1d3 di danno più Bava fermentata di Lucos)
	

	\textbf{Freccia}
	
	grado di Sfida: 1 
	
	Tipo: meccanico 
	
	DC Sopravvivenza: 20 
	
	DC Disattivare Congegni: 20 
	
	Attivatore: contatto 
	
	Ripristino: nessuno 
	
	Effetto: Attacco a distanza 12 metri +15 (1d8+1/×3)
	

	\textbf{Fossa}
	
	grado di Sfida: 1 
	
	Tipo: meccanico 
	
	DC Sopravvivenza: 20 
	
	DC Disattivare Congegni: 20 
	
	Attivatore: posizione 
	
	Ripristino: manuale 
	
	Effetto: fossa profonda 3 metri (2d6 danni da caduta) 
	
	TS: Riflessi DC 20 evita 
	
	Bersaglio: bersagli multipli (tutti i bersagli raggio di 3 metri)
	

	\textbf{Lama Falciante}
	
	grado di Sfida: 1 
	
	Tipo: meccanico 
	
	DC Sopravvivenza: 20 
	
	DC Disattivare Congegni: 20 
	
	Attivatore: posizione 
	
	Ripristino: manuale 
	
	Effetto: Attacco in mischia +10 (1d8+1/×3) 
	
	Bersaglio: bersagli multipli (tutti i bersagli in una linea entro 3 metri)
	

	\textbf{Fossa con Spuntoni}
	
	grado di Sfida: 2 
	
	Tipo: meccanico 
	
	DC Sopravvivenza: 20 
	
	DC Disattivare Congegni: 20 
	
	Attivatore: posizione 
	
	Ripristino: manuale 
	
	Effetto: fossa profonda 3 metri m (1d6 danni da caduta) + spuntoni (Attacco in mischia +10, 1d4 spuntoni per bersaglio per 1d4+2 danni ciascuno) 
	
	TS: Riflessi DC 20 evita 
	
	Bersaglio: bersagli multipli (tutti i bersagli in un quadrato di 3 metri di lato)
	

	\textbf{Mani Brucianti}
	
	grado di Sfida: 2 
	
	Tipo: magico 
	
	DC Sopravvivenza: 26 
	
	DC Disattivare Congegni: 26 
	
	Attivatore: prossimità (Allarme) 
	
	Ripristino: nessuno 
	
	Effetto: 2d4 danni da fuoco 
	
	TS: Riflessi DC 11 dimezza 
	
	Bersaglio: bersagli multipli (tutti i bersagli in un cono di 6 metri di lunghezza e 3 metri di finale)
	

	\textbf{Giavellotto}
	
	grado di Sfida: 2 
	
	Tipo: meccanico 
	
	DC Sopravvivenza: 20 
	
	DC Disattivare Congegni: 20 
	
	Attivatore: posizione 
	
	Ripristino: nessuno 
	
	Effetto: Attacco a distanza 12 metri +15 (1d6+6), entro raggio 6 metri
	

	\textbf{Freccia Acida}
	
	grado di Sfida: 3 
	
	Tipo: magico 
	
	DC Sopravvivenza: 27 
	
	DC Disattivare Congegni: 27 
	
	Attivatore: prossimità (Allarme) 
	
	Ripristino: nessuno 
	
	Effetto: Attacco distanza di 16 metri (2d4 danni da acido per 4 round)
	

	\textbf{Fossa Celata}
	
	grado di Sfida: 3 
	
	Tipo: meccanico 
	
	DC Sopravvivenza: 25 
	
	DC Disattivare Congegni: 20
	Attivatore: posizione 
	
	Ripristino: manuale 
	
	Effetto: fossa profonda media (3d6 danni da caduta) 
	
	TS: Riflessi DC 20 evita 
	
	Bersaglio: bersagli multipli (tutti i bersagli in un quadrato di 3 metri di lato)
	

	\textbf{Arco Elettrico}
	
	grado di Sfida: 4 
	
	Tipo: meccanico 
	
	DC Sopravvivenza: 25 
	
	DC Disattivare Congegni: 20 
	
	Attivatore: contatto 
	
	Ripristino: nessuno 
	
	Effetto: Arco elettrico, 4d6 danni da elettricità
	
	TS: Riflessi DC 20 dimezza 
	
	Bersaglio: bersagli multipli (tutti i bersagli in una linea a distanza 6 metri)
	

	\textbf{Falce a Parete}
	
	grado di Sfida: 4 
	
	Tipo: meccanico 
	
	DC Sopravvivenza: 20 
	
	DC Disattivare Congegni: 20 
	
	Attivatore: posizione 
	
	Ripristino: automatico 
	
	Effetto: Attacco in mischia +20 (2d4+6/×4)
	

	\textbf{Blocco in Caduta}
	
	grado di Sfida: 5 
	
	Tipo: meccanico 
	
	DC Sopravvivenza: 20 
	
	DC Disattivare Congegni: 20 
	
	Attivatore: posizione 
	
	Ripristino: manuale 
	
	Effetto: Attacco in mischia +15 (6d6) 
	
	Bersaglio: bersagli multipli (tutti i bersagli in un quadrato di 3 metri di lato)
	

	\textbf{Aria infuocata}
	
	grado di Sfida: 5 
	
	Tipo: magico 
	
	DC Sopravvivenza: 28 
	
	DC Disattivare Congegni: 28 
	
	Attivatore: prossimità (Allarme) 
	
	Ripristino: nessuno 
	
	Effetto: 6d6 danni da fuoco, distanza 3 metri
	
	TS: Riflessi DC 14 dimezza 
	
	Bersaglio: bersagli multipli (tutti i bersagli in un'esplosione di raggio 3 metri)
	

	\textbf{Colpo Infuocato}
	
	grado di Sfida: 6 
	
	Tipo: magico 
	
	DC Sopravvivenza: 30 
	
	DC Disattivare Congegni: 30 
	
	Attivatore: prossimità (Allarme) 
	
	Ripristino: nessuno 
	
	Effetto: 8d6 danni da fuoco, distanza 3 metri
	
	TS: Riflessi DC 17 dimezza 
	
	Bersaglio: bersagli multipli (tutti i bersagli in un cilindro di raggio 3 metri)

	\textbf{Freccia Avvelenata}
	
	grado di Sfida: 6 
	
	Tipo: meccanico 
	
	DC Sopravvivenza: 20 
	
	DC Disattivare Congegni: 20 
	
	Attivatore: posizione 
	
	Ripristino: nessuno 
	
	Effetto: Attacco a distanza 18 metri +15 (1d6 più Veleno ×3)
	

	\textbf{Zanne Gelide}
	
	grado di Sfida: 7 
	
	Tipo: meccanico 
	
	DC Sopravvivenza: 25 
	
	DC Disattivare Congegni: 20 
	
	Attivatore: posizione 
	
	Durata: 3 round 
	
	Ripristino: nessuno 
	
	Effetto: distanza 3 metri (spruzzo di acqua gelata, 3d6 danni da freddo) 
	
	TS: Riflessi DC 20 dimezza 
	
	Bersaglio: bersagli multipli (tutti i bersagli in una stanza di 3x3x3 metri)
	

	\textbf{Trappola a Gas}
	
	grado di Sfida: 8 
	
	Tipo: meccanico 
	
	DC Sopravvivenza: 25 
	
	DC Disattivare Congegni: 20 
	
	Attivatore: posizione 
	
	Ripristino: riparabile 
	
	Effetto: Gas velenoso 
	
	Bersaglio: bersagli multipli (tutti i bersagli che si trovano in una stanza 3x3x3 metri)
	

	\textbf{Raffica di Frecce}
	
	grado di Sfida: 9 
	
	Tipo: meccanico 
	
	DC Sopravvivenza: 25 
	
	DC Disattivare Congegni: 25 
	
	Attivatore: visivo ( Occhio Arcano) 
	
	Ripristino: riparabile 
	
	Effetto: Attacco a distanza +20 (6d6) 
	
	Bersaglio: bersagli multipli (tutti i bersagli in una linea di 6 metri)
	

	\textbf{Fossa Celata con Spuntoni}
	
	grado di Sfida: 8 
	
	Tipo: meccanico 
	
	DC Sopravvivenza: 25 
	
	DC Disattivare Congegni: 20 
	
	Attivatore: posizione 
	
	Ripristino: manuale 
	
	Effetto: Fossa profonda 15 m (5d6 danni da caduta) + spuntoni (Attacco in mischia +15, 1d4 spuntoni per bersaglio per 1d6+5 danni ciascuno) 
	
	TS: Riflessi DC 20 evita 
	
	Bersaglio: bersagli multipli (tutti i bersagli in un cubo con lato 3x3x3 metri)
	

	\textbf{Pavimento Folgorante}
	
	grado di Sfida: 9 
	
	Tipo: magico 
	
	DC Sopravvivenza: 26 
	
	DC Disattivare Congegni: 26 
	
	Attivatore: prossimità (Allarme) 
	
	Durata: 1d6 round 
	
	Ripristino: nessuno 
	
	Effetto: Attacco di contatto in mischia +9, 4d6 danni da Elettricità
	
	Bersaglio: bersagli multipli (tutti i bersagli in una stanza di 6x6x3 metri)

	\textbf{Risucchio di Energia}
	
	grado di Sfida: 10 
	
	Tipo: magico 
	
	DC Sopravvivenza: 34 
	
	DC Disattivare Congegni: 34 
	
	Attivatore: visivo (Visione del Vero) 
	
	Ripristino: nessuno 
	
	Effetto: Attacco di contatto a distanza 18 metri +10, Punti Ferita max calano di 10d4
	
	TS: Tempra DC 23 nega dopo 24 ore
	

	\textbf{Stanza di Lame}
	
	grado di Sfida: 10 
	
	Tipo: meccanico 
	
	DC Sopravvivenza: 25 
	
	DC Disattivare Congegni: 20 
	
	Attivatore: posizione 
	
	Durata: 1d4 round 
	
	Ripristino: riparabile 
	
	Effetto: Attacco in mischia +20 (3d8+3) 
	
	Bersaglio: bersagli multipli (tutti i bersagli che si trovano in una stanza di 3x3x3 metri)
	

	\textbf{Cono di Schegge di Ghiaccio}
	
	grado di Sfida: 11 
	
	Tipo: magico 
	
	DC Sopravvivenza: 30 
	
	DC Disattivare Congegni: 30 
	
	Attivatore: prossimità (Allarme) 
	
	Ripristino: nessuno 
	
	Effetto: cono di lance di ghiaccio, 15d6 danni da freddo 
	
	TS: Riflessi DC 17 dimezza 
	
	Bersaglio: bersagli multipli (tutti i bersagli in un cono di 18 metri di lunghezza e 6 metri finali)
	

	\textbf{Lancia Mortale}
	
	grado di Sfida: 18 
	
	Tipo: meccanico 
	
	DC Sopravvivenza: 30 
	
	DC Disattivare Congegni: 30 
	
	Attivatore: visivo
	
	Ripristino: manuale 
	
	Effetto: Attacco a distanza 36 metri +20 (1d8+6 più veleno)
	

	\textbf{Inferno di fuoco}
	
	grado di Sfida: 13 
	
	Tipo: magico 
	
	DC Sopravvivenza: 31 
	
	DC Disattivare Congegni: 31 
	
	Attivatore: prossimità (Allarme) 
	
	Ripristino: nessuno 
	
	Effetto: 60 danni da fuoco 
	
	TS: Riflessi DC 14 dimezza 
	
	Bersaglio: bersagli multipli (tutti i bersagli in un'esplosione di 6 metri di raggio)
	

	\textbf{Masso Schiacciante}
	
	grado di Sfida: 15 
	
	Tipo: meccanico 
	
	DC Sopravvivenza: 30 
	
	DC Disattivare Congegni: 20 
	
	Attivatore: posizione 
	
	Ripristino: manuale 
	
	Effetto: Attacco in mischia +15 (16d6) 
	
	Bersaglio: bersagli multipli (tutti i bersagli in un quadrato di 3 metri di lato)
	

	\textbf{Attacco Potenziato}
	
	grado di Sfida: 16 
	
	Tipo: magico 
	
	DC Sopravvivenza: 33 
	
	DC Disattivare Congegni: 33 
	
	Attivatore: visivo (Visione del Vero) 
	
	Ripristino: nessuno 
	
	Effetto: +9 contatto a distanza 18 metri, 30d6 danni, TS: Tempra DC 19 riduce a 5d6 danni
	

	\textbf{Ferimento}
	
	grado di Sfida: 14 
	
	Tipo: magico 
	
	DC Sopravvivenza: 31 
	
	DC Disattivare Congegni: 31 
	
	Attivatore: contatto 
	
	Ripristino: nessuno 
	
	Effetto: Sanguinamento 5, attacco di contatto in mischia +6
	
	TS: Tempra DC 19 annulla
	

	\textbf{Galleria di Fulmini}
	
	grado di Sfida: 17 
	
	Tipo: magico 
	
	DC Sopravvivenza: 29 
	
	DC Disattivare Congegni: 29 
	
	Attivatore: prossimità (Allarme) 
	
	Durata: 1d6 round 
	
	Ripristino: nessuno 
	
	Effetto: 8d6 danni da Elettricità)
	
	TS: Riflessi DC 16 dimezza 
	
	Bersaglio: tutti i bersagli in una corridoio di 12x3x3 metri
	

	\textbf{Fossa Avvelenata}
	
	grado di Sfida: 12 
	
	Tipo: meccanico 
	
	DC Sopravvivenza: 25 
	
	DC Disattivare Congegni: 20 
	
	Attivatore: posizione 
	
	Ripristino: manuale 
	
	Effetto: Fossa profonda 15 m (5d6 danni da caduta) + spuntoni (attacco in mischia +15, 1d4 spuntoni per bersaglio per 1d6+5 danni ciascuno più veleno)
	
	TS: Riflessi DC 25 evita 
	
	Bersaglio: bersagli multipli (tutti i bersagli in un quadrato di 3x3 metri)
	

	\textbf{Sciame di Meteore}
	
	grado di Sfida: 19 
	
	Tipo: magico 
	
	DC Sopravvivenza: 34 
	
	DC Disattivare Congegni: 34 
	
	Attivatore: visivo
	
	Ripristino: nessuno 
	
	Effetto: 4 meteore a bersagli separati, +9 contatto a distanza 27 metri, 2d6 da impatto più 6d6 danni da fuoco
	
	TS: Riflessi DC 23 dimezza danni da fuoco
	
	Bersaglio: bersagli multipli (quattro bersagli, due dei quali non possono trovarsi ad una distanza superiore ai 12m l'uno dall'altro)
	

	\textbf{Distruzione}
	
	grado di Sfida: 20 
	
	Tipo magico 
	
	DC Sopravvivenza: 34 
	
	DC Disattivare Congegni: 34 
	
	Attivatore: prossimità (Allarme) 
	
	Ripristino: nessuno 
	
	Effetto: Tiro Salvezza Morte
	
	TS: Tempra DC 23 riduce a 5d12 danni altrimenti 10d12
	

\end{multicols}

\pagebreak

\subsection{Reputazione e Fama* - Opzionale}\index{Reputazione}\index{Fama}


\begin{enfasi}{
Fama e onore vanno talvolta più facilmente a chi non li ricerca. (Tito Livio)}\end{enfasi}\medskip

\begin{multicols}{2}

Benché alcuni eroi si accontentino delle ricompense delle loro imprese o si nascondano dietro una scorza di umiltà, altri cercano di vivere per sempre nelle saghe e nei canti delle loro epiche imprese. La storia misura il successo di un eroe con racconti di trionfo e audacia, ripetuti per generazioni.

Un eroe che non possa rievocare a nessuno la sua storia cade presto nell'oblio, insieme ai suoi sforzi non raccontati. Il racconto delle prodi gesta diviene il metro con cui si misura un eroe, e scolpisce tanto la sua identità quanto la sua reputazione.

La reputazione rappresenta il modo in cui il pubblico in generale percepisce positivamente o negativamente il personaggio. Questa percezione lo precede, parla per lui in sua assenza e determina come sarà trattato da chi ne ha sentito parlare. La reputazione implica cose diverse per tipi di personaggio diversi, in base ai valori sociali e culturali di regioni differenti. Un personaggio che incarni le qualità di eroe in una regione potrebbe essere considerato un depravato o uno scostumato in un'altra. Un'icona ampiamente riverita e rispettata in madrepatria potrebbe scivolare dalla fama all'oblio se si reca in un regno confinante.

Quando si usano queste regole per la reputazione, il Narratore deve stabilire cosa significa la reputazione per i giocatori e i PNG della campagna. Per esempio, una campagna a tema vichingo potrebbe basare la reputazione sui saccheggi. 

Se si riesce ad acquisire una reputazione forte o notevole, si potrebbe essere lodati per le proprie azioni e ricompensati con risorse superiori a quelle ottenibili dagli individui meno noti. Allo stesso modo, si può usare la reputazione per influenzare socialmente, politicamente o economicamente la gente.

La Fama aumenta e diminuisce in base alle proprie azioni. La Fama attuale determina la reputazione complessiva, la Sfera di Notorietà definisce i luoghi in cui si possono applicare i benefici della reputazione.

\end{multicols}

\textbf{Tabella: come acquisire punti Fama}\index{Tabella come acquisire punti Fama}

\medskip

\begin{tabularx}{0.95\textwidth}{lX}
\textbf{Eventi}	&\textbf{Modificatore di Fama}\\
\toprule
\textbf{Eventi Positivi}&\\
Acquisire un tesoro notevole da un degno avversario										&	+1\\
Consacrare un tempio al proprio Patrono													&	+1\\
Creare un Oggetto Magico potente														&	+12\\
Aumentare di un Livello																	&	+1\\
Individuare e disarmare tre o più trappole con CR adeguato di fila						&	+1\\
Effettuare una scoperta storica, scientifica o magica degna di nota						&	+1\\
Possedere un oggetto leggendario o un artefatto											&	+14\\
Ricevere una medaglia o una simile onorificenza da una figura pubblica					&	+1\\
Restituire un Oggetto Magico significativo o una reliquia al suo proprietario			&	+1\\
Saccheggiare la roccaforte di un nobile potente	(nemico)								&	+1\\
Sconfiggere in singolar tenzone un nemico con un CR più alto del proprio livello		&	+15\\
Come gruppo vincere un incontro di combattimento con un APL +3 più						&	+1\\
Sconfiggere in combattimento un diffamatore pubblico									&	+2\\
Superare una prova di Professione con DC 30 o più per creare un'opera d'arte o un oggetto &	+2\\
Superare una prova pubblica di Intimidire con DC 30 o più (devono esserci testimoni)	&	+2\\
Superare una prova pubblica di Intrattenere con DC 30 o più (devono esserci testimoni)	&	+2\\
Completare un'avventura con una difficoltà adeguato al proprio livello					&	+3\\
Ottenere un titolo formale (lady, lord, cavaliere e così via)							&	+3\\
Sconfiggere un rivale chiave (della campagna) in combattimento							&	+5\\
\textbf{Eventi Negativi}&\\
Essere condannato per un crimine lieve													&	-1\\
Accompagnarsi a una persona disdicevole													&	-18\\
Essere condannato per un crimine grave non violento										&	-2\\
Fuggire pubblicamente da un incontro con un avversario più debole						&	-3\\
Attaccare gente innocente																&	-5\\
Essere condannato per un crimine grave violento											&	-5\\
Perdere pubblicamente un incontro con un avversario più debole							&	-5\\
Essere condannato per omicidio															&	-8\\
Essere condannato per tradimento														&	-10\\
\end{tabularx}

\bigskip

\begin{multicols}{2}

\subsubsection{Fama}

Si comincia il gioco con una Fama pari al proprio livello del personaggio + il proprio modificatore di Carisma. La fama va da -100 a 100, con 0 che rappresenta la mancanza di notorietà. 

Nel corso della campagna, parole e gesta aiutano a costruirsi una reputazione. Benché un avventuriero compia molte imprese, non tutte sono abbastanza significative da garantire un cambiamento di Fama. Se possibile, il Narratore dovrebbe attenersi a quelle imprese che influenzano direttamente la storia o la campagna, e non assegnare punti per le vittorie secondarie.

La significatività di un'impresa specifica dovrebbe essere a discrezione del Narratore, ma la Tabella: Eventi di Fama fornisce alcuni esempi. Se la Fama scende sotto 0, vedi Discredito e Infamia più avanti.

\subsubsection{Sfera di Notorietà}

La reputazione di un personaggio viaggia di pari passo al racconto delle sue imprese. Anche se è un grande eroe nella sua terra, quando viaggia altrove scoprirà presto che la sua reputazione diminuisce e che prima o poi giungerà in regioni in cui è completamente sconosciuto. Maggiore è la reputazione, più si estende l'Area Influenzata.

La Fama determina il raggio massimo della Sfera di Notorietà. La Sfera di Notorietà ha un raggio di 150 chilometri, e in genere aumenta di altri 150 chilometri quando la Fama arriva a 10, 20, 30, 40 e 55.

Accrescere la Sfera di Notorietà non è sempre automatico, si può esprimere un parere su dove si concentri la propria reputazione. Per esempio, si potrebbe richiedere che la propria sfera si estenda più a sud verso una grande città e ignori le tribù barbariche a est, o che si estenda all'interno verso un altro paese anziché all'esterno, sull'oceano.

Sebbene la reputazione possa diffondersi per caso, in genere lo fa volutamente, perché i cantastorie girovaghi abbelliscono le storie delle gesta di un personaggio per renderle più divertenti, i suoi alleati amplificano le più comuni imprese, i suoi nemici ripetono pettegolezzi su di lui per assoldare altri e combatterlo, o il personaggio stesso racconta la sua storia ad ascoltatori compiaciuti.

Il luogo in cui si narrano queste storie determina dove si sarà conosciuti e crea una Sfera di Notorietà: un'eroica maga potrebbe assoldare dei Bardi per vantarsi della propria magia in un regno vicino che prevede di visitare, mentre un Barbaro antagonista potrebbe spingere verso sud i sopravvissuti feriti delle sue razzie, per diffondere la paura tra le sue prossime vittime.

Le seguenti azioni e condizioni influenzano il modificatore alle prove di Carisma, Diplomazia o Intimidire allo scopo di espandere la Sfera di Notorietà.

\medskip

\textbf{Tabella: Modificatori della Sfera di Notorietà}\index{Tabella Modificatori della Sfera di Notorietà}

\end{multicols}

\medskip

\begin{tabularx}{0.95\textwidth}{Xl}
\textbf{Azione}	&\textbf{Modificatore alla Prova}\\
\toprule
Alleati o servitori diffondono storie delle imprese del PG prima del suo arrivo	&+5\\
Un Bardo diffonde storie o canti delle imprese del PG prima del suo arrivo&	+1/2 Punteggio Intrattene del Bardo\\
Si hanno contatti con i PNG dell'insediamento&	+1\\
Si hanno nemici nell'insediamento&	+1\\
Distanza dalla propria Sfera di Notorietà&	-1 per 15 chilometri\\
La lingua principale dell'insediamento è diversa dalla propria&	-5\\
\end{tabularx}

\begin{multicols}{2}

\subsubsection{Il Livello di Fama}

\begin{itemize}


\item Il punteggio di Fama  è ciò che rende popolare il personaggio.

\item Un punteggio di fama entro i 10 punti ne faranno un eroe locale, di una piccola città.

\item Un punteggio di fama tra i 10 e 20 punti ne faranno un personaggio pubblico, noto a tutti in una piccola città o eroe di quartiere in una grande città.

\item Un punteggio tra i 20 e 30 punti rende il personaggio noto a tutti anche in una grande città, le sue gesta sono anche conosciute in regione magari non con tutti i dettagli.

\item Un punteggio tra i 30 e 40 è una vera è propria celebrità nella sua città, conosciuto per nome anche nelle città limitrofe e rispettato in tutta la regione.

\item Una fama tra i 40 e 50 punti rendono il personaggio una vera eminenza rispettata nello stato.

\item Un punteggio oltre i 55 punte fanno del personaggio una leggenda le cui gesta vengono tramandata ed ingigantite nei secoli a venire.

\end{itemize}

\subsubsection{Discredito e Infamia}


Se la propria Fama scende sotto 0, la reputazione è basata sull'infamia piuttosto che sulla Fama. Si consideri la Fama come un numero positivo anziché un numero negativo per tutte le regole legate a Fama, Sfera di Notorietà e Punti Prestigio (per esempio, una Fama da antagonista di -20 equivale a una Fama da eroe di 20 per i suoi ammiratori.


Nel caso in cui un evento possa accrescere la Fama, si può scegliere di aumentare la Fama (portandola più vicina a 0) o di diminuirla (rendendola un numero negativo maggiore). Per esempio, se la Fama di un personaggio è 20 e si tira pubblicamente un 30 su una prova di Professione per creare una spada  (cosa che in genere vale +2), si può aumentare la Fama a 18 o diminuirla a 22. 

Gli eventi negativi che diminuiscono la Fama contano sempre come negativi (un antagonista che attacchi gente innocente non ispira simpatia al pubblico).

Se si ha una Fama negativa, i PNG non malvagi spesso avranno reazioni maldisposte o ostili (vedi Tabella: Reazioni alla Fama Negativa). Si noti che se si ha una reputazione di persona potente e pericolosa, i PNG potrebbero evitare il PG anziché confrontarsi con lui.


\end{multicols}

\textbf{Tabella: Reazioni alla Fama Negativa}\index{Tabella Reazioni alla Fama Negativa}

\medskip

\begin{tabularx}{0.95\textwidth}{lX}
\textbf{Fama}&\textbf{	Reazione}\\
\toprule
-5&	Mercanti, mercenari e locandieri addebitano un al PG 10\% aggiuntivo per scoraggiare dal fare affari nella loro comunità.\\
-8&	Mercanti, mercenari e locandieri si rifiutano di fare affari. Al PG che entra in un negozio viene immediatamente chiesto di andarsene. Se si rifiuta, il proprietario chiama le autorità o i concittadini per buttarlo fuori.\\
-10&	Quando il PG si avvicina, i negozi chiudono le finestre e sbarrano le porte. La maggior parte dei cittadini si rifiuta di parlargli. Altri lo spronano ad andarsene immediatamente. Se resta più di 24 ore o agisce platealmente contro i cittadini, la sua Fama diminuisce di 5 e i cittadini si radunano per scacciare il PG.\\
-15&	Incendiata dalla spudorata audacia del PG di presentarsi nella comunità, una folla inferocita si raduna. Se il PG non se ne va entro pochi minuti, la folla comincia a bombardarlo con frutta marcia, rami e pietre.\\
-20	&Una folla inferocita si forma subito dopo l'ingresso in città del PG. Non volendo attendere un processo potenzialmente corrotto, cercano di catturarlo e giustiziarlo per i suoi crimini.\\
-25	&Una figura autoritaria ha promulgato un editto di arresto contro il PG, compresa una ricompensa per chiunque lo catturi. La cosa è risaputa e molti vogliono riscuoterla.\\
-30	&Una figura autoritaria ha messo una taglia sulla testa del PG. La cosa è risaputa e molti vogliono riscuoterla.\\

\end{tabularx}


\pagebreak

\section{Veleni e Pozioni}\index{Veleni}\index{Pozioni}

\label{veleni-e-pozioni}


\begin{enfasi}{
Un giorno, un uomo fu colpito da una freccia avvelenata. Gli amici e i parenti, in ansia, chiamarono un medico. Quando gli si avvicinarono per prendere la freccia, l'uomo disse loro: "Prima di farlo, vorrei sapere chi mi ha trafitto con questa freccia... Era uno schiavo, un re, o un bramino? Era grande? Piccolo? Di che colore era la sua pelle? Dove viveva? E la freccia com'è stata costruita? Quale veleno è stato impiegato? ..."
		
Mentre si stava ponendo tutte queste domande... il veleno fece il suo effetto e l'uomo ferito finì per morire. (Budda)}\end{enfasi}\medskip

\begin{multicols}{2}

\subsection{Tipo di Veleno e Pozione}

\lettrine[lines=2, lhang=0.33, loversize=0.25, findent=1.5em]{I}{ veleni} e pozioni possono distinguersi in base a come si viene in contatto con loro.
Non tutti i veleni sono tossici se ingeriti o inalati.

I veleni possono essere divisi in base al metodo con il quale devono venire in contatto con la vittima.


Per identificare una pozione è necessario una prova di Erboristeria a DC 12 + la rarità della pianta oppure in caso di Veleni la difficoltà è pari al Tiro Salvezza dello stesso.


Le pozioni se non descritto diversamente devono essere bevute (ingestione).

\textbf{Contatto}: sono contratti nel momento in cui qualcuno tocca il veleno con la pelle nuda. Tali veleni possono essere usati come veleni da ferimento. I veleni a contatto hanno solitamente un tempo di insorgenza di 1 round. Un veleno a contatto può essere un unguento, balsamo, liquido di qualsiasi densità o anche polvere se specifica per contatto e non inalazione.

\textbf{Ingestione}: si attivano quando una creatura li mangia o li beve. I veleni ad ingestione hanno solitamente un tempo di insorgenza di 10 minuti.

\textbf{Ferimento}: vengono trasferiti soprattutto con gli attacchi di alcune creature e tramite armi cosparse di veleno. I veleni a ferimento non hanno solitamente un tempo di insorgenza.

\textbf{Inalazione}: si attivano nel momento in cui una creatura entra in un'area che contiene tali veleni. Molti veleni ad inalazione riempiono un volume pari ad un cubo con spigolo di 3x3x3 metri per dose. Le creature possono tentare di trattenere il fiato mentre si trovano all'interno dell'area per evitare di inalare la tossina. Le creature che trattengono il fiato hanno devono fare una prova su Costituzione (3d6+Costituzione) a difficoltà 10 ogni round per non inalare il gas. Ogni round in cui si trattiene il fiato la prova di difficoltà aumenta di 1.
Vedi anche le regole per trattenere il fiato e soffocare in Ambiente.

\subsection{Insorgenza ed Effetto}\index{Insorgenza veleno}\index{Tempo di attivazione veleno}

Per insorgenza si intende quanto tempo ci mette il veleno o pozione a fare effetto. Se il tempo di insorgenza è 1 turno significa che per gli effetti del veleno/pozione ed il Tiro Salvezza lo si effettua dopo 10 minuti. Se nella tabella del veleno/pozione insorgenza non è specificata significa che l'effetto è immediato dopo l'entrata in contatto con il veleno.

L'effetto di un veleno/pozione è immediato dopo l'insorgenza. Verificare la descrizione del veleno per capirne l'effetto. Se il Tiro Salvezza su Tempra riesce il veleno non ha fatto effetto e si può ritenere neutralizzato.

Ci sono alcuni casi in cui e' presente la voce Frequenza, in questi rare occasioni il Tiro Salvezza va ripetuto ogni volta che passa la Frequenza indicata, in caso di fallimento del Tiro Salvezza i danni indicati vengoni riapplicati. 

\begin{center}
	\includegraphics[height=0.5\linewidth]{immagini/potion.png}
\end{center}

\begin{narratoresmall}
I veleni qui proposti sono alcuni dei tanti presenti e possibili. Usali come linee guida. Se per tua etica e stile non ti piacciono i veleni, specialmente quelli più cattivi, suggerisco di usare le Pozioni Generiche che trovi in fondo al capitolo. Sono veleni più blandi e meno personali, probabilmente più facilmente usabili anche dai giocatori.
\end{narratoresmall}

\subsubsection{Avvelenati}\index{Avvelenati}

\textbf{Prima dose}: Quando si viene esposti a un veleno per la prima volta (durante la propria azione o quella di qualcun altro), è necessario effettuare un Tiro Salvezza per evitare di venire avvelenati.

\textbf{Successo}: Si resiste al veleno. Non si subiscono effetti negativi e non sono necessari ulteriori Tiri Salvezza.

\textbf{Fallimento}: Siete stati avvelenati e si subisce subito l'effetto elencato.

\textbf{Più dosi}: Se si vieni esposti a più dosi dello stesso veleno nello stesso round la difficoltà del Tiro Salvezza aumenta di 1 per dose aggiuntiva.\index{Velono più dosi}

\textbf{In tempi diversi}: se si viene esposti al veleno in tempi diversi, ogni volta ci sarà un nuovo Tiro Salvezza e si subiranno gli eventuali effetti nei tempi previsti.


Se si viene esposti a veleni diversi è necessario effettuare un Tiro Salvezza per ogni tipo di veleno assunto.

\medskip

\begin{tcolorbox}[width=0.425\textwidth,title = Veleno ?]
{Il veleno è un arma a doppio taglio. Finché la usi tu va bene ma se te la usano contro, magari la stessa, diventa un problema. Ci sono anche degli aspetti etici nell'usare i veleni, valuta se i tuoi Tratti ti permettono di usare dei veleni e che tipi.
}\end{tcolorbox}

\subsection{Applicare il Veleno}\index{Applicare il Veleno}

Applicare il veleno ad un'arma o ad una munizione richiede 3 Azioni.

Ogni volta che un personaggio applica o prepara un veleno per l'uso deve tirare 3d6+Intelligenza e se ottiene 4 o meno con il tiro di dadi è entrato in contatto con il veleno e deve effettuare un Tiro Salvezza contro il veleno come di norma. Ciò non consuma la dose di veleno.

Ogni volta che un personaggio attacca con un'arma avvelenata, se ottiene 4 o meno con il tiro di dadi per il Tiro per Colpire, si espone agli effetti del veleno. Ciò consuma il veleno sull'arma.
Un pozione di veleno è sufficiente per coprire di veleno un arma media oppure 3 frecce. Il veleno viene così consumato e rimane attivo sull'arma finché questa non colpisce.

Una creatura che sotto gli effetti di un veleno, che si siano già' scatenati o meno, ha la condizione di Avvelenato.

\subsection{Creare Veleni Naturali}\index{Creare Veleni Naturali}

I veleni naturali possono essere realizzati usando Erboristeria. La DC per preparare un veleno è uguale alla DC del Tiro Salvezza su Tempra che richiede -5. Se gli ingredienti si comprano il costo per preparare la pozione è metà del costo di vendita indicato, se si cercano in natura il costo per produzione scende ad un quarto. Il tempo per preparare queste pozioni/droghe è pari alla DC/2 in ore.

Ottenendo un 3 o 4 naturale con la prova di Erboristeria ci si espone al veleno durante la sua preparazione. Il tempo necessario per preparare i veleni è pari alla DC in ore.

Gli esempi seguenti rappresentano solo alcuni dei possibili veleni.


\begin{center}
	\includegraphics[height=0.4\linewidth]{immagini/poison.png}
\end{center}

\begin{narratoresmall}
I veleni fanno parte della lunga tradizione dei problemi ed avversità nei giochi di ruolo. Quando volete usare un veleno pensate innanzitutto perché si trova li, per chi doveva essere usato, per quale scopo. Non e' detto che tutti i veleni debbano uccidere, un abile ladro potrebbe anche usare veleni stordenti o che indeboliscono la volontà del suo obiettivo giusto quel tanto che basta a farsi aprire la cassaforte.
\end{narratoresmall}

\end{multicols}

\medskip

\begin{center}
	\includegraphics[width=0.70\linewidth]{immagini/funeralebarca.png}
\end{center}

\bigskip

\textbf{Tabella: Veleni}\index{Tabella Veleni}

\medskip

\begin{tabularx}{1\textwidth}{m{4.5cm}lllm{7cm}l} %{XlllXl}
\toprule
%\begin{xltabular}{0.99\textwidth}{llllll} %{XlllXl}
	\textbf{Nome Veleno}  & \textbf{Uso} & \textbf{TS} & \textbf{Ins.} & \textbf{Effetto (danno)} & \textbf{Costo}\\
	\toprule
	Mistura Rossa \index{Mistura Rossa}   & F  & 13    & -  & -1d6 TC/TS per 10 minuti   & 10\\
	\toprule
	Nocciolo di Dennar \index{Nocciolo di Dennar}  & I  & 13    & 1 turno  & -1d2 Forza, per 3gg    & 15\\
	\toprule
	Succo di Ythis\index{Succo di Ythis} & I  & 14    & 1 turno  & -1d2 Intelligenza, per 1g  & 20\\
	\toprule
	Sangue di Thrun \index{Sangue di Thrun}   & C  & 26    & -   & -1d3 Costituzione   & 80\\
	\toprule
	Erba puntuta rosa \index{Erba puntuta rosa}    & I  & 22    & 1 turno  & -1d6 Destrezza, per 1 ora  & 60\\
	\toprule
	Dita di Daraka\index{Dita di Daraka} & F  & 17    & -   & -1d6 Forza, per 1 ora   & 35\\
	\toprule
	Polline di Rosa di Omro\index{Polline di Rosa di Omro}   & I  & 15    & -   & -1d3 Costituzione e Destrezza, per 1 ora   & 25\\
		\toprule
	Fumi di Curna\index{Fumi di Curna}   & R  & 18    & -   & -1d3 Saggezza   & 40\\
		\toprule
	Olio di Nabar \index{Olio di Nabar}  & R-F& 20    & -   & Confuso per 2d6 round    & 50\\
		\toprule
	Bacche Azzurre di fosso \index{Bacche Azzurre di fosso}  & I  & 21    & 1 turno  & -1d3 Intelligenza e Saggezza per 6 ore& 55\\
		\toprule
	Pelle di Rospo Azzurro \index{Pelle di Rospo Azzurro}    & C  & 22    & 1 minuto & Paralizzato per 1d6 turni& 60\\
		\toprule
	Cenere di Corteccia Gialla \index{Cenere di Corteccia Gialla} & F  & 15    & 6 round  & Privo di sensi per 1d3 ore    & 25\\
		\toprule
	Fiocco bianco di Mucot \index{Fiocco bianco di Mucot}    & C  & 20    & -   & Dorme per 2d12 ore  & 20\\
		\toprule
	Bava fermentata di Lucos \index{Bava fermentata di Lucos}& F  & 15    & -   & 1d8 Punti Ferita    & 25\\
		\toprule
	Bacca Viola di Barsar\index{Bacca Viola di Barsar}  & I  & 18    & 1 turno  & Incapace di eseguire azioni violente per 3d8 ore  & 40   \\
		\toprule
	Lingua di Kreex \index{Lingua di Kreex}   & F  & 20    & -   & La ferita sanguina. +1 danno da sanguinamento per round per 2 minuti Max +5 sanguinamento & 50   \\
		\toprule
	Fegato di Toporagno Viola \index{Fegato di Toporagno Viola}   & I  & 25    & 1 ora    & 2d6 di danno a Saggezza e Intelligenza. Permanente   & 75   \\
		\toprule
	Muschio Giallo \index{Muschio Giallo}& I  & 20    & 1 round  & la creatura guadagna una taglia. Sovradosaggi sono possibili. Durata 10 minuti  & 50\\
		\toprule
	Veleno di Serpe del Sangue \index{Veleno di Serpe del Sangue} & F  & 25    & -   & Paralisi per 1d6 ore -1d4 punti Forza per 7 giorni   & 75   \\
		\toprule
	Profumo di Ragmor \index{Profumo di Ragmor}    & R  & 16    & -   & -1d3 Carisma, per 1 giorno & 30\\
		\toprule
	Grasso di Toporagno Viola \index{Grasso di Toporagno Viola}   & C  & 13    & 1 round  & 2d12 Punti Ferita & 15\\
		\toprule
	Veleno di Ottalm\index{Veleno di Ottalm}  & F  & 20    & -   & Morte o -1d2 Costituzione permanente  & 50\\
\end{tabularx}

\medskip

\textbf{Applicazione}: \textbf{I}(ngestione), \textbf{F}(erimento), \textbf{C}(ontatto), \textbf{R}(espirazione).

Il Tiro Salvezza è sempre su Tempra se non specificato diversamente

I punti caratteristica persi si recuperano al ritmo di 1 al giorno se non indicato diversamente.

\begin{center}
	\includegraphics[width=0.25\linewidth]{immagini/mandragola2.png}
	
	\textit{Pianta di Mandragola}
\end{center}


\subsection{Pozioni naturali}\index{Pozioni}


\begin{enfasi}{
Io credo che una foglia d'erba non sia meno di una giornata di lavoro compiuto dagli astri. (Walt Whitman)
}\end{enfasi}

\begin{multicols}{2}

Il tempo per preparare queste pozioni/droghe è pari alla DC/2 in ore, mentre la difficoltà della prova di Erboristeria è pari alla DC -5. Se gli ingredienti si comprano il costo per preparare la pozione è metà del costo di vendita indicato, se si cercano in natura il costo per produzione scende ad un quarto.

Se la prova di DC Erboristeria ha successo se ne preparano 1d2 pozioni (da 1 dose).

Non si può beneficiare di più di una dose di pozioni naturali (per tipo) al giorno, a differenza di quelle magiche.

\end{multicols}

\textbf{Tabella: Elenco Pozioni}\index{Tabella Elenco Pozioni}

\medskip

\begin{tabularx}{0.99\textwidth}{llllXll}
\textbf{Nome}  & \textbf{Uso} & \textbf{Ins.} & \textbf{DC} & \textbf{Effetto}& \textbf{Loc.} & \textbf{Costo (mo)} \\
\toprule
Arlandas\index{Arlandas} & R  & 1 ora& 24& Rinsalda le fratture & CF5 & 100  \\
\toprule
Burthelas \index{Burthelas}   & I  & 1 turno   & 32& Rigenera le mani& HD7 & 410  \\
\toprule
Musekiss\index{Musekiss} & C  & 1 ora& 30& Rigenera arti inferiori   & TH9 & 550  \\
\toprule
Bacche di Ljust \index{Bacche di Ljust} & I  & 1 round   & 16& Preso la sera recuperi il doppio dei Punti Ferita minimo 4) & AZ6 & 30   \\
\toprule
Culcoa\index{Culcoa}& C  & 1 round   & 16& Recuperi 2d6 da danno da fuoco & TS7 & 30   \\
\toprule
Jojopo\index{Jojopo}& C  & 1 round   & 15& Recuperi 2d6 da danno da ghiaccio   & FM6 & 25   \\
\toprule
Kelventare\index{Kelventare}  & I  & 1d4 round & 28& Recuperi 2d6 Punti Ferita   & TT7 & 90   \\
\toprule
Harfy \index{Harfy} & C  & -    & 12& Interrompe il sanguinamento    & SS6 & 10   \\
\toprule
Arlan\index{Arlan}  & C  & -    & 15& Cura 1d6+3 Punti Ferita   & TT5 & 25   \\
\toprule
Darsurion\index{Darsurion}    & C  & 1 round   & 25& Cura 1d4 Punti Ferita& CM4 & 75   \\
\toprule
Draaf \index{Draaf} & C  & 1 round   & 20& Cura 1d8 Punti Ferita& SO6 & 50   \\
\toprule
Garioe\index{Garioe}& I  & 1 round   & 25& Cura 2d6 Punti Ferita& AZ7 & 75   \\
\toprule
Geffnull \index{Geffnull}& I  & 5 round   & 28& Cura 3d8+3 Punti Ferita   & EV8 & 90   \\
\toprule
Mirenna\index{Mirenna}   & I  & 1 round   & 20& Cura 5 Punti Ferita  & CM6 & 50   \\
\toprule
Rewky\index{Rewky}  & I  & -    & 25& Cura 2d8 Punti Ferita& TD6 & 75   \\
\toprule
Wickalim\index{Wickalim} & I  & -    & 15& Cura 2 Punti Ferita  & TD4 & 25   \\
\toprule
Lingua Rossa di Xabax\index{Lingua Rossa di Xabax}& C  & 1 turno   & 20& Cura 2d6 Punti Ferita ma se c'è malattia o veleno la rimuove ma causando 2d6 di danno & TA7 & 50   \\
\toprule
Yaveth\index{Yaveth}& I  & -    & 20& Cura 2d8 Punti Ferita& MO5 & 50   \\
\toprule
Bacio di Ljust\index{Bacio di Ljust}    & C  & 1 round   & 35& Cura 100 Punti Ferita& HO8 & 125  \\
\toprule
Polline di Rosa Verde\index{Polline di Rosa Verde}& R  & 3 turni   & 25& Recuperi 2d4 danni Intelligenza  aggezza  & FA8 & 75   \\
\toprule
Arkasun\index{Arkasun}   & C  & -    & 25& Cura 1d6 Punti Ferita a turno per 3 turni& MT7 & 75   \\
\toprule
Attarna\index{Attarna}   & I  & 1 turno   & 20& Concede un nuovo Tiro Salvezza per Malattie con un +4    & TF7 & 50   \\
\toprule
Delrean\index{Delrean}   & C  & 1 round   & 15& Allontana insetti per 1 giorno & CC6 & 2    \\
\toprule
Delrean Plus\index{Delrean Plus}   & I  & 1 round   & 18& Allontana insetti per 3 giorni & CC6 & 5    \\
\toprule
Melandrir\index{Melandrir}    & I  & 1 round   & 15& Concede un nuovo Tiro Salvezza per Malattie con +5  & CF7 & 25   \\
\toprule
Uovo di Urk\index{Uovo di Urk}& I  & 1 turno   & 12& 1 giorno di cibo& FH7 & 2    \\
\toprule
Barannie\index{Barannie} & I  & -    & 15& Rimuove nausea   & MD6 & 10   \\
\toprule
Eldrin'tail\index{Eldrin'tail}& I  & -    & 15& Concede un nuovo Tiro Salvezza su Veleni  & FH7 & 25   \\
\toprule
\end{tabularx}

\begin{tabularx}{0.99\textwidth}{llllXll}
\textbf{Nome}  & \textbf{Uso} & \textbf{Ins.} & \textbf{DC} & \textbf{Effetto}& \textbf{Loc.} & \textbf{Costo (mo)} \\
\toprule
Harlindar\index{Harlindar}    & I  & 1 turno   & 15& Fa abortire& SS7 & 20   \\
\toprule
Klandor\index{Klandor}   & I  & -    & 15& Rimuove paralisi& HB6 & 25   \\
\toprule
Klynkyx\index{Klynkyx}   & C  & 6 turno   & 15& Fa cadere tutti i capelli per 1d6+4 gg   & MO6 & 8    \\
\toprule
Arduuar\index{Arduuar}   & I  & 1 round   & 25& Rimuove Veleni  & SZ7 & 75   \\
\toprule
Nazamuse \index{Nazamuse}& I  & -    & 30& Rimuove Veleni e Malattie naturali & EW9 & 100  \\
\toprule
Muschio argentato\index{Muschio Argentato}& I  & -    & 25& Rimuove Malattie magiche & MU8 & 250  \\
\toprule
Nelthalion \index{Nelthalion} & I  & -    & 15& Fa vomitare& SR3 & 1    \\
\toprule
Uscaboo \index{Uscaboo}  & R  & 1 turno   & 25& Rimuove cecità  & MO7 & 75   \\
\toprule
Ucsaboo \index{Ucsaboo}  & C  & 1 turno   & 30& Rigenera occhi  & MO8 & 200  \\
	\toprule
Febfendi \index{Febfendi}& C  & 1 turno   & 25& Rigenera orecchie    & CF7 & 75   \\
	\toprule
Siranmuse\index{Siranmuse}    & I  & 1 giorno  & 30& Rigenera organi interni   & SS8 & 350  \\
	\toprule
Klagul\index{Klagul}& C  & 1 turno   & 20& Pulisce i denti & SS4 & 30   \\
	\toprule
Corteccia di Aklent\index{Corteccia di Aklent}    & I  & 1 turno   & 10& La corteccia masticata per almeno 10 round concede per le 24 ore successive un +1 Tiro Salvezza vs Veleno  & MT6 & 5    \\
	\toprule
Petali di Lisbeth \index{Petali di Lisbeth}  & I  & 1 turno   & 15&+2 Intelligenza, -2 Destrezza per 10 minuti & MC6 & 20   \\
	\toprule
Estratto radice Gisenosa\index{Estratto di radice Gisenosa}  & I  & 3 turni  & 15& Cura tosse e raffreddore  & MT6 & 5    \\
	\toprule
Gylvert\index{Gylvert}   & I  & -    & 25& Concede respirare sott'acqua per 4 ore   & MO7 & 75 \\
	\toprule
Gusterbloon \index{Gusterbloon}    & C  & 1 round   & 20& La pelle diventa più scura concedendo un +4 alla prove di Nascondersi & CM5 & 40  \\
	\toprule
Lievito Muschio Bianco \index{Lievito di Muschio Bianco} & I  & -    & 12& I prodotti da forno che usano questo lievito causano meteorismo incontrollabile ed incredibilmente puzzolente per 12 ore & CA3 & 1    \\
	\toprule
Estratto di Bacca Illa\index{Estratto di Bacca Illa bruciata}& I  & - & 15& +2 Iniziativa, +2 Destrezza, -4 Tiro Salvezza su Volontà, per 10 minuti    & MS6 & 5    \\
	\toprule
Corteccia Dagmather\index{Polvere di corteccia di Dagmather}    & R  & 1 round   & 25& Rimuove condizione esausto e affaticato  & SS5 & 20   \\
	\toprule
Radice secca di Kathaus\index{Radice secca di Kathaus} & R  & -    & 20& +2 Forza e Destrezza per 1 ora & FW6 & 25   \\
\end{tabularx}

\begin{multicols}{2}

\bigskip

\subsubsection{Note sulle pianticine...}

\textbf{Corteccia di Aklent}: chiamato anche \emph{Cespuglio Puzzola} per il suo pungente e caratteristico odore.\\

\textbf{Estratto di radice Gisenosa}: pianta tipo cardo, estremamente spinosa. Tende a crescere circondata dal \emph{Tribulus terrestris} o "baciapiedi".\\

\textbf{Muschio Argentato}: molto simile, per un non esperto, al Muschio Bianco. Si raccolgono le bacche.

\end{multicols}

\subsection{Dove trovare le piante}

\textbf{Tabella decodifica codici località}\index{Tabella decodifica codici località}

\smallskip

Es: Gusterbloon FT5. La prima Lettera indica il CLIMA, la Seconda indica l'AMBIENTE, la Terza indica la RARITA'. La rarità indica la possibilità, su un d10, di trovare l'erba/pianta ricercata. Tirare 1d10 e fare meno del numero indicato, chiaramente se c'è corrispondenza di clima ed ambiente.

\medskip

\textbf{Tabella: della corrispondenza Pozioni - Luoghi}\index{Tabella della corrispondenza Pozioni - Luoghi}

\medskip


\begin{tabularx}{0.99\textwidth}{ll|lX|lX}
	\textbf{1' Lettera} & \textbf{Clima}   &  \textbf{2' Lettera} & \textbf{Ambiente} & \textbf{2'  Lettera} & \textbf{Ambiente}\\
	\toprule
	A       & Arido 			 & A         & Alpino & B         & Gole\\ 
	C       & Freddo			 & C         & Foresta di Conifere  & D         & Foresta Decidua\\
	E       & Ghiacci perenni	 & F         & Argini fiumi e torrenti & G  & Campi ghiacciati\\
	F       & Freddo severo   	 & H         & Campi secchi &J         & Giungla, Foreste piovose\\
	H       & Umido e caldo  	 & M         & Montagna & N         & Oceano, distese salate\\
	M       & Temperato          & S         & Erba bassa & T         & Erba alta\\
	S       & Semi arido         & U         & Caverne e sotterranei & V         & Vulcanica\\ 
	T       & Temperato fresco   & W         & Discariche / Rifiuti & Z         & Deserto\\
	X       & Sconosciuto        & X         & Sconosciuto&&\\
\end{tabularx}


\subsection{Droghe (opzionale)}\index{Droga}

\textbf{Tabella: Elenco Droghe}\index{Tabella Elenco Droghe}

\medskip

\begin{tabularx}{0.99\textwidth}{XlllXrr}
\textbf{Nome}  & \textbf{Uso} & \textbf{Ins.} & \textbf{DC} & \textbf{Effetto}& \textbf{Loc.} & \textbf{Costo} \\
	\toprule
	Foglie fermentate di Luside\index{Foglie fermentate di Luside}  & I  & 1 turno  & 17& Allucinazioni sensoriali per 2d4 ore. +2 Carisma ed Intelligenza & SF7 & 5    \\
	\toprule
	Ferpillon \index{Ferpillon}& I  & 1 round   & 30& Fa dormire per 24 ore& SC5 & 50   \\
	\toprule
	Unto Grigio \index{Unto Grigio} & I  & 1 round   & 24& Rimuove condizionamenti mentali fino a Difficoltà 26  & AH9 & 80   \\
	\toprule
	Cenere di Arpasur \index{Cenere di Arpasur}    & R  & 1 round   & 20& Rimuove condizione di affaticato    & FT6 & 10    \\
	\toprule
	Carne secca di Ragno Viola \index{Carne secca di Ragno Viola}   & I  & 1 round   & 24& +4 Forza -4 Intelligenza per 1 turno& SH7 & 30   \\
	\toprule
	Estratto alcolico di Melzaa\index{Estratto alcolico di Melzaa}  & I  & -    & 20& +1d4 Forza, +1d4 Destrezza . -4 Tiro Salvezza su Volontà. Per 3 ore   & AF6 & 25   \\
	\toprule
	Essenza profumata di Inut\index{Essenza profumata do Inut} & R  & -    & 15& +2 Intelligenza, per 1d8 ore& HB6 & 15   \\
	\toprule
	Polline di Julnnaus\index{Polline di Julnnaus} & R  & -    & 20& +3 Costituzione per 2 ore & FO6 & 25   \\
	\toprule
	Polline del fiore di Erain \index{Polline del fiore di Erain} & R  & -    & 20& +2 Forza e Intelligenza e Destrezza. +3d6 Punti Ferita temporanei, per 1 ora  & FT7 & 75   \\
\end{tabularx}

\begin{multicols}{2}

\medskip

\textbf{L'utilizzo delle droghe e' completamente opzionale, e' il Narratore a decidere la loro presenza e disponibilità}. 

Le droghe danno dipendenza. Terminato l'effetto effettuare un Tiro Salvezza su Volontà a difficoltà 15 o prenderne un altra dose, il successivo Tiro Salvezza avrà difficoltà +1 e così via.

Ogni qual volta si prende una nuova dose entro 2 settimane dalla prima il Tiro Salvezza per non diventare dipendenti aumenta di 1. Vedi Dipendenza in Svantaggi.

\end{multicols}

\subsection{Pozioni generiche}\index{Pozioni generiche}\index{Pozioni}

Il Narratore è libero di usare tutte le pozioni e veleni indicate sopra oppure usare delle pozioni generiche pronte all'uso, comprabili in quasi tutti i negozi di erboristeria o di pozioni.

Nella tabella i costi ed effetti di queste pozioni, l'insorgenza è sempre immediata, la durata per le cure è immediata, per le altre è 1 ora (quindi la pozione Rimuovi Veleno ti "immunizza" per 1 ora contro un veleno).

\textbf{Tabella: delle pozioni generiche}\index{Tabella delle pozioni generiche}

\medskip

\begin{tabularx}{0.99\textwidth}{lXll}
	\textbf{Nome Pozione}&  \textbf{Effetto}&  \textbf{Costo (mo)}& \textbf{Applicazione}\\ 
	\toprule
	Cura					& recuperi 1d8+1 Punti Ferita 							& 50 & Ingestione\\ 
	Cura potenziata			& recuperi 3d8+3 Punti Ferita 							& 250  & Ingestione\\ 
	Indebolente				& -1d6 TC. TS DC 15 Tempra 						& 25 & Ingestione\\ 
	Indebolente potenziata	& -1d6 TC. TS DC 18 Tempra						& 50 & Ferimento \\ 
	Veleno					& subisci 2d4+2 di danno. TS DC 15 Tempra 		& 25 & Ingestione \\ 
	Veleno potenziata		& subisci 2d4+2 di danno. TS DC 18 Tempra 		& 50 & Ferimento \\ 
	Rimuovi Veleno			& annulla l'insorgenza di un veleno se presa entro l'attivazione, oppure concede un nuovo TS con +4 & 75 & Ingestione\\
\end{tabularx} 


\pagebreak

\section{Movimento e Trasporto}\index{Trasporto}\index{Movimento}

\label{movimento-e-trasporto}

\begin{enfasi}{
Il mio piede sinistro funziona benissimo, ma non riuscirei comunque a camminare se non ci fosse il destro! (Madagascar 3 - Ricercati in Europa, Film )

\medskip

Quando non puoi più correre, cammina veloce; quando non puoi più camminare veloce, cammina; quando non puoi più camminare, usa il bastone; però non trattenerti mai. (Madre Teresa di Calcutta)}
\end{enfasi}\medskip


\begin{multicols}{2}

\lettrine[lines=2, lhang=0.33, loversize=0.25, findent=1.5em]{I}{l} movimento si può distinguere in base a quale situazione coinvolge.

\medskip

\begin{itemize}
	\item Tattico, quando si combatte, si usano le distanze precise ed i quadretti di 1 metro di lato
	\item Locale, per esplorare una zona, misurato in metri al minuto.
	\item Via Terra, per muoversi da un posto all'altro, misurato in km all'ora o al giorno.
\end{itemize}

\subsection{Tipi di Movimento}

Quando si muovono nelle differenti scale di movimento (Tattico, Locale Via Terra), le creature generalmente camminano o corrono.

\textbf{Camminare}:\index{Camminare} Camminare rappresenta un movimento non affrettato ma deciso di circa 4 km all'ora per un umano senza Ingombro.

\textbf{Correre}\index{Correre}: Significa muoversi di circa 12 km all'ora per un umano.

Il personaggio che corre ha un malus di 1d6 nel Tiro per Colpire e di 4 nella Difesa nel round in cui corre.
Correre come azione di movimento raddoppia la velocità di movimento e non la triplica. Solo in situazioni di non combattimento (movimento locale) la corsa triplica il movimento (movimento locale, via terra)

\subsection{Tabella: Movimento e Distanza e Velocità: a Piedi}\index{Movimento a Piedi}\index{Tabella Movimento e Distanza e Velocità : a Piedi}

Questa tabella mostra i valori base di movimento a terra in situazioni di non combattimento.

\bigskip

\begin{tabularx}{0.43\textwidth}{lccc}
	\multirow{2}*{Tipo di movimento} &
	\multicolumn{3}{c}{Movimento}        \\
	\cmidrule(lr){2-4} & 6m    & 9m   & 12m                \\
	\midrule
\multicolumn{4}{c}{\textbf{Movimento Tattico)}}\\
	Camminare                        & 6m   & 9m   & 12m  \\
	Correre (x2)                     & 12m  & 18m  & 24m  \\
\multicolumn{4}{c}{\textbf{Un minuto (locale)}}                           \\
	Camminare & 36m  & 54m  & 72m \\
	Correre (x3)                     & 108m & 162m & 216m \\
\multicolumn{4}{c}{\textbf{Un'ora (via terra)}}                           \\
	Camminare                        & 3km  & 4km  & 6km  \\
	Correre (x3)                     & 9km  & 12km & 18km \\
\multicolumn{4}{c}{\textbf{Un giorno (via terra}}                        \\
	Camminare                        & 24km & 32km & 54km \\
\end{tabularx}


\subsection{Movimento Tattico}\index{Movimento Tattico}

Durante un combattimento si utilizza il Movimento tattico.
Le distanze vengono misurate in quadretti da un metro, il movimento è gestito tramite le Azioni di Movimento.


Un personaggio può usare 1 Azione (di Movimento) per muoversi fino a tutto il proprio movimento. Può effettuare più volte nel round, fino a 3 volte, l'azione Movimento, spostandosi quindi del triplo del suo movimento.

Può anche effettuare una Azione di Scatto\index{Scatto} ovvero percorrere il doppio del suo movimento in una sola Azione. Incappa così però nelle penalità per chi corre (-1d6 colpire, -4 Difesa) e potenzialmente subendo gli attacchi di opportunità degli avversari.

Un personaggio può effettuare fino a 3 Azioni di Scatto, ovvero corre per tutto il round percorrendo quindi il suo movimento * 6.

\subsubsection{Movimento Ostacolato}\index{Terreno difficile}

Terreno difficile, ostacoli o scarsa visibilità possono impedire i movimenti. Quando il movimento è ostacolato si va a metà della velocità. Quindi sono necessari 2 Azioni per coprire la propria distanza di 9 metri (se si è umano senza ingombro..). Oppure con una Azione di movimento si copre solo 4 metri.

Se esiste più di una condizione particolare, aggiungere tra loro tutti i costi aggiuntivi applicabili.

In alcune situazioni il movimento è talmente ostacolato che la distanza percorribile per Azione è minima.. in tal caso si possono utilizzare tutte e 3 le Azioni per muoversi di una Azione di movimento (9/6 metri) in qualsiasi direzione.

Non applicate questa regola per attraversare terreni impraticabili o per muoversi quando non è possibile farlo in alcun modo.

Non si può \textbf{Scattare} (Correre) o \textbf{Caricare} \index{Carica su Terreno difficile} \index{Scatto su terreno difficile}agevolmente attraverso un \textbf{percorso che ostacola il movimento}, ovvero terreno difficile. Il giocatore può tentare una prova di Acrobatica a DC 20 per riuscire a caricare o correre, comunque percorrendo solo la metà della distanza. La prova di Acrobatica non è necessaria, pur percorrendo metà movimento, se si ha il Vantaggio Rinoceronte.

\subsection{Movimento Locale}\index{Movimento Locale}

I personaggi che esplorano una zona usano il movimento locale, misurato in metri al minuto.


In queste situazione non è fondamentale misurare la distanza in maniera precisa ma appena la situazione diventa "problematica" o richiede attenzione la mappa si converte in movimento tattico, quadrettata e misurata.

\medskip

\begin{itemize}
	\item
	      Camminare: Un personaggio può camminare senza problemi in scala locale per 8 ore al giorno.
	\item
	      Correre: Un personaggio può Correre per un numero di round pari al triplo del proprio punteggio di Costituzione su scala locale senza bisogno di riposarsi (minimo un round). 
\end{itemize}


\subsection{Movimento Via Terra}\index{Movimento Via Terra}

I personaggi che percorrono lunghe distanze usano il movimento via terra. Il movimento via terra è misurato in ore o giorni. Un giorno rappresenta 8 ore di tempo di viaggio reale. Per imbarcazioni a remi, un giorno significa remare per 10 ore. Per navi a vela rappresenta 24 ore di movimento.

\textbf{Camminare}\index{Camminare}

Si può camminare per 8 ore in un giorno di viaggio senza problemi.

Camminare più a lungo può sfinire (vedi Marcia forzata, sotto).

\textbf{Andare Veloci}\index{Andare Veloci}

Si può andare veloci (movimento*2) per 1 ora senza problemi. Andare veloci per una seconda ora compresa tra due cicli di sonno provoca 1 Danno Non Letale, e ogni ora aggiuntiva provoca il doppio dei danni subiti nell'ora precedente. Un personaggio che subisce Danni Non Letali da andatura veloce è considerato Affaticato.

Un personaggio Affaticato non può Correre o Caricare.

\textbf{Correre}\index{Correre}

Non è possibile Correre per un lungo periodo di tempo. Tentativi di Correre e riposarsi a cicli funzionano come Andare Veloci.

\textbf{Terreno}\index{Terreno}

Il terreno su cui si viaggia influenza quale distanza viene percorsa in un'ora o in un giorno (vedi Tabella: Terreno e Movimento Via Terra). Una strada maestra è una strada principale, dritta e lastricata. Una strada comune è solitamente un cammino impervio. Un sentiero è come una strada comune tranne per il fatto che permette di viaggiare solo in fila indiana e non avvantaggia un gruppo che viaggia con veicoli. Un terreno libero è una zona selvaggia senza sentieri segnati.

\bigskip

\textbf{Tabella: Terreno e Movimento Via Terra* (Opzionale)}\index{Tabella Terreno e Movimento Via Terra* (Opzionale)}

Nella tabella sono indicati i moltiplicatori per la distanza percorsa.

\medskip

\begin{tabularx}{0.45\textwidth}{XXXX}
	\textbf{Terreno}  & \textbf{Strada maestra} & \textbf{Strada comune} & \textbf{Sentiero non battuto}\\
\toprule
	Brughiera         & x1                      & x1                     & x3/4\\
	Collina           & x1                      & x3/4                   & x1/2\\
	Deserto Sabbioso  & x1                      & x1/2                   & x1/2\\
	Foresta           & x1                      & x3/4                   & x1/2\\
	Giungla           & x1                      & x3/4                   & x1/4\\
	Montagna          & x3/4                    & x3/4                   & x1/2\\
	Palude            & x1                      & \texttimes 3/4         & \texttimes 1/2\\
	Pianura           & x1                      & \texttimes 3/4         & \texttimes 1/2\\
	Tundra Ghiacciata & x1                      & \texttimes 3/4         & \texttimes 3/4\\
\end{tabularx}

\bigskip

\textbf{Marcia Forzata}\index{Marcia Forzata}

In un giorno di cammino normale, si può camminare per 8 ore. Il resto del giorno viene sfruttato per fare e disfare il campo, riposarsi e mangiare.

Se si cammina di più di 8 ore è necessario effettuare un Tiro Salvezza su Tempra a difficoltà 11 +1 per ogni giorno consecutivo di marcia forzata o si diventa Affaticati.

La marcia forzata può essere tenuta per un numero di giorni pari al valore di Costituzione prima di incorrere nell'Affaticamento indipendentemente dall'esito del Tiro Salvezza.


\textbf{Movimento in sella}\index{Movimento in sella}

Una cavalcatura che porta un cavaliere può muoversi con andatura veloce. Tuttavia, i danni che subisce sono danni normali invece che non letali. Può anche essere costretta a una marcia forzata, ma le sue prove di Costituzione falliscono automaticamente e di nuovo i danni che subisce sono danni normali. Anche le cavalcature sono considerate Affaticate quando subiscono danni da andatura veloce o marcia forzata.

\end{multicols}

\medskip

\begin{center}
	\includegraphics[height=0.3\linewidth]{immagini/carretto.png}
\end{center}

\subsection{Tabella: Cavalcature e Veicoli}\index{Cavalcature}\index{Veicoli}\index{Tabella Cavalcature e Veicoli}\index{Cavallo movimento}

\medskip

\label{tabella-cavalcature-e-veicoli}\index{Cane}\index{Pony}\index{Carretto}\index{Zattera}\index{Barca}\index{Nave}

\begin{tabularx}{0.95\textwidth}{lXX}
	\textbf{Cavalcatura o Veicolo (peso trasportato)} & \textbf{All'ora} & \textbf{Al giorno}\\
\toprule
	Cane da Galoppo     & 6km              & 48km              \\
	Cane da Galoppo (50.5-150 kg)*      & 4.5km            & 36km              \\
	Cavallo Leggero     & 7.5km            & 60km              \\
	Cavallo Leggero (115,5-345 kg)*     & 5.25km           & 42km              \\
	Cavallo Pesante     & 7.5km            & 60km              \\
	Cavallo Pesante (150.5-450 kg)*     & 5.52km           & 42km              \\
	Pony& 6km              & 48km              \\
	Pony (75,5-225 kg)* & 4.5km            & 36km              \\
	Carretto o Carro    & 3km              & 24km              \\
	\textbf{Imbarcazione}               &  &   \\
	Zattera o Chiatta (pertica o rimorchio)             & 0.75km           & 7.5km             \\
	Barcone (a Remi)**  & 1.5km            & 15km              \\
	Barca a Remi**      & 2.25km           & 22.5km            \\
	Nave a Vela (vele)  & 3km              & 72km              \\
	Nave da Guerra (vele e remi)        & 3.75km           & 90km              \\
	Nave Lunga (vele e remi)            & 4.5km            & 108km             \\
	Galea (remi e vele) & 6km              & 144km             \\
\end{tabularx}

\begin{multicols}{2}

\bigskip

*I quadrupedi, come i cavalli, possono portare carichi superiori rispetto ai personaggi. Vedi Capacità di Trasporto per maggiori informazioni.

**Zattere, chiatte e barconi sono usati su laghi e fiumi. Se seguono la corrente, sommare la velocità della corrente (di solito 4,5 km/h) alla velocità dell'imbarcazione. Oltre a essere spinta con i remi per 10 ore, l'imbarcazione può anche essere trasportata dalla corrente per altre 14 ore, se qualcuno è in grado di guidarla, e quindi si aggiungono altri 63 km alla distanza giornaliera percorsa. Queste imbarcazioni non possono essere spinte a remi contro una corrente molto forte, ma possono essere tirate controcorrente da animali da soma sulla riva.

\subsection{Fuga e Inseguimento}\index{Fuga}\index{Inseguimento}

Nel movimento round per round è impossibile per un personaggio lento sfuggire ad un personaggio veloce senza qualche tipo di aiuto. Allo stesso modo, non è un problema per un personaggio veloce sfuggire ad uno più lento.

Quando la velocità dei due personaggi coinvolti è uguale, c'è un metodo abbastanza semplice per risolvere un inseguimento: se una creatura sta inseguendo un'altra ed entrambe si muovono alla stessa velocità, e l'inseguimento prosegue almeno per alcuni round, occorre che inseguitore ed inseguito effettuino 3 Tiri Salvezza consecutivi su Riflessi contrapposti.

Chi vince la sfida riesce a fare perdere le proprie tracce o agguantare il fuggitivo.

Nel caso di un lungo inseguimento dove non c'è possibilità di nascondersi o fare perdere le proprie tracce, eseguite 3 Tiri Salvezza su Tempra contrapposti per determinare quale delle due parti può mantenere più a lungo il ritmo. Chi ottiene più successi riesce a fuggire oppure è l'inseguitore che riesce a raggiungerla.

\subsection{Capacità di Carico e Trasporto: Peso Ingombro}\index{Capacità di Carico}\index{Ingombro}

\label{sec:capacita-di-carico-e-trasporto-ingombro}

\subsubsection{Peso e Ingombro}\index{Ingombro}\index{Peso}

Portare tesori, pezzi di drago, armature complete per non parlare di armi sproporzionate o arieti da sfondamento, carrucole e paranco, rendono difficile il movimento.

Quando valutate il peso trasportato ragionate anche sull'ingombro!
Portale un rotolo di 12 metri x 6 metri di seta non è una attività fisica impegnativa, saranno pochi chili, ma l'ingombro è tale da non poter permettere ulteriore carico.

Ci possono essere oggetti leggeri ma estremamente ingombranti (tronchi cavi, tappeti di seta appunto..) oppure piccoli ma pesantissimi (sfere di mercurio, vestiti intessuti d'oro), per tutti questi oggetti il valore di peso deve essere ragionato anche in funzione dell'ingombro.

I valori di peso degli oggetti si sommano tra di loro per dare il carico totale portato.

\subsubsection{Capacità di Carico}

Consulta la tabella Capacità di carico per capire la tua capacità di carico.

\bigskip

\end{multicols}

\textbf{Tabella: Capacità di carico}\index{Tabella Capacità di carico}

\medskip

\begin{tabularx}{0.95\textwidth}{XXXXX}
\toprule
	\multirow{2}*{Forza} & \multicolumn{3}{c}{Capacità di Carico}&\\
	\cmidrule(lr){2-5} & Normale (kg)  & Appesantita (kg)  & Massimo (kg) & Spinta (kg)\\ 
	\cmidrule(lr){2-5} &   & -3 Des, -1/3 Mov,  & -6 Des, Mov=1/3 & Mov=1/4\\ 
-3 &0-5 & 5-7 & 7-15&entro 22\\
-2&0-7& 7-10 & 10-20&entro 25\\
-1& 0-10& 10-15 & 15-30&entro 45\\
0&  0-15& 15-22 & 22-45&entro 65\\
+1& 0-20& 20-30&30-60&entro 90\\
+2& 0-35& 35-50&50-100&1entro 50\\
+3& 0-60& 60-90&90-180&entro 270\\
+4& 0-95& 95-140&140-280&entro 420\\
+5& 0-140& 140-220&220-440&entro 660\\
\end{tabularx}

\begin{multicols}{2}

\subsubsection{Spingere, Trascinare o Sollevare}

Puoi trasportare, tra zaino, armatura, armi ed equipaggiamento un peso pari a quanto indicato su \textbf{Normale} senza alcuna penalità.

Se trasporti un peso che ricade nell'area \textbf{Appensantita} hai un -3 alle prove basate sulla Destrezza ed il movimento diminuisce di un terzo.

Il peso indicato come \textbf{Massimo} significa il massimo trasportabile. Hai un -6 alle prove basate sulla Destrezza, il tuo movimento diminuisce ad un terzo.

Se il peso è oltre il Massimo ma entro \textbf{Spinta} puoi appoggiarlo e spingerlo (se fattibile), in questo caso la tua velocità si ridurrà ad quarto.

Un oggetto se ha "\textbf{delle ruote}" su cui muoversi si considera che abbia un peso pari ad un quarto per quanto riguarda la Spinta.



\emph{Ricordate che l'armatura e scudo indossati hanno un peso dimezzato rispetto a quanto segnato.}


\medskip

In caso più creature spingano il "grave" considerate la Forza piena della creatura più forte più metà della Forza delle creature con forza minore (minimo 1). Valutate chiaramente anche quante persone possono riuscire contemporaneamente a spingere data la dimensione dell'oggetto da trasportare.

\subsubsection{Creature Più Grandi e Più Piccole}

\textbf{Creature bipedi più grandi} possono trasportare più Ingombro in base alla categoria di taglia:

\begin{itemize}
	\item Piccolissima: Carico Normale = Carico Normale /16
	\item Minuta: Carico Normale = Carico Normale /8
	\item Minuscola: Carico Normale = Carico Normale /4
	\item Piccola: Carico Normale = Carico Normale /2 +3
	\item Grandi: Carico Normale = Carico Normale x2 +5
	\item Enormi: Carico Normale = Carico Normale x4 +5
	\item Mastodontica: Carico Normale = Carico Normale x8 +5
	\item Colossale: Carico Normale = Carico Normale x16 +5
\end{itemize}

\bigskip

\subsubsection{Le creature quadrupedi possono trasportare pesi superiori }

\begin{center}
	\includegraphics[height=0.5\linewidth]{immagini/cavallo.png}
\end{center}

\begin{itemize}
	\item Piccolissima: Carico Normale = Carico Normale *1/4
	\item Minuta: Carico Normale = Carico Normale x1/2
	\item Minuscola: Carico Normale = Carico Normale x3/4
	\item Piccola: Carico Normale = Carico Normale
	\item Media: Carico Normale = Carico Normale x2
	\item Grandi: Carico Normale = Carico Normale x3
	\item Enormi: Carico Normale = Carico Normale x6
	\item Mastodontica: Carico Normale = Carico Normale x12
	\item Colossale: Carico Normale = Carico Normale x24
\end{itemize}

Un cavallo essendo Grande e quadrupede, con Forza 3, puo trasportare senza problemi fino ad un massimo di 60kg di base *2 perché Grande * 3 perché quadrupede, quindi 360 kg non incorrendo in penalità.

\subsection{Altri Tipi di Movimento}

\label{altri-tipi-di-movimento}

\subsubsection{Nuotare}\index{Nuotare}

Una creatura con una velocità di Nuotare può muoversi attraverso l'acqua alla sua velocità indicata senza fare prove di Nuotare. Si guadagna un bonus di +8 su qualsiasi prova di Nuotare per eseguire un'azione particolare o evitare un pericolo. La creatura può sempre scegliere di prendere 10 su una prova di Nuotare, anche se distratti o in pericolo quando si nuota. Una tale creatura può utilizzare l'azione di correre mentre nuota, a condizione che nuoti in linea retta.

Se non si ha il tipo di movimento Nuotare muoversi nell'acqua si considera come "terreno" difficile, e quindi ci si muovo a metà della velocità indicata da movimento.

\subsubsection{Scalare}\index{Scalare}

Una creatura con una velocità di Scalare ha un bonus di +8 su tutti le prove di Arrampicarsi. La creatura se  deve fare una prova di Arrampicarsi per arrampicarsi su qualsiasi parete o pendenza con una DC superiore a 0,  può sempre scegliere di prendere 10, anche se di fretta o minacciata durante la salita.

Se una creatura con una velocità di Scalare tenta una scalata rapida (vedi sopra), e come se eseguisse una Azione di Scatto e fa una singola prova di Arrampicarsi con una penalità di -5.

Una creatura mantiene il suo bonus di Destrezza alla Difesa (se presente) durante la salita, e gli avversari non ottengono bonus speciale per i loro attacchi contro di esso. 

Se non si il tipo di movimento Scalare si considera come "terreno" difficile, e quindi ci si muovo a metà della velocità indicata da movimento.

\subsubsection{Scavare}\index{Scavare}

Una creatura con una velocità di Scavare può scavare tunnel attraverso la terra, ma non attraverso la roccia a meno che il testo descrittivo non dica il contrario. Le creature non possono caricare o correre mentre scavano.

La maggior parte delle creature scavatrici non lascia tunnel che altre creature possono utilizzare (sia perché il materiale attraverso cui scavano riempie il tunnel dietro di loro o perché in realtà non spostano materiale quando scavano), vedere la descrizione della singola creatura per i dettagli.

\subsubsection{Camminare - Velocità Su Terreno}

La Velocità sul terreno é la normale velocità per personaggi che non scalano, nuotano o volano.

\subsubsection{Volare}\index{Volare}

Volare per una creatura dotata di questa abilità è come camminare per una creatura "terrestre". Una creatura dotata di volo usa le sua azioni per muoversi ma difficilmente sarà influenzata dal terreno difficoltoso.

\begin{center}
	\includegraphics[width=0.9\linewidth]{immagini/grifonicastello.png}
\end{center}


\end{multicols}

\pagebreak

\section{Masterizzare}\index{Masterizzare}\index{Narratore}

\label{masterizzare}


\subsection{Il Narratore}

\begin{enfasi}{
Chi comanda al racconto non è la voce: è l'orecchio. (Italo Calvino)}\end{enfasi}\medskip

\begin{multicols}{2}

\label{il-narratore}

\lettrine[lines=2, lhang=0.33, loversize=0.25, findent=1.5em]{M}{entre} il giocatore interpreta un personaggio in un'avventura, il Narratore è colui che la gestisce. Ha certamente molto più lavoro, ma ricreare un mondo intero affinché i propri amici lo esplorino, può dare molte soddisfazioni.

Il ruolo del Narratore non è facile ma concede enormi privilegi. Vedere i propri amici giocare, divertirsi, "ammattirsi" dietro dubbi, indovinelli e situazioni da te create dà tantissimo divertimento e momenti di vera convivialità.

Il tuo ruolo è quello del grande orchestratore, pianificatore o anche paesaggista se preferisci, con poche semplici pennellate delinei la struttura e saranno poi i giocatori ad aggiungere dettagli e situazioni.

Il tuo "lavoro" è fondamentale ed importantissimo, la bontà della sessione di gioco dipende da te. Il tuo scopo è fare divertire, coinvolgere ma anche terrorizzare e fare discutere.

Non sei tu il protagonista ne l'avventura, ma i giocatori, i tuoi amici, non rubare la scena ma come un gran ballo sii il direttore d'orchestra dove gli strumenti sono le situazioni che hai previsto (l'avventura) ed i ballerini i giocatori.


\subsection{Punti Esperienza}\index{Punti Esperienza}\index{PX}

\label{punti-esperienza}

In DBS il passaggio di livello non è vincolato da un numero di mostri affrontati o dai tesori ottenuti, bensì dal fattore di difficoltà degli incontri e da come i giocatori hanno giocato.

In DBS il suggerimento primario è premiare i giocatori che più si sono impegnati per il gruppo, quelli che maggiormente hanno contribuito al buon esito dell'avventura e della sessione. I punti esperienza non misurano solo il successo ma anche la partecipazione al gioco.

E' quindi possibile avere personaggio con punti esperienza diversi e potenzialmente anche livelli diversi.

Alternativamente potete decidere di fare passare di livello ogni qualvolta lo ritenete necessario al buon gioco ed all'avventura, uniformando l'esperienza dei personaggi.

Di seguito troverete le regole per stabilire come costruire gli incontri, scontri e come premiare i giocatori/personaggi.

Prendete questa tabella dei punti esperienza

\subsubsection{Tabella: Punti Esperienza / Livello}\index{Tabella punti Esperienza / Livello}

\label{tabella-punti-esperienza-livello}

\begin{tabularx}{0.45\textwidth}{lX|lX}
	\textbf{Livello} & \textbf{Punti Esperienza}&\textbf{Livello} & \textbf{Punti Esperienza}\\
	\toprule
	1 & 0-14xp & 11&310 xp\\
	2& 15 xp & 12&350 xp\\
	3& 25 xp&13&390 xp\\
	4 &60 xp &14&430 xp\\
	5 &95 xp & 15& 470 xp\\
	6 &130 xp & 16& 510 xp\\
	7 &165 xp & 17&555 xp\\
	8 &200 xp & 18&600 xp\\
	9 &235 xp & 19&645 xp\\
	10& 270 xp & 20 & 690 xp\\
	&&20+& +50 xp \\
\end{tabularx}

\begin{itemize}
\item
\textbf{Per ogni incontro designato per sfidare il gruppo in maniera media o difficile assegnate 1 punto esperienza.}
\item
\textbf{Per ogni incontro designato per essere potenzialmente mortale assegnate 2 punti esperienza.}
\item
\textbf{Per ogni incontro finale, il climax dell'avventura, assegnate 3 punti esperienza, questi punti più che per lo scontro "con il Boss finale" vanno assegnati come merito per aver portato a termine una lunga avventura.}
\end{itemize}

Nel Mostruario sono indicati dei PE per i mostri (es. Sfida 13 (10.000 PE). Ignorateli. Usate il grado di sfida. Oppure se volete un sistema ancor più OSR usateli in maniera esclusiva. Consultate la Tabella opzione Punti Esperienza per Livello.

Continuerei a suggerire di premiare i giocatori anche in base al divertimento fornito, idee, azioni epiche, tesori trovati.

\medskip

\textbf{Tabella: opzionale Punti Esperienza per Livello}\index{Tabella opzionale Punti Esperienza per Livello}

\begin{tabularx}{0.45\textwidth}{lX|lX}
\textbf{Livello} & \textbf{Punti Esperienza}&\textbf{Livello} & \textbf{Punti Esperienza}\\
\toprule
1&	0 &2	&250\\
3&	1,100&4	&3,100\\
5&	6,500&6	&11,800\\
7&	19,700&8&	30,60\\
9&	45,600&10&	64,800\\
11&	89,600&12&	121,000\\
13&	159,000	&14&	207,000\\
15&	266,000	&16&	341,000\\
17&	433,000&18&	550,000\\
19&	697,000&20	&883,000\\
\end{tabularx}

I piu' attenti avranno notato che non e' la tabella standard, la differenza e' voluta e ricercata.

\medskip

Questi punti saranno assegnati al gruppo e quindi a tutti i giocatori, purché abbiano almeno cercato di partecipare agli scontri/sfide.

Se il gruppo per propria "incapacità" o per "sfortuna" trasforma un incontro facile (da 0 punti esperienza) in un incontro mortale, non dovete dare 2 punti esperienza. Cercate al più di premiare lo spirito di gruppo, le energie spese e se possibile la creatività nell'uscirne vivi nonostante tutto.

Quando dico "incontro" non pensate al semplice scontro con i mostri, per incontro si intende qualsiasi evento di ruolo che sfidi e metta alla prova i giocatori. Questa sfida può essere una arguta discussione con il nobile che non li vuole pagare al termine di una missione, alla sfida di un indovinello, rebus, delle trappole ben piazzate.


\begin{center}
	\includegraphics[width=0.8\linewidth]{immagini/deathbeowulf.png}
	
	\textit{Henry Justice Ford}
\end{center}


\bigskip

Ogni qual volta il giocatore o il gruppo:\index{Bonus PX}\index{Bonus Esperienza}
\begin{itemize}
	\item
	      \textbf{raggiunga gli obiettivi prefissati} (premio di gruppo o singolo);
	\item
	      \textbf{faccia un ottimo gioco di ruolo} (premio al giocatore);
	\item
	      \textbf{sfrutti a pieno ed anzi sia alternativo nell'uso delle proprie Abilità e capacità (senza cadere nel powerplayer)} (premio al giocatore);
	\item
	      \textbf{risolva i problemi in maniera creativa e fantasiosa e funzionale} (premio al giocatore);
	\item
	      \textbf{abbia buona collaborazione e sappia interagire come gruppo gruppo verso i png} (premio al gruppo od al giocatore);
	\item
	      \textbf{scopra o avvii indizi di avventura e creazione di nuovi plot} (premio al giocatore);
	\item
	      \textbf{raccolga 5000 monete d'oro (o tesoro equivalente)}(premio di gruppo);
\end{itemize}

\bigskip

Premiate il giocatore/giocatori con 1 punto esperienza. Questi punti vanno dati per sessione di gioco al chi si li è meritati come singolo o gruppo. Non c'e' bisogno di dare i punti esperienza a fine sessione di gioco, tenetene traccia e informate i giocatori quando c'e' un momento di pausa, di riflessione su quanto accaduto e fatto.

In questo sistema sono necessarie circa 10 sessioni per passare di livello, potenzialmente anche molte meno se i giocatori si dimostrano bravi ed interpretano personaggi e situazioni in maniera brillante.

Fate in modo che ogni sessione possa assegnare 1-4 punti almeno. Costruite la sessione perché tutti i giocatori possano essere partecipi e nessuno si senta escluso.

Nel limite del possibile ogni sessione dovrebbe includere una parte di ruolo, una parte di esplorazione, una parte di combattimento (anche molte più di una), una parte di riposo.

\bigskip

\begin{narratoresmall}
Può sembrare anacronistico quando c'è già in sviluppo la sesta edizione del più famoso gioco di ruolo tornare a premiare i giocatori in base all'oro preso ai mostri. 

Posso però garantirvi che qualora il vostro gruppo sia particolarmente "povero" di giocate di ruolo o semplicemente preferisca uno stile più combattivo, sapere che l'oro raccolto equivale ad esperienza può rendere molto più dinamico ed avvincente l'andare in avventura.

DBS si rifà al movimento OSR e come tale la fase di esplorazione e combattimento ha un proprio peso importante e vitale.
\end{narratoresmall}

\subsection{Incontri}\index{Incontri}

\label{incontri}

\begin{enfasismall}{Che è la vita senza speranza? Una gittata di dadi fra le tenebre, fra i deliri. Ambrogio Bazzero}
\end{enfasismall}


Un incontro è un momento di tensione e speranza, paura e sfida. E' l'occasione di mostrare e manifestare le proprie capacità e di lavorare come gruppo.

Un incontro non è l'occasione per fare sfoggio del proprio potere assoluto, sia come Narratore, che come Giocatore. Il Narratore saprà \xcancel{punire} fare capire il giocatore che vuole essere oltre il gruppo e non parte di esso.

Troverete nelle pagine seguenti le istruzioni per creare delle sfide facili (0 punti esperienza), medie e alte (1 punto esperienza), straordinarie (2 punti esperienza) ed epiche (3 punti esperienza).

Sarete comunque sempre voi, il Narratore, a stabilire e sapere se una sfida è provante o meno, se è sfidante e critica per i giocatori e quindi valutarne sia l'impatto come punti esperienza che come difficoltà.

Un incontro è un evento che mette i personaggi di fronte ad un problema specifico che devono risolvere. Molti sono combattimenti con i mostri o i PNG ostili, ma ce ne sono altri tipi: un corridoio irto di trappole, un'interazione politica con un re sospettoso, un passaggio pericoloso sopra un ponticello di corda traballante, un argomento scomodo con un PNG amichevole che ritiene che un personaggio lo abbia tradito, o qualsiasi cosa che aggiunga un pò di drammaticità al gioco.

Rompicapi, sfide interpretative e prove di abilità sono i metodi classici per la risoluzione degli incontri, ma gli incontri più complessi da costruire sono i più comuni incontri di combattimento.

Uno scontro può anche nascere palesemente sbilanciato, sarà l'accortezza dei giocatori a capire quando scappare!

Nel progettare un incontro di combattimento, in primo luogo decidete che livello di sfida volete far fronteggiare ai PG, quindi seguite i punti descritti qui di seguito.

\textbf{Determinare APL}: \index{APL}Determinate il livello medio dei personaggi: questo è il Livello Medio del Gruppo (APL in breve, Average Party Level). Dovreste arrotondate questo valore al numero intero più vicino (questa è una delle poche eccezioni alla regola dell'arrotondamento per difetto).

Si noti che questa guida di riferimento alla creazione di un incontro presuppone un gruppo di quattro o cinque PG. Se il vostro gruppo ha sei o più giocatori, aggiungete uno al loro livello medio. Se il vostro gruppo contiene tre o meno giocatori, sottraete uno dal loro livello medio. Per esempio, se il vostro gruppo consiste di sei giocatori, due di 4° livello e quattro di 5° livello, il APL è il 6° (28 livelli totali, diviso per sei giocatori, arrotondando per eccesso e aggiungendo uno al risultato finale).

\begin{center}
	\includegraphics[width=0.8\linewidth]{immagini/impegnativa.png}
	
	\textit{Henry Justice Ford}	
\end{center}

\textbf{Determinare il grado di Sfida}: Il Grado di Sfida (o grado di Sfida, GS) è un numero di convenienza usato per indicare i rischi relativi presentati da un mostro, una trappola, un pericolo o un altro incontro: più il grado di Sfida è alto, più pericoloso è l'incontro. Riferitevi alla Tabella: Determinare gli Incontri per determinare il Grado di Sfida che il vostro gruppo dovrebbe affrontare, in base alla difficoltà della sfida che volete e al APL.

\medskip

\textbf{Tabella: Determinare gli Incontri}\index{Tabella Determinare gli Incontri}

\medskip

\begin{tabular}{ll}
	\textbf{difficoltà} & \textbf{Grado di Sfida (grado di Sfida)}\\
\toprule
	Facile               & APL -1\\
	Media& APL\\
	Alta & APL +1\\
	Straordinaria        & APL +2\\
	Epica& APL +3\\
\end{tabular}

\subsubsection{Costruire l'Incontro}


Per costruire un incontro come prima cosa calcolate il valore dell' APL.

Per sviluppare il vostro incontro, aggiungete le creature, le trappole ed i pericoli finché non arrivate a vostro APL programmato.

Parti calcolando le sfide con grado di Sfida più alto dell'incontro, completando il resto con sfide minori.

Per esempio, volete che il vostro gruppo di sei personaggi di 7° livello affronti alcuni Gargoyle (grado di Sfida 2 ciascuno), degli Xorn (grado di Sfida 5) e il loro capo, un Gigante delle Pietre (grado di Sfida 7). I personaggi hanno APL 8 e la Tabella: Determinare gli Incontri stabilisce che una sfida Alta per un APL 8 è un incontro di grado di Sfida 9.

Partendo da un grado di Sfida stabilito (9) seguite questa tabella per stabilire quanti "mostri inserire nello scontro".

\medskip

\textbf{Tabella: Peso grado di Sfida per calcolo APL}\index{Tabella Peso grado di Sfida per calcolo APL}

\medskip

\begin{tabularx}{0.45\textwidth}{XXX}
	\textbf{Sfida obiettivo} & \textbf{Sfida creatura rispetto ad obiettivo APL} & \textbf{"Peso" per singola creatura}\\
\toprule
	grado di Sfida        & -7                    & 5\\
	                      & -6                    & 10\\
	                      & -5                    & 15\\
	                      & -4                    & 20\\
	                      & -3                    & 30\\
	                      & -2                    & 50\\
	                      & -1                    & 75\\
	                      & 0                     & 90\\
	                      & +1					  & 100\\	
\end{tabularx}

\bigskip

\textbf{Per raggiungere l'obiettivo dobbiamo sommare "i pesi" fino a raggiungere 100, ovvero il 100\% della sfida.}

Nel nostro esempio un Gigante delle Pietre ha grado di Sfida 7, ovvero un grado di Sfida -2 rispetto al nostro obiettivo di difficoltà grado di Sfida 9, lo Xorn ha grado di Sfida 5 ovvero -4 rispetto al grado di Sfida 9, i Gargoyle hanno grado di Sfida 2 ovvero -7 rispetto al grado di Sfida 9.

Un nemico con grado di Sfida -2 ha peso 50, un grado di Sfida -4 ha peso 20, un grado di Sfida -7 ha un peso di 5.

Per raggiungere l'obiettivo di un grado di Sfida 9 metterò 1 grado di Sfida -2 ( ovvero 1 gigante delle pietre), 2 grado Sfida -4 (ovvero due Xorn) e 2 grado di Sfida -7 (ovvero 2 gargoyle). Il Totale sarà 50 (Gigante) +2*20 (due Xorn)+2 + 2*5 (gargoyle) = 50+40+10 = 100. Obiettivo raggiunto!

Avversari con grado di Sfida inferiore a 7 rispetto al APL si contano, "pesano", solo se sono superiori a 20 come unità.

\subsubsection{Aggiungere i PNG}

Una creatura che possiede livelli, Abilità, competenze, che potrebbe essere un personaggio si considera un PNG. Queste creature possono svolgere un ruolo molto importante e non vanno usate come semplici "mostri". Dategli uno spessore e creerete dei personaggi indimenticabili.

\subsubsection{Modifiche ad Hoc del grado di Sfida}

Mentre potete modificare il grado di Sfida specifico del mostro avanzandolo, applicando modifiche o livelli, potete anche aggiustare la difficoltà dell'incontro applicando modifiche ad hoc all'incontro o alla creatura in sé.

Qui descritti ci sono tre modi aggiuntivi con cui potete alterare la difficoltà dell'incontro.


\begin{center}
	\includegraphics[width=0.7\linewidth]{immagini/tesoro2.png}
\end{center}


\subsubsection{Terreno Favorevole ai PG}

Un incontro contro un mostro che non è nel suo elemento preferito (come uno Yeti incontrato in una caverna piena di lava, o un Drago enorme incontrato in una stanza molto piccola) da ai personaggi un vantaggio. Sviluppate l'incontro normalmente considerate l'incontro come se avesse un grado di Sfida più basso del suo grado di Sfida reale.

\subsubsection{Terreno Sfavorevole ai PG}

I mostri sono progettati con il presupposto che siano incontrati nel loro terreno preferito: incontrare un Aboleth sott'acqua non aumenta il grado di Sfida dell'incontro, anche se nessun personaggio è in grado di respirare sott'acqua.

Se, d'altra parte, il terreno ha un impatto più significativo sull'incontro (come un incontro contro una creatura con Vista Cieca in una zona che sopprime ogni fonte di luce), si possono, nel caso, aumentare il grado di Sfida dell'incontro fosse di un grado più alto.


\subsubsection{Modifiche all'Equipaggiamento dei PNG}

Potete aumentare o diminuire la difficoltà data dai PNG modificandone l'Equipaggiamento. Un PNG incontrato senza Equipaggiamento dovrebbe avere un grado di Sfida ridotto di 1 (a condizione che la perdita di Equipaggiamento sia realmente controproducente per il PNG), mentre un PNG che ha un Equipaggiamento equivalente a quello di un personaggio (come indicato sulla Tabella: Ricchezza dei Personaggi per Livello) ha un grado di Sfida superiore di 1 al suo grado di Sfida reale.

Occorre prestare attenzione ad assegnare ai PNG questo equipaggiamento supplementare, specie ai livelli più alti, in cui potete consumare l'intero tesoro della vostra avventura in un colpo solo!

\begin{narratoresmall}
	Ma quanti scontri gestire per sessione ?
	
	Non c'e' una risposta esatta (2/4 al giorno ???), molto dipende dal tipo di giocatori che avete e da che gioco fanno.
	Rimanete concentrati sull'avventura, non andate a pensare che e' meglio non stancare i personaggi se no non reggono il prossimo scontro. Non dovete preoccuparvi eccessivamente dei personaggi, questo e' compito dei giocatori. Voi dovete ragionare in base all'ambiente dove siete ed a chi c'e' in giro.
	
	In ogni caso il buon senso aiuta sempre. Stancate i personaggi ma fate in modo che non sia un "Total party kill" ad ogni sessione.
\end{narratoresmall}

\subsubsection{Assegnare i PX}

I personaggi avanzano di livello sconfiggendo mostri, superando sfide e completando l'avventura: nel farlo guadagnano i Punti Esperienza (PX in breve). Potete assegnare Punti Esperienza non appena una sfida viene superata, ma ciò potrebbero interrompere il flusso del gioco. E' più facile assegnare i punti esperienza alla fine di una sessione di gioco (o più sessioni) che permetta ai personaggi di riflettere su quanto accaduto. Può usare il tempo a disposizione fra le sessioni di gioco per aggiornare la scheda.

\subsubsection{Disporre Tesori}

Mentre i personaggi avanzano di livello, anche la quantità di tesori che trasportano ed usano aumenta. In DBS si suppone che tutti i personaggi di pari livello abbiano più o meno la stessa quantità di tesoro e oggetti magici. Poiché il reddito primario per un personaggio deriva dai tesori e dai bottini ricavati dalle avventure, è importante moderare la ricchezza e i tesori nelle vostre avventure.\\
Per aiutarvi nel disporre i tesori, la quantità di oggetti magici e di bottino che i personaggi ricevono per le loro avventure è legata al grado di Sfida degli incontri che affrontano: più è alto il grado di Sfida dell'incontro, maggiore sarà il tesoro assegnato.

\textbf{La Tabella: Ricchezza dei Personaggi per Livello} per Livello indica la quantità di tesoro che ogni personaggio dovrebbe avere ad un livello specifico. Si noti che questa tabella si basa su un modello standard di gioco.

Le avventure con magia rara potrebbero assegnare soltanto la metà di questo valore, mentre i giochi più epici potrebbero raddoppiarlo. Si presume che parte del tesoro sia consumato nel corso di un'avventura (come pozioni e pergamene) e che alcuni degli oggetti meno utilizzati siano venduti per metà del loro valore per acquistare un Equipaggiamento più utile.

La Tabella: Ricchezza dei Personaggi per Livello può anche essere usata per stanziare l'Equipaggiamento per i personaggi che cominciano dopo il 1° livello, come un nuovo personaggio creato per sostituirne uno morto. I personaggi non dovrebbero spendere più della metà della loro ricchezza totale su un singolo oggetto.

Per un metodo equilibrato, i personaggi che vengono creati dopo il 1° livello dovrebbero spendere il 25\% della loro ricchezza per le armi, il 25\% per armatura e oggetti di protezione, il 25\% per altri oggetti magici, il 15\% per oggetti che si consumano come bacchette, pergamene e pozioni e il 10\% per un Equipaggiamento normale e monete. Tipi di personaggio differenti potrebbero spendere diversamente la loro ricchezza rispetto a come suggerito; ad esempio, gli incantatori arcani potrebbero spendere di più per oggetti magici e a consumo che per le armi.

\textbf{La Tabella: Valori del Tesoro per Incontro} elenca la quantità di tesoro che ogni incontro dovrebbe assegnare in base al livello medio dei personaggi e alla velocità di progressione di PX della campagna. Gli incontri facili dovrebbero assegnare un tesoro di un livello più basso rispetto al livello medio dei PG. Gli incontri più pericolosi, difficili ed eroici dovrebbero assegnare rispettivamente un tesoro di uno, due o tre livelli superiore al livello medio dei PG. Se nel gioco la magia è rara, dimezzate questi valori. Se il gioco è più epico, raddoppiate questi valori.

\medskip

\textbf{Tabella: Valori del Tesoro per Incontro}\index{Tesoro}\index{Tabella Valori del Tesoro per Incontro}

\begin{tabularx}{0.45\textwidth}{XX|XX}

	\textbf{Livello Medio Gruppo} & \textbf{per Incontro (mo)} & \textbf{Livello Medio Gruppo} & \textbf{per Incontro (mo)}\\
\toprule
	1             & 130        & 11            & 3500\\
	2             & 250        & 12            & 4500\\
	3             & 400        & 13            & 5500\\
	4             & 5500       & 14            & 7500\\
	5             & 750        & 15            & 9000\\
	6             & 1000       & 16            & 12000\\
	7             & 1300       & 17            & 16000\\
	8             & 1700       & 18            & 20000\\
	9             & 2100       & 19            & 25000\\
	10            & 2500       & 20            & 32000\\
\end{tabularx}
\bigskip

Usate questa tabella per avere un indicazione del valore del tesoro del "mostro".

Un tesoro Doppio significa che un gruppo di esempio 7 APL troverà l'equivalente di 2600 mo circa.

Un tesoro accidentale indica che potranno esserci tesori, oggetti, monete solo "accidentalmente" ottenuti, magari uccidendo altre creature. Detta diversamente la creatura non raccoglie i tesori, ma potrebbe comunque avere accumulato qualche cosa.

Suggerisco di valutare un tesoro accidentale la metà di un tesoro normale. Un tesoro accidentale si presta benissimo a fornire oggetti particolari che possono dare indicazioni o fornire nuovi plot di avventura (per cosa era questa ampolla forgiata in adamantio ?)...


\begin{narratoresmall}
Un ultimo consiglio e' fare in modo che i tesori trovati nei dungeon siano distribuiti secondo questo criterio:

- un terzo li avranno i mostri addosso

- un terzo saranno nascosti dietro passaggi segreti o trappole

- un terzo sarà sparso in giro

Questo stimolerà i giocatori a continuare l'esplorazione, affrontare i mostri e cercare attivamente nel dungeon.
\end{narratoresmall}

\medskip

\textbf{Tabella: Ricchezza dei Personaggi per Livello}\index{Tabella Ricchezza dei Personaggi per Livello}

\bigskip

\begin{tabular}{ll|ll}
	\textbf{Livello} & \textbf{Ricchezza} & \textbf{Livello} & \textbf{Ricchezza}\\
\toprule
	1            & 100               & 11            & 82000\\
	2            & 500               & 12            & 55000\\
	3            & 1500              & 13            & 75000\\
	4            & 3000              & 14            & 100000\\
	5            & 5000              & 15            & 120000\\
	6            & 8000              & 16            & 160000\\
	7            & 12000             & 17            & 210000\\
	8            & 17000             & 18            & 270000\\
	9            & 25000             & 19            & 350000\\
	10           & 35000             & 20            & 500000\\
\end{tabular}

\medskip

Gli incontri contro dei PNG solitamente ricompensano con un tesoro tre volte superiore a quello con un mostro, grazie all'Equipaggiamento del PNG. Per compensare, assicuratevi che i personaggi affrontino un paio di incontri supplementari che assegnano poco in fatto di tesori.

Animali, Vegetali, Costrutti, Non Morti non intelligenti, Melme e trappole sono ottimi "incontri con poco tesoro". In alternativa, se i personaggi affrontano un certo numero di creature con poco o nessun tesoro, dovrebbero avere l'occasione di ottenere un certo numero di oggetti di valore più significativo nell'immediato futuro per compensare lo squilibrio. Come regola generale, i personaggi non dovrebbero possedere alcun oggetto magico di valore superiore alla metà della ricchezza totale del personaggio, pertanto controllate bene prima di ricompensare i personaggi con oggetti molti costosi.

\subsubsection{Costruire un Bottino}\index{Costruire un Bottino}

Spesso è sufficiente dire ai vostri giocatori che hanno trovato 5.000 mo in gemme e 10.000 mo in gioielli. Ma a volte, è più interessante fornire dei particolari. Dare a un tesoro una personalità può non solo aiutare la verosimiglianza del gioco, ma può a volte innescare nuove avventure.

Le informazioni nelle pagine seguenti possono aiutarvi per determinare tipi di tesori in modo casuale: per molti degli oggetti sono stati dati dei valori, ma potete assegnarli come ritenete meglio. E' più facile collocare gli oggetti più costosi prima: se volete potete anche determinare gli oggetti magici in modo casuale usando le tabelle in Oggetti Magici, per stabilire quali oggetti siano presenti nel tesoro.

Una volta che avete consumato una parte considerevole del valore del tesoro, il resto può semplicemente essere composto da monete sparse e oggetti non magici con valori definiti in base alle vostre esigenze.

\textbf{Monete}: Le monete in un tesoro possono essere di rame, argento, oro e platino: quelle d'argento e d'oro sono le più comuni, ma potete decidere diversamente. Per le monete ed il loro valore di cambio andate all'Equipaggiamento.

\textbf{Gemme}: Anche se potete assegnare qualsiasi valore ad una gemma, alcune possono valere di più delle altre. Utilizzate le categorie di valore qui sotto (e le pietre preziose associate) come guida di riferimento quando assegnate i valori alle pietre preziose.

\textbf{Gemme di Bassa Qualità} (10 mo): agata; azzurrite; quarzo blu; ematite; lapislazzuli; malachite; ossidiana; rodocrosite; occhio di tigre; turchese; perla di fiume (irregolare).

\textbf{Gemme Semi Preziose} (50 mo): eliotropio, corniola; calcedonio; crisoprasio; citrino; diaspro; lunaria; onice; crisolito; cristallo di roccia (quarzo chiaro); sardonica; sardonice; quarzo rosato, affumicato o rosa di stella; zircone.

\textbf{Pietre Preziose di Media Qualità} (100 mo): ambra; ametista; crisoberillo; corallo; granato rosso o verde-marrone; giada; giaietto; perla bianca, dorata, rosa o argentata; spinello rosso, marrone-rosso o verde scuro; tormalina.

\textbf{Pietre Preziose di Alta Qualità} (500 mo): alessandrite; acquamarina; granato viola; perla nera; spinello blu scuro; topazio giallo oro.

\textbf{Gioielli} (1.000 mo): smeraldo; opale bianco, nero, o di fuoco; zaffiro blu; corindone giallo fuoco o vermiglio; zaffiro a stella blu o nero.

\textbf{Gioielli Eccezionali} (5.000 mo o piu'): smeraldo verde brillante cristallino, diamante, giacinto, rubino.

\textbf{Tesori non Magici} Questa categoria include monili, abiti raffinati, merci, oggetti alchemici, oggetti perfetti e altri.

Diversamente delle gemme, molti di questi oggetti hanno valori stabiliti, ma potete sempre aumentare il valore dell'oggetto decorandolo con pietre preziose o con fatture particolarmente artistiche.

Questo aumento di costo non conferisce capacità aggiuntive: una scimitarra di Ferro Freddo impreziosita da gemme del valore di 40.000 mo funziona come una normale scimitarra di Ferro Freddo da 330 mo. Qui di seguito trovate numerosi esempi di tesori non magici, con i valori tipici.

\textbf{Oggetti d'Arte Raffinati} (100 mo o piu'): Anche se alcuni oggetti d'arte sono composti di materiali preziosi, il valore della maggior parte di pitture, sculture, opere letterarie, abiti raffinati, e simili consiste nella fattura con cui sono realizzati e nella bravura di chi li ha realizzati. Gli oggetti d'arte sono spesso ingombranti o difficili da spostare, e fragili, rendendone il recupero ed il trasporto un'avventura a sé.

\textbf{Monili Minori} (50 mo): Questa categoria comprende monili realizzati con materiali come ottone, bronzo, rame, avorio, o legni esotici, a volte impreziositi con gemme di bassa qualità molto piccole o difettate. I monili minori includono anelli, braccialetti e orecchini.

\textbf{Monili Normali} (100-500 mo): La maggior parte dei monili è realizzata con argento, oro, giada, o corallo, e decorata spesso con gemme semi preziose o pietre preziose di qualità media. I monili normali comprendono tutti i tipi di monili minori più bracciali, collane e spille.

\textbf{Monili Preziosi} (500 mo o piu'): I monili preziosi sono realizzati in oro, mithral, platino, o simili metalli rari. Tali oggetti comprendono i tipi di monili normali più scettri, pendenti ed altri grandi oggetti.

\textbf{Attrezzi fatti ottimamente} (100-300 mo): Questa categoria include attrezzi per le Professioni o Competenze: vedi Equipaggiamento per i dettagli e i costi di questi oggetti.\\

\medskip

\includegraphics[width=0.8\linewidth]{immagini/Hoxne_Hoard_1.png}

\textit{Riproduzione del tesoro di Hoxne}

\medskip

\textbf{Oggetti Comuni} (fino a 1.000 mo): Ci sono molti oggetti di valore di natura alchemica o comune che possono essere utilizzati come tesoro. La maggior parte degli oggetti alchemici sono oggetti portabili e stimabili, ma anche altri come serrature, simboli sacri, cannocchiali, vini prelibati o abiti raffinati possono costituire parti interessanti di un tesoro. Anche le merci commerciali possono servire da tesoro: 5 kg di zafferano, per esempio, valgono 150 mo.

\textbf{Mappe del Tesoro e Oggetti d'Informazione} (variabili): Gli oggetti come mappe del tesoro, documenti legali di navi e case, liste di informatori o dei turni di guardia, parole d'accesso, e simili possono essere divertenti oggetti da trovare in un tesoro: potete stabilire il valore di questi oggetti come volete e possono essere di doppia utilità in quanto possono generare idee per nuove avventure.

\textbf{Oggetti Magici}

Naturalmente, la scoperta di un Oggetto Magico è il vero premio per qualsiasi avventuriero. Fate attenzione a collocare gli Oggetti Magici in un tesoro: è molto più soddisfacente per molti giocatori trovare un oggetto magico piuttosto che comprarlo, così non è sbagliato mettere degli oggetti che poi verranno usati dai giocatori!

Anche se in genere dovreste collocare gli oggetti con attenta riflessione sui loro probabili effetti sulla vostra campagna, può essere divertente generare gli oggetti magici in un tesoro a caso. Fate attenzione, comunque! è facile, con un pò di fortuna (o sfortuna) dei dadi gonfiare il vostro gioco con troppo tesoro o privarlo dello stesso. Il collocamento di oggetti magici casuali dovrebbe essere temperato sempre dal buon senso del Narratore.

Vedi capitolo sugli oggetti magici per le tabelle di creazione del tesoro magico.

Anche gli incantesimi sono veri e propri tesori e premi al pari di oggetti magici. Valutate con attenzione quali possono essere trovati. Ricordate che una abilità magica non è un incantesimo copiabile, solo quelli presenti nei tomi, pergamene, bastoni e quant'altro appositamente creato per essere un ricettacolo di incantesimi è idoneo alla copia.

\subsection{Recitare}\index{Recitare}

\label{recitare}

Un gioco di ruolo non è un semplice tirare dadi, è un incontro di pensieri, opinioni, sfide, lotte. E' un gioco catartico, liberatorio, evolutivo ed istruttivo.

E' giusto che ci sia combattimento, lotta, sangue paura ed azione, allo stesso modo deve esserci la possibilità di giocare i propri personaggi con i loro svantaggi, vantaggi, poteri e storie e anche drammi personali.

Il giocatore deve sempre impersonare il personaggio, immedesimarsi e partecipare attivamente.

Ci possono essere situazioni di contorno, gestite velocemente, che vengono fatte in terza persona, eppure ogni volta che si rende necessario giocare allora deve essere vero, fatto dal giocatore calandosi appieno nel personaggio.

\medskip

Quando un giocatore interpreta bene e descrive l'azione che va a svolgere in maniera partecipativa, coinvolgente, ispirata, dategli un premio, concedete un bonus di +1 all'azione che sta svolgendo

\medskip

Fatelo presente al giocatore che grazie alla sua interpretazione ha quel bonus.

Allo stesso tempo potrebbero esserci situazioni che si rivelano sgradevoli da gestire e giocare per qualche giocatore. Fate molta attenzione in questo caso, andare contro la sensibilità di un giocatore, di un amico, non è come andare contro l'etica o morale di un personaggio. Se percepite un senso di disagio ed imbarazzo fermate subito il gioco e chiarite la situazione con i giocatori e riprendete solo quando vi sarete accordati su come modificare la situazione per evitare che accada di nuovo.

\subsection{Circa DBS ed i tiro di dadi}

DBS usa un sistema di tiro di dadi peculiare andando a mescolare una distribuzione 3d6 (normalizzata) ad il potenziale dei 6 che esplodono. Questo sistema riesce a garantire una buona varianza e pur se accentrando i risultati intorno ai valori centrali della distribuzione lascia aperto il limite superiore a tiri particolarmente fortunati.

Se volete divertirvi a studiare la curva corrispondente vi consiglio il sito  \href{https://anydice.com/}{Anydice} e se volete gia' lo schema usato in Dungeon Bell System potete usare questo link \href{https://anydice.com/program/945}{3d6 explore}

\subsection{Le avventure in DBS} \hypertarget{OSR}{} \index{OSR}\index{Avventura in DBS}

Suggerisco la lettura integrale dell'articolo  \href{https://lithyscaphe.blogspot.com/p/principia-apocrypha.html} {Principia Apocrypha} ( https://lithyscaphe.blogspot.com/p/principia-apocrypha.html ), quello che segue è un sunto da me adattato e modificato delle linee guida che seguo quando masterizzo DBS.

DBS segue i principi dell' \href{https://it.wikipedia.org/wiki/Old_School_Renaissance}{OSR }(wikipedia). Le avventure in DBS mirano ad essere letali, avere un mondo liberamente esplorabile, una trama abbozzata, spingere sul problem-solving ed avere un sistema di ricompense incentrato sull'esplorazione e sui tesori e sulla partecipazione al gruppo. DBS non si cura troppo del bilanciamento degli incontri e apprezza l'intraprendenza dei giocatori e cattura le loro idee mettendole nell'avventura.

Per me l'OSR non sono tabelle di incontri casuali e randomizzazione caotica, e' piuttosto lo spirito di avventura, meraviglia, paura, gloria e stupore che si sviluppa nelle avventure. Non siate troppo lineari, troppo prevedibili, aggiungete nelle avventure quel giusto mix che le rendono sempre uniche.

Se il metodo può non piacere usate quello che più vi aggrada, personalmente nel corso dei decenni ho imparato ad apprezzare e vedere apprezzato la spontaneità e naturalezza che i cardini dell'OSR portano nel gioco.

\bigskip

\textbf{Queste le regole di base, per il Narratore, per condurre le avventure.}\index{Linee guida per i Narratori}

\medskip

\begin{itemize}

\item
Tu sei il Narratore, tue le Regole, tuo il Mondo.

Non farti limitare dall'avventura, dal sistema, dall'elenco dei mostri, sentiti sempre libero di modificare e adattare in base alle necessità dell'avventura e del gruppo

\item
Ricordati di esser giusto e corretto. Improvvisa, adatta quanto vuoi ma sii coerente. Se stabilisci una regola (o una modifica ad una regola) seguila fino in fondo.

Allo stesso tempo se ti serve una regola e non la trovi usa il buon senso, è sicuramente la scelta giusta in quel momento.

Rispetta i dadi ed i risultati ottenuti, come capiteranno ai giocatori  capiteranno risultati particolari anche a te. E' giusto così.

\item
Non devi salvare il culo ai personaggi. Non sei il loro amico ne il loro nemico. Il tuo ruolo è di raccontare storie che nascono dalle storie dei personaggi, dalle loro azioni ed inazioni.

\item
Abbozza la storia, scrivi le parti centrali o da leggere ai giocatori ma non farti dominare o vincolare da quello che ti aspetti. Spesso e volentieri i giocatori ti stupiranno, meglio sapere dove si muovono e cosa hanno intorno per poter reagire sempre puntualmente.

Sono i giocatori a dare la direzione all'avventura e tu a dipanarla.

\item
Apprezza il caso e crea situazioni diverse dove i giocatori possono scegliere strade diverse o intrecciarne di nuove. E' la tua fortuna avere dei giocatori creativi che sanno sorprenderti.

\item
Non costringere nessuno a fare qualcosa, lascia sbagliare i giocatori, lascia che paghino le loro scelte. Non devi ostacolarli ne devi imbeccarli per una direzioni. Richiede da parte tua una immaginazione e capacità di adattamento non indifferente, ma sicuramente l'avventura ne gioverà.

\item
I personaggi sono esploratori, per definizione. Focalizza sull'esplorazione, più si esplora più si creano situazioni, più si creano agganci nell'avventura, più si conoscono altri pgn più ci sono zone da esplorare.

Fa capire che i tesori sono esperienza, in senso letterale e pratico. Non dovrai mai spingerli tu in un dungeon ma la loro brama di esperienza e tesoro.

\item
Fai risolvere i problemi ai giocatori e non ai personaggi. Lascia ruolare le scene, sono sempre meglio di un tiro di dado. Incoraggia il giocatori ad interagire e chiedere una prova solo come ultima chance. Proponi problemi che non debbano essere risolti per forza con un tiro di dado bensì piuttosto tramite più azioni, anche complesse.

Premia le azioni creative e le scelte coraggiose, prima più di tutto l'intelligenza e il volere trovare situazioni alternative e creative.

\item
Fa che i giocatori ti chiedano informazioni, si confrontino con l'ambiente e tra di loro. Incoraggia l'interazione con il mondo esterno e solo come ultima possibilità concedi un tiro di dado.

\item
E' naturale che i personaggi sappiano quello che sanno i giocatori, cerca di limitare questo interscambio, a beneficio delle competenze e capacità di tutti.

\item
Grandi sfide e rischi danno sempre grandi ricompense. Non deludere i giocatori (se non per scopo di avventura) negandogli il giusto tesoro o esperienza, piu' si addentreranno in profondità piu' i pericoli saranno letali maggiore sarà la ricompensa.

\item
Non devono esistere abitudini, consuetudini. Non creare uno standard. 
Cerca sempre di sorprendere i giocatori con mostri fuori luogo (ma che abbiano un senso), trappole anomale, ambienti alternativi. Situazioni diverse stimoleranno i giocatori a risolvere in maniera diversa ogni problema.

Prepara soluzioni diverse e accetta soluzioni diverse. Metti nell'avventura problemi e situazioni che nell'insieme permettano la soluzione, ogni stanza non dovrà essere un asettico ambiente ma contenere indizi e soluzioni per altri problemi anche senza una diretta soluzione.

\item
Accetta la morte. Un combattimento se tale è sempre letale, non avere paura di ferire o uccidere i personaggi. Falli ragionare, studiare il nemico, capire quale è il migliore approccio; ed infine stupiscili. I personaggi devono prima battere i nemici in astuzia e pianificazione, se vogliono sopravvivere. 
Se proteggi i personaggi la partita mancherà di tensione e i giocatori risolveranno tutti i problemi con la forza bruta.
I dungeon non devono essere ambienti per forza da svuotare dai mostri. Lo scopo dei mostri e limitare e orientare le azioni, consumare le opzioni.

Se i giocatori cercano sempre e comunque lo scontro frontale allora daglielo, come richiedono.

\item
Mantieni l'attenzione alta. Fa in modo che il passare del tempo abbia conseguenze, se i giocatori temono lo scorrere del tempo faranno scelte più ardite o forse sbagliate. Mantieni la tensione fra il desiderio di esplorare e fare bottino e il terrore di restare troppo a lungo. 

\item
Tu sei la sorgente delle informazioni, i giocatori le elaborano, i personaggi le usano.

Non nascondere informazioni che i personaggi devono sapere o sanno già, non dovrai fare il professore ma allo stesso modo fa in modo che siano consapevoli di ciò che hanno intorno.
Allo stesso tempo non devi rivelare tutto subito, falli indagare, curiosare. Come una cipolla le informazioni che otterranno saranno nascoste sotto strati di altre informazioni magari di minore importanza.

\item
Gli indizi creano situazioni. Lascia che i tuoi indizi, specifici e curiosi, attirino l'attenzione dei giocatori. Come un esca su un amo attira i giocatori in situazioni di dubbio, dove indagare e capire cosa succede.

Non infarcire l'avventura di dettagli inutili, lascia spazio alla creatività e immaginazione dei giocatori, i dettagli che però fornirai dovranno non solo avere un senso ma essere necessari all'avventura.

\item
Se i giocatori tendono a dimenticare le informazioni utili date cerca di sfruttare un PNG che abbia memoria o invitali a prendere appunti, non c'e' nulla di male nell'essere preparati.

\item
L'avventura non è mai statica ne tanto meno il mondo dove si muovono i personaggi.
Il mondo ha la stessa importanza se non di più dell'avventura stessa. Azioni dei giocatori possono scatenare accadimenti a livello globale. Pensate sempre alle conseguenze dei gesti.

\item
Se usi i PNG (personaggi non giocanti) non farli essere delle semplici macchiette, fa in modo che i personaggi si possano affezionare e considerare il PNG uno del gruppo alla pari di tutti gli altri.

\item
I mostri non devono essere stupidi per forza. Falli parlare, ragionare, scappare.. anche loro vogliono vivere!

\end{itemize}

\end{multicols}

\pagebreak

\section{Creare Oggetti Magici}\index{Creare Oggetti Magici}

\begin{enfasi}{
Creare è vivere due volte. (Albert Camus)}\end{enfasi}\medskip

\begin{multicols}{2}

\label{creare-oggetti-magici}

\lettrine[lines=2, lhang=0.33, loversize=0.25, findent=1.5em]{P}{er} Creare Oggetti Magici è necessario avere le Abilità Creazione oggetti magici.

I costi qui elencati sono quelli di produzione, il ricavo si può attestare almeno attorno al 20\% del prezzo di produzione.

La Difficoltà indicata è relativa ad una prova di Competenza Magica dell'incantesimo che si vuole inserire nell'oggetto.

In caso di fallimento nella prova di magia per creare un oggetto magico si sprecano componenti pari a metà del costo di produzione, oltre a tutto il tempo impiegato.

Conoscere l'incantesimo (o averlo a disposizione tramite Pergamena) che si applica all'oggetto è un requisito di ogni oggetto magico che si crea.

\subsubsection{Creare Anelli Magici}\index{Anelli Magici}

Per creare un anello magico, un personaggio ha bisogno di una fonte di calore. Ha anche bisogno di una provvista di materiali, di cui il più ovvio è un anello o pezzi di anello da assemblare. Il costo dei materiali è compreso nel costo della creazione dell'anello.

Il costo dell'anello è pari a Difficoltà*Difficoltà*50.

Quindi un Anello con Invisibilità costa 19*19*50=18050 mo

La Difficoltà della prova di magia per creare l'oggetto è pari al doppio della Difficoltà dell'incantesimo da applicare all'anello.

\begin{center}
	\includegraphics[width=0.3\linewidth]{immagini/onering2.png}
	
	\textit{Non c'è bisogno di dire che Anello sia...}
\end{center}

Un anello permette di fissare un incantesimo in un anello per rendere l'effetto sempre attivo.
L'anello deve avere un valore intrinseco pari almeno a 10mo*Difficoltà dell'incantesimo che deve ospitare.

Un anello può ospitare un incantesimo di Difficoltà 36 o se più incantesimi la massima singola Difficoltà è 31.

E' anche possibile inserire un incantesimo ad attivazione, in questo caso consultare i costi delle Verghe.

Forgiare un anello richiede 1 giorno per ogni 1.000 mo del prezzo base.

Talento di creazione oggetto richiesto: Creare Oggetti Magici Superiori. Se l'incantatore ha la Professione Gioielliere ottiene un +4 alle prova di creazione dell'oggetto.

In caso di più incantesimi i costi e tempi si sommano.

\subsection{Creare Armature Magiche}\index{Creare Armature Magiche}

Per creare un'armatura magica, un personaggio ha bisogno di una fonte di calore e di alcuni attrezzi per lavorare il ferro, il legno o il cuoio. Ha anche bisogno di una provvista di materiali, di cui il più ovvio è l'armatura stessa o i pezzi di armatura da assemblare. Un'armatura che va incantata deve essere di qualità.

\begin{center}
	\includegraphics[width=0.5\linewidth]{immagini/Rustning_Gustav_Vasa.png}
	
	\textit{Armour for Gustav I of Sweden by Kunz Lochner, c. 1540 (Livrustkammaren)}
\end{center}

Se i prerequisiti per la creazione dell'armatura comprendono degli incantesimi, l'incantatore deve conoscere detti incantesimi.

Un armatura magica +1 costa 2250 mo, +2 9000 mo, +3 15000 mo, +4 30000 mo, +5 60000 mo più il prezzo dell'armatura stessa.

Per riuscire a creare l'armatura è necessaria una prova di magia a difficoltà rispettivamente 15, 22, 27, 35, 42.

Infondere un incantesimo in una armatura ha un costo come se si andasse a creare un anello con quell'incantesimo.

Creare armature magiche richiede un giorno per ogni 1.000 mo del valore del prezzo base.

Talento di creazione oggetto richiesto: Creare Oggetti Magici. Se l'incantatore ha la Professione Fabbro ottiene un +4 alle prova di creazione dell'oggetto.

\subsection{Creare Armi Magiche}\index{Creare Armi Magiche}

Per creare un'arma magica, un personaggio ha bisogno di una fonte di calore e alcuni attrezzi per lavorare il ferro od il materiale con cui è fatta l'arma. Ha anche bisogno di una provvista di materiali, di cui il più ovvio è l'arma stessa o i pezzi di arma da assemblare. Solo un'arma di qualità può essere incantata per diventare un'arma magica, e il suo costo va aggiunto al costo totale di incantamento per determinare il valore finale di mercato.

Un'arma magica deve avere almeno bonus di +1 per avere una qualsiasi capacità speciale o incantesimo.

\medskip

\begin{center}
\includegraphics[width=0.5\linewidth]{immagini/exacaliburfuori.png}

\textit{The drawing of the sword from the stone, Henrietta Elizabeth Marshall's Our Island Story (1906)}	
\end{center}

\medskip

Se i prerequisiti per la creazione dell'arma comprendono degli incantesimi, l'incantatore deve conoscere detti incantesimi.

Nel momento della creazione, l'incantatore deve decidere se l'arma emana luce o meno, come effetto secondario della magia infusa nell'arma. Questa decisione non influenza il prezzo o il tempo di creazione, ma una volta che l'oggetto è completato, la decisione è definitiva.

Creare armi doppie viene considerato analogo a creare due armi per quanto riguarda il costo, il tempo e le Capacità Speciali.

Un Arma +1 costa 1500 mo, +2 5000 mo, +3 15000 mo, +4 35000 mo, +5 60000 mo più il prezzo dell'arma (influente solo se è di un qualche materiale raro o prezioso).

Una Freccia +1 costa 25 mo, +2 100 mo, +3 400 mo.

Per riuscire a creare un arma magica, o un set (12) di proiettili, è necessaria una prova di magia a difficoltà rispettivamente 15, 22, 27, 35, 42.

Infondere un incantesimo in un arma ha un costo come se si andasse a creare un anello con quell'incantesimo.

Creare un'arma magica richiede una giornata per ogni 1.000 mo del valore del prezzo base.

Talento di creazione oggetto richiesto: Creare Oggetti Magici.  Se l'incantatore ha la Professione Armaiolo ottiene un +4 alle prova di creazione dell'oggetto.

\subsection{Creare Bacchette}\index{Creare Bacchette}

\bigskip

\textbf{Costi Base delle Bacchette}

Il costo della Bacchetta è pari a Difficoltà*Difficoltà*10.

Quindi una Bacchetta con Invisibilità costa 19*19*10=3610 mo

La Difficoltà della prova di magia per creare l'oggetto è pari 10 + la Difficoltà dell'incantesimo da applicare alla bacchetta.

Una bacchetta è un oggetto magico che conserva in se un incantesimo caricato in precedenza.

Per ricaricare una bacchetta un incantatore deve infondere lo stesso incantesimo e avere l'Abilità Creare oggetti magici. La bacchetta recupera una carica ma l'incantatore oltre ad avere usato una magia spende l'equivalente di 5*Difficoltà monete d'oro in componenti.

Una Bacchetta può contenere come massima Difficoltà di incantesimo 26.

Per creare una bacchetta, un personaggio ha bisogno di una provvista di materiali, di cui il più ovvio è una bacchetta o i pezzi di una bacchetta da assemblare. Le bacchette sono sempre pienamente cariche (20 cariche) all'atto della creazione.

L'incantatore deve conoscere l'incantesimo che inserisce nella Bacchetta.

\begin{center}
	\includegraphics[width=0.5\linewidth]{immagini/wand.png}
\end{center}

Creare una bacchetta richiede 1 giorno per ogni 1.000 mo del valore del prezzo base.

Talento di creazione oggetto richiesto: Creare Oggetti Magici.  Se l'incantatore ha la Professione Falegname ottiene un +4 alle prova di creazione dell'oggetto.

\subsection{Creare Bastoni}\index{Creare Bastoni}

\textbf{Costi Base dei Bastoni}

\bigskip

Il costo del Bastone è pari a Difficoltà*Difficoltà*15.

Quindi un Bastone con Invisibilità costa 19*19*15=5415 mo

La Difficoltà della prova di magia per creare l'oggetto è pari 15 + la Difficoltà dell'incantesimo da applicare al Bastone.


\bigskip

Un Bastone è un oggetto magico dove si caricano una o più incantesimi.

Quando un bastone viene attivato è possibile usare un incantesimo alla volta.

Per creare un bastone, un personaggio ha bisogno di una provvista di materiali, di cui il più ovvio è un bastone o i pezzi di un bastone da assemblare.

I bastoni sono sempre pienamente carichi, 10 cariche, all'atto della creazione.

\begin{center}
	\includegraphics[width=0.3\linewidth]{immagini/staff2.png}
\end{center}


Una Bastone può contenere come massima Difficoltà di incantesimo 31, o in caso di diversi incantesimi la massima Difficoltà contenibile è 29.

Creare un bastone richiede 1 giorno per ogni 1.000 mo del prezzo base.

Talento di creazione oggetto richiesto: Creare Oggetti Magici Superiori.  Se l'incantatore ha la Professione Falegname ottiene un +4 alle prova di creazione dell'oggetto.


\subsection{Creare Pergamene}\index{Creare Pergamene}\index{Pergamene}\index{Isy Scrol}
\index{Pergamene facili}\index{Comprare incatesimi}
\medskip

Esistono due tipologie di Pergamene magiche, quelle eseguibili da tutti (dette ISY SCROL, o Facili) e quelle invece che richiedono la capacità magica di lanciare incantesimi, ovvero CA maggiore o uguale a 1.

Le pergamene facili hanno un costo di produzione pari a Difficoltà*Difficoltà*2.

Le pergamene normali, non facili, hanno un costo di produzione pari a Difficoltà*Difficoltà.


Il costo della pergamena va valutato in base alla rarità e Difficoltà dell'incantesimo. Un incantesimo molto raro o di alto Difficoltà ( Difficoltà maggiore uguale 29) può facilmente costare Difficoltà*Difficoltà*Difficoltà.

La Difficoltà della prova di magia per creare una pergamena è pari 10 (in caso di ISY SCROL) + la Difficoltà dell'incantesimo da applicare alla Pergamena. In caso di pergamene normali la Difficoltà della prova di magia per creare la pergamena è  5 + Difficoltà dell'incantesimo.

\begin{center}
	\includegraphics[width=0.4\linewidth]{immagini/scrool3.png}
\end{center}


Se una pergamena include più incantesimi il costo è pari alla somma dei vari incantesimi. Su una pergamena ISY SCROL non possono esserci incantesimi da pergamena normali e vice versa.

L'incantatore deve conoscere gli incantesimi che inserisce nella pergamena. Per preparare una pergamena è necessario 10 minuti di lavoro per Difficoltà.

Una pergamena può contenere come massima Difficoltà di incantesimo 36. In caso di più incantesimi la massima difficoltà contenibile è 31.

\medskip

Per leggere una pergamena è necessario:

\textbf{in caso di pergamene ISY SCROL}:

- per comprendere il contenuto è sufficiente una prova di Arcana a difficoltà DC 10

- per poter leggere e lanciare l'incantesimo della pergamena è necessaria una prova di Intelligenza (o Arcana se conosciuta) a difficoltà 12.


\textbf{in caso di pergamene normali}:

- per comprenderne il contenuto è necessaria una prova di Arcana a Difficoltà 15

- per poter leggere e lanciare l'incantesimo della pergamena è necessaria una prova di competenza magica a Difficoltà 20.


Talento di creazione dell'oggetto richiesto: Creare Oggetti Magici.  Se l'incantatore ha la Professione Calligrafo ottiene un +2 alle prova di creazione dell'oggetto, con l'abilità Decifrare scritti magici ottiene +2 nella creazione e +4 nel comprendere e lanciare l'incantesimo contenuto.

Competenza usata nella creazione: Arcana o Professione (scrivano).

\begin{center}
	\includegraphics[width=0.6\linewidth]{immagini/potion2.png}
	
	\textit{A witch, raising her arm above a flaming cauldron, recites a spell; a young woman kneels in front of the cauldron. Mezzotint by J. Dixon after J.H. Mortimer, 1773}
\end{center}

\subsection{Creare Pozioni}\index{Creare Pozioni}\index{Pozioni}

Una pozione contiene l'infuso di un solo incantesimo, ogni pozione è quindi monouso.

\medskip

Il costo della Pozione è pari a Difficoltà*Difficoltà.

Quindi una Pozione con Invisibilità costa 19*19=361 mo

La Difficoltà della prova di magia per creare l'oggetto è pari 5 + la Difficoltà dell'incantesimo da applicare alla Pozione.


\bigskip

Per creare una pozione, un personaggio ha bisogno di un piano di lavoro orizzontale e alcuni contenitori per mescere i liquidi, oltre a una fonte di calore per bollire l'infuso.

Una Pozione può contenere come massima Difficoltà di incantesimo 21.

Tutti gli ingredienti e i materiali per mescere una pozione devono essere freschi e mai usati.

L'incantatore deve conoscere l'incantesimo che inserisce nella pozione.

Talento di creazione oggetto richiesto: Distillare Pozioni (che concede un bonus di +4 alla prova).  Se l'incantatore ha la Professione Erborista oppure Alchimista ottiene un +2 alle prova di creazione dell'oggetto.



\subsection{Creare Verghe}\index{Creare Verghe}\index{Verghe}

Una verga è una bacchetta speciale che è capace di rigenerare le proprie cariche. Sono oggetti preziosi e molto costosi.

Per creare una verga, un personaggio ha bisogno di una provvista di materiali, di cui il più ovvio è una verga o i pezzi di una verga da assemblare.

\medskip

Il costo della Verga è pari a Difficoltà*Difficoltà*40.

Quindi una Verga con Invisibilità costa 19*19*40=14440 mo

La Difficoltà della prova di magia per creare l'oggetto è pari al doppio della Difficoltà dell'incantesimo da applicare alla Verga.

\bigskip

Una verga è in grado di lanciare 1 volta al giorno il proprio incantesimo. 

Moltiplicare il costo per 4 se è in grado di lanciarla 2 volte, moltiplicare per 8 se è in grado di lanciarla 3 volte al giorno.


Si può anche lanciare una volta in più nel giorno l'incantesimo contenuta nella verga, dopo di che la verga si distrugge.

Una Verga può contenere come massima Difficoltà di incantesimo 21.

L'incantatore deve conoscere l'incantesimo che inserisce nella Verga.

\begin{center}
	\includegraphics[width=0.25\linewidth]{immagini/Rod_of_asclepius.png}
	
	\textit{Il bastone di Asclepio è un antico simbolo greco associato alla medicina. Consiste in un serpente attorcigliato intorno ad una verga.}
\end{center}

Creare una verga richiede 1 giorno per ogni 1.000 mo del prezzo base.

Talento di creazione oggetto richiesto: Creare Oggetti Magici Superiori.  Se l'incantatore ha la Professione Gioielliere ottiene un +4 alle prova di creazione dell'oggetto.

\subsection{Aggiungere Nuove Capacità}\index{Aggiungere Nuove Capacita}

A volte, la mancanza di fondi o tempo rende impossibile realizzare l'oggetto magico voluto, ma fortunatamente è possibile potenziare o modificare un oggetto magico creato. Solo il tempo, l'oro ed i vari prerequisiti richiesti dalla nuova capacità che si vuole aggiungere all'oggetto magico pongono delle restrizione sul tipo di poteri addizionali che uno può infondere.

Il costo per aggiungere capacità addizionali ad un oggetto è lo stesso che se l'oggetto non fosse magico, meno il valore dell'oggetto originale. Quindi, una spada lunga +1 può diventare una spada lunga vorpal +2, e il costo della creazione è uguale a quello di una spada lunga vorpal +2 meno il costo di una spada lunga +1.

Quando si determina il prezzo di un oggetto magico inventato bisogna considerare molti fattori. Il modo più semplice per decidere il prezzo è confrontare il nuovo oggetto a un oggetto che ha già un prezzo, e usare tale prezzo come guida.

\end{multicols}

\pagebreak

\section{Regole su Oggetti Magici}\index{Regole su Oggetti Magici}\hypertarget{identificareom}{}

\begin{multicols}{2}

\lettrine[lines=2, lhang=0.33, loversize=0.25, findent=1.5em]{Q}{ueste} sono le indicazioni su l'utilizzo degli oggetti magici.


\label{oggetti-magici}
\begin{itemize}
	\item
	      Un personaggio può portare innumerevoli (12) oggetti magici su di sé ma per determinare il bonus alla Difesa non si possono sommare più di 2 oggetti (es. 1 anello magico ed un braccialetto). Armatura e Scudo non si considerano in questo conteggio.
	\item
	      Lo stesso principio vale per il bonus ai Tiri Salvezza, puoi sommare solo i bonus provenienti da due oggetti.
	\item
	      Se il bonus è alle Caratteristiche si conta solo quello con il bonus maggiore.
	\item
	      Un personaggio non può portare più di due anelli magici altrimenti entrano in risonanza causando 1d6 di danno a round per ogni anello oltre il secondo.
	\item
  		  Per riconoscere un oggetto magico e le sue capacità è necessario l'incantesimo Identificare. Una prova di Arcana a Difficoltà 25 può dare indicazioni di massima sui poteri.	      
\end{itemize}


\end{multicols}

\subsection{Tabella: Generazione Casuale delle Rarità degli Oggetti Magici}\index{Tabella Rarità Generazione Casuale}

Tira 1d100 e consulta la tabella, il risultato indica l'ammontare di tesoro casuale per rarità che i personaggi trovano.

\medskip

\begin{tabularx}{0.95\textwidth}{XXXXXX}
\textbf{D100} & \textbf{Comune} & \textbf{Non Comune} & \textbf{Raro} & \textbf{Molto Raro} & \textbf{Leggendario}\\
\toprule
1-5						&				&						&				&					& Maledetto\\	
6-35					&1&&&&\\	
36-40					&2&&&&\\					
41-45					&3&&&&\\ 
46-50					&0				&1			&&&\\	
51-55					&1				&1			&&&\\
56-60					&2				&1			&&&\\
61-65					&2				&2			&&&\\				
66-75					&3				&2			&&&\\
76-80					&3				&3			&&&\\				
81-85					&2				&1			&1&&\\				
86-90					&3				&2			&2&&\\
91-92					&3				&3			&2&&\\
93-94					&3				&3			&3&&\\
95-96					&3				&2			&2							&1		&\\
97-98					&3				&3			&3							&1		&\\
99						&3				&3			&3							&2		&\\
100					&3				&3			&3							&2		&1\\				
\end{tabularx}

\medskip


\subsection{Tabella: Generazione Casuale degli Oggetti Magici}\index{Generazione Casuale}\index{Tabella Generazione Casuale degli Oggetti Magici}

\label{tabella-generazione-casuale-degli-oggetti-magici}

\begin{tabularx}{0.95\textwidth}{XXXX}
	\textbf{Oggetto Comune} & \textbf{ Oggetto Non Comune} & \textbf{Oggetto Raro} & \textbf{Oggetto}    \\
\toprule
	01--04         & 01--10         & 01--10           & Armature e scudi\\
	05--09         & 11--20         & 11--20           & Armi\\
	10--44         & 21--30         & 21--25           & Pozioni\\
	45--46         & 31--40         & 26--35           & Anelli\\
	---            & 41--50         & 36--45           & Verghe\\
	47--81         & 51--65         & 46--55           & Pergamene\\
	---            & 66--68         & 56--75           & Bastoni\\
	82--91         & 69--83         & 76--80           & Bacchette\\
	92--100        & 84--100        & 81--100          & Oggetti meravigliosi\\
\end{tabularx}

\bigskip

\begin{multicols}{2}

\subsection{Taglia e Oggetti Magici}

\label{taglia-e-oggetti-magici}

Quando un capo di vestiario o un gioiello magici vengono scoperti, il più delle volte la taglia non è un problema: molti vestiti magici sono di facile utilizzo per tutti oppure si adattano magicamente a chi li indossa. Di regola, la taglia non dovrebbe impedire ai personaggi di varia tipologia fisica l'utilizzo di un oggetto magico.

Ci possono essere delle rare eccezioni, specie con gli oggetti realizzati per una razza specifica.

Le armi e le armature rinvenute casualmente hanno una probabilità del 30\% di essere Piccole (01--30), del 60\% di essere Medie (31--90), e del 10\% di essere di un'altra taglia.

\subsection{Oggetti Magici sul Corpo}\index{Oggetti Magici sul Corpo}

\label{oggetti-magici-sul-corpo}

Molti oggetti magici devono essere indossati da un personaggio che voglia usarli o beneficiare delle loro capacità. Per una creatura di forma umanoide è possibile indossare fino a 12 oggetti magici alla volta. Ognuno di questi oggetti deve essere indossato sopra una parte specifica del corpo denominata "slot".

Un corpo di forma umanoide può portare addosso Equipaggiamento magico consistente di un oggetto per ognuno dei gruppi seguenti, legato alla parte del corpo sulla quale viene indossato l'oggetto.

\textbf{Anello} (due al massimo): anelli.

\textbf{Vesti}: corazze, armature, tuniche e vesti

\textbf{Cintura}: cinture.

\textbf{Collo}: amuleti, collane, medaglioni, scarabei, spille e talismani.

\textbf{Mani}: guanti e guanti d'arme.

\textbf{Occhi}: occhi, occhiali e lenti.

\textbf{Piedi}: scarpe, stivali e pantofole.

\textbf{Polso}: braccialetti e bracciali.

\textbf{Scudo}: scudi.

\textbf{Spalle}: cappe e mantelli.

\textbf{Testa}: cappelli, diademi, elmi, maschere, corone, fasce e filatteri

\textbf{Torace}: camicie, giubbe, maglie e manti.

\medskip

Naturalmente, un personaggio può possedere quanti oggetti vuole di uno stesso tipo. Ma oggetti magici dello stesso tipo addizionali, oltre a quelli previsti negli slot, non funzioneranno.

\subsection{Tiri Salvezza Contro i Poteri degli Oggetti Magici}\index{Tiri Salvezza}

\label{tiri-salvezza-contro-i-poteri-degli-oggetti-magici}

Gli oggetti magici normalmente riproducono incantesimi o altri effetti magici. Per un Tiro Salvezza contro la magia o un effetto magico generato da un oggetto magico, la DC è sempre la Difficoltà dell'incantesimo generato se non specificato diversamente.

\subsection{Danneggiare gli Oggetti Magici}\index{Danneggiare gli Oggetti Magici}

\label{danneggiare-gli-oggetti-magici}

Un oggetto magico non deve compiere un Tiro Salvezza a meno che non sia incustodito, sia il bersaglio specifico dell'effetto, o il suo possessore ottenga un 3 naturale al suo Tiro Salvezza.

Gli oggetti magici hanno sempre diritto a un Tiro Salvezza contro qualcosa che potrebbe danneggiarli, anche quando un oggetto normale dello stesso tipo non avrebbe alcuna possibilità di effettuare un Tiro Salvezza. Gli oggetti magici usano sempre lo stesso bonus ai Tiri Salvezza, indipendentemente dal tipo (Tempra, Riflessi o Volontà). Il bonus ai Tiri Salvezza di un oggetto magico è pari a 2 + 1/2 della Difficoltà dell'incantesimo più potente che ospitano (oppure un +4 per ogni +1 che hanno). Le sole eccezioni a questa regola sono gli oggetti magici intelligenti, che effettuano i Tiri Salvezza su Volontà basandosi sul loro punteggio di Saggezza.


\subsection{Riparare gli Oggetti Magici}\index{Riparare gli Oggetti Magici}
\label{riparare-gli-oggetti-magici}

Per riparare un oggetto magico occorrono materiali e tempo, pari alla metà del tempo e del costo per crearlo.


\subsection{Cariche, Dosi e Usi Multipli}\index{Cariche}\index{Dosi}\index{Usi Multipli}

\label{cariche-dosi-e-usi-multipli}

Molti oggetti, e in modo particolare le bacchette e i bastoni, hanno un potere limitato al numero di cariche che contengono. Normalmente gli oggetti dotati di cariche non superano mai il massimo di 20 cariche (10 per i bastoni). Se oggetti simili vengono trovati come parte casuale di un tesoro, si tira un 5d6 e si divide per 2 per determinare il numero delle cariche rimaste (arrotondando per difetto, minimo 1). Se un oggetto ha un numero massimo di cariche diverso da 20, si tira casualmente per stabilire quante cariche sono rimaste.

I prezzi dati si riferiscono sempre agli oggetti al massimo delle loro cariche (quando un oggetto viene creato, ha sempre il massimo delle cariche). Se un oggetto perde di valore perché non ha più cariche (il che è valido per quasi tutti gli oggetti a cariche), il valore dell'oggetto parzialmente usato è pari al numero di cariche rimaste. Nel caso di oggetti che invece potrebbero avere un'utilità anche se privi di cariche, soltanto parte del valore dell'oggetto sarà basato sul numero di cariche rimaste.


\end{multicols}

\subsection{Acquisire Oggetti Magici}\index{Acquisire Oggetti Magici}\index{Tabella Acquisire Oggetti Magici}

\label{acquisire-oggetti-magici}

\bigskip

\begin{tabular}{lllll}
	\textbf{Dimensioni Comunità} & \textbf{Valore Base} & \textbf{Comune} & \textbf{Non Comune} & \textbf{Raro}\\
\toprule
	Insediamento  & 50mo & 1d2 oggetti     && \\
	Borgo         & 200mo& 1d4 oggetti     && \\
	Villaggio     & 500mo& 1d6 oggetti     & 1d2 oggetti    & \\
	Piccolo paese & 1000mo               & 1d4 oggetti     & 1d2 oggetti    & \\
	Grande paese  & 2000mo               & 1d6 oggetti     & 1d4 oggetti    & 1d2 oggetti\\
	Piccola città & 4000mo               & 2d4 oggetti     & 1d6 oggetti    & 1d4 oggetti\\
	Grande città  & 8000mo               & 3d4 oggetti      & 2d4 oggetti    & 1d6 oggetti\\
	Metropoli     & 16000mo              & {*}             & 3d4 oggetti    & 2d4 oggetti\\
\end{tabular}

{*} In una metropoli si trovano quasi tutti gli oggetti magici minori.

\begin{multicols}{2}

\bigskip

Gli oggetti magici sono preziosi e la maggior parte delle grandi città ha almeno uno o due fornitori di oggetti magici, dal semplice venditore di pozioni ad un fabbro specializzato nel forgiare spade magiche. Naturalmente, non ogni oggetto in questo manuale è disponibile in ogni città.

Le linee guida seguenti aiutano i Narratori a determinare quali oggetti sono disponibili in una specifica comunità. Esse presuppongono una campagna con un livello medio di magia. Alcune città potrebbero deviare di molto da questa linea di base a discrezione del Narratore. Il Narratore dovrebbe tenere una lista degli oggetti disponibili da ogni mercante e dovrebbe rimpinguare occasionalmente le scorte con nuove acquisizioni.

Il numero ed i tipi di oggetti magici disponibili in una comunità dipendono dalla sua dimensione. Ogni comunità ha un valore base legato ad essa (vedi Tabella: Oggetti Magici Disponibili).

c'è una probabilità del 75\% che qualsiasi oggetto di quel valore o inferiore si possa trovare in vendita facilmente in quella comunità. Inoltre, la comunità ha un certo numero di altri oggetti in vendita. Questi oggetti sono determinati a caso e sono ripartiti in categorie (minore, medio o maggiore).

Dopo aver determinato il numero di oggetti disponibili in ogni categoria, consultate la Tabella: Generazione Casuale degli Oggetti Magici per determinare il tipo di ogni oggetto (pozione, pergamena, anello, arma,ecc.) prima di passare alle tabelle specifiche per stabilire l'oggetto esatto. Ritirate ogni volta che gli oggetti non si adeguano al valore base della comunità.

Se l'uso della magia nella campagna in cui si gioca è raro, occorre dimezzare il valore base e il numero di oggetti in ogni comunità. Nelle campagne con magia estremamente rara o senza magia potrebbero non esserci affatto oggetti magici in vendita I Narratori che conducono questo tipo di campagne dovrebbe prevedere delle modifiche alle sfide affrontate dai personaggi data la mancanza di oggetti magici.

Le campagne con abbondanti oggetti magici potrebbero avere comunità con il doppio del valore base stabilito e degli oggetti magici casuali disponibili. In alternativa, si potrebbe stabilire che tutte le comunità siano di una categoria di dimensione maggiore allo scopo di stabilire gli oggetti magici disponibili. In una campagna con magia molto comune, tutti gli oggetti magici si possono acquistare in una metropoli.

Oggetti e attrezzi non magici sono in genere disponibili in una comunità di qualsiasi dimensione a meno che l'oggetto non sia molto costoso, come un'armatura completa, o fatto di un materiale insolito, come una spada lunga in adamantio. Questi oggetti dovrebbero seguire la linea guida del valore base per determinare la loro disponibilità, a discrezione del Narratore.

\end{multicols}

\pagebreak

\section{Gli Oggetti Magici}

\begin{multicols}{2}

\lettrine[lines=2, lhang=0.33, loversize=0.25, findent=1.5em]{T}{roverete} qui diversi oggetti magici da distribuire con parsimonia ed attenzione ai personaggi, come tesoro per i mostri uccisi o ricompensa per un eroica missione.

In molti oggetti è segnata la rarità di poterli trovare, usate questa indicazione per avere un idea su come darli ai giocatori o farli trovare.

Se non indicato diversamente una capacità magica usabile una volta al giorno si ricarica all'alba successiva all'uso.

\subsubsection{Note su Armature, Scudi, Armi magiche}

\textbf{Armi}: un'arma con una capacità speciale deve avere almeno bonus di +1. Le armi non possono avere la stessa capacità speciale più di una volta.


\textbf{Armature}: ogni +2 magico si abbassa di 1 la penalità di Prove Destrezza, ed ogni +1 si abbassa di 1 la penalità alle prove di Competenza Magica.


\textbf{Scudi}: ogni +1 magico si abbassa di 1 il malus alla prova di CM.


\textbf{Armature e scudi}: non possono avere bonus alla Difesa superiore a +5.


\textbf{Attivare capacità magiche}: se non indicato diversamente attivare un abilità magica di un oggetto costa 2 Azioni.


Un oggetto magico che \textbf{manifesta incantesimi} non deve fare alcuna prova di magia per riuscirci. La Difficoltà di un eventuale Tiro Salvezza è pari alla Difficoltà dell'incantesimo manifestato se non specificato diversamente.

Per \textbf{identificare le capacità} di un oggetto magico è necessaria una prova di Arcana a difficoltà 25 (o farsi aiutare dall'incantesimo Identificare).


Un \textbf{arma può essere identificata} a seguito di due critici in un Tiro per Colpire oppure dedicando 1 ora di allenamento.


Un oggetto magico che fornisce un bonus (o malus...) statico applica il suo valore anche se l'oggetto non è stato identificato, sarà il Narratore ad applicare silenziosamente questo bonus alla Difesa, Tiro per Colpire, Tiri Salvezza... informando il giocatore che percepisce come l'oggetto interagisce con la situazione.


\textbf{Il costo di Armi e Armature:} di dimensioni superiori alle Medie è almeno il doppio (o quadruplo in base alla taglia). Armature piccole o Armi piccole pur richiedendo meno materiale costano la medesima cifra delle armi e armature medie.


\subsubsection{Elenco degli Oggetti Magici}

Gli oggetti magici sono presentati in ordine alfabetico. La descrizione di un oggetto magico fornisce il nome dell'oggetto, la sua categoria, rarità e le sue proprietà magiche.

Benché i costi siano riportati è sempre bene concedere gli oggetti magici come premi, tesoro, a seguito di missione.

In linea di massima un oggetto Comune, l'unico che potrebbe trovarsi facilmente in una grande città, puoi costare dai 50 ai 100 mo, uno Non Comune tra i 150 ed i 500 mo, uno Raro tra i 500 e i 5000 mo, uno Molto Raro fino a 50000 mo.

Oggetti con un bonus oltre il +3, o Leggendari, non si comprano mai, deve essere un epica avventura a farli trovare.

Gli oggetti senza prezzo semplicemente non sono in vendita e quelli con un prezzo indicato con un + ZXY mo significa che sono abilità e capacità da aggiungere al costo di un altro oggetto.


\medskip

\includegraphics[width=0.8\linewidth]{immagini/Excalibur_His_Sheath.png}

\textit{Il fodero di Excalibur}

\medskip

Anche gli incantesimi sono oggetti magici e come tali, se il Narratore permette, possono essere acquistati (orrore! non c'è nulla di più bello trovare un nuovo incantesimo tra i tesori di un'avventura).

Un incantesimo costa in monete d'oro Difficoltà*Difficoltà*Difficoltà/2 

\medskip 

o se preferite \(\frac{Difficolta'^{3}}{2}\) (adoro Latex!)


\bigskip

\index{Ali del Volo}\textbf{Ali del Volo}

\textit{Oggetto meraviglioso, raro} - 54000 mo

Mentre indossi questa cappa, puoi usare due azioni per pronunciare la sua parola di comando, trasformandola in un paio di ali da pipistrello o da uccello che spuntano dalla tua schiena per 1 ora o finché non ripeti la parola di comando con un'azione. Le ali ti forniscono velocità di volo 18 metri. Quando scompaiono, non potrai più usarle per 1d12 ore.


\index{Ammazza Draghi}\textbf{Ammazza Draghi}

\textit{Arma (qualsiasi spada), raro} - 8000 mo

Ottieni un bonus di +1 ai Tiri per Colpire e danno effettuati con quest'arma magica. Quando colpisci un drago con quest'arma, il drago subisce 3d6 danni aggiuntivi del tipo dell'arma. Ai fini di quest'arma, "drago" è qualsiasi creatura del tipo drago. 


\index{Ammazza Giganti}\textbf{Ammazza Giganti}

\textit{Arma (qualsiasi ascia o spada), raro} - 7000 mo

Ottieni un bonus di +1 ai Tiri per Colpire e danno effettuati con quest'arma magica. Quando colpisci un gigante con quest'arma, il gigante subisce 2d6 danni aggiuntivi del tipo dell'arma e deve superare un Tiro Salvezza su Tempra con DC 18 o cadere prono. Ai fini di quest'arma, "gigante" qualsiasi creatura del tipo gigante.


\index{Ampolla di Ferro}\textbf{Ampolla di Ferro}

\textit{Oggetto meraviglioso, leggendario}

Questa bottiglia di ferro ha un tappo di ottone. Puoi usare due azioni per pronunciare la parola di comando dell'ampolla, prendendo come bersaglio una creatura visibile entro 18 metri da te. Se il bersaglio è nativo di un piano di esistenza diverso da quello in cui ti trovi, deve superare un Tiro Salvezza su Volontà con DC 21 o venir intrappolato nell'ampolla. Se il bersaglio è già stato intrappolato nell'ampolla, riceve +1d6 al Tiro Salvezza. Una volta intrappolata, la creatura rimarrà nell'ampolla finché non sarà liberata. L'ampolla può contenere solo una creatura alla volta. Una creatura intrappolata nell'ampolla non ha bisogno di respirare, mangiare o dormire e non invecchia. Puoi usare due azioni per rimuovere il tappo dell'ampolla e liberare la creatura che contiene. La creatura sarà amichevole verso di te e i tuoi compagni per 1 ora e obbedirà ai vostri comandi per quella durata. Se non le impartisci comandi o gliene dai uno che provocherebbe la sua morte, si difenderà ma non compirà altre azioni. Al termine della durata, la creatura agirà in base al suo normale comportamento


L'incantesimo identificare rivela che una creatura si trova all'interno dell'ampolla, ma l'unico modo per determinare che sorta di creatura sia è di aprire l'ampolla. Un'ampolla di ferro appena scoperta potrebbe già contenere una creatura scelta dal Narratore o determinata casualmente. 


\medskip

\begin{tabular}{ll}
\toprule
d100 &Contiene\\
1-50 &Vuota\\
51-66 &Demone \\
67 &Angelo Deva\\
68-69 &Diavolo (superiore)\\
70-73 &Diavolo (inferiore)\\
74-75 &Genio Djinni\\
76-77 &Genio Efreeti\\
78-83 &Elementale (qualsiasi)\\
84-86 &Persecutore invisibile\\
87-90 &Megera notturna\\
91 &Angelo Planetar\\
92-95 &Salamandra\\
96 &Angelo Solar\\
97-99 &Succube/Incubo\\
100 &Xorn\\
\end{tabular}
\medskip

\index{Amuleto dei Piani}\textbf{Amuleto dei Piani}

\textit{Oggetto meraviglioso, raro} - 160000 mo

Mentre indossi questo amuleto, puoi usare due azioni per nominare un luogo con cui sei familiare e che si trovi su di un altro piano di esistenza. Effettua una prova di Intelligenza con DC 18. Se la prova riesce, lanci l'incantesimo spostamento planare tramite l'amuleto. Se la prova fallisce, tu e ciascuna creatura e oggetto entro 4,5 metri da te venite trasportati in una destinazione casuale. Tira un 1d8. Da 1 a 4, raggiungi una destinazione casuale sul piano che hai nominato. Da 5 a 8, raggiungi un piano dell'esistenza determinato casualmente.


\index{Amuleto di Protezione dalla Individuazione e Localizzazione}\textbf{Amuleto di Protezione dalla Individuazione e Localizzazione} - 20000 mo

\textit{Oggetto meraviglioso, non comune}

Mentre indossi questo amuleto sei celato alla magia di divinazione. Non puoi essere preso come bersaglio da queste magie o percepito tramite sensori magici di scrutamento.


\index{Amuleto della Salute}\textbf{Amuleto della Salute}

\textit{Oggetto meraviglioso, raro} - 8000 mo

Mentre indossi questo amuleto hai un +4 ai Tiri Salvezza su Tempra.


\index{Anello Accumula Incantesimi}\textbf{Anello Accumula Incantesimi}

\textit{Anello, raro} - 24000 mo

Questo anello immagazzina gli incantesimi lanciati su di esso, conservandoli fino a che chi lo indossa non ne faccia uso. L'anello può accumulare fino a 3 Incantesimi per un totale di Difficoltà massimo di 61, con un massimo di Singola Difficoltà 26 .

Qualsiasi creatura può lanciare un incantesimo di Difficoltà da 12 a 26 sull'anello, toccandolo mentre lancia l'incantesimo.

L'incantesimo non ha effetto, oltre quello di essere immagazzinato nell'anello. Se l'anello non può contenere l'incantesimo, l'incantesimo viene speso senza effetti.

Mentre indossi questo anello, puoi lanciare gli incantesimi che contiene. L'incantesimo usa la Difficoltà minima per determinare la DC del Tiro Salvezza dell'incantesimo, il bonus per colpire dell'incantesimo e la caratteristica da incantatore dell'incantatore originale, ma per il resto è considerato come se fosse stato lanciato da te.

Un incantesimo lanciato tramite questo anello non è più contenuto al suo interno, e libera spazio per altri incantesimi.


\index{Anello dell'Ariete}\textbf{Anello dell'Ariete}

\textit{Anello, raro} - 5000 mo

Mentre indossi questo anello, puoi usare due azioni per spendere da 1 a 3 cariche per attaccare una creatura visibile entro 18 metri da te.

L'anello produce una testa di ariete spettrale ed effettua il suo Tiro per Colpire con un bonus di +7. Se colpisci, per ogni carica spesa, il bersaglio subisce 2d10 danni da forza e viene spinto di 1,5 metri lontano da te.

In alternativa, puoi spendere da 1 a 3 cariche dell'anello con due azioni per tentare di rompere un oggetto visibile entro 18 metri da te che non sia indossato o trasportato.

L'anello effettua una prova con Forza +5 ogni carica spesa.

Questo anello ha 3 cariche, e recupera 1d3 cariche spese ogni mattina all'alba.


\index{Anello di Caduta Morbida}\textbf{Anello di Caduta Morbida}

\textit{Anello, raro} - 2000 mo

Mentre cadi e indossi questo anello, scendi di 9 metri per round e non subisci danni dalla caduta.


\index{Anello di Camminare sull'Acqua}\textbf{Anello di Camminare sull'Acqua}

\textit{Anello, non comune} - 1500 mo

Mentre indossi questo anello, puoi stare in piedi o muoverti su qualsiasi superficie liquida come se fosse terreno solido.


\index{Anello del Calore}\textbf{Anello del Calore}

\textit{Anello, non comune } - 5000 mo

Mentre indossi questo anello, hai resistenza ai danni da freddo. Inoltre, tu e tutto quello che indossi e trasporti siete immuni agli effetti delle temperature basse fino a -45° C.


\index{Anello del Comando degli Elementali dell'Acqua}\textbf{Anello del Comando degli Elementali dell'Acqua}

\textit{Anello, leggendario}
Questo anello è collegato al Piano Elementale dell'Acqua.

Mentre lo indossi, hai +1d6 ai Tiri per Colpire contro gli elementali del Piano Elementale dell'Acqua, ed essi hanno -1d6 ai Tiri per Colpire effettuati contro di te.

Puoi spendere 2 cariche dell'anello per lanciare dominare mostri su di un elementale dell'acqua.
 Inoltre, puoi stare in piedi e camminare sulle superfici liquide come se fossero terreno solido.
 
Puoi parlare e comprendere l'Aquan. 

Se aiuti a uccidere un elementale dell'acqua mentre indossi l'anello, ottieni accesso alle seguenti proprietà aggiuntive:

\medskip

\begin{itemize}
\item
Puoi respirare sott'acqua e hai velocità di nuovo pari alla tua velocità di passeggio.
\item
Puoi lanciare tramite l'anello i seguenti incantesimi, spendendo il numero di cariche richieste: creare o distruggere acqua (1 carica), controllare tempo atmosferico (3 cariche), muro di ghiaccio (3 cariche) o tempesta di ghiaccio (2 cariche).
L'anello ha 5 cariche. Recupera 1d4 + 1 cariche ogni giorno all'alba.

Gli incantesimi lanciati tramite l'anello hanno DC del Tiro Salvezza 21.

\end{itemize}

\index{Anello del Comando degli Elementali dell'Aria}\textbf{Anello del Comando degli Elementali dell'Aria}

\textit{Anello, leggendario}

Questo anello è collegato al Piano Elementale dell'Aria.

Mentre lo indossi, hai +1d6 ai Tiri per Colpire contro gli elementali del Piano Elementale dell'Aria, ed essi hanno -1d6 ai Tiri per Colpire effettuati contro di te.

Puoi spendere 2 cariche dell'anello per lanciare dominare mostri su di un elementale dell'aria. Inoltre, quando cadi, scendi di 18 metri per round e non subisci danni dalla caduta.

Puoi parlare e comprendere l'Auran. 

Se aiuti a uccidere un elementale dell'aria mentre indossi l'anello, ottieni accesso alle seguenti proprietà aggiuntive:

\medskip

\begin{itemize}
	\item
Hai resistenza ai danni da fulmine.
\item
Hai velocità di volo pari alla tua velocità di passeggio e puoi fluttuare.
\item
Puoi lanciare tramite l'anello i seguenti incantesimi, spendendo il numero di cariche richieste: catena di fulmini (3 cariche), folata di vento (2 cariche) o muro di vento (1 carica).
\end{itemize}
\medskip

L'anello ha 5 cariche. Recupera 1d4 + 1 cariche ogni giorno all'alba.

Gli incantesimi lanciati tramite l'anello hanno DC del Tiro Salvezza 21.


\index{Anello del Comando degli Elementali del Fuoco}\textbf{Anello del Comando degli Elementali del Fuoco}
\textit{Anello, leggendario}

Questo anello è collegato al Piano Elementale del Fuoco.

Mentre lo indossi, hai +1d6 ai Tiri per Colpire contro gli elementali del Piano Elementale del Fuoco, ed essi hanno -1d6 ai Tiri per Colpire effettuati contro di te.

Puoi spendere 2 cariche dell'anello per lanciare dominare mostri su di un elementale del fuoco.

Inoltre, hai resistenza ai danni da fuoco. 

Puoi parlare e comprendere l'Ignan.

Se aiuti a uccidere un elementale del fuoco mentre indossi l'anello, ottieni accesso alle seguenti proprietà aggiuntive:

\medskip

\begin{itemize}
	\item
Hai immunità ai danni da fuoco.
	\item
	Puoi lanciare tramite l'anello i seguenti incantesimi, spendendo il numero di cariche richieste: mani brucianti (1 carica), muro di fuoco (3 cariche) o palla di fuoco (2 cariche).
\end{itemize}

\medskip

L'anello ha 5 cariche. Recupera 1d4 + 1 cariche ogni giorno all'alba.

Gli incantesimi lanciati tramite l'anello hanno DC del Tiro Salvezza 21.


\index{Anello del Comando degli Elementali della Terra}\textbf{Anello del Comando degli Elementali della Terra}

\textit{Anello, leggendario}

Questo anello è collegato al Piano Elementale della Terra.

Mentre lo indossi, hai +1d6 ai Tiri per Colpire contro gli elementali del Piano Elementale della Terra, ed essi hanno -1d6 ai Tiri per Colpire effettuati contro di te.

Puoi spendere 2 cariche dell'anello per lanciare dominare mostri su di un elementale della terra.
Inoltre, puoi muoverti su terreno difficile composto da macerie, pietre o terra come se fosse terreno normale.

Puoi parlare e comprendere il Terran.

Se aiuti a uccidere un elementale della terra mentre indossi l'anello, ottieni accesso alle seguenti proprietà aggiuntive:


\medskip

\begin{itemize}
\item
Hai resistenza ai danni da acido.
\item
Puoi muoverti attraverso la terra o la roccia solida come se fossero terreno difficile. Se vi termini il tuo round, vieni proiettato fuori nello spazio non occupato più vicino che hai occupato per ultimo. 
\item
Puoi lanciare tramite l'anello i seguenti incantesimi, spendendo il numero di cariche richieste: scolpire pietra (2 cariche), muro di pietra (3 cariche) o pelle di pietra (1 carica).
\end{itemize}

\medskip

L'anello ha 5 cariche. Recupera 1d4 + 1 cariche ogni giorno all'alba.

Gli incantesimi lanciati tramite l'anello hanno DC del Tiro Salvezza 21.


\index{Anello di Elusione}\textbf{Anello di Elusione}

\textit{Anello, raro} - 5000 mo

Mentre indossi questo anello e fallisci un Tiro Salvezza su Riflessi, puoi usare la tua reazione per spendere 1 carica per riuscire il Tiro Salvezza che hai appena fallito. Questo anello ha 3 cariche, e recupera 1d3 cariche spese ogni mattina all'alba.


\index{Anello dell'Evocazione dello Djinni}\textbf{Anello dell'Evocazione dello Djinni}

\textit{Anello, leggendario}

Mentre indossi quest'anello, puoi pronunciarne la parola di comando con due azioni per evocare uno specifico djinni del Piano Elementale dell'Aria. Lo djinni compare in uno spazio non occupato a tua scelta, entro 36 metri da te. Resta finché rimani concentrato (come se ti concentrassi su di un incantesimo), per un massimo di 1 ora, o finché non scende a 0 punti ferita.Poi ritorna al suo piano natio.

Finché resta evocato, lo djinni è amichevole verso di te e i tuoi compagni. Obbedisce a qualsiasi comando gli dai, non importa la lingua usata. Se non gli impartisci ordini, lo djinni si difenderà dagli attacchi ma non effettuerà nessun'altra azione.

Dopo la partenza dello djinni, esso non potrà più essere evocato prima che siano passate 24 ore, e se lo djinni muore l'anello perde la sua magia. 


\index{Anello di Influenza sugli Animali}\textbf{Anello di Influenza sugli Animali}

\textit{Anello, raro} - 4000 mo

Mentre indossi questo anello, puoi usare due azioni per spendere 1 delle sue cariche per lanciare tramite esso uno dei seguenti incantesimi: amicizia con gli animali (DC del Tiro Salvezza 15), parlare con gli animali, paura (DC del Tiro Salvezza 15, prende come bersaglio solo bestie che hanno Intelligenza -3 o meno).

Questo anello ha 3 cariche, e recupera 1d3 cariche spese ogni giorno all'alba.


\index{Anello di Invisibilità}\textbf{Anello di Invisibilità}

\textit{Anello, leggendario} - 10000 mo

Mentre indossi quest'anello, puoi renderti invisibile con due azioni. Tutto ciò che indossi o trasporti diventa invisibile assieme a te. Resti invisibile finché l'anello non viene rimosso, attacchi o lanci un incantesimo, o finché non usi due azioni per tornare visibile.


\index{Anello di Libertà di Azione}\textbf{Anello di Libertà di Azione}

\textit{Anello, raro} - 20000 mo

Mentre indossi questo anello, il terreno difficile non ti costa movimento aggiuntivo. Inoltre, la magia non può né ridurre la tua velocità né renderti paralizzato o intralciato.

\index{Anello del Nuoto}\textbf{Anello del Nuoto}

\textit{Anello, non comune} - 3000 mo

Mentre indossi questo anello, hai velocità di nuoto 12 metri.


\index{Anello di Protezione}\textbf{Anello di Protezione}

\textit{Anello, raro} - 3500 mo, 8000mo, 15000 mo

Mentre indossi questo anello, hai un bonus da +1 a +3 alla Difesa e ai Tiri Salvezza.

\index{Anello Respingi Incantesimi}\textbf{Anello Respingi Incantesimi}

\textit{Anello, leggendario} - 30000 mo

Mentre indossi quest'anello, hai +1d6 ai Tiri Salvezza contro qualsiasi incantesimo che prende come bersaglio solo te e non un'area di effetto. Inoltre, se fai un critico sul Tiro Salvezza e l'incantesimo ha Difficoltà 29 o più basso, l'incantesimo non ha effetto su di te e invece prende come bersaglio l'incantatore.


\index{Anello di Rigenerazione}\textbf{Anello di Rigenerazione}

\textit{Anello, molto raro} - 12000 mo

Mentre indossi questo anello, recuperi 1d6 punti ferita ogni 10 minuti, purché ti rimanga almeno 1 punto ferita. Se perdi una parte del corpo, l'anello fa sì che la parte mancante ricresca e ritorni alla sua completa funzionalità in 1d6 + 1 giorni, purché per tutto il periodo ti rimanga sempre almeno 1 punto ferita. 


\index{Anello di Resistenza}\textbf{Anello di Resistenza}

\textit{Anello, raro} - 6000 mo

Mentre indossi questo anello, hai resistenza a un tipo di danno. La gemma incastonata nell'anello indica il tipo di danno, che viene scelto o determinato casualmente dal Narratore.


\medskip

\begin{tabular}{lll}
\textbf{d10} & \textbf{Tipo di Danno} & \textbf{Gemma}\\

\toprule
1 &Acido &Perla\\
2& Forza &Zaffiro\\
3& Freddo &Tormalina\\
4& Fulmine &Citrino\\
5& Fuoco &Granato\\
6& Vuoto& Giaietto\\
7& Psichico &Giada\\
8& Luce &Topazio\\
9& Tuono &Spinello\\
10& Veleno &Ametista\\
\end{tabular}

\medskip

\index{Anello del Salto}\textbf{Anello del Salto}

\textit{Anello, non comune} - 2500 mo

Mentre indossi questo anello, con due azioni puoi lanciare tramite esso l'incantesimo saltare a volontà, ma il bersaglio puoi essere solo tu. 


\index{Anello dello Scudo Mentale}\textbf{Anello dello Scudo Mentale}

\textit{Anello, non comune} - 16000 mo

Mentre indossi questo anello, sei immune alla magia che permette alle altre creature di leggere i tuoi pensieri, determinare se stai mentendo, conoscere i tuoi Tratti, o apprendere che tipo di creatura sei. Le creature possono comunicare telepaticamente con te solo se glielo concedi.

Puoi usare due azioni per far diventare invisibile l'anello fino a che un'altra azione non lo renderà di nuovo visibile, finché non lo rimuovi o muori. Se muori mentre indossi questo anello, la tua anima vi viene catturata, a meno che non ospiti già un'altra anima. Puoi decidere di rimanere nell'anello o raggiungere la vita ultraterrena. Finché la tua anima resta nell'anello, puoi comunicare telepaticamente con qualsiasi creatura lo indossi. Chi lo indossa non può impedire questa forma di comunicazione telepatica.


\index{Anello delle Stelle Cadenti}\textbf{Anello delle Stelle Cadenti}

\textit{Anello, molto raro} - 14000 mo

Mentre indossi questo anello a luce fioca o all'oscurità, puoi lanciare tramite esso luci danzanti e luce a volontà. Lanciare uno dei due incantesimi tramite l'anello richiede due azioni. L'anello ha 6 cariche per le seguenti altre proprietà. 

L'anello recupera 1d6 cariche spese ogni giorno all'alba.


\textit{Luminescenza}. Spendi 1 carica con due azioni per lanciare tramite l'anello l'incantesimo luminescenza. 
\textit{Palla di fulmini}. Puoi spendere 2 cariche con due azioni per creare da una a quattro sfere di fulmini di 1 metro di diametro. Più sfere crei, meno potente sarà ciascuna sfera individualmente.
Ogni sfera compare in uno spazio non occupato visibile entro 36 metri da te. La sfera dura finché ti concentri su di essa (come se ti concentrassi su di un incantesimo), fino a un massimo di 1 minuto. Ogni sfera irradia luce fioca in un raggio di 9 metri. Con due azioni puoi muovere ciascuna sfera di massimo 9 metri, ma senza superare i 36 metri di distanza da te. Quando una creatura, a parte te, si trova entro 1,5 metri da una sfera, la sfera scarica i fulmini contro quella creatura e poi scompare. Quella creatura deve effettuare un Tiro Salvezza su Riflessi con DC 18. Se fallisce il Tiro Salvezza, la creatura subisce danni da fulmine in base al numero di sfere da te creato (4 sfere, 2d4 danni; 3 sfere, 2d6 danni; 2 sfere, 5d4 danni; 1 sfera, 4d12 danni).

\textit{Stelle Cadenti}. Puoi spendere da 1 a 3 cariche con due azioni. Per ogni carica spesa, scagli una scintilla di luce dall'anello in un punto visibile entro 18 metri da te. Ogni creatura, in cubo di 4,5 metri di lato originante da quel punto, viene ricoperta di scintille e deve effettuare un Tiro Salvezza di Destrezza DC 15, subendo 5d4 danni da fuoco se lo fallisce, o la metà di questi danni se lo supera.


\index{Anello di Telecinesi}\textbf{Anello di Telecinesi}

\textit{Anello, molto raro} - 80000 mo

Mentre indossi questo anello, puoi lanciare a volontà l'incantesimo telecinesi, ma puoi prendere come bersaglio solo oggetti che non siano indossati o trasportati.


\index{Anello dei Tre Desideri}\textbf{Anello dei Tre Desideri}

\textit{Anello, leggendario}

Mentre indossi quest'anello, puoi usare due azioni per spendere 1 delle sue 3 cariche per lanciare tramite esso l'incantesimo desiderio. L'anello perde la sua magia quando usi l'ultima carica.


\index{Anello della Vista ai Raggi X}\textbf{Anello della Vista ai Raggi X}


\textit{Anello, raro} - 6000 mo

Mentre indossi questo anello, puoi usare due azioni per pronunciarne la parola di comando. Quando lo fai, puoi vedere attraverso la materia solida per 1 minuto. Questa vista ha un raggio di 9 metri. Per te, gli oggetti solidi all'interno del raggio appaiono trasparenti e non impediscono alla luce di attraversarli.

Questa vista può penetrare 30 centimetri di pietra, 2,5 centimetri di metallo comune o fino a 90 centimetri di legno o terra. Le sostanze più dense bloccano la vista, così come un sottile foglio di piombo. Ogni qualvolta usi di nuovo l'anello prima di aver terminato una notte di riposo devi superare un Tiro Salvezza su Tempra con DC 18 o diventare affaticato.


\index{Apparato del Granchio}\textbf{Apparato del Granchio}

\textit{Oggetto meraviglioso, leggendario} - 15000 mo

Quest'oggetto appare come un barile di ferro sigillato di taglia Grande e del peso di 250 chili. Il barile nasconde un fermo, che può essere trovato superando una prova di Intelligenza con DC 25. Rimuovere il fermo apre uno scomparto a una delle estremità dell'apparato, che permette a due creature di taglia Media o inferiore di entrarvi dentro. All'estremità opposta sono disposte dieci leve, ciascuna in posizione neutrale, in grado di muoversi verso l'alto o il basso. Quando vengono impiegate determinate leve, l'apparato si trasforma e assomiglia a un'aragosta gigante.

L'apparato è un oggetto Grande con le seguenti statistiche.

Difesa: 20

Punti Ferita: 200

Velocità: 9 m, nuoto 9 m (o 0 m entrambi se le gambe e la coda non vengono estese)

Immunità ai Danni: psichico, veleno

Per essere usato come veicolo, l'apparato necessita un pilota. Quando lo sportello dell'apparato viene chiuso, il compartimento è a tenuta stagna, e non fa filtrare aria o acqua. I compartimenti conservano aria sufficiente per 10 ore, divise per il numero di creature all'interno. L'apparato galleggia in acqua e può anche spingersi sott'acqua fino a una profondità di 270 metri. Al di sotto di questa soglia, l'apparato subisce 2d6 danni contundenti al minuto a causa della pressione. Una creatura all'interno del compartimento può usare due azioni per muovere verso l'alto o il basso fino a due leve. Dopo ciascun uso, la leva torna alla sua posizione neutrale. Ogni leva, da sinistra a destra, funziona come mostrato sulla tabella seguente.


1: Estende gambe e coda, permettendo all'apparato di camminare e nuotare. Ritrae gambe e coda, riducendo la velocità dell'apparato a 0 e rendendolo incapace di beneficiare di bonus alla velocità.

2: Apre l'oblò frontale. Chiude l'oblò frontale.

3: Apre gli oblò laterali (due per lato). Chiude gli oblò laterali (due per lato).

4: Estende due chele dal lato frontale dell'apparato. Ritrae le chele.

5: Effettua un attacco con arma da mischia con ciascuna chela estesa: +8 al Tiro per Colpire, portata 1,5 m, un bersaglio. Colpisce: 7 (2d6) danni contundenti. Effettua un attacco con arma da mischia con ciascuna chela estesa: +8 al Tiro per Colpire, portata 1,5 m, un bersaglio. Colpisce: Il bersaglio è afferrato (DC 18 per fuggire).

6: L'apparato cammina o nuota in avanti. L'apparato cammina o nuota indietro.

7: L'apparato svolta di 90 gradi a sinistra. L'apparato svolta di 90 gradi a destra.

8: Delle fessure frontali emettono luce intensa in un raggio di 9 metri e luce fioca per ulteriori  metri. Spegne le luci.

9: L'apparato affonda di 6 metri nei liquidi. L'apparato risale di 6 metri dai liquidi.

10: Sblocca e apre il portellone posteriore. Chiude e sigilla il portellone posteriore.


\index{Armi Magiche}\textbf{Arma} +1, +2, +3, +4, +5

\textit{Arma (qualsiasi)} - +1 1800 mo, +2 6000 mo, +3 17000 mo, +4 45000 mo, +5 80000 mo

Hai un bonus ai Tiri per Colpire e ai tiri di danno effettuati con quest'arma. Il bonus è determinato dalla rarità dell'arma.

Alcune armi magiche possiedono delle ulteriori proprietà, come l'emettere luce.


\index{Arma Perfida}\textbf{Arma Perfida}

\textit{Arma (qualsiasi), raro} - 1500 mo

Quando ottieni 17 o 18 al Tiro per Colpire con quest'arma magica, il bersaglio subisce 7 danni aggiuntivi del tipo dell'arma.


\index{Armature Magiche}\textbf{Armatura} +1, +2, +3, +4, +5

\textit{Armatura (qualsiasi)} - +1 2500 mo, +2 10000 mo, +3 18000 mo, +4 35000 mo, +5 80000 mo

Mentre indossi questa armatura ricevi un bonus alla tua Difesa. Il bonus è determinato dalla rarità dell'armatura.


\index{Armatura di Adamantio}\textbf{Armatura di Adamantio}

Armatura (media o pesante, ma non di pelle), non comune - +700 mo oltre il prezzo base dell'armatura

Mentre la indossi, qualsiasi colpo critico che subisci diventa un colpo normale (ma non esplosione del danno). 


\index{Armatura di Cuoio Borchiato Elegante}\textbf{Armatura di Cuoio Borchiato Elegante}

\textit{Armatura (cuoio borchiato), raro} - 2800 mo

Mentre la indossi, ottieni un bonus di +1 alla Difesa. Puoi anche usare due azioni per pronunciare la parola di comando dell'armatura e far sì che l'armatura assuma l'aspetto di un comune abito o qualche altro tipo di armatura. Decidi tu l'aspetto, compreso il colore, lo stile e gli accessori, ma l'armatura mantiene il suo normale ingombro e peso. L'aspetto illusorio dura finché non usi di nuovo questa proprietà o ti togli l'armatura.


\index{Armatura Demoniaca}\textbf{Armatura Demoniaca}

\textit{Armatura (di piastre), molto raro} - 2000 mo

Mentre la indossi, ottieni un bonus di +1 alla Difesa, e puoi comprendere e parlare l'Abissale. Inoltre, le manopole artigliate dell'armatura trasformano i colpi disarmati effettuati con le tue mani in armi magiche che infliggono danni taglienti, con un bonus di +1 ai Tiri per Colpire e ai tiri di danno e il d8 come dado di danno.

\textit{Maledizione}. Una volta indossata questa armatura maledetta, non potrai più rimuoverla a meno che non diventi bersaglio dell'incantesimo rimuovi maledizione o una simile magia. Mentre indossi questa armatura, hai -1d6 ai Tiri per Colpire contro i demoni e ai Tiri Salvezza contro i loro incantesimi e capacità speciali.


\index{Armatura di Maglia Elfica}\textbf{Armatura di Maglia Elfica}

\textit{Armatura (giaco di maglia), raro} - 4000 mo

Mentre la indossi, ottieni un bonus di +1 alla Difesa. Sei considerato competente con quest'armatura anche se non sei competente con le armature medie. Il peso di questo giaco è la metà del normale (\textit{felpa!}).


\index{Armatura di Mithral}\textbf{Armatura di Mithral}

\textit{Armatura (media o pesante, ma non di pelle), non comune} +800 mo oltre il prezzo base dell'armatura

Il mithral è un metallo leggero e flessibile. Un giaco di maglia o un pettorale di mithral possono essere indossati sotto abiti normali. Riduce di 1 la categoria di peso per determinare malus alle prove di Destrezza e Magia


\index{Armatura di Piastre dell'Eterealità}\textbf{Armatura di Piastre dell'Eterealità}
\textit{Armatura (di piastre), leggendario} - 48000 mo

Mentre la indossi, con due azioni puoi pronunciare la sua parola di comando per ottenere l'effetto dell'incantesimo forma eterea che dura 10 minuti o finché non rimuovi l'armatura o usi due azioni per pronunciare di nuovo la parola di comando. Questa proprietà dell'armatura non può essere usata di nuovo fino alla prossima alba.


\index{Armatura di Piastre dell'Invulnerabilità}\textbf{Armatura di Piastre dell'Invulnerabilità}

\textit{Armatura (di piastre), leggendario}

Mentre indossi questa armatura hai resistenza ai danni non magici. Inoltre, puoi usare due azioni per renderti immune ai danni non magici per 10 minuti o fino a che non starai più indossando questa armatura. Una volta usata l'armatura in questo modo, non potrai più usarla così fino alla prossima alba. 


\index{Armatura di Piastre Nanica}\textbf{Armatura di Piastre Nanica}

\textit{Armatura (di piastre), molto raro}

Mentre la indossi, ottieni un bonus di +4 alla Tiro per Colpire. Inoltre, se un effetto ti muove sul terreno contro la tua volontà, puoi usare la tua reazione per ridurre di 3 metri la distanza di cui sei mosso.

\index{Armatura di Resistenza}\textbf{Armatura di Resistenza}

\textit{Armatura (leggera, media o pesante), raro} - 9000 mo

Mentre la indossi, hai resistenza ad un tipo di danno. Il Narratore sceglie il tipo tra le opzioni seguenti: acido, forza, freddo, fulmine, fuoco, Vuoto, psichico, Luce, tuono o veleno.


\index{Armatura di Scaglie di Drago}\textbf{Armatura di Scaglie di Drago}

\textit{Armatura (di scaglie), molto raro} - 4000 mo

L'armatura di scaglie di drago è fatta con le scaglie di una specie di drago.

Mentre la indossi, ottieni un bonus di +1 alla Difesa, hai +1d6 ai Tiri Salvezza contro la Presenza Spaventosa e le armi a soffio dei draghi, e hai resistenza a un tipo di danno determinato dalla specie di drago che ha fornito le scaglie (Argento/freddo,Bianco/freddo, Blu/fulmine, Bronzo/fulmine, Nero/acido, Oro/fuoco, Ottone/fuoco, Rame/acido, Rosso/fuoco, Verde/veleno).

Inoltre, con due azioni puoi focalizzare i tuoi sensi per determinare magicamente la distanza e la direzione in cui si trovi il drago più vicino entro 45 chilometri che sia della stessa specie dell'armatura. Quest'azione speciale non può essere più usata fino alla prossima alba.


\index{Armatura di Vulnerabilità}\textbf{Armatura di Vulnerabilità}

\textit{Armatura (di piastre), raro} - 2000 mo

Mentre la indossi, hai resistenza a uno dei seguenti tipi di danno: contundente, perforante o tagliente. Il Narratore sceglie il tipo.

\textit{Maledizione}. L'armatura è maledetta, cosa che viene rivelata solo quando le viene lanciato sopra l'incantesimo identificare o la si indossa

Mentre sei maledetto, hai vulnerabilità a due dei tre tipi di danno associati con l'armatura (che
non siano quello a cui hai resistenza).


\index{Ascia del Berserker}\textbf{Ascia del Berserker}

\textit{Arma (qualsiasi ascia), raro} - 3000 mo

Hai un bonus di +1 ai Tiri per Colpire e ai tiri di danno effettuati con quest'arma magica. Inoltre, mentre quest'arma è la la tua arma principale, i tuoi punti ferita massimi aumentano di 1 per ogni livello di esperienza che possiedi.

\textit{Maledizione}. Finché rimani maledetto, non vorrai separarti dall'ascia, tenendola sempre a portata. Inoltre hai -1d6 ai Tiri per Colpire con armi diverse da questa, a meno che non ci sia alcun avversario che tu possa vedere o sentire nel giro di 18 metri da te.

Ogni qualvolta una creatura ostile ti danneggi mentre l'ascia è in tuo possesso, devi superare un Tiro Salvezza su Volontà DC 18 o entrare in berserk. Mentre sei in berserk, ogni round devi usare la tua azione per attaccare con l'ascia la creatura che si trovi più vicina a te entro 18 metri


\index{Bacchetta dei Dardi Incantati}\textbf{Bacchetta dei Dardi Incantati}

\textit{Bacchetta, non comune} - 8000 mo

Mentre impugni questa bacchetta, puoi usare due azioni per spendere 1 o più delle sue cariche per lanciare tramite essa l'incantesimo dardo incantato. Ogni carica genera 1 dardo.

La bacchetta ha 7 cariche. La bacchetta recupera 1d6 + 1 cariche spese all'alba di ciascun giorno.

Tuttavia, se spendi l'ultima carica della bacchetta, tira 3d6 se ottieni 5 o meno la bacchetta si riduce in polvere ed è distrutta.


\index{Bacchetta dei Fulmini}\textbf{Bacchetta dei Fulmini}

\textit{Bacchetta, raro} - 32000 mo

Mentre impugni questa bacchetta, puoi usare due azioni per spendere 1 o più cariche per lanciare tramite essa l'incantesimo fulmine (DC del Tiro Salvezza 18). 

Questa bacchetta ha 7 cariche. La bacchetta recupera 1d6 + 1 cariche spese all'alba di ciascun giorno. Tuttavia, se spendi l'ultima carica della bacchetta, tira 3d6 se ottieni 5 o meno la bacchetta si riduce in polvere ed è distrutta.


\index{Bacchetta di Individuazione del Magico}\textbf{Bacchetta di Individuazione del Magico}

\textit{Bacchetta, non comune} - 1500 mo

Mentre impugni questa bacchetta, con due azioni puoi spendere 1 carica per lanciare tramite essa l'incantesimo individuazione del magico. Questa bacchetta ha 3 cariche, e recupera 1d3 cariche spese ogni mattina all'alba.


\index{Bacchetta di Individuazione dei Nemici}\textbf{Bacchetta di Individuazione dei Nemici}

\textit{Bacchetta, raro} - 4000 mo

Mentre impugni questa bacchetta, puoi usare due azioni e spendere 1 carica per pronunciarne la parola di comando. Per il minuto successivo, conosci in che direzione si trovi la creatura ostile più vicina entro 18 metri da te, ma non la distanza che vi separa. La bacchetta può percepire la presenza di creature ostili che siano eteree, invisibili, camuffate, o nascoste, oltre che di quelle in piena vista. L'effetto termina se smetti di impugnare la bacchetta.

Questa bacchetta ha 7 cariche. La bacchetta recupera 1d6 + 1 cariche spese all'alba di ciascun giorno. Tuttavia, se spendi l'ultima carica della bacchetta, tira 3d6 se ottieni 5 o meno, la bacchetta si riduce in polvere ed è distrutta.


\index{Bacchetta del Mago da Guerra}\textbf{Bacchetta del Mago da Guerra, +1, +2 o +3}

\textit{Bacchetta, non comune (+1), raro (+2), o molto raro} (+3) - 1500 mo, 5500 mo, 25000 mo

Mentre impugni questa bacchetta, ottieni un bonus ai Tiri per Colpire con gli incantesimi determinato dalla rarità della bacchetta. Inoltre, ignori ka copertura leggera quando effettui un attacco con incantesimo.


\index{Bacchetta della Metamorfosi}\textbf{Bacchetta della Metamorfosi}

\textit{Bacchetta, molto raro} - 32000 mo

Mentre impugni questa bacchetta, puoi usare due azioni per spendere 1 carica per lanciare tramite essa l'incantesimo metamorfosi (DC del Tiro Salvezza 18). Questa bacchetta ha 7 cariche. La bacchetta recupera 1d6 + 1 cariche spese all'alba di ciascun giorno. Tuttavia, se spendi l'ultima carica della bacchetta, tira 3d6 se ottieni 5 o meno, la bacchetta si riduce in polvere ed è distrutta.


\index{Bacchetta delle Meraviglie}\textbf{Bacchetta delle Meraviglie}

\textit{Bacchetta, raro}

Mentre impugni questa bacchetta, puoi spendere 1 carica con due azioni e scegliere un bersaglio entro 36 metri da te. Il bersaglio può essere una creatura, un oggetto o un punto nello spazio. Il Narratore decide o determina casualmente cosa accadrà quando fai uso della bacchetta. Gli incantesimi lanciati tramite la bacchetta hanno DC del Tiro Salvezza 18. Se l'incantesimo normalmente ha una gittata espressa in metri, la gittata diventa 36 metri qualora non lo sia già. Se un effetto copre un'area, devi centrare l'incantesimo sul bersaglio e includervelo. Se un effetto più agire su più soggetti possibili, il Narratore determina casualmente chi sia affetto.

Questa bacchetta ha 7 cariche. La bacchetta recupera 1d6 + 1 cariche ogni giorno all'alba. Se spendi l'ultima carica della bacchetta, tira 3d6 se ottieni 5 o meno, la bacchetta si riduce in polvere ed è distrutta.

Ogni volta che fai uso della bacchetta delle meraviglie tira un d100 e consulta questa tabella.


\end{multicols}

\medskip

\begin{tabularx}{0.95\textwidth}{lX}
	\textbf{d100}& \textbf{Contenuti}\\
\toprule
	01-05 &Lanci lentezza.\\
	06-10 &Lanci fuoco delle fate.\\
	11-15 &Sei stordito fino all'inizio del tuo prossimo round, e ritieni che sia accaduto qualcosa di stupefacente.\\
	16-20 &Lanci folata di vento.\\
	21-25 &Lanci individuazione dei pensieri sul bersaglio da te scelto. Se 	il tuo a bersaglio non è una creatura, subisci invece 1d6 danni.\\
	26-30 &Lanci nube maleodorante.\\
	31-33 &Pioggia abbondante precipita in un raggio di 18 metri centrato 	sul bersaglio. L'area diventa oscurata leggermente. La pioggia 	continua a cadere fino all'inizio del tuo prossimo round.\\
	34-36 &Compare un animale nello spazio non occupato più vicino al 	bersaglio. L'animale non è sotto il tuo controllo e agisce come 	di norma. Tira un d100 per determinare che specie di animale compaia.	01-25, un rinoceronte; 26-50, un elefante; 51-100, un ratto.\\
	37-46 &Lanci fulmine.\\
	47-49 &Una nube di 600 enormi farfalle riempie un raggio di 9 metri 	intorno al bersaglio. L'area diventa oscurata pesantemente. Le 	farfalle restano per 10 minuti.\\
	50-53 &Ingrandisci il bersaglio come se avessi lanciato l'incantesimo 	ingrandire/ridurre. Se il bersaglio non può essere soggetto 	all'incantesimo, o se non è una creatura, tu diventi il bersaglio.\\
	54-58 &Lanci oscurità.\\
	59-62 &Erba folta spunta in un raggio di 18 metri intorno al bersaglio.	Se vi è già dell'erba, questa cresce di dieci volte e resta così per 	1 minuto.\\
	63-65 &Un oggetto a scelta del Narratore scompare sul Piano Etereo.	L'oggetto non deve essere né indossato né trasportato, deve essere entro 36 metri dal bersaglio, e non più grande di 3 metri in ciascuna dimensione.\\
	66-69 &Ti rimpicciolisci come se avessi lanciato su di te l'incantesimo ingrandire/ridurre.\\
	70-79 &Lanci palla di fuoco.\\
	80-84 &Lanci invisibilità su di te.\\
	85-87 &Sul bersaglio crescono delle foglie. Se hai scelto un punto nello spazio come bersaglio, le foglie spunteranno sulla creatura più vicina a quel punto. A meno che non vengano strappate, le foglie diventeranno marroni e cadranno dopo 24 ore.\\
	88-90& Un flusso di 1d4 x 10 gemme del valore di 1 mo ciascuna scaturisce dalla punta della bacchetta in una linea lunga 9 metri e larga 1,5 metri. Ogni gemma infligge 1 danno contundente, e 	il loro danno totale è diviso equamente tra tutte le creature sulla linea.\\
	91-95 &Una raffica di luci scintillanti e colorate si estende da te in un raggio di 9 metri. Tu e tutte le creature nell'area dovete superare un Tiro Salvezza su Tempra con DC 15 o restare accecati per 1 minuto. Una creatura può ripetere il Tiro Salvezza al termine di ciascun suo round, terminando l'effetto su di sé in	caso lo superi.\\
	96-97 &La pelle del bersaglio assume un colorito blu intenso per 1d10 giorni. Se hai scelto un punto nello spazio, il soggetto sarà la creatura più vicina a quel punto.\\
	98-00 &Se il bersaglio è una creatura, deve effettuare un Tiro Salvezza di 	Tempra con DC 18. Se il bersaglio non è una creatura, il bersaglio diventi tu e sarai tu a effettuare il Tiro Salvezza. Se il	Tiro Salvezza fallisce di 5 o più, il bersaglio è pietrificato. Se il Tiro Salvezza fallisce di meno, il bersaglio è intralciato e inizia a trasformarsi in pietra. Mentre è intralciato a questo modo, il bersaglio deve ripetere il Tiro Salvezza al termine di ciascun suo round, diventando pietrificato in caso di fallimento o terminando l'effetto in caso di successo. Il bersaglio resta pietrificato finché non sarà liberato dall'incantesimo pietra in carne o simili magie.\\	
\end{tabularx}

\medskip

\begin{multicols}{2}

\index{Bacchetta delle Palle di Fuoco}\textbf{Bacchetta delle Palle di Fuoco}

\textit{Bacchetta, raro} - 32000 mo

Mentre impugni questa bacchetta, puoi usare due azioni per spendere 1 o più cariche per lanciare tramite essa l'incantesimo palla di fuoco (DC del Tiro Salvezza 18). 

Questa bacchetta ha 7 cariche. La bacchetta recupera 1d6 + 1 cariche spese all'alba di ciascun giorno. Tuttavia, se spendi l'ultima carica della bacchetta, tira 3d6 se ottieni 5 o meno la bacchetta si riduce in polvere ed è distrutta.


\index{Bacchetta della Paralisi}\textbf{Bacchetta della Paralisi}

\textit{Bacchetta, raro} - 16000 mo

Mentre impugni questa bacchetta, puoi usare due azioni per spendere 1 carica per far sì che un sottile raggio parta dalla sua punta verso una creatura visibile entro 18 metri da te. Il bersaglio deve superare un tiro Salvezza su Tempra con DC 17 o restare paralizzato per 1 minuto.

Al termine di ciascun round del bersaglio, questi può effettuare un tiro Salvezza su Tempra DC 15, terminando l'effetto su di sé in caso lo superi. 

Questa bacchetta ha 7 cariche. La bacchetta recupera 1d6 + 1 cariche spese all'alba di ciascun giorno. Tuttavia, se spendi l'ultima carica della bacchetta, tira 3d6 se ottieni 5 o meno la bacchetta si riduce in polvere ed è distrutta.


\index{Bacchetta della Paura}\textbf{Bacchetta della Paura}

\textit{Bacchetta, raro} - 10000 mo

Questa bacchetta ha 7 cariche per le seguenti proprietà. La bacchetta recupera 1d6 + 1 cariche spese all'alba di ciascun giorno. Tuttavia, se spendi l'ultima carica della bacchetta, tira 3d6 se ottieni 5 o meno la bacchetta si riduce in polvere ed è distrutta.

\textbf{Comando}.
Mentre impugni questa bacchetta, puoi usare due azioni per spendere 1 carica e comandare a un'altra
creatura di scappare o strisciare, come per l'incantesimo comando (DC del Tiro Salvezza 18).

\textbf{Cono di Paura}.
Mentre impugni questa bacchetta, puoi usare due azioni per spendere 2 cariche, facendo sì che la punta della bacchetta emetta luce in un cono di 18 metri. Ogni creatura nel cono deve superare un Tiro Salvezza su Volontà con DC 18 o restare spaventata da te per 1 minuto. Mentre è spaventata in questo modo, una creatura deve spendere i suoi turni cercando di muoversi più lontano possibile da te, e non può muoversi volontariamente entro 9 metri da te.

Inoltre non può effettuare reazioni. Come sua azione, la creatura può usare solo l'azione Scattare o cercare di scappare da un effetto che le impedisca di muoversi. Se non può muoversi da nessuna parte, la creatura può usare l'azione Schivare. Al termine di ciascun suo round, la creatura può ripetere il Tiro Salvezza, terminando l'effetto su di sé in caso lo superi. 


\index{Bacchetta della Ragnatela}\textbf{Bacchetta della Ragnatela}

\textit{Bacchetta, non comune} - 8000 mo

Mentre la impugni, puoi usare due azioni per spendere 1 carica per lanciare tramite essa l'incantesimo ragnatela (DC del Tiro Salvezza 18). Questa bacchetta ha 7 cariche. La bacchetta recupera 1d6 + 1 cariche spese all'alba di ciascun giorno. Tuttavia, se spendi l'ultima carica della bacchetta, tira 3d6 se ottieni 5 o meno la bacchetta si riduce in polvere ed è distrutta.


\index{Bacchetta dei Segreti}\textbf{Bacchetta dei Segreti}

Bacchetta, non comune - 1500 mo

Mentre impugni questa bacchetta, puoi usare due azioni per spendere 1 carica, e se una porta segreta o trappola si trova entro 9 metri da te, la bacchetta pulsa e punta a quella più vicina a te. La bacchetta ha 3 cariche. La bacchetta recupera 1d3 cariche spese all'alba di ciascun giorno.


\index{Bacchetta del Vincolo}\textbf{Bacchetta del Vincolo}

\textit{Bacchetta, raro} - 10000 mo

Questa bacchetta ha 7 cariche per le seguenti proprietà. La bacchetta recupera 1d6 + 1 cariche spese all'alba di ciascun giorno. Tuttavia, se spendi l'ultima carica della bacchetta, tira 3d6 se ottieni 5 o meno la bacchetta si riduce in polvere ed è distrutta.

Incantesimi. Mentre impugni questa bacchetta, puoi usare due azioni e spendere alcune delle sue cariche per lanciare uno dei seguenti incantesimi (DC del Tiro Salvezza 21):

\textbf{blocca mostri} (5 cariche) o \textbf{blocca persone} (2 cariche).

\textbf{Fuga Assistita}. Mentre impugni questa bacchetta,puoi usare la tua reazione e spendere 1 carica per ottenere +1d6 ai Tiri Salvezza che effettui per evitare di restare paralizzato o intralciato, o puoi spendere 1 carica per ottenere +1d6 su qualsiasi prova effettuata per sfuggire un tentativo di afferrare. 


\index{Barca Pieghevole}\textbf{Barca Pieghevole}

\textit{Oggetto meraviglioso, raro} - 12000 mo

Questo oggetto sembra una scatola di legno che misura 30 centimetri di lunghezza, 15 centimetri di larghezza e 15 centimetri di profondità. Pesa 2 chili e galleggia. Può essere aperta per porvi oggetti all'interno. Quest'oggetto possiede tre parole di comando, ciascuna delle quali richiede due azioni per essere pronunciata. Una parola di comando fa sì che la scatola si dispieghi in una barca lunga 3 metri, larga 1,2 metri e profonda 50 centimetri. La barca ha un paio di remi, un'ancora, un albero e una vela. La barca può contenere fino a quattro creature di taglia Media.

La seconda parola di comando fa sì che la scatola si dispieghi in una nave lunga 7,2 metri, larga 2,5 metri e profonda 2 metri. La nave ha un ponte, file di voga, cinque serie di remi, un timone direzionale, un'ancora, una cabina e un albero con la vela quadrata. La nave può contenere quindici creature di taglia Media.

La terza parola di comando fa sì che la barca pieghevole ritorni a piegarsi nella scatola, purché nessuna creatura sia a bordo. Qualsiasi oggetto a bordo che non possa entrare nella scatola resta fuori della scatola mentre questa si piega. Qualsiasi oggetto a bordo che possa entrare nella scatola, vi entra. 


\index{Bastone di Avvizzimento}\textbf{Bastone di Avvizzimento}

\textit{Bastone, raro} - 3000 mo

Il bastone può essere impugnato come un bastone da combattimento magico. Se colpisci, infligge danni come un normale bastone da combattimento, e puoi spendere 1 carica per infliggere 2d10 danni da Vuoto aggiuntivi al bersaglio.

Inoltre, il bersaglio deve superare un Tiro Salvezza su Tempra con DC 18 o avere -1d6 per 1 ora a qualsiasi prova di caratteristica o Tiro Salvezza che richieda Costituzione

Questo bastone ha 3 cariche e recupera 1d3 cariche spese ogni mattina all'alba.


\index{Bastone dei Boschi}\textbf{Bastone dei Boschi}

\textit{Bastone, raro} - 44000 mo

Il bastone può essere impugnato come un bastone da combattimento magico che conferisce un bonus di +2 ai Tiri per Colpire e danno effettuati con esso. Quando lo impugni hai anche un bonus di +2 ai Tiri per Colpire con incantesimi.
Questo bastone ha 10 cariche per le seguenti proprietà. Recupera 1d6 + 4 cariche spese ogni giorno all'alba. Se spendi l'ultima carica del bastone,tira 3d6 se ottieni 5 o meno il bastone si annerisce, si trasforma in cenere, ed è distrutto.

\textit{Incantesimi}. Puoi usare due azioni per spendere 1 o più cariche del bastone per lanciare tramite esso uno dei seguenti incantesimi, utilizzando la tua DC del Tiro Salvezza degli incantesimi: amicizia con gli animali (1 carica), localizza animali e piante (1 carica), muro di spine (6 cariche), parlare con gli animali (3 cariche), pelle coriacea (2 cariche) o risveglio (5 cariche). Puoi inoltre usare due azioni per lanciare tramite il bastone l'incantesimo passare senza tracce senza 
spendere cariche.

\textit{Forma d'Albero}. Puoi usare due azioni per piantare un'estremità del bastone su terreno fertile e spendere 1 carica per trasformare il bastone in un albero vigoroso. L'albero è alto 18 metri, con un tronco di 1,5 metri di diametro; in cima i suoi rami si estendono per 6 metri. L'albero sembra un albero normale ma irradia una debole aura di magia di trasmutazione, qualora sia bersagli o dell'incantesimo individuazione del magico. Mentre sei in contatto con l'albero e usi un'altra azione per pronunciarne la parola di comando, riporti il bastone alla sua forma normale. Qualsiasi creatura sull'albero, cade quando questo si ritrasforma in bastone.


\index{Bastone dello Charme}\textbf{Bastone dello Charme}

Bastone, raro - 12000 mo

Mentre impugni questo bastone, puoi usare due azioni per spendere 1 carica per lanciare tramite esso charme su persone, comando o comprendere linguaggi, utilizzando la tua DC dei Tiri Salvezza degli incantesimi. Il bastone può essere usato come bastone da combattimento magico.

Se stai impugnando il bastone e fallisci un Tiro Salvezza contro un incantesimo di ammaliamento che prende come bersaglio solo te e non un'area, puoi trasformare il Tiro Salvezza fallito in un successo. Non potrai più usare questa proprietà del bastone fino all'alba del giorno successivo.

Se riesci in un Tiro Salvezza contro un incantesimo di ammaliamento che prende come bersaglio solo te, con o senza l'intervento del bastone, puoi usare la tua reazione per spendere 1 carica dal bastone e rivolgere l'incantesimo contro chi lo ha lanciato, come se l'incantesimo fosse stato lanciato da te.

Il bastone ha 10 cariche, e recupera 1d8 + 2 cariche spese ogni giorno all'alba. Se spendi l'ultima carica, tira 3d6 se ottieni 5 o meno il bastone diventa un bastone da combattimento normale.


\index{Bastone del Colpire}\textbf{Bastone del Colpire}

\textit{Bastone, molto raro} - 25000 mo

Questo bastone può essere impugnato come un bastone da combattimento magico che conferisce un bonus di +3 ai Tiri per Colpire e di danno effettuati con esso.

Quando colpisci con un attacco da mischia facendo uso del bastone, puoi spendere fino a 3 delle sue cariche. Per ogni carica spesa, il bersaglio subisce 1d6 danni da forza aggiuntivi. 

Il bastone ha 10 cariche, e recupera 1d6 + 4 cariche spese ogni giorno all'alba. Se spendi l'ultima carica, tira 3d6 se ottieni 5 o meno il bastone diventa un bastone da combattimento normale.


\index{Bastone del Fuoco}\textbf{Bastone del Fuoco}

\textit{Bastone, molto raro} - 16000 mo

Mentre impugni questo bastone, hai resistenza al danno da fuoco.

Inoltre, puoi usare due azioni per spendere 1 o più delle sue cariche per lanciare tramite esso uno dei seguenti incantesimi, utilizzando la tua DC dei Tiri Salvezza degli incantesimi: mani brucianti (1 carica), muro di fuoco (4 cariche) o palla di fuoco (3 cariche).

Il bastone ha 10 cariche, e recupera 1d6 + 4 cariche spese ogni giorno all'alba. Se spendi l'ultima carica del bastone, tira 3d6 se ottieni 5 o meno il bastone si annerisce, si trasforma in cenere, ed è distrutto.


\index{Bastone del Gelo}\textbf{Bastone del Gelo}

\textit{Bastone, molto raro} - 26000 mo

Mentre impugni questo bastone, hai resistenza ai danni da freddo.

Inoltre, puoi usare due azioni per spendere 1 o più delle sue cariche per lanciare tramite esso uno dei seguenti incantesimi, utilizzando la tua DC dei Tiri Salvezza degli 

\textit{incantesimi}: cono di freddo (5 cariche), muro di ghiaccio (4 cariche), nube di nebbia (1 carica) o tempesta di ghiaccio (4 cariche).

Il bastone ha 10 cariche, e recupera 1d6 + 4 cariche spese ogni giorno all'alba. Se spendi l'ultima carica del bastone, tira 3d6 se ottieni 5 o meno il bastone si trasforma in acqua ed è distrutto.


\index{Bastone di Guarigione}\textbf{Bastone di Guarigione}

\textit{Bastone, raro} - 13000 mo

Mentre lo impugni, puoi usare due azioni per spendere 1 o più delle sue cariche per lanciare tramite esso uno dei seguenti incantesimi: cura ferite leggere (1 carica), ristorare inferiore (2 cariche), o cura ferite di massa (5 cariche). Questo bastone ha 10 cariche, e recupera 1d6 + 4 cariche spese ogni giorno all'alba. Se spendi l'ultima carica del bastone, tira 3d6 se ottieni 5 o meno il bastone svanisce in un lampo di luce, perso per sempre.


\index{Bastone degli Insetti Sciamanti}\textbf{Bastone degli Insetti Sciamanti}

\textit{Bastone, raro} - 160000 mo

Questo bastone ha 10 cariche che puoi impiegare per usare le proprietà sotto descritte e recupera 1d6 + 4 cariche spese ogni giorno all'alba. Se spendi l'ultima carica del bastone, tira 3d6 se ottieni 5 o meno uno sciame di insetti consuma e distrugge il bastone, e poi si disperde.

\textit{Incantesimi}. Mentre impugni questo bastone, puoi usare due azioni per spendere le sue cariche ed lanciare uno dei seguenti incantesimi, utilizzando la DC dell'incantesimo: insetto gigante (4 cariche) o piaga degli insetti (5 cariche).

\textit{Nube di Insetti}. Mentre impugni questo bastone, puoi usare due azioni e spendere 1 carica per fa sì che uno sciame di insetti innocui si diffonda in un raggio di 9 metri intorno a te. Gli insetti rimangono per 10 minuti, rendendo l'area oscurata pesantemente per tutti tranne te. Lo sciame si muove assieme a te, rimanendo centrato su di te. Un vento di almeno 15 chilometri all'ora disperde lo sciame e termina l'effetto.


\index{Bastone del Pitone}\textbf{Bastone del Pitone}

\textit{Bastone, non comune} - 2000 mo

Puoi usare due azioni per pronunciare la parola di comando del bastone e scagliarlo sul terreno fino a 3 metri di distanza. Il bastone diventa un serpente costrittore gigante sotto il tuo controllo e agisce al proprio conteggio di iniziativa. Utilizzando due azioni per pronunciare di nuovo la parola di comando, riporti il bastone alla sua forma normale nello spazio precedentemente occupato dal serpente.

Durante il tuo round puoi impartire ordini mentali al serpente finché si trova entro 18 metri da te e non sei inabile. Decidi tu quali azioni effettuerà il serpente e dove si muoverà durante il suo prossimo round, oppure puoi impartirgli un comando generico, come quello di attaccare i tuoi nemici o difendere un luogo. Se il serpente viene ridotto a 0 punti ferita, muore e ritorna alla sua forma di bastone. Poi, il bastone si frantuma ed è distrutto. Se il serpente si ritrasforma in forma di bastone prima di perdere tutti i suoi punti ferita, recupera tutti quelli persi.


\index{Bastone del Potere}\textbf{Bastone del Potere}

\textit{Bastone, molto raro}

Questo bastone può essere impugnato come un bastone da combattimento magico che conferisce un bonus di +2 ai Tiri per Colpire e danno effettuati con esso. Mentre lo impugni, ricevi un bonus di +2 alla Difesa, ai Tiri Salvezza, e ai Tiri per Colpire con incantesimi. Questo bastone ha 20 cariche per le seguenti proprietà. Recupera 2d8 + 4 cariche spese ogni giorno all'alba. Se spendi l'ultima carica del bastone, tira 3d6 se ottieni 5 o meno il bastone mantiene il suo bonus di +2 ai Tiri per Colpire e danno ma perde tutte le altre proprietà. Se il risultato è 20, il bastone recupera 1d8 + 2 cariche.

\textit{Colpo di Potere}. Quando colpisci con un attacco in mischia usando questo bastone, puoi spendere 1 carica per infliggere 1d6 danni da forza aggiuntivi al bersaglio. 

\textit{Incantesimi}. Mentre impugni questo bastone, puoi usare due azioni per spendere 1 o più delle sue cariche per lanciare tramite esso uno dei seguenti incantesimi, utilizzando la tua DC del Tiro Salvezza degli incantesimi e la tua abilità da incantatore: blocca mostri (5 cariche), cono di freddo (5 cariche), globo di invulnerabilità (6 cariche), levitazione (2 cariche), muro di forza (5 cariche), palla di fuoco (3 cariche), dardo incantato (1 carica), raggio di indebolimento (1 carica) o fulmine (3 cariche).

\textit{Colpo di Vendetta}. Puoi usare due azioni per spezzare il bastone sul tuo ginocchio o contro una superficie solida, eseguendo un colpo di vendetta. Il bastone viene distrutto e libera la sua magia rimanente in un'esplosione che si espande fino a riempire una sfera di 9 metri di raggio centrata su di esso. 

Hai il 50\% di probabilità di viaggiare istantaneamente in un piano di esistenza a caso, evitando così l'esplosione. Se non riesci a evitare l'effetto, subisci danni da forza pari a 16 x il numero di cariche nel bastone. Ogni altra creatura nell'area deve effettuare un Tiro Salvezza su Riflessi con DC 21. Se il Tiro Salvezza fallisce, la creatura subisce un ammontare di danno basato sulla distanza dal punto di origine dell'esplosione, come mostrato sulla tabella seguente.

Se il Tiro Salvezza riesce, la creatura subisce la metà di questi danni.


\medskip

\begin{tabularx}{0.45\textwidth}{lX}
\toprule
\textbf{Distanza dall'Origine} &\textbf{Danno}\\
3 metri o meno &8 x il numero di cariche nel bastone\\
Fino a 6 metri& 6 x il numero di cariche nel bastone\\
Fino a 9 metri& 4 x il numero di cariche nel bastone\\
\end{tabularx}

\medskip

\index{Bastone dei Tuoni e Fulmini}\textbf{Bastone dei Tuoni e Fulmini}

\textit{Bastone, molto raro} - 10000 mo

Il bastone può essere impugnato come un bastone da combattimento magico che conferisce un bonus di +2 ai Tiri per Colpire e danno effettuati con esso. Inoltre ha le seguenti proprietà. Quando viene usata una di queste proprietà, non se ne potrà più far uso fino all'alba successiva.

\textit{Fulmine}. Quando colpisci con un attacco in mischia usando il bastone, puoi far sì che il bersaglio subisca 2d6 danni da fulmine aggiuntivi.

\textit{Tuono}. Quando colpisci con un attacco in mischia usando il bastone, puoi far sì che il bastone emetta il suono di un tuono, udibile fino a 90 metri di distanza. Il bersaglio colpito deve superare un Tiro Salvezza su Tempra con DC 21 o restare stordito fino al termine del tuo prossimo round.

\textit{Colpo Fulminante}. Puoi usare due azioni per far sì che una fulmine balzi dalla punta del bastone in una linea larga 1,5 metri e lunga 36 metri. Ogni creatura sulla linea deve effettuare un Tiro Salvezza su Riflessi con DC 21, subendo 9d6 danni da fulmine se lo fallisce, o la metà di questi danni se lo supera.

\textit{Rombo di Tuono}. Puoi usare due azioni per far sì che il bastone produca un rombo di tuono assordante, udibile fino a 180 metri di distanza. Ogni creatura entro 18 metri da te (te escluso) deve effettuare un Tiro Salvezza su Tempra con DC 21. Se fallisce il Tiro Salvezza, la creatura subisce 2d6 danni da tuono e resta assordata per 1 minuto. Se supera il Tiro Salvezza, subisce la metà dei danni e non è assordata.

\textit{Tuoni e Fulmini}. Puoi usare due Azioni per usare le proprietà Colpo Fulminante e Rombo di Tuono assieme. Farlo non consuma l'uso giornaliero di quelle proprietà, ma solo l'uso di questa.


\index{Battaglio dell'Apertura}\textbf{Battaglio dell'Apertura}

\textit{Oggetto meraviglioso, raro} - 1500 mo

Questo tubo metallico cavo misura circa 30 centimetri di lunghezza e pesa 0,5 chili. Puoi batterlo con due azioni, puntandolo verso un oggetto entro 36 metri che può essere aperto, come una porta o una serratura. Il battaglio emette un suono limpido, e una serratura o laccio dell'oggetto si apre a meno che il suono sia impedito dal raggiungere l'oggetto. Se non rimangono serrature o lacci da aprire, l'oggetto si apre da sé.

Il battaglio può essere usato dieci volte. Dopo la decima, si spacca e diventa inutilizzabile.


\index{Borsa Conservante}\textbf{Borsa Conservante}


Sussistono diverse tipologie di Borse Conservanti e tutte hanno in comune la capacità di poter contenere molto di più di quello che dovrebbero date le loro dimensioni.

Le Borse Conservanti si suddividono in 4 tipi (Tipo I, II, III; IV) a seconda dalla capacità di conservazione che hanno.

Se la borsa è sovraccarica, perforata o strappata, la borsa si rompe ed è distrutta e il suo contenuto sparpagliato per il Piano Astrale. Se la borsa viene rivoltata, i suoi contenuti vengono espulsi, illesi, ma la borsa dev'essere rimessa nel verso giusto prima che possa essere riutilizzata. Le creature che respirano, piazzate nella borsa, possono sopravvivervi per un numero di minuti pari a 10 diviso il numero di creature (minimo 1 minuto), dopodiché inizieranno a soffocare.

Piazzare una borsa conservante all'interno dello spazio extradimensionale generato da uno zainetto pratico, un buco portatile o simile oggetto, distrugge entrambi gli oggetti e apre un portale verso il Piano Astrale. Il portale origina nel punto in cui un oggetto è stato posto all'interno dell'altro. Qualsiasi creatura entro 3 metri dal portale viene risucchiata al suo interno e ricompare in un posto a caso sul Piano Astrale, poi il portale si richiude. Il portale è a senso unico e non può essere riaperto.


Alcuni Maghi preferiscono creare Bauli Conservanti, che funzionano nell'identico modo delle borse conservanti.

\index{Borsa Conservante - Tipo I}\textbf{Borsa Conservante - Tipo I}

\textit{Oggetto meraviglioso, non comune} - 500 mo

Questo è il modello più piccolo delle borse conservanti. All'apparenza è un piccolo sacchetto di 20 cm di diametro con una bocca larga circa altrettanto.
Non è possibile fare entrare oggetti che abbiano una larghezza superiore ai 20 cm ed una lunghezza superiore ai 50cm.
La capacità massima è di 20 kg.

\index{Borsa Conservante - Tipo II}\textbf{Borsa Conservante - Tipo II}

\textit{Oggetto meraviglioso, non comune} - 1000 mo

Questo è il modello medio delle borse conservanti. All'apparenza è un sacchetto di 40 cm di diametro con una bocca larga circa altrettanto.
Non è possibile fare entrare oggetti che abbiano una larghezza superiore ai 40 cm ed una lunghezza superiore ai 100cm.
La capacità massima è di 100 kg.

\index{Borsa Conservante - Tipo III}\textbf{Borsa Conservante - Tipo III}

\textit{Oggetto meraviglioso, non comune} - 750 mo

All'apparenza è un sacco di 80 cm di diametro con una bocca larga circa altrettanto.
Non è possibile fare entrare oggetti che abbiano una larghezza superiore ai 80 cm ed una lunghezza superiore ai 150cm.
La capacità massima è di 200 kg.

\index{Borsa Conservante - Tipo IV}\textbf{Borsa Conservante - Tipo IV}

\textit{Oggetto meraviglioso, non comune} - 5000 mo

All'apparenza è un saccone di 120 cm di diametro con una bocca larga circa altrettanto.
Non è possibile fare entrare oggetti che abbiano una larghezza superiore ai 120 cm ed una lunghezza superiore ai 200cm.
La capacità massima è di 300 kg.


\index{Borsa Divorante}\textbf{Borsa Divorante}

\textit{Oggetto meraviglioso, molto raro} - 2000 mo

La borsa appare come una borsa conservante. Se la borsa viene rivolta le sue proprietà smettono di funzionare.

La creatura extradimensionale attaccata alla borsa può percepire qualsiasi cosa vi venga posto all'interno. La materia animale o vegetale posta interamente dentro la borsa viene divorata ed è persa per sempre. Quando una parte di una creatura vivente viene posta nella borsa, c'è una probabilità del 50\% che la creatura venga trascinata dentro la borsa. Una creatura all'interno della borsa può usare due azioni per cercare di fuggirne superando una prova di Forza con DC 18.

Un'altra creatura può usare due azioni per afferrare la creatura all'interno della borsa e tirarla fuori, superando una prova di Forza con DC 20 (e sempre che non venga a sua volta trascinata dentro la borsa). Qualsiasi creatura che inizi il proprio round all'interno della borsa viene divorata, il suo corpo distrutto.

All'interno della borsa possono essere posti oggetti inanimati, fino a 27 dm3 di materiale.
Tuttavia, una volta al giorno, la borsa inghiotte qualsiasi oggetto posto al suo interno e lo risputa fuori in un altro piano di esistenza. Il Narratore determina il momento e il piano. Se la borsa venisse fatta a pezzi o strappata, è distrutta, e qualsiasi cosa contenga verrebbe trasportata in un luogo casuale del Piano Astrale.
 

\index{Borsa dei Fagioli}\textbf{Borsa dei Fagioli}

\textit{Oggetto meraviglioso, raro}

All'interno di questa borsa si trovano 3d4 fagioli secchi. La borsa pesa 250 grammi più 125 grammi per ogni fagiolo che contiene.

Se riversi il contenuto della borsa sul terreno, i fagioli esplodono in un raggio di 3 metri. Ogni creatura nell'area, te compreso, deve effettuare un Tiro Salvezza di Riflessi con DC 18, subendo 5d4 danni da fuoco se lo fallisce, o la metà di questi danni se lo supera.

Il fuoco incendia gli oggetti infiammabili nell'area che non siano indossati o trasportati. Se rimuovi il fagiolo dalla borsa, lo pianti nel terreno o la sabbia, e lo innaffi, il fagiolo produrrà un effetto 1 minuto dopo, a partire dal punto del terreno in cui è stato piantato. Il Narratore sceglie l'effetto o lo determina casualmente.


\end{multicols}

\medskip

\begin{tabularx}{0.95\textwidth}{lX}
\textbf{d100} & \textbf{Effetto}\\
\toprule
01 &Spuntano 5d4 funghi. Se una creatura mangia un fungo, tira un dado. Se il risultato è dispari, costui deve superare un tiro Salvezza su Tempra con DC 15 o subire 5d6 danni da veleno e restare avvelenato per 1 ora. Se il risultato è pari, costui ottiene 5d6 punti ferita temporanei per 1 ora.\\
02-10 &Erutta un geyser che sputa acqua, birra, succo di frutta, tè, aceto, vino od olio (a discrezione del Narratore) 9 metri in aria per 1d12 round.\\
11-20 &Spunta un uomo albero. C'è una probabilità del 50\% che l'uomo albero sia caotico malvagio e ti attacchi.\\
21-30 &Una statua di pietra animata con le tue fattezze si leva dal terreno. Essa comincerà a minacciarti verbalmente. Se dovessi andartene e altre persone giungessero sul posto, la statua ti descriverebbe come il più pericoloso dei criminali, e li esorterebbe ad cercarti e attaccarti. Se ti trovi sullo stesso piano di esistenza della statua, essa saprà sempre dove sei. Dopo 24 ore la statua diventerà inanimata.\\
31-40 &Un fuoco da campo che produce fiamme blu spunta dal terreno e brucia per 24 ore (o finché non viene spento).\\
41-50 &Sputano 1d6 + 6 funghi urlatori.\\
51-60 &Compaiono 1d4 + 8 rospi fucsia. Ogniqualvolta un rospo viene toccato, si trasforma in un mostro di taglia Grande o inferiore a scelta del Narratore. Il mostro resta per 1 minuto e poi scompare in un sbuffo di fumo fucsia. 61-70 Un bulette esce dal terreno e attacca.\\
71-80 &Cresce un albero da frutta. Possiede 1d10 + 20 frutti. 1d8 di questi funzionano come una pozione magica determinata a caso, mentre uno di loro funge da veleno ingerito del tipo determinato dal Narratore. L'albero svanisce dopo 1 ora. I frutti raccolti invece rimangono, e mantengono la propria magia per 30 giorni. \\
81-90 &Compare un nido con 1d4 + 3 uova. Qualsiasi creatura che mangi un uovo deve effettuare un Tiro Salvezza su Tempra con DC 28. Se il Tiro Salvezza riesce, la
creatura aumenta permanentemente il suo punteggio di caratteristica più basso di 1, scegliendo casualmente in caso di parità. Se il Tiro Salvezza fallisce, la creatura subisce 10d6 danni da forza a causa di un'esplosione magica al suo interno.\\
91-99 &Spunta dal terreno una piramide dalla base quadrata di 18 metri. All'interno c'è un sarcofago che contiene una mummia sovrana. La piramide è considerata come la tana della mummia sovrana, e il suo sarcofago contiene un tesoro a scelta del Narratore.\\
100 &Un enorme pianta di fagioli cresce sul posto, fino a un'altezza a scelta del Narratore. La cima conduce dovunque voglia il Narratore, che sia il castello di un gigante delle nuvole o un altro piano di esistenza.
\end{tabularx}

\begin{multicols}{2}

\medskip

\index{Borsa dei Trucchi}\textbf{Borsa dei Trucchi}

\textit{Oggetto meraviglioso, non comune}

Questa borsa dall'aspetto normale appare vuota.

Allungare la mano all'interno della borsa, tuttavia, rivela la presenza di un piccolo oggetto peloso. La borsa pesa 250 grammi.

Puoi usare due azioni per estrarre l'oggetto peloso dalla borsa e scagliarlo fino a 6 metri di distanza. Quando l'oggetto atterra, si trasforma in una creatura determinata dal lancio di un d8 e consultando la tabella che corrisponde al colore della borsa. Vedi l'elenco dei mostri per le statistiche della creatura. La creatura svanisce all'alba successiva o quando viene ridotta a 0 punti ferita. 

La creatura è amichevole verso di te e i tuoi compagni, e agisce durante il tuo round. Puoi usare due azioni  per ordinare alla creatura di muoversi e quale azione debba effettuare durante il suo prossimo round, o darle ordini generici, come quello di attaccare i tuoi nemici. In assenza di simili ordini, la creatura agisce in maniera appropriata alla sua natura e resterà per 10minuti prima di svanire.

Una volta che tre oggetti pelosi sono stati estratti dalla borsa, questa non potrà più essere usata fino alla prossima alba.


\index{Bottiglia dell'Efreeti}\textbf{Bottiglia dell'Efreeti}

\textit{Oggetto meraviglioso, molto raro}

Questa bottiglia di ottone dipinta pesa 500 grammi. Quando usi due azioni per rimuoverne il tappo, una nube di denso fumo fuoriesce dalla bottiglia. Al termine del tuo round, il fumo si dissipa in un lampo di fuoco innocuo, e un efreeti compare in uno spazio non occupato entro 9 metri da te. La prima volta che la bottiglia viene aperta, il Narratore determina casualmente cosa accade.


\medskip

\begin{tabularx}{0.45\textwidth}{lX}
\textbf{3d6} &\textbf{Effetto}\\
\toprule
3-5 & L'efreeti ti attacca. Dopo aver combattuto per 5 round, l'efreeti scompare e la bottiglia perde la sua magia.\\
6-16 &L'efreeti ti obbedisce per 1 ora, agendo ai tuoi comandi. Poi torna nella bottiglia, e un nuovo tappo lo può contenere. Il tappo non potrà essere rimosso prima che siano passate 24 ore. Le prossime due volte che la bottiglia viene aperta, si ripresenta lo stesso effetto. Se la bottiglia viene aperta una quarta volta, l'efreeti scappa e scompare, e la bottiglia perde la sua magia.\\
17-18 & L'efreeti può lanciare l'incantesimo desiderio a tuo favore per tre volte. Scompare quando conferisce il desiderio finale o dopo 1 ora, allorché la bottiglia perde la sua magia.
\end{tabularx}

\medskip

\index{Bottiglia Fumante}\textbf{Bottiglia Fumante}

\textit{Oggetto meraviglioso, non comune} - 1200 mo

Dalla bocca di questa bottiglia di ottone fuoriesce continuamente del fumo, trattenuto dal suo tappo di piombo. La bottiglia pesa 500 grammi. Quando usi due azioni per rimuovere il tappo, una nube di denso fumo si sparge in un raggio di 18 metri intorno alla bottiglia. L'area della nube è oscurata pesantemente. Per ciascun minuto in cui la bottiglia resta aperta e all'interno della nube, il raggio aumenta di 3 metri finché non raggiunge il raggio massimo di 36 metri.

La nube persiste fino a quando la bottiglia resta aperta. Chiudere la bottiglia richiede che tu pronunci la sua parola di comando con due azioni. Una volta chiusa la bottiglia, la nube si disperde dopo 10 minuti. Un vento moderato (dai 15 ai 30 km/h) può disperdere il fumo in1 minuto, e un vento forte (più di 30 km/h) può disperderlo in 1 round. 


\index{Bracciali dell'Arciere}\textbf{Bracciali dell'Arciere}

\textit{Oggetto meraviglioso, non comune} - 1500 mo

Mentre indossi questi bracciali, hai competenza con l'arco lungo e l'arco corto, e ottieni un bonus di +2 ai tiri di danno degli attacchi a distanza effettuati con queste armi.


\index{Bracciali della Difesa}\textbf{Bracciali della Difesa}
\textit{Oggetto meraviglioso, raro} - 6000 mo, 15000 mo, 30000 mo, 45000 mo, 60000 mo

Mentre indossi questi bracciali, hai un bonus di +1, +2, +3, +4+, +5 alla tua Difesa se non indossi nessuna armatura e non usi nessuno scudo.

\index{Bracciali della Difesa Maggiore}\textbf{Bracciali della Difesa}
\textit{Oggetto meraviglioso, leggendario} - 12000 mo, 24000 mo, 36000 mo, 50000 mo, 75000 mo

Questi bracciali funzionano come un armatura pur non essendo tali. Vieni avvolto in uno scudo magico invisibile che ti concede Difesa 15, 17, 19, 21, 23. La Difesa può essere aumentata con scudi, anelli... o comunque oggetti che migliorano la Difesa, tranne armature.

\index{Braciere del Comando degli Elementali del Fuoco}\textbf{Braciere del Comando degli Elementali del Fuoco} - 8000 mo

\textit{Oggetto meraviglioso, raro}

Mentre il fuoco arde all'interno di questo braciere di ottone, puoi usare due azioni per pronunciare la parola di comando del braciere ed evocare un elementale del fuoco, come se avessi lanciato l'incantesimo evoca elementali. Il braciere non può di nuovo essere usato a questo modo, fino alla prossima alba. 

Il braciere pesa 2,5 chili.


\index{Brando del Gelo}\textbf{Brando del Gelo}

\textit{Arma (qualsiasi spada), molto raro} - 2500 mo

Quando colpisci con un attacco che usi questa spada magica, il bersaglio subisce 1d6 danni da freddo aggiuntivi. Inoltre, mentre impugni questa spada, hai resistenza ai danni da fuoco.

A temperature gelide, la lama irradia luce intensa in un raggio di 3 metri e luce fioca per ulteriori 3 metri. Quando estrai quest'arma, puoi estinguere tutte le fiamme non magiche entro 9 metri da te. Questa proprietà non può essere usata più di una volta all'ora.

\index{Brocca dell'Acqua Infinita}\textbf{Brocca dell'Acqua Infinita}

\textit{Oggetto meraviglioso, non comune} - 12000 mo

Quest'ampolla tappata emette un suono di liquido quando viene smossa, come se contenesse acqua. La brocca pesa 1 chilo. Puoi usare due azioni per rimuovere il tappo e pronunciare una delle tre parole di comando, e a quel punto un ammontare di acqua fresca o acqua salata (a tua scelta) si riverserà fuori dell'ampolla, fino all'inizio del tuo prossimo round. Scegli una delle opzioni seguenti:


\medskip

\begin{itemize}
\item
"Ruscello" produce 4 litri d'acqua.
\item
"Fontana" produce 20 litri d'acqua.
\item
"Geyser" produce 150 litri d'acqua che vengono proiettati da un geyser lungo 9 metri e largo 30 centimetri. Con due azioni, mentre impugni la brocca, puoi prendere come bersaglio del geyser una creatura visibile entro 9 metri da te. 

Il bersaglio deve superare un Tiro Salvezza su Tempra con DC 15 o subire 1d4 danni contundenti e cadere prono. Invece di una creatura, puoi prendere come bersaglio un oggetto che non sia indossato o trasportato e che non pesi più di 100 chili. L'oggetto viene ribaltato o spinto 4,5 metri lontano da te.
\end{itemize}

\medskip

\index{Vano Portatile}\textbf{Vano Portatile}

\textit{Oggetto meraviglioso, raro} - 10000 mo

Questo elegante tessuto nero, soffice come la seta, si piega fino alle dimensioni di un fazzoletto. Si dispiega in uno strato circolare di 1,8 metri di diametro. Puoi usare due azioni per dispiegare un Vano portatile e piazzarlo sopra o contro una superficie solida, sulla quale il Vano portatile crea un foro extradimensionale profondo 3 metri. Lo spazio cilindrico all'interno del foro si trova su di un piano diverso, e quindi non può essere usato per aprire dei passaggi. Qualsiasi creatura all'interno di un Vano portatile aperto può uscirne fuori arrampicandosi fuori di esso.

Puoi usare due azioni per chiudere un Vano portatile prendendo i margini del tessuto e ripiegandolo. Piegare il tessuto chiude il Vano, e qualsiasi creatura od oggetto al suo interno rimane nello spazio extradimensionale. Non importa quello che contiene, il Vano non pesa nulla.

Se il Vano viene ripiegato, una creatura all'interno dello spazio dimensionale del Vano può usare due azioni per effettuare una prova di Forza con DC 10. Se la prova riesce, la creatura riesce a liberarsi e ricompare entro 1,5 metri dal Vano portatile o della creatura che lo trasporta. Una creatura che respira può sopravvivere all'interno di un buco portatile chiuso per un massimo di 10 minuti, dopodiché iniziare a soffocare.

Piazzare un Vano portatile all'interno dello spazio extradimensionale creato da una borsa conservante, uno zainetto pratico o simile oggetto distrugge istantaneamente entrambi gli oggetti e apre un portale verso il Piano Astrale. Il portale origina dal punto in cui un oggetto è stato piazzato all'interno dell'altro. Qualsiasi creatura entro 3 metri dal portale viene risucchiata al suo interno e depositata in un luogo casuale del Piano Astrale. Poi il portale si chiude. Il portale è a senso unico e non può essere riaperto.



\index{Buco Portatile}\textbf{Buco Portatile}

\textit{Oggetto meraviglioso, raro} - 10000 mo

Questo elegante tessuto nero, soffice come la seta, si piega fino alle dimensioni di un fazzoletto. Si dispiega in uno strato circolare di 1 metri di diametro. Puoi usare 1 round per dispiegare un buco portatile e piazzarlo sopra o contro una superficie solida, sulla quale il Buco portatile crea un foro profondo 3 metri.  Qualsiasi creatura abbastanza piccola può usare il Buco Portatile per attraversare la parete o superficie su cui è appoggiato purché sia profonda meno di 3 metri.

Puoi usare 1 round per chiudere un Buco portatile prendendo i margini del tessuto e ripiegandolo. Piegare il tessuto chiude il buco, e qualsiasi creatura od oggetto al suo interno viene espulso con una probabilità del 50\% di uscire da una parte o dall'altra.

Piazzare un buco portatile all'interno dello spazio extradimensionale creato da una borsa conservante, Vano portatile, uno zainetto pratico o simile oggetto distrugge istantaneamente entrambi gli oggetti e apre un portale verso il Piano Astrale. Il portale origina dal punto in cui un oggetto è stato piazzato all'interno dell'altro. Qualsiasi creatura entro 3 metri dal portale viene risucchiata al suo interno e depositata in un luogo casuale del Piano Astrale. Poi il portale si chiude. Il portale è a senso unico e non può essere riaperto.


\index{Candela di Invocazione}\textbf{Candela di Invocazione}

\textit{Oggetto meraviglioso, molto raro}

Questa lunga e sottile candela è dedicata a una divinità e ne condivide i Tratti. I Tratti della candela possono essere individuati tramite un rituale di 1 ora di affiancamento alla candela.

Il Narratore sceglie il Patrono e i Tratti associato a esso o lo determina casualmente.

La magia della candela si attiva quando la candela viene accesa con due azioni. Dopo aver bruciato per 4 ore, la candela è distrutta. Puoi decidere di spegnerla anticipatamente per riutilizzarla più tardi. Dedurre il tempo che rimane alla candela prima di estinguersi a incrementi di 1 minuto, per determinare per quanto abbia bruciato la candela.

Quando è accesa, la candela irradia luce fioca in un raggio di 9 metri. Qualsiasi creatura all'interno della luce Devota o Seguace a quello della candela effettua Tiri per Colpire, Tiri Salvezza e prove di abilità con +1d6.

In alternativa, quando accendi la candela per la prima volta, puoi lanciare l'incantesimo portale. Farlo distrugge la candela.


\index{Cappa dell'Aracnide}\textbf{Cappa dell'Aracnide}

\textit{Oggetto meraviglioso, molto raro} - 5000 mo

Mentre indossi questo elegante abito di seta nera intessuto con fili d'argento, ottieni i seguenti benefici:

\medskip

\begin{itemize}
\item
Hai resistenza ai danni da veleno.
\item
Hai velocità di scalata pari alla tua velocità di passeggio.
\item
Puoi muoverti verso l'alto, il basso e lungo superfici verticali e a testa in giù sui soffitti, tenendo le mani libere.
\item
Non puoi essere catturato da alcuna sorta di ragnatela e ti muovi attraverso le ragnatele come
fossero terreno difficile.
\item
Puoi usare due azioni per lanciare l'incantesimo ragnatela (DC del Tiro Salvezza 15). La ragnatela creata dall'incantesimo riempie il doppio della sua normale area. Una volta usata, questa proprietà della cappa non può essere usata di nuovo fino alla prossima alba.
\end{itemize}

\medskip

\index{Cappa di Distorsione}\textbf{Cappa di Distorsione}

\textit{Oggetto meraviglioso, raro} - 60000 mo

Mentre indossi questa cappa, essa proietta un'illusione che ti fa apparire come se stessi in un punto vicino alla tua reale posizione, facendo sì che tutte le creature abbiano -1d6 ai Tiri per Colpire contro di te. Se subisci danni, la proprietà cessa di funzionare fino all'inizio del tuo prossimo round. Questa proprietà è soppressa mentre sei inabile, intralciato o altrimenti impossibilitato a muoverti.


\index{Cappa degli Elfi}\textbf{Cappa degli Elfi}

\textit{Oggetto meraviglioso, non comune} - 5000 mo

Mentre indossi questa cappa tirando su il cappuccio, le prove di Saggezza (Percezione) effettuate per notarti hanno -1d6, e hai +1d6 alle prove di Destrezza effettuate per nasconderti. Tirare su o giù il cappuccio richiede due azioni.


\index{Cappa della Manta}\textbf{Cappa della Manta}

\textit{Oggetto meraviglioso, non comune} - 6000 mo

Mentre indossi questa cappa con il cappuccio tirato su, puoi respirare sott'acqua e hai velocità di nuoto 18 metri. Tirare su o giù il cappuccio richiede 1 azione.


\index{Cappa del Pipistrello}\textbf{Cappa del Pipistrello}
\textit{Oggetto meraviglioso, raro} - 6000 mo

Mentre indossi questa cappa, hai +1d6 alle prove di Destrezza. In aree di luce fioca o oscurità, puoi afferrare i bordi della cappa con entrambe le mani e usarla per muoverti a velocità di volo 12 metri. Se dovessi smettere di tenere i bordi della cappa mentre voli a questo modo, perdi la tua velocità di volo. Mentre indossi la cappa in un'area di luce fioca o oscurità, puoi usare la tua azione per lanciare metamorfosi su di te, trasformandoti in un pipistrello. Quando sei in forma di pipistrello, mantieni i tuoi punteggi di Intelligenza, Saggezza e Carisma. La cappa non può essere impiegata di nuovo in questo modo fino alla prossima alba.


\index{Cappa di Protezione}\textbf{Cappa di Protezione}

\textit{Oggetto meraviglioso, non comune} - 3500 mo, 6000 mo, 15000 mo

Mentre indossi questa cappa, ottieni un bonus di +1,+2,+3 alla Difesa e ai Tiri Salvezza.


\index{Cappello del Camuffamento}\textbf{Cappello del Camuffamento}

\textit{Oggetto meraviglioso, non comune} - 5000 mo

Mentre indossi questo cappello, puoi usare due azioni per lanciare a volontà l'incantesimo camuffare sé stesso. L'incantesimo termina quando il cappello viene rimosso.


\index{Ceppi Dimensionali}\textbf{Ceppi Dimensionali}

\textit{Oggetto meraviglioso, raro} - 4000

Puoi usare 2 Azioni per piazzare queste manette su di una creatura inabile. Le manette si adattano a qualsiasi creatura da taglia Piccola a Grande. Oltre a servire da comuni manette, i ceppi impediscono a una creatura legata con essi dall'usare qualsiasi metodo di movimento extradimensionale, compreso il teletrasporto o il viaggio verso piani diversi dell'esistenza. Tuttavia non impediscono a una creatura di attraversare un portale interdimensionale.

Tu e qualsiasi creatura da te indicata quando fai uso dei ceppi potete usare due azioni per rimuoverli. Una volta ogni 30 giorni, la creatura legata può effettuare una prova di Forza con DC 40. Se la supera, la creatura si libera e distrugge i ceppi. 


\index{Cerchietto dell'Esplosione}\textbf{Cerchietto dell'Esplosione}

\textit{Oggetto meraviglioso, non comune} - 1500 mo

Mentre indossi questo cerchietto, puoi usare due azioni per lanciare tramite esso l'incantesimo raggio rovente. Quando effettui gli attacchi dell'incantesimo, puoi farlo con bonus di CM +5. Il cerchietto non potrà essere usato di nuovo a questo modo fino alla prossima alba.


\index{Cintura dei Giganti}\textbf{Cintura dei Giganti}

\textit{Oggetto meraviglioso, rarità varia}

Mentre indossi questa cinta, il tuo punteggio raggiunge il punteggio conferito dalla cinta. Se il tuo punteggio di Forza è già pari o superiore al punteggio della cinta, l'oggetto non ha effetto su di te.

Esistono quattro varianti di questa cinta, corrispondenti ciascuna a una specie di veri giganti. La cinta del gigante di pietra e la cinta del gigante del gelo appaiono diverse, ma hanno lo stesso effetto.

\medskip

\begin{tabular}{lll}
	\textbf{Tipo}& \textbf{Forza} &\textbf{Rarità}\\
	\toprule
	\textbf{Gigante di collina} &5& Raro\\
	\textbf{Gigante di pietra/del gelo}& 6 &Molto raro\\
	\textbf{Gigante del fuoco} &7& Molto raro\\
	\textbf{Gigante delle nuvole} &8& Leggendario\\
	\textbf{Gigante delle tempeste}& 9& Leggendario\\
\end{tabular}

\medskip

\index{Cintura dei Nani}\textbf{Cintura dei Nani}

\textit{Oggetto meraviglioso, raro} - 6000 mo

Mentre indossi questa cinta, ottieni i seguenti benefici: 

Il tuo punteggio di Costituzione aumenta di 1, fino a un massimo di 5.

Hai +1d6 alle prove di Carisma effettuate per interagire con i nani.

Inoltre, mentre sei indossi la cintura hai il 50\% di probabilità ogni giorno all'alba di vederti spuntare una folta barba, se può crescerti, oppure di vedere la tua ancora più folta, se già la hai.

Se non sei un nano, ottieni i seguenti benefici aggiuntivi quando indossi questa cintura:

Hai +1d6 ai Tiri Salvezza contro veleno e hai resistenza ai danni da veleno.

Hai scurovisione con una gittata di 18 metri. Puoi parlare, leggere e scrivere in Nanico.


\index{Collana dell'Adattamento}\textbf{Collana dell'Adattamento}

\textit{Oggetto meraviglioso, non comune} - 1500 mo

Mentre indossi questa collana, puoi respirare normalmente in qualsiasi ambiente, e hai +1d6 ai Tiri Salvezza effettuati contro gas e vapori nocivi. 


\textbf{Colla Suprema}

\textit{Oggetto meraviglioso, leggendario} - 400 mo

Questa sostanza bianco lattea e viscosa può formare un legame adesivo permanente tra qualsiasi due oggetti. Deve essere contenuto in una giara o ampolla che è stata ricoperta all'interno di olio di scivolosità. Quando viene trovata, il suo contenitore ne tiene 1d6 + 1 per 30 grammi.

30 grammi di colla possono coprire una superficie quadrata di 30 centimetri di lato. La colla ci mette 1 minuto per fissarsi. Una volta fissata la colla, il legame creato può essere spezzato solo dal solvente universale o l'olio della forma eterea, o tramite l'incantesimo desiderio.


\index{Collana delle Palle di Fuoco}\textbf{Collana delle Palle di Fuoco}

\textit{Oggetto meraviglioso, raro} - a seconda delle sfere presenti: 500 mo, 1000 mo, 1600 mo, 2300 mo, 3100 mo, 4000 mo, 4500 mo, 5000 mo, 5500 mo, 6000 mo

Da questa collana pendono 1d6 + 3 sfere. Puoi usare due azioni per staccare una sfera e lanciarla fino a 18 metri di distanza. Quando essa raggiunge il termine della sua traiettoria, la sfera detona come un incantesimo palla di fuoco (DC 18). 


\index{Collana del Rosario}\textbf{Collana del Rosario}

\textit{Oggetto meraviglioso, raro} - 3000 mo + variabile

Questa collana possiede 1d4 + 2 sfere magiche fatte di acquamarina, perla nera o topazio. Possiede anche diverse sfere non magiche. Se una sfera magica venisse rimossa dalla collana, quella sfera perderebbe la sua magia.

Esistono sei tipi di sfere magiche. Il Narratore decide il tipo di ciascuna sfera facente parte della collana. Una collana può avere più di una sfera dello stesso tipo. Per usarla, devi indossare la collana. Ogni sfera contiene un incantesimo che puoi lanciare con due azioni, con DC dell'Incantesimo in caso di Tiro Salvezza. Una volta che l'incantesimo di una sfera magica è stato lanciato, non potrai usare di nuovo quella sfera fino all'alba successiva.


\medskip

\begin{tabularx}{0.45\textwidth}{llX}
	\textbf{3d6} &\textbf{Sfera di...} &\textbf{Incantesimo}\\
	\toprule
	3-5 &Benedizione &benedizione\\
	6-11& Cura &cura ferite serie o ristorare inferiore\\
	12-14 &Favore& ristorare superiore\\
	15-16& Punire &punizione marchiante\\
	17 &Vento& camminare nel vento\\
	18 &Convocare &alleato planare\\
\end{tabularx}

\medskip

\index{Corda da Arrampicata}\textbf{Corda da Arrampicata}

\textit{Oggetto meraviglioso, non comune} - 2000 mo

Questa corda di seta lunga 18 metri, pesa 1,5 chili e può sostenere fino a 1.500 chili. Se impugni un'estremità della corda e usi due azioni per pronunciare la parola di comando, la corda si anima. Con due azioni puoi comandare all'altra estremità di muoversi verso una destinazione di tua scelta. Quell'estremità si muove di 3 metri durante il tuo round quando riceve il tuo primo comando, e di 3 metri durante ciascun round successivo finché non raggiunge la sua destinazione, fino alla sua lunghezza massima, o finché non le dici di fermarsi. Puoi anche dire alla corda di stringersi o sganciarsi da un oggetto, annodarsi o snodarsi, o riavvolgersi per essere trasportata. Se dici alla corda di compiere un nodo, grossi nodi compariranno a intervalli di 30 centimetri lungo la corda. Mentre è annodata, la corda diminuisce fino a un lunghezza di 15 metri e conferisce +1d6 alle prove effettuate per arrampicarvisi.

La corda ha Difesa 20, Durezza 3 e 20 punti ferita. Recupera 1 punto ferita ogni 5 minuti finché ha almeno 1 punto ferita. Se la corda scende a 0 punti ferita, è distrutta. 


\index{Corda dell'Intralciamento}\textbf{Corda dell'Intralciamento}

\textit{Oggetto meraviglioso, raro} - 4000 mo

Questa corda è lunga 9 metri e pesa 1,5 chili. Se tieni un'estremità della corda e usi due azioni per pronunciare la sua parola di comando, l'altra estremità scatterà in avanti per impigliare una creatura visibile entro 6 metri da te. Il bersaglio deve superare un Tiro Salvezza su Riflessi con DC 18 o restare intralciato. Puoi rilasciare la creatura usando due azioni per pronunciare una seconda parola di comando. Un bersaglio intralciato dalla corda può usare due azioni per effettuare una prova di Forza o Artista della Fuga con DC 18 (a scelta del bersaglio). Se la supera, la creatura non è più intralciata dalla corda.

La corda ha Difesa 20 e 20 punti ferita. Recupera 1 punto ferita ogni 5 minuti finché ha almeno 1 punto ferita. se la corda scende a 0 punti ferita, è distrutta.


\index{Corno di Distruzione}\textbf{Corno di Distruzione}

\textit{Oggetto meraviglioso, raro} - 750 mo

Puoi usare due azioni per pronunciare la parola di comando del corno e poi suonarlo, emettendo uno scoppio tonante in un cono di 9 metri e udibile fino a 180 metri di distanza. Ogni creatura all'interno del cono deve effettuare un tiro Salvezza su Tempra con DC 18. Se il Tiro Salvezza fallisce, la creatura subisce 5d6 danni da tuono e resta assordata per 1 minuto. Se il Tiro Salvezza riesce, la creatura subisce la metà dei danni e non è assordata. Le creature e gli oggetti fatti di vetro o cristallo hanno -1d6 al Tiro Salvezza e subiscono 10d6 danni da tuono anziché 5d6.

Ogni uso della magia del corno ha il 20\% di probabilità di farlo esplodere. L'esplosione infligge 10d6 danni da fuoco a chi lo suona e distrugge il corno.


\index{Corno del Valhalla}\textbf{Corno del Valhalla}

\textit{Oggetto meraviglioso, raro (argento, ottone), molto raro (bronzo) o leggendario (ferro)} - 6000 mo base

Puoi usare due Azioni per suonare questo corno. Come risposta, entro 18 metri da te appaiono gli spiriti guerrieri di Asgard. Questi spiriti usano le statistiche dei berserker. Essi ritornano ad Asgard dopo 1 ora o quando scendono a 0 punti ferita. Una volta usato, il corno non potrà essere usato di nuovo prima che siano passati 7 giorni.

Esistono tre tipi di corno del Valhalla, ognuno fatto di un metallo differente. Il tipo di corno determina quanti spiriti guerrieri rispondano all'evocazione, oltre al requisito necessario per l'uso. Il Narratore sceglie il tipo di corno o lo determina casualmente.


\medskip

\begin{tabular}{llll}
\textbf{d100} &\textbf{Tipo Berserker} &\textbf{Evocati} & \textbf{Requisiti}\\
	\toprule
01-40 &Argento& 2d4 + 2 &CA 0\\
41-75 &Ottone& 3d4 + 3 &CA 1\\
76-90 &Bronzo& 4d4 + 4 &CA 2\\
91-100& Ferro& 5d4 + 5 &CA 3\\
\end{tabular}

\medskip

Se suoni il corno senza soddisfarne i requisiti, gli spiriti guerrieri evocati ti attaccheranno. Se soddisfi i requisiti, essi saranno amichevoli verso di te e i tuoi compagni, ed eseguiranno i tuoi ordini.


\index{Cubo di Forza}\textbf{Cubo di Forza}

\textit{Oggetto meraviglioso, raro} - 16000 mo

Questo cubo ha 2,5 centimetri di spigolo. Ogni faccia ha un marchio unico che può essere premuto. Il cubo inizia con 36 cariche, e recupera 3d6 cariche spese ogni giorno all'alba. Puoi usare due Azioni per premere una delle facce del cubo, spendendo un numero di cariche basate sulla faccia del cubo, come mostrato sulla tabella Facce del Cubo di Forza.

Ogni faccia ha un effetto diverso.Se al cubo non rimangono più cariche, non succede nulla. Altrimenti, si erge una barriera di forza invisibile, che forma un cubo di 4,5 metri di spigolo. La barriera è centrata su di te, si muove con te, e dura per 1 minuto, fino a che non usi due azioni per premere la sesta faccia del cubo, o il cubo esaurisce le cariche. Puoi cambiare l'effetto della barriera premendo una faccia diversa del cubo e spendendo il numero di cariche richiesto, resettandone la durata.

Se il tuo movimento fa sì che la barriera entri a contatto con un oggetto solido che non può attraversare il cubo, finché rimane la barriera non potrai avvicinarti all'oggetto. 


\medskip

\begin{tabularx}{0.45\textwidth}{llX}
\textbf{Faccia} & \textbf{Cariche}& \textbf{Effetto}\\
\toprule
1& 1& Gas, vento e nebbia non possono penetrare la barriera\\
2& 2 &La materia non vivente non può attraversare la barriera. Muri, pavimenti e soffitti possono attraversarla a tua discrezione.\\
3 &3 &La materia vivente non può attraversare la barriera.\\
4 &4 &Gli effetti dell'incantesimo non possono attraversare la barriera.\\
5 &5 &Nulla può attraversare la barriera. Muri,pavimenti e soffitti possono attraversarla a tua discrezione.\\
6 &0& La barriera si disattiva. \\
\end{tabularx}


\medskip

Il cubo perde cariche quando la barriera viene presa come bersaglio da certi incantesimi o entra a contatto con certi incantesimi o effetti di oggetti magici, come indicato nella tabella seguente.

\medskip

\begin{tabular}{ll}
\textbf{Incantesimo o Oggetto} &\textbf{Cariche Perse}\\
\toprule
Corno dell'esplosione &1d10\\
Disintegrazione &1d12\\
Muro di fuoco &1d4\\
Passapareti& 1d6\\
Spruzzo prismatico &3d6\\
\end{tabular}

\medskip

\index{Difensore}\textbf{Difensore}

\textit{Arma (qualsiasi spada), leggendario} - 35000 mo, 45000 mo, 70000 mo

Ottieni un bonus di +3,+4,+5 ai Tiri per Colpire e danno effettuati con quest'arma magica.

La prima volta che attacchi con questa spada durante un tuo round, puoi trasferire parte o tutto il suo bonus alla tua Difesa, invece di usare il bonus sugli attacchi di questo round. Il bonus così modificato rimane efficace fino all'inizio del tuo prossimo round, ma dovrai impugnare la spada per ottenere il bonus alla CA da parte sua. 

\index{Elmo della Comprensione dei Linguaggi}\textbf{Elmo della Comprensione dei Linguaggi}

\textit{Oggetto meraviglioso, non comune} - 600 mo

Mentre indossi questo elmo, puoi usare due azioni per lanciare a volontà tramite esso l'incantesimo comprendere linguaggi.

\index{Elmo della Lucentezza}\textbf{Elmo della Lucentezza}

\textit{Oggetto meraviglioso, molto raro}

Questo elmo luminoso è incastonato con 1d10 diamanti, 2d10 rubini, 3d10 opali di fuoco e 4d10 opali.

Qualsiasi gemma estratta dall'elmo si riduce in polvere. Quando tutte le gemme sono rimosse o distrutte, l'elmo perde la sua magia.


Mentre lo indossi ottieni i seguenti benefici: 

\medskip

\begin{itemize}
\item
Puoi usare due azioni per lanciare uno dei seguenti incantesimi, usando una delle gemme dell'elmo del tipo specificato come componente: luce diurna (opale), muro di fuoco (rubino), palla di fuoco (opale di fuoco) o spruzzo prismatico (diamante). Quando l'incantesimo viene lanciato la gemma è distrutta e scompare dall'elmo.

\item
Finché possiede almeno un diamante, l'elmo emette luce fioca in un raggio di 9 metri quando almeno un non morto si trova entro quest'area. Qualsiasi non morto che inizi il suo round all'interno dell'area subisce 1d6 danni da Luce.

\item
Finché l'elmo possiede almeno un rubino, hai resistenza ai danni da fuoco.
\end{itemize}

\medskip

Finché l'elmo possiede almeno un opale di fuoco, puoi usare due azioni e pronunciare una parola di comando per far sì che un'arma che stai impugnando venga avvolta dalle fiamme. Le fiamme emettono luce intensa in un raggio di 3 metri e luce fioca per ulteriori 3 metri. Le fiamme sono innocue per te e per l'arma. Quando colpisci con un attacco sferrato con l'arma infiammata, il bersaglio subisce
1d6 danni da fuoco aggiuntivi. Le fiamme perdurano fino a quando non userai due azioni per pronunciare la parola di comando di nuovo o fino a quando non lascerai cadere o rinfodererai l'arma.

Se stai indossando l'elmo e subisci danni da fuoco in seguito al fallimento di un Tiro Salvezza contro un incantesimo con 3 uno l'elmo emette un fascio di luce tramite le gemme rimanenti. Ogni creatura entro 18 metri dall'elmo, a parte te, deve superare un Tiro Salvezza su Riflessi con DC 21 o venire colpita dal fascio, subendo danni di Luce uguali al numero di gemme nell'elmo. Poi, le gemme e l'elmo vengono distrutti.

\index{Elmo della Telepatia}\textbf{Elmo della Telepatia}

\textit{Oggetto meraviglioso, non comune} - 12000 mo

Mentre indossi questo elmo, puoi usare due azioni per lanciare tramite esso l'incantesimo individuazione dei pensieri (DC del Tiro Salvezza 13). Finché mantieni la concentrazione sull'incantesimo, puoi usare due azioni  per inviare un messaggio telepatico alla creatura su cui sei concentrato. Essa può replicare (usando due azioni per farlo) fino a quando continui a concentrarti su di lei.

Mentre ti concentri su di una creatura con individuazione dei pensieri, puoi usare due azioni per lanciare tramite l'elmo l'incantesimo suggestione (DC del Tiro Salvezza 13) su quella creatura. Una volta usata, la proprietà suggestione non potrà essere usata di nuovo fino alla prossima alba.

\index{Elmo del Teletrasporto}\textbf{Elmo del Teletrasporto}

\textit{Oggetto meraviglioso, raro} - 64000 mo

Mentre indossi questo elmo, puoi usare due azioni e spendere 1 carica per lanciare l'incantesimo teletrasporto tramite esso. L'elmo ha 3 cariche, e ne recupera 1d3 ogni mattina all'alba.

\index{Faretra Efficiente}\textbf{Faretra Efficiente}

\textit{Oggetto meraviglioso, non comune} - 2500 mo

Ciascuno dei tre compartimenti della faretra è collegato a uno spazio extradimensionale che le permetta di trasportare numerosi oggetti non pesando mai più di 1 chilo.

Il compartimento più piccolo può contenere fino a 60 frecce, saette od oggetti simili. Il compartimento mediano può contenere fino a 18 giavellotti od oggetti simili. Il compartimento più lungo può contenere fino a 6 oggetti lunghi, come archi, bastoni da combattimento o lance. Puoi estrarre qualsiasi oggetto contenuto nella faretra come se lo stessi prendendo da una normale faretra o fodero.

\index{Fasce di Ferro del Vincolo}\textbf{Fasce di Ferro del Vincolo}

\textit{Oggetto meraviglioso, raro} - 5000 mo

Questa sfera di ferro arrugginita misura 7,5 centimetri di diametro e pesa 500 grammi. Puoi usare due azioni per pronunciare una parola di comando e scagliare la sfera contro una creatura visibile di taglia Enorme o inferiore entro 18 metri da te. La sfera si muove nell'aria, aprendosi in un reticolato di fasce metalliche. Effettua un Tiro per Colpire a distanza, se colpisci, il bersaglio è intralciato fino a quando non effettuerai due azioni per pronunciare una parola di comando e liberarlo. Farlo, o mancare l'attacco, fa sì che le fasce si contraggano e ritornino a essere una sfera.

Una creatura, compresa quella intralciata, può usare due azioni per effettuare una prova di Forza con DC 25 per spezzare le fasce di ferro. Se la riesce, l'oggetto viene distrutto, e la creatura intralciata è libera. Se la prova fallisce, qualsiasi ulteriore tentativo effettuato dalla creatura fallisce automaticamente fino a quando non saranno trascorse 24 ore.

Una volta che le fasce sono state usate non potranno più esserlo fino alla prossima alba.

\index{Bandana dell'Intelligenza}\textbf{Bandana dell'Intelligenza}

\textit{Oggetto meraviglioso, non comune } - 8000 mo

Mentre indossi questa bandana il tuo Intelligenza è +4. La fascetta non ha effetto se hai già Intelligenza è già +4 o più alta.

\index{Filtro d'Amore}\textbf{Filtro d'Amore}

\textit{Pozione, non comune} - 120

Resterai affascinato per 1 ora dalla prima creatura che vedrai entro 10 minuti da quando avrai bevuto questo filtro. Se la creatura è di una specie o genere da cui sei normalmente attratto,finché sei affascinato la considererai il tuo unico e grande amore.

\index{Fortezza Istantanea}\textbf{Fortezza Istantanea}

\textit{Oggetto meraviglioso, raro} - 75000 mo

Puoi usare due azioni per porre questo cubo di metallo di 2,5 centimetri di spigolo sul terreno e pronunciarne la parola di comando. Il cubo cresce rapidamente fino a diventare una fortezza che resterà fino a quando userai due azioni per pronunciare la parola di comando che la congeda, la quale funziona solo quando la fortezza è vuota.

La fortezza è una torre quadrata, 6 metri per lato e alta 9 metri, con feritoie su tutti i lati e spalti in cima. Il suo interno è diviso in due piani, con una scala che corre lungo una parete a congiungerli. La scala termina con una botola che si apre sul tetto. Quando viene attivata, la torre presenta una piccola porta sul lato rivolto verso di te. La porta si apre solo al tuo comando, che puoi pronunciare con due azioni. È immune all'incantesimo scassinare e magie simili, come quella del battaglio dell'apertura.

Ogni creatura nell'area in cui la fortezza compare deve effettuare un Tiro Salvezza su Riflessi con DC 17, subendo 10d10 danni contundenti se lo fallisce, o la metà di questi danni se lo riesce. In entrambi i casi, la creatura viene spinta in uno spazio fuori della fortezza ma in sua prossimità. Gli oggetti nell'area che non sono indossati o trasportati subiscono gli stessi danni e vengono spinti automaticamente.

La torre è fatta di adamantio, e la sua magia le impedisce di venir ribaltata. Il tetto, la porta e le mura hanno 100 punti ferita ognuno, immunità ai danni dalle armi non magiche a eccezione delle armi da assedio, e resistenza a tutti gli altri danni.

Solo l'incantesimo desiderio può riparare la fortezza. Ciascun lancio di desiderio fa sì che il tetto, la porta o una delle pareti recuperi 50 punti ferita.

\index{Gemma Elementale}\textbf{Gemma Elementale}

\textit{Oggetto meraviglioso, non comune} - 1200 mo

Questa gemma contiene una scintilla di energia elementale. Quando usi due azioni per infrangere la gemma, questa evoca un elementale come se tu avessi lanciato l'incantesimo evoca elementali, e la magia della gemma svanisce. Il tipo di gemma determina l'elementale evocato dall'incantesimo.


\medskip

\begin{tabularx}{0.45\textwidth}{XX}
\textbf{Gemma Elementale} &\textbf{Evocato}\\
\toprule
Corindone rosso& Elementale del fuoco\\
Diamante giallo& Elementale della terra\\
Smeraldo &Elementale dell'acqua\\
Zaffiro &blu Elementale dell'aria\\
\end{tabularx}

\medskip

\index{Gemma della Luminosità}\textbf{Gemma della Luminosità}

\textit{Oggetto meraviglioso, non comune} - 5000 mo

Questo prisma ha 50 cariche. Mentre lo impugni, puoi usare due azioni per pronunciare una delle tre parole di comando per provocare uno dei seguenti effetti:

\begin{itemize}
\item
La prima parola di comando fa sì che la gemma produca una luce intensa nel raggio di 9 metri e luce fioca per ulteriori 9 metri. L'effetto non consuma cariche. Dura finché non userai due azioni per ripetere la parola di comando o finché non impiegherai un'altra funzione della gemma.

\item
La seconda parola di comando spende 1 carica e fa sì che la gemma proietti una fascio di luce luminoso contro una creatura visibile entro 18 metri da te. La creatura deve superare un Tiro Salvezza su Tempra con DC 17 o restare accecata per 1 minuto. 

\item
La terza parola di comando spende 5 cariche e fa sì che la gemma irradi una luce accecante in un cono di 9 metri originante da te. Ogni creatura all'interno del cono deve effettuare un Tiro Salvezza come se fosse stata colpita dal fascio creato dalla seconda parola di comando.

\end{itemize}

\medskip

Quando tutte le cariche della gemma sono state spese, la gemma diventa un comune gioiello del valore di 50 mo.

\index{Gemma della Vista}\textbf{Gemma della Vista}

\textit{Oggetto meraviglioso, raro} - 32000 mo

Con due azioni, puoi pronunciare la parola di comando della gemma e spendere 1 carica. Per i successivi 10 minuti, quando guardi attraverso la gemma possiedi la visione del vero fino a 36 metri di distanza. La gemma ha 3 cariche, e recupera 1d3 cariche spese ogni giorno all'alba.

\index{Giavellotto dei Fulmini}\textbf{Giavellotto dei Fulmini}

\textit{Arma (giavellotto), non comune} - 1500 mo

Questo giavellotto è un'arma magica. Quando lo scagli e pronunci la sua parola di comando, si trasforma in un fulmine, formando una linea larga 1 metro che si estende da te verso un bersaglio entro 36 metri. Ogni creatura sulla linea, escluso te e il bersaglio, deve effettuare un Tiro Salvezza su Riflessi con DC 15, subendo 4d6 danni da fulmine se lo fallisce o la metà di questi danni se lo riesce. Il fulmine ridiventa un giavellotto quando raggiunge il bersaglio. Se lo colpisce, il bersaglio subisce i danni del giavellotto più 4d6 danni da fulmine.
\
La proprietà del giavellotto non può più essere usata fino alla prossima alba. Nel frattempo, il giavellotto può essere comunque usato come arma magica. 

\index{Guanti Afferra Proiettili}\textbf{Guanti Afferra Proiettili}

\textit{Oggetto meraviglioso, non comune} - 3000 mo

Questi quanti sembrano quasi fondersi con la tua pelle quando li indossi. Quando un attacco con arma a distanza ti colpisce mentre li indossi, puoi usare una Reazione per ridurre il danno di 1d10 + Destrezza, purché tu abbia una mano libera. Se riduci il danno a 0, e il proiettile è piccolo a sufficienza da essere tenuto in mano, puoi afferrarlo.

\index{Guanti d'Arme del Potere Ogre}\textbf{Guanti d'Arme del Potere Ogre}

\textit{Oggetto meraviglioso, non comune}

Mentre indossi queste manopole la tua Forza è 4 . Le manopole non hanno effetto se la tua Forza è già 4 o più.

\index{Guanti del Nuoto e della Scalata}\textbf{Guanti del Nuoto e della Scalata}

\textit{Oggetto meraviglioso, non comune} - 2000 mo

Mentre indossi entrambi questi guanti, la scalata e il nuoto non ti costano movimento aggiuntivo. Inoltre, hai un bonus di +1d6 alle prove di Costituzione e Saggezza effettuate mentre scali o nuoti.

\index{Incensiere del Comando degli Elementali dell'Aria}\textbf{Incensiere del Comando degli Elementali dell'Aria}

\textit{Oggetto meraviglioso, raro} - 8000 mo

Mentre l'incenso brucia all'interno di questo incensiere, puoi usare due azioni per pronunciare la parola di comando del braciere ed evocare un elementale dell'aria, come se avessi lanciato l'incantesimo evoca elementali. L'incensiere non può di nuovo essere usato a questo modo fino alla prossima alba. Questo incensiere largo 15 centimetri e alto 30 centimetri assomiglia a un calice dalla copertura decorata. Pesa 0,5 chili.

\index{Ladra delle Nove Vite}\textbf{Ladra delle Nove Vite}

\textit{Arma (qualsiasi spada), molto raro} - 12000 mo

Ottieni un bonus di +2 ai Tiri per Colpire e danno effettuati con quest'arma magica. Se ottieni un colpo critico contro una creatura che ha meno di 100 punti ferita, questa deve superare un tiro Salvezza su Tempra con DC 17 o venire immediatamente uccisa, mentre la spada ne risucchia la forza vitale dal corpo (i costrutti e i non morti sono immuni a questa proprietà). 

La spada ha 1d8 + 1 cariche, e perde 1 carica quando una creatura viene uccisa. Quando la spada non ha più cariche, perde questa proprietà.

\index{Lama della Fortuna}\textbf{Lama della Fortuna}

\textit{Arma (qualsiasi spada), leggendario}

Ottieni un bonus di +1,+2,+3 ai Tiri per Colpire e danno effettuati con quest'arma magica. Finché hai addosso la spada ricevi anche un bonus di +1 ai Tiri Salvezza.

\textit{Fortuna}. Se hai addosso la spada, puoi affidarti alla sua fortuna (non richiede due azioni) per ripetere un Tiro per Colpire, prova di caratteristica o Tiro Salvezza il cui risultato non ti soddisfa. Sei obbligato a usare il secondo risultato del dado. Questa proprietà non può essere usata di nuovo fino alla prossima alba.

\textit{Desiderio}. Mentre la impugni, puoi usare due azioni per spendere 1 carica e lanciare tramite essa l'incantesimo desiderio. Questa proprietà non può essere usata di nuovo
fino alla prossima alba. La spada ha 1d4-1 cariche, e perde questa proprietà se finisce le cariche.

\index{Lama del Sole}\textbf{Lama del Sole}

\textit{Arma (spada lunga), raro} - 12000 mo

Quest'oggetto sembra l'impugnatura di una spada lunga, ma senza lama. Quando ne afferri l'impugnatura, puoi usare due azioni per far sì che una lama di pura luminescenza si formi, o faccia sparire la lama inserita nell'impugnatura.

Finché la spada esiste, questa spada lunga magica ha la proprietà Versatile.

Se sei competente con le spade corte o le spade lunghe, sei competente anche con la lama del sole.

Ottieni un bonus di +2 ai Tiri per Colpire e danno effettuati con quest'arma, che infligge danni da Luce anziché danni taglienti. Quando colpisci con essa una creatura non morta, il bersaglio subisce 1d8 danni da Luce aggiuntivi.

La lama luminosa della spada emette luce intensa in un raggio di 4,5 metri e luce fioca per ulteriori 4,5 metri. La luce è luce solare. Finché la lama è attiva, puoi usare due azioni per espandere o ridurre il raggio della luce intensa e fioca di 1,5 metri ciascuno, fino a un massimo di 9 metri o un minimo di 3 metri ciascuno.

\index{Lanciatore Nanico}\textbf{Lanciatore Nanico}

\textit{Arma (martello da guerra), molto raro} - 18000 mo

Ottieni un bonus di +3 ai Tiri per Colpire e danno effettuati con quest'arma magica. Ha Gittata di 6 metri.

Quando colpisci con un attacco a distanza usando quest'arma, essa infligge 1d8 danni aggiuntivi o, se il bersaglio è un gigante, 2d8 danni aggiuntivi. Subito dopo l'attacco, l'arma vola indietro nella tua mano. 


\index{Lanterna della Rivelazione}\textbf{Lanterna della Rivelazione}

\textit{Oggetto meraviglioso, non comune} - 5000 mo

Mentre è accesa, questa lanterna coperta brucia per 6 ore con 1 pinta d'olio, irradiando luce intensa in un raggio di 9 metri e luce fioca per ulteriori 9 metri. Le creature e gli oggetti invisibili sono resi visibili mentre si trovano sotto la luce intensa della lanterna. 

Puo usare due azioni per abbassare la copertura, riducendo la luce a fioca con un raggio di 1,5 metri.

\index{Lingua di Fuoco}\textbf{Lingua di Fuoco}

\textit{Arma (qualsiasi spada), raro} - 5000 mo

Puoi usare due azioni per pronunciare la parola di comando di questa spada magica, facendo sì che dalla sua lama eruttino fiamme. Queste fiamme irradiano luce intensa in un raggio di 12 metri e luce fioca per ulteriori 12 metri. Mentre la spada è in fiamme, infligge 2d6 danni da fuoco aggiuntivi a qualsiasi bersaglio colpisca. Le fiamme durano fino a che non usi due azioni per pronunciare di nuovo la parola di comando o finché non lasci cadere o rinfoderi l'arma. 

\index{Mantella del Ciarlatano}\textbf{Mantella del Ciarlatano}

\textit{Oggetto meraviglioso, raro} - 8000 mo

Mentre indossi questa mantella che odora lievemente di zolfo, puoi usarla per lanciare l'incantesimo porta dimensionale con due azioni. La proprietà di questa mantella non può essere usata di nuovo fino all'alba. Quando scompari, ti lasci alle spalle una nube di fumo, e riappari alla tua destinazione all'interno di una simile nube di fumo. Questo fumo oscura leggermente lo spazio che hai lasciato e quello dove riappari, e si dissipa alla fine del tuo prossimo round. Un vento leggero o più forte disperde il fumo.

\index{Mantello della Resistenza agli Incantesimi}\textbf{Mantello della Resistenza agli Incantesimi}

\textit{Oggetto meraviglioso, raro} - 30000 mo

Mentre indossi questa cappa, hai +2 ai Tiri Salvezza contro incantesimi.

\index{Manuale degli Esercizi Pratici}\textbf{Manuale degli Esercizi Pratici}

\textit{Oggetto meraviglioso, molto raro}

Questo libro descrive esercizi atletici, e le sue parole sono soffuse di magia. Se trascorri 48 ore in un periodo di 6 giorni o meno a studiare i contenuti del libro e praticarne le indicazioni, il tuo punteggio di Costituzione aumenta di 1, poi il manuale perde la sua magia, per recuperarla dopo un secolo.

\index{Manuale dei Golem}\textbf{Manuale dei Golem}

\textit{Oggetto meraviglioso, molto raro}

Questo tomo contiene le informazioni e incantamenti necessari a costruire un tipo particolare di golem. Il Narratore sceglie il tipo di golem che è possibile costruire o lo determina casualmente. Per decifrare e usare il manuale devi avere almeno CM 10. Una creatura che non possa usare il manuale dei golem e provi a leggerlo, subisce 6d6 danni psichici.

\medskip

\begin{tabular}{llll}
	3d6 &Golem &Tempo &Costo\\
	\toprule
	3-4 &Argilla &30 giorni &65.000 mo\\
	5-16 &Carne &60 giorni& 50.000 mo\\
	17  &Ferro &120 giorni &100.000 mo\\
	18 &Pietra& 90 giorni &80.000 mo\\
\end{tabular}


\medskip

Per creare un golem, devi trascorrere il tempo sopra indicato, lavorando senza interruzione con il manuale a disposizione e riposando per non più di 8 ore al giorno. Devi anche pagare il costo specificato per acquistare i materiali necessari.

Una volta finito di creare il golem, il libro viene consumato da fiamme arcane. Il golem si anima quando le ceneri del manuale saranno sparse su di esso. Sarà sotto il tuo controllo e comprende e obbedisce gli ordini pronunciati da te.

\index{Manuale della Rapidità d'Azione}\textbf{Manuale della Rapidità d'Azione}

\textit{Oggetto meraviglioso, molto raro}

Questo libro contiene esercizi di coordinazione ed equilibrio, e le sue parole sono soffuse di magia. Se trascorri 48 ore in un periodo di 6 giorni o meno a studiare i contenuti del libro e praticarne le indicazioni, il tuo punteggio di Destrezza aumenta di 1. Poi il manuale perde la sua magia, per recuperarla dopo un secolo.

\index{Martello dei Fulmini}\textbf{Martello dei Fulmini}

\textit{Arma (maglio), leggendario} - 20000 mo

Ottieni un bonus di +2 ai Tiri per Colpire e di danno effettuati con quest'arma magica.

\textit{Anatema dei Giganti}. Devi indossare una \textit{cintura dei giganti} (qualsiasi varietà) e i \textit{guanti d'arme della forza ogre} per poter usare quest'arma. 

Mentre usi il martello il tuo punteggio di Forza aumenta di 2 (fino ad un massimo di 15).

Quando ottieni un critico sul Tiro per Colpire effettuato con quest'arma contro un gigante, il gigante deve superare un tiro Salvezza su Tempra con DC 21 o morire.

Puoi spendere 1 carica ed effettuare un attacco con arma a distanza con il martello, scagliandolo come se avesse la proprietà da lancio con un gittata normale di 6 metri. Se l'attacco colpisce, il martello produce un tuono udibile fino a 90 metri di distanza. Il bersaglio e tutte le creature entro 9 metri da esso devono superare un tiro Salvezza su Tempra con DC 21 o restare stordite fino al termine del tuo prossimo round. 

Il martello ha 5 cariche, e recupera 1d4 + 1 cariche spese ogni giorno all'alba.

\index{Mazza della Distruzione}\textbf{Mazza della Distruzione}

\textit{Arma (mazza), raro} - 12000 mo

Quando colpisci un immondo o un non morto con quest'arma magica, quella creatura subisce 2d6 danni da Luce aggiuntivi. Se, dopo aver subito il danno, al bersaglio rimangono 25 punti ferita o meno, questi deve superare un Tiro Salvezza su Volontà con DC 17 o venire distrutto. Se il Tiro Salvezza riesce, la creatura resta spaventata da te fino al termine del tuo prossimo round.

Mentre impugni quest'arma, essa irradia luce intensa in un raggio di 6 metri e luce fioca per ulteriori 6 metri.

\index{Mazza della Punizione}\textbf{Mazza della Punizione}

\textit{Arma (mazza), raro} - 7000 mo

Ottieni un bonus di +1 ai Tiri per Colpire e danno effettuati con quest'arma magica. Il bonus aumenta a +3 quando usi quest'arma per attaccare un costrutto. 

Quando ottieni un critico al Tiro per Colpire effettuato con quest'arma, il bersaglio subisce 7 danni contundenti aggiuntivi, o 14 danni contundenti aggiuntivi se è un costrutto. Se, dopo aver subito questi danni, a un costrutto restano 25 punti ferita o meno, viene distrutto.

\index{Mazza del Terrore}\textbf{Mazza del Terrore}

\textit{Arma (mazza), raro} - 8000 mo

Mentre la impugni, puoi usare due azioni e spendere 1 carica per scatenare un'ondata di terrore.
Ogni creatura di tua scelta, in un raggio di 9 metri a partire da te, deve superare un Tiro Salvezza su Volontà con DC 17 o restare spaventata da te per 1 minuto. Mentre è spaventata a questo modo, una creatura deve impiegare i suoi turni a cercare di muoversi più lontano possibile da te, e non può consapevolmente muoversi in uno spazio che sia entro 9 metri da te. Inoltre non può effettuare reazioni. Come sua azione, può usare solo l'azione di Movimento per scappare. Se non può muoversi da nessuna parte, la creatura può usare l'Azione Difesa Totale. 
 
Al termine di ciascun suo round, la creatura può ripetere il Tiro Salvezza, terminando l'effetto per sé in caso lo superi. Quest'arma magica ha 3 cariche, e recupera 1d3 cariche ogni giorno all'alba.

\index{Mazzo delle Illusioni}\textbf{Mazzo delle Illusioni}

\textit{Oggetto meraviglioso, non comune} - 6500 mo

Questa scatola contiene un set di carte di pergamena. Un mazzo completo contiene 34 carte, ognuna raffigurante una creatura diversa. Le creature rappresentate vengono lasciate alla discrezionalità del Narratore. Di solito i mazzi trovati in giro sono privi di 3d6-3 carte.

La magia del mazzo funziona solo se le carte vengono pescate a caso (potete usare un mazzo di normali carte da gioco modificato per simulare il mazzo delle illusioni). Puoi usare due azioni per pescare una carta dal mazzo e scagliarla in un punto sul terreno a 9 metri da te.

L'illusione di una o più creature si forma sopra la carta lanciata e rimane finché non viene dissolta. La creatura illusoria sembra reale, della taglia appropriata, e si comporta come fosse una vera creatura, eccetto che non può recare danni. Finché resti entro 36 metri dalla creatura illusoria e puoi vederla, puoi usare due azioni per muoverla magicamente in qualsiasi punto entro 9 metri dalla carta. Qualsiasi interazione fisica con la creatura illusoria la rivela come illusione, dato che gli oggetti le passano attraverso. Qualcuno che usi due azioni per ispezionare visivamente la creatura, la identifica come illusoria superando una prova di Intelligenza con DC 17. La creatura le apparirà quindi trasparente.
L'illusione permane finché la carta non viene mossa o l'illusione dissolta. Quando l'illusione termina, l'immagine sulla carta scompare, e quella carta non potrà più essere usata.

\end{multicols}

\medskip

\begin{tabular}{ll|ll}
\textbf{Carta da Gioco}& \textbf{Illusione}&\textbf{Carta da Gioco}& \textbf{Illusione}\\
\toprule
Asso di cuori &Drago rosso&Asso di quadri& Beholder\\
Re di cuori &Cavaliere e quattro guardie&Re di quadri & Arcimago e magio apprendista\\
Regina di cuori &Succube o incubo&Regina di quadri &Megera notturna\\
Fante di cuori &Druido&Fante di quadri &Assassino\\
Dieci di cuore &Gigante delle nuvole&Dieci di quadri &Gigante del fuoco\\
Nove di cuori &Ettin&Nove di quadri &Oni\\
Otto di cuori& Bugbear&Otto di quadri &Gnoll\\
Due di cuori &Goblin&Due di quadri &Coboldo\\
Asso di picche &Lich&Asso di fiori& Golem di ferro\\
Re di picche &Sacerdote e due accoliti&Re di fiori &Capitano bandito e tre banditi\\
Regina di picche& Medusa&Regina di fiori &Erinni\\
Fante di picche &Veterano&Fante di fiori &Berserker\\
Dieci di picche &Gigante del gelo&Dieci di fiori &Gigante di collina\\
Nove di picche &Troll&Nove di fiori &Ogre\\
Otto di picche &Hobgoblin&Otto di fiori& Orco\\
Due di picche &Goblin&Due di fiori &Coboldo\\
Jolly (2) &Tu (il proprietario del mazzo)&&\\
\end{tabular}

\begin{multicols}{2}

\medskip

\index{Mazzo delle Meraviglie}\textbf{Mazzo delle Meraviglie}

\textit{Oggetto meraviglioso, leggendario}

Di solito lo si trova in un borsello o una scatola, che contiene delle carte fatte d'avorio o vello. La maggior parte di questi mazzi (il 75\%) ha solo tredici carte, mentre i restanti mazzi ne hanno ventidue.

Prima di pescare una carta, devi dichiarare quante carte intendi pescare e poi pescarle casualmente (puoi usare un mazzo di carte da gioco modificato per simulare il mazzo). Qualsiasi carta pescata in eccesso di questo numero non ha effetto. Altrimenti, appena peschi una carta dal mazzo, la sua magia ha effetto.

Devi pescare ciascuna carta entro 1 ora dalla pescata precedente. Se non peschi il numero scelto di carte, il numero di carte rimanenti uscirà fuori dal mazzo spontaneamente e avrà effetto in contemporanea. Una volta estratta una carta, questa svanirà dall'esistenza. A meno che la carta non sia il Matto o il Buffone, la carta ricompare nel mazzo, rendendo possibile pescare due volte la stessa carta.

\medskip

\end{multicols}

\begin{tabularx}{0.95\textwidth}{lX|lX}
\textbf{Carta da Gioco}& \textbf{Carta}&\textbf{Carta da Gioco}& \textbf{Carta}\\
\toprule
Asso di quadri& Visir*&Asso di cuori &Fato*\\
Re di quadri &Sole&Re di cuori &Trono\\
Regina di quadri& Luna&Regina di cuori& Chiave\\
Fante di quadri &Stella&Fante di cuori& Cavaliere\\
Due di quadri &Cometa*&Due di cuori &Gemma*\\
Asso di fiori &Speroni*&Asso di picche& Dongione*\\
Re di fiori &Il Vuoto&Re di picche& Rovina\\
Regina di fiori& Fiamme&Regina di picche &Euriale\\
Fante di fiori &Teschio&Fante di cuori& Furfante\\
Due di fiori &Idiota&Due di picche &Appeso*\\
Jolly &Matto*&Jolly &Buffone\\
\end{tabularx}

\begin{multicols}{2}

\medskip

* Solo in mazzo da 22 carte

\textit{Appeso} (solo in mazzo da 22). La tua mente è sconvolta, e cambi 2 Tratti

\textit{Buffone}. Ottieni 35 PX o puoi pescare due carte aggiuntive oltre alle tue pescate dichiarate.

\textit{Cavaliere}. Ottieni i servigi di un guerriero con CA 4 livello che compare in uno spazio a tua scelta entro 9 metri da te. Il guerriero è della tua stessa razza e ti servirà lealmente fino alla morte, credendo che sia stato il fato a portarlo al tuo servizio. Il personaggio è controllato da te.

\textit{Chiave}. Un'arma magica rara, molto rara o leggendaria con la quale sei competente compare tra le tue mani. Il Narratore determina di che tipo di arma si tratta. 

\textit{Cometa} (solo in mazzo da 22). Se sconfiggi da solo il prossimo mostro o gruppo ostile che incontrerai, otterrai abbastanza punti esperienza da guadagnare un livello. Altrimenti, questa carta non avrà effetto.

\textit{Euriale}. Sei maledetto dalla carta e subisci una penalità di -2 a tutti i Tiri Salvezza finché resterai maledetto a questo modo. Solo un Patrono o la magia della carta del Fato può porre fine a questa maledizione.

\textit{Fato} (solo in mazzo da 22). La struttura della realtà si dissolve e riforma, permettendoti di evitare o cancellare un evento come se non fosse mai accaduto. Puoi usare la magia di questa carta non appena l'hai pescata o aspettare un qualsiasi altro momento fino alla tua morte.

\textit{Fiamme}. Un potente diavolo diventa tuo nemico. Il diavolo cercherà di rovinare e infestare la tua esistenza, assaporando le tue sofferenze fino al momento in cui cercherà di ucciderti. Questa inimicizia durerà fino alla morte tua o del diavolo.

\textit{Furfante}. Un personaggio non dei giocatori a scelta del Narratore diventa ostile nei tuoi confronti. L'identità del nuovo nemico è ignota fino a quando il PNG o qualcun altro la rivelerà. Nulla a meno di un desiderio o intervento divino potrà porre fine all'ostilità del PNG nei tuoi confronti.

\textit{Gemma} (solo in mazzo da 22). Davanti ai tuoi piedi compaiono venticinque gioielli del valore di 2.000 mo ciascuno o cinquanta gemme del valore di 1.000 mo ciascuna.

\textit{Idiota} (solo in mazzo da 22). Riduci permanentemente il tuo punteggio di Intelligenza di 2 (fino a un punteggio minimo di 3). Puoi pescare un'ulteriore carta prima delle tue altre pescate dichiarate. 

\textit{Luna}. Ricevi la capacità di lanciare l'incantesimo desiderio 1d3 volte.

\textit{Matto} (solo in mazzo da 22). Perdi 10.000 PE, scarti questa carta, e peschi di nuovo dal mazzo, contando entrambe le pescate come solo una delle tue pescate. Se perdere quel numero di PE ti farebbe perdere un livello, rimarrai invece con il numero di PE appena sufficienti per mantenere il tuo livello.

\textit{Rovina}. Perdi tutte le ricchezze che hai con te, a parte gli altri oggetti magici. Attività, edifici e le terre che possiedi vengono perse nel modo che altera di meno la realtà. Qualsiasi documento che provi che tu sia il proprietario di qualcosa che hai perso a causa di questa carta, scompare. 

\textit{Sole}. Ottieni 35 PX, e un oggetto meraviglioso (determinato dal Narratore) compare tra le tue mani. 

\textit{Sotterraneo} (solo in mazzo da 22). Scompari e vieni sepolto in uno stato di animazione sospesa all'interno di una sfera extradimensionale. Tutto ciò che stavi indossando o trasportando rimane nello spazio da te occupato quando sei scomparso. Rimarrai imprigionato finché non sarai ritrovato e rimosso dalla sfera. Non puoi essere localizzato tramite nessuna magia di divinazione, ma l'incantesimo desiderio può rivelare la posizione della tua prigione. Non si pescano ulteriori carte.

\textit{Speroni} (solo in mazzo da 22). Ogni oggetto magico che indossi o trasporti viene disintegrato. Gli artefatti in tuo possesso non vengono disintegrati, ma svaniscono. 

\textit{Stella}. Aumenta un tuo punteggio di caratteristica di 1. Il punteggio può superare il 5, ma non può superare 7.
\textit{Teschio}. Evochi un avatara della morte (uno spettrale scheletro umanoide avvolto in una vestaglia nera e sbrindellata, il quale impugna una falce spettrale). Esso compare in uno spazio a scelta del Narratore entro 3 metri da te e ti attacca, avvisando tutti gli altri che devi vincere la battaglia da solo. L'avatara combatte fino alla tua morte o finché non scende a 0 punti ferita, al che svanisce. Se qualcuno cerca di aiutarti, costui evocherà il proprio avatara della morte. Una creatura uccisa da un avatara della morte non può essere riportata in vita. 

\textit{Trono}. Ottieni +4 in Oratoria. Inoltre, ottieni il diritto di proprietà su di una piccola rocca da qualche parte nel mondo. Tuttavia, la rocca è attualmente occupata da mostri, che dovrai cacciare prima di poterla rivendicare come tua.

\textit{Visir} (solo in mazzo da 22). In qualsiasi momento di tua scelta, entro un anno da quando hai pescato questa carta, puoi chiedere, meditando, risposta a una tua domanda e ricevere una risposta veritiera a essa. A parte fornire informazioni, la risposta può aiutarti a risolvere un problema complesso o un dilemma. In altre parole, la conoscenza è fornita assieme alla saggezza su come impiegarla.

\textit{Vuoto}. Questa carta nera è indice di disastro. La tua anima viene rapita dal corpo e imprigionata all'interno di un oggetto in un luogo a scelta del Narratore. Una o più potenti creature proteggono questo luogo. Finché la tua anima è così intrappolata, il tuo corpo è inabile. L'incantesimo desiderio non è in grado di ripristinare la tua anima, ma può rivelare il luogo in cui si trova l'oggetto che la contiene. Non si pescano più carte.

\textit{Avatara della Morte}

Non morto media, neutrale malvagio

\textbf{Forza} +3

\textbf{Destrezza}' +3

\textbf{Intelligenza} +3

\textbf{Saggezza} +3

\textbf{Carisma} +3

\textbf{Difesa} 20

\textbf{Punti Ferita} metà dei punti ferita del suo evocatore

\textbf{Movimento}: Velocità 18 m, volo 18 m (fluttua)

\textbf{Immunità ai Danni}: Vuoto, veleno

\textbf{Immunità alle Condizioni}: affascinato, avvelenato, paralizzato, pietrificato, spaventato, svenuto

\textbf{Sensi}: scurovisione 18 m, visione del vero 18 m\\
\textbf{Linguaggi}: tutti i linguaggi conosciuti dal suo evocatore 

\textbf{Sfida} - (0 PE)

\textbf{Movimento Incorporeo}. L'avatara può attraversare creature e oggetti come fossero terreno difficile. Subisce 5 (1d10) danni da forza se termina il proprio round all'interno di un oggetto.
 
\textbf{Immunità allo Scacciare}. L'avatara è immune agli effetti che scacciano i non morti.

\textbf{Azioni}

\textbf{Falce Mietitrice}. L'avatara affonda la sua falce spettrale in una creatura entro 1,5 metri da esso, infliggendo 7 (1d8 + 3) danni perforanti più 4 (1d8) danni da Energia Negativa.


\index{Medaglione dei Pensieri}\textbf{Medaglione dei Pensieri}

\textit{Oggetto meraviglioso, non comune} - 3000 mo

Mentre indossi questo medaglione, puoi usare due azioni e spendere 1 carica per lanciare tramite esso l'incantesimo individuazione dei pensieri (DC del Tiro Salvezza 15). Il medaglione ha 3 cariche, e recupera 1d3 cariche spese ogni giorno all'alba.

\index{Miniatura dal Potere Meraviglioso}\textbf{Miniatura dal Potere Meraviglioso}

\textit{Oggetto meraviglioso, rarità in base alla miniatura}

Una miniatura dal potere meraviglioso è una statuetta di una bestia, piccola a sufficienza da entrare in tasca. Se usi due azioni per pronunciare una parola di comando e lanciare la miniatura in un punto del terreno entro 18 metri da te, la miniatura diventa una creatura vivente. Se lo spazio in cui la creatura dovesse apparire è occupato da un'altra creatura od oggetto, o se non c'è spazio sufficiente per la creatura, la miniatura non si trasforma.

La creatura è amichevole nei confronti tuoi e dei tuoi compagni. Comprende i tuoi linguaggi e obbedisce agli ordini impartitele. Se non le impartisci ordini, la creatura si difende ma non effettua altre azioni. Vedi il Bestiario per le altre statistiche della creatura.

La creatura resta per la durata specificata per ciascuna miniatura. Al termine della durata, la creatura ritorna alla sua forma di miniatura. Si trasforma anticipatamente se scende a 0 punti ferita o se usi due azioni per pronunciare la parola di comando di nuovo mentre la tocchi. Dopo che la creatura è tornata a essere una miniatura, le sue proprietà non possono più essere usate fino a quando non sarà trascorso un certo ammontare di tempo, come specificato nella descrizione della miniatura.

\textit{Cane di Onice} (Raro). Questa statuetta di onice raffigura un cane. Può diventare un mastino per un massimo di 6 ore. Il mastino ha Intelligenza 8 e può parlare Comune. Inoltre ha scurovisione con una gittata di 18 metri e può vedere le creature e gli oggetti invisibili entro quella gittata. Una volta usata, non può essere usata di nuovo prima che siano passati 7 giorni. 

\textit{Caprone d'Avorio (Raro)}. Queste statuette d'avorio di caproni sono sempre create in set da tre. Ogni caprone ha un aspetto unico e funziona in modo diverso dagli altri. Le loro proprietà sono le seguenti:

- Il caprone del terrore può diventare un caprone gigante per un massimo di 3 ore. Il caprone non può attaccare, ma puoi rimuoverne i corni e usarli come armi. Un corno diventa una lancia da cavaliere +1 mentre l'altro diventa una spada lunga +2.

Rimuovere un corno richiede due azioni, e le armi scompaiono e i corni ricompaiono quando il caprone torna alla sua forma di miniatura. Inoltre, il caprone irradia un'aura di terrore con raggio 9 metri finché lo cavalchi. Qualsiasi creatura a te ostile che inizi il proprio round all'interno dell'aura deve superare un Tiro Salvezza su Volontà con DC 17 o restare
spaventata dal caprone per 1 minuto, o finché il caprone non torna alla forma di miniatura. La creatura spaventata può ripetere il Tiro Salvezza al termine di ciascun suo round, terminando l'effetto se lo supera. Una volta che ha riuscito il Tiro Salvezza contro questo effetto, una creatura è immune all'aura del caprone per le successive 24 ore. Una volta usata, la miniatura non può essere usato di nuovo prima che siano passati 15 giorni.

- Il caprone del travaglio può diventare un caprone gigante per un massimo di 3 ore. Una volta usato, non può essere usato di nuovo prima che siano passati 30 giorni.
- Il caprone del viaggio può diventare un caprone Grande con le stesse statistiche di un cavallo da corsa. Ha 24 cariche, e ciascuna ora o porzione di essa che trascorre in forma di bestia costa 1 carica. Finché ha cariche, lo puoi usare quanto ti pare. Una volta terminate le cariche, ritorna a essere una miniatura e non può essere usato di nuovo prima che siano passati 7 giorni, allorché avrà recuperato tutte le sue cariche.

\textit{Corvo d'Argento} (Non Comune). Questa statuetta d'argento raffigura un corvo. Può diventare un corvo per un massimo di 6 ore. Una volta usata, non può essere usata di nuovo prima che siano passati 2 giorni. Mentre è in forma di corvo, la miniatura ti permette di lanciare a volontà l'incantesimo messaggero animale su di essa.

\textit{Destriero di Ossidiana} (Molto Raro). Questa statuetta di ossidiana liscia diventa un incubo per un massimo di 24 ore. L'incubo combatte solo per difendersi. Una volta usata, non può essere usata di nuovo prima che siano passati 5 giorni.

\textit{Elefante di Marmo} (Raro). Questa statuetta di marmo è larga e alta circa 10 centimetri. Può diventare un elefante per un massimo di 24 ore. Una volta usata, non può essere usata di nuovo prima che siano passati 7 giorni.

\textit{Grifone di Bronzo} (Raro). Questa statuetta di bronzo raffigura un grifone rampante. Può diventare un grifone per un massimo di 6 ore. Una volta usata, non può essere usata di nuovo prima che siano passati 5 giorni. 

\textit{Gufo Serpentino} (Raro). Questa statuetta serpentina di un gufo può diventare un gufo gigante per un massimo di 8 ore. Una volta usata, non può essere usata di nuovo prima che siano passati 2 giorni. Se vi trovate sullo stesso piano di esistenza, il gufo può comunicare telepaticamente con te a qualsiasi distanza.

\textit{Leoni d'Oro} (Raro). Queste statuette d'oro di leoni sono sempre create a coppie. Puoi usare una o entrambe le miniature contemporaneamente. Ciascuna può diventare un leone per un massimo di 1 ora. Una volta usato uno dei leoni, questi non può essere usato di nuovo prima che siano passati 7 giorni. 

\index{Munizione dell'Uccisione}\textbf{Munizione dell'Uccisione}

\textit{Arma (freccia, saetta), molto raro} - 700 mo

Se una creatura appartenente al tipo, razza o gruppo a cui la freccia dell'uccisione è associata subisce danni dalla freccia, la creatura deve effettuare un tiro Salvezza su Tempra con DC 21, subendo 6d10 danni perforanti aggiuntivi se lo fallisce, o la metà di questi danni se lo riesce.

Una volta che la freccia dell'uccisione ha inflitto danni aggiuntivi alla creatura, diventa una freccia non magica. 

\index{Occhi Affascinanti}\textbf{Occhi Affascinanti}

\textit{Oggetto meraviglioso, non comune} - 3000

Mentre indossi queste lenti di cristallo davanti agli occhi, puoi spendere 1 carica con due azioni per lanciare l'incantesimo charme su persone (DC del Tiro Salvezza 15) su di un umanoide entro 9 metri da te, purché tu e il bersaglio vi possiate vedere. Le lenti hanno 3 cariche e recuperano tutte quelle spese ogni giorno all'alba.

\index{Occhi dell'Aquila}\textbf{Occhi dell'Aquila}

\textit{Oggetto meraviglioso, non comune} - 2500 mo

Mentre indossi queste lenti di cristallo davanti agli occhi, hai +1d6 alle prove di Consapevolezza basate sulla vista. In condizioni di visibilità limpida, puoi distinguere i dettagli anche di creature e oggetti molto distanti delle dimensioni di 50 centimetri.

\index{Occhi della Vista Dettagliata}\textbf{Occhi della Vista Dettagliata}

\textit{Oggetto meraviglioso, non comune} - 2500 mo

Mentre indossi queste lenti di cristallo davanti agli occhi, puoi vedere molto meglio del normale fino a una distanza di 30 centimetri. Hai +1d6 alle prove di Consapevolezza basata su vista mentre perlustri un'area o studi un oggetto a distanza.

\index{Occhiali da Notte}\textbf{Occhiali da Notte}

\textit{Oggetto meraviglioso, non comune} - 1500 mo

Mentre indossi queste lenti scure, possiedi la scurovisione, con una gittata di 18 metri. Se già possiedi la scurovisione, indossare questi occhiali ne aumenta la gittata di 18 metri.

\index{Olio di Affilatezza}\textbf{Olio di Affilatezza}

\textit{Pozione, molto raro} - 3200 mo

Quest'olio può ricoprire un'arma tagliente o perforante o fino a 5 munizioni taglienti o perforanti. Applicare l'olio richiede 1 minuto. Per 1 ora, l'arma ricoperta dall'olio è magica e ha un bonus di +3 ai Tiri per Colpire e danno. 

\index{Olio di Forma Eterea}\textbf{Olio di Forma Eterea}

\textit{Pozione, raro} - 2000 mo

Una dose di olio è sufficiente a ricoprire una creatura di taglia Media o inferiore, e l'equipaggiamento che indossa e trasporta (è necessaria un'ulteriore fiala per ogni categoria di taglia sopra la Media). Applicare l'olio richiede 10 minuti. Dopodiché la creatura ottiene l'effetto dell'incantesimo forma eterea per 1 ora.

\index{Olio di Scivolosità}\textbf{Olio di Scivolosità}

\textit{Pozione, non comune} - 500 mo

L'olio può coprire una creatura di taglia Media o inferiore, insieme a tutto l'equipaggiamento che indossa o trasporta (è necessaria un'ulteriore fiala per ogni categoria di taglia sopra la Media). Applicare l'olio richiede 10 minuti. La creatura ottiene poi il beneficio dell'incantesimo libertà di movimento per 8 ore. In alternativa, con due azioni si può versare l'olio sul terreno, duplicando per 8 ore l'effetto dell'incantesimo unto su quell'area.

\index{Palla di Cristallo}\textbf{Palla di Cristallo}

\textit{Oggetto meraviglioso, molto raro o leggendario} - 50000 mo

Una tipica palla di cristallo ha il diametro di circa 15 centimetri. Mentre la tocchi, puoi lanciare tramite essa l'incantesimo scrutare (DC del Tiro Salvezza 21). Le seguenti palle di cristallo varianti sono oggetti leggendari e hanno proprietà aggiuntive.

\textit{Palla di Cristallo di Lettura del Pensiero}. Questa palla di cristallo è di circa 12 centimetri di diametro. Mentre la tocchi, puoi lanciare tramite di essa l'incantesimo scrutare (DC del Tiro Salvezza 21). Puoi usare due azioni per lanciare l'incantesimo individuazione dei pensieri (DC del Tiro Salvezza 21) mentre stai scrutando tramite questa palla di cristallo, prendendo come bersaglio le creature che puoi vedere e si trovano entro 9 metri dal sensore dell'incantesimo. Non devi concentrarti su questo individuazione dei pensieri per mantenerlo per la sua durata, che termina quando termina scrutare.

\textit{Palla di Cristallo di Telepatia}. Questa palla di cristallo è di circa 12 centimetri di diametro. Mentre la tocchi, puoi lanciare tramite di essa l'incantesimo scrutare (DC del Tiro Salvezza 21). Mentre scruti tramite questa palla di cristallo, puoi comunicare telepaticamente con le creature che puoi vedere e si trovano entro 9 metri dal sensore dell'incantesimo. Puoi anche usare due azioni per lanciare l'incantesimo suggestione (DC del Tiro Salvezza 21) su una di queste creature tramite il sensore. Non devi concentrarti su questa suggestione per mantenerla per la sua durata, che termina se termina scrutare. Una volta usato, il potere suggestione della palla di cristallo non può essere usato di nuovo fino alla prossima alba.

\textit{Palla di Cristallo di Visione del Vero}. Questa palla di cristallo è di circa 12 centimetri di diametro. Mentre la tocchi, puoi lanciare tramite di essa l'incantesimo scrutare (DC del Tiro Salvezza 21). Mentre scruti con questa palla di cristallo, hai visione del vero con un raggio di 36 metri centrato sul sensore dell'incantesimo.

\index{Pantofole del Ragno}\textbf{Pantofole del Ragno}

\textit{Oggetto meraviglioso, non comune} - 5000 mo

Mentre indossi queste scarpe leggere, puoi muoverti verso l'alto, il basso, e lungo superfici verticali e a testa in giù sul soffitto, lasciando libere le mani. Hai una velocità di scalata pari alla velocità di passeggio. Tuttavia, le pantofole non ti permettono di muoverti a questo modo su superfici scivolose, come quelle coperte da ghiaccio o da olio.

\index{Pergamena degli Incantesimi}\textbf{Pergamena degli Incantesimi}

\textit{Pergamena, varia} - vedi costi creazione pergamena

Una pergamena degli incantesimi riporta le parole di un singolo incantesimo, scritte in un codice mistico.

Per leggere una pergamena è necessario:

\textbf{in caso di pergamene ISY SCROL}:

- per comprendere il contenuto è sufficiente una prova di Arcana a difficoltà DC 10

- per poter leggere e lanciare l'incantesimo della pergamena è necessaria una prova di Intelligenza (o Arcana se conosciuta) a difficoltà 12.


\textbf{in caso di pergamene normali}:

- per comprenderne il contenuto è necessaria una prova di Arcana a Difficoltà 15

- per poter leggere e lanciare l'incantesimo della pergamena è necessaria una prova di competenza magica a Difficoltà 20.


Lanciare l'incantesimo leggendolo da una pergamena richiede il normale tempo di lancio dell'incantesimo. Una volta che l'incantesimo è stato lanciato, le parole sulla pergamena svaniscono, e la pergamena viene ridotta in polvere. Se il lancio viene interrotto, la pergamena non si dissolve.

\index{Perla del Potere}\textbf{Perla del Potere}

\textit{Oggetto meraviglioso, non comune} - 6000 mo

Mentre hai la perla con te, puoi usare due azioni per recuperare un uso di incantesimi al giorno.Una volta usata, la perla non potrà essere usata di nuovo fino alla prossima alba.


\index{Pietra Arcana}\textbf{Pietra Arcana}

\textit{Oggetto meraviglioso, rarità varia}

Esistono diversi tipi di pietra arcana, ogni tipo una specifica combinazione di forme e colori.

Quando usi due azioni per lanciare una di queste pietre in aria, la pietra inizia a orbitare intorno alla tua testa alla distanza di 1d3 x 30 centimetri e ti conferisce un beneficio.
 Dopodiché, un'altra creatura dovrà usare due azioni per afferrare o imbrigliare la pietra e separarla da te, riuscendo in un Tiro per Colpire contro Difesa 24 o superando una prova di Destrezza con DC 31. Puoi usare due azioni per afferrare e mettere da parte la pietra, terminandone l'effetto.
 
Una pietra ha CA 24, 10 punti ferita e resistenza a tutti i danni. Mentre orbita intorno alla tua testa è considerata un oggetto indossato.

\textit{Destrezza} (molto raro). Mentre orbita intorno alla tua testa il tuo punteggio di Destrezza aumenta di 1, fino a un massimo di 5. 3000 mo

\textit{Assorbimento} (molto raro). Mentre orbita intorno alla tua testa, puoi usare una tua Azione per cancellare un incantesimo di Difficoltà 23 o inferiore lanciato da una creatura visibile e che prende a bersaglio solo te. Una volta che la pietra ha cancellato 5 Incantesimi, si esaurisce e diventa grigia opaca, perdendo la sua magia. 6000 ,p

\textit{Autorità} (molto raro). Mentre orbita intorno alla tua testa il tuo punteggio di Carisma aumenta di 1, fino a un massimo di 5. 3000 mo 

\textit{Consapevolezza} (raro). Mentre orbita intorno alla tua testa non puoi essere sorpreso. 12000 mo

\textit{Forza} (molto raro). Mentre orbita intorno alla tua testa il tuo punteggio di Forza aumenta di 1, fino a un massimo di 5. 3000 mo

\textit{Intelligenza} (molto raro). Mentre orbita intorno alla tua testa il tuo punteggio di Intelligenza aumenta di 1, fino a un massimo di 5. 3000 mo

\textit{Intuizione} (molto raro). Mentre orbita intorno alla tua testa il tuo punteggio di Saggezza aumenta di 2, fino a un massimo di 5. 3000 mo

\textit{Protezione} (raro). Mentre orbita intorno alla tua testa ottieni un bonus di +1 alla Difesa. 10000 mo

\textit{Sostentamento} (raro). Mentre orbita intorno alla tua testa non hai bisogno di mangiare né di bere. 3500 mo

\index{Pietra della Buona Sorte}\textbf{Pietra della Buona Sorte}

\textit{Oggetto meraviglioso, non comune} - 4500 mo

Finché la pietra è con te, ottieni un bonus di +1 alle prove di caratteristica e ai Tiri Salvezza. 

\index{Pietra del Controllo degli Elementali della Terra}\textbf{Pietra del Controllo degli Elementali della Terra} - 8000 mo

\textit{Oggetto meraviglioso, raro}

Se la pietra tocca terra, puoi usare due azioni per pronunciare la parola di comando ed evocare un elementale della terra, come se avessi lanciato l'incantesimo evocare elementali. La pietra non può di nuovo essere usata a questo modo, fino alla prossima alba. La pietra pesa 2,5 chili.

\index{Piffero delle Fogne}\textbf{Piffero delle Fogne}

\textit{Oggetto meraviglioso, non comune} - 2000 mo

Devi essere competente con gli strumenti a fiato per usare questo piffero. Mentre usi questo piffero, i ratti normali e i ratti giganti sono indifferenti nei tuoi confronti e non ti attaccheranno a meno che non li minacci o li danneggi. Se con due azioni suoni il piffero, puoi usare due azioni per spendere da 1 a 3 cariche, richiamando uno sciame di ratti per ogni carica spesa, purché ci siano abbastanza ratti entro 750 metri da te da richiamare in questa maniera (a discrezione del Narratore). Se non ci sono abbastanza ratti da formare uno sciame, la carica è sprecata. Gli sciami richiamati si muovono verso la musica tramite la rotta più breve possibile, ma non sono in alcun altro modo sotto il tuo controllo. Il piffero ha 3 cariche e recupera 1d3 cariche spese ogni giorno all'alba.

Ogni qualvolta uno sciame di ratti che non sia sotto il controllo di un'altra creatura si avvicina entro 9 metri da te mentre stai suonando il piffero, puoi effettuare una prova di Carisma contesa dalla prova di Saggezza dello sciame. Se perdi la contesa, lo sciame si comporta come di norma e non può essere di nuovo distratto dalla musica del piffero per le successive 24 ore. Se vinci la contesa, lo sciame è attratto dalla musica del piffero e diventa amichevole nei confronti tuoi e dei tuoi compagni fino a che continui a suonare il piffero con due azioni ogni round. Uno sciame amichevole obbedisce ai tuoi comandi. Se non impartisci ordini a uno sciame amichevole, questo si difenderà ma non compirà altre azioni.

Se uno sciame amichevole all'inizio del round non può udire la musica del piffero, il tuo controllo su quello sciame termina, e lo sciame si comporta come farebbe normalmente e non può essere attirato nuovamente dalla musica del piffero per le successive 24 ore.

\index{Piffero dello Spavento}\textbf{Piffero dello Spavento}

\textit{Oggetto meraviglioso, non comune} - 6000 mo

Devi essere competente con gli strumenti a fiato per usare questo piffero. Puoi usare due azioni per suonarlo e spendere 1 carica per creare un suono incantevole e spettrale. Ogni creatura entro 9 metri da te e che ti oda suonare deve superare un Tiro Salvezza su Volontà con DC 17 o restare spaventata da te per 1 minuto. Se lo desideri, tutte le creature nell'area che non ti siano ostili possono superare automaticamente il loro Tiro Salvezza. Una creatura che fallisca il Tiro Salvezza può ripeterlo alla fine del suo round, terminando l'effetto su di sé in caso lo superi. Una creatura che superi il Tiro Salvezza è immune all'effetto di questo piffero per 24 ore. Il piffero ha 3 cariche e recupera 1d3 cariche spese ogni giorno all'alba.

\index{Pigmenti delle Meraviglie}\textbf{Pigmenti delle Meraviglie}

\textit{Oggetto meraviglioso, molto raro} - 400 mo

Trovati solitamente in 1d4 vasetti all'interno di eleganti scatole di legno assieme a un pennello (del peso totale di 500 grammi), questi pigmenti ti permettono di creare oggetti tridimensionali, dipingendoli a due dimensioni. La pittura fluisce dal pennello per formare l'oggetto desiderato mentre ti concentri sull'immagine

Ogni vasetto di pittura è sufficiente a coprire 90 m2 di una superficie, permettendoti di creare oggetti inanimati e caratteristiche del terreno (porte, fosse, fiori, alberi, celle, stanze o armi) che occupino un totale di 270 m3. Ci vogliono 10 minuti per coprire 90 m2.

Quando completi il dipinto, l'oggetto o la caratteristica del terreno dipinta diventa un oggetto reale, non magico. Quindi, dipingere una porta su di una parete crea una vera porta che può essere aperta per accedere a ciò che si trova oltre di essa. Dipingere una fossa sul pavimento crea una vera fossa, la cui profondità è conteggiata nell'area totale degli oggetti che puoi creare.

Nulla di ciò che viene creato dai pigmenti può avere un valore superiore ai 25 mo. Se dipingi un oggetto di valore superiore (un diamante o una pila d'oro), l'oggetto sembrerà autentico, ma un attento esame rivelerà che è fatto di gesso, ossa o qualche altro materiale privo di valore.

Se dipingi una forma di energia, come fuoco o fulmine, l'energia compare ma si dissipa non appena completi il dipinto, senza recare danni a niente. 

\index{Piuma Arcana}\textbf{Piuma Arcana}

\textit{Oggetto meraviglioso, raro}

Questo minuscolo oggetto assomiglia a una piuma. Esistono diversi tipi di piume arcane, ciascuno dotato di un singolo effetto monouso. Il Narratore sceglie il tipo di piuma arcana.

\textit{Albero}. Devi trovarti all'aperto per poter usare questa piuma arcana. Puoi usare due azioni per appoggiarla a uno spazio non occupato sul terreno. La piuma svanisce e al suo posto spunta un albero di quercia non magico. L'albero è alto 18 metri e ha un tronco di 1,5 metri di diametro. In cima, i suoi rami si estendono per un massimo di 6 metri. 50 mo

\textit{Ancora}. Puoi usare due azioni per appoggiare la piuma arcana a una barca o nave. Per le successive 24 ore, il vascello non potrà essere mosso in alcun modo. Toccare di nuovo il vascello con la piuma arcana termina questo effetto. Quando l'effetto termina, la piuma svanisce. 50 mo

\textit{Frusta}. Puoi usare due azioni per lanciare la piuma arcana verso un punto entro 3 metri da te. La piuma svanisce e al suo posto compare una frusta fluttuante. Puoi poi usare due azioni per effettuare un attacco con incantesimo in mischia contro una creatura entro 3 metri dalla frusta, con un bonus di attacco +9. Se colpisci, il bersaglio subisce 1d6 + 5 danni da forza. Durante il tuo round, con due azioni puoi dirigere la frusta affinché voli per un massimo di 6 metri e ripeta l'attacco contro una creatura entro 3 metri da essa. La frusta svanisce dopo 1 ora, quando usi due azioni per congedarla, o quando sei inabile o muori. 250 mo

\textit{Nave Cigno}. Puoi usare due azioni per appoggiare la piuma arcana su di una massa d'acqua di almeno 18 metri di diametro. La piuma svanisce e al suo posto compare una barca lunga 15 metri e larga 6 metri dalla forma di cigno. La barca si sposta da sola e si muove in acqua alla velocità di 9 chilometri all'ora. Puoi usare due azioni, mentre sei a bordo per comandarle di muoversi o voltare di 90 gradi. La barca può trasportare fino a trentadue creature di taglia Media o inferiore. Una creatura Grande conta come quattro creature Medie, mentre una creatura Enorme conta come nove creature Medie. La barca svanisce dopo 24 ore. Puoi congedare la barca con due azioni. 3000 mo

\textit{Uccello}. Puoi usare due azioni per lanciare la piuma arcana 1,5 metri nell'aria. La piuma svanisce e un enorme uccello multicolore ne prende il posto. L'uccello ha le statistiche di un Roc, ma obbedisce a comandi semplici e non può attaccare. Può trasportare fino a 250 chili mentre vola alla sua velocità massima (24 chilometri all'ora per un massimo di 216 chilometri al giorno, con un'ora di riposo ogni 3 ore di volo), o 500 chili di peso a metà velocità. L'uccello svanisce dopo aver volato per la distanza massima possibile in un giorno o se scende a 0 punti ferita. Puoi congedare l'uccello con due azioni. 3000 mo

\textit{Ventaglio}. Se ti trovi su di una barca o una nave, puoi usare due azioni per lanciare la piuma arcana fino a 3 metri in aria. La piuma svanisce e un gigantesco ventaglio compare al suo posto. Il ventaglio galleggia e crea un vento forte abbastanza da gonfiare le vele della nave, aumentandone la velocità di 7,5 chilometri all'ora per 8 ore. Puoi congedare il ventaglio con due azioni. 250 mo

\index{Polvere dell'Aridità}\textbf{Polvere dell'Aridità}

\textit{Oggetto meraviglioso, raro} - 120 mo

Questa piccola confezione contiene 1d6 + 4 pizzichi di polvere. Puoi usare due azioni per spargere un pizzico di polvere sull'acqua La polvere trasforma un cubo d'acqua di 4,5 metri di spigolo in una pallina delle dimensioni di una biglia, che fluttua o si deposita nel punto in cui è stata gettata la polvere. Il peso della pallina è trascurabile.

Chiunque può usare due azioni per spaccare la pallina contro una superficie dura, facendo sì che la pallina si rompa e liberi l'acqua assorbita dalla polvere. Farlo esaurisce la magia della pallina.

Un elementale composto principalmente d'acqua e che venga esposto a un pizzico di questa polvere, deve effettuare un tiro Salvezza su Tempra con DC 15, subendo 10d6 danni da Vuoto se lo fallisce, o la metà di questi danni se lo riesce.

\index{Polvere della Sparizione}\textbf{Polvere della Sparizione}

\textit{Oggetto meraviglioso, non comune} - 300 mo

Rinvenuta in piccoli sacchetti, questa polverina sembra sabbia molto sottile. In un sacchetto ce n'è a sufficienza per un uso. Quando usi due azioni per lanciare la polvere in aria, tu e ciascuna creatura e oggetto entro 3 metri da te diventate invisibili per 2d4 minuti. La durata è la stessa per tutti i soggetti, e quando la magia prende effetto la polvere si consuma. Se una creatura sotto l'effetto della polvere attacca o lancia un incantesimo, l'invisibilità ha fine solo per quella creatura.

\index{Polvere dello Starnuto e del Soffocamento}\textbf{Polvere dello Starnuto e del Soffocamento}

\textit{Oggetto meraviglioso, non comune} - 480 mo

Trovata in piccoli contenitori, questa polverina sembra sabbia sottile. Appare simile alla polvere della sparizione, e l'incantesimo identificare la rivela come tale. Ce n'è a sufficienza per un uso. Quando usi due azioni per lanciare una manciata di polvere in aria, tu e tutte le creature che necessitano di respirare e si trovino entro 9 metri da te dovete superare un tiro Salvezza su Tempra con DC 17 o smettere di respirare, e iniziare a starnutire in maniera incontrollabile. Una creatura afflitta a questo modo è inabile e soffoca. Finché è cosciente, la creatura può ripetere il Tiro Salvezza alla fine di ciascun suo round, terminando l'effetto in caso lo superi. Anche l'incantesimo ristorare inferiore può terminare l'effetto che affligge la creatura.

\index{Portale Cubico}\textbf{Portale Cubico}

\textit{Oggetto meraviglioso, leggendario} - 40000 mo

Questo cubo di 7,5 centimetri di spigolo irradia una palpabile energia magica. Le sei facce del cubo sono ciascuna collegata a un diverso piano di esistenza, uno dei quali è il Piano Materiale. Le altre facce sono collegate a piani determinati dal Narratore.

Puoi usare due azioni per premere una faccia del cubo per lanciare tramite esso l'incantesimo portale, aprendo un passaggio verso il piano collegato a quella faccia. In alternativa, se usi due azioni per premere una faccia due volte, puoi lanciare l'incantesimo spostamento planare (DC del Tiro Salvezza 17) tramite il cubo e trasportarne i bersagli al piano collegato a quella faccia. Il cubo ha 3 cariche. Ogni uso del cubo spende 1 carica. Il cubo recupera 1d3 cariche spese ogni giorno all'alba.

\index{Pozione di Amicizia con gli Animali}\textbf{Pozione di Amicizia con gli Animali} 

\textit{Pozione, non comune} - 200 mo

Quando bevi questa pozione, per 1 ora puoi lanciare a volontà l'incantesimo amicizia con gli animali (DC del Tiro Salvezza 15).

\index{Pozione di Arrampicata}\textbf{Pozione di Arrampicata}

\textit{Pozione, comune} - 250 mo

Quando bevi questa pozione, per 1 ora ottieni velocità di scalata pari alla tua velocità di passeggio. Durante questo periodo hai +1d6 alle prove di Resistenza che compi per effettuare una scalata.

\index{Pozione di Chiaroveggenza}\textbf{Pozione di Chiaroveggenza}

\textit{Pozione, raro} - 1000 mo

Quando bevi questa pozione, ottieni l'effetto dell'incantesimo chiaroveggenza.

\index{Pozione di Crescita}\textbf{Pozione di Crescita}

\textit{Pozione, non comune} - 300 mo

Quando bevi questa pozione, per 1d4 ore ottieni l'effetto "ingrandire" dell'incantesimo ingrandire/ridurre (non richiede concentrazione).

\index{Pozione di Eroismo}\textbf{Pozione di Eroismo}

\textit{Pozione, raro} - 200

Quando bevi questa pozione, ottieni 10 punti ferita temporanei che durano 1 ora. Per la stessa durata sei sotto l'effetto dell'incantesimo benedizione (non richiede concentrazione).

\index{Pozione di Forma Gassosa}\textbf{Pozione di Forma Gassosa}

\textit{Pozione, raro}

Quando bevi questa pozione, per 1 ora o finché non terminerai l'effetto con due azioni ottieni l'effetto dell'incantesimo forma gassosa (non richiede concentrazione).

\index{Pozione di Forza dei Giganti}\textbf{Pozione di Forza dei Giganti}

\textit{Pozione, rarità varia}

Quando bevi questa pozione, per 1 ora il tuo punteggio di Forza cambia. Il tipo di gigante determina il punteggio (vedi la tabella seguente). La pozione non ha effetto se il tuo punteggio di Forza è pari o superiore al nuovo punteggio.

La pozione della forza del gigante del gelo e la pozione della forza del gigante di pietra hanno lo stesso effetto.

\item - Gigante di colline, Forza 5, Non comune - 500 mo

\item - Gigante di pietra o del gelo, Forza 6, Raro - 1000 mo

\item - Gigante del fuoco, Forza 7, Raro - 2000 mo

\item - Gigante delle nuvole, Forza 8, Molto raro - 5000 mo

\item - Gigante delle tempeste, Forza 9, Leggendario - 10000 mo

\index{Pozione di Guarigione}\textbf{Pozione di Guarigione}

\textit{Pozione, rarità varia}

Quando bevi da questa pozione, recuperi un numero di punti ferita che varia a seconda della rarità della pozione, come mostrato sulla tabella Pozioni di Guarigione.

\item - Guarigione Comune, Punti Ferita 2d4 + 2, 75 mo

\item - Guarigione maggiore, Punti Ferita 4d4 + 4, 220 mo

\item - Guarigione superiore, Punti Ferita 8d4 + 8, 2200 mo

\item - Guarigione suprema, Punti Ferita 10d4 + 20, 22000 mo

	
\index{Pozione di Invisibilità}\textbf{Pozione di Invisibilità}

\textit{Pozione, molto raro} - 200 mo

Quando bevi questa pozione, per 1 ora diventi invisibile. Mentre sei invisibile, tutto ciò che trasporti o indossi resta anch'esso invisibile assieme a te. L'effetto ha termine qualora tu attacchi o lanci un incantesimo. 

\index{Pozione di Lettura del Pensiero}\textbf{Pozione di Lettura del Pensiero}

\textit{Pozione, raro} - 200 mo

Quando bevi questa pozione, ottieni l'effetto dell'incantesimo individuazione dei pensieri (DC del Tiro Salvezza 15). 

\index{Pozione di Resistenza}\textbf{Pozione di Resistenza}

\textit{Pozione, non comune} - 300 mo

Quando bevi questa pozione, per 1 ora ottieni resistenza a un tipo di danno. Il Narratore sceglie il tipo di danno o lo determina casualmente (Acido, Freddo, Fuoco, Forza, Fulmine, Vuoto, Veleno, Psichico, Luce, Tuono)

\index{Pozione di Rimpicciolimento}\textbf{Pozione di Rimpicciolimento}

\textit{Pozione, raro} - 300 mo

Quando bevi questa pozione, per 1d4 ore ottieni l'effetto "ridurre" dell'incantesimo ingrandire/ridurre (non richiede concentrazione).

\index{Pozione di Respirare Sott'Acqua}\textbf{Pozione di Respirare Sott'Acqua}

\textit{Pozione, non comune} - 200 mo

Dopo aver bevuto questa pozione, puoi respirare sott'acqua per 1 ora.

\index{Pozione di Veleno}\textbf{Pozione di Veleno}

\textit{Pozione, non comune} - 100 mo

Questo distillato assomiglia, odora e ha il sapore di una pozione di guarigione o di un'altra pozione benefica. Tuttavia è in realtà un veleno mascherato da magie di illusione. L'incantesimo identificare ne rivela la vera natura.

Se lo bevi, subisci 3d6 danni da veleno, e devi superare un tiro Salvezza su Tempra con DC 15 o restare avvelenato. All'inizio di ciascun tuo round, finché resti avvelenato a questo modo, subisci 3d6 danni da veleno. Puoi ripetere il Tiro Salvezza al termine di ciascun tuo round. Se il Tiro Salvezza riesce, il danno da veleno subito nei turni successivi diminuisce di 1d6. Il veleno cessa i suoi effetti quando il danno scende a 0d6.

\index{Pozione di Velocità}\textbf{Pozione di Velocità}

\textit{Pozione, molto raro} - 400 mo

Quando bevi questa pozione, ottieni l'effetto dell'incantesimo velocità per 1 minuto (non richiede concentrazione).

\index{Pozione di Volo}\textbf{Pozione di Volo}

\textit{Pozione, molto raro} - 500 mo

Quando bevi questa pozione, per 1 ora ottieni velocità di volo pari alla tua normale velocità di passeggio e puoi fluttuare. Se la pozione ha termine mentre stai volando, cadi a meno che non possiedi qualche altro metodo per restare in aria.

\index{Pozzo dei Molti Mondi}\textbf{Pozzo dei Molti Mondi}

\textit{Oggetto meraviglioso, leggendario}

Questo elegante tessuto nero, soffice come la seta, è avvolto fino alle dimensioni di un fazzoletto. Si dispiega in un foglio circolare di 1,8 metri di diametro. Puoi usare due azioni per dispiegare e piazzare il pozzo dei molti mondi su di una superficie solida, su cui crea un portale bidirezionale verso un altro mondo o piano di esistenza. Ogni volta che l'oggetto apre un portale, il Narratore decide il posto a cui conduce. Puoi usare due azioni per chiudere un portale aperto afferrando i margini del tessuto e ripiegandoli. Una volta che un pozzo dei molti mondi ha aperto un portale, non potrà farlo di nuovo prima che siano passate 1d8 ore.

\index{Pugnale del Veleno}\textbf{Pugnale del Veleno}

\textit{Arma (pugnale), raro} - 3000 mo

Hai un bonus di +1 ai Tiri per Colpire e tiri di danno per gli attacchi effettuati con quest'arma magica. 

Una volta al giorno, puoi usare due azioni per far sì che un denso veleno nero ricopra la lama. Il veleno resta per 1 minuto o finché non colpisci con un attacco usando quest'arma. Quando colpisci una creatura con il pugnale avvelenato, il bersaglio deve effettuare un tiro Salvezza su Tempra con DC 17.

Se fallisce il Tiro Salvezza, il bersaglio diventa avvelenato per 1 minuto e subisce 2d10 danni da veleno. Il pugnale non può essere usato di nuovo a questo modo fino alla prossima
alba.

\index{Sacro Vendicatore}\textbf{Sacro Vendicatore}

\textit{Arma (qualsiasi spada), leggendario}

Ottieni un bonus di +3 ai Tiri per Colpire e danno effettuati con quest'arma magica. Quando con essa colpisci un immondo o un non morto, quella creatura subisce 2d10 danni da Luce aggiuntivi.

Mentre impugni la spada sguainata, essa crea un'aura di 3 metri di raggio attorno a te. Tu e tutte le creature a te amichevoli all'interno dell'aura ottenete +1d6 ai Tiri Salvezza contro incantesimi e altri effetti magici. Se hai CA 13 o più, il raggio dell'aura aumenta a 9 metri.

\index{Scarabeo di Protezione}\textbf{Scarabeo di Protezione}

\textit{Anello, leggendario} - 36000 mo

Se tieni questo medaglione a forma di scarabeo tra le tue mani per 1 round, su di esso compare un'iscrizione che ne rivela la natura magica. Mentre è addosso a te, fornisce due benefici:

- Hai +1d6 ai Tiri Salvezza contro incantesimi.

- Lo scarabeo ha 12 cariche. Se fallisci un Tiro Salvezza contro un incantesimo di necromanzia o un effetto nocivo originante da una creatura non morta, puoi usare la tua reazione per spendere 1 carica e trasformare il Tiro Salvezza fallito in un successo. Lo scarabeo si riduce in polvere ed è distrutto quando viene spesa la sua ultima carica.

\index{Scimitarra di Velocità}\textbf{Scimitarra di Velocità}

\textit{Arma (scimitarra), molto raro} - 9000 mo

Ottieni un bonus di +2 all'iniziativa ed ai Tiri per Colpire, +1 al danno, con quest'arma magica. Inoltre, come azione immediata durante ciascun tuo round puoi effettuare un attacco aggiuntivo.

\index{Scopa Volante}\textbf{Scopa Volante}

\textit{Oggetto meraviglioso, non comune} - 8000 mo

Questa scopa di legno, del peso di circa 1,5 chili, funziona come una normale scopa fino a quando non vi siedi sopra e ne pronunci la parola di comando. Essa inizia così a fluttuare sotto di te e può essere cavalcata in aria. Ha velocità di volo 15 metri. Può trasportare fino a 200 chili, ma la sua velocità di volo diventa 9 metri se dovesse trasportare più di 100 chili. Quando atterri, la scopa smette di fluttuare.

Pronunciando la parola di comando, nominando il posto e se vi sei familiare, puoi inviare la scopa da sola in un posto fino a 1,5 chilometri da te. La scopa tornerà da te quando pronuncerai un'altra parola di comando, purché si trovi ancora entro 1,5 chilometri da te.

\index{Scudo Magico}\textbf{Scudo} +1, +2, +3, +4, +5

\textit{Scudi (piccoli, medi, pesanti)}: +1 1500 mo, +2 4000 mo, +3 9000 mo, +4 20000 mo, +5 35000 mo

Mentre impugni questo scudo, hai un bonus alla Difesa determinato dal bonus magico dello scudo. Questo bonus è in aggiunta al normale bonus alla Difesa fornito dallo scudo. 
Ogni +1 magico si abbassa di 1 il malus alla prova di CM.

\index{Scudo Afferra Frecce}\textbf{Scudo Afferra Frecce}

\textit{Armatura (scudo), raro}

Mentre impugni questo scudo, hai un bonus di +2 alla Difesa contro gli attacchi a distanza. Questo bonus è in aggiunta al normale bonus dello scudo alla Difesa. Inoltre, ogni volta che una creatura effettua un attacco a distanza contro un bersaglio entro 1,5 metri da te, puoi usare la tua reazione per divenire il bersaglio dell'attacco.

\index{Scudo Animato}\textbf{Scudo Animato}

\textit{Armatura (scudo), molto raro} - 6000 mo

Mentre impugni questo scudo, con due azioni puoi pronunciare una parola di comando e farlo animare. Lo scudo fluttuerà nell'aria all'interno del tuo spazio per proteggerti come se lo stessi impugnando, lasciandoti libera la mano.

Lo scudo resta animato per 1 minuto, finché non usi due azioni per terminarne l'effetto, sei inabile o muori: a quel punto lo scudo cadrà a terra o tornerà nella tua mano se ne hai una libera. 

\index{Scudo dell'Attrazione dei Proiettili}\textbf{Scudo dell'Attrazione dei Proiettili}

\textit{Armatura (scudo), raro} - 2000 mo

Mentre impugni questo scudo apparentemente hai resistenza ai danni da parte degli attacchi con arma a distanza. 

\textit{Maledizione} Questo scudo è maledetto. 

Togliersi lo scudo non pone fine alla maledizione. Ogni qualvolta un attacco con arma a distanza viene effettuato contro un bersaglio entro 3 metri da te, la maledizione fa sì che diventi tu il bersaglio dell'attacco. 

\index{Scudo di Difesa dagli Incantesimi}\textbf{Scudo di Difesa dagli Incantesimi}

\textit{Armatura (scudo), molto raro} - 50000 mo

Mentre impugni questo scudo, hai +1d6 ai Tiri Salvezza contro incantesimi e altri effetti magici, e gli attacchi con incantesimo subiscono -1d6 quando effettuati contro di te.

\index{Sfera dell'Annientamento}\textbf{Sfera dell'Annientamento}

\textit{Oggetto meraviglioso, leggendario}

Questa sfera nera di 50 centimetri di diametro è in realtà un foro nella struttura del multiverso, che fluttua nello spazio ed è stabilizzata dal campo magico che la circonda.

La sfera annienta tutta la materia che attraversa e tutta la materia che l'attraversa. L'unica eccezione sono gli artefatti. A meno che l'artefatto non sia suscettibile ai danni della sfera dell'annientamento, esso può attraversare la sfera senza problemi. Qualsiasi altra cosa tocchi la sfera e non ne sia completamente avvolta e annientata da essa, subisce 4d10 danni da forza.

La sfera resta immobile fino a quando qualcuno non la controlla. Se ti trovi entro 18 metri da una sfera incontrollata, puoi impiegare due azioni per effettuare una prova di Arcana con DC 30. Se la superi, la sfera levita in una direzione a tua scelta, per un numero di metri pari a 1,5 x il Intelligenza (minimo 1,5 metri). Se fallisci, la sfera si muove di 3 metri verso di te. Una creatura nel cui spazio entri la sfera, deve superare un Tiro Salvezza di Riflessi con DC 15 o venire toccata da essa, subendo 4d10 danni da forza.

Se tenti di controllare una sfera che si trova sotto il controllo di un'altra creatura, effettui una prova contesa di arcana contro arcana dell'altra creatura. Il vincitore della contesa ottiene il controllo della sfera e può farla levitare come di norma. 

Se la sfera entra in contatto con un portale planare, come quello creato dall'incantesimo portale, o uno spazio extradimensionale, come quello all'interno di un buco portatile, il Narratore determina casualmente ciò che accade, utilizzando la tabella seguente. 

\medskip

\begin{tabularx}{0.45\textwidth}{lX}
	\textbf{3d6}& \textbf{Risultato}\\
	\toprule
	3-10 &La sfera è distrutta\\
	12-16& La sfera si muove attraverso il portale o all'interno dello spazio extradimensionale.\\
	17-18 &Un squarcio spaziale spedisce ogni creatura e oggetto entro 54 metri dalla sfera, sfera inclusa, in un piano dell'esistenza casuale.\\
\end{tabularx}

\medskip

\index{Solvente Universale}\textbf{Solvente Universale}

\textit{Oggetto meraviglioso, leggendario} - 300 mo

Questo tubetto contiene un liquido bianco con un forte odore di alcool. Puoi usare due azioni per versarne i contenuti su di una superficie a portata. Il liquido dissolve istantaneamente 1.000 cm2 di adesivo con cui entra in contatto, compresa la colla suprema.

\index{Spada dell'Affilatezza}\textbf{Spada dell'Affilatezza}

\textit{Arma (qualsiasi spada che infligga danni taglienti), molto raro } + 15000 mo

Quando attacchi un oggetto con quest'arma magica e colpisci, massimizza i dadi di danno della tua arma contro il bersaglio.

Quando attacchi una creatura con quest'arma e fai un critico al Tiro per Colpire, il bersaglio subisce 14 danni taglienti aggiuntivi. Se effettui due critici recidi uno degli arti del bersaglio: l'effetto di questa perdita è determinato dal Narratore. Se la creatura non ha arti da recidere, verrà tagliata una parte del suo corpo.

Inoltre, puoi pronunciare la parola di comando della spada per far sì che la lama irradi luce intensa in un raggio di 3 metri e luce fioca per ulteriori 3 metri.

Pronunciando di nuovo la parola di comando o rinfoderando la spada, la luce si spegne.

\index{Spada Danzante}\textbf{Spada Danzante}

\textit{Arma (qualsiasi spada), molto raro} + 3000 mo

Puoi usare due azioni per scagliare questa spada magica nell'aria e pronunciare la parola di comando.

Quando lo fai, la spada inizia a fluttuare, vola fino a 9 metri, e attacca una creatura a tua scelta entro 1,5 metri da essa. La spada usa il tuo Tiro per Colpire e il la tua Forza per il danno.

Mentre la spada fluttua, puoi usare due azioni per farla volare di massimo 9 metri verso un altro punto entro 9 metri da te. Come parte della stessa azione, puoi far sì che la spada attacchi una creatura entro 1,5 metri da essa.

Dopo che la spada fluttuante avrà attaccato per la quarta volta, volerà per un massimo di 9 metri e cercherà di tornare tra le tue mani. Se non hai mani libere, la spada cadrà sul terreno ai tuoi piedi. Se la spada non ha strada libera verso di te, si muoverà più vicino possibile a te e poi cadrà a terra. Cesserà di fluttuare anche nel caso in cui tu la afferri o ti allontani più di 9 metri da essa.

\index{Spada Ruba Vita}\textbf{Spada Ruba Vita}

\textit{Arma (qualsiasi spada), raro} + 5000 mo

Quando attacchi una creatura con quest'arma magica e ottieni un critico al Tiro per Colpire, il bersaglio, a parte i costrutti e i non morti, subisce 10 danni da Vuoto aggiuntivi. Inoltre, tu guadagni 10 punti ferita temporanei.


\index{Spada del Sanguinamento}\textbf{Spada del Sanguinamento}

\textit{Arma (qualsiasi spada), raro} + 15000 mo

I punti ferita persi a causa dei danni di quest'arma, possono essere recuperati solo tramite riposo naturale, anziché tramite la rigenerazione, la magia o altri metodi.

Una volta per round, quando colpisci una creatura con un attacco usando quest'arma magica, puoi far sanguinare il bersaglio. All'inizio di ciascun round della creatura sanguinante, essa subisce 1d4 danni da sanguinamento per ogni volta che l'hai ferita a questo modo, ed essa può effettuare un tiro Salvezza su Tempra con DC 17, terminando l'effetto su tutte le ferite sanguinanti in caso di successo. In alternativa, la creatura sanguinante, o una creatura entro 1,5 metri da essa, può usare due azioni per effettuare una prova di Pronto Soccorso con DC 17, terminando l'effetto del sanguinamento in caso la superi.

\index{Spada Vorpal}\textbf{Spada Vorpal}
\textit{Arma (qualsiasi spada che infligga danni taglienti), leggendario}

Ottieni un bonus di +5 ai Tiri per Colpire e danno effettuati con quest'arma magica. Inoltre, l'arma ignora la resistenza ai danni taglienti. Quando attacchi una creatura che abbia almeno una testa con quest'arma e ottieni un critico al Tiro per Colpire, tagli una delle teste della creatura. La creatura muore se non può sopravvivere senza la perdita della testa.

Una creatura è immune a questo effetto se è immune ai danni taglienti, non possiede o non ha bisogno di una testa o il Narratore decide che la creatura è troppo grossa perché la sua testa sia recisa da quest'arma. 

Una creatura del genere subisce invece 6d8 danni taglienti aggiuntivi dal colpo subito. 

\index{Specchio Intrappola Vita}\textbf{Specchio Intrappola Vita}

\textit{Oggetto meraviglioso, molto raro} - 18000 mo

Quando questo specchio alto 120 centimetri viene guardato in maniera indiretta, la sua superficie mostra una vaga immagine della creatura. Lo specchio pesa 25 chili, ha Difesa 11, 10 punti ferita e vulnerabilità ai danni contundenti. Si frantuma ed è distrutto quando viene ridotto a 0 punti ferita.

Se lo specchio è appeso a una superficie verticale e ti trovi entro 1,5 metri da esso, puoi usare due azioni per pronunciare la sua parola di comando e attivarlo. Rimarrà attivo fino a quando non pronuncerai di nuovo la parola di comando.

Qualsiasi creatura, a parte te, che veda il suo riflesso nello specchio attivato mentre si trova entro 9 metri da esso deve superare un Tiri Salvezza su Volontà con DC 17 o finire intrappolata, insieme a tutto ciò che indossa o trasporta, in una delle dodici celle extradimensionali dello specchio. Questo Tiro Salvezza riceve +1d6 se la creatura conosce la natura dello specchio, e i costrutti riescono automaticamente il Tiro Salvezza.

Una cella extradimensionale è uno spazio infinito colmo di una densa foschia che riduce la visibilità a 3 metri. Le creature intrappolate nelle celle dello specchio non invecchiano, e non hanno bisogno di mangiare, bere o dormire. Una creatura intrappolata all'interno di una cella può fuggirne usando la magia che permette di viaggiare tra i piani. Altrimenti, la creatura è confinata nella cella fino a quando non sarà liberata.

Se lo specchio intrappola una creatura ma le sue dodici celle extradimensionali sono già occupate, lo specchio libera una delle creature intrappolate a caso per alloggiare il nuovo prigioniero. La creatura liberata compare in uno spazio non occupato in vista dello specchio ma rivolta dalla parte opposta. Se lo specchio viene infranto, tutte le creature che contiene sono liberate e ricompaiono in uno spazio non occupato in sua prossimità.

Mentre ti trovi entro 1,5 metri dallo specchio, puoi usare due azioni per pronunciare il nome di una delle creature intrappolate al suo interno o richiamare un particolare numero di cella. La creatura nominata o contenuta nella cella nominata appare come immagine sulla superficie dello specchio. Dopodiché tu e la creatura nominata potete comunicare normalmente.

In un modo simile, puoi usare due azioni per pronunciare una seconda parola di comando e liberare una delle creature intrappolate nello specchio. La creatura liberata compare, insieme a tutte le sue proprietà, nello spazio non occupato più vicino allo specchio e rivolta nella direzione opposta a esso. 

\index{Spilla della Difesa}\textbf{Spilla della Difesa}

\textit{Oggetto meraviglioso, non comune} - 7500 mo

Mentre indossi questa spilla, hai resistenza ai danni da forza, e hai immunità al danno generato dall'incantesimo dardo incantato.

\index{Stivali Alati}\textbf{Stivali Alati}

\textit{Oggetto meraviglioso, non comune}

Mentre indossi questi stivali, hai una velocità di volo pari alla tua velocità di passeggio. Puoi usare questi stivali per volare per un massimo di 4 ore, tutte insieme o divise in brevi voli, ciascuno dei quali impiega un minimo di 1 minuto di durata. Se la durata termina mentre stai volando, scendi alla velocità di 9 metri per round finché non atterri.

Gli stivali recuperano 2 ore di capacità di volo ogni 12 ore che non sono in uso. 

\index{Stivali della Corsa e del Salto}\textbf{Stivali della Corsa e del Salto}

\textit{Oggetto meraviglioso, non comune} - 5000 mo

Mentre indossi questi stivali, la tua velocità di passeggio diventa 9 metri, a meno che non sia superiore, e la tua velocità non viene ridotta qualora tu sia ingombrato o stia indossando un'armatura pesante. 

Inoltre, salti tre volte la normale distanza, fino ad un massimo di 9 metri.

\index{Stivali degli Elfi}\textbf{Stivali degli Elfi}

\textit{Oggetto meraviglioso, non comune} - 3000 mo

Mentre indossi questi stivali, i tuoi passi non emettono suoni, quale che sia la superficie che stai attraversando. Hai +1d6 alle prove di Destrezza che si basano sul muoversi silenziosamente.

\index{Stivali dell'Inverno}\textbf{Stivali dell'Inverno}

\textit{Oggetto meraviglioso, non comune} - 10000 mo

Mentre indossi questi stivali, ottieni i seguenti benefici:

- Hai resistenza ai danni da freddo.

- Ignori il terreno difficile prodotto da neve o ghiaccio.

- Puoi tollerare le temperature fino ai -45° C senza bisogno di ulteriori protezioni. Se indossi abiti pesanti, puoi tollerare temperature fino a -75° C.

\index{Stivali della Levitazione}\textbf{Stivali della Levitazione}

\textit{Oggetto meraviglioso, raro} - 5000 mo

Mentre indossi questi stivali, puoi usare a volontà due azioni per lanciare l'incantesimo levitazione su di te.

\index{Stivali della Velocità}\textbf{Stivali della Velocità}

\textit{Oggetto meraviglioso} - 5000 mo

Mentre indossi questi stivali, puoi usare due azioni per raddoppiare la tua velocità di passeggio, e qualsiasi creatura che effettui un attacco di opportunità contro di te, ha -1d6 al Tiro per Colpire. Puoi terminare l'effetto quando vuoi.

Quando la proprietà degli stivali è stata usata per un totale di 10 minuti, la magia cessa di funzionare fino all'arrivo della prossima alba. 

\index{Talismano di Protezione dal Veleno}\textbf{Talismano di Protezione dal Veleno}

\textit{Oggetto meraviglioso, raro} - 5000 mo

Mentre indossi questo pendente i veleni non hanno alcun effetto su di te. Sei immune alla condizione avvelenato e hai immunità ai danni da veleno. 

\index{Talismano della Salute}\textbf{Talismano della Salute}

\textit{Oggetto meraviglioso, non comune} - 5000 mo

Mentre indossi questo pendente sei immune alla possibilità di contrarre qualsiasi malattia. Se sei già infetto da una malattia, i suoi effetti vengono sospesi finché indossi questo pendente.

\index{Talismano della Sfera}\textbf{Talismano della Sfera}

\textit{Oggetto meraviglioso, leggendario}

Quando effettui una prova di Arcana controllare una sfera dell'annientamento mentre stai impugnando questo talismano hai un bonus di 5. Inoltre, quando inizi il round con il controllo di una sfera dell'annientamento, puoi usare due azioni per farla levitare di 3 metri più un numero di metri aggiuntivi pari a 3 x il tuo valore di Intelligenza.

\index{Tappeto Volante}\textbf{Tappeto Volante}

\textit{Oggetto meraviglioso, molto raro} - 15000 mo

Puoi pronunciare la parola di comando del tappeto con due azioni per far fluttuare e volare il tappeto. Esso si muove in base alle direzioni indicategli a voce, purché ti trovi entro 9 metri da esso.

Esistono quattro taglie di tappeto volante. Il Narratore sceglie la taglia del tappeto o la determina casualmente.

\medskip

\begin{tabular}{llll}
d100 &Taglia &Capacità &Velocità di Volo\\
01-20& 90 cm x 1,5 m &100 kg &24 metri\\
21-55& 1,2 m x 1,8 m &200 kg &18 metri\\
56-80& 1,5 m x 2,1 m &300 kg &12 metri\\
81-100& 1,8 m x 2,7 m& 400 kg& 9 metri\\
\end{tabular}

\medskip

Il tappeto può trasportare fino al doppio del peso indicato sulla tabella, ma vola a velocità dimezzata se trasporta più della sua capacità di carico.

\index{Tomo dell'Autorità e dell'Influenza}\textbf{Tomo dell'Autorità e dell'Influenza}

\textit{Oggetto meraviglioso, molto raro}

Questo libro contiene indicazioni su come influenzare e affascinare il prossimo, e le sue parole sono soffuse di magia. Se trascorri 48 ore in un periodo di 6 giorni o meno a studiare i contenuti del libro e praticarne le indicazioni, il tuo punteggio di Carisma aumenta di 1. Poi il manuale perde la sua magia, per recuperarla dopo un secolo.

\index{Tomo della Comprensione}\textbf{Tomo della Comprensione}

\textit{Oggetto meraviglioso, molto raro}

Questo libro contiene esercizi di intuizione e discernimento, e le sue parole sono soffuse di magia. Se trascorri 48 ore in un periodo di 6 giorni o meno a studiare i contenuti del libro e praticarne le indicazioni, il tuo punteggio di Saggezza aumenta di 1, e così fa il tuo punteggio massimo per quella caratteristica. Poi il manuale perde la sua magia, per recuperarla dopo un secolo.

\index{Tomo del Pensiero Limpido}\textbf{Tomo del Pensiero Limpido}

\textit{Oggetto meraviglioso, molto raro}

Questo libro contiene esercizi di memoria e logica, e le sue parole sono soffuse di magia. Se trascorri 48 ore in un periodo di 6 giorni o meno a studiare i contenuti del libro e praticarne le indicazioni, il tuo punteggio di Intelligenza aumenta di 1. Poi il manuale perde la sua magia, per recuperarla dopo un secolo.

\index{Tridente del Comando dei Pesci}\textbf{Tridente del Comando dei Pesci}

\textit{Arma (tridente), non comune} - 1000 mo

Questo tridente è un'arma magica. Finché lo porti con te, puoi usare due azioni e spendere 1 carica per lanciare tramite esso dominare bestie (DC del Tiro Salvezza 15) su di una bestia che abbia una velocità di nuoto innata. Il tridente ha 3 cariche, e recupera 1d3 cariche spese ogni giorno all'alba.

\index{Unguento Ristorativo}\textbf{Unguento Ristorativo}

\textit{Oggetto meraviglioso, non comune} - 5000 mo

Questa giara di vetro, 7,5 centimetri di diametro, contiene 1d4 + 1 dosi di una densa mistura. La giara e i suoi contenuti pesano 250 grammi. Con due azioni, si può inghiottire o applicare sulla pelle una dose di unguento. La creatura che lo riceve recupera 2d8 + 2 punti ferita, smette di essere avvelenata e viene curata da qualsiasi malattia.

\index{Ventaglio Arcano}\textbf{Ventaglio Arcano}

\textit{Oggetto meraviglioso, non comune} - 1500 mo

Mentre impugni questo ventaglio, puoi usare due azioni per lanciare tramite esso l'incantesimo folata di vento (DC del Tiro Salvezza 15). Una volta usato, il ventaglio
non dovrebbe essere usato di nuovo fino alla prossima alba. Ogni volta che venga usato prima di allora, c'è una probabilità cumulativa del 20\% che non funzioni e si rompa in inutili brandelli privi di magia. 

\index{Verga dell'Assorbimento}\textbf{Verga dell'Assorbimento}

\textit{Verga, molto raro} - 50000 mo

Mentre impugni questa verga, puoi usare una Azione per assorbire un incantesimo che prenda come bersaglio solo te e privo di un'area di effetto. L'effetto dell'incantesimo assorbito è cancellato, e l'energia dell'incantesimo (non l'incantesimo stesso) viene assorbita dalla verga. Nel corso della sua esistenza la verga può assorbire e contenere fino ad una somma di 181 Difficoltà di incantesimi. Una volta che la verga ha assorbito 8 incantesimi (max Difficoltà 23), non ne potrà più assorbire. Se sei il bersaglio di un incantesimo che la verga non può contenere, la verga non ha alcun effetto sull'incantesimo. 

Quando prendi in mano la verga, sai quanti incantesimi la verga ha assorbito finora.

Se sei un incantatore e impugni la verga, puoi convertire l'energia contenuta per avere +1d6 alla prova di magia per incantesimo assorbito.

\index{Verga Inamovibile}\textbf{Verga Inamovibile}

\textit{Verga, non comune} - 5000 mo

Questa verga di ferro piatta ha un pulsante a un'estremità. Puoi usare due azioni per premere il pulsante, che fa sì che la verga resti magicamente fissata sul posto. Fino a quando tu o un'altra creatura userete due azioni per premere di nuovo il pulsante, la verga non si muoverà, anche se dovesse sfidare la gravità. La verga può sostenere fino a 4.000 chili di peso. Un peso maggiore fa sì che la verga si disattivi e cada. Una creatura può usare due azioni per effettuare una prova di Forza con DC 30, spostando la verga di 3 metri in caso di successo.

\index{Verga della Forza Sovrana}\textbf{Verga della Forza Sovrana}

\textit{Verga, leggendario} - 30000 mo

Questa verga ha una testa flangiata, e funziona come una mazza magica che conferisce un bonus di +3 ai Tiri per Colpire e danno effettuati con essa. La verga ha delle proprietà associate ai sei diversi pulsanti che sono disposti lungo il manico. Possiede anche altre tre proprietà descritte di seguito.

\textbf{Sei Pulsanti}. Puoi premere uno dei sei pulsanti della verga con due azioni. L'effetto del pulsante dura finché non premi un pulsante differente o premi di nuovo lo stesso pulsante, facendo tornare la verga alla sua forma normale.

Se premi il \textit{pulsante 1}, la verga diventa un'arma lingua di fuoco, e una lama infuocata fuoriesce dall'estremità opposta alla testa flangiata.

Se premi il \textit{pulsante 2}, la testa flangiata della verga si ripiega e fuoriescono due lame a mezzaluna, che trasformano la verga in un'ascia da battaglia magica che conferisce un bonus di +3 ai Tiri per Colpire e danno effettuati con essa.

Se premi il \textit{pulsante 3}, la testa flangiata della verga si ripiega, e una punta di lancia esce fuori dall'estremità della verga, mentre il manico si allunga fino a 1,8 metri, trasformando la verga in una lancia magica che conferisce un bonus di +3 ai Tiri per Colpire e danno effettuati con essa.

Se premi il \textit{pulsante 4}, la verga si trasforma in un'asta per scalare lunga fino a 15 metri, come specificato da te. Sulle superfici dure come il granito, uno spuntone sul fondo e tre in cima tengono l'asta fissa sul posto. Sbarre orizzontali lunghe 7,5 centimetri si dipanano lungo i lati della verga, a 30 centimetri di distanza l'uno dall'altro, per formare una scala. L'asta può sostenere 2.000 chili. Un peso superiore o la mancanza di un ancoraggio solido fa sì che la verga torni alla sua forma normale.
\
Se premi il \textit{pulsante 5}, la verga si trasforma in un ariete da sfondamento e conferisce a chi lo usa un bonus di +10 alle prove di Forza effettuate per sfondare porte, barricate o altre barriere.

Se premi il \textit{pulsante 6}, la verga assume o rimane nella sua forma normale e indica il nord magnetico (non accade nulla se questa funzione della verga viene impiegata in zone prive di un nord magnetico). La verga ti fornisce anche un'approssimativa conoscenza della profondità sottoterra e della tua altezza sul livello del mare. Risucchiare Vita. Quando colpisci una creatura con un attacco in mischia utilizzando la verga, puoi obbligare il bersaglio a effettuare un tiro Salvezza su Tempra con DC 21. Se lo fallisce, il bersaglio subisce 4d6 danni da Vuoto aggiuntivi, e tu recuperi un numero di punti ferita pari alla metà del danno da Vuoto inflitto. Una volta usata, questa proprietà non più essere usata fino all'alba del giorno successivo.

\textbf{Paralizzare}. Quando colpisci una creatura con un attacco da mischia utilizzando la verga, puoi obbligare il bersaglio a effettuare un Tiro Salvezza su Tempra con DC 21. Se lo fallisce, il bersaglio è paralizzato per 1 minuto. Il bersaglio può ripetere il Tiro Salvezza al termine di ciascun suo round, terminando l'effetto su di sé in caso lo superi. Una volta usata, questa proprietà non può più essere usata fino all'alba del giorno successivo. Terrorizzare. Mentre impugni questa verga, puoi obbligare ogni creatura che vedi entro 9 metri da te a effettuare un Tiri Salvezza su Volontà con DC 21. Se lo fallisce, il bersaglio è spaventato da te per 1 minuto. Il bersaglio spaventato può ripetere il Tiro Salvezza al termine di ciascun suo round, terminando l'effetto su di sé in caso lo superi. Una volta usata, questa proprietà non può più essere usata fino all'alba del giorno successivo.

\index{Verga della Prontezza}\textbf{Verga della Prontezza}

\textit{Verga, molto raro} - 25000 mo

Questa verga dalla testa flangiata ha le seguenti proprietà.

\textit{Prontezza}. Mentre impugni questa verga, hai +1d6 alle prove di Saggezza e ai tiri di iniziativa. Incantesimi. Mentre impugni questa verga, puoi usare due azioni per lanciare tramite essa uno dei seguenti incantesimi: individuazione del bene e del male, individuazione del magico, individuazione del veleno e delle malattie o vedere invisibilità.

\textit{Aura Protettiva}. Con due azioni, puoi piantare l'estremità appuntita della verga nel terreno. A quel punto la testa della verga irradierà luce intensa in un raggio di 18 metri e luce fioca per ulteriori 18 metri. All'interno di questa luce intensa, tu e qualsiasi creatura a te amichevole otterrete un bonus di +1 alla Difesa e ai Tiri Salvezza e potrete percepire la posizione di qualsiasi creatura invisibile ostile che si trovi anch'essa all'interno della luce intensa. La testa della verga smette di emettere luce e termina l'effetto dopo 10 minuti, o quando una creatura usa due azioni per estrarre la verga dal terreno. Questa proprietà non può essere usata di nuovo fino all'alba del giorno successivo.

\index{Verga della Sicurezza}\textbf{Verga della Sicurezza}

\textit{Verga, molto raro} - 90000 mo

Mentre impugni questa verga, puoi usare due azioni per attivarla. Di conseguenza la verga trasporta te e fino ad altre 199 altre creature consenzienti visibili in un paradiso collocato in uno spazio extraplanare. Sarai tu a scegliere la forma di questo paradiso. Potrebbe essere un placido giardino, una gradevole radura, un'allegra taverna, un immenso palazzo, un'isola tropicale, o una fantastica fiera o qualsiasi altra cosa tu riesca a immaginare. Quale che sia la sua natura, il paradiso contiene cibo e bevande sufficienti ad alimentare i suoi visitatori. Tutto ciò con cui si può interagire nello spazio extraplanare può esistere solo al suo interno.

Per ogni ora trascorsa in questo paradiso, un visitatore recupera punti ferita come se avesse speso 1 Dado Vita. Inoltre, finché le creature restano nel paradiso non invecchiano, sebbene il tempo trascorra normalmente. I visitatori possono restare nel paradiso per un massimo di 200 giorni diviso il numero di creature presenti (arrotondare per difetto). 

Quando il tempo termina o usi due azioni per farlo terminare, tutti i visitatori ricompaiono nel luogo da loro occupato quando hai attivato la verga, o nello spazio non occupato più vicino a quello. La verga non potrà essere usata di nuovo prima che siano passati dieci giorni.

\index{Verga della Sovranità}\textbf{Verga della Sovranità}

\textit{Verga, raro} - 16000 mo

Puoi usare due azioni e presentare la verga e richiedere obbedienza a ciascuna creatura visibile entro 36 metri da te di tua scelta. Ogni bersaglio deve superare un Tiri Salvezza su Volontà con DC 17 o restare affascinato da te per 8 ore. Mentre è affascinata in questa maniera, la creatura ti considera un capo fidato. Se le viene recato danno da te o dai tuoi compagni, o le viene ordinato di fare qualcosa contrario alla sua natura, il bersaglio smetterà di essere affascinato in questa maniera. La verga non può essere usata di nuovo prima della prossima alba.

\index{Verga Tentacolare}\textbf{Verga Tentacolare}

\textit{Verga, raro} - 5000 mo

Questa verga è un'arma magica che termina in tre tentacoli gommosi. Mentre impugni la verga, puoi usare due azioni per dirigere ciascun tentacolo per attaccare una creatura visibile entro 4,5 metri da te. Ogni tentacolo effettua un Tiro per Colpire da mischia con un bonus di +9. Se colpisci, il tentacolo infligge 1d6 danni contundenti. Se colpisci un bersaglio con tutti e tre i tentacoli, esso deve effettuare un Tiro Salvezza di Costituzione con DC 15. Se lo fallisce, la velocità della creatura è dimezzata, ha s+1d6 ai Tiri Salvezza di Riflessi, e per 1 minuto non può usare le sue reazioni. Inoltre, durante ciascun suo round, egli può effettuare due azioni o due azioni ma non entrambe. Il bersaglio può ripetere il Tiro Salvezza al termine di ciascun suo round, terminando l'effetto su di sé in caso lo superi.


\index{Vestaglia dei Colori Scintillanti}\textbf{Vestaglia dei Colori Scintillanti}

\textit{Oggetto meraviglioso, molto raro} - 6000 mo

Questa vestaglia ha 3 cariche, e recupera 1d3 cariche spese ogni giorno all'alba. Quando la indossi, puoi usare due azioni e spendere 1 carica per far sì che l'indumento produca una trama mutevole di colori abbaglianti fino al termine del tuo prossimo round. Durante questo periodo, la vestaglia emana luce intensa in un raggio di 9 metri e luce fioca per ulteriori 9 metri. Le creature che ti vedono hanno -1d6 ai Tiri per Colpire contro di te. Inoltre, qualsiasi creatura sotto la luce intensa e che ti veda quando il potere della vestaglia viene attivato, deve superare un Tiri Salvezza su Volontà con DC 17 o restare stordita fino al termine dell'effetto.

\index{Vestaglia degli Occhi}\textbf{Vestaglia degli Occhi}

\textit{Oggetto meraviglioso, raro} - 30000 mo

Questa vestaglia è adornata da un disegno di occhi. Mentre la indossi, ottieni i seguenti benefici:

- La vestaglia ti permette di vedere in tutte le direzioni e hai +1d6 alle prove di Consapevolezza basate sulla vista.

- Hai scurovisione con una portata di 36 metri.

- Puoi vedere creature e oggetti invisibili, oltre che nel Piano Etereo, fino a una gittata di 36 metri.

Gli occhi della vestaglia non possono essere chiusi o distolti, e mentre indossi questa vestaglia non viene mai considerato a occhi chiusi o distolti.

L'incantesimo luce lanciato sulla vestaglia o l'incantesimo luce diurna lanciato entro 1,5 metri dalla vestaglia ti rendono accecato per 1 minuto. Al termine di ciascun tuo round, puoi effettuare un tiro Salvezza su Tempra (DC 13 per luce o DC 17 per luce diurna), ponendo fine alla condizione accecato in caso lo superi.

\index{Vestaglia degli Oggetti Utili}\textbf{Vestaglia degli Oggetti Utili}

\textit{Oggetto meraviglioso, non comune} - 200 mo

Mentre indossi questa vestaglia ricoperta da toppe di varie forme e colori, puoi usare due azioni per staccare una delle toppe, facendola diventare l'oggetto o la creatura che rappresenta. Quando l'ultima toppa viene rimossa, la vestaglia diventa un indumento normale. La vestaglia possiede due di ciascuna delle seguenti toppe:

Asta di 3 metri, Corda di canapa (15 metri,arrotolata), Lanterna a lente sporgente (piena e accesa), Pugnale, Sacco, Specchio d'acciaio.

Inoltre, la vestaglia ha 4d4 altre toppe. Il Narratore sceglie le toppe o le determina a caso, scegliendo tra proprietà totalmente diverse da quelle già presenti.

Tira un d100 sulla tabella seguente per scoprire le proprietà delle altre 4d4 toppe della vestaglia degli oggetti utili.

\end{multicols}

\medskip

\begin{tabularx}{0.95\textwidth}{lX}
\textbf{d100} & \textbf{Effetto}\\
\toprule
01-08 &Borsello con 100 mo.\\
09-15& Forziere d'argento (lungo 30 cm, largo e profondo 15 cm) del valore di 500 mo.\\
16-22& Porta di ferro (larga e alta massimo 3 metri, sbarrata dal lato di tua scelta), che puoi piazzare su qualsiasi apertura a portata; si adatta per entrare nell'apertura, fissandosi e creando dei cardini.\\
23-30 &10 gemme del valore di 100 mo ciascuna.\\
31-44 &Una scala di legno (7,5 metri).\\
45-51 &Un cavallo da corsa con sacche da sella 52-59 Fossa (un cubo di 3 metri di spigolo), che puoi piazzare sul terreno entro 3 metri da te.\\
60-68 &4 pozioni di guarigione. \\
69-75 &Barca a remi (lunga 3,5 metri).\\
76-83& Pergamena degli incantesimi contenente un incantesimo di livello dal 1° al 3°.\\
84-90& Due mastini.\\
91-96 &Finestra (60 x 120 cm, profonda massimo 60 cm), che puoi piazzare su qualsiasi superficie verticale a portata.\\
97-100 &Ariete portatile.\\
\end{tabularx}

\begin{multicols}{2}

\medskip

\index{Vestaglia delle Stelle}\textbf{Vestaglia delle Stelle}

\textit{Oggetto meraviglioso, molto raro} - 60000 mo

Mentre indossi questa vestaglia, ottieni un bonus di +1 ai Tiri Salvezza.

Sei stelle, posizionate sulla parte superiore frontale della vestaglia, sono più grosse delle altre. Mentre indossi questa vestaglia, puoi usare due azioni per estrarre una delle stelle e usarla per lanciare dardo incantato. Ogni giorno al tramonto, la stella rimossa ricompare sulla vestaglia. Mentre indossi la vestaglia, puoi usare due azioni per entrare nel Piano Astrale assieme a tutto ciò che indossi o trasporti. Resterai lì fino a quando userai due azioni per ritornare al piano in cui ti trovavi prima. Ricompari nell'ultimo spazio da te occupato, o se quello spazio è occupato, nello spazio non occupato più vicino.

\index{Zainetto Pratico}\textbf{Zainetto Pratico}

\textit{Oggetto meraviglioso, raro} - 7000 mo

Questo zaino ha una sacca centrale e due laterali, ciascuna delle quali è in realtà uno spazio extradimensionale. Ogni sacca laterale può contenere 10 chili di materiale, che non ecceda un volume di 60 dm3

- La grande sacca centrale può contenere fino a 240 dm3 o 40 chili di materiale. Lo zaino pesa sempre 2,5 chili, quali che siano i suoi contenuti.

Piazzare un oggetto all'interno dello zainetto segue le normali regole di interazione con gli oggetti. Recuperare un oggetto dallo zainetto richiede l'uso di due azioni. Quando cerchi un oggetto nello zainetto, questo magicamente si troverà sempre in cima alla pila degli oggetti che questo contiene.

Lo zainetto ha alcune limitazioni. Se sovraccarico, o un oggetto affilato lo taglia o si strappa, lo zainetto si spacca e viene distrutto. Se lo zainetto è distrutto, ciò che conteneva è perso per sempre, sebbene un artefatto ricomparirà sempre da qualche parte nel multiverso. Se lo zainetto viene rivoltato, ciò che contiene viene espulso, senza recargli danno, e lo zainetto deve essere rimesso al verso giusto prima che possa essere usato di nuovo. Se una creatura che respira viene posta all'interno dello zainetto, vi può sopravvivere per al massimo 10 minuti, prima di cominciare a soffocare.

Piazzare lo zainetto all'interno dello spazio extradimensionale creato da una borsa conservante, un buco portatile o un oggetto simile distrugge immediatamente entrambi gli oggetti e apre un portale verso il Piano Astrale. Il portale origina dal punto in cui gli oggetti sono stati posti l'uno dentro l'altro. Qualsiasi creatura entro 3 metri dal portale viene risucchiata attraverso di esso e trascinata in un luogo casuale del Piano Astrale. Poi il portale si chiude. Il portale è a senso unico e non può essere riaperto.

\index{Zoccoli della Velocità}\textbf{Zoccoli della Velocità}

\textit{Oggetto meraviglioso, raro} - 5000 mo

Questi zoccoli di ferro si trovano in set da quattro. Quando tutti e quattro gli zoccoli sono fissati a un cavallo o creatura simile, aumentano la velocità di passeggio di quella creatura di 9 metri. 

\index{Zoccoli dello Zefiro}\textbf{Zoccoli dello Zefiro}

\textit{Oggetto meraviglioso, molto raro} - 1500 mo

Questi zoccoli di ferro si trovano in set da quattro. Quando tutti e quattro gli zoccoli sono fissati a un cavallo o creatura simile, permettono a quella creatura di muoversi normalmente, mentre fluttua a circa 10 centimetri dal terreno. Questo effetto vuol dire che la creatura può attraversare o passare sopra superfici non solide o instabili, come l'acqua o la lava. La creatura non lascia tracce e ignora il terreno difficile. Inoltre, la creatura può muoversi alla sua normale velocità per un massimo di 12 ore al giorno senza subire l'affaticamento a causa della marcia forzata.



\end{multicols}

\pagebreak

\subsection{Descrizione delle Capacità Speciali delle Armi Magiche}

\lettrine[lines=2, lhang=0.33, loversize=0.25, findent=1.5em]{U}{n'} arma con una capacità speciale deve avere almeno bonus magico +1.

Vengono qui elencati le capacita' magiche che un'armatura, scudo o arma puo' avere oltre al generico bonus magico (+1,+2....). Usate questo elenco come linee guida ed esempi, stessa cosa per i prezzi, usateli come indicazione di rarità.

\begin{multicols}{2}

\subsubsection{Accumula Incantesimi}\index{Accumula Incantesimi}

Un'arma Accumula Incantesimi permette a un incantatore di immagazzinare un incantesimo con bersaglio fino Difficoltà 21 nell'arma. L'incantesimo deve avere un tempo di lancio standard di 2 Azioni. Ogni volta che l'arma colpisce una creatura e quest'ultima subisce dei danni, chi impugna l'arma con una azione immediata può liberare l'incantesimo.\\
Una volta che l'incantesimo viene lanciato, un incantatore può immagazzinarvi all'interno qualsiasi altro incantesimo con bersaglio, sempre fino a Difficoltà 21.

L'arma rivela magicamente a chi la impugna il nome dell'incantesimo attualmente contenuto. Un'arma Accumula Incantesimi creata casualmente ha una probabilità del 50\% di avere già un incantesimo contenuto al suo interno. Questa capacità speciale può essere aggiunta solo ad armi da mischia.

Un'arma Accumula Incantesimi emette una forte aura della scuola Invocazione, più l'aura dell'incantesimo contenuto.

Aura Invocazione forte e variabile; Requisiti di Creare Oggetti Magici Superiori, Costo +3000 mo.

\subsubsection{Adattiva}\index{Adattiva}

Questa capacità può essere aggiunta solo agli archi compositi. Un arco Adattivo reagisce alla forza di chi lo impugna, agendo come un arco con un bonus di Forza pari a quello di chi lo sta impugnando. Chi lo impugna può scoccare con un bonus di Forza inferiore (e causare meno danni) se lo desidera.

Aura Trasmutazione debole; Requisiti di Creare Oggetti Magici, Deformare Legno; Costo +1500 mo.

\subsubsection{Affilata}\index{Affilata}

Questa capacita' in caso di critico consente di contare il numero di 6 tirati aumentandolo di 1.
Solo le armi da mischia taglienti o perforanti possono essere affilate.

Aura Trasmutazione moderata; Requisiti di Creare Oggetti Magici Superiori, Estremità Affilata; Costo +5000 mo.

\subsubsection{Anatema}\index{Anatema}

Un'arma Anatema eccelle nell'attaccare certe creature. Contro il nemico prescelto, il suo bonus effettivo diventa di +2. L'arma, inoltre, infligge +2d6 danni addizionali contro tale nemico. Per determinare casualmente il nemico prescelto dell'arma si usa la tabella seguente:

\medskip

\begin{tabular}{ll}
d\% 	&Nemico prescelto\\
01-05 	&Aberrazioni\\
06-09 	&Bestie\\
10-16 	&Costrutti\\
17-22 	&Draghi\\
23-27 	&Fatati\\
28-60 	&Umanoidi (scegliere sottotipo)\\
61-65 	&Creature Magiche\\
66-70 	&Giganti\\
71-72 	&Melme\\
73-88 	&Immondi\\
89-90 	&Piante\\
91-98 	&Non Morti\\
99-100 	&Insetti\\
\end{tabular}

\medskip

Aura Evocazione moderata; Requisiti di Creare Oggetti Magici Superiori, Evoca Mostri I; Costo +3000 mo.

\subsubsection{Cacciatore}\index{Cacciatore}

Un'arma del Cacciatore aiuta chi la impugna a localizzare e a catturare la preda. Quando l'arma viene tenuta in mano, chi la impugna ottiene il bonus dell'arma alle prove di Sopravvivenza effettuate per seguire le tracce di qualsiasi creatura che l'arma ha danneggiato nel corso del giorno precedente. Infligge +1d6 danni alle creature di cui sono state seguite le tracce con Sopravvivenza da chi la impugna nel corso del giorno precedente.

\textbf{Dettagli}: Aura Divinazione moderata; Requisiti di Creare Oggetti Magici Superiori, Individuazione di Animali o Vegetali; Costo +3000 mo.

\subsubsection{Conduttiva}\index{Conduttiva}

Un'arma Conduttiva è in grado di incanalare l'energia di una Abilità magica che richieda un attacco di contatto in mischia o a distanza per colpire il suo bersaglio. 

Quando chi la impugna effettua con successo un attacco del tipo appropriato, può scegliere di spendere due utilizzi della sua capacità magica per incanalarla attraverso l'arma, al fine di colpire l'avversario, che subisce gli effetti dell'attacco dell'arma e quelli della capacità speciale (come incanalare energia, imposizione delle mani...).

Questa capacità speciale dell'arma può essere utilizzata solamente una volta per round (anche se possiede più armi Conduttive).

\textbf{Dettagli}: Aura Necromanzia moderata; Requisiti di Creare Oggetti Magici Superiori, Mano Arcana; Costo +3000 mo.


\subsubsection{Coraggiosa}\index{Coraggiosa}

Questa capacità speciale può essere aggiunta solo a un'arma da mischia. Un'arma Coraggiosa fortifica il coraggio e il morale in battaglia di chi la indossa. Chi la impugna ottiene un Bonus ai Tiri Salvezza contro la paura pari al bonus dell'arma.\\ 

\textbf{Dettagli}: Aura Ammaliamento debole; Requisiti di Creare Oggetti Magici, Eroismo, Paura; Costo +3000 mo.

\subsubsection{Corrosiva}\index{Corrosiva}

A comando, un'arma Corrosiva si ricopre di uno strato di acido che infligge 1d6 danni aggiuntivi da acido quando colpisce il bersaglio. L'acido non danneggia chi la impugna. L'effetto permane fino a quando non viene impartito un nuovo comando.

\textbf{Dettagli}: Aura Invocazione moderata; Requisiti di Creare Oggetti Magici, Freccia Acida; Costo +3000 mo.


\subsubsection{Crudele}\index{Crudele}

Un'arma Crudele si alimenta di paura e sofferenza. Quando chi la impugna colpisce una creatura spaventata, Scossa o In Preda al Panico con un'arma crudele, quest'ultima diventa Inferma per 1 round. Quando chi la impugna usa l'arma per rendere Priva di Sensi o uccidere una creatura, ottiene 5 Punti Ferita Temporanei che durano per 10 minuti.

\textbf{Dettagli}: Aura Necromanzia debole; Requisiti di Creare Oggetti Magici, Paura, Costo +3000 mo.

\subsubsection{Danzante}\index{Danzante}

Come azione standard, un'arma Danzante può essere lasciata libera in modo che combatta da sola. L'arma combatte per 4 round usando il Difesa di colui che l'ha lasciata libera e poi cade a terra. 

Rimane sempre accanto alla persona che l'ha liberata, anche se si sposta con mezzi fisici o magici. Se colui che l'ha lasciata libera ha una mano libera può riprendere l'arma che sta attaccando da sola, come azione immediata, ma una volta ripresa, la spada non potrà più danzare (attaccare da sola) prima di 4 round.

Questa capacità può essere aggiunta solo ad armi da mischia.

\textbf{Dettagli}: Aura Trasmutazione forte; Requisiti di Creare Oggetti Magici Superiori, Animare Oggetti; Costo +25000 mo.

\subsubsection{Designante}\index{Designante}

Questa capacità speciale può essere aggiunta solo ad armi a distanza o munizioni. Ogni volta che un'arma a distanza o una munizione con questa capacità colpisce una creatura, designa magicamente il bersaglio. Tutti gli alleati ottengono Bonus +2 ai Tiri per Colpire per 1 round. Più colpi andati a segno sullo stesso bersaglio non incrementano i bonus o la loro durata.

\textbf{Dettagli}: Aura Ammaliamento moderato; Requisiti di Creare Oggetti Magici Superiori, Luce; Costo +6000 mo.

\subsubsection{Difensiva}\index{Difensiva}

Un'arma Difensiva permette a chi la impugna di trasferire una parte o tutto il bonus dell'arma alla propria Difesa come un bonus cumulabile con eventuali altri bonus. Come azione immediata, chi la impugna può scegliere come disporre del bonus dell'arma all'inizio del round, prima di usarla, e il bonus alla Difesa dura fino al turno successivo.

\textbf{Dettagli}: Aura Abiurazione moderata; Requisiti di Creare Oggetti Magici Superiori, Scudo; Costo +3000 mo.

\subsubsection{Distanza}\index{Distanza}

Questa capacità speciale può essere aggiunta solo a proiettili. Un proiettile della Distanza ha il doppio della gittata data dall'arma che lancia.

\textbf{Dettagli}: Aura Divinazione moderata; Requisiti di Creare Oggetti Magici, Chiaroudienza/Chiaroveggenza; Costo +3000 mo.


\subsubsection{Distruzione}\index{Distruzione}

Un'arma della Distruzione è la rovina di tutti i Non Morti. Ogni creatura Non Morta colpita in combattimento deve superare un Tiro Salvezza su Volontà con CD 14 o viene distrutta o subire 2d8 di danni aggiuntivi da Luce. Un'arma della Distruzione deve essere un'arma da mischia da botta.

\textbf{Dettagli}: Aura Evocazione forte; Requisiti di Creare Oggetti Magici Superiori, Guarigione; Costo +6000 mo.

\subsubsection{Duello}\index{Duello}

Questa capacità può essere conferita solo ad un'arma da mischia. Un'arma del Duello (che deve essere un'arma che può essere utilizzata con il talento Arma Accurata) garantisce a chi la impugna bonus +4 alle prove di Iniziativa, purché l'arma sia stata estratta e impugnata quando viene effettuata la prova di Iniziativa. 

\textbf{Dettagli}: Aura Trasmutazione debole; Requisiti di Creare Oggetti Magici, Grazia del Gatto; Costo +7.000 mo.

\subsubsection{Energia Luminosa}\index{Energia Luminosa}

Un'arma di Energia Luminosa si trasforma per la maggior parte in Luce, anche se ciò non influisce sul suo peso. Fornisce sempre luce come una torcia (6 metri di raggio).

Un'arma di Energia Luminosa ignora la materia non vivente. Bonus di Armatura e Scudo alla Difesa (compreso qualsiasi bonus a quell'armatura) non contano contro di essa perché l'arma passa attraverso l'armatura.
Destrezza, Armatura Naturale e bonus magici si applicano normalmente.

Un'arma di Energia Luminosa non può ferire Costrutti ed oggetti. Causa +2d6 danni ai Non Morti.

\textbf{Dettagli}: Aura Trasmutazione forte; Requisiti di Creare Oggetti Magici Meravigliosi, Fiamma Perenne, Esplosione Solare; Costo +45000 mo.

\subsubsection{Estingui Fuoco}\index{Estingui Fuoco}

Questa capacità speciale può essere aggiunta solo ad armi da mischia. Un'arma Estingui Fuoco è in grado di estinguere un fuoco non magico di taglia Media o inferiore. Quando usata contro una creatura del Fuoco, infligge 1d6 danni addizionali. Chi impugna un'arma Estingui Fuoco ottiene Bonus di Competenza +2 ai Tiri Salvezza contro gli effetti basati sul fuoco e l'arma stessa è immune ai danni da fuoco.

\textbf{Dettagli}: Aura Trasmutazione debole; Requisiti di Creare Oggetti Magici, Gelare il Metallo; Costo +3000 mo.

\subsubsection{Ferimento}\index{Ferimento}

Questa capacità può essere aggiunta solo ad armi da mischia. Un'arma da Ferimento infligge 1 danno da Sanguinamento quando colpisce una creatura. Danni multipli di quest'arma aumentano il danno da Sanguinamento fino ad un massimo di 5.
Le creature sanguinanti subiscono il danno da Sanguinamento all'inizio del loro round.

Il Sanguinamento può essere fermato con una prova di Pronto Soccorso con DC 15 o con un incantesimo qualsiasi che curi i danni ai Punti Ferita. Un Colpo Critico non moltiplica il danno da Sanguinamento. Le creature immuni ai Colpi Critici sono immuni ai danni da Sanguinamento inflitti da quest'arma.

\textbf{Dettagli}: Aura Invocazione moderata; Requisiti di Creare Oggetti Magici Superiori, Contagio; Costo +6000 mo.

\subsubsection{Folgorante}\index{Folgorante}

A comando, un'arma Folgorante viene avvolta da elettricità crepitante che infligge 1d6 danni addizionali da elettricità per ogni colpo andato a segno. Quest'elettricità non danneggia chi impugna l'arma. L'effetto rimane sempre attivo finché l'arma e' sguainata.

\textbf{Dettagli}: Aura Invocazione moderata; Requisiti di Creare Oggetti Magici Superiori, Fulmine; Costo +3000 mo.

\subsubsection{Furia Innata}\index{Furia Innata}

Questa capacità speciale può essere aggiunta solo ad armi da mischia. Un'arma della Furia Innata trae potere dalla rabbia e dalla frustrazione che prova chi la impugna quando combatte dei nemici che si rifiutano di morire. Ogni volta che chi la impugna infligge danni ad un avversario con l'arma, il suo bonus aumenta di +1 quando effettua attacchi contro quel nemico (fino a un bonus massimo totale di +5). Questo bonus aggiuntivo svanisce se l'avversario muore, o se chi impugna l'arma la usa per attaccare una creatura diversa, manca con il tiro per colpire o passa 1 ora.

\textbf{Dettagli}: Aura Ammaliamento moderato; Requisiti di Creare Oggetti Magici Superiori, Eroismo; Costo +4000 mo.

\subsubsection{Gelida}\index{Gelida}

A comando, un'arma Gelida viene avvolta da un gelo terribile che infligge 1d6 danni da freddo per ogni colpo andato a segno. Questo freddo non danneggia chi impugna l'arma. L'effetto rimane sempre attivo finché l'arma e' sguainata.

\textbf{Dettagli}: Aura Invocazione moderata; Requisiti di Creare Oggetti Magici Superiori, Gelare il Metallo; Costo +3000 mo.

\subsubsection{Gloriosa}\index{Gloriosa}

Un'arma Gloriosa illumina con una luce abbagliante pari a quella un incantesimo Luce Diurna quando viene estratta. Chi la impugna non può sopprimere questa luce, anche se può essere soppressa temporaneamente da qualsiasi effetto che può sopprimere Luce Diurna. \\
Quando un'arma Gloriosa effettua un Colpo Critico, il bersaglio è Accecato fino all'inizio del round successivo del possessore (Tempra CD 14 nega). Solo un'arma da mischia può avere la capacità Gloriosa.

\textbf{Dettagli}: Aura Invocazione moderata; Requisiti di Creare Oggetti Magici, Cecità/Sordità,  Luce Diurna; Costo +6000 mo.

\subsubsection{Guardiana}\index{Guardiana}

Questa capacità può essere aggiunta solo ad armi da mischia. Un'arma Guardiana permette a chi la impugna di trasferire una parte o tutto il bonus dell'arma ai suoi Tiri Salvezza come bonus che si cumula con tutti gli altri. Come azione immediata, chi impugna l'arma sceglie come distribuire il bonus dell'arma all'inizio del suo turno prima di usare l'arma. Il bonus a tutti i Tiri Salvezza dura fino al suo turno successivo. Solo il bonus proprio dell'arma può essere sacrificato, non si può usare nessun altro bonus derivante da altri effetti.

Se un'arma ha sia la capacità Difensivo che Guardiana, sacrificare un singolo punto del bonus migliora la Difesa o i Tiri Salvezza, ma non entrambi.

\textbf{Dettagli}: Aura Abiurazione moderata; Requisiti di Creare Oggetti Magici Superiori, Resistenza; Costo +3000 mo.

\subsubsection{Immorale}\index{Immorale}

Questa capacità può essere aggiunta solo ad armi da mischia. Quando un'arma Immorale colpisce un avversario, produce un lampo di Vuoto che riecheggia tra chi la impugna e il suo bersaglio. L'energia infligge 2d6 danni addizionali all'avversario e 1d6 danni a chi la impugna.

\textbf{Dettagli}: Aura Invocazione moderata; Requisiti di Creare Oggetti Magici Superiori, Debilitazione; Costo +3000 mo.

\subsubsection{Impulso Vitale}\index{Impulso Vitale}

Questa capacità speciale può essere aggiunta solo ad armi da mischia. Un'arma dell'Impulso Vitale aumenta e sostiene l'energia vitale di chi la impugna, mentre è nel mezzo del combattimento. Chi la impugna guadagna un bonus ai Tiri Salvezza contro gli effetti di Necromanzia (inclusi danni alle caratteristiche, risucchi di caratteristica e riduzioni punti ferita massimi dovuti ai poteri dei Non Morti) pari al bonus dell'arma. Inoltre, ogni volta che chi la impugna ottiene Punti Ferita Temporanei da qualsiasi fonte, vi aggiunge il bonus dell'arma.

\textbf{Dettagli}: Aura Evocazione moderata; Requisiti di Creare Oggetti Magici Superiori, Cura Ferite Gravi, Ristorare superiore; Costo +6000 mo.

\subsubsection{Infuocata}\index{Infuocata}

A comando, un'arma Infuocata viene avvolta da fiamme che infliggono 1d6 danni da fuoco per ogni colpo andato a segno. Questo fuoco non danneggia chi impugna l'arma. L'effetto rimane attivo finché non viene disattivato con un altro comando.

\textbf{Dettagli}: Aura Invocazione moderata; Requisiti di Creare Oggetti Magici Superiori, Palla di Fuoco; Costo +3000 mo.

\subsubsection{Letale}\index{Letale}

Questa capacità speciale può essere aggiunta solo ad armi da mischia che normalmente infliggono Danni Non Letali, da stordimento Tutti i danni di un'arma Letale sono normali (letali). A comando, azione immediata, l'arma sopprime questa capacità finché chi la impugna non gli ordina di riattivarla.

\textbf{Dettagli}: Aura Necromanzia debole; Requisiti di Creare Oggetti Magici Superiori, Infliggi Ferite Leggere; Costo +3000 mo.

\subsubsection{Marina}\index{Marina}

Questa capacità speciale può essere aggiunta solo ad armi da mischia. Un'arma Marina funziona tranquillamente negli ambienti acquatici. Con l'arma in mano, chi la impugna ottiene un bonus alle prove di Nuotare pari al doppio del bonus dell'arma. \\
Inoltre, chi la impugna non subisce le normali penalità ai Tiri per Colpire e per i danni dovuti al trovarsi sott'acqua, come se fosse soggetto a un incantesimo Libertà di Movimento.

\textbf{Dettagli}: Aura Necromanzia moderata; Requisiti di Creare Oggetti Magici Superiori, Libertà di Movimento,Costo +3000 mo.

\subsubsection{Mascheramento}\index{Mascheramento}

A un'arma del Mascheramento può essere comandato di mutare la sua forma e apparire come un altro oggetto di taglia simile. L'arma conserva tutte le sue proprietà (compreso il peso) anche quando è mascherata, ma non irradia magia. Solo Visione del Vero o altre magie simili rivelano la reale natura dell'arma trasformata. Dopo che un'arma del Mascheramento è stata usata per attaccare, questa capacità speciale viene soppressa per 1 minuto.

\textbf{Dettagli}: Aura Illusione moderata; Requisiti di Creare Oggetti Magici Superiori, Arma Magica, Camuffare Se Stesso; Costo +2.000 mo.

\subsubsection{Munizione Fantasma}\index{Munizione Fantasma}

Questa capacità può essere conferita solo alle munizioni. Una munizione con questa capacità speciale delle armi si dissolve 1 round dopo essere stata scagliata. In aggiunta, se il Proiettile colpisce un bersaglio, la ferita causata si richiude non appena la munizione si disintegra. Il Proiettile infligge danni normalmente, ma non lascia alcuna traccia visibile di violenza.

Il prezzo si riferisce a 50 Munizioni Fantasma.

\textbf{Dettagli}: Aura Trasmutazione moderata; Requisiti di Creare Oggetti Magici Superiori, Disintegrazione, Rendere Integro; Costo +1.000.

\subsubsection{Munizioni Infinite}\index{Munizioni Infinite}

Solo archi e balestre possono essere rese armi dalle Munizioni Infinite. Ogni volta che un'arma dalle Munizioni Infinite viene incoccata, una singola freccia o quadrello non magico viene creato spontaneamente dalla sua magia, quindi chi lo impugna non ha mai bisogno di caricare l'arma con delle munizioni.

Se chi lo impugna tenta di caricare l'arma con altre munizioni, la freccia o il quadrello creato svanisce immediatamente e si può caricare l'arma come di norma. Questa capacità non riduce l'ammontare di tempo necessario per caricare o scoccare con l'arma. La freccia o il quadrello creato svanisce se rimosso dall'arma; persiste solo se scagliato. A differenza di una normale munizione per arco o balestra, queste frecce e quadrelli vengono sempre distrutti quando scagliati.

\textbf{Dettagli}: Aura Evocazione moderata; Requisiti di Creare Oggetti Magici Superiori, Creazione Minore; Costo +6000 mo.

\subsubsection{Pietosa}\index{Pietosa}

Tutto il danno inflitto dall'arma e' temporaneo.

A comando, l'arma sopprime questa capacità fino a quando non le viene ordinato di riattivarla (permettendole di infliggere danni letali).

\textbf{Dettagli}: Aura Evocazione debole; Requisiti di Creare Oggetti Magici, Cura Ferite Leggere; Costo +3000 mo.

\subsubsection{Planare}\index{Planare}

Un'arma Planare è efficace contro tutti i tipi di esseri extradimensionali, essendo in grado di superare la loro resistenza ai danni fisici. Quando usata per attaccare gli Esterni, un'arma Planare ignora 5 punti della loro Riduzione del Danno o Resistenze.

\textbf{Dettagli}: Aura Evocazione moderata; Requisiti di Creare Oggetti Magici Superiori, Spostamento Planare; Costo +3000 mo.

\subsubsection{Prensile}\index{Prensile}

Questa capacità può essere conferita solo alle fruste. Una Frusta Prensile può, come azione di movimento, aggrapparsi a un oggetto come se fosse un Rampino. La Frusta può poi essere usata per scalare superfici o dondolare attraverso una stanza o qualsiasi area all'aperto. 

\textbf{Dettagli}: Aura Ammaliamento moderato; Requisiti di Creare Oggetti Magici Superiori, Trucco della Corda; Costo +2.500.

\subsubsection{Ricercante}\index{Ricercante}

Questa capacità può essere aggiunta solo ad armi a distanza. Un'arma Ricercante vira verso il suo bersaglio, negando qualsiasi probabilità di mancarlo che si potrebbe applicare, come quella dovuta all'Occultamento. Chi la impugna deve comunque mirare l'arma nel quadretto giusto. Le Frecce scoccate per errore in uno spazio vuoto, per esempio, non virano per colpire gli avversari Invisibili, se ce n'è qualcuno nelle vicinanze.

\textbf{Dettagli}: Aura Divinazione forte; Requisiti di Creare Oggetti Magici Superiori, Visione del Vero; Costo +3000 mo.

\subsubsection{Ritornante}\index{Ritornante}

Un'arma Ritornante può teletrasportarsi nelle mani del suo possessore come azione immediata, anche se si trova in possesso di un'altra creatura. Questa capacità ha un raggio massimo di 30 metri e gli effetti che bloccano il teletrasporto impediscono il ritorno di un'arma Ritornante. Un'arma Ritornante deve essere in possesso di una creatura per almeno 24 ore perché questa capacità funzioni.

\textbf{Dettagli}: Aura Evocazione moderata; Requisiti di Creare Oggetti Magici Superiori, Teletrasporto; Costo +3000 mo.

\subsubsection{Sacra}\index{Sacra}

Un'arma Sacra è infusa di potere sacro che rende l'arma allineata con un Patrono specifico. 

Infligge 2d6 danni addizionali a tutte le creature che non sono Devoti o Seguaci di quel Patrono. Tenere l'arma in mano se non sei un Devoto o Seguace causa 1d6 di danno a round.

\textbf{Dettagli}: Aura Invocazione moderata; Tratti comuni 12; Requisiti di Creare Oggetti Magici Superiori; Costo +6000 mo.


\subsubsection{Sprezzante}\index{Sprezzante}

Questa capacità speciale può essere aggiunta solo ad armi da mischia. Un'arma Sprezzante aiuta chi la impugna a sopravvivere in condizioni disperate. Rimane nelle mani di chi la impugna anche se quest'ultimo è In Preda al Panico, Stordito o Privo di Sensi. Chi la impugna aggiunge il suo bonus come bonus alle prove per Pronto Soccorso quando e' svenuto  o morente ed aggiunge lo stesso anche ai TS contro incantesimi che causano morte istantanea.

\textbf{Dettagli}: Aura Abiurazione forte; Requisiti di Creare Oggetti Magici Superiori, Stabilizzare; Costo +6000 mo.

\subsubsection{Tocco Fantasma}\index{Tocco Fantasma}

Un'arma del Tocco Fantasma infligge danni alle creature Incorporee normalmente, indipendentemente dal suo bonus. 

\textbf{Dettagli}: Aura Evocazione moderata; Requisiti di Creare Oggetti Magici Superiori, Spostamento Planare; Costo +3000 mo.

\subsubsection{Tonante}\index{Tonante}

Un'arma Tonante crea un tremendo frastuono simile a quello di un tuono, quando mette a segno un Colpo Critico. L'energia sonora non danneggia chi tiene in mano l'arma e infligge 1d8 danni sonori addizionali per ogni Colpo Critico andato a segno.Chi è soggetto a un Colpo Critico da un'arma Tonante deve effettuare un Tiro Salvezza su Tempra con CD 14 o resta Sordo in modo permanente.

\textbf{Dettagli}: Aura Necromanzia debole; Requisiti di Creare Oggetti Magici, Cecità/Sordità; Costo +3000 mo.

\subsubsection{Trasformante}\index{Trasformante}

Questa capacità si può aggiungere solo ad armi da mischia. Un'arma Trasformante altera la sua forma a comando di chi la impugna, diventando una qualsiasi altra arma da mischia con dimensioni simili Ad esempio, una Spada Lunga trasformante può assumere la forma di una qualsiasi altra arma da mischia a una mano Media, come una Scimitarra, un Mazzafrusto Leggero od un Tridente, ma non un'arma da mischia leggera o a due mani Media (come una Spada Corta Media o uno Spadone a due mani).\\
L'arma conserva tutte le sue capacità, compresi bonus e capacità speciali dell'arma, ad eccezione di quelle proibite dalla sua nuova forma attuale. Se lasciata incustodita, l'arma ritorna alla sua forma originaria.

\textbf{Dettagli}: Aura Trasmutazione moderata; Requisiti di Creare Oggetti Magici Superiori, Creazione Maggiore; Costo +5.000 mo.

\subsubsection{Velocità}\index{Velocità}

Quando compie più attacchi (2 Azioni), chi impugna un'arma di Velocità può compiere un attacco addizionale con l'arma. L'attacco aggiuntivo non ha le penalità degli attacchi multipli. Questa capacita' non e' cumulabile con incantesimi o effetti simili.

\textbf{Dettagli}: Aura Trasmutazione moderata; Requisiti di Creare Oggetti Magici Superiori, Velocità; Costo +15000 mo.

\subsection{Descrizione delle Capacità Speciali delle Armature Magiche e degli Scudi Magici}

Gran parte delle armature e degli scudi magici hanno solo bonus, ma certi possiedono alcune delle capacità speciali descritte qui sotto. Un'armatura o uno scudo con capacità speciali devono avere almeno bonus +1.

\subsubsection{Accecante}\index{Accecante}

Uno scudo dotato di questo incantamento emana una luce accecante per un massimo di due volte al giorno su comando di chi lo impugna. Tutti coloro che si trovano entro 6 metri dallo scudo, eccetto chi lo impugna, devono superare un Tiro Salvezza su Riflessi con CD 14 o restano Accecati per 1d4 round.

\textbf{Dettagli}: Aura Invocazione moderata; Requisiti di Costruzione Creare Oggetti Magici Superiori, Luce Incandescente; Costo +3000 mo.

\subsubsection{Amorfa}\index{Amorfa}

Una volta al giorno a comando, chi indossa l'armatura (insieme a qualsiasi equipaggiamento indossi) può assumere la forma di un liquido viscoso che è in grado di passare attraverso qualsiasi spazio nel quale potrebbe ragionevolmente scorrere del fango denso. Mentre si usa questa capacità, la propria velocità viene ridotta a 3 metri e si possono effettuare solo azioni di movimento. Si può assumere questa forma per 1 minuto o finché non si spende un'azione di movimento per tornare alla propria forma naturale. Un'armatura Amorfa deve essere fatta principalmente di cuoio, stoffa o altro materiale organico e flessibile.

\textbf{Dettagli}: Aura Trasmutazione moderata; Requisiti di Costruzione Creare Oggetti Magici Superiori, Metamorfosi, Costo +2.250 mo.

\subsubsection{Animato}\index{Animato}

Come azione di movimento, uno scudo Animato può essere lasciato da solo a difendere il suo possessore. Per i 4 round successivi, lo scudo conferisce il suo bonus a chi lo ha lasciato e poi cade. Mentre è animato, lo scudo conferisce il suo Bonus ed i bonus dati da qualsiasi altra capacità speciale abbia, ma non può intraprendere azioni di sua volontà, come quelle conferite dalle capacità magiche.
Se chi lo lascia ha una mano libera, può afferrarlo quando gli effetti magici svaniscono come azione immediata. Una volta che uno scudo è stato ripreso, non può essere animato nuovamente se non dopo almeno 4 round. Questa proprietà non può essere aggiunta a uno Scudo di tipo Pesante.

\textbf{Dettagli}: Aura Trasmutazione forte; Requisiti di Costruzione Creare Oggetti Magici Superiori, Animare Oggetti; Costo +6000 mo.

\subsubsection{Antiemorragica}\index{Antiemorragica}

Un'armatura Antiemorragica aiuta a fermare la perdita di sangue dalle ferite di chi la indossa, stringendo automaticamente come un laccio emostatico nei punti appropriati mentre riduce anche magicamente l'entità della ferita.

Un'armatura Antiemorragica riduce i danni ai Punti Ferita di 1 per colpo subito e si e' non si può subire danni da sanguinamento. 

\textbf{Dettagli}: Aura Trasmutazione moderata; Requisiti di Costruzione Creare Oggetti Magici Superiori, Cura Ferite Critiche, Ristorare Inferiore o Stabilizzare; Costo +3000 mo.

\subsubsection{Ariete}\index{Ariete}

Questi scudi sono molto solidi e spesso recano l'emblema di un ariete o un toro. Quando chi indossa uno scudo ariete effettua un attacco con lo scudo come parte di una Carica, il bonus alla Difesa dello scudo si applica ai Tiri per Colpire e per i danni. Questo non si cumula con nessun altro potenziamento che possegga lo scudo. Questa capacita' non e' applicabile agli scudi di tipo leggero.

\textbf{Dettagli}: Aura Invocazione debole; Requisiti di Costruzione Creare Oggetti Magici Superiori, Pugno di Forza; Costo +3000 mo.


\subsubsection{Attaccabrighe}\index{Attaccabrighe}

Chi indossa un'armatura Attaccabrighe ottiene bonus +2 ai Tiri per Colpire e per i danni degli attacchi senz'armi. I suoi colpi senz'armi contano come armi magiche al fine di superare la Riduzione del Danno. La capacità Attaccabrighe può essere applicata solo alle armature leggere.

\textbf{Dettagli}: Aura Trasmutazione debole; Requisiti di Costruzione Creare Oggetti Magici, Forza del Toro; Costo +15000 mo

\subsubsection{Bilanciata}\index{Bilanciata}

Questa armatura respinge tutto ciò che rischia di buttare a terra chi la indossa. Il portatore ottiene bonus +4 contro chi prova a spingerlo o buttarlo a terra.\\
Buttarsi a terra mentre si indossa un'armatura Bilanciata è un'azione di movimento invece di una azione immediata. La capacità Bilanciata può essere applicata ad armature leggere o medie, ma non a quelle pesanti o agli scudi.

\textbf{Dettagli}: Aura Trasmutazione debole; Requisiti di Costruzione Creare Oggetti Magici, Grazia del Gatto; Costo +3000 mo.

\subsubsection{Brillante}\index{Brillante}

Armature e scudi con la capacità speciale Brillante irradiano luce come una torcia quando indossati, che può essere soppressa o riattivata a comando. L'aspetto dell'oggetto di solito è caratterizzato da colori vivaci e una brillante lucentezza anche quando non illuminato. Una volta al giorno, il portatore può comandare all'armatura o allo scudo di brillare con l'intensità di un incantesimo Luce Diurna per 10 minuti o finché non gli viene comandato di affievolirla.\\
Questa armatura va pulita almeno 1 volta a settimana o perde i poteri per una settimana.

\textbf{Dettagli}: Aura Invocazione moderata; Requisiti di Costruzione Creare Oggetti Magici, Luce Diurna; Costo +3.750 mo.

\subsubsection{Carico}\index{Carico}

Un'armatura del Carico distribuisce il peso trasportato da chi la indossa più efficacemente, permettendogli di trasportarne di più senza subire gli effetti dell'ingombro. La capacità di ingombro di chi la indossa viene aumentata del 50\%.

\textbf{Dettagli}: Aura Trasmutazione debole; Requisiti di Costruzione Creare Oggetti Magici, Armatura Passiva; Costo +2.000 mo.

\subsubsection{Denegante}\index{Denegante}

Una volta al giorno, quando colui che indossa l'armatura è bersaglio di un Colpo Critico o Esplosione del danno effettuato con un'arma da mischia, può automaticamente negare questo Critico e renderlo un attacco normale. Questa capacità può essere applicata solo alle armature pesanti.

\textbf{Dettagli}: Aura Abiurazione forte; Requisiti di Costruzione Artigiano Mitico Meravigliosi, Creare Oggetti Magici, Desiderio Limitato o Miracolo; Costo +25000 mo.

\subsubsection{Determinazione}\index{Determinazione}

Uno scudo o un'armatura dotati di questa capacità ha la capacità di combattere in circostanze apparentemente impossibili. Una volta al giorno, quando il possessore raggiungere 0 o meno Punti Ferita, l'oggetto attiva automaticamente l'incantesimo Cure ferite serie.

\textbf{Dettagli}: Aura Evocazione moderata; Requisiti di Costruzione Creare Oggetti Magici Superiori, Cura ferite serie; Costo +15.000 mo.

\subsubsection{Felpa}\index{Felpa}

Un armatura con la capacità Felpa conta per quanto concerne le penalità di indossare un armatura ad una armatura di pelle. Il personaggio riesce a muoversi quasi senza difficoltà con questa armatura.

\textbf{Dettagli}: Aura Trasmutazione forte; Requisiti di Costruzione Creare Oggetti Magici Superiori; Costo +6.000 mo.

\subsubsection{Forma Eterea}\index{Forma Eterea}

A comando, questa proprietà permette a chi indossa l'armatura di diventare Etereo (come per l'incantesimo Transizione Eterea) una volta al giorno. Il personaggio può rimanere Etereo per quanto tempo desidera ma, una volta tornato alla normalità, per quel giorno non può più diventare Etereo.

\textbf{Dettagli}: Aura Trasmutazione forte; Requisiti di Costruzione Creare Oggetti Magici Meravigliosi, Transizione Eterea; Costo +24.500 mo.

\subsubsection{Invulnerabilità}\index{Invulnerabilità}

Questa armatura garantisce a chi la indossa una Riduzione del Danno di 5/magia. Un armatura con Invulnerabilità emette un'\textbf{Dettagli}: Aura di Abiurazione forte.

\textbf{Dettagli}: Aura Abiurazione forte (Invocazione se viene usato Miracolo); Requisiti di Costruzione Creare Oggetti Magici Mitici, Desiderio; Costo +15000 mo.

\subsubsection{Irrintracciabile}\index{Irrintracciabile}

Un'armatura Irrintracciabile alleggerisce i passi di chi la indossa e ne camuffa l'aspetto. Le prove di Sopravvivenza per seguire le tracce del portatore subiscono penalità -5, e chi indossa l'armatura ottiene Bonus di Competenza +5 alle prove di Furtività. Soltanto le armature di cuoio o di pelle possono essere Irrintracciabili.

\textbf{Dettagli}: Aura Trasmutazione debole; Requisiti di Costruzione Creare Oggetti Magici, Passare Senza Tracce; Costo +3.750 mo.

\subsubsection{Mascheramento}\index{Mascheramento}

A comando, un'armatura di questo tipo muta la sua forma e appare come un normale set di vestiti. L'armatura conserva tutte le sue proprietà (compreso il peso) anche quando è mascherata. Solo Visione del Vero o altre magie simili rivelano la reale natura dell'armatura trasformata.

\textbf{Dettagli}: Aura Illusione moderata; Requisiti di Costruzione Creare Oggetti Magici Superiori, Camuffare Se Stesso; Costo +1.350 mo.

\subsubsection{Ombra}\index{Ombra}

Quest'armatura rende chi la indossa sfocato ogni volta che tenta di nascondersi, fornendo Bonus di +5 alle sue prove di Nascondersi nelle ombre. La penalità di armatura alla prova si applica normalmente.

\textbf{Dettagli}: Aura Illusione debole; Requisiti di Costruzione Creare Oggetti Magici, Invisibilità, Silenzio; Costo +1.875 mo.

\subsubsection{Ospitale}\index{Ospitale}

Un'armatura o uno scudo con questa capacità speciale nasconde animali vivi all'interno della sua iconografia per tenerli al sicuro. Il portatore con una parola di comando immagazzina magicamente un animale a cui è legato, come un Famiglio o una Cavalcatura. L'animale immagazzinato appare come simbolo sull'armatura o sullo scudo, che si tratti di un'imitazione dell'aspetto dell'animale o di una rappresentazione più simbolica e astratta.

Mentre è immagazzinato, l'animale dorme e non dà alcun beneficio (come il bonus alle abilità di un Famiglio) a chi la indossa. La taglia degli animali immagazzinabili dipende dal tipo di armatura o scudo. Le armature leggere o medie e gli scudi leggeri o pesanti possono immagazzinare un animale di taglia massima pari a quella di chi li indossa. Un'armatura pesante o uno scudo torre possono immagazzinare un animale fino a una categoria taglia superiore rispetto a chi li indossa. Una seconda parola di comando rilascia l'animale immagazzinato nell'armatura o nello scudo ospitale. Un animale liberato si risveglia immediatamente, appare in uno spazio adiacente al portatore e può intraprendere azioni nel round in cui appare.

Dato che l'animale immagazzinato dorme anziché essere in animazione sospesa (o persino in letargo), invecchia e ha fame al ritmo normale mentre è immagazzinato. Un'armatura o uno scudo Ospitale rilascia automaticamente un animale immagazzinato 24 ore dopo che vi è stato immagazzinato all'interno.

\textbf{Dettagli}: Aura Evocazione moderata; Requisiti di Costruzione Creare Oggetti Magici Superiori, Scrigno Segreto; Costo +3.750 mo.


\subsubsection{Percettiva}\index{Percettiva}

Un'armatura Percettiva viene in soccorso quando chi la indossa è stato Accecato, si trova nel buio totale (se il portatore non ha Scurovisione o la capacità Vedere al Buio), o si trova in un'oscurità magica. Quando una di queste condizioni influenza chi indossa l'armatura, un'armatura percettiva gli concede immediatamente Vista Cieca in un raggio di 1,5 metri. Non appena il portatore torna a vedere, i sensi addizionali cessano. Chi indossa l'armatura non può ottenere queste capacità chiudendo gli occhi.

\textbf{Dettagli}: Aura Divinazione forte; Requisiti di Costruzione Creare Oggetti Magici Meravigliosi, Visione del Vero; Costo +15000 mo.

\subsubsection{Resistenza al Veleno}\index{Resistenza al Veleno}

Un'armatura o uno scudo con questa capacità speciale conferisce a chi lo indossa Bonus di +3 ai Tiri Salvezza contro il veleno.

\textbf{Dettagli}: Aura Trasmutazione debole; Requisiti di Costruzione Creare Oggetti Magici Superiori, Neutralizza Veleno; Costo +1.125 mo.

\subsubsection{Resistenza all'Energia}\index{Resistenza all'Energia}

Questo tipo di armatura o scudo protegge contro un tipo di energia (Fuoco, Luce, Suono, Elettricità, Energia Positiva, Energia Negativa, Freddo, Vuoto) ed è decorata da disegni che raffigurano l'elemento dal quale protegge. L'armatura o lo scudo assorbono i primi 10 danni di energia per attacco che verrebbero subiti normalmente da chi li indossa.

\textbf{Dettagli}: Aura Abiurazione debole; Requisiti di Costruzione Creare Oggetti Magici, Resistere all'Energia; Costo +9.000 mo.

\subsubsection{Resistenza all'Energia Superiore}\index{Resistenza all'Energia Superiore}

Questo tipo di armatura o scudo protegge contro un tipo di energia (Fuoco, Luce, Suono, Elettricità, Energia Positiva, Energia Negativa, Freddo, Vuoto) ed è decorata da disegni che raffigurano l'elemento dal quale protegge. L'armatura o lo scudo concedono Resistenza all'energia indicata.

\textbf{Dettagli}: Aura Abiurazione moderata; Requisiti di Costruzione Creare Oggetti Magici Superiori, Resistere all'Energia; Costo +21.000 mo.

\subsubsection{Selvatica}\index{Selvatica}

Un'armatura con questa capacità speciale generalmente sembra fatta di pelle animale magicamente indurita. Chi indossa un'armatura o uno scudo con questa capacità conserva la Difesa anche mentre e' trasformato in un animale (vuoi per Incantesimo o Abilità).\\
Le armature e gli scudi con questa capacità di solito recano motivi di foglie. Mentre chi la indossa è in Forma Selvatica, l'armatura non è visibile.

\textbf{Dettagli}: Aura Trasmutazione moderata; Requisiti di Costruzione Creare Oggetti Magici Superiori, Metamorfosi Funesta; Costo +15000 mo.

\subsubsection{Soffio del Dragone}\index{Soffio del Dragone}

Uno scudo con questa capacità speciale di solito è realizzato con le fauci di un drago spalancate sulla parte anteriore. Uno scudo con la capacità speciale Soffio del Dragone è legato a un tipo di energia (veleno, elettricità, freddo o fuoco).Lo scudo recupera 1d4 cariche ad ogni alba e ne puo' tenere fino a 10. \\
A comando, 2 Azioni, chi lo indossa può consumare da 1 a 5 cariche dello scudo per fargli emettere un Soffio in un cono di 4,5 metri che infligge 1d4 danni da energia per carica consumata (Riflessi CD 11 dimezza). Questo danno è dello stesso tipo di energia legato allo scudo. Uno scudo non può avere più di una capacità Soffio del Dragone.

\textbf{Dettagli}: Aura Invocazione debole; Requisiti di Costruzione Creare Oggetti Magici, Mani Brucianti; Costo +2.500 mo.

\subsubsection{Titanica}\index{Titanica}

Un'armatura con la proprietà Titanica è quasi comicamente fuori misura, anche se l'effetto è solo esteriore e l'interno accoglie una creatura come di norma, senza necessitare modifiche. Una creatura che indossa un'armatura Titanica è considerata di una categoria di taglia superiore, questo anche al fine dell'uso di oggetti ed armi o dell'essere influenzati da attacchi speciali che dipendono dalla taglia, come Inghiottire e Travolgere.\\

\textbf{Dettagli}: Aura Trasmutazione moderata; Requisiti di Costruzione Creare Oggetti Magici Superiori, Ingrandire Persone; Costo +15000 mo.

\subsubsection{Tocco Fantasma}\index{Tocco Fantasma}

Quest'armatura o scudo sembra quasi trasparente. Il valore di Difesa dato dall'armatura viene conteggiato contro gli attacchi delle creature corporee e Incorporee. Inoltre l'armatura o lo scudo possono essere raccolti, spostati e indossati in qualsiasi momento dalle creature corporee e Incorporee. Le creature Incorporee ottengono il bonus dell'oggetto contro attacchi corporei e incorporei, e mantengono comunque la capacità di passare attraverso gli oggetti solidi.

\textbf{Dettagli}: Aura Trasmutazione forte; Requisiti di Costruzione Creare Oggetti Magici Meravigliosi, Forma Eterea; Costo +15000 mo.
\end{multicols}

\pagebreak

\section{Oggetti Maledetti}\index{Oggetti Maledetti}

\begin{enfasiwide}{Quando un empio maledice l'avversario, maledice se stesso. (Siracide)
		
\medskip		
		
Se maledici una persona ci saranno due fosse. Proverbio giapponese}
\end{enfasiwide}\medskip

\begin{multicols}{2}

\label{oggetti-maledetti}

\lettrine[lines=2, lhang=0.33, loversize=0.25, findent=1.5em]{G}{li} oggetti maledetti sono oggetti magici dotati di un'influenza potenzialmente negativa sul personaggio. A volte tendono a confondere il male con il bene, costringendo il loro possessore a fare scelte difficili.

Gli oggetti maledetti non sono quasi mai realizzati intenzionalmente, ma piuttosto sono il risultato di un lavoro mal riuscito, di artigiani con poca esperienza o della mancanza di componenti adeguati.

Quando una prova di creazione di un oggetto magico fallisce di 10 o piu', tirate sulla tabella per determinare il tipo di maledizione che l'oggetto possiede.

\medskip

\textbf{Maledizioni Comuni degli Oggetti}

\medskip

\begin{tabular}{ll}
	\textbf{\%} & \textbf{Maledizione}\\
	\toprule
	01-15       & Inganno\\
	16-40       & Effetto o Bersaglio Opposto\\
	41-50       & Funzionamento Discontinuo\\
	51-65       & Requisito\\
	66-90       & Inconveniente\\
	91-100      & Effetto completamente diverso\\
\end{tabular}

\medskip

Gli oggetti maledetti sono identificati come qualsiasi altro oggetto magico con una sola eccezione: a meno che la prova effettuata per identificare l'oggetto riesca e ci sia pure un critico (2 volte 6) la maledizione non viene individuata. Se la prova non riesco con critico, ma riesce comunque, tutto quello che viene rivelato è l'originale scopo dell'oggetto magico.

Se si sa che l'oggetto è maledetto, la natura della maledizione può essere determinata usando la DC \hyperlink{identificareom}{standard} per identificare l'oggetto.

\begin{center}
	\includegraphics[width=0.75\linewidth]{immagini/vasobasano.png}
	
	\textit{Vaso di Basano. Questo vaso è stato realizzato nella seconda metà del XV secolo ed è realizzato in argento.}
\end{center}


\begin{narratoresmall}
Una maledizione e' sempre un \textit{inconveniente} particolare, che non si usa a caso. Ragionate attentamente sugli oggetti maledetti che farete trovare ai personaggi perché vi chiederanno molte informazioni e dovrete essere pronti.
	
Non c'e' bisogno che la maledizione sia eccessiva e limitante puo' essere benissimo ridicola o particolare, fate in modo che sia caratterizzante. Il personaggio non deve sentirsi (tranne se lo volete) condannato in eterno, sfruttate l'occasione per costruire nuove avventure e spirito di gruppo.
\end{narratoresmall}


\subsection{Rimuovere Oggetti Maledetti}\index{Rimuovere Oggetti Maledetti}

Mentre alcuni oggetti maledetti possono essere semplicemente posati, altri esercitano una forte compulsione sul possessore a tenerli con sé, a qualsiasi costo. Altri riappaiono anche se abbandonati o è impossibile gettarli via.

Questi oggetti possono essere rimossi solo dopo che sul personaggio o l'oggetto viene lanciato l'incantesimo Rimuovi Maledizione. 

Se l'oggetto e' stato maledetto tramite l'incantesimo Scagliare Maledizione, o comunque il Narratore decide che l'oggetto ha una maledizione specifica per quel personaggio allora la prova di magia per il Rimuovere Maledizione deve essere almeno pari alla Difficoltà con il quale e' stata scagliata la maledizione.

Se la prova ha successo, l'oggetto può essere rimosso nel round successivo, ma la maledizione rimane e colpisce nuovamente se l'oggetto viene usato un'altra volta.

Ogni oggetto maledetto ha un proprio metodo per essere distrutto, dall'essere gettato in un vulcano attivo, ad essere colpito dal martello del dio del Tuono (o Patrono...) oppure divorato da un Verme colossale delle sabbie se non colpito dal soffio di un drago rosso e un drago bianco contemporaneamente.

\subsection{Effetti Comuni degli Oggetti Maledetti}

Gli effetti più comuni degli oggetti maledetti sono i seguenti. I Narratore possono inventare nuovi effetti particolari per specifici oggetti maledetti.


\subsubsection{Inganno}

Chi utilizza l'oggetto continua a credere che sia ciò che sembra a prima vista, ma in realtà non ha alcun potere, a parte quello di ingannare. Chi lo usa è mentalmente spinto a credere che funzioni, e non può essere convinto del contrario se non con l'uso di Rimuovi maledizione

\begin{center}
	\includegraphics[width=0.70\linewidth]{immagini/mirror.png}
	
	\textit{The mirror in The Myrtles Plantation.}
\end{center}

\subsubsection{Effetto o Bersaglio Opposto}

Questi oggetti maledetti tendono ad avere dei difetti di funzionamento che in alcuni casi generano effetti diametralmente opposti a quelli desiderati dal loro creatore, mentre in altri casi tendono a colpire chi li utilizza invece di qualcun altro.

Ma la cosa più interessante è che questi oggetti potrebbero anche non essere uno svantaggio per chi li possiede. La categoria degli oggetti magici dagli effetti opposti include anche le armi che infliggono penalità ai Tiri per Colpire e per i danni, invece che bonus.

Visto che un personaggio non dovrebbe sapere immediatamente quale sia il bonus di un oggetto magico, non dovrebbe venire a conoscenza nemmeno della natura della sua maledizione. Una volta che lo verrà a sapere, comunque, l'oggetto potrà essere abbandonato a meno che su di esso non vi sia qualche effetto magico che costringa il suo possessore a tenerlo e ad usarlo. In questi casi, per liberarsi dall'oggetto sarà necessario l'Incantesimo Rimuovi maledizione.

Alcune maledizione particolarmente forti, a discrezione del Narratore, possono essere rimosse da Rimuovi Maledizioni che abbia raggiunto un certo punteggio di Difficoltà (o sia riuscito con almeno 1 critico nella prova di magia).

\subsection{Funzionamento Discontinuo}

Gli oggetti discontinui funzionano esattamente come dovrebbero, quando funzionano. Stabilite se l'oggetto e' Inaffidabile, Condizionato oppure Incotrollabile.

\medskip
\subsubsection{Inaffidabile}

Ogni volta che l'oggetto viene attivato, c'è una probabilità del 5\% che non funzioni.

\subsubsection{Condizionato}

Questo oggetto funziona solo in determinate situazioni. Per determinare quali siano, scegliete una condizione di attivazione o consultato la tabella poco sotto.

\subsubsection{Incontrollabile}

Un oggetto incontrollabile tende ad attivarsi casualmente. Tirare un d\% ogni giorno. Con un risultato di 01--05 l'oggetto si attiva spontaneamente in un certo momento del giorno.

\medskip

\begin{tabularx}{0.45\textwidth}{lX}
	\textbf{\%} & \textbf{Situazione}\\
	\toprule
	01-03       & Temperatura sotto lo zero\\
	04-05       & Temperatura sopra lo zero\\
	06-10       & Durante il giorno\\
	11-15       & Durante la notte\\
	16-20       & Esposto alla luce solare\\
	21-25       & In assenza di luce solare\\
	26-34       & Sott'acqua\\
	35-37       & Fuori dall'acqua\\
	38-45       & Sottoterra\\
	46-55       & In superficie\\
	56-60       & Entro 3 metri da un tipo di creatura casuale\\
	61-64       & Entro 3 metri da una razza o tipo di creatura casuale\\
	65-72       & Entro 3 metri da un incantatore\\
	73-80       & Entro 3 metri da un Seguace o Devoto di un Patrono specifico\\
	81-85       & Nelle mani di un personaggio non incantatore\\
	86-90       & Nelle mani di un personaggio incantatore\\
	91-95       & Nelle mani di una creatura con particolare Tratto\\
	96          & Nelle mani di una creatura di un particolare sesso\\
	97-99       & Nei giorni non sacri o durante particolari ricorrenze astronomiche\\
	100         & A più di 150 km da un determinato luogo\\
\end{tabularx}

\subsection{Requisito}

Alcuni oggetti hanno requisiti molto più difficili da soddisfare perché funzionino. Per far funzionare l'oggetto in questione, potrebbe essere necessario soddisfare una delle seguenti condizioni:

\begin{itemize}
	\item Il personaggio deve mangiare il doppio del normale.
	\item Il personaggio deve dormire il doppio del normale.
	\item Il personaggio deve compiere almeno una missione specifica.
	\item Il personaggio deve sacrificare (distruggere) un valore pari a 100 mo di oggetti o materiali preziosi al giorno.
	\item Il personaggio deve giurare lealtà ad un nobile in particolare, o alla sua famiglia.
	\item Il personaggio deve abbandonare tutti gli altri oggetti magici.
	\item Il personaggio deve essere un Seguace o Devoto di uno specifico Patrono
	\item Il personaggio deve avere un numero minimo di gradi in una particolare competenza.
	\item Il personaggio deve sacrificare parte della propria energia vitale (1 punto di Costituzione permanente) la prima volta che usa l'oggetto.
	\item L'oggetto deve essere purificato con l'acqua santa ogni giorno.
	\item L'oggetto deve essere bagnato in almeno mezzo litro di sangue (animale o umanoide) al giorno.
	\item L'oggetto deve essere usato per uccidere una creatura vivente al giorno.
	\item L'oggetto deve essere usato almeno una volta al giorno, o smette di funzionare per il suo attuale possessore.
	\item Quando viene brandito, l'oggetto deve spillare sangue (solo armi). Non può essere messo da parte o cambiato con un altro oggetto finché non ha messo a segno un colpo.
\end{itemize}

\medskip

\begin{center}
	\includegraphics[width=0.8\linewidth]{immagini/donnalemb.png}
	
	\textit{Donna di Lemb o Statua della Dea della Morte, 3500 AC}
\end{center}

\medskip

I requisiti dipendono così tanto dalla convenienza dell'oggetto che non dovrebbero mai essere determinati a caso. Un oggetto intelligente con un requisito spesso impone il proprio requisito grazie alla sua personalità.

Se il requisito non viene soddisfatto, l'oggetto smette di funzionare. Se invece viene soddisfatto, di solito l'oggetto funziona per un giorno intero prima di dover di nuovo soddisfare il requisito (anche se alcuni requisiti vanno soddisfatti una volta sola, altri una volta al mese e altri ancora in continuazione).

\subsection{Inconveniente}

Gli oggetti che hanno degli inconvenienti hanno solitamente degli effetti positivi su chi li usa, ma hanno anche degli aspetti negativi. Anche se a volte gli inconvenienti vengono alla luce solo quando gli oggetti sono utilizzati (o tenuti in mano, nel caso di oggetti come le armi), di solito rimangono presenti fino a quando il personaggio non si libera dell'oggetto in questione.

A meno che non sia indicato diversamente, gli inconvenienti rimangono attivi per tutto il tempo in cui l'oggetto rimane in possesso del personaggio. La DC dei Tiro Salvezza per evitare questi effetti è pari a 10 + DC della maledizione (se non avete stabilito la difficoltà impostate il Tiro Salvezza, solitamente su Volontà, a DC 25)


\end{multicols}

\medskip

\begin{narratorewide}
L'elenco e' di esempio per poter generare casualmente degli effetti sul possessore dell'oggetto. Prendete spunto e siate creativi!  Non fate però che una maledizione renda impossibile giocare il personaggio piuttosto deve essere vissuta come l'occasione per provare, fare, qualcosa di diverso. Non gettate mai un oggetto maledetto a caso nel mucchio dei tesori, pensate sempre cosa potrà accadere e quali conseguenze si genereranno. Un oggetto maledetto richiede sempre un alto livello di attenzione e pianificazione da parte del Narratori
\end{narratorewide}

\bigskip

\textbf{Tabella: Effetti degli oggetti magici maledetti}\index{Tabella Effetti degli oggetti magici maledetti}

\medskip

\begin{tabular}{ll}
	\textbf{\%} & \textbf{Inconveniente}\\
	\toprule
	01-02	& I capelli del personaggio crescono di 2,5 cm all'ora.\\
	02-04	& Le unghie del personaggio crescono di 1 cm ogni 8 ore\\	
	05-06   & L'altezza del personaggio diminuisce di 5d10 cm \\
	07-09   & L'altezza del personaggio aumenta di 5d10 cm \\
	10-11   & La temperatura intorno all'oggetto è di 5° C più fredda del normale.\\
	12-13   & La temperatura intorno all'oggetto è di 20° C più fredda del normale.\\
	14-15   & La temperatura intorno all'oggetto è di 5° C più calda del normale.\\
	16-17   & La temperatura intorno all'oggetto è di 20° C più calda del normale.\\
	18-20   & Il colore dei capelli del personaggio cambia.\\
	21-23   & II colore della pelle del personaggio cambia.\\
	24	& Il colore dei capelli del personaggio cambia ogni ora\\
	25	& Il colore della pelle del personaggio cambia ogni ora\\
	26      & Delle corna come un montone crescono sulla testa del personaggio\\
	27      & Un palco di corna come un alce crescono sulla testa del personaggio\\
	28-29   & II personaggio ora porta un segno distintivo (un tatuaggio, una strana	luminescenza ecc.).\\
	30-32   & II sesso del Personaggio cambia.\\
	33-34   & La razza o la specie del Personaggio cambiano.\\
	35      & II personaggio viene colpito da una Malattia determinata casualmente, che non può essere curata.\\
	36-39   & L'oggetto emette costantemente suoni sgradevoli (lamenti, maledizioni, insulti...).\\
	40      & L'oggetto ha un aspetto ridicolo (colori sgargianti, forma,brilla di un alone rosa ecc.).\\
	41      & Un unicorno blu, visibile solo con la magia, di dimensioni piccole vola sempre\\
	& attorno al Personaggio dando consigli inutili e facendo battute stupide.\\
	42	& Ogni giorno ti prende una improvvisa voglia e capacità di fare l'uncinetto per almeno 1 ora.\\
	43-45   & II personaggio diventa estremamente possessivo nei confronti dell'oggetto.\\
	46-49   & II personaggio ha una paura incontrollabile di perdere l'oggetto o che venga danneggiato.\\
	50      & Un Tratto viene cambiato\\
	51	& Il metabolismo del personaggio cambia e diventa esclusivamente carnivoro\\
	52	& Il metabolismo del personaggio cambia e diventa esclusivamente vegetariano\\
	53-54   & II personaggio deve attaccare la creatura a lui più vicina (probabilità del 5\% ogni giorno).\\
	55-57   & II personaggio rimane Stordito per 1d4 round ogni volta che l'oggetto è servito al suo scopo\\
	58-60   & Il personaggio diventa sordo\\
	61-64   & I Punti ferita massimi calano di 10 (rimanendo con un minimo di 1).\\
	65      & I Punti ferita massimi calano di 20 (rimanendo con un minimo di 1).\\
	66-70   & II PG deve effettuare un Tiro Salvezza su Volontà ogni giorno con modificatore Intelligenza\\& o subisce 1 danno a Intelligenza.\\
	71-75   & Il PG deve effettuare un Tiro Salvezza su Volontà ogni giorno o subisce 1 danno a Saggezza.\\
	76-80   & II PG deve effettuare un Tiro Salvezza su Volontà ogni giorno con modificatore Carisma \\&o subisce 1 danno a Carisma.\\
	81-85   & II PG deve effettuare un Tiro Salvezza su Tempra ogni giorno o subisce 1 danno a Forza.\\
	86-90   & II PG deve effettuare un Tiro Salvezza su Tempra ogni giorno o subisce 1 danno a Destrezza.\\
	91-95   & II PG deve effettuare un Tiro Salvezza su Tempra ogni giorno o subisce 1 danno a Costituzione.\\
	96      & II personaggio viene trasformato in una creatura specifica (probabilità del 5\% ogni giorno).\\
	97      & II personaggio non può più usare oggetti magici o Incantesimi con Difficoltà oltre 26\\
	98      & II personaggio non può più usare oggetti magici o Incantesimi con Difficoltà oltre 21\\
	99      & II personaggio non può più usare Incantesimi\\
	100     & Tira due volte\\
\end{tabular}

\pagebreak

\section{Yeru}\index{Yeru}\index{Atilantis}

\begin{enfasiwide}{
Così la Terra è davvero tonda. Però non immaginavo che fosse azzurra. Perché gli uomini che vivono su un pianeta tanto bello non fanno altro che combattere tra loro? (Nadia - Il mistero della pietra azzurra)
		
\medskip

		
Il pianeta non ci appartiene, siamo noi ad appartenergli. Noi siamo di passaggio, lui rimane. (Pierre Rabhi)}\end{enfasiwide}\medskip

\begin{multicols}{2}

\label{yeru}

\lettrine[lines=2, lhang=0.33,loversize=0.25, findent=1.5em]{Y}{eru} è il pianeta di riferimento di DBS. Un pianeta spaccato sia fisicamente che magicamente.

Intorno a Yeru ruotano due stelle Sparka e Andhakara.\index{Sparka}\index{Andhakara}

Sparka è di un caldo colore dorato è colei che porta calore e luce, attorno a lei Yeru fa un giro completo in 336 giorni da 24 ore l'uno.

Sparka illumina sempre e solo l'emisfero nord di Yeru, chiamato Curyan \index{Curyan}.

Andhakara illumina sempre e solo l'emisfero sud di Yeru, Tiya, ed è invece una stella azzurra e fredda, priva di vita, è colei che porta tempeste energetiche e strani accadimenti naturali. Porta una fredda penombra.

Se le 14 (06-20) ore diurne vedono Sparka e Andhakara protagoniste in questa loro danza nel cielo; le 10 ore notturne vedono come totali protagoniste le due lune di Yeru di nome Idam e Kenatu.

Gli abitanti di Yeru le chiamano le loro lune anche se in realtà non sono propriamente solo lune ma veri e propri pianeti abitati.

Le due lune sono grandi ed imponenti sul cielo notturno, Idam di un colore grigio rossastro e Kenata di un caldo grigio madreperlato comandano le maree e influenzano con la loro presenza la navigazione.

Yeru ha una distribuzione delle terre peculiare ed unica, frutto del capriccio degli Dei della Genesi (Ljust e Calicante), potete immaginarlo come un sistema speculare sull'equatore.

Le terre non si uniscono all'equatore, lasciando circa 200 km di mare aperto.

Le terre emerse che compongono emisfero nord ed emisfero sud sono fra loro quasi simmetriche e simbiotiche. Forma e suddivisione delle grandi isole sono fra loro molto similari. Ma dal punto di vista climatico ci sono profonde diversità.

La zona di mare aperto di confine è selvaggia ed imperscrutabile. Le più profonde e potenti tempeste scaricano di continuo la loro energie ed anche la magia non riesce a penetrare. Nell'occhio di questo perenne e gigantesco maelstrom c'è la civilizzata e potentissima Alantia, da molti ritenuta una isola leggendaria e culla della civiltà.

Molte zone di Yeru sono ancora non mappate e inesplorate, il caos primigenio governa queste zone e ogni cosa diventa possibile. 

Poche sono le città che superano i 50000 abitanti. Ogni stato ha una capitale che per il beffardo destino di Yeru molto spesso viene distrutta o scompare. La legge è spesso assente e solo quella del più forte vige.


Ampie lande di si dipanano dove antichi resti di civiltà scomparse sono rifugio di nuovi abitanti. Strati su strati di civiltà storiche sotto i propri piedi con tesori, segreti, caverne e protettori.

Curyan è governata dalla forza della vita, questa regione vive una sorta di perenne calda stagione con gradazioni di temperatura e fenomeni atmosferici che variano a seconda della latitudine.

Si incrociano zone dal clima torrido e umido ad altre con un caldo secco e senza precipitazioni; esistono fenomeni quali tempeste di sabbia nelle zone desertiche e tempeste tropicali forti e devastanti nelle lussureggianti baie centrali.
Ci sono territori piacevolmente caldi ed altri rinfrescati da brezze fresche provenienti dai ghiacciai del nord.

Tiya invece è un emisfero semi avvizzito, la luce che arriva basta appena a permettere l'agricoltura e gli animali d'allevamento hanno un aspetto pallido ed emaciato.

La zona più ricca è quella più prossima all'equatore dove solo di poco si attenua la fredda luce di Andhakara fa spazio a qualche raggio di Sparka.
In questa stretta fascia l'agricoltura è più fiorente e ci sono meno fenomeni meteorologici devastanti.

E' l'emisfero dove vige la legge del più forte, dove si lotta per vivere e pochi sono gli stati che hanno un sistema di protezione efficace.

Il mare che abbraccia l'equatoriale è forte e tumultuoso, pochissime barche si avventurano da un continente all'altro, questo porta che scambi tra Tiya e Curyan siano quasi nulli via mare, solo pochissimi capitani, e di nascosto, osano attraversare il maelstrom.

\subsection{Avventure in Yeru}\index{Avventure in Yeru}

Il "problema" per gli avventurieri ed esploratori è l'estrema diversificazione e mutevolezza non solo dell'ambiente ma anche delle culture e civiltà che si possono incontrare.
Dato che ogni mille anni qualcosa di molto importante cambia è possibile che intere isole scompaiano o appaiano con civiltà una volta dimenticate, intere città siano improvvisamente sommerse dall'acqua o vegetazione, o peggio ancora interi imperi diventino belligeranti di non morti (si, è capitato anche questo, assieme ad un paio di apocalisse zombi durate diversi secoli).

Non si è mai certi di quello che ci può trovare a Yeru!.
Oltretutto gli accadimenti peggiori sono quelli che colpiscono la ricca e vitale Curyan mentre eventi più positivi prendono Tiya.

Yeru non si potrà mai dire esplorato, la stessa zona può cambiare da un giorno all'altro perché un Patrono ha deciso così. Curiosi, ineffabili, volubili sono capaci di costruire in un battito di mani l'avventura della vita solo per godersi lo spettacolo.

Che tu possa essere di Curyan o Tiya la tua vita non sarà facile, nulla ti verrà regalato. I giovani di entrambi i continenti scappano dalla miseria, soprusi, violenza per intraprendere una nuova vita ancora piu' ricca di miseria, soprusi e violenza ma che almeno è solo la loro, frutto delle proprie scelte.

Quando hai deciso di intraprendere questa nuova carriera, oppure sei stato strappato alla precedente, sapevi che non sarebbe stato facile, che Yeru stesso, tramite i suoi Patroni, avrebbe fatto di tutto per sconfiggerti ed umiliarti, ma la Legge del Premio e' superiore anche ai Patroni e avresti avuto il tuo premio anche se cosa rimaneva un incognita.


\subsection{Luoghi notevoli di Yeru}

\subsubsection{Deserto di Kranguran}

In questo immenso deserto si celano giganteschi mostri. Alcuni nascosti sotto la sabbia come enormi dinosauri usano il senso tellurico per cacciare le lore prede.

Ogni creatura in questo deserto e' gigantesca, mostruosa e sproporzionata nell'aspetto, quasi fosse nato dall'incubo di qualcuno.

La stessa vegetazione nelle poche oasi presenti e' enorme e ipertrofica.

\subsubsection{Città di Knandir}

Questo ricca, prosperosa e popolosa citta' antica fu distrutta nell'arco di una notte da un gigantesco cataclisma.

Si dice che il volere di Cattalm fu talmente pervasivo che tutti gli edifici vennero distrutte o severamente danneggiate. Non soddisfatto dell'opera condanno la città a essere sfasata rispetto alla realtà, facendola scomparire agli occhi di tutti gli altri.

I pochi abitanti sopravvissuti perirono tra atroci sofferenze, condannati a non poter uscire, a non aver da magiare o bere.

La città venne maledetta e nei pochi giorni dell'anno in cui e' possibile raggiungerla ogni persona che ci mette piede per saccheggiare gli immensi tesori contenuti sembra condannato a non uscire piu, vittima della maledizione o dei numerosi fantasmi, spiriti e non morti dei precedenti abitanti.

La città oltretutto non appare mai nello stesso posto ma si sposta seguendo uno schema non ben compreso. Si dice che la Pergamena perduta di Knandir ne spieghi gli spostamenti.


\end{multicols}

\pagebreak

\subsection{I Portali}\index{Portali}

\begin{enfasiwide}{
Non aprire mai le porte a coloro che le aprono anche senza il tuo permesso. (Stanislaw Jerzy Lec)}\end{enfasiwide}\medskip

\begin{multicols}{2}

\label{i-portali}

In un mondo dove i trasferimenti marittimi non funzionano se non tra isola e isola dello stesso emisfero come pure l'incantesimo di Teletrasporto la capacità di usare portali per trasferire merci e persone ha preso piede in maniera significativa.

Questo proliferare di piccoli, grandi, duraturi o istantanei tunnel ha causato uno squarcio nel tessuto dimensionale di Yeru generando a sua volta un proliferare di tunnel spontanei più o meno grandi e duraturi.

E questi Portali sono la causa di tantissimi problemi sia a Tiya che a Curyan in quanto non solo legano i due emisferi ma collegano tutta Yeru ad altri mondi (o almeno così si pensa dato che pochi sono tornati per riferirlo..).

Ci sono portali conosciuti e stabili, fino ad ora, che collegano Tiya a Curyan, quasi tutto sotto il controllo, per non dire dentro il castello, di reali o potenti.

Ci sono zone dove più frequentemente si aprono portali ma la destinazione non è sempre certa.

Poi ci sono i portali dei draghi. I draghi non sono nativi di Yeru ma sono stati attirati da queste porte magiche, causando scompiglio e terrore a Tiya e Curyan.

I draghi hanno ben compreso la natura di Yeru e con la loro fine intelligenza e innata capacità di plasmare la magia hanno costruito i loro portali facendo venire centinaia di draghi. Tutti malvagi.

Si, su Yeru non ci sono draghi "buoni" se non con poche eccezioni.

Si è sempre cercato di distruggere i portali dei draghi, con sacrificio e sangue. Molti sono stati distrutti, altri sono stati generati. E' una guerra senza fine, l'unica che può unire i popoli e destini dei due emisferi.

\end{multicols}

\subsection{Draghi}\index{Draghi}
\label{draghi}

\begin{enfasiwide}{
		Oh maledetti possa Lynx chiudervi tutti i portali\\
		Oh assassini possa Sumkjir sterminarvi\\
		Oh devastatori possa Nedraf rompervi le ossa!\\
		(imprecazioni contro i Draghi)}\end{enfasiwide}\medskip

\begin{multicols}{2}

I Draghi non sono nativi di Yeru bensì arrivati poco meno di 300 anni fa portandosi dietro morte e distruzione sia per Curyan che per Tiya.

Tàhil, potente drago rosso, usando la magia dei viaggi interdimensionali voleva trovare nuovi tesori e terre da soggiogare purtroppo per lui si avventurò troppo nello spazio vuoto, dove anche la luce le stelle non arriva ma si flette e torna indietro.

In questi non luoghi venne soggiogato, dominato, la sua mente rifatta da esseri oltre l'umana comprensione, pura follia e chaos.

Morto, ricostruito, distrutto, riassemblato, annichilito, ricomposto un innumerevole volte del suo essere originario non rimase nulla, solo lucida aggressiva follia.

Quando questi esseri passarono ad un nuovo gioco Tàhil creò un nuovo portale magico, non avendo più una percezione o ricordo di casa questo lo portò su Yeru, un mondo diverso e ambiguo dove subito si scontro contro un Patrono, Lynx.

Il suo corpo era stato rifatto, riforgiato, ricomposto della stessa non materia, la sua mente un vortice di puro potere caotico.

Il Patrono, forse una divinità per quel mondo, si dimostrò un avversario facile e si divertì a scarificare il corpo di quella debole entità.

Ormai giunto al colpo finale il Patrono aprì un portale sotto di e si lascio cadere dentro chiudendolo immediatamente.

Tàhil comprese subito la magia usata e le potenzialità del mondo, come la sua magia fosse ancora più efficace.
Apri centinaia di portali richiamando in quel mondo ogni abominio avesse mai immaginato e.. draghi, tanti, migliaia, tutti i draghi oscuri che potesse mai sognare.

L'invasione di Yeru era iniziata. Immediatamente Tàhil ribadì' il suo potere e la sua leadership uccidendo in un solo giorno decine e decine di draghi di tutti i colori presenti (rosso, verde, blu, bianco, nero e viola).
Gli altri draghi su sottomisero al suo volere, rimanendo incatenati alla volontà di Tàhil.

Tutta Yeru venne devastata per oltre 60 anni dai draghi, pochi dei draghi perirono per mano degli avventurieri del mondo.

Con il massimo sforzo Gradh riuscì a coinvolgere i Patroni della Genesi e solo così Tàhil accetto un incontro.

La storia è nota, Calicante vide in Tàhil un arma diversa e sommamente potente per portare distruzione ed entropia in Yeru, Ljust vide in Deynos l'aiuto e la conoscenza di quelle arcane creature.
Si strinse una alleanza ed una finta tregua.

Calicante privo di un pò della follia Tàhil, ma non tutta, amava il chaos e la morte che che in tanti modi riusciva a portare.

Ljust ampliò i poteri di Deynos perché potesse trasformare i draghi oscuri catturati in draghi buoni.

Esattamente, draghi buoni.

Fino a quel momento l'unico, solo, drago dimostratosi sinceramente buono era Deynos, che per puro caso, se non per errore aveva usato un portale creato da Tàhil prima che questo si chiudesse.

Nessun altro drago buono è mai giunto su Yeru, i pochissimi presenti sono quelli che catturati sono stati trasformati da Deynos.

Tàhil è diventato un jolly, un arma di pura aggressività buttata nel mondo, perché alzi il livello di caos, vendetta, violenza di Yeru.
I draghi, quando non comandati da Tàhil, vanno in giro per Tiya e Curyan a distruggere, ammassare ricchezze e creare leggende.

Un drago è una creatura praticamente incontrastata, le più potenti balliste possono ferirli ma il loro soffio è morte certa.

Ogni qual volta Tàhil decide che è necessario rinforzare le sue truppe apre nuovi portali facendo giungere nuovi draghi da sottomettere e dominare.

E' una lotta impari per i poveri yeruiti, ogni volta che con estremo sacrificio un drago muore altri due arrivano. Dove sia la tana di Tàhil è un mistero, celato dal potere stesso di Calicante che neanche Ljust è riuscita a carpire.

La speranza è che prima o poi un leggendario gruppo di eroi possa scovare la sua tana ed uccidere il maggior nemico di Yeru.

Nota: purtroppo non è vero che i draghi hanno solo 300 anni, in realtà l'ultima vittoria del millennio ha fatto dimenticare che queste creature sono su Yeru da molti millenni e da altrettanto seminano distruzione, chaos e morte.

\subsubsection{I Colori dei Draghi}

Ogni Drago ha sue caratteristiche tipiche e peculiari.
Tutti i draghi su Yeru obbediscono a Tàhil, ciecamente e senza resistenza, almeno finché  vengono trasferiti su Yeru tramite un suo portale.

Ogni volta che Tàhil vuole convocare un drago apre un portale e da questo esce un drago, di colore casuale. Appena arrivato la magia di dominazione di Tàhil soggioga il drago che non può più ribellarsi ai suoi ordini.

Il drago conduce la vita che più gli "aggrada", solitamente questa contempla la distruzione di qualche città e centinaia di morti, solo quando richiamato da Tàhil interrompe le sue attività per volare dove richiesto. Oppure Tàhil può inviare un ordine mentale senza bisogno che il drago si sposti.
Negli ultimi 300 anni solo una volta sono stati convocati tutti i draghi, diversamente solo i più anziani e potenti vengono richiamati perché agiscano come luogotenenti di Tàhil presso il territorio.


Per i Draghi Tàhil è la loro divinità.

\subsubsection{Drago Nero} \index{Drago Nero}

I Draghi Neri sono violenti ed aggressivi, vivono in paludi e acquitrini e che generalmente governano come padroni indiscussi.

I Draghi Neri sono creature minacciose che hanno grandi corna curve in avanti.
La testa si collega ad un collo relativamente corto e ad un corpo da lucertola grossa e muscoloso.

Hanno ali piccolissime che si trovano sui lati, ma riescono comunque a volare grazie alla magia.
Hanno le zampe palmate per permettere loro di nuotare con maggiore facilità nelle zone paludose dove vivono.

I Draghi Neri tendono a fare le loro tane al centro della palude o acquitrino.
Considerano quel territorio il loro e nessuno può bagnarsi senza subire la loro ira.

Una tana di drago nero può essere un ammasso gigantesco di tronchi ma anche una caverna sotterranea sommersa d'acqua, se non il fondo di un lago.
Potendo respirare sott'acqua non si fanno preoccupazione su dove costruire la loro dimora.

La loro casa è sempre protetta da trappole e dai loro seguaci malvagi che gli portano cibo, possibilmente vivo.

L'ambiente dove vive un drago nero ne subisce i suoi effetti, vapori acidi, distruzione, corruzione sono immediatamente percepibili.

Il Drago Nero rappresentano i Tratti dell'egoismo e violenza odiando ogni forma di vita, compreso gli stessi draghi neri.

I Draghi neri hanno +1d6 nelle prove di magia con la Scuola di Necromanzia ed è immune all'acido.


\subsubsection{Drago Blu} \index{Drago Blu}

I Draghi Blu abitano tra le nuvole, volando (e levitando) tra le tempeste.

I Draghi Blu hanno un aspetto serpentiforme, allungato ed legante, con corna lunghe all'indietro.

La faccia di un Drago Blu è meno segnata da increspature e rimane liscia.
Sono gli unici draghi a non avere ali pur volando meglio di ogni altro drago.

La loro magica ma naturale capacità di volo unita al fatto di nutrirsi di elettricità ne fa creature prettamente volanti che quasi mai scendono a terra (e mai toccano terra considerandola impura e sporca!), preferiscono rimanere tra le nubi, specialmente tra quelle più scure e cariche di energia per nutrirsi

La tana del Drago Blu solitamente è tra i picchi più alti delle montagne possibilmente tanto alte da arrivare alle nubi. Questa non è mai coperta e spesso assomiglia a giganteschi nidi.

I Draghi Blu possono assimilare carne ma non vegetali, non traggono nutrimenti da ciò che mangiano avendo un metabolismo puramente elettrico.

Sono draghi sociali, che amano stare con i loro simili e sono molto protettivi con la loro prole.
Solitamente non si trova mai un nido da solo, ma interi altopiani dominati da decine di draghi.

Non vanno d'accordo con i draghi viola che disprezzano per la scelta di aver rinunciato al volo per vivere sottoterra.

I Draghi Blu padroneggiano l'elettricità e ne sono immuni ai danni (magici o meno).


\subsubsection{Drago Verde} \index{Drago Verde}

I Draghi verdi amano le foreste e la natura incontaminata dove si reputano i padroni e re indiscussi.

I potenti draghi verdi hanno la testa tondeggiante e pronunciate orecchie all'indietro, le corna sono corte e non appuntite.
Gli artigli e le fauci sono devastanti, potenti e capace di tranciare qualsiasi cosa.
Il naso è largo e le narici aperte come se dovesse soffiare in qualsiasi momento.

Il soffio dei draghi verde è veleno, così che possa uccidere le creature viventi ma non le piante.

La tana di un drago verde è sempre vicino ad una sorgente d'acqua, possibilmente nella parte più lussureggiante ed incontaminata della foresta.

Un Drago verde non ama volare e preferisce saltare schiacciando con il suo peso e dilaniare con i suoi artigli.

Tra i tanti draghi quello verde è forse quello che farà parlare gli avventurieri se si dimostrano rispettosi ed impauriti dalla sua regalità.

I Draghi Verdi padroneggiano i Veleni e ne sono immuni sia a quelli magici che naturali.

\subsubsection{Drago Bianco} \index{Drago Bianco}

I Draghi Bianchi sono tra i più selvaggi e "animali" di tutti i draghi.
Amano i posti freddi e ghiacciati, trovando rifugio nelle valli più fredde come i picchi ghiacciati delle montagne e le steppe gelide.

I Draghi Bianchi hanno un aspetto selvaggio quasi sempre mostrano i denti e gli artigli sono estratti per muoversi agilmente sul terreno ghiacciato.
Non hanno penalità di movimento su questi terreni.

Sfruttano il loro naturale camuffamento per aggredire e catturare le prede, sono ottimi cacciatori, molto intelligenti nello sfruttare l'ambiente.

Poco inclini alla magia sanno però soffiare schegge di ghiaccio molto più frequentemente di altri draghi. E' immune gli attacchi basati sul freddo e ghiaccio.

Le loro tane sono caverne ghiacciate nelle montagne o scavate nei ghiacciai più massicci.

I Draghi Bianchi padroneggiano il ghiaccio e ne sono immuni sia a quello magico che naturale.


\subsubsection{Drago Porpora} \index{Drago Porpora}

I Draghi Porpora vivono sotto terra e si sono perfettamente adattati alla vita sotterranea.
Capaci di vedere al buio come se fosse pieno giorno, dotati di Senso Tellurico, hanno perso la capacità di volare ma acquisito quella di scavare con la stessa velocità come se corressero.

Un Drago Porpora è molto territoriale e stabilità un perimetro (di circa 5 km di raggio) crea, se non già presente un intricata serie di cunicoli e caverne per i suoi servi.

Un Drago Porpora è molto protettivo con le sue creature, con chi gli porta da mangiare e gli offre tesori.

Dall'aspetto serpentino hanno denti fini e artigli enormi che continuamente crescono.

E' forte e coraggioso, arrogante ma non sfrontato. Non ha paura di combattere se pensa di vincere. Porta sempre la battaglia sottoterra dove può creare fosse per fare precipitare i nemici o scappare se necessario.

Un Drago Porpora soffia un potentissimo attacco sonico che spesso crea crolli nelle caverne, crolli che sono completamente indifferente a lui. E' immune agli attacchi sonori.


\subsubsection{Drago Giallo} \index{Drago Giallo}

I Draghi Gialli hanno squame di vari toni di giallo che con la crescita prendono ad assomigliare sempre di più al colore delle sabbie dove dimorano, dal giallo chiaro all'ocra mattone.

Sono molto intelligenti ma essendo per natura solitari non hanno interesse a comunicare con le altre razze.

Vivono nei deserti dove spesso tendono agguati alle loro prede nascondendosi sul fondo di ampie buche scavate nella sabbia.
Appena percepiscono un movimento sopra di loro escono e divorano qualunque creatura.
Hanno una passione per la carne dei nani che trovano saporita anche se asciutta.

Il Drago Giallo pur se intelligente è una macchina di morte e difficilmente scende a patti, solo se si trova in serio pericolo.

Ha una naturale resistenza al danno (dimezzano) al fuoco ed alle armi non magiche.

Un Drago Giallo ha un soffio rovente, anche se non propriamente fuoco. E' immune agli attacchi di Fuoco.


\subsubsection{Drago Rosso} \index{Drago Rosso}

Il Drago Rosso si crede il Re dei Draghi per via della sua potenza fisica e del soffio capace di sciogliere la pietra.

I Draghi Rossi sono i draghi più grandi sia per corporatura che per apertura alare.
Spesso le scaglie, di un rosso scuro quasi di sangue, hanno bordi affilati ed allungati.

I Draghi Rossi prediligono le montagne calde e se possibile direttamente direttamente dentro un vulcano.

Combattono sfruttando la loro mole le ali il morso artigli.. insomma tutto ciò che sono ed hanno a disposizione. Un Drago Rosso combatte sempre fino alla morte non si ritira ne scappa ne rinuncia ad una sfida, l'orgoglio di cui sono tronfi non gli permette di mostrarsi deboli.

Un drago rosso è immune al fuoco naturale e magico.


%Tàhil rosso
%Dyenos argento 
%Curyan vita
%Tiya scuro


\end{multicols}

\subsection{Il Calendario}\index{Calendario}

\begin{enfasiwide}{
Mi è capitato spesso di finire su un calendario. Ma mai per una data precisa. (Marilyn Monroe)
		
Tutto ebbe inizio la tredicesima ora del tredicesimo giorno del tredicesimo mese... Eravamo lì per discutere degli errori di stampa dei calendari acquistati dalla scuola. (I Simpson)} \end{enfasiwide}\medskip


\begin{multicols}{2}

\label{il-calendario}

Basato sul ciclo presenta 12 mesi da 28 giorni.

Questi i nomi dei mesi a partire da quello che si definisce inizio anno
\bigskip

1°) Ianas

2°) Prineva

3°) Marc

4°) Epral

5°) Meea

6°) Vernam

7°) Ilai

8°) Arkast

9°) Cester

10°) Koper

11°) Narava

12°) Kartan

\bigskip
La settimana è a sua volta diviso in 7 giorni di nome

Kalint

Iratam

Munrat

Arai

Venran

Kittam

Viltar


Il giorno è diviso in 24 ore

\end{multicols}

\subsection{I cicli millenari}\index{I cicli millenari}

\begin{enfasiwide}{
		Vidi poi un angelo che scendeva dal cielo con la chiave dell'Abisso e una gran catena in mano.
		
		Afferrò il dragone, il serpente antico - cioè il diavolo, satana - e lo incatenò per mille anni; 
		
		lo gettò nell'Abisso, ve lo rinchiuse e ne sigillò la porta sopra di lui, perché non seducesse più le nazioni, fino al compimento dei mille anni. Dopo questi dovrà essere sciolto per un pò di tempo. (Apocalisse 20,1-3, apostolo Giovanni)
	}\end{enfasiwide}\medskip


\begin{multicols}{2}

Dice il mito che ogni mille anni il Yeru muoia per rinascere nuovamente, più bello di prima.

Non è proprio così ma ci si avvicina molto.

E' noto a pochi eruditi di Atmos che ogni mille anni i Patroni riconosciuti e da cui molti traggono i poteri scompaiano e lascino il posto, dopo esattamente 1 anno a nuovi Patroni.

Improvvisamente gli incantesimi cessano di funzionare, solo gli oggetti magici che possono assorbire e conservare la magia funzionano (come ad esempio una Pozione od un Bastone che abbia delle cariche, ma non oggetti che si ricaricano automaticamente), neanche i Devoti o Seguaci non hanno più accesso a nessuna scuola di magia.

Con qualche eccezione. I Patroni della Genesi, Atmos e Lynx ed il Vincitore sono gli unici a rimanere costanti a non mutare e solo i loro Devoti e Seguaci possono continuare ad usare le Scuole conosciute.

A partire dal sesto mese i vecchi Seguaci e Devoti incominciano a sentire delle voci, a sognare nuovi volti e nuovi Patroni.

Ogni nuovo Patrono, in base ai Tratti che comanda, avvicina un credente e cerca di convincerlo ad accettarlo come nuovo Patrono.

Questo Seguace/Devoto dovrà avere almeno due Tratti in comune con il nuovo Patrono per essergli Seguace ed almeno 3 per rimanergli Devoto (come sempre).

Gli incantatori solo al termine dell'anno potranno usare gli incantesimi, indipendentemente che seguano un Patrono o meno.

E' un periodo estremamente turbolento ed agitato dove scoppiano guerre e vendette approfittando dell'assenza della magia. Per molti è un periodi di puro odio e violenza dove vengono sfogati gli istinti più bassi sapendo poi che non si sarà giudicati da nessuna divinità.

La verità è che ogni mille anni i Patroni delle Genesi giudicano le lore creature, i Patroni, valutando chi ha fatto meglio e chi peggio. E' una sfida tra Calicante ed LJust a chi ha, tramite i Patroni, ottenuto più Seguaci e Devoti.

Il Patrono che più di tutti si è dimostrato capace di mantenere e conquistare persone rimarrà anche nel millennio successivo, questo sarà il Vincitore ed i suoi credenti ne canteranno per altri mille anni la gloria e potenza.

Inebriato dalla vittoria il Patrono della Genesi esprimerà un desiderio che l'altro dovrà cercare di rispettare il più possibile. Ovvio che lui/lei stessa potrebbe manifestarlo ma la soddisfazione di fare fare all'altro qualcosa che detesta è superiore a ogni cosa.

Ed è per questo che ogni mille anni succede qualcosa, oltre alla nascita di nuovi Patroni. Può essere un continente, mare.. luna, animali... qualcosa di imponente cambia per tutti gli yeruiti. E' un periodo di sconvolgimenti globali.

Solo i sommi Devoti di Atmos conoscono questa verità come conoscono che i Patroni della Genesi dopo la vittoria giacciono insieme per sei mesi generando i nuovi Patroni.

\bigskip

Per il Narratore: valutate bene quando fare incominciare le vostre campagne, in base alla durata ed all'anno potreste incorrere in questi accadimenti.
Sfruttate a vostro favore e beneficio di avventura il mutamento dei Patroni, fate giocare un pò del "riposo" dalla magia, aiutate i giocatori con personaggi più magici a riprendersi.

\end{multicols}

\pagebreak

\section{I Piani}

\begin{multicols}{2}

\lettrine[lines=2, lhang=0.33, loversize=0.25, findent=1.5em]{A}{nche} se avventure infinite vi attendono nel gioco, ci sono altri mondi oltre questo, altri continenti, altri pianeti, altre galassie. Tuttavia anche oltre l'esistenza di innumerevoli pianeti esistono altri mondi, dimensioni completamente differenti dalla realtà, conosciuti come piani di esistenza. Tranne per rari punti di collegamento che permettono di viaggiare tra loro, ogni piano è un universo a sé con le sue proprie leggi naturali.


Sebbene il numero di piani sia limitato solo dall'immaginazione, essi possono ricondursi tutti a cinque tipi generali: il Piano Materiale, i Piani di Transizione, i Piani Interni, i Piani Esterni e gli innumerevoli semipiani.


\subsection{Cos'è un Piano?}

I piani di esistenza in realtà sono differenti collegamenti intrecciati. Eccetto per rari punti di collegamento, ogni piano è effettivamente un universo fine a se stesso, con le proprie leggi naturali. I piani si suddividono in una serie di tipologie generali: il Piano Materiale, i Piani di Transizione, i Piani Interni, i Piani Esterni e i Semipiani.


\textit{Piano Materiale}: Il Piano Materiale tende ad essere quello più simile alla terra e a funzionare facendo uso delle stesse regole naturali. La sua "dimensione" dipende dalla campagna: si può conformare solo al mondo di gioco effettivo, o comprendere un intero universo di pianeti, lune, stelle e galassie. Il Piano Materiale è il piano di base per il gioco.

\textit{Piani di Transizione}: Questi piani hanno un importante elemento in comune: sono tutti sovrapposti agli altri piani e servono a viaggiare tra realtà in sovrapposizione. Questi piani sono fortemente interconnessi col Piano Materiale, ed è possibile accedervi usando numerosi incantesimi. Sono dotati anche essi di abitanti nativi. Qui di seguito sono descritti alcuni Piani di Transizione.

\begin{itemize}
	\item
	\textit{Piano Astrale}: Un vuoto argenteo che connette il Piano Materiale ai Piani Interni e ai Piani Esterni, il Piano Astrale è il mezzo attraverso cui le anime dei defunti giungono all'aldilà. Un viaggiatore nel Piano Astrale vede il piano come un infinito vuoto periodicamente punteggiato da minuscoli sprazzi di realtà fisica distaccatisi dagli innumerevoli piani sovrapposti. Incantatori potenti utilizzano il Piano Astrale per una breve frazione di secondo quando si teletrasportano, o possono usarlo per viaggiare tra i piani con incantesimi come Proiezione Astrale.
	
	\item
	\textit{Piano Etereo}: Il Piano Etereo è una dimensione nebulosa e nascosta coesistente col Piano Materiale e il Piano delle Ombre. I viaggiatori che attraversano il Piano Etereo sperimentano il mondo reale come fosse insostanziale e si possono muovere tra gli oggetti solidi senza essere visti nel mondo reale. Creature bizzarre abitano il Piano Etereo, così come fantasmi e sogni, molte delle quali possono a volte estendere la loro influenza nel mondo reale in modi misteriosi e terrificanti. Incantatori potenti utilizzano il Piano Etereo con incantesimi come Forma Eterea, Intermittenza e Transizione Eterea.
	
	\item
	\textit{Piano delle Ombre}: Il misterioso e mortale Piano delle Ombre è una versione grigia e priva di colori del Piano Materiale. Si sovrappone al Piano Materiale ma è più piccolo, ed è per molti versi un "riflesso" distorto e perverso del Piano Materiale, infuso di energia negativa (vedi Piani Interni) e abitato da terribili mostri come ombre o creature ancora peggiori. Incantatori potenti utilizzano il Piano delle Ombre per percorrere rapidamente immense distanze sul Piano Materiale con Camminare nelle Ombre, o attingono all'essenza mutevole del piano per creare effetti e mostri quasi reali con gli incantesimi Ombra di una Evocazione o Ombre.
	
\end{itemize}

\medskip

\textit{Piani Interni}: Questi piani sono le incarnazioni degli elementi base che costruiscono l'universo. Possono essere visti come "contenenti" il Piano Materiale, ma non sono ad esso sovrapposti come i Piani di Transizione. Sono composti da un unico tipo di energia o di elemento, superiore a tutti gli altri. Gli stessi abitanti di uno specifico Piano Interno sono composti dall'elemento del piano. Tra i Piani Interni ci sono:


\medskip

\begin{itemize}
	\item
	Piani Elementali: I quattro classici Piani Interni sono Piano dell'Acqua, Piano dell'Aria, Piano del Fuoco e Piano della Terra. Da questi piani provengono le creature note come elementali, ma sono abitati anche da altre bizzarre creature, come geni, xorn, mephit e cacciatori invisibili.
	
	\item
	Piani di Energia: Esistono due piani di energia, Il Piano dell'Energia Positiva (da cui provengono le scintille vitali) ed il Piano dell'Energia Negativa (da cui proviene la corruzione della non morte). L'energia di entrambi i piani è infusa nella realtà, ed il flusso di questa energia scorre in tutte le creature dalla nascita alla morte. I Devoti utilizzano il potere di questi due piani quando Incanalano Energia.
	
\end{itemize}

\medskip
\end{multicols}
\pagebreak
\begin{center}
\includegraphics[width=0.96\linewidth]{immagini/mappaplanare3.png}\\
\medskip
\textit{Mappa Planare. Concessa in licenza dall'autore.\\  https://www.reddit.com/r/ImaginaryGolarion/comments/97rog0/pathfinder\_map\_to\_the\_planes}		
\end{center}
\pagebreak
\begin{multicols}{2}

\textit{Piani Esterni}: Oltre i regni mortali, oltre gli elementi della realtà, ci sono i Piani Esterni. Vasti oltre ogni immaginazione è ad essi che giungono le anime dei morti ed è qui che dimorano le divinità. Ognuno di essi ha un suo insieme di Tratti, che rappresenta un aspetto morale o etico particolare, e i loro abitanti tendono a comportarsi seguendo questi Tratti.

I Piani Esterni sono anche il luogo del riposo finale degli spiriti provenienti dal Piano Materiale, sia che siano destinati a una calma introspezione che alla dannazione eterna. Gli abitanti dei Piani Esterni formano le mitologie delle civiltà, comprendendo angeli e demoni, titani e diavoli, e innumerevoli altre incarnazioni del possibile. Ogni mondo di gioco dovrebbe avere Piani Esterni diversi che si conformino ai temi e alle necessità specifiche, ma i classici Piani Esterni includono il Paradiso (Legale Buono), l'Abisso (Caotico Malvagio), l'Inferno (Legale Malvagio) e l'Elysium (Caotico Buono). Incantatori potenti possono entrare in contatto con i Piani Esterni per guida e consiglio con incantesimi come Comunione e Contattare Altri Piani, o possono evocare alleati con Evoca Mostri e Evoca Alleato Naturale.

\textit{Semipiani}: Questa categoria serve a raccogliere tutti gli altri spazi extradimensionali che funzionano come i piani ma che hanno accesso e dimensioni misurabili e limitate. Gli altri tipi di piani hanno in teoria dimensioni infinite, ma un semipiano potrebbe essere lungo anche soltanto poche centinaia di metri. Ci sono innumerevoli semipiani alla deriva nella realtà, e mentre molti sono connessi al Piano Astrale e al Piano Etereo, altri sono tagliati completamente fuori dai Piani di Transizione e sono raggiungibili solo attraverso portali ben nascosti o magie oscure.


\subsection{Piani a Più Strati}
L'infinito può essere ripartito in infiniti più piccoli e i piani in piani più piccoli correlati tra loro. Questi strati sono a tutti gli effetti dei piani di esistenza separati, e ogni strato può avere le sue particolari caratteristiche. Gli strati sono collegati tra loro attraverso numerosi portali planari, vortici naturali, sentieri e confini mutevoli.


L'accesso a un piano a più strati da un'altra provenienza di solito avviene su uno strato specifico, il primo strato del piano, che può essere sia quello più in alto che quello più in basso, in base al piano in questione. Molti punti d'accesso fissi (come portali e vortici naturali) conducono fino a questo strato, che diventa il portale d'accesso verso gli altri strati del piano. Anche l'incantesimo \emph{Spostamento Planare} deposita l'incantatore sul primo strato del piano.


\subsection{Interazione Planare}
Due piani che sono separati tra loro non si sovrappongono e non si collegano direttamente l'uno all'altro. Sono come pianeti su orbite diverse. Il solo modo di spostarsi da un piano all'altro è attraversare un terzo piano, come un Piano di Transizione.


\textit{Piani Adiacenti}: Quei piani che si collegano gli uni agli altri in punti specifici vengono considerati adiacenti. Laddove si toccano esiste una connessione attraverso la quale i viaggiatori possono uscire da una realtà ed entrare nell'altra.

\textit{Piani Coesistenti}: Se è possibile creare in ogni punto un legame tra due piani, i due piani sono coesistenti. Questi piani si sovrappongono l'uno all'altro completamente. Un piano coesistente può essere raggiunto da qualsiasi punto del piano a cui è sovrapposto. Quando ci si muove su un piano coesistente, spesso si può vedere o interagire col piano ad esso sovrapposto.

\subsection{Caratteristiche Planari}
Ogni piano di esistenza ha le sue peculiarità; le leggi naturali del suo universo. Le caratteristiche planari si suddividono in aree generali. Tutti i piani hanno le seguenti caratteristiche.


\textit{Caratteristiche Fisici}: Determinano le leggi fisiche e naturali del piano, compreso il funzionamento della gravità e del tempo.

\textit{Caratteristiche Elementali ed Energetici}: L'influenza di forze elementali ed energetiche è determinata da questi tratti.

\textit{Tratti}: Proprio come i personaggi possono essere legali neutrali o caotici buoni, così molti piani sono legati ad una particolare morale o etica.

\textit{Caratteristiche Magiche}: La magia funziona in modo diverso da piano a piano; i tratti magici delimitano il confine tra ciò che la magia può fare e non può fare su ogni piano.

\textit{Caratteristiche Fisiche}

Le due più importanti leggi naturali determinate dai tratti fisici riguardano la funzione della gravità e del tempo. Altre caratteristiche fisiche riguardano la grandezza e la forma di un piano ed il modo con cui si possa alterarne la natura.


\subsection{Gravità}

La direzione di attrazione gravitazionale può essere inusuale, e potrebbe addirittura cambiare direzioni all'interno dello stesso piano.


\subsection{Tempo}

Il ritmo con cui trascorre il tempo può variare nei diversi piani, sebbene rimanga costante all'interno di un qualunque piano specifico. Il tempo è sempre soggettivo per lo spettatore. La stessa soggettività si applica ai vari piani. I viaggiatori potrebbero scoprire che stanno guadagnando o perdendo tempo muovendosi tra i piani, ma dal loro punto di vista il tempo trascorre in modo naturale.


\textit{Tempo Normale}: Definisce il trascorrere del tempo sul Piano Materiale. Un'ora su un piano caratterizzato da tempo normale equivale ad 1 ora sul Piano Materiale. A meno che non sia diversamente specificato nella descrizione di un piano, si presume che esso sia caratterizzato da tempo normale.
\textit{Tempo Irregolare}: Alcuni piani sono caratterizzati da un tempo che rallenta ed accelera, per cui un individuo potrebbe perdere o guadagnare tempo mentre si muove tra piani come questo ed altri. Per gli abitanti di un simile piano, il tempo trascorre in modo naturale e lo spostamento passa inosservato.


\textit{Tempo Fluente}: Su alcuni piani, il flusso temporale è considerevolmente più veloce o più lento. Qualcuno potrebbe viaggiare verso un altro piano, trascorrervi un anno e poi ritornare al Piano Materiale per scoprire che sono passati soltanto 6 secondi. Qualsiasi cosa del piano in cui si è tornati ha vissuto appena qualche secondo in più. Per il viaggiatore e gli oggetti, gli incantesimi e gli effetti in funzione su di lui, quell'anno di lontananza è stato completamente reale. Quando progettate il funzionamento del tempo su piani con tempo fluente, pensate per prima al flusso temporale del Piano Materiale, dopodiché al flusso presente nell'altro piano.


\textit{Assenza di Tempo}: Sui piani che presentano questa caratteristica il tempo trascorre ma i suoi effetti sono limitati. Il modo in cui l'assenza di tempo influenza determinate attività e condizioni quali la fame, la sete, l'invecchiamento, gli effetti del veleno e la guarigione varia da piano a piano. Il pericolo di un piano con assenza di tempo è che quando un individuo abbandona tale piano per giungere in un altro dove il tempo scorre normalmente, condizioni come la fame e l'invecchiamento si verificano con effetto retroattivo. Se un piano ha assenza di tempo in relazione alla magia, qualsiasi incantesimo lanciato con durata non istantanea diviene permanente finché dissolto.


\subsection{Caratteristiche Elementali ed Energetiche}

Quattro elementi base e due tipologie di energia si combinano per plasmare ogni cosa; gli elementi sono acqua, aria, fuoco e terra; le tipologie di energia sono positiva e negativa. Il Piano Materiale rispecchia un bilanciamento di questi elementi ed energie: in esso è possibile trovarle tutte. Ognuno dei Piani Interni è dominato da un elemento o tipologia di energia. Molti piani non hanno alcuna caratteristica elementale o energetica; tali caratteristiche vengono specificate nella descrizione di un piano solo se presenti.


\textit{Acqua Dominante}: I piani con questa caratteristica sono per lo più liquidi. I visitatori che non possono respirare sott'acqua o che non riescono a raggiungere una sacca d'aria probabilmente muoiono annegati. Le creature del Fuoco si trovano estremamente a disagio nei piani con acqua dominante. Tali creature, costituite di fuoco, subiscono 1d10 danni ogni round.

\textit{Aria Dominante}: Costituiti essenzialmente da spazio aperto, i piani con questa caratteristica ospitano giusto qualche pezzo di pietra fluttuante o di altro materiale solido. Di solito sono caratterizzati da atmosfera respirabile, sebbene un piano simile potrebbe presentare nubi di gas acido o tossico. Le creature del Sottotipo Terra si trovano a disagio sui piani con aria dominante vista la scarsa quantità o assenza di terra naturale con cui entrare in contatto. Tuttavia, esse non subiscono alcun danno effettivo.

\textit{Terra Dominante}: I piani con questa caratteristica sono per lo più solidi. I viaggiatori che vi giungono sono a rischio di soffocamento a meno che non raggiungano una caverna o un altro anfratto. Peggio ancora, gli individui che non possiedono la capacità di Scavare restano intrappolati sotto terra e devono scavare da sè una via d'uscita (1 metro per round). 
Le creature del Sottotipo Aria si trovano a disagio sui piani con terra dominante visto che li considerano angusti e claustrofobici, ma a parte avere difficoltà nei movimenti non incappano in altri inconvenienti.

\textit{Fuoco Dominante}: I piani con questa caratteristica sono costituiti da fiamme che bruciano continuamente senza esaurire la loro fonte di alimentazione. I piani con fuoco dominante sono estremamente ostili per le creature del Piano Materiale, e coloro che non hanno Resistenza o Immunità al fuoco vengono inceneriti in poco tempo. Legno, carta, stoffa senza protezione ed altri materiali infiammabili prendono fuoco quasi istantaneamente, così come coloro che indossano vestiario non protetto ed infiammabile. In aggiunta, gli individui subiscono 3d10 danni da fuoco per ogni round in cui restano in un piano con fuoco dominante. Le creature del Sottotipo Acqua si trovano estremamente a disagio sui piani con fuoco dominante. Tali creature, costituite d'acqua, subiscono il doppio del danno ogni round.

\textit{Energia Negativa Dominante}: I piani con questa caratteristica sono recessi vasti e vuoti che risucchiano l'essenza vitale dei viaggiatori che li attraversano. Tendono ad essere piani desertici e tormentati, spogliati del colore e riempiti da venti che trasportano i deboli lamenti di coloro che sono morti al loro interno. Esistono due tipi di tratti basati su energia negativa dominante: energia negativa dominante minore e superiore. Nei primi, le creature viventi subiscono 1d6 danni per round. A 0 Punti Ferita o meno, queste si riducono in cenere.

I secondi sono persino più pericolosi. Ogni round, coloro che ne sono all'interno devono effettuare un Tiro Salvezza su Tempra con DC 25 i punti ferita massimi calano di 6 se muore in questa maniera diventa un Wraith. L'incantesimo Interdizione alla Morte protegge il viaggiatore dal danno e dal risucchio d'energia di un piano con energia negativa dominante.

\textit{Energia Positiva Dominante}: L'abbondanza di vita contraddistingue i piani che presentano questa caratteristica. Come per i piani con energia negativa dominante, anche i piani con energia positiva dominante possono essere minori e superiori.
Un piano con energia positiva dominante minore è una tumultuosa esplosione di vita in tutte le sue forme. I colori sono più luminosi, i fuochi più caldi, i rumori più forti e le sensazioni più intense grazie all'energia positiva diffusa nel piano. Tutti gli individui in un piano con energia positiva dominante rigenerano 2 Punti Ferita a round.

I piani con energia positiva dominante superiore vanno persino oltre. Una creatura su uno di questi piani deve effettuare un Tiro Salvezza su Tempra con DC 15 per evitare di rimanere Accecata per 10 round dalla luminescenza dei dintorni. Il semplice fatto di trovarsi sul piano conferisce rigenerazione 5pf a round. In aggiunta, coloro che hanno il massimo dei propri Punti Ferita guadagnano 5 Punti Ferita Temporanei aggiuntivi per round. Tali Punti Ferita Temporanei svaniscono 3d6 round dopo che la creatura ha lasciato il piano. Tuttavia, una creatura deve effettuare un Tiro Salvezza su Tempra con DC 20 per ogni round che questi Punti Ferita Temporanei superano i suoi normali Punti Ferita totali. Fallire questo Tiro Salvezza fa sì che la creatura esploda in un tripudio d'energia, morendo.


\subsection{Tratti}

Alcuni piani hanno una predisposizione verso uno specifico insieme di Tratti. Gli abitanti di questi piani condividono per lo più tale insieme di Tratti o parte di questi, persino creature potenti come le divinità. L'insieme di Tratti di un piano ne influenza le interazioni sociali. I personaggi che hanno dei Tratti diversi da quelli della maggior parte degli abitanti potrebbero avere delle difficoltà quando si confrontano con i nativi e le situazioni del piano. I Tratti hanno molteplici componenti. Prima di tutto ci sono le componenti morali ed etiche. In secondo luogo potrebbe esserci una specifica indicazione se questo insieme di Tratti si manifesta in maniera moderata o in modo più marcato. Molti piani non hanno Tratti; quest'ultimi sono specificati nella descrizione di un piano solo se presenti


\subsection{Caratteristiche Magiche}

Le caratteristiche magiche di un piano definisce la magia su quel piano rispetto al Piano Materiale. In luoghi particolari di un piano (come quelli sotto il diretto controllo delle divinità) potrebbero applicarsi una diversa caratteristica magica.


\textit{Magia Normale}: Questa caratteristica magica implica che tutti gli incantesimi e le capacità soprannaturali agiscano come da descrizione. A meno che non sia diversamente descritto, un piano si presume che abbia il tratto magia normale.

\textit{Magia Morta}: Contraddistingue i piani dove la magia non esiste affatto. Un piano con la caratteristica magia morta funziona sotto tutti gli aspetti come un Campo Antimagia. Gli incantesimi di Divinazione non possono individuare qualcuno che si trovi in un piano di magia morta, né un incantatore può utilizzare l'incantesimo Teletrasporto per muoversi dentro e fuori di esso. L'unica eccezione alla regola "assenza di magia" è rappresentata dai portali planari permanenti, che funzionano comunque normalmente.

\textit{Magia Potenziata}: Sui piani con questa caratteristica magica particolari incantesimi e capacità magiche sono più facili da usare o producono effetti più potenti rispetto a come operano nel Piano Materiale. I nativi di un piano sono consapevoli di quali incantesimi e capacità magiche siano potenziate, ma i viaggiatori planari potrebbero scoprirlo di loro iniziativa. Se un incantesimo è potenziato, esso funziona come se avesse fatto un critico nella prova di magia.

\textit{Magia Ostacolata}: Particolari incantesimi e capacità magiche sono più difficili da utilizzare sui piani con questa caratteristica, spesso perché la natura del piano li ostacola. Per lanciare un incantesimo ostacolato, l'incantatore deve riuscire a battere di 10 la Difficoltà dell'incantesimo. Se la prova fallisce, l'incantesimo non ha effetto ma viene comunque sprecato. Se la prova riesce, l'incantesimo ha effetto normalmente.

\textit{Magia Limitata}: I piani con questa caratteristica consentono il solo uso di incantesimi e capacità magiche che soddisfino particolari requisiti. La magia può essere limitata nei suoi effetti da determinate scuole o sottoscuole, da effetti con certi descrittori o da effetti di un dato livello (o da qualsiasi combinazione di questi aspetti). Gli incantesimi e le capacità magiche che non soddisfano i requisiti semplicemente non hanno effetto.

\textit{Magia Selvaggia}: Su un piano con la caratteristica di magia selvaggia, incantesimi e capacità magiche funzionano in modo totalmente diverso e a volte pericoloso. C'è la possibilità che qualsiasi incantesimo o capacità magica utilizzata su un piano di magia selvaggia non abbia effetto. Quando l'incantatore lancia una magia deve effettuare due prove se anche solo una fallisce comporta che accada qualcosa di insolito; tirate un d100 e consultate la

\medskip

\end{multicols}

\vspace{5cm}

\textbf{Tabella: Effetti della Magia Selvaggia.}\index{Tabella Effetti della Magia Selvaggia}

\medskip


\begin{tabularx}{0.95\textwidth}{lX}
	d100	&Effetto\\
	\toprule
	01-19	&L'incantesimo rimbalza sull'incantatore con effetto normale. Se l'incantesimo non può influenzare l'incantatore, non produce alcun effetto.\\
	20-23	&Una fossa circolare del diametro di 4,5 metri si apre sotto i piedi dell'incantatore; la sua profondità è di 3 metri per livello dell'incantatore.\\
	24-27	&L'incantesimo non ha effetto, ma il bersaglio o i bersagli di quest'ultimo vengono colpiti da una pioggia di piccoli oggetti (qualsiasi cosa, dai fiori alla frutta rancida), che scompaiono non appena hanno colpito. L'attacco continua per 1 round. Durante questo periodo i bersagli sono Accecati e devono effettuare prove di Concentrazione (DC 15 + il livello dell'incantesimo) per lanciare incantesimi.\\
	28-31	&L'incantesimo colpisce un bersaglio o un'area casuale. Scegliete in modo casuale un bersaglio differente fra quelli entro il raggio d'azione dell'incantesimo o centrate quest'ultimo in un luogo a caso che rientri in tale raggio d'azione. Per generare casualmente la direzione, tirate 1d8 e contate in senso orario, partendo da sud. Per generare casualmente il raggio d'azione tirate 3d6. Moltiplicate il risultato per 1 metro per gli incantesimi a corto raggio, 6 metri per quelli a medio raggio e 24 metri per quelli a lungo raggio.\\
	32-35	&L'incantesimo funziona normalmente, ma qualsiasi componente materiale non viene consumata. L'incantesimo non scompare dalla mente dell'incantatore (uno slot incantesimo o un incantesimo preparato possono ancora essere utilizzati). Allo stesso modo, un oggetto non perde cariche e l'effetto non influenza il limite di utilizzo di un oggetto o di una capacità magica.\\
	36-39	&L'incantesimo non ha effetto. Invece, qualcuno (amico o nemico) entro 9 metri dall'incantatore riceve l'effetto di un incantesimo Guarigione.\\
	40-43	&L'incantesimo non ha effetto. Invece, degli effetti di Oscurità Profonda e Silenzio coprono un raggio di 9 metri attorno all'incantatore per 2d4 round.\\
	44-47	&L'incantesimo non ha effetto. Invece, un effetto di Inversione della Gravità copre un raggio di 9 metri attorno all'incantatore per 1 round.\\
	48-51	&L'incantesimo ha effetto, ma dei colori scintillanti turbinano attorno all'incantatore per 1d4 round. Considerate quest'area come un effetto di Polvere Luccicante con un Tiro Salvezza con DC 10 + il livello dell'incantesimo che ha generato tale risultato.\\
	52-59	&Non accade nulla. L'incantesimo non ha effetto. Qualsiasi componente materiale viene utilizzata. L'incantesimo o lo slot incantesimo viene utilizzato, un oggetto perde cariche e l'effetto influenza il limite di utilizzo di un oggetto o di una capacità magica.\\
	60-71	&Non accade nulla. L'incantesimo non ha effetto. Qualsiasi componente materiale non viene utilizzata. L'incantesimo non scompare dalla mente dell'incantatore (uno slot incantesimo o un incantesimo preparato possono ancora essere utilizzati). Un oggetto non perde cariche e l'effetto non influenza il limite di utilizzo di un oggetto o di una capacità magica.\\
	72-98	&L'incantesimo ha effetto normalmente.\\
	99-100	&L'incantesimo ha effetto potenziato. La prova di magia genera automaticamente un critico\\
\end{tabularx}

\addvspace{2cm}

\begin{multicols}{2}


\subsection{Piano Materiale}
Il Piano Materiale è il fulcro della maggior parte delle cosmologie e definisce cosa può considerarsi normale. Si tratta del piano su cui si focalizzano gran parte delle campagne.\\
Il Piano Materiale presenta i seguenti tratti:\\
\textit{Gravità Normale}\\
\textit{Tempo Normale}\\
\textit{Nessun Tratto Elementale o Energetico}: Tuttavia, luoghi specifici potrebbero presentare tali tratti.\\
\textit{Moderatamente Neutrale}: Anche se in alcuni punti potrebbe presentare elevate concentrazioni di male o bene, legge o caos.\\
Magia Normale\\

\subsection{Piano delle Ombre}
Il Piano delle Ombre è una dimensione poco illuminata che allo stesso tempo coincide e coesiste con il Piano Materiale. Si sovrappone al Piano Materiale tanto quanto al Piano Etereo, per cui il viaggiatore planare può sfruttare il Piano delle Ombre per coprire grandi distanze rapidamente. Il Piano delle Ombre coincide anche con altri piani. Tramite l'incantesimo giusto, un personaggio può servirsi del Piano delle Ombre per visitare altre realtà. Il Piano delle Ombre è un mondo in bianco e nero: l'ambiente è privo di colori. Se non fosse per questo, somiglierebbe al Piano Materiale. Nonostante l'assenza di fonti luminose, alcune piante, animali e umanoidi, considerano il Piano delle Ombre come loro dimora.

Il Piano delle Ombre presenta le seguenti caratteristiche:

\textit{Geografia imperfetta}: Parti del Piano delle Ombre fluiscono continuamente verso altri piani. Dunque, nonostante la presenza di punti di riferimento, creare una mappa precisa del piano è quasi impossibile. Inoltre, alcuni incantesimi, come Ombra di una Evocazione e Ombra di una Invocazione, modificano la struttura di base del Piano delle Ombre. Questi incantesimi all'interno del Piano delle Ombre sono particolarmente utili sia per gli esploratori che per i nativi .

\textit{Tratti}: Indisciplinato, Libero, Superficiale, Vendicativo, Pessimista

\textit{Magia Potenziata}: Gli incantesimi che lavorano con l'ombra vengono potenziati sul Piano delle Ombre. Inoltre, specifici incantesimi diventano più potenti. Gli incantesimi Ombra di una Evocazione e Ombra di una Invocazione hanno il 30\% della potenza degli incantesimi che copiano (invece del 20\%). Ombra di una Evocazione Superiore e Ombra di una Invocazione Superiore hanno il 70\% della potenza degli incantesimi che copiano (invece del 60\%), mentre un incantesimo Ombre il 90\% (invece dell'8o\%). Nonostante la natura oscura del Piano delle Ombre, gli incantesimi che generano, utilizzano o manipolano l'oscurità non vengono influenzati dal piano.

\textit{Magia Ostacolata}: Gli incantesimi di luce o che utilizzino o generino luce o fuoco vengono ostacolati sul Piano delle Ombre. Gli incantesimi che generano luce sono meno efficaci in generale, dal momento che su questo piano tutte le fonti luminose hanno raggio d'azione dimezzato.


\subsection{Piano dell'Energia Negativa}
Per un osservatore c'è ben poco da vedere sul Piano dell'Energia Negativa. È un luogo buio e vuoto, una fossa infinita in cui il viaggiatore potrebbe precipitare finché il piano non abbia cancellato luce e vita. Il Piano dell'Energia Negativa è il più ostile fra i Piani Interni, il più indifferente ed intollerante nei confronti della vita. Soltanto le creature immuni ai suoi effetti di risucchio possono sopravvivere qui.

Il Piano dell'Energia Negativa presenta le seguenti caratteristiche:

\textit{Energia Negativa Dominante Superiore}: Alcune zone all'interno del piano hanno solo la caratteristica energia negativa dominante minore, ma queste isole tendono ad essere disabitate.

\textit{Magia Potenziata}: Incantesimi e capacità magiche che utilizzano l'energia negativa vengono potenziati. Le Abilità che sfruttano l'energia negativa, come Incanalare Energia negativa, ottengono bonus +4 alla DC del Tiro Salvezza per resistere alla capacità.

\textit{Magia Ostacolata}: Incantesimi e capacità magiche che utilizzano l'energia positiva (inclusi gli incantesimi di guarigione) vengono ostacolati. I personaggi su questo piano subiscono una Difficoltà di +10 nel lanciare incantesimi che curano o rimuovono effetti negativi.


\subsection{Piano dell'Energia Positiva}
Il Piano dell'Energia Positiva non ha superficie ed è simile al Piano dell'Aria con il suo spazio totalmente aperto. Tuttavia, ogni angolo di questo piano è illuminato vivacemente da una potenza innata. Tale potere è pericoloso per le forme mortali, non predisposte a subirlo. Nonostante gli effetti benefici è uno dei Piani Interni più ostili. Un personaggio sprovvisto di difese, traboccherà di potenza non appena l'energia positiva viene convogliata su di lui. Ma, visto che la sua forma mortale non è in grado di contenere tale potere, verrà incenerito, come un granello di polvere catturato all'estremità di una supernova. Le visite al Piano dell'Energia Positiva sono di breve durata, ed anche in tal caso i viaggiatori devono essere adeguatamente protetti.
Il Piano dell'Energia Positiva presenta le seguenti caratteristiche:

\textit{Energia Positiva Dominante Superiore}: Alcune regione del piano, invece, sono caratterizzate da energia positiva dominante minore, ma tali isole tendono ad essere disabilitate.

\textit{Magia Potenziata}: Incantesimi e capacità magiche che usano l'energia positiva vengono potenziati. Le Abilità che sfruttano l'energia positiva, come Incanalare Energia positiva, ottengono bonus +4 alla DC per resistere alla capacità.

\textit{Magia Ostacolata}: Incantesimi e capacità magiche che utilizzano l'energia negativa (inclusi gli incantesimi infliggi) vengono ostacolati.


\subsection{Piano Elementale dell'Acqua}
Il Piano dell'Acqua è un mare senza fondale o superficie, un ambiente liquido illuminato da una luce diffusa. E' uno dei Piani Interni più ospitali, una volta che il viaggiatore supera il problema di respirare sott'acqua. Gli infiniti oceani di questo piano spaziano tra il freddo gelido e il caldo incandescente e tra acqua salata e acqua dolce. Gli insediamenti permanenti del piano si generano attorno a pezzi di relitti sospesi in questo fluido senza fine, andando alla deriva con le maree.

Il Piano dell'Acqua presenta le seguenti caratteristiche:

\textit{Acqua Dominante}

\textit{Magia Potenziata}: Incantesimi e capacità magiche che usano l'acqua vengono potenziati.

\textit{Magia Ostacolata}: Incantesimi e capacità magiche che usano o creano fuoco e gli incantesimi che evocano elementali del fuoco o Esterni del Sottotipo Fuoco vengono ostacolati.


\subsection{Piano Elementale dell'Aria}
Il Piano dell'Aria è un piano vuoto, costituito da cielo in ogni direzione. Si tratta del più confortevole e vivibile dei piani interni ed è la dimora di tutti i generi di creature dell'aria. Infatti, le creature volanti ottengono grande vantaggio su questo piano. Sebbene i viaggiatori possano sopravvivere bene qui anche senza la capacità di volare, sono comunque svantaggiati.
Il Piano dell'Aria presenta le seguenti caratteristiche:

\textit{Aria Dominante}

\textit{Magia Potenziata}: Incantesimi e capacità magiche che usano, manipolano o generano aria vengono potenziati.

\textit{Magia Ostacolata}: Incantesimi e capacità magiche che usano o generano terra e gli incantesimi che evocano elementali della terra o Esterni del Sottotipo Terra vengono ostacolate.


\subsection{Piano Elementale del Fuoco}
Sul Piano del Fuoco ogni cosa è illuminata. Il suolo è costituito nient'altro che da vasti e mutevoli strati di fuoco compresso. L'aria viene smossa dal calore delle continue piogge di fuoco ed il liquido più comune è il magma. Gli oceani sono composti da fiamma liquida e le montagne fanno colare lava fusa. Qui il fuoco perdura senza alimentazione o aria, ma gli elementi infiammabili introdotti sul piano vengono consumati rapidamente.

Il Piano del Fuoco presenta le seguenti caratteristiche:

\textit{Fuoco Dominante}

\textit{Magia Potenziata}: Incantesimi e capacità magiche che utilizzano, manipolano o generano fuoco vengono potenziati.

\textit{Magia Ostacolata}: Incantesimi e capacità magiche che usano o generano acqua e gli incantesimi che evocano elementali dell'acqua o Esterni del Sottotipo Acqua vengono ostacolati.


\subsection{Piano Elementale della Terra}
Il Piano della Terra è un luogo solido composto da terra e pietra. Un viaggiatore imprudente potrebbe ritrovarsi seppellito da questa vasta massa solida: i suoi resti polverizzati resteranno di monito per chi oserà seguirlo. Nonostante la sua natura solida e rigida, il Piano della Terra ha consistenza variabile, spaziando da terreno soffice a vene di metallo più duro e prezioso.

Il Piano della Terra presenta le seguenti caratteristiche:

\textit{Terra Dominante}

\textit{Magia Potenziata}: Incantesimi e capacità magiche che utilizzano, manipolano o generano terra o pietra vengono potenziati.

\textit{Magia Ostacolata}: Incantesimi e capacità magiche che utilizzano o generano aria e gli incantesimi che evocano elementali dell'aria o Esterni del Sottotipo Aria vengono ostacolati.


\subsection{Piano Etereo}
Il Piano Etereo coesiste col Piano Materiale e spesso anche con altri piani. Lo stesso Piano Materiale è visibile dal Piano Etereo, ma appare silenzioso e indistinto; i colori si confondo fra loro e i confini sono sfocati. Sebbene sia possibile vedere il Piano Materiale dal Piano Etereo, quest'ultimo è di solito invisibile a coloro che si trovano sul Piano Materiale. Normalmente, le creature del Piano Etereo non possono attaccare quelle del Piano Materiale e viceversa. Un viaggiatore che si trovi sul Piano Etereo è invisibile, incorporeo e totalmente silenzioso per qualcuno del Piano Materiale.

Il Piano Etereo presenta le seguenti caratteristiche:

\textit{Assenza di Gravità}

\textit{Magia Normale}: Gli incantesimi funzionano normalmente sul Piano Etereo, anche se non attraversano il Piano Materiale. Le uniche eccezioni sono gli incantesimi e le capacità magiche e che influenzano le entità eteree.

Nessun attacco magico passa dal Piano Etereo al Piano Materiale, compresi gli attacchi di forza.


\subsection{Piano Astrale}
Il Piano Astrale è lo spazio tra piani interni ed esterni e confina con tutti i piani. Quando un personaggio attraversa un portale o proietta il suo spirito su un altro piano di esistenza, viaggia attraverso il Piano Astrale. Anche gli incantesimi che consentono movimento istantaneo attraverso un piano influenzano per poco il Piano Astrale. Quest'ultimo è una grande distesa infinita di limpido cielo argenteo, sia sopra che sotto. Occasionali pezzi di materia solida possono essere trovati qui, ma la maggior parte del Piano Astrale è uno spazio aperto e sconfinato.

Il Piano Astrale presenta le seguenti caratteristiche:

\textit{Assenza di Tempo}: L'età, la fame, la sete, le sofferenze (come Malattie, Maledizioni e Veleni) e la guarigione naturale non hanno effetti nel Piano Astrale, sebbene riprendano il proprio funzionamento quando il viaggiatore lascia il piano.

\textit{Magia Potenziata}: Tutti gli incantesimi e le capacità magiche usate nel Piano Astrale hanno velocità di 1 Azione. Gli incantesimi e le capacità magiche già velocizzati non vengono influenzati, così come gli incantesimi degli oggetti magici. Gli incantesimi velocizzati in tal modo sono comunque preparati e lanciati al loro livello originario.



\subsection{Altri Piani}

Quelli che seguono sono alcuni dei Piani più conosciuti (interni od esterni), molti altri ne esistono anche se solo pochi arditi sono riusciti a raggiungerli (spontaneamente!).

\subsection{Abaddon}
Reame di distese desolate sotto un cielo putrido, Abaddon è avvolto da una nera nebbia nauseante, e dall'opprimente crepuscolo di un'eclissi solare senza fine. Il veneficio Stige nasce in Abaddon, prima di immettersi come un serpente contorto negli altri piani. Abaddon è uno dei Piani Esterni più ostile: regno dei Daemon, immondi di male puro indifferenti al conflitto tra legge e caos, che rappresentano l'oblio e la distruzione. I Daemon governati da quattro arcidaemon con poteri simili a divinità, sono temuti come divoratori di anime.

Abaddon presenta le seguenti caratteristiche:

\textit{Tratti}: Distruttivo, Implacabile, Incontentabile, Irrazionale, Iracondo, Sadico

\textit{Magia Potenziata}: Gli incantesimi e le capacità magiche malvagie vengono potenziati.

\textit{Magia Ostacolata}: Gli incantesimi e le capacità magiche benevole bene vengono ostacolati.

\subsection{Abisso}
L'Abisso, piano multi-strato, circonda la Sfera Esterna come la buccia estesa di una cipolla; è formato da giganteschi canyon e gole che si spalancano nel tessuto dei Piani Esterni ed è delimitato dalle nefaste acque del Fiume Stige. Gli infiniti strati dell'Abisso, confinanti con tutti i Piani Esterni, sono collegati l'un l'altro da sentieri in costante spostamento. Nell'Abisso non ci sono regole, leggi, ordine o speranza. L'Abisso rappresenta la corruzione della libertà, un regno da incubo di orrore assoluto dove il desiderio e la sofferenza assumono forma demoniaca, terra di proliferazione di innumerevoli razze di Demoni, tra gli esseri più antichi di tutto il creato.

L'Abisso presenta le seguenti caratteristiche:

\textit{Tratti}: Anarchico, Vendicativo, Permaloso, Arrogante, Doppiogiochista

\textit{Fortemente Caotico e Fortemente Malvagio}

\textit{Magia Potenziata}: Gli incantesimi e le capacità magiche caotiche o malvagie vengono potenziati.

\textit{Magia Ostacolata}: Gli incantesimi e le capacità magiche legali o buone vengono ostacolati.


\subsection{Elysium}
Una vasta terra di natura incontaminata e selvaggia, Elysium è il piano del caos benevolo, della libertà e indipendenza, personificati nei nativi Azata. Nell'Elysium, la cooperazione disinteressata e la feroce competizione si scontrano violentemente, ma tali conflitti non mettono mai in ombra i nobili concetti di coraggio, creatività e bene non ostacolati da regole o leggi.

Elysium presenta le seguenti caratteristiche:

\textit{Tratti}: Buono, Caritatevole, Anarchico, Innovativo, Competitivo

\textit{Magia Potenziata}: Gli incantesimi e le capacità magiche caotiche o buone vengono potenziati.

\textit{Magia Ostacolata}: Gli incantesimi e le capacità magiche legali o malvagie vengono ostacolati.


\subsection{Inferno}
I nove strati dell'Inferno formano un labirinto strutturato di male premeditato dove il tormento va di pari passo con la purificazione. Piano di città di ferro, desolazioni in fiamme, ghiacciai congelati e picchi vulcanici infiniti, l'Inferno è suddiviso in nove strati concentrici, ciascuno sotto il crudele dominio di un arcidiavolo. Tortura, angoscia e sofferenza sono inevitabili nell'Inferno, ma sono impartite metodicamente, non per dispetto o capriccio, e supportano un disegno pianificato sotto i vigili sguardi dei disciplinati ranghi di Diavoli minori dell'Inferno. I nove strati dell'Inferno, dal primo all'ultimo, sono Averno, Dite, Erebo, Flegetonte, Stigia, Malebolge, Cocito, Caina e Nessus.

L'Inferno presenta le seguenti caratteristiche:

\textit{Tratti}: Malvagio, Disciplinato, Iracondo, Sadico, Arrogante

\textit{Fortemente Legale e Fortemente Malvagio}

\textit{Magia Potenziata}: Gli incantesimi e le capacità magiche legali o malvage vengono potenziati.

\textit{Magia Ostacolata}: Gli incantesimi e le capacità magiche caotiche o buone vengono ostacolati.


\subsection{Limbo}
Un vasto oceano di caos sfrenato e di potenziale inutilizzato circonda ciascuno dei Piani Esterni e confina con essi. Questo è il Limbo: splendido, mortale e davvero infinito. Dalle sue profondità inesplorate sono nati tutti gli altri piani e nelle sue viscere ribelli, alla fine, farà ritorno tutto il creato. Dove il mare informe del Limbo bagna le coste di altri piani, la sua massa assume un certo grado di stabilità, ed è in queste terre di confine che il viaggio è più sicuro, anche se ancora irto di pericoli derivanti dagli abitanti corrotti dal caos del Limbo. Più in profondità nel piano, i Protean nativi del Limbo si tuffano nel Caos Primordiale, creando e distruggendo la materia grezza del caos con incomprensibile trasporto.\
Il Limbo presenta le seguenti caratteristiche:

\textit{Tratti}: Anarchico, Incoerente, Istintivo, Irrazionale, Indisciplinato

\textit{Tempo Irregolare}

\textit{Magia Selvaggia e Magia Normale}: Sulle poche isole stabili del Limbo, è più probabile che la magia sia normale. In qualsiasi altro luogo la magia è selvaggia.


\subsection{Nirvana}
Il Nirvana è un paradiso imparziale esistente tra i due estremi di Elysium e Paradiso. Le sue meravigliose montagne, colline e fitte foreste rispondo alle aspettative di paradiso pastorale da parte del viaggiatore, ma il Nirvana contiene anche misteri che conducono all'illuminazione. Il Nirvana è un santuario e un luogo di riposo per tutti coloro che sono in cerca della redenzione o dell'illuminazione. Gli Agathion nativi del Nirvana hanno volontariamente messo da parte la propria trascendenza per custodire gli enigmi del piano, mentre i celestiali combattono le forze del male presenti fra i piani.

il Nirvana presenta le seguenti caratteristiche:

\textit{Tratti}: Buono, Gentile, Calmo, Semplice, Sicuro

\textit{Magia Potenziata}: Incantesimi e capacità magiche buone vengono potenziati.

\textit{Magia Ostacolata}: Incantesimi e capacità magiche malvage vengono ostacolati.


\subsection{Paradiso}
La svettante montagna del Paradiso torreggia al di sopra della Sfera Esterna. Questo ordinato regno di onore e compassione è suddiviso in sette strati. I declivi del Paradiso sono pieni di città ordinate e ben strutturate e di giardini e frutteti puliti e curati. Sebbene abbiano iniziato le proprie esistenze come mortali, gli Arconti nativi del Paradiso vedono la legge ed il bene come le due metà inscindibili dello stesso sommo concetto e si schierano contro le corruzioni cosmiche del caos e del male.

Il Paradiso presenta le seguenti caratteristiche:

\textit{Tratti}: Buono, Rigido, Combattivo, Pratico. Sincero, Valoroso

\textit{Magia Potenziata}: Gli incantesimi e le capacità magiche legali o buone vengono potenziati.

\textit{Magia Ostacolata}: Gli incantesimi e le capacità magiche caotiche o malvage vengono ostacolati.


\subsection{Purgatorio}
Ogni anima transita nel Purgatorio per essere giudicata prima di essere inviata verso la sua destinazione ultima. Vasti cimiteri e terre desolate riempiono le sue cupe distese, insieme a polverosi e riecheggianti tribunali preposti al giudizio dei morti. il Purgatorio è la dimora degli Eoni, una razza che incorpora la dualistica natura dell'esistenza e i cui membri sono costantemente in guerra e in pace fra loro e con se stessi.

Il Purgatorio presenta le seguenti caratteristiche:

\textit{Assenza di Tempo}: L'età, la fame, la sete, le sofferenze (come Malattie, Maledizioni e Veleni) e la guarigione naturale non hanno effetto nel Purgatorio, sebbene riprendano il proprio funzionamento quando il viaggiatore lascia il piano.

\textit{Magia Potenziata}: Incantesimi e capacità magiche che riguardano la morte o riposo vengono potenziati.


\subsection{Utopia}
Utopia è una roccaforte dell'ordine contrapposta al caos del Limbo e alle infinite orde demoniache dell'Abisso. Una grande città di perfezione eterna, le cui strade ed edifici sono modelli di architettura ed estetica: ogni cosa è in ordine e nulla accade per caso. Anche se Utopia non è governata da nessuna razza, Assiomiti ed Inevitabili ne fanno la loro dimora, cercando costantemente di espandere la loro città perfetta.

Utopia presenta le seguenti caratteristiche:

\textit{Tratti}: Rigido, Disciplinato, Serio, Diretto, Freddo

\textit{Magia Potenziata}: Gli incantesimi e le capacità magiche legali vengono potenziati.

\textit{Magia Ostacolata}: Gli incantesimi e le capacità magiche caotiche vengono ostacolati.


\end{multicols}

\pagebreak

%\begin{document}
%	\author{Andres Zanzani}
%	\title{\centering Dungeon Bell System (DBS)\\\vspace{0.5cm}Monster Manual\\\vspace{0.5cm}\includegraphics[scale=0.365] {copertina-monster}}
%	\date{\today\\v1.0.4\\}
%\maketitle
%\newpage~
%\setcounter{page}{1}
%\begin{multicols}{2}  
%{\small \tableofcontents{}}
%\end{multicols}
%\pagebreak
%\thispagestyle{empty}	
%\newpage~\newpage~
%}	
\section{Mostruario di TUS}

\textbf{Arrivano i Mostri...}

\begin{enfasiwide}{Chi lotta con i mostri deve guardarsi di non diventare, così facendo, un mostro. E se tu scruterai a lungo in un abisso, anche l'abisso scruterà dentro di te. (Friedrich Nietzsche)
		
\medskip
		
I mostri possono essere sconfitti soltanto dai loro simili. (Claymore)

\medskip

La tragedia dei mostri è di essere troppo grandi e potenti per essere accettati dal genere umano. (Ishiro Honda)
}\end{enfasiwide}\medskip

\begin{multicols}{2}

\lettrine[lines=2, lhang=0.33, loversize=0.25, findent=1.5em]{B}{envenuti} in un universo ricco di nemici cattivi violenti subdoli intelligenti meschini giganteschi.. e quant'altro tu vorrai. I mostri sono il caposaldo di qualsiasi gioco di ruolo fantasy.

Vengono qui spiegati e presentati dei mostri, non certo tutti ne tanto meno esaustivi, usateli per popolare di incubi le avventure dei vostri compagni.

\medskip

\includegraphics[width=0.8\linewidth]{immagini/sangiorgioedrago.png}
\begin{center}
\textit{San Giorgio e il drago (1460 circa) di Paolo Uccello. National Gallery di Londra}
\end{center}

\subsection{Introduzione}

Un avventura non e' solo un insieme di mostri ma di situazioni, di luoghi, di sorprese, insomma di tutto cio' che puo' affascinare, coinvolgere stupire, impegnare i giocatori. Ma anche i mostri servono. Picchiare ha un aspetto catartico, liberatorio.

Inserite nell'avventura mostri difficili e pericolosi dove serve ma ogni tanto, raramente fate sentire i giocatori potenti, fagli affrontare mostri che in pochissimi round possono risolvere. Descrivete il combattimento enfatizzando i colpi, i critici, il dolore ed il sangue dei mostri. Fate capire quanto possano essere potenti i personaggi.

Altre volte fate che i mostri incutano timore perché' sono grossi, affamati, magici e cattivi, e' necessario che i giocatori abbiano paura per i loro personaggi, che non diano mai per scontato la vittoria.

La sicurezza nel descrivere la situazione, poche battute, il fissare negli occhi i giocatori..coinvolgete i giocatori innanzitutto, una volta che i giocatori avranno la vostra attenzione anche i personaggi saranno piu' attenti. Cercate di mettere mostri coerenti all'ambiente, all'avventura, alla situazione. Non tirate a caso su tabelle, uno scontro ben organizzato da molta piu' soddisfazione che mostri a caso che "spawnano".

Non riducete tutto a un MMORG dove l'obiettivo e' solo uccidere tutto e tutti, ci possono essere sempre tante scelte se ti impegni un po'.

\begin{tcolorbox}[width=0.425\textwidth,title = Affrontare i mostri]
{
Lascia che questo vecchio ti dia un paio di consiglio giovane avventuriero!		

- Non tutti i nemici si sconfiggono con la spada, molte volte serve anche una mazza!

- A volte le armi e la forza bruta non bastano. Se non hai compagni che possono lanciare incantesimi assicurati di avere sempre la possibilità di appiccare un fuoco.

- Scappa. E' sempre una opzione valida se hai modo e vedi che la situazione non promette niente di buono.

- Organizzati! non entrare nel dungeon a testa bassa senza mai fermarti tranne quando sei morto! Riposati, esplora, controlla l'ambiente e quando sei sicuro e stai meglio prosegui! anche i tuoi nemici si organizzano e si riposano intanto.

- A volte si puo' anche parlare con i nemici, non sempre ma anche loro non vogliono morire.

- Se devi uccidere fallo con cattiveria e velocità. Non perdere tempo e ottimizza i colpi, risparmia le energie e preparati immediatamente ad un altro scontro.

}\end{tcolorbox}

\subsection{Modificare le Creature}

Nonostante la versatile collezione di mostri in questo documento, potresti comunque trovarti in imbarazzo quando si tratta di trovare la creatura perfetta per una tua avventura. Sentiti libro di modificare le creature esistenti e trasformarle in qualcosa che ti sia più utile, magari prendendo in prestito uno o due tratti da un mostro diverso.

Tieni a mente che modificare un mostro potrebbe cambiarne il grado di sfida. 

\subsection{Taglia}

Un mostro può essere di taglia Minuscola, Piccola, Media, Grande, Enorme o Mastodontica e Colossale. La tabella Categorie di Taglia mostra quanto spazio una creatura di una specifica taglia occupi in combattimento.

\medskip

\textbf{Categorie di Taglia}

\begin{tabularx}{0.45\textwidth}{llX}
\toprule
\textbf{Taglia}& \textbf{Spazio} & \textbf{Esempio}\\
Minuscola & 25 x 25 cm& Gatto, spiritello\\
Piccola & 0,5 x 0,5 m & Goblin, cane\\
Media & 1 x 1 m & Orco\\
Grande & 3 x 3 m& Ogre\\
Enorme & 5 x 5 m & Gigante, Ent\\
Mastodontico & 6 x 6 m&Kraken, Drago\\
Colossale & 12 x 12 m&Drago anziano\\
\end{tabularx}

\subsection{Tipo}

Il tipo di un mostro si riferisce alla sua natura basilare. Certi incantesimi, oggetti magici, Abilità e altri effetti del gioco interagiscono in modi speciali con le creature di un tipo specifico. Ad esempio, una \emph{freccia} \emph{ammazza draghi} infligge danni extra non solo ai draghi ma anche a tutte le altre creature del tipo drago, come i draghi tartaruga e le viverne.

Il gioco comprende i seguenti tipi di mostri:

\medskip\textbf{Aberrazioni}, creature totalmente aliene. Molte di esse possiedono innate abilità magiche che attingono alla mente aliena della creatura anziché dalle forze mistiche del mondo. Esempi classici di aberrazioni sono aboleti, osservatori, scortica mente e i batraci del caos.

\medskip\textbf{Bestie}, creature non umanoidi che sono una componente naturale di un mondo fantasy. Alcune possiedono poteri magici, ma la maggior parte è priva di Intelligenza e non ha alcuna forma di società o linguaggio. Esempi classici di bestie sono tutte le specie di animali comuni, i dinosauri e le versioni giganti degli animali. 

\medskip\textbf{Celestiali}, creature native dei Piani Superiori. Molti di loro sono servitori delle divinità, impiegati come messaggeri o agenti nel mondo dei mortali e per i piani.\\
I celestiali sono di natura buona, esempi classici di celestiali sono angeli, couatl e pegasi.

\medskip\textbf{Costrutti}, sono creati e non partoriti. Alcuni sono programmati dai loro creatori per seguire una semplice serie di istruzioni, mentre altri sono senzienti e capaci di pensare per proprio conto. I golem sono i costrutti più rappresentativi.

\medskip\textbf{Draghi}, sono grandi creature rettili di antica origine ed enorme potere. I veri draghi, compresi i buoni draghi metallici e i malvagi draghi cromatici, sono molto intelligenti e possiedono doti magiche innate. In questa categoria si collocano anche creature lontanamente imparentate con i veri draghi, ma meno potenti, meno intelligenti e meno magiche, come le viverne e gli pseudodraghi.

\medskip\textbf{Elementali}, sono creature native dei piani elementali. Alcune creature di questo tipo sono poco più che masse animate del rispettivo elemento, e includono le creature chiamate semplicemente elementali. Altre creature possiedono forme biologiche infuse di energia elementale. Le razze dei geni, compresi djinn ed efreet, formano le civiltà più importanti dei piani elementali. Altre creature elementali sono gli azer, i persecutori  invisibili e le bizzarrie d'acqua. 

\medskip\textbf{Fatati}, sono creature magiche strettamente legate alle forze della natura. Vivono in radure crepuscolari e foreste nebbiose. Esempi di fatati sono driadi, pixie e satiri.

\medskip\textbf{Giganti}, troneggiano sugli umani e i loro simili. Sono di forma umana, sebbene alcuni abbiano più teste (ettin) o deformità (fomori). Le sei varianti dei veri giganti sono gigante di collina, gigante di pietra, gigante del gelo, gigante del fuoco, gigante delle nuvole, gigante delle tempeste. Oltre questi, anche ogri e troll sono giganti. 

\medskip\textbf{Immondi}, creature perverse native dei Piani Inferi. Alcune sono al servizio di  divinità, ma molte di più operano agli ordini di arcidiavoli e principi demoni. A volte sacerdoti e maghi malvagi evocano gli immondi nel mondo materiale perché eseguano le loro volontà. Se un celestiale malvagio è una rarità, un immondo buono è praticamente inconcepibile. Gli immondi includono demoni, diavoli, segugi infernali e rakshasa. 

\medskip\textbf{Melme}, sono creature gelatinose che difficilmente hanno una forma fissa. Vivono principalmente sottoterra, stabilendosi in grotte e sotterranei, nutrendosi di rifiuti, carcasse o creature tanto sfortunate da incapparvi. I protoplasmi neri e i cubi gelatinosi sono tra le melme più riconoscibili.

\medskip\textbf{Mostruosità}, sono mostri nel senso più stretto del termine creature spaventose che non sono comuni, né davvero naturali, e quasi mai benigne. Alcune sono il risultato di esperimenti magici andati male (come l'orsogufo), mentre altri sono il prodotto di terribili maledizioni (tra cui annoveriamo il minotauro). Sfuggono a qualsiasi categorizzazione, e in qualche modo servono da categoria onnicomprensiva per quelle creature che non corrispondono a nessun altro tipo di mostro. 

\begin{center}
	\includegraphics[width=0.8\linewidth]{immagini/sanmichelesatana.png}\\
	\textit{San Michele sconfigge Satana. Raffaello ed aiuti (1518). Museo del Louvre}
\end{center}

\medskip\textbf{Non Morti}, sono creature un tempo vive condotte ad un orribile stato di non morte tramite la pratica della magia negromantica o qualche blasfema maledizione. Tra i non morti si annoverano cadaveri ambulanti, come vampiri e zombi, e spiriti incorporei, come fantasmi e spettri.

\medskip\textbf{Piante}, in questo contesto si tratta di creature vegetali, non della normale flora. La maggior parte di esse sono mobili, e alcune sono carnivore. L'esempio più classico di piante sono i tumuli ambulanti e gli ent. Anche le creature fungoidi come le spore gassose e i miconidi rientrano in questa categoria.

\medskip\textbf{Umanoidi}, sono la popolazione principale dei mondi di gioco, civilizzati e selvaggi, comprendono gli umani e un'ampia gamma di altre specie. Possiedono una lingua e una cultura, poche o nessuna abilità magica innata (sebbene molti umanoidi possano apprendere gli incantesimi), e una forma bipede. Le razze più comuni di umanoide sono quelle più adatte come personaggi del giocatore: umani, nani, elfi e halfling. Quasi altrettanto numerose, ma più brutali e selvagge, e quasi tutte malvagie, sono le razze goblinoidi (goblin, hobgoblin e bugbear), orchi, gnoll, lucertoloidi e coboldi.\\

\medskip

Queste categorie possono essere a loro volta raggruppate in tipologie di Creature:
\smallskip
\begin{itemize}
\item
Le \textbf{Creature Naturali}: sono Insetti, Rettili, Bestie, Umanoidi, Piante, Creature acquatiche, Monstrusita', Melme
\item
Le \textbf{Creature Magiche} sono: Immondi, Fatati, Spiriti, Non morti, Giganti, Celestiali, Costrutti, Aberrazioni (tutto ciò che e' alieno o innaturale) e Draghi.

Se una Creatura Naturale ha poteri magici allora si considera anche come Creatura Magica.
\end{itemize}


\medskip\textbf{Etichette}

Un mostro può presentare una o più etichette indicate tra parentesi, a seguire il suo tipo. Ad esempio un orco ha il tipo \emph{umanoide (orco)}. Le etichette tra parentesi forniscono ulteriori categorizzazioni per determinate creature. Le etichette non hanno delle proprie regole specifiche, ma alcuni elementi del gioco, come gli oggetti magici, vi possono fare riferimento. Ad esempio, una lancia particolarmente efficace contro i demoni, funzionerebbe contro qualsiasi mostro che abbia l'etichetta demone.

\subsection{Tratti}

I mostri non presentano l'elenco dettagliato dei tratti, troverete solo l'indicazione sugli assi del Chaos, Legge, Bene e Male. Ricordatevi che sono indicazioni, le eccezioni possono capitare specialmente nelle specie più intelligenti.
Determinate creature sono \textbf{disallineate}, ovvero non hanno una condotta morale o etica.

\subsection{Difesa}

Un mostro che indossa un'armatura o trasporta uno scudo ha una Difesa che tiene conto dell'armatura, lo scudo e della Destrezza. Altrimenti, la Difesa di un mostro è basata sul suo valore di Destrezza l'armatura naturale, se la possiede. Se un mostro possiede un'armatura naturale, indossa armature o trasporta uno scudo, viene indicato tra parentesi dopo il valore della sua Difesa.

Qualora il mostro fosse \textbf{colto di sorpresa} sottraete alla Difesa il valore di Destrezza e Scudo se presente.

Se il mostro viene colpito con un \textbf{effetto a tocco} (Difesa a tocco) la Difesa sarà 10 + Destrezza + bonus magici.

\subsection{Punti Ferita}

Di solito quando scende a 0 punti ferita, un mostro muore o viene distrutto.

I punti ferita di un mostro sono presentati sia come un insieme di dadi che come valore medio. Ad esempio, un mostro con 2d8 punti ferita ha di media 9 punti ferita (2 x 4,5).

La taglia di un mostro determina il dado impiegato per calcolare i suoi punti ferita, come mostrato sulla tabella Dadi Vita per Taglia.

\subsection{Dadi Vita per Taglia Mostro}

\medskip
\begin{tabular}{lll}
\toprule
Taglia & Dado Vita & PF per Dado\\
Minuscola &d4&2,5\\
Piccola &d6&3,5\\
Media&d8 &4,5\\
Grande&d10&5,5\\
Enorme&d12&6,5\\
Mastodontico&d20&10,5\\
Colossale&2d12&12\\
\end{tabular}
\medskip

Anche il valore di Costituzione di un mostro influenza il numero di punti ferita che possiede. Il suo valore di Costituzione viene moltiplicato per il numero di Dadi Vita che possiede e il risultato viene sommato ai suoi punti ferita. Ad esempio, un mostro ha Costituzione 1 e 2d8 Dadi Vita, e avrà quindi 2d8+2 punti ferita (media 11).

\subsection{Movimento}

Il Movimento di un mostro ti dice di quanto si possa muovere durante il suo round per Azione di Movimento

Tutte le creature possiedono un movimento di passeggio, detto semplicemente movimento del mostro. Le creature che non possiedono alcuna forma di spostamento terreno hanno velocità di passeggio 0 metri.

Alcune creature possiedono uno o più dei seguenti modi di movimento aggiuntivi.

\begin{center}
	\includegraphics[width=0.7\linewidth]{immagini/roc.png}\\
	\textit{Henry Justice Ford}
\end{center}


\medskip\textbf{Nuoto}

Un mostro che possiede una velocità di nuoto non deve spendere movimento extra per nuotare (non e' terreno difficile)

\medskip\textbf{Scalata}

Un mostro che possiede una velocità di scalata può usare tutto o solo parte del suo movimento per muoversi su superfici verticali. Il mostro non deve spendere movimento extra per scalare.

\medskip\textbf{Scavo}

Un mostro che possiede una velocità di scavo può usare la sua velocità per attraversare sabbia, terra, fango, ecc. Un mostro non può scavare attraverso la roccia solida a meno che non possieda un tratto speciale che glielo permetta.

\medskip\textbf{Volo}

Un mostro che possiede una velocità di volo può usare tutto o solo parte del suo movimento per volare. Alcuni mostri hanno l'abilità di \textbf{fluttuare}, che li rende difficili da abbattere. Il mostro smette di fluttuare quando muore.


\subsection{Punteggi di Caratteristica}

Ogni mostro possiede sei punteggi di caratteristica (Forza, Destrezza, Costituzione, Intelligenza, Saggezza, Carisma)

\subsection{Competenze}

La voce Competenze è riservata a quei mostri che sono capaci in una o più competenze. Ad esempio, un mostro che è molto attento e furtivo potrebbe avere bonus alle prove di Saggezza (Consapevolezza) e Destrezza. \\
Si possono applicare anche altri modificatori. Ad esempio, un mostro potrebbe avere un bonus più grande del previsto per tenere conto della sua grande perizia.

\subsection{Vulnerabilità, Resistenze e Immunità}

Alcune creature possiedono vulnerabilità, resistenze o immunità ad un certo tipo di danno. Creature particolari sono addirittura resistenti o immuni agli attacchi non magici (un attacco magico è un attacco sferrato tramite un incantesimo, un oggetto magico, o un'altra fonte di magia). \\
E' anche possibile che sia indicato uno specifico bonus magico minimo per poter danneggiare la creatura.\\
Inoltre, certe creature sono immuni a determinate condizioni. Se  un mostro è immune ad un effetto di gioco che non viene considerato danno o condizione, possiede invece un tratto speciale.


Nella tabella sottostante viene indicato quale incantamento magico dell'arma e' necessario per superare l'immunita' indicata.\\
E' anche indicato il livello minimo di attacco naturale nel caso si colpisca con calci e pugni.

\medskip

\textbf{Tabella: Equivalenza Armi Magiche}\index{Tabella Equivalenza Armi Magiche}

\medskip

\begin{tabularx}{0.45\textwidth}{lXX}
	\toprule
	\textbf{Immunità} & \textbf{Bonus Arma} & \textbf{Attacco Naturale}\\
	Incantamento +1         & +1              & Livello 3\\
	Incantamento +2         & +2              & Livello 6\\
	Ferro Freddo / Argento  & +2              & Livello 9\\
	Adamantio               & +3              & Livello 12\\
	Incantamento +3         & +3              & Livello 15\\
	Incantamento +4         & +4              & Livello 18\\
\end{tabularx}


\subsection{Sensi}

La voce Sensi elenca qualsiasi senso speciale di cui il mostro sia in possesso. I sensi speciali sono descritti di seguito. Se non e' presente la voce Sensi, la creatura ha dei sensi standard (visione...)

\subsubsection{Percezione Tellurica}

Un mostro con percezione tellurica può individuare e trovare le origini delle vibrazioni entro uno specifico raggio, purché il mostro e la fonte della vibrazione siano in contatto con lo stesso terreno o sostanza. La percezione tellurica non può essere impiegata per individuare creature volanti o incorporee. Molte creature scavatrici, come gli ankheg e i colossi di terra, possiedono questo senso speciale.

\subsubsection{Visione Crepuscolare}

Una creatura con Visione Crepuscolare può vedere nella piu' tenue delle luci, ma non nell'oscurità' completa. Molte creature che vivono sottoterra possiedono questo senso
speciale.  Vedi capitolo \hyperlink{visioneeluce}{Caratteristiche Speciali}.

\subsubsection{Visione del Vero}

Un mostro con la visione del vero può, fino ad una specifica gittata, vedere attraverso l'oscurità normale e magica, vedere creature e oggetti invisibili, automaticamente individuare le illusioni e riuscire i Tiri Salvezza contro di loro, e percepire la forma originale di un mutaforma o di una creatura trasformata dalla magia. Inoltre, la creatura può vedere nel Piano Etereo fino alla stessa gittata.


\subsubsection{Vista Cieca}

Una creatura con vista cieca può percepire l'ambiente circostante, senza fare affidamento alla vista, fino ad una specifica gittata. 

Le creature senza occhi, come i grimlock e le melme, e le creature con ecolocazione o sensi potenziati, come i pipistrelli e i draghi puri, possiedono questo senso. 

Se un mostro è cieco di natura, la cosa viene annotata tra parentesi, ad indicare che la gittata della sua vista cieca definisce anche la gittata massima della sua percezione.

\medskip
\begin{center}
	
	\includegraphics[width=0.7\linewidth]{immagini/ciclope.png}
	
	\textit{Henry Justice Ford}
	
\end{center}

\subsection{Linguaggi}

Le lingue che un mostro può parlare sono riportate in ordine alfabetico. A volte un mostro può capire una lingua ma non parlarla, e la cosa viene indicata a questa voce. Una ``-'' indica che la creatura non parla né comprende alcuna lingua.

\subsection{Telepatia}

La telepatia è un'abilità magica che permette ad un mostro di comunicare mentalmente con un'altra creatura nel raggio di azione specificato. La creatura contattata non e' necessario che parli la stessa lingua del mostro per comunicare in questo modo. Una creatura senza telepatia può ricevere e rispondere a messaggi telepatici ma non può iniziare o terminare una conversazione telepatica.

Un mostro telepatico non ha bisogno di vedere la creatura contattata e può terminare il contatto telepatico in qualsiasi momento. Il contatto è infranto non appena le due creature non si trovano più entro il raggio di azione o se il mostro telepatico contatta un'altra creatura a gittata. Un mostro telepatico può iniziare o terminare una conversazione  telepatica senza dover usare un'azione, ma mentre il mostro è inabile, non può dare inizio ad un contatto telepatico, e qualsiasi contatto in corso viene terminato.

Una creatura nell'area di un \emph{campo anti-magia} o in qualsiasi altro posto in cui la magia non funziona non può inviare o ricevere messaggi telepatici.

\subsection{Sfida}

Il \textbf{grado di sfida} (CR) di un mostro vi dice quanto sia grande la minaccia che pone. Una compagnia di quattro avventurieri equipaggiata in maniera appropriata e riposata dovrebbe essere in grado di sconfiggere un mostro dal grado di sfida pari al proprio livello medio senza subire perdite. Ad esempio, una compagnia di quattro personaggi di 3° livello dovrebbe ritenere un mostro di grado di sfida 3 una degna sfida, ma non letale.

I mostri che sono significativamente più deboli dei personaggi di 1° livello hanno un grado di sfida inferiore ad 1. I mostri con un grado di sfida 0 non presentano problemi eccetto in grandi numeri; quelli privi di reali attacchi non valgono punti esperienza.

Alcuni mostri presentano una sfida superiore a quelle che anche una compagnia di 20° livello sia in grado di gestire. Questi mostri hanno grado di sfida 21 o superiore e sono progettati proprio per mettere alla prova le capacità dei personaggi.


\subsection{Tratti Speciali}

I tratti speciali (che compaiono dopo il grado di sfida di un mostro ma prima di qualsiasi azione o reazione) sono peculiarità che avranno probabilmente un ruolo in un incontro di combattimento e che richiedono delle spiegazioni.

\subsection{Incantesimi}

Un mostro con il privilegio Incantesimi e' in grado di lanciare Incantesimi.\\
Un mostro può lanciare un incantesimo dalla sua lista senza effettuare la prova di magia e senza la possibilità di effettuare tiri critici o meno. La DC e' quella dell'incantesimo + Intelligenza o Saggezza a seconda della caratteristica migliore.


\includegraphics[width=0.7\linewidth]{immagini/lich2.png}

\medskip

\textit{Lich - Battle of Wesnoth}


\subsection{Incantesimi Innati}

Un mostro con l'abilità innata di lanciare incantesimi ha il tratto speciale Incantesimi.
Se non indicato diversamente non e' necessario effettuare la prova di magia e l'incantesimo viene lanciato alla sua Difficoltà senza conteggiare alcun critico.\\
Gli incantesimi innati di un mostro non possono essere scambiati con altri incantesimi. 

\subsection{Azioni}

Anche i mostri agiscono secondo lo schema delle 3 Azioni disponibile per round. Possono essere segnate abilità e capacità che gli permettono di eseguire un numero piu' elevato di Azioni.

Quando un mostro svolge le sue azioni, può scegliere tra le opzioni della sezione Azioni del suo blocco statistiche o impiegare una delle azioni disponibili a tutte le creature, come Scattare o Nascondersi.

\subsubsection{Attacchi da Mischia e a Distanza}

L'azione più comune che un mostro effettuerà in combattimento, sarà un attacco da mischia o a distanza. Possono essere attacchi con incantesimi o attacchi con armi, dove l'arma può essere un manufatto o un'arma naturale, come gli artigli o la coda chiodata.

\emph{\textbf{Creatura contro Bersaglio}.} Il bersaglio di un attacco da mischia o a distanza è di solito una creatura o un bersaglio, la differenza nel fatto che un ``bersaglio'' può essere una creatura o un oggetto.

\emph{\textbf{Colpisce.}} Qualsiasi danno inflitto o altro effetto che avviene come risultato di un attacco che colpisce il bersaglio viene descritto nell'annotazione ``\emph{Colpisce}''. Puoi scegliere se prendere il danno medio o tirare i dadi; per questo  motivo vengono presentati sia il danno medio che una formula di dadi. 

\textbf{\emph{Manca}.} Se un attacco ha un effetto prodotto da un colpo a vuoto, quell'informazione viene fornita dall'annotazione ``\emph{Manca}''.

\emph{\textbf{Danni.}} Se un mostro impugna armi manufatte, infligge danni appropriati all'arma. I mostri più grossi di solito impugnano armi di dimensioni superiori che infliggono danni extra quando colpiscono. Raddoppiare i dadi dell'arma se la creatura è Grande, triplicarli se Enorme e quadruplicarli se Mastodontica.

Una creatura ha -1d6 ai tiri per colpire con un'arma costruita per una taglia superiore alla sua.  Il Narratore può decidere che le armi di due o più taglie più grandi di quella dell'attaccante sono del tutto impossibili da usare.

\subsubsection{Multiattacco}

Una creatura che può effettuare più attacchi durante il suo round ha l'abilità Multiattacco. Una creatura non può usare Multiattacco quando effettua un attacco di opportunità, il quale deve essere un singolo attacco da mischia.

\begin{center}
	\includegraphics[width=0.6\linewidth]{immagini/polpo.png}
	
	\textit{Alphonse de Neuville - Hetzel edition of 20000 Lieues Sous les Mers}
\end{center}

\subsubsection{Regole dell'Afferrare per i Mostri}

Molti mostri possiedono un attacco speciale che gli permette di afferrare rapidamente la preda. Quando un mostro colpisce con un simile attacco, non deve effettuare un'ulteriore prova di caratteristica per determinare se l'afferrare riesce, a meno che l'attacco non dica altrimenti.

Una creatura afferrata dal mostro può usare un azione per tentare di sfuggirgli. Per farlo, deve riuscire una prova di Forza contro la DC di fuga nel blocco statistiche del mostro. Se non viene fornita una DC di fuga, assumere che la DC sia uguale a 10 + Forza del mostro.

\subsubsection{Munizioni}

Un mostro porta con sé munizioni sufficienti per effettuare i suoi attacchi a distanza. Puoi presumere che un mostro abbia 2d4 proiettili per un attacco con armi da lancio, e 2d10 proiettili per un'arma a proiettili come un arco o una balestra.


\includegraphics[width=0.7\linewidth]{immagini/cupido.png}

\textit{Eros con il suo arco. Musei Capitolini}

\subsubsection{Reazioni}

Se un mostro può compiere qualcosa di speciale con le sue reazioni, è riportato qui. Se una creatura non ha reazioni speciali, questa sezione è assente.

\subsubsection{Uso Limitato}

Alcune abilità speciali hanno restrizioni sul numero di volte che possono essere usate.

\textbf{\emph{X/Giorno}.} L'annotazione ``X/Giorno'' indica un'abilità speciale che può essere usata X volte prima che il sorga l'alba per recuperare gli usi consumati. Ad esempio, ``1/Giorno'' indica un'abilità speciale che può essere usata una volta prima che il mostro debba aspettare la nuova alba.

\emph{\textbf{Ricarica X-Y.}} L'annotazione ``Ricarica X-Y'' indica che il mostro può usare un'abilità speciale una volta e che l'abilità ha una probabilità casuale di ricaricarsi ogni round seguente di combattimento. All'inizio di ciascun round del mostro, tira un d6. Se il risultato è uno dei numeri dell'annotazione di ricarica, il mostro recupera l'uso dell'abilità speciale. L'abilità si ricarica anche all'alba di un nuovo giorno.

Ad esempio, "Ricarica 5-6" indica che un mostro può usare la sua abilità speciale una volta. Poi, all'inizio del round del mostro, recupera l'uso dell'abilità se tira 5 o 6 su di un d6.

\subsection{Equipaggiamento}

Il blocco statistiche si riferisce all'equipaggiamento, oltre le armi o le armature utilizzate dal mostro. Una creatura che normalmente indossa abiti, come un umanoide, si assume sia abbigliato in maniera appropriata.

Puoi equipaggiare i mostri con ulteriore equipaggiamento o ninnoli come preferisci, utilizzando il capitolo \hyperlink{equipaggiamento}{Equipaggiamento} come fonte di ispirazione, e sei tu a decidere quanto dell'equipaggiamento del mostro è recuperabile dopo che la creatura è stata uccisa o se qualsiasi parte del suo equipaggiamento sia ancora utilizzabile. Ad esempio, un'armatura ammaccata fatta per un mostro difficilmente sarà utilizzabile da qualcun altro.  Se un mostro incantatore necessita di componenti  materiali per lanciare i suoi incantesimi, dai per  scontato che abbia le componenti materiali per lanciare  gli incantesimi nel suo blocco statistiche.  

\subsection{Azioni Aggiuntive}

Certe creature possono  possono eseguire azioni speciali al di fuori del proprio  round, e alcune possono estendere il proprio potere  all'ambiente, provocando l'avvenimento di effetti magici  straordinari nelle loro vicinanze.

Una creatura con azioni aggiuntive può effettuare un certo  numero di azioni speciali -- dette azioni aggiuntive -- al  di fuori del suo round. Solo un'azione aggiuntiva può  essere usata alla volta e solo al termine del round di  un'altra creatura. Una creatura con azioni aggiuntive recupera  all'inizio del suo round le azioni aggiuntive che ha  usato. Non è obbligata ad usare le sue azioni aggiuntive, e non può usare le azioni aggiuntive mentre è inabile o altrimenti incapace di effettuare  azioni. Se sorpresa, non può usarle fin dopo il suo  primo round di combattimento.

Se una creatura assume la forma di una creatura con azioni aggiuntive, magari tramite un incantesimo, non ne  ottiene però le azioni aggiuntive, le azioni da tana, o  gli effetti regionali.

\subsubsection{La Tana di una Creatura}

Una creatura con azioni aggiuntive può presentare una sezione che ne descrive la tana e gli effetti speciali che vi può  creare mentre si trova lì, o per propria volontà o  semplicemente grazie alla sua presenza. Questa  sezione si applica solo alle creature leggendarie che  trascorrono molto tempo nelle loro tane ed è altamente  probabile che vi vengano incontrate.

\subsubsection{Azioni da Tana}

Se una creatura con azioni aggiuntive ha un'azione da tana, può  usarla per imbrigliare la magia ambientale della sua  tana. Al conteggio di iniziativa 20, perdendo i pareggi,  la creatura può usare una delle sue opzioni di azioni da  tana. Non può farlo mentre è inabile o altrimenti  incapace di effettuare azioni. Se sorpresa, non può  farne uso fino a dopo il suo primo round di combattimento.

\subsubsection{Effetti Regionali}

La semplice presenza di una creatura con azioni aggiuntive può  avere effetti strani e meravigliosi sull'ambiente, come  indicato in questa sezione. Gli effetti regionali terminano all'istante o si dissipano col tempo una volta  morta la creatura con azioni aggiuntive.


\subsubsection{Tipologie di Tesoro}

Ogni tipologia di creatura puo' preferire un tipo di tesoro (inteso come oggetti, monete, gemme...) diverso. Questi sono solo suggerimenti su come costruire il tesoro del mostro.

\medskip

\begin{itemize}

\item \textbf{Aberrazione}
Molte aberrazioni hanno scarsa considerazione per i tesori, possedendo solo quel che prendono dai resti delle loro vittime precedenti. Altre sono avversari astute che usano vari oggetti magici e tesori per potenziare le loro capacità.

\item \textbf{Animale}
Gli animali non si curano affatto dei tesori, lasciando invece monete e oggetti con i resti dei loro pasti. Per quelli con un tesoro, quest'ultimo in genere si trova nelle loro tane, sparso tra le ossa e gli altri scarti.

\item \textbf{Bestia Magica}
Curandosi poco dei valori, la maggior parte delle bestie magiche è unicamente in cerca del suo prossimo pasto. I nascondigli di queste creature sono spesso disseminati di ninnoli preziosi e oggetti magici.

\item \textbf{Costrutto}
Il solo tesoro portato dai costrutti generalmente è parte del costrutto stesso, come un'arma o un oggetto magico. I costrutti, però, vengono tipicamente usati per sorvegliare tesori o oggetti magici di maggior valore.

\item \textbf{Drago}
Noti per i loro preziosi tesori, i draghi spesso rimuginano su pile di monete, gemme, oggetti magici e altri oggetti costosi.

\item \textbf{Esterno}
Gli esterni sono tra i tipi di creature più vari e di conseguenza potrebbero avere davvero qualsiasi tipo di tesoro su di loro o nascosto nei loro rifugi. Il Narratore dovrebbe considerare la singola creatura quando determina il tipo di tesoro che più si adatta a quell'esterno.

\item \textbf{Folletto}
Sopra ogni altra cosa, i folletti danno valore agli oggetti belli e magici. Hanno scarsa considerazione per gli strumenti di scambio e commercio usati dalle razze più civilizzate, come monete e valori.

\item \textbf{Melma}
Le melme non concepiscono cose come i tesori e lasciano tutto quel che trovano nella loro ricerca del prossimo pasto. Qualsiasi tesoro possano trasportare è completamente accidentale.

\item \textbf{Non Morto}
I tesori trasportati dai non morti variano a seconda che si tratti o meno di una creatura intelligente. I non morti privi di intelletto tipicamente hanno solo i miseri valori che portavano con sé in vita, raramente davvero utilizzabili come tesori, mentre quelli intelligenti sfruttano una vasta gamma di oggetti magici per distruggere i viventi.

\item \textbf{Parassita}
Come le altre creature senza mente, i parassiti non bramano tesori, sebbene queste creature talvolta si trovino a infestare aree dove sono custoditi dei valori.

\item \textbf{Umanoide}
Le creature di questo tipo sono molto variegate, ma persino gli umanoidi più primitivi usano equipaggiamento e oggetti magici in qualche misura. In gruppi più grandi, come le comunità, gli umanoidi spesso possiedono una grande quantità di tesori che custodiscono collettivamente.

\item \textbf{Vegetale}
Come gli animali, le creature vegetali non danno peso ai tesori, e tutto ciò che si potrebbe trovare dove crescono rappresenta semplicemente i resti non digeriti di una vittima precedente.

\end{itemize}

\end{multicols}

\bigskip

\begin{narratorewide}

Le creature qui presentate vogliono essere un esempio, corposo, dei nemici che i tuoi amici potrebbero incontrare. Attenzione, non e' detto che siano tutti nemici o per forza che abbiano intenzione negative.

Creature piu' civilizzate avranno una loro condotta etica e morale individuale, anche all'interno di uno stesso gruppo di "nemici" c'e' chi potrebbe essere più "nemico" o semplicemente indifferente.

Sfrutta le peculiarità e unicità delle creature per creare incontri non scontati e sfidanti dal punto di vista tattico. Non essere scontato ma neanche assurdo nelle scelte, deve sempre esserci della coerenza nel piazzare le creature.	
	
\end{narratorewide}


\bigskip

\begin{enfasiwide}{
		
Amon Goth: Il controllo è potere. Questo è il potere.

Oskar Schindler: E' per questo che ci temono?

Amon Goth: Abbiamo il potere di uccidere. Per questo ci temono.

Oskar Schindler: Ci temono perché abbiamo il potere di uccidere arbitrariamente. Un uomo commette un reato, doveva pensarci, lo facciamo uccidere e ci sentiamo in pace. O lo uccidiamo noi stessi e ci sentiamo ancora meglio. Questo non è il potere però! Questa è giustizia, è una cosa diversa dal potere. Il potere è quando abbiamo ogni giustificazione per uccidere e non lo facciamo.

Amon Goth: E' questo il potere?

Oskar Schindler: L'avevano gli imperatori questo. Un uomo ruba qualcosa, viene portato davanti all'imperatore e si lascia cadere per terra tremante, implora per avere pietà. E' conscio che sta per andarsene. E l'imperatore lo perdona, invece. Quell'uomo, immeritevole, lo lascia libero. 

(Schindler's list - La lista di Schindler, Film, 1993)
}\end{enfasiwide}\medskip




\pagebreak
\subsection{I Mostri}

\begin{enfasiwide}{
Io sono il mostro che gli uomini che respirano bramerebbero uccidere. Io sono Dracula. (Dracula di Bram Stoker)

\medskip

Nipote del signor Wing: Senta mister, ci sono tre regole da seguire, però.

Rand: Ah, sì? E quali sarebbero?

Nipote del signor Wing: Lo tenga lontano dalla luce, lui odia la luce forte, specialmente quella del sole. Morirebbe. E lo tenga lontano dall'acqua, non lo faccia bagnare. Ma la cosa più importante, la regola che non dovrà mai dimenticare è che anche se lui piange, anche se fa scena e la supplica lei non dovrà mai, mai dargli da mangiare dopo la mezzanotte. Ha capito? (Gremlins, Film, 1984)} \end{enfasiwide}\medskip

\bigskip

\begin{multicols}{2}

\medskip\index{Mostri - Aboleth}\textbf{Aboleth}

\emph{Grande aberrazione, legale malvagio}

\textbf{FORZA} +5

\textbf{DESTREZZA} -1

\textbf{COSTITUZIONE} +2

\textbf{INTELLIGENZA} +4

\textbf{SAGGEZZA} +2

\textbf{CARISMA} +4

\textbf{Iniziativa} +4 -- \textbf{Difesa} 22

\textbf{Punti Ferita} 135 (18d10 + 36)

\textbf{Movimento} 3 m, nuoto 12 m

\textbf{Tiri Salvezza} Tempra +8, Riflessi +5, Volontà +11

\textbf{Competenze} Consapevolezza +10, Storia +12

\textbf{Sensi} scurovisione 36 m

\textbf{Linguaggi} Linguaggio delle Profondità, telepatia 36 m

\textbf{Sfida} 10 (5.900 PE)

\emph{\textbf{Anfibio.}} L'aboleth può respirare aria e acqua.

\emph{\textbf{Nube di Muco.}} Mentre è sott'acqua, l'aboleth è avvolto da muco mutante. Una creatura che entri a contatto con l'aboleth, o che lo colpisca con un attacco da mischia mentre si trova entro 1 metro da esso, deve effettuare un Tiro Salvezza di Tempra CD 14. Se lo fallisce, la creatura resta ammalata per 1d4 ore. La creatura ammalata può respirare solo sott'acqua.

\emph{\textbf{Sonda Telepatica.}} Se una creatura comunica telepaticamente con l'aboleth, e l'aboleth può vederla, l'aboleth ne apprende i più grandi desideri.

\textbf{Azioni}

\emph{\textbf{Multiattacco.}} L'aboleth effettua tre attacchi con i tentacoli

\emph{\textbf{Tentacolo.} Attacco con arma da mischia}: +9 a colpire, portata 3 m, un bersaglio.

\emph{Colpisce:} 12 (2d6 + 5) danni da botta. Se il bersaglio è una creatura, deve riuscire un Tiro Salvezza di Tempra CD 14 o divenire ammalato. La malattia non produce alcun effetto per 1 minuto e può essere rimossa da qualsiasi magia che curi le malattie. Dopo 1 minuto, la pelle della creatura ammalata diventa trasparente e viscida, la creatura non può recuperare punti ferita a meno che non sia sott'acqua, e la malattia può essere rimossa solo da \emph{guarire} o un altro incantesimo cura malattie di Difficoltà 29 o più. Quando la creatura si trova al di fuori di un corpo d'acqua, subisce 6 (1d12) danni da acido ogni 10 minuti a meno che la sua pelle non venga bagnata prima che siano passati questi 10 minuti.

\emph{\textbf{Coda.} Attacco con arma da mischia}: +9 a colpire, portata 3 m, un bersaglio.

\emph{Colpisce:} 15 (3d6 + 5) danni da botta.

\emph{\textbf{Schiavizzare (3/Giorno).}} L'aboleth prende a bersaglio una creatura che può vedere entro 9 metri da esso. Il bersaglio deve riuscire un Tiro Salvezza di Volontà CD 14 o restare affascinato magicamente dall'aboleth finché l'aboleth muore o i due si trovano su piani di esistenza differenti. Il bersaglio affascinato è sotto il controllo dell'aboleth e non può effettuare reazioni. L'aboleth e il bersaglio possono comunicare telepaticamente tra di loro a qualsiasidistanza. 

Ogniqualvolta il bersaglio affascinato subisce danni, può ripetere il Tiro Salvezza. Se lo riesce, l'effetto termina. Non più di una volta ogni 24 ore, può ripetere il Tiro Salvezza quando si trova almeno a 1,5 chilometri di distanza dall'aboleth.

\textbf{Azioni Aggiuntive}

L'aboleth può effettuare 3 Azioni aggiuntive, scelte tra le opzioni seguenti. Può usare solo un'opzione leggendaria alla volta e solo al termine del turno di un'altra creatura. L'aboleth recupera le Azioni aggiuntive spese all'inizio del proprio turno.

\textbf{Individuare.} L'aboleth effettua una prova di Saggezza (Consapevolezza).

\textbf{Risucchio Psichico (Costa 2 Azioni).} Una creatura affascinata dall'aboleth subisce 10 (3d6) danni psichici, e l'aboleth recupera un numero di punti ferita pari al danno subito dalla creatura.

\textbf{Spazzata di Coda.} L'aboleth effettua un attacco di coda.

\textbf{Ecologia}\\
Ambiente: Qualsiasi Acquatico\\
Organizzazione: Solitario, coppia, nidiata (3-6) o branco (7-19)\\
Tesoro: Doppio\\
\textbf{Descrizione}\\
Come suggerisce il loro aspetto primitivo, gli ermafroditi aboleth sono fra le più antiche forme di vita al mondo. Già antichi quando gli dei cominciarono ad interessarsi del Piano Materiale, gli aboleth hanno sempre vissuto lontani dagli altri mortali: sono alieni, freddi e sempre occupati ad intessere piani. Un tempo governavano il mondo in un vasto impero, ed oggi vedono le altre forme di vita come cibo o schiavi... a volte entrambe le cose assieme. Disprezzano gli dei, poiché ritengono di essere loro i veri signori del creato, un aboleth è lungo 7 metri e pesa circa 3,2 tonnellate. Nelle più oscure profondità del mare, gli aboleth abitano ancora nelle loro grottesche città, ciclopiche e nauseabonde. Sono serviti da innumerevoli schiavi presi da ogni nazione, sia terrestre che marina, e quelli terrestri sono doppiamente schiavi dei loro padroni e del loro muco, che permette loro di respirare sott'acqua, gli aboleth incontrati da soli sono di solito esploratori provenienti da queste città nascoste, in cerca di nuovi schiavi.\\



\subsection{Angeli}

\medskip\index{Mostri - Angelo Deva}\textbf{Angelo Deva}

\emph{Medio celestiale, legale buono}

\textbf{FORZA} +4

\textbf{DESTREZZA} +4

\textbf{COSTITUZIONE} +4

\textbf{INTELLIGENZA} +3

\textbf{SAGGEZZA} +5

\textbf{CARISMA} +5

\textbf{Iniziativa} +4 -- \textbf{Difesa} 22

\textbf{Punti Ferita} 136 (16d8 + 64)

\textbf{Movimento} 9 m, volo 27 m

\textbf{Tiri Salvezza} Tempra +16, Riflessi +13, Volontà +11

\textbf{Competenze} Percepire Emozioni +9, Consapevolezza +9

\textbf{Resistenze ai Danni} da Luce; da botta, perforante e tagliente di attacchi non magici

\textbf{Immunità alle Condizioni} affascinato, affaticamento, spaventato

\textbf{Sensi} scurovisione 36 m

\textbf{Linguaggi} tutte, telepatia 36 m

\textbf{Sfida} 10 (5.900 PE)

\emph{\textbf{Armi Angeliche.}} Gli attacchi con arma del deva sono magici. Quando il deva colpisce con qualsiasi arma, l'arma infligge 4d8 danni da Luce aggiuntivi (già compresi nell'attacco).

\emph{\textbf{Incantesimi Innati.}} La caratteristica da incantatore innato del deva è il Carisma (CD 17 per i Tiri Salvezza degli incantesimi). Il deva può lanciare in maniera innata i seguenti incantesimi, con l'uso delle sole componenti verbali: 

A volontà: \emph{individuazione del bene e del male}

1/giorno: \emph{comunione, rianimare morti}

\emph{\textbf{Resistenza alla Magia.}} Il deva ha +1d6 ai Tiri Salvezza contro incantesimi e altri effetti magici.

\textbf{Azioni}

\emph{\textbf{Multiattacco.}} Il deva effettua due attacchi da mischia.

\emph{\textbf{Mazza.} Attacco con arma da mischia}: +8 a colpire, portata 1 m, un bersaglio.

\emph{Colpisce:} 7 (1d6 + 4) danni da botta più 18 (4d8) danni da Luce.

\emph{\textbf{Tocco Guaritore (3/Giorno).}} Il deva entra a contatto con un'altra creatura. Il bersaglio recupera magicamente 20 (4d8 + 2) punti ferita ed è libero da qualsiasi cecità, malattia, maledizione, sordità o veleno.

\emph{\textbf{Mutare Forma.}} Il deva può trasformarsi magicamente in un umanoide o bestia il cui grado di sfida sia pari o inferiore al proprio, o tornare alla sua vera forma. Alla morte ritorna alla sua vera forma. Qualsiasi equipaggiamento stia indossando o trasportando viene assorbito o trasportato nella nuova forma (a scelta del deva).

Nella nuova forma, il deva mantiene le sue statistiche di gioco e la facoltà di parlare, ma la sua Difesa, metodi di movimento, Forza, Destrezza e sensi speciali vengono rimpiazzati da quelli della nuova forma, e ottiene qualsiasi statistica o capacità (Azioni aggiuntive e azioni da tana) possedute dalla sua nuova forma e non dalla sua originale.

\textbf{Ecologia}
Ambiente: Qualsiasi piano di con Tratti buono\\
Organizzazione: Solitario, coppia, o squadriglia (3-6)\\
Tesoro: Doppio (Spadone Infuocato +1, altro tesoro)\\
\textbf{Descrizione}\\
I deva movanici compongono i ranghi della fanteria delle armate celesti, sebbene trascorrano la maggior parte del loro tempo pattugliando il Piano Positivo, quello Negativo e quello Materiale. Sul Piano Positivo sorvegliano le anime buone erranti, e questa volta li mette in conflitto con gli Jyoti. Sul Piano Negativo combattono i non morti, gli Sceaduinar e altri strani esseri che cacciano nel famelico vuoto. Le loro rare volte sul Piano Materiale hanno solitamente lo scopo di portare aiuto a potenti mortali, quando un grande pericolo minaccia di far cadere nelle mani del male un intero regno.\\


\medskip\index{Mostri - Angelo Planetar}\textbf{Angelo Planetar}

\emph{Grande celestiale, legale buono}

\textbf{FORZA} +7

\textbf{DESTREZZA} +5

\textbf{COSTITUZIONE} +7

\textbf{INTELLIGENZA} +4

\textbf{SAGGEZZA} +6

\textbf{CARISMA} +7

\textbf{Iniziativa} +5 -- \textbf{Difesa} 27

\textbf{Punti Ferita} 200 (16d10 + 112)

\textbf{Movimento} 12 m, volo 36 m

\textbf{Tiri Salvezza} Tempra +19, Riflessi +11, Volontà +19

\textbf{Competenze} Consapevolezza +11

\textbf{Resistenze ai Danni} da Luce;

\textbf{Immunità alle Condizioni} affascinato, affaticamento, spaventato, armi +1

\textbf{Sensi} visione del vero 36 m

\textbf{Linguaggi} tutte, telepatia 36 m

\textbf{Sfida} 16 (15.000 PE)

\emph{\textbf{Armi Angeliche.}} Gli attacchi con arma del planetar sono magici. Quando colpisce con qualsiasi arma, l'arma infligge 5d8 danni da Luce aggiuntivi (già compresi nell'attacco).

\emph{\textbf{Consapevolezza Divina.}} Il planetar riconosce immediatamente le bugie.

\emph{\textbf{Incantesimi Innati.}} La caratteristica da incantatore innato del planetario è il Carisma (CD 20 per i Tiri Salvezza degli incantesimi). Il planetario può lanciare in maniera innata i seguenti incantesimi, senza bisogno di componenti materiali:

A volontà: \emph{individuazione del bene e del male}, \emph{invisibilità} (solo personale)

3/giorno: \emph{barriera di lame, colpo infuocato, dissolvi il bene e il} \emph{male, rianimare morti}

1/giorno: \emph{comunione, controllare tempo atmosferico, piaga degli insetti}

\emph{\textbf{Resistenza alla Magia.}} Il planetar ha +1d6 ai Tiri Salvezza contro incantesimi e altri effetti magici.

\textbf{Azioni}

\emph{\textbf{Multiattacco.}} Il planetar effettua due attacchi da mischia.

\emph{\textbf{Spadone.} Attacco con arma da mischia}: +12 a colpire, portata 1 m, un bersaglio.

\emph{Colpisce:} 21 (4d6 + 7) danni taglienti più 22 (5d8) danni da Luce.

\emph{\textbf{Tocco Guaritore (4/Giorno).}} Il planetar entra a contatto con un'altra creatura. Il bersaglio recupera magicamente 30 (6d8 + 3) punti ferita ed è libero da qualsiasi cecità, malattia, maledizione, sordità o veleno.

\textbf{Ecologia}
Ambiente: Qualsiasi piano con Tratti buono\\
Organizzazione: Solitario o coppia\\
Tesoro: Doppio (Spadone Sacro +3)\\
\textbf{Descrizione}\\
I planetar sono i generali delle armate celestiali volti alla distruzione del male. Un planetar è di norma alto 2,7 metri e pesa circa 250 kg. Sono ottimi diplomatici, ma contro gli immondi preferiscono una guerra piuttosto che negoziare una pace.\\


\medskip\index{Mostri - Angelo Solar}\textbf{Angelo Solar}

\emph{Grande celestiale, legale buono}

\textbf{FORZA} +8

\textbf{DESTREZZA} +6

\textbf{COSTITUZIONE} +8

\textbf{INTELLIGENZA} +7

\textbf{SAGGEZZA} +7

\textbf{CARISMA} +10

\textbf{Iniziativa} +7 -- \textbf{Difesa} 31

\textbf{Punti Ferita} 243 (18d10 + 144) 

\textbf{Movimento} 15 m, volo 45 m

\textbf{Tiri Salvezza} Tempra +25, Riflessi +14, Volontà +23

\textbf{Competenze} Consapevolezza +14

\textbf{Resistenze ai Danni} da Luce;

\textbf{Immunità ai Danni} da Vuoto, veleno, armi +2

\textbf{Immunità alle Condizioni} affascinato, avvelenato, affaticamento, spaventato

\textbf{Sensi} visione del vero 36 m

\textbf{Linguaggi} tutte, telepatia 36 m 

\textbf{Sfida} 21 (33.000 PE)

\emph{\textbf{Armi Angeliche.}} Gli attacchi con arma del solar sono magici. Quando colpisce con qualsiasi arma, l'arma infligge 6d8 danni da Luce aggiuntivi (già compresi nell'attacco).

\emph{\textbf{Consapevolezza Divina.}} Il solar riconosce immediatamente le bugie.

\emph{\textbf{Incantesimi Innati.}} La caratteristica da incantatore innato del solar è il Carisma (CD 25 per i Tiri Salvezza degli incantesimi). Il solar può lanciare in maniera innata i seguenti incantesimi, senza bisogno di componenti materiali:

A volontà: \emph{individuazione del bene e del male}, \emph{invisibilità} (solo personale)

3/giorno: \emph{barriera di lame, colpo infuocato, dissolvi il bene e il male, resurrezione}

1/giorno: \emph{comunione, controllare tempo atmosferico}

\emph{\textbf{Resistenza alla Magia.}} Il solar ha +1d6 ai Tiri Salvezza contro incantesimi e altri effetti magici.

\textbf{Azioni}

\emph{\textbf{Multiattacco.}} Il solar effettua due attacchi con lo spadone.

\emph{\textbf{Spadone.} Attacco con arma da mischia}: +15 a colpire, portata 1 m, un bersaglio.

\emph{Colpisce:} 22 (4d6 + 8) danni taglienti più 27 (6d8) danni da Luce.

\emph{\textbf{Arco Lungo dell'Uccisione.} Attacco con arma a distanza}: +13 a colpire, gittata 45m, un bersaglio.

\emph{Colpisce:} 15 (2d8 + 6) danni perforanti più 27 (6d8) danni da Luce. Se il bersaglio è una creatura con 100 punti ferita o meno, deve riuscire un Tiro Salvezza di Tempra CD 15 o morire.

\emph{\textbf{Spada Volante.}} Il solare libera il suo spadone perché fluttui magicamente in uno spazio non occupato entro 1 metro da lui.  Se il solare può vedere la spada, con un'azione bonus le può ordinare mentalmente di volare per un massimo di 15 metri ed effettuare un attacco contro un bersaglio o ritornare nella mano del solare. Se la spada fluttuante è bersaglio di un effetto, si considera come se fosse impugnata dal solare. Se il solare muore, la spada fluttuante cade a terra.

\emph{\textbf{Tocco Guaritore (4/Giorno).}} Il solare entra a contatto con un'altra creatura. Il bersaglio recupera magicamente 40 (8d8 + 4) punti ferita ed è libero da qualsiasi cecità, malattia, maledizione, sordità o veleno.

Il solare può effettuare 3 azioni aggiuntive, scelte tra le opzioni seguenti. Può usare solo un'Azione Aggiuntiva alla volta e solo al termine del round di un'altra creatura. Il solare recupera le azioni aggiuntive spese all'inizio del proprio round. 

\textbf{Esplosione Incandescente (Costa 2 Azioni).} Il solare emette energia magica divina. Ogni creatura di sua scelta, in un raggio di 3 metri, deve effettuare un Tiro Salvezza su Riflessi DC  30, subendo 14 (4d6) danni da fuoco più 14 (4d6) danni da Luce se fallisce il Tiro Salvezza, o la metà se lo riesce. 

\textbf{Sguardo Accecante (Costa 3 Azioni).} Il solare prende a bersaglio una creatura entro 9 metri e che possa vedere. Se il bersaglio può vedere il solare, il bersaglio deve riuscire un Tiro Salvezza su Tempra DC  18 o restare accecato finché un incantesimo come \emph{ristorare inferiore} non rimuoverà la cecità.

\textbf{Teletrasporto.} Il solare si teletrasporta magicamente fino a 36 metri di distanza, insieme a tutto l'equipaggiamento che sta indossando o trasportando, in uno spazio non occupato e che può vedere.

\textbf{Ecologia}\\
Ambiente: Qualsiasi piano con Tratti buono\\
Organizzazione: Solitario o coppia\\
Tesoro: Doppio (Armatura Completa +5, Spadone Danzante +5, Arco Lungo Composito +5 [For +9])\\
\textbf{Descrizione}\\
I solar sono i più potenti fra gli angeli, solitamente braccio destro di una divinità o campioni di cause che portano beneficio ad un intero mondo o piano. Un solar ha di solito aspetto quasi umano, anche se alcuni di essi somigliano ad altre razze umanoidi ed alcuni hanno anche forme più inusuali. Un solar è alto circa 2,7 metri, pesa circa 250 kg ed ha una voce profonda ed imperiosa, impossibile da ignorare. La maggior parte di essi ha pelle argentata o dorata.\\
Benedetti con una serie di capacità magiche più potenti, i solar sono avversari terribili in grado di uccidere da soli le più potenti creature malvagie. Fra i celestiali sono considerati i più eccellenti cercatori di tracce, ed i migliori fra loro, si dice, sono in grado di seguire tracce vecchie di giorni lasciate da un Diavolo della Fossa attraverso il Piano Astrale. Alcuni di essi prendono il manto di uccisori di mostri e danno la caccia a potenti immondi e non morti come i divoratori, le megere notturne, le Ombre Notturne ed i Diavoli della Fossa, compiendo perfino incursioni nei piani  malvagi e nel Piano dell'Energia Negativa per distruggere queste creature alla fonte, prima che possano fare del male ai mortali. Alcuni fra i solar più antichi hanno portato a termine la loro missione, ed hanno fama di uccisori di creature oggi estinte.\\
I solar accettano il ruolo di guardiani, di solito di concetti soprannaturali o di oggetti o creature di grande importanza. Su un mondo, un gruppo di solar protegge i condotti di energia del sole contro i tentativi di spegnerlo e di portare tenebre eterne operati da razze malvagie come gli Elfi. Su un altro, sette solar vegliano su sette catene mistiche che tengono gli dèi del male imprigionati in un semipiano. Su un altro ancora, un solar con una spada fiammeggiante protegge il Paradiso Terrestre, impedendo a tutte le creature di entrare.\\
Nei mondi in cui gli dèi possono prendere forma fisica, i solar vengono inviati per diventare profeti e guru (spesso in guisa di mortali), gettando così le fondamenta di culti che diverranno grandi religioni. Nei mondi oppressi dal male, i solar sono i sacerdoti clandestini che portano speranza agli oppressi o che si lasciano martirizzare così che la loro essenza possa esplodere nelle regioni circostanti e crescere nel cuore dei futuri eroi.\\
Pur non essendo divinità, il potere dei solar si avvicina a quello dei semidei, e spesso fanno da consiglieri per le divinità più giovani o deboli. In alcune fedi politeiste, i mortali venerano uno o più solar come aspetti o servitori alla pari delle vere divinità (comunque mai senza l'approvazione della divinità in questione) o considerano i solar più famosi come figli, consorti o amanti delle vere divinità (cosa che, a seconda della divinità, potrebbe corrispondere al vero).\\
A differenza degli altri angeli, la maggior parte dei solar viene creata come diretta servitrice degli dèi, amalgamando anime buone ed energia divina pura, ma sempre più spesso questi potenti angeli vengono creati attraverso la "promozione" di angeli minori come deva e planetar. Raramente accade che anime particolarmente potenti e pure ascendano direttamente allo status di solar. I più antichi fra loro sono anteriori alla creazione dei mortali, e sono fra le prime creazioni degli dèi. Questi solar sono campioni fra i loro simili, ed hanno poca o nessuna interazione con i mortali, concentrandosi invece su concetti astratti quali la gravità, l'entropia, la materia oscura ed il male primevo.\\
I solar che passano molto tempo sul Piano Materiale, specialmente quelli che prendono la forma di mortali, sono a volta fonte di linee di sangue aasimar o mezzo-celestiali nelle famiglie umane, a volte a causa di una storia d'amore, a volte semplicemente per la vicinanza dei mortali alle loro emanazioni celestiali. È raro che vi siano loro discendenti diretti, e quando ciò accade è sempre una madre mortale a portare in grembo il figlio: anche se i solar possono apparire di qualsiasi sesso, gli dèi non hanno concesso loro la possibilità di dare alla luce un figlio. È per questo che i solar tendono a cercare un'amante mortale. Gli altri solar hanno poca considerazione di un loro simile che dà un figlio ad una mortale, quindi i padri solar tendono ad evitare i contatti con la loro progenie, per evitare di attirare vergogna su di sé. I solar, però, tendono a controllare da lontano i loro figli e, in tempo di difficoltà, ad aiutarli, anche se in modi misteriosi e discreti.\\
Tutti gli angeli rispettano il potere e la saggezza dei solar, e sebbene essi tendano a lavorare da soli, a volte comandano armate guidate da planetar e fanno da generali per le grandi incursioni contro le legioni dell'Inferno o le orde dell'Abisso.\\


\medskip\index{Mostri - Ankheg}\textbf{Ankheg}

\emph{Grande mostruosità, disallineato}

\textbf{FORZA} +3

\textbf{DESTREZZA} +0

\textbf{COSTITUZIONE} +1

\textbf{INTELLIGENZA} -5

\textbf{SAGGEZZA} +1

\textbf{CARISMA} -2

\textbf{Iniziativa} +0 -- \textbf{Difesa} 15 , 12 mentre è prono

\textbf{Punti Ferita} 39 (6d10 + 6)

\textbf{Movimento} 9 m, scavo 3 m

\textbf{Sensi} scurovisione 18 m, percezione tellurica 18 m, 

\textbf{Linguaggi} -

\textbf{Sfida} 2 (450 PE)

\textbf{Azioni}

\emph{\textbf{Morso.} Attacco con arma da mischia}: +5 a colpire, portata 1 m, un bersaglio.

\emph{Colpisce:} 10 (2d6 + 3) danni taglienti più 3 (1d6) danni da acido. Se il bersaglio è una creatura di taglia Grande o inferiore, è afferrata (CD 13 per fuggire). Fino al termine dell'afferrare, l'ankheg può mordere solo la creatura afferrata e ha +1d6 ai tiri di attacco contro di essa.

\emph{\textbf{Spruzzo Acido (Ricarica 6).}} L'ankheg sputa acido in una linea lunga 9 metri e larga 1 metro, purché non stia afferrando nessuna creatura. Ogni creatura su quella linea deve effettuare un Tiro Salvezza di Riflessi CD 13, e subire 10 (3d6) danni da acido se fallisce il Tiro Salvezza, o la metà di questi danni se lo riesce.

\textbf{Ecologia}\\
Ambiente: Pianure temperate o calde\\
Organizzazione: Solitario, coppia o nido (3-6)\\
Tesoro: Accidentale\\
\textbf{Descrizione}\\
Gli ankheg sono una piaga fin troppo comune per le zone rurali. Questo mostro scavatore grande come un cavallo in genere evita le zone abitate più popolose, ma la sua predilezione per la carne del bestiame e degli esseri umani li tiene lontano dalle zone disabitate. Il loro habitat favorito è rappresentato delle campagne rurali, dato che il terriccio smosso rende loro molto facile muoversi scavando. Si narra di ankheg più grandi che vivono nei deserti remoti, che si cibano di Scorpioni e Cammelli e vengono raramente in contatto con la civiltà (un ankheg del deserto è un ankheg avanzato Enorme).\\
In combattimento, gli ankheg preferiscono attaccare con il morso. Contro più avversari, un ankheg Afferra uno dei bersagli e tenta di ritirarsi sottoterra. Una creatura trascinata sottoterra può respirare, anche se con difficoltà (anche l'ankheg deve farlo, quindi i tunnel sono abbastanza porosi), ma spesso viene mangiata viva prima che i suoi compagni possano salvarla.\\
Gli ankheg scavano con le loro gambe e le mandibole, muovendosi velocissimi attraverso il terriccio, la sabbia e la ghiaia (non la roccia). Un ankheg che scava si ferma spesso a costruire tunnel, cospargendo le pareti con una densa secrezione orale. Se un ankheg vuole costruire un tunnel mentre scava deve muoversi a metà della propria velocità di scavare. Un tipico tunnel di ankheg è alto e largo 3 metri, di forma vagamente circolare e lungo da 18 a 45 metri ([1d10+5]×10). I gruppi di ankheg condividono lo stesso territorio e creano complesse reti di tunnel sotto le campagne, a volte creando voragini nei punti in cui troppi di essi scavano allo stesso tempo.\\
Anche se gli ankheg somigliano ad immensi insetti sono più intelligenti e, con un po' di tempo e un buon addestratore, possono diventare animali domestici o da carico. Il fatto che anche "addomesticati" gli ankheg tendano a sputare acido quando spaventati o sorpresi li rende poco sicuri nelle regioni più civilizzate, ma fra le razze selvagge, come gli Hobgoblin, i Trogloditi e soprattutto gli Orchi sono popolari come guardiani o perfino animali da salotto. Un ankheg può raggiungere una lunghezza di 3 metri e pesare circa 400 kg.\\


\medskip\index{Mostri - Arpia}\textbf{Arpia}

\emph{Media mostruosità, caotico malvagio}

\textbf{FORZA} +1

\textbf{DESTREZZA} +1

\textbf{COSTITUZIONE} +1

\textbf{INTELLIGENZA} -2

\textbf{SAGGEZZA} +0

\textbf{CARISMA} +1

\textbf{Iniziativa} +1 -- \textbf{Difesa} 12

\textbf{Punti Ferita} 38 (7d8 + 7)

\textbf{Movimento} 6 m, volo 12 m

\textbf{Linguaggi} Comune

\textbf{Sfida} 1 (200 PE)

\textbf{Azioni}

\emph{\textbf{Multiattacco.}} L'armatura effettua due attacchi: uno con gli artigli e uno con il randello.

\emph{\textbf{Artigli.} Attacco con arma da mischia}: +3 a colpire, portata 1 m, un bersaglio.

\emph{Colpisce:} 6 (2d4 + 1) danni taglienti.

\emph{\textbf{Randello.} Attacco con arma da mischia}: +3 a colpire, portata 1 m, un bersaglio.

\emph{Colpisce:} 3 (1d4 + 1) danni da botta.

\emph{\textbf{Canto Ammaliatore.}} L'arpia canta una melodia magica. Ogni umanoide e gigante entro 90 metri dall'arpia e che possa udire la canzone deve riuscire un Tiro Salvezza di Volontà CD 11 o restare affascinato fino al termine della canzone. L'arpia deve effettuare un'azione bonus durante il suo prossimo turno per continuare a cantare. Può smettere di cantare in qualsiasi momento. Il canto ha termine se l'arpia è inabile.

Mentre è affascinato dall'arpia, un bersaglio è inabile e ignora le canzoni di altre arpie. Se il bersaglio affascinato si trova a più di 1 metro dall'arpia, il bersaglio deve muoversi durante il proprio turno per dirigersi verso l'arpia usando la via più diretta. Egli non eviterà attacchi di opportunità, ma prima di muoversi in un terreno pericoloso, come lava o un pozzo, e prima di subire danno da qualsiasi fonte che non sia l'arpia, il bersaglio potrà ripetere il Tiro Salvezza. Una creatura può ripetere il Tiro Salvezza al termine di ciascun proprio turno. Se il Tiro Salvezza ha successo, l'effetto ha termine per quel bersaglio.

Un bersaglio che riesce il Tiro Salvezza è immune al canto di quell'arpia per le successive 24 ore.

\textbf{Ecologia}\\
Ambiente: Paludi Temperate\\
Organizzazione: Solitario, coppia o stormo (3-12)\\
Tesoro: Standard (Armatura di Cuoio, Morning Star e altro tesoro)\\
\textbf{Descrizione}\\
Spesso viste come creature malvagie e corrotte, le arpie sanno come gli altri pensano e agiscono. Questa capacità percettiva offre loro un vantaggio nel trovare i loro pasti preferiti. Sebbene le creature selvatiche cadano facilmente vittime del canto ammaliatore, queste malvagie donne-uccello preferiscono pasti conditi con complessi pensieri senzienti. Le facili prede rendono il pasto noioso.\\
Anche se in definitiva selvagge e senza alcun rimorso per le loro azioni, diverse arpie vivono presso le società umanoidi e si divertono a sfruttare le creature che reputano potenziali pasti.\\
Le arpie tendono ad indossare ninnoli e ciondoli rubati alle loro vittime, perché amano compiacersi dei brillanti ornamenti degli uomini. Da vicino queste creature trasudano del puzzo delle loro vittime divorate e raramente lasciano che le creature non ancora ammaliate si avvicinino troppo, cosicché non sentano l'odore del sangue e della putrefazione sulle loro penne. Per questo motivo, molte arpie si cospargono di profumi e oli aromatici.\\
Le arpie sono marcatamente differenti a seconda della regione in cui vivono. Alcune assomigliano ad una mescolanza di avvoltoi e donne, mentre altri portano sulle penne i tratti regali di falchi e falconi. Rare nidiate di arpie, in luoghi isolati e tropicali del mondo, hanno anche piume colorate come i pappagalli.\\


\medskip\index{Mostri - Azer}\textbf{Azer}

\emph{Media elementale, legale neutrale}

\textbf{FORZA} +3

\textbf{DESTREZZA} +1

\textbf{COSTITUZIONE} +2

\textbf{INTELLIGENZA} +1

\textbf{SAGGEZZA} +1

\textbf{CARISMA} +0

\textbf{Iniziativa} +1 -- \textbf{Difesa} 18 (armatura naturale, scudo)

\textbf{Punti Ferita} 39 (6d8 + 12)

\textbf{Movimento} 9 m

\textbf{Tiri Salvezza} Tempra +2, Riflessi +1, Volontà +1

\textbf{Immunità ai Danni} fuoco, veleno

\textbf{Immunità alle Condizioni} avvelenato

\textbf{Linguaggi} Ignan

\textbf{Sfida} 2 (450 PE)

\emph{\textbf{Armi Riscaldate.}} Quando l'azer colpisce con un'arma da mischia in metallo, infligge 3 (1d6) danni da fuoco aggiuntivi (già inclusi nell'attacco).

\emph{\textbf{Corpo Riscaldato.}} Una creatura che entri a contatto con l'azer o lo colpisca con un attacco da mischia mentre si trova entro 1 metro da lui subisce 5 (1d10) danni da fuoco.

\emph{\textbf{Fuoco Vivente.}} Un azer non ha bisogno di cibo, bevande o di dormire.

\emph{\textbf{Illuminazione.}} L'azer irradia luce intensa in un raggio di 3 metri e luce fioca per ulteriori 3 metri.

\textbf{Azioni}

\emph{\textbf{Martello da Guerra.} Attacco con arma da mischia}: +5 a colpire, portata 1 m, un bersaglio.

\emph{Colpisce:} 7 (1d8 + 3) danni da botta, o 8 (1d10 + 3) danni da botta se usato a due mani per effettuare un attacco da mischia, più 3 (1d6) danni da fuoco.

\textbf{Ecologia}\\
Ambiente: Qualsiasi terreno (Piano del Fuoco)\\
Organizzazione: Solitario, coppia, gruppo (3-6), squadra (11-20 più 2 sergenti di 3° livello e 1 capo di 3°-6° livello) o clan (30-100 più 50\% di non combattenti più 1 sergente di 3° livello ogni 20 adulti, 5 tenenti di 5° livello e 3 capitani di 7° livello)\\
Tesoro: Standard (Corazza a Scaglie Perfetta, Martello da Guerra Perfetto, Martello Leggero, altro tesoro)\\
\textbf{Descrizione}\\
Una Razza orgogliosa e industriosa proveniente dal Piano del Fuoco, gli Azer lavorano nelle loro fortezze di bronzo e d'ottone, sempre pronti a combattere la loro lunga e ribollente guerra contro gli Efreet. Gli Azer vivono in una società in cui ogni membro sa qual è il suo posto. Nati con specifici doveri, solitamente legati alle attività del padre o della madre, gli Azer si dedicano a queste occupazioni per tutta la vita. Un sistema di caste provvede a tenere ulteriormente in riga la società Azer. I nobili, che regnano senza dover rendere conto a nessuno, indossano kilt di ottone decorato come simbolo della loro casta, mentre quelli dei mercanti e dei proprietari di negozi sono in resistente bronzo. I kilt di rame sono indossati dalla casta lavoratrice, composta da servitori, artigiani e braccianti.\\
Capaci di incanalare calore tramite le Armi e gli attrezzi in metallo, gli Azer non utilizzano quasi mai Armi non metalliche, e prediligono il corpo a corpo agli attacchi a distanza. Sono soliti fare prigionieri, riportandoli alle loro fortezze e obbligandoli a lavorare per loro per un anno e un giorno.\\
Nella leggendaria Città d'Ottone abitano più di mezzo milione di Azer. La maggior parte di questi sfortunati Azer vive una vita di Schiavitù sotto gli Efreet. Gli Azer soggiogati a questa Schiavitù continuano a eseguire i loro doveri senza porre domande, preferendo aspettare la conclusione dei loro contratti o sperando che i loro padroni muoiano o vengano sconfitti. La dedizione all'ordine arde intensa in questa Razza, al punto che alcuni degli Schiavi Azer fungono da supervisori sulla loro stessa gente. Al di fuori della Città d'Ottone, gli Azer sono liberi di vivere le loro vite, spesso in altre metropoli Planari, creando oggetti, vendendo merci e gestendo taverne.\\
A un occhio non allenato gli Azer si somigliano tra loro in modo impressionante. Sono alti 1,2 metri ma pesano 100 kg.\\


\medskip\index{Mostri - Basilisco}\textbf{Basilisco}

\emph{Media mostruosità, disallineato}

\textbf{FORZA} +3

\textbf{DESTREZZA} -1

\textbf{COSTITUZIONE} +2

\textbf{INTELLIGENZA} -4

\textbf{SAGGEZZA} -1

\textbf{CARISMA} -2

\textbf{Iniziativa} -1 -- \textbf{Difesa} 17

\textbf{Punti Ferita} 52 (8d8 + 16)

\textbf{Movimento} 6 m

\textbf{Sensi} scurovisione 18 m

\textbf{Linguaggi} -

\textbf{Sfida} 3 (700 PE)

\emph{\textbf{Sguardo Pietrificante.}} Se una creatura comincia il suo turno entro 9 metri dal basilisco e i due si possono vedere vicendevolmente, se non  inabile il basilisco può obbligare la creatura ad effettuare un Tiro Salvezza di Tempra CD 12. Se la creatura fallisce il Tiro Salvezza, inizia magicamente a trasformarsi in pietra ed è intralciata. La creatura deve ripetere il Tiro Salvezza al termine del suo prossimo turno. Se lo riesce, l'effetto termina. Se lo fallisce, la creatura è pietrificata finché non viene liberata dall'incantesimo \emph{ristorare} \emph{superiore} o altra magia.

Una creatura che non sia sorpresa, può distogliere lo sguardo per evitare il Tiro Salvezza all'inizio del suo turno. In quel caso, non potrà vedere il basilisco fino all'inizio del suo prossimo turno, quando potrà distogliere nuovamente lo sguardo. Se nel frattempo dovesse guardare il basilisco, dovrebbe immediatamente effettuare il Tiro Salvezza.

Se il basilisco si trova entro 9 metri dal suo riflesso a luce intensa e lo vede, lo scambia per un rivale e diventa il bersaglio del proprio sguardo.

\textbf{Azioni}

\emph{\textbf{Morso.} Attacco con arma da mischia}: +5 a colpire, portata 1 m, un bersaglio.

\emph{Colpisce:} 10 (2d6 + 3) danni perforanti più 7 (2d6) danni da veleno.

\textbf{Ecologia}\\
Ambiente: Qualsiasi\\
Organizzazione: Solitario, coppia o colonia (3-6)\\
Tesoro: Accidentale\\
\textbf{Descrizione}\\
Il basilisco, spesso chiamato "Re dei Serpenti" è un rettile a otto zampe di indole aggressiva che ha la capacità di trasformare le creature in pietra con il suo sguardo. La leggenda narra che, come la Cockatrice, i primi basilischi nacquero da uova deposte da serpenti e covate da galli, ma ben poco nella fisiologia del basilisco lascia spazio a questa teoria.\\
I basilischi vivono in quasi tutti gli ambienti asciutti, dalla foresta al deserto, e la loro pelle tende a rispecchiare l'ambiente che li circonda: un basilisco del deserto può essere bronzeo o marrone, mentre uno che vive nelle foreste può essere di colore verde acceso. Tendono a usare come rifugio le grotte, le tane o altre zone riparate. Questi rifugi sono spesso segnalati da statue raffiguranti persone e animali in pose naturali, che non sono altro che i resti pietrificati degli sventurati imbattutisi in un basilisco.\\
I basilischi hanno la capacità di consumare le creature pietrificate; l'acido prodotto dal loro stomaco dissolve ed estrae sostanze nutrienti dalla pietra, sebbene il processo sia lento e inefficiente, il che li rende pigri e inerti. Di conseguenza, i basilischi raramente attaccano o cacciano le prede che evitano il loro sguardo, contando sulla loro Furtività e l'elemento di sorpresa al fine di non rimanere senza cibo. Quando non sono in attesa dei piccoli mammiferi, uccelli o rettili che fanno parte della loro dieta, i basilischi passano il tempo a dormire nelle tane. Coloro che sono abbastanza coraggiosi da catturare i basilischi o da nascondere un tesoro vicino a loro, scoprono che questi esseri possono fare da custodi o da cani da guardia.\\
Un basilisco adulto è lungo quasi 4 metri, di cui la metà occupata dalla lunga coda, e pesa 135 chili. Alcune razze presentano delle piccole corna ricurve sul naso o piccole creste di pungiglioni ossuti sopra la testa simili a una corona. Sebbene siano creature in genere solitarie che si riuniscono solo per accoppiarsi e deporre le uova, in zone particolarmente pericolose possono riunirsi in piccoli gruppi per proteggersi e attaccare gli intrusi in massa.\\
Per motivi ignoti, le donnole e i furetti sono immuni allo sguardo del basilisco, e a volte si intrufolano nelle tane mentre l'adulto è a caccia per cibarsi dei suoi piccoli. Alcune leggende narrano che il sangue di un basilisco può tramutare comuni pietre in un altro materiale, ma probabilmente si tratta di testimoni che hanno mal interpretato la ristorazione magica di creature o di parti del corpo pietrificate in precedenza.\\


\medskip\index{Mostri - Behir}\textbf{Behir}

\emph{Enorme mostruosità, neutrale malvagio}

\textbf{FORZA} +6

\textbf{DESTREZZA} +3

\textbf{COSTITUZIONE} +4

\textbf{INTELLIGENZA} -2

\textbf{SAGGEZZA} +2

\textbf{CARISMA} +1

\textbf{Iniziativa} +3 -- \textbf{Difesa} 23

\textbf{Punti Ferita} 168 (16d12 + 64)

\textbf{Movimento} 15 m, scalata 12 m

\textbf{Competenze} Muoversi Silenziosamente / Nascondersi +7, Consapevolezza +6

\textbf{Immunità al Danno} fulmine

\textbf{Sensi} scurovisione 27 m

\textbf{Linguaggi} Draconico

\textbf{Sfida} 11 (7.200 PE)

\textbf{Azioni}

\emph{\textbf{Multiattacco.}} Il behir effettua due attacchi: uno con il morso e uno per stritolare.

\emph{\textbf{Morso.} Attacco con arma da mischia}: +10 a colpire, portata 3 m, un bersaglio.

\emph{Colpisce:} 22 (3d10 + 6) danni perforanti.

\emph{\textbf{Stritolare.} Attacco con arma da mischia}: +10 a colpire, portata 1 m, una creatura di taglia Grande o inferiore.

\emph{Colpisce:} 17 (2d10 + 6) danni da botta più 17 (2d10 + 6) danni taglienti. Il bersaglio è afferrato (CD 16 per fuggire) Se il behir non sta già stritolando un'altra creatura, il bersaglio è afferrato e intralciato fino al termine dell'afferrare.

\emph{\textbf{Inghiottire.}} Il behir effettua una attacco di morso contro un bersaglio di taglia Media o inferiore che sta afferrando. Se l'attacco colpisce, il bersaglio è inghiottito, e l'afferrare ha termine. Il bersaglio inghiottito è accecato e intralciato, ha copertura totale contro gli attacchi e altri effetti all'esterno del behir, e subisce 21 (6d6) danni da acido all'inizio di ciascun turno del behir. Il behir può inghiottire solo una creatura alla volta.

Se il behir subisce 30 o più danni in un singolo turno da una creatura che ha inghiottito, deve riuscire un Tiro Salvezza di Tempra CD 14 al termine di quel turno o vomitare la creatura, che ricade prona in uno spazio entro 3 metri dal behir. Se il behir muore, una creatura inghiottita non è più intralciata da esso e può uscire dal cadavere utilizzando 5 metri di movimento, uscendo prona.

\emph{\textbf{Soffio di Fulmine (Ricarica 5-6).}} Il behir esala fulmini in una linea lunga 6 metri e larga 1 metro. Ogni creatura su quella linea deve effettuare un Tiro Salvezza di Riflessi CD 16 e subire 66 (12d10) danni da fulmine se fallisce il Tiro Salvezza, o la metà di questi danni se lo riesce.

\textbf{Ecologia}\\
Ambiente: Colline e Deserti Caldi\\
Organizzazione: Solitario o coppia\\
Tesoro: Doppio\\
\textbf{Descrizione}\\
Istintivo e bramoso, il behir trascorre gran parte del tempo a strisciare per le colline sabbiose e le rocce del deserto che formano il suo territorio, dando la caccia a tutte le creature che osano entrare nel suo territorio. Le sue sei paia di zampe robuste e dotate di artigli restano piegate ai suoi fianchi per gran parte del tempo, e si stendono solo in combattimento per afferrare i nemici, per Correre al galoppo o per Scalare i pendii delle scogliere a picco, tane di queste creature.\\
In media il behir è lungo 12 metri e pesa circa 1800 Kg. Oltre alle due corna prominenti sulla testa, molti hanno aculei decorativi a intervalli regolari lungo la spina dorsale.\\
Pur essendo territoriale e bestiale nella sua furia, il behir non è né stupido né necessariamente malvagio anche se, a causa del suo egocentrismo e della tendenza a rivendicare come sua ogni cosa esistente, entra spesso in conflitto con le altre razze. In quanto tale, un behir può essere corrotto o convinto da intrepidi negoziatori disposti ad avvicinarglisi. In questi casi, la tendenza di un behir ad attaccare prima e a ragionare poi (o non ragionare affatto) significa che chiunque cerchi di trovare un accordo deve avere dei validi motivi e far subito colpo sul behir con un'offerta allettante.\\
Spesso si dice che i behir siano in qualche modo legati ai draghi blu, ma la vera natura di questo Legame rimane un mistero. Molti draghi negano qualsiasi Legame e non vedono di buon occhio i behir per la loro scarsa intelligenza: un affronto che fa infuriare i behir, di per sé già impulsivi. Proprio per questo, molti behir portano rancore verso i draghi e sono pronti ad attaccare qualunque drago entri nel loro territorio.\\


\medskip\index{Mostri - Bugbear}\textbf{Bugbear}

\emph{Media umanoide (goblinoide), caotico malvagio}

\textbf{FORZA} +2

\textbf{DESTREZZA} +2

\textbf{COSTITUZIONE} +1

\textbf{INTELLIGENZA} -1

\textbf{SAGGEZZA} +0

\textbf{CARISMA} -1

\textbf{Iniziativa} +2 -- \textbf{Difesa} 17

\textbf{Punti Ferita} 27 (5d8 + 5)

\textbf{Movimento} 9 m

\textbf{Competenze} Muoversi Silenziosamente / Nascondersi +6, Sopravvivenza +2

\textbf{Sensi} scurovisione 18 m

\textbf{Linguaggi} Comune, Goblin

\textbf{Sfida} 1 (200 PE)

\emph{\textbf{Attacco di Sorpresa.}} Se il bugbear sorprende una creatura e la colpisce con un attacco durante il primo round di combattimento, il bersaglio subisce 7 (2d6) danni aggiuntivi
dall'attacco.

\emph{\textbf{Bruto.}} Un'arma da mischia infligge un dado aggiuntivo di danno quando il bugbear colpisce con essa (già incluso nell'attacco).

\textbf{Azioni}

\emph{\textbf{Mazza Chiodata.} Attacco con arma da mischia}: +4 a colpire, portata 1 m, un bersaglio.

\emph{Colpisce:} 11 (2d8 + 2) danni perforanti.

\emph{\textbf{Giavellotto.} Attacco con arma da mischia o a Distanza}: +4 a colpire, portata 1 m o gittata 9m, un bersaglio.

\emph{Colpisce:} 9 (2d6 + 2) danni perforanti in mischia o 5 (1d6 + 2) danni perforanti a gittata.

\textbf{Ecologia}\\
Ambiente: Montagne temperate\\
Organizzazione: Solitario, coppia, gruppo (3-6) o banda da guerra (7-12 più 2 Guerrieri di 1° livello e 1 capitano di 3°-5° livello)\\
Tesoro: Equipaggiamento da PNG (Armatura di Cuoio, Scudo Leggero di Legno, Morning Star, 3 Giavellotti, altro tesoro)\\
\textbf{Descrizione}\\
Il bugbear è il più grande degli esponenti della razza Goblinoide, un bruto dai movimenti pesanti che supera di almeno una testa la maggior parte degli Umani. Sono solitari che preferiscono vivere ed uccidere da soli piuttosto che in tribù, sebbene non sia insolito trovare una piccola banda di Bugbear che collabora o vive con una tribù di Goblin od Hobgoblin fungendo da guardia d'élite o carnefici.\\
I bugbear non formano grandi insediamenti come i goblin o nazioni come gli hobgoblin; preferiscono qualcosa di più piccolo e caotico che li lasci liberi di fare quello che preferiscono (uccidere e torturare) a un livello più personale. Gli umani sono le prede preferite dei bugbear, e la maggior parte di essi annovera la carne umana come uno degli alimenti principali della propria dieta. Macabri trofei quali orecchie e dita sono decorazioni comuni tra i bugbear.\\
I bugbear, quando si rivolgono alla religione, prediligono le divinità dell'omicidio e della violenza, con i vari signori dei demoni tra i preferiti.\\
Un tipico bugbear è alto 2,1 metri e pesa 200 kg.\\


\medskip\index{Mostri - Bulette}\textbf{Bulette}

\emph{Grande mostruosità, disallineato}

\textbf{FORZA} +4

\textbf{DESTREZZA} +0

\textbf{COSTITUZIONE} +5

\textbf{INTELLIGENZA} -4

\textbf{SAGGEZZA} +0

\textbf{CARISMA} -3

\textbf{Iniziativa} +0 -- \textbf{Difesa} 20

\textbf{Punti Ferita} 94 (9d10 + 45)

\textbf{Movimento} 12 m, scavo 12 m

\textbf{Competenze} Consapevolezza +6

\textbf{Sensi} scurovisione 18 m, percezione tellurica 18 m

\textbf{Linguaggi} -

\textbf{Sfida} 5 (1.800 PE)

\emph{\textbf{Salto da Fermo.}} Un bulette può saltare in lungo fino a 9  metri e in alto fino a 5 m con o senza la rincorsa.

\textbf{Azioni}

\emph{\textbf{Morso.} Attacco con arma da mischia}: +7 a colpire, portata 1 m, un bersaglio.

\emph{Colpisce:} 30 (4d12 + 4) danni perforanti.

\emph{\textbf{Salto Letale.}} Se il bulette può saltare di almeno 4  metri come parte del suo movimento, può usare poi questa azione per  atterrare in piedi in uno spazio che contiene una o più creature.  Ciascuna di queste creature deve riuscire un Tiro Salvezza di Tempra o Riflessi CD 16 (a scelta del bersaglio) o venire gettata prona e subire  14 (3d6 + 4) danni da botta più 14 (3d6 + 4) danni taglienti. Se il  Tiro Salvezza riesce, la creatura subisce solo la metà dei danni, non è  gettata prona, e viene spinta di 1 metro fuori dello spazio del  bulette in uno spazio non occupato a scelta della creatura. Se non ci  sono spazi non occupati a gittata, la creatura cade prona nello spazio  del bulette.

\textbf{Ecologia}\\
Ambiente: Colline Temperate\\
Organizzazione: Solitario o coppia\\
Tesoro: Nessuno\\
\textbf{Descrizione}\\
Creazione di uno sconosciuto mago del passato, il bulette ora è diventato un feroce predatore di collina. Scavando rapidamente sotto il terreno, fende la superficie con la sua pinna dorsale lasciandosi dietro una scia caratteristica. Il bulette balza fuori, liberandosi da pietre e terriccio, per fare a pezzi la sua preda senza rimorsi, dando così origine al suo soprannome di "squalo terrestre".\\
I bulette sono noti per il pessimo carattere, e attaccano creature molto più grandi di loro senza alcuna paura. Bestie solitarie tranne per le occasionali coppie in fase riproduttiva, passano la maggior parte del tempo pattugliando i loro territori, che possono superare i 4 km2, cacciando e punendo gli intrusi con una furia in grado di scuotere i pendii delle colline.\\
I bulette sono macchine perfette per divorare e distruggere ossa, armature e anche oggetti magici con le loro possenti mascelle e l'acido ribollente del loro stomaco. In mancanza d'altro, un bulette potrebbe sgranocchiare oggetti comuni, ma per qualche ragione non mangia volontariamente carne di elfo, segno forse di un coinvolgimento della magia elfica nella loro creazione, o di nani, anche se può far strage dei membri di entrambe le razze. Gli Halfling, invece, sono tra i cibi preferiti di queste bestie, e non ci sono Halfling assennati che si avventurino nel territorio di un bulette a cuor leggero.\\
Il bulette è un combattente astuto, e sorprende i nemici con agilità impressionante. Una delle sue tattiche preferite è lanciarsi alla carica e balzare sulla preda attaccando con i suoi artigli affilati come rasoi. Si dice che la carne dietro la cresta dorsale della bestia sia particolarmente tenera, e che quanti vogliano o riescano ad attendere che la pinna venga sollevata nella concitazione del combattimento o dell'accoppiamento possano tentare di sferrare un colpo mortale in quel punto, anche se la maggior parte di quelli che hanno affrontato uno squalo terrestre concordano sul fatto che il miglior modo per vincere un combattimento con un bulette sia evitarlo del tutto.\\


\medskip\index{Mostri - Centauro}\textbf{Centauro}

\emph{Grande mostruosità, neutrale buono}

\textbf{FORZA} +4

\textbf{DESTREZZA} +2

\textbf{COSTITUZIONE} +2

\textbf{INTELLIGENZA} -1

\textbf{SAGGEZZA} +1

\textbf{CARISMA} +0

\textbf{Iniziativa} +2 -- \textbf{Difesa} 13

\textbf{Punti Ferita} 45 (6d10 + 12)

\textbf{Movimento} 15 m

\textbf{Competenze} Acrobatica +6, Consapevolezza +3, Sopravvivenza +3

\textbf{Linguaggi} Elfico, Silvano

\textbf{Sfida} 2 (450 PE)

\emph{\textbf{Carica.}} Se il centauro si muove di almeno 9 metri diretto verso il bersaglio e colpisce con un attacco di picca durante lo stesso turno, il bersaglio subisce 10 (3d6) danni perforanti aggiuntivi.

\textbf{Azioni}

\emph{\textbf{Multiattacco.}} Il centauro effettua due attacchi: uno con
la picca e uno con gli zoccoli o due con l'arco lungo.

\emph{\textbf{Picca.} Attacco con arma da mischia}: +6 a colpire,
portata 3 m, un bersaglio.

\emph{Colpisce:} 9 (1d10 + 4) danni perforanti.

\emph{\textbf{Zoccoli.} Attacco con arma da mischia}: +6 a colpire,
portata 1 m, un bersaglio.

\emph{Colpisce:} 11 (2d6 + 4) danni da botta.

\emph{\textbf{Arco Lungo.} Attacco con arma a Distanza}: +4 a colpire, gittata 45m, un bersaglio.

\emph{Colpisce:} 6 (1d8 + 2) danni perforanti.

\textbf{Ecologia}\\
Ambiente: Pianure e foreste temperate\\
Organizzazione: Solitario, coppia, banda (3-10), tribù (11-30 più 3 cacciatori di 3° livello e 1 capo di 6° livello)\\
Tesoro: Standard (Corazza di Piastre, Scudo Pesante di Metallo, Spada Lunga, Lancia, altro tesoro)\\
\textbf{Descrizione}\\
Leggendari cacciatori e abili guerrieri, i centauri sono in parte uomini e in parte cavalli. Generalmente collocata ai margini della civilizzazione, questa stoica popolazione varia enormemente come aspetto: di solito il colore della pelle è molto abbronzato ma simile a quello degli umani delle regioni limitrofe, mentre la parte inferiore del corpo ha le tonalità degli equini locali. Hanno capelli e occhi di colore scuro e i tratti del volto piuttosto marcati, mentre la loro stazza totale dipende dalla taglia del cavallo di cui hanno la parte inferiore del corpo. Quindi, anche se un centauro medio è alto in piedi 2,1 metri e pesa più di 1.000 kg, esistono molteplici varianti regionali, dagli esili corridori delle pianure ai massicci cacciatori di montagna.\\
I centauri vivono in media circa 60 anni. Distanti dalle altre razze e in conflitto con gli altri della loro specie, i centauri sono una razza antica che lentamente comincia ad accettare il mondo moderno. Anche se la maggioranza dei centauri vive ancora in tribù vagando per vaste pianure o ai margini di mistiche foreste, alcuni hanno abbandonato i modi isolazionisti dei loro antenati per stabilirsi in città cosmopolite. Spesso questi spiriti liberi sono considerati dei reietti e vengono disprezzati dalle loro tribù, e pertanto la decisione di abbandonarle è una scelta pesante. In alcuni casi, comunque, intere tribù guidate da capi progressisti hanno cominciato a commerciare o stringere alleanze con altre comunità di umanoidi, specie Elfi, a volte Gnomi, e più raramente Umani o Nani. Molte razze rimangono caute nei confronti dei centauri, però, per lo più a causa di leggende che li ritraggono come creature territoriali e feroci e dei periodici scontri violenti che essi hanno con i coloni testardi e i paesi in via di espansione.\\


\medskip\index{Mostri - Chimera}\textbf{Chimera}

\emph{Grande mostruosità, caotico malvagio}

\textbf{FORZA} +4

\textbf{DESTREZZA} +0

\textbf{COSTITUZIONE} +4

\textbf{INTELLIGENZA} -4

\textbf{SAGGEZZA} +2

\textbf{CARISMA} +0

\textbf{Iniziativa} +0 -- \textbf{Difesa} 17

\textbf{Punti Ferita} 114 (12d10 + 48)

\textbf{Movimento} 9 m, volo 18 m

\textbf{Competenze} Consapevolezza +8

\textbf{Sensi} scurovisione 18 m

\textbf{Linguaggi} comprende il Draconico ma non può parlare

\textbf{Sfida} 6 (2.300 PE)

\textbf{Azioni}

\emph{\textbf{Multiattacco.}} La chimera effettua tre attacchi: uno con  il morso, uno con le corna e uno con gli artigli. Quando il soffio infuocato  è disponibile, può usare il soffio al posto del morso o delle corna.

\emph{\textbf{Artigli.} Attacco con arma da mischia}: +7 a colpire,  portata 1 m, un bersaglio.

\emph{Colpisce:} 11 (2d6 + 4) danni taglienti.

\emph{\textbf{Corna.} Attacco con arma da mischia}: +7 a colpire,  portata 1 m, un bersaglio.

\emph{Colpisce:} 10 (1d12 + 4) danni da botta.

\emph{\textbf{Morso.} Attacco con arma da mischia}: +7 a colpire,  portata 1 m, un bersaglio.

\emph{Colpisce:} 11 (2d6 + 4) danni perforanti.

\emph{\textbf{Soffio Infuocato (Ricarica 5-6).}} La testa di drago esala  fuoco in un cono di 5 metri. Ogni creatura in quell'area deve  effettuare un Tiro Salvezza di Riflessi CD 15 e subire 31 (7d8) danni  da fuoco se fallisce il Tiro Salvezza, o la metà di questi danni se lo  riesce.

\textbf{Ecologia}\\
Ambiente: Colline Temperate\\
Organizzazione: Solitario, coppia, branco (3-6) o stormo (7-12)\\
Tesoro: Standard\\
\textbf{Descrizione}\\
Le chimere sono mostruose creature nate dal male primordiale. Odiose e fameliche, cacciano sia a terra che in aria. La testa di drago di una chimera può essere di qualunque tipo di drago malvagio, con il soffio corrispondente e le ali generalmente dotate delle stesse scaglie della testa. Le chimere parlano con tre voci che si sovrappongono, ma lo fanno raramente, tipicamente solo per adulare una creatura più potente. Una chimera è alta al garrese 1 metro, raggiungendo la lunghezza di 3 metri e il peso di 350 kg.\\
Le chimere preferiscono la carne, ma possono sopravvivere di vegetali se necessario (anche se quando sono costrette a farlo il loro umore peggiora ulteriormente). Il fatto che volino significa che possono scegliere con attenzione le loro prede, e generalmente cacciano in vaste aree cercando quelle facili. Sono troppo stupide e belligeranti per acquisire seguaci, anche se a volte una tribù di coboldi può far loro delle offerte. Al contrario, sono abbastanza intelligenti e caparbie da essere mediocri animali domestici, e solo una creatura molto più potente di loro può riuscire a sottometterle. Possono formare collaborazioni paritarie con umanoidi rispettosi o creature simili, e acconsentono anche ad essere usate come cavalcature. Un branco di chimere ha una gerarchia simile a quella dei leoni, con un maschio dominante che comanda il gruppo e la maggior parte delle cacce svolte dalle femmine. Una chimera solitaria può essere un giovane maschio solitario o una femmina con i cuccioli nelle vicinanze.\\


\medskip\index{Mostri - Chuul}\textbf{Chuul}

\emph{Grande aberrazione, caotico malvagio}

\textbf{FORZA} +4

\textbf{DESTREZZA} +0

\textbf{COSTITUZIONE} +3

\textbf{INTELLIGENZA} -3

\textbf{SAGGEZZA} +0

\textbf{CARISMA} -3

\textbf{Iniziativa} +0 -- \textbf{Difesa} 18

\textbf{Punti Ferita} 93 (11d10 + 33)

\textbf{Movimento} 9 m, nuoto 9 m

\textbf{Competenze} Consapevolezza +4

\textbf{Immunità ai Danni} veleno

\textbf{Immunità alle Condizioni} avvelenato

\textbf{Sensi} scurovisione 18 m

\textbf{Linguaggi} comprende la Linguaggio delle Profondità ma non può
parlare

\textbf{Sfida} 4 (1.100 PE)

\emph{\textbf{Anfibio.}} Il chuul può respirare aria e acqua.

\emph{\textbf{Senso della Magia.}} Il chuul percepisce la magia entro 36  metri da sé. Questo tratto funziona come l'incantesimo  \emph{individuazione} \emph{del magico} ma di per sé non è magico.  

\textbf{Azioni}

\emph{\textbf{Multiattacco.}} Il chuul effettua due attacchi con le  chele. Se il chuul sta afferrando una creatura, può anche usare i suoi tentacoli una volta.

\emph{\textbf{Chele.} Attacco con arma da mischia}: +6 a colpire,  portata 3 m, un bersaglio.

\emph{Colpisce:} 11 (2d6 + 4) danni da botta. Un bersaglio è  afferrato (CD 14 per fuggire) se è di taglia Grande o inferiore e il  chuul non sta già afferrando altre due creature.

\emph{\textbf{Tentacoli.}} Una creatura afferrata dal chuul deve  riuscire un Tiro Salvezza di Tempra CD 13 o restare avvelenata per  1 minuto. Fino al termine dell'avvelenamento, il bersaglio è  paralizzato. Il bersaglio può ripetere il Tiro Salvezza al termine di  ciascun suo turno, terminando l'effetto per sé in caso di successo.

\textbf{Ecologia}\\
Ambiente: Paludi Temperate\\
Organizzazione: Solitario, coppia o branco (3-6)\\
Tesoro: Standard\\
\textbf{Descrizione}\\
I chuul sono predatori corazzati simili ai crostacei, sempre in agguato sotto la superficie degli stagni e dei pantani poco profondi, che escono dal loro nascondiglio per afferrare le loro prede con le loro chele e poi paralizzarle con i tentacoli della bocca prima di mangiarle vive.\\
I chuul sono eccellenti nuotatori, ma preferiscono attaccare le creature terrestri o abituate ad acque poco profonde. Una volta afferrate le loro vittime, i chuul spesso le trascinano nell'acqua profonda. I lucertoloidi sono le prede preferite dei chuul, anche se le pallide specie di chuul che vivono nei sotterranei preferiscono morlock, duergar, incauti drow e altri sfortunati che si avvicinano troppo ai loro corsi d'acqua sotterranei, ad eccezione dei trogloditi il cui sapore i chuul trovano particolarmente disgustoso.\\
I chuul sono sorprendentemente intelligenti e molti si impegnano in inutili speculazioni sulle loro origini e motivazioni. Parlano un cinguettante e gorgogliante dialetto del Comune, ma pochi di essi sono inclini a chiacchierare con quanti non siano della loro razza, e se esiste una società chuul al di fuori del frenetico periodo degli amori, nessuno lo ha ancora scoperto. Al contrario, le menti dei chuul sembrano dedite solo alla ricerca del luogo perfetto in cui tendere un'imboscata per attaccare altre creature intelligenti e a come decorare le loro elaborate tane con trofei delle loro vittime. Anche se i chuul sembrano non interessati all'utilizzo di utensili, hanno un bisogno compulsivo di collezionare quelli delle loro vittime.\\
Un tipico chuul è alto 2,4 metri e pesa 325 kg.\\


\medskip\index{Mostri - Coboldo}\textbf{Coboldo}

\emph{Piccola umanoide (coboldo), legale malvagio}

\textbf{FORZA} -2

\textbf{DESTREZZA} +2

\textbf{COSTITUZIONE} -1

\textbf{INTELLIGENZA} -1

\textbf{SAGGEZZA} -2

\textbf{CARISMA} -1

\textbf{Iniziativa} +2 -- \textbf{Difesa} 13

\textbf{Punti Ferita} 5 (2d6 - 2)

\textbf{Movimento} 9 m

\textbf{Sensi} scurovisione 18 m

\textbf{Linguaggi} Comune, Draconico

\textbf{Sfida} 1/8 (25 PE)

\emph{\textbf{Sensibilità alla Luce}}. Mentre è alla luce del sole, il coboldo ha -1d6 ai tiri per colpire, oltre che alle prove di Saggezza (Consapevolezza) basate sulla vista.

\emph{\textbf{Tattiche di Branco.}} Il coboldo ha +1d6 ai tiri per colpire contro una creatura se almeno uno degli alleati del coboldo si trova entro 1 metro dalla creatura e quell'alleato non è inabile.

\textbf{Azioni}

\emph{\textbf{Pugnale.} Attacco con arma da mischia}: +4 a colpire,
portata 1 m, un bersaglio.

\emph{Colpisce:} 4 (1d4 + 2) danni perforanti.

\emph{\textbf{Fionda.} Attacco con arma a distanza}: +4 a colpire, gittata 9m, un bersaglio.

\emph{Colpisce:} 4 (1d4 + 2) danni da botta.

\textbf{Ecologia}\\
Ambiente: Foreste temperate o sotterranei\\
Organizzazione: solitario, gruppo (2-4), nido (5-30 più un ugual numero di non combattenti, 1 sergente di 3° livello ogni 20 adulti e 1 capo di 4°-6° livello) o tribù (31-300 più di 35\% di non combattenti, 1 sergente di 3° livello ogni 20 adulti, 2 tenenti di 4° livello, 1 capo di 6°-8° livello e 5-16 Ratti Crudeli)\\
Tesoro: Equipaggiamento da PNG (Armatura di Cuoio, Lancia, Fionda, altro tesoro)\\
\textbf{Descrizione}\\
I coboldi sono creature dell'oscurità, che si incontrano più facilmente in enormi dedali sotterranei o negli angoli bui delle foreste dove il sole non batte mai. A causa della somiglianza fisica, i coboldi si proclamano a gran voce eredi della stirpe draconica e destinati a governare la terra sotto l'ala dei loro grandi cugini divini, ma la maggior parte dei draghi li considera poco più che insetti fastidiosi. Ma, anche se proclamano discendenze divine e l'evidenza del loro destino, i coboldi sono consapevoli della loro debolezza. Codardi ed intriganti, non lottano mai apertamente se possono evitarlo, tendendo invece imboscate e trappole, rintanandosi nei loro dedali dietro una coltre di rozzi ma ingegnosi trabocchetti, o rovesciandosi sul nemico in vaste orde ululanti.\\
La tonalità dei coboldi varia anche tra i fratelli della stessa covata, spaziando tra i colori dei draghi cromatici, con una predominanza del rosso, e più di rado bianco, verde, blu e nero.\\


\medskip\index{Mostri - Cockatrice}\textbf{Cockatrice}

\emph{Piccola mostruosità, disallineato}

\textbf{FORZA} -2

\textbf{DESTREZZA} +1

\textbf{COSTITUZIONE} +1

\textbf{INTELLIGENZA} -4

\textbf{SAGGEZZA} +1

\textbf{CARISMA} -3

\textbf{Iniziativa} +1 -- \textbf{Difesa} 12

\textbf{Punti Ferita} 27 (6d6 + 6)

\textbf{Movimento} 6 m, volo 12 m

\textbf{Sensi} scurovisione 18 m

\textbf{Linguaggi} -

\textbf{Sfida} 1/2 (100 PE)

\textbf{Azioni}

\emph{\textbf{Morso.} Attacco con arma da mischia}: +3 a colpire, portata 1 m, una creatura.

\emph{Colpisce:} 3 (1d4 + 1) danni perforanti, e il bersaglio deve riuscire un Tiro Salvezza di Tempra CD 11 per non essere magicamente pietrificato. Se fallisce il Tiro Salvezza, la creatura inizia a trasformarsi in pietra ed è intralciata. Al termine del turno successivo deve ripetere il Tiro Salvezza. Se lo riesce, l'effetto ha termine. Se lo fallisce, la creatura è pietrificata per 24 ore.

\textbf{Ecologia}\\
Ambiente: Pianure temperate\\
Organizzazione: Solitario, coppia, squadriglia (3-5) o stormo (6-12)\\
Tesoro: Nessuno\\
\textbf{Descrizione}\\
Stupide, malevole e repellenti, le cockatrici sono evitate dalle altre creature per la loro capacità di trasformare la carne in pietra. Le leggende affermano che la prima cockatrice emerse da un uovo deposto da un gallo e covato da un rospo. Che questa storia sia vera o no, le cockatrici odierne si riproducono fra loro in tane terrificanti e sporche scavate a casaccio da almeno una dozzina di creature chioccianti. I maschi sono molto più numerosi delle femmine in questi stormi, e si distinguono solo per barbigli e creste. Una tipica cockatrice è alta poco più di 60 centimetri e pesa 2,5 kg.\\
Anche se la loro dieta consiste principalmente di semi e insetti pietrificati (che nello stomaco della creatura fungono sia da gastroliti che da nutrimento), le cockatrici difendono ferocemente il loro territorio da tutto ciò che ritengono una minaccia, e i vagabondaggi dei maschi raminghi in cerca di nuovi luoghi dove costruire tane a volte li portano ad involontari contatti con gli umani, con risultati devastanti.\\
La strana capacità della cockatrice di trasformare le altre creature in pietra è la sua miglior difesa, e la sua tana è invariabilmente piena di resti dei nemici pietrificati. Per ironia della sorte, tuttavia, donnole e furetti, le creature che più probabilmente finiscono nei nidi delle cockatrici per mangiarne le uova, sembrano completamente immuni a questo effetto. Per ragioni sconosciute, le cockatrici sono sia terrorizzate che furiose con i galli comuni, e c'è la stessa probabilità che fuggano o attacchino quando avviene un confronto.\\


\medskip\index{Mostri - Couatl}\textbf{Couatl}

\emph{Media celestiale, legale buono}

\textbf{FORZA} +3

\textbf{DESTREZZA} +5

\textbf{COSTITUZIONE} +3

\textbf{INTELLIGENZA} +4

\textbf{SAGGEZZA} +5

\textbf{CARISMA} +4

\textbf{Iniziativa} +5 -- \textbf{Difesa} 21

\textbf{Punti Ferita} 97 (13d8 + 39)

\textbf{Movimento} 9 m, volo 9 m

\textbf{Tiri Salvezza} Tempra +9, Riflessi +13, Volontà +14

\textbf{Resistenze al Danno} da Luce

\textbf{Immunità al Danno} psichico; da botta, perforante e tagliente di attacchi non magici

\textbf{Sensi} visione del vero 36 m

\textbf{Linguaggi} tutte, telepatia 36 m 

\textbf{Sfida} 4 (1.100 PE)

\emph{\textbf{Armi Magiche.}} Gli attacchi con armi del couatl sono magici.

\emph{\textbf{Incantesimi Innati.}} La caratteristica da incantatore innato del couatl è il Carisma. Il couatl può lanciare questi incantesimi in maniera innata, usando solo componenti verbali:

A volontà: \emph{individuazione del bene e del male, individuazione del magico, individuazione dei pensieri}

3/giorno ciascuno: \emph{benedizione, creare cibo e acqua, cura ferite,} \emph{protezione dai veleni, ristorare inferiore, santuario, scudo} 1/giorno ciascuno: \emph{ristorare superiore, scrutare, sogno}

\emph{\textbf{Mente Protetta.}} Il couatl è immune allo scrutare e qualsiasi effetto che percepisca le sue emozioni, legga i suoi pensieri o individui la sua posizione.

\textbf{Azioni}

\emph{\textbf{Morso.} Attacco con arma da mischia}: +8 a colpire, portata 1 m, una creatura.

\emph{Colpisce:} 8 (1d6 + 5) danni perforanti, e il bersaglio deve riuscire un Tiro Salvezza di Tempra CD 13 o restare avvelenato per 24 ore. Fino al termine dell'avvelenamento, il bersaglio è privo di sensi. Un'altra creatura può effettuare un'azione per risvegliare il bersaglio.

\emph{\textbf{Stritolare.} Attacco con arma da mischia}: +6 a colpire, portata 3 m, una creatura di taglia Media o inferiore.

\emph{Colpisce:} 10 (2d6 + 3) danni da botta, e il bersaglio è afferrato (CD 15 per fuggire). Fino al termine dell'afferrare, il bersaglio è intralciato, e il couatl non può stritolare un altro bersaglio.

\emph{\textbf{Mutare Forma.}} Il couatl può trasformarsi magicamente in un umanoide o bestia il cui grado di sfida sia pari o inferiore al proprio, o tornare alla sua vera forma. Alla morte ritorna alla sua vera forma. Qualsiasi equipaggiamento stia indossando o trasportando viene assorbito o trasportato nella nuova forma (a scelta del couatl).

Nella nuova forma, il couatl mantiene le sue statistiche di gioco e la facoltà di parlare, ma la sua Difesa, metodi di movimento, Forza, Destrezza e altre azioni vengono rimpiazzati da quelli della nuova forma, e ottiene qualsiasi statistica o capacità (Azioni aggiuntive e azioni da tana) possedute dalla sua nuova forma e non dalla sua originale. Se la nuova forma ha un attacco di morso, il couatl può usare il proprio morso nella nuova forma.

\textbf{Ecologia}\\
Ambiente: foreste calde\\
Organizzazione: Solitario, coppia o stormo (3-6)\\
Tesoro: Standard\\
\textbf{Descrizione}\\
I couatl sono servitori di divinità legali buone, anche se alcuni operano in maniera indipendente da qualsiasi entità superiore. Rispettati ed ammirati per la loro saggezza e bellezza, cercano di portare i mortali sulla retta via e usano i loro poteri per combattere il male, specie quelli noti per viaggiare tra i piani. Alcuni couatl sono visti come divinità benevole da società isolate e, anche se i couatl rabbrividiscono al solo pensiero di fingere di essere una divinità, consentono che si perpetuino questi malintesi poiché permettono loro di guidare queste società su sentieri di pace e cooperazione con i loro vicini. Un couatl è lungo circa 3,6 metri, con un'apertura alare di circa 5 metri e pesa 900 kg.\\
Come esterni nativi, i couatl devono mangiare. Preferiscono gli stessi alimenti dei veri serpenti, come mammiferi e uccelli, anche se è noto che divorano gli umanoidi malvagi. Poiché preferiscono passare il tempo a perseguire i loro intenti anziché cacciare, apprezzano le offerte di cibo, in particolare piccoli cinghiali e volatili. Un couatl talvolta mostra il suo apprezzamento a un avventuriero o a un gruppo che gli ha reso un servizio donandogli 1d4 delle sue brillanti piume colorate. Queste piume ottenute gratuitamente, se usate come componente materiale aggiuntivo, permettono ad un incantatore che lancia Alleato Planare di evocare quello specifico couatl senza pagare il normale costo in oro o altri valori, a condizione che il couatl approvi il servizio richiesto dall'incantatore.\\


\medskip\index{Mostri - Cumulo Strisciante}\textbf{Cumulo Strisciante}

\emph{Grande pianta, disallineato}

\textbf{FORZA} +4

\textbf{DESTREZZA} -1

\textbf{COSTITUZIONE} +3

\textbf{INTELLIGENZA} -3

\textbf{SAGGEZZA} +0

\textbf{CARISMA} -3

\textbf{Iniziativa} -1 -- \textbf{Difesa} 18

\textbf{Punti Ferita} 136 (16d10 + 48)

\textbf{Movimento} 6 m, nuoto 6 m

\textbf{Competenze} Muoversi Silenziosamente / Nascondersi +2

\textbf{Resistenze al Danno} freddo, fuoco

\textbf{Immunità al Danno} fulmine

\textbf{Immunità alle Condizioni} accecato, assordato, affaticamento

\textbf{Sensi} vista cieca 18 m (cieco oltre questo raggio)

\textbf{Linguaggi} -

\textbf{Sfida} 5 (1.800 PE)

\emph{\textbf{Assorbimento dei Fulmini.}} Ogni qual volta il cumulo strisciante subisce danni da fulmine, non subisce danni e recupera un numero di punti ferita pari al danno da fulmine inferto.

\textbf{Azioni}

\emph{\textbf{Multiattacco.}} Il cumulo strisciante effettua due attacchi di schianto. Se entrambi gli attacchi colpiscono una creatura di taglia Media o inferiore, il bersaglio è afferrato (CD 14 per fuggire) e il cumulo strisciante usa Avvolgere su di esso.

\emph{\textbf{Schianto.} Attacco con arma da mischia}: +7 a colpire, portata 1 m, un bersaglio.

\emph{Colpisce:} 13 (2d8 + 4) danni da botta.

\emph{\textbf{Avvolgere.}} Il cumulo strisciante avvolge una creatura di taglia Media o inferiore che ha afferrato. Il bersaglio avvolto è accecato, intralciato e impossibilitato a respirare, e deve riuscire un Tiro Salvezza di Tempra CD 14 all'inizio di ciascun turno del tumulo o subire 13 (2d8 + 4) danni da botta. Se il cumulo si muove, il bersaglio avvolto si muove con esso. Il cumulo può avvolgere solo una creatura alla volta.

\textbf{Ecologia}\\
Ambiente: Foreste o Paludi Temperate\\
Organizzazione: Solitario\\
Tesoro: Standard\\
\textbf{Descrizione}\\
I cumuli striscianti, chiamati anche soltanto striscianti, sembrano masse vegetali in decomposizione. Sono piante carnivore intelligenti, con un debole per la carne elfica. Il cervello e gli organi sensoriali si trovano nella parte superiore del corpo. Di solito i cumuli striscianti hanno una circonferenza di 2,4 metri e sono alti da 1,8 a 2,7 metri. Pesano circa 1.900 kg.\\
I cumuli striscianti sono strane creature, più simili a un groviglio di rampicanti parassiti che ad una singola pianta dotata di radici. Sono onnivori, capaci di trarre sostentamento da qualsiasi cosa, avvinghiandosi agli alberi per succhiarne la linfa, inserendo le radici nel terreno per assorbire nutrienti semplici o consumando la carne e le ossa dalle prede.\\
I cumuli striscianti sono incredibilmente furtivi nel loro ambiente naturale. Si confondono con il terreno circostante e possono attendere immobili per giorni l'arrivo di una potenziale preda. Possono essere praticamente ovunque ed attaccare in qualsiasi momento senza alcun preavviso e senza curarsi che ci siano o meno sopravvissuti, fintanto che hanno da mangiare.\\
Di solito i cumuli striscianti conducono un'esistenza nomade e solitaria in profonde foreste e fetide paludi ma possono essere trovati anche sottoterra, in mezzo a boschetti di funghi. Voci preoccupanti parlano di gruppi di cumuli striscianti che si radunano intorno a grandi tumuli nelle profondità di giungle e paludi, spesso durante violente tempeste di fulmini. Il motivo di questo comportamento è sconosciuto, e molti saggi si chiedono se dietro non ci sia uno scopo oscuro ed alieno.\\


\subsection{Demoni}

\medskip\index{Mostri - Balor}\textbf{Balor}

\emph{Enorme immondo (demone), caotico malvagio}

\textbf{FORZA} +8

\textbf{DESTREZZA} +2

\textbf{COSTITUZIONE} +6

\textbf{INTELLIGENZA} +5

\textbf{SAGGEZZA} +3

\textbf{CARISMA} +6

\textbf{Iniziativa} +5 -- \textbf{Difesa} 29

\textbf{Punti Ferita} 262 (21d12 + 126)

\textbf{Movimento} 12 m, volo 24 m

\textbf{Tiri Salvezza} Tempra +29, Riflessi +17, Volontà +25

\textbf{Resistenze al Danno} freddo, fulmine; 

\textbf{Immunità al Danno} fuoco, veleno , armi +1

textbf{Immunità alle Condizioni} avvelenato

\textbf{Vulnerabilità al Danno} ferro freddo

\textbf{Sensi} visione del vero 36 m

\textbf{Linguaggi} Abissale, telepatia 36 m

\textbf{Sfida} 19 (22.000 PE)

\emph{\textbf{Armi Magiche.}} Gli attacchi con arma del demone sono magici.

\emph{\textbf{Aura di Fuoco.}} All'inizio di ciascun turno del demone, ciascuna creatura entro 1 metro da lui subisce 10 (3d6) danni da fuoco, e gli oggetti infiammabili che si trovano nell'aura e che non sono indossati o trasportati prendono fuoco. Una creatura che entri a contatto con il demone o lo colpisca con un attacco da mischia mentre si trova entro 1 metro da esso subisce 10 (3d6) danni da fuoco.

\emph{\textbf{Resistenza alla Magia.}} Il demone ha +1d6 ai Tiri Salvezza contro incantesimi e altri effetti magici.

\emph{\textbf{Spasmo Mortale.}} Quando il demone muore, esplode; ciascuna creatura entro 9 metri da esso deve effettuare un Tiro Salvezza di Riflessi CD 20, subendo 70 (20d6) danni da fuoco se fallisce il Tiro Salvezza, o la metà di questi danni se lo riesce. L'esplosione appicca il fuoco agli oggetti infiammabili che non sono indossati o trasportati, e distrugge le armi del demone.

\textbf{Azioni}

\emph{\textbf{Multiattacco.}} Il demone effettua due attacchi: uno con la spada lunga e uno con la frusta.

\emph{\textbf{Frusta.} Attacco con arma da mischia}: +14 a colpire, portata 9 m, un bersaglio.

\emph{Colpisce:} 15 (2d6 + 8) danni taglienti più 10 (3d6) danni da fuoco, e il bersaglio deve riuscire un Tiro Salvezza di Tempra CD 20 o venire trascinato 7 metri verso il demone.

\emph{\textbf{Spada Lunga.} Attacco con arma da mischia}: +14 a colpire, portata 3 m, un bersaglio.

\emph{Colpisce:} 21 (3d8 + 8) danni taglienti più 13 (3d8) danni da fulmine. Se il demone ottiene un colpo critico, tira il danno tre volte, invece che due.

\emph{\textbf{Teletrasporto.}} Il demone si teletrasporta magicamente, insieme a tutto l'equipaggiamento che indossa o trasporta, in uno spazio non occupato e che può vedere entro 36 metri.

\textbf{Ecologia}\\
Ambiente: Qualsiasi (Abisso)\\
Organizzazione: Solitario o banda di guerra (1 Balor e 2-5 Glabrezu)\\
Tesoro: Standard (Spada Lunga Sacrilega+1, Frusta Infuocata+1, altro tesoro)\\
\textbf{Descrizione}\\
Quando la gente sussurra terrificanti racconti di creature demoniache, immagina per lo più un'imponente figura di fuoco e carne, un incubo cornuto armato di frusta e spada fiammeggianti, che vola nella notte in cerca delle sue prede. Il demone che queste persone temono è il balor, e questa paura è pienamente giustificata, dal momento che pochi demoni possono eguagliare il possente balor in forza o in brutalità.\\
Nell'Abisso, i balor sono per lo più al servizio dei signori dei demoni, in qualità di generali o capitani (quando non si tratti di balor estremamente potenti, noti come signori dei balor). Un balor solitamente comanda vaste legioni di demoni e, sebbene spesso consenta a questi servi bramosi e sbavanti di combattere le sue battaglie, è tutt'altro che un codardo. Se si presenta l'opportunità di unirsi ad uno scontro, sono pochi i balor che scelgono di trattenersi.\\
Un balor è alto 4,2 metri e pesa 2.250 kg. Solo le anime mortali più crudeli possono alimentare la creazione di un balor: a differenza degli altri demoni, spesso occorrono numerose anime di potenti malvagi per far nascere un nuovo balor.\\


\medskip\index{Mostri - Dretch}\textbf{Dretch}

\emph{Piccola immondo (demone), caotico malvagio}

\textbf{FORZA} +0

\textbf{DESTREZZA} +0

\textbf{COSTITUZIONE} +1

\textbf{INTELLIGENZA} -3

\textbf{SAGGEZZA} -1

\textbf{CARISMA} -4

\textbf{Iniziativa} +0 -- \textbf{Difesa} 12

\textbf{Punti Ferita} 18 (4d6 + 4)

\textbf{Movimento} 6 m

\textbf{Resistenze al Danno} freddo, fulmine, fuoco

\textbf{Immunità al Danno} veleno

\textbf{Immunità alle Condizioni} avvelenato

\textbf{Vulnerabilità al Danno} ferro freddo

\textbf{Sensi} scurovisione 18 m

\textbf{Linguaggi} Abissale, telepatia 18 m (funziona solo con le creature che comprendono l'Abissale)

\textbf{Sfida} 1/4 (50 PE)

\textbf{Azioni}

\emph{\textbf{Multiattacco.}} Il demone effettua due attacchi: uno con il morso e uno con gli artigli.

\emph{\textbf{Artigli.} Attacco con arma da mischia}: +2 a colpire, portata 1 m, un bersaglio.

\emph{Colpisce:} 5 (2d4) danni taglienti.

\emph{\textbf{Morso.} Attacco con arma da mischia}: +2 a colpire, portata 1 m, un bersaglio.

\emph{Colpisce:} 3 (1d6) danni perforanti.

\emph{\textbf{Nube Fetida (1/Giorno).}} Un disgustoso gas verde si estende in un raggio di 3 metri dal demone. Il gas si propaga intorno agli angoli, e la sua area è oscurata leggermente. Rimane per 1 minuto o finché non viene disperso da un forte vento. Qualsiasi creatura che inizi il proprio turno in quell'area deve riuscire un Tiro Salvezza di Tempra CD 11 o restare avvelenata fino all'inizio del suo prossimo turno. Mentre è avvelenato in questo modo, il bersaglio, durante il suo turno, può effettuare solo un'azione o un'azione bonus, ma non entrambe, e non può effettuare reazioni.

\textbf{Ecologia}\\
Ambiente: Qualsiasi (Abisso)\\
Organizzazione: Solitario, coppia, banda (3-5), gruppo (6-12) o folla (13+)\\
Tesoro: Nessuno\\
\textbf{Descrizione}\\
Anche il più infimo demone dell'Abisso è pericoloso e possiede la necessità impellente di spargere rovina e sgomento. Il miserabile dretch è tanto orripilante e fetido quanto crudele, anche se non possiede la forza ed il potere per riuscire a soddisfare la sua voglia di brutalizzare gli altri nel suo reame nativo. Lo scopo dell'esistenza dei dretch è quello di servire demoni più potenti come vittime sacrificabili, e solo pochi fortunati riescono a sopravvivere abbastanza a lungo da evolversi.\\
I dretch sono i bersagli preferiti dai dilettanti in evocazioni abissali. Relativamente deboli e facili da intimorire, i dretch spesso possono essere obbligati a lunghi periodi di servitù utilizzando vaghe promesse di opportunità di sfogare le loro frustrazioni e la loro rabbia contro avversari più deboli. Eppure il potenziale evocatore di dretch farebbe meglio a ricordarsi che questi demoni sono codardi ed infidi quanto gli altri demoni. Un dretch che si trova di fronte un nemico più potente sarà assai lieto di scambiare qualsiasi informazione di cui disponga in cambio della sua miserevole vita.\\
A differenza della maggior parte dei demoni, la sciatta personalità del dretch ed il suo disprezzo per il lavoro fisico prolungato raramente danno dei risultati. I dretch avanzati sono rari, ma quelli che riescono a trovare la forza in se stessi per diventare più di quello che erano al momento della loro creazione divengono i sovrani poveri dell'Abisso, crudeli ed amareggiati, che regnano su parassiti, anime spezzate, non morti privi di intelletto e altri dretch. I loro imperi sono limitati a tratti abbandonati di fogne sotto città dimenticate, instabili distese paludose evitate dalle menti più sensate ed altri sgraditi angoli dell'Abisso che persino i demoni considerano scomodi o ripugnanti. Eppure per i signori dei dretch questi regni sono i loro imperi, e li difendono con pietosa tenacia.\\
Un dretch è alto 1,2 metri e pesa 90 kg. I dretch solitamente si formano dalle anime di mortali malvagi ed indolenti: è sufficiente solo un piccolo frammento di anima per dare origine ad una nascita così orripilante. Una sola anima spesso può causare l'apparizione di una piccola armata di dretch, e la vista di un'orda di dretch appena nati che si liberano dalla protomateria pulsante dell'Abisso è al contempo nauseante e terrificante.\\


\medskip\index{Mostri - Glabrezu}\textbf{Glabrezu}

\emph{Grande immondo (demone), caotico malvagio}

\textbf{FORZA} +5

\textbf{DESTREZZA} +2

\textbf{COSTITUZIONE} +5

\textbf{INTELLIGENZA} +4

\textbf{SAGGEZZA} +3

\textbf{CARISMA} +3

\textbf{Iniziativa} +4 -- \textbf{Difesa} 22

\textbf{Punti Ferita} 157 (15d10 + 75)

\textbf{Movimento} 12 m

\textbf{Tiri Salvezza} Tempra +18, Riflessi +4, Volontà +11

\textbf{Resistenze al Danno} freddo, fulmine, fuoco; da botta, perforante e tagliente di attacchi non magici

\textbf{Immunità al Danno} veleno

\textbf{Immunità alle Condizioni} avvelenato

\textbf{Vulnerabilità al Danno} ferro freddo

\textbf{Sensi} visione del vero 36 m

\textbf{Linguaggi} Abissale, telepatia 36 m 

\textbf{Sfida} 9 (5.000 PE)

\emph{\textbf{Incantesimi Innati.}} La caratteristica da incantatore del demone è l'Intelligenza. Il demone può lanciare questi incantesimi in maniera innata, senza bisogno di componenti materiali:

A volontà: \emph{dissolvi magie, individuazione del magico, oscurità}

1/giorno ciascuno: \emph{confusione, parola del potere stordire, volare}

\emph{\textbf{Resistenza alla Magia.}} Il demone ha +1d6 ai Tiri Salvezza contro incantesimi e altri effetti magici.

\textbf{Azioni}

\emph{\textbf{Multiattacco.}} Il demone effettua quattro attacchi: due con le chele e due con i pugni. In alternativa, può effettuare due attacchi con le chele e lanciare un incantesimo.

\emph{\textbf{Chela.} Attacco con arma da mischia}: +9 a colpire, portata 3 m, un bersaglio.

\emph{Colpisce:} 16 (2d10 + 5) danni da botta. Se il bersaglio è una creatura di taglia Media o inferiore, è afferrato (CD 15 per fuggire). Il glabrezu possiede due chele, ciascuna delle quali può afferrare un bersaglio.

\emph{\textbf{Pugno.} Attacco in mischia con arma}: +9 a colpire, portata 1 m, un bersaglio.

\emph{Colpisce:} 7 (2d4 + 2) danni da botta.

\textbf{Ecologia}\\
Ambiente: Qualsiasi (Abisso)\\
Organizzazione: Solitario o drappello (1 glabrezu, 1 Succube e 2-5 Vrock)\\
Tesoro: Standard\\
\textbf{Descrizione}\\
Mentre la Succube è un demone che adesca la sua preda sfruttandone i desideri e le necessità carnali, il glabrezu è un tentatore di altro genere. Feroce e dalla forma bestiale, il glabrezu è in realtà un maestro di inganni e bugie. Con la sua abilità di nascondere la sua vera forma dietro piacenti illusioni, usa la sua magia per esaudire i desideri degli umanoidi mortali, come forma di ricompensa per coloro che soccombono ai suoi inganni e raggiri. Un desiderio esaudito da un glabrezu appaga la necessità di chi lo esprime nel modo più rovinoso possibile, sebbene queste conseguenze possano non rivelarsi immediatamente tali. Un fabbro che fatica ad affermarsi potrebbe desiderare fama ed abilità nella professione scelta, solo per scoprire che il suo miglior patrono è un crudele e sadico omicida che usa le armi per promuovere i propri distruttivi desideri. Un uomo solo che esprime il desiderio di avere una compagna, potrebbe vedere il suo desiderio avverarsi con una sua vecchia fiamma ritornata alla "vita" in forma di vampiro, ed altri esempi di questo tipo. Il glabrezu è assai creativo nel soddisfare i desideri di un mortale.\\
Un glabrezu è alto 5,4 metri e pesa poco più di 3.000 kg. Questi perfidi demoni si originano dalle anime dei traditori, dei falsi e dei sovversivi: anime di mortali che, in vita, giurarono il falso o utilizzarono il tradimento e l'inganno per rovinare le vite altrui.\\


\medskip\index{Mostri - Hezrou}\textbf{Hezrou}

\emph{Grande immondo (demone), caotico malvagio}

\textbf{FORZA} +4

\textbf{DESTREZZA} +3

\textbf{COSTITUZIONE} +5

\textbf{INTELLIGENZA} 5 (-2)

\textbf{SAGGEZZA} +1

\textbf{CARISMA} +1

\textbf{Iniziativa} +3 -- \textbf{Difesa} 20

\textbf{Punti Ferita} 136 (13d10 + 65)

\textbf{Movimento} 9 m

\textbf{Tiri Salvezza} Tempra +16, Riflessi +3, Volontà +9

\textbf{Resistenze al Danno} freddo, fulmine, fuoco; da botta, perforante e tagliente di attacchi non magici

\textbf{Immunità al Danno} veleno

\textbf{Immunità alle Condizioni} avvelenato

\textbf{Vulnerabilità al Danno} ferro freddo

\textbf{Sensi} scurovisione 36 m

\textbf{Linguaggi} Abissale, telepatia 36 m 

\textbf{Sfida} 8 (3.900 PE)

\emph{\textbf{Fetore.}} Qualsiasi creatura che inizi il suo turno entro 3 metri dal demone, deve riuscire un Tiro Salvezza di Tempra CD 14 o restare avvelenata fino all'inizio del proprio turno. Se riesce il Tiro Salvezza, la creatura è immune al fetore del demone gracidante per 24 ore.

\emph{\textbf{Resistenza alla Magia.}} Il demone ha +1d6 ai Tiri Salvezza contro incantesimi e altri effetti magici.

\textbf{Azioni}

\emph{\textbf{Multiattacco.}} Il demone effettua tre attacchi: uno con il morso e due con gli artigli.

\emph{\textbf{Artiglio.} Attacco con arma da mischia}: +7 a colpire, portata 1 m, un bersaglio.

\emph{Colpisce:} 11 (2d6 + 4) danni taglienti.

\emph{\textbf{Morso.} Attacco con arma da mischia}: +7 a colpire, portata 1 m, un bersaglio.

\emph{Colpisce:} 15 (2d10 + 4) danni perforanti.

\textbf{Ecologia}\\
Ambiente: Qualsiasi (Abisso)\\
Organizzazione: Solitario o banda (2-4)\\
Tesoro: Standard\\
\textbf{Descrizione}\\
l'hezrou vive nelle vaste paludi, acquitrini e corsi d'acqua dell'Abisso, a suo agio sia nell'acqua che sulla terraferma. La presenza di un hezrou ha un effetto dannoso su flora, causando nodosità e mutazioni, e acque circostanti, rendendole maleodoranti e dal sapore salmastro, peculiarità più facilmente individuabili nel Piano Materiale che nell'Abisso. L'esposizione prolungata a questa corruzione causa orrende trasformazioni e deformità. Spesso intere comunità isolate di mutanti deformi devono il loro aspetto contorto non tanto ai loro depravati costumi quanto alla vicinanza di un hezrou.\\
Sebbene sia abbastanza intelligente, si può onestamente dire che un hezrou sprechi il proprio intelletto. Questi esseri preferiscono i piaceri più semplici: dormire, il gusto della tortura, la beatitudine di cibarsi di carne vivente o la gioia di sentire qualcosa di bello rompersi e sbriciolarsi nella stretta dei loro pugni. Non cercano spesso di costruire imperi o porsi alla testa di culti, sebbene pochi hezrou rifiuterebbero potenziali seguaci che vengano ad offrirglisi di loro spontanea volontà.\\
Queste mostruose e bestiali creature nascono dalle anime di mortali malvagi che hanno avvelenato se stessi, i loro parenti o il loro ambiente, ad esempio, drogati, assassini ed alchimisti che non si sono preoccupati di come i loro esperimenti avvelenassero il mondo naturale.\\


\medskip\index{Mostri - Marilith}\textbf{Marilith}

\emph{Grande immondo (demone), caotico malvagio}

\textbf{FORZA} +4

\textbf{DESTREZZA} +5

\textbf{COSTITUZIONE} +5

\textbf{INTELLIGENZA} +4

\textbf{SAGGEZZA} +3

\textbf{CARISMA} +5

\textbf{Iniziativa} +5 -- \textbf{Difesa} 26

\textbf{Punti Ferita} 189 (18d10 + 90)

\textbf{Movimento} 12 m

\textbf{Tiri Salvezza} Tempra +25, Riflessi +18, Volontà +13

\textbf{Resistenze al Danno} freddo, fulmine, fuoco

\textbf{Immunità al Danno} veleno, armi +1

\textbf{Immunità alle Condizioni} avvelenato

\textbf{Vulnerabilità al Danno} ferro freddo

\textbf{Sensi} visione del vero 36 m

\textbf{Linguaggi} Abissale, telepatia 36 m 

\textbf{Sfida} 16 (15.000 PE)

\emph{\textbf{Armi Magiche.}} Gli attacchi con armi del demone sono magici.

\emph{\textbf{Reattivo.}} Il demone può effettuare una reazione durante ciascun turno di combattimento.

\emph{\textbf{Resistenza alla Magia.}} Il demone ha +1d6 ai Tiri Salvezza contro incantesimi e altri effetti magici.

\textbf{Azioni}

\emph{\textbf{Multiattacco.}} Il demone effettua sette attacchi: sei con le spade lunghe e uno con la coda.

\emph{\textbf{Coda.} Attacco con arma da mischia}: +9 a colpire, portata 3 m, una creatura.

\emph{Colpisce:} 15 (2d10 + 4) danni da botta. Se il bersaglio è di taglia Media o inferiore, è afferrato (CD 19 per fuggire). Fino al termine dell'afferrare, il bersaglio è intralciato, e il demone può colpire automaticamente il bersaglio con la coda, ma non può effettuare attacchi di coda contro altri bersagli.

\emph{\textbf{Spada Lunga.} Attacco con arma da mischia}: +9 a colpire, portata 1 m, un bersaglio.

\emph{Colpisce:} 13 (2d8 + 4) danni taglienti.

\textbf{Reazioni}

\emph{\textbf{Parata.}} Il demone somma 5 alla sua Difesa contro un attacco da mischia che lo colpirebbe. Per farlo, il demone deve poter vedere il suo attaccante e impugnare un'arma da mischia.

\textbf{Ecologia}\\
Ambiente: Qualsiasi (Abisso)\\
Organizzazione: Solitario, coppia o plotone (1 marilith, 1-3 Glabrezu e 3-14 Babau)\\
Tesoro: Doppio (6 Spade Lunghe, altro tesoro)\\
\textbf{Descrizione}\\
Sovrane di orde demoniache e regine di nazioni abissali, le temibili marilith servono i signori dei demoni come governanti, consigliere e persino amanti, eppure la loro supremazia come strateghe le rende particolarmente richieste come generali e comandanti d'armate. Le marilith più potenti non sono al servizio di nessuno e comandano invece fameliche legioni demoniache.\\
Una marilith è alta da 1,8 a 2,7 metri, lunga 6 metri dalla testa alla punta della coda, e pesa 2.000 kg. Solo le anime malvagie più arroganti ed orgogliose, solitamente quelle di crudeli sovrani, sadici generali e signori della guerra particolarmente violenti, possono causare la nascita di una marilith.\\


\medskip\index{Mostri - Nalfeshnee}\textbf{Nalfeshnee}

\emph{Grande immondo (demone), caotico malvagio}

\textbf{FORZA} +5

\textbf{DESTREZZA} +0

\textbf{COSTITUZIONE} +6

\textbf{INTELLIGENZA} +4

\textbf{SAGGEZZA} +1

\textbf{CARISMA} +2

\textbf{Iniziativa} +4 -- \textbf{Difesa} 25

\textbf{Punti Ferita} 184 (16d10 + 96)

\textbf{Movimento} 6 m, volo 9 m

\textbf{Tiri Salvezza} Tempra +22, Riflessi +9, Volontà +21

\textbf{Resistenze al Danno} freddo, fulmine, fuoco; da botta, perforante e tagliente di attacchi non magici

\textbf{Immunità al Danno} veleno

\textbf{Immunità alle Condizioni} avvelenato

\textbf{Vulnerabilità al Danno} ferro freddo

\textbf{Sensi} scurovisione 36 m

\textbf{Linguaggi} Abissale, telepatia 36 m 

\textbf{Sfida} 13 (10.000 PE)

\emph{\textbf{Resistenza alla Magia.}} Il demone ha +1d6 ai Tiri Salvezza contro incantesimi e altri effetti magici.

\textbf{Azioni}

\emph{\textbf{Multiattacco.}} Il demone usa, se possibile, Aureola di Orrore. Poi effettua tre attacchi: uno con il morso e due con gli artigli.

\emph{\textbf{Artiglio.} Attacco con arma da mischia}: +10 a colpire, portata 3 m, un bersaglio.

\emph{Colpisce:} 15 (3d6 + 5) danni taglienti.

\emph{\textbf{Morso.} Attacco con arma da mischia}: +10 a colpire, portata 1 m, un bersaglio.

\emph{Colpisce:} 32 (5d10 + 5) danni perforanti.

\emph{\textbf{Aureola di Orrore (Ricarica 5-6).}} Il demone emette una luce magica multicolore e scintillante. Ogni creatura entro 5 metri dal demone e che possa vedere la luce, deve riuscire un Tiro Salvezza su Volontà CD 15 o restare spaventata per 1 minuto. Una creatura può ripetere il Tiro Salvezza al termine di ciascun suo turno, terminando l'effetto per sé se lo riesce. Se il Tiro Salvezza della creatura riesce o l'effetto ha termine per essa, la creatura è immune all'Aureola di
Orrore del demone gemente per le successive 24 ore.

\emph{\textbf{Teletrasporto.}} Il demone si teletrasporta, insieme a tutto l'equipaggiamento che sta indossando o trasportando, in uno spazio non occupato che possa vedere fino a 36 metri di distanza.

\textbf{Ecologia}
Ambiente: Qualsiasi (Abisso)\\
Organizzazione: Solitario o banda di guerra (1 nalfeshnee, 1 Hezrou e 2-5 Vrock)\\
Tesoro: Standard\\
\textbf{Descrizione}\\
Sono pochi i demoni che comprendono le meccaniche interne che regolano l'Abisso come i nalfeshnee, e non è raro che questi demoni servano l'Abisso stesso invece che un signore dei demoni. Alcuni sovrintendono i reami organici che generano i nuovi demoni, mentre altri custodiscono luoghi di particolare importanza nei recessi nascosti del piano. Spesso il regno di un nalfeshnee nell'Abisso è superiore per forze e dimensioni al più grande dei regni mortali, in quanto questi demoni hanno una predisposizione naturale a governare ed imporre una sorta di ordine al caos dell'Abisso. Gli evocatori mortali spesso li richiamano per il loro folle ma impareggiabile intelletto, esaminando accuratamente gli accordi presi con questi demoni onde evitare eventuali conseguenze nascoste e risvolti non voluti, in quanto un nalfeshnee raramente accetta qualcosa che, in qualche modo contorto, non gli consenta di soddisfare le necessità ed i desideri dell'Abisso.\\
I nalfeshnee sono alti 6 metri e pesano 4.000 kg. Sono creati dalle anime di malvagi mortali avari o bramosi, in particolare di coloro che hanno regnato su imperi di schiavitù, furto, brigantaggio e altri vizi ancora più violenti.\\


\medskip\index{Mostri - Quasit}\textbf{Quasit}

\emph{Minuscola immondo (demone, mutaforma), caotico malvagio}

\textbf{FORZA} -3

\textbf{DESTREZZA} +3

\textbf{COSTITUZIONE} +0

\textbf{INTELLIGENZA} -2

\textbf{SAGGEZZA} +0

\textbf{CARISMA} +0

\textbf{Iniziativa} +3 -- \textbf{Difesa} 14

\textbf{Punti Ferita} 7 (3d4)

\textbf{Movimento} 12 m (3 m, volo 12 m in forma di pipistrello; 12 m, scalata 12 m in forma di centopiedi; 12 m, nuoto 12 m in forma di rospo)

\textbf{Competenze} Muoversi Silenziosamente / Nascondersi +5

\textbf{Resistenze al Danno} freddo, fulmine, fuoco; da botta, perforante e tagliente di attacchi non magici

\textbf{Immunità al Danno} veleno 

\textbf{Immunità alle Condizioni}
avvelenato

\textbf{Sensi} scurovisione 36 m

\textbf{Linguaggi} Abissale, Comune

\textbf{Sfida} 1 (200 PE)

\emph{\textbf{Mutaforma.}} Il demone può usare la sua azione per trasformarsi in una forma bestiale da pipistrello, centopiedi o rospo, o per tornare alla sua vera forma. Le sue statistiche sono le stesse in tutte le forme, sebbene gli attacchi possano variare per alcune di esse. Qualsiasi equipaggiamento stia indossando o trasportando non viene trasformato. Alla morte ritorna alla sua vera forma.

\emph{\textbf{Resistenza alla Magia.}} Il demone ha +1d6 ai Tiri Salvezza contro incantesimi e altri effetti magici.

\textbf{Azioni}

\emph{\textbf{Artigli (Morso in Forma di Bestia).} Attacco con arma da  mischia}: +4 a colpire, portata 1 m, un bersaglio. \emph{Colpisce:} 5 (1d4 + 3) danni perforanti. Se il bersaglio è una creatura, deve riuscire un Tiro Salvezza di Tempra CD 10 o subire 5 (2d4) danni da veleno e restare avvelenato per 1 minuto. La creatura può ripetere il Tiro Salvezza al termine di ciascun suo turno, ponendo termine all'effetto se lo riesce.

\emph{\textbf{Invisibilità.}} Il demone resta invisibile finché non attacca o termina la sua concentrazione. Qualsiasi cosa che il demone stia trasportando o indossando resta invisibile finché rimane in contatto con il demone.

\emph{\textbf{Spavento (1/Giorno).}} Una creatura scelta dal demone che si trovi entro 6 metri da lui, deve riuscire un Tiro Salvezza su Volontà CD 10 o restare spaventata per 1 minuto. Il bersaglio può ripetere il Tiro Salvezza al termine di ciascun suo turno, con -1d6 se il demone è in linea di visuale, ponendo termine all'effetto prematuramente se riesce il Tiro Salvezza.

\textbf{Ecologia}\\
Ambiente: Qualsiasi (Abisso)\\
Organizzazione: Solitario o stormo (2-12)\\
Tesoro: Standard\\
\textbf{Descrizione}\\
Il quasit è forse il demone meno potente, ma non è tra i meno rispettati: persino i quasit si ritengono superiori alle orde di Dretch e, fedeli alla propria natura, i Dretch mancano del coraggio o degli stimoli necessari a dimostrare loro che si sbagliano. Il ruolo primario in vita di un quasit è quello di famiglio al servizio di un incantatore, ma quei quasit che sfuggono a questa umiliante servitù acquisiscono una volontà propria e sono molto più pericolosi. Un quasit tipico è alto 45 centimetri e pesa solo 4 kg.\\
Unici tra le orde demoniache, i quasit non nascono dalle anime di malvagi mortali deceduti, ma da anime viventi: quando un incantatore cerca di richiamare a sé un quasit come famiglio, la sua anima sfiora l'Abisso ed esso reagisce, creando dalla sua materia un quasit collegato all'anima dell'incantatore e generando un potente legame tra i due.\\
I quasit appena creati vengono alla luce direttamente nel Piano Materiale, dove diventano famigli e, finché sono soggetti alla volontà del loro padrone, lo odiano e disprezzano, dal momento che possono percepire il pulsare delle sua anima e sanno che potrebbero aspirare a qualcosa di più. Un quasit serve, eppure osserva e vigila nell'attesa di errori che possano costare la vita al suo signore, o meglio, che gli consentano di rivoltarsi contro il proprio padrone. Quando il padrone di un quasit muore, questi può cercare di seguirne l'anima nel Grande Oltre, superando un Tiro Salvezza su Volontà con DC 15. Questo effetto funziona come Spostamento Planare ma influisce solo sul quasit e lo trasporta nell'Abisso, facendo diventare sua l'anima del padrone, in forma di larva, piuttosto che utilizzarla per creare nuove forme di vita demoniache. In questo modo, un quasit può usare l'anima appena catturata per contrattare con abitanti più potenti dei piani inferiori, e magari raggiungere un'abietta "promozione" che lo trasformi in una forma di vita più potente.\\
Raramente un quasit decide di ignorare la morte del proprio padrone e di rimanere nel Piano Materiale in cerca di altri modi per divertirsi: solitamente insediandosi in un'area urbana dove ci sono molti individui da tormentare\\


\medskip\index{Mostri - Vrock}\textbf{Vrock}

\emph{Grande immondo (demone), caotico malvagio}

\textbf{FORZA} +3

\textbf{DESTREZZA} +2

\textbf{COSTITUZIONE} +4

\textbf{INTELLIGENZA} -1

\textbf{SAGGEZZA} +1

\textbf{CARISMA} -1

\textbf{Iniziativa} +2 -- \textbf{Difesa} 18

\textbf{Punti Ferita} 104 (11d10 + 44)

\textbf{Movimento} 12 m, volo 18 m

\textbf{Tiri Salvezza} Tempra +13, Riflessi +10, Volontà +6

\textbf{Resistenze al Danno} freddo, fulmine, fuoco; da botta, perforante e tagliente di attacchi non magici

\textbf{Immunità al Danno} veleno 

\textbf{Immunità alle Condizioni} avvelenato

\textbf{Sensi} scurovisione 36 m

\textbf{Linguaggi} Abissale, telepatia 36 m

\textbf{Sfida} 6 (2.300 PE)

\emph{\textbf{Resistenza alla Magia.}} Il demone ha +1d6 ai Tiri Salvezza contro incantesimi e altri effetti magici.

\textbf{Azioni}

\emph{\textbf{Multiattacco.}} Il demone effettua due attacchi: uno con il becco e uno con gli speroni.

\emph{\textbf{Becco.} Attacco con arma da mischia}: +6 a colpire, portata 1 m, un bersaglio.

\emph{Colpisce:} 10 (2d6 + 3) danni perforanti.

\emph{\textbf{Speroni.} Attacco con arma da mischia}: +6 a colpire, portata 1 m, un bersaglio.

\emph{Colpisce:} 14 (2d10 + 3) danni taglienti.

\emph{\textbf{Spore (Ricarica 6).}} Una nube di spore tossiche si diffonde in un raggio di 5 metri intorno al demone. Le spore si propagano intorno agli angoli. Ogni creatura in quell'area deve riuscire un Tiro Salvezza di Tempra CD 14 o restare avvelenata. Mentre   avvelenato in questo modo, un bersaglio subisce 5 (1d10) danni da   veleno all'inizio di ciascun suo turno. Il bersaglio può ripetere il   Tiro Salvezza al termine di ciascun suo turno, ponendo termine   all'effetto se lo riesce. Anche svuotare una fiala di acqua sacra sul   bersaglio pone termine all'effetto.

\emph{\textbf{Strillo Stordente (1/Giorno).}} Il demone emette uno strillo orripilante. Ogni creatura entro 6 metri da esso e che lo possa udire, e non sia un demone, deve riuscire un Tiro Salvezza su Tempra CD 14 o restare stordita fino al termine del prossimo turno del demone.

\textbf{Ecologia}\\
Ambiente: Qualsiasi (Abisso)\\
Organizzazione: Solitario, coppia o banda (3-10)\\
Tesoro: Standard\\
\textbf{Descrizione}\\
Profani campioni dell'Abisso, i vrock incarnano tutta la rabbia, l'odio e la violenza di questo reame. Tanto voraci e grottescamente opportunisti quanto il saprofago a cui assomigliano, i vrock si deliziano nello spargimento di sangue, godendo del suono e delle sensazioni derivanti dallo strappare gli intestini ancora pulsanti da una creatura vivente.\\
Un vrock tipico è alto 2,4 metri e pesa 200 kg. Queste creature solitamente si originano dalle anime di malvagi mortali colmi di odio e di collera, in particolare coloro che erano criminali professionisti, mercenari o assassini.\\



\medskip\index{Mostri - Destriero da Incubo}\textbf{Destriero da Incubo}

\emph{Grande immondo, neutrale malvagio}

\textbf{FORZA} +4

\textbf{DESTREZZA} +2

\textbf{COSTITUZIONE} +3

\textbf{INTELLIGENZA} +0

\textbf{SAGGEZZA} +1

\textbf{CARISMA} +2

\textbf{Iniziativa} +2 -- \textbf{Difesa} 15

\textbf{Punti Ferita} 68 (8d10 + 24)

\textbf{Movimento} 18 m, volo 24 m

\textbf{Immunità al Danno} fuoco

\textbf{Linguaggi} comprende Abissale, Comune e Infernale ma non può parlare

\textbf{Sfida} 3 (700 PE)

\emph{\textbf{Conferire Resistenza al Fuoco.}} Il destriero da incubo può conferire resistenza al danno da fuoco a chiunque lo cavalchi.

\emph{\textbf{Illuminazione.}} Il destriero da incubo irradia luce intensa in un raggio di 3 metri e luce fioca per ulteriori 3 metri.

\textbf{Azioni}

\emph{\textbf{Zoccoli.} Attacco con arma da mischia}: +6 a colpire, portata 1 m, un bersaglio.

\emph{Colpisce:} 13 (2d8 + 4) danni da botta più 7 (2d6) danni da fuoco.

\emph{\textbf{Passo Etereo.}} Il destriero da incubo e fino a tre creature consenzienti entro 1 metro da esso possono entrare magicamente nel Piano Etereo dal Piano Materiale e viceversa.

\textbf{Ecologia}\\
Ambiente: Qualsiasi (Abaddon)\\
Organizzazione: Solitario\\
Tesoro: Nessuno\\
\textbf{Descrizione}\\
Gli incubi sono fiammeggianti messaggeri di morte. Permettono solo alle creature più malvagie di cavalcarli, e non sono mai soltanto cavalcature, ma correi nella distruzione provocata dai loro cavalieri.\\



\subsection{Diavoli}

\medskip\index{Mostri - Diavolo Barbuto}\textbf{Diavolo Barbuto}

\emph{Media immondo (diavolo), legale malvagio}

\textbf{FORZA} +3

\textbf{DESTREZZA} +2

\textbf{COSTITUZIONE} +2

\textbf{INTELLIGENZA} -1

\textbf{SAGGEZZA} +0

\textbf{CARISMA} +0

\textbf{Iniziativa} +2 -- \textbf{Difesa} 15

\textbf{Punti Ferita} 52 (8d8 + 16)

\textbf{Movimento} 9 m

\textbf{Tiri Salvezza} Tempra +9, Riflessi +7, Volontà +3

\textbf{Resistenze al Danno} freddo; da botta, perforante e tagliente di attacchi non magici o che non siano argentati

\textbf{Immunità al Danno} fuoco, veleno

\textbf{Immunità alle Condizioni} avvelenato

\textbf{Sensi} scurovisione 36 m

\textbf{Linguaggi} Infernale, telepatia 36 m

\textbf{Sfida} 3 (700 PE)

\emph{\textbf{Resistenza alla Magia.}} Il diavolo ha +1d6 ai Tiri Salvezza contro incantesimi e altri effetti magici.

\emph{\textbf{Risoluto.}} Il diavolo non può essere spaventato finché riesce a vedere una creatura alleata entro 9 metri da lui.

\emph{\textbf{Vista del Diavolo.}} La scurovisione del diavolo non è limitata dall'oscurità magica.

\textbf{Azioni}

\emph{\textbf{Multiattacco.}} Il diavolo effettua due attacchi: uno con la barba e uno con il falcione.

\emph{\textbf{Barba.} Attacco con arma da mischia}: +5 a colpire, portata 1 m, una creatura.

\emph{Colpisce:} 6 (1d8 + 2) danni perforanti, e il bersaglio deve riuscire un Tiro Salvezza di Tempra CD 12 o restare avvelenato per 1 minuto. Mentre è avvelenato in questo modo, il bersaglio non può recuperare punti ferita. Il bersaglio può ripetere il Tiro Salvezza al termine di ciascun suo turno, terminando l'effetto se riesce il Tiro Salvezza.

\emph{\textbf{Falcione.} Attacco con arma da mischia}: +5 a colpire, portata 3 m, un bersaglio.

\emph{Colpisce:} 8 (1d10 + 3) danni taglienti. Se il bersaglio è una creatura, ad esclusione di costrutti e non morti, deve riuscire un Tiro Salvezza su Tempra 12 o perdere 5 (1d10) punti ferita all'inizio di ciascun suo turno a causa della ferita infernale. Ogni volta che il diavolo colpisce il bersaglio ferito con questo attacco, il danno inflitto dalla ferita aumenta di 5 (1d10). Qualsiasi creatura può effettuare un'azione per bloccare la ferita con una prova riuscita di Saggezza (Pronto Soccorso) CD 12. La ferita si richiude anche nel caso in cui il bersaglio riceva della magia guaritrice.

\textbf{Ecologia}\\
Ambiente: Qualsiasi (Inferno)\\
Organizzazione: Solitario, coppia, squadra (3-10) o truppa (10-40)\\
Tesoro: Standard (Falcione, altro tesoro)\\
\textbf{Descrizione}\\
Guerrieri scelti delle legioni infernali, i diavoli barbuti, o barbazu, combattono selvaggiamente in nome dei loro signori infernali e in battaglia comandano orde brutali di dannati. Si radunano e si addestrano con i loro falcioni forgiati negli inferi, tra le volte del terzo girone dell'Inferno, Erebo, ma ritornano inevitabilmente nel primo girone, Averno, per servire al fianco del temibile signore Barbatos.\\
I barbazu amano effettuare attacchi di carica con i loro falcioni e cercano di mantenere una distanza di 3 metri tra loro ed i loro avversari, così che possono utilizzare le loro caratteristiche armi ad asta con la massima efficacia. Contro un avversario che ha una portata superiore (oppure è in grado di evitare la tattica preferita del diavolo), gettano i falcioni e si affidano ai loro artigli ed alle orribili barbe. In posizione eretta i diavoli barbuti sono alti più di 1,8 metri (sebbene la posizione accovacciata che tengono in battaglia li faccia spesso sembrare più bassi) e pesano più di 100 kg.\\


\medskip\index{Mostri - Diavolo delle Catene}\textbf{Diavolo delle Catene}

\emph{Media immondo (diavolo), legale malvagio}

\textbf{FORZA} +4

\textbf{DESTREZZA} +2

\textbf{COSTITUZIONE} +4

\textbf{INTELLIGENZA} +0

\textbf{SAGGEZZA} +1

\textbf{CARISMA} +2

\textbf{Iniziativa} +2 -- \textbf{Difesa} 20

\textbf{Punti Ferita} 85 (10d8 + 40)

\textbf{Movimento} 9 m

\textbf{Tiri Salvezza} Tempra +9, Riflessi +4, Volontà +3

\textbf{Resistenze al Danno} freddo; da botta, perforante e tagliente di attacchi non magici che non siano argentati

\textbf{Immunità al Danno} fuoco, veleno 

\textbf{Immunità alle Condizioni} avvelenato

\textbf{Sensi} scurovisione 36 m

\textbf{Linguaggi} Infernale, telepatia 36 m 

\textbf{Sfida} 8 (3.900 PE)

\emph{\textbf{Resistenza alla Magia.}} Il diavolo ha +1d6 ai Tiri Salvezza contro incantesimi e altri effetti magici.

\emph{\textbf{Vista del Diavolo.}} La scurovisione del diavolo non è limitata dall'oscurità magica.

\textbf{Azioni}

\emph{\textbf{Multiattacco.}} Il diavolo effettua due attacchi con la catena.

\emph{\textbf{Catena.} Attacco con arma da mischia}: +8 a colpire, portata 3 m, un bersaglio.

\emph{Colpisce:} 11 (2d6 + 4) danni taglienti. Il bersaglio è afferrato (CD 14 per fuggire) se il diavolo non sta già afferrando un'altra creatura. Fino al termine dell'afferrare, il bersaglio è intralciato e subisce 7 (2d6) danni perforanti all'inizio di ciascun suo turno.

\emph{\textbf{Animare Catene (Ricarica dopo un 1 ora).}} Fino a quattro catene che il diavolo possa vedere e si trovano entro 18 metri da lui producono dei bordi affilati e si animano sotto il controllo del diavolo, purché quelle catene non siano né indossate né trasportate da qualcun altro.

Ogni catena animata è un oggetto con Difesa 20, 20 punti ferita, resistenza ai danni perforanti, e immunità ai danni psichici e da tuono. Quando il diavolo usa Multiattacco durante il suo turno, può usare ciascuna catena animata per effettuare un ulteriore attacco di catena. Una catena animata può afferrare una creatura per conto proprio ma non può effettuare attacchi mentre afferra. Una catena animata ritorna al suo stato inanimato se viene ridotta a 0 punti ferita o se il diavolo è reso inabile o muore.

\textbf{Reazioni}

\emph{\textbf{Maschera Snervante.}} Quando una creatura che il diavolo può vedere inizia il proprio turno entro 9 metri dal diavolo, il diavolo può creare un'illusione per assomigliare all'amore perduto o un acerrimo rivale di quella creatura. Se la creatura può vedere il diavolo, deve riuscire un Tiro Salvezza di Volontà CD 14 o rimanere spaventata fino al termine del suo turno.

\textbf{Ecologia}\\
Ambiente: Qualsiasi\\
Organizzazione: Solitario, coppia, anello (3-6) o catena (7-20)\\
Tesoro: Standard\\
\textbf{Descrizione}
Spesso classificati dai profani tra le fila dei diavoli infernali, i sadomasochistici non sono veri diavoli. Anche se alcuni sono noti per vivere all'Inferno, essi esistono al di fuori delle gerarchie stabilite dagli dei degli inferi e dai suoi arcidiavoli e a volte si possono trovare su altri piani, in particolare sul Piano delle Ombre. Molti suggeriscono che siano nativi dell'Inferno che esisteva prima dell'avvento della stirpe diabolica, anche se altri ipotizzano che siano stati portati sul piano da qualche sadica potenza. Indipendentemente dalle loro origini vagano per i piani assecondano il loro desiderio di causare e ricevere sofferenza, ricercando il dolore attraverso violenti rapimenti e sadiche depravazioni.\\


\medskip\index{Mostri - Diavolo Cornuto}\textbf{Diavolo Cornuto}

\emph{Grande immondo (diavolo), legale malvagio}

\textbf{FORZA} +6

\textbf{DESTREZZA} +3

\textbf{COSTITUZIONE} +5

\textbf{INTELLIGENZA} +1

\textbf{SAGGEZZA} +3

\textbf{CARISMA} +3

\textbf{Iniziativa} +3 -- \textbf{Difesa} 23

\textbf{Punti Ferita} 178 (17d10 + 85)

\textbf{Movimento} 6 m, volo 18 m

\textbf{Tiri Salvezza} Tempra +18, Riflessi +17, Volontà +13

\textbf{Resistenze al Danno} freddo; da botta, perforante e tagliente di attacchi che non siano argentati

\textbf{Immunità al Danno} fuoco, veleno, armi +1

\textbf{Immunità alle Condizioni} avvelenato

\textbf{Sensi} scurovisione 36 m

\textbf{Linguaggi} Infernale, telepatia 36 m 

\textbf{Sfida} 11 (7.200 PE)

\emph{\textbf{Resistenza alla Magia.}} Il diavolo ha +1d6 ai Tiri Salvezza contro incantesimi e altri effetti magici.

\emph{\textbf{Vista del Diavolo.}} La scurovisione del diavolo non è limitata dall'oscurità magica.

\textbf{Azioni}

\emph{\textbf{Multiattacco.}} Il diavolo effettua tre attacchi da mischia: due con il forcone e uno con la coda. Può usare Scagliare Fiamma al posto di qualsiasi attacco da mischia.

\emph{\textbf{Coda.} Attacco con arma da mischia}: +10 a colpire, portata 3 m, un bersaglio.

\emph{Colpisce:} 10 (1d8 + 6) danni perforanti. Se il bersaglio è una creatura, ad esclusione di costrutti e non morti, deve riuscire un Tiro Salvezza su Tempra 17 o perdere 10 (3d6) punti ferita all'inizio di ciascun suo turno a causa della ferita infernale. Ogni volta che il diavolo ferisce il bersaglio con questo attacco, il danno inflitto dalla ferita aumenta di 10 (3d6). Qualsiasi creatura può effettuare un'azione per bloccare la ferita riuscendo una prova di Saggezza (Pronto Soccorso) CD 12. La ferita si richiude anche nel caso in cui il bersaglio riceva magia guaritrice.

\emph{\textbf{Forcone.} Attacco con arma da mischia}: +10 a colpire, portata 3 m, un bersaglio.

\emph{Colpisce:} 15 (2d8 + 6) danni perforanti.

\emph{\textbf{Pungiglione.} Attacco con arma da mischia}: +8 a colpire, portata 3 m, un bersaglio.

\emph{Colpisce:} 13 (2d8 + 4) danni perforanti più 17 (5d6) danni da veleno, e il bersaglio deve riuscire un Tiro Salvezza di Tempra CD 14, o restare avvelenato per 1 minuto. Il bersaglio può ripetere il Tiro Salvezza al termine di ciascun suo turno, terminando l'effetto se lo
riesce.

\emph{\textbf{Scagliare Fiamma.} Attacco con incantesimo a Distanza}: +7 a colpire, gittata 45 m, un bersaglio.

\emph{Colpisce:} 14 (4d6) danni da fuoco. Se il bersaglio è un oggetto infiammabile che non sia indossato o trasportato, prende fuoco.

\textbf{Ecologia}\\
Ambiente: Qualsiasi (Inferno)\\
Organizzazione: Solitario, coppia o stormo (3-10)\\
Tesoro: Standard (Catena Chiodata Sacrilega+1, altro tesoro)\\
\textbf{Descrizione}\\
Tra i più letali guerrieri degli arcidiavoli ed abili comandanti dei diavoli minori, i diavoli cornuti divulgano le regole dell'Inferno dovunque passano. Questi diavoli maggiori sono addestrati, forgiati e riforgiati per essere tra i più implacabili ed obbedienti guerrieri del multiverso. I diavoli cornuti delle truppe degli eserciti infernali sono noti come cornugon, mentre i più grandi tra loro sono chiamati malebranche.\\
Un diavolo cornuto tipico raggiunge la ragguardevole altezza di 2,7 metri, è dotato di ali con un'apertura di 4,2 metri, e pesa 350 kg.\\


\medskip\index{Mostri - Diavolo della Fossa}\textbf{Diavolo della Fossa}

\emph{Grande immondo (diavolo), legale malvagio}

\textbf{FORZA} +8

\textbf{DESTREZZA} +2

\textbf{COSTITUZIONE} +7

\textbf{INTELLIGENZA} +6

\textbf{SAGGEZZA} +4

\textbf{CARISMA} +7

\textbf{Iniziativa} +6 -- \textbf{Difesa} 29

\textbf{Punti Ferita} 300 (24d10 + 168) 

\textbf{Movimento} 9 m, volo 18 m

\textbf{Tiri Salvezza} Tempra +24, Riflessi +21, Volontà +18

\textbf{Resistenze al Danno} freddo; da botta, perforante e tagliente che non siano argentati

\textbf{Immunità al Danno} fuoco, veleno , armi +2

\textbf{Immunità alle Condizioni} avvelenato

\textbf{Sensi} visione del vero 36 m

\textbf{Linguaggi} Infernale, telepatia 36 m

\textbf{Sfida} 20 (25.000 PE)

\emph{\textbf{Arma Magica.}} Gli attacchi con arma del diavolo della fossa sono magici.

\emph{\textbf{Aura di Paura.}} Qualsiasi creatura ostile al diavolo che inizi il suo turno entro 6 metri da esso, deve effettuare un Tiro Salvezza su Volontà CD 21, a meno che il diavolo non sia inabile. Se fallisce il Tiro Salvezza, la creatura è spaventata fino all'inizio del suo prossimo turno. Se il Tiro Salvezza della creatura riesce, la creatura è immune all'Aura di Paura del diavolo per le successive 24 ore.

\emph{\textbf{Incantesimi Innati.}} La caratteristica da incantatore  diavolo della fossa è il Carisma. Il diavolo della fossa può lanciare questi incantesimi in maniera innata, senza bisogno di componenti materiali:

A volontà: \emph{individuazione del magico, palla di fuoco}

3/giorno ciascuno: \emph{blocca mostri, muro di fuoco}

\emph{\textbf{Resistenza alla Magia.}} Il diavolo ha +1d6 ai Tiri Salvezza contro incantesimi e altri effetti magici.

\textbf{Azioni}

\emph{\textbf{Multiattacco.}} Il diavolo effettua quattro attacchi: uno con il morso, uno con l'artiglio, uno con la mazza e uno con la coda.

\emph{\textbf{Artiglio.} Attacco con arma da mischia}: +14 a colpire, portata 3 m, un bersaglio.

\emph{Colpisce:} 17 (2d8 + 8) danni taglienti.

\emph{\textbf{Coda.} Attacco con arma da mischia}: +14 a colpire, portata 3 m, un bersaglio.

\emph{Colpisce:} 24 (3d10 + 8) danni da botta.

\emph{\textbf{Mazza.} Attacco con arma da mischia}: +14 a colpire, portata 3 m, un bersaglio.

\emph{Colpisce:} 15 (2d6 + 8) danni da botta più 21 (6d6) danni da fuoco.

\emph{\textbf{Morso.} Attacco con arma da mischia}: +14 a colpire, portata 1 m, un bersaglio.

\emph{Colpisce:} 22 (4d6 + 8) danni perforanti. Il bersaglio deve riuscire un Tiro Salvezza di Tempra CD 21 o restare avvelenato. Mentre è avvelenato in questo modo, il bersaglio non può recuperare punti ferita, e subisce 21 (6d6) danni da veleno all'inizio di ciascun suo turno. Il bersaglio avvelenato può ripetere il Tiro Salvezza al termine di ciascun suo turno, terminando l'effetto su di sé.

\textbf{Ecologia}\\
Ambiente: Qualsiasi (Inferno)\\
Organizzazione: Solitario, coppia o concilio (3-9)\\
Tesoro: Doppio\\
\textbf{Descrizione}
Sovrani di reami infernali, generali delle armate dell'Inferno e consiglieri degli arcidiavoli, i diavoli della fossa sono la personificazione del terribile e spaventoso apice della razza diabolica.\\
Massicci, dal fisico indomito e dotati di ingegnosi intelletti malvagi, questi diabolici tiranni possiedono grande autonomia sia al servizio degli arcidiavoli che nella loro sovranità su distese infernali di schiavi o quando sono impegnati a soggiogare i mondi mortali. Solidi muscoli si tendono sui loro giganteschi corpi, corazzati da spesse placche taglienti capaci di bloccare quasi tutti gli attacchi. Le fauci dotate di zanne grandi come pugnali ed i loro volti bestiali nascondono alcune tra le menti più insidiose dell'Inferno.\\
Nati nelle profondità di Nessus, il nono e più profondo girone dell'Inferno, i diavoli della fossa vengono creati dai ranghi dei cornugon e dei gelugon solamente dagli arcidiavoli e dai loro duchi. Sebbene molti viaggino fino ai gironi superiori e oltre l'Inferno, al comando delle legioni infernali, la maggior parte rimane nel Nessus, al servizio delle corti dei potenti dell'Inferno o in oscure congreghe dagli innominabili propositi.\\

I diavoli della fossa sono sempre alti più di 4,2 metri, con una apertura alare di oltre 6 metri ed un peso superiore ai 500 kg.\\

I diavoli della fossa sono signori del fuoco e prediligono i territori lambiti dalle fiamme. All'Inferno, questa loro predisposizione fa sì che Averno, Dite, Malebolge, Nessus, e Flegetonte siano i gironi che più facilmente ospitano i loro templi-cittadelle avvolti dalle fiamme. Fanatici ossessionati dalla superiorità diabolica e dalla più ferrea obbedienza, i diavoli della fossa, se lasciati agire indisturbati, radunano immensi eserciti, rastrellando le fosse dell'Inferno alla ricerca dei lemure più depravati per trasformarli in veri diavoli. Una volta certi di aver creato le legioni perfette, volgono la loro attenzione ai semipiani ed ai mondi mortali più vulnerabili, pregustandone la conquista.\\

Servitori degli arcidiavoli o di altri unici signori della guerra infernali, i diavoli della fossa si votano alla loro causa, obbedendo alla volontà dei nobili scelti da Asmodeus nella speranza che, un giorno, riescano ad ottenere il favore del Principe dell'Oscurità o dell'Inferno stesso. Pur obbedienti alle gerarchie della propria razza, sono anche severi nel farne rispettare le regole e, se un diavolo della fossa si trovasse a servire un padrone indegno, si riterrebbe in dovere di deporlo. Pertanto, siano essi signori o servitori, i diavoli della fossa incarnano la volontà delle implacabili leggi dell'inferno e si assicurano che solo i diavoli più potenti possano (o osino) prosperare.\\

Solo i più potenti tra gli incantatori mortali possono od osano evocare un diavolo della fossa. Le reazioni di questo tipo di diavoli all'evocazione sono rapide e premeditate, solitamente caratterizzate da una furia incontenibile all'idea che un essere così insignificante possa sprecare il loro tempo immortale. Chi non riesce a fronteggiarne la bruciante rabbia viene ucciso e la sua anima dannata all'Inferno e posta al servizio del diavolo evocato. Chi riesce a controllare questi diavoli maggiori riesce anche ad intrigarli.\\

Un diavolo della fossa può servire rispettosamente un signore mortale per secoli, ma il suo scopo rimane sempre lo stesso: corromperne sempre più l'anima, assicurarsi la sua completa dannazione e, quando questi alla fine muore, rivendicarne l'anima ed iniziare il processo per farne un servitore lemure totalmente corrotto.\\

I diavoli della fossa sono consapevoli di essere immortali e sono abbastanza intelligenti da avere una pazienza incredibilmente disciplinata. Pertanto i diavoli della fossa più antichi vedono nelle loro legioni i volti degli innumerevoli folli che un tempo hanno preteso di ritenersi loro padroni.\\


\medskip\index{Mostri - Diavolo del Ghiaccio}\textbf{Diavolo del Ghiaccio}

\emph{Grande immondo (diavolo), legale malvagio}

\textbf{FORZA} +5

\textbf{DESTREZZA} +2

\textbf{COSTITUZIONE} +4

\textbf{INTELLIGENZA} +4

\textbf{SAGGEZZA} +2

\textbf{CARISMA} +4

\textbf{Iniziativa} +4 -- \textbf{Difesa} 25

\textbf{Punti Ferita} 180 (19d10 + 76)

\textbf{Movimento} 12 m

\textbf{Tiri Salvezza} Tempra +15, Riflessi +14, Volontà +12

\textbf{Resistenze al Danno} da botta, perforante e tagliente di attacchi che non siano argentate

\textbf{Immunità al Danno} freddo, fuoco, veleno , armi +1

\textbf{Immunità alle Condizioni} avvelenato

\textbf{Sensi} vista cieca 18 m, scurovisione 36 m

\textbf{Linguaggi} Infernale, telepatia 36 m

\textbf{Sfida} 14 (11.500 PE)

\emph{\textbf{Resistenza alla Magia.}} Il diavolo ha +1d6 ai Tiri Salvezza contro incantesimi e altri effetti magici.

\emph{\textbf{Vista del Diavolo.}} La scurovisione del diavolo non è limitata dall'oscurità magica.

\textbf{Azioni}

\emph{\textbf{Multiattacco.}} Il diavolo effettua tre attacchi: uno con il morso, uno con gli artigli e uno con la coda. In alternativa effettua due attacchi: uno con la coda e uno con lancia.

\emph{\textbf{Artigli.} Attacco con arma da mischia}: +10 a colpire, portata 1 m, un bersaglio.

\emph{Colpisce:} 10 (2d4 + 5) danni taglienti più 10 (3d6) danni da freddo.

\emph{\textbf{Coda.} Attacco con arma da mischia}: +10 a colpire, portata 3 m, un bersaglio.

\emph{Colpisce:} 12 (2d6 + 5) danni da botta più 10 (3d6) danni da freddo.

\emph{\textbf{Lancia di Ghiaccio.} Attacco con arma da mischia}: +10 a colpire, portata 3 m, un bersaglio.

\emph{Colpisce:} 14 (2d8 + 5) danni perforanti più 10 (3d6) danni da freddo. Se il bersaglio è una creatura, deve riuscire un Tiro Salvezza su Tempra CD 15, o avere per 1 minuto la velocità ridotta di 3 metri; durante ciascun suo turno può effettuare solo un'azione o un'azione bonus, ma non entrambe; non può effettuare reazioni. Il bersaglio può ripetere il Tiro Salvezza al termine di ciascun suo turno, terminando l'effetto su di sé in caso di successo.

\emph{\textbf{Morso.} Attacco con arma da mischia}: +10 a colpire,
portata 1 m, un bersaglio.

\emph{Colpisce:} 12 (2d6 + 5) danni perforanti più 10 (3d6) danni da
freddo.

\emph{\textbf{Muro di Ghiaccio (Ricarica 6).}} Il diavolo forma magicamente un muro di ghiaccio opaco su di una superficie solida che possa vedere entro 18 metri da lui. Il muro è spesso 30 centimetri e largo fino a 9 metri per un massimo di 3 metri di altezza, oppure   una cupola semisferica di massimo 6 metri di diametro. Quando la   parete appare, ogni creatura nel suo spazio viene spinta fuori da esso   tramite la via più breve. La creatura sceglie su quale lato del muro   finire, a meno che la creatura non sia inabile. La creatura poi   effettua un Tiro Salvezza di Riflessi CD 17, subendo 35 (10d6) danni da freddo se lo fallisce, o la metà di questi danni se lo riesce.

Il muro rimane per 1 minuto o finché il diavolo non è reso inabile o muore. Il muro può essere danneggiato e bucato; ogni sezione di 3 metri ha Difesa 5, 30 punti ferita, vulnerabilità al danno da fuoco, e immunità al danno da acido, freddo, da Vuoto, psichico e da veleno. Se una sezione viene distrutta, lascia una patina di aria gelida nello spazio che occupava prima il muro. Ogni volta che una creatura finisce per muoversi attraverso quest'aria gelida durante un turno, consenziente o meno, deve effettuare un Tiro Salvezza di Tempra CD 17, subendo 17 (5d6)danni da freddo se lo  fallisce, o la metà di questi danni se lo riesce. L'aria gelida si dissipa quando il resto del muro svanisce.


\medskip\index{Mostri - Diavolo d'Ossa}\textbf{Diavolo d'Ossa}

\emph{Grande immondo (diavolo), legale malvagio}

\textbf{FORZA} +4

\textbf{DESTREZZA} +3

\textbf{COSTITUZIONE} +4

\textbf{INTELLIGENZA} +1

\textbf{SAGGEZZA} +2

\textbf{CARISMA} +3

\textbf{Iniziativa} +3 -- \textbf{Difesa} 24

\textbf{Punti Ferita} 142 (15d10 + 60)

\textbf{Movimento} 12 m, volo 12 m

\textbf{Tiri Salvezza} Tempra +12, Riflessi +12, Volontà +7

\textbf{Competenze} Ingannare +7, Percepire Emozioni +6

\textbf{Resistenze al Danno} freddo; da botta, perforante e tagliente di attacchi non magici o che non siano argentati

\textbf{Immunità al Danno} fuoco, veleno

\textbf{Immunità alle Condizioni} avvelenato

\textbf{Sensi} scurovisione 36 m

\textbf{Linguaggi} Infernale, telepatia 36 m 

\textbf{Sfida} 9 (5.000 PE)

\emph{\textbf{Resistenza alla Magia.}} Il diavolo ha +1d6 ai Tiri Salvezza contro incantesimi e altri effetti magici.

\emph{\textbf{Vista del Diavolo.}} La scurovisione del diavolo non è limitata dall'oscurità magica.

\textbf{Azioni}

\emph{\textbf{Multiattacco.}} Il diavolo effettua tre attacchi: due con gli artigli e uno con il pungiglione oppure uno con la sua arma inastata uncinata e uno con il pungiglione.

\emph{\textbf{Arma Inastata Uncinata.} Attacco con arma da mischia}: +8 a colpire, portata 3 m, un bersaglio.

\emph{Colpisce:} 17 (2d12 + 4) danni perforanti. Se il bersaglio è una creatura di taglia Enorme o inferiore, è afferrato (CD 14 per fuggire). Fino al termine dell'afferrare, il diavolo non può usare la sua arma inastata su di un altro bersaglio.

\emph{\textbf{Artiglio.} Attacco con arma da mischia}: +8 a colpire, portata 3 m, un bersaglio.

\emph{Colpisce:} 8 (1d8 + 4) danni taglienti.

\emph{\textbf{Pungiglione.} Attacco con arma da mischia}: +8 a colpire, portata 3 m, un bersaglio.

\emph{Colpisce:} 13 (2d8 + 4) danni perforanti più 17 (5d6) danni da veleno, e il bersaglio deve riuscire un Tiro Salvezza di Tempra CD 14, o restare avvelenato per 1 minuto. Il bersaglio può ripetere il Tiro Salvezza al termine di ciascun suo turno, terminando l'effetto se lo riesce.

\textbf{Ecologia}\\
Ambiente: Qualsiasi (Inferno)\\
Organizzazione: Solitario, squadra (2-3), concilio (4-10) o contingente (1-3 diavoli del ghiaccio, 2-6 diavoli cornuti e 1-4 diavoli d'ossa\\
Tesoro: Standard (Lancia Gelida+1, altro tesoro)\\
\textbf{Descrizione}\\
Strateghi illuminati delle armate dell'Inferno, gli insettoidi diavoli del ghiaccio sono tra le menti più ingegnose e crudeli nelle legioni dell'inferno. Noto come gelugon tra le fila dei diavoli, un diavolo del ghiaccio nasconde nel suo petto un cuore ghiacciato trafugato ad un mortale, che gli permette di prendere decisioni libero da emozioni. Nati nel girone ghiacciato di Cocito, il settimo girone infernale, la maggior parte dei diavoli del ghiaccio migra a Caina, l'ottavo girone, dove complotta per dannare il mondo da corti di gelido acciaio. Sebbene abbiano le sembianze più aliene e mostruose tra tutti i diavoli, a pochi altri viene accordato un maggiore rispetto.\\
In combattimento un gelugon manda avanti i suoi sottoposti, così da poter valutare le tattiche, i punti di forza e le debolezze dell'avversario nelle retrovie, e fornire loro supporto con le sue capacità magiche, evitando di coglierli nell'area di effetto dei suoi incantesimi: atteggiamento non dovuto ad un senso di cameratismo, bensì alla fredda e logica verità che i suoi alleati possono sopravvivere più a lungo in uno scontro se non sono esposti a fuoco amico.\\
I gelugon sono alti 3,6 metri e pesano approssimativamente 350 kg.\\


\medskip\index{Mostri - Diavolo Spinoso}\textbf{Diavolo Spinoso}

\emph{Piccola immondo (diavolo), legale malvagio}

\textbf{FORZA} +0

\textbf{DESTREZZA} +2

\textbf{COSTITUZIONE} +1

\textbf{INTELLIGENZA} +0

\textbf{SAGGEZZA} +2

\textbf{CARISMA} -1

\textbf{Iniziativa} +2 -- \textbf{Difesa} 14

\textbf{Punti Ferita} 22 (5d6 + 5)

\textbf{Movimento} 6 m, volo 12 m

\textbf{Resistenze al Danno} freddo; da botta, perforante e tagliente di attacchi non magici o che non siano argentati

\textbf{Immunità al Danno} fuoco, veleno

\textbf{Immunità alle Condizioni} avvelenato

\textbf{Sensi} scurovisione 36 m

\textbf{Linguaggi} Infernale, telepatia 36 m

\textbf{Sfida} 2 (450 PE)

\emph{\textbf{Resistenza alla Magia.}} Il diavolo ha +1d6 ai Tiri Salvezza contro incantesimi e altri effetti magici.

\emph{\textbf{Sorvolare.}} Il diavolo non provoca attacchi di opportunità quando vola via dalla portata di un nemico.

\emph{\textbf{Spine Limitate.}} Il diavolo possiede dodici spine caudali. Le spine usate ricrescono a mezzanotte.

\emph{\textbf{Vista del Diavolo.}} La scurovisione del diavolo non è limitata dall'oscurità magica.

\textbf{Azioni}

\emph{\textbf{Multiattacco.}} Il diavolo effettua due attacchi: uno con il morso e uno con il suo forcone o due con le sue spine caudali.

\emph{\textbf{Forcone.} Attacco con arma da mischia}: +2 a colpire, portata 1 m, un bersaglio.

\emph{Colpisce:} 3 (1d6) danni perforanti.

\emph{\textbf{Morso.} Attacco con arma da mischia}: +2 a colpire, portata 1 m, un bersaglio.

\emph{Colpisce:} 5 (2d4) danni taglienti.

\emph{\textbf{Spina Caudale.} Attacco con arma a Distanza}: +4 a colpire, gittata 6m, un bersaglio.

\emph{Colpisce:} 4 (1d4 + 2) danni perforanti più 3 (1d6) danni da fuoco.

\textbf{Ecologia}\\
Ambiente: Qualsiasi (Inferno)\\
Organizzazione: Solitario, coppia, gruppo (3-5) o plotone (6-11)\\
Tesoro: Standard\\
\textbf{Descrizione}\\
Sentinelle delle volte dell'Inferno, carcerieri delle anime più nere e armi viventi delle forge infernali, i diavoli uncinati, noti ai diabolisti come hamatula, impongono ai dannati i loro ceppi e custodiscono il nefasto operato dei diavoli maggiori. Un hamatula ama sentire il sangue caldo sulle proprie spine e preferisce gettarsi nella mischia quando gli viene offerta l'opportunità di combattere.\\
Gli hamatula sono collezionisti ed organizzatori, e sono gli alleati favoriti di bramosi evocatori, dal momento che spesso portano con sé tesori tentatori dalle volte dell'Inferno o conoscono il sentiero per ottenere mortali ricchezze. Se lasciati agire liberamente, nei nascondigli di questi diavoli spesso fanno mostra i trofei trafitti di vecchie vittime, appesi come perverse collezioni di insetti su muri insanguinati.\\
La maggior parte dei diavoli uncinati è alta dai 2,1 metri in su e pesa 150 kg, sebbene i loro corpi asciutti e muscolosi sembrino più grossi per via degli spuntoni in continua crescita che fuoriescono dai loro corpi, taglienti come lame.\\


\medskip\index{Mostri - Erinni}\textbf{Erinni}

\emph{Media immondo (diavolo), legale malvagio}

\textbf{FORZA} +4

\textbf{DESTREZZA} +3

\textbf{COSTITUZIONE} +4

\textbf{INTELLIGENZA} +2

\textbf{SAGGEZZA} +2

\textbf{CARISMA} +4

\textbf{Iniziativa} +3 -- \textbf{Difesa} 24 (armatura di piastre)

\textbf{Punti Ferita} 153 (18d8 + 72)

\textbf{Movimento} 9 m, volo 18 m

\textbf{Tiri Salvezza} Tempra +11, Riflessi +12, Volontà +7

\textbf{Resistenze al Danno} freddo; da botta, perforante e tagliente di attacchi non magici o che non siano argentati

\textbf{Immunità al Danno} fuoco, veleno

\textbf{Immunità alle Condizioni} avvelenato

\textbf{Sensi} visione del vero 36 m

\textbf{Linguaggi} Infernale, telepatia 36 m

\textbf{Sfida} 12 (8.400 PE)

\emph{\textbf{Armi Diaboliche.}} Gli attacchi con arma dell'erinni sono magici e infliggono 13 (3d8) danni da veleno aggiuntivi quando colpiscono (già incluso negli attacchi).

\emph{\textbf{Resistenza alla Magia.}} L'erinni ha +1d6 ai Tiri Salvezza contro incantesimi e altri effetti magici.

\textbf{Azioni}

\emph{\textbf{Multiattacco.}} L'erinni effettua tre attacchi.

\emph{\textbf{Spada Lunga.} Attacco con arma da mischia}: +8 a colpire, portata 1 m, un bersaglio.

\emph{Colpisce:} 8 (1d8 + 4) danni taglienti, o 9 (1d10 + 4) danni taglienti se usata con due mani, più 13 (3d8) danni da veleno.

\emph{\textbf{Arco Lungo.} Attacco con arma a Distanza}: +7 a colpire, gittata 45m, un bersaglio.

\emph{Colpisce:} 7 (1d8 + 4) danni perforanti più 13 (3d8) danni da veleno, e il bersaglio deve riuscire un Tiro Salvezza di Tempra CD 14 o restare avvelenato. Il veleno rimane finché non viene rimosso da un incantesimo \emph{ristorazione inferiore} o simile.

\textbf{Reazioni}

\emph{\textbf{Parata.}} L'erinni somma 4 alla sua Difesa contro un attacco da mischia che lo colpirebbe. Per farlo, l'erinni deve poter vedere il suo attaccante e impugnare un'arma da mischia.

\textbf{Ecologia}\\
Ambiente: Qualsiasi (Inferno)\\
Organizzazione: Solitario o trio\\
Tesoro: Triplo (Arco Lungo Composito Infuocato+1 [Forza +5], corda, Spada Lunga+1)\\
\textbf{Descrizione}\\
Note con molti nomi, i Caduti, le Ali Cineree e le Furie, i diavoli conosciuti come erinni insultano la loro forma angelica con la loro brama di vendetta e sanguinosa giustizia. Carnefici, non giudici, le erinni volteggiano sopra i cornicioni affilati come lame di Dite, il secondo girone cosmopolita dell'Inferno, sempre attente a cogliere ogni occasione di battaglia, che sia a difesa dell'inferno, per il capriccio dei loro diabolici signori o per l'appassionata chiamata di capricciosi evocatori mortali. Tutte le erinni intrecciano con i loro stessi capelli letali corde viventi, che utilizzano in battaglia per intralciare e sollevare in aria i loro nemici, schernendoli e condannandoli per le loro trasgressioni prima di lasciarli precipitare da grandi altezze.\\
Le erinni sono angeli bellissimi ed oscuri che accrescono deliberatamente la propria sensualità con cicatrici e lividi. Eppure, nonostante la loro bellezza, le erinni non sono seduttrici: mancano loro la sottigliezza e la pazienza richieste per questa raffinata arte emotiva, poiché preferiscono risolvere i loro problemi con atti di rapida ed atroce violenza. Spesso una erinni tratterrà il suo colpo mortale mentre tenta di uccidere un nemico, solo per prolungarne le sofferenze. La morte è in genere l'unico modo per sfuggire alle attenzioni di una erinni, e quelle più potenti sono abilissime nel tenere i loro nemici in vita ma inermi, così da prolungare il loro tormento, arrivando addirittura a mantenerli vivi con la magia. Si dice che le più potenti torturatrici erinni siano dotate di capacità che permettono alle sofferenze da loro inflitte di perdurare anche dopo la morte del soggetto.\\
La maggior parte delle erinni è alta poco meno di 1,8 metri e pesa circa 70 kg, e le loro ali nere piumate hanno una apertura superiore ai 3 metri.\\


\medskip\index{Mostri - Imp}\textbf{Imp}

\emph{Minuscola immondo (diavolo, mutaforma), legale malvagio}

\textbf{FORZA} -2

\textbf{DESTREZZA} +3

\textbf{COSTITUZIONE} +1

\textbf{INTELLIGENZA} +0

\textbf{SAGGEZZA} +1

\textbf{CARISMA} +2

\textbf{Iniziativa} +3 -- \textbf{Difesa} 14

\textbf{Punti Ferita} 10 (3d4 + 3)

\textbf{Movimento} 6 m, volo 12 m (6 m in forma di ratto; 6 m, volo 18 m in forma di corvo; 6 m, scalata 6 m in forma di ragno)

\textbf{Tiri Salvezza} Tempra +1, Riflessi +6, Volontà +4

\textbf{Competenze} Muoversi Silenziosamente / Nascondersi +5, Ingannare +4, Percepire Emozioni +3

\textbf{Resistenze al Danno} freddo; da botta, perforante e tagliente di attacchi non magici o che non siano argentati

\textbf{Immunità al Danno} fuoco, veleno

\textbf{Immunità alle Condizioni} avvelenato

\textbf{Sensi} scurovisione 36 m 

\textbf{Linguaggi} Infernale, Comune

\textbf{Sfida} 1 (200 PE)

\emph{\textbf{Mutaforma.}} Il diavolo può usare la sua azione per trasformarsi in una forma bestiale da ratto, corvo o ragno, o per tornare alla sua vera forma. Le sue statistiche sono le stesse in tutte le forme, sebbene gli attacchi possano variare per alcune di esse. Qualsiasi equipaggiamento stia indossando o trasportando non viene trasformato. Alla morte ritorna alla sua vera forma.

\emph{\textbf{Resistenza alla Magia.}} Il diavolo ha +1d6 ai Tiri Salvezza contro incantesimi e altri effetti magici.

\emph{\textbf{Vista del Diavolo.}} La scurovisione del diavolo non è limitata dall'oscurità magica.

\textbf{Azioni}

\emph{\textbf{Pungiglione (Morso in Forma di Bestia).} Attacco con arma da mischia}: +5 a colpire, portata 1 m, una creatura.

\emph{Colpisce:} 5 (1d4 + 3) danni perforanti, e il bersaglio deve effettuare un Tiro Salvezza di Tempra CD 11, subendo 10 (3d6) danni da veleno se lo fallisce, o la metà di questi danni se lo riesce. 

\emph{\textbf{Invisibilità.}} Il diavolo resta invisibile finché non attacca o termina la sua concentrazione. Qualsiasi cosa che il diavolo stia trasportando o indossando, resta invisibile finché rimane in contatto con il diavolo.

\textbf{Ecologia}\\
Ambiente: Qualsiasi (Inferno)\\
Organizzazione: Solitario, coppia o stormo (3-10)\\
Tesoro: Standard\\
\textbf{Descrizione}\\
Nati direttamente dalle fosse dell'Inferno, gli imp sono i diavoli meno potenti, anche se queste crudeli ed invadenti creature svolgono un ruolo importante nella corruzione delle anime mortali. Libere dalle gerarchie e dai doveri delle armate infernali, gli imp si dilettano ad ogni opportunità di viaggiare fino al Piano Materiale e di tentare astutamente i mortali, spingendoli a compiere atti sempre più depravati.\\

Volontariamente al servizio di incantatori nel ruolo di famigli, gli imp recitano la parte dei fedeli servitori, offrendo spesso ai loro padroni astuti consigli ed infernali intuizioni. In realtà, gli imp operano per inviare anime all'Inferno, accertandosi che l'anima del loro padrone, insieme a molte altre, sia destinata alla dannazione dopo la morte.\\

Gli imp variano molto in aspetto, in un ampio spettro di tratti bestiali e grotteschi, sebbene molti di essi abbiano la forma di un umanoide alato dalla pelle rossiccia, con lineamenti bulbosi. Il tipico imp è alto solamente 60 centimetri, ha un'apertura alare di 90 centimetri e pesa 5 kg.\\

Un imp su mille è dotato dell'abilità di comunicare telepaticamente con creature entro 15 metri ed il potere di modificare la propria forma in quella di un animale Piccolo o Minuscolo, come per effetto di un incantesimo Forma Ferina II. Questi imp consolari sono assai apprezzati dai diavoli potenti, che li inviano come servitori ai loro seguaci preferiti o per corrompere eroi mortali. Un imp consolare può essere evocato per mezzo del talento Famiglio Migliorato, ma solo da un incantatore di 8° livello o superiore. I diabolisti narrano di altre razze di imp con capacità altrettanto specializzate, ma se queste creature esistono realmente si tratta di casi estremamente rari.\\

Diversamente dagli altri diavoli, gli imp si ritrovano spesso liberi e soli nel Piano Materiale, in particolare dopo che sono stati evocati per servire come famigli ed i loro padroni sono morti (spesso, indirettamente, a causa delle macchinazioni dell'imp stesso). Senza alcun mezzo per poter fare ritorno a casa questi imp, liberi da ogni legame con padroni arcani, possono diventare pericolosi seccatori o persino porsi a capo di piccole tribù di sanguinosi umanoidi, quali Goblin o Coboldi.\\


\medskip\index{Mostri - Lemure}\textbf{Lemure}

\emph{Media immondo (diavolo), legale malvagio}

\textbf{FORZA} +0

\textbf{DESTREZZA} -3

\textbf{COSTITUZIONE} +0

\textbf{INTELLIGENZA} -5

\textbf{SAGGEZZA} +0

\textbf{CARISMA} -4

\textbf{Iniziativa} -3 -- \textbf{Difesa} 8

\textbf{Punti Ferita} 13 (3d8)

\textbf{Movimento} 5 metri

\textbf{Tiri Salvezza} Tempra +4, Riflessi +3, Volontà +0

\textbf{Resistenze al Danno} freddo

\textbf{Immunità al Danno} fuoco, veleno

\textbf{Immunità alle Condizioni} affascinato, avvelenato, spaventato

\textbf{Sensi} scurovisione 36 m

\textbf{Linguaggi} comprende l'Infernale ma non può parlare

\textbf{Sfida} 0 (10 PE)

\emph{\textbf{Rinvigorimento Diabolico.}} Un lemure che muore nei Nove Inferi ritorna in vita con tutti i suoi punti ferita in 1d10 giorni a meno che non venga ucciso da una creatura di allineamento buono su cui sia stato eseguito l'incantesimo \emph{benedire} o i suoi resti vengano
cosparsi di acqua sacra.

\emph{\textbf{Vista del Diavolo.}} La scurovisione del diavolo non è limitata dall'oscurità magica.

\textbf{Azioni}

\emph{\textbf{Pugno.} Attacco con arma da mischia}: +3 a colpire, portata 1 m, un bersaglio.

\emph{Colpisce:} 2 (1d4) danni da botta.

\textbf{Ecologia}\\
Ambiente: Qualsiasi (Inferno)\\
Organizzazione: Solitario, coppia, gruppo (3-5), sciame (6-17) o schiera (10-40 o più)\\
Tesoro: Nessuno\\
\textbf{Descrizione}\\
I più infimi tra i diavoli, i lemure hanno origine dalle fila delle anime condannate all'inferno, masse informi di carne tremolante. La scintilla di istinto o di memoria che sopravvive nella loro coscienza addormentata solitamente dà forma ai loro tratti, che imitano quelli dei suoi torturatori o delle anime torturate che li circondano. Grotteschi ed inutili, i tratti di un lemure non rivelano nulla di quello che è stato un tempo. Molti sfoggiano diversi orribili volti o sono nulla più di colonne ribollenti di carne cancerosa. Solamente i loro arti bitorzoluti, che agitano in continuazione, sembrano funzionare correttamente, e vengono usati solo per distruggere qualsiasi forma di vita non infernale che si avvicini troppo.\\
I lemure in movimento si consolidano in forme alte più di 1,2 metri e pesanti più di 100 kg, sebbene questi disgustosi diavoli, quando stanno riposando, spesso hanno l'indistinto aspetto di masse di carne disciolta dai tratti deformi.\\

Sebbene siano tra le più rivoltanti creature esistenti, i lemure rivestono un ruolo vitale nella perversa ecologia dell'Inferno. Quando, al termine della sua esistenza mortale, un'anima viene dannata, sia perché adoratrice di forze diaboliche che per mancanza di fede in altre divinità, essa si unisce alle masse delle anime sofferenti che riempiono le pianure dell'Averno, il primo girone dell'Inferno. Qui iniziano i tormenti, mentre diavoli minori le sospingono assieme ad altri spiriti, preparandole all'arduo viaggio fino ad uno dei gironi dell'inferno più profondi, solitamente uno adatto alla punizione appropriata per i crimini commessi dall'anima, oppure semplicemente verso il dominio di un diavolo che necessita di nuovi schiavi. Una volta giunte nel regno della loro dannazione, le anime affrontano innumerevoli secoli di tormento per mano dei diavoli, di altri esseri malvagi e delle letali macchinazioni dell'Inferno stesso. Mentre l'essenza mortale impazzisce lentamente, queste creature dimenticano le loro vite, divenendo prima selvagge e infine poco più che automi guidati dall'odio e dalla paura. Dopo eoni di questa esistenza, il crudele procedimento dell'Inferno distrugge totalmente l'anima oppure, nel caso degli spiriti più profani, riconsacra questi esseri dimenticati sotto la forma di lemure, la forma di vita più elementare dei diavoli, insensate orde di carne putrescente e diabolica. Questi esseri ripugnanti si radunano in grandi masse, rivoltanti ondate formate da migliaia e migliaia di queste creature.\\

I diavoli maggiori sono in grado di riconoscere i più corrotti tra loro e, per mezzo di torture misteriose o grazie ai poteri stessi dell'Inferno, le riplasmano in veri diavoli, appena rinati e pronti a servire obbedienti nelle legioni dei dannati.\\


\subsection{Dinosauri}

\medskip\index{Mostri - Plesiosauro}\textbf{Plesiosauro}

\emph{Grande bestia, disallineato}

\textbf{FORZA} +4

\textbf{DESTREZZA} +2

\textbf{COSTITUZIONE} +3

\textbf{INTELLIGENZA} -4

\textbf{SAGGEZZA} +1

\textbf{CARISMA} -3

\textbf{Iniziativa} +2 -- \textbf{Difesa} 14

\textbf{Punti Ferita} 68 (8d10 + 24)

\textbf{Movimento} 6 m, nuoto 12 m

\textbf{Tiri Salvezza} Tempra +18, Riflessi +11, Volontà +9

\textbf{Competenze} Muoversi Silenziosamente / Nascondersi +4, Consapevolezza +3

\textbf{Linguaggi} -

\textbf{Sfida} 2 (450 PE)

\emph{\textbf{Trattenere il Fiato.}} Il plesiosauro può trattenere il fiato per 1 ora.

\textbf{Azioni}

\emph{\textbf{Morso.} Attacco con arma da mischia}: +6 a colpire, portata 3 m, un bersaglio.

\emph{Colpisce:} 14 (3d6 + 4) danni perforanti.

\textbf{Ecologia}\\
Ambiente: Acquatico Caldo\\
Organizzazione: Solitario, coppia o branco (3-6)\\
Tesoro: Nessuno\\
\textbf{Descrizione}\\
Il plesiosauro è un rettile acquatico dal lungo collo. Sebbene tecnicamente non sia un dinosauro, questa creatura ed i suoi simili si trovano spesso a cacciare in laghi ed oceani nei quali è facile trovare dei dinosauri.\\


\medskip\index{Mostri - Tirannosauro}\textbf{Tirannosauro}

\emph{Enorme bestia, disallineato}

\textbf{FORZA} +7

\textbf{DESTREZZA} +0

\textbf{COSTITUZIONE} +4

\textbf{INTELLIGENZA} -4

\textbf{SAGGEZZA} +1

\textbf{CARISMA} -1

\textbf{Iniziativa} +0 -- \textbf{Difesa} 17

\textbf{Punti Ferita} 136 (13d12 + 52)

\textbf{Movimento} 15 m

\textbf{Tiri Salvezza} Tempra +15, Riflessi +12, Volontà +10

\textbf{Competenze} Consapevolezza +4

\textbf{Linguaggi} -

\textbf{Sfida} 8 (3.900 PE)

\textbf{Azioni}

\emph{\textbf{Multiattacco.}} Il tirannosauro effettua due attacchi: uno con il morso e uno con la coda. Non può effettuare entrambi gli attacchi contro lo stesso bersaglio.

\emph{\textbf{Coda.} Attacco con arma da mischia}: +10 a colpire, portata 3 m, un bersaglio.

\emph{Colpisce:} 20 (3d8 + 7) danni da botta.

\emph{\textbf{Morso.} Attacco con arma da mischia}: +10 a colpire, portata 3 m, un bersaglio.

\emph{Colpisce:} 33 (4d12 + 7) danni perforanti. Se il bersaglio è una creatura di taglia Media o inferiore, è afferrato (CD 17 per fuggire). Fino al termine dell'afferrare, il bersaglio è intralciato, e il tirannosauro non può usare il morso contro un altro bersaglio.

\textbf{Ecologia}\\
Ambiente: Foreste e Pianure Calde\\
Organizzazione: Solitario, coppia o branco (3-6)\\
Tesoro: Nessuno\\
\textbf{Descrizione}\\
Il tirannosauro è un predatore primario che misura 12 metri di lunghezza e pesa 7.000 kg.\\


\medskip\index{Mostri - Triceratopo}\textbf{Triceratopo}

\emph{Enorme bestia, disallineato}

\textbf{FORZA} +6

\textbf{DESTREZZA} -1

\textbf{COSTITUZIONE} +3

\textbf{INTELLIGENZA} -4

\textbf{SAGGEZZA} +0

\textbf{CARISMA} -3

\textbf{Iniziativa} -1 -- \textbf{Difesa} 16

\textbf{Punti Ferita} 95 (10d12 + 30)

\textbf{Movimento} 15 m

\textbf{Tiri Salvezza} Tempra +15, Riflessi +8, Volontà +5

\textbf{Linguaggi} -

\textbf{Sfida} 5 (1.800 PE)

\emph{\textbf{Carica Travolgente.}} Se il triceratopo si muove di almeno 6 metri diretto verso una creatura e la colpisce con un attacco di incornata durante lo stesso turno, il bersaglio deve riuscire un Tiro Salvezza su Tempra CD 13 o cadere prono. Se il bersaglio è prono, il triceratopo può effettuare un attacco di pestone contro di lui come azione bonus.

\textbf{Azioni}

\emph{\textbf{Incornata.} Attacco con arma da mischia}: +9 a colpire,
portata 1 m, un bersaglio.

\emph{Colpisce:} 24 (3d10 + 6) danni perforanti.

\emph{\textbf{Pestone.} Attacco con arma da mischia}: +9 a colpire,
portata 1 m, una creatura prona.

\emph{Colpisce:} 22 (3d10 + 6) danni da botta.

\textbf{Ecologia}\\
Ambiente: Pianure Calde\\
Organizzazione: Solitario, coppia o branco (5-8)\\
Tesoro: Nessuno\\
\textbf{Descrizione}\\
Il triceratopo è un erbivoro irascibile e caparbio. Un tipico triceratopo è lungo 9 metri e pesa 10.000 kg.\\


\medskip\index{Mostri - Doppelganger}\textbf{Doppelganger}

\emph{Media mostruosità (mutaforma), neutrale}

\textbf{FORZA} +0

\textbf{DESTREZZA} +4

\textbf{COSTITUZIONE} +2

\textbf{INTELLIGENZA} +0

\textbf{SAGGEZZA} +1

\textbf{CARISMA} +2

\textbf{Iniziativa} +4 -- \textbf{Difesa} 16

\textbf{Punti Ferita} 52 (8d8 + 16)

\textbf{Movimento} 9 m

\textbf{Tiri Salvezza} Tempra +4, Riflessi +5, Volontà +6

\textbf{Competenze} Ingannare +6, Percepire Emozioni +3

\textbf{Immunità alle Condizioni} affascinato

\textbf{Sensi} scurovisione 18 m

\textbf{Linguaggi} Comune

\textbf{Sfida} 3 (700 PE)

\emph{\textbf{Mutaforma.}} Il doppelganger può usare la sua azione per cambiare la propria forma in quella di un umanoide Piccolo o Medio che abbia visto, o per tornare alla sua vera forma. Le sue statistiche, a parte la taglia, sono le stesse in tutte le forme. Qualsiasi equipaggiamento stia indossando o trasportando non viene trasformato. Alla morte ritorna alla sua vera forma.

\emph{\textbf{Appostato.}} Nel primo round di combattimento, il doppelganger ha +1d6 ai tiri di attacco contro qualsiasi creatura abbia preso di sorpresa.

\emph{\textbf{Attacco di Sorpresa.}} Se il doppelganger sorprende una creatura e la colpisce con un attacco durante il primo round di combattimento, il bersaglio subisce 10 (3d6) danni aggiuntivi dall'attacco.

\textbf{Azioni}

\emph{\textbf{Multiattacco.}} Il doppelganger effettua due attacchi da
mischia.

\emph{\textbf{Schianto.} Attacco con arma da mischia}: +6 a colpire,
portata 1 m, un bersaglio.

\emph{Colpisce:} 7 (1d6 + 4) danni da botta.

\emph{\textbf{Leggere Pensieri.}} Il doppelganger legge magicamente i pensieri di superficie di una creatura entro 18 metri da lui. L'effetto può penetrare le barriere, ma 1 metro di legno o terra, 50 centimetri di pietra, 5 centimetri di metallo, o un sottile foglio di piombo lo blocca. Mentre il bersaglio è a gittata, il doppelganger può continuare a leggerne i pensieri, purché la concentrazione del doppelganger non venga infranta (come la concentrazione di un incantesimo). Mentre legge la mente di un bersaglio, il doppelganger ha +1d6 alle prove di Saggezza e Carisma contro il bersaglio.

\textbf{Ecologia}\\
Ambiente: Qualsiasi\\
Organizzazione: Solitario, coppia o banda (3-6)\\
Tesoro: Equipaggiamento da PNG\\
\textbf{Descrizione}\\
I doppelganger sono strani esseri che possono assumere la forma di coloro che incontrano. Nella sua forma naturale, la creatura somiglia più o meno ad un umanoide, ma snello e fragile, con membra magre e tratti facciali non del tutto formati. La sua carnagione è pallida, è glabro ed i suoi occhi sono bianchi e vacui.\\
I doppelganger preferiscono infiltrarsi in società dove possono ammassare ricchezza e potere, e vedono scarse prospettive nel fondare città coi loro simili. I doppelganger più giovani sperimentano le loro abilità su piccole tribù di orchi o goblin, poi si spostano in società più complesse come comunità naniche, elfiche e umane. Piuttosto che diventare bersagli occupando posizioni di comando, preferiscono mantenere il potere da dietro il trono, o usare molteplici identità per manipolare cittadini influenti o intere gilde.\\
I doppelganger fanno un uso eccellente della loro mimica naturale per tendere imboscate, trappole con esca e infiltrarsi nelle società umanoidi. Anche se di solito non sono malvagi, sono interessati solo a sé stessi e considerano tutti gli altri come giocattoli da manipolare ed ingannare. Piace loro moltissimo invadere le società umane per soddisfare i loro desideri; alcuni amano i complessi giochi politici mentre altri cercano continuamente di cambiare razza, sesso e partner amorosi. Nonostante non sia la norma, quei doppelganger che usano i loro doni per scopi crudeli e sadici sono molto famosi, e questi mutaforma sono i principali responsabili della sinistra reputazione della loro razza. Certamente, una creatura capace di cambiare forma è avvantaggiata quando cerca di evitare di essere catturata per i suoi crimini, ed alcuni doppelganger particolarmente malevoli godono nel troncare relazioni amorose inscenando tradimenti.\\

Delle voci insistenti parlano di doppelganger ancor più potenti capaci non solo di cambiare il loro aspetto, ma anche di far proprie abilità, ricordi ed anche capacità straordinarie e soprannaturali delle creature che scelgono di impersonare.\\


\subsection{Draghi Cromatici}

\medskip\index{Mostri - Drago Bianco Antico}\textbf{Drago Bianco Antico}

\emph{Mastodontica drago, caotico malvagio}

\textbf{FORZA} +8

\textbf{DESTREZZA} +0

\textbf{COSTITUZIONE} +8

\textbf{INTELLIGENZA} +0

\textbf{SAGGEZZA} +1

\textbf{CARISMA} +2

\textbf{Iniziativa} +0 -- \textbf{Difesa} 30

\textbf{Punti Ferita} 333 (18d20 + 144)

\textbf{Movimento} 12 m, nuoto 12 m, scavo 12 m, volo 24 m

\textbf{Tiri Salvezza} Tempra +19, Riflessi +14, Volontà +16

\textbf{Competenze} Muoversi Silenziosamente / Nascondersi +6, Consapevolezza +13

\textbf{Immunità al Danno} freddo, arma +1

\textbf{Sensi} scurovisione 36 m, vista cieca 18 m

\textbf{Linguaggi} Comune, Draconico

\textbf{Sfida} 20 (25.000 PE)

\emph{\textbf{Camminare sul Ghiaccio.}} Il drago può muoversi e arrampicarsi su superfici ghiacciate senza bisogno di effettuare prove di caratteristica. Inoltre, il terreno difficile composto di ghiaccio o neve non gli costa movimento aggiuntivo.

\emph{\textbf{Resistenza Leggendaria (3/Giorno).}} Se il drago fallisce un Tiro Salvezza, può scegliere invece di riuscire.

\textbf{Azioni}

\emph{\textbf{Multiattacco.}} Il drago può usare la sua Presenza Spaventosa. Poi effettuare tre attacchi: uno con il morso e due con gli artigli.

\emph{\textbf{Artiglio.} Attacco con arma da mischia}: +14 a colpire, portata 3 m, un bersaglio.

\emph{Colpisce:} 15 (2d6 + 8) danni taglienti.

\emph{\textbf{Coda.} Attacco con arma da mischia}: +14 a colpire, portata 6 m, un bersaglio.

\emph{Colpisce:} 17 (2d8 + 8) danni da botta.

\emph{\textbf{Morso.} Attacco con arma da mischia}: +14 a colpire, portata 5 metri, un bersaglio.

\emph{Colpisce:} 19 (2d10 + 8) danni perforanti più 9 (2d8) danni da freddo.

\emph{\textbf{Presenza Spaventosa.}} Ogni creatura scelta dal drago, che si trovi entro 36 metri da esso e consapevole della sua presenza, deve riuscire un Tiro Salvezza di Volontà CD 16 o restare spaventata per 1 minuto. Una creatura può ripetere il Tiro Salvezza al termine di ciascun suo turno, terminando l'effetto se lo riesce. Se il Tiro Salvezza della creatura ha successo o l'effetto ha termine per essa, la creatura è immune alla Presenza Spaventosa del drago per le successive 24 ore.

\emph{\textbf{Soffio Gelido (Ricarica 5-6).}} Il drago esala un'esplosione di ghiaccio in un cono di 27 metri. Ogni creatura in quell'area deve effettuare un Tiro Salvezza di Tempra CD 22 e subire 72 (16d8) danni da freddo se fallisce il Tiro Salvezza, o la metà di questi danni se lo riesce.

\textbf{Azioni Aggiuntive}

Il drago può effettuare 3 Azioni aggiuntive, scelte tra le opzioni seguenti. Può usare solo un'opzione leggendaria alla volta e solo al termine del turno di un'altra creatura. Il drago recupera le Azioni aggiuntive spese all'inizio del proprio turno.

\textbf{Attacco di Ala (Costa 2 Azioni).} Il drago batte le ali. Ogni creatura entro 5 metri dal drago deve riuscire un Tiro Salvezza su Riflessi CD 22 o subire 15 (2d6 + 8) danni da botta e venir gettato prono. Il drago può poi volare fino a metà del suo movimento di volo. \textbf{Attacco di Coda.} Il drago effettua un attacco di coda. \textbf{Individuare.} Il drago effettua una prova di Saggezza (Consapevolezza).

\textbf{Ecologia}\\
Ambiente: Montagne Fredde\\
Organizzazione: Solitario\\
Tesoro: Triplo\\
\textbf{Descrizione}\\
Anche se molti lo considerano il più debole e bestiale tra i draghi cromatici, il drago bianco supplisce alla sua mancanza di astuzia con la pura e semplice ferocia. I draghi bianchi vivono su remote e gelide cime montuose, e nei bassopiani artici, facendo la loro tana in scintillanti caverne piene di ghiaccio e neve. Preferiscono che i loro pasti siano completamente congelati.\\

\textbf{Incantesimi}\index{Incantesimi da Drago Bianco}\\
Il drago e' anche capace, se di eta' giovanile o piu' anziano, di lanciare incantesimi.\\
Puo' lanciare un numero di incantesimi pari al suo valore di Carisma, senza aver necessita' di componenti magici.\\
Non deve fare prove di magia, si considera che riesca sempre con un punteggio pari alla Difficoltà + il valore del Carisma.\\
Se deve fare un Tiro per Colpire ha CA = Grado di Sfida con un bonus al colpire pari alla Forza\\
Gli incantesimi a disposizione sono:\\
- Scudo di Fuoco\\
- Tempesta di ghiaccio\\
- Tempesta di Nevischio\\


\medskip\index{Mostri - Drago Bianco Adulto}\textbf{Drago Bianco Adulto}

\emph{Enorme drago, caotico malvagio}

\textbf{FORZA} +6

\textbf{DESTREZZA} +0

\textbf{COSTITUZIONE} +6

\textbf{INTELLIGENZA} -1

\textbf{SAGGEZZA} +1

\textbf{CARISMA} +1

\textbf{Iniziativa} +0 -- \textbf{Difesa} 25

\textbf{Punti Ferita} 200 (16d12 + 96)

\textbf{Movimento} 12 m, nuoto 12 m, scavo 9 m, volo 24 m

\textbf{Tiri Salvezza} Tempra +13, Riflessi +9, Volontà +10

\textbf{Competenze} Muoversi Silenziosamente / Nascondersi +5, Consapevolezza +11

\textbf{Immunità al Danno} freddo

\textbf{Sensi} scurovisione 36 m, vista cieca 18 m

\textbf{Linguaggi} Comune, Draconico

\textbf{Sfida} 13 (10.000 PE)

\emph{\textbf{Camminare sul Ghiaccio.}} Il drago può muoversi e arrampicarsi su superfici ghiacciate senza bisogno di effettuare prove di caratteristica. Inoltre, il terreno difficile composto di ghiaccio o neve non gli costa movimento aggiuntivo.

\emph{\textbf{Resistenza Leggendaria (3/Giorno).}} Se il drago fallisce un Tiro Salvezza, può scegliere invece di riuscire. 

\textbf{Azioni}

\emph{\textbf{Multiattacco.}} Il drago può usare la sua Presenza Spaventosa e poi effettuare tre attacchi: uno con il morso e due con gli artigli.

\emph{\textbf{Artiglio.} Attacco con arma da mischia}: +11 a colpire, portata 1 m, un bersaglio.

\emph{Colpisce:} 13 (2d6 + 6) danni taglienti.

\emph{\textbf{Coda.} Attacco con arma da mischia}: +11 a colpire, portata 5 metri, un bersaglio.

\emph{Colpisce:} 15 (2d8 + 6) danni da botta.

\emph{\textbf{Morso.} Attacco con arma da mischia}: +11 a colpire, portata 3 m, un bersaglio.

\emph{Colpisce:} 17 (2d10 + 6) danni perforanti più 4 (1d8) danni da freddo.

\emph{\textbf{Presenza Spaventosa.}} Ogni creatura scelta dal drago, che si trovi entro 36 metri da esso e consapevole della sua presenza, deve riuscire un Tiro Salvezza di Volontà CD 14 o restare spaventata per 1 minuto. Una creatura può ripetere il Tiro Salvezza al termine di ciascun suo turno, terminando l'effetto se lo riesce. Se il Tiro Salvezza della creatura ha successo o l'effetto ha termine per essa, la creatura è immune alla Presenza Spaventosa del drago per le successive 24 ore.

\emph{\textbf{Soffio Gelido (Ricarica 5-6).}} Il drago esala un'esplosione di ghiaccio in un cono di 18 metri. Ogni creatura in quell'area deve effettuare un Tiro Salvezza di Tempra CD 19 e subire 54 (12d8) danni da freddo se fallisce il Tiro Salvezza, o la metà di questi danni se lo riesce. 

\textbf{Azioni Aggiuntive}

Il drago può effettuare 3 Azioni aggiuntive, scelte tra le opzioni seguenti. Può usare solo un'opzione leggendaria alla volta e solo al termine del turno di un'altra creatura. Il drago recupera le Azioni aggiuntive spese all'inizio del proprio turno.

\textbf{Attacco di Ala (Costa 2 Azioni).} Il drago batte le ali. Ogni creatura entro 3 metri dal drago deve riuscire un Tiro Salvezza su Riflessi CD 19 o subire 13 (2d6 + 6) danni da botta e venir gettato prono. Il drago può poi volare fino a metà del suo movimento di volo. \textbf{Attacco di Coda.} Il drago effettua un attacco di coda 
.
\textbf{Individuare.} Il drago effettua una prova di Saggezza (Consapevolezza).

\textbf{Ecologia}\\
Ambiente: Montagne Fredde\\
Organizzazione: Solitario\\
Tesoro: Triplo\\
\textbf{Descrizione}\\
Anche se molti lo considerano il più debole e bestiale tra i draghi cromatici, il drago bianco supplisce alla sua mancanza di astuzia con la pura e semplice ferocia. I draghi bianchi vivono su remote e gelide cime montuose, e nei bassopiani artici, facendo la loro tana in scintillanti caverne piene di ghiaccio e neve. Preferiscono che i loro pasti siano completamente congelati.\\
\textbf{Incantesimi}\index{Incantesimi da Drago Bianco}\\
Il drago e' anche capace, se di eta' giovanile o piu' anziano, di lanciare incantesimi.\\
Puo' lanciare un numero di incantesimi pari al suo valore di Carisma, senza aver necessita' di componenti magici.\\
Non deve fare prove di magia, si considera che riesca sempre con un punteggio pari alla Difficoltà + il valore del Carisma.\\
Se deve fare un Tiro per Colpire ha CA = Grado di Sfida con un bonus al colpire pari alla Forza\\
Gli incantesimi a disposizione sono:\\
- Scudo di Fuoco\\
- Tempesta di ghiaccio\\
- Tempesta di Nevischio\\


\medskip\index{Mostri - Drago Bianco Giovane}\textbf{Drago Bianco Giovane}

\emph{Grande drago, caotico malvagio}

\textbf{FORZA} +4

\textbf{DESTREZZA} +0

\textbf{COSTITUZIONE} +4

\textbf{INTELLIGENZA} -2

\textbf{SAGGEZZA} +0

\textbf{CARISMA} +1

\textbf{Iniziativa} +0 -- \textbf{Difesa} 20

\textbf{Punti Ferita} 133 (14d10 + 56)

\textbf{Movimento} 12 m, nuoto 12 m, scavo 6 m, volo 24 m

\textbf{Tiri Salvezza} Tempra +8, Riflessi +7, Volontà +5

\textbf{Competenze} Muoversi Silenziosamente / Nascondersi +3, Consapevolezza +6

\textbf{Immunità al Danno} freddo

\textbf{Sensi} scurovisione 36 m, vista cieca 9 m

\textbf{Linguaggi} Comune, Draconico

\textbf{Sfida} 6 (2.300 PE)

\emph{\textbf{Camminare sul Ghiaccio.}} Il drago può muoversi e arrampicarsi su superfici ghiacciate senza bisogno di effettuare prove di caratteristica. Inoltre, il terreno difficile composto di ghiaccio o neve non gli costa movimento aggiuntivo.

\textbf{Azioni}

\emph{\textbf{Multiattacco.}} Il drago può usare la sua Presenza Spaventosa. Poi effettuare tre attacchi: uno con il morso e due con gli artigli.

\emph{\textbf{Artiglio.} Attacco con arma da mischia}: +7 a colpire, portata 1 m, un bersaglio.

\emph{Colpisce:} 11 (2d6 + 4) danni taglienti.

\emph{\textbf{Morso.} Attacco con arma da mischia}: +7 a colpire, portata 3 m, un bersaglio.

\emph{Colpisce:} 15 (2d10 + 4) danni perforanti più 4 (1d8) danni da freddo.

\emph{\textbf{Soffio Gelido (Ricarica 5-6).}} Il drago esala un'esplosione di ghiaccio in un cono di 9 metri. Ogni creatura in quell'area deve effettuare un Tiro Salvezza di Tempra CD 15 e subire 45 (10d8) danni da freddo se fallisce il Tiro Salvezza, o la metà di questi danni se lo riesce.

\textbf{Ecologia}\\
Ambiente: Montagne Fredde\\
Organizzazione: Solitario\\
Tesoro: Triplo\\
\textbf{Descrizione}\\
Anche se molti lo considerano il più debole e bestiale tra i draghi cromatici, il drago bianco supplisce alla sua mancanza di astuzia con la pura e semplice ferocia. I draghi bianchi vivono su remote e gelide cime montuose, e nei bassopiani artici, facendo la loro tana in scintillanti caverne piene di ghiaccio e neve. Preferiscono che i loro pasti siano completamente congelati.\\
\textbf{Incantesimi}\index{Incantesimi da Drago Bianco}\\
Il drago e' anche capace, se di eta' giovanile o piu' anziano, di lanciare incantesimi.\\
Puo' lanciare un numero di incantesimi pari al suo valore di Carisma, senza aver necessita' di componenti magici.\\
Non deve fare prove di magia, si considera che riesca sempre con un punteggio pari alla Difficoltà + il valore del Carisma.\\
Se deve fare un Tiro per Colpire ha CA = Grado di Sfida con un bonus al colpire pari alla Forza\\
Gli incantesimi a disposizione sono:\\
- Scudo di Fuoco\\
- Tempesta di ghiaccio\\
- Tempesta di Nevischio\\


\medskip\index{Mostri - Drago Bianco Cucciolo}\textbf{Drago Bianco Cucciolo}

\emph{Media drago, caotico malvagio}

\textbf{FORZA} +2

\textbf{DESTREZZA} +0

\textbf{COSTITUZIONE} +2

\textbf{INTELLIGENZA} -3

\textbf{SAGGEZZA} +0

\textbf{CARISMA} +0

\textbf{Iniziativa} +0 -- \textbf{Difesa} 17

\textbf{Punti Ferita} 32 (5d8 + 10)

\textbf{Movimento} 9 m, nuoto 9 m, scavo 5 metri, volo 18 m

\textbf{Tiri Salvezza} Tempra +2, Riflessi +1, Volontà +1

\textbf{Competenze} Muoversi Silenziosamente / Nascondersi +2, Consapevolezza +4

\textbf{Immunità al Danno} freddo

\textbf{Sensi} scurovisione 18 m, vista cieca 3 m

\textbf{Linguaggi} Draconico

\textbf{Sfida} 2 (450 PE)

\textbf{Azioni}

\emph{\textbf{Morso.} Attacco con arma da mischia}: +7 a colpire, portata 3 m, un bersaglio.

\emph{Colpisce:} 15 (2d10 + 4) danni perforanti più 4 (1d8) danni da freddo.

\emph{\textbf{Soffio Gelido (Ricarica 5-6).}} Il drago esala un'esplosione di ghiaccio in un cono di 5 metri. Ogni creatura in quell'area deve effettuare un Tiro Salvezza di Tempra CD 12 e subire 22 (5d8) danni da freddo se fallisce il Tiro Salvezza, o la metà di questi danni se lo riesce.

\textbf{Ecologia}\\
Ambiente: Montagne Fredde\\
Organizzazione: Solitario\\
Tesoro: Triplo\\
\textbf{Descrizione}\\
Anche se molti lo considerano il più debole e bestiale tra i draghi cromatici, il drago bianco supplisce alla sua mancanza di astuzia con la pura e semplice ferocia. I draghi bianchi vivono su remote e gelide cime montuose, e nei bassopiani artici, facendo la loro tana in scintillanti caverne piene di ghiaccio e neve. Preferiscono che i loro pasti siano completamente congelati.\\


\medskip\index{Mostri - Drago Blu Antico}\textbf{Drago Blu Antico}

\emph{Mastodontica drago, legale malvagio}

\textbf{FORZA} +9

\textbf{DESTREZZA} +0

\textbf{COSTITUZIONE} +8

\textbf{INTELLIGENZA} +4

\textbf{SAGGEZZA} +3

\textbf{CARISMA} +5

\textbf{Iniziativa} +4 -- \textbf{Difesa} 34

\textbf{Punti Ferita} 481 (26d20 + 208)

\textbf{Movimento} 12 m, scavo 12 m, volo 24 m

\textbf{Tiri Salvezza} Tempra +21, Riflessi +13, Volontà +19

\textbf{Competenze} Muoversi Silenziosamente / Nascondersi +7, Consapevolezza +17

\textbf{Immunità al Danno} fulmine, arma +1

\textbf{Sensi} scurovisione 36 m, vista cieca 18 m

\textbf{Linguaggi} Comune, Draconico

\textbf{Sfida} 23 (50.000 PE)

\emph{\textbf{Resistenza Leggendaria (3/Giorno).}} Se il drago fallisce un Tiro Salvezza, può scegliere invece di riuscire.

\textbf{Azioni}

\emph{\textbf{Multiattacco.}} Il drago può usare la sua Presenza Spaventosa. Poi effettuare tre attacchi: uno con il morso e due con gli artigli.

\emph{\textbf{Artiglio.} Attacco con arma da mischia}: +16 a colpire,
portata 3 m, un bersaglio.

\emph{Colpisce:} 16 (2d6 + 9) danni taglienti.

\emph{\textbf{Coda.} Attacco con arma da mischia}: +16 a colpire, portata 6 m, un bersaglio.

\emph{Colpisce:} 18 (2d8 + 9) danni da botta.

\emph{\textbf{Morso.} Attacco con arma da mischia}: +16 a colpire, portata 5 metri, un bersaglio.

\emph{Colpisce:} 20 (2d10 + 9) danni perforanti più 11 (2d10) danni da fulmine.

\emph{\textbf{Presenza Spaventosa.}} Ogni creatura scelta dal drago, che si trovi entro 36 metri da esso e consapevole della sua presenza, deve riuscire un Tiro Salvezza di Volontà CD 20 o restare spaventata per 1 minuto. Una creatura può ripetere il Tiro Salvezza al termine di ciascun suo turno, terminando l'effetto se lo riesce. Se il Tiro Salvezza della creatura ha successo o l'effetto ha termine per essa, la creatura è immune alla Presenza Spaventosa del drago per le successive 24 ore.

\emph{\textbf{Soffio Fulminante (Ricarica 5-6).}} Il drago esala fulmini in una linea lunga 36 metri e larga 3 metri. Ogni creatura su quella linea deve effettuare un Tiro Salvezza di Riflessi CD 23 e subire 88 (16d10) danni da fulmine se fallisce il Tiro Salvezza, o la metà di questi danni se lo riesce.

\textbf{Azioni Aggiuntive}

Il drago può effettuare 3 Azioni aggiuntive, scelte tra le opzioni seguenti. Può usare solo un'opzione leggendaria alla volta e solo al termine del turno di un'altra creatura. Il drago recupera le Azioni aggiuntive spese all'inizio del proprio turno.

\textbf{Attacco di Ala (Costa 2 Azioni).} Il drago batte le ali. Ogni creatura entro 5 metri dal drago deve riuscire un Tiro Salvezza su Riflessi CD 24 o subire 16 (2d6 + 9) danni da botta e venir gettato prono. Il drago può poi volare fino a metà del suo movimento di volo.

\textbf{Attacco di Coda.} Il drago effettua un attacco di coda.

\textbf{Individuare.} Il drago effettua una prova di Saggezza (Consapevolezza).

\textbf{Individuare.} Il drago effettua una prova di Saggezza (Consapevolezza).\\
\textbf{Ecologia}\\
Ambiente: Picchi montuosi\\
Organizzazione: Solitario\\
Tesoro: Triplo\\
\textbf{Descrizione}\\
I draghi blu sono intriganti consumati ed ossessivamente ordinati. In combattimento, i draghi blu preferiscono prendere di sorpresa i nemici, se possibile, e non esitano a ritirarsi se le cose si mettono male. Preferiscono fare la loro tana vicino a quelli che controllano, qualche volta anche entro i confini di una città.\\
\textbf{Incantesimi}\index{Incantesimi da Drago Blu}\\
Il drago e' anche capace, se di eta' giovanile o piu' anziano, di lanciare incantesimi.\\
Puo' lanciare un numero di incantesimi pari al suo valore di Carisma, senza aver necessita' di componenti magici.\\
Non deve fare prove di magia, si considera che riesca sempre con un punteggio pari alla Difficoltà + il valore del Carisma.\\
Se deve fare un Tiro per Colpire ha CA = Grado di Sfida con un bonus al colpire pari alla Forza\\
Gli incantesimi a disposizione sono:\\
- Catena di fulmini\\
- Gabbia di Forza\\
- Teletrasporto\\
- Forma Eterea\\


\medskip\index{Mostri - Drago Blu Adulto}\textbf{Drago Blu Adulto}

\emph{Enorme drago, legale malvagio}

\textbf{FORZA} +7

\textbf{DESTREZZA} +0

\textbf{COSTITUZIONE} +6

\textbf{INTELLIGENZA} +3

\textbf{SAGGEZZA} +2

\textbf{CARISMA} +4

\textbf{Iniziativa} +3 -- \textbf{Difesa} 27

\textbf{Punti Ferita} 225 (18d12 + 108)

\textbf{Movimento} 12 m, scavo 12 m, volo 24 m

\textbf{Tiri Salvezza} Tempra +15, Riflessi +10, Volontà +13

\textbf{Competenze} Muoversi Silenziosamente / Nascondersi +5, Consapevolezza +12

\textbf{Immunità al Danno} fulmine

\textbf{Sensi} scurovisione 36 m, vista cieca 18 m

\textbf{Linguaggi} Comune, Draconico

\textbf{Sfida} 16 (15.000 PE)

\emph{\textbf{Resistenza Leggendaria (3/Giorno).}} Se il drago fallisce un Tiro Salvezza, può scegliere invece di riuscire.

\textbf{Azioni}

\emph{\textbf{Multiattacco.}} Il drago può usare la sua Presenza Spaventosa. Poi effettuare tre attacchi: uno con il morso e due con gli artigli.

\emph{\textbf{Artiglio.} Attacco con arma da mischia}: +12 a colpire, portata 1 m, un bersaglio.

\emph{Colpisce:} 14 (2d6 + 7) danni taglienti.

\emph{\textbf{Coda.} Attacco con arma da mischia}: +12 a colpire, portata 5 metri, un bersaglio.

\emph{Colpisce:} 16 (2d8 + 7) danni da botta.

\emph{\textbf{Morso.} Attacco con arma da mischia}: +12 a colpire, portata 3 m, un bersaglio.

\emph{Colpisce:} 18 (2d10 + 7) danni perforanti più 5 (1d10) danni da fulmine.

\emph{\textbf{Presenza Spaventosa.}} Ogni creatura scelta dal drago, che si trovi entro 36 metri da esso e consapevole della sua presenza, deve riuscire un Tiro Salvezza di Volontà CD 17 o restare spaventata per 1 minuto. Una creatura può ripetere il Tiro Salvezza al termine di ciascun suo turno, terminando l'effetto se lo riesce. Se il Tiro Salvezza della creatura ha successo o l'effetto ha termine per essa, la creatura è immune alla Presenza Spaventosa del drago per le successive 24 ore.

\emph{\textbf{Soffio Fulminante (Ricarica 5-6).}} Il drago esala fulmini in una linea lunga 27 metri e larga 1 metro. Ogni creatura su quella linea deve effettuare un Tiro Salvezza di Riflessi CD 19 e subire 66 (12d10) danni da fulmine se fallisce il Tiro Salvezza, o la metà di questi danni se lo riesce.

\textbf{Azioni Aggiuntive}

Il drago può effettuare 3 Azioni aggiuntive, scelte tra le opzioni seguenti. Può usare solo un'opzione leggendaria alla volta e solo al termine del turno di un'altra creatura. Il drago recupera le Azioni aggiuntive spese all'inizio del proprio turno.

\textbf{Attacco di Ala (Costa 2 Azioni).} Il drago batte le ali. Ogni creatura entro 3 metri dal drago deve riuscire un Tiro Salvezza su Riflessi CD 20 o subire 14 (2d6 + 7) danni da botta e venir gettato prono. Il drago può poi volare fino a metà della del suo movimento di  volo.

\textbf{Attacco di Coda.} Il drago effettua un attacco di coda.

\textbf{Individuare.} Il drago effettua una prova di Saggezza (Consapevolezza).

\textbf{Ecologia}\\
Ambiente: Picchi montuosi\\
Organizzazione: Solitario\\
Tesoro: Triplo\\
\textbf{Descrizione}\\
I draghi blu sono intriganti consumati ed ossessivamente ordinati. In combattimento, i draghi blu preferiscono prendere di sorpresa i nemici, se possibile, e non esitano a ritirarsi se le cose si mettono male. Preferiscono fare la loro tana vicino a quelli che controllano, qualche volta anche entro i confini di una città.\\
\textbf{Incantesimi}\index{Incantesimi da Drago Blu}\\
Il drago e' anche capace, se di eta' giovanile o piu' anziano, di lanciare incantesimi.\\
Puo' lanciare un numero di incantesimi pari al suo valore di Carisma, senza aver necessita' di componenti magici.\\
Non deve fare prove di magia, si considera che riesca sempre con un punteggio pari alla Difficoltà + il valore del Carisma.\\
Se deve fare un Tiro per Colpire ha CA = Grado di Sfida con un bonus al colpire pari alla Forza\\
Gli incantesimi a disposizione sono:\\
- Catena di fulmini\\
- Gabbia di Forza\\
- Teletrasporto\\
- Forma Eterea\\


\medskip\index{Mostri - Drago Blu Giovane}\textbf{Drago Blu Giovane}

\emph{Enorme drago, legale malvagio}

\textbf{FORZA} 21(+5)

\textbf{DESTREZZA} +0

\textbf{COSTITUZIONE} +4

\textbf{INTELLIGENZA} +2

\textbf{SAGGEZZA} +1

\textbf{CARISMA} +3

\textbf{Iniziativa} +2 -- \textbf{Difesa} 23

\textbf{Punti Ferita} 152 (16d10 + 64)

\textbf{Movimento} 12 m, scavo 12 m, volo 24 m

\textbf{Tiri Salvezza} Tempra +10, Riflessi +8, Volontà +8

\textbf{Competenze} Muoversi Silenziosamente / Nascondersi +4, Consapevolezza +9

\textbf{Immunità al Danno} fulmine

\textbf{Sensi} scurovisione 36 m, vista cieca 9 m 

\textbf{Linguaggi} Comune, Draconico

\textbf{Sfida} 9 (5.000 PE)

\textbf{Azioni}

\emph{\textbf{Multiattacco.}} Il drago può effettuare tre attacchi: uno con il morso e due con gli artigli.

\emph{\textbf{Artiglio.} Attacco con arma da mischia}: +9 a colpire, portata 1 m, un bersaglio.

\emph{Colpisce:} 12 (2d6 + 5) danni taglienti.

\emph{\textbf{Morso.} Attacco con arma da mischia}: +9 a colpire, portata 3 m, un bersaglio.

\emph{Colpisce:} 16 (2d10 + 5) danni perforanti più 5 (1d10) danni da fulmine.

\emph{\textbf{Soffio Fulminante (Ricarica 5-6).}} Il drago esala fulmini in una linea lunga 18 metri e larga 1 metro. Ogni creatura su quella linea deve effettuare un Tiro Salvezza di Riflessi CD 16 e subire 55 (10d10) danni da fulmine se fallisce il Tiro Salvezza, o la metà di questi danni se lo riesce.

\textbf{Ecologia}\\
Ambiente: Picchi montuosi\\
Organizzazione: Solitario\\
Tesoro: Triplo\\
\textbf{Descrizione}\\
I draghi blu sono intriganti consumati ed ossessivamente ordinati. In combattimento, i draghi blu preferiscono prendere di sorpresa i nemici, se possibile, e non esitano a ritirarsi se le cose si mettono male. Preferiscono fare la loro tana vicino a quelli che controllano, qualche volta anche entro i confini di una città.\\
\textbf{Incantesimi}\index{Incantesimi da Drago Blu}\\
Il drago e' anche capace, se di eta' giovanile o piu' anziano, di lanciare incantesimi.\\
Puo' lanciare un numero di incantesimi pari al suo valore di Carisma, senza aver necessita' di componenti magici.\\
Non deve fare prove di magia, si considera che riesca sempre con un punteggio pari alla Difficoltà + il valore del Carisma.\\
Se deve fare un Tiro per Colpire ha CA = Grado di Sfida con un bonus al colpire pari alla Forza\\
Gli incantesimi a disposizione sono:\\
- Catena di fulmini\\
- Gabbia di Forza\\
- Teletrasporto\\
- Forma Eterea\\


\medskip\index{Mostri - Drago Blu Cucciolo}\textbf{Drago Blu Cucciolo}

\emph{Enorme drago, legale malvagio}

\textbf{FORZA} +3

\textbf{DESTREZZA} +0

\textbf{COSTITUZIONE} +2

\textbf{INTELLIGENZA} +1

\textbf{SAGGEZZA} +0

\textbf{CARISMA} +2

\textbf{Iniziativa} +1 -- \textbf{Difesa} 19

\textbf{Punti Ferita} 52 (8d8 + 16)

\textbf{Movimento} 9 m, scavo 5 metri, volo 18 m

\textbf{Tiri Salvezza} Tempra +4, Riflessi +1, Volontà +1

\textbf{Competenze} Muoversi Silenziosamente / Nascondersi +2, Consapevolezza +4

\textbf{Immunità al Danno} fulmine

\textbf{Sensi} scurovisione 18 m, vista cieca 3 m

\textbf{Linguaggi} Draconico

\textbf{Sfida} 3 (700 PE)

\textbf{Azioni}

\emph{\textbf{Morso.} Attacco con arma da mischia}: +5 a colpire, portata 1 m, un bersaglio.

\emph{Colpisce:} 8 (1d10 + 3) danni perforanti più 3 (1d6) danni da fulmine.

\emph{\textbf{Soffio Fulminante (Ricarica 5-6).}} Il drago esala fulmini in una linea lunga 9 metri e larga 1 metro. Ogni creatura su quella linea deve effettuare un Tiro Salvezza di Riflessi CD 12 e subire 22 (4d10) danni da fulmine se fallisce il Tiro Salvezza, o la metà di questi danni se lo riesce.

\textbf{Ecologia}\\
Ambiente: Picchi montuosi\\
Organizzazione: Solitario\\
Tesoro: Triplo\\
\textbf{Descrizione}\\
I draghi blu sono intriganti consumati ed ossessivamente ordinati. In combattimento, i draghi blu preferiscono prendere di sorpresa i nemici, se possibile, e non esitano a ritirarsi se le cose si mettono male. Preferiscono fare la loro tana vicino a quelli che controllano, qualche volta anche entro i confini di una città.\\

\medskip\textbf{Drago Giallo Antico}\index{Mostri - Drago Giallo Antico}\\
\emph{Mastodontica drago, neutrale malvagio}\\
\textbf{Forza}: +10\\
\textbf{Destrezza}: +1\\
\textbf{Costituzione}: +8\\
\textbf{Intelligenza}: +3\\
\textbf{Saggezza}: +2\\
\textbf{Carisma}: +4\\	
\textbf{Difesa}: 27 (armatura naturale) - \textbf{Iniziativa}: +4\\
\textbf{Punti Ferita}: 481 (26d20 + 208)\\
\textbf{Movimento}: 12 m, scavo 24 m, scalata 24, volo 12 m\\
\textbf{Tiri Salvezza}: Tempra +21, Riflessi +13, Volontà +19\\
\textbf{Competenze}: Criminalità +7, Consapevolezza +17\\
\textbf{Immunità al Danno}: fulmine\\
\textbf{Sensi}: Scurovisione 36 m, vista cieca 18 m\\
\textbf{Linguaggi} Comune, Draconico\\
\textbf{Sfida}: 23 (50.000 PE)\smallskip\\
\emph{\textbf{Resistenza Leggendaria (3/Giorno).}} Se il drago fallisce un Tiro Salvezza, può scegliere invece di riuscire. \\
\smallskip\textbf{Azioni}\\
\emph{\textbf{Multiattacco.}} Il drago può usare la sua Presenza Spaventosa. Poi effettuare tre attacchi: uno con il morso e due con gli artigli.\\
\emph{\textbf{Artiglio.} Attacco con arma da mischia}: +25 al colpire, portata 3 m, un bersaglio.\\
\emph{Colpisce:} 16 (2d6 + 9) danni taglienti.\\
\emph{\textbf{Coda.} Attacco con arma da mischia}: +25 al colpire, portata 6 m, un bersaglio.\\
\emph{Colpisce:} 18 (2d8 + 9) danni contundenti.\\
\emph{\textbf{Morso.} Attacco con arma da mischia}: +25 al colpire, portata 5 metri, un bersaglio.\\
\emph{Colpisce:} 20 (2d10 + 9) danni perforanti più 11 (2d10) danni da fulmine.\\
\emph{\textbf{Presenza Spaventosa.}} Ogni creatura scelta dal drago, che si trovi entro 36 metri da esso e consapevole della sua presenza, deve riuscire un Tiro Salvezza su Volontà DC  25 o restare spaventata per 1 minuto. Una creatura può ripetere il Tiro Salvezza al termine di ciascun suo round, terminando l'effetto se lo riesce. Se il Tiro Salvezza della creatura ha successo o l'effetto ha termine per essa, la creatura è immune alla Presenza Spaventosa del drago per le successive 24 ore.\\
\emph{\textbf{Soffio Incendiario (Ricarica 5-6).}} Il drago esala aria rovente in una linea lunga 36 metri e larga 3 metri. Ogni creatura su quella linea deve effettuare un Tiro Salvezza su Riflessi DC 30 e subire 88 (16d10) danni da fuoco se fallisce il Tiro Salvezza, o la metà di questi danni se lo riesce.\\
\textbf{Azioni Aggiuntive}\\
Il drago può effettuare 3 azioni aggiuntive, scelte tra le opzioni seguenti. Può usare solo un'Azione Aggiuntiva alla volta e solo al termine del round di un'altra creatura. Il drago recupera le Azioni Aggiuntive spese all'inizio del proprio round.\\
\textbf{Attacco di Ala (Costa 2 Azioni).} Il drago batte le ali. Ogni creatura entro 5 metri dal drago deve riuscire un Tiro Salvezza su Riflessi DC  31 o subire 16 (2d6 + 9) danni contundenti e venir gettato prono. Il drago può poi volare fino a metà della sua velocità di volo.\\
\textbf{Attacco di Coda.} Il drago effettua un attacco di coda.\\
\textbf{Individuare.} Il drago effettua una prova di Saggezza (Consapevolezza).\\
\textbf{Ecologia}\\
Ambiente: Deserti Caldi\\
Organizzazione: Solitario\\
Tesoro: Triplo\\
\textbf{Descrizione}\\
I draghi gialli sono predatori famelici e combattenti indisciplinati. Amano la caccia ed uccidere, sono consumati predatori che istintivamente aggrediscono chiunque sia nel loro territorio. Il deserto e' il loro terreno dove scavano trappole grezze per catturare le loro povere vittime.
Il soffio di un drago giallo e' un onda di calore (danno da fuoco).
\\
\textbf{Incantesimi}\index{Incantesimi da Drago Giallo}\\
Il drago e' anche capace, se di eta' giovanile o piu' anziano, di lanciare incantesimi.\\
Puo' lanciare un numero di incantesimi pari al suo valore di Carisma, senza aver necessita' di componenti magici.\\
Non deve fare prove di magia, si considera che riesca sempre con un punteggio pari alla Difficoltà + il valore del Carisma.\\
Se deve fare un Tiro per Colpire ha CA = Grado di Sfida con un bonus al colpire pari alla Forza\\
Gli incantesimi a disposizione sono:\\
- Riscaldare Metallo\\
- Palla di Fuoco\\
- Scudo di Fuoco\\


\medskip\index{Mostri - Drago Nero Antico}\textbf{Drago Nero Antico}

\emph{Mastodontica drago, caotico malvagio}

\textbf{FORZA} +8

\textbf{DESTREZZA} +2

\textbf{COSTITUZIONE} +7

\textbf{INTELLIGENZA} +3

\textbf{SAGGEZZA} +2

\textbf{CARISMA} +4

\textbf{Iniziativa} +3 -- \textbf{Difesa} 33

\textbf{Punti Ferita} 367 (21d20 + 147)

\textbf{Movimento} 12 m, scalata 12 m, volo 24 m

\textbf{Tiri Salvezza} Tempra +20, Riflessi +13, Volontà +18

\textbf{Competenze} Muoversi Silenziosamente / Nascondersi +9, Consapevolezza +16

\textbf{Immunità al Danno} acido, arma +1

\textbf{Sensi} scurovisione 36 m, vista cieca 18 m

\textbf{Linguaggi} Comune, Draconico

\textbf{Sfida} 21 (33.000 PE)

\emph{\textbf{Anfibio.}} Il drago può respirare aria e acqua.

\emph{\textbf{Resistenza Leggendaria (3/Giorno).}} Se il drago fallisce un Tiro Salvezza, può scegliere invece di riuscire.

\textbf{Azioni}

\emph{\textbf{Multiattacco.}} Il drago può usare la sua Presenza Spaventosa. Poi effettuare tre attacchi: uno con il morso e due con gli artigli.

\emph{\textbf{Artiglio.} Attacco con arma da mischia}: +15 a colpire, portata 3 m, un bersaglio.

\emph{Colpisce:} 15 (2d6 + 8) danni taglienti.

\emph{\textbf{Coda.} Attacco con arma da mischia}: +15 a colpire, portata 6 m, un bersaglio.

\emph{Colpisce:} 17 (2d8 + 8) danni da botta.

\emph{\textbf{Morso.} Attacco con arma da mischia} : +15 a colpire, portata 5 metri, un bersaglio.

\emph{Colpisce:} 19 (2d10 + 8) danni perforanti più 9 (4d6) danni da acido.

\emph{\textbf{Presenza Spaventosa.}} Ogni creatura scelta dal drago, che si trovi entro 36 metri da esso e consapevole della sua presenza, deve riuscire un Tiro Salvezza di Volontà CD 19 o restare spaventata per 1 minuto. Una creatura può ripetere il Tiro Salvezza al termine di ciascun suo turno, terminando l'effetto se lo riesce. Se il Tiro Salvezza della creatura ha successo o l'effetto ha termine per essa, la creatura è immune alla Presenza Spaventosa del drago per le successive 24 ore.

\emph{\textbf{Soffio Acido (Ricarica 5-6).}} Il drago esala acido in una linea di 27 metri larga 3 metri. Ogni creatura in quell'area deve effettuare un Tiro Salvezza di Riflessi CD 22 e subire 67 (15d8) danni da acido se fallisce il Tiro Salvezza, o la metà di questi danni se lo riesce.

\textbf{Azioni Aggiuntive}

Il drago può effettuare 3 Azioni aggiuntive, scelte tra le opzioni seguenti. Può usare solo un'opzione leggendaria alla volta e solo al termine del turno di un'altra creatura. Il drago recupera le Azioni aggiuntive spese all'inizio del proprio turno.

\textbf{Attacco di Ala (Costa 2 Azioni).} Il drago batte le ali. Ogni creatura entro 5 metri dal drago deve riuscire un Tiro Salvezza su Riflessi CD 23 o subire 15 (2d6 + 8) danni da botta e venir gettato prono. Il drago può poi volare fino a metà del suo movimento di volo.  

\textbf{Attacco di Coda.} Il drago effettua un attacco di coda.

\textbf{Individuare.} Il drago effettua una prova di Saggezza (Consapevolezza).

\textbf{Ecologia}\\
Ambiente: Paludi Calde\\
Organizzazione: Solitario\\
Tesoro: Triplo\\
\textbf{Descrizione}\\
Signori delle paludi e degli acquitrini più cupi, i draghi neri sono i padroni incontrastati del loro territorio, che dominano con crudeltà e infondendo terrore in chi abita nelle vicinanze. I draghi neri si stanziano nelle zone più remote delle paludi, specie in caverne sul fondo di pozze fetide e buie. Dentro, ammassano i loro luridi tesori e dormono tra radici e fango. I draghi neri amano il cibo un po' marcio e spesso lasciano un pasto a marcire in una pozza per giorni prima di consumarlo. I draghi neri preferiscono tesori che non si decompongono o degradano, accumulando tesori di monete, pietre preziose, gioielli e altri oggetti di pietra o metallo.\\
\textbf{Incantesimi}\index{Incantesimi da Drago Nero}\\
Il drago e' anche capace, se di eta' giovanile o piu' anziano, di lanciare incantesimi.\\
Puo' lanciare un numero di incantesimi pari al suo valore di Carisma, senza aver necessita' di componenti magici.\\
Non deve fare prove di magia, si considera che riesca sempre con un punteggio pari alla Difficoltà + il valore del Carisma.\\
Se deve fare un Tiro per Colpire ha CA = Grado di Sfida con un bonus al colpire pari alla Forza\\
Gli incantesimi a disposizione sono:\\
- Dito della morte\\
- Disintegrazione\\
- Blocca Mostri\\


\medskip\index{Mostri - Drago Nero Adulto}\textbf{Drago Nero Adulto}

\emph{Enorme drago, caotico malvagio}

\textbf{FORZA} +6

\textbf{DESTREZZA} +2

\textbf{COSTITUZIONE} +5

\textbf{INTELLIGENZA} +2

\textbf{SAGGEZZA} +1

\textbf{CARISMA} +3

\textbf{Iniziativa} +2 -- \textbf{Difesa} 28

\textbf{Punti Ferita} 195 (17d12 + 85)

\textbf{Movimento} 12 m, scalata 12 m, volo 24 m

\textbf{Tiri Salvezza} Tempra +14, Riflessi +10, Volontà +12

\textbf{Competenze} Muoversi Silenziosamente / Nascondersi +7, Consapevolezza +11

\textbf{Immunità al Danno} acido

\textbf{Sensi} scurovisione 36 m, vista cieca 18 m 

\textbf{Linguaggi} Comune, Draconico

\textbf{Sfida} 17 (18.000 PE)

\emph{\textbf{Anfibio.}} Il drago può respirare aria e acqua.

\emph{\textbf{Resistenza Leggendaria (3/Giorno).}} Se il drago fallisce un Tiro Salvezza, può scegliere invece di riuscire.

\textbf{Azioni}

\emph{\textbf{Multiattacco.}} Il drago può usare la sua Presenza Spaventosa. Poi effettuare tre attacchi: uno con il morso e due con gli artigli.

\emph{\textbf{Artiglio.} Attacco con arma da mischia}: +11 a colpire, portata 1 m, un bersaglio.

\emph{Colpisce:} 13 (2d6 + 6) danni taglienti.

\emph{\textbf{Coda.} Attacco con arma da mischia}: +11 a colpire, portata 5 metri, un bersaglio.

\emph{Colpisce:} 15 (2d8 + 6) danni da botta.

\emph{\textbf{Morso.} Attacco con arma da mischia}: +11 a colpire, portata 3 m, un bersaglio.

\emph{Colpisce:} 17 (2d10 + 6) danni perforanti più 4 (1d8) danni da acido.

\emph{\textbf{Presenza Spaventosa.}} Ogni creatura scelta dal drago, che si trovi entro 36 metri da esso e consapevole della sua presenza, deve riuscire un Tiro Salvezza di Volontà CD 16 o restare spaventata per 1 minuto. Una creatura può ripetere il Tiro Salvezza al termine di ciascun suo turno, terminando l'effetto se lo riesce. Se il Tiro Salvezza della creatura ha successo o l'effetto ha termine per essa, la creatura è immune alla Presenza Spaventosa del drago per le successive 24 ore.

\emph{\textbf{Soffio Acido (Ricarica 5-6).}} Il drago esala acido in una linea di 18 metri larga 1 metro. Ogni creatura in quell'area deve effettuare un Tiro Salvezza di Riflessi CD 18 e subire 54 (12d8) danni da acido se fallisce il Tiro Salvezza, o la metà di questi danni se lo
riesce.

\textbf{Azioni Aggiuntive}

Il drago può effettuare 3 Azioni aggiuntive, scelte tra le opzioni seguenti. Può usare solo un'opzione leggendaria alla volta e solo al termine del turno di un'altra creatura. Il drago recupera le Azioni aggiuntive spese all'inizio del proprio turno.

\textbf{Attacco di Ala (Costa 2 Azioni).} Il drago batte le ali. Ogni creatura entro 3 metri dal drago deve riuscire un Tiro Salvezza su Riflessi CD 19 o subire 13 (2d6 + 6) danni da botta e venir gettato prono. Il drago può poi volare fino a metà della del suo movimento di volo.

\textbf{Attacco di Coda.} Il drago effettua un attacco di coda.

\textbf{Individuare.} Il drago effettua una prova di Saggezza (Consapevolezza).

\textbf{Ecologia}\\
Ambiente: Paludi Calde\\
Organizzazione: Solitario\\
Tesoro: Triplo\\
\textbf{Descrizione}\\
Signori delle paludi e degli acquitrini più cupi, i draghi neri sono i padroni incontrastati del loro territorio, che dominano con crudeltà e infondendo terrore in chi abita nelle vicinanze. I draghi neri si stanziano nelle zone più remote delle paludi, specie in caverne sul fondo di pozze fetide e buie. Dentro, ammassano i loro luridi tesori e dormono tra radici e fango. I draghi neri amano il cibo un po' marcio e spesso lasciano un pasto a marcire in una pozza per giorni prima di consumarlo. I draghi neri preferiscono tesori che non si decompongono o degradano, accumulando tesori di monete, pietre preziose, gioielli e altri oggetti di pietra o metallo.\\
\textbf{Incantesimi}\index{Incantesimi da Drago Nero}\\
Il drago e' anche capace, se di eta' giovanile o piu' anziano, di lanciare incantesimi.\\
Puo' lanciare un numero di incantesimi pari al suo valore di Carisma, senza aver necessita' di componenti magici.\\
Non deve fare prove di magia, si considera che riesca sempre con un punteggio pari alla Difficoltà + il valore del Carisma.\\
Se deve fare un Tiro per Colpire ha CA = Grado di Sfida con un bonus al colpire pari alla Forza\\
Gli incantesimi a disposizione sono:\\
- Dito della morte\\
- Disintegrazione\\
- Blocca Mostri\\


\medskip\index{Mostri - Drago Nero Giovane}\textbf{Drago Nero Giovane}

\emph{Grande drago, caotico malvagio}

\textbf{FORZA} +4

\textbf{DESTREZZA} +2

\textbf{COSTITUZIONE} +3

\textbf{INTELLIGENZA} +1

\textbf{SAGGEZZA} +0

\textbf{CARISMA} +2

\textbf{Iniziativa} +2 -- \textbf{Difesa} 22

\textbf{Punti Ferita} 127 (15d10 + 45)

\textbf{Movimento} 12 m, scalata 12 m, volo 24 m

\textbf{Tiri Salvezza} Tempra +9, Riflessi +8, Volontà +7

\textbf{Competenze} Muoversi Silenziosamente / Nascondersi +5, Consapevolezza +6

\textbf{Immunità al Danno} acido

\textbf{Sensi} scurovisione 36 m, vista cieca 9 m

\textbf{Linguaggi} Comune, Draconico

\textbf{Sfida} 7 (2.900 PE)

\emph{\textbf{Anfibio.}} Il drago può respirare aria e acqua.

\textbf{Azioni}

\emph{\textbf{Multiattacco.}} Il drago può effettuare tre attacchi: uno con il morso e due con gli artigli.

\emph{\textbf{Artiglio.} Attacco con arma da mischia}: +10 a colpire, portata 1 m, un bersaglio.

\emph{Colpisce:} 11 (2d6 + 4) danni taglienti.

\emph{\textbf{Morso.} Attacco con arma da mischia}: +7 a colpire, portata 3 m, un bersaglio.

\emph{Colpisce:} 11 (2d10 + 4) danni perforanti più 4 (1d8) danni da acido.

\emph{\textbf{Soffio Acido (Ricarica 5-6).}} Il drago esala acido in una linea di 9 metri larga 1 metro. Ogni creatura in quell'area deve effettuare un Tiro Salvezza di Riflessi CD 14 e subire 49 (11d8) danni da acido se fallisce il Tiro Salvezza, o la metà di questi danni se lo riesce.

\textbf{Ecologia}\\
Ambiente: Paludi Calde\\
Organizzazione: Solitario\\
Tesoro: Triplo\\
\textbf{Descrizione}\\
Signori delle paludi e degli acquitrini più cupi, i draghi neri sono i padroni incontrastati del loro territorio, che dominano con crudeltà e infondendo terrore in chi abita nelle vicinanze. I draghi neri si stanziano nelle zone più remote delle paludi, specie in caverne sul fondo di pozze fetide e buie. Dentro, ammassano i loro luridi tesori e dormono tra radici e fango. I draghi neri amano il cibo un po' marcio e spesso lasciano un pasto a marcire in una pozza per giorni prima di consumarlo. I draghi neri preferiscono tesori che non si decompongono o degradano, accumulando tesori di monete, pietre preziose, gioielli e altri oggetti di pietra o metallo.\\
\textbf{Incantesimi}\index{Incantesimi da Drago Nero}\\
Il drago e' anche capace, se di eta' giovanile o piu' anziano, di lanciare incantesimi.\\
Puo' lanciare un numero di incantesimi pari al suo valore di Carisma, senza aver necessita' di componenti magici.\\
Non deve fare prove di magia, si considera che riesca sempre con un punteggio pari alla Difficoltà + il valore del Carisma.\\
Se deve fare un Tiro per Colpire ha CA = Grado di Sfida con un bonus al colpire pari alla Forza\\
Gli incantesimi a disposizione sono:\\
- Dito della morte\\
- Disintegrazione\\
- Blocca Mostri\\


\medskip\index{Mostri - Drago Nero Cucciolo}\textbf{Drago Nero Cucciolo}

\emph{Media drago, caotico malvagio}

\textbf{FORZA} +2

\textbf{DESTREZZA} +2

\textbf{COSTITUZIONE} +1

\textbf{INTELLIGENZA} +0

\textbf{SAGGEZZA} +0

\textbf{CARISMA} +1

\textbf{Iniziativa} +2 -- \textbf{Difesa} 18

\textbf{Punti Ferita} 33 (6d8 + 6)

\textbf{Movimento} 9 m, scalata 9 m, volo 18 m

\textbf{Tiri Salvezza} Tempra +2, Riflessi +2, Volontà +0

\textbf{Competenze} Muoversi Silenziosamente / Nascondersi +4, Consapevolezza +4

\textbf{Immunità al Danno} acido

\textbf{Sensi} scurovisione 18 m, vista cieca 3 m

\textbf{Linguaggi} Draconico

\textbf{Sfida} 2 (450 PE)

\emph{\textbf{Anfibio.}} Il drago può respirare aria e acqua.

\textbf{Azioni}

\emph{\textbf{Morso.} Attacco con arma da mischia}: +4 a colpire, portata 1 m, un bersaglio.

\emph{Colpisce:} 7 (1d10 + 2) danni perforanti più 2 (1d4) danni da acido.

\emph{\textbf{Soffio Acido (Ricarica 5-6).}} Il drago esala acido in una linea di 5 metri larga 1 metro. Ogni creatura in quell'area deve effettuare un Tiro Salvezza di Riflessi CD 11 e subire 22 (5d8) danni da acido se fallisce il Tiro Salvezza, o la metà di questi danni se lo riesce.

\textbf{Ecologia}\\
Ambiente: Paludi Calde\\
Organizzazione: Solitario\\
Tesoro: Triplo\\
\textbf{Descrizione}\\
Signori delle paludi e degli acquitrini più cupi, i draghi neri sono i padroni incontrastati del loro territorio, che dominano con crudeltà e infondendo terrore in chi abita nelle vicinanze. I draghi neri si stanziano nelle zone più remote delle paludi, specie in caverne sul fondo di pozze fetide e buie. Dentro, ammassano i loro luridi tesori e dormono tra radici e fango. I draghi neri amano il cibo un po' marcio e spesso lasciano un pasto a marcire in una pozza per giorni prima di consumarlo. I draghi neri preferiscono tesori che non si decompongono o degradano, accumulando tesori di monete, pietre preziose, gioielli e altri oggetti di pietra o metallo.\\


\medskip\index{Mostri - Drago Rosso Antico}\textbf{Drago Rosso Antico}

\emph{Mastodontica drago, caotico malvagio}

\textbf{FORZA} +10

\textbf{DESTREZZA} +0

\textbf{COSTITUZIONE} +9

\textbf{INTELLIGENZA} +4

\textbf{SAGGEZZA} +2

\textbf{CARISMA} +6

\textbf{Iniziativa} +4 -- \textbf{Difesa} 34

\textbf{Punti Ferita} 546 (28d20 + 252)

\textbf{Movimento} 12 m, scalata 12 m, volo 24 m

\textbf{Tiri Salvezza} Tempra +22, Riflessi +13, Volontà +21

\textbf{Competenze} Muoversi Silenziosamente / Nascondersi +7, Consapevolezza +16

\textbf{Immunità al Danno} fuoco, arma +1

\textbf{Sensi} scurovisione 36 m, vista cieca 18 m

\textbf{Linguaggi} Comune, Draconico

\textbf{Sfida} 24 (62.000 PE)

\emph{\textbf{Resistenza Leggendaria (3/Giorno).}} Se il drago fallisce un Tiro Salvezza, può scegliere invece di riuscire.

\textbf{Azioni}

\emph{\textbf{Multiattacco.}} Il drago può usare la sua Presenza Spaventosa e poi effettuare tre attacchi: uno con il morso e due con gli artigli.

\emph{\textbf{Artiglio.} Attacco con arma da mischia}: +17 a colpire, portata 3 m, un bersaglio.

\emph{Colpisce:} 17 (2d6 + 10) danni taglienti.

\emph{\textbf{Coda.} Attacco con arma da mischia}: +17 a colpire, portata 6 m, un bersaglio.

\emph{Colpisce:} 19 (2d8 + 10) danni da botta.

\emph{\textbf{Morso.} Attacco con arma da mischia}: +17 a colpire, portata 5 metri, un bersaglio.

\emph{Colpisce:} 21 (2d10 + 10) danni perforanti più 14 (4d6) danni da fuoco.

\emph{\textbf{Presenza Spaventosa.}} Ogni creatura scelta dal drago, che si trovi entro 36 metri da esso e consapevole della sua presenza, deve riuscire un Tiro Salvezza di Volontà CD 21 o restare spaventata per 1 minuto. Una creatura può ripetere il Tiro Salvezza al termine di ciascun suo turno, terminando l'effetto se lo riesce. Se il Tiro Salvezza della creatura ha successo o l'effetto ha termine per essa, la creatura è immune alla Presenza Spaventosa del drago per le successive 24 ore.

\emph{\textbf{Soffio Infuocato (Ricarica 5-6).}} Il drago esala fuoco in un cono di 27 metri. Ogni creatura in quell'area deve effettuare un Tiro Salvezza su Riflessi CD 24 e subire 91 (26d6) danni da fuoco se fallisce il Tiro Salvezza, o la metà di questi danni se lo riesce.

\textbf{Azioni Aggiuntive}

Il drago può effettuare 3 Azioni aggiuntive, scelte tra le opzioni seguenti. Può usare solo un'opzione leggendaria alla volta e solo al termine del turno di un'altra creatura. Il drago recupera le Azioni aggiuntive spese all'inizio del proprio turno.

\textbf{Attacco di Ala (Costa 2 Azioni).} Il drago batte le ali. Ogni creatura entro 5 metri dal drago deve riuscire un Tiro Salvezza su Riflessi CD 25 o subire 17 (2d6 + 10) danni da botta e venir gettato prono. Il drago può poi volare fino a metà del suo movimento di volo.

\textbf{Attacco di Coda.} Il drago effettua un attacco di coda.

\textbf{Individuare.} Il drago effettua una prova di Saggezza (Consapevolezza).

\textbf{Ecologia}\\
Ambiente: Montagne calde\\
Organizzazione: Solitario\\
Tesoro: Triplo\\
\textbf{Descrizione}\\
Poche creature sono più crudeli e terribili del possente drago rosso. Sovrano dei cromatici, il terribile drago rosso porta rovina e morte nelle terre minacciate dalla sua presenza.\\
\textbf{Incantesimi}\index{Incantesimi da Drago Rosso}\\
Il drago e' anche capace, se di eta' giovanile o piu' anziano, di lanciare incantesimi.\\
Puo' lanciare un numero di incantesimi pari al suo valore di Carisma, senza aver necessita' di componenti magici.\\
Non deve fare prove di magia, si considera che riesca sempre con un punteggio pari alla Difficoltà + il valore del Carisma.\\
Se deve fare un Tiro per Colpire ha CA = Grado di Sfida con un bonus al colpire pari alla Forza\\
Gli incantesimi a disposizione sono:\\
- Palla di Fuoco\\
- Nube Incendiaria\\
- Muro di fuoco\\


\medskip\index{Mostri - Drago Rosso Adulto}\textbf{Drago Rosso Adulto}

\emph{Enorme drago, caotico malvagio}

\textbf{FORZA} +8

\textbf{DESTREZZA} +0

\textbf{COSTITUZIONE} +7

\textbf{INTELLIGENZA} +3

\textbf{SAGGEZZA} +1

\textbf{CARISMA} +5

\textbf{Iniziativa} +3 -- \textbf{Difesa} 28

\textbf{Punti Ferita} 256 (19d12 + 133) 

\textbf{Movimento} 12 m, scalata 12 m, volo 24 m

\textbf{Tiri Salvezza} Tempra +16, Riflessi +10, Volontà +15

\textbf{Competenze} Muoversi Silenziosamente / Nascondersi +6, Consapevolezza +13

\textbf{Immunità al Danno} fuoco

\textbf{Sensi} scurovisione 36 m, vista cieca 18 m 

\textbf{Linguaggi} Comune, Draconico

\textbf{Sfida} 17 (18.000 PE)

\emph{\textbf{Resistenza Leggendaria (3/Giorno).}} Se il drago fallisce un Tiro Salvezza, può scegliere invece di riuscire.

\textbf{Azioni}

\emph{\textbf{Multiattacco.}} Il drago può usare la sua Presenza Spaventosa e poi effettuare tre attacchi: uno con il morso e due con gli artigli.

\emph{\textbf{Artiglio.} Attacco con arma da mischia}: +14 a colpire, portata 1 m, un bersaglio.

\emph{Colpisce:} 15 (2d6 + 8) danni taglienti.

\emph{\textbf{Coda.} Attacco con arma da mischia}: +14 a colpire, portata 5 metri, un bersaglio.

\emph{Colpisce:} 17 (2d8 + 8) danni da botta.

\emph{\textbf{Morso.} Attacco con arma da mischia}: +14 a colpire, portata 3 m, un bersaglio.

\emph{Colpisce:} 19 (2d10 + 8) danni perforanti più 7 (2d6) danni da
fuoco.

\emph{\textbf{Presenza Spaventosa.}} Ogni creatura scelta dal drago, che si trovi entro 36 metri da esso e consapevole della sua presenza, deve riuscire un Tiro Salvezza di Volontà CD 19 o restare spaventata per 1 minuto. Una creatura può ripetere il Tiro Salvezza al termine di ciascun suo turno, terminando l'effetto se lo riesce. Se il Tiro Salvezza della creatura ha successo o l'effetto ha termine per essa, la creatura è  immune alla Presenza Spaventosa del drago per le successive 24 ore.

\emph{\textbf{Soffio Infuocato (Ricarica 5-6).}} Il drago esala fuoco in un cono di 18 metri. Ogni creatura in quell'area deve effettuare un Tiro Salvezza su Riflessi CD 21 e subire 63 (18d6) danni da fuoco se fallisce il Tiro Salvezza, o la metà di questi danni se lo riesce.

\textbf{Azioni Aggiuntive}

Il drago può effettuare 3 Azioni aggiuntive, scelte tra le opzioni seguenti. Può usare solo un'opzione leggendaria alla volta e solo al termine del turno di un'altra creatura. Il drago recupera le Azioni aggiuntive spese all'inizio del proprio turno.

\textbf{Attacco di Ala (Costa 2 Azioni).} Il drago batte le ali. Ogni creatura entro 3 metri dal drago deve riuscire un Tiro Salvezza su Riflessi CD 22 o subire 15 (2d6 + 8) danni da botta e venir gettato prono. Il drago può poi volare fino a metà del suo movimento di volo. 

\textbf{Attacco di Coda.} Il drago effettua un attacco di coda. 

\textbf{Individuare.} Il drago effettua una prova di Saggezza (Consapevolezza).

\textbf{Ecologia}\\
Ambiente: Montagne calde\\
Organizzazione: Solitario\\
Tesoro: Triplo\\
\textbf{Descrizione}\\
Poche creature sono più crudeli e terribili del possente drago rosso. Sovrano dei cromatici, il terribile drago rosso porta rovina e morte nelle terre minacciate dalla sua presenza.\\
\textbf{Incantesimi}\index{Incantesimi da Drago Rosso}\\
Il drago e' anche capace, se di eta' giovanile o piu' anziano, di lanciare incantesimi.\\
Puo' lanciare un numero di incantesimi pari al suo valore di Carisma, senza aver necessita' di componenti magici.\\
Non deve fare prove di magia, si considera che riesca sempre con un punteggio pari alla Difficoltà + il valore del Carisma.\\
Se deve fare un Tiro per Colpire ha CA = Grado di Sfida con un bonus al colpire pari alla Forza\\
Gli incantesimi a disposizione sono:\\
- Palla di Fuoco\\
- Nube Incendiaria\\
- Muro di fuoco\\


\medskip\index{Mostri - Drago Rosso Giovane}\textbf{Drago Rosso Giovane}

\emph{Grande drago, caotico malvagio}

\textbf{FORZA} +6

\textbf{DESTREZZA} +0

\textbf{COSTITUZIONE} +5

\textbf{INTELLIGENZA} +2

\textbf{SAGGEZZA} +0

\textbf{CARISMA} +4

\textbf{Iniziativa} +2 -- \textbf{Difesa} 23

\textbf{Punti Ferita} 178 (17d10 + 85)

\textbf{Movimento} 12 m, scalata 12 m, volo 24 m

\textbf{Tiri Salvezza} Tempra +11, Riflessi +8, Volontà +10

\textbf{Competenze} Muoversi Silenziosamente / Nascondersi +4, Consapevolezza +8

\textbf{Immunità al Danno} fuoco

\textbf{Sensi} scurovisione 36 m, vista cieca 9 m

\textbf{Linguaggi} Comune, Draconico

\textbf{Sfida} 10 (5.900 PE)

\textbf{Azioni}

\emph{\textbf{Multiattacco.}} Il drago può effettuare tre attacchi: uno con il morso e due con gli artigli.

\emph{\textbf{Artiglio.} Attacco con arma da mischia}: +10 a colpire, portata 1 m, un bersaglio.

\emph{Colpisce:} 13 (2d6 + 6) danni taglienti.

\emph{\textbf{Morso.} Attacco con arma da mischia}: +10 a colpire, portata 3 m, un bersaglio.

\emph{Colpisce:} 17 (2d10 + 6) danni perforanti più 3 (1d6) danni da fuoco.

\emph{\textbf{Soffio Infuocato (Ricarica 5-6).}} Il drago esala fuoco in un cono di 9 metri. Ogni creatura in quell'area deve effettuare un Tiro Salvezza su Riflessi CD 17 e subire 56 (16d6) danni da fuoco se fallisce il Tiro Salvezza, o la metà di questi danni se lo riesce.

\textbf{Ecologia}\\
Ambiente: Montagne calde\\
Organizzazione: Solitario\\
Tesoro: Triplo\\
\textbf{Descrizione}\\
Poche creature sono più crudeli e terribili del possente drago rosso. Sovrano dei cromatici, il terribile drago rosso porta rovina e morte nelle terre minacciate dalla sua presenza.\\
\textbf{Incantesimi}\index{Incantesimi da Drago Rosso}\\
Il drago e' anche capace, se di eta' giovanile o piu' anziano, di lanciare incantesimi.\\
Puo' lanciare un numero di incantesimi pari al suo valore di Carisma, senza aver necessita' di componenti magici.\\
Non deve fare prove di magia, si considera che riesca sempre con un punteggio pari alla Difficoltà + il valore del Carisma.\\
Se deve fare un Tiro per Colpire ha CA = Grado di Sfida con un bonus al colpire pari alla Forza\\
Gli incantesimi a disposizione sono:\\
- Palla di Fuoco\\
- Nube Incendiaria\\
- Muro di fuoco\\


\medskip\index{Mostri - Drago Rosso Cucciolo}\textbf{Drago Rosso Cucciolo}

\emph{Media drago, caotico malvagio}

\textbf{FORZA} +4

\textbf{DESTREZZA} +0

\textbf{COSTITUZIONE} +3

\textbf{INTELLIGENZA} +1

\textbf{SAGGEZZA} +0

\textbf{CARISMA} +2

\textbf{Iniziativa} +1 -- \textbf{Difesa} 19

\textbf{Punti Ferita} 75 (10d8 + 30)

\textbf{Movimento} 9 m, scalata 9 m, volo 18 m

\textbf{Tiri Salvezza} Tempra +4, Riflessi +3, Volontà +1

\textbf{Competenze} Muoversi Silenziosamente / Nascondersi +2, Consapevolezza +4

\textbf{Immunità al Danno} fuoco

\textbf{Sensi} scurovisione 18 m, vista cieca 3 m

\textbf{Linguaggi} Draconico

\textbf{Sfida} 4 (1.100 PE)

\textbf{Azioni}

\emph{\textbf{Morso.} Attacco con arma da mischia}: +6 a colpire, portata 1 m, un bersaglio.

\emph{Colpisce:} 9 (1d10 + 4) danni perforanti più 3 (1d6) danni da fuoco.

\emph{\textbf{Soffio Infuocato (Ricarica 5-6).}} Il drago esala fuoco in un cono di 5 metri. Ogni creatura in quell'area deve effettuare un Tiro Salvezza di Riflessi CD 13 e subire 24 (7d6) danni da fuoco se fallisce il Tiro Salvezza, o la metà di questi danni se lo riesce.

\textbf{Ecologia}\\
Ambiente: Montagne calde\\
Organizzazione: Solitario\\
Tesoro: Triplo\\
\textbf{Descrizione}\\
Poche creature sono più crudeli e terribili del possente drago rosso. Sovrano dei cromatici, il terribile drago rosso porta rovina e morte nelle terre minacciate dalla sua presenza.\\


\medskip\index{Mostri - Drago Verde Antico}\textbf{Drago Verde Antico}

\emph{Mastodontica drago, legale malvagio}

\textbf{FORZA} +8

\textbf{DESTREZZA} +1

\textbf{COSTITUZIONE} +7

\textbf{INTELLIGENZA} +5

\textbf{SAGGEZZA} +3

\textbf{CARISMA} +4

\textbf{Iniziativa} +5 -- \textbf{Difesa} 32

\textbf{Punti Ferita} 385 (22d20 + 154) 

\textbf{Movimento} 12 m, nuoto 12 m, volo 24 m

\textbf{Tiri Salvezza} Tempra +20, Riflessi +12, Volontà +20

\textbf{Competenze} Muoversi Silenziosamente / Nascondersi +8, Ingannare +11, Percepire Emozioni +10, Consapevolezza + 15

\textbf{Immunità al Danno} veleno , arma +1

\textbf{Immunità alle Condizioni}
avvelenato

\textbf{Sensi} scurovisione 36 m, vista cieca 18 m

\textbf{Linguaggi} Comune, Draconico 

\textbf{Sfida} 22 (41.000 PE)

\emph{\textbf{Anfibio.}} Il drago può respirare aria e acqua.

\emph{\textbf{Resistenza Leggendaria (3/Giorno).}} Se il drago fallisce un Tiro Salvezza, può scegliere invece di riuscire.

\textbf{Azioni}

\emph{\textbf{Multiattacco.}} Il drago può usare la sua Presenza Spaventosa. Poi effettuare tre attacchi: uno con il morso e due con gli artigli.

\emph{\textbf{Artiglio.} Attacco con arma da mischia}: +15 a colpire, portata 3 m, un bersaglio.

\emph{Colpisce:} 15 (2d6 + 8) danni taglienti.

\emph{\textbf{Coda.} Attacco con arma da mischia}: +15 a colpire, portata 6 m, un bersaglio.

\emph{Colpisce:} 17 (2d8 + 8) danni da botta.

\emph{\textbf{Morso.} Attacco con arma da mischia}: +15 a colpire, portata 5 metri, un bersaglio.

\emph{Colpisce:} 19 (2d10 + 8) danni perforanti più 10 (3d6) danni da veleno.

\emph{\textbf{Presenza Spaventosa.}} Ogni creatura scelta dal drago, che si trovi entro 36 metri da esso e consapevole della sua presenza, deve riuscire un Tiro Salvezza di Volontà CD 19 o restare spaventata per 1 minuto. Una creatura può ripetere il Tiro Salvezza al termine di ciascun suo turno, terminando l'effetto se lo riesce. Se il Tiro Salvezza della creatura ha successo o l'effetto ha termine per essa, la creatura è immune alla Presenza Spaventosa del drago per le successive 24 ore.

\emph{\textbf{Soffio Velenoso (Ricarica 5-6).}} Il drago esala gas velenosi in un cono di 27 metri. Ogni creatura in quell'area deve effettuare un Tiro Salvezza di Tempra CD 22 e subire 77 (22d6) danni da veleno se fallisce il Tiro Salvezza, o la metà di questi danni se lo riesce.

\textbf{Azioni Aggiuntive}

Il drago può effettuare 3 Azioni aggiuntive, scelte tra le opzioni seguenti. Può usare solo un'opzione leggendaria alla volta e solo al termine del turno di un'altra creatura. Il drago recupera le Azioni aggiuntive spese all'inizio del proprio turno.

\textbf{Attacco di Ala (Costa 2 Azioni).} Il drago batte le ali. Ogni creatura entro 5 metri dal drago deve riuscire un Tiro Salvezza su Riflessi CD 23 o subire 15 (2d6 + 8) danni da botta e venire gettato prono. Il drago può poi volare fino a metà del suo movimento di volo.

\textbf{Attacco di Coda.} Il drago effettua un attacco di coda.

\textbf{Individuare.} Il drago effettua una prova di Saggezza (Consapevolezza).

\textbf{Ecologia}\\
Ambiente: Foreste Temperate\\
Organizzazione: Solitario\\
Tesoro: Triplo\\
\textbf{Descrizione}\\
I draghi verdi vivono nelle antiche foreste del mondo, vagando in cerca di preda sotto giganteschi tetti di foglie. Di tutti i draghi cromatici, i draghi verdi sono forse quelli con cui ci si può più facilmente accordare diplomaticamente.\\
\textbf{Incantesimi}\index{Incantesimi da Drago Verde}\\
Il drago e' anche capace, se di eta' giovanile o piu' anziano, di lanciare incantesimi.\\
Puo' lanciare un numero di incantesimi pari al suo valore di Carisma, senza aver necessita' di componenti magici.\\
Non deve fare prove di magia, si considera che riesca sempre con un punteggio pari alla Difficoltà + il valore del Carisma.\\
Se deve fare un Tiro per Colpire ha CA = Grado di Sfida con un bonus al colpire pari alla Forza.\\
Gli incantesimi a disposizione sono:\\
- Nube Mortale\\
- Terreno illusorio\\
- Rimuovi veleno\\


\medskip\index{Mostri - Drago Verde Adulto}\textbf{Drago Verde Adulto}

\emph{Enorme drago, legale malvagio}

\textbf{FORZA} +6

\textbf{DESTREZZA} +1

\textbf{COSTITUZIONE} +5

\textbf{INTELLIGENZA} +4

\textbf{SAGGEZZA} +2

\textbf{CARISMA} +3

\textbf{Iniziativa} +4 -- \textbf{Difesa} 27

\textbf{Punti Ferita} 207 (18d12 + 90)

\textbf{Movimento} 12 m, nuoto 12 m, volo 24 m

\textbf{Tiri Salvezza} Tempra +14, Riflessi +9, Volontà +14

\textbf{Competenze} Muoversi Silenziosamente / Nascondersi +6, Ingannare +8, Percepire Emozioni +7, Consapevolezza +12

\textbf{Immunità al Danno} veleno

\textbf{Immunità alle Condizioni} avvelenato

\textbf{Sensi} scurovisione 36 m, vista cieca 18 m

\textbf{Linguaggi} Comune, Draconico

\textbf{Sfida} 15 (13.000 PE)

\emph{\textbf{Anfibio.}} Il drago può respirare aria e acqua.

\emph{\textbf{Resistenza Leggendaria (3/Giorno).}} Se il drago fallisce un Tiro Salvezza, può scegliere invece di riuscire.

\textbf{Azioni}

\emph{\textbf{Multiattacco.}} Il drago può usare la sua Presenza Spaventosa. Poi effettuare tre attacchi: uno con il morso e due con gli artigli.

\emph{\textbf{Artiglio.} Attacco con arma da mischia}: +11 a colpire, portata 1 m, un bersaglio.

\emph{Colpisce:} 13 (2d6 + 6) danni taglienti.

\emph{\textbf{Coda.} Attacco con arma da mischia}: +11 a colpire, portata 5 metri, un bersaglio.

\emph{Colpisce:} 15 (2d8 + 6) danni da botta.

\emph{\textbf{Morso.} Attacco con arma da mischia}: +11 a colpire, portata 3 m, un bersaglio.

\emph{Colpisce:} 17 (2d10 + 6) danni perforanti più 7 (2d6) danni da veleno.

\emph{\textbf{Presenza Spaventosa.}} Ogni creatura scelta dal drago, che si trovi entro 36 metri da esso e consapevole della sua presenza, deve riuscire un Tiro Salvezza di Volontà CD 16 o restare spaventata per 1 minuto. Una creatura può ripetere il Tiro Salvezza al termine di ciascun suo turno, terminando l'effetto se lo riesce. Se il Tiro Salvezza della creatura ha successo o l'effetto ha termine per essa, la creatura è immune alla Presenza Spaventosa del drago per le successive 24 ore.

\emph{\textbf{Soffio Velenoso (Ricarica 5-6).}} Il drago esala gas velenosi in un cono di 18 metri. Ogni creatura in quell'area deve effettuare un Tiro Salvezza di Tempra CD 18 e subire 56 (16d6) danni da veleno se fallisce il Tiro Salvezza, o la metà di questi danni se lo riesce.

\textbf{Azioni Aggiuntive}

Il drago può effettuare 3 Azioni aggiuntive, scelte tra le opzioni seguenti. Può usare solo un'opzione leggendaria alla volta e solo al termine del turno di un'altra creatura. Il drago recupera le Azioni aggiuntive spese all'inizio del proprio turno.

\textbf{Attacco di Ala (Costa 2 Azioni).} Il drago batte le ali. Ogni creatura entro 3 metri dal drago deve riuscire un Tiro Salvezza su Riflessi CD 19 o subire 13 (2d6 + 6) danni da botta e venir  gettato prono. Il drago può poi volare fino a metà del suo movimento di volo.

\textbf{Attacco di Coda.} Il drago effettua un attacco di coda.

\textbf{Individuare.} Il drago effettua una prova di Saggezza (Consapevolezza).

\textbf{Ecologia}\\
Ambiente: Foreste Temperate\\
Organizzazione: Solitario\\
Tesoro: Triplo\\
\textbf{Descrizione}\\
I draghi verdi vivono nelle antiche foreste del mondo, vagando in cerca di preda sotto giganteschi tetti di foglie. Di tutti i draghi cromatici, i draghi verdi sono forse quelli con cui ci si può più facilmente accordare diplomaticamente.\\
\textbf{Incantesimi}\index{Incantesimi da Drago Verde}\\
Il drago e' anche capace, se di eta' giovanile o piu' anziano, di lanciare incantesimi.\\
Puo' lanciare un numero di incantesimi pari al suo valore di Carisma, senza aver necessita' di componenti magici.\\
Non deve fare prove di magia, si considera che riesca sempre con un punteggio pari alla Difficoltà + il valore del Carisma.\\
Se deve fare un Tiro per Colpire ha CA = Grado di Sfida con un bonus al colpire pari alla Forza\\
Gli incantesimi a disposizione sono:\\
- Nube Mortale\\
- Terreno illusorio\\
- Rimuovi veleno\\


\medskip\index{Mostri - Drago Verde Giovane}\textbf{Drago Verde Giovane}

\emph{Grande drago, legale malvagio}

\textbf{FORZA} +4

\textbf{DESTREZZA} +1

\textbf{COSTITUZIONE} +3

\textbf{INTELLIGENZA} +3

\textbf{SAGGEZZA} +1

\textbf{CARISMA} +2

\textbf{Iniziativa} +3 -- \textbf{Difesa} 22

\textbf{Punti Ferita} 136 (16d10 + 48)

\textbf{Movimento} 12 m, nuoto 12 m, volo 24 m

\textbf{Tiri Salvezza} Tempra +9, Riflessi +7, Volontà +9

\textbf{Competenze} Muoversi Silenziosamente / Nascondersi +4, Ingannare +5, Consapevolezza +7

\textbf{Immunità al Danno} veleno

\textbf{Immunità alle Condizioni} avvelenato

\textbf{Sensi} scurovisione 36 m, vista cieca 9 m
\textbf{Linguaggi} Comune, Draconico

\textbf{Sfida} 8 (3.900 PE)

\emph{\textbf{Anfibio.}} Il drago può respirare aria e acqua.

\textbf{Azioni}

\emph{\textbf{Multiattacco.}} Il drago può effettuare tre attacchi: uno con il morso e due con gli artigli. 

\emph{\textbf{Artiglio.} Attacco con arma da mischia}: +7 a colpire, portata 1 m, un bersaglio.

\emph{Colpisce:} 11 (2d6 + 4) danni taglienti.

\emph{\textbf{Morso.} Attacco con arma da mischia}: +7 a colpire, portata 3 m, un bersaglio.

\emph{Colpisce:} 15 (2d10 + 4) danni perforanti più 7 (2d6) danni da veleno.

\emph{\textbf{Soffio Velenoso (Ricarica 5-6).}} Il drago esala gas velenosi in un cono di 9 metri. Ogni creatura in quell'area deve effettuare un Tiro Salvezza di Tempra CD 14 e subire 42 (12d6) danni da veleno se fallisce il Tiro Salvezza, o la metà di questi danni se lo riesce.

\textbf{Ecologia}\\
Ambiente: Foreste Temperate\\
Organizzazione: Solitario\\
Tesoro: Triplo\\
\textbf{Descrizione}\\
I draghi verdi vivono nelle antiche foreste del mondo, vagando in cerca di preda sotto giganteschi tetti di foglie. Di tutti i draghi cromatici, i draghi verdi sono forse quelli con cui ci si può più facilmente accordare diplomaticamente.\\
\textbf{Incantesimi}\index{Incantesimi da Drago Verde}\\
Il drago e' anche capace, se di eta' giovanile o piu' anziano, di lanciare incantesimi.\\
Puo' lanciare un numero di incantesimi pari al suo valore di Carisma, senza aver necessita' di componenti magici.\\
Non deve fare prove di magia, si considera che riesca sempre con un punteggio pari alla Difficoltà + il valore del Carisma.\\
Se deve fare un Tiro per Colpire ha CA = Grado di Sfida con un bonus al colpire pari alla Forza\\
Gli incantesimi a disposizione sono:\\
- Nube Mortale\\
- Terreno illusorio\\
- Rimuovi veleno\\

\medskip\index{Mostri - Drago Verde Cucciolo}\textbf{Drago Verde Cucciolo}

\emph{Media drago, legale malvagio}

\textbf{FORZA} +2

\textbf{DESTREZZA} +1

\textbf{COSTITUZIONE} +1

\textbf{INTELLIGENZA} +2

\textbf{SAGGEZZA} +0

\textbf{CARISMA} +1

\textbf{Iniziativa} +2 -- \textbf{Difesa} 18

\textbf{Punti Ferita} 38 (7d8 + 7)

\textbf{Movimento} 9 m, nuoto 9 m, volo 18 m

\textbf{Tiri Salvezza} Tempra +3, Riflessi +1, Volontà +0

\textbf{Competenze} Muoversi Silenziosamente / Nascondersi +3, Consapevolezza +4 

\textbf{Immunità al Danno} veleno 

\textbf{Immunità alle Condizioni} avvelenato

\textbf{Sensi} scurovisione 18 m, vista cieca 3 m

\textbf{Linguaggi} Draconico

\textbf{Sfida} 2 (450 PE)

\emph{\textbf{Anfibio.}} Il drago può respirare aria e acqua.

\textbf{Azioni}

\emph{\textbf{Morso.} Attacco con arma da mischia}: +4 a colpire, portata 1 m, un bersaglio.

\emph{Colpisce:} 7 (1d10 + 2) danni perforanti più 3 (1d6) danni da veleno.

\emph{\textbf{Soffio Velenoso (Ricarica 5-6).}} Il drago esala gas velenosi in un cono di 5 metri. Ogni creatura in quell'area deve effettuare un Tiro Salvezza di Tempra CD 11 e subire 21 (6d6) danni da veleno se fallisce il Tiro Salvezza, o la metà di questi danni se lo riesce.

\textbf{Ecologia}\\
Ambiente: Foreste Temperate\\
Organizzazione: Solitario\\
Tesoro: Triplo\\
\textbf{Descrizione}\\
I draghi verdi vivono nelle antiche foreste del mondo, vagando in cerca di preda sotto giganteschi tetti di foglie. Di tutti i draghi cromatici, i draghi verdi sono forse quelli con cui ci si può più facilmente accordare diplomaticamente.\\


\subsection{Draghi Metallici}

\medskip\index{Mostri - Drago d'Argento Antico}\textbf{Drago d'Argento Antico}

\emph{Mastodontica drago, legale buono}

\textbf{FORZA} +10

\textbf{DESTREZZA} +0

\textbf{COSTITUZIONE} +9

\textbf{INTELLIGENZA} +4

\textbf{SAGGEZZA} +2

\textbf{CARISMA} +6

\textbf{Iniziativa} +4 -- \textbf{Difesa} 34

\textbf{Punti Ferita} 487 (25d20 + 225)

\textbf{Movimento} 12 m, volo 24 m

\textbf{Tiri Salvezza} Tempra +21, Riflessi +15, Volontà +23

\textbf{Competenze} Arcano +11, Muoversi Silenziosamente / Nascondersi +7, Consapevolezza +16, Storia +11

\textbf{Immunità al Danno} freddo, arma +1

\textbf{Sensi} scurovisione 36 m, vista cieca 18 m

\textbf{Linguaggi} Comune, Draconico

\textbf{Sfida} 23 (50.000 PE)

\emph{\textbf{Resistenza Leggendaria (3/Giorno).}} Se il drago fallisce un Tiro Salvezza, può scegliere invece di riuscire.

\textbf{Azioni}

\emph{\textbf{Multiattacco.}} Il drago può usare la sua Presenza Spaventosa. Poi effettuare tre attacchi: uno con il morso e due con gli
artigli.

\emph{\textbf{Artiglio.} Attacco con arma da mischia}: +17 a colpire, portata 3 m, un bersaglio.

\emph{Colpisce:} 17 (2d6 + 10) danni taglienti.

\emph{\textbf{Coda.} Attacco con arma da mischia}: +17 a colpire, portata 6 m, un bersaglio.

\emph{Colpisce:} 19 (2d8 + 10) danni da botta.

\emph{\textbf{Morso.} Attacco con arma da mischia}: +17 a colpire, portata 5 metri, un bersaglio.

\emph{Colpisce:} 21 (2d10 + 10) danni perforanti.

\emph{\textbf{Presenza Spaventosa.}} Ogni creatura scelta dal drago, che si trovi entro 36 metri da esso e consapevole della sua presenza, deve riuscire un Tiro Salvezza di Volontà CD 21 o restare spaventata per 1 minuto. Una creatura può ripetere il Tiro Salvezza al termine di ciascun suo turno, terminando l'effetto se lo riesce. Se il Tiro Salvezza della creatura ha successo o l'effetto ha termine per essa, la creatura è immune alla Presenza Spaventosa del drago per le successive 24 ore.

\emph{\textbf{Arma a Soffio (Ricarica 5-6).}} Il drago usa una delle seguenti armi a soffio:

\emph{Soffio Gelido.} Il drago esala un'esplosione ghiacciata in un cono di 27 metri. Ogni creatura nell'area deve effettuare un Tiro Salvezza su Tempra CD 24, subendo 67 (15d8) danni da freddo se fallisce il Tiro Salvezza, o la metà di questi danni se lo riesce.

\emph{Soffio Paralizzante.} Il drago esala un gas paralizzante in un cono di 24 metri. Ogni creatura nell'area deve riuscire un Tiro Salvezza su Tempra 24 o restare paralizzata per 1 minuto. Una creatura può ripetere il Tiro Salvezza al termine di ciascun suo turno, terminando l'effetto per sé in caso di successo.

\emph{\textbf{Mutare Forma.}} Il drago può trasformarsi magicamente in un umanoide o bestia il cui grado di sfida sia pari o inferiore al proprio, o tornare alla sua vera forma. Alla morte ritorna alla sua vera forma. 

Qualsiasi equipaggiamento stia indossando o trasportando viene assorbito o trasportato nella nuova forma (a scelta del drago).

Nella nuova forma, il drago mantiene il suo allineamento, punti ferita, Dadi Vita, la facoltà di parlare, le competenze, la Resistenza Leggendaria, le azioni da tana, e i punteggi di Intelligenza, Saggezza  e Carisma, oltre a questa azione. Le sue statistiche e capacità vengono altrimenti rimpiazzate da quelle della nuova forma, eccetto Azioni aggiuntive della nuova forma.

\textbf{Azioni Aggiuntive}

Il drago può effettuare 3 Azioni aggiuntive, scelte tra le opzioni seguenti. Può usare solo un'opzione leggendaria alla volta e solo al termine del turno di un'altra creatura. Il drago recupera le  Azioni aggiuntive spese all'inizio del proprio turno.

\textbf{Attacco di Ala (Costa 2 Azioni).} Il drago batte le ali. Ogni  creatura entro 5 metri dal drago deve riuscire un Tiro Salvezza  su Riflessi CD 25 o subire 17 (2d6 + 10) danni da botta e  venir gettato prono. Il drago può poi volare fino a metà della sua  velocità di volo.

\textbf{Attacco di Coda.} Il drago effettua un attacco di coda.

\textbf{Individuare.} Il drago effettua una prova di Saggezza (Consapevolezza).

\textbf{Ecologia}\\
Ambiente: Montagne Temperate\\
Organizzazione: Solitario\\
Tesoro: Triplo\\
\textbf{Descrizione}\\
Tra tutti i draghi, quelli d'argento sono i più coraggiosi, e si attengono ad un codice cavalleresco che impone loro di aiutare i deboli, sconfiggere il male e comportarsi in modo onorevole.\\
\textbf{Incantesimi}\index{Incantesimi da Drago Argento}\\
Il drago e' anche capace, se di eta' giovanile o piu' anziano, di lanciare incantesimi.\\
Puo' lanciare un numero di incantesimi pari al suo valore di Carisma, senza aver necessita' di componenti magici.\\
Non deve fare prove di magia, si considera che riesca sempre con un punteggio pari alla Difficoltà + il valore del Carisma.\\
Se deve fare un Tiro per Colpire ha CA = Grado di Sfida con un bonus al colpire pari alla Forza\\
Gli incantesimi a disposizione sono:\\
- Cono di freddo\\
- Tempesta di ghiaccio\\
- Creazione\\
- Capanna\\

\medskip\index{Mostri - Drago d'Argento Adulto}\textbf{Drago d'Argento Adulto}

\emph{Enorme drago, legale buono}

\textbf{FORZA} +8

\textbf{DESTREZZA} +0

\textbf{COSTITUZIONE} +7

\textbf{INTELLIGENZA} +3

\textbf{SAGGEZZA} +1

\textbf{CARISMA} +5

\textbf{Iniziativa} +3 -- \textbf{Difesa} 27

\textbf{Punti Ferita} 243 (18d12 + 126)

\textbf{Movimento} 12 m, volo 24 m

\textbf{Tiri Salvezza} Tempra +15, Riflessi +12, Volontà +17

\textbf{Competenze} Arcano +8, Muoversi Silenziosamente / Nascondersi +5, Consapevolezza +11, Storia +8

\textbf{Immunità al Danno} freddo

\textbf{Sensi} scurovisione 36 m, vista cieca 18 m 

\textbf{Linguaggi} Comune, Draconico

\textbf{Sfida} 16 (15.000 PE)

\emph{\textbf{Resistenza Leggendaria (3/Giorno).}} Se il drago fallisce un Tiro Salvezza, può scegliere invece di riuscire.

\textbf{Azioni}

\emph{\textbf{Multiattacco.}} Il drago può usare la sua Presenza Spaventosa. Poi effettuare tre attacchi: uno con il morso e due con gli artigli.

\emph{\textbf{Artiglio.} Attacco con arma da mischia}: +13 a colpire, portata 1 m, un bersaglio.

\emph{Colpisce:} 15 (2d6 + 8) danni taglienti.

\emph{\textbf{Coda.} Attacco con arma da mischia}: +13 a colpire, portata 5 metri, un bersaglio.

\emph{Colpisce:} 17 (2d8 + 8) danni da botta.

\emph{\textbf{Morso.} Attacco con arma da mischia}: +13 a colpire, portata 3 m, un bersaglio.

\emph{Colpisce:} 19 (2d10 + 8) danni perforanti.

\emph{\textbf{Presenza Spaventosa.}} Ogni creatura scelta dal drago, che si trovi entro 36 metri da esso e consapevole della sua presenza, deve riuscire un Tiro Salvezza di Volontà CD 18 o restare spaventata per 1 minuto. Una creatura può ripetere il Tiro Salvezza al termine di ciascun suo turno, terminando l'effetto se lo riesce. Se il Tiro Salvezza della creatura ha successo o l'effetto ha termine per essa, la creatura è immune alla Presenza Spaventosa del drago per le successive 24 ore.

\emph{\textbf{Arma a Soffio (Ricarica 5-6).}} Il drago usa una delle seguenti armi a soffio:

\emph{Soffio Gelido.} Il drago esala un'esplosione ghiacciata in un cono di 18 metri. Ogni creatura nell'area deve effettuare un Tiro Salvezza su Tempra CD 20, subendo 58 (13d8) danni da freddo se fallisce il Tiro Salvezza, o la metà di questi danni se lo riesce.

\emph{Soffio Paralizzante.} Il drago esala un gas paralizzante in un cono di 18 metri. Ogni creatura nell'area deve riuscire un Tiro Salvezza su Tempra 20 o restare paralizzata per 1 minuto. Una creatura può ripetere il Tiro Salvezza al termine di ciascun suo turno, terminando l'effetto per sé in caso di successo.

\emph{\textbf{Mutare Forma.}} Il drago può trasformarsi magicamente in un umanoide o bestia il cui grado di sfida sia pari o inferiore al proprio, o tornare alla sua vera forma. Alla morte ritorna alla sua vera forma. Qualsiasi equipaggiamento stia indossando o trasportando viene assorbito o trasportato nella nuova forma (a scelta del drago).

Nella nuova forma, il drago mantiene il suo allineamento, punti ferita, Dadi Vita, la facoltà di parlare, le competenze, la Resistenza Leggendaria, le azioni da tana, e i punteggi di Intelligenza, Saggezza e Carisma, oltre a questa azione. Le sue statistiche e capacità vengono altrimenti rimpiazzate da quelle della nuova forma, eccetto Azioni aggiuntive della nuova forma.

\textbf{Azioni Aggiuntive}

Il drago può effettuare 3 Azioni aggiuntive, scelte tra le opzioni seguenti. Può usare solo un'opzione leggendaria alla volta e solo al termine del turno di un'altra creatura. Il drago recupera le Azioni aggiuntive spese all'inizio del proprio turno.

\textbf{Attacco di Ala (Costa 2 Azioni).} Il drago batte le ali. Ogni creatura entro 3 metri dal drago deve riuscire un Tiro Salvezza su Riflessi CD 21 o subire 15 (2d6 + 8) danni da botta e venir gettato prono. Il drago può poi volare fino a metà del suo movimento di volo.

\textbf{Attacco di Coda.} Il drago effettua un attacco di coda.

\textbf{Individuare.} Il drago effettua una prova di Saggezza (Consapevolezza).

\textbf{Ecologia}\\
Ambiente: Montagne Temperate\\
Organizzazione: Solitario\\
Tesoro: Triplo\\
\textbf{Descrizione}\\
Tra tutti i draghi, quelli d'argento sono i più coraggiosi, e si attengono ad un codice cavalleresco che impone loro di aiutare i deboli, sconfiggere il male e comportarsi in modo onorevole.\\
\textbf{Incantesimi}\index{Incantesimi da Drago Argento}\\
Il drago e' anche capace, se di eta' giovanile o piu' anziano, di lanciare incantesimi.\\
Puo' lanciare un numero di incantesimi pari al suo valore di Carisma, senza aver necessita' di componenti magici.\\
Non deve fare prove di magia, si considera che riesca sempre con un punteggio pari alla Difficoltà + il valore del Carisma.\\
Se deve fare un Tiro per Colpire ha CA = Grado di Sfida con un bonus al colpire pari alla Forza\\
Gli incantesimi a disposizione sono:\\
- Cono di freddo\\
- Tempesta di ghiaccio\\
- Creazione\\
- Capanna\\

\medskip\index{Mostri - Drago d'Argento Giovane}\textbf{Drago d'Argento Giovane}

\emph{Grande drago, legale buono}

\textbf{FORZA} +6

\textbf{DESTREZZA} +0

\textbf{COSTITUZIONE} +5

\textbf{INTELLIGENZA} +2

\textbf{SAGGEZZA} +0

\textbf{CARISMA} +4

\textbf{Iniziativa} +2 -- \textbf{Difesa} 23

\textbf{Punti Ferita} 168 (16d10 + 80)

\textbf{Movimento} 12 m, volo 24 m

\textbf{Tiri Salvezza} Tempra +10, Riflessi +8, Volontà +12

\textbf{Competenze} Arcano +6, Muoversi Silenziosamente / Nascondersi +4, Consapevolezza +8, Storia +6

\textbf{Immunità al Danno} freddo

\textbf{Sensi} scurovisione 36 m, vista cieca 9 m

\textbf{Linguaggi} Comune, Draconico

\textbf{Sfida} 9 (5.000 PE)

\textbf{Azioni}

\emph{\textbf{Multiattacco.}} Il drago può effettuare tre attacchi: uno con il morso e due con gli artigli.

\emph{\textbf{Artiglio.} Attacco con arma da mischia}: +10 a colpire, portata 1 m, un bersaglio.

\emph{Colpisce:} 13 (2d6 + 6) danni taglienti.

\emph{\textbf{Morso.} Attacco con arma da mischia}: +10 a colpire, portata 3 m, un bersaglio.

\emph{Colpisce:} 17 (2d10 + 6) danni perforanti.

\emph{\textbf{Arma a Soffio (Ricarica 5-6).}} Il drago usa una delle seguenti armi a soffio:

\emph{Soffio Gelido.} Il drago esala un'esplosione ghiacciata in un cono di 9 metri. Ogni creatura nell'area deve effettuare un Tiro Salvezza su Tempra CD 17, subendo 54 (12d8) danni da freddo se fallisce il Tiro Salvezza, o la metà di questi danni se lo riesce.

\emph{Soffio Paralizzante.} Il drago esala un gas paralizzante in un cono di 9 metri. Ogni creatura nell'area deve riuscire un Tiro Salvezza su Tempra 17 o restare paralizzata per 1 minuto. Una creatura può ripetere il Tiro Salvezza al termine di ciascun suo turno, terminando l'effetto per sé in caso di successo.

\textbf{Ecologia}\\
Ambiente: Montagne Temperate\\
Organizzazione: Solitario\\
Tesoro: Triplo\\
\textbf{Descrizione}\\
Tra tutti i draghi, quelli d'argento sono i più coraggiosi, e si attengono ad un codice cavalleresco che impone loro di aiutare i deboli, sconfiggere il male e comportarsi in modo onorevole.\\
\textbf{Incantesimi}\index{Incantesimi da Drago Argento}\\
Il drago e' anche capace, se di eta' giovanile o piu' anziano, di lanciare incantesimi.\\
Puo' lanciare un numero di incantesimi pari al suo valore di Carisma, senza aver necessita' di componenti magici.\\
Non deve fare prove di magia, si considera che riesca sempre con un punteggio pari alla Difficoltà + il valore del Carisma.\\
Se deve fare un Tiro per Colpire ha CA = Grado di Sfida con un bonus al colpire pari alla Forza\\
Gli incantesimi a disposizione sono:\\
- Cono di freddo\\
- Tempesta di ghiaccio\\
- Creazione\\
- Capanna\\

\medskip\index{Mostri - Drago d'Argento Cucciolo}\textbf{Drago d'Argento Cucciolo}

\emph{Media drago, legale buono}

\textbf{FORZA} +4

\textbf{DESTREZZA} +0

\textbf{COSTITUZIONE} +3

\textbf{INTELLIGENZA} +1

\textbf{SAGGEZZA} +0

\textbf{CARISMA} +2

\textbf{Iniziativa} +1 -- \textbf{Difesa} 18

\textbf{Punti Ferita} 45 (6d8 + 18)

\textbf{Movimento} 9 m, volo 18 m

\textbf{Tiri Salvezza} Tempra +3, Riflessi +3, Volontà +2

\textbf{Competenze} Muoversi Silenziosamente / Nascondersi +2, Consapevolezza +4

\textbf{Immunità al Danno} freddo

\textbf{Sensi} scurovisione 18 m, vista cieca 3 m

\textbf{Linguaggi} Draconico

\textbf{Sfida} 2 (450 PE)

\textbf{Azioni}

\emph{\textbf{Morso.} Attacco con arma da mischia}: +6 a colpire, portata 1 m, un bersaglio.

\emph{Colpisce:} 9 (1d10 + 4) danni perforanti.

\emph{\textbf{Arma a Soffio (Ricarica 5-6).}} Il drago usa una delle seguenti armi a soffio:

\emph{Soffio Gelido.} Il drago esala un'esplosione ghiacciata in un cono di 5 metri. Ogni creatura nell'area deve effettuare un Tiro Salvezza su Tempra 13, subendo 18 (4d8) danni da freddo se fallisce il Tiro Salvezza, o la metà di questi danni se lo riesce.

\emph{Soffio Paralizzante.} Il drago esala un gas paralizzante in un cono di 5 metri. Ogni creatura nell'area deve riuscire un Tiro Salvezza su Tempra 13 o restare paralizzata per 1 minuto. Una creatura può ripetere il Tiro Salvezza al termine di ciascun suo turno, terminando l'effetto per sé in caso di successo.

\textbf{Ecologia}\\
Ambiente: Montagne Temperate\\
Organizzazione: Solitario\\
Tesoro: Triplo\\
\textbf{Descrizione}\\
Tra tutti i draghi, quelli d'argento sono i più coraggiosi, e si attengono ad un codice cavalleresco che impone loro di aiutare i deboli, sconfiggere il male e comportarsi in modo onorevole.\\


\medskip\index{Mostri - Drago di Bronzo Antico}\textbf{Drago di Bronzo Antico}

\emph{Mastodontica drago, caotico buono}

\textbf{FORZA} +9

\textbf{DESTREZZA} +0

\textbf{COSTITUZIONE} +8

\textbf{INTELLIGENZA} +4

\textbf{SAGGEZZA} +3

\textbf{CARISMA} +5

\textbf{Iniziativa} +4 -- \textbf{Difesa} 33

\textbf{Punti Ferita} 444 (24d20 + 192)

\textbf{Movimento} 12 m, nuoto 12 m, volo 24 m

\textbf{Tiri Salvezza} Tempra +21, Riflessi +13, Volontà +21

\textbf{Competenze} Muoversi Silenziosamente / Nascondersi +7, Percepire Emozioni +10, Consapevolezza +17

\textbf{Immunità al Danno} fulmine, arma +1

\textbf{Sensi} scurovisione 36 m, vista cieca 18 m

\textbf{Linguaggi} Comune, Draconico

\textbf{Sfida} 22 (41.000 PE)

\emph{\textbf{Anfibio.}} Il drago può respirare aria e acqua.

\emph{\textbf{Resistenza Leggendaria (3/Giorno).}} Se il drago fallisce un Tiro Salvezza, può scegliere invece di riuscire.

\textbf{Azioni}

\emph{\textbf{Multiattacco.}} Il drago può usare la sua Presenza Spaventosa. Poi effettuare tre attacchi: uno con il morso e due con gli artigli.

\emph{\textbf{Artiglio.} Attacco con arma da mischia}: +16 a colpire, portata 3 m, un bersaglio.

\emph{Colpisce:} 16 (2d6 + 9) danni taglienti.

\emph{\textbf{Coda.} Attacco con arma da mischia}: +16 a colpire, portata 6 m, un bersaglio.

\emph{Colpisce:} 18 (2d8 + 9) danni da botta.

\emph{\textbf{Morso.} Attacco con arma da mischia}: +16 a colpire, portata 5 metri, un bersaglio.

\emph{Colpisce:} 20 (2d10 + 9) danni perforanti.

\emph{\textbf{Presenza Spaventosa.}} Ogni creatura scelta dal drago, che si trovi entro 36 metri da esso e consapevole della sua presenza, deve riuscire un Tiro Salvezza di Volontà CD 20 o restare spaventata per 1 minuto. Una creatura può ripetere il Tiro Salvezza al termine di ciascun suo turno, terminando l'effetto se lo riesce. Se il Tiro Salvezza della creatura ha successo o l'effetto ha termine per essa, la creatura è immune alla Presenza Spaventosa del drago per le successive 24 ore.

\emph{\textbf{Arma a Soffio (Ricarica 5-6).}} Il drago usa una delle seguenti armi a soffio:

\emph{Soffio Fulminante.} Il drago esala fulmini in una linea lunga 36 metri e larga 3 metri. Ogni creatura sulla linea deve effettuare un Tiro Salvezza su Riflessi CD 23, subendo 88 (16d10) danni da fulmine se fallisce il Tiro Salvezza, o la metà di questi danni se lo riesce. \emph{Soffio Repulsivo.} Il drago esala dell'energia repulsiva in un cono di 9 metri. Ogni creatura in quell'area deve riuscire un Tiro Salvezza su Tempra CD 23, altrimenti viene allontana di 18 metri dal
drago.

\emph{\textbf{Mutare Forma.}} Il drago può trasformarsi magicamente in un umanoide o bestia il cui grado di sfida sia pari o inferiore al proprio, o tornare alla sua vera forma. Alla morte ritorna alla sua vera forma. Qualsiasi equipaggiamento stia indossando o trasportando viene assorbito o trasportato nella nuova forma (a scelta del drago).

Nella nuova forma, il drago mantiene il suo allineamento, punti ferita, Dadi Vita, la facoltà di parlare, le competenze, la Resistenza Leggendaria, le azioni da tana, e i punteggi di Intelligenza, Saggezza e Carisma, oltre a questa azione. Le sue statistiche e capacità vengono altrimenti rimpiazzate da quelle della nuova forma, eccetto Azioni aggiuntive della nuova forma.

\textbf{Azioni Aggiuntive}

Il drago può effettuare 3 Azioni aggiuntive, scelte tra le opzioni seguenti. Può usare solo un'opzione leggendaria alla volta e solo al termine del turno di un'altra creatura. Il drago recupera le Azioni aggiuntive spese all'inizio del proprio turno.

\textbf{Attacco di Ala (Costa 2 Azioni).} Il drago batte le ali. Ogni creatura entro 5 metri dal drago deve riuscire un Tiro Salvezza su Riflessi CD 24 o subire 16 (2d6 + 9) danni da botta e venir gettato prono. Il drago può poi volare fino a metà del suo movimento di volo.

\textbf{Attacco di Coda.} Il drago effettua un attacco di coda.

\textbf{Individuare.} Il drago effettua una prova di Saggezza (Consapevolezza).

\textbf{Ecologia}\\
Ambiente: Zone Costiere Temperate\\
Organizzazione: Solitario\\
Tesoro: Triplo\\
\textbf{Descrizione}\\
I draghi di bronzo sono noti per allearsi con viaggiatori ed avventurieri se causa e ricompensa sono giuste e adeguate\\
\textbf{Incantesimi}\index{Incantesimi da Drago Bronzo}\\
Il drago e' anche capace, se di eta' giovanile o piu' anziano, di lanciare incantesimi.\\
Puo' lanciare un numero di incantesimi pari al suo valore di Carisma, senza aver necessita' di componenti magici.\\
Non deve fare prove di magia, si considera che riesca sempre con un punteggio pari alla Difficoltà + il valore del Carisma.\\
Se deve fare un Tiro per Colpire ha CA = Grado di Sfida con un bonus al colpire pari alla Forza\\
Gli incantesimi a disposizione sono:\\
- Controllare Tempo Atmosferico\\
- Invocare il Fulmine\\


\medskip\index{Mostri - Drago di Bronzo Adulto}\textbf{Drago di Bronzo Adulto}

\emph{Enorme drago, caotico buono}

\textbf{FORZA} +7

\textbf{DESTREZZA} +0

\textbf{COSTITUZIONE} +6

\textbf{INTELLIGENZA} +3

\textbf{SAGGEZZA} +2

\textbf{CARISMA} +4

\textbf{Iniziativa} +3 -- \textbf{Difesa} 27

\textbf{Punti Ferita} 212 (17d12 + 102)

\textbf{Movimento} 12 m, nuoto 12 m, volo 24 m

\textbf{Tiri Salvezza} Tempra +15, Riflessi +10, Volontà +15

\textbf{Competenze} Muoversi Silenziosamente / Nascondersi +5, Percepire Emozioni +7, Consapevolezza +12

\textbf{Immunità al Danno} fulmine

\textbf{Sensi} scurovisione 36 m, vista cieca 18 m

\textbf{Linguaggi} Comune, Draconico

\textbf{Sfida} 15 (13.000 PE)

\emph{\textbf{Anfibio.}} Il drago può respirare aria e acqua.

\emph{\textbf{Resistenza Leggendaria (3/Giorno).}} Se il drago fallisce un Tiro Salvezza, può scegliere invece di riuscire.

\textbf{Azioni}

\emph{\textbf{Multiattacco.}} Il drago può usare la sua Presenza Spaventosa e poi effettuare tre attacchi: uno con il morso e due con gli
artigli.

\emph{\textbf{Artiglio.} Attacco con arma da mischia}: +12 a colpire, portata 1 m, un bersaglio.

\emph{Colpisce:} 14 (2d6 + 7) danni taglienti.

\emph{\textbf{Coda.} Attacco con arma da mischia}: +12 a colpire, portata 5 metri, un bersaglio.

\emph{Colpisce:} 16 (2d8 + 7) danni da botta.

\emph{\textbf{Morso.} Attacco con arma da mischia}: +12 a colpire, portata 3 m, un bersaglio.

\emph{Colpisce:} 18 (2d10 + 7) danni perforanti.

\emph{\textbf{Presenza Spaventosa.}} Ogni creatura scelta dal drago, che si trovi entro 36 metri da esso e consapevole della sua presenza, deve riuscire un Tiro Salvezza di Volontà CD 17 o restare spaventata per 1 minuto. Una creatura può ripetere il Tiro Salvezza al termine di ciascun suo turno, terminando l'effetto se lo riesce. Se il Tiro Salvezza della creatura ha successo o l'effetto ha termine per essa, la creatura è immune alla Presenza Spaventosa del drago per le successive 24 ore. 

\emph{\textbf{Arma a Soffio (Ricarica 5-6).}} Il drago usa una delle seguenti armi a soffio:

\emph{Soffio Fulminante.} Il drago esala fulmini in una linea lunga 27 metri e larga 1 metro. Ogni creatura sulla linea deve effettuare un Tiro Salvezza di Riflessi CD 19, subendo 66 (12d10) danni da fulmine se fallisce il Tiro Salvezza, o la metà di questi danni se lo riesce. \emph{Soffio Repulsivo.} Il drago esala dell'energia repulsiva in un cono di 9 metri. Ogni creatura in quell'area deve riuscire un Tiro Salvezza su Tempra CD 19, altrimenti viene allontana di 18 metri dal drago.

\emph{\textbf{Mutare Forma.}} Il drago può trasformarsi magicamente in un umanoide o bestia il cui grado di sfida sia pari o inferiore al proprio, o tornare alla sua vera forma. Alla morte ritorna alla sua vera forma. Qualsiasi equipaggiamento stia indossando o trasportando viene assorbito o trasportato nella nuova forma (a scelta del drago).

Nella nuova forma, il drago mantiene il suo allineamento, punti ferita, Dadi Vita, la facoltà di parlare, le competenze, la Resistenza Leggendaria, le azioni da tana, e i punteggi di Intelligenza, Saggezza e Carisma, oltre a questa azione. Le sue statistiche e capacità vengono altrimenti rimpiazzate da quelle della nuova forma, eccetto Azioni aggiuntive della nuova forma.

\textbf{Azioni Aggiuntive}

Il drago può effettuare 3 Azioni aggiuntive, scelte tra le opzioni seguenti. Può usare solo un'opzione leggendaria alla volta e solo al termine del turno di un'altra creatura. Il drago recupera le Azioni aggiuntive spese all'inizio del proprio turno.

\textbf{Attacco di Ala (Costa 2 Azioni).} Il drago batte le ali. Ogni creatura entro 3 metri dal drago deve riuscire un Tiro Salvezza su Riflessi CD 20 o subire 14 (2d6 + 7) danni da botta e venir gettato prono. Il drago può poi volare fino a metà del suo movimento di volo.

\textbf{Attacco di Coda.} Il drago effettua un attacco di coda. 

\textbf{Individuare.} Il drago effettua una prova di Saggezza (Consapevolezza).

\textbf{Ecologia}\\
Ambiente: Zone Costiere Temperate\\
Organizzazione: Solitario\\
Tesoro: Triplo\\
\textbf{Descrizione}\\
I draghi di bronzo sono noti per allearsi con viaggiatori ed avventurieri se causa e ricompensa sono giuste e adeguate\\
\textbf{Incantesimi}\index{Incantesimi da Drago Bronzo}\\
Il drago e' anche capace, se di eta' giovanile o piu' anziano, di lanciare incantesimi.\\
Puo' lanciare un numero di incantesimi pari al suo valore di Carisma, senza aver necessita' di componenti magici.\\
Non deve fare prove di magia, si considera che riesca sempre con un punteggio pari alla Difficoltà + il valore del Carisma.\\
Se deve fare un Tiro per Colpire ha CA = Grado di Sfida con un bonus al colpire pari alla Forza\\
Gli incantesimi a disposizione sono:\\
- Controllare Tempo Atmosferico\\
- Invocare il Fulmine\\


\medskip\index{Mostri - Drago di Bronzo Giovane}\textbf{Drago di Bronzo Giovane}

\emph{Grande drago, caotico buono}

\textbf{FORZA} +5

\textbf{DESTREZZA} +0

\textbf{COSTITUZIONE} +4

\textbf{INTELLIGENZA} +2

\textbf{SAGGEZZA} +1

\textbf{CARISMA} +3

\textbf{Iniziativa} +2 -- \textbf{Difesa} 22

\textbf{Punti Ferita} 142 (15d10 + 60)

\textbf{Movimento} 12 m, nuoto 12 m, volo 24 m

\textbf{Tiri Salvezza} Tempra +10, Riflessi +8, Volontà +10

\textbf{Competenze} Muoversi Silenziosamente / Nascondersi +3, Percepire Emozioni +4, Consapevolezza +7

\textbf{Immunità al Danno} fulmine

\textbf{Sensi} scurovisione 36 m, vista cieca 9 m

\textbf{Linguaggi} Comune, Draconico

\textbf{Sfida} 8 (3.900 PE)

\emph{\textbf{Anfibio.}} Il drago può respirare aria e acqua.

\textbf{Azioni}

\emph{\textbf{Multiattacco.}} Il drago può usare effettuare tre attacchi: uno con il morso e due con gli artigli.

\emph{\textbf{Artiglio.} Attacco con arma da mischia}: +8 a colpire, portata 1 m, un bersaglio.

\emph{Colpisce:} 12 (2d6 + 5) danni taglienti.

\emph{\textbf{Morso.} Attacco con arma da mischia}: +8 a colpire, portata 3 m, un bersaglio.

\emph{Colpisce:} 16 (2d10 + 5) danni perforanti.

\emph{\textbf{Arma a Soffio (Ricarica 5-6).}} Il drago usa una delle seguenti armi a soffio:

\emph{Soffio Fulminante.} Il drago esala fulmini in una linea lunga 18 metri e larga 1 metro. Ogni creatura sulla linea deve effettuare un Tiro Salvezza di Riflessi CD 15, subendo 55 (10d10) danni da fulmine se fallisce il Tiro Salvezza, o la metà di questi danni se lo riesce.

\emph{Soffio Repulsivo.} Il drago esala dell'energia repulsiva in un cono di 9 metri. Ogni creatura in quell'area deve riuscire un Tiro Salvezza su Tempra CD 15, altrimenti viene allontana di 12 metri dal drago.

\textbf{Ecologia}\\
Ambiente: Zone Costiere Temperate\\
Organizzazione: Solitario\\
Tesoro: Triplo\\
\textbf{Descrizione}\\
I draghi di bronzo sono noti per allearsi con viaggiatori ed avventurieri se causa e ricompensa sono giuste e adeguate\\
\textbf{Incantesimi}\index{Incantesimi da Drago Bronzo}\\
Il drago e' anche capace, se di eta' giovanile o piu' anziano, di lanciare incantesimi.\\
Puo' lanciare un numero di incantesimi pari al suo valore di Carisma, senza aver necessita' di componenti magici.\\
Non deve fare prove di magia, si considera che riesca sempre con un punteggio pari alla Difficoltà + il valore del Carisma.\\
Se deve fare un Tiro per Colpire ha CA = Grado di Sfida con un bonus al colpire pari alla Forza\\
Gli incantesimi a disposizione sono:\\
- Controllare Tempo Atmosferico\\
- Invocare il Fulmine\\


\medskip\index{Mostri - Drago di Bronzo Cucciolo}\textbf{Drago di Bronzo Cucciolo}

\emph{Media drago, caotico buono}

\textbf{FORZA} +3

\textbf{DESTREZZA} +0

\textbf{COSTITUZIONE} +2

\textbf{INTELLIGENZA} +1

\textbf{SAGGEZZA} +0

\textbf{CARISMA} +2

\textbf{Iniziativa} +1 -- \textbf{Difesa} 18

\textbf{Punti Ferita} 32 (5d8 + 10)

\textbf{Movimento} 9 m, nuoto 9 m, volo 18 m

\textbf{Tiri Salvezza} Tempra +2, Riflessi +1, Volontà +1

\textbf{Competenze} Muoversi Silenziosamente / Nascondersi +2, Consapevolezza +4

\textbf{Immunità al Danno} fulmine

\textbf{Sensi} scurovisione 18 m, vista cieca 3 m

\textbf{Linguaggi} Draconico

\textbf{Sfida} 2 (450 PE)

\emph{\textbf{Anfibio.}} Il drago può respirare aria e acqua.

\textbf{Azioni}

\emph{\textbf{Morso.} Attacco con arma da mischia}: +5 a colpire,
portata 1 m, un bersaglio.

\emph{Colpisce:} 8 (1d10 + 3) danni perforanti.

\emph{\textbf{Arma a Soffio (Ricarica 5-6).}} Il drago usa una delle seguenti armi a soffio:

\emph{Soffio Fulminante.} Il drago esala fulmini in una linea lunga 12 metri e larga 1 metro. Ogni creatura sulla linea deve effettuare un Tiro Salvezza di Riflessi CD 12, subendo 16 (3d10) danni da fulmine se fallisce il Tiro Salvezza, o la metà di questi danni se lo riesce.

\emph{Soffio Repulsivo.} Il drago esala dell'energia repulsiva in un cono di 9 metri. Ogni creatura in quell'area deve riuscire un Tiro Salvezza su Tempra CD 12, altrimenti viene allontana di 9 metri dal drago.

\textbf{Ecologia}\\
Ambiente: Zone Costiere Temperate\\
Organizzazione: Solitario\\
Tesoro: Triplo\\
\textbf{Descrizione}\\
I draghi di bronzo sono noti per allearsi con viaggiatori ed avventurieri se causa e ricompensa sono giuste e adeguate\\



\medskip\index{Mostri - Drago d'Oro Antico}\textbf{Drago d'Oro Antico}

\emph{Mastodontica drago, legale buono}

\textbf{FORZA} +10

\textbf{DESTREZZA} +2

\textbf{COSTITUZIONE} +9

\textbf{INTELLIGENZA} +4

\textbf{SAGGEZZA} +3

\textbf{CARISMA} +9

\textbf{Iniziativa} +4 -- \textbf{Difesa} 34

\textbf{Punti Ferita} 546 (28d20 + 252)

\textbf{Movimento} 12 m, nuoto 12 m, volo 24 m

\textbf{Tiri Salvezza} Tempra +23, Riflessi +14, Volontà +24

\textbf{Competenze} Muoversi Silenziosamente / Nascondersi +9, Percepire Emozioni +10, Consapevolezza +17, Ingannare +16

\textbf{Immunità al Danno} fuoco, arma +1

\textbf{Sensi} scurovisione 36 m, vista cieca 18 m

\textbf{Linguaggi} Comune, Draconico

\textbf{Sfida} 24 (62.000 PE)

\emph{\textbf{Anfibio.}} Il drago può respirare aria e acqua.

\emph{\textbf{Resistenza Leggendaria (3/Giorno).}} Se il drago fallisce un Tiro Salvezza, può scegliere invece di riuscire.

\textbf{Azioni}

\emph{\textbf{Multiattacco.}} Il drago può usare la sua Presenza Spaventosa. Poi effettuare tre attacchi: uno con il morso e due con gli artigli.

\emph{\textbf{Artiglio.} Attacco con arma da mischia}: +17 a colpire, portata 3 m, un bersaglio.

\emph{Colpisce:} 17 (2d6 + 10) danni taglienti.

\emph{\textbf{Coda.} Attacco con arma da mischia}: +17 a colpire, portata 6 m, un bersaglio.

\emph{Colpisce:} 19 (2d8 + 10) danni da botta.

\emph{\textbf{Morso.} Attacco con arma da mischia}: +17 a colpire, portata 5 metri, un bersaglio.

\emph{Colpisce:} 21 (2d10 + 10) danni perforanti.

\emph{\textbf{Presenza Spaventosa.}} Ogni creatura scelta dal drago, che si trovi entro 36 metri da esso e consapevole della sua presenza, deve riuscire un Tiro Salvezza di Volontà CD 24 o restare spaventata per 1 minuto. Una creatura può ripetere il Tiro Salvezza al termine di ciascun suo turno, terminando l'effetto se lo riesce. Se il Tiro Salvezza della creatura ha successo o l'effetto ha termine per essa, la creatura è immune alla Presenza Spaventosa del drago per le successive 24 ore.

\emph{\textbf{Arma a Soffio (Ricarica 5-6).}} Il drago usa una delle seguenti armi a soffio:

\emph{Soffio Infuocato.} Il drago esala fuoco in un cono di 27 metri. Ogni creatura nell'area deve effettuare un Tiro Salvezza di Riflessi CD 24, subendo 71(13d10) danni da fuoco se fallisce il Tiro Salvezza, o la metà di questi danni se lo riesce.

\emph{Soffio Indebolente.} Il drago esala del gas in un cono di 27 metri. Ogni creatura in quell'area deve riuscire un Tiro Salvezza su Tempra CD 24 o avere -1d6 ai tiri di attacco basati sulla Forza, prove di Forza, e Tiri Salvezza su Tempra per 1 minuto. Una creatura può ripetere il Tiro Salvezza al termine di ciascun suo turno, terminando l'effetto su di sé in caso di successo.

\emph{\textbf{Mutare Forma.}} Il drago può trasformarsi magicamente in un umanoide o bestia il cui grado di sfida sia pari o inferiore al proprio, o tornare alla sua vera forma. Alla morte ritorna alla sua vera forma. Qualsiasi equipaggiamento stia indossando o trasportando viene assorbito o trasportato nella nuova forma (a scelta del drago).

Nella nuova forma, il drago mantiene il suo allineamento, punti ferita, Dadi Vita, la facoltà di parlare, le competenze, la Resistenza Leggendaria, le azioni da tana, e i punteggi di Intelligenza, Saggezza e Carisma, oltre a questa azione. Le sue statistiche e capacità vengono altrimenti rimpiazzate da quelle della nuova forma, eccetto Azioni aggiuntive della nuova forma.

\textbf{Azioni Aggiuntive}

Il drago può effettuare 3 Azioni aggiuntive, scelte tra le opzioni seguenti. Può usare solo un'opzione leggendaria alla volta e solo al termine del turno di un'altra creatura. Il drago recupera le Azioni aggiuntive spese all'inizio del proprio turno.

\textbf{Attacco di Ala (Costa 2 Azioni).} Il drago batte le ali. Ogni creatura entro 5 metri dal drago deve riuscire un Tiro Salvezza su Riflessi CD 25 o subire 17 (2d6 + 10) danni da botta e venir gettato prono. Il drago può poi volare fino a metà del suo movimento di volo.

\textbf{Attacco di Coda.} Il drago effettua un attacco di coda.

\textbf{Individuare.} Il drago effettua una prova di Saggezza (Consapevolezza).

\textbf{Ecologia}\\
Ambiente: Pianure calde\\
Organizzazione: Solitario\\
Tesoro: Triplo\\
\textbf{Descrizione}\\
I draghi d'oro sono l'emblema della virtù. Gli altri draghi metallici li riveriscono come agenti delle potenze divine e membri esemplari della razza draconica, e spesso li cercano per consigli o aiuto.\\
\textbf{Incantesimi}\index{Incantesimi da Drago d'Oro}\\
Il drago e' anche capace, se di eta' giovanile o piu' anziano, di lanciare incantesimi.\\
Puo' lanciare un numero di incantesimi pari al suo valore di Carisma, senza aver necessita' di componenti magici.\\
Non deve fare prove di magia, si considera che riesca sempre con un punteggio pari alla Difficoltà + il valore del Carisma.\\
Se deve fare un Tiro per Colpire ha CA = Grado di Sfida con un bonus al colpire pari alla Forza\\
Gli incantesimi a disposizione sono:\\
- Blocca Mostri\\
- Muro di Fuoco\\
- Porta Dimensionale\\

\medskip\index{Mostri - Drago d'Oro Adulto}\textbf{Drago d'Oro Adulto}

\emph{Enorme drago, legale buono}

\textbf{FORZA} +8

\textbf{DESTREZZA} +2

\textbf{COSTITUZIONE} +7

\textbf{INTELLIGENZA} +3

\textbf{SAGGEZZA} +2

\textbf{CARISMA} +7

\textbf{Iniziativa} +3 -- \textbf{Difesa} 28

\textbf{Punti Ferita} 256 (19d12 + 133)

\textbf{Movimento} 12 m, nuoto 12 m, volo 24 m

\textbf{Tiri Salvezza} Tempra +17, Riflessi +11, Volontà +18

\textbf{Competenze} Muoversi Silenziosamente / Nascondersi +8, Percepire Emozioni +8, Consapevolezza +14, Ingannare +13 

\textbf{Immunità al Danno} fuoco

\textbf{Sensi} scurovisione 36 m, vista cieca 18 m

\textbf{Linguaggi} Comune, Draconico

\textbf{Sfida} 17 (18.000 PE)

\emph{\textbf{Anfibio.}} Il drago può respirare aria e acqua.

\emph{\textbf{Resistenza Leggendaria (3/Giorno).}} Se il drago fallisce un Tiro Salvezza, può scegliere invece di riuscire.

\textbf{Azioni}

\emph{\textbf{Multiattacco.}} Il drago può usare la sua Presenza Spaventosa. Poi effettuare tre attacchi: uno con il morso e due con gli artigli.

\emph{\textbf{Artiglio.} Attacco con arma da mischia}: +14 a colpire, portata 1 m, un bersaglio.

\emph{Colpisce:} 15 (2d6 + 8) danni taglienti.

\emph{\textbf{Coda.} Attacco con arma da mischia}: +14 a colpire, portata 5 metri, un bersaglio.

\emph{Colpisce:} 17 (2d8 + 8) danni da botta.

\emph{\textbf{Morso.} Attacco con arma da mischia}: +14 a colpire, portata 3 m, un bersaglio.

\emph{Colpisce:} 19 (2d10 + 8) danni perforanti.

\emph{\textbf{Presenza Spaventosa.}} Ogni creatura scelta dal drago, che si trovi entro 36 metri da esso e consapevole della sua presenza, deve riuscire un Tiro Salvezza di Volontà CD 21 o restare spaventata per 1 minuto. Una creatura può ripetere il Tiro Salvezza al termine di ciascun suo turno, terminando l'effetto se lo riesce. Se il Tiro Salvezza della creatura ha successo o l'effetto ha termine per essa, la creatura è immune alla Presenza Spaventosa del drago per le successive 24 ore.

\emph{\textbf{Arma a Soffio (Ricarica 5-6).}} Il drago usa una delle seguenti armi a soffio:

\emph{Soffio Infuocato.} Il drago esala fuoco in un cono di 18 metri. Ogni creatura nell'area deve effettuare un Tiro Salvezza di Riflessi CD 21, subendo 66 (12d10) danni da fuoco se fallisce il Tiro Salvezza, o la metà di questi danni se lo riesce.

\emph{Soffio Indebolente.} Il drago esala del gas in un cono di 18 metri. Ogni creatura in quell'area deve riuscire un Tiro Salvezza su Tempra CD 21 o avere -1d6 ai tiri di attacco basati sulla Forza, prove di Forza, e Tiri Salvezza su Tempra per 1 minuto. Una creatura può ripetere il Tiro Salvezza al termine di ciascun suo turno, terminando l'effetto su di sé in caso di successo.

\emph{\textbf{Mutare Forma.}} Il drago può trasformarsi magicamente in un umanoide o bestia il cui grado di sfida sia pari o inferiore al proprio, o tornare alla sua vera forma. Alla morte ritorna alla sua vera forma. Qualsiasi equipaggiamento stia indossando o trasportando viene assorbito o trasportato nella nuova forma (a scelta del drago).

Nella nuova forma, il drago mantiene il suo allineamento, punti ferita, Dadi Vita, la facoltà di parlare, le competenze, la Resistenza Leggendaria, le azioni da tana, e i punteggi di Intelligenza, Saggezza e Carisma, oltre a questa azione. Le sue statistiche e capacità vengono altrimenti rimpiazzate da quelle della nuova forma, eccetto Azioni aggiuntive della nuova forma. 

\textbf{Azioni Aggiuntive}

Il drago può effettuare 3 Azioni aggiuntive, scelte tra le opzioni seguenti. Può usare solo un'opzione leggendaria alla volta e solo al termine del turno di un'altra creatura. Il drago recupera le Azioni aggiuntive spese all'inizio del proprio turno.

\textbf{Attacco di Ala (Costa 2 Azioni).} Il drago batte le ali. Ogni creatura entro 3 metri dal drago deve riuscire un Tiro Salvezza su Riflessi CD 22 o subire 15 (2d6 + 8) danni da botta e venir gettato prono. Il drago può poi volare fino a metà del suo movimento di volo.

\textbf{Attacco di Coda.} Il drago effettua un attacco di coda.

\textbf{Individuare.} Il drago effettua una prova di Saggezza (Consapevolezza).

\textbf{Ecologia}\\
Ambiente: Pianure calde\\
Organizzazione: Solitario\\
Tesoro: Triplo\\
\textbf{Descrizione}\\
I draghi d'oro sono l'emblema della virtù. Gli altri draghi metallici li riveriscono come agenti delle potenze divine e membri esemplari della razza draconica, e spesso li cercano per consigli o aiuto.\\
\textbf{Incantesimi}\index{Incantesimi da Drago d'Oro}\\
Il drago e' anche capace, se di eta' giovanile o piu' anziano, di lanciare incantesimi.\\
Puo' lanciare un numero di incantesimi pari al suo valore di Carisma, senza aver necessita' di componenti magici.\\
Non deve fare prove di magia, si considera che riesca sempre con un punteggio pari alla Difficoltà + il valore del Carisma.\\
Se deve fare un Tiro per Colpire ha CA = Grado di Sfida con un bonus al colpire pari alla Forza\\
Gli incantesimi a disposizione sono:\\
- Blocca Mostri\\
- Muro di Fuoco\\
- Porta Dimensionale\\

\medskip\index{Mostri - Drago d'Oro Giovane}\textbf{Drago d'Oro Giovane}

\emph{Grande drago, legale buono}

\textbf{FORZA} +6

\textbf{DESTREZZA} +2

\textbf{COSTITUZIONE} +5

\textbf{INTELLIGENZA} +3

\textbf{SAGGEZZA} +1

\textbf{CARISMA} +5

\textbf{Iniziativa} +3 -- \textbf{Difesa} 23

\textbf{Punti Ferita} 178 (17d10 + 85)

\textbf{Movimento} 12 m, nuoto 12 m, volo 24 m

\textbf{Tiri Salvezza} Tempra +12, Riflessi +9, Volontà +13

\textbf{Competenze} Muoversi Silenziosamente / Nascondersi +6, Percepire Emozioni +5, Consapevolezza +9, Ingannare +9

\textbf{Immunità al Danno} fuoco

\textbf{Sensi} scurovisione 36 m, vista cieca 9 m

\textbf{Linguaggi} Comune, Draconico

\textbf{Sfida} 10 (5.900 PE)

\emph{\textbf{Anfibio.}} Il drago può respirare aria e acqua.

\textbf{Azioni}

\emph{\textbf{Multiattacco.}} Il drago può effettuare tre attacchi: uno con il morso e due con gli artigli.

\emph{\textbf{Artiglio.} Attacco con arma da mischia}: +10 a colpire, portata 1 m, un bersaglio.

\emph{Colpisce:} 13 (2d6 + 6) danni taglienti.

\emph{\textbf{Morso.} Attacco con arma da mischia}: +10 a colpire, portata 3 m, un bersaglio.

\emph{Colpisce:} 17 (2d10 + 6) danni perforanti.

\emph{\textbf{Arma a Soffio (Ricarica 5-6).}} Il drago usa una delle seguenti armi a soffio:

\emph{Soffio Infuocato.} Il drago esala fuoco in un cono di 9 metri. Ogni creatura nell'area deve effettuare un Tiro Salvezza di Riflessi CD 17, subendo 55 (10d10) danni da fuoco se fallisce il Tiro Salvezza, o la metà di questi danni se lo riesce.

\emph{Soffio Indebolente.} Il drago esala del gas in un cono di 9 metri. Ogni creatura in quell'area deve riuscire un Tiro Salvezza di Tempra CD 17 o avere -1d6 ai tiri di attacco basati sulla Forza, prove di Forza, e Tiri Salvezza su Tempra per 1 minuto. Una creatura può ripetere il Tiro Salvezza al termine di ciascun suo turno, terminando l'effetto su di sé in caso di successo.

\textbf{Ecologia}\\
Ambiente: Pianure calde\\
Organizzazione: Solitario\\
Tesoro: Triplo\\
\textbf{Descrizione}\\
I draghi d'oro sono l'emblema della virtù. Gli altri draghi metallici li riveriscono come agenti delle potenze divine e membri esemplari della razza draconica, e spesso li cercano per consigli o aiuto.\\
\textbf{Incantesimi}\index{Incantesimi da Drago d'Oro}\\
Il drago e' anche capace, se di eta' giovanile o piu' anziano, di lanciare incantesimi.\\
Puo' lanciare un numero di incantesimi pari al suo valore di Carisma, senza aver necessita' di componenti magici.\\
Non deve fare prove di magia, si considera che riesca sempre con un punteggio pari alla Difficoltà + il valore del Carisma.\\
Se deve fare un Tiro per Colpire ha CA = Grado di Sfida con un bonus al colpire pari alla Forza\\
Gli incantesimi a disposizione sono:\\
- Blocca Mostri\\
- Muro di Fuoco\\
- Porta Dimensionale\\

\medskip\index{Mostri - Drago d'Oro Cucciolo}\textbf{Drago d'Oro Cucciolo}

\emph{Media drago, legale buono}

\textbf{FORZA} +4

\textbf{DESTREZZA} +2

\textbf{COSTITUZIONE} +3

\textbf{INTELLIGENZA} +2

\textbf{SAGGEZZA} +0

\textbf{CARISMA} +3

\textbf{Iniziativa} +2 -- \textbf{Difesa} 19

\textbf{Punti Ferita} 60 (8d8 + 24)

\textbf{Movimento} 9 m, nuoto 9 m, volo 18 m

\textbf{Tiri Salvezza} Tempra +3, Riflessi +2, Volontà +1

\textbf{Competenze} Muoversi Silenziosamente / Nascondersi +4, Consapevolezza +4

\textbf{Immunità al Danno} fuoco

\textbf{Sensi} scurovisione 18 m, vista cieca 3 m

\textbf{Linguaggi} Draconico

\textbf{Sfida} 3 (700 PE)

\emph{\textbf{Anfibio.}} Il drago può respirare aria e acqua.

\textbf{Azioni}

\emph{\textbf{Morso.} Attacco con arma da mischia}: +6 a colpire, portata 1 m, un bersaglio.

\emph{Colpisce:} 9 (1d10 + 4) danni perforanti.

\emph{\textbf{Arma a Soffio (Ricarica 5-6).}} Il drago usa una delle seguenti armi a soffio:

\emph{Soffio Infuocato.} Il drago esala fuoco in un cono di 5 metri. Ogni creatura nell'area deve effettuare un Tiro Salvezza di Riflessi CD 13, subendo 22 (4d10) danni da fuoco se fallisce il Tiro Salvezza, o la metà di questi danni se lo riesce.

\emph{Soffio Indebolente.} Il drago esala del gas in un cono di 5 metri. Ogni creatura in quell'area deve riuscire un Tiro Salvezza su Tempra CD 13 o avere -1d6 ai tiri di attacco basati sulla Forza, prove di Forza, e Tiri Salvezza su Tempra per 1 minuto. Una creatura può ripetere il Tiro Salvezza al termine di ciascun suo turno, terminando l'effetto su di sé in caso di successo.

\textbf{Ecologia}\\
Ambiente: Pianure calde\\
Organizzazione: Solitario\\
Tesoro: Triplo\\
\textbf{Descrizione}\\
I draghi d'oro sono l'emblema della virtù. Gli altri draghi metallici li riveriscono come agenti delle potenze divine e membri esemplari della razza draconica, e spesso li cercano per consigli o aiuto.\\


\medskip\index{Mostri - Drago d'Ottone Antico}\textbf{Drago d'Ottone Antico}

\emph{Mastodontica drago, caotico buono}

\textbf{FORZA} +8

\textbf{DESTREZZA} +0

\textbf{COSTITUZIONE} +7

\textbf{INTELLIGENZA} +3

\textbf{SAGGEZZA} +2

\textbf{CARISMA} +4

\textbf{Iniziativa} +3 -- \textbf{Difesa} 30

\textbf{Punti Ferita} 297 (17d20 + 119)

\textbf{Movimento} 12 m, scavo 12 m, volo 24 m

\textbf{Tiri Salvezza} Tempra +20, Riflessi +13, Volontà +18

\textbf{Competenze} Muoversi Silenziosamente / Nascondersi +6, Consapevolezza +14, Ingannare +10, Storia +9 

\textbf{Immunità al Danno} fuoco, arma +1

\textbf{Sensi} scurovisione 36 m, vista cieca 18 m

\textbf{Linguaggi} Comune, Draconico

\textbf{Sfida} 20 (25.000 PE)

\emph{\textbf{Resistenza Leggendaria (3/Giorno).}} Se il drago fallisce un Tiro Salvezza, può scegliere invece di riuscire.

\textbf{Azioni}

\emph{\textbf{Multiattacco.}} Il drago può usare la sua Presenza Spaventosa. Poi effettuare tre attacchi: uno con il morso e due con gli artigli.

\emph{\textbf{Artiglio.} Attacco con arma da mischia}: +14 a colpire, portata 3 m, un bersaglio.

\emph{Colpisce:} 15 (2d6 + 8) danni taglienti.

\emph{\textbf{Coda.} Attacco con arma da mischia}: +14 a colpire, portata 6 m, un bersaglio.

\emph{Colpisce:} 17 (2d8 + 8) danni da botta.

\emph{\textbf{Morso.} Attacco con arma da mischia}: +14 a colpire, portata 5 metri, un bersaglio.

\emph{Colpisce:} 19 (2d10 + 8) danni perforanti.

\emph{\textbf{Presenza Spaventosa.}} Ogni creatura scelta dal drago, che si trovi entro 36 metri da esso e consapevole della sua presenza, deve riuscire un Tiro Salvezza di Volontà CD 18 o restare spaventata per 1 minuto. Una creatura può ripetere il Tiro Salvezza al termine di ciascun suo turno, terminando l'effetto se lo riesce. Se il Tiro Salvezza della creatura ha successo o l'effetto hatermine per essa, la creatura è immune alla Presenza Spaventosa del drago per le successive 24 ore.

\emph{\textbf{Arma a Soffio (Ricarica 5-6).}} Il drago usa una delle seguenti armi a soffio:

\emph{Soffio Infuocato.} Il drago esala fuoco in una linea lunga 27 metri e larga 3 metri. Ogni creatura sulla linea deve effettuare un Tiro Salvezza su Riflessi CD 21, subendo 56 (16d6) danni da fuoco se fallisce il Tiro Salvezza, o la metà di questi danni se lo riesce.

\emph{Soffio Soporifero.} Il drago esala del gas soporifero in un cono di 27 metri. Ogni creatura in quell'area deve riuscire un Tiro Salvezza su Tempra 21 o cadere svenuta per 10 minuti. Questo effetto
termina se la creatura svenuta subisce danni o qualcuno impiega un'azione per risvegliarla.

\emph{\textbf{Mutare Forma.}} Il drago può trasformarsi magicamente in un umanoide o bestia il cui grado di sfida sia pari o inferiore al proprio, o tornare alla sua vera forma. Alla morte ritorna alla sua vera forma. Qualsiasi equipaggiamento stia indossando o trasportando viene assorbito o trasportato nella nuova forma (a scelta del drago).

Nella nuova forma, il drago mantiene il suo allineamento, punti ferita, Dadi Vita, la facoltà di parlare, le competenze, la Resistenza Leggendaria, le azioni da tana, e i punteggi di Intelligenza, Saggezza e Carisma, oltre a questa azione. Le sue statistiche e capacità vengono altrimenti rimpiazzate da quelle della nuova forma, eccetto Azioni aggiuntive della nuova forma.

\textbf{Azioni Aggiuntive}

Il drago può effettuare 3 Azioni aggiuntive, scelte tra le opzioni seguenti. Può usare solo un'opzione leggendaria alla volta e solo al termine del turno di un'altra creatura. Il drago recupera le Azioni aggiuntive spese all'inizio del proprio turno.

\textbf{Attacco di Ala (Costa 2 Azioni).} Il drago batte le ali. Ogni creatura entro 5 metri dal drago deve riuscire un Tiro Salvezza su Riflessi CD 22 o subire 15 (2d6 + 8) danni da botta e venir gettato prono. Il drago può poi volare fino a metà del suo movimento di volo.

\textbf{Attacco di Coda.} Il drago effettua un attacco di coda.

\textbf{Individuare.} Il drago effettua una prova di Saggezza (Consapevolezza).

\textbf{Ecologia}\\
Ambiente: Deserti Caldi\\
Organizzazione: Solitario\\
Tesoro: Triplo\\
\textbf{Descrizione}\\
Ottimi conversatori, i draghi d'ottone preferiscono parlare invece che combattere. I draghi d'ottone fanno la tana vicino agli insediamenti umanoidi, dove possono udire le notizie e i pettegolezzi più recenti.\\
\textbf{Incantesimi}\index{Incantesimi da Drago d'Ottone}\\
Il drago e' anche capace, se di eta' giovanile o piu' anziano, di lanciare incantesimi.\\
Puo' lanciare un numero di incantesimi pari al suo valore di Carisma, senza aver necessita' di componenti magici.\\
Non deve fare prove di magia, si considera che riesca sempre con un punteggio pari alla Difficoltà + il valore del Carisma.\\
Se deve fare un Tiro per Colpire ha CA = Grado di Sfida con un bonus al colpire pari alla Forza\\
Gli incantesimi a disposizione sono:\\
- Inviare\\
- Trama Ipnotica\\
- Lingue\\


\medskip\index{Mostri - Drago d'Ottone Adulto}\textbf{Drago d'Ottone Adulto}

\emph{Enorme drago, caotico buono}

\textbf{FORZA} +6

\textbf{DESTREZZA} +0

\textbf{COSTITUZIONE} +5

\textbf{INTELLIGENZA} +2

\textbf{SAGGEZZA} +1

\textbf{CARISMA} +3

\textbf{Iniziativa} +2 -- \textbf{Difesa} 25

\textbf{Punti Ferita} 172 (15d12 + 75)

\textbf{Movimento} 12 m, scavo 9 m, volo 24 m

\textbf{Tiri Salvezza} Tempra +14, Riflessi +10, Volontà +12

\textbf{Competenze} Muoversi Silenziosamente / Nascondersi +5, Consapevolezza +11, Ingannare +8, Storia +7

\textbf{Immunità al Danno} fuoco

\textbf{Sensi} scurovisione 36 m, vista cieca 18 m 

\textbf{Linguaggi} Comune, Draconico

\textbf{Sfida} 13 (10.000 PE)

\emph{\textbf{Resistenza Leggendaria (3/Giorno).}} Se il drago fallisce un Tiro Salvezza, può scegliere invece di riuscire.

\textbf{Azioni}

\emph{\textbf{Multiattacco.}} Il drago può usare la sua Presenza Spaventosa. Poi effettuare tre attacchi: uno con il morso e due con gli artigli.

\emph{\textbf{Artiglio.} Attacco con arma da mischia}: +11 a colpire, portata 1 m, un bersaglio.

\emph{Colpisce:} 13 (2d6 + 6) danni taglienti.

\emph{\textbf{Coda.} Attacco con arma da mischia}: +11 a colpire, portata 5 metri, un bersaglio.

\emph{Colpisce:} 15 (2d8 + 6) danni da botta.

\emph{\textbf{Morso.} Attacco con arma da mischia}: +11 a colpire, portata 3 m, un bersaglio.

\emph{Colpisce:} 17 (2d10 + 6) danni perforanti.

\emph{\textbf{Presenza Spaventosa.}} Ogni creatura scelta dal drago, che si trovi entro 36 metri da esso e consapevole della sua presenza, deve riuscire un Tiro Salvezza di Volontà CD 16 o restare spaventata per 1 minuto. Una creatura può ripetere il Tiro Salvezza al termine di ciascun suo turno, terminando l'effetto se lo riesce. Se il Tiro Salvezza della creatura ha successo o l'effetto ha termine per essa, la creatura è immune alla Presenza Spaventosa del drago per le successive 24 ore.

\emph{\textbf{Arma a Soffio (Ricarica 5-6).}} Il drago usa una delle seguenti armi a soffio:

\emph{Soffio Infuocato.} Il drago esala fuoco in una linea lunga 18 metri e larga 1 metro. Ogni creatura sulla linea deve effettuare un Tiro Salvezza di Riflessi CD 18, subendo 45 (13d6) danni da fuoco se fallisce il Tiro Salvezza, o la metà di questi danni se lo riesce. 

\emph{Soffio Soporifero.} Il drago esala del gas soporifero in un cono di 18 metri. Ogni creatura in quell'area deve riuscire un Tiro Salvezza su Tempra 18 o cadere svenuta per 10 minuti. Questo effetto termina se la creatura svenuta subisce danni o qualcuno impiega un'azione per risvegliarla.

\textbf{Azioni Aggiuntive}

Il drago può effettuare 3 Azioni aggiuntive, scelte tra le opzioni seguenti. Può usare solo un'opzione leggendaria alla volta e solo al termine del turno di un'altra creatura. Il drago recupera le Azioni aggiuntive spese all'inizio del proprio turno.

\textbf{Attacco di Ala (Costa 2 Azioni).} Il drago batte le ali. Ogni creatura entro 3 metri dal drago deve riuscire un Tiro Salvezza su Riflessi CD 19 o subire 13 (2d6 + 6) danni da botta e venir gettato  prono. Il drago può poi volare fino a metà del suo movimento di volo.

\textbf{Attacco di Coda.} Il drago effettua un attacco di coda. 

\textbf{Individuare.} Il drago effettua una prova di Saggezza (Consapevolezza).

\textbf{Ecologia}\\
Ambiente: Deserti Caldi\\
Organizzazione: Solitario\\
Tesoro: Triplo\\
\textbf{Descrizione}\\
Ottimi conversatori, i draghi d'ottone preferiscono parlare invece che combattere. I draghi d'ottone fanno la tana vicino agli insediamenti umanoidi, dove possono udire le notizie e i pettegolezzi più recenti.\\
\textbf{Incantesimi}\index{Incantesimi da Drago d'Ottone}\\
Il drago e' anche capace, se di eta' giovanile o piu' anziano, di lanciare incantesimi.\\
Puo' lanciare un numero di incantesimi pari al suo valore di Carisma, senza aver necessita' di componenti magici.\\
Non deve fare prove di magia, si considera che riesca sempre con un punteggio pari alla Difficoltà + il valore del Carisma.\\
Se deve fare un Tiro per Colpire ha CA = Grado di Sfida con un bonus al colpire pari alla Forza\\
Gli incantesimi a disposizione sono:\\
- Inviare\\
- Trama Ipnotica\\
- Lingue\\

\medskip\index{Mostri - Drago d'Ottone Giovane}\textbf{Drago d'Ottone Giovane}

\emph{Grande drago, caotico buono}

\textbf{FORZA} +4

\textbf{DESTREZZA} +0

\textbf{COSTITUZIONE} +3

\textbf{INTELLIGENZA} +1

\textbf{SAGGEZZA} +0

\textbf{CARISMA} +2

\textbf{Iniziativa} +1 -- \textbf{Difesa} 20

\textbf{Punti Ferita} 110 (13d10 + 39)

\textbf{Movimento} 12 m, scavo 6 m, volo 24 m

\textbf{Tiri Salvezza} Tempra +9, Riflessi +8, Volontà +7

\textbf{Competenze} Muoversi Silenziosamente / Nascondersi +3, Consapevolezza +6, Ingannare +5

\textbf{Immunità al Danno} fuoco

\textbf{Sensi} scurovisione 36 m, vista cieca 9 m

\textbf{Linguaggi} Comune, Draconico

\textbf{Sfida} 6 (2.300 PE)

\textbf{Azioni}

\emph{\textbf{Multiattacco.}} Il drago può effettuare tre attacchi: uno con il morso e due con gli artigli.

\emph{\textbf{Artiglio.} Attacco con arma da mischia}: +7 a colpire, portata 1 m, un bersaglio.

\emph{Colpisce:} 11 (2d6 + 4) danni taglienti.

\emph{\textbf{Morso.} Attacco con arma da mischia}: +7 a colpire, portata 3 m, un bersaglio.

\emph{Colpisce:} 15 (2d10 + 4) danni perforanti.

\emph{\textbf{Arma a Soffio (Ricarica 5-6).}} Il drago usa una delle seguenti armi a soffio:

\emph{Soffio Infuocato.} Il drago esala fuoco in una linea lunga 12 metri e larga 1 metro. Ogni creatura sulla linea deve effettuare un Tiro Salvezza di Riflessi CD 14, subendo 42 (12d6) danni da fuoco se fallisce il Tiro Salvezza, o la metà di questi danni se lo riesce. \emph{Soffio Soporifero.} Il drago esala del gas soporifero in un cono di 9 metri. Ogni creatura in quell'area deve riuscire un Tiro Salvezza su Tempra 14 o cadere svenuta per 5 minuti. Questo effetto termina se la creatura svenuta subisce danni o qualcuno impiega un'azione per risvegliarla.

\textbf{Ecologia}\\
Ambiente: Deserti Caldi\\
Organizzazione: Solitario\\
Tesoro: Triplo\\
\textbf{Descrizione}\\
Ottimi conversatori, i draghi d'ottone preferiscono parlare invece che combattere. I draghi d'ottone fanno la tana vicino agli insediamenti umanoidi, dove possono udire le notizie e i pettegolezzi più recenti.\\
\textbf{Incantesimi}\index{Incantesimi da Drago d'Ottone}\\
Il drago e' anche capace, se di eta' giovanile o piu' anziano, di lanciare incantesimi.\\
Puo' lanciare un numero di incantesimi pari al suo valore di Carisma, senza aver necessita' di componenti magici.\\
Non deve fare prove di magia, si considera che riesca sempre con un punteggio pari alla Difficoltà + il valore del Carisma.\\
Se deve fare un Tiro per Colpire ha CA = Grado di Sfida con un bonus al colpire pari alla Forza\\
Gli incantesimi a disposizione sono:\\
- Inviare\\
- Trama Ipnotica\\
- Lingue\\

\medskip\index{Mostri - Drago d'Ottone Cucciolo}\textbf{Drago d'Ottone Cucciolo}

\emph{Media drago, caotico buono}

\textbf{FORZA} +2

\textbf{DESTREZZA} +0

\textbf{COSTITUZIONE} +1

\textbf{INTELLIGENZA} +0

\textbf{SAGGEZZA} +0

\textbf{CARISMA} +1

\textbf{Iniziativa} +0 -- \textbf{Difesa} 17

\textbf{Punti Ferita} 16 (3d8 + 3)

\textbf{Movimento} 9 m, scavo 5 metri, volo 18 m

\textbf{Tiri Salvezza} Tempra +2, Riflessi +0, Volontà +1

\textbf{Competenze} Muoversi Silenziosamente / Nascondersi +2, Consapevolezza +4

\textbf{Immunità al Danno} fuoco

\textbf{Sensi} scurovisione 18 m, vista cieca 3 m

\textbf{Linguaggi} Draconico

\textbf{Sfida} 1 (200 PE)

\textbf{Azioni}

\emph{\textbf{Morso.} Attacco con arma da mischia}: +4 a colpire, portata 1 m, un bersaglio.

\emph{Colpisce:} 7 (1d10 + 2) danni perforanti.

\emph{\textbf{Arma a Soffio (Ricarica 5-6).}} Il drago usa una delle seguenti armi a soffio:

\emph{Soffio Infuocato.} Il drago esala fuoco in una linea lunga 6 metri e larga 1 metro. Ogni creatura sulla linea deve effettuare un Tiro Salvezza su Riflessi CD 11, subendo 14 (4d6) danni da fuoco se fallisce il Tiro Salvezza, o la metà di questi danni se lo riesce.

\emph{Soffio Soporifero.} Il drago esala del gas soporifero in un cono di 5 metri. Ogni creatura in quell'area deve riuscire un Tiro Salvezza su Tempra 11 o cadere svenuta per 1 minuto. Questo effetto termina se la creatura svenuta subisce danni o qualcuno impiega un'azione per risvegliarla.

\textbf{Ecologia}\\
Ambiente: Deserti Caldi\\
Organizzazione: Solitario\\
Tesoro: Triplo\\
\textbf{Descrizione}\\
Ottimi conversatori, i draghi d'ottone preferiscono parlare invece che combattere. I draghi d'ottone fanno la tana vicino agli insediamenti umanoidi, dove possono udire le notizie e i pettegolezzi più recenti.\\



\medskip\index{Mostri - Drago di Rame Antico}\textbf{Drago di Rame Antico}

\emph{Mastodontica drago, caotico buono}

\textbf{FORZA} +8

\textbf{DESTREZZA} +1

\textbf{COSTITUZIONE} +7

\textbf{INTELLIGENZA} +5

\textbf{SAGGEZZA} +3

\textbf{CARISMA} +4

\textbf{Iniziativa} +5 -- \textbf{Difesa} 33

\textbf{Punti Ferita} 350 (20d20 + 140) 

\textbf{Movimento} 12 m, scalata 12 m, volo 24 m

\textbf{Tiri Salvezza} Tempra +20, Riflessi +13, Volontà +19

\textbf{Competenze} Muoversi Silenziosamente / Nascondersi +8, Ingannare +11, Consapevolezza +17

\textbf{Immunità al Danno} acido, arma +1

\textbf{Sensi} scurovisione 36 m, vista cieca 18 m

\textbf{Linguaggi} Comune, Draconico

\textbf{Sfida} 21 (33.000 PE)

\emph{\textbf{Resistenza Leggendaria (3/Giorno).}} Se il drago fallisce un Tiro Salvezza, può scegliere invece di riuscire.

\textbf{Azioni}

\emph{\textbf{Multiattacco.}} Il drago può usare la sua Presenza Spaventosa. Poi effettuare tre attacchi: uno con il morso e due con gli artigli.

\emph{\textbf{Artiglio.} Attacco con arma da mischia}: +15 a colpire, portata 3 m, un bersaglio.

\emph{Colpisce:} 15 (2d6 + 8) danni taglienti.

\emph{\textbf{Coda.} Attacco con arma da mischia}: +15 a colpire, portata 6 m, un bersaglio.

\emph{Colpisce:} 17 (2d8 + 8) danni da botta.

\emph{\textbf{Morso.} Attacco con arma da mischia}: +15 a colpire, portata 5 metri, un bersaglio.

\emph{Colpisce:} 19 (2d10 + 8) danni perforanti.

\emph{\textbf{Presenza Spaventosa.}} Ogni creatura scelta dal drago, che si trovi entro 36 metri da esso e consapevole della sua presenza, deve riuscire un Tiro Salvezza di Volontà CD 19 o restare spaventata per 1 minuto. Una creatura può ripetere il Tiro Salvezza al termine di ciascun suo turno, terminando l'effetto se lo riesce. Se il Tiro Salvezza della creatura ha successo o l'effetto ha termine per essa, la creatura è immune alla Presenza Spaventosa del drago per le successive 24 ore.

\emph{\textbf{Arma a Soffio (Ricarica 5-6).}} Il drago usa una delle seguenti armi a soffio:

\emph{Soffio Acido.} Il drago esala acido in una linea lunga 27 metri e larga 3 metri. Ogni creatura sulla linea deve effettuare un Tiro Salvezza su Riflessi CD 22, subendo 63 (14d8) danni da acido se fallisce il Tiro Salvezza, o la metà di questi danni se lo riesce.

\emph{Soffio Rallentante.} Il drago esala del gas in un cono di 27 metri. Ogni creatura in quell'area deve riuscire un Tiro Salvezza su Tempra CD 22. Se fallisce il Tiro Salvezza, la creatura non può usare la sua reazione, ha la velocità dimezzata, e non può effettuare più di un attacco durante il suo turno. Inoltre, la creatura può usare un'azione o un'azione bonus, ma non entrambe. Questi effetti permangono 1 minuto. La creatura può ripetere il Tiro Salvezza al termine di ciascun suo turno, terminando l'effetto su di sé in caso di successo.

\emph{\textbf{Mutare Forma.}} Il drago può trasformarsi magicamente in un umanoide o bestia il cui grado di sfida sia pari o inferiore al proprio, o tornare alla sua vera forma. Alla morte ritorna alla sua vera forma. Qualsiasi equipaggiamento stia indossando o trasportando viene assorbito o trasportato nella nuova forma (a scelta del drago).

Nella nuova forma, il drago mantiene il suo allineamento, punti ferita, Dadi Vita, la facoltà di parlare, le competenze, la Resistenza Leggendaria, le azioni da tana, e i punteggi di Intelligenza, Saggezza e Carisma, oltre a questa azione. Le sue statistiche e capacità

vengono altrimenti rimpiazzate da quelle della nuova forma, eccetto Azioni aggiuntive della nuova forma.

\textbf{Azioni Aggiuntive}

Il drago può effettuare 3 Azioni aggiuntive, scelte tra le opzioni seguenti. Può usare solo un'opzione leggendaria alla volta e solo al termine del turno di un'altra creatura. Il drago recupera le Azioni aggiuntive spese all'inizio del proprio turno.

\textbf{Attacco di Ala (Costa 2 Azioni).} Il drago batte le ali. Ogni creatura entro 5 metri dal drago deve riuscire un Tiro Salvezza su Riflessi CD 23 o subire 15 (2d6 + 8) danni da botta e venir gettato prono. Il drago può poi volare fino a metà del suo movimento di volo.

\textbf{Attacco di Coda.} Il drago effettua un attacco di coda.

\textbf{Individuare.} Il drago effettua una prova di Saggezza (Consapevolezza).

\textbf{Ecologia}\\
Ambiente: Colline Calde\\
Organizzazione: Solitario\\
Tesoro: Triplo\\
\textbf{Descrizione}\\
Questo drago capriccioso durante il combattimento cerca di ostacolare e frustrare i suoi nemici.\\
\textbf{Incantesimi}\index{Incantesimi da Drago di Rame}\\
Il drago e' anche capace, se di eta' giovanile o piu' anziano, di lanciare incantesimi.\\
Puo' lanciare un numero di incantesimi pari al suo valore di Carisma, senza aver necessita' di componenti magici.\\
Non deve fare prove di magia, si considera che riesca sempre con un punteggio pari alla Difficoltà + il valore del Carisma.\\
Se deve fare un Tiro per Colpire ha CA = Grado di Sfida con un bonus al colpire pari alla Forza\\
Gli incantesimi a disposizione sono:\\
- Metamorfosi\\
- Confusione\\
- Nube maleodorante\\


\medskip\index{Mostri - Drago di Rame Adulto}\textbf{Drago di Rame Adulto}

\emph{Enorme drago, caotico buono}

\textbf{FORZA} +6

\textbf{DESTREZZA} +1

\textbf{COSTITUZIONE} +5

\textbf{INTELLIGENZA} +4

\textbf{SAGGEZZA} +2

\textbf{CARISMA} +3

\textbf{Iniziativa} +4 -- \textbf{Difesa} 25

\textbf{Punti Ferita} 184 (16d12 + 80)

\textbf{Movimento} 12 m, scalata 12 m, volo 24 m

\textbf{Tiri Salvezza} Tempra +14, Riflessi +10, Volontà +13

\textbf{Competenze} Muoversi Silenziosamente / Nascondersi +6, Ingannare +8, Consapevolezza +12

\textbf{Immunità al Danno} acido

\textbf{Sensi} scurovisione 36 m, vista cieca 18 m

\textbf{Linguaggi} Comune, Draconico

\textbf{Sfida} 14 (11.500 PE)

\emph{\textbf{Resistenza Leggendaria (3/Giorno).}} Se il drago fallisce un Tiro Salvezza, può scegliere invece di riuscire.

\textbf{Azioni}

\emph{\textbf{Multiattacco.}} Il drago può usare la sua Presenza Spaventosa. Poi effettuare tre attacchi: uno con il morso e due con gli artigli.

\emph{\textbf{Artiglio.} Attacco con arma da mischia}: +11 a colpire, portata 1 m, un bersaglio.

\emph{Colpisce:} 13 (2d6 + 6) danni taglienti.

\emph{\textbf{Coda.} Attacco con arma da mischia}: +11 a colpire, portata 5 metri, un bersaglio.

\emph{Colpisce:} 15 (2d8 + 6) danni da botta.

\emph{\textbf{Morso.} Attacco con arma da mischia}: +11 a colpire, portata 3 m, un bersaglio.

\emph{Colpisce:} 17 (2d10 + 6) danni perforanti.

\emph{\textbf{Presenza Spaventosa.}} Ogni creatura scelta dal drago, che si trovi entro 36 metri da esso e consapevole della sua presenza, deve riuscire un Tiro Salvezza di Volontà CD 16 o restare spaventata per 1 minuto. Una creatura può ripetere il Tiro Salvezza al termine di ciascun suo turno, terminando l'effetto se lo riesce. Se il Tiro Salvezza della creatura ha successo o l'effetto ha termine per essa, la creatura è immune alla Presenza Spaventosa del drago per le successive 24 ore.

\emph{\textbf{Arma a Soffio (Ricarica 5-6).}} Il drago usa una delle seguenti armi a soffio:

\emph{Soffio Acido.} Il drago esala acido in una linea lunga 18 metri e larga 1 metro. Ogni creatura sulla linea deve effettuare un Tiro Salvezza su Riflessi CD 18, subendo 54 (12d8) danni da acido se fallisce il Tiro Salvezza, o la metà di questi danni se lo riesce.

\emph{Soffio Rallentante.} Il drago esala del gas in un cono di 18 metri. Ogni creatura in quell'area deve riuscire un Tiro Salvezza su Tempra CD 18. Se fallisce il Tiro Salvezza, la creatura non può usare la sua reazione, ha la velocità dimezzata, e non può effettuare più di un attacco durante il suo turno. Inoltre, la creatura può usare un'azione o un'azione bonus, ma non entrambe. Questi effetti permangono 1 minuto. La creatura può ripetere il Tiro Salvezza al termine di ciascun suo turno, terminando l'effetto su di sé in caso di successo.

\textbf{Azioni Aggiuntive}

Il drago può effettuare 3 Azioni aggiuntive, scelte tra le opzioni seguenti. Può usare solo un'opzione leggendaria alla volta e solo al termine del turno di un'altra creatura. Il drago recupera le Azioni aggiuntive spese all'inizio del proprio turno.

\textbf{Attacco di Ala (Costa 2 Azioni).} Il drago batte le ali. Ogni creatura entro 3 metri dal drago deve riuscire un Tiro Salvezza su Riflessi CD 19 o subire 13 (2d6 + 6) danni da botta e venir gettato prono. Il drago può poi volare fino a metà del suo movimento di volo.

\textbf{Attacco di Coda.} Il drago effettua un attacco di coda. 

\textbf{Individuare.} Il drago effettua una prova di Saggezza (Consapevolezza).

\textbf{Ecologia}\\
Ambiente: Colline Calde\\
Organizzazione: Solitario\\
Tesoro: Triplo\\
\textbf{Descrizione}\\
Questo drago capriccioso durante il combattimento cerca di ostacolare e frustrare i suoi nemici.\\
\textbf{Incantesimi}\index{Incantesimi da Drago di Rame}\\
Il drago e' anche capace, se di eta' giovanile o piu' anziano, di lanciare incantesimi.\\
Puo' lanciare un numero di incantesimi pari al suo valore di Carisma, senza aver necessita' di componenti magici.\\
Non deve fare prove di magia, si considera che riesca sempre con un punteggio pari alla Difficoltà + il valore del Carisma.\\
Se deve fare un Tiro per Colpire ha CA = Grado di Sfida con un bonus al colpire pari alla Forza\\
Gli incantesimi a disposizione sono:\\
- Metamorfosi\\
- Confusione\\
- Nube maleodorante\\


\medskip\index{Mostri - Drago di Rame Giovane}\textbf{Drago di Rame Giovane}

\emph{Grande drago, caotico buono}

\textbf{FORZA} +4

\textbf{DESTREZZA} +1

\textbf{COSTITUZIONE} +3

\textbf{INTELLIGENZA} +3

\textbf{SAGGEZZA} +1

\textbf{CARISMA} +2

\textbf{Iniziativa} +3 -- \textbf{Difesa} 21

\textbf{Punti Ferita} 119 (14d10 + 42)

\textbf{Movimento} 12 m, scalata 12 m, volo 24 m

\textbf{Tiri Salvezza} Tempra +9, Riflessi +8, Volontà +8

\textbf{Competenze} Muoversi Silenziosamente / Nascondersi +4, Ingannare +5, Consapevolezza +7

\textbf{Immunità al Danno} acido

\textbf{Sensi} scurovisione 36 m, vista cieca 9 m

\textbf{Linguaggi} Comune, Draconico

\textbf{Sfida} 7 (2.900 PE)

\textbf{Azioni}

\emph{\textbf{Multiattacco.}} Il drago può effettuare tre attacchi: uno con il morso e due con gli artigli.

\emph{\textbf{Artiglio.} Attacco con arma da mischia}: +7 a colpire, portata 1 m, un bersaglio.

\emph{Colpisce:} 11 (2d6 + 4) danni taglienti.

\emph{\textbf{Morso.} Attacco con arma da mischia}: +7 a colpire, portata 3 m, un bersaglio.

\emph{Colpisce:} 15 (2d10 + 4) danni perforanti.

\emph{\textbf{Arma a Soffio (Ricarica 5-6).}} Il drago usa una delle seguenti armi a soffio:

\emph{Soffio Acido.} Il drago esala acido in una linea lunga 12 metri e larga 1 metro. Ogni creatura sulla linea deve effettuare un Tiro Salvezza su Riflessi CD 14, subendo 40 (9d8) danni da acido se fallisce il Tiro Salvezza, o la metà di questi danni se lo riesce.

\emph{Soffio Rallentante.} Il drago esala del gas in un cono di 9 metri. Ogni creatura in quell'area deve riuscire un Tiro Salvezza su Tempra CD 14. Se fallisce il Tiro Salvezza, la creatura non può usare la sua reazione, ha la velocità dimezzata, e non può effettuare più di un attacco durante il suo turno. Inoltre, la creatura può usare un'azione o un'azione bonus, ma non entrambe. Questi effetti permangono 1 minuto. La creatura può ripetere il Tiro Salvezza al termine di ciascun suo turno, terminando l'effetto su di sé in caso di successo.

\textbf{Ecologia}\\
Ambiente: Colline Calde\\
Organizzazione: Solitario\\
Tesoro: Triplo\\
\textbf{Descrizione}\\
Questo drago capriccioso durante il combattimento cerca di ostacolare e frustrare i suoi nemici.\\
\textbf{Incantesimi}\index{Incantesimi da Drago di Rame}\\
Il drago e' anche capace, se di eta' giovanile o piu' anziano, di lanciare incantesimi.\\
Puo' lanciare un numero di incantesimi pari al suo valore di Carisma, senza aver necessita' di componenti magici.\\
Non deve fare prove di magia, si considera che riesca sempre con un punteggio pari alla Difficoltà + il valore del Carisma.\\
Se deve fare un Tiro per Colpire ha CA = Grado di Sfida con un bonus al colpire pari alla Forza\\
Gli incantesimi a disposizione sono:\\
- Metamorfosi\\
- Confusione\\
- Nube maleodorante\\

\textbf{Drago di Rame Cucciolo}

\emph{Media drago, caotico buono}

\textbf{FORZA} +2

\textbf{DESTREZZA} +1

\textbf{COSTITUZIONE} +1

\textbf{INTELLIGENZA} +2

\textbf{SAGGEZZA} +0

\textbf{CARISMA} +1

\textbf{Iniziativa} +2 -- \textbf{Difesa} 17

\textbf{Punti Ferita} 22 (4d8 + 4)

\textbf{Movimento} 9 m, scalata 9 m, volo 18 m

\textbf{Tiri Salvezza} Tempra +2, Riflessi +2, Volontà +0

\textbf{Competenze} Muoversi Silenziosamente / Nascondersi +3, Consapevolezza +4

\textbf{Immunità al Danno} acido

\textbf{Sensi} scurovisione 18 m, vista cieca 3 m

\textbf{Linguaggi} Draconico

\textbf{Sfida} 1 (200 PE)

\textbf{Azioni}

\emph{\textbf{Morso.} Attacco con arma da mischia}: +4 a colpire, portata 1 m, un bersaglio.

\emph{Colpisce:} 7 (1d10 + 2) danni perforanti.

\emph{\textbf{Arma a Soffio (Ricarica 5-6).}} Il drago usa una delle seguenti armi a soffio:

\emph{Soffio Acido.} Il drago esala acido in una linea lunga 6 metri e larga 1 metro. Ogni creatura sulla linea deve effettuare un Tiro Salvezza su Riflessi CD 11, subendo 18 (4d8) danni da acido se fallisce il Tiro Salvezza, o la metà di questi danni se lo riesce.

\emph{Soffio Rallentante.} Il drago esala del gas in un cono di 5 metri. Ogni creatura in quell'area deve riuscire un Tiro Salvezza su Tempra CD 11. Se fallisce il Tiro Salvezza, la creatura non può usare la sua reazione, ha la velocità dimezzata, e non può effettuare più di un attacco durante il suo turno. Inoltre, la creatura può usare un'azione o un'azione bonus, ma non entrambe. Questi effetti permangono 1 minuto. La creatura può ripetere il Tiro Salvezza al termine di ciascun suo turno, terminando l'effetto su di sé in caso di successo.

\textbf{Ecologia}\\
Ambiente: Colline Calde\\
Organizzazione: Solitario\\
Tesoro: Triplo\\
\textbf{Descrizione}\\
Questo drago capriccioso durante il combattimento cerca di ostacolare e frustrare i suoi nemici.\\


\medskip\index{Mostri - Drider}\textbf{Drider}

\emph{Grande mostruosità, caotico malvagio}

\textbf{FORZA} +3

\textbf{DESTREZZA} +3

\textbf{COSTITUZIONE} +4

\textbf{INTELLIGENZA} +1

\textbf{SAGGEZZA} +2

\textbf{CARISMA} +1

\textbf{Iniziativa} +3 -- \textbf{Difesa} 22

\textbf{Punti Ferita} 123 (13d10 + 52)

\textbf{Movimento} 9 m, scalata 9 m

\textbf{Tiri Salvezza} Tempra +7, Riflessi +5, Volontà +9

\textbf{Competenze} Muoversi Silenziosamente / Nascondersi +9, Consapevolezza +5

\textbf{Sensi} scurovisione 36 m

\textbf{Linguaggi} Elfico, Linguaggio delle Profondità

\textbf{Sfida} 6 (2.300 PE)

\emph{\textbf{Camminare sulla Tela.}} Il drider ignora le restrizioni al movimento provocate dalle ragnatele.

\emph{\textbf{Discendenza Fatata.}} Il drider ha +1d6 ai Tiri Salvezza per non restare affascinato, e la magia non può far addormentare un drider.

\emph{\textbf{Incantesimi Innati.}} La caratteristica da incantatore innato del drider è la Saggezza. Il drider può lanciare in maniera innata i seguenti incantesimi, senza bisogno di componenti materiali:

A volontà: \emph{luci danzanti}

1/Giorno: \emph{luminescenza, oscurità}

\emph{\textbf{Scalare come Ragno.}} Il drider può scalare superfici difficili, compreso lo stare a testa in giù sul soffitto, senza bisogno di effettuare una prova di abilità.

\textbf{Azioni}

\emph{\textbf{Multiattacco.}} Il drider effettua tre attacchi con la spada lunga o con l'arco lungo. Può rimpiazzare uno di questi attacchi con un attacco di morso.

\emph{\textbf{Morso.} Attacco con arma da mischia}: +6 a colpire, portata 1 m, una creatura.

\emph{Colpisce:} 2 (1d4) danni perforanti più 9 (2d8) danni da veleno.

\emph{\textbf{Spada Lunga.} Attacco con arma da mischia}: +6 a colpire, portata 1 m, un bersaglio.

\emph{Colpisce:} 7 (1d8 + 3) danni taglienti, o 8 (1d8 + 3) danni taglienti se usata con due mani.

\emph{\textbf{Arco Lungo.} Attacco con arma a Distanza}: +6 a colpire, gittata 45m, un bersaglio.

\emph{Colpisce:} 7 (1d8 + 3) danni perforanti più 4 (1d8) danni da veleno.

\textbf{Ecologia}\\
Ambiente: Qualsiasi sotterraneo\\
Organizzazione: Solitario, coppia o gruppo (3-8)\\
Tesoro: doppio (Mazza Pesante Perfetta, Arco Lungo Composito Perfetto [Forza +2] con 20 Frecce, altro tesoro)\\
\textbf{Descrizione}\\
Creato dal corpo di un drow, alterato e mutato attraverso speciali veleni ed elisir per assumere le caratteristiche di un ragno gigante, il drider è una creatura pericolosa.\\
I drider sono sessualmente dimorfici. La parte inferiore da ragno del corpo di un drider femmina è lucente ed aggraziata, spesso simile al corpo di una vedova nera, mentre il busto superiore di drow mantiene le sue curve allettanti e il bel viso (con l'eccezione delle venefiche zanne acuminate). La parte inferiore del corpo di un drider maschio è tozza come una tarantola, mentre quella superiore ha un fisico asciutto e supporta un'orrenda faccia più da ragno che da drow, completa di mandibole zannute.\\


\medskip\index{Mostri - Driade}\textbf{Driade}

\emph{Media fatato, neutrale}

\textbf{FORZA} +0

\textbf{DESTREZZA} +1

\textbf{COSTITUZIONE} +0

\textbf{INTELLIGENZA} +2

\textbf{SAGGEZZA} +2

\textbf{CARISMA} +4

\textbf{Iniziativa} +2 -- \textbf{Difesa} 12 (17 con \emph{pelle di corteccia})

\textbf{Punti Ferita} 22 (5d8)

\textbf{Vulnerabilità al Danno} ferro freddo

\textbf{Movimento} 9 m

\textbf{Tiri Salvezza} Tempra +5, Riflessi +9, Volontà +7

\textbf{Competenze} Muoversi Silenziosamente / Nascondersi +5, Consapevolezza +4

\textbf{Sensi} scurovisione 18 m

\textbf{Linguaggi} Elfico, Silvano

\textbf{Sfida} 1 (200 PE)

\emph{\textbf{Camminata Arborea.}} Uno volta durante il suo turno, la driade può usare 3 metri di movimento per entrare magicamente in un albero vivo a sua portata ed emergere da un altro albero vivo entro 18 metri dal primo albero, ricomparendo in uno spazio non occupato entro 1 metro dal secondo albero. Entrambi gli alberi devono essere di taglia Grande o superiore.

\emph{\textbf{Incantesimi Innati.}} La caratteristica da incantatore innato della driade è il Carisma (CD 14 per i Tiri Salvezza degli incantesimi). La driade può lanciare in maniera innata i seguenti incantesimi, senza aver bisogno di componenti materiali. A volontà: 

\emph{arte del druido}

3/giorno ciascuno: \emph{bacche benefiche}, \emph{intralciare} 1/giorno:
\emph{passare senza tracce, pelle coriacea, randello} \emph{incantato}

\emph{\textbf{Parlare con Animali e Piante.}} La driade può comunicare con bestie e piante come se parlassero la stessa lingua.

\emph{\textbf{Resistenza alla Magia.}} La driade ha +1d6 ai Tiri Salvezza contro incantesimi e altri effetti magici.

\textbf{Azioni}

\emph{\textbf{Randello.} Attacco con arma da mischia}: +2 a colpire (+6 a colpire con \emph{bastone}), portata 1 m, un bersaglio.

\emph{Colpisce:} 2 (1d4) danni da botta, o 8 (1d8 + 4) danni da botta con \emph{bastone}

\emph{\textbf{Fascino Fatato.}} La driade può prendere a bersaglio un umanoide o bestia entro 9 metri da lei e che possa vedere. Se il bersaglio può vedere la driade, deve riuscire un Tiro Salvezza su Volontà CD 14 o restare affascinato dalla magia. Le creature affascinate considerano la driade un'amica fidata da ascoltare e proteggere. Sebbene il bersaglio non sia sotto il controllo della driade, interpreterà le richieste o le azioni della driade nel modo più favorevole possibile.

Ogni volta che la driade o i suoi alleati arrecano danno al bersaglio, esso può ripetere il Tiro Salvezza, terminando l'effetto in caso di successo. Altrimenti, l'effetto permane 24 ore o finché la driade muore, si trova su di un piano di esistenza diverso rispetto al bersaglio, o termina l'effetto con un'azione bonus. Se il Tiro Salvezza del bersaglio riesce, il bersaglio sarà immune al Fascino Fatato della driade per le successive 24 ore.

La driade non può tenere affascinati più di un umanoide o tre bestie alla volta.

\textbf{Ecologia}\\
Ambiente: Foreste Temperate\\
Organizzazione: Solitario, coppia o boschetto (3-8)\\
Tesoro: standard (Arco Lungo Perfetto con 20 Frecce, Pugnale, altro tesoro)\\
\textbf{Descrizione}\\
Le driadi sono folletti-albero che amano i boschi appartati lontani dagli umanoidi bisognosi di legname. L'interesse principale delle driadi è la propria sopravvivenza e quella delle adorate foreste, e sono note per costringere magicamente i viaggiatori ad aiutarle in quei compiti che non possono espletare. Sono amichevoli con druidi e guardiaboschi non malvagi, dato che riconoscono la loro empatia o il loro rispetto per la natura.\\
Le driadi sono benevole guardiane degli alberi, e sebbene non siano violente di natura, possono bloccare e sventare le minacce alle loro dimore o trasformare i nemici in alleati. Alcune tengono uno o più umanoidi ammaliati nel proprio territorio per difenderlo o per sviare gli assalitori. I nemici resi inabili in genere vengono trascinati ai confini della foresta dagli alleati delle driadi e scacciati, ma quelli malvagi o ostili vengono uccisi una volta finito il combattimento.\\


\medskip\index{Mostri - Duergar}\textbf{Duergar}

\emph{Media umanoide (nano), legale malvagio}

\textbf{FORZA} +2

\textbf{DESTREZZA} +0

\textbf{COSTITUZIONE} +2

\textbf{INTELLIGENZA} +0

\textbf{SAGGEZZA} +0

\textbf{CARISMA} -1

\textbf{Iniziativa} +2 -- \textbf{Difesa} 17 (armatura di scaglie, scudo)

\textbf{Punti Ferita} 26 (4d8 + 8)

\textbf{Movimento} 8 m

\textbf{Tiri Salvezza} Tempra +4, Riflessi +0, Volontà +1

\textbf{Resistenza al Danno} veleno

\textbf{Sensi} scurovisione 36 m

\textbf{Linguaggi} Nanico, Linguaggio delle Profondità 

\textbf{Sfida} 1 (200 PE)

\emph{\textbf{Resilienza Duerga.}} Il duergar ha +1d6 ai Tiri Salvezza contro veleni, incantesimi e illusioni, oltre al resistere al restare affascinato o paralizzato.

\emph{\textbf{Sensibilità alla Luce}}. Mentre è alla luce del sole, il duergar ha -1d6 ai tiri di attacco, oltre che alle prove di Saggezza (Consapevolezza) basate sulla vista.

\textbf{Azioni}

\emph{\textbf{Ingrandire (Ricarica dopo un 1 ora).}} Per 1 minuto, il duergar aumenta magicamente di taglia, insieme a tutto ciò che sta trasportando o indossando. Mentre è ingrandito, il duergar è di taglia Grande, raddoppia i dadi di danno degli attacchi con armi basate sulla Forza (già incluso negli attacchi), e ha +1d6 alle prove di Forza e ai Tiri Salvezza di Forza. Se il duergar non ha sufficiente spazio per diventare Grande, ottiene la massima taglia concessa dallo spazio a disposizione.

\emph{\textbf{Piccone da Guerra.} Attacco con arma da mischia}: +4 a colpire, portata 1 m, un bersaglio.

\emph{Colpisce:} 6 (1d8 + 2) danni perforanti, o 11 (2d8 + 2) danni perforanti quando ingrandito.

\emph{\textbf{Giavellotto.} Attacco con arma da mischia o a Distanza}: +4 a colpire, portata 1 m o gittata 9m, un bersaglio. \emph{Colpisce:} 5 (1d6 + 2) danni perforanti o 9 (2d6 + 2) danni
perforanti quando ingrandito.

\emph{\textbf{Invisibilità (Ricarica dopo un 1 ora).}} Il duergar diventa magicamente invisibile al massimo per un'ora (come se stesse mantenendo la concentrazione per un incantesimo) o finché non attacca, lancia un incantesimo, usa Ingrandire o la sua concentrazione viene spezzata. Tutto l'equipaggiamento che il duergar indossa o trasporta diventa invisibile assieme a lui.

\textbf{Ecologia}\\
Ambiente: Qualsiasi sotterraneo\\
Organizzazione: solitario, gruppo (2-5), squadra (6-12 più 3 sergenti di 3° livello e 1 capo di 3°-8° livello), o clan (13-80 più 25\% di bambini non combattenti più 1 sergente di 3° livello ogni 5 adulti, 3-6 tenenti di 3°-6° livello, e 1-4 capitani di 9° livello)\\
Tesoro: equipaggiamento da PNG (Cotta di Maglia, Scudo Pesante di Metallo, Martello da Guerra, Balestra Leggera con 20 Quadrelli, 3d6 mo, altro tesoro)\\
\textbf{Descrizione}\\
Lontani parenti dei Nani, più cupi e deformi, i Duergar sono creature dal pessimo carattere che odiano gli intrusi nei loro reami sotterranei, ma mai più dei Nani. Vivono in comunità nelle profondità del sottosuolo. Hanno pelle grigio opaco, come fosse sporca di polvere o cenere, ma questa tonalità naturale permette di mimetizzarsi meglio nelle zone sotterranee. Sono una Razza di schiavisti, ma mentre costringono i prigionieri non Nani a lavori massacranti, uccidono senza remore i Nani catturati. In combattimento, i Duergar tirano di balestra, e poi passano al Martello da Guerra qualche round dopo. Se in inferiorità numerica, o se c'è un pericolo (e spazio) adeguato, un Duergar userà la sua capacità Ingrandire ed attacchera'.

\medskip\index{Mostri - Elementale dell'Acqua}\textbf{Elementale dell'Acqua}

\emph{Grande elementale, neutrale}

\textbf{FORZA} +4

\textbf{DESTREZZA} +2

\textbf{COSTITUZIONE} +4

\textbf{INTELLIGENZA} -3

\textbf{SAGGEZZA} +0

\textbf{CARISMA} -1

\textbf{Iniziativa} +4 -- \textbf{Difesa} 17

\textbf{Punti Ferita} 114 (12d10 + 48)

\textbf{Movimento} 9 m, nuoto 27 m

\textbf{Tiri Salvezza} Tempra +9, Riflessi +8, Volontà +2

\textbf{Resistenze al Danno} acido; da botta, perforante e tagliente di attacchi non magici

\textbf{Immunità al Danno} veleno

\textbf{Immunità alle Condizioni} afferrato, avvelenato, intralciato, paralizzato, pietrificato, privo di sensi, prono, affaticamento

\textbf{Sensi} scurovisione 18 m

\textbf{Linguaggi} Aquan

\textbf{Sfida} 5 (1.800 PE)

\emph{\textbf{Congelamento.}} Se l'elementale subisce danno da freddo, gela parzialmente; il suo movimento è ridotto di 6 metri fino al termine del suo prossimo turno.

\emph{\textbf{Forma d'Acqua.}} L'elementale può entrare nello spazio di una creatura ostile e fermarsi lì. Può muoversi attraverso uno spazio stretto fino a 3 centimetri senza doversi stringere.

\emph{\textbf{Natura Elementale.}} Un elementale non ha bisogno di aria,
cibo, bevande o sonno.

\textbf{Azioni}

\emph{\textbf{Multiattacco.}} L'elementale effettua due attacchi di
schianto.

\emph{\textbf{Schianto.} Attacco con arma da mischia}: +7 a colpire,
portata 1 m, un bersaglio.

\emph{Colpisce:} 13 (2d8 + 4) danni da botta.

\emph{\textbf{Sommergere (Ricarica 4-6).}} Ogni creatura nello spazio dell'elementale deve effettuare un Tiro Salvezza di Tempra CD 15. Se lo fallisce, il bersaglio subisce 13 (2d8 + 4) danni da botta. Se è di taglia Grande o inferiore, il bersaglio è anche afferrato (CD 14 per fuggire). Fino al termine dell'afferrare, il bersaglio è intralciato e non può respirare a meno che non sia in grado di respirare acqua. Se il Tiro Salvezza riesce, il bersaglio viene spinto fuori dallo spazio
dell'elementale.

L'elementale può afferrare una creatura Grande o fino a due Medie o più piccole alla volta. All'inizio di ciascun turno dell'elementale, ogni bersaglio afferrato subisce 13 (2d8 + 4) danni da botta. Una creatura entro 1 metro dall'elementale può trascinare fuori da esso una creatura o oggetto, impiegando un'azione per tentare di riuscire una prova di Forza CD 14.

\textbf{Ecologia}\\
Ambiente: Qualsiasi (Piano dell'Acqua)\\
Organizzazione: Solitario, coppia o gruppo (3-8)\\
Tesoro: Nessuno\\
\textbf{Descrizione}\\
Gli elementali dell'acqua sono creature pazienti e inflessibili composte di acqua vivente, dolce o salata. Preferiscono coprire d'acqua i loro avversari o trascinarveli dentro per ottenere un vantaggio.\\
Come gli altri elementali, tutti gli elementali dell'acqua hanno aspetto e forma unici. Molti sono creature dall'aspetto simile ad un'ondata con faccia vagamente umanoide e onde più piccole ai lati che fungono da braccia. Un'altra forma comune è quella di una qualche creatura acquatica, come uno squalo o un polpo, ma fatta interamente d'acqua.\\
Un elementale dell'acqua grande è alto 4,8 metri e pesa 1125kg.\\

\medskip\index{Mostri - Elementale dell'Aria}\textbf{Elementale dell'Aria}

\emph{Grande elementale, neutrale}

\textbf{FORZA} +2

\textbf{DESTREZZA} +5

\textbf{COSTITUZIONE} +2

\textbf{INTELLIGENZA} -2

\textbf{SAGGEZZA} +0

\textbf{CARISMA} -2

\textbf{Iniziativa} +5 -- \textbf{Difesa} 18

\textbf{Punti Ferita} 90 (12d10 + 24)

\textbf{Movimento} 0 m, volo 27 m (fluttua)

\textbf{Tiri Salvezza} Tempra +9, Riflessi +13, Volontà +2

\textbf{Resistenze al Danno} fulmine, tuono; da botta, perforante e tagliente di attacchi non magici

\textbf{Immunità al Danno} veleno

\textbf{Immunità alle Condizioni} afferrato, avvelenato, intralciato, paralizzato, pietrificato, privo di sensi, prono, affaticamento

\textbf{Sensi} scurovisione 18 m

\textbf{Linguaggi} Auran

\textbf{Sfida} 5 (1.800 PE)

\emph{\textbf{Forma d'Aria.}} L'elementale può entrare nello spazio di una creatura ostile e fermarsi lì. Può muoversi attraverso uno spazio stretto fino a 3 centimetri senza doversi stringere.

\emph{\textbf{Natura Elementale.}} Un elementale non ha bisogno di aria, cibo, bevande o sonno.

\textbf{Azioni}

\emph{\textbf{Multiattacco.}} L'elementale effettua due attacchi di schianto.

\emph{\textbf{Schianto.} Attacco con arma da mischia}: +8 a colpire, portata 1 m, un bersaglio.

\emph{Colpisce:} 14 (2d8 + 5) danni da botta.

\emph{\textbf{Turbine (Ricarica 4-6).}} Ogni creatura nello spazio dell'elementale deve effettuare un Tiro Salvezza di Tempra CD 13. Se lo fallisce, il bersaglio subisce 15 (3d8 + 2) danni da botta e viene scagliato a 6 metri di distanza dall'elementale in una direzione casuale e cadere prono. Se un bersaglio lanciato colpisce un oggetto, come un muro o il pavimento, subisce 3 (1d6) danni da botta per ogni 3 metri per cui è stato lanciato. Se il bersaglio viene lanciato contro un'altra creatura, quella creatura deve riuscire un Tiro Salvezza di Riflessi CD 13 o subire lo stesso danno e cadere prona.

Se il Tiro Salvezza riesce, il bersaglio subisce la metà del danno da botta e non viene scagliato via né cade prono.

\textbf{Ecologia}\\
Ambiente: Piano dell'Aria\\
Organizzazione: Solitario, coppia o gruppo (3-8)\\
Tesoro: Nessuno\\
\textbf{Descrizione}\\
Gli elementali dell'aria sono rapide creature volanti fatte d'aria. Primitivi e territoriali, non amano essere evocati o controllati dai mortali, e preferiscono trascorrere il loro tempo sul Piano dell'Aria, volando attraverso il cielo infinito.\\
Sebbene tutti gli elementali dell'aria della stessa taglia abbiano le stesse statistiche, l'aspetto esatto di ognuno varia molto da individuo a individuo: uno può apparire come un vortice animato di vento e fumo, mentre un altro come una creatura di fumo simile ad un uccello con occhi scintillanti e ali di vento.\\
Un elementale dell'aria preferisce attaccare le creature che volano, non solo per i vantaggi che ha grazie alla sua padronanza dell'aria, ma anche perché detesta toccare il terreno. Un elementale dell'aria può muoversi sott'acqua, e anche se non corre alcun rischio di annegamento, non ha gradi in Nuotare e sott'acqua perde gran parte della sua mobilità e velocità.\\
Un elementale dell'Aria Grande è alto 4,8 m e pesa 2 kg.\\


\medskip\index{Mostri - Elementale del Fuoco}\textbf{Elementale del Fuoco}

\emph{Grande elementale, neutrale}

\textbf{FORZA} +0

\textbf{DESTREZZA} +3

\textbf{COSTITUZIONE} +3

\textbf{INTELLIGENZA} -2

\textbf{SAGGEZZA} +0

\textbf{CARISMA} -2

\textbf{Iniziativa} +3 -- \textbf{Difesa} 16

\textbf{Punti Ferita} 102 (12d10 + 36)

\textbf{Movimento} 15 m

\textbf{Tiri Salvezza} Tempra +8, Riflessi +11, Volontà +4

\textbf{Resistenze al Danno} da botta, perforante e tagliente di attacchi non magici

\textbf{Immunità al Danno} fuoco, veleno

\textbf{Immunità alle Condizioni} afferrato, avvelenato, intralciato, paralizzato, pietrificato, prono, privo di sensi, affaticamento

\textbf{Sensi} scurovisione 18 m

\textbf{Linguaggi} Ignan

\textbf{Sfida} 5 (1.800 PE)

\emph{\textbf{Forma di Fuoco.}} L'elementale può spostarsi attraverso uno spazio fino a 3 centimetri di larghezza senza stringersi. Una creatura che entri a contatto o colpisca l'elementale con un attacco da mischia mentre si trova entro 1 metro da esso subisce 5 (1d10) danni da fuoco. Inoltre, l'elementale può entrare nello spazio di una creatura ostile e fermarsi lì. La prima volta che entra nello spazio di una creatura in un turno, la creatura subisce 5 (1d10) danni da fuoco e prende fuoco; finché qualcuno non impiega un'azione per spegnere le fiamme, la creatura subirà 5 (1d10) danni da fuoco all'inizio di ciascun proprio turno.

\emph{\textbf{Illuminazione.}} L'elementale emette luce intensa in un raggio di 9 metri e luce fioca per ulteriori 9 metri.

\emph{\textbf{Natura Elementale.}} Un elementale non ha bisogno di aria, cibo, bevande o sonno.

\emph{\textbf{Suscettibilità all'Acqua.}} L'elementale subisce 1 danno da freddo per ogni 1 metro che si muove in acqua o per ogni 4 litri d'acqua che gli vengono spruzzati addosso.

\textbf{Azioni}

\emph{\textbf{Multiattacco.}} L'elementale effettua due attacchi di contatto.

\emph{\textbf{Contatto.} Attacco con arma da mischia}: +6 a colpire, portata 1 m, un bersaglio.

\emph{Colpisce:} 10 (2d8 + 5) danni da fuoco. Se il bersaglio è una creatura o un oggetto infiammabile, prende fuoco. Finché una creatura non impiega un'azione per spegnere le fiamme, la creatura subirà 5 (1d10) danni da fuoco all'inizio di ciascun proprio turno.

\textbf{Ecologia}
Ambiente: Qualsiasi (Piano del Fuoco)\\
Organizzazione: Solitario, coppia o gruppo (3-8)\\
Tesoro: Nessuno\\
\textbf{Descrizione}\\
Gli elementali del fuoco sono creature veloci e crudeli fatte di fiamme viventi. Si divertono a spaventare quelli più deboli di loro, e terrorizzano qualsiasi creatura che possano incendiare. Un elementale del fuoco non può entrare nel'acqua o in qualsiasi liquido ininfiammabile. Una massa d'acqua è una barriera impenetrabile a meno che l'elementale possa scavalcarla o saltarla, oppure venga coperta con materiale infiammabile (come uno strato d'olio).\\
Gli elementali del fuoco hanno un aspetto variabile; in genere si manifestano in forma di spire serpentine fatte di fumo e fiamme, ma alcuni elementali del fuoco prendono sembianze più simili a quelle di umani, demoni o altri mostri per aumentare il terrore quando compaiono improvvisamente. Il corpo di un elementale del fuoco sembra fatto di fiamme o sbuffi di scintille, fumo o cenere semistabili.\\

Un elementale del fuoco grande è alto 4,8 metri\\

\medskip\index{Mostri - Elementale della Terra}\textbf{Elementale della Terra}

\emph{Grande elementale, neutrale}

\textbf{FORZA} +5

\textbf{DESTREZZA} -1

\textbf{COSTITUZIONE} +5

\textbf{INTELLIGENZA} -3

\textbf{SAGGEZZA} +0

\textbf{CARISMA} -3

\textbf{Iniziativa} -1 -- \textbf{Difesa} 20

\textbf{Punti Ferita} 126 (12d10 + 60)

\textbf{Movimento} 9 m, scavo 9 m

\textbf{Tiri Salvezza} Tempra +9, Riflessi +1, Volontà +6

\textbf{Vulnerabilità al Danno} tuono

\textbf{Resistenze al Danno} da botta, perforante e tagliente di attacchi non magici

\textbf{Immunità al Danno} veleno

\textbf{Immunità alle Condizioni} avvelenato, paralizzato, pietrificato, prono, privo di sensi, affaticamento,

\textbf{Sensi} percezione tellurica 18 m, scurovisione 18 m

\textbf{Linguaggi} Terran

\textbf{Sfida} 5 (1.800 PE)

\emph{\textbf{Mostro d'Assedio.}} L'elementale infligge danni doppi agli oggetti e le strutture.

\emph{\textbf{Natura Elementale.}} Un elementale non ha bisogno di aria, cibo, bevande o sonno.

\emph{\textbf{Planata Terrestre.}} L'elementale può scavare attraversa la terra e la pietra non magiche e non lavorate. Quando lo fa, l'elementale non disturba il materiale che sposta. 
\textbf{Azioni}

\emph{\textbf{Multiattacco.}} L'elementale effettua due attacchi di schianto.

\emph{\textbf{Schianto.} Attacco con arma da mischia}: +8 a colpire, portata 3 m, un bersaglio.

\emph{Colpisce:} 14 (2d8 + 5) danni da botta.

\textbf{Ecologia}
Ambiente: Qualsiasi (Piano della Terra)\\
Organizzazione: Solitario, coppia o gruppo (3-8)\\
Tesoro: Nessuno\\
\textbf{Descrizione}\\
Gli elementali della terra sono creature lente ed ostinate fatte di pietra o terra. Quando stanno completamente fermi sono indistinguibili da un mucchio di pietre o una piccola collina.\\

Quando un elementale della terra si mette pesantemente in movimento, il suo aspetto esteriore può variare, anche se le sue statistiche restano identiche a quelle dei suoi simili della stessa taglia. Gli elementali della terra sono fatti per lo più di roccia, terra o cristallo, con gemme scintillanti come occhi. Quelli più grandi hanno l'aspetto di umanoidi di pietra. Ciuffi di vegetazione spesso crescono sul suolo che costituisce parte del corpo di un elementale della terra.\\

Un elementale della terra grande è alto 4,8 metri e pesa 3000 kg.\\


\medskip\index{Mostri - Ettercap}\textbf{Ettercap}

\emph{Media mostruosità, neutrale malvagio}

\textbf{FORZA} +2

\textbf{DESTREZZA} +2

\textbf{COSTITUZIONE} +1

\textbf{INTELLIGENZA} -2

\textbf{SAGGEZZA} +1

\textbf{CARISMA} 8 (-2)

\textbf{Iniziativa} +2 -- \textbf{Difesa} 14

\textbf{Punti Ferita} 44 (8d8 + 8)

\textbf{Movimento} 9 m, scalata 9 m

\textbf{Tiri Salvezza} Tempra +6, Riflessi +4, Volontà +6

\textbf{Competenze} Muoversi Silenziosamente / Nascondersi +4, Consapevolezza +3, Sopravvivenza +3

\textbf{Sensi} scurovisione 18 m

\textbf{Linguaggi} -

\textbf{Sfida} 2 (450 PE)

\emph{\textbf{Camminare sulla Tela.}} L'ettercap ignora le restrizioni al movimento provocate dalle ragnatele.

\emph{\textbf{Scalare come Ragno.}} L'ettercap può scalare superfici difficili, compreso lo stare a testa in giù sul soffitto, senza bisogno di effettuare una prova di caratteristica.

\emph{\textbf{Senso della Tela.}} Mentre è in contatto con una ragnatela, l'ettercap sa l'esatta posizione di qualsiasi altra creatura in contatto con la stessa ragnatela.

\textbf{Azioni}

\emph{\textbf{Multiattacco.}} L'ettercap effettua due attacchi: uno con il morso e uno con gli artigli

\emph{\textbf{Artigli.} Attacco con arma da mischia}: +4 a colpire, portata 1 m, un bersaglio.

\emph{Colpisce:} 7 (2d4 + 2) danni taglienti.

\emph{\textbf{Morso.} Attacco con arma da mischia}: +4 a colpire, portata 1 m, un bersaglio.

\emph{Colpisce:} 6 (1d8 + 2) danni perforanti più 4 (1d8) danni da veleno. Il bersaglio deve riuscire un Tiro Salvezza di Tempra CD 11 o restare avvelenato per 1 minuto. La creatura può ripetere il Tiro Salvezza al termine di ciascun suo turno, terminando l'effetto se riesce il Tiro Salvezza.

\emph{\textbf{Ragnatela (Ricarica 5-6).} Attacco con arma a Distanza}: +4 a colpire, gittata 9m, una creatura di taglia Grande o minore. \emph{Colpisce:} La creatura è intralciata dalla ragnatela. Con un'azione, la creatura intralciata può effettuare una prova di Forza CD 11, liberandosi dalla tela se la riesce. L'effetto termina se la tela è distrutta. La tela ha Difesa 10, 5 punti ferita, vulnerabilità ai danni da fuoco, e immunità ai danni da botta, da veleno e psichici.

\textbf{Ecologia}\\
Ambiente: Foreste Temperate\\
Organizzazione: solitario, coppia o nido (3-6 più 2-8 ragni giganti)\\
Tesoro: Standard\\
\textbf{Descrizione}\\
Gli ettercap sono alti di solito 1,8 metri e pesano circa 100 kg. Sono solitari e raramente si uniscono ad altri della loro razza, tranne per l'accoppiamento. Quando fanno gruppo, tendono ad attrarre varie specie di ragni, formando uno strano connubio di ettercap e aracnidi.\\
Gli ettercap sono noti per la costruzione di astute trappole fatte di ragnatele e altri materiali naturali, che usano per catturare prede. Costruiscono rifugi di ragnatela, tra i rami più alti gli alberi lontano dagli altri predatori terrestri, e usano ragni mostruosi come vedette e guardiani.\\
Gli ettercap non sono coraggiosi, ma le loro trappole spesso impediscono al nemico di estrarre le armi. Un ettercap attacca con artigli e morsi velenosi. In genere evita la mischia con gli avversari che possono ancora muoversi e fugge se si liberano.\\


\medskip\index{Mostri - Ettin}\textbf{Ettin}

\emph{Grande gigante, caotico malvagio}

\textbf{FORZA} +5

\textbf{DESTREZZA} -1

\textbf{COSTITUZIONE} +3

\textbf{INTELLIGENZA} -2

\textbf{SAGGEZZA} +0

\textbf{CARISMA} -1

\textbf{Iniziativa} -1 -- \textbf{Difesa} 14

\textbf{Punti Ferita} 85 (10d10 + 30)

\textbf{Movimento} 12 m

\textbf{Tiri Salvezza} Tempra +9, Riflessi +2, Volontà +5

\textbf{Competenze} Consapevolezza +4

\textbf{Linguaggi} Gigante, Goblinoide

\textbf{Sfida} 4 (1.100 PE)

\emph{\textbf{Due Teste.}} L'ettin ha +1d6 alle prove di Saggezza (Consapevolezza) e sui Tiri Salvezza contro le condizioni accecato, affascinato, assordato, privo di sensi, spaventato e stordito.

\emph{\textbf{Veglia.}} Quando una delle due teste dell'ettin è addormentata, l'altra è sveglia.

\textbf{Azioni}

\emph{\textbf{Multiattacco.}} L'ettin effettua due attacchi: uno con l'ascia da battaglia e uno con la mazza chiodata.

\emph{\textbf{Ascia da Battaglia.} Attacco con arma da mischia}: +7 a colpire, portata 1 m, un bersaglio.

\emph{Colpisce:} 14 (2d8 + 5) danni taglienti.

\emph{\textbf{Mazza Chiodata.} Attacco con arma da mischia}: +7 a colpire, portata 1 m, un bersaglio.

\emph{Colpisce:} 14 (2d8 + 5) danni perforanti.

\textbf{Ecologia}\\
Ambiente: Colline fredde\\
Organizzazione: Solitario, coppia, gruppo (3-6), truppa (1-2 più 1-2 Orsi Bruni, banda (3-6 più 1-2 Orsi Bruni) o colonia (3-6 più 1-2 Orsi Bruni e 7-12 Orchi, o 9-16 Goblin)\\
Tesoro: Standard (Armatura di Cuoio, 2 Mazzafrusti Leggeri, 4 Giavellotti, altro tesoro)\\
\textbf{Descrizione}\\
Gli ettin, o giganti a due teste, sono cacciatori notturni malevoli e imprevedibili. Le due teste gli concedono impareggiabili poteri di percezione, facendone dei guardiani eccellenti.\\
Gli ettin sembrano Giganti delle Colline o Giganti delle Rocce, ma il volto zannuto tradisce una discendenza orchesca. Hanno pelle marrone rosata e non si lavano mai se non vi sono costretti, cosa che li rende così sporchi e sudici che la loro pelle sembra spessa e grigia.\\
Gli adulti sono alti 3,9 metri e pesano 2.600 kg. Gli ettin vivono circa 75 anni.\\
Gli ettin non hanno un loro linguaggio ma parlano un gergo misto di Gigante, Goblin e Orchesco. Le creature che parlano uno qualsiasi di questi linguaggi possono comunicare con un ettin effettuando una prova di Intelligenza con DC 15. La prova si effettua una volta per ogni frammento di informazione; se l'altra creatura parla due di questi linguaggi la DC è 10, mentre per qualcuno che li parla tutti e tre è 5.\\
Sebbene gli ettin non siano molto intelligenti, sono guerrieri astuti. Preferiscono tendere imboscate alle loro vittime anziché ingaggiarle in combattimento, ma una volta che la battaglia è cominciata, un ettin combatte furiosamente fino alla morte del nemico.\\
Gli ettin sono creature solitarie, si stabiliscono nella sicurezza di cave rocciose e cavità, spesso circondate da buche e fossi, e tengono a volte degli orsi delle caverne come animali da compagnia o guardiani.\\
Un ettin particolarmente potente può attrarre un gruppo di seguaci, specie Goblin o Orchi. Comunque, questi assembramenti sono più che altro delle eccezioni, e raramente durano a lungo, con gli individualisti ettin che vanno per la loro strada appena le opportunità di saccheggio e rapina diminuiscono o se il capo viene ucciso.\\
In genere formano delle coppie riproduttive per allevare la prole solo per brevi periodi prima di riprendere ognuno la propria strada. I giovani ettin maturano rapidamente, raggiungendo la taglia adulta in un anno, potendo così provvedere a se stessi.\\

\textbf{Ecologia}
Ambiente: qualsiasi\\
Organizzazione: solitario\\
Tesoro: equipaggiamento da PNG\\
\textbf{Descrizione}\\
Quando ad un'anima non è concesso il riposo a causa di qualche grave ingiustizia, vera o presunta, a volte essa torna come fantasma. Questi esseri sono eternamente angosciati, privi di sostanza e incapaci di rimettere le cose a posto. Sebbene i fantasmi possano avere qualsiasi Tratto, molti si aggrappano al mondo dei viventi con un forte senso di odio e rabbia, e come risultato diventano  malvagi; anche una creatura buona dopo morta può diventare un fantasma odioso e crudele.\\

Più di altri mostri, il fantasma deve avere un background ben delineato. Perché questo personaggio è diventato un fantasma? Quali leggende lo circondano? Un incontro con un fantasma non dovrebbe mai avvenire in modo accidentale: ci sono molti altri non morti incorporei, come Wraith e Spettri, per questo. Un incontro adeguato con un fantasma dovrebbe avvenire in una scena al culmine di un lungo periodo di tensione costruito con servitori minori o manifestazioni di spiriti non morti. L'esempio di fantasma sopra rappresenta una principessa umana assassinata da un amante infedele; dopo un confronto, lui la legò con delle catene e la gettò nel pozzo del castello, dove morì annegata. Le capacità del fantasma sono state selezionate in base al background, mostrando come si possa creare un potente antagonista. Applicando l'archetipo a creature con livelli e quindi Abilità proprie o con capacità razziali significative si possono creare fantasmi molto più potenti.\\

Quando viene creato un fantasma, questi ottiene le "copie" degli oggetti a cui in vita dava particolare valore (a condizione che gli originali non siano in possesso di altre creature). L'equipaggiamento funziona normalmente per il fantasma ma passa attraverso gli oggetti o le creature materiali. Un'arma +1 o con un potenziamento superiore, tuttavia, può danneggiare le creature materiali, ma tali attacchi infliggono la metà dei danni (50\%) a meno che non sia un'arma del tocco fantasma. Un fantasma può usare scudi e armature solo se hanno la capacità Tocco Fantasma.\\

Gli oggetti originali vengono lasciati indietro, proprio come le spoglie fisiche del fantasma. Se un'altra creatura impugna l'originale, la copia incorporea svanisce. Questa perdita fa inevitabilmente infuriare il fantasma, che non si ferma davanti a nulla per riportare l'oggetto nel posto in cui giaceva originariamente (e riguadagnarne l'utilizzo).\\

\medskip\index{Mostri - Fantasma}\textbf{Fantasma}

\emph{Media non morto, qualsiasi allineamento}

\textbf{FORZA} -2

\textbf{DESTREZZA} +1

\textbf{COSTITUZIONE} +0

\textbf{INTELLIGENZA} +0

\textbf{SAGGEZZA} +1

\textbf{CARISMA} +3

\textbf{Iniziativa} +1 -- \textbf{Difesa} 13

\textbf{Punti Ferita} 45 (10d8)

\textbf{Movimento} 0 m, volo 12 m (fluttua)

\textbf{Tiri Salvezza} Tempra +7, Riflessi +6, Volontà +7

\textbf{Resistenze al Danno} acido, fulmine, fuoco, tuono; da botta, perforante, tagliente di attacchi non magici

\textbf{Immunità ai Danni} freddo, da Vuoto, veleno

\textbf{Immunità alle Condizioni} affascinato, afferrato, avvelenato, intralciato, paralizzato, pietrificato, prono, affaticamento, spaventato

\textbf{Sensi} scurovisione 18 m

\textbf{Linguaggi} qualsiasi lingua conosciuta in vita

\textbf{Sfida} 4 (1.100 PE)

\emph{\textbf{Movimento Incorporeo.}} Il fantasma può attraversare altre creature e oggetti come se fossero terreno difficile. Subisce 5 (1d10) danni da forza se termina il suo turno all'interno di un oggetto. 

\emph{\textbf{Natura Non Morta.}} Il fantasma non ha bisogno di aria, cibo, bevande o di dormire.

\emph{\textbf{Vista Eterea.}} Il fantasma può vedere 18 metri nel Piano Etereo quando si trova sul Piano Materiale, e vice versa.

\textbf{Azioni}

\emph{\textbf{Tocco Avvizzente.} Attacco con arma da mischia}: +5 a colpire, portata 1 m, un bersaglio.

\emph{Colpisce:} 17 (4d6 + 3) danni da Vuoto.

\emph{\textbf{Eterealità.}} Il fantasma entra nel Piano Etereo dal Piano Materiale, o vice versa. È visibile sul Piano Materiale mentre è nel Margine Etereo, e vice versa, ma non può interagire con nulla che si trovi sull'altro piano.

\emph{\textbf{Possessione (Ricarica 6).}} Un umanoide, entro 1 metro e visibile al fantasma, deve riuscire un Tiro Salvezza di Volontà CD 13 o venire posseduto dal fantasma; il fantasma poi scompare, e il bersaglio è inabile e perde il controllo del suo corpo. Il fantasma ora controlla il corpo ma non priva il bersaglio della sua consapevolezza. Il fantasma non può essere bersaglio di attacchi, incantesimi, o altri effetti, eccetto quelli che scacciano i non morti, e mantiene il suo allineamento, Intelligenza, Saggezza, Carisma e immunità all'essere affascinato e spaventato. Per il resto usa altrimenti le statistiche del bersaglio posseduto, ma non accede al sapere e competenze del bersaglio.

La possessione dura finché il corpo scende a 0 punti ferita, il fantasma la termina con un'azione bonus, o il fantasma viene scacciato o espulso da un effetto come l'incantesimo \emph{dissolvi il bene e il male}. Quando la possessione termina, il fantasma riappare in uno spazio non occupato entro 1 metro dal corpo. Il bersaglio è immune alla Possessione di questo fantasma per 24 ore dopo aver riuscito il Tiro Salvezza o al termine della possessione.

\emph{\textbf{Viso Orripilante.}} Ogni creatura che non sia non morta, entro 18 metri dal fantasma e che lo possa vedere, deve riuscire un Tiro Salvezza di Volontà CD 13 o essere spaventata per 1 minuto. Se il Tiro Salvezza fallisce di 5 o più, il bersaglio invecchia anche di 1d4 x 10 anni. Un  bersaglio spaventato può ripetere il Tiro Salvezza al termine di ciascun proprio turno, terminando l'effetto per sé, qualora riuscisse il tiro  salvezza. Se il Tiro Salvezza del bersaglio riesce e per lui l'effetto ha fine, il bersaglio è immune al Viso Orripilante del fantasma per le successive 24  ore. L'effetto di invecchiamento può essere invertito con l'incantesimo \emph{ristorare superiore}, ma solo se eseguito entro 24 dall'effetto di  invecchiamento.

\textbf{Ecologia}
Ambiente: qualsiasi\\
Organizzazione: solitario\\
Tesoro: equipaggiamento da PNG\\
\textbf{Descrizione}\\
Quando ad un'anima non è concesso il riposo a causa di qualche grave ingiustizia, vera o presunta, a volte essa torna come fantasma. Questi esseri sono eternamente angosciati, privi di sostanza e incapaci di rimettere le cose a posto. Sebbene i fantasmi possano avere qualsiasi Tratto, molti si aggrappano al mondo dei viventi con un forte senso di odio e rabbia, e come risultato diventano  malvagi; anche una creatura buona dopo morta può diventare un fantasma odioso e crudele.\\

Più di altri mostri, il fantasma deve avere un background ben delineato. Perché questo personaggio è diventato un fantasma? Quali leggende lo circondano? Un incontro con un fantasma non dovrebbe mai avvenire in modo accidentale: ci sono molti altri non morti incorporei, come Wraith e Spettri, per questo. Un incontro adeguato con un fantasma dovrebbe avvenire in una scena al culmine di un lungo periodo di tensione costruito con servitori minori o manifestazioni di spiriti non morti. L'esempio di fantasma sopra rappresenta una principessa umana assassinata da un amante infedele; dopo un confronto, lui la legò con delle catene e la gettò nel pozzo del castello, dove morì annegata. Le capacità del fantasma sono state selezionate in base al background, mostrando come si possa creare un potente antagonista. Applicando l'archetipo a creature con livelli e quindi Abilità proprie o con capacità razziali significative si possono creare fantasmi molto più potenti.\\

Quando viene creato un fantasma, questi ottiene le "copie" degli oggetti a cui in vita dava particolare valore (a condizione che gli originali non siano in possesso di altre creature). L'equipaggiamento funziona normalmente per il fantasma ma passa attraverso gli oggetti o le creature materiali. Un'arma +1 o con un potenziamento superiore, tuttavia, può danneggiare le creature materiali, ma tali attacchi infliggono la metà dei danni (50\%) a meno che non sia un'arma del tocco fantasma. Un fantasma può usare scudi e armature solo se hanno la capacità Tocco Fantasma.\\

Gli oggetti originali vengono lasciati indietro, proprio come le spoglie fisiche del fantasma. Se un'altra creatura impugna l'originale, la copia incorporea svanisce. Questa perdita fa inevitabilmente infuriare il fantasma, che non si ferma davanti a nulla per riportare l'oggetto nel posto in cui giaceva originariamente (e riguadagnarne l'utilizzo).\\


\medskip\index{Mostri - Fauci Gorgoglianti}\textbf{Fauci Gorgoglianti}

\emph{Media aberrazione, neutrale}

\textbf{FORZA} +0

\textbf{DESTREZZA} -1

\textbf{COSTITUZIONE} +3

\textbf{INTELLIGENZA} -4

\textbf{SAGGEZZA} +0

\textbf{CARISMA} -2

\textbf{Iniziativa} -1 -- \textbf{Difesa} 10

\textbf{Punti Ferita} 67 (9d8 + 27)

\textbf{Movimento} 3 m, nuoto 3 m

\textbf{Tiri Salvezza} Tempra +8, Riflessi +4, Volontà +5

\textbf{Immunità alle Condizioni} prono

\textbf{Sensi} scurovisione 18 m

\textbf{Linguaggi} -

\textbf{Sfida} 2 (450 PE)

\emph{\textbf{Gorgoglio.}} Finché la fauce è in grado di vedere una creatura e non è inabile, pronuncia frasi incoerenti. Ogni creatura che inizi il suo turno entro 6 metri dalla fauce e può udire il suo gorgoglio deve effettuare un Tiro Salvezza di Volontà CD 10. Se lo fallisce, la creatura non può effettuare reazioni fino all'inizio del suo prossimo turno e tira un d8 per determinare cosa farà durante il proprio turno. Da 1 a 4, la creatura non fa nulla. Con 5 o 6, la creatura non svolge nessun'azione o azione bonus e usa tutto il suo movimento per muoversi in una direzione determinata casualmente. Con 7 o 8, la creatura effettua un attacco da mischia contro una creatura determinata a caso entro la sua portata o non fa nulla se non è in grado di effettuare un simile attacco.

\emph{\textbf{Terreno Aberrante.}} Il terreno in un raggio di 3 metri intorno alla fauce è considerato terreno difficile. Ogni creatura che inizi il suo turno in quell'area deve riuscire un Tiro Salvezza di Tempra CD 10 o vedere il suo movimento ridotto a 0 fino all'inizio del suo turno successivo.

\textbf{Azioni}

\emph{\textbf{Multiattacco.}} La fauce gorgogliante effettua un attacco di morso e, se può, uno Sputo Accecante.

\emph{\textbf{Morso.} Attacco con arma da mischia}: +2 a colpire, portata 1 m, una creatura.

\emph{Colpisce:} 17 (5d6) danni perforanti. Se il bersaglio è di taglia Media o inferiore, deve riuscire un Tiro Salvezza di Tempra CD 10 o venir gettato prono. Se il bersaglio viene ucciso da questo danno, viene assorbito dalla fauce.

\emph{\textbf{Sputo Accecante (Ricarica 5-6).}} La fauce sputa un globo chimico ad un punto visibile entro 5 metri da essa. Il globo esplode all'impatto in un lampo accecante di luce. Ogni creatura entro 1 metro dal lampo deve riuscire un Tiro Salvezza di Riflessi CD 13 o restare accecata fino al termine del prossimo turno della fauce.

\textbf{Ecologia}\\
Ambiente: Qualsiasi Sotterraneo\\
Organizzazione: Solitario\\
Tesoro: Standard\\
\textbf{Descrizione}\\
Disgustosa, nauseante e affamata: queste sono le uniche parole che descrivono in modo appropriato la fauce gorgogliante. Bestie ripugnanti che si nascondono nelle grotte, nelle fogne e negli incubi, le fauci non hanno altro senso sociale, ecologico o religioso diverso dalla loro capacità di far impazzire coloro che le ascoltano. Alcuni studiosi credono che le fauci gorgoglianti siano una variante più piccola del molto più pericoloso shoggoth, mentre altri teorizzano che sia una punizione di qualche potente entità o divinità inflitta a coloro che l'hanno offesa.\\




\subsection{Funghi}

\medskip\index{Mostri - Fungo Stridente}\textbf{Fungo Stridente}

\emph{Media pianta, disallineato}

\textbf{FORZA} -5

\textbf{DESTREZZA} -5

\textbf{COSTITUZIONE} +0

\textbf{INTELLIGENZA} -5

\textbf{SAGGEZZA} -4

\textbf{CARISMA} -5

\textbf{Iniziativa} -5 -- \textbf{Difesa} 6

\textbf{Punti Ferita} 13 (3d8)

\textbf{Movimento} 0 m

\textbf{Tiri Salvezza}: Fort -3, Riflessi +3, Will -4

\textbf{Immunità alle Condizioni} accecato, assordato, spaventato

\textbf{Sensi} vista cieca 9 m (cieco oltre questo raggio)

\textbf{Linguaggi} -

\textbf{Sfida} 0 (10 PE)

\emph{\textbf{Falso Aspetto.}} Mentre il fungo stridente rimane immobile, è indistinguibile da un normale fungo.

\textbf{Azioni}

\emph{\textbf{Strillo.}} Quando una luce intensa o una creatura si trova entro 9 metri dal fungo stridente, esso emette un strillo udibile fino a 90 metri di distanza. Il fungo stridente continua a strillare finché la fonte del disturbo non si è portata fuori gittata e per altri 1d4 turni successivi.

\textbf{Ecologia}\\
Ambiente: Qualsiasi sotterraneo\\
Organizzazione: Solitario, coppia o macchia (3-12)\\
Tesoro: Accidentale\\
\textbf{Descrizione}\\
Un fungo stridente e' alto circa 50 cm, dall'ampio cappello marrone. Una volta emesso l'urlo il cappello si sgonfia. \\
Si racconta di Cuochi Duergar specializzati nel cuocere questi funghi in pietanze sopraffine.
I piu' bravi riescono anche a non fare sgonfiare il cappello.\\


\medskip\index{Mostri - Fungo Violetto}\textbf{Fungo Violetto}

\emph{Media pianta, disallineato}

\textbf{FORZA} -4

\textbf{DESTREZZA} -5

\textbf{COSTITUZIONE} +0

\textbf{INTELLIGENZA} -5

\textbf{SAGGEZZA} -4

\textbf{CARISMA} -5

\textbf{Iniziativa} -5 -- \textbf{Difesa} 6

\textbf{Punti Ferita} 18 (4d8)

\textbf{Movimento} 2 m

\textbf{Tiri Salvezza}: Fort -3, Ref -3, Will -3

\textbf{Immunità alle Condizioni} accecato, assordato, spaventato 

\textbf{Sensi} vista cieca 9 m (cieco oltre questo raggio)

\textbf{Linguaggi} -

\textbf{Sfida} 1/4 (50 PE)

\emph{\textbf{Falso Aspetto.}} Mentre il fungo violetto rimane immobile, è indistinguibile da un normale fungo.

\textbf{Azioni}

\emph{\textbf{Multiattacco.}} Il fungo effettua 1d4 attacchi con Contatto Putrido.

\emph{\textbf{Contatto Putrido.} Attacco con arma da mischia}: +2 a colpire,  portata 3 m, un bersaglio.

\emph{Colpisce:} 4 (1d8) danni da Vuoto.

\textbf{Ecologia}\\
Ambiente: Qualsiasi sotterraneo\\
Organizzazione: Solitario, coppia o macchia (3-12)\\
Tesoro: Accidentale\\
\textbf{Descrizione}\\
I funghi viola sono uno dei più noti e temuti pericoli delle caverne. Un viaggiatore può spesso notare i segni lasciati dal fungo viola su coloro che vivono o cacciano nei luoghi in cui questi funghi carnivori si appostano. Queste profonde e orribili cicatrici sembrano solchi scavati nella carne: i segni di un incontro ravvicinato con un fungo viola.\\
Un fungo viola si nutre della materia organica putrefatta, ma a differenza della maggioranza dei funghi non è un consumatore passivo. I viticci di un fungo viola possono colpire con inaspettata rapidità e sono ricoperti di un veleno virulento che causa la putrefazione delle carni con nauseante velocità. Questo potente veleno, se trascurato, può far marcire rapidamente un intero braccio o una gamba, lasciandosi dietro solo ossa che presto si corroderanno anch'esse.\\
Sebbene i funghi viola possano muoversi, lo fanno solo per attaccare o cacciare la preda. Un fungo viola con un flusso regolare di putredine di cui nutrirsi si accontenta di restare in un posto. Molti abitanti del sottosuolo, in particolare Trogloditi e Vegepigmei, sfruttano questo comportamento a loro vantaggio e posizionano molteplici funghi viola in giunzioni ed entrate chiave delle loro caverne come guardiani, assicurandosi di fornire loro cadaveri a sufficienza per evitare che si addentrino nel rifugio in cerca di cibo.\\
Alcune specie di Boleto Stridente hanno un aspetto piuttosto simile a quello dei funghi viola, sebbene manchino di ramificazioni tentacolari. Non è strano trovare boleti stridenti e funghi viola nello stesso groviglio, specialmente nelle aree dove altre creature coltivano questi funghi come guardiani.\\

Un fungo viola è alto 1,2 metri e pesa 25 kg.\\


\medskip\index{Mostri - Fuoco Fatuo}\textbf{Fuoco Fatuo}

\emph{Minuscola non morto, caotico malvagio}

\textbf{FORZA} -5

\textbf{DESTREZZA} +9

\textbf{COSTITUZIONE} +0

\textbf{INTELLIGENZA} +1

\textbf{SAGGEZZA} +2

\textbf{CARISMA} +0

\textbf{Iniziativa} +9 -- \textbf{Difesa} 20

\textbf{Punti Ferita} 22 (9d4)

\textbf{Movimento} 0 m, volo 15 m (fluttua)

\textbf{Tiri Salvezza}: Tempra +3, Riflessi +12, Volontà +9

\textbf{Immunità ai Danni} fulmine, veleno

\textbf{Resistenze al Danno} acido, freddo, fuoco, da Vuoto, tuono; contendente, perforante e tagliente di attacchi non magici

\textbf{Immunità alle Condizioni} afferrato, avvelenato, intralciato, paralizzato, privo di sensi, prono, affaticamento

\textbf{Sensi} scurovisione 36 m

\textbf{Linguaggi} le lingue che conosceva in vita

\textbf{Sfida} 2 (450 PE)

\emph{\textbf{Consumare Vita.}} Con un'azione bonus, il fuoco fatuo può prendere a bersaglio una creatura che può vedere entro 1 metro da esso e che abbia 0 punti ferita e sia ancora in vita. Il bersaglio deve riuscire un Tiro Salvezza di Tempra CD 10 contro questa magia o morire. Se il bersaglio muore, il fuoco fatuo recupera 10 (3d6) punti ferita.

\emph{\textbf{Effimero.}} Il fuoco fatuo non può indossare né trasportare nulla.

\emph{\textbf{Illuminazione Variabile.}} Il fuoco fatuo promana luce intensa in un raggio da 1 a 6 metri e luce fioca per un numero di metri aggiuntivi pari al raggio scelto. Il fuoco fatuo può modificare questo raggio con un'azione bonus.

\emph{\textbf{Movimento Incorporeo.}} Il fuoco fatuo può muoversi attraverso altre creature e oggetti come se fossero terreno difficile. Subisce 5 (1d10) danni da forza se termina il suo turno all'interno di un oggetto.

\emph{\textbf{Natura Non Morta.}} Il fuoco fatuo non ha bisogno di aria, cibo o bevande.

\textbf{Azioni}

\emph{\textbf{Scossa.} Attacco con incantesimo in mischia}: +4 a colpire, portata 1 m, una creatura.

\emph{Colpisce:} 9 (2d8) danni da fulmine.

\emph{\textbf{Invisibilità.}} Il fuoco fatuo e la sua luce diventano magicamente invisibili finché non attacca o usa Consumare Vita, o finché la sua concentrazione non termina (come se si stesse concentrando su di un incantesimo).

\textbf{Ecologia}
Ambiente: Qualsiasi Palude\\
Organizzazione: Solitario, coppia o sequenza (3-4)\\
Tesoro: Accidentale\\
\textbf{Descrizione}\\
Ogni cacciatore e agricoltore che viva vicino ad un acquitrino o a una palude ha dato un nome a queste sfere di luce fioca: jack lanterna, candele dei defunti, fuochi che camminano, luci dei pini, luci fantasma, luci di giunco; ma tutti sanno che si tratta di pericolosi predatori e false guide nell'oscurità.\\

Malvagie creature che si nutrono delle forti emanazioni psichiche delle creature terrorizzate, i fuochi fatui traggono piacere nel mettere i viaggiatori creduloni in situazioni pericolose. Nelle terre selvagge, dove sono molto comuni, i fuochi fatui preferiscono tattiche semplici come posizionarsi su scogli o sabbie mobili dove possono essere scambiati facilmente per lanterne (specialmente se possono predisporre la trappola nei pressi di vere lanterne di segnalazione), così da attirare i viaggiatori verso il pericolo. In rare occasioni, i fuochi fatui in cerca di vita facile si spostano in una città e si stabiliscono vicino ai patiboli o seguono, invisibili, un'armata, così da nutrirsi della paura degli uomini morenti; perché la stragrande maggioranza scelga di rimanere nelle paludi, dove le vittime scarseggiano, rimane un mistero.\\

I fuochi fatui possono contare solo sulla loro scossa elettrica in situazioni pericolose, quindi preferiscono lasciare che altre creature o pericoli si occupino delle loro vittime mentre loro fluttuano nelle vicinanze e banchettano.\\

I fuochi fatui possono brillare di qualunque colore desiderino, ma sono più spesso gialli, bianchi, verdi o blu. Possono anche variare la loro luminosità per creare un disegno: molti fuochi fatui amano creare forme che somigliano vagamente a teschi nella loro luminescenza per aumentare il terrore nelle loro vittime. I loro veri corpi sono globi di materiale spugnoso traslucido appena visibili di circa 30 centimetri che pesano 1,5 kg e possono emettere luce su tutta la loro superficie. La luce dei fuochi fatui brilla approssimativamente come una torcia, e sebbene non sembrino utilizzare suoni per comunicare, sentono perfettamente e possono far vibrare i loro corpi così rapidamente da imitare il linguaggio.\\

Nonostante siano denigrati dalla maggioranza delle creature senzienti, i fuochi fatui sono in realtà alquanto intelligenti, sebbene ragionino in modo completamente alieno. A volte si organizzano in gruppi chiamati "sequenze"; la loro società e i loro scopi rimangono completamente sconosciuti, così come le loro origini, sebbene talvolta siano noti per stringere patti con chi offre loro una grande quantità di vittime adeguatamente terrorizzate.\\

I fuochi fatui non hanno età e sono di fatto immortali, a meno che non muoiano di morte violenta; i fuochi fatui più antichi possono essere ottimi depositari di conoscenze del passato, sebbene convincere una di queste crudeli creature a cooperare possa essere piuttosto complicato.\\


\medskip\index{Mostri - Fustigatore}\textbf{Fustigatore}

\emph{Grande mostruosità, neutrale malvagio}

\textbf{FORZA} +4

\textbf{DESTREZZA} -1

\textbf{COSTITUZIONE} +3

\textbf{INTELLIGENZA} -2

\textbf{SAGGEZZA} +3

\textbf{CARISMA} -2

\textbf{Iniziativa} -1 -- \textbf{Difesa} 23

\textbf{Punti Ferita} 93 (11d10 + 33)

\textbf{Movimento} 3 m, scalata 3 m

\textbf{Tiri Salvezza}: Tempra +13, Riflessi +5, Volontà +13

\textbf{Competenze} Muoversi Silenziosamente / Nascondersi +5, Consapevolezza +6

\textbf{Sensi} scurovisione 18 m

\textbf{Linguaggi} -

\textbf{Sfida} 5 (1.800 PE)

\emph{\textbf{Falso Aspetto.}} Quando il fustigatore rimane immobile, è indistinguibile da una normale formazione rocciosa, come una stalagmite.

\emph{\textbf{Scalare come Ragno.}} Il fustigatore può scalare superfici difficili, compreso lo stare a testa in giù sul soffitto, senza bisogno di effettuare una prova di abilità.

\emph{\textbf{Viticci Afferranti.}} Il fustigatore può avere fino a sei viticci alla volta. Ogni viticcio può essere attaccato (CA 20; 10 punti ferita; immunità ai danni psichici e da veleno). Distruggere un viticcio non infligge danni al fustigatore, che può produrre un viticcio di rimpiazzo nel suo prossimo turno. Un viticcio può essere anche rotto se una creatura effettua un'azione e riesce una prova di Forza CD 15 contro di esso.

\textbf{Azioni}

\emph{\textbf{Multiattacco.}} Il fustigatore può effettuare quattro attacchi con i suoi viticci, usare avvolgere e effettuare un attacco con il morso.

\emph{\textbf{Morso.} Attacco con arma da mischia}: +7 a colpire, portata 1 m, un bersaglio.

\emph{Colpisce:} 22 (4d8 + 4) danni perforanti.

\emph{\textbf{Viticcio.} Attacco con arma da mischia}: +7 a colpire, portata 15 m, una creatura.

\emph{Colpisce:} Il bersaglio è afferrato (CD 15 per fuggire). Fino al termine dell'afferrare, il bersaglio è intralciato e ha -1d6 alle prove di Forza e ai Tiri Salvezza su Tempra, mentre il fustigatore non può usare lo stesso viticcio contro un altro bersaglio.

\emph{\textbf{Avvolgere.}} Il fustigatore trascina le creature afferrate da lui di 7 metri verso di lui.

\textbf{Ecologia}
Ambiente: Qualsiasi Sotterraneo\\
Organizzazione: Solitario, coppia o gruppo (3-6)\\
Tesoro: Standard\\
\textbf{Descrizione}\\
Il fustigatore è un cacciatore da agguato. Capace di modificare la colorazione e la forma del suo corpo, un fustigatore nascosto sembra una stalagmite di pietra o ghiaccio (o in luoghi dal soffitto basso, una colonna di pietra o ghiaccio). Nelle aree prive di questi tratti per nascondersi un fustigatore può comprimere il suo corpo fino a sembrare un masso. Le sferze che può estroflettere non sono di carne ma di uno spesso materiale semiliquido simile a cera parzialmente fusa ma con la resistenza di una catena di ferro e la capacità di intirizzire la carne e indebolire le forze. Il fustigatore può usare queste sferze con grande maestria e farle volare fino a 15 metri per rubare gli oggetti che attraggono la sua attenzione.\\

Nonostante la sua forma aliena e mostruosa, il fustigatore è uno degli abitanti più intelligenti del sottosuolo. Non formano vaste società (anche se spesso si trovano a vivere insieme ad altre creature del sottosuolo come Divoracervelli e Neothelid, con cui a volte si alleano), ma spesso si aggregano in piccoli gruppi. Particolarmente interessato alla filosofia della vita e della morte, e agli aspetti più sottili delle religioni più sinistre e crudeli del mondo, un fustigatore può parlare o discutere per ore con quelli che inizialmente aveva semplicemente cercato di mangiare. Alcune storie parlano di oratori e filosofi particolarmente dotati che sono stati tenuti per giorni o anche anni come animali domestici o compagni di conversazione da gruppi di fustigatori; alla fine, però, se non riescono a scappare, l'appetito dei fustigatori finisce per avere la meglio sulla loro curiosa intelligenza, specialmente nei casi in cui questi animali da compagnia superano costantemente l'arguzia e la pazienza dei loro guardiani.\\
Un fustigatore è alto 2,7 metri e pesa 1.100 kg.\\


\medskip\index{Mostri - Gargoyle}\textbf{Gargoyle}

\emph{Media elementale, caotico malvagio}

\textbf{FORZA} +2

\textbf{DESTREZZA} +0

\textbf{COSTITUZIONE} +3

\textbf{INTELLIGENZA} -2

\textbf{SAGGEZZA} +0

\textbf{CARISMA} -2

\textbf{Iniziativa} +0 -- \textbf{Difesa} 16

\textbf{Punti Ferita} 52 (7d8 + 21)

\textbf{Movimento} 9 m, volo 18 m

\textbf{Tiri Salvezza}: Tempra +4, Riflessi +6, Volontà +4

\textbf{Resistenze al Danno} da botta, perforante e tagliente di attacchi non magici o che non siano di adamantio

\textbf{Immunità ai Danni} veleno

\textbf{Immunità alle Condizioni} avvelenato, pietrificato, affaticamento

\textbf{Sensi} scurovisione 18 m

\textbf{Linguaggi} Terran

\textbf{Sfida} 2 (450 PE)

\emph{\textbf{Falso Aspetto.}} Mentre la gargoyle rimane immobile, è indistinguibile da una statua inanimata.

\emph{\textbf{Natura Elementale.}} Una gargoyle non ha bisogno di aria, cibo, bevande o sonno.

\textbf{Azioni}

\emph{\textbf{Multiattacco.}} La gargoyle effettua due attacchi: uno con il morso e uno con gli artigli.

\emph{\textbf{Artigli.} Attacco con arma da mischia}: +4 a colpire, portata 1 m, un bersaglio.

\emph{Colpisce:} 5 (1d6 + 2) danni taglienti.

\emph{\textbf{Morso.} Attacco con arma da mischia}: +4 a colpire, portata 1 m, un bersaglio.

\emph{Colpisce:} 5 (1d6 + 2) danni perforanti.

\emph{Colpisce:} 5 (1d6 + 2) danni perforanti.\\
\textbf{Ecologia}
Ambiente: Qualsiasi\\
Organizzazione: Solitario, coppia o stormo (3-12)\\
Tesoro: Standard\\
\textbf{Descrizione}\\
I gargoyle spesso sembrano essere statue alate di pietra, poiché possono rimanere immobili indefinitamente per poi sorprendere i nemici. I gargoyle tendono a comportamenti ossessivo-compulsivi, tanto diversi quanto abbondante è la loro specie. Libri, ninnoli rubati, armi e trofei raccolti dai nemici caduti sono solo alcuni esempi dei tipi di oggetti che un gargoyle può collezionare per decorare la sua tana e il suo territorio.\\

I gargoyle tendono ad avere uno stile di vita solitario, anche se a volte formano temibili stormi detti "ali" per protezione e divertimento. In certe condizioni, una tribù di gargoyle può persino allearsi con altre creature, ma anche la più stabile di queste alleanze può crollare per ragioni infime; i gargoyle sono solo traditori, meschini e vendicativi.\\

I gargoyle sono noti per abitare nel cuore delle città più grandi, accovacciati tra le decorazioni di pietra delle cattedrali e degli edifici dove si nascondono in bella vista di giorno piombando giù per nutrirsi di vagabondi, mendicanti e altri sfortunati la notte.\\

Più a lungo una tribù di gargoyle dimora in un'area di edifici o rovine, più i suoi membri cominciano ad assomigliare allo stile architettonico della zona. I cambiamenti subiti dall'aspetto di un gargoyle sono lenti e sottili, ma nel corso degli anni possono diventare radicali.\\

Un'insolita variante del gargoyle non abita tra edifici e rovine ma sotto le onde del mare. Queste creature sono note come kapoacinth; hanno le stesse statistiche base dei gargoyle normali, eccetto che hanno il sottotipo acquatico e le loro ali gli garantiscono una velocità di nuotare di 12 metri (ma sono inutili per volare). I kapoacinth abitano nelle regioni costiere poco profonde dove possono strisciare fuori dalla spuma per dare la caccia ai residenti della zona. È più probabile che formino stormi, poiché i kapoacinth preferiscono la vita di gruppo a quella solitaria.\\



\subsection{Geni}

\medskip\index{Mostri - Djinni}\textbf{Djinni}

\emph{Grande elementale, caotico buono}

\textbf{FORZA} +5

\textbf{DESTREZZA} +2

\textbf{COSTITUZIONE} +6

\textbf{INTELLIGENZA} +2

\textbf{SAGGEZZA} +3

\textbf{CARISMA} +5

\textbf{Iniziativa} +2 -- \textbf{Difesa} 23

\textbf{Punti Ferita} 161 (14d10 + 84)

\textbf{Movimento} 9 m, volo 27 m

\textbf{Tiri Salvezza} Tempra +4, Riflessi +9, Volontà +7

\textbf{Immunità al Danno} fulmine, tuono

\textbf{Sensi} scurovisione 36 m

\textbf{Linguaggi} Auran

\textbf{Sfida} 11 (7.200 PE)

\emph{\textbf{Decesso Elementale.}} Se il djinni muore, il suo corpo si disintegra in una brezza calda, lasciando dietro di sé solo l'equipaggiamento che il djinni stava indossando o trasportando.

\emph{\textbf{Incantesimi Innati.}} La caratteristica da incantatore innato del djinni è il Carisma 17, +9 a colpire con attacchi da incantesimo). Può lanciare in maniera innata i seguenti incantesimi, senza bisogno di componenti materiali:

A volontà: \emph{individuazione del bene e del male, individuazione del magico, onda tonante}

3/giorno ciascuno: \emph{camminare nel vento, creare cibo e acqua} (può creare vino al posto dell'acqua), \emph{linguaggi}

1/giorno ciascuno: \emph{creazione}, \emph{evoca elementali} (solo elementale dell'aria), \emph{forma gassosa, immagine maggiore}, \emph{invisibilità,} \emph{spostamento planare}

\textbf{Azioni}

\emph{\textbf{Multiattacco.}} Il djinni effettua tre attacchi di
scimitarra.

\emph{\textbf{Scimitarra.} Attacco con arma da mischia}: +9 a colpire, portata 1 m, un bersaglio.

\emph{Colpisce:} 12 (2d6 + 5) danni taglienti più 3 (1d6) danni da fulmine o tuono (a scelta del gin).

\emph{\textbf{Creare Turbine.}} Un cilindro d'aria turbinante di 1 metro di raggio e alto 9 metri si forma magicamente in un punto visibile al djinni entro 36 metri da esso. Il turbine resta finché il djinni mantiene la concentrazione (come se si stesse concentrando su di un incantesimo). Qualsiasi creatura salvo il djinni che entri nel turbine deve riuscire un Tiro Salvezza di Tempra CD 18 o restare intralciata da esso. Il djinni può muovere il turbine di massimo 18 metri con un'azione, e le creature intralciate dal turbine si muovono con esso. Il turbine termina se il djinni lo perde di vista.

Una creatura può usare la sua azione per liberare una creatura intralciata dal turbine, compresa se stessa, riuscendo una prova di Forza CD 18. Se la prova riesce, la creatura non è più intralciata e si sposta nello spazio più vicino all'esterno del turbine.

\textbf{Ecologia}
Ambiente: Qualsiasi (Piano dell'Aria)\\
Organizzazione: Solitario, coppia, compagnia (3-6) o banda (7-10)\\
Tesoro: Standard (Scimitarra Perfetta, altro tesoro)\\
\textbf{Descrizione}\\
I Djinn (singolare djinni) sono Geni provenienti dal Piano Elementale dell'Aria. Si dice che siano fatti di nuvole e abbiano la forza delle tempeste più potenti. Un Djinni è alto circa 3 metri e pesa circa 500 kg.\\

I Djinn disdegnano il Combattimento fisico, preferendo usare i loro poteri Magici e capacità aeree contro i nemici. Un Djinni sconfitto in Combattimento generalmente prende il volo e diventa un turbine per molestare chi lo insegue. Quando non ha altra scelta che combattere in mischia, la maggioranza dei Djinn preferisce impugnare Scimitarre a Due Mani Perfette.\\

Verso gli altri Geni, i Djinn vanno d'accordo con gli Janni e i Marid. Sono frequentemente in contrasto con gli Shaitan, e sono nemici giurati degli Efreeti, disprezzando questi Geni feroci più di qualsiasi altra delle Razze di Geni. Il conflitto tra gli Efreeti e i Djinn è così leggendario che molti incantatori tentano (con vari gradi di successo) di assicurarsi il servizio di un Djinni promettendogli aiuto nella causa contro gli odiati nemici.\\


\medskip\index{Mostri - Efreeti}\textbf{Efreeti}

\emph{Grande elementale, legale malvagio}

\textbf{FORZA} +6

\textbf{DESTREZZA} +1

\textbf{COSTITUZIONE} +7

\textbf{INTELLIGENZA} +3

\textbf{SAGGEZZA} +2

\textbf{CARISMA} +3

\textbf{Iniziativa} +3 -- \textbf{Difesa} 23

\textbf{Punti Ferita} 200 (16d10 + 112)

\textbf{Movimento} 12 m, volo 18 m

\textbf{Tiri Salvezza} Tempra +7, Riflessi +10, Volontà +9

\textbf{Immunità al Danno} fuoco

\textbf{Sensi} scurovisione 36 m

\textbf{Linguaggi} Ignan

\textbf{Sfida} 11 (7.200 PE)

\emph{\textbf{Decesso Elementale.}} Se l'efreeti muore, il suo corpo si disintegra in un lampo di fuoco e uno sbuffo di fumo, lasciando dietro di sé solo l'equipaggiamento che l'efreeti stava indossando o trasportando.

\emph{\textbf{Incantesimi Innati.}} La caratteristica da incantatore innato dell'efreeti è il Carisma, +7 a colpire con attacchi da incantesimo). Può lanciare in maniera innata i seguenti incantesimi, senza bisogno di componenti materiali: 

A volontà: \emph{individuazione del magico}

3/giorno ciascuno: \emph{ingrandire/ridurre, linguaggi}

1/giorno ciascuno: \emph{evoca elementali} (solo elementale del fuoco), \emph{forma gassosa, immagine maggiore}, \emph{invisibilità, muro di fuoco, spostamento planare}

\textbf{Azioni}

\emph{\textbf{Multiattacco.}} L'efreeti effettua due attacchi di scimitarra o usa due volte Scagliare Fiamma.

\emph{\textbf{Scimitarra.} Attacco con arma da mischia}: +10 a colpire, portata 1 m, un bersaglio.

\emph{Colpisce:} 13 (2d6 + 6) danni taglienti più 7 (2d6) danni da fuoco.

\emph{\textbf{Scagliare Fiamma.} Attacco con arma a Distanza}: +7 a colpire, gittata 36 m, un bersaglio.

\emph{Colpisce:} 17 (5d6) danni da fuoco.

\textbf{Ecologia}
Ambiente: Qualsiasi (Piano del Fuoco)\\
Organizzazione: Solitario, coppia, compagnia (3-6) o banda (7-12)\\
Tesoro: Standard (Falchion Perfetto, altro tesoro)\\
\textbf{Descrizione}\\
Gli Efreet (singolare Efreeti) sono Geni provenienti dal Piano del Fuoco. Sono alti 3,6 metri e pesano circa 1.000 kg.\\
Gli Efreet hanno pochi alleati tra gli altri Geni: odiano i Djinni, e li attaccano a vista, non sopportano i Marid, e vedono i Janni come deboli e fragili. Gli Efreet spesso cooperano bene con gli Shaitan, eppure anche queste alleanze sono temporanee.\\


\subsection{Ghoul}

\medskip\index{Mostri - Ghast}\textbf{Ghast}

\emph{Media non morto, caotico malvagio}

\textbf{FORZA} +3

\textbf{DESTREZZA} +3

\textbf{COSTITUZIONE} +0

\textbf{INTELLIGENZA} +0

\textbf{SAGGEZZA} +0

\textbf{CARISMA} -1

\textbf{Iniziativa} +3 -- \textbf{Difesa} 14

\textbf{Punti Ferita} 36 (8d8)

\textbf{Movimento} 9 m

\textbf{Tiri Salvezza}: Tempra +2, Riflessi +2, Volontà +5

\textbf{Resistenze al Danno} da Vuoto

\textbf{Immunità al Danno} veleno

\textbf{Immunità alle Condizioni} affascinato, avvelenato, affaticamento

\textbf{Sensi} scurovisione 18 m

\textbf{Linguaggi} Comune

\textbf{Sfida} 2 (450 PE)

\emph{\textbf{Fetore.}} Qualsiasi creatura che inizi il suo turno entro 1 metro dal ghast deve riuscire un Tiro Salvezza di Tempra CD 10 o restare avvelenata fino all'inizio del suo prossimo turno. Se riesce il Tiro Salvezza, la creatura è immune al Fetore del ghast per le successive 24
ore.

\emph{\textbf{Ribellione allo Scacciare.}} Il ghast e tutti i ghoul entro 9 metri da esso hanno +1d6 ai Tiri Salvezza contro gli effetti che scacciano i non morti.

\textbf{Azioni}

\emph{\textbf{Artigli.} Attacco con arma da mischia}: +5 a colpire, portata 1 m, un bersaglio.

\emph{Colpisce:} 10 (2d6 + 3) danni taglienti. Se il bersaglio è una creatura, diversa da un non morto, deve riuscire un Tiro Salvezza su Tempra CD 10 o restare paralizzata per 1 minuto. Il bersaglio può ripetere il Tiro Salvezza al termine di ciascun suo turno, terminando l'effetto se riesce il Tiro Salvezza. 

\emph{\textbf{Morso.} Attacco con arma da mischia}: +3 a colpire, portata 1 m, una creatura.

\emph{Colpisce:} 12 (2d8 + 3) danni perforanti.

\textbf{Ecologia}\\
Ambiente: Qualsiasi terreno\\
Organizzazione: Solitario, gruppo (2-4) o branco (7-12)\\
Tesoro: Standard\\
\textbf{Descrizione}\\
I ghast sono Ghoul con l'archetipo semplice avanzato. La paralisi di un ghast ha effetto anche sugli Elfi. I ghast si aggirano in branchi o comandano gruppi di Ghoul comuni. Il fetore di morte e putrefazione che circonda queste creature è travolgente.\\


\medskip\index{Mostri - Ghoul}\textbf{Ghoul}

\emph{Media non morto, caotico malvagio}

\textbf{FORZA} +1

\textbf{DESTREZZA} +2

\textbf{COSTITUZIONE} +0

\textbf{INTELLIGENZA} -2

\textbf{SAGGEZZA} +0

\textbf{CARISMA} -2

\textbf{Iniziativa} +2 -- \textbf{Difesa} 13

\textbf{Punti Ferita} 22 (5d8)

\textbf{Movimento} 9 m

\textbf{Tiri Salvezza}: Tempra +1, Riflessi +2, Volontà +4

\textbf{Immunità al Danno} veleno

\textbf{Immunità alle Condizioni} affascinato, avvelenato, affaticamento

\textbf{Sensi} scurovisione 18 m

\textbf{Linguaggi} Comune

\textbf{Sfida} 1 (200 PE)

\textbf{Azioni}

\emph{\textbf{Artigli.} Attacco con arma da mischia}: +4 a colpire, portata 1 m, un bersaglio.

\emph{Colpisce:} 7 (2d4 + 2) danni taglienti. Se il bersaglio è una creatura, diversa da un elfo o un non morto, deve riuscire un Tiro Salvezza su Tempra CD 10 o restare paralizzata per 1 minuto. Il bersaglio può ripetere il Tiro Salvezza al termine di ciascun suo turno, terminando l'effetto se riesce il Tiro Salvezza.

\emph{\textbf{Morso.} Attacco con arma da mischia}: +2 a colpire, portata 1 m, una creatura.

\emph{Colpisce:} 9 (2d6 + 2) danni perforanti.

\textbf{Ecologia}
Ambiente: Qualsiasi terreno\\
Organizzazione: Solitario, gruppo (2-4) o branco (7-12)\\
Tesoro: Standard\\
\textbf{Descrizione}\\
I ghoul sono non morti che frequentano i cimiteri e mangiano i cadaveri. Le leggende sostengono che i primi ghoul fossero umani cannibali che una fame innaturale ha riportato indietro dalla morte, oppure umani che in vita si nutrivano dei resti in decomposizione dei loro simili e che morirono (e poi rinacquero) a causa di un'orrenda malattia; la vera origine di questi non morti necrofagi è incerta.\\
I ghoul si appostano ai margini della civilizzazione (dentro o nei pressi dei cimiteri o nelle fogne cittadine) dove possono reperire ampie scorte del loro cibo preferito. Sebbene preferiscano i corpi in putrefazione e spesso seppelliscano le loro vittime per migliorarne il sapore, mangiano i morti freschi se hanno abbastanza fame.\\

Anche se molti ghoul di superficie vivono in modo primitivo, delle voci parlano di città di ghoul nelle profondità del sottosuolo comandate da sacerdoti che adorano antiche divinità crudeli o strani signori dei demoni della fame. Questi ghoul "civilizzati" non sono meno orribili nelle loro abitudini alimentari, e in effetti il loro concetto di tavola ben imbandita per banchetti è forse anche più orrendo dell'idea di un pasto fresco prelevato da una bara.\\


\subsection{Giganti}

\medskip\index{Mostri - Gigante di Collina}\textbf{Gigante di Collina}

\emph{Enorme gigante, caotico malvagio}

\textbf{FORZA} +5

\textbf{DESTREZZA} -1

\textbf{COSTITUZIONE} +4

\textbf{INTELLIGENZA} -3

\textbf{SAGGEZZA} -1

\textbf{CARISMA} -2

\textbf{Iniziativa} -1 -- \textbf{Difesa} 16

\textbf{Punti Ferita} 105 (10d12 + 40)

\textbf{Movimento} 12 m

\textbf{Tiri Salvezza}: Tempra +11, Riflessi +2, Volontà +3

\textbf{Competenze} Consapevolezza +2

\textbf{Linguaggi} Gigante

\textbf{Sfida} 5 (1.800 PE)

\textbf{Azioni}

\emph{\textbf{Multiattacco.}} Il gigante effettua due attacchi con il randello pesante.

\emph{\textbf{Randello Pesante.} Attacco con arma da mischia}: +8 a colpire, portata 3 m, un bersaglio.

\emph{Colpisce:} 18 (3d8 + 5) danni da botta.

\emph{\textbf{Sasso.} Attacco con arma a Distanza}: +8 a colpire, gittata 18m, un bersaglio.

\emph{Colpisce:} 21 (3d10 + 5) danni da botta.

\textbf{Ecologia}\\
Ambiente: Colline Temperate\\
Organizzazione: Solitario, gruppo (2-5), banda (6-8), gruppo di razzia (9-12 più 1d4 Lupi Crudeli) o tribù (13-30 più 35\% non combattente più 1 capo combattente di 4°-6° livello, 11-16 Lupi Crudeli, 1-4 Ogre e 13-20 schiavi orchi)\\
Tesoro: Standard (Armatura di Pelle, Randello Pesante, altro tesoro)\\
\textbf{Descrizione}\\
I giganti delle colline hanno pelle che varia dal marrone chiaro al rossastro, capelli castani o neri, ed occhi dello stesso colore. Indossano strati di pelli rozzamente conciate con ancora il pelo. Raramente lavano o riparano i propri indumenti, e preferiscono semplicemente aggiungere nuovi strati man mano che i vecchi si logorano. Gli adulti sono alti circa 3 metri e pesano più o meno 550 kg. I giganti delle colline possono vivere fino a 200 anni, anche se raramente raggiungono quest'età.\\
I giganti delle colline preferiscono combattere dall'alto di sporgenze e rupi, da dove possono colpire gli avversari con rocce e massi, limitando così il rischio personale. Amano effettuare attacchi di oltrepassare contro creature più piccole all'inizio del combattimento, e solo dopo prendono posizione e iniziano a roteare i loro massicci randelli.\\
I giganti delle colline sono per natura nomadi e preferiscono viaggiare da un luogo all'altro per razziare e saccheggiare. Sebbene gradiscano di più i climi temperati, non disdegnano di viaggiare lontano dal loro ambiente favorito, se la razzia è abbondante e prospera. Si tratta, nel complesso, di creature molto egoiste, che raramente affrontano battaglie che non siano sicuri di vincere. I giganti delle colline sono noti per l'abitudine di spingersi l'un l'altro se devono confrontarsi con avversari temibili e non esitano a sacrificare un compagno per salvarsi la pelle. Bande erranti di giganti delle colline sono diffuse sulle colline temperate, e la loro costante aggressività li rende uno dei pericoli più temuti in questo ambiente.\\

I giganti delle colline solitari e non malvagi sono molto rari, ma li si può trovare qualche volta in altre società umanoidi, anche se non sono quasi mai accettati nelle città principali o nei centri popolati. Si trovano a proprio agio come lavoratori e soldati nelle remote città di frontiera, e spesso fungono da rudimentali diplomatici per negoziare con le bande di giganti delle colline razziatori. Sfortunatamente, i giganti delle colline che abbandonano il proprio stile di vita razziale per la civiltà vengono derisi e spesso uccisi a vista dai loro fratelli nomadi. Tuttavia, questi giganti delle colline "civilizzati" possono trovare il proprio posto nella società e molti sono riusciti a vivere un'esistenza pacifica e tranquilla.\\


\medskip\index{Mostri - Gigante del Fuoco}\textbf{Gigante del Fuoco}

\emph{Enorme gigante, legale malvagio}

\textbf{FORZA} +7

\textbf{DESTREZZA} -1

\textbf{COSTITUZIONE} +6

\textbf{INTELLIGENZA} +0

\textbf{SAGGEZZA} +2

\textbf{CARISMA} +1

\textbf{Iniziativa} +0 -- \textbf{Difesa} 27 (armatura di piastre)

\textbf{Punti Ferita} 162 (13d12 + 78)

\textbf{Movimento} 9 m

\textbf{Tiri Salvezza}: Tempra +14, Riflessi +4, Volontà +9

\textbf{Competenze} Acrobatica +11, Consapevolezza +6

\textbf{Immunità ai Danni} fuoco

\textbf{Linguaggi} Gigante

\textbf{Sfida} 9 (5.000 PE)

\textbf{Azioni}

\emph{\textbf{Multiattacco.}} Il gigante effettua due attacchi con lo spadone.

\emph{\textbf{Spadone.} Attacco con arma da mischia}: +11 a colpire, portata 3 m, un bersaglio.

\emph{Colpisce:} 28 (6d6 + 7) danni taglienti.

\emph{\textbf{Sasso.} Attacco con arma a Distanza}: +11 a colpire, gittata 18m, un bersaglio.

\emph{Colpisce:} 29 (4d10 + 7) danni da botta.

\textbf{Ecologia}
Ambiente: Montagne calde\\
Organizzazione: Solitario, gruppo (2-5), banda (6-12 più un 35\% non combattenti e 1 adepto o Devoto di 1°-2° livello), gruppo di razziatori (6-12 più 1 adepto o mago di 3°-5° livello, 2-5 Segugi Infernali e 2-3 Troll o Ettin) o tribù (20-30 più 1 adepto, mago o Devoto di 6°-7° livello; 1 re Guerriero o guardiaboschi di 8°-9° livello; e 17-38 Segugi Infernali, 12-22 Troll, 7-12 Ettin e 1-2 Draghi Rossi Giovani)\\
Tesoro: Standard (Mezza Armatura, Spadone, altro tesoro)\\
\textbf{Descrizione}\\
I giganti del fuoco sono i giganti più rigidi e marziali, sempre pronti alla guerra e a trattare brutalmente chiunque incontrino. La loro rigida struttura di comando prevede soldati, ufficiali e persino generali, e che tutti obbediscano agli ordini del loro re senza discutere.\\

I giganti del fuoco hanno capelli arancione brillante che splendono e scintillano come se fossero in fiamme. Un maschio adulto è alto tra i 3,6 e i 4,8 metri, con una cassa toracica di circa 2,7 metri, e pesa circa 3.500 kg. Le femmine sono leggermente più basse e snelle. I giganti del fuoco possono vivere fino a 350 anni.\\

I giganti del fuoco indossano abiti di tessuti robusti o di pelle di color arancione, giallo, nero o rosso. I guerrieri indossano elmi e mezze armature di acciaio brunito e impugnano grandi spadoni che mulinano per il campo di battaglia. In gruppi numerosi, i giganti del fuoco combattono con tattiche di gruppo brutali ed efficienti, e non esitano a sacrificare qualche compagno per tendere un'imboscata al nemico.\\

I giganti del fuoco preferiscono i luoghi caldi: più caldi sono meglio è. Si possono trovare nei deserti, nei vulcani, nelle fonti termali e nelle profondità della terra nei pressi di camini lavici. Vivono in castelli, insediamenti fortificati o grandi caverne, e l'architettura di questi luoghi riflette il loro stile di vita rigido e militaristico, con gli ufficiali che abitano in alloggi migliori di quelli dei loro sottoposti.\\



\medskip\index{Mostri - Gigante del Gelo}\textbf{Gigante del Gelo}

\emph{Enorme gigante, neutrale malvagio}

\textbf{FORZA} +6

\textbf{DESTREZZA} -1

\textbf{COSTITUZIONE} +5

\textbf{INTELLIGENZA} -1

\textbf{SAGGEZZA} +0

\textbf{CARISMA} +1

\textbf{Iniziativa} -1 -- \textbf{Difesa} 19 (armatura composita)

\textbf{Punti Ferita} 138 (12d12 + 60)

\textbf{Movimento} 12 m

\textbf{Tiri Salvezza} Tempra +14, Riflessi +3, Volontà +6

\textbf{Competenze} Acrobatica +9, Consapevolezza +3

\textbf{Immunità ai Danni} freddo

\textbf{Linguaggi} Gigante

\textbf{Sfida} 8 (3.900 PE)

\textbf{Azioni}

\emph{\textbf{Multiattacco.}} Il gigante effettua due attacchi con l'ascia bipenne.

\emph{\textbf{Ascia Bipenne.} Attacco con arma da mischia}: +9 a colpire, portata 3 m, un bersaglio.

\emph{Colpisce:} 25 (3d12 + 6) danni taglienti.

\emph{\textbf{Sasso.} Attacco con arma a Distanza}: +9 a colpire, gittata 18m, un bersaglio.

\emph{Colpisce:} 28 (4d10 + 6) danni da botta.

\textbf{Ecologia}\\
Ambiente: Montagne fredde\\
Organizzazione: Solitario, banda (3-5), gruppo (6-12 più 35\% non combattenti e 1 mago o Devoto di 1°-2° livello), gruppo di razziatori (6-12 più 35\% non combattenti, 1 Devoto o mago di 3°-5° livello, 1-4 Lupi Invernali e 2-3 Ogre) o tribù (21-30 più 1 adepto, mago o Devoto di 6°-7° livello; 1 jarl Barbaro o guardiaboschi 7°-9° livello; e 15-36 Lupi Invernali, 13-22 Ogre e 1-2 Draghi Bianchi Giovani)\\
Tesoro: Standard (Giaco di Maglia, Ascia Bipenne, altro tesoro)\\
\textbf{Descrizione}\\
Un gigante del gelo ha capelli azzurri o giallo sporco, e occhi in genere dello stesso colore. Si vestono con pelli e pellicce, adornandosi con qualsiasi gioiello possiedano. I giganti del gelo combattenti indossano anche giachi di maglia ed elmi di metallo decorati con corna e piume. Un maschio adulto è alto 5 metri e pesa circa 1.400 kg. Le femmine sono leggermente più basse e snelle, ma per il resto sono identiche ai maschi. I giganti del gelo possono vivere fino a 250 anni.\\

I giganti del gelo sono molto temuti, poiché la brama di distruzione e guerra ed il loro comportamento sprezzante li spingono a manifestazioni di brutalità sempre maggiori. I giganti del gelo iniziano attaccando a distanza, scagliando rocce finché finiscono le munizioni o l'avversario si avvicina, poi lo affrontano con le loro enormi asce. Una delle tattiche preferite è tendere un'imboscata nascondendosi sotto la neve al di sopra di un pendio ghiacciato o innevato, dove gli avversari avranno difficoltà a raggiungerli, e poi iniziano causando una valanga prima di scendere in battaglia. I giganti del gelo possono nascondersi molto bene negli ambienti nevosi e sono dei maestri nella furtività nel loro dominio.\\

I giganti del gelo sopravvivono cacciando e razziando da soli, dato che vivono in ambienti freddi e desolati. I gruppi di giganti del gelo sono divisi quasi equamente tra quelli che vivono in insediamenti di fortuna o castelli abbandonati e quelli che vagabondano per il gelido nord, come nomadi in cerca di bottino e provviste. I capi dei giganti del gelo si chiamano jarl e richiedono obbedienza assoluta ai loro seguaci. In ogni momento uno jarl può essere sfidato in combattimento per il comando della tribù. Queste sfide tipicamente finiscono con la morte di uno dei contendenti. Un singolo jarl può spesso contare su una dozzina o più di tribù più piccole di giganti del gelo come estensione della sua. In questi casi, i capi delle tribù minori sono noti come capitani o signori della guerra.\\

I giganti del gelo amano prendere prigionieri e li usano sia come schiavi che come materia prima. Di solito ogni gruppo di giganti del gelo tiene 1-2 schiavi umanoidi incatenati ad un addestratore di schiavi: il più meschino e crudele del gruppo dopo lo jarl. Hanno anche una certa passione per gli animali domestici mostruosi: Draghi Bianchi e Lupi Invernali sono scelte popolari, ma nella tana di un gigante del gelo si possono trovare anche Remorhaz e Yeti.\\



\medskip\index{Mostri - Gigante delle Nuvole}\textbf{Gigante delle Nuvole}

\emph{Enorme gigante, neutrale buono (50\%) o neutrale malvagio (50\%)}

\textbf{FORZA} +8

\textbf{DESTREZZA} +0

\textbf{COSTITUZIONE} +6

\textbf{INTELLIGENZA} +1

\textbf{SAGGEZZA} +3

\textbf{CARISMA} +3

\textbf{Iniziativa} +1 -- \textbf{Difesa} 19

\textbf{Punti Ferita} 200 (16d12 + 96)

\textbf{Movimento} 12 m

\textbf{Tiri Salvezza} Tempra +16, Riflessi +6, Volontà +10

\textbf{Competenze} Percepire Emozioni +7, Consapevolezza +7

\textbf{Linguaggi} Comune, Gigante

\textbf{Sfida} 9 (5.000 PE)

\emph{\textbf{Incantesimi Innati.}} La caratteristica da incantatore del gigante è il Carisma. Il gigante può lanciare questi incantesimi in maniera innata, senza bisogno di componenti materiali:

A volontà: \emph{individuazione del magico, luce, nube di nebbia}

3/giorno ciascuno: \emph{caduta morbida, passo nebbioso, telecinesi}

1/giorno ciascuno: \emph{controllare tempo atmosferico, forma gassosa}

\emph{\textbf{Olfatto Affinato.}} Il gigante ha +1d6 alle prove di Saggezza (Consapevolezza) basate sull'olfatto.

\textbf{Azioni}

\emph{\textbf{Multiattacco.}} Il gigante effettua due attacchi con la morning star.

\emph{\textbf{Morning star.} Attacco con arma da mischia}: +12 a colpire, portata 3 m, un bersaglio.

\emph{Colpisce:} 21 (3d8 + 8) danni perforanti.

\emph{\textbf{Sasso.} Attacco con arma a Distanza}: +12 a colpire, gittata 18m, un bersaglio.

\emph{Colpisce:} 30 (4d10 + 8) danni da botta.

\textbf{Ecologia}\\
Ambiente: Montagne Temperate\\
Organizzazione: Solitario, gruppo (2-5), famiglia (2-5 più 35\% non combattenti più 1 mago o Devoto di 4°-7° livello e 2-5 Grifoni) o tribù (6-20 più 1 oracolo mago o Devoto di 7°-12° livello e 2-5 Grifoni)\\
Tesoro: Standard (Giaco di Maglia, Morning Star, altro tesoro)\
\textbf{Descrizione}\\
Il colore pelle dei giganti delle nuvole varia dal bianco latte al blu polvere. I maschi adulti sono alti circa 5,4 metri e pesano approssimativamente 2.500 kg. Le femmine sono leggermente più basse e snelle. I giganti delle nuvole possono vivere fino a 400 anni, vestono con abiti preziosi e gioielli. Per molti l'aspetto indica lo status. Migliori sono i vestiti e più raffinati i gioielli, più importante è chi li indossa. Inoltre apprezzano la musica, e la maggioranza suona uno o più strumenti (l'arpa è uno dei preferiti).\\

I giganti delle nuvole possono avere Tratti insolitamente vari; circa metà sono buoni e metà malvagi. I giganti delle nuvole buoni costruiscono strade che collegano i loro insediamenti con le strade degli umani per promuovere il commercio. Non è insolito vedere un gigante delle nuvole buono camminare tra gli uomini, ad esempio, in una città umana nei pressi di un'alta catena montuosa. I giganti delle nuvole malvagi tendono a non creare insediamenti stabili e anzi preferiscono vivere in rozzi rifugi su alti picchi, da cui scendono solo per depredare i villaggi di quello di cui potrebbero aver bisogno. Queste due filosofie portano spesso allo scoppio di guerre violente e durature tra tribù vicine.\\

Sono molte le leggende che parlano di magiche città dei giganti delle nuvole situate tra le nuvole stesse, che fluttuano sui venti e circumnavigano il mondo. Mentre i giganti delle nuvole riconoscono che si tratta per lo più di fantasie, alcuni sostengono di averle viste e hanno dedicato la loro intera esistenza a ritrovarle.\\


\medskip\index{Mostri - Gigante di Pietra}\textbf{Gigante di Pietra}

\emph{Enorme gigante, neutrale}

\textbf{FORZA} +6

\textbf{DESTREZZA} +2

\textbf{COSTITUZIONE} +5

\textbf{INTELLIGENZA} +0

\textbf{SAGGEZZA} +1

\textbf{CARISMA} -1

\textbf{Iniziativa} +2 -- \textbf{Difesa} 21

\textbf{Punti Ferita} 126 (11d12 + 55)

\textbf{Movimento} 12 m

\textbf{Tiri Salvezza} Tempra +12, Riflessi +6, Volontà +7

\textbf{Competenze} Acrobatica +12, Consapevolezza +4

\textbf{Sensi} scurovisione 18 m

\textbf{Linguaggi} Gigante

\textbf{Sfida} 7 (2.900 PE)

\emph{\textbf{Mimetismo di Pietra.}} Il gigante ha +1d6 alle prove di Destrezza (Nascondersi) effettuate per nascondersi su terreni rocciosi.

\textbf{Azioni}

\emph{\textbf{Multiattacco.}} Il gigante effettua due attacchi con il randello pesante.

\emph{\textbf{Randello Pesante.} Attacco con arma da mischia}: +9 a colpire, portata 5 metri, un bersaglio.

\emph{Colpisce:} 19 (3d8 + 6) danni da botta.

\emph{\textbf{Sasso.} Attacco con arma a Distanza}: +9 a colpire, gittata 18m, un bersaglio.

\emph{Colpisce:} 28 (4d10 + 6) danni da botta. Se il bersaglio è una creatura, deve riuscire un Tiro Salvezza di Tempra CD 17 o cadere prona.

\textbf{Reazioni}

\emph{\textbf{Afferrare Sassi.}} Se un sasso o un simile oggetto viene scagliato al gigante, il gigante può, riuscendo un Tiro Salvezza su Riflessi CD 10, afferrare il proiettile e non subire danni da botta da esso.

\textbf{Ecologia}
Ambiente: Montagne temperate\\
Organizzazione: Solitario, gruppo (2-5), banda (4-8), gruppo di caccia (9-12 più 1 Anziano) o tribù (13-30 più 35\% non combattenti, 1-3 Anziani e 4-6 Orsi Crudeli)\\
Tesoro: Standard (Randello Pesante, altro tesoro)\\
\textbf{Descrizione}\\
I giganti delle rocce preferiscono spessi indumenti di cuoio, tinti con tonalità di marrone e grigio per confondersi con la pietra che li circonda. Gli adulti sono alti circa 3,6 metri, pesano circa 750 kg e possono vivere fino a 800 anni.\\
I giganti delle rocce, se possibile, combattono a distanza, ma se non possono evitare la mischia usano giganteschi randelli di pietra. Una delle tattiche favorite dai giganti delle rocce è di stare immobili, mimetizzandosi con il paesaggio, per poi avanzare scagliando rocce e sorprendere i nemici.\\

I giganti delle rocce preferiscono vivere in enormi caverne sulle cime rocciose. Raramente vivono a più di qualche giorno di viaggio da altre bande di giganti delle rocce e allevano greggi condivisi di capre e altro bestiame.\\

I giganti delle rocce più vecchi tendono ad allontanarsi dalla tribù per molto tempo, per vivere in solitudine da qualche parte o tentando di inserirsi in altre civiltà umanoidi. Dopo decadi di esilio auto imposto, chi fa ritorno è noto come Gigante delle Rocce Anziano.\\


\medskip\index{Mostri - Gigante delle Tempeste}\textbf{Gigante delle Tempeste}

\emph{Enorme gigante, caotico buono}

\textbf{FORZA} +9

\textbf{DESTREZZA} +2

\textbf{COSTITUZIONE} +5

\textbf{INTELLIGENZA} +3

\textbf{SAGGEZZA} +4

\textbf{CARISMA} +4

\textbf{Iniziativa} +3 -- \textbf{Difesa} 23 (armatura di scaglie)

\textbf{Punti Ferita} 230 (20d12 + 100)

\textbf{Movimento} 15 m, nuoto 15 m

\textbf{Tiri Salvezza} Tempra +17, Riflessi +8, Volontà +13

\textbf{Competenze} Arcano +8, Acrobatica +14, Consapevolezza +9, Storia +8

\textbf{Resistenze al Danno} freddo

\textbf{Immunità al Danno} fulmine, tuono

\textbf{Linguaggi} Comune, Gigante

\textbf{Sfida} 13 (10.000 PE)

\emph{\textbf{Anfibio.}} Il gigante può respirare aria e acqua.

\emph{\textbf{Incantesimi Innati.}} La caratteristica da incantatore del gigante è il Carisma. Il gigante può lanciare questi incantesimi in maniera innata, senza bisogno di componenti materiali:

A volontà: \emph{caduta controllata, individuazione del magico,} \emph{levitazione, luce}

3/giorno ciascuno: \emph{controllare tempo atmosferico, respirare} \emph{sott'acqua}

\textbf{Azioni}

\emph{\textbf{Multiattacco.}} Il gigante effettua due attacchi con lo spadone.

\emph{\textbf{Spadone.} Attacco con arma da mischia}: +14 a colpire, portata 3 m, un bersaglio.

\emph{Colpisce:} 30 (6d6 + 9) danni taglienti.

\emph{\textbf{Sasso.} Attacco con arma a Distanza}: +14 a colpire, gittata 18m, un bersaglio.

\emph{Colpisce:} 35 (4d12 + 9) danni da botta.

\emph{\textbf{Colpo Fulminante (Ricarica 5-6).}} Il gigante scaglia una folgore magica ad un punto visibile entro 150 metri da sé. Ogni creatura entro 3 metri da quel punto deve effettuare un Tiro Salvezza su Riflessi CD 17, subendo 54 (12d8) danni da fulmine se lo fallisce, o la metà se lo supera.

\textbf{Ecologia}\\
Ambiente: Qualsiasi caldo\\
Organizzazione: Solitario o famiglia (2-5 più 1 mago o Devoto di livello 7°-10°, 1-2 Roc, 2-6 Grifoni e 2-8 Squali)\\
Tesoro: Standard (Corazza di Piastre Perfetta, Arco Lungo Composito Perfetto [Forza +14] con 20 Frecce, Spadone Perfetto, altro tesoro)\\
\textbf{Descrizione}\\
I giganti delle tempeste tendono ad avere carnagione abbronzata, anche se rari esemplari hanno pelle viola, capelli viola o blu scuri e occhi grigio argento o porpora. Il colore viola è considerato di buon auspicio tra i giganti delle tempeste, e coloro che lo posseggono tendono a diventare capi tra la loro gente. Gli adulti sono normalmente alti 6,3 metri e pesano 6.000 kg. I giganti delle tempeste possono vivere fino a 600 anni.\\

Quando sono a riposo, preferiscono indossare tuniche corte e ampie cinte ai fianchi, sandali o piedi nudi e una fascia per capelli. Indossano pochi gioielli di semplice ma ottima fattura, i più comuni sono cavigliere (preferite dai giganti a piedi scalzi), anelli o diademi. Ma quando si equipaggiano per la battaglia, indossano corazze di piastre scintillanti e impugnano enormi spadoni e archi.\\

I giganti delle tempeste sono tendenzialmente solitari, preferendo abitare lungo remote coste o su isole tropicali. Come suggerisce il loro nome, sono inclini a violenti sbalzi di umore. I giganti delle tempeste sono facili all'ira di fronte al male e possono essere nemici brutali e pericolosi quando vengono insultati. In battaglia, preferiscono scagliare una pioggia di frecce sui loro nemici, estraendo gli spadoni solo dopo che gli avversari si sono avvicinati.\\

I giganti delle tempeste vivono in belle torri, castelli o in insediamenti cinti da mura e amano coltivare la terra. Possiedono enormi giardini ben curati e gestiscono centinaia di acri di coltivazioni per gruppo. Spesso impiegano altri umanoidi, come Elfi o Umani, come supporto per condurre le loro immense fattorie. Una enclave di giganti delle tempeste spesso si assume la responsabilità della sicurezza di un'intera isola o linea di costa.\\


\medskip\index{Mostri - Gnoll}\textbf{Gnoll}

\emph{Media umanoide (gnoll), caotico malvagio}

\textbf{FORZA} +2

\textbf{DESTREZZA} +1

\textbf{COSTITUZIONE} +0

\textbf{INTELLIGENZA} -2

\textbf{SAGGEZZA} +0

\textbf{CARISMA} -2

\textbf{Iniziativa} +1 -- \textbf{Difesa} 16 (armatura di pelle, scudo)

\textbf{Punti Ferita} 22 (5d8)

\textbf{Movimento} 9 m

\textbf{Tiri Salvezza}: Tempra +4, Riflessi +0, Volontà +0

\textbf{Sensi} scurovisione 18 m

\textbf{Linguaggi} Gnoll

\textbf{Sfida} 1/2 (100 PE)

\emph{\textbf{Rabbia.}} Quando lo gnoll riduce una creatura a 0 punti ferita con un attacco da mischia durante il proprio turno, può svolgere un'azione bonus per muoversi fino a metà del suo movimento ed effettuare un attacco di morso.

\textbf{Azioni}

\emph{\textbf{Morso.} Attacco con arma da mischia}: +4 a colpire, portata 1 m, una creatura.

\emph{Colpisce:} 4 (1d4 + 2) danni perforanti.

\emph{\textbf{Lancia.} Attacco con arma da mischia o a Distanza}: +4 a colpire, portata 1 m o gittata 6 m, un bersaglio.

\emph{Colpisce:} 5 (1d6 + 2) danni perforanti o 6 (1d8 + 2) danni perforanti se usata con due mani per effettuare un attacco da mischia.

\emph{\textbf{Arco Lungo.} Attacco con arma a Distanza}: +3 a colpire, gittata 45m, un bersaglio.

\emph{Colpisce:} 5 (1d8 + 1) danni perforanti.

\textbf{Ecologia}\\
Ambiente: Pianure calde, deserti\\
Organizzazione: Solitario, coppia, gruppo di caccia (2-5 e 1-2 Iene), banda (10-100 adulti più 50\% piccoli non combattenti, 1 sergente di 3° livello ogni 20 adulti, 1 capo di 4°-6° livello e 5-8 Iene) o tribù (20-200 più 1 sergente di 3° livello ogni 20 adulti, 1 o 2 luogotenenti di 4° o 5° livello, 1 capo di 6°-8° livello, 7-12 Iene e 4-7 ienodonti)\\
Tesoro: equipaggiamento da PNG (Armatura di Cuoio, Scudo Pesante di Legno, Lancia, altro tesoro)\\
\textbf{Descrizione}\\
Gli gnoll sono una razza di umanoidi grandi e grossi che assomigliano alle iene non solo per il semplice aspetto; mostrano un'evidente affinità con questi animali spazzini, tanto da tenerli come animali di compagnia, e riflettono molti dei comportamenti di tali animali.\\

Gli gnoll sono abili cacciatori, ma preferiscono di gran lunga ripulire o trafugare una carcassa piuttosto che cacciare una preda. Questa pigrizia li spinge a procurarsi degli schiavi di qualsiasi specie disponibile per costringerli a scavare tane, raccogliere provviste e acqua e perfino cacciare per i loro padroni gnoll.\\

Le altre creature che non siano iene o gnoll diventano pasto o schiavi, a seconda del temperamento della tribù. Anche un compagno morto o caduto diventa un pasto fresco per uno gnoll, che può onorare un membro famoso della tribù con una breve preghiera o cucinarne interamente uno morto di una devastante malattia: altrimenti, gli gnoll non vedono un loro simile morto molto diversamente da qualsiasi altra creatura. Gli gnoll più "civilizzati" non mangiano i loro prigionieri: li tengono, invece, come schiavi, per difendere o migliorare la loro tana o per scambiarli con altre tribù o bande schiaviste.\\

Gli gnoll provano gusto per il combattimento, ma solo quando sono in superiorità numerica. In altre situazioni, preferiscono evitare il combattimento tranne che come mezzo per ottenere una carcassa da un altro cacciatore, o come un'ingegnosa imboscata per abbattere un lauto pasto. Questi uomini iena non vedono alcun valore nel coraggio o nell'eroismo e preferiscono invece fuggire una volta chiaro che la vittoria non è raggiungibile, sostenendo che è meglio scappare con la coda tra le gambe piuttosto che perderla del tutto.\\

Durante il combattimento, gli gnoll usano una strana combinazione di tattiche da branco e strategie individuali. Se uno gnoll è sicuro di vincere, tenta di abbattere l'avversario più debole piuttosto che aiutare i suoi compagni. Se gli gnoll sono in difficoltà, si coalizzano contro un avversario potente e tentano di eliminarlo, sperando di costringere alla fuga i suoi alleati.\\

I capi gnoll hanno competenze da guardiaboschi ma non e' impossibile trovare anche gnoll Devoti a qualche famelico Patrono. Difficilmente padroneggiano in maniera efficage la magia.\\


\medskip\index{Mostri - Gnomo delle Profondità (Svirfneblin)}\textbf{Gnomo delle Profondità (Svirfneblin)}

\emph{Piccola umanoide (gnomo), neutrale buono}

\textbf{FORZA} +2

\textbf{DESTREZZA} +2

\textbf{COSTITUZIONE} +2

\textbf{INTELLIGENZA} +1

\textbf{SAGGEZZA} +0

\textbf{CARISMA} -1

\textbf{Iniziativa} +2 -- \textbf{Difesa} 16 (giaco di maglia)

\textbf{Punti Ferita} 16 (3d6 + 6)

\textbf{Movimento} 6 m

\textbf{Tiri Salvezza}: Tempra +6, Riflessi +6, Volontà +2

\textbf{Competenze} Muoversi Silenziosamente / Nascondersi +4, Consapevolezza +2

\textbf{Sensi} scurovisione 36 m

\textbf{Linguaggi} Gnomica, Linguaggio delle Profondità, Terran

\textbf{Sfida} 1/2 (100 PE)

\emph{\textbf{Astuzia Gnomesca.}} Lo gnomo ha +1d6 ai Tiri Salvezza contro la magia.

\emph{\textbf{Camuffamento di Pietra.}} Lo gnomo ha +1d6 alle prove di Destrezza (Nascondersi) effettuate per nascondersi su terreni rocciosi.

\emph{\textbf{Incantesimi Innati.}} La caratteristica da incantatore innato dello gnomo è l'Intelligenza. Lo gnomo può lanciare questi incantesimi in maniera innata, senza bisogno di componenti:

A volontà: \emph{anti-individuazione} (personale)

1/giorno ciascuno: \emph{camuffare sé stesso, cecità/sordità, sfocatura}

\textbf{Azioni}

\emph{\textbf{Piccone da Guerra.} Attacco con arma da mischia}: +4 a colpire, portata 1 m, un bersaglio.

\emph{Colpisce:} 6 (1d8 + 2) danni perforanti.

\emph{\textbf{Dardo Avvelenato.} Attacco con arma a Distanza}: +4 a colpire, gittata 9m, un bersaglio.

\emph{Colpisce:} 4 (1d4 + 2) danni perforanti, e il bersaglio deve riuscire un Tiro Salvezza di Tempra CD 12 o restare avvelenato per 1 minuto. Il bersaglio può ripetere il Tiro Salvezza al termine di ciascun suo turno, terminando l'effetto su di sé in caso di successo.

\textbf{Ecologia}
Ambiente: Qualsiasi sotterraneo\\
Organizzazione: Solitario, compagnia (2-4), squadra (5-20 più 1 capo 3°-6° e due sergenti di 3° livello), o banda (30-50 più 1 sergente di 3° livello ogni 20 adulti, 5 tenenti di 5° livello, 3 capitani di 7° livello, e 2-5 Elementali della Terra Medi)\\
Tesoro: Equipaggiamento da PNG (Piccone Pesante, Balestra Leggera con 10 Quadrelli, altro tesoro)\\
\textbf{Descrizione}\\
Gli svirfneblin, o "gnomi delle profondità", sono una branca della razza gnomesca. Dimorano nel sottosuolo, in città nascoste, al sicuro dagli elfi scuri e da altre razze sotterranee. La loro pelle è del colore della roccia, di solito grigia o marrone. I maschi sono calvi e le femmine hanno radi capelli grigi. Il legame di uno svirfneblin con il reame dei folletti è molto più forte di quello degli Gnomi di superficie; questo rende gli svirfneblin stranamente distaccati dalle loro emozioni oppure soggetti ad improvvise e violente manifestazioni emotive. Gli svirfneblin hanno combattuto a lungo contro i Duergar e non riescono a distinguere bene i Duergar dai Nani.\\



\subsection{Golem}

\medskip\index{Mostri - Golem di Argilla}\textbf{Golem di Argilla}

\emph{Grande costrutto, disallineato}

\textbf{FORZA} +5

\textbf{DESTREZZA} -1

\textbf{COSTITUZIONE} +4

\textbf{INTELLIGENZA} -4

\textbf{SAGGEZZA} -1

\textbf{CARISMA} -5

\textbf{Iniziativa} -1 -- \textbf{Difesa} 19

\textbf{Punti Ferita} 133 (14d10 + 56)

\textbf{Movimento} 6 m

\textbf{Tiri Salvezza}: Tempra +4, Riflessi +3, Volontà +4

\textbf{Immunità al Danno} acido, psichico, veleno; da botta, perforante e tagliente di attacchi non magici o che non siano di adamantio

\textbf{Immunità alle Condizioni} affascinato, avvelenato, paralizzato, pietrificato, affaticamento, spaventato

\textbf{Sensi} scurovisione 18 m

\textbf{Linguaggi} comprende le lingue del suo creatore ma non può parlare

\textbf{Sfida} 9 (5.000 PE)

\emph{\textbf{Berserk.}} Ogni volta che il golem inizia il suo turno con 60 punti ferita o meno, tira un d6. Se ottieni 6, il golem va in berserk. Durante ogni suo turno mentre è in berserk, il golem attacca la creatura più vicina che può vedere. Se non c'è nessuna creatura abbastanza vicina da muoversi e attaccarla, il golem attacca un oggetto, con preferenza per gli oggetti più piccoli di lui. Una volta che il golem è andato in berserk, continuerà ad esserlo finché non viene distrutto o recupera tutti i suoi punti ferita.

\emph{\textbf{Armi Magiche.}} Gli attacchi con armi del golem sono magici.

\emph{\textbf{Assorbimento dell'Acido.}} Ogni volta che il golem è vittima di danni da acido, non subisce danni ma invece recupera un pari numero di punti ferita. 

\emph{\textbf{Forma Immutabile.}} Il golem è immune a qualsiasi incantesimo o effetto che altererebbe la sua forma.

\emph{\textbf{Natura di Costrutto.}} Un golem non ha bisogno di aria, cibo, bevande o sonno.

\emph{\textbf{Resistenza alla Magia.}} Il golem ha +1d6 ai Tiri Salvezza contro incantesimi e altri effetti magici.

\textbf{Azioni}

\emph{\textbf{Multiattacco.}} Il golem effettua due attacchi di schianto.

\emph{\textbf{Schianto.} Attacco con arma da mischia}: +8 a colpire, portata 1 m, un bersaglio.

\emph{Colpisce:} 16 (2d10 + 5) danni da botta. Se il bersaglio è una creatura, deve riuscire un Tiro Salvezza di Tempra CD 15 o vedere i suoi punti ferita massimi ridotti di un ammontare pari al danno subito. Il bersaglio muore se l'attacco riduce i suoi punti ferita massimi a 0. La riduzione resta finché non viene rimossa dall'incantesimo \emph{ristorare superiore} o altra magia.

\emph{\textbf{Velocità (Ricarica 5-6).}} Fino al termine del suo prossimo turno, il golem ottiene un bonus magico di +2 alla Difesa, ha +1d6 ai Tiri Salvezza su Riflessi, e può usare gli attacchi di schianto come azione bonus.

\textbf{Ecologia}\\
Ambiente: Qualsiasi\\
Organizzazione: Solitario o gruppo (2-4)\\
Tesoro: Nessuno\\
\textbf{Descrizione}\\
I golem di argilla non indossano abiti, eccezion fatta per un indumento di cuoio trattato o metallo attorno ai fianchi. Mediamente sono alti più di 2,4 metri e pesano 300 chili.\\
\textbf{Costruzione}
Un golem d'argilla può essere scolpito a partire da un unico blocco d'argilla del peso minimo di 500 chili, trattato con polveri e oli rari per il valore di 1,500 mo.\\


\medskip\index{Mostri - Golem di Carne}\textbf{Golem di Carne}

\emph{Media costrutto, neutrale}

\textbf{FORZA} +4

\textbf{DESTREZZA} -1

\textbf{COSTITUZIONE} +4

\textbf{INTELLIGENZA} -2

\textbf{SAGGEZZA} +0

\textbf{CARISMA} -3

\textbf{Iniziativa} -1 -- \textbf{Difesa} 12

\textbf{Punti Ferita} 93 (11d8 + 44)

\textbf{Movimento} 9 m

\textbf{Tiri Salvezza}: Tempra +3, Riflessi +2, Volontà +3

\textbf{Immunità al Danno} fulmine, veleno; da botta, perforante e tagliente di attacchi non magici o che non siano di adamantio

\textbf{Immunità alle Condizioni} affascinato, avvelenato, paralizzato, pietrificato, affaticamento, spaventato

\textbf{Sensi} scurovisione 18 m

\textbf{Linguaggi} comprende le lingue del suo creatore ma non può
parlare

\textbf{Sfida} 5 (1.800 PE)

\emph{\textbf{Berserk.}} Ogni volta che il golem inizia il suo turno con 40 punti ferita o meno, tira un d6. Se ottieni 6, il golem va in berserk. Durante ogni suo turno mentre è in berserk, il golem attacca la creatura più vicina che possa vedere. Se non c'è nessuna creatura abbastanza vicina da muoversi e attaccarla, il golem attacca un oggetto, con preferenza per gli oggetti più piccoli di lui. Una volta che il golem è andato in berserk, continuerà ad esserlo finché non viene distrutto o recupera tutti i suoi punti ferita.

\emph{\textbf{Armi Magiche.}} Gli attacchi con armi del golem sono magici.

\emph{\textbf{Assorbimento dei Fulmini.}} Ogni volta che il golem sia vittima di un danno da fulmine, non subisce danni ma invece recupera un pari numero di punti ferita.

\emph{\textbf{Avversione al Fuoco.}} Se il golem subisce danni da fuoco, ha -1d6 ai tiri di attacco e le prove di competenza fino alla fine del suo prossimo turno.

\emph{\textbf{Forma Immutabile.}} Il golem è immune a qualsiasi incantesimo o effetto che altererebbe la sua forma.

\emph{\textbf{Natura di Costrutto.}} Un golem non ha bisogno di aria, cibo, bevande o sonno.

\emph{\textbf{Resistenza alla Magia.}} Il golem ha +1d6 ai Tiri Salvezza contro incantesimi e altri effetti magici.

\textbf{Azioni}

\emph{\textbf{Multiattacco.}} Il golem effettua due attacchi di
schianto.

\emph{\textbf{Schianto.} Attacco con arma da mischia}: +7 a colpire,
portata 1 m, un bersaglio.

\emph{Colpisce:} 13 (2d8 + 4) danni da botta.

\textbf{Ecologia}\\
Ambiente: Qualsiasi\\
Organizzazione: Solitario o gruppo (2-4)\\
Tesoro: Nessuno\\
\textbf{Descrizione}\\
Un golem di carne è una mostruosa collezione di parti anatomiche umanoidi trafugate e cucite insieme. La sua carne cadaverica ha tonalità verde pallido o giallognola. Un golem di carne indossa qualsiasi tipo di vestito che il suo creatore desideri, normalmente solo un logoro paio di pantaloni. Non ha Equipaggiamento né armi. Un golem di carne è alto più di 2,4 metri e pesa 250 kg.\\

Un golem di carne non parla, anche se può emettere una specie di ringhio rauco. Cammina e si muove con un'andatura a scatti, come se non avesse il pieno controllo del proprio corpo.\\

Anche se molti golem di carne sono privi di ragione, si narra di golem eccezionali che in qualche modo hanno mantenuto i ricordi della vita precedente. La testa (e quindi il cervello) di questi golem di carne deve essere la giusta combinazione di freschezza e (nella vita precedente) decisione, ma di assoluta importanza sembrano essere anche la fortuna e il caso affinchè durante la loro creazione si conservi l'intelletto. Certamente quelli che costruiscono golem di carne preferiscono avere schiavi privi di intelletto piuttosto che dotati di una propria volontà, di conseguenza i golem di carne intelligenti sono rari.\\



\medskip\index{Mostri - Golem di Ferro}\textbf{Golem di Ferro}

\emph{Grande costrutto, disallineato}

\textbf{FORZA} +7

\textbf{DESTREZZA} -1

\textbf{COSTITUZIONE} +5

\textbf{INTELLIGENZA} -4

\textbf{SAGGEZZA} +0

\textbf{CARISMA} -5

\textbf{Iniziativa} -1 -- \textbf{Difesa} 28

\textbf{Punti Ferita} 210 (20d10 + 100)

\textbf{Movimento} 9 m

\textbf{Tiri Salvezza}: Tempra +6, Riflessi +5, Volontà +6

\textbf{Immunità al Danno} fuoco, psichico, veleno; da botta, perforante e tagliente di attacchi non magici o che non siano di adamantio

\textbf{Immunità alle Condizioni} affascinato, avvelenato, paralizzato, pietrificato, affaticamento, spaventato

\textbf{Sensi} scurovisione 36 m

\textbf{Linguaggi} comprende le lingue del suo creatore ma non può parlare

\textbf{Sfida} 16 (15.000 PE)

\emph{\textbf{Armi Magiche.}} Gli attacchi con armi del golem sono magici.

\emph{\textbf{Assorbimento del Fuoco.}} Ogni volta che il golem sia vittima di un danno da fuoco, non subisce danni ma invece recupera un pari numero di punti ferita.

\emph{\textbf{Forma Immutabile.}} Il golem è immune a qualsiasi incantesimo o effetto che altererebbe la sua forma.

\emph{\textbf{Natura di Costrutto.}} Un golem non ha bisogno di aria, cibo, bevande o sonno.

\emph{\textbf{Resistenza alla Magia.}} Il golem ha +1d6 ai Tiri Salvezza contro incantesimi e altri effetti magici.

\textbf{Azioni}

\emph{\textbf{Multiattacco.}} Il golem effettua due attacchi da mischia.

\emph{\textbf{Schianto.} Attacco con arma da mischia}: +13 a colpire, portata 1 m, un bersaglio.

\emph{Colpisce:} 20 (3d8 + 7) danni da botta.

\emph{\textbf{Spada.} Attacco con arma da mischia}: +13 a colpire, portata 3 m, un bersaglio.

\emph{Colpisce:} 23 (3d10 + 7) danni taglienti.

\emph{\textbf{Soffio Velenoso (Ricarica 6).}} Il golem esala un gas velenoso in un cono di 5 metri. Ogni creatura in quell'area deve effettuare un Tiro Salvezza di Tempra CD 19, subendo 45 (10d8) danni da veleno se fallisce il Tiro Salvezza, o la metà di questi danni se lo riesce.

\textbf{Ecologia}\\
Ambiente: Qualsiasi\\
Organizzazione: Solitario o gruppo (2-4)\\
Tesoro: Nessuno\\
\textbf{Descrizione}\\
Un golem di ferro ha un corpo di forma umanoide in ferro. Il creatore può dargli qualsiasi forma desideri, ma presenta quasi sempre un'armatura di qualche tipo, sia essa cerimoniale e preziosa o semplice e d'uso. Rispetto ad un golem di pietra ha sembianze molto più definite. I golem di ferro, talvolta, portano con sé un'arma, anche se il più delle volte tendono a preferirle i loro attacchi schianto.\\

Un golem di ferro è alto 3,6m e pesa circa 2.500 chili. Un golem di ferro non può parlare né emettere voce. Inoltre, non emette nessun odore riconoscibile.\\

Anche se la pratica della costruzione di golem di ferro è gradualmente caduta in disuso, i membri venerabili di alcune grandi civiltà del passato consideravano la capacità di forgiare golem di ferro dalla forza e dalle dimensioni sconcertanti un motivo di vanto. Questi golem (di taglia maggiore o uguale a Enorme), in alcuni angoli remoti del mondo, esistono ancora, e ancora eseguono meccanicamente ordini impartiti loro da imperi ormai scomparsi.\\

\textbf{Costruzione}
Per costruire un golem di ferro occorrono 2.500 kg di ferro, fuso con tinture rare del valore minimo di 10.000 mo.\\


\medskip\index{Mostri - Golem di Pietra}\textbf{Golem di Pietra}

\emph{Grande costrutto, disallineato}

\textbf{FORZA} +6

\textbf{DESTREZZA} -1

\textbf{COSTITUZIONE} +5

\textbf{INTELLIGENZA} -4

\textbf{SAGGEZZA} +0

\textbf{CARISMA} -5

\textbf{Iniziativa} -1 -- \textbf{Difesa} 22

\textbf{Punti Ferita} 178 (17d10 + 85)

\textbf{Movimento} 9 m

\textbf{Tiri Salvezza}: Tempra +4, Riflessi +3, Volontà +4

\textbf{Immunità al Danno} psichico, veleno; da botta, perforante e tagliente di attacchi non magici o che non siano di adamantio

\textbf{Immunità alle Condizioni} affascinato, avvelenato, paralizzato, pietrificato, affaticamento, spaventato

\textbf{Sensi} scurovisione 36 m

\textbf{Linguaggi} comprende le lingue del suo creatore ma non può parlare

\textbf{Sfida} 10 (5.900 PE)

\emph{\textbf{Armi Magiche.}} Gli attacchi con armi del golem sono magici.

\emph{\textbf{Forma Immutabile.}} Il golem è immune a qualsiasi incantesimo o effetto che altererebbe la sua forma.

\emph{\textbf{Natura di Costrutto.}} Un golem non ha bisogno di aria, cibo, bevande o sonno.

\emph{\textbf{Resistenza alla Magia.}} Il golem ha +1d6 ai Tiri Salvezza contro incantesimi e altri effetti magici.

\textbf{Azioni}

\emph{\textbf{Multiattacco.}} Il golem effettua due attacchi di schianto.

\emph{\textbf{Schianto.} Attacco con arma da mischia}: +10 a colpire, portata 1 m, un bersaglio.

\emph{Colpisce:} 19 (3d8 + 6) danni da botta.

\emph{\textbf{Lentezza (Ricarica 5-6).}} Il golem prende a bersaglio una o più creature entro 3 metri da lui e che possa vedere. Ciascun bersaglio deve effettuare un Tiro Salvezza di Volontà CD 17 contro questa magia. Se fallisce il Tiro Salvezza, il bersaglio non può usare reazioni, ha la velocità dimezzata, e durante il proprio turno non può effettuare più di un attacco. Inoltre, durante il proprio turno il bersaglio può effettuare un'azione o un'azione bonus, ma non entrambe. Questi effetti durano per 1 minuto. Il bersaglio può ripetere il Tiro Salvezza al termine di ciascun suo turno, terminando l'effetto per sé, in caso di successo.

\textbf{Ecologia}\\
Ambiente: Qualsiasi\\
Organizzazione: Solitario o gruppo (2-4)\\
Tesoro: Nessuno\\
\textbf{Descrizione}\\
Un golem di pietra ha un corpo umanoide fatto di pietra, spesso stilizzato per soddisfare il suo creatore. Ad esempio, può essere scolpito in modo da indossare un'armatura, con particolari simboli scolpiti sulla corazza, o avere dei disegni intarsiati nella pietra dei suoi arti. La testa è spesso scolpita per sembrare un elmo o la testa di qualche bestia. Sebbene possa essere scolpito con uno scudo o un'arma di pietra come una spada, queste scelte estetiche non influenzano le sue capacità in combattimento.\\

Come per la maggior parte dei golem, un golem di pietra non può parlare e non emette altro suono se non quelo della pietra che sfrega sulla pietra quando si muove. Un golem di pietra è alto 2,7 metri e pesa cirda 1.000 kg.\\

\textbf{Costruzione}
Il corpo di un golem di pietra viene scolpito da un unico blocco di pietra dura, come il granito, del peso di almeno 1.500 kg. La pietra deve essere di qualità eccezionale, e costare 5.000 mo.\\


\medskip\index{Mostri - Gorgone}\textbf{Gorgone}

\emph{Grande mostruosità, disallineato}

\textbf{FORZA} +5

\textbf{DESTREZZA} +0

\textbf{COSTITUZIONE} +4

\textbf{INTELLIGENZA} -4

\textbf{SAGGEZZA} +1

\textbf{CARISMA} -2

\textbf{Iniziativa} +0 -- \textbf{Difesa} 22

\textbf{Punti Ferita} 114 (12d10 + 48)

\textbf{Movimento} 12 m

\textbf{Tiri Salvezza}: Tempra +13, Riflessi +6, Volontà +7

\textbf{Competenze} Consapevolezza +4

\textbf{Immunità alle Condizioni} Pietrificato

\textbf{Sensi} scurovisione 18 m

\textbf{Linguaggi} -

\textbf{Sfida} 5 (1.800 PE)

\emph{\textbf{Carica Travolgente.}} Se la gorgone si muove di almeno 6 metri in linea retta verso il bersaglio e lo colpisce con un attacco di incornata durante lo stesso turno, il bersaglio deve riuscire un Tiro Salvezza su Tempra CD 16 o cadere prono. Se il bersaglio è prono, la gorgone può effettuare un attacco di zoccoli contro di lui come azione bonus.

\textbf{Azioni}

\emph{\textbf{Incornata.} Attacco con arma da mischia}: +8 a colpire, portata 1 m, un bersaglio.

\emph{Colpisce:} 18 (2d12 + 5) danni perforanti.

\emph{\textbf{Zoccoli.} Attacco con arma da mischia}: +8 a colpire, portata 1 m, un bersaglio.

\emph{Colpisce:} 16 (2d10 + 5) danni da botta.

\emph{\textbf{Soffio Pietrificante (Ricarica 5-6).}} La gorgone esala un gas pietrificante in un cono di 9 metri. Ogni creatura in quell'area deve riuscire un Tiro Salvezza di Tempra CD 13. Se il Tiro Salvezza fallisce, il bersaglio inizia a trasformarsi in pietra ed è intralciato. Il bersaglio intralciato deve ripetere il Tiro Salvezza al termine del suo prossimo turno. Se lo riesce, l'effetto sul bersaglio ha termine. Se lo fallisce, il bersaglio è pietrificato finché non viene liberato dall'incantesimo \emph{ripristino superiore} o simile magia.

\textbf{Ecologia}\\
Ambiente: Pianure Temperate, Colline Rocciose e Sotterranei\\
Organizzazione: Solitario, coppia, branco (3-4) o mandria (5-12)\\
Tesoro: Nessuno\\
\textbf{Descrizione}\\
Le gorgoni sono creature magiche e irascibili: sebbene a prima vista possano sembrare dei costrutti, sotto le piastre metalliche dall'aspetto artificiale sono fatte di carne e ossa. Come tori aggressivi, sfidano qualsiasi creatura sconosciuta che incontrano, spesso travolgendo il cadavere del loro avversario o frantumando i suoi resti pietrificati finché la creatura non è più riconoscibile. Le femmine sono pericolose quanto i maschi, e i due sessi hanno l'identico aspetto. Una tipica gorgone è alta 1,8 metri e lunga 2,4 metri. Pesa circa 2.000 kg.\\

Le gorgoni ricavano il loro nutrimento consumando minerali, in particolare la pietra delle loro vittime pietrificate, e ogni statua da loro creata viene completamente divorata. Non possono digerire metallo o gemme, così il loro sterco (che assomiglia a polvere grigia dall'odore acre) spesso contiene piccoli cristalli grezzi e pepite d'oro. La loro aggressività verso tutte le altre creature fa sì che nei loro pascoli siano pochi, se non nessuno, i predatori e le prede. Ogni mandria è guidata da un toro dominante; le gorgoni solitarie sono generalmente tori adolescenti allontanati dalla mandria del toro dominante.\\

La loro carne è dura e muscolosa (una volta che viene rimossa l'armatura), e per coloro che la assaggiano è abbastanza nutriente. Molte tribù di giganti della pietra credono che mangiare la carne di gorgone aumenti la loro armatura naturale. Le corna di gorgone polverizzate valgono 250 mo come componente materiale alternativo per gli oggetti magici ed incantesimi che agiscono fulla Forza o Pietra.\\


\medskip\index{Mostri - Grick}\textbf{Grick}

\emph{Media mostruosità, neutrale}

\textbf{FORZA} +2

\textbf{DESTREZZA} +2

\textbf{COSTITUZIONE} +0

\textbf{INTELLIGENZA} -4

\textbf{SAGGEZZA} +2

\textbf{CARISMA} -3

\textbf{Iniziativa} +2 -- \textbf{Difesa} 15

\textbf{Punti Ferita} 27 (6d8)

\textbf{Movimento} 9 m, scalata 9 m

\textbf{Tiri Salvezza}: Tempra +3, Riflessi +3, Volontà +2

\textbf{Resistenza al Danno} da botta, perforante e tagliente di attacchi non magici

\textbf{Sensi} scurovisione 18 m

\textbf{Linguaggi} -

\textbf{Sfida} 2 (450 PE)

\emph{\textbf{Camuffamento di Pietra.}} Il grick ha +1d6 alle prove di Destrezza (Nascondersi) effettuate per nascondersi su terreno roccioso.

\textbf{Azioni}

\emph{\textbf{Multiattacco.}} Il grick effettua un attacco con i suoi tentacoli. Se l'attacco colpisce, il grick può effettuare un attacco di becco contro lo stesso bersaglio.

\emph{\textbf{Tentacoli.} Attacco con arma da mischia}: +4 a colpire, portata 1 m, un bersaglio.

\emph{Colpisce:} 9 (2d6 + 2) danni taglienti.

\emph{\textbf{Becco.} Attacco con arma da mischia}: +4 a colpire,
portata 1 m, un bersaglio.

\emph{Colpisce:} 5 (1d6 + 2) danni perforanti.

\textbf{Ecologia}: \\
Ambiente: Qualsiasi Sotterraneo\\
Organizzazione: Solitario o ammasso (2-5)\\
Tesoro: Accidentale\\

\textbf{Descrizione}
Il vermiforme grick è il terrore delle caverne e dei cunicoli in cui abita, attendendo in agguato nei pressi di tunnel molto trafficati o città sotterranee, per balzare fuori dal buio e catturare le sue prede. È raro che tali prede vengano consumate sul posto. Il grick preferisce portare il cibo fresco nella sua tana, uno stretto cunicolo o sull'alta sporgenza di una caverna, dove può consumarlo con piccoli morsi, in tranquillità.\\
Le origini del grick sono ignote. E anche se ha una rudimentale intelligenza, non ha alcuna società di cui parlare, e la maggior parte delle volte si incontrano singoli esemplari. Nelle occasioni in cui i viaggiatori sfortunati ne incontrano più di uno, i gruppi di grick non sembrano comunicare o lavorare tra loro: ognuno attacca invece obiettivi individuali e si ritira col suo bottino non appena riesce ad abbattere un avversario. Predatori capaci, i grick hanno anche una strana pelle resistente alle armi che li rende particolarmente pericolosi. Molti avventurieri inesperti sono periti sotto l'attacco di un grick semplicemente perché non erano in grado di danneggiare la creatura con le loro armi non magiche. Coloro che hanno familiarità con i grick (soprattutto i Nani, i Morlock e i Trogloditi) sanno che la migliore strategia per affrontarli è quella di ritirarsi e attendere rinforzi più potenti o magici.\\
I grick contano sul loro colore scuro e sulla capacità di scalare i muri per tenersi fuori vista, finché non sono pronti a scattare all'attacco. In più occasioni quando il cibo scarseggia in una determinata regione, i grick si dirigono verso la superficie e vagano per il deserto in cerca di prede, ma questi soggiorni sono quasi sempre per necessità, e alla fine i grick trovano presto entrate a nuove tane sotterranee. Preferiscono le tenebre e la comodità di un "tetto" sopra la testa, evitando il cielo aperto e facendo molto per restare coperto da alberi, nuvole basse o edifici.\\



\medskip\index{Mostri - Grifone}\textbf{Grifone}

\emph{Grande mostruosità, disallineato}

\textbf{FORZA} +4

\textbf{DESTREZZA} +2

\textbf{COSTITUZIONE} +3

\textbf{INTELLIGENZA} -4

\textbf{SAGGEZZA} +1

\textbf{CARISMA} -1

\textbf{Iniziativa} +2 -- \textbf{Difesa} 13

\textbf{Punti Ferita} 59 (7d10 + 21)

\textbf{Movimento} 9 m, volo 24 m

\textbf{Tiri Salvezza}: Tempra +7, Riflessi +6, Volontà +4

\textbf{Competenze} Consapevolezza +5

\textbf{Sensi} scurovisione 18 m

\textbf{Linguaggi} -

\textbf{Sfida} 2 (450 PE)

\emph{\textbf{Vista Affinata.}} Il grifone ha +1d6 nelle prove di Saggezza (Consapevolezza) basate sulla vista.

\textbf{Azioni}

\emph{\textbf{Multiattacco.}} Il grifone effettua due attacchi: uno con il becco e uno con gli artigli.

\emph{\textbf{Artigli.} Attacco con arma da mischia}: +6 a colpire, portata 1 m, un bersaglio.

\emph{Colpisce:} 11 (2d6 + 4) danni taglienti.

\emph{\textbf{Becco.} Attacco con arma da mischia}: +6 a colpire, portata 1 m, un bersaglio.

\emph{Colpisce:} 8 (1d8 + 4) danni perforanti.

\textbf{Ecologia}\\
Ambiente: Colline Temperate\\
Organizzazione: Solitario, coppia o branco (6-10)\\
Tesoro: Accidentale\\
\textbf{Descrizione}\\
I grifoni sono potenti predatori aerei, che piombano dai loro altissimi nidi per afferrare le loro prede con il becco e gli artigli. Aggressivi e territoriali, non sono semplici bestie, bensì combattenti astuti e compagni leali verso coloro che si guadagnano il loro rispetto, combattendo fino alla morte per proteggere i loro amici e i loro simili.\\

Del peso di oltre 250 kg e lungo 2,4 metri, dal becco aguzzo alla coda crestata, il grifone possiede un profilo imponente che è da tempo usato in araldica e in altre iconografie come simbolo di potenza, autorità e giustizia. In realtà, il grifone è meno interessato a concetti astratti e più a cacciare per nutrirsi e difendersi. Sebbene a volte possano essere addestrati o diventino amici per servire da cavalcatura, i grifoni non possiedono un'innata affinità con gli umanoidi, e spesso entrano in sanguinosi conflitti con le razze civilizzate nel tentativo di procurarsi il loro cibo preferito: carne di cavallo. La gente di città può meravigliarsi di fronte allo stile maestoso di un grifone addestrato e alla sua apertura alare di 7 metri, ma quei contadini costretti a condividere il territorio con la sua specie sanno che conviene affrettarsi in casa e mettere al sicuro le loro greggi quando sentono le urla di caccia delle bestie.\\

I grifoni si accoppiano per la vita, e spesso per anni cercano vendetta per l'uccisione del compagno o di un figlio. È stata proprio per questa innata caparbietà e fiera lealtà che li ha portati nell'uso domestico come cavalcature e guardiani di tesori. Nonostante l'insito pericolo, il commercio di grifoni catturati e di uova rubate è fervido, con le uova che valgono fino a 3.500 mo l'una e i giovani vivi fino a 7.000. I personaggi che desiderano un grifone come cavalcatura, però, dovrebbero sapere che comprare o addomesticare con la violenza le creature intelligenti come i grifoni è ritenuto schiavitù dalla maggior parte delle divinità buone, e guadagnarsi la spontanea lealtà di un grifone non è un compito facile. Raggiungere un mutuo accordo (o perfino l'amicizia) è una strada molto più elegante e sicura per assicurarsi un grifone come cavalcatura.\\

Prima che lo si possa cavalcare in combattimento, un grifone deve fare pratica nel portare il peso del suo cavaliere. Per essere ben addestrato, un grifone deve per prima cosa avere un atteggiamento amichevole verso il suo addestratore (con una prova di Addestrare Animali, Diplomazia o Intimidire). Dopodiché, 6 settimane di pratica e una prova riuscita di Addestrare Animali con DC 20 sono sufficienti perché la bestia sia a suo agio con il carico e, per la loro intelligenza, si può ritenere che i grifoni addestrati conoscano tutti i trucchi elencati nella descrizione dell'abilità Addestrare Animali, ed è anche possibile che imparino nuovi comandi, impartendo semplici richieste in Comune.\\

I grifoni possono portare fino a 150 kg come carico leggero, 300 kg come carico medio e 450 kg come carico pesante. Per cavalcare un grifone è necessaria una sella esotica.\\


\medskip\index{Mostri - Grimlock}\textbf{Grimlock}

\emph{Media umanoide (grimlock), neutrale malvagio}

\textbf{FORZA} +3

\textbf{DESTREZZA} +1

\textbf{COSTITUZIONE} +1

\textbf{INTELLIGENZA} -1

\textbf{SAGGEZZA} -1

\textbf{CARISMA} -2

\textbf{Iniziativa} +1 -- \textbf{Difesa} 12

\textbf{Punti Ferita} 11 (2d8 + 2)

\textbf{Movimento} 9 m

\textbf{Tiri Salvezza}: Tempra +3, Riflessi +1, Volontà +0

\textbf{Competenze} Acrobatica +5, Muoversi Silenziosamente / Nascondersi +3, Consapevolezza +3

\textbf{Immunità alle Condizioni} accecato

\textbf{Sensi} vista cieca 9 m o 3 m se assordato (cieco oltre questo raggio) 

\textbf{Linguaggi} Linguaggio delle Profondità

\textbf{Sfida} 1/4 (50 PE)

\emph{\textbf{Camuffamento di Pietra.}} Il grimlock ha +1d6 alle prove di Destrezza (Nascondersi) effettuate per nascondere su terreni rocciosi.

\emph{\textbf{Sensi Ciechi.}} Il grimlock non può usare la vista cieca mentre è assordato e non più fiutare.

\emph{\textbf{Olfatto e Udito Affinati.}} Il grimlock ha +1d6 alle prove di Saggezza (Consapevolezza) basate su udito o olfatto. 

\textbf{Azioni}

\emph{\textbf{Randello d'Osso Appuntito.} Attacco con arma da mischia}:
+5 a colpire, portata 1 m, un bersaglio.

\emph{Colpisce:} 5 (1d4 + 3) danni da botta più 2 (1d4) danni perforanti.

\emph{\textbf{Arco Lungo.} Attacco con arma a Distanza}: +3 a colpire, gittata 45m, un bersaglio.

\emph{Colpisce:} 5 (1d8 + 1) danni perforanti.

\textbf{Ecologia}\\
I Grimlock abitano gli insediamenti abbandonati di altre Razze e sono spesso trovati come Schiavi di altre creature più organizzate, come i Duergar ed Elfi. Si ritiene che si trattino di una propaggine ancora più degenerata dei Morlock, che viaggiano da Sekamina per cacciare i Grimlock per il cibo e considerano la loro carne una delicatezza.\\
\textbf{Descrizione}\\
I Grimlock sono creature umane cieche e selvagge che abitano nel regno delle Darklands di Nar-Voth, dove si organizzano in piccoli gruppi tribali.\\


\medskip\index{Mostri - Guardiano Protettore}\textbf{Guardiano Protettore}

\emph{Grande costrutto, disallineato}

\textbf{FORZA} +4

\textbf{DESTREZZA} -1

\textbf{COSTITUZIONE} +4

\textbf{INTELLIGENZA} -2

\textbf{SAGGEZZA} +0

\textbf{CARISMA} -4

\textbf{Iniziativa} -1 -- \textbf{Difesa} 21

\textbf{Punti Ferita} 142 (15d10 + 60)

\textbf{Movimento} 9 m

\textbf{Tiri Salvezza}: Tempra +6, Riflessi +1, Volontà +2

\textbf{Immunità al Danno} veleno

\textbf{Immunità alle Condizioni} affascinato, avvelenato, paralizzato, affaticamento, spaventato

\textbf{Sensi} scurovisione 18 m, vista cieca 3 m

\textbf{Linguaggi} comprende i comandi forniti in qualsiasi lingua ma non può parlare

\textbf{Sfida} 7 (2.900 PE)

\emph{\textbf{Accumulare Incantesimi.}} Un incantatore che indossi l'amuleto del guardiano protettore può far sì che il guardiano accumuli un incantesimo di Difficoltà 23 o più basso. Per farlo, l'incantatore deve lanciare l'incantesimo sul guardiano. L'incantesimo non ha effetto ma viene accumulato all'interno del guardiano. Quando gli viene comandato di farlo da chi indossa l'amuleto o si presenta una situazione predeterminata dall'incantatore, il guardiano lancia l'incantesimo accumulato con tutti i parametri predisposti dall'incantatore originale, senza bisogno di componenti. Quando l'incantesimo viene lanciato o qualsiasi nuovo incantesimo viene accumulato, tutti gli incantesimi precedentemente accumulati vengono persi.

\emph{\textbf{Natura di Costrutto.}} Il guardiano non ha bisogno di aria, cibo, bevande o sonno.

\emph{\textbf{Rigenerazione.}} Il guardiano protettore recupera 10 punti ferita all'inizio del proprio turno se ne possiede ancora almeno 1.

\emph{\textbf{Vincolato.}} Il guardiano protettore è vincolato magicamente ad un amuleto. Finché il guardiano e l'amuleto sono sullo stesso piano di esistenza, chi indossa l'amuleto può richiamare telepaticamente il guardiano perché lo raggiunga, e il guardiano saprà la distanza e la direzione in cui si trova l'amuleto. Se il guardiano si trova entro 18 metri da chi indossa l'amuleto, metà dei danni subiti da chi lo indossa (arrotondati per difetto) vengono trasferiti al guardiano. Se l'amuleto viene distrutto, il guardiano è inabile finché non viene creato un amuleto di rimpiazzo. L'amuleto del guardiano può essere soggetto ad un attacco diretto qualora non sia indossato o trasportato da nessuno. Ha Difesa 10, 10 punti ferita e immunità ai danni psichici e da veleno. Costruire un amuleto richiede 1 settimana e costa 10.000 mo in componenti.

\textbf{Azioni}

\emph{\textbf{Multiattacco.}} Il golem effettua due attacchi di pugno.

\emph{\textbf{Pugno.} Attacco con arma da mischia}: +7 a colpire,
portata 1 m, un bersaglio.

\emph{Colpisce:} 11 (2d6 + 4) danni da botta.

\textbf{Reazioni}

\emph{\textbf{Scudo.}} Quando una creatura attacca chi indossa l'amuleto del guardiano, il guardiano conferisce un bonus di +2 alla sua Difesa, se entro 1 metro dal suo controllore.

\medskip\index{Mostri - Hobgoblin}\textbf{Hobgoblin}

\emph{Media umanoide (goblinoide), legale malvagio}

\textbf{FORZA} +1

\textbf{DESTREZZA} +1

\textbf{COSTITUZIONE} +1

\textbf{INTELLIGENZA} +0

\textbf{SAGGEZZA} +0

\textbf{CARISMA} -1

\textbf{Iniziativa} +1 -- \textbf{Difesa} 19 (armatura di maglia, scudo)

\textbf{Punti Ferita} 11 (2d8 + 2)

\textbf{Movimento} 9 m

\textbf{Tiri Salvezza}: Tempra +5, Riflessi +2, Volontà +1

\textbf{Sensi} scurovisione 18 m

\textbf{Linguaggi} Comune, Goblin

\textbf{Sfida} 1/2 (100 PE)

\emph{\textbf{+1d6 Marziale.}} Una volta per turno, l'hobgoblin può infliggere 7 (2d6) danni aggiuntivi ad una creatura che colpisce con un attacco con arma, se quella creatura si trova entro 1 metro da un alleato dell'hobgoblin che non sia inabile.

\textbf{Azioni}

\emph{\textbf{Spada Lunga.} Attacco con arma da mischia}: +3 a colpire, portata 1 m, un bersaglio.

\emph{Colpisce:} 5 (1d8 + 1) danni taglienti o 6 (1d10 + 1) danni taglienti se usata con due mani.

\emph{\textbf{Arco Lungo.} Attacco con arma a Distanza}: +3 a colpire, gittata 45m, un bersaglio.

\emph{Colpisce:} 5 (1d8 + 1) danni perforanti.

\textbf{Ecologia}\\
Ambiente: Colline Temperate\\
Organizzazione: Gruppo (4-9), banda da guerra (10-24) o tribù (25+ più 50\% non combattenti, 1 sergente di 3° livello per 20 adulti, 1 o 2 luogotenenti di 4° o 5° livello, 1 capo di 6°-8° livello, 6-12 Leopardi e 1-4 Ogre o 1-2 Troll)\\
Tesoro: Equipaggiamento da PNG (Corazza di Cuoio Borchiato, Scudo Leggero di Metallo, Spada Lunga, Arco Lungo con 20 Frecce, altro tesoro)\\
\textbf{Descrizione}\\
Gli Hobgoblin sono militaristi e prolifici, una combinazione che li rende molto pericolosi in alcune regioni. Procreano rapidamente, rimpiazzando i membri caduti con nuovi soldati mantenendo costante il loro numero indipendentemente dalle sorti della guerra. Generalmente basta poco perché dichiarino guerra, ma nella maggior parte dei casi il motivo è per catturare nuovi Schiavi: la vita da Schiavi in un covo di Hobgoblin è brutale e breve, e nuovi Schiavi sono sempre necessari per rimpiazzare quelli che muoiono o che vengono mangiati.\\
Tra tutte le Razze Goblinoidi quella degli Hobgoblin è di gran lunga la più civilizzata.
Vedono i più grandi e solitari Bugbear come strumenti da assoldare e usare dove necessario, di solito per specifiche missioni che richiedono l'omicidio e il furto, e guardano alla specie più piccola dei Goblin con un misto di vergogna e frustrazione. Gli Hobgoblin ammirano la tenacia dei Goblin, sebbene la natura imprevedibile e la passione per il fuoco dei loro minuscoli parenti li rende sgradite aggiunte a tribù o insediamenti Hobbgoblin. Tuttavia, la maggior parte delle tribù Hobgoblin include un piccolo gruppo di Goblin, che normalmente si nascondono negli angoli peggiori dell'insediamento.\\
Molte tribù Hobgoblin uniscono l'amore per la guerra con l'intelletto acuto. La Scienza delle macchine d'assedio, l'Alchimia e le complesse imprese di ingegneria affascinano la maggior parte degli Hobgoblin, e quelli particolarmente dotati vengono trattati da eroi e ottengono sempre delle posizioni di alto rango nella tribù. Gli Schiavi dalle menti raffinate vengono apprezzati, perciò le incursioni nelle città Naniche sono cosa ordinaria.\\
È risaputo che gli Hobgoblin diffidano della Magia e la disprezzano, in particolar modo quella Arcana. I loro Sciamani sono considerati con un misto di paura e rispetto, e vengono normalmente costretti a vivere da soli ai margini del covo della tribù. Non si è mai sentito di Hobgoblin che pratichino la Magia Arcana o, come dicono gli Hobgoblin, la "Magia degli Elfi". Questa è la causa del loro odio per la Magia: gli Hobgoblin odiano gli Elfi.\\
Un Hobgoblin è alto 1 metro e pesa 80 kg.\\



\medskip\index{Mostri - Idra}\textbf{Idra}

\emph{Enorme mostruosità, disallineato}

\textbf{FORZA} +5

\textbf{DESTREZZA} +1

\textbf{COSTITUZIONE} +5

\textbf{INTELLIGENZA} -4

\textbf{SAGGEZZA} +0

\textbf{CARISMA} -2

\textbf{Iniziativa} +1 -- \textbf{Difesa} 19

\textbf{Punti Ferita} 172 (15d12 + 75)

\textbf{Movimento} 9 m, nuoto 9 m

\textbf{Tiri Salvezza}: Tempra +8, Riflessi +7, Volontà +3

\textbf{Competenze} Consapevolezza +6

\textbf{Sensi} scurovisione 18 m

\textbf{Linguaggi} -

\textbf{Sfida} 8 (3.900 PE)

\emph{\textbf{Teste Multiple.}} L'idra ha cinque teste. Finché ha più di una testa, l'idra ha +1d6 ai Tiri Salvezza contro le condizioni accecata, affascinata, assordata, spaventata, stordita o svenuta.

Ogni volta che l'idra subisce 25 o più danni in un singolo turno, una delle sue teste muore. Se tutte le teste muoiono, anche l'idra muore.

Al termine del suo turno, l'idra ricresce due teste per ciascuna delle sue teste uccise dal suo ultimo turno, a meno che non abbia subito danno da fuoco dal suo ultimo turno. L'idra recupera 10 punti ferita per ogni testa ricresciuta in questo modo.

\emph{\textbf{Teste Reattive.}} Per ogni testa posseduta oltre la prima, l'idra riceve una reazione extra che può essere usata solo per compiere attacchi di opportunità.

\emph{\textbf{Trattenere il Fiato.}} L'idra può trattenere il fiato per 1 ora.

\emph{\textbf{Veglia.}} Mentre l'idra dorme, almeno una delle sue teste resta sveglia.

\textbf{Azioni}

\emph{\textbf{Multiattacco.}} L'idra effettua tanti attacchi di morso quante sono le sue teste.

\emph{\textbf{Morso.} Attacco con arma da mischia}: +8 a colpire, portata 3 m, un bersaglio.

\emph{Colpisce:} 10 (1d10 + 5) danni perforanti.

\textbf{Ecologia}\\
Ambiente: Paludi Temperate\\
Organizzazione: Solitario\\
Tesoro: Standard\\
\textbf{Descrizione}\\
L'idra e' un drago a piu' teste, ma stupido.\\


\medskip\index{Mostri - Ippogrifo}\textbf{Ippogrifo}

\emph{Grande mostruosità, disallineato}

\textbf{FORZA} +3

\textbf{DESTREZZA} +1

\textbf{COSTITUZIONE} +1

\textbf{INTELLIGENZA} -4

\textbf{SAGGEZZA} +1

\textbf{CARISMA} -1

\textbf{Iniziativa} +1 -- \textbf{Difesa} 12

\textbf{Punti Ferita} 19 (3d10 + 3)

\textbf{Movimento} 12 m, volo 18 m

\textbf{Tiri Salvezza}: Tempra +5, Riflessi +5, Volontà +2

\textbf{Competenze} Consapevolezza +5

\textbf{Linguaggi} -

\textbf{Sfida} 1 (200 PE)

\emph{\textbf{Vista Affinata.}} L'ippogrifo ha +1d6 nelle prove di Saggezza (Consapevolezza) basate sulla vista.

\textbf{Azioni}

\emph{\textbf{Multiattacco.}} L'ippogrifo effettua due attacchi: uno con il becco e uno con gli artigli.

\emph{\textbf{Artigli.} Attacco con arma da mischia}: +5 a colpire, portata 1 m, un bersaglio.

\emph{Colpisce:} 10 (2d6 + 3) danni taglienti.

\emph{\textbf{Becco.} Attacco con arma da mischia}: +5 a colpire,
portata 1 m, un bersaglio.

\emph{Colpisce:} 8 (1d10 + 3) danni perforanti.

\textbf{Ecologia}\\
Ambiente: Colline Temperate o Pianure\\
Organizzazione: Solitario, coppia o stormo (7-12)\\
Tesoro: Nessuno\\
\textbf{Descrizione}\\
l'ippogrifo ha le ali, le zampe anteriori e la testa di un grande rapace e la coda e il corpo di un magnifico cavallo. Dato che i cavalli sono il cibo preferito dei grifoni, gli studiosi affermano che un mago con senso dell'umorismo tanto tempo fa creò come scherzo questa sfortunata fusione tra un cavallo e un falco.\\

Le piume dell'ippogrifo hanno una colorazione simile a quelle di un falco o di un'aquila; tuttavia, alcuni allevatori sono riusciti a produrre degli esemplari con piume completamente bianche o color carbone. Il torso di un ippogrifo e le estremità posteriori sono molto spesso di colore baio, nocciola o grigio, con alcuni manti che mostrano colorazioni pezzate o anche palomino. Un ippogrifo è lungo 3,3 metri e pesa fino a 680 kg.\\

I territoriali ippogrifi proteggono ferocemente il loro dominio. Gli ippogrifi devono anche sorvegliare i cieli a causa degli altri predatori, dato che sono il cibo preferito di grifoni, viverne e giovani draghi. Gli ippogrifi nidificano nelle vaste praterie erbose, aspre colline e f luenti praterie. Ippogrifi eccezionalmente resistenti stabiliscono le loro dimore all'interno di nicchie o mura di canyon, da cui setacciano i deserti rocciosi alla ricerca di coyote, cervi e a volte umanoidi. Gli ippogrifi preferiscono i mammiferi, tuttavia brucano erba dopo qualsiasi pasto di carne per aiutare la digestione. Queste loro abitudini dietetiche possono risultare pericolose sia per gli allevatori che per le loro mandrie, così spesso le comunità di allevatori mettono delle ricompense su di loro. Le vittime di queste partite di caccia vengono spesso imbalsamate, e di frequente degli ippogrifi imbalsamati decorano le taverne di frontiera e gli avamposti sperduti.\\

Di gran lunga più facili da addestrare rispetto ai grifoni e intelligenti quanto i cavalli, gli ippogrifi vengono addestrati come animali da monta da alcune compagnie scelte di soldati a cavallo, che pattugliano i cieli e piombano addosso ai nemici inconsapevoli. Sebbene siano bestie magiche, se catturati da giovani gli ippogrifi possono venire addestrati grazie a Addestrare Animali come fossero degli animali. Un ippogrifo adulto è molto più difficile da addestrare, e per farlo bisogna seguire le normali regole per l'addestramento delle bestie magiche utilizzando tale abilità. Una sella per ippogrifo deve venire fatta in modo tale da non intralciare i movimenti delle ali della creatura; queste selle sono sempre selle esotiche.\\

Gli ippogrifi sono ovipari: come regola generale, il nido di un ippogrifo contiene solo un uovo alla volta. L'uovo di ippogrifo vale 200 mo, ma un giovane ippogrifo in salute vale 500 mo. Un ippogrifo completamente addestrato come cavalcatura può veder salire il proprio valore fino a 5000 mo o anche di più. Un ippogrifo può trasportare 90 kg come carico leggero, 180 kg come carico medio e 270 kg come carico pesante.\\


\medskip\index{Mostri - Kraken}\textbf{Kraken}

\emph{Mastodontica mostruosità (titano), caotico malvagio}

\textbf{FORZA} +10

\textbf{DESTREZZA} +0

\textbf{COSTITUZIONE} +7

\textbf{INTELLIGENZA} +6

\textbf{SAGGEZZA} +4

\textbf{CARISMA} +5

\textbf{Iniziativa} +6 -- \textbf{Difesa} 30

\textbf{Punti Ferita} 472 (27d20 + 189)

\textbf{Movimento} 6 m, nuoto 18 m

\textbf{Tiri Salvezza}: Tempra +21, Riflessi +12, Volontà +11

\textbf{Immunità al Danno} fulmine, armi +1

\textbf{Immunità alle Condizioni} paralizzato, spaventato

\textbf{Sensi} visione del vero 36 m 

\textbf{Linguaggi} comprende Abissale, Celestiale, Infernale e Druidico ma non può parlare, telepatia 36 m 

\textbf{Sfida} 23 (50.000 PE)

\emph{\textbf{Anfibio.}} Il kraken può respirare aria e acqua.

\emph{\textbf{Libertà di Movimento.}} Il kraken ignora i terreni difficili, e gli effetti magici non possono ridurne la velocità o far sì che diventi intralciato. Può spendere 1 metro di movimento per liberarsi dalle restrizioni non magiche o dall'essere afferrato.

\emph{\textbf{Mostro d'Assedio.}} Il kraken infligge danni doppi agli oggetti e le strutture.

\textbf{Azioni}

\emph{\textbf{Multiattacco.}} Il kraken effettua tre attacchi di tentacolo, ciascuno dei quali può essere rimpiazzato da un uso di Fiondare.

\emph{\textbf{Morso.} Attacco con arma da mischia}: +17 a colpire, portata 1 m, un bersaglio.

\emph{Colpisce:} 23 (3d8 + 10) danni perforanti. Se il bersaglio è una creatura di taglia Grande o inferiore afferrato dal kraken, quella creatura viene inghiottita, e l'afferrare ha termine. Mentre è inghiottita, la creatura è accecata e intralciata, ha copertura totale contro gli attacchi e altri effetti provenienti dall'esterno del kraken, e subisce 42 (12d6) danni da acido all'inizio di ciascun turno del kraken.

Se il kraken subisce 50 o più danni in un singolo turno da una creatura al suo interno, il kraken deve riuscire un Tiro Salvezza di Tempra CD 25 o vomitare tutte le creature inghiottite, che cadono prone in uno spazio entro 3 metri dal kraken. Se il kraken muore, una creatura inghiottita non risulta più intralciata da esso e può fuggire dal cadavere usando 5 metri di movimento, uscendo prona.

\emph{\textbf{Tentacolo.} Attacco con arma da mischia}: +17 a colpire, portata 9 m, un bersaglio.

\emph{Colpisce:} 20 (3d6 + 10) danni da botta, e il bersaglio è afferrato (CD 18 per fuggire). Fino al termine dell'afferrare, il bersaglio è intralciato. Il kraken ha dieci tentacoli, ciascuno dei
quali può afferrare un bersaglio.

\emph{\textbf{Fiondare.}} Un oggetto impugnato o una creatura afferrata dal kraken, di taglia Grande o inferiore viene lanciato di 18 metri in una direzione casuale e gettata prona. Se il bersaglio lanciato colpisce una superficie solida, subisce 3 (1d6) danni da botta per ogni 3 metri percorsi. Se il bersaglio viene lanciato contro un'altra creatura, quella creatura deve riuscire un Tiro Salvezza di Riflessi CD 18 o subire lo stesso danno e cadere prona.

\emph{\textbf{Tempesta di Fulmini.}} Il kraken crea magicamente tre saette di energia, ciascuna delle quali può colpire un bersaglio entro 36 metri e che il kraken possa vedere. Il bersaglio deve effettuare un Tiro Salvezza di Riflessi CD 23, e subire 22 (4d10) danni da fulmine se fallisce il Tiro Salvezza, o la metà se lo riesce.

\textbf{Azioni Aggiuntive}

Il kraken può effettuare 3 Azioni aggiuntive, scelte tra le opzioni seguenti. Può usare solo un'opzione leggendaria alla volta e solo al termine del turno di un'altra creatura. Il kraken recupera le Azioni aggiuntive spese all'inizio del proprio turno.

\textbf{Attacco di Tentacolo o Fiondare.} Il kraken effettua un attacco di tentacolo o usa Fiondare.

\textbf{Nube di Inchiostro (Costa 3 Azioni).} Mentre si trova sott'acqua, il kraken espelle una nube di inchiostro con un raggio di 18 metri. La nube si propaga intorno agli angoli, e quell'area è oscurata pesantemente per tutte le creature tranne il kraken. Ciascuna creatura a parte il kraken che termini il suo turno nell'area deve riuscire un Tiro Salvezza su Tempra 23, subendo 16 (3d10) danni da veleno se fallisce il Tiro Salvezza, o la metà se lo riesce. Una forte corrente disperde lanube, che altrimenti svanisce al termine del prossimo turno  del kraken. \textbf{Tempesta di Fulmini (Costa 2 Azioni).} Il kraken usa Tempesta di Fulmini.

\textbf{Ecologia}\\
Ambiente: Qualsiasi Oceano\\
Organizzazione: Solitario\\
Tesoro: Triplo\\
\textbf{Descrizione}\\
Il leggendario kraken è una delle più grandi paure dei marinai, perché è una creatura della taglia di una balena, può colpire delle profondità senza esser visto, può comandare i venti e le condizioni meteorologiche necessarie alla nave per muoversi, e possiede il crudele intelletto della maggior parte dei più spietati e creativi criminali del mondo. Alcuni credono che i kraken siano una punizione divina, mentre altri li ritengono i veri signori delle profondità, che considerano le razze che respirano aria nient'altro che bestiame.\\

Molte leggende sono sorte in merito al fatto che comprenda il linguaggio druidico.

Un kraken è lungo quasi 30 metri e pesa 2.000 kg.\\


\medskip\index{Mostri - Lamia}\textbf{Lamia}

\emph{Grande mostruosità, caotico malvagio}

\textbf{FORZA} +3

\textbf{DESTREZZA} +1

\textbf{COSTITUZIONE} +2

\textbf{INTELLIGENZA} +2

\textbf{SAGGEZZA} +2

\textbf{CARISMA} +3

\textbf{Iniziativa} +2 -- \textbf{Difesa} 15

\textbf{Punti Ferita} 97 (13d10 + 26)

\textbf{Movimento} 9 m

\textbf{Tiri Salvezza}: Tempra +6, Riflessi +9, Volontà +11

\textbf{Competenze} Muoversi Silenziosamente / Nascondersi +3, Ingannare +7, Percepire Emozioni +4,

\textbf{Sensi} scurovisione 18 m

\textbf{Linguaggi} Abissale, Comune

\textbf{Sfida} 4 (1.100 PE)

\emph{\textbf{Incantesimi Innati.}} La caratteristica da incantatore innato della lamia è il Carisma- La lamia può lanciare in maniera innata i seguenti incantesimi, senza bisogno di componenti materiali:

A volontà: \emph{camuffare sé stesso} (qualsiasi forma umanoide)\emph{,} \emph{immagine maggiore}

3/Giorno ciascuno: \emph{charme su persone, immagine speculare,}

\emph{scrutare, suggestione}

1/Giorno: \emph{restrizione}

\textbf{Azioni}

\emph{\textbf{Multiattacco.}} La lamia effettua due attacchi: uno con
gli artigli e uno con il pugnale o il Tocco Intossicante.

\emph{\textbf{Artigli.} Attacco con arma da mischia}: +5 a colpire, portata 1 m, un bersaglio.

\emph{Colpisce:} 14 (2d10 + 3) danni taglienti.

\emph{\textbf{Pugnale.} Attacco con arma da mischia}: +5 a colpire, portata 1 m, un bersaglio.

\emph{Colpisce:} 5 (1d4 + 3) danni perforanti.

\emph{\textbf{Tocco Intossicante.} Attacco con incantesimo in mischia}: +5 a colpire, portata 1 m, una creatura.

\emph{Colpisce:} Il bersaglio è maledetto per 1 ora da questa magia. Fino al termine della maledizione, il bersaglio ha -1d6 ai Tiri Salvezza su Volontà e a tutte le prove di competenza.

\textbf{Ecologia}\\
Ambiente: Deserti Temperati\\
Organizzazione: Solitario, coppia o setta (3-12)\\
Tesoro: Doppio (Pugnale+1, altro tesoro)\\

\textbf{Descrizione}\\
Eredi piene d'odio di un'antica maledizione, le lamie hanno l'aspetto di donne snelle ed attraenti dalla cintola in su, mentre sotto hanno il corpo di un possente leone. Anche le loro fattezze umanoidi portano tratti distintivi dei felini, i loro occhi sono stretti e ferini e i loro denti somigliano alle zanne dei predatori. Una tipica lamia in piedi è alta 1,8 metri, è lunga 2,4 metri e pesa più di 325 kg.\\

Le lamie sono attratte da torrioni in rovina, città abbandonate e monumenti dimenticati che soddisfano i rozzi canoni estetici di queste letali cacciatrici; specie quelli in zone aride o sterili. Tuttavia, le lamie prediligono i templi decrepiti. Provano gioia nel vedere in rovina i templi di divinità buone e deviano dalla loro strada per mettere in difficoltà questi fiorenti luoghi sacri.\\

Le lamie vedono le femmine più anziane del loro gruppo come capi, madri e sciamane, attaccandosi a loro con fanatica reverenza. Anche se le lamie rifuggono dalla maggior parte delle religioni, vedendole come la fonte della maledizione che le ha costrette in queste forme bestiali, le lamie anziane affermano di udire i sussurri del vento che flagella il deserto e di conoscere i freddi capricci delle stelle, e fanno affidamento su queste sorgenti mistiche per guidare il loro popolo.\\

Le lamie presentate qui sono solo gli esponenti più comuni e meno potenti di questa razza maledetta; altre hanno forme serpentine, volanti e anche più perverse.\\


\medskip\index{Mostri - Lich}\textbf{Lich}

\emph{Media non morto, qualsiasi allineamento malvagio}

\textbf{FORZA} +0

\textbf{DESTREZZA} +3

\textbf{COSTITUZIONE} +3

\textbf{INTELLIGENZA} +5

\textbf{SAGGEZZA} +2

\textbf{CARISMA} +3

\textbf{Iniziativa} +5 -- \textbf{Difesa} 28

\textbf{Punti Ferita} 135 (18d8 + 54)

\textbf{Movimento} 9 m

\textbf{Tiri Salvezza}: Tempra +6, Riflessi +7, Volontà +11

\textbf{Resistenze al Danno} freddo, fulmine, da Vuoto

\textbf{Immunità al Danno} veleno; da botta, perforante e tagliente di attacchi non magici

\textbf{Immunità alle Condizioni} affascinato, avvelenato, paralizzato, affaticamento, spaventato

\textbf{Sensi} visione del vero 36 m

\textbf{Linguaggi} Comune più altre cinque lingue

\textbf{Sfida} 21 (33.000 PE)

\emph{\textbf{Incantesimi.}} Il lich ha CM 18. La sua caratteristica da incantatore è l'Intelligenza, +5 a colpire con attacchi da incantesimo). Il lich ha preparati i seguenti incantesimi:

Trucchetti (a volontà): \emph{mano magica, prestidigitazione, raggio} \emph{di gelo}

Difficoltà 16 (4 slot): \emph{dardo incantato, individuazione del magico,} \emph{onda tonante, scudo}

Difficoltà 19 (3 slot): \emph{freccia acida, immagine speculare,} \emph{individuazione dei pensieri, invisibilità}

Difficoltà 21 (3 slot): \emph{animare morti, controincantesimo, dissolvi} \emph{magie, palla di fuoco}

Difficoltà 23 (3 slot): \emph{inaridire, porta dimensionale}

Difficoltà 26 (3 slot): \emph{nube mortale, scrutare}

Difficoltà 29 (1 slot): \emph{disintegrazione, globo di invulnerabilità}

Difficoltà 31 (1 slot): \emph{dito della morte, spostamento planare}

Difficoltà 34 (1 slot): \emph{dominare mostri, parola del potere stordire}

Difficoltà 36 (1 slot): \emph{parola del potere uccidere}

\emph{\textbf{Natura Non Morta.}} Il lich non ha bisogno di aria, cibo, bevande o sonno.

\emph{\textbf{Resistenza Leggendaria (3/Giorno).}} Se il lich fallisce un Tiro Salvezza, può scegliere invece di riuscirvi.

\emph{\textbf{Resistenza allo Scacciare.}} Il lich ha +1d6 ai Tiri Salvezza contro gli effetti che scacciano i non morti.

\emph{\textbf{Rinvigorimento.}} Se possiede un filatterio, il lich distrutto ottiene un nuovo corpo in 1d10 giorni, recuperando tutti i suoi punti ferita e ritornando in attività. Il nuovo corpo compare entro 1 metro dal filatterio.

\emph{\textbf{Sacrifici di Anime.}} Un lich deve periodicamente nutrire di anime il suo filatterio per sostenere la magia che mantiene il suo corpo e la sua coscienza. Per farlo usa l'incantesimo \emph{imprigionare}. Invece di scegliere una delle normali opzioni dell'incantesimo, il lich lo impiega per intrappolare magicamente il corpo e l'anima del bersaglio all'interno del filatterio. Il filatterio deve trovarsi sullo stesso piano del lich, perché questo incantesimo funzioni. Il filatterio di un lich può contenere solo una creatura alla volta, e \emph{dissolvi magie} lanciato come incantesimo di Difficoltà 36 sul filatterio libera qualsiasi creatura imprigionata al suo interno. Una creatura imprigionata nel filatterio per 24 ore viene consumata e distrutta, dopodiché nulla salvo un intervento divino potrà riportarla in vita.

Un lich che dimentichi o non riesca a mantenere il suo corpo con le anime sacrificate inizia a cascare a pezzi, e potrebbe infine trasformarsi in un semilich.

\textbf{Azioni}

\emph{\textbf{Tocco Paralizzante.} Attacco con incantesimo in mischia}: +12 a colpire, portata 1 m, una creatura.

\emph{Colpisce:} 10 (3d6) danni da freddo. Il bersaglio deve riuscire un Tiro Salvezza di Tempra CD 18 o restare paralizzato per 1 minuto. Il bersaglio può ripetere il Tiro Salvezza al termine di ciascun suo turno, terminando l'effetto su di sé in caso di successo.

\textbf{Azioni Aggiuntive}

Il lich può effettuare 3 Azioni aggiuntive, scelte tra le opzioni seguenti. Può usare solo un'opzione leggendaria alla volta e solo al termine del turno di un'altra creatura. Il lich recupera le Azioni aggiuntive spese all'inizio del proprio turno.

\emph{\textbf{Distruggere Vita (Costa 3 Azioni).}} Ogni creatura ad eccezione dei non morti entro 6 metri dal lich deve effettuare un Tiro Salvezza su Tempra CD 18 contro questa magia, subendo 21 (6d6) danni da Vuoto se fallisce il Tiro Salvezza, o la metà di questi danni se lo riesce.

\emph{\textbf{Sguardo Spaventoso (Costa 2 Azioni).}} Il lich fissa il suo sguardo su di una creatura visibile entro 3 metri da esso. Il bersaglio deve riuscire un Tiro Salvezza di Volontà CD 18 contro questa magia o restare spaventato per 1 minuto. Il bersaglio spaventato può ripetere il Tiro Salvezza al termine di ciascun suo turno, terminando l'effetto su di sé in caso di successo. Se il Tiro Salvezza del bersaglio è riuscito o l'effetto per lui ha termine, il bersaglio è immune allo sguardo del lich per le successive 24 ore.

\emph{\textbf{Tocco Paralizzante (Costa 2 Azioni).}} Il lich usa il suo Tocco Paralizzante.

\emph{\textbf{Trucchetto.}} Il lich lancia un trucchetto.

\textbf{Ecologia}\\
Ambiente: Qualsiasi\\
Organizzazione: Solitario\\
Tesoro: Equipaggiamento da PNG (Anello di Protezione +2, Fascia della Sapienza +2 (Consapevolezza), Stivali della Levitazione, pergamena di Dominare Persone, pergamena di Teletrasporto, pozione di Invisibilità)\\

\textbf{Descrizione}
Poche creature sono più temute dei lich. Apice delle arti Necromantiche, il lich è un incantatore che ha scelto di rinunciare alla vita ed ingannare la morte diventando non morto. Anche se molti di coloro che raggiungono simili vette di potenza farebbero di tutto per raggiungere l'immortalità, l'idea di diventare un lich è aborrita da molte creature. Il processo prevede di estrarre la forza vitale dell'incantatore e imprigionarla in un filatterio preparato in modo speciale; l'incantatore cede la sua vita, ma rimane intrappolato tra la vita e la morte, e fintanto che il suo filatterio rimane intatto può continuare le sue ricerche e il suo lavoro senza temere il passare del tempo.\\



\medskip\index{Mostri - Lucertoloide}\textbf{Lucertoloide}

\emph{Media umanoide (lucertoloide), neutrale}

\textbf{FORZA} +2

\textbf{DESTREZZA} +0

\textbf{COSTITUZIONE} +1

\textbf{INTELLIGENZA} -2

\textbf{SAGGEZZA} +1

\textbf{CARISMA} -2

\textbf{Iniziativa} +0 -- \textbf{Difesa} 16 (armatura naturale, scudo)

\textbf{Punti Ferita} 22 (4d8 + 4)

\textbf{Movimento} 9 m, nuoto 9 m

\textbf{Tiri Salvezza}: Tempra +4, Riflessi +0, Volontà +0

\textbf{Competenze} Muoversi Silenziosamente / Nascondersi +4, Consapevolezza +3, Sopravvivenza +5

\textbf{Linguaggi} Draconico

\textbf{Sfida} 1/2 (100 PE)

\emph{\textbf{Trattenere il Fiato.}} Il lucertoloide può trattenere il fiato per 15 minuti.

\textbf{Azioni}

\emph{\textbf{Multiattacco.}} Il lucertoloide effettua due attacchi in mischia, ciascuno con un'arma diversa.

\emph{\textbf{Giavellotto.} Attacco con arma da mischia o a Distanza}: +4 a colpire, portata 1 m o gittata 9m, un bersaglio. \emph{Colpisce:} 5 (1d6 + 2) danni perforanti.

\emph{\textbf{Morso.} Attacco con arma da mischia}: +4 a colpire, portata 1 m, un bersaglio.

\emph{Colpisce:} 5 (1d6 + 2) danni perforanti.

\emph{\textbf{Randello Pesante.} Attacco con arma da mischia}: +4 a colpire, portata 1 m, un bersaglio.

\emph{Colpisce:} 5 (1d6 + 2) danni da botta.

\emph{\textbf{Scudo Appuntito.} Attacco con arma da mischia}: +4 a colpire, portata 1 m, un bersaglio.

\emph{Colpisce:} 5 (1d6 + 2) danni perforanti.

\textbf{Ecologia}\\
Ambiente: paludi temperate\\
Organizzazione: solitario, coppia, banda (3-12) o tribù (13-60)\\
Tesoro: Equipaggiamento da PNG (Scudo Pesante di Legno, Morning Star, 3 Giavellotti)\\

\textbf{Descrizione}\\
I lucertoloidi sono rettili predatori orgogliosi e potenti che fanno le loro case comuni in sparuti villaggi nei recessi di paludi e acquitrini. Privi di interesse verso la colonizzazione delle terre aride e soddisfatti delle loro semplici armi e dei rituali che li hanno serviti bene per millenni, i lucertoloidi sono visti da molte delle altre razze come selvaggi retrogradi, ma all'interno delle loro isolate comunità sono in realtà un popolo vitale ricco di tradizioni e con una storia orale che risale a prima che l'uomo camminasse in posizione eretta.\\

La maggior parte dei lucertoloidi è alta dagli 1,8 ai 2,1 metri e pesa dai 100 ai 125 kg, ed ha i possenti muscoli coperti da scaglie grigie, verdi o marroni. Alcune razze hanno piccole creste dorsali o collari dai colori brillanti, e tutte nuotano bene spostandosi con rapidi movimenti della loro possente coda lunga 1,2 metri. Anche se sono pienamente a loro agio in acqua, trattengono il fiato e tornano alle loro abitazioni poste su colline artificiali per riprodursi e dormire. Poiché il loro sangue da rettile li rende lenti al freddo, molti lucertoloidi cacciano e lavorano durante il giorno e si ritirano nelle loro dimore di notte per rannicchiarsi con gli altri della loro tribù a condividere il calore di grandi fuochi di torba.\\

Anche se generalmente sono neutrali, il comportamento scostante dei lucertoloidi, il loro strenuo rifiuto dei "doni" della civilizzazione, e la leggendaria ferocia in battaglia li fa mal giudicare dalla maggioranza degli umanoidi. Questi tratti derivano da buone ragioni, tuttavia, poiché il loro basso tasso di riproduzione non ha eguali tra gli umanoidi a sangue caldo, e se le tribù non difendessero i loro territori paludosi fino all'ultimo respiro si troverebbero presto sopraffatte da orde di mammiferi. Per quanto riguarda la loro propensione a mangiare i corpi dei morti sia amici che nemici, i pratici lucertoloidi sono lesti a sottolineare che la vita è dura nella palude, e nulla deve andare sprecato.\\

I lucertoloidi presentati qui vivono in ambienti paludosi. Le tribù lucertoloidi possono vivere altrettanto bene in altri ambienti, ma come velocità ottengono Scalare 5 metri al posto di Nuotare.\\



\subsection{Mannari}

\medskip\index{Mostri - Cinghiale Mannaro}\textbf{Cinghiale Mannaro}

\emph{Media umanoide (umano, mutaforma), neutrale malvagio}

\textbf{FORZA} +3

\textbf{DESTREZZA} +0

\textbf{COSTITUZIONE} +2

\textbf{INTELLIGENZA} +0

\textbf{SAGGEZZA} +0

\textbf{CARISMA} -1

\textbf{Iniziativa} +0 -- \textbf{Difesa} 12 in forma umanoide, 13 in forma di cinghiale o ibrida

\textbf{Punti Ferita} 78 (12d8 + 24)

\textbf{Movimento} 9 m (12 m in forma di cinghiale)

\textbf{Tiri Salvezza}: Tempra +7, Riflessi +1, Volontà +4

\textbf{Competenze} Consapevolezza +2

\textbf{Immunità al Danno} da botta, perforante e tagliente di attacchi non magici o che non siano argentati 

\textbf{Linguaggi} Comune (non può parlare in forma di cinghiale)

\textbf{Sfida} 4 (1.100 PE)

\emph{\textbf{Carica (Solo Forma di Cinghiale o Ibrida).}} Se il cinghiale mannaro si muove in linea retta di almeno 5 metri verso un bersaglio e poi lo colpisce con le zanne durante lo stesso turno, il bersaglio subisce 7 (2d6) danni taglienti aggiuntivi. Se il bersaglio è una creatura, deve riuscire un Tiro Salvezza di Tempra CD 13 o cadere prono.

\emph{\textbf{Implacabile (Ricarica dopo un 1 ora).}} Se il cinghiale mannaro subisce 14 danni o meno che lo ridurrebbero a 0 punti ferita, scende invece a 1 punto ferita.

\emph{\textbf{Mutaforma.}} Il cinghiale mannaro può usare la sua azione per trasformarsi in un ibrido cinghiale-umanoide o in un cinghiale, o per tornare alla sua vera forma, che è umanoide. Le sue statistiche, a parte la Difesa, sono le stesse in tutte le forme. Qualsiasi equipaggiamento stia indossando o trasportando non viene trasformato. Alla morte ritorna alla sua vera forma.

\textbf{Azioni}

\emph{\textbf{Multiattacco (Solo in Forma Umanoide o Ibrida).}} Il cinghiale mannaro effettua due attacchi, di cui solo uno può essere con le zanne.

\emph{\textbf{Maglio (Soltanto in Forma Umanoide o Ibrida).} Attacco con arma da mischia}: +5 a colpire, portata 1 m, un bersaglio. \emph{Colpisce:} 10 (2d6 + 3) danni da botta.

\emph{\textbf{Zanne (Soltanto in Forma di Cinghiale o Ibrida).} Attacco con arma da mischia}: +5 a colpire, portata 1 m, un bersaglio. \emph{Colpisce:} 10 (2d6 + 3) danni taglienti. Se il bersaglio è un umanoide, deve riuscire un Tiro Salvezza di Tempra CD 12 o venire maledetto dalla licantropia del cinghiale mannaro.

\textbf{Ecologia}\\
Ambiente: Qualsiasi Foresta o Pianura\\
Organizzazione: Solitario, coppia, famiglia (3-8) o truppa (3-8 più 1-4 Cinghiali)\\
Tesoro: Equipaggiamento da PNG (Armatura di Cuoio Borchiato, 2 Pugnali, altro tesoro)\\
\textbf{Descrizione}\\
Nella loro forma umanoide, i cinghiali mannari tendono a essere tozzi, con nasi all'insù, pelo ispido e incisivi prominenti. Hanno capelli rossi, castani o neri ma alcuni sono anche biondi, canuti o calvi. Hanno di norma peluria sul labbro superiore, e i maschi di solito non riescono a far crescere la barba. Poiché sono testardi e aggressivi, hanno piccole comunità di loro simili e non si mischiano ai non licantropi: di solito vivono in piccole fattorie dall'aspetto assolutamente normale. Tendono ad avere grandi famiglie e molti figli.\\



\medskip\index{Mostri - Lupo Mannaro}\textbf{Lupo Mannaro}

\emph{Media umanoide (umano, mutaforma), caotico malvagio}

\textbf{FORZA} +2

\textbf{DESTREZZA} +1

\textbf{COSTITUZIONE} +2

\textbf{INTELLIGENZA} +0

\textbf{SAGGEZZA} +0

\textbf{CARISMA} +0

\textbf{Iniziativa} +1 -- \textbf{Difesa} 13 in forma umanoide, 14 in forma di lupo o ibrida

\textbf{Punti Ferita} 58 (9d8 + 18)

\textbf{Movimento} 9 m (12 m in forma di lupo)

\textbf{Tiri Salvezza}: Tempra +5, Riflessi +1, Volontà +2

\textbf{Competenze} Muoversi Silenziosamente / Nascondersi +3, Consapevolezza +4

\textbf{Immunità al Danno} da botta, perforante e tagliente di attacchi non magici o che non siano argentati 

\textbf{Linguaggi} Comune (non può parlare in forma di lupo)

\textbf{Sfida} 3 (700 PE)

\emph{\textbf{Mutaforma.}} Il lupo mannaro può usare la sua azione per trasformarsi in un ibrido lupo-umanoide o in un lupo, o per tornare alla sua vera forma, che è umanoide. Le sue statistiche, a parte la Difesa, sono le stesse in tutte le forme. Qualsiasi equipaggiamento stia indossando o trasportando non viene trasformato. Alla morte ritorna alla sua vera forma.

\emph{\textbf{Udito e Olfatto Affinato.}} Il lupo mannaro ha +1d6 nelle prove di Saggezza (Consapevolezza) basate su udito o olfatto.

\textbf{Azioni}

\emph{\textbf{Multiattacco (Soltanto in Forma Umanoide o Ibrida).}} Il lupo mannaro effettua due attacchi: uno con il morso e uno con gli artigli o la lancia.

\emph{\textbf{Artigli (Soltanto in Forma Ibrida).} Attacco con arma da mischia}: +4 a colpire, portata 1 m, una creatura. \emph{Colpisce:} 7 (2d4 + 2) danni taglienti.

\emph{\textbf{Lancia (Soltanto in Forma Umanoide).} Attacco con arma da mischia o a Distanza}: +4 a colpire, portata 1 m o gittata 6m, una creatura.

\emph{Colpisce:} 5 (1d6 + 2) danni perforanti o 6 (1d8 + 2) danni perforanti se usata con due mani in un attacco di mischia.

\emph{\textbf{Morso (Soltanto in Forma di Lupo o Ibrida).} Attacco con arma da mischia}: +4 a colpire, portata 1 m, un bersaglio.

\emph{Colpisce:} 6 (1d8 + 2) danni perforanti. Se il bersaglio è un umanoide, deve riuscire un Tiro Salvezza di Tempra CD 12 o venir maledetto dalla licantropia del lupo mannaro.

\textbf{Ecologia}\\
Ambiente: Qualsiasi Terreno\\
Organizzazione: Solitario, coppia o branco (3-6)\\
Tesoro: Equipaggiamento da PNG (Cotta di Maglia, Spada Lunga, Balestra Leggera con 20 Quadrelli, altro tesoro)\\
\textbf{Descrizione}\\
Nella forma umana i lupi mannari somigliano a persone normali, anche se alcuni tendono ad avere un aspetto ferino e capelli ribelli. Sopracciglia che crescono unendosi, dito indice più lungo del medio e strane voglie sul palmo della mano sono tutti segni comunemente accettati che una persona sia in realtà un lupo mannaro. Naturalmente, questi segni rivelatori non sono sempre accurati, perché questi tratti fisici esistono anche nelle persone normali, ma nelle zone dove i lupi mannari sono un problema comune, questi tratti possono essere considerati schiaccianti a prescindere.\\


\medskip\index{Mostri - Orso Mannaro}\textbf{Orso Mannaro}

\emph{Media umanoide (umano, mutaforma), neutrale buono}

\textbf{FORZA} +4

\textbf{DESTREZZA} +0

\textbf{COSTITUZIONE} +3

\textbf{INTELLIGENZA} +0

\textbf{SAGGEZZA} +1

\textbf{CARISMA} +1

\textbf{Iniziativa} +0 -- \textbf{Difesa} 13 in forma umanoide, 14

in forma di orso o ibrida

\textbf{Punti Ferita} 135 (18d8 + 54)

\textbf{Movimento} 9 m (12 m, scalata 9 m in forma di orso o forma ibrida)

\textbf{Tiri Salvezza}:  Tempra +5, Riflessi +6, Volontà +2

\textbf{Competenze} Consapevolezza +7

\textbf{Immunità al Danno} da botta, perforante e tagliente di attacchi non magici o che non siano argentati

\textbf{Linguaggi} Comune (non può parlare in forma di orso)

\textbf{Sfida} 5 (1.800 PE)

\emph{\textbf{Mutaforma.}} L'orso mannaro può usare la sua azione per trasformarsi in un ibrido orso-umanoide o in un orso, o per tornare alla sua vera forma, che è umanoide. Le sue statistiche, a parte la Difesa, sono le stesse in tutte le forme. Qualsiasi equipaggiamento stia indossando o trasportando non viene trasformato. Alla morte ritorna alla sua vera forma.

\emph{\textbf{Olfatto Affinato.}} L'orso mannaro ha +1d6 nelle prove di Saggezza (Consapevolezza) basate sull'olfatto.

\textbf{Azioni}

\emph{\textbf{Multiattacco.}} In forma di orso, l'orso mannaro effettua due attacchi di artiglio. In forma umanoide, effettua due attacchi di ascia bipenne. In forma ibrida, può attaccare come un orso o un umanoide.  

\emph{\textbf{Artiglio (Soltanto in Forma di Orso o Ibrida).} Attacco con arma da mischia}: +7 a colpire, portata 1 m, un bersaglio. \emph{Colpisce:} 13 (2d8 + 2) danni taglienti.

\emph{\textbf{Ascia Bipenne (Soltanto in Forma Umanoide o Ibrida).} Attacco con arma da mischia}: +7 a colpire, portata 1 m, un bersaglio. \emph{Colpisce:} 10 (1d12 + 4) danni taglienti.

\emph{\textbf{Morso (Soltanto in Forma di Orso o Ibrida).} Attacco con arma da mischia}: +7 a colpire, portata 1 m, un bersaglio.

\emph{Colpisce:} 15 (2d10 + 4) danni perforanti. Se il bersaglio è un umanoide, deve riuscire un Tiro Salvezza di Tempra CD 14 o venir maledetto dalla licantropia dell'orso mannaro.


\textbf{Ecologia}\\
Ambiente: Qualsiasi Foresta\\
Organizzazione: Solitario, coppia, famiglia (3-6) o truppa (3-6 più 1-4 orsi Neri o Grigi)\\
Tesoro: Equipaggiamento da PNG (Giaco di Maglia, Ascia da Battaglia Perfetta, 2 Asce da Lancio Perfette, altro tesoro)\\
\textbf{Descrizione}\\
Nelle loro forme umanoidi, gli orsi mannari tendono a essere muscolosi e con spalle larghe, tratti aspri e occhi scuri. Hanno capelli rossi, castani o neri e sembrano abituati a una vita di duro lavoro. Anche se i più benigni fra i licantropi, sono evitati dalla maggior parte delle persone normali, che temono la loro trasformazione animalesca. Per la maggior parte vivono in zone boschive isolate o in piccole unità familiari della loro stessa specie. Evitano di affrontare gli stranieri, ma non esitano se devono scacciare umanoidi malvagi dai loro territori.\\

\medskip\index{Mostri - Ratto Mannar}\textbf{Ratto Mannaro}

\emph{Media umanoide (umano, mutaforma), legale malvagio}

\textbf{FORZA} +0

\textbf{DESTREZZA} +2

\textbf{COSTITUZIONE} +1

\textbf{INTELLIGENZA} +0

\textbf{SAGGEZZA} +0

\textbf{CARISMA} -1

\textbf{Iniziativa} +2 -- \textbf{Difesa} 13

\textbf{Punti Ferita} 33 (6d8 + 6)

\textbf{Movimento} 9 m

\textbf{Tiri Salvezza}: Tempra +2, Riflessi +5, Volontà +3

\textbf{Competenze} Muoversi Silenziosamente / Nascondersi +4, Consapevolezza +2

\textbf{Immunità al Danno} da botta, perforante e tagliente di attacchi non magici o che non siano argentati

\textbf{Sensi} scurovisione 18 m (solo in forma di ratto)

\textbf{Linguaggi} Comune (non può parlare in forma di ratto)

\textbf{Sfida} 2 (450 PE)

\emph{\textbf{Mutaforma.}} Il ratto mannaro può usare la sua azione per trasformarsi in un ibrido ratto-umanoide o in un ratto, o per tornare alla sua vera forma, che è umanoide. Le sue statistiche, a parte la Difesa, sono le stesse in tutte le forme. Qualsiasi equipaggiamento stia indossando o trasportando non viene trasformato. Alla morte ritorna alla sua vera forma.

\emph{\textbf{Olfatto Affinato.}} Il ratto mannaro ha +1d6 nelle prove di Saggezza (Consapevolezza) basate sull'olfatto.

\textbf{Azioni}

\emph{\textbf{Multiattacco (Solo in Forma Umanoide o Ibrida).}} Il ratto mannaro effettua due attacchi, di cui solo uno può essere con il morso.

\emph{\textbf{Spada Corta (Soltanto in Forma Umanoide o Ibrida).} Attacco con arma da mischia}: +4 a colpire, portata 1 m, un bersaglio. \emph{Colpisce:} 5 (1d6 + 2) danni perforanti.

\emph{\textbf{Balestra a mano (Soltanto in Forma Umanoide o Ibrida).} Attacco con arma a Distanza}: +4 a colpire, gittata 9m, un bersaglio.

\emph{Colpisce:} 5 (1d6 + 2) danni perforanti.

\emph{\textbf{Morso (Soltanto in Forma di Ratto o Ibrida).} Attacco con arma da mischia}: +4 a colpire, portata 1 m, un bersaglio.

\emph{Colpisce:} 4 (1d4 + 2) danni perforanti. Se il bersaglio è un umanoide, deve riuscire un Tiro Salvezza di Tempra CD 11 o venir maledetto dalla licantropia del ratto mannaro.

\textbf{Ecologia}\\
Ambiente: Qualsiasi Urbano\\
Organizzazione: Solitario, coppia, branco (5-10) o gilda (11-30 più 5-12 Ratti Crudeli)\\
Tesoro: Equipaggiamento da PNG (Armatura di Cuoio Borchiato Perfetta, Spada Corta, Balestra Leggera con 20 Quadrelli, altro tesoro)\\
\textbf{Descrizione}\\
I ratti mannari naturali sono bassi, asciutti e muscolosi, con occhi attenti e vispi, e hanno movimenti nervosi. I maschi spesso hanno sottili baffi striminziti.\\


\medskip\index{Mostri - Tigre Mannara}\textbf{Tigre Mannara}

\emph{Media umanoide (umano, mutaforma), neutrale}

\textbf{FORZA} +3

\textbf{DESTREZZA} +2

\textbf{COSTITUZIONE} +3

\textbf{INTELLIGENZA} +0

\textbf{SAGGEZZA} +1

\textbf{CARISMA} +0

\textbf{Iniziativa} +2 -- \textbf{Difesa} 14

\textbf{Punti Ferita} 120 (16d8 + 48)

\textbf{Movimento} 9 m (12 m in forma di tigre)

\textbf{Tiri Salvezza}: Tempra +2, Riflessi +7, Volontà +4

\textbf{Competenze} Muoversi Silenziosamente / Nascondersi +4, Consapevolezza +5

\textbf{Immunità al Danno} da botta, perforante e tagliente di attacchi non magici che non siano argentati

\textbf{Sensi} scurovisione 18 m

\textbf{Linguaggi} Comune (non può parlare in forma di tigre)

\textbf{Sfida} 4 (1.1100 PE)

\emph{\textbf{Balzo.}} Se la tigre mannara si muove di almeno 5 metri in linea retta verso una creatura e la colpisce con un attacco di artiglio durante lo stesso turno, il bersaglio deve riuscire un Tiro Salvezza su Tempra CD 14 o cadere prono. Se il bersaglio è prono, la tigre mannara può effettuare un attacco di morso contro di esso come azione bonus.

\emph{\textbf{Mutaforma.}} La tigre mannara può usare la sua azione per trasformarsi in un ibrido tigre-umanoide o in una tigre, o per tornare alla sua vera forma, che è umanoide. Le sue statistiche, a parte la Difesa, sono le stesse in tutte le forme. Qualsiasi equipaggiamento stia indossando o trasportando non viene trasformato. Alla morte ritorna alla sua vera forma.

\emph{\textbf{Olfatto e Udito Affinato.}} La tigre mannara ha +1d6 nelle prove di Saggezza (Consapevolezza) basate su olfatto e udito.

\textbf{Azioni}

\emph{\textbf{Multiattacco (Solo in Forma Umanoide o Ibrida).}} In forma umanoide, la tigre mannara effettua due attacchi di scimitarra o due attacchi di arco lungo. In forma ibrida, può attaccare come un umanoide o effettuare due attacchi di artiglio.

\emph{\textbf{Artiglio (Soltanto in Forma di Tigre o Ibrida).} Attacco con arma da mischia}: +5 a colpire, portata 1 m, un bersaglio. \emph{Colpisce:} 7 (1d8 + 3) danni taglienti.

\emph{\textbf{Morso (Soltanto in Forma di Tigre o Ibrida).} Attacco con arma da mischia}: +5 a colpire, portata 1 m, un bersaglio.

\emph{Colpisce:} 8 (1d10 + 3) danni perforanti. Se il bersaglio è un umanoide, deve riuscire un Tiro Salvezza di Tempra CD 13 o venir maledetto dalla licantropia della tigre mannara.

\emph{\textbf{Scimitarra (Soltanto in Forma Umanoide o Ibrida).} Attacco con arma da mischia}: +5 a colpire, portata 1 m, un bersaglio. \emph{Colpisce:} 6 (1d6 + 3) danni taglienti.

\emph{\textbf{Arco Lungo (Soltanto in Forma Umanoide o Ibrida).} Attacco con arma a Distanza}: +4 a colpire, gittata 45m, un bersaglio.

\emph{Colpisce:} 6 (1d8 + 2) danni perforanti.

\textbf{Ecologia}
Ambiente: Qualsiasi Pianura o Palude\\
Organizzazione: Solitario o coppia\\
Tesoro: Equipaggiamento da PNG (Armatura di Cuoio Borchiato, Spada Corta, 2 Pugnali, altro tesoro)\\
\textbf{Descrizione}\\
Le tigri mannare in forma umanoide hanno grandi occhi, nasi allungati, zigomi sporgenti e capelli castani o rossi, oppure bianchi, neri o grigio-blu. I loro movimenti sono attenti e aggraziati, e chi li guarda potrebbe scambiarli per un ottimo tagliaborse, un danzatore aggraziato o un'abile cortigiana.\\


\medskip\index{Mostri - Manticora}\textbf{Manticora}

\emph{Grande mostruosità, legale malvagio}

\textbf{FORZA} +3

\textbf{DESTREZZA} +3

\textbf{COSTITUZIONE} +3

\textbf{INTELLIGENZA} -2

\textbf{SAGGEZZA} +1

\textbf{CARISMA} -1

\textbf{Iniziativa} +3 -- \textbf{Difesa} 16

\textbf{Punti Ferita} 68 (8d10 + 24)

\textbf{Movimento} 9 m, volo 15 m

\textbf{Tiri Salvezza}: Tempra +9, Riflessi +7, Volontà +3

\textbf{Sensi} scurovisione 18 m

\textbf{Linguaggi} Comune

\textbf{Sfida} 3 (700 PE)

\emph{\textbf{Ricrescere Spine della Coda.}} La manticora possiede ventiquattro spine nella coda. Le spine usate ricrescono all'alba.

\textbf{Azioni}

\emph{\textbf{Multiattacco.}} La manticora effettua tre attacchi: uno con il morso e due con gli artigli o tre con le spine della coda.

\emph{\textbf{Artiglio.} Attacco con arma da mischia}: +5 a colpire, portata 1 m, un bersaglio.

\emph{Colpisce:} 6 (1d6 + 3) danni taglienti.

\emph{\textbf{Morso.} Attacco con arma da mischia}: +5 a colpire, portata 1 m, un bersaglio.

\emph{Colpisce:} 7 (1d8 + 3) danni perforanti.

\emph{\textbf{Spine della Coda.} Attacco con arma a Distanza}: +5 a colpire, gittata 30m, un bersaglio.

\emph{Colpisce:} 7 (1d8 + 3) danni perforanti.

\textbf{Ecologia}
Ambiente: Colline e Paludi Calde\\
Organizzazione: Solitario, coppia o branco (3-6)\\
Tesoro: Standard\\

\textbf{Descrizione}\\
Le manticore sono feroci predatori che controllano vaste aree in cerca di carne fresca. Una tipica manticora è lunga circa 3 metri e pesa circa 500 kg. Alcune hanno la testa simile a quella di un umano, in genere barbuto. Maschi e femmine sono molto simili.\\

Le manticore mangiano qualsiasi tipo di carne, anche quella delle carogne, ma preferiscono quella umana e raramente si lasciano sfuggire un'occasione di gustare questa delizia. Sono abbastanza furbe e sociali da stringere patti con umanoidi malvagi per formare alleanze o da costringerli ad offre tributi, e molte creature potenti le incaricano di sorvegliare o controllare un posto o una zona. Prediligono fare le loro tane in posti alti, come le sommità delle colline e le caverne tra le rupi.\\

Anche se le manticore sono simili a delle creazioni magiche, sono da tempo annoverate tra le specie naturali. Curiosamente, le manticore sembrano stranamente feconde e possono incrociarsi con numerose altre specie dalla forma simile, inclusi Leoni, Tigri, Lamie, Sfingi e Chimere.\\


\medskip\index{Mostri - Manto Assassino}\textbf{Manto Assassino}

\emph{Grande aberrazione, caotico neutrale}

\textbf{FORZA} +3

\textbf{DESTREZZA} +2

\textbf{COSTITUZIONE} +1

\textbf{INTELLIGENZA} +1

\textbf{SAGGEZZA} +1

\textbf{CARISMA} +2

\textbf{Iniziativa} +2 -- \textbf{Difesa} 18

\textbf{Punti Ferita} 78 (12d10 + 12)

\textbf{Movimento} 3 m, volo 12 m

\textbf{Tiri Salvezza}: Tempra +6, Riflessi +5, Volontà +7

\textbf{Competenze} Muoversi Silenziosamente / Nascondersi +5

\textbf{Sensi} scurovisione 18 m

\textbf{Linguaggi} Linguaggio delle Profondità

\textbf{Sfida} 8 (3.900 PE)

\emph{\textbf{Falso Aspetto.}} Mentre il manto assassino resta immobile senza esporre la parte inferiore del corpo, è indistinguibile da un manto di pelle nera.

\emph{\textbf{Sensibilità alla Luce}}. Mentre è alla luce del sole, il manto assassino ha -1d6 ai tiri per colpire, oltre che alle prove di Saggezza (Consapevolezza) basate sulla vista.

\emph{\textbf{Trasferimento di Danno.}} Mentre è appiccicato ad una creatura, il manto assassino subisce solo la metà dei danni che gli sono inferti (arrotondare per difetto), e la creatura vittima del manto assassino subisce l'altra metà.

\textbf{Azioni}

\emph{\textbf{Multiattacco.}} Il manto assassino effettua due attacchi:
uno con il morso e uno con la coda.

\emph{\textbf{Morso.} Attacco con arma da mischia}: +6 a colpire, portata 1 m, una creatura.

\emph{Colpisce:} 10 (2d6 + 3) danni perforanti, e se il bersaglio è di taglia Grande o inferiore, il manto assassino vi si appiccica. Se il manto assassino ha +1d6 contro il bersaglio, si appiccica alla sua testa e il bersaglio è accecato e impossibilitato a respirare finché il manto assassino vi rimane appiccicato. Mentre appiccicato il manto assassino può effettuare questo attacco solo   contro il bersaglio e ha +1d6 al tiro per colpire. Il manto   assassino può staccarsi spendendo 1 metro di movimento. Una   creatura, compreso il bersaglio, può effettuare la sua azione per   staccare il manto assassino riuscendo una prova di Forza CD 16. 


\emph{\textbf{Coda.} Attacco con arma da mischia}: +6 a colpire, portata 3 m, una creatura.

\emph{Colpisce:} 7 (1d8 + 3) danni taglienti.

\emph{\textbf{Apparizioni (Ricarica dopo un 1 ora).}} Qualora non si trovi sotto luce intensa, il manto assassino crea tre duplicati illusori di sé stesso, che si muovono assieme ad esso e ne imitano le azioni, scambiandosi di posizione per rendere impossibile capire quale sia il reale manto assassino. Se l'originale si trova in un'area di luce intensa, i duplicati svaniscono.

Ogniqualvolta una creatura prenda a bersaglio il manto assassino con un attacco o un incantesimo nocivo mentre sono ancora presenti dei duplicati, quella creatura determina casualmente se prende a bersaglio il manto assassino o uno dei duplicati. Una creatura che non possa vedere o che si affida a sensi diversi dalla vista ignora questo effetto magico.

Un duplicato possiede la Difesa e usa i Tiri Salvezza del manto assassino. Se un attacco colpisce un duplicato, o se un duplicato fallisce un Tiro Salvezza contro un effetto che infligge danni, svanisce.

\emph{\textbf{Gemito.}} Ogni creatura entro 18 metri dal manto assassino, che possa udire il suo gemito e che non sia un'aberrazione, deve riuscire un Tiro Salvezza su Volontà CD 13 o essere spaventata fino al termine del prossimo turno del manto assassino. Se il Tiro Salvezza della creatura
riesce, la creatura è immune al gemito del manto assassino per le successive 24 ore.

\textbf{Ecologia}
Ambiente: Sotterranei\\
Organizzazione: Solitario, coppia, schiera (3-6) o stormo (7-12)\\
Tesoro: Standard\\
\textbf{Descrizione}\\
Simili a mante volanti orribilmente malvagie, i manti assassini sono creature misteriose e paranoiche. Un tipico esemplare ha un'apertura alare di 2,4 metri e pesa 50 kg.\\

Le loro motivazioni sono misteriose e confuse, e diffidano perfino dei loro simili. La strana forma permette loro di essere scambiati per mantelli, arazzi o altri oggetti comuni, e alcune storie narrano di manti assassini che si alleano con altre creature, facendosi trasportare sulla loro schiena e contribuendo alla protezione dei loro alleati per ragioni imperscrutabili. Alcuni esemplari sono sacerdoti di antiche divinità, al comando di culti di manti assassini e Skum intenti a celebrare orribili riti dagli scopi sinistri.\\


\medskip\index{Mostri - Mantoscuro}\textbf{Mantoscuro}

\emph{Piccola mostruosità, disallineato}

\textbf{FORZA} +3

\textbf{DESTREZZA} +1

\textbf{COSTITUZIONE} +1

\textbf{INTELLIGENZA} -4

\textbf{SAGGEZZA} +0

\textbf{CARISMA} -3

\textbf{Iniziativa} +1 -- \textbf{Difesa} 12

\textbf{Punti Ferita} 22 (5d6 + 5)

\textbf{Movimento} 3 m, volo 9 m

\textbf{Tiri Salvezza}: Tempra +5, Riflessi +3, Volontà +0

\textbf{Competenze} Muoversi Silenziosamente / Nascondersi +3

\textbf{Sensi} vista cieca 18 m

\textbf{Linguaggi} -

\textbf{Sfida} 1/2 (100 PE)

\emph{\textbf{Ecolocazione.}} Il mantoscuro non può usare la vista cieca se assordato.

\emph{\textbf{Falso Aspetto.}} Mentre il mantoscuro rimane immobile, è indistinguibile da una formazione rocciosa come una stalattite o una stalagmite.

\textbf{Azioni}

\emph{\textbf{Spaccare.} Attacco con arma da mischia}: +5 a colpire, portata 1 m, una creatura.

\emph{Colpisce:} 6 (1d6 + 3) danni da botta e il mantoscuro si appiccica alla creatura. Se il bersaglio è di taglia Media o inferiore il mantoscuro ha +1d6 al tiro per colpire, si appiccica avvolgendo la testa del bersaglio, che è accecato e impossibilitato a respirare finché il mantoscuro resta appiccicato in questo modo.

Mentre è appiccicato al bersaglio, il mantoscuro non può attaccare nessun'altra creatura salvo il bersaglio, ma ha +1d6 ai suoi tiri per colpire. La velocità del mantoscuro diventa 0 e non può trarre beneficio da nessun bonus alla velocità, muovendosi assieme al bersaglio.

Una creatura può staccare il mantoscuro con un'azione e riuscendo una prova di Forza CD 13. Durante il suo turno, il mantoscuro può staccarsi dal bersaglio da solo usando 1 metro di movimento.

\emph{\textbf{Aura di Oscurità (1/Giorno).}} Un'oscurità magica con 5 metri di raggio si estende dal mantoscuro, muovendosi con esso, e propagandosi oltre gli angoli. L'oscurità permane finché il mantoscuro mantiene la concentrazione, massimo 10 minuti (come se si stesse concentrando su di un incantesimo). La scurovisione non può penetrare questa oscurità, né essa può essere rischiarata da alcuna luce naturale. Se qualsiasi parte dell'oscurità si sovrappone ad un'area di luce generata da un incantesimo di Difficoltà 19 o inferiore, l'incantesimo che sta creando la luce viene dissolto.

\textbf{Ecologia}
Ambiente: Qualsiasi (sotterraneo)\\
Organizzazione: Solitario, coppia o nidiata (3-12)\\
Tesoro: Nessuno\\
\textbf{Descrizione}\\
l'apertura tentacolare di un mantoscuro ha un'ampiezza di poco inferiore agli 1 m; quando è appeso alla volta di una caverna, mascherato da stalattite, la sua lunghezza varia tra i 60 ed i 90 cm. Un esemplare tipico di mantoscuro pesa 20 kg. La testa ed il corpo della creatura sono solitamente del colore del basalto o del granito scuro, ma i suoi tentacoli membranosi possono cambiare colore per adattarsi all'ambiente circostante.\\

I mantoscuro non sono scalatori particolarmente abili, ma sono in grado di appendersi alla volta di una caverna come i pipistrelli, agganciati per mezzo degli uncini posti in fondo ai loro tentacoli, così che il loro corpo penzolante risulti quasi indistinguibile da una stalattite. Da questa postazione nascosta la creatura attende che la preda passi sotto di lei e, a questo punto, si stacca lanciandosi verso di essa, sbattendo contro il bersaglio e tentando di avvolgervi attorno i suoi membranosi tentacoli. Se il mantoscuro manca la preda, risale e si lancia nuovamente contro la preda, fino a quando quest'ultima non viene sconfitta o il mantoscuro è gravemente ferito (nel qual caso svolazza sul soffitto per nascondersi, sperando che la sua "preda" lo lasci perdere). La capacità innata di questa creatura di celare la zona circostante per mezzo dell'oscurità magica le offre un ulteriore vantaggio contro gli avversari che necessitano della luce per vedere.\\

I mantoscuro preferiscono vivere e cacciare nelle caverne e nei cunicoli più vicini alla superficie, dal momento che questi offrono un più frequente passaggio di prede che questi mostri possono cacciare. Non si limitano però a queste caverne buie e talvolta possono essere incontrati in fortezze abbandonate o persino nelle fogne delle città affollate. Qualsiasi luogo dove abbondi il cibo e ci sia un soffitto a cui appendersi è un possibile covo per un mantoscuro.\\

Il ciclo vitale di un mantoscuro è rapido: i piccoli diventano adulti nell'arco di pochi mesi e la maggior parte muore di vecchiaia dopo pochi anni. Di conseguenza le generazioni di mantoscuro si susseguono rapidamente e nel corso degli anni l'evoluzione di queste creature è altrettanto rapida. Per questa ragione l'ecosistema di una caverna può avere effetti importanti sull'aspetto, le capacità e le tattiche di un mantoscuro. In caverne acquatiche possono svilupparsi mantoscuri in grado di nuotare, mentre creature che abitano luoghi soggetti a vulcanismo potrebbero sviluppare una specifica resistenza al fuoco. Altre varianti di mantoscuro potrebbero avere pelli più resistenti ed invece di cadere per stritolare la preda potrebbero semplicemente gettarsi cercando di trafiggerla analogamente a vere e proprie stalattiti. Si mormora che le caverne più oscure e profonde nascondano mantoscuri di dimensioni incredibili, in grado di soffocare contemporaneamente diversi bersagli di dimensioni umane nel loro abbraccio avvolgente.\\


\medskip\index{Mostri - Medusa}\textbf{Medusa}

\emph{Media mostruosità, legale malvagio}

\textbf{FORZA} +0

\textbf{DESTREZZA} +2

\textbf{COSTITUZIONE} +3

\textbf{INTELLIGENZA} +1

\textbf{SAGGEZZA} +1

\textbf{CARISMA} +2

\textbf{Iniziativa} +2 -- \textbf{Difesa} 18

\textbf{Punti Ferita} 127 (17d8 + 51)

\textbf{Movimento} 9 m

\textbf{Tiri Salvezza}: Tempra +6, Riflessi +8, Volontà +7

\textbf{Competenze} Muoversi Silenziosamente / Nascondersi +5, Ingannare +5, Percepire Emozioni +4, Consapevolezza +4

\textbf{Sensi} scurovisione 18 m

\textbf{Linguaggi} Comune

\textbf{Sfida} 6 (2.300 PE)

\emph{\textbf{Sguardo Pietrificante.}} Se una creatura comincia il suo turno entro 9 metri da una medusa di cui possa vedere gli occhi, la medusa, qualora la non sia inabile e possa vedere a sua volta la creatura, può obbligarla ad effettuare un Tiro Salvezza di Tempra CD 14. Se la creatura fallisce il Tiro Salvezza di 5 o più, viene pietrificata all'istante, altrimenti inizia magicamente a trasformarsi in pietra ed è intralciata. La creatura intralciata deve ripetere il Tiro Salvezza al termine del suo prossimo turno. Se lo riesce, l'effetto termina. Se lo fallisce, la creatura è pietrificata finché non viene liberata dall'incantesimo \emph{ristorare superiore} o altra magia.

Una creatura che non sia sorpresa può distogliere lo sguardo per evitare il Tiro Salvezza all'inizio del proprio turno. In quel caso, non potrà vedere la medusa fino all'inizio del suo prossimo turno, quando potrà distogliere nuovamente lo sguardo. Se nel frattempo dovesse guardare la medusa, dovrebbe immediatamente effettuare il Tiro Salvezza.

Se la medusa vede il suo riflesso su di una superficie riflettente entro 9 metri da lei in un'area di luce intensa, a causa della propria maledizione subirà gli effetti del suo stesso sguardo.

\textbf{Azioni}

\emph{\textbf{Multiattacco.}} La medusa effettua tre attacchi -- uno con i capelli serpentini e due con la spada corta -- o due attacchi a distanza con l'arco lungo.

\emph{\textbf{Capelli Serpentini.} Attacco con arma da mischia}: +5 a colpire, portata 1 m, un bersaglio.

\emph{Colpisce:} 4 (1d4 + 2) danni perforanti più 14 (4d6) danni da veleno.

\emph{\textbf{Spada Corta.} Attacco con arma da mischia}: +5 a colpire, portata 1 m, un bersaglio.

\emph{Colpisce:} 5 (1d6 + 2) danni perforanti.

\emph{\textbf{Arco Lungo.} Attacco con arma a Distanza}: +5 a colpire, gittata 45m, un bersaglio.

\emph{Colpisce:} 6 (1d8 + 2) danni perforanti più 7 (2d6) danni da veleno.

\textbf{Ecologia}\\
Ambiente: Paludi temperate e sotterranei\\
Organizzazione: Solitario\\
Tesoro: Doppio (Pugnale, Arco Lungo Perfetto con 20 Frecce, altro tesoro)\\
\textbf{Descrizione}\\
Le meduse sono creature simili agli umani con serpenti al posto dei capelli. Dalla distanza di 9 metri o più, una medusa può passare facilmente per una bella donna se indossa qualcosa che copre la sua chioma serpentina; quando indossa un abbigliamento che ne cela la testa e il volto può essere scambiata per un'umana anche a distanza ravvicinata. Le meduse usano bugie e travestimenti per celare il loro volto fino a che gli avversari non sono abbastanza vicini da usare il loro sguardo pietrificante, anche se gli piace giocare con la loro preda e possono usare delle frecce fiammeggianti per intrappolare i nemici a distanza. Alcune si divertono a creare intricate decorazioni con le loro vittime, usando la pietrificazione per dare un certo tocco ai loro nascondigli paludosi, ma molte meduse hanno cura di nascondere le prove dei loro scontri precedenti così che i loro nuovi nemici non si accorgano della loro pericolosa presenza.\\

Avvezze a nascondersi, le meduse cittadine generalmente sono ladre, mentre quelle delle zone selvagge spesso finiscono per essere guardiaboschi. Le meduse delle leggende più note, tuttavia, sono quelle che prendono livelli da incatatore. Carismatiche ed intelligenti, le meduse urbane sono spesso coinvolte in gilde di ladri ed altri aspetti del mondo criminale. Le meduse possono formare alleanze con creature cieche o non morti intelligenti, entrambi immuni al loro sguardo pietrificante. Le meduse incantatrici fungono spesso da oracoli o profetesse, vivendo generalmente in remote zone di leggendaria potenza o dalla storia infausta. Queste meduse oracoli traggono grande diletto dal loro ruolo, e se ci si presenta con i giusti doni e adulazioni, i segreti che offrono possono essere veramente utili. Naturalmente, i nascondigli di queste potenti creature sono decorati con le statue di coloro che le hanno offese, come monito ad usare le dovute cautele durante gli incontri.\\

Tutte le meduse sono femmine. Raramente, una medusa decide di prendere un maschio umanoide come compagno, generalmente grazie all'aiuto di una Elisir d'Amore o qualche magia simile, ed hanno sempre cura di non pietrificare il loro prigioniero, a meno che non si siano annoiate della sua compagnia.\\


\subsection{Mefiti}

\medskip\index{Mostri - Mefito di Ghiaccio}\textbf{Mefito di Ghiaccio}

\emph{Piccola elementale, neutrale malvagio}

\textbf{FORZA} -2

\textbf{DESTREZZA} +1

\textbf{COSTITUZIONE} +0

\textbf{INTELLIGENZA} -1

\textbf{SAGGEZZA} +0

\textbf{CARISMA} +1

\textbf{Iniziativa} +1 -- \textbf{Difesa} 12

\textbf{Punti Ferita} 21 (6d6)

\textbf{Movimento} 9 m, volo 9 m

\textbf{Tiri Salvezza}: Tempra +2, Riflessi +5, Volontà +3

\textbf{Competenze} Muoversi Silenziosamente / Nascondersi +3, Consapevolezza +2

\textbf{Vulnerabilità ai Danni} da botta, fuoco

\textbf{Immunità ai Danni} freddo, veleno

\textbf{Immunità alle Condizioni} avvelenato

\textbf{Sensi} scurovisione 18 m

\textbf{Linguaggi} Aquan, Auran

\textbf{Sfida} 1/2 (100 PE)

\emph{\textbf{Falso Aspetto.}} Mentre il mefito rimane immobile, è indistinguibile da un ordinario frammento di ghiaccio.

\emph{\textbf{Incantesimi Innati (1/Giorno).}} Il mefito può lanciare in maniera innata \emph{nube di nebbia}, senza bisogno di componenti materiali. La sua caratteristica da incantatore innato è il Carisma.

\emph{\textbf{Natura Elementale.}} Un mefito non ha bisogno di cibo, bevande o sonno.

\emph{\textbf{Scoppio Mortale.}} Quando il mefito muore, esplode in uno scoppio di frammenti di ghiaccio. Ogni creatura entro 1 metro da esso deve effettuare un Tiro Salvezza di Riflessi CD 10 o subire 4 (1d8) danni taglienti in caso di fallimento, o la metà di questi danni in caso
di successo.

\textbf{Azioni}

\emph{\textbf{Artigli.} Attacco con arma da mischia}: +3 a colpire,
portata 1 m, una creatura.

\emph{Colpisce:} 3 (1d4 + 1) danni taglienti più 2 (1d4) danni da
freddo.

\emph{\textbf{Soffio Gelido (Ricarica 6).}} Il mefito esala un cono di 5 metri di aria fredda. Ogni creatura nell'area deve effettuare un Tiro Salvezza di Riflessi CD 10, subendo 5 (2d4) danni da freddo in caso di fallimento, o la metà di questi danni in caso di successo.

\textbf{Ecologia}\\
Ambiente: Qualsiasi (piano elementale dell'aria)\\
Organizzazione: Solitario, coppia, gruppo (3-6) o stormo (7-12)\\
Tesoro: Standard\\
\textbf{Descrizione}\\
I mephit sono i servitori di potenti creature elementali. I siti e le locazioni chiave dei piani elementali sono pieni di mephit che si affannano per svolgere un importante dovere o incarico.\\

I mephit del ghiaccio comunemente si trovano sul Piano dell'Aria. Questi mephit sono distanti e crudeli.\\


\medskip\index{Mostri - Mefito di Magma}\textbf{Mefito di Magma}

\emph{Piccola elementale, neutrale malvagio}

\textbf{FORZA} -1

\textbf{DESTREZZA} +1

\textbf{COSTITUZIONE} +1

\textbf{INTELLIGENZA} -2

\textbf{SAGGEZZA} +0

\textbf{CARISMA} +0

\textbf{Iniziativa} +1 -- \textbf{Difesa} 12

\textbf{Punti Ferita} 22 (5d6 + 5)

\textbf{Movimento} 9 m, volo 9 m

\textbf{Tiri Salvezza}: Tempra +2, Riflessi +5, Volontà +3

\textbf{Competenze} Muoversi Silenziosamente / Nascondersi +3

\textbf{Vulnerabilità ai Danni} freddo

\textbf{Immunità ai Danni} fuoco, veleno

\textbf{Immunità alle Condizioni} avvelenato

\textbf{Sensi} scurovisione 18 m

\textbf{Linguaggi} Ignan, Terran

\textbf{Sfida} 1/2 (100 PE)

\emph{\textbf{Falso Aspetto.}} Mentre il mefito rimane immobile, è indistinguibile da un'ordinaria pozza di magma.

\emph{\textbf{Incantesimi Innati (1/Giorno).}} Il mefito può lanciare in maniera innata \emph{riscaldare metallo} (CD del Tiro Salvezza dell'incantesimo 10), senza bisogno di componenti materiali. La sua caratteristica da incantatore innato è il Carisma.

\emph{\textbf{Natura Elementale.}} Un mefito non ha bisogno di cibo, bevande o sonno.

\emph{\textbf{Scoppio Mortale.}} Quando il mefito muore, esplode in uno scoppio di lava. Ogni creatura entro 1 metro da esso deve effettuare un Tiro Salvezza di Riflessi CD 11 o subire 7 (2d6) danni da fuoco in caso di fallimento, o la metà di questi danni in caso di successo.

\textbf{Azioni}

\emph{\textbf{Artigli.} Attacco con arma da mischia}: +3 a colpire, portata 1 m, una creatura.

\emph{Colpisce:} 3 (1d4 + 1) danni taglienti più 2 (1d4) danni da fuoco.

\emph{\textbf{Soffio Infuocato (Ricarica 6).}} Il mefito esala un cono di 5 metri di fuoco. Ogni creatura nell'area deve effettuare un tiro salvezza su Riflessi CD 11, subendo 7 (2d6) danni da fuoco in caso di fallimento, o la metà di questi danni in caso di successo.

\textbf{Ecologia}\\\
Ambiente: Qualsiasi (piano elementale del fuoco)\\
Organizzazione: Solitario, coppia, gruppo (3-6) o stormo (7-12)\\
Tesoro: Standard\\
\textbf{Descrizione}\\
I mephit sono i servitori di potenti creature elementali. I siti e le locazioni chiave dei piani elementali sono pieni di mephit che si affannano per svolgere un importante dovere o incarico.\\
I mephit del magma comunemente si trovano sul Piano del Fuoco. Questi mephit sono stupidi bruti.\\


\medskip\index{Mostri - Mefito di Polvere}\textbf{Mefito di Polvere}

\emph{Piccola elementale, neutrale malvagio}

\textbf{FORZA} -3

\textbf{DESTREZZA} +2

\textbf{COSTITUZIONE} +0

\textbf{INTELLIGENZA} -1

\textbf{SAGGEZZA} +0

\textbf{CARISMA} +0

\textbf{Iniziativa} +2 -- \textbf{Difesa} 13

\textbf{Punti Ferita} 17 (5d6)

\textbf{Movimento} 9 m, volo 9 m

\textbf{Tiri Salvezza}: Tempra +2, Riflessi +5, Volontà +3

\textbf{Competenze} Muoversi Silenziosamente / Nascondersi +4, Consapevolezza +2

\textbf{Vulnerabilità ai Danni} fuoco

\textbf{Immunità ai Danni} veleno

\textbf{Immunità alle Condizioni} avvelenato

\textbf{Sensi} scurovisione 18 m

\textbf{Linguaggi} Auran, Terran

\textbf{Sfida} 1/2 (100 PE)

\emph{\textbf{Incantesimi Innati (1/Giorno).}} Il mefito può eseguire in
maniera innata \emph{sonno} (CD del Tiro Salvezza dell'incantesimo 10),
senza bisogno di componenti materiali. La sua abilità da incantatore
innato è il Carisma.

\emph{\textbf{Natura Elementale.}} Un mefito non ha bisogno di cibo,
bevande o sonno.

\emph{\textbf{Scoppio Mortale.}} Quando il mefito muore, esplode in uno scoppio di polvere. Ogni creatura entro 1 metro da esso deve riuscire un Tiro Salvezza di Tempra CD 10 o restare accecata per 1 minuto. Una creatura accecata può ripetere il Tiro Salvezza durante ciascun suo turno, terminando l'effetto su di sé in caso di successo.

\textbf{Azioni}

\emph{\textbf{Artigli.} Attacco con arma da mischia}: +4 a colpire,
portata 1 m, una creatura.

\emph{Colpisce:} 4 (1d4 + 2) danni taglienti.

\emph{\textbf{Soffio Accecante (Ricarica 6).}} Il mefito esala un cono di 5 metri di polvere accecante. Ogni creatura nell'area deve riuscire un Tiro Salvezza di Riflessi CD 10 o restare accecata per 1 minuto. Una creatura accecata può ripetere il Tiro Salvezza durante ciascun suo turno, terminando l'effetto su di sé in caso di successo.

\textbf{Ecologia}\\
Ambiente: Qualsiasi (piano elementale dell'aria)\\
Organizzazione: Solitario, coppia, gruppo (3-6) o stormo (7-12)\\
Tesoro: Standard\\
\textbf{Descrizione}\\
I mephit sono i servitori di potenti creature elementali. I siti e le locazioni chiave dei piani elementali sono pieni di mephit che si affannano per svolgere un importante dovere o incarico.\\
I mephit della polvere comunemente si trovano sul Piano dell'Aria. Questi mephit sono irritanti ed insistenti.\\

\medskip\index{Mostri - Mefito di Vapore}\textbf{Mefito di Vapore}

\emph{Piccola elementale, neutrale malvagio}

\textbf{FORZA} -3

\textbf{DESTREZZA} +0

\textbf{COSTITUZIONE} +0

\textbf{INTELLIGENZA} +0

\textbf{SAGGEZZA} +0

\textbf{CARISMA} +1

\textbf{Iniziativa} +0 -- \textbf{Difesa} 11

\textbf{Punti Ferita} 21 (6d6)

\textbf{Movimento} 9 m, volo 9 m

\textbf{Tiri Salvezza}: Tempra +2, Riflessi +5, Volontà +3

\textbf{Immunità ai Danni} fuoco, veleno

\textbf{Immunità alle Condizioni} avvelenato

\textbf{Sensi} scurovisione 18 m

\textbf{Linguaggi} Aquan, Ignan

\textbf{Sfida} 1/4 (50 PE)

\emph{\textbf{Incantesimi Innati (1/Giorno).}} Il mefito può eseguire in maniera innata \emph{sfocatura}, senza bisogno di componenti materiali. La sua abilità da incantatore innato è il Carisma.

\emph{\textbf{Natura Elementale.}} Un mefito non ha bisogno di cibo,
bevande o sonno.

\emph{\textbf{Scoppio Mortale.}} Quando il mefito muore, esplode in nube
di vapore. Ogni creatura entro 1 metro da esso deve riuscire un tiro
salvezza su Riflessi CD 10 o subire 4 (1d8) danni da fuoco.

\textbf{Azioni}

\emph{\textbf{Artigli.} Attacco con arma da mischia}: +2 a colpire,
portata 1 m, una creatura.

\emph{Colpisce:} 2 (1d4) danni taglienti più 2 (1d4) danni da fuoco.

\emph{\textbf{Soffio Vaporoso (Ricarica 6).}} Il mefito esala un cono di 5 metri di vapore caldo. Ogni creatura nell'area deve effettuare un Tiro Salvezza di Riflessi CD 10, subendo 4 (1d8) danni da fuoco in caso di fallimento, o la metà di questi danni in caso di successo.

\textbf{Ecologia}\\
Ambiente: Qualsiasi (piano elementale del fuoco)\\
Organizzazione: Solitario, coppia, gruppo (3-6) o stormo (7-12)\\
Tesoro: Standard\\
\textbf{Descrizione}\\
I mephit sono i servitori di potenti creature elementali. I siti e le locazioni chiave dei piani elementali sono pieni di mephit che si affannano per svolgere un importante dovere o incarico.\\
I mephit del vapore comunemente si trovano sul Piano del Fuoco. Questi mephit sono insolenti e sprezzanti.\\



\subsection{Megere}

\medskip\index{Mostri - Megera Marina}\textbf{Megera Marina}

\emph{Media fatato, caotico malvagio}

\textbf{FORZA} +3

\textbf{DESTREZZA} +1

\textbf{COSTITUZIONE} +3

\textbf{INTELLIGENZA} +1

\textbf{SAGGEZZA} +1

\textbf{CARISMA} +1

\textbf{Iniziativa} +1 -- \textbf{Difesa} 15

\textbf{Punti Ferita} 52 (7d8 + 21)

\textbf{Vulnerabilità al Danno} ferro freddo

\textbf{Movimento} 9 m, nuoto 12 m

\textbf{Tiri Salvezza}: Tempra +5, Riflessi +7, Volontà +5

\textbf{Sensi} scurovisione 18 m

\textbf{Linguaggi} Aquan, Comune, Gigante

\textbf{Sfida} 2 (450 PE)

\emph{\textbf{Anfibio.}} La megera può respirare aria e acqua.

\emph{\textbf{Aspetto Orripilante.}} Qualsiasi umanoide che inizi il suo turno entro 9 metri dalla megera e ne può vedere la vera forma deve effettuare un Tiro Salvezza di Volontà CD 11. Se fallisce il Tiro Salvezza, la creatura resta spaventata per 1 minuto. Una creatura può ripetere il Tiro Salvezza al termine di ciascun suo turno, con -1d6 se la megera è in linea di visuale, e terminando l'effetto se riesce il Tiro Salvezza. Se il Tiro Salvezza della creatura riesce o l'effetto ha termine su di essa, la creatura è immune all'Aspetto Orripilante per le successive 24 ore.

A meno che il bersaglio non sia sorpreso o la rivelazione della vera forma della megera non sia improvvisa, il bersaglio può distogliere lo sguardo e evitare di effettuare il Tiro Salvezza iniziale. Fino all'inizio del suo prossimo turno, una creatura che distolga lo sguardo
ha -1d6 ai tiri di attacco contro la megera.

\textbf{Azioni}

\emph{\textbf{Artigli.} Attacco in mischia con arma}: +5 a colpire,
portata 1 m, un bersaglio.

\emph{Colpisce:} 10 (2d6 + 3) danni taglienti.

\emph{\textbf{Aspetto Illusorio.}} La megera ricopre se stessa e tutto
quello che sta indossando o trasportando in un'illusione magica che le
dona l'aspetto di una creatura ripugnante all'incirca della stessa
taglia e forma umanoide. L'illusione termina se la megera effettua
un'azione bonus per terminarla o se muore.

I cambiamenti apportati da questo effetto non sono in grado di superare le ispezioni fisiche. Ad esempio, la megera potrebbe apparire come una creatura priva di artigli, ma una persona in contatto con le sue mani li avvertirebbe. Altrimenti, una creatura deve effettuare un'azione per ispezionare visivamente l'illusione e riuscire una prova di Intelligenza CD 16 per comprendere che la megera si è camuffata.

\emph{\textbf{Occhiata Mortale.}} La megera prende a bersaglio una creatura spaventata visibile entro 9 metri da lei. Se il bersaglio può vedere la megera, deve riuscire un Tiro Salvezza di Volontà CD 11 contro questa magia o scendere a 0 punti ferita.

\textbf{Ecologia}\\
Ambiente: qualsiasi acquatico\\
Organizzazione: solitario o congrega (3 megere di qualsiasi specie)\\
Tesoro: standard\\
\textbf{Descrizione}\\
Queste perfide e mostruose megere possiedono dei tratti terrificanti che pochi osano fissare, traggono piacere dalla discordia e dalla morte dei marinai, e tormentano la gente di mare con ineluttabili sciagure. Le megere marine hanno sempre un aspetto terribile e, malgrado la loro natura famelica, in genere sono creature emaciate che sembrano sul punto di morir di fame. Sono alte 1,8 metri e pesano 75 kg.\\

Le megere marine preferiscono vivere vicino alla riva dove i pescherecci e i mercantili sono più comuni, e comunque lontano dalle aree urbane di modo che le loro azioni non attraggano troppo l'attenzione di possibili nemici, anche se non è insolito che una megera marina coraggiosa o avida si stabilisca in una città portuale o alla foce di un fiume profondo.\\

Le megere marine formano congreghe simili a quelle delle altre megere, ma la loro natura acquatica generalmente le spinge ad astenersi dal formare congreghe miste. Nel caso in cui una Megera Verde abiti lungo la costa (spesso in una palude salmastra o in una palude costiera), una congrega è formata da due megere marine che rispettano la Megera Verde come madre e capo. Molto comunemente, una congrega di megere marine consiste in un gruppo di megere marine particolarmente amiche e vicine.\\


\medskip\index{Mostri - Megera Notturna}\textbf{Megera Notturna}

\emph{Media immondo, neutrale malvagio}

\textbf{FORZA} +4

\textbf{DESTREZZA} +2

\textbf{COSTITUZIONE} +3

\textbf{INTELLIGENZA} +3

\textbf{SAGGEZZA} +2

\textbf{CARISMA} +3

\textbf{Iniziativa} +3 -- \textbf{Difesa} 20

\textbf{Punti Ferita} 112 (15d8 + 45)

\textbf{Movimento} 9 m

\textbf{Tiri Salvezza}: Tempra +14, Riflessi +8, Volontà +11

\textbf{Competenze} Muoversi Silenziosamente / Nascondersi +6, Ingannare +7, Percepire Emozioni +6, Consapevolezza +6,

\textbf{Resistenze al Danno} freddo, fuoco; da botta, perforante e tagliente di attacchi non magici o non siano argentati

\textbf{Sensi} scurovisione 36 m

\textbf{Linguaggi} Abissale, Comune, Infernale, Druidico

\textbf{Sfida} 5 (1.800 PE)

\emph{\textbf{Incantesimi Innati.}} La caratteristica da incantatore innato della megera è il Carisma (CD 14 per i Tiri Salvezza degli incantesimi, +6 a colpire con attacchi da incantesimo). La megera può lanciare in maniera innata i seguenti incantesimi, senza aver bisogno di
componenti materiali.

A volontà: \emph{dardo incantato, individuazione del magico} 2/giorno ciascuno: \emph{raggio di indebolimento, sonno, spostamento} \emph{planare} (personale)

\emph{\textbf{Resistenza alla Magia.}} La megera ha +1d6 ai tiri salvezza contro incantesimi e altri effetti magici.

\textbf{Azioni}

\emph{\textbf{Artigli (Solo in Forma di Megera).} Attacco con arma da 	mischia}: +7 a colpire, portata 1 m, un bersaglio. \emph{Colpisce:} 13 (2d8 + 4) danni taglienti.

\emph{\textbf{Forma Eeterea.}} La megera entra magicamente nel Piano Etereo dal Piano Materiale, e viceversa. Per farlo deve essere in possesso di un \emph{cuore di pietra}.

\emph{\textbf{Infestare Incubi (1/Giorno).}} Mentre si trova sul Piano Etereo, la megera entra magicamente in contatto con un umanoide addormentato che si trova sul Piano Materiale. L'incantesimo \emph{protezione dal bene e dal male} lanciato sul bersaglio previene questo contatto, così come \emph{cerchio magico}. Finché il contatto persiste, il bersaglio soffre di orribili visioni. Se queste visioni durano per almeno 1 ora, il bersaglio non ottiene benefici dal suo riposo, e i suoi punti ferita massimi sono ridotti di 5 (1d10). Se questo effetto riduce i punti ferita massimi del bersaglio a 0, il bersaglio muore, e se il bersaglio era malvagio, la sua anima resta intrappolata nella \emph{borsa} \emph{delle anime} della megera. La riduzione dei punti ferita massimi del bersaglio rimane finché non viene rimossa dall'incantesimo \emph{ristorare} \emph{superiore} o simile magia.

\emph{\textbf{Mutare Forma.}} La megera può trasformarsi magicamente in una femmina umanoide di taglia Piccola o Media, o tornare alla sua vera forma. Le sue statistiche sono le stesse in qualsiasi forma. Tutto l'equipaggiamento che stava trasportando o indossando non viene trasformato. Alla morte, ritorna alla sua vera forma.



\medskip\index{Mostri - Megera Verde}\textbf{Megera Verde}

\emph{Media fatato, neutrale malvagio}

\textbf{FORZA} +4

\textbf{DESTREZZA} +1

\textbf{COSTITUZIONE} +3

\textbf{INTELLIGENZA} +1

\textbf{SAGGEZZA} +2

\textbf{CARISMA} +2

\textbf{Iniziativa} +1 -- \textbf{Difesa} 19

\textbf{Punti Ferita} 82 (11d8 + 33)

\textbf{Vulnerabilità al Danno} ferro freddo

\textbf{Movimento} 9 m

\textbf{Tiri Salvezza}: Temp. +6, Rifl. +7, Vol. +7

\textbf{Competenze} Arcano +3, Muoversi Silenziosamente / Nascondersi +3, Ingannare +4, Consapevolezza +4

\textbf{Sensi} scurovisione 18 m

\textbf{Linguaggi} Comune, Draconico, Silvano

\textbf{Sfida} 3 (700 PE)

\emph{\textbf{Anfibio.}} La megera può respirare aria e acqua.

\emph{\textbf{Imitazione.}} La megera può imitare suoni animali e voci umanoidi. Una creatura che senta questi rumori può determinare che si tratti di un'imitazione riuscendo una prova di Saggezza CD 14.

\emph{\textbf{Incantesimi Innati.}} La caratteristica da incantatore innato della megera è il Carisma (CD 12 per i Tiri Salvezza degli incantesimi). La megera può lanciare in maniera innata i seguenti incantesimi, senza aver bisogno di componenti materiali.

A volontà: \emph{illusione minore, luci danzanti, beffa maligna}

\textbf{Azioni}

\emph{\textbf{Artigli.} Attacco con arma da mischia}: +6 a colpire,
portata 1 m, un bersaglio.

\emph{Colpisce:} 13 (2d8 + 4) danni taglienti.

\emph{\textbf{Aspetto Illusorio.}} La megera ricopre sé stessa e tutto quello che sta indossando o trasportando in un'illusione magica che le dona l'aspetto di un'altra creatura all'incirca della stessa taglia e forma umanoide. L'illusione termina se la megera effettua un'azione bonus per terminarla o se muore.

I cambiamenti apportati da questo effetto non sono in grado di superare le ispezioni fisiche. Ad esempio, la megera potrebbe apparire come una creatura dalla pelle liscia, ma il contatto rivelerebbe la sua pelle ruvida. Altrimenti, una creatura deve effettuare un'azione per ispezionare visivamente l'illusione e riuscire una prova di Intelligenza CD 20 per comprendere che si tratta di una megera camuffata.

\emph{\textbf{Passaggio Invisibile.}} La megera può rendersi invisibile finché non attacca o lancia un incantesimo, o finché non termina la concentrazione (come se si stesse concentrando su di un incantesimo). Mentre è invisibile, non lascia traccia fisica del suo passaggio, quindi le sue tracce possono essere seguite solo dalla magia. Tutto l'equipaggiamento che sta trasportando o indossando diventa invisibile assieme a lei.

\textbf{Ecologia}
Ambiente: Paludi temperate\\
Organizzazione: Solitario o congrega (3 megere di qualsiasi tipo)\\
Tesoro: Standard\\
\textbf{Descrizione}\\
Terrificanti vecchie rugose che frequentano ripugnanti paludi e foreste intricate, le megere verdi nutrono un odio intenso per tutto ciò che è bello e puro. Facendo uso delle loro svariate capacità illusorie, queste vegliarde si dilettano nell'uccidere gli innocenti, nello sconvolgere gli animi nobili e nell'avvilire i cuori puri. Amano utilizzare Camuffare Se Stesso per assumere le forme di giovani e attraenti ragazze così da sedurre e strappare giovani uomini ai loro affetti e parenti, e corrompere nobili e onesti cittadini con ogni sorta di depravazione e scandalo. Alcune megere verdi preferiscono rivelare la loro reale natura ai loro amati in un momento attentamente architettato per spingere l'uomo alla pazzia per l'orrore e la vergogna. Altre prolungano il loro amoreggiamento e fanno di tutto per rovinare completamente la vita degli uomini da loro sedotti prima di mostrare loro la verità. Infine, i più fortunati di questi sventurati finiscono per essere divorati dalla megera verde loro amante: per gli sfortunati, il destino finale può essere molto peggiore, dato che la crudele fantasia della megera verde è immensa. Una tipica megera verde è alta tra 1,5 e 1,8 metri e pesa poco meno di 80 kg.\\


\subsection{Melme}

\medskip\index{Mostri - Ameba Paglierina}\textbf{Ameba Paglierina}

\emph{Grande melma, disallineato}

\textbf{FORZA} +2

\textbf{DESTREZZA} -2

\textbf{COSTITUZIONE} +2

\textbf{INTELLIGENZA} -4

\textbf{SAGGEZZA} -2

\textbf{CARISMA} -5

\textbf{Iniziativa} +2 -- \textbf{Difesa} 9

\textbf{Punti Ferita} 45 (6d10 + 12)

\textbf{Movimento} 3 m, scalata 3 m

\textbf{Tiri Salvezza}: Tempra +8, Riflessi -3, Volontà -3

\textbf{Resistenze al Danno} acido

\textbf{Immunità al Danno} fulmine, tagliente

\textbf{Immunità alle Condizioni} accecato, affascinato, assordato, prono, affaticamento, spaventato

\textbf{Sensi} vista cieca 18 m (cieca oltre questo raggio)

\textbf{Linguaggi} -

\textbf{Sfida} 2 (450 PE)

\emph{\textbf{Amorfo.}} L'ameba può muoversi attraverso uno spazio fino a 3 centimetri di larghezza senza doversi stringere.

\emph{\textbf{Natura di Melma.}} L'ameba non necessita di dormire.

\emph{\textbf{Scalare come Ragno.}} L'ameba può scalare superfici
difficili, compreso lo stare a testa in giù sul soffitto, senza bisogno
di effettuare una prova di abilità.

\textbf{Azioni}

\emph{\textbf{Pseudopodo.} Attacco con arma da mischia}: +4 a colpire,
portata 1 m, un bersaglio.

\emph{Colpisce:} 9 (2d6 + 2) danni da botta più 3 (1d6) danni da
acido.

\textbf{Reazioni}

\emph{\textbf{Divisione.}} Quando un'ameba Media o più grande subisce danni da fulmine o taglienti, si divide in due nuove amebe che hanno almeno 10 punti ferita. Ogni nuova ameba ha un numero di punti ferita pari alla metà dell'ameba originale, arrotondati per difetto. Le nuove amebe sono di una taglia più piccola di quella originale.

\textbf{Ecologia}
Ambiente: Sotterranei o Paludi Temperati\\
Organizzazione: Solitario\\
Tesoro: Nessuno\\
\textbf{Descrizione}\\
Le Ameba Paglierina sono masse animate di protoplasma di colore simile ad un repellente amalgama di giallo, arancio e marrone. Quando a riposo, il loro corpo piatto e pulsante è alto circa 15 centimetri e si estende tutto intorno; in movimento, si raccolgono in una forma vagamente sferica e sembrano quasi spostarsi rotolando. I loro corpi malleabili permettono loro di attraversare fessure e buchi molto più piccoli dello spazio che occupano. Le creature che vivono sottoterra spesso sigillano tutte le aperture per difendersi dalle Ameba Paglierina.\\

l'acido altamente specializzato dell'Ameba Paglierina dissolve solo la carne. Questa scoperta ha portato molti maestri avvelenatori ed alchimisti a cercarne esemplari per studiarli. Da questi esperimenti sono nate diverse armi specifiche ideate per distruggere i corpi. Si racconta dell'esistenza di un veleno ad azione lenta che distrugge ad una ad una le cellule delle creature viventi, il cui segreto è ben conservato dal suo creatore.\\

Alcune note in un tomo dimenticato parlano di un rituale funebre utilizzato in luoghi lontani. Invece di bruciare il corpo, esso veniva sigillato in un sarcofago di pietra con una Ameba Paglierina, che ne dissolveva il corpo. In seguito, i becchini inserivano la gelatina in un'urna completa di targa in bronzo con il nome del deceduto. Questa pratica protegge gli oggetti interrati con il morto (che viene ridotto in poco tempo ad uno scheletro lucido) e l'essenza della creatura, che si riteneva vivere ancora all'interno della gelatina.\\

L'Ameba Paglierina sono alte circa 15 centimetri, hanno un diametro che puo' arrivare a 3 metri e pesano circa 1.300 chili. In combattimento, si raccolgono su loro stesse e producono lunghi pseudopodi umidi per colpire ed afferrare qualunque cosa si muova.\\

Anche se la tipica Ameba Paglierina ha le statistiche qui presentate, nelle profondità della terra questi predatori possono raggiungere dimensioni mostruose. Si parla anche di Ameba Paglierina che hanno sviluppato altri modi di catturare la preda. Ad esempio, gelatine che avvelenano con il tocco e che espellono nubi di gas tossico che fa bruciare occhi e bocca, lasciando indifesi ma coscienti mentre questa bestia protoplasmatica scivola sui corpi e se ne ciba.


\medskip\index{Mostri - Cubo Gelatinoso}\textbf{Cubo Gelatinoso}

\emph{Grande melma, disallineato}

\textbf{FORZA} +2

\textbf{DESTREZZA} -4

\textbf{COSTITUZIONE} +5

\textbf{INTELLIGENZA} -5

\textbf{SAGGEZZA} -2

\textbf{CARISMA} -5

\textbf{Iniziativa} -4 -- \textbf{Difesa} 7

\textbf{Punti Ferita} 84 (8d10 + 40)

\textbf{Movimento} 5 metri

\textbf{Tiri Salvezza}: Tempra +9, Riflessi -4, Volontà -4

\textbf{Immunità alle Condizioni} accecato, affascinato, assordato, prono, affaticamento, spaventato

\textbf{Sensi} vista cieca 18 m (cieca oltre questo raggio)

\textbf{Linguaggi} -

\textbf{Sfida} 2 (450 PE)

\emph{\textbf{Cubo di Melma.}} Il cubo occupa il suo intero spazio. Le altre creature possono entrare nello spazio, ma rimangono vittima del Sommergere del cubo e hanno -1d6 al Tiro Salvezza.

Le creature all'interno del cubo sono visibili ma godono di copertura totale.

Una creatura entro 1 metro dal cubo può effettuare un'azione per tirare una creatura od oggetto fuori dal cubo. Farlo richiede la riuscita di una prova di Forza CD 12, e la creatura che effettua il tentativo subisce 10 (3d6) danni da acido.

Il cubo può contenere solo una creatura Grande o un massimo di quattro creature Medie o più piccole alla volta.

\emph{\textbf{Natura di Melma.}} Il cubo non necessita di dormire.

\emph{\textbf{Trasparente.}} Anche quando è in piena vista, è necessario riuscire una prova di Saggezza (Consapevolezza) CD 15 per notare un cubo che non si è mosso o non ha attaccato. Una creatura che cerchi di entrare nello spazio del cubo mentre è inconsapevole della sua presenza resta sorpresa dal cubo.

\textbf{Azioni}

\emph{\textbf{Pseudopodo.} Attacco con arma da mischia}: +4 a colpire, portata 1 m, un bersaglio.

\emph{Colpisce:} 10 (3d6) danni da acido.

\emph{\textbf{Sommergere.}} Il cubo si muove fino al massimo del suo movimento. Nel farlo, può entrare nello spazio di una creatura di taglia Grande o più piccola. Ogni volta che il cubo entra nello spazio di una creatura, la creatura deve effettuare un Tiro Salvezza di Riflessi CD 12.

Se il Tiro Salvezza riesce, la creatura può scegliere di essere spinta indietro o di lato di 1 metro. Una creatura che decida di non farsi spingere subisce le conseguenze di un Tiro Salvezza fallito.

Se il Tiro Salvezza fallisce, il cubo entra nello spazio della creatura, che subisce 10 (3d6) danni da acido ed è sommersa. La creatura sommersa non può respirare, è intralciata e subisce 21 (6d6) danni da acido all'inizio del turno del cubo. Quando il cubo si muove, la creatura sommersa si muove con esso.

Una creatura sommersa può tentare di fuggire effettuando un'azione per compiere una prova di Forza CD 12. Se la riesce, la creatura sfugge e ed entra nello spazio di sua scelta entro 1 metro dal cubo.

\textbf{Ecologia}
Ambiente: Qualsiasi sotterraneo\\
Organizzazione: Solitario\\
Tesoro: Accidentale\\
\textbf{Descrizione}\\
Tra i predatori più insoliti e peculiari dei dungeon, i cubi gelatinosi trascorrono la loro esistenza vagabondando senza meta per i cunicoli sotterranei e le oscure caverne, inglobando materiali organici come piante, rifiuti, carogne e anche creature viventi. La materia che il cubo non può digerire, come metalli e pietra, riempie di detriti il volume della creatura, e a volte questa può espellerne una parte dal suo corpo. Spesso il tesoro e gli averi delle vittime passate restano dentro il cubo gelatinoso: immagine spettrale dei loro resti materiali.\\

Alcuni saggi credono che queste creature si siano evolute delle Melme Grigie. Alcuni esseri usano i cubi gelatinosi come guardiani di dungeon e fortificazioni sotterranee, intrappolando queste immense creature in casse di metallo massiccio e trasportandole con poteri o magie fino al loro posto di guardia finale. Sono dei meccanismi di smaltimento rifiuti particolarmente efficaci; una tribù può intrappolare un cubo gelatinoso in una fossa o un'altra area che non possa scalare usandolo come letamaio o anche trappola mortale, a seconda dell'ingegnosità delle creature che l'hanno catturato.\\

I cubi gelatinosi in genere hanno uno spigolo di 3 metri e pesano più di 7.500 kg, sebbene alcuni esploratori sotterranei affermino che nel sottosuolo esistano esemplari più grandi. In zone in cui il cibo abbonda, i cubi gelatinosi possono vivere per centinaia, se non migliaia, di anni. Tuttavia, se viene a mancare la materia organica per più di 6 mesi, un cubo gelatinoso comincia a deperire, e le sue pareti iniziano a colare, disfacendosi rapidamente in muco liquido finché l'intero corpo non collassa e scompare completamente.\\


\medskip\index{Mostri - Melma Grigia}\textbf{Melma Grigia}

\emph{Media melma, disallineato}

\textbf{FORZA} +1

\textbf{DESTREZZA} -2

\textbf{COSTITUZIONE} +3

\textbf{INTELLIGENZA} -5

\textbf{SAGGEZZA} -2

\textbf{CARISMA} -4

\textbf{Iniziativa} -2 -- \textbf{Difesa} 9

\textbf{Punti Ferita} 22 (3d8 + 9)

\textbf{Movimento} 3 m, scalata 3 m

\textbf{Tiri Salvezza}: Tempra +9, Riflessi -4, Volontà -4

\textbf{Resistenze al Danno} acido, freddo, fuoco

\textbf{Immunità alle Condizioni} accecato, affascinato, assordato, prono, affaticamento, spaventato

\textbf{Sensi} vista cieca 18 m (cieca oltre questo raggio)

\textbf{Linguaggi} -

\textbf{Sfida} 1/2 (100 PE)

\emph{\textbf{Amorfo.}} La melma può muoversi attraverso uno spazio fino a  centimetri di larghezza senza doversi stringere.

\emph{\textbf{Corrodere Metallo.}} Qualsiasi arma non magica fatta di metallo che colpisca la melma si corrode. Dopo aver inflitto il danno, l'arma subisce una penalità permanente e cumulativa di -1 ai tiri di danno. Se la penalità arriva a -5, l'arma è distrutta. Le munizioni non magiche fatte di metallo che colpiscano la melma, si distruggono dopo aver inflitto il danno.

La melma può divorare metallo non magico dello spessore di 5 centimetri in un 1 round.

\emph{\textbf{Falso Aspetto.}} Quando la melma rimane immobile, è indistinguibile da una pozza d'olio o una pietra bagnata.

\emph{\textbf{Natura di Melma.}} La melma non necessita di dormire.

\textbf{Azioni}

\emph{\textbf{Pseudopodo.} Attacco con arma da mischia}: +3 a colpire, portata 1 m, un bersaglio.

\emph{Colpisce:} 4 (1d6 + 1) danni da botta più 7 (2d6) danni da acido, e se il bersaglio sta indossando un'armatura di metallo, questa viene parzialmente dissolta e subisce una penalità permanente e cumulativa di -1 alla Difesa che offre. L'armatura è distrutta se la penalità riduce la sua Difesa a 10.

\textbf{Ecologia}\\
Ambiente: Paludi fredde e sotterranei\\
Organizzazione: Solitario\\
Tesoro: Nessuno\\
\textbf{Descrizione}\\
Strisciando attraverso le fredde paludi e gli acquitrini nebbiosi o, a volte in sotterranei e caverne, le melme grigie consumano ogni sostanza organica che incontrano. Sebbene priva di intelligenza, la melma grigia è una delle creature che dà non pochi problemi per la sua trasparenza. Anche se non può arrampicarsi facilmente sui muri o nuotare, la sua abitudine di nascondersi nel fango spesso lungo le rive paludose o di rimanere immobile in pozze dall'aspetto innocuo sul pavimento grigio di un sotterraneo, la rendono molto difficile da notare e da evitare.\\

Alcuni saggi credono che le melme grigie siano il risultato di un esperimento alchemico fallito, mentre altri teorizzano che le prime melme grigie siano nate spontaneamente da un pozzo di detriti magici. Naturalmente, queste teorie che non le considerano organismi viventi, bensì il risultato di una sfortunata mistura di fluidi caustici e residui magici, sono derisi da chi vive nelle zone infestate da queste creature, che non hanno una storia di inquinamento magico.\\


\medskip\index{Mostri - Protoplasma Nero}\textbf{Protoplasma Nero}

\emph{Grande melma, disallineato}

\textbf{FORZA} +3

\textbf{DESTREZZA} -3

\textbf{COSTITUZIONE} +3

\textbf{INTELLIGENZA} -5

\textbf{SAGGEZZA} -2

\textbf{CARISMA} -5

\textbf{Iniziativa} -3 -- \textbf{Difesa} 9

\textbf{Punti Ferita} 85 (10d10 + 30)

\textbf{Movimento} 6 m, scalata 6 m

\textbf{Tiri Salvezza}: Tempra +9, Riflessi -2, Volontà -2

\textbf{Immunità al Danno} acido, freddo, fulmine, tagliente

\textbf{Immunità alle Condizioni} accecato, affascinato, assordato, prono, affaticamento, spaventato

\textbf{Sensi} vista cieca 18 m (cieco oltre questo raggio)

\textbf{Linguaggi} -

\textbf{Sfida} 4 (1.100 PE)

\emph{\textbf{Amorfo.}} Il protoplasma nero può muoversi attraverso uno spazio fino a 3 centimetri di larghezza senza doversi stringere.

\emph{\textbf{Forma Corrosiva.}} Una creatura che entri a contatto col protoplasma nero o lo colpisca con un attacco da mischia mentre si trova entro 1 metro da esso subisce 4 (1d8) danni da acido. Qualsiasi arma non magica fatta di metallo o legno che colpisca il protoplasma nero si corrode. Dopo aver inflitto il danno, l'arma subisce una penalità permanente e cumulativa di -1 ai tiri di danno. Se la penalità arriva a -5, l'arma è distrutta. Le munizioni non magiche fatte di metallo o legno che colpiscano il protoplasma nero, si distruggono dopo aver inflitto il danno.

Il protoplasma nero può divorare legno o metallo non magico dello spessore di 5 centimetri in un 1 round.

\emph{\textbf{Natura di Melma.}} Il protoplasma nero non necessita di dormire.

\emph{\textbf{Scalare come Ragno.}} Il protoplasma nero può scalare superfici difficili, compreso lo stare a testa in giù sul soffitto, senza bisogno di effettuare una prova di abilità.

\textbf{Azioni}

\emph{\textbf{Pseudopodo.} Attacco con arma da mischia}: +5 a colpire, portata 1 m, un bersaglio.

\emph{Colpisce:} 6 (1d6 + 3) danni da botta più 18 (4d8) danni da acido. Inoltre, un'armatura non magica indossata dal bersaglio viene parzialmente dissolta e subisce una penalità permanente e cumulativa di -1 alla Difesa che offre. L'armatura è distrutta se la penalità riduce la sua Difesa a 10.

\textbf{Reazioni}

\emph{\textbf{Divisione.}} Quando un protoplasma nero di taglia Media o più grande subisce danni da fulmine o taglienti, si divide in due nuovi protoplasma neri di almeno 10 punti ferita ciascuno. Ogni nuovo protoplasma nero ha un numero di punti ferita pari alla metà del protoplasma nero originale, arrotondati per difetto. I nuovi protoplasmi neri sono di una taglia più piccola di quella originale.

\textbf{Ecologia}\\
Ambiente: Qualsiasi sotterraneo\\
Organizzazione: Solitario\\
Tesoro: Nessuno\\
\textbf{Descrizione}\\
I protoplasmi neri sono gli spazzini del mondo sotterraneo, costantemente alla ricerca di cibo. Possono percepire corpi organici o metallici nel raggio di 18 metri e attaccano in modo istintivo tali oggetti o esseri finché non li dissolvono, o finché la melma non viene uccisa. Un protoplasma nero si riproduce staccando un pezzo del proprio corpo e formando un nuovo protoplasma più piccolo che raggiunge l'età adulta nel giro di un mese. Alcune tra le creature più intelligenti nel mondo sotterraneo usano i protoplasmi neri per smaltire in modo naturale la spazzatura, creando cave di pietra atte ad ospitare il protoplasma, per poi gettarvi i rifiuti organici o i nemici.\\
Gli esemplari più grandi di protoplasmi neri sono stati avvistati nelle regioni più profonde del mondo: individui Mastodontici che possiedono fino a 30 DV. Si dice che esistano anche protoplasmi colorati: alcuni bianchi che vivono nelle zone artiche, marroni nelle paludi e di colore rossiccio che popolano il deserto.\\


\medskip\index{Mostri - Mimic}\textbf{Mimic}

\emph{Media mostruosità (mutaforma), neutrale}

\textbf{FORZA} +3

\textbf{DESTREZZA} +1

\textbf{COSTITUZIONE} +2

\textbf{INTELLIGENZA} -3

\textbf{SAGGEZZA} +1

\textbf{CARISMA} -1

\textbf{Iniziativa} +1 -- \textbf{Difesa} 13

\textbf{Punti Ferita} 58 (9d8 + 18)

\textbf{Movimento} 5 metri

\textbf{Tiri Salvezza}: Tempra +5, Riflessi +5, Volontà +6

\textbf{Competenze} Muoversi Silenziosamente / Nascondersi +5

\textbf{Immunità al Danno} acido

\textbf{Immunità alle Condizioni} prono

\textbf{Sensi} scurovisione 18 m

\textbf{Linguaggi} -

\textbf{Sfida} 2 (450 PE)

\emph{\textbf{Aderente (Solo Forma di Oggetto).}} Il mimic aderisce a qualsiasi cosa con cui entri in contatto. Una creatura di taglia Enorme o inferiore a cui il mimic aderisce è considerata afferrata da esso (CD 13 per fuggire). Le prove di caratteristica effettuare per fuggire da
questo afferrare hanno -1d6.

\emph{\textbf{Afferratore.}} Il mimic ha +1d6 ai tiri per colpire contro una creatura da esso afferrata.

\emph{\textbf{Falso Aspetto (Solo Forma di Oggetto).}} Mentre il mimic rimane immobile, è indistinguibile da un comune oggetto.

\emph{\textbf{Mutaforma.}} Il mimic può usare la sua azione per trasformarsi in un oggetto, o per tornare alla sua vera forma amorfa. Le sue statistiche sono le stesse in qualsiasi forma. Qualsiasi equipaggiamento stia indossando o trasportando non si trasforma. Alla morte ritorna al suo vero aspetto.

\textbf{Azioni}

\emph{\textbf{Morso.} Attacco con arma da mischia}: +5 a colpire, portata 1 m, un bersaglio.

\emph{Colpisce:} 7 (1d8 + 3) danni perforanti più 4 (1d8) danni da acido.

\emph{\textbf{Pseudopodo.} Attacco con arma da mischia}: +5 a colpire, portata 1 m, un bersaglio.

\emph{Colpisce:} 7 (1d8 + 3) danni da botta. Se il mimic è in forma di oggetto, il bersaglio è vittima del tratto Aderente.

\textbf{Ecologia}
Ambiente: Qualsiasi\\
Organizzazione: Solitario\\
Tesoro: Accidentale\\
\textbf{Descrizione}\\
Si ritiene che i mimic siano il risultato del tentativo di un alchimista di dar vita ad un oggetto inanimato attraverso l'applicazione di un reagente mistico, la cui formula è andata perduta. Nel corso degli anni, queste creature strane ma intelligenti hanno appreso la capacità di trasformarsi in simulacri degli oggetti manufatti, in particolare nei luoghi frequentati poco da un ristretto numero di creature, dove aumentano le loro probabilità di successo con un attacco alle loro vittime.\\
Anche se i mimic non sono intrinsecamente malvagi, alcuni saggi suggeriscono che attacchino gli uomini e le altre creature intelligenti più per passatempo che per sfamarsi. Il desiderio di ingannare gli altri è parte del loro essere, e i loro attacchi a sorpresa rappresentano il culmine di questo desiderio.\\
Un tipico mimic ha un volume di 2 metri cubi (1 m per 1 m per 2 m) e pesa circa 450 kg. Leggende e storie parlano di mimic di taglie maggiori, con la capacità di assumere la forma di case, navi o interi complessi sotterranei che guarniscono con dei tesori (sia veri che falsi) per attirare al loro interno il loro ignaro cibo.\\


\medskip\index{Mostri - Minotauro}\textbf{Minotauro}

\emph{Grande mostruosità, caotico malvagio}

\textbf{FORZA} +4

\textbf{DESTREZZA} +0

\textbf{COSTITUZIONE} +3

\textbf{INTELLIGENZA} -2

\textbf{SAGGEZZA} +3

\textbf{CARISMA} -1

\textbf{Iniziativa} +0 -- \textbf{Difesa} 16

\textbf{Punti Ferita} 76 (9d10 + 27)

\textbf{Movimento} 12 m

\textbf{Tiri Salvezza}: Tempra +6, Riflessi +5, Volontà +5

\textbf{Competenze} Consapevolezza +7

\textbf{Sensi} scurovisione 18 m

\textbf{Linguaggi} Abissale

\textbf{Sfida} 3 (700 PE)

\emph{\textbf{Carica.}} Se il minotauro si muove di almeno 3 metri diretto verso un bersaglio e lo colpisce con un attacco di incornata durante lo stesso turno, il bersaglio subisce 9 (2d8) danni perforanti aggiuntivi. Se il bersaglio è una creatura, deve riuscire un Tiro Salvezza su Tempra CD 14 o venire spinto via fino a 3 metri di distanza e cadere prono.

\emph{\textbf{Incauto.}} All'inizio del suo turno, il minotauro può ottenere +1d6 su tutti i tiri per colpire con armi da mischia effettuati durante quel turno, ma i tiri per colpire contro di esso hanno +1d6 fino all'inizio del suo prossimo turno.

\emph{\textbf{Ricordare Labirinto.}} Il minotauro può ricordare perfettamente qualsiasi tragitto abbia percorso.

\textbf{Azioni}

\emph{\textbf{Ascia Bipenne.} Attacco con arma da mischia}: +6 a colpire, portata 1 m, un bersaglio.

\emph{Colpisce:} 17 (2d12 + 4) danni taglienti.

\emph{\textbf{Incornata.} Attacco con arma da mischia}: +6 a colpire, portata 1 m, un bersaglio.

\emph{Colpisce:} 13 (2d8 + 4) danni perforanti.

\textbf{Ecologia}\\
Ambiente: Rovine Temperate e Sotterranei\\
Organizzazione: Solitario, coppia o gruppo (3-4)\\
Tesoro: Standard (Ascia Bipenne, altro tesoro)\\
\textbf{Descrizione}\\
Nessuno porta rancore come un minotauro. Disprezzati dalle razze civilizzate e nati secoli fa da una maledizione divina, i minotauri hanno cacciato, ucciso e divorato gli umanoidi inferiori per punire offese vere o presunte più a lungo di quanto riescono a ricordare. La maggior parte delle culture ha leggende su come i minotauri furono creati da divinità vendicative o offese che punirono gli umani deformando le loro sembianze, sottraendo loro bellezza e intelligenza, e dotandoli di teste di toro. Eppure la maggioranza dei minotauri moderni disprezza queste leggende e non crede di essere lo scherzo di qualche divinità, ma modelli di perfezione divina creati dal crudele e potente signore dei demoni Baphomet.\\
I nascondigli tradizionali dei minotauri sono i labirinti, sia i dedali costruiti per confondere e sconcertare, sia quelli naturali creati da un intrico di caverne o altri passaggi sotterranei. Grazie alla loro astuzia naturale, i minotauri usano i loro nascondigli labirintici per scoraggiare gli incauti nemici che cercano di scovarli o che semplicemente incappano nei loro nascondigli e si perdono, dando lentamente la caccia agli intrusi che cercano inutilmente di trovare una via d'uscita. Solo quando la disperazione ha nettamente preso il sopravvento il minotauro colpisce le sue perdute vittime. Quando hanno a che fare con un gruppo, spesso i minotauri lasciano scappare uno creatura, affinché diffonda il suo terribile racconto e attiri altri, che sperano di uccidere queste bestie, nei loro labirinti. Naturalmente, per i minotauri, questi aspiranti eroi rappresentano delle pietanze deliziose.\\
I minotauri si possono trovare anche al servizio di un mostro o una creatura malvagia più potente, e lo servono fintanto che possono cacciare e mangiare a loro piacimento. Generalmente questo significa fare la guardia a qualche potente oggetto o preziosa locazione, ma può anche significare lavorare come mercenario, dando la caccia ai nemici del padrone.\\
I minotauri sono combattenti relativamente diretti, usando le loro corna per incornare orribilmente le creature viventi più vicine quando cominciano a combattere.\\


\subsection{Mummie}

\medskip\index{Mostri - Mummia}\textbf{Mummia}

\emph{Media non morto, legale malvagio}

\textbf{FORZA} +3

\textbf{DESTREZZA} -1

\textbf{COSTITUZIONE} +2

\textbf{INTELLIGENZA} -2

\textbf{SAGGEZZA} +0

\textbf{CARISMA} +1

\textbf{Iniziativa} -1 -- \textbf{Difesa} 13

\textbf{Punti Ferita} 58 (9d8 + 18)

\textbf{Movimento} 6 m

\textbf{Tiri Salvezza}: Tempra +4, Riflessi +2, Volontà +8

\textbf{Vulnerabilità al Danno} fuoco

\textbf{Resistenze al Danno} da botta, perforante e tagliente di attacchi non magici

\textbf{Immunità al Danno} da Vuoto, veleno

\textbf{Immunità alle Condizioni} affascinato, avvelenato, paralizzato, affaticamento, spaventato

\textbf{Sensi} scurovisione 18 m 

\textbf{Linguaggi} le lingue che conosceva in vita

\textbf{Sfida} 3 (700 PE)

\emph{\textbf{Natura Non Morta.}} Un mummia non ha bisogno di aria, cibo, bevande o sonno.

\textbf{Azioni}

\emph{\textbf{Multiattacco.}} La mummia può usare la sua Occhiata Temibile ed effettuare un attacco con il pugno putrefacente.

\emph{\textbf{Pugno Putrefacente.} Attacco con arma da mischia}: +5 a colpire, portata 1 m, un bersaglio.

\emph{Colpisce:} 10 (2d6 + 3) danni da botta più 10 (3d6) danni da Vuoto. Se il bersaglio è una creatura deve riuscire un Tiro Salvezza su Tempra 12 o venire maledetto dalla putrefazione della mummia. Il bersaglio maledetto non può recuperare punti ferita, e i suoi punti ferita massimi diminuiscono di 10 (3d6) ogni 24 ore di durata della maledizione. Se la maledizione riduce i punti ferita massimi del bersaglio a 0, il bersaglio muore, e il suo corpo si tramuta in polvere. La maledizione dura finché non viene rimossa dall'incantesimo \emph{rimuovi maledizione} o altra magia.

\emph{\textbf{Occhiata Temibile.}} La mummia prende a bersaglio una creatura che possa vedere e si trovi entro 18 metri da lei. Se il bersaglio può vedere la mummia, deve riuscire un Tiro Salvezza su Volontà CD 11 contro questa magia o restare spaventato fino al termine del prossimo turno della mummia. Se il bersaglio fallisce il Tiro Salvezza di 5 o più, è anche paralizzato per la stessa durata. Un bersaglio che riesca il Tiro Salvezza è immune all'Occhiata Terribile di tutte le mummie (ma non delle mummie sovrane) per le successive 24 ore.

\medskip\index{Mostri - Mummia Sovrana}\textbf{Mummia Sovrana}

\emph{Media non morto, legale malvagio}

\textbf{FORZA} +4

\textbf{DESTREZZA} +0

\textbf{COSTITUZIONE} +3

\textbf{INTELLIGENZA} +0

\textbf{SAGGEZZA} +4

\textbf{CARISMA} +3

\textbf{Iniziativa} +0 -- \textbf{Difesa} 25

\textbf{Punti Ferita} 97 (13d8 + 39)

\textbf{Movimento} 6 m

\textbf{Tiri Salvezza}: Tempra +12, Riflessi +6, Volontà +16

\textbf{Competenze} Religione +5, Storia +5

\textbf{Vulnerabilità al Danno} fuoco

\textbf{Immunità al Danno} da Vuoto, veleno; armi +1

\textbf{Immunità alle Condizioni} affascinato, avvelenato, paralizzato, affaticamento, spaventato

\textbf{Sensi} scurovisione 18 m

\textbf{Linguaggi} le lingue che conosceva in vita

\textbf{Sfida} 15 (13.000 PE)

\emph{\textbf{Cuore della Mummia Sovrana.}} Come parte del rituale che crea una mummia sovrana, il cuore e le viscere della creatura vengono rimossi dal cadavere e piazzati all'interno di contenitori sigillati. Questi contenitori sono di solito fatti in pietra o ceramica, incisi o dipinti con geroglifici religiosi.

Finché il suo cuore avvizzito rimane intatto, la mummia sovrana non può essere permanentemente distrutta. Quando scende a 0 punti ferita, la mummia sovrana si riduce in polvere e si riforma a piena forza 24 ore più tardi, riemergendo dalla polvere in prossimità della giara sigillata che contiene il suo cuore. Per impedire che una mummia sovrana si riformi e distruggerla una volta per tutte, bisogna ridurne il cuore in cenere. Per questo motivo, la mummia sovrana di solito tiene il cuore e le viscere nascoste all'interno di una tomba nascosta.

Il cuore della mummia sovrana ha Difesa 5, 25 punti ferita e immunità a tutti i danni eccetto il fuoco.

\emph{\textbf{Incantesimi.}} La mummia ha CM 10. La sua caratteristica da incantatore è la Saggezza, +9 a colpire con attacchi da incantesimo. La mummia ha preparati i seguenti incantesimi: Trucchetti (a volontà): \emph{fiamma sacra, taumaturgia}

Difficoltà 16 (4 slot): \emph{comando, dardo tracciante, scudo della fede}

Difficoltà 19 (3 slot): \emph{arma spirituale, blocca persone, silenzio}

Difficoltà 21 (3 slot): \emph{animare morti, dissolvi magie}

Difficoltà 23 (3 slot): \emph{divinazione, guardiano della fede}

Difficoltà 26 (2 slot): \emph{contagio, piaga degli insetti}

Difficoltà 29 (1 slot): \emph{ferire}

\emph{\textbf{Natura Non Morta.}} Un mummia non ha bisogno di aria, cibo, bevande o sonno.

\emph{\textbf{Resistenza alla Magia.}} La mummia sovrana ha +1d6 ai Tiri Salvezza contro incantesimi o altri effetti magici.

\emph{\textbf{Rinvigorimento.}} Una mummia sovrana forma un nuovo corpo entro 24 ore se il suo cuore resta intatto, recuperando tutti i punti ferita e potendo agire nuovamente. Il nuovo corpo compare entro 1 metro dal cuore della mummia sovrana.

\textbf{Azioni}

\emph{\textbf{Multiattacco.}} La mummia può usare la sua Occhiata Temibile ed effettuare un attacco con il pugno putrefacente.

\emph{\textbf{Pugno Putrefacente.} Attacco con arma da mischia}: +9 a colpire, portata 1 m, un bersaglio.

\emph{Colpisce:} 14 (3d6 + 4) danni da botta più 21 (6d6) danni da Vuoto. Se il bersaglio è una creatura deve riuscire un Tiro Salvezza su Tempra 16 o venire maledetto dalla putrefazione della mummia. Il bersaglio maledetto non può recuperare punti ferita, e i suoi punti ferita massimi diminuiscono di 10 (3d6) ogni 24 ore di durata della maledizione. Se la maledizione riduce i punti ferita massimi del bersaglio a 0, il bersaglio muore, e il suo corpo si tramuta in polvere. Lamaledizione dura finché non viene rimossa dall'incantesimo  \emph{rimuovere maledizione} o altra magia.

\emph{\textbf{Occhiata Temibile.}} La mummia prende a bersaglio una creatura che possa vedere e si trovi entro 18 metri da lei. Se il bersaglio può vedere la mummia, deve riuscire un Tiro Salvezza su Volontà CD 16 contro questa magia o restare spaventato fino al termine del prossimo turno della mummia. Se il bersaglio fallisce il Tiro Salvezza di 5 o più, è anche paralizzato per la stessa durata. Un bersaglio che riesca il Tiro Salvezza è immune all'Occhiata Terribile di tutte le mummie (ma non delle mummie sovrane) per le successive 24 ore.

\textbf{Azioni Aggiuntive}

La mummia sovrana può effettuare 3 Azioni aggiuntive, scelte tra le opzioni seguenti. Può usare solo un'opzione leggendaria alla volta e solo al termine del turno di un'altra creatura. La mummia sovrana recupera le Azioni aggiuntive spese all'inizio del proprio turno.

\emph{\textbf{Attaccare.}} La mummia sovrana effettua un attacco con il pugno putrefacente o usa la sua Occhiata Temibile.

\emph{\textbf{Incanalare Energia Negativa (Costa 2 Azioni).}} La mummia sovrana può scatenare magicamente l'energia negativa. Le creature entro 18 metri dalla mummia sovrana, comprese quelle dietro barriere o angoli, non possono recuperare punti ferita fino al termine del prossimo turno della mummia sovrana.

\emph{\textbf{Parola Blasfema (Costa 2 Azioni).}} La mummia sovrana pronuncia una parola blasfema. Ciascuna creatura, esclusi i non morti, entro 3 metri dalla mummia sovrana e che possa udire questa frase magica deve riuscire un Tiro Salvezza di Tempra CD 16 o restare stordita fino al termine del prossimo turno della mummia sovrana.

\emph{\textbf{Polvere Accecante.}} Polvere e sabbia accecanti turbinano magicamente intorno alla mummia sovrana. Ogni creatura entro 1 metro dalla mummia sovrana deve riuscire un Tiro Salvezza di Tempra CD 16 o restare accecata fino al termine del prossimo turno della creatura.

\emph{\textbf{Turbine di Sabbia (Costa 2 Azioni).}} La mummia sovrana può trasformarsi magicamente in un turbine di sabbia, muovendosi di massimo 18 metri, e tornando poi alla sua forma normale. Mentre è in forma di turbine, la mummia sovrana è immune a tutti i danni, e non può essere afferrata, pietrificata, gettata prona, intralciata o stordita. L'equipaggiamento indossato o trasportato dalla mummia sovrana rimane in suo possesso.

\subsection{Naga}

\medskip\index{Mostri - Naga Guardiano}\textbf{Naga Guardiano}

\emph{Grande mostruosità, legale buono}

\textbf{FORZA} +4

\textbf{DESTREZZA} +4

\textbf{COSTITUZIONE} +3

\textbf{INTELLIGENZA} +3

\textbf{SAGGEZZA} +4

\textbf{CARISMA} +4

\textbf{Iniziativa} +4 -- \textbf{Difesa} 23

\textbf{Punti Ferita} 127 (15d10 + 45)

\textbf{Movimento} 12 m

\textbf{Tiri Salvezza}: Tempra +9, Rif +12, Volontà +12

\textbf{Immunità ai Danni} veleno

\textbf{Immunità alle Condizioni} affascinato, avvelenato 

\textbf{Sensi} scurovisione 18 m 

\textbf{Linguaggi} Celestiale, Comune 

\textbf{Sfida} 10 (5.900 PE)

\emph{\textbf{Incantesimi.}} Il naga ha CM 11. La sua caratteristica da incantatore è la Saggezza (+8 a colpire con attacchi con incantesimo), e ha bisogno solo delle componenti verbali per lanciare i suoi incantesimi. Il naga prepara i seguenti incantesimi:

Trucchetti (a volontà): \emph{fiamma sacra, riparare, taumaturgia}

Difficoltà 16 (4 slot): \emph{comando, cura ferite, scudo della fede}

Difficoltà 19 (3 slot): \emph{bloccare persone, calmare emozioni}

Difficoltà 21 (3 slot): \emph{chiaroveggenza, scagliare maledizione}

Difficoltà 23 (3 slot): \emph{esilio, libertà di movimento}

Difficoltà 26 (2 slot): \emph{colpo infuocato, costrizione}

Difficoltà 29 (1 slot): \emph{visione del vero}

\emph{\textbf{Rinvigorimento.}} Se muore, il naga ritorna in vita in 1d6 giorni e recupera tutti i suoi punti ferita. Solo l'incantesimo \emph{desiderio} può impedire a questo tratto di funzionare.

\textbf{Azioni}

\emph{\textbf{Morso.} Attacco con arma da mischia}: +8 a colpire, portata 3 m, una creatura.

\emph{Colpisce:} 8 (1d8 + 4) danni perforanti, e il bersaglio deve effettuare un Tiro Salvezza di Tempra CD 15, subendo 45 (10d8) danni da veleno se fallisce il Tiro Salvezza, o la metà di questi danni se lo riesce.

\emph{\textbf{Sputare Veleno.} Attacco con arma a Distanza}: +8 a colpire, gittata 5m, una creatura.

\emph{Colpisce:} Il bersaglio deve effettuare un Tiro Salvezza su Tempra CD 15, subendo 45 (10d8) danni da veleno se fallisce il Tiro Salvezza, o la metà di questi danni se lo riesce.

\textbf{Ecologia}\\
Ambiente: Pianure Temperate\\
Organizzazione: Solitario, coppia o nido (3-6)\\
Tesoro: Standard\\
\textbf{Descrizione}\\
Sebbene abbiano un aspetto feroce, con scaglie brillanti, cappucci simili a quelli dei cobra e potenti corpi serpentini, i naga guardiani fungono da coscienziosi protettori di luoghi di eccezionale potere e sacralità. Spesso le loro scaglie sfoggiano disegni elaborati simili a quelli degli esotici serpenti della giungla. Un tipico naga guardiano raggiunge la lunghezza di 4,2 metri e un peso approssimativo di 175 kg.\\
Mentre alcuni naga guardiani aderiscono a pratiche esotiche di divinità antiche o dimenticate, altri sono semplicemente attratti da siti dalla spiccata bellezza naturale, quali templi su imponenti cascate, pinnacoli naturali e cime di montagne, custodendoli con il massimo della reverenza e del senso del dovere. Spesso questi naga si uniscono a fedi ancora attive, servendo come protettori di santuari o antichi tesori. Una coppia di naga può stabilirsi nei pressi di un sito che ritengono meritevole di protezione, covandovi una nidiata e crescendovi la prole. Quando i giovani raggiungono l'età adulta, possono scegliere di partire per cercare la propria casa o rimanere a proteggere la zona sorvegliata dai loro genitori. A volte, un naga guardiano che custodisce delle rovine od un tempio è solo l'ultimo di una successione di sentinelle che si sono avvicendate nel corso dei secoli. Queste sentinelle spesso prendono lo stesso nome dei loro predecessori sembrando un unico individuo eccezionalmente longevo.\\


\medskip\index{Mostri - Naga Spirituale}\textbf{Naga Spirituale}

\emph{Grande mostruosità, caotico malvagio}

\textbf{FORZA} +4

\textbf{DESTREZZA} +3

\textbf{COSTITUZIONE} +2

\textbf{INTELLIGENZA} +3

\textbf{SAGGEZZA} +2

\textbf{CARISMA} +3

\textbf{Iniziativa} +3 -- \textbf{Difesa} 19

\textbf{Punti Ferita} 75 (10d10 + 20)

\textbf{Movimento} 12 m

\textbf{Tiri Salvezza}: Tempra +8, Rif +10, Volontà +10

\textbf{Immunità al Danno} veleno

\textbf{Immunità alle Condizioni} affascinato, avvelenato

\textbf{Sensi} scurovisione 18 m

\textbf{Linguaggi} Abissale, Comune

\textbf{Sfida} 8 (3.900 PE)

\emph{\textbf{Incantesimi.}} Il naga ha CM 10. La sua abilità da incantatore è l'Intelligenza (+6 a colpire con attacchi con incantesimo), e ha bisogno solo delle componenti verbali per eseguire i suoi incantesimi. Il naga prepara i seguenti incantesimi:

Trucchetti (a volontà): \emph{illusione minore, mano magica, raggio di}
\emph{gelo}

Difficoltà 16 (4 slot): \emph{charme su persone, individuazione del
	magico,} \emph{sonno}

Difficoltà 19 (3 slot): \emph{blocca persone, individuazione dei pensieri}

Difficoltà 21 (3 slot): \emph{fulmine, respirare sott'acqua}

Difficoltà 23 (3 slot): \emph{inaridire, porta dimensionale}

Difficoltà 26 (2 slot): \emph{dominare persone}

\emph{\textbf{Rinvigorimento.}} Se muore, il naga ritorna in vita in 1d6 giorni e recupera tutti i suoi punti ferita. Solo l'incantesimo \emph{desiderio} può impedire a questo tratto di funzionare.

\textbf{Azioni}

\emph{\textbf{Morso.} Attacco con arma da mischia}: +7 a colpire, portata 3 m, una creatura.

\emph{Colpisce:} 7 (1d8 + 4) danni perforanti, e il bersaglio deve effettuare un Tiro Salvezza di Tempra CD 13, subendo 31 (7d8) danni da veleno se fallisce il Tiro Salvezza, o la metà di questi danni se lo riesce.

\subsection{Oggetti Animati}

\medskip\index{Mostri - Armatura Animata}\textbf{Armatura Animata}

\emph{Media costrutto, disallineato}

\textbf{FORZA} +2

\textbf{DESTREZZA} +0

\textbf{COSTITUZIONE} +1

\textbf{INTELLIGENZA} -5

\textbf{SAGGEZZA} -4

\textbf{CARISMA} -5

\textbf{Iniziativa} +0 -- \textbf{Difesa} 19

\textbf{Punti Ferita} 33 (6d8 + 6)

\textbf{Movimento} 7 m

\textbf{Tiri Salvezza}: Tempra +2, Rif +0, Volontà -4

\textbf{Immunità al Danno} psichico, veleno

\textbf{Immunità alle Condizioni} accecato, affascinato, assordato, avvelenato, paralizzato, pietrificato, affaticamento, spaventato

\textbf{Sensi} vista cieca 18 m (cieco oltre questo raggio)

\textbf{Linguaggi} -

\textbf{Sfida} 1 (200 PE)

\emph{\textbf{Falso Aspetto.}} Mentre l'armatura rimane immobile, è indistinguibile da una normale armatura.

\emph{\textbf{Suscettibilità all'Anti Magia.}} L'armatura è inabile se si trova nell'area di un \emph{campo anti-magia}. Se è bersaglio di \emph{dissolvi} \emph{magie}, l'armatura deve riuscire un Tiro Salvezza su Tempra contro la CD del Tiro Salvezza dell'incantesimo o restare svenuta per 1 minuto.

\textbf{Azioni}

\emph{\textbf{Multiattacco.}} L'armatura effettua due attacchi da mischia.

\emph{\textbf{Schianto.} Attacco con arma da mischia}: +4 a colpire, portata 1 m, un bersaglio.

\emph{Colpisce:} 5 (1d6 + 2) danni da botta.

\medskip\index{Mostri - Spada Volante}\textbf{Spada Volante}

\emph{Piccola costrutto, disallineato}

\textbf{FORZA} +1

\textbf{DESTREZZA} +2

\textbf{COSTITUZIONE} +0

\textbf{INTELLIGENZA} -5

\textbf{SAGGEZZA} -3

\textbf{CARISMA} -5

\textbf{Iniziativa} +2 -- \textbf{Difesa} 18

\textbf{Punti Ferita} 17 (5d6)

\textbf{Movimento} 0 m, volo 15 m (fluttua)

\textbf{Tiri Salvezza}  Tempra +1, Riflessi +3, Volontà -4

\textbf{Immunità al Danno} psichico, veleno

\textbf{Immunità alle Condizioni} accecato, affascinato, assordato, avvelenato, paralizzato, pietrificato, spaventato

\textbf{Sensi} vista cieca 18 m (cieco oltre questo raggio)

\textbf{Linguaggi} -

\textbf{Sfida} 1/4 (50 PE)

\emph{\textbf{Falso Aspetto.}} Mentre l'arma rimane immobile e non sta volando, è indistinguibile da una normale spada.

\emph{\textbf{Suscettibilità all'Anti Magia.}} La spada è inabile se si trova nell'area di un \emph{campo anti-magia}. Se è bersaglio di \emph{dissolvi} \emph{magie}, la spada deve riuscire un Tiro Salvezza su Tempra contro la CD del Tiro Salvezza dell'incantesimo o restare svenuta per 1 minuto.

\textbf{Azioni}

\emph{\textbf{Spada Lunga.} Attacco con arma da mischia}: +3 a colpire, portata 1 m, un bersaglio.

\emph{Colpisce:} 5 (1d8 + 1) danni taglienti.


\medskip\index{Mostri - Tappeto del Soffocamento}\textbf{Tappeto del Soffocamento}

\emph{Grande costrutto, disallineato}

\textbf{FORZA} +3

\textbf{DESTREZZA} +2

\textbf{COSTITUZIONE} +0

\textbf{INTELLIGENZA} -5

\textbf{SAGGEZZA} -4

\textbf{CARISMA} -5

\textbf{Iniziativa} +2 -- \textbf{Difesa} 13

\textbf{Punti Ferita} 33 (6d10)

\textbf{Movimento} 3 m

\textbf{Tiri Salvezza}: Tempra +4, Riflessi +2, Volontà -4

\textbf{Immunità al Danno} psichico, veleno

\textbf{Immunità alle Condizioni} accecato, affascinato, assordato, avvelenato, paralizzato, pietrificato, spaventato

\textbf{Sensi} vista cieca 18 m (cieco oltre questo raggio)

\textbf{Linguaggi} -

\textbf{Sfida} 2 (450 PE)

\emph{\textbf{Falso Aspetto.}} Mentre il tappeto resta immobile, è indistinguibile da un normale tappeto.

\emph{\textbf{Suscettibilità all'Anti Magia.}} Il tappeto è inabile mentre si trova nell'area di un \emph{campo anti-magia}. Se è il bersaglio di \emph{dissolvi} \emph{magie}, il tappeto deve riuscire un Tiro Salvezza di Tempra contro la CD del Tiro Salvezza dell'incantatore o cadere privo di sensi per 1 minuto.

\emph{\textbf{Trasferimento di Danno.}} Mentre afferra una creatura, il tappeto subisce solo la metà dei danni che gli sono inferti, e la creatura afferrata dal tappeto subisce l'altra metà.

\textbf{Azioni}

\emph{\textbf{Soffocare.} Attacco con arma da mischia}: +5 a colpire, portata 1 m, una creatura di taglia Media o inferiore.

\emph{Colpisce:} La creatura è afferrata (CD 13 per fuggire). Fino al termine dell'afferrare, il bersaglio è intralciato, accecato e rischia di soffocare, ma il tappeto non può soffocare un altro bersaglio. Inoltre, all'inizio di ciascun turno del bersaglio, il bersaglio subisce 10 (2d6 + 3) danni da botta.

\medskip\index{Mostri - Ogre}\textbf{Ogre}

\emph{Grande gigante, caotico malvagio}

\textbf{FORZA} +4

\textbf{DESTREZZA} -1

\textbf{COSTITUZIONE} +3

\textbf{INTELLIGENZA} -3

\textbf{SAGGEZZA} -2

\textbf{CARISMA} -2

\textbf{Iniziativa} -1 -- \textbf{Difesa} 12 (armatura di pelle)

\textbf{Punti Ferita} 59 (7d10 + 21)

\textbf{Movimento} 12 m

\textbf{Tiri Salvezza}: Tempra +6, Riflessi +0, Volontà +1

\textbf{Sensi} scurovisione 18 m

\textbf{Linguaggi} Comune, Gigante

\textbf{Sfida} 2 (450 PE)

\textbf{Azioni}

\emph{\textbf{Randello Pesante.} Attacco con arma da mischia}: +6 a colpire, portata 1 m, un bersaglio.

\emph{Colpisce:} 13 (2d8 + 4) danni da botta. 

\emph{\textbf{Giavellotto.} Attacco con arma da mischia o a Distanza}: +6 a colpire, portata 1 m o gittata 9m, un bersaglio. 

\emph{Colpisce:} 11 (2d6 + 4) danni perforanti.

\textbf{Ecologia}\\
Ambiente: Colline fredde o temperate\\
Organizzazione: Solitario, coppia, gruppo (3-4) o famiglia (5-16)\\
Tesoro: Standard (Armatura di Pelle, Randello Pesante, 4 Giavellotti, altro)\\
\textbf{Descrizione}\\
Nelle storie riguardanti gli ogre ci sono elementi orrendi: brutalità e ferocia, cannibalismo e tortura. Poi stupri, smembramenti, necrofilia, incesto, mutilazioni e altri esempi di crudeltà. Coloro che non hanno mai incontrato gli ogre ritengono queste storie un avvertimento. Chi è sopravvissuto ad un simile incontro sa che le storie sono niente in confronto alla realtà.\\

Gli ogre godono della sofferenza altrui. Se non hanno a disposizione le razze più piccole da schiacciare fra le loro grasse mani o da violare in amplessi violenti, si divertono fra loro. Per gli ogre non esiste tabù. Si potrebbe pensare che, lasciata a sé stessa, una tribù di ogre si farebbe a pezzi da sola e che soltanto i più forti sopravvivrebbero: se c'è una cosa che gli ogre rispettano, però, è la famiglia.\\

Le tribù ogre sono conosciute come famiglie, e molte delle loro deformità sono causate dalla pratica comune dell'incesto. Il capo della tribù è spesso il padre, ma in alcuni casi un'ogre femmina è in grado di reclamare il titolo di madre. Le tribù ogre litigano fra loro, cosa che li tiene impegnati ed impedisce loro di tormentare i loro vicini. Di quando in quando, però, emerge un patriarca particolarmente violento o temuto, capace di unire più famiglie sotto il suo comando.\\

Le regioni abitate degli ogre sono luoghi tristi e degradati, dato che questi giganti vivono nello squallore e non sentono il bisogno di essere in armonia con quanto li circonda. Il confine fra le terre civilizzate e quelle degli ogre è un luogo di disperazione abitato da reietti, dove vivono gli Ogremanni, progenie deformi che nascono dalle razzie che gli ogre effettuano nelle terre degli umani.\\

I giochi degli ogre sono violenti e crudeli: le vittime utilizzate come giocattolo sono fortunate a morire il primo giorno. Il crudele senso dell'umorismo degli ogre è il solo caso in cui mostrano di possedere creatività: i metodi e gli strumenti di tortura ogre sembrano usciti dagli incubi.\\

La grande forza e la mancanza di immaginazione li rendono particolarmente adatti ai lavori pesanti, nelle miniere, come fabbri o nel disboscamento. I giganti più potenti (soprattutto quelli delle Colline e delle Rocce) spesso soggiogano le famiglie ogre perché diventino loro servitori.\\
Un ogre adulto è alto sui 3 metri e pesa circa 325 kg.\\


\medskip\index{Mostri - Ombra}\textbf{Ombra}

\emph{Media non morto, caotico malvagio}

\textbf{FORZA} -2

\textbf{DESTREZZA} +2

\textbf{COSTITUZIONE} +1

\textbf{INTELLIGENZA} -2

\textbf{SAGGEZZA} +0

\textbf{CARISMA} -1

\textbf{Iniziativa} +2 -- \textbf{Difesa} 13

\textbf{Punti Ferita} 16 (3d8 + 3)

\textbf{Movimento} 12 m

\textbf{Tiri Salvezza}: Tempra +3, Riflessi +3, Volontà +4

\textbf{Competenze} Muoversi Silenziosamente / Nascondersi +4 (+6 a luce fioca o oscurità)

\textbf{Vulnerabilità al Danno} da Luce

\textbf{Resistenze al Danno} acido, freddo, fulmine, fuoco, tuono; da botta, perforante e tagliente di attacchi non magici

\textbf{Immunità al Danno} da Vuoto, veleno

\textbf{Immunità alle Condizioni} afferrato, avvelenato, intralciato, paralizzato, pietrificato, prono, affaticamento, spaventato

\textbf{Sensi} scurovisione 18 m

\textbf{Linguaggi} -

\textbf{Sfida} 1/2 (100 PE)

\emph{\textbf{Amorfo.}} L'ombra può muoversi attraverso uno spazio stretto fino a 3 centimetri senza stringersi.

\emph{\textbf{Debolezza alla Luce del Sole.}} Mentre si trova alla luce del sole, l'ombra ha -1d6 ai tiri per colpire, le prove di competenza e i Tiri Salvezza.

\emph{\textbf{Furtività d'Ombra.}} Quando si trova a luce fioca o all'oscurità, l'ombra può effettuare l'azione Nascondersi come azione bonus. 

\emph{\textbf{Natura Non Morta.}} Un'ombra non necessita aria, cibo, bevande o sonno.

\textbf{Azioni}

\emph{\textbf{Risucchio di Forza.} Attacco con arma da mischia}: +4 a colpire, portata 1 m, una creatura.

\emph{Colpisce:} 9 (2d6 + 2) danni da Vuoto, e il punteggio di Forza del bersaglio viene ridotto di 1d4. Il bersaglio muore se ciò riduce la sua Forza a 0. Altrimenti, la riduzione resta finché il bersaglio non riposa 8 ore.

Se un umanoide non malvagio muore a causa di questo attacco, entro 1d4 ore dal suo cadavere si animerà una nuova ombra.

\textbf{Ecologia}
Ambiente: Qualsiasi\\
Organizzazione: Solitario, coppia, gruppo (3–6) o sciame (7–12)\\
Tesoro: Standard\\

\textbf{Descrizione}\\
La malvagia ombra si muove lungo il confine tra il buio delle tenebre e la dura verità della luce. L’ombra preferisce infestare le rovine che la civiltà si lascia alle spalle, dove dà la caccia alle creature viventi tanto sciocche da incappare nel suo territorio. L’ombra è un orribile non morto, e come tale non ha scopi o motivazioni apparenti oltre a risucchiare forza vitale e vitalità dagli esseri viventi.


\medskip\index{Mostri - Omuncolo}\textbf{Omuncolo}

\emph{Minuscola costrutto, neutrale}

\textbf{FORZA} -3

\textbf{DESTREZZA} +2

\textbf{COSTITUZIONE} +0

\textbf{INTELLIGENZA} +0

\textbf{SAGGEZZA} +0

\textbf{CARISMA} -2

\textbf{Iniziativa} +2 -- \textbf{Difesa} 14

\textbf{Punti Ferita} 5 (2d4)

\textbf{Movimento} 6 m, volo 12 m

\textbf{Tiri Salvezza}:  Tempra +0, Riflessi +4, Volontà +1

\textbf{Immunità al Danno} veleno

\textbf{Immunità alle Condizioni} affascinato, avvelenato

\textbf{Sensi} scurovisione 18 m, vista cieca 3 m

\textbf{Linguaggi} comprende le lingue del suo creatore ma non può parlare

\textbf{Sfida} 0 (10 PE)

\emph{\textbf{Legame Telepatico.}} Mentre l'omuncolo si trova sullo stesso piano di esistenza del suo padrone, può comunicare magicamente al suo padrone quello che percepisce, e i due possono comunicare telepaticamente.

\textbf{Azioni}

\emph{\textbf{Morso.} Attacco con arma da mischia}: +4 a colpire, portata 1 m, una creatura.

\emph{Colpisce:} 1 danno perforante, e il bersaglio deve riuscire un Tiro Salvezza di Tempra CD 10 o restare avvelenato per 1 minuto. Se il Tiro Salvezza viene fallito di 5 o più, il bersaglio resta invece avvelenato per 5 (1d10) minuti e mentre è avvelenato in questo modo è anche privo di sensi.

\medskip\index{Mostri - Oni}\textbf{Oni}

\emph{Grande gigante, legale malvagio}

\textbf{FORZA} +4

\textbf{DESTREZZA} +0

\textbf{COSTITUZIONE} +3

\textbf{INTELLIGENZA} +2

\textbf{SAGGEZZA} +1

\textbf{CARISMA} +2

\textbf{Iniziativa} +2 -- \textbf{Difesa} 20 (cotta di maglia)

\textbf{Punti Ferita} 110 (13d10 + 39)

\textbf{Movimento} 9 m, volo 9 m

\textbf{Tiri Salvezza}: Tempra +7, Riflessi +4, Volontà +6

\textbf{Competenze} Arcano +5, Ingannare +8, Consapevolezza +4 

\textbf{Sensi} scurovisione 18 m 

\textbf{Linguaggi} Comune, Gigante

\textbf{Sfida} 7 (2.900 PE)

\emph{\textbf{Armi Magiche.}} Gli attacchi con armi dell'oni sono magici.

\emph{\textbf{Incantesimi Innati.}} La caratteristica da incantatore dell'oni è il Carisma. L'oni può lanciare questi incantesimi in maniera innata, senza bisogno di componenti materiali:

A volontà: \emph{invisibilità, oscurità}

1/giorno: \emph{charme su persone, cono di freddo, forma gassosa,}
\emph{sonno}

\emph{\textbf{Rigenerazione.}} Se ha almeno 1 punto ferita, l'oni recupera 10 punti ferita all'inizio del suo turno.

\textbf{Azioni}

\emph{\textbf{Multiattacco.}} L'oni effettua due attacchi, con gli artigli o con il falcione.

\emph{\textbf{Artiglio (Solo Forma di Oni).} Attacco con arma da mischia}: +7 a colpire, portata 1 m, un bersaglio. \emph{Colpisce:} 8 (1d8 + 4) danni taglienti.

\emph{\textbf{Falcione.} Attacco con arma da mischia}: +7 a colpire, portata 3 m, un bersaglio.

\emph{Colpisce:} 15 (2d10 + 4) danni taglienti, o 9 (1d10 + 4) danni taglienti in forma Piccola o Media.

\emph{\textbf{Mutare Forma.}} L'oni può trasformarsi magicamente in un umanoide Piccolo o Medio, in un gigante Grande, o tornare alla sua vera forma. A parte la taglia, le sue statistiche sono le stesse in ciascuna forma. L'unico equipaggiamento che viene trasformato è il falcione, che rimpicciolisce in modo da essere impugnato anche in forma umanoide. Se l'oni muore, ritorna alla sua vera forma, e il falcione ritorna alla sua taglia originale.

\medskip\index{Mostri - Orco}\textbf{Orco}

\emph{Media umanoide (orco), caotico malvagio}

\textbf{FORZA} +3

\textbf{DESTREZZA} +1

\textbf{COSTITUZIONE} +3

\textbf{INTELLIGENZA} -2

\textbf{SAGGEZZA} +0

\textbf{CARISMA} +0

\textbf{Iniziativa} +1 -- \textbf{Difesa} 14 (armatura di pelle)

\textbf{Punti Ferita} 15 (2d8 + 6)

\textbf{Movimento} 9 m

\textbf{Tiri Salvezza}: Tempra +3, Riflessi +1, Volontà +1

\textbf{Competenze} Intimidire +2

\textbf{Sensi} scurovisione 18 m

\textbf{Linguaggi} Comune, Goblinoide

\textbf{Sfida} 1/2 (100 PE)

\emph{\textbf{Aggressivo.}} Come azione bonus, l'orco può muoversi fino a metà del suo movimento verso una creatura ostile che possa vedere.

\textbf{Azioni}

\emph{\textbf{Ascia Bipenne.} Attacco con arma da mischia}: +5 a colpire, portata 1 m, un bersaglio.

\emph{Colpisce:} 9 (1d12 + 3) danni taglienti.

\emph{\textbf{Giavellotto.} Attacco con arma da mischia o a Distanza}: +5 a colpire, portata 1 m o gittata 9m, un bersaglio. \emph{Colpisce:} 6 (1d6 + 3) danni perforanti.

\textbf{Ecologia}\\
Ambiente: Colline e montagne temperate o sotterranei\\
Organizzazione: solitario, gruppo (2-4), squadra (11-20 più 2 sergenti di 3° livello e 1 capo di 3°-6° livello) o banda \\
Tesoro: Equipaggiamento da PNG (Armatura di Cuoio Borchiato, Falchion, 4 Giavellotti, altro tesoro)\\
\textbf{Descrizione}\\
La differenza principale fra gli orchi e gli umanoidi civilizzati, oltre alla loro forza bruta ed all'intelligenza inferiore, è il loro carattere. Come cultura, gli orchi sono violenti ed aggressivi, ed il forte domina il debole attraverso paura e brutalità. Prendono ciò che vogliono con la forza e non si fanno scrupoli a prendere interi villaggi come schiavi se ne hanno la possibilità. Non si curano delle comodità, ed i loro villaggi e campi tendono ad essere luoghi sporchi e precari, pieni di risse fra ubriachi, arene per i combattimenti ed altri divertimenti sadici. Privi della pazienza necessaria a coltivare e capaci di allevare solo gli animali più robusti ed autosufficienti, gli orchi ritengono più semplice prendere agli altri il frutto del loro lavoro. Sono arroganti e lesti ad infuriarsi quando sfidati, ma si preoccupano dell'onore solo finché farlo porta loro beneficio.\\

Un orco maschio adulto è alto 1,8 metri e pesa circa 105 kg. Gli orchi e gli umani possono accoppiarsi, anche se di solito ciò avviene durante le razzie, e non come unione consensuale. Molte tribù orchesche allevano i mezzorchi di proposito, dato che sono ottimi strateghi e capitribù.\\


\medskip\index{Mostri - Orsogufo}\textbf{Orsogufo}

\emph{Grande mostruosità, disallineato}

\textbf{FORZA} +5

\textbf{DESTREZZA} +1

\textbf{COSTITUZIONE} +3

\textbf{INTELLIGENZA} -4

\textbf{SAGGEZZA} +1

\textbf{CARISMA} -2

\textbf{Iniziativa} +1 -- \textbf{Difesa} 15

\textbf{Punti Ferita} 59 (7d10 + 21)

\textbf{Movimento} 12 m

\textbf{Tiri Salvezza}: Tempra +10, Riflessi +5, Volontà +2

\textbf{Competenze} Consapevolezza +3

\textbf{Sensi} scurovisione 18 m

\textbf{Linguaggi} -

\textbf{Sfida} 3 (700 PE)

\emph{\textbf{Olfatto e Vista Affinati.}} L'orsogufo ha +1d6 nelle prove di Saggezza (Consapevolezza) basate su olfatto o vista. 

\textbf{Azioni}

\emph{\textbf{Multiattacco.}} L'orsogufo effettua due attacchi: uno con il becco e uno con gli artigli.

\emph{\textbf{Artigli.} Attacco con arma da mischia}: +7 a colpire, portata 1 m, un bersaglio.

\emph{Colpisce:} 14 (2d8 + 5) danni taglienti.

\emph{\textbf{Becco.} Attacco con arma da mischia}: +7 a colpire, portata 1 m, una creatura.

\emph{Colpisce:} 10 (1d10 + 5) danni perforanti.

\textbf{Ecologia}\\
\textbf{Ambiente: Foreste Temperate}
Organizzazione: Solitario, coppia o branco (3-8)\\
Tesoro: Accidentale\\
\textbf{Descrizione}\\
Le origini dell'orsogufo sono oggetto di dibattito fra gli studiosi delle creature mostruose. La maggior parte di essi concorda che fu un Mago, in passato, a crearne il primo esemplare unendo un orso con un gufo gigante; forse come esperimento su qualche folle concetto della natura della vita, ma più probabilmente a causa della sua totale pazzia. Quale che fosse lo scopo originale di una creazione tanto folle come l'orsogufo, la creatura ha iniziato a riprodursi, ed è divenuta uno dei predatori più conosciuto delle zone boschive.\\
Gli orsigufo sono selvaggi predatori, noti per il loro pessimo temperamento, la loro aggressività e la loro ferocia. Tendono ad attaccare tutto ciò che si muove loro davanti, anche se questo non mostra intenzioni bellicose. Molti studiosi che hanno incontrato queste creature nelle terre selvagge hanno notato che hanno sempre occhi iniettati di sangue che ruotano tutto attorno poco prima di un attacco. Questo è generalmente visto come segno di follia, che suggerisce che tutti gli orsigufo nascano con un bisogno patologico di combattere ed uccidere, ma i ricercatori più realisti ritengono sia dovuto alla struttura dei loro occhi acuti.\\
Gli orsigufo abitano le zone più interne e nascoste dei boschi, e preparano le loro tane all'interno di foreste intricate o di buie e profonde caverne. Possono cacciare sia di giorno che di notte, a seconda delle abitudini delle prede che popolano i territori circostanti alla loro tana.\\
Gli orsigufo adulti vivono in coppia e cacciano le prede in branco, lasciando i cuccioli nelle tane. In una tana si possono trovare di solito 1d6 cuccioli, che possono valere fino a 3.000 mo nei mercati cittadini.\\

Anche se è pressochè impossibile addomesticarli a causa della loro natura selvaggia, gli orsigufo possono essere sfruttati come guardiani di un territorio specifico, sempre che vengano lasciati liberi di spostarsi al suo interno per cacciare. Gli addestratori professionisti chiedono fino a 2.000 mo per addestrare un orsogufo perché diventi un guardiano che obbedisca a comandi semplici (DC 23 per un cucciolo di orsogufo, DC 30 per un orsogufo adulto).\\


\medskip\index{Mostri - Otyugh}\textbf{Otyugh}

\emph{Grande aberrazione, neutrale}

\textbf{FORZA} +3

\textbf{DESTREZZA} +0

\textbf{COSTITUZIONE} +4

\textbf{INTELLIGENZA} -2

\textbf{SAGGEZZA} +1

\textbf{CARISMA} -2

\textbf{Iniziativa} +0 -- \textbf{Difesa} 17

\textbf{Punti Ferita} 114 (12d10 + 48)

\textbf{Movimento} 9 m

\textbf{Tiri Salvezza}: Tempra +3, Riflessi +2, Volontà +6

\textbf{Sensi} scurovisione 36 m

\textbf{Linguaggi} Otyugh

\textbf{Sfida} 5 (1.800 PE)

\emph{\textbf{Telepatia Limitata.}} L'otyugh può trasmettere magicamente dei semplici messaggi e immagini a qualsiasi creatura entro 36 metri da esso e che possa comprendere una lingua. Questa forma di telepatia non permette alla creatura ricevente di rispondere telepaticamente.

\textbf{Azioni}

\emph{\textbf{Multiattacco.}} L'otyugh effettua tre attacchi: uno con il morso e due con i tentacoli.

\emph{\textbf{Morso.} Attacco con arma da mischia}: +6 a colpire, portata 1 m, un bersaglio.

\emph{Colpisce:} 12 (2d8 + 3) danni perforanti. Se il bersaglio è una creatura, deve riuscire un Tiro Salvezza di Tempra CD 15 contro malattia o restare avvelenato finché la malattia non viene curata. Ogni 24 ore successive, il bersaglio deve ripetere il Tiro Salvezza, riducendo il suo massimo di punti ferita di 5 (1d10) se lo fallisce. Se il Tiro Salvezza riesce, la malattia è passata. Il bersaglio muore se la malattia riduce i suoi punti ferita massimi a 0.

Questa riduzione dei punti ferita massimi del personaggio, perdura finché la malattia non viene curata.
\
\emph{\textbf{Tentacolo.} Attacco con arma da mischia}: +6 a colpire, portata 3 m, un bersaglio.

\emph{Colpisce:} 7 (1d8 + 3) danni da botta più 4 (1d8) danni perforanti. Se il bersaglio è di taglia Media o inferiore, è afferrato (CD 13 per fuggire) e intralciato fino al termine dell'afferrare. L'otyugh ha due tentacoli, ciascun dei quali può afferrare un bersaglio diverso.

\emph{\textbf{Schianto di Tentacolo.}} L'otyugh schianta le creature afferrate dai suoi tentacoli, l'una contro l'altra o sul pavimento. Ogni creatura deve riuscire un Tiro Salvezza di Tempra CD 14 o subire 10 (2d6 + 3) danni da botta e restare stordita fino al termine del prossimo turno dell'otyugh. Se il Tiro Salvezza riesce, il bersaglio subisce la metà dei danni da botta e non è stordito.

\textbf{Ecologia}\\
Ambiente: Qualsiasi Sotterraneo\\
Organizzazione: Solitario, coppia o gruppo (3-4)\\
Tesoro: Standard\\
\textbf{Descrizione}\\
Gli otyugh sono creature particolarmente luride ed orride che vivono in luoghi che le persone sane di mente tendono ad evitare. Le loro tane si trovano nelle fogne, nei pozzi neri, nelle discariche e nelle paludi più mefitiche: più un luogo è sporco, più attira gli otyugh. Amano il ruolo dello spazzino, e vagano per le caverne sotterranee in cerca di nuovi bocconcini in mezzo ai rifiuti. Una volta trovati, si ingozzano e riportano alla loro tana quello che non riescono a consumare in una volta sola. Gli otyugh passano parecchio tempo nelle loro luride tane, che riempiono di carogne e letame, che rilasciano effluvi mefitici.\\
Le creature intelligenti che vivono nelle zone sotterranee vicino agli otyugh a volte formano alleanze di convenienza con essi. Forniscono loro rifiuti e carne cruda agli otyugh, rendendoli un vero e proprio mezzo di smaltimento. In cambio, gli otyugh lasciano in pace i loro benefattori, non li attaccano e possono anche fare da guardiani.\\
La cosa che la maggior parte delle razze trova terrificante degli otyugh non è la loro dieta o l'odore delle loro tane, ma il fatto che creature con i loro gusti non siano solo spazzini senza cervello. Gli otyugh si mostrano infatti sorprendentemente intelligenti, ed amano formare alleanze con coloro che li riforniscono di cibi più raffinati di letame e sporcizia. La maggior parte degli otyugh si rende conto che le altre creature li trovano rivoltanti, ma sono pochi quelli a cui importa davvero.\\


\medskip\index{Mostri - Pegaso}\textbf{Pegaso}

\emph{Grande celestiale, caotico buono}

\textbf{FORZA} +4

\textbf{DESTREZZA} +2

\textbf{COSTITUZIONE} +3

\textbf{INTELLIGENZA} +0

\textbf{SAGGEZZA} +2

\textbf{CARISMA} +1

\textbf{Iniziativa} +2 -- \textbf{Difesa} 13

\textbf{Punti Ferita} 59 (7d10 + 21)

\textbf{Movimento} 18 m, volo 27 m

\textbf{Tiri Salvezza} Tempra +7, Riflessi +6, Volontà +4

\textbf{Competenze} Consapevolezza +6

\textbf{Linguaggi} comprende Celestiale, Comune, Elfico e Silvano ma non può parlare

\textbf{Sfida} 2 (450 PE)

\textbf{Azioni}

\emph{\textbf{Zoccoli.} Attacco con arma da mischia}: +6 a colpire, portata 1 m, un bersaglio.

\emph{Colpisce:} 11 (2d6 + 4) danni da botta.

\textbf{Ecologia}
Ambiente: Pianure Temperate e Calde\\
Organizzazione: Solitario, coppia o branco (6-10)\\
Tesoro: Nessuno\\
\textbf{Descrizione}\\
Il pegaso è un magnifico cavallo alato che a volte serve la causa del bene. Seppur molto apprezzati come cavalcature volanti, i pegasi sono creature timide che difficilmente stringono amicizie. Un tipico pegaso è alto 1,8 metri al garrese, pesa 750 kg ed ha un'apertura alare di 6 metri. La maggior parte dei pegasi è bianca, ma a volte alcuni esemplari hanno colori diversi.\\

Il pegaso, nonostante le apparenze, è intelligente quanto un umano. Chi cerca di addestrarne uno a fare da cavalcatura, scoprirà che il pegaso è ricalcitrante e perfino violento. Un pegaso non può parlare, ma capisce il Comune e preferisce la compagnia di creature buone. Il metodo corretto per convincere un pegaso a fare da cavalcatura è farselo amico con Diplomazia, favori e buone azioni. Un pegaso ha di norma atteggiamento indifferente verso le creature buone, maldisposto verso quelle neutrali ed ostile verso quelle malvagie. Prima che possa servire come cavalcatura, un pegaso deve essere reso amichevole tramite una prova di Diplomazia o in altro modo. Cavalcare un pegaso richiede una sella esotica o Cavalcare a pelo, dato che una sella normale interferisce con le sue ali. Un pegaso può combattere portando un cavaliere, ma il cavaliere non può attaccare a sua volta se non supera una prova di Cavalcare. I pegasi addestrati non temono il combattimento ed il cavaliere non deve effettuare una prova di Cavalcare per controllarlo.\\

I pegasi depongono uova che sul mercato valgono 2.000 mo l'una, mentre i piccoli arrivano alle 3.000 mo a testa. Essendo creature intelligenti e buone, vendere uova e piccoli è essenzialmente schiavismo: nelle società buone chi lo fa è disprezzato o punito dalla legge.\\

I pegasi maturano come i cavalli. Gli addestratori professionisti chiedono 1.000 per addestrare un pegaso, che servirà un cavaliere buono o neutrale fedelmente per tutta la vita.\\

Un carico leggero per un pegaso è fino a 150 kg; un carico medio è 150,5-300 kg; un carico pesante è 300,5-450 kg.\\

In alcuni pegasi il sangue di un antenato che era un eroico stallone è ancora forte. Questi campioni hanno la durata della vita di un umano, l'archetipo avanzato, manovrabilità perfetta, resistenza al fuoco 10, un bonus razziale di +4 ai Tiri Salvezza contro i Veleni e immunità alla Pietrificazione. Alcuni riescono a dire poche parole in Celestiale o Comune. Si rendono conto della loro superiorità agli altri cavalli ed ai pegasi, e non devono essere addestrati a volare con un cavaliere, ma permettono solo ai più grandi eroi di cavalcarli.\\


\medskip\index{Mostri - Persecutore Invisibile}\textbf{Persecutore Invisibile}

\emph{Media elementale, neutrale}

\textbf{FORZA} +3

\textbf{DESTREZZA} +4

\textbf{COSTITUZIONE} +2

\textbf{INTELLIGENZA} +0

\textbf{SAGGEZZA} +2

\textbf{CARISMA} +0

\textbf{Iniziativa} +4 -- \textbf{Difesa} 17

\textbf{Punti Ferita} 104 (16d8 + 32)

\textbf{Movimento} 15 m, volo 15 m (fluttua)

\textbf{Tiri Salvezza}: Tempra +13, Riflessi +11, Volontà +4

\textbf{Competenze} Muoversi Silenziosamente / Nascondersi +10, Consapevolezza +8

\textbf{Resistenze al Danno} da botta, perforante e tagliente di attacchi non magici

\textbf{Immunità ai Danni} veleno

\textbf{Immunità alle Condizioni} afferrato, avvelenato, intralciato, paralizzato, pietrificato, privo di sensi, prono, affaticamento

\textbf{Sensi} scurovisione 18 m

\textbf{Linguaggi} Auran, comprende il Comune ma non lo parla

\textbf{Sfida} 6 (2.300 PE)

\emph{\textbf{Cacciatore Infallibile.}} Il convocatore assegna una preda
al persecutore. Il persecutore sa la direzione e la distanza a cui si
trova la preda finché entrambi si trovano sullo stesso piano di
esistenza. Il persecutore conosce anche la posizione del suo
convocatore.

\emph{\textbf{Invisibilità.}} Il persecutore è invisibile.

\emph{\textbf{Natura Elementale.}} Un persecutore invisibile non ha bisogno di aria, cibo, bevande o sonno.

\textbf{Azioni}

\emph{\textbf{Multiattacco.}} La persecutore effettua due attacchi di schianto.

\emph{\textbf{Schianto.} Attacco con arma da mischia}: +6 a colpire, portata 1 m, un bersaglio.

\emph{Colpisce:} 10 (2d6 + 3) danni da botta.

\medskip\index{Mostri - Pseudodrago}\textbf{Pseudodrago}

\emph{Minuscola drago, neutrale buono}

\textbf{FORZA} -2

\textbf{DESTREZZA} +2

\textbf{COSTITUZIONE} +1

\textbf{INTELLIGENZA} +0

\textbf{SAGGEZZA} +1

\textbf{CARISMA} +0

\textbf{Iniziativa} +2 -- \textbf{Difesa} 14

\textbf{Punti Ferita} 7 (2d4 + 2)

\textbf{Movimento} 5 metri, volo 18 m

\textbf{Tiri Salvezza}: Tempra +4, Riflessi +5, Volontà +4

\textbf{Competenze} Muoversi Silenziosamente / Nascondersi +4, Consapevolezza +3

\textbf{Sensi} scurovisione 18 m, vista cieca 3 m

\textbf{Linguaggi} comprende il Comune e il Draconico ma non parla

\textbf{Sfida} 1/4 (50 PE)

\emph{\textbf{Resistenza alla Magia.}} Lo pseudodrago ha +1d6 ai Tiri Salvezza contro incantesimi e altri effetti magici.

\emph{\textbf{Sensi Affinati.}} Lo pseudodrago ha +1d6 alle prove di Saggezza (Consapevolezza) basate su vista, udito e olfatto.

\emph{\textbf{Telepatia Limitata.}} Lo pseudodrago può comunicare semplici idee, emozioni e immagini telepaticamente con qualsiasi creatura entro 30 metri da esso che può comprendere una lingua.

\textbf{Azioni}

\emph{\textbf{Morso.} Attacco con arma da mischia}: +4 a colpire, portata 1 m, un bersaglio.

\emph{Colpisce:} 4 (1d4 + 2) danni perforanti.

\emph{\textbf{Pungiglione.} Attacco con arma da mischia}: +4 a colpire, portata 1 m, una creatura.

\emph{Colpisce:} 4 (1d4 + 2) danni perforanti, e il bersaglio deve riuscire un Tiro Salvezza di Tempra CD 11 o restare avvelenato per 1 ora. Se il risultato del Tiro Salvezza è 6 o meno, il bersaglio cade privo di sensi per la stessa durata, o finché non subisce danni o un'altra creatura usa un'azione per risvegliarlo.

\textbf{Ecologia}\\
Ambiente: Foreste temperate\\
Organizzazione: Solitario, coppia o nido (3-5)\\
Tesoro: Standard\\
\textbf{Descrizione}\\
Gli pseudodraghi sono piccoli parenti dei veri draghi, giocosi e timidi. Parlano cinguettando, sibilando, ringhiando e facendo le fusa, ma possono comunicare telepaticamente con qualsiasi creatura intelligente. Se avvicinati pacificamente con offerte di cibo, sono disposti a condividere informazioni su quanto si trova nel loro territorio, ma minacce e violenza li fanno fuggire.\\

Gli pseudodraghi sono carnivori e mangiano insetti, roditori, uccellini e serpenti, anche se mangiano uova ed amano burro, formaggio e pesce. A volte cacciano a terra come le lucertole o volando come gli uccelli predatori. Intelligenti come la maggior parte degli umanoidi, non amano essere trattati come animali domestici, e preferiscono essere considerati amici. Diffidano delle creature malvagie, possono unirsi a incantatori e Devoti come Famigli e alcuni hanno stretto amicizia con Druidi e guardiaboschi o collaborano con i draghi buoni come sentinelle. Gli pseudodraghi diventano Famigli solo se apprezzano la personalità dell'incantatore (e se questi ha l'Abilita' Famiglio e Carisma almeno 1), ma possono anche legarsi a persone delle quali apprezzano la compagnia. Uno pseudodrago potrebbe seguire in questo modo un personaggio per giorni, settimane, anni o perfino per tutta la vita, posto che siano ben nutriti e trattati con affetto.\\

Raggiunta l'età adulta, il corpo di uno pseudodrago è lungo 30 centimetri con una coda di 60 centimetri, e pesa circa 3,5 kg. Le uova di uno pseudodrago sono grandi come quelle di gallina, ma di consistenza simile al cuoio e macchiate di marrone, e le femmine le depongono in gruppi di 2-5 ogni primavera. Un nido di pseudodraghi (che costituiscono un gruppo familiare, e non sono nati dallo stesso gruppo di uova) di solito consiste di una coppia di adulti e diversi cuccioli quasi adulti.\\


\medskip\index{Mostri - Rakshasa}\textbf{Rakshasa}

\emph{Media immondo, legale malvagio}

\textbf{FORZA} +2

\textbf{DESTREZZA} +3

\textbf{COSTITUZIONE} +4

\textbf{INTELLIGENZA} +1

\textbf{SAGGEZZA} +3

\textbf{CARISMA} +5

\textbf{Iniziativa} +3 -- \textbf{Difesa} 23

\textbf{Punti Ferita} 110 (13d8 + 52)

\textbf{Movimento} 12 m

\textbf{Tiri Salvezza}: Tempra +9, Riflessi +12, Volontà +8 

\textbf{Competenze} Ingannare +10, Percepire Emozioni +8

\textbf{Vulnerabilità al Danno} perforante di armi magiche impugnate da
creatura buone

\textbf{Immunità al Danno} da botta, armi +1

\textbf{Sensi} scurovisione 18 m

\textbf{Linguaggi} Comune, Infernale

\textbf{Sfida} 13 (10.000 PE)

\emph{\textbf{Immunità alla Magia Limitata.}} Il rakshasa è immune agli affetti o all'individuazione tramite incantesimi di Difficoltà 29 o più basso a meno che non desideri esserne soggetto. Ha +1d6 ai Tiri Salvezza contro tutti gli altri incantesimi ed effetti magici.

\emph{\textbf{Incantesimi Innati.}} La caratteristica da incantatore del rakshasa il Carisma (+10 a colpire   con attacchi con incantesimi). Il rakshasa può lanciare in maniera innata i seguenti incantesimi, senza aver bisogno di componenti materiali:

A volontà: \emph{camuffare sé stesso, illusione minore, individuazione} \emph{dei pensieri, mano magica}

3/Giorno ciascuno: \emph{charme su persone, immagine maggiore,} \emph{individuazione del magico, invisibilità, suggestione} 1/Giorno: \emph{dominare persone, spostamento planare, visione del} \emph{vero, volare}

\textbf{Azioni}

\emph{\textbf{Multiattacco.}} Il rakshasa può effettuare due attacchi di artiglio.

\emph{\textbf{Artiglio.} Attacco con arma da mischia}: +7 a colpire, portata 1 m, un bersaglio.

\emph{Colpisce:} 9 (2d6 + 2) danni taglienti, e se il bersaglio è una creatura rimane maledetto. La maledizione magica ha effetto ogni qualvolta il bersaglio riposa, riempiendo i pensieri del bersaglio di immagini e sogni orribili. Il bersaglio maledetto non riceve beneficio dall'aver terminato un riposo. La maledizione perdura finché non viene rimossa dall'incantesimo \emph{rimuovi maledizione} o simile magia.

\textbf{Ecologia}
Ambiente: Qualsiasi\\
Organizzazione: Solitario, coppia o culto (3-12)\\
Tesoro: Doppio (Pugnale+1, altro tesoro)\\
\textbf{Descrizione}\\
Il rakshasa è uno spirito maligno che si traveste da creatura umanoide così da poter seguire la sua preda in incognito. Personificazione dei tabù della maggioranza delle società e capace di assumere l'aspetto di quelli che cerca di corrompere, un rakshasa compie moltissime azioni orribili. Se fossero umani, la loro blasfemia, il cannibalismo e gli atti ancora peggiori che compiono li marchierebbero come criminali meritevoli del più crudele degli inferni.\\
Quando non ha un altro aspetto, il rakshasa appare come un umanoide con la testa di un animale. Spesso ha il capo di un grosso felino (come tigri o pantere) o serpente (quali cobra o vipere) e, seppur sia più raro, può avere testa di gorilla, sciacallo, avvoltoio, elefante, mantide, lucertola, rinoceronte, cinghiale e molte altre ancora. In molti casi, il tipo di testa posseduta da un rakshasa dice qualcosa della sua personalità: un rakshasa dalla testa di tigre è furtivo e famelico, mentre uno con la testa di cinghiale può essere ghiotto e crudele. Queste differenze raramente incidono sulle statistiche base del rakshasa, anche se esistono varianti più potenti della standard con molteplici teste, poteri magici più potenti, e strane e letali capacità speciali aggiuntive.\\
I rakshasa disprezzano le religioni; riconoscono il potere degli dei, ma si vedono come i soli esseri degni di venerazione da parte delle razze mortali. I Devoti rakshasa sono quindi piuttosto rari. Sebbene i rakshasa siano esterni, sono anche creature del Piano Materiale, e alcuni credono che i primi rakshasa scelsero questo esilio al posto di qualche altro ruolo offertogli da un dio da tempo dimenticato. Anche se in genere sono solitari, non è raro trovare grandi famiglie di rakshasa che lavorano insieme per provocare la caduta di una civiltà mortale dall'interno, attraverso il succedersi di molte generazioni.\\
Un rakshasa è alto 1,8 metri e pesa 90 kg.\\


\medskip\index{Mostri - Remorhaz}\textbf{Remorhaz}

\emph{Enorme mostruosità, disallineato}

\textbf{FORZA} +7

\textbf{DESTREZZA} +1

\textbf{COSTITUZIONE} +5

\textbf{INTELLIGENZA} -3

\textbf{SAGGEZZA} +0

\textbf{CARISMA} -3

\textbf{Iniziativa} +1 -- \textbf{Difesa} 23

\textbf{Punti Ferita} 195 (17d12 + 85)

\textbf{Movimento} 9 m, scavo 6 m

\textbf{Tiri Salvezza}: Tempra +11, Riflessi +7, Volontà +4

\textbf{Immunità ai Danni} freddo, fuoco

\textbf{Sensi} scurovisione 18 m, senso tellurico 18 m

\textbf{Linguaggi} -

\textbf{Sfida} 11 (7.200 PE)

\emph{\textbf{Corpo Riscaldato.}} Una creatura che entri a contatto con il remorhaz o lo colpisca con un attacco da mischia mentre si trova entro 1 metro da esso, subisce 10 (3d6) danni da fuoco.

\textbf{Azioni}

\emph{\textbf{Morso.} Attacco in mischia con arma}: +11 a colpire, portata 3 m, un bersaglio.

\emph{Colpisce:} 40 (6d10 + 7) danni perforanti più 10 (3d6) danni da fuoco. Se il bersaglio è una creatura, è afferrato (CD 17 per fuggire). Fino al termine dell'afferrare, il bersaglio è intralciato, e il remorhaz non può attaccare con il morso un altro bersaglio.

\emph{\textbf{Inghiottire.}} Il remorhaz effettua una attacco di morso contro un bersaglio di taglia Media o inferiore che sta afferrando. Se l'attacco colpisce, la creatura subisce il danno da morso ed è inghiottita, e l'afferrare ha termine. Il bersaglio inghiottito è accecato e intralciato, ha copertura totale contro gli attacchi e altri effetti all'esterno del remorhaz, e subisce 21 (6d6) danni da acido all'inizio di ciascun turno del remorhaz.

Se il remorhaz subisce 30 o più danni in un singolo turno da una creatura al suo interno, il remorhaz deve riuscire un Tiro Salvezza su Tempra CD 15 al termine di quel turno o vomitare tutte le creature inghiottite, che cadono prone in uno spazio entro 3 metri dal remorhaz. Se il remorhaz muore, una creatura inghiottita non   più intralciata da esso e può uscire dal cadavere utilizzando 5 metri di movimento, uscendo prona.

\textbf{Ecologia}\\
Ambiente: Deserti Freddi e Ghiacciai\\
Organizzazione: Solitario\\
Tesoro: Nessuno\\
\textbf{Descrizione}\\
In un mondo di ghiaccio e neve, i remorhaz sono particolarmente temuti per il terribile fuoco che brucia dentro i loro corpi. Questo fuoco interiore fa sì che le piastre lungo il suo dorso divengano roventi quando la creatura è particolarmente arrabbiata, eccitata o nel panico. Le creature che si sono adattate alle regioni artiche spesso sono particolarmente vulnerabili al fuoco, il che rende la principale difesa del remorhaz incredibilmente potente e gli assicura il ruolo di pericoloso predatore delle zone ghiacciate. I remorhaz vivono in estesi labirinti scavati nel cuore dei ghiacciai. Queste bestie usano il loro calore per scavare tunnel nel ghiaccio, tunnel le cui lisce pareti vitree si ricongelano rapidamente lungo la loro scia creando numerosi dedali incredibilmente stabili.\\

Anche se il remorhaz ha molto in comune con i più piccoli parassiti di superficie, questa bestia è sorprendentemente intelligente. Sebbene incapace di palare, il tipico remorhaz capisce bene il Gigante, e spesso le tribù di giganti ne approfittano per stipulare alleanze con questi bestioni. I Giganti del Gelo sono particolarmente ossessionati da essi; questi giganti affrontano le crudeli e letali bruciature che un remorhaz può infliggere per diventare "amici del verme" ottenendo una potente arma da usare contro i loro nemici, un assassino capace di scavare attraverso il pavimento delle fortificazioni glaciali per colpire direttamente con la maggiore debolezza di un gigante del gelo: il fuoco. Altri giganti usano queste bestie come forge viventi, poiché il loro dorso è abbastanza caldo da sciogliere il metallo.\\
Un remorhaz è lungo 7 metri e pesa 5.000 kg.\\



\medskip\index{Mostri - Rugginofago}\textbf{Rugginofago}

\emph{Media Mostruosità, disallineato}

\textbf{FORZA} +1

\textbf{DESTREZZA} +1

\textbf{COSTITUZIONE} +1

\textbf{INTELLIGENZA} -4

\textbf{SAGGEZZA} +1

\textbf{CARISMA} -2

\textbf{Iniziativa} +1 -- \textbf{Difesa} 15

\textbf{Punti Ferita} 27 (5d8 + 5)

\textbf{Movimento} 12 m

\textbf{Tiri Salvezza}: Tempra +2, Riflessi +4, Volontà +5

\textbf{Sensi} scurovisione 18 m

\textbf{Linguaggi} -

\textbf{Sfida} 1/2 (100 PE)

\emph{\textbf{Fiuto del Ferro.}} Il rugginofago può individuare, con l'olfatto, l'esatta posizione di metalli ferrosi entro 9 metri.

\emph{\textbf{Arrugginire Metallo.}} Qualsiasi arma non magica fatta di metallo che colpisca il rugginofago si corrode. Dopo aver inflitto il danno, l'arma subisce una penalità permanente e cumulativa di - 1 ai tiri di danno. Se la penalità scende fino a -5, l'arma è distrutta. Le munizioni non magiche fatte di metallo e che colpiscono il rugginofago, sono considerate distrutte dopo aver inflitto il danno.

\textbf{Azioni}

\emph{\textbf{Morso.} Attacco con arma da mischia}: +3 a colpire, portata 1 m, un bersaglio.

\emph{Colpisce:} 5 (1d8 + 1) danni perforanti.

\emph{\textbf{Antenne.}} Il rugginofago corrode gli oggetti di metallo ferroso non magici che può vedere e si trovano entro 1 metro. Se l'oggetto non è indossato o trasportato, il contatto col rugginofago ne distrugge un cubo di 30 centimetri di spigolo. Se l'oggetto è indossato o trasportato da una creatura, la creatura può effettuare un Tiro Salvezza su Riflessi CD 11 per evitare il contatto con il rugginofago.

Se l'oggetto con cui entra in contatto è un'armatura o scudo di metallo indossati o trasportati, questi subiscono una penalità permanente e cumulativa di -1 alla Difesa che forniscono. Le armature ridotte a Difesa 0 o gli scudi che scendono ad un bonus di +0 sono distrutti. Se l'oggetto con cuientra in contatto è un'arma di metallo impugnata da qualcuno, la  arrugginisce come descritto nel tratto Arrugginire Metallo.

\textbf{Ecologia}
Ambiente: Qualsiasi Sotterraneo\\
Organizzazione: Solitario, coppia o nido (3-10)\\
Tesoro: Accidentale (nessun tesoro di metallo)\\
\textbf{Descrizione}\\
Di tutte le bestie terrificanti che un esploratore può incontrare nel sottosuolo, solo il rugginofago ha come obbiettivo quello cui l'avventuriero medio da più valore: il suo tesoro.\\
Lungo in genere 1 metro e dal peso di almeno 100 kg, il rugginofago somiglia ad un crostaceo e sarebbe già abbastanza spaventoso anche senza l'alieno processo nutritivo da cui prende il nome. I rugginofagi mangiano gli oggetti di metallo, preferendo quelli di ferro e di leghe ferrose come l'acciaio ma divorano anche mithral, Adamantio e metalli incantati con uguale facilità. Qualsiasi metallo toccato dalle delicate antenne del rugginofago o dalla sua pelle corazzata si corrode e si riduce in polvere in pochi secondi, facendone una delle bestie più temute dagli avventurieri sotterranei e dai Nani minatori che devono difendere le loro forge e competere con loro per l'oro.\\
Anche se i rugginofagi non hanno una tendenza innata per la violenza, la loro fame insaziabile li spinge a caricare qualunque cosa gli si avvicini con addosso abbastanza metallo, e qualsiasi resistenza viene accolta con inaspettata ferocia. Non è insolito che i rugginofagi in zone povere di metallo seguano le vittime in fuga per giorni usando la loro capacità di fiutare metalli, purché queste abbiano ancora oggetti di metallo intatti.\\
Fortunatamente, spesso è possibile sfuggire alle attenzioni di un rugginofago lanciandogli un oggetto di metallo denso, come uno scudo, e correndo nella direzione opposta. Quanti frequentano le aree infestate dai rugginofagi imparano velocemente a tenere a portata di mano armi di legno o pietra.\\



\medskip\index{Mostri - Sahuagin}\textbf{Sahuagin}

\emph{Media umanoide (sahuagin), legale malvagio}

\textbf{FORZA} +1

\textbf{DESTREZZA} +0

\textbf{COSTITUZIONE} +1

\textbf{INTELLIGENZA} +1

\textbf{SAGGEZZA} +1

\textbf{CARISMA} -1

\textbf{Iniziativa} +1 -- \textbf{Difesa} 13

\textbf{Punti Ferita} 22 (4d8 + 4)

\textbf{Movimento} 9 m, nuoto 12 m

\textbf{Tiri Salvezza}: Tempra +4, Riflessi +4, Volontà +4

\textbf{Competenze} Consapevolezza +5

\textbf{Sensi} scurovisione 36 m

\textbf{Linguaggi} Sahuagin

\textbf{Sfida} 1/2 (100 PE)

\emph{\textbf{Anfibio Limitato.}} Il sahuagin può respirare aria e acqua, ma deve restare sommerso almeno una volta ogni 4 ore per evitare di soffocare.

\emph{\textbf{Frenesia Sanguinaria.}} Il sahuagin ha +1d6 ai tiri per colpire in mischia contro qualsiasi creatura che non sia al massimo dei suoi punti ferita.

\emph{\textbf{Telepatia con gli Squali}}. Il sahuagin può comandare magicamente qualsiasi squalo entro 36 metri da sé, usando una forma limitata di telepatia. 

\textbf{Azioni}

\emph{\textbf{Multiattacco.}} Il sahuagin può effettuare due attacchi da mischia:  uno con il morso e uno con gli artigli o la lancia.

\emph{\textbf{Artigli.} Attacco con arma da mischia}: +3 a colpire, portata 1 m, un bersaglio.

\emph{Colpisce:} 3 (1d4 + 1) danni taglienti.

\emph{\textbf{Lancia.} Attacco con arma da mischia o a Distanza}: +3 a colpire, portata 1 m o gittata 6m, un bersaglio.

\emph{Colpisce:} 4 (1d6 + 1) danni perforanti, o 5 (1d8 + 1) danni perforanti se usata con due mani per effettuare un attacco da mischia.

\emph{\textbf{Morso.} Attacco con arma da mischia}: +3 a colpire, portata 1 m, un bersaglio.

\emph{Colpisce:} 3 (1d4 + 1) danni perforanti.

\textbf{Ecologia}\\
Ambiente: Oceani Temperati o Caldi\\
Organizzazione: Solitario, coppia, squadra (5-8), pattuglia (11-20 più 1 tenente di 3° livello e 1-2 Squali), banda (20-80 più 100\% non combattenti, 1 tenente di 3° livello e 1 capitano di 4° livello ogni 20 adulti, e 1-2 Squali) o tribù (70-160 più 100\% non combattenti, 1 tenente di 3° livello ogni 20 adulti, 1 capitano di 4° livello ogni 40 adulti, 9 guardie di 4° livello, 1-4 novizie di 3°-6° livello, 1 sacerdotessa di 7° livello, 1 barone di 6°-8° livello, e 5-8 Squali)
Tesoro: Equipaggiamento da PNG (Tridente, Balestra Pesante con 10 Quadrelli, altro tesoro)\\
\textbf{Descrizione}\\
Famelici e crudeli, i sahuagin sono, sfortunatamente, tra le razze oceaniche più prosperose. Grandi città sono state costruite da questa razza nelle buie profondità delle fosse oceaniche, e alcune fortezze sorgono nei pressi delle coste da dove lanciano assalti continui contro i nemici che respirano aria che vivono vicino alla riva. Orgogliosi e bellicosi, i sahuagin si alleano raramente con altri, e vedono le altre razze acquatiche, come aboleth, marinidi e simili come concorrenti. Le sole creature che sembrano rispettare oltre ai loro simili sono gli squali; in questi implacabili predatori, infatti, i sahuagin rivedono molto di loro stessi. Un sahuagin è alto 2,1 metri e pesa circa 125 kg.\\

I sahuagin sono soggetti a mutazioni genetiche, e quando nasce un mutante assurge quasi sempre ai ranghi nobiliari o di comando nella società. La mutazione sahuagin più comune consiste in un paio di braccia extra (che concedono due attacchi addizionali con gli artigli o la possibilità di maneggiare più armi). Alcuni parlano dei rari malenti, sahuagin che non sembrano uomini squalo ma elfi acquatici, malgrado condividano la sete di sangue e la natura crudele dei loro simili. I malenti spesso servono come spie o assassini i governanti sahuagin, ma si narra di intere tribù composte di malenti in remote zone del mare.\\


\medskip\index{Mostri - Salamandra}\textbf{Salamandra}

\emph{Grande elementale, neutrale malvagio}

\textbf{FORZA} +4

\textbf{DESTREZZA} +2

\textbf{COSTITUZIONE} +2

\textbf{INTELLIGENZA} +0

\textbf{SAGGEZZA} +0

\textbf{CARISMA} +1

\textbf{Iniziativa} +2 -- \textbf{Difesa} 18

\textbf{Punti Ferita} 90 (12d10 + 24)

\textbf{Movimento} 9 m

\textbf{Tiri Salvezza}: Tempra +10, Riflessi +7, Volontà +6

\textbf{Vulnerabilità al Danno} freddo

\textbf{Resistenze al Danno} da botta, perforante e tagliente di attacchi non magici

\textbf{Immunità ai Danni} fuoco

\textbf{Sensi} scurovisione 18 m

\textbf{Linguaggi} Ignan

\textbf{Sfida} 5 (1.800 PE)

\emph{\textbf{Armi Riscaldate.}} Qualsiasi arma da mischia metallica che la salamandra impugni infligge 3 (1d6) danni da fuoco aggiuntivi per colpo (già incluso nell'attacco).

\emph{\textbf{Corpo Riscaldato.}} Una creatura che entri a contatto con la salamandra o la colpisce con un attacco da mischia mentre si trova entro 1 metro da essa subisce 7 (2d6) danni da fuoco.

\textbf{Azioni}

\emph{\textbf{Multiattacco.}} La salamandra effettua due attacchi: uno con la lancia e uno con la coda.

\emph{\textbf{Coda.} Attacco con arma da mischia}: +7 a colpire, portata 3 m, un bersaglio.

\emph{Colpisce:} 11 (2d6 + 4) danni da botta più 7 (2d6) danni da fuoco, e il bersaglio è afferrato (CD 14 per fuggire). Fino al termine dell'afferrare, il bersaglio è intralciato, la salamandra può colpire automaticamente il bersaglio con la coda, e la salamandra non può effettuare attacchi di coda contro altri bersagli.

\emph{\textbf{Lancia.} Attacco con arma da mischia o a Distanza}: +7 a colpire, portata 1 m, gittata 6m, un bersaglio.

\emph{Colpisce:} 11 (2d6 + 4) danni perforanti, o 13 (2d8 +4) danni perforanti se usata con due mani per effettuare un attacco da mischia, più 3 (1d6) danni da fuoco.

\textbf{Ecologia}
Ambiente: Qualsiasi (Piano del Fuoco)\\
Organizzazione: Solitario, coppia o gruppo (3-5)\\
Tesoro: Standard (Lancia, altro tesoro ininfiammabile)\\
\textbf{Descrizione}\\
Le Salamandre sono native del Piano del Fuoco, dove le loro legioni di fieri combattenti sono molto temute dagli altri abitanti del Piano. Poiché molte delle più forti Razze Elementali del Fuoco Schiavizzano le Salamandre per la loro Abilità nella metallurgia e capacità combattiva, le Salamandre odiano gli Efreet e gli altri con fervore.\\

Anche se i loro nascondigli superano i 250 gradi C di temperatura, le Salamandre possono tollerare temperature più basse. Generalmente lo fanno se costrette, e sono anche più burbere e irascibili del normale in questi ambienti. Sebbene provenga dal Piano del Fuoco, la Razza delle Salamandre si identifica di più con l'Abisso, e ha un grande rispetto per i Demoni (in particolare quelli associati col fuoco, come i Balor e certi Signori dei Demoni legati alle fiamme). Per questo non è insolito incontrare un grosso gruppo di Salamandre nell'Abisso.\\

Le Salamandre sono spesso evocate nel Piano Materiale per servire come guardiani o, più comunemente, come fabbricanti di Armature, Armi e altri oggetti metallurgici, dato che la loro Abilità in questo campo è leggendaria. Le Salamandre infestano anche quelle aree del Piano Materiale dove il confine tra questo mondo e il Piano del Fuoco si è fatto labile, come vicino e dentro i Vulcani.\\

Abitando zone così estreme, le Salamandre posseggono solo tesori che resistono alle alte temperature, come Spade, Armature, gioielli, Verghe e altri oggetti che hanno un alto punto di fusione. La società delle Salamandre è crudele e basata sul potere e sulla capacità di soggiogare chi è inferiore a loro. Gli esseri inferiori alle Salamandre che causano problemi affrontano una morte lenta e dolorosa.\\



\medskip\index{Mostri - Satiro}\textbf{Satiro}

\emph{Media fatato, caotico neutrale}

\textbf{FORZA} +1

\textbf{DESTREZZA} +3

\textbf{COSTITUZIONE} +0

\textbf{INTELLIGENZA} +1

\textbf{SAGGEZZA} +0

\textbf{CARISMA} +2

\textbf{Iniziativa} +3 -- \textbf{Difesa} 15 (armatura di cuoio)

\textbf{Punti Ferita} 31 (7d8)

\textbf{Vulnerabilità al Danno} ferro freddo

\textbf{Movimento} 12 m

\textbf{Tiri Salvezza}: Tempra +4, Riflessi +8, Volontà +8

\textbf{Competenze} Muoversi Silenziosamente / Nascondersi +5, Intrattenere +6, Consapevolezza +2

\textbf{Linguaggi} Comune, Elfico, Silvano

\textbf{Sfida} 1/2 (100 PE)

\emph{\textbf{Resistenza alla Magia.}} Il satiro ha +1d6 ai Tiri Salvezza contro incantesimi e altri effetti magici.

\textbf{Azioni}

\emph{\textbf{Incornata.} Attacco con arma da mischia}: +3 a colpire, portata 1 m, un bersaglio.

\emph{Colpisce:} 6 (2d4 + 1) danni da botta.

\emph{\textbf{Spada Corta.} Attacco con arma da mischia}: +5 a colpire, portata 1 m, un bersaglio.

\emph{Colpisce:} 6 (1d6 + 3) danni perforanti.

\emph{\textbf{Arco Corto.} Attacco con arma a Distanza}: +5 a colpire, gittata 24m, un bersaglio.

\emph{Colpisce:} 6 (1d6 + 3) danni perforanti.

\textbf{Ecologia}\\
Ambiente: Foreste Temperate\\
Organizzazione: Solitario, coppia, banda (3-6) o festino (7-11)\\
Tesoro: Standard (Pugnale, Arco Corto più 20 Frecce, flauto di pan perfetto, altro tesoro)\\
\textbf{Descrizione}\\
I satiri, conosciuti in molte regioni come fauni, sono creature debosciate ed edoniste delle parti più profonde e primordiali delle foreste. Adorano il vino, la musica e i piaceri della carne, sono rinomati come libertini e bellimbusti che corteggiano le fanciulle sprovvedute e i pastorelli e si lasciano dietro una scia di spiegazioni imbarazzanti e gravidanze indesiderate.\\

Anche se i loro corpi sono quasi sempre quelli di uomini attraenti e ben proporzionati, le capacità seduttive dei satiri risiedono nel loro talento musicale. Con l'aiuto del suo flauto, un satiro è capace di tessere una vasta gamma di incantesimi melodici ideati per affascinare gli altri e farli accondiscendere ai suoi capricciosi desideri.\\

Oltre ad amoreggiare costantemente, i satiri spesso fungono da guardiani delle loro foreste, e quanti riescono a trasformare la lussuria del fauno in ira probabilmente si troveranno di fronte i più pericolosi tra gli animali che circondano il fauno. Inoltre, anche se i satiri tendono a mettere il loro divertimento al di sopra dei diritti altrui, non covano alcun risentimento contro quelli che seducono.\\

I bambini nati da questi incontri sono sempre satiri di sangue puro e vengono generalmente portati via dai loro sfrenati padri subito dopo la nascita.\\


\medskip\index{Mostri - Scheletro}\textbf{Scheletro}

\emph{Media non morto, legale malvagio}

\textbf{FORZA} +0

\textbf{DESTREZZA} +2

\textbf{COSTITUZIONE} +2

\textbf{INTELLIGENZA} -2

\textbf{SAGGEZZA} -1

\textbf{CARISMA} -3

\textbf{Iniziativa} +2 -- \textbf{Difesa} 14 (pezzi di armatura)

\textbf{Punti Ferita} 13 (2d8 + 4)

\textbf{Movimento} 9 m

\textbf{Tiri Salvezza}: Tempra +0, Riflessi +2, Volontà +2

\textbf{Vulnerabilità al Danno} da botta

\textbf{Resistenze al Danno} perforante e tagliente di attacchi non magici

\textbf{Immunità al Danno} veleno

\textbf{Immunità alle Condizioni} avvelenato, affaticamento

\textbf{Sensi} scurovisione 18 m

\textbf{Linguaggi} comprende tutte le lingue che parlava in vita ma non può parlare

\textbf{Sfida} 1/4 (50 PE)

\emph{\textbf{Natura Non Morta.}} Lo scheletro non necessita aria, cibo, bevande o sonno.

\textbf{Azioni}

\emph{\textbf{Spada Corta.} Attacco con arma da mischia}: +4 a colpire, portata 1 m, un bersaglio.

\emph{Colpisce:} 5 (1d6 + 2) danni perforanti.

\emph{\textbf{Arco Corto.} Attacco con arma a Distanza}: +4 a colpire, gittata 24m, un bersaglio.

\emph{Colpisce:} 5 (1d6 + 2) danni perforanti.

\textbf{Ecologia}\\
Ambiente: Qualsiasi\\
Organizzazione: Qualsiasi\\
Tesoro: Nessuno (Giaco di Maglia Rotto, Scimitarra Rotta)\\
\textbf{Descrizione}\\
Gli scheletri sono ossa di morti animate, portate alla non vita da magie sacrileghe. Per la maggior parte, gli scheletri sono automi privi di volontà, ma possiedono un'astuzia malvagia concessa loro dalla forza che li anima: un'astuzia che permette loro di portare armi ed indossare armature.\\


\medskip\index{Mostri - Scheletro di Cavallo da Guerra}\textbf{Scheletro di Cavallo da Guerra}

\emph{Grande non morto, legale malvagio}

\textbf{FORZA} +4

\textbf{DESTREZZA} +1

\textbf{COSTITUZIONE} +2

\textbf{INTELLIGENZA} -4

\textbf{SAGGEZZA} -1

\textbf{CARISMA} -3

\textbf{Iniziativa} +1 -- \textbf{Difesa} 14 (pezzi di bardatura)

\textbf{Punti Ferita} 22 (3d10 + 6)

\textbf{Movimento} 18 m

\textbf{Tiri Salvezza}: Tempra +4, Riflessi +3, Volontà +1

\textbf{Vulnerabilità al Danno} da botta

\textbf{Resistenze al Danno} perforante e tagliente di attacchi non magici

\textbf{Immunità al Danno} veleno

\textbf{Immunità alle Condizioni} avvelenato, affaticamento

\textbf{Sensi} scurovisione 18 m 

\textbf{Linguaggi} -

\textbf{Sfida} 1/2 (100 PE)

\emph{\textbf{Natura Non Morta.}} Lo scheletro non necessita aria, cibo, bevande o sonno.

\textbf{Azioni}

\emph{\textbf{Zoccoli.} Attacco con arma da mischia}: +6 a colpire, portata 1 m, un bersaglio.

\emph{Colpisce:} 11 (2d6 + 4) danni da botta.

\medskip\index{Mostri - Scheletro di Minotauro}\textbf{Scheletro di Minotauro}

\emph{Grande non morto, legale malvagio}

\textbf{FORZA} +4

\textbf{DESTREZZA} +0

\textbf{COSTITUZIONE} +2

\textbf{INTELLIGENZA} -2

\textbf{SAGGEZZA} -1

\textbf{CARISMA} -3

\textbf{Iniziativa} +0 -- \textbf{Difesa} 13

\textbf{Punti Ferita} 67 (9d10 + 18)

\textbf{Movimento} 12 m

\textbf{Tiri Salvezza}: Tempra +6, Riflessi +3, Volontà +2

\textbf{Vulnerabilità al Danno} da botta

\textbf{Immunità al Danno} veleno

\textbf{Resistenze al Danno} perforante e tagliente di attacchi non magici

\textbf{Immunità alle Condizioni} avvelenato, affaticamento

\textbf{Sensi} scurovisione 18 m

\textbf{Linguaggi} comprende l'Abissale ma non può parlare

\textbf{Sfida} 2 (450 PE)

\emph{\textbf{Carica.}} Se lo scheletro di minotauro si muove di almeno 3 metri in linea retta verso il bersaglio e poi lo colpisce con un attacco di incornata durante lo stesso turno, il bersaglio subisce 9 (2d8) danni perforanti aggiuntivi. Se il bersaglio è una creatura, deve riuscire un Tiro Salvezza di Tempra CD 14 o venire spinto di 3 metri indietro e cadere prono.

\emph{\textbf{Natura Non Morta.}} Lo scheletro non necessita aria, cibo, bevande o sonno.

\textbf{Azioni}

\emph{\textbf{Ascia Bipenne.} Attacco con arma da mischia}: +6 a colpire, portata 1 m, un bersaglio.

\emph{Colpisce:} 17 (2d12 + 4) danni taglienti.

\emph{\textbf{Incornata.} Attacco con arma da mischia}: +6 a colpire, portata 1 m, un bersaglio.

\emph{Colpisce:} 13 (2d8 + 4) danni perforanti.

\medskip\index{Mostri - Segugio Infernale}\textbf{Segugio Infernale}

\emph{Media immondo, legale malvagio}

\textbf{FORZA} +3

\textbf{DESTREZZA} +1

\textbf{COSTITUZIONE} +2

\textbf{INTELLIGENZA} -2

\textbf{SAGGEZZA} +1

\textbf{CARISMA} -2

\textbf{Iniziativa} +1 -- \textbf{Difesa} 17

\textbf{Punti Ferita} 45 (7d8 + 14)

\textbf{Movimento} 15 m

\textbf{Tiri Salvezza}: Tempra +6, Riflessi +5, Volontà +1

\textbf{Competenze} Consapevolezza +5

\textbf{Immunità al Danno} fuoco

\textbf{Sensi} scurovisione 18 m

\textbf{Linguaggi} comprende l'Infernale ma non può parlare

\textbf{Sfida} 3 (700 PE)

\emph{\textbf{Udito e Olfatto Affinato.}} Il segugio ha +1d6 nelle prove di Saggezza (Consapevolezza) basate su udito od olfatto.

\emph{\textbf{Tattiche di Branco.}} Il segugio ha +1d6 ai tiri per colpire contro una creatura se almeno uno degli alleati del segugio si trova entro 1 metro dalla creatura e quell'alleato non è inabile.

\textbf{Azioni}

\emph{\textbf{Morso.} Attacco con arma da mischia}: +5 a colpire,
portata 1 m, un bersaglio.

\emph{Colpisce:} 7 (1d6 + 3) danni perforanti più 7 (2d6) danni da
fuoco.

\emph{\textbf{Soffio Infuocato (Ricarica 5-6).}} Il segugio esala fuoco in un cono di 5 metri. Ogni creatura in quell'area deve effettuare un Tiro Salvezza di Riflessi CD 12, e subire 21 (6d6) danni da fuoco se fallisce il Tiro Salvezza, o la metà di questi danni se lo riesce.



\subsection{Sfingi}

\medskip\index{Mostri - Androsfinge}\textbf{Androsfinge}

\emph{Grande mostruosità, legale neutrale}

\textbf{FORZA} +6

\textbf{DESTREZZA} +0

\textbf{COSTITUZIONE} +5

\textbf{INTELLIGENZA} +3

\textbf{SAGGEZZA} +4

\textbf{CARISMA} +6

\textbf{Iniziativa} +3 -- \textbf{Difesa} 26

\textbf{Punti Ferita} 199 (19d10 + 95)

\textbf{Movimento} 12 m, volo 18 m

\textbf{Tiri Salvezza}: Tempra +12, Riflessi +8, Volontà +7

\textbf{Competenze} Arcano +9, Consapevolezza +10, Religione +15

\textbf{Immunità al Danno} psichico; da botta, perforante e tagliente di attacchi non magici

\textbf{Immunità alle Condizioni} affascinato, spaventato 

\textbf{Sensi} visione del vero 36 m 

\textbf{Linguaggi} Comune, Sfinge

\textbf{Sfida} 17 (18.000 PE)

\emph{\textbf{Armi Magiche.}} Gli attacchi con armi della sfinge sono magici.

\emph{\textbf{Imperscrutabile.}} La sfinge è immune a qualsiasi effetto in grado di percepirne le emozioni o leggerne i pensieri, oltre che a qualsiasi incantesimo di divinazione che rifiuti. Le prove di Saggezza (Percepire Emozioni) per discernere le intenzioni o la sincerità della sfinge hanno -1d6.

\emph{\textbf{Incantesimi.}} La sfinge ha CM 12.
La sua caratteristica da incantatore è la Saggezza (CD del Tiro Salvezza degli incantesimi 18, +10 a colpire con attacchi con incantesimo). Non ha bisogno di componenti materiali per lanciare i suoi incantesimi. La sfinge tiene preparati i seguenti incantesimi:

Trucchetti (a volontà): \emph{fiamma sacra, salvare i morenti,} \emph{taumaturgia}

Difficoltà 16 (4 slot): \emph{comando, individuazione del magico,} \emph{individuare male e bene}

Difficoltà 19 (3 slot): \emph{ristorare inferiore, zona di verità}

Difficoltà 21 (3 slot): \emph{dissolvi magie, linguaggi}

Difficoltà 23 (3 slot): \emph{esilio, libertà di movimento}

Difficoltà 26 (2 slot): \emph{colpo infuocato, ristorare superiore}

Difficoltà 29 (1 slot): \emph{banchetto degli eroi}

\textbf{Azioni}

\emph{\textbf{Multiattacco.}} La sfinge può effettuare due attacchi di artiglio.

\emph{\textbf{Artiglio.} Attacco con arma da mischia}: +12 a colpire, portata 1 m, un bersaglio.

\emph{Colpisce:} 17 (2d6 + 10) danni taglienti.

\emph{\textbf{Ruggito (3/Giorno).}} La sfinge emette un ruggito magico. Ogni volta che ruggisce prima di una nuova alba, il ruggito più forte e l'effetto è diverso, come dettagliato di seguito. Ogni creatura entro 150 metri dalla sfinge e capace di udirne il ruggito deve effettuare un Tiro Salvezza.

\textbf{Primo Ruggito.} Ogni creatura che fallisce un Tiro Salvezza su Volontà CD 18 resta spaventata per 1 minuto. Una creatura spaventata può ripetere il Tiro Salvezza al termine di ciascun suo turno, terminandone l'effetto per sé, se lo riesce.

\textbf{Secondo Ruggito.} Ogni creatura che fallisce un Tiro Salvezza su Volontà CD 18 resta assordata e spaventata per 1 minuto. Una creatura spaventata è paralizzata e può ripetere il Tiro Salvezza al termine di ciascun suo turno, terminandone l'effetto per sé, se lo riesce.

\textbf{Terzo Ruggito.} Ogni creatura effettua un Tiro Salvezza su Tempra CD 18. Chi fallisce il Tiro Salvezza subisce 44 (8d10) danni da tuono ed è gettato prono. Se il Tiro Salvezza riesce, la creatura subisce la metà di questi danni e non viene gettata prona. 

\textbf{Azioni Aggiuntive}

La sfinge può effettuare 3 Azioni aggiuntive, scelte tra le opzioni seguenti. Può usare solo un'opzione leggendaria alla volta e solo al termine del turno di un'altra creatura. La sfinge recupera le Azioni aggiuntive spese all'inizio del proprio turno.

\textbf{Attacco di Artiglio.} La sfinge effettua un attacco di artiglio. 

\textbf{Eseguire un Incantesimo (Costa 3 Azioni).} La sfinge lancia un incantesimo dalla lista degli incantesimi preparati, utilizzando uno slot incantesimo come di norma. 

\textbf{Teletrasporto (Costa 2 Azioni).} La sfinge si teletrasporta magicamente, insieme a tutto l'equipaggiamento che sta indossando o trasportando, in uno spazio non occupato che possa vedere, fino a 36 metri di distanza.


\textbf{Ecologia}\\
Ambiente: Colline o Deserti Caldi\\
Organizzazione: Solitario\\
Tesoro: Standard\\
\textbf{Descrizione}\\
Le androsfingi, le più potenti tra le sfingi comuni, ritengono di rappresentare tutto ciò che c'è di degno e nobile nella loro specie e si atteggiano come se il peso del mondo intero poggiasse sul loro buon esempio. Guardano le Criosfingi con sufficienza paternalistica, le Ieracosfingi con malcelato disgusto e le Ginosfingi come le uniche altre sfingi degne del loro tempo.\\

Le androsfingi ostentano una facciata scorbutica e astiosa nei confronti degli stranieri. Non si sforzano in alcun modo di celare il loro fastidio quando sono irritate. Tendono inoltre a essere gelose del loro territorio, anche se meno delle altre sfingi. Quasi inevitabilmente lanciano avvertimenti e proclami roboanti prima di attaccare, e rispettano quasi sempre un appello a trattare. Le androsfingi barattano informazioni e conversazioni, e non tesori, in cambio di un passaggio sicuro.\\
Le androsfingi sono alte 3,6 metri e pesano 500 kg.\\


\medskip\index{Mostri - Ginosfinge}\textbf{Ginosfinge}

\emph{Grande mostruosità, legale neutrale}

\textbf{FORZA} +4

\textbf{DESTREZZA} +2

\textbf{COSTITUZIONE} +3

\textbf{INTELLIGENZA} +4

\textbf{SAGGEZZA} +4

\textbf{CARISMA} +4

\textbf{Iniziativa} +4 -- \textbf{Difesa} 23

\textbf{Punti Ferita} 136 (16d10 + 48)

\textbf{Movimento} 12 m, volo 18 m

\textbf{Tiri Salvezza}: Tempra +11, Riflessi +9, Volontà +10

\textbf{Competenze} Arcano +14, Consapevolezza +9, Religione +9, Storia +14

\textbf{Resistenze al Danno} da botta, perforante e tagliente di attacchi non magici

\textbf{Immunità al Danno} psichico

\textbf{Immunità alle Condizioni} affascinato, spaventato 

\textbf{Sensi} visione del vero 36 m

\textbf{Linguaggi} Comune, Sfinge

\textbf{Sfida} 11 (7.200 PE)

\emph{\textbf{Armi Magiche.}} Gli attacchi con armi della sfinge sono magici.

\emph{\textbf{Imperscrutabile.}} La sfinge è immune a qualsiasi effetto in grado di percepirne le emozioni o leggerne i pensieri, oltre che a qualsiasi incantesimo di divinazione che rifiuti. Le prove di Saggezza (Percepire Inganni) per discernere le intenzioni o la sincerità della sfinge hanno -1d6.

\emph{\textbf{Incantesimi.}} La sfinge ha CM 9. La sua abilità da incantatore è l'Intelligenza (CD del Tiro Salvezza degli incantesimi 17, +9 a colpire con attacchi da incantesimo). Non ha bisogno di componenti materiali per eseguire i suoi incantesimi. La sfinge tiene preparati i seguenti incantesimi: Trucchetti (a volontà): \emph{illusione minore, mano magica,} \emph{prestidigitazione}

Difficoltà 16 (4 slot): \emph{identificare, individuazione del magico, scudo}

Difficoltà 19 (3 slot): \emph{localizza oggetto, oscurità, suggestione} 

Difficoltà 21 (3 slot): \emph{dissolvi magie, linguaggi, rimuovi maledizione}

Difficoltà 23 (3 slot): \emph{esilio, invisibilità superiore}

Difficoltà 26 (2 slot): \emph{conoscenza delle leggende}

\textbf{Azioni}

\emph{\textbf{Multiattacco.}} La sfinge può effettuare due attacchi di artiglio.

\emph{\textbf{Artiglio.} Attacco con arma da mischia}: +9 a colpire, portata 1 m, un bersaglio.

\emph{Colpisce:} 13 (2d8 + 4) danni taglienti.

\textbf{Azioni Aggiuntive}

La sfinge può effettuare 3 Azioni aggiuntive, scelte tra le opzioni seguenti. Può usare solo un'opzione leggendaria alla volta e solo al termine del turno di un'altra creatura. La sfinge recupera le Azioni aggiuntive spese all'inizio del proprio turno.

\textbf{Attacco di Artiglio.} La sfinge effettua un attacco di artiglio. 

\textbf{Eseguire un Incantesimo (Costa 3 Azioni).} La sfinge esegue un incantesimo dalla lista degli incantesimi preparati, utilizzando uno slot incantesimo come di norma.

\textbf{Teletrasporto (Costa 2 Azioni).} La sfinge si teletrasporta magicamente, insieme a tutto l'equipaggiamento che sta indossando o trasportando, in uno spazio non occupato che possa vedere, fino a 36 metri di distanza.

\textbf{Ecologia}
Ambiente: Deserti e colline caldi\\
Organizzazione: Solitario, coppia o culto (3-6)\\
Tesoro: Doppio\\
\textbf{Descrizione}\\
Anche se esistono diversi tipi di sfinge, quella alla quale gli studiosi si riferiscono come Ginosfinge (un nome che molte sfingi trovano offensivo) è una creatura saggia e maestosa ma al contempo terrificante se arrabbiata. Meno moraliste delle loro controparti maschili (le Androsfingi, creature totalmente differenti da quella presentata qui), le sfingi sono prudenti e metodiche quando prendono delle decisioni, e sono orgogliose della loro fredda logica e della loro imparzialità. Hanno poca pazienza con le varianti inferiori di sfingi, considerandole poco più che animali. Le sfingi amano gli enigmi e gli indovinelli complicati, e fanno tesoro di fatti insoliti e dilemmi arcani molto più che di oro o gemme.\\

Pur non essendo grandi studiose in senso tradizionale, il grande apprezzamento delle sfingi per gli enigmi le porta a compiere ricerche in una grande varietà di materie, rendendole spesso una preziosa fonte di informazioni, specialmente quando fanno uso delle loro capacità magiche. Di solito sono felici di avere contatti con altre razze, ed offrono regolarmente beni materiali in cambio di informazioni o di indovinelli nuovi ed interessanti. Sono eccellenti guardiane di templi, tombe ed altri luoghi importanti, fintanto che vengono intrattenute in maniera adeguata. Le sfingi danno grande importanza alla gentilezza, ma possono essere capricciose: possono decidere altruisticamente di dividere i loro ultimi enigmi con dei viaggiatori ma non ci pensano due volte a divorarli se non vi prestano abbastanza attenzione o non forniscono alcun indizio utile alla loro risoluzione.\\

Una tipica sfinge è lunga 3 metri e pesa circa 400 kg. Anche se le loro ali possono tenerle in aria per lunghi periodi di tempo, sono delle volatrici scarse, e preferiscono atterrare prima di iniziare a combattere, attaccando con i loro poderosi artigli. Nonostante siano estremamente territoriali, le sfingi tendono ad avvisare gli intrusi varie volte prima di attaccare.\\


\medskip\index{Mostri - Spiritello}\textbf{Spiritello}

\emph{Minuscola fatato, neutrale buono}

\textbf{FORZA} -4

\textbf{DESTREZZA} +4

\textbf{COSTITUZIONE} +0

\textbf{INTELLIGENZA} +2

\textbf{SAGGEZZA} +1

\textbf{CARISMA} +0

\textbf{Iniziativa} +4 -- \textbf{Difesa} 16 (armatura di cuoio)

\textbf{Punti Ferita} 2 (1d4)

\textbf{Vulnerabilità al Danno} ferro freddo

\textbf{Movimento} 3 m, volo 12 m

\textbf{Tiri Salvezza}: Tempra +0, Riflessi +5, Volontà +2

\textbf{Competenze} Muoversi Silenziosamente / Nascondersi +8 (la prova è fatta con -1d6 se lo spiritello sta volando), Consapevolezza +3

\textbf{Linguaggi} Comune, Elfico, Silvano

\textbf{Sfida} 1/4 (50 PE)

\textbf{Azioni}

\emph{\textbf{Spada Lunga.} Attacco con arma da mischia}: +2 a colpire,

portata 1 m, un bersaglio.

\emph{Colpisce:} 1 danno tagliente.

\emph{\textbf{Arco Corto.} Attacco con arma a Distanza}: +6 a colpire, gittata 12 m, un bersaglio.

\emph{Colpisce:} 1 danno perforante. Se il bersaglio è una creatura, deve riuscire un Tiro Salvezza di Tempra CD 10 o restare avvelenata per 1 minuto. Se il risultato di questo Tiro Salvezza è 5 o meno, il bersaglio cade privo di sensi per la stessa durata, o finché subisce danni o un'altra creatura usa un'azione per risvegliarlo. 

\emph{\textbf{Invisibilità.}} Lo spiritello resta invisibile finché non attacca o termina la sua concentrazione. Qualsiasi cosa che lo spiritello stia trasportando o indossando resta invisibile finché rimane in contatto con lo spiritello.

\emph{\textbf{Vista del Cuore.}} Lo spiritello entra in contatto con una creatura e ne apprende l'attuale stato emotivo. Se il bersaglio fallisce un Tiro Salvezza di Tempra CD 10, lo spiritello apprende anche l'allineamento della creatura. Celestiali, immondi e non morti falliscono automaticamente questo Tiro Salvezza.

\textbf{Descrizione}\\
Gli spiritelli si riuniscono in gruppi nelle profondità di regioni boschive, uniti nella causa per proteggere la natura. Intere tribù di spiritelli si sono dichiarate protettrici di una determinata persona, di un luogo o di una creatura di particolare rilievo nelle loro terre, anche nel caso in cui l'essere non desideri o non necessiti di alcuna protezione.\\

Il corpo di uno spiritello è luminoso per natura, sebbene la creatura possa variare il colore e l'intensità della luce emessa dal suo corpo a piacimento. Subito dopo la sua morte, il corpo di uno spiritello si dissolve in una nebbia luccicante. Gli spiritelli sono i più piccoli tra i folletti, alti poco più di 22 centimetri e di un peso che raramente supera 1 kg.\\

Sotto molti aspetti gli spiritelli sono più primitivi della maggior parte dei folletti. Apprezzano la compagnia dei propri simili, ma tendono a diffidare degli altri folletti e presumono che qualsiasi umanoide o creatura che non hanno espressamente scelto di proteggere voglia far loro del male. Persino gli animali vengono da loro solitamente considerati pericolosi. La ragione di questa diffidenza è per buona parte dovuta alla taglia minuscola di queste creature, che le rende facili prede per i predatori. Pertanto la reazione iniziale di uno spiritello di fronte a un pericolo è darsi alla fuga: in genere utilizza le sue capacità magiche per rallentare o distrarre gli inseguitori, e in seguito si affida alla sua velocità di volare e alla sua taglia per riuscire a fuggire.\\

Sebbene gli spiritelli di per sé abbiano una natura incolta e selvaggia, hanno una sana curiosità per tutte le cose dotate di magia innata. Sono particolarmente attratti dai luoghi di grande potere magico latente, quali le rovine di antichi templi. Questa curiosità li rende anche insolitamente adatti al ruolo di famigli. Un incantatore caotico neutrale di 5° livello può ottenere uno spiritello come famiglio se ha l'Abilita' Famiglio.\\


\medskip\index{Mostri - Strige (Uccello Stigeo)}\textbf{Strige (Uccello Stigeo)}

\emph{Minuscola bestia, disallineato}

\textbf{FORZA} -3

\textbf{DESTREZZA} +3

\textbf{COSTITUZIONE} +0

\textbf{INTELLIGENZA} -4

\textbf{SAGGEZZA} -1

\textbf{CARISMA} -2

\textbf{Iniziativa} +3 -- \textbf{Difesa} 15

\textbf{Punti Ferita} 2 (1d4)

\textbf{Movimento} 3 m, volo 12 m

\textbf{Tiri Salvezza}: Tempra +2, Riflessi +6, Volontà +1

\textbf{Sensi} scurovisione 18 m

\textbf{Linguaggi} -

\textbf{Sfida} 1/8 (25 PE)

\textbf{Azioni}

\emph{\textbf{Risucchio di Sangue.} Attacco con arma da mischia}: +5 a colpire, portata 1 m, una creatura.

\emph{Colpisce:} 5 (1d4 + 3) danni perforanti e lo strige si attacca al bersaglio. Mentre è attaccato, lo strige non attacca. Invece, all'inizio di ciascun turno dello strige, il bersaglio perde 5 (1d4 + 3) punti ferita a causa della perdita di sangue.

Lo strige può staccarsi spendendo 1 metro di movimento. Lo fa automaticamente dopo aver risucchiato 10 punti ferita dal bersaglio o alla morte del bersaglio. Una creatura, compreso il bersaglio, può usare la sua azione per staccare lo strige.

\textbf{Ecologia}
Ambiente: Paludi temperate e calde\\
Organizzazione: Solitario, colonia (2-4), stormo (5-8), nugolo (9-14) o sciame (15-40)\\
Tesoro: Nessuno\\
\textbf{Descrizione}\\
Gli strige sono pericolosi succhiasangue che infestano le paludi e predano animali selvatici, bestiame ed ignari viaggiatori. Pur essendo deboli individualmente, sciami di queste creature sono capaci di prosciugare un uomo in pochi minuti, lasciando dietro a loro solo un cadavere essiccato.\\

Più simili ai mammiferi che agli insetti, gli strige si alzano in volo con le loro quattro ali di carne, cercando prede a sangue caldo. Spesso si nascondono vicino a pozze di acqua bevibile aspettando che i viaggiatori abbassino la guardia per poi attaccarli e bere a sazietà, conficcando le loro proboscidi nelle vene scoperte. Dopo essersi nutriti, volano via a nascondersi tra la fanghiglia e tra i canneti per deporre le loro uova e riposare finché la fame non li spinge a cacciare di nuovo.\\

Di solito gli strige sono lunghi circa 30 centimetri, con un'apertura alare di circa il doppio, e pesano meno di 0,5 kg. Sono color rosso ruggine o marrone rossastro, ed hanno il ventre color giallo sporco, ma quelli che non si sono nutriti adeguatamente sono di colore rosa pallido.\\


\medskip\index{Mostri - Succube}\textbf{Succube}

\emph{Media immondo (mutaforma), neutrale malvagio}

\textbf{FORZA} -1

\textbf{DESTREZZA} +3

\textbf{COSTITUZIONE} +1

\textbf{INTELLIGENZA} +2

\textbf{SAGGEZZA} +1

\textbf{CARISMA} +5

\textbf{Iniziativa} +3 -- \textbf{Difesa} 17

\textbf{Punti Ferita} 66 (12d8 + 12)

\textbf{Movimento} 9 m, volo 18 m

\textbf{Tiri Salvezza}: Tempra +7, Riflessi +9, Volontà +10

\textbf{Competenze} Furtività 5, Percepire Emozioni +5, Consapevolezza +5, Ingannare +9

\textbf{Resistenze al Danno} freddo, fulmine, fuoco, veleno; da botta, perforante e tagliente di attacchi non magici

\textbf{Sensi} scurovisione 18 m

\textbf{Linguaggi} Abissale, Comune, Infernale, telepatia 18 m

\textbf{Sfida} 4 (1.100 PE)

\emph{\textbf{Legame Telepatico.}} L'immondo ignora le restrizioni di raggio di azione della sua telepatia quando comunica con una creatura che ha affascinato. I due non sono neppure costretti a trovarsi sullo stesso piano di esistenza.

\emph{\textbf{Mutaforma.}} L'immondo può usare la sua azione per trasformarsi in un umanoide di taglia Piccola o Media, o per tornare alla sua vera forma. Senza le ali, l'immondo perde la velocità di volo. A parte la taglia e la velocità, le sue statistiche sono le stesse in tutte le forme. Qualsiasi equipaggiamento stia indossando o trasportando non viene trasformato. Alla morte ritorna alla sua vera forma.

\textbf{Azioni}

\emph{\textbf{Artiglio (Solo Forma Immonda).} Attacco con arma da
	mischia}:

+5 a colpire, portata 1 m, un bersaglio.

\emph{Colpisce:} 6 (1d6 + 3) danni taglienti.

\emph{\textbf{Affascinare.}} Un umanoide visibile all'immondo entro 9 metri da esso deve riuscire un Tiro Salvezza di Volontà CD 15 o restare magicamente affascinato per 1 giorno. Il bersaglio affascinato obbedisce ai comandi verbali o telepatici dell'immondo. Se il bersaglio subisce danni o riceve un comando suicida, può ripetere il Tiro Salvezza, terminando l'effetto se lo riesce. Se ilbersaglio riesce il tiro  salvezza contro l'effetto, o se l'effetto termina, il bersaglio è immune all'Affascinare dell'immondo per le successive 24 ore.

L'immondo può tenere affascinato solo un bersaglio alla volta. Se ne affascina un altro, l'effetto sul bersaglio precedente termina.

\emph{\textbf{Bacio Risucchiante.}} L'immondo bacia una creatura affascinata o una creatura consenziente. Il bersaglio deve effettuare un Tiro Salvezza di Tempra CD 15 contro questa magia, subendo 32 (5d10 + 5) danni psichici se lo fallisce, o la metà di questi danni se lo riesce. I punti ferita massimi del bersaglio vengono ridotti di un ammontare pari ai danni subiti. Questa riduzione perdura finché non sorge l'alba. Il bersaglio muore se questo effetto riduce i suoi punti ferita massimi a 0.

\emph{\textbf{Forma Eterea.}} L'immondo entra magicamente nel Piano Etereo dal Piano Materiale, e viceversa.

\textbf{Ecologia}\\
Ambiente: Qualsiasi (Abisso)\\
Organizzazione: Solitario, coppia o harem (3-12)\\
Tesoro: doppio\\
\textbf{Descrizione}\\
Tra le orde demoniache una succube spesso può raggiungere altissimi livelli di potere, utilizzando le sue manipolazioni ed il suo fascino sensuale, e molte guerre demoniache imperversano a causa delle subdole macchinazioni di queste creature. Una succube si origina dalle anime di malvagi mortali particolarmente libidinosi ed avidi.\\


\medskip\index{Mostri - Tarrasque}\textbf{Tarrasque}

\emph{Mastodontica mostruosità (titano), disallineato}

\textbf{FORZA} +10

\textbf{DESTREZZA} +0

\textbf{COSTITUZIONE} +10

\textbf{INTELLIGENZA} -4

\textbf{SAGGEZZA} +0

\textbf{CARISMA} +0

\textbf{Iniziativa} +0 -- \textbf{Difesa} 35

\textbf{Punti Ferita} 676 (33d20 + 330)

\textbf{Movimento} 12 m

\textbf{Tiri Salvezza}: Tempra +31, Riflessi +22, Volontà +12

\textbf{Immunità al Danno} fuoco, veleno; armi +2

\textbf{Immunità alle Condizioni} affascinato, avvelenato, paralizzato, spaventato

\textbf{Sensi} vista cieca 36 m

\textbf{Linguaggi} -

\textbf{Sfida} 30 (155.000 PE)

\emph{\textbf{Carapace Riflettente.}} Ogni volta che il tarrasque è il bersaglio di un incantesimo \emph{dardo incantato}, un incantesimo a linea, o un incantesimo che richiede un tiro di attacco a gittata, tira un d6. Da 1 a 5, il tarrasque lo ignora. Con 6, il tarrasque lo ignora, e l'effetto viene riflesso contro l'incantatore come se fosse originato dal tarrasque, trasformando l'incantatore nel bersaglio. 

\emph{\textbf{Mostro d'Assedio.}} Il tarrasque infligge danni doppi agli oggetti e le strutture.

\emph{\textbf{Resistenza Leggendaria (3/Giorno).}} Se il tarrasque fallisce un Tiro Salvezza, può scegliere invece di riuscire.

\emph{\textbf{Resistenza alla Magia.}} Il tarrasque ha +1d6 ai Tiri Salvezza contro incantesimi o altri effetti magici.

\textbf{Azioni}

\emph{\textbf{Multiattacco.}} Il tarrasque può usare la sua Presenza Spaventosa. Poi effettua cinque attacchi: uno con il morso, due con gli artigli, uno con le corna, e uno con la coda. Al posto del morso può usare Inghiottire.

\emph{\textbf{Artiglio.} Attacco con arma da mischia}: +19 a colpire, portata 5 metri, un bersaglio.

\emph{Colpisce:} 28 (4d8 + 10) danni taglienti.

\emph{\textbf{Coda.} Attacco con arma da mischia}: +19 a colpire,
portata 6 m, un bersaglio.

\emph{Colpisce:} 24 (4d6 + 10) danni da botta. Se il bersaglio è una creatura, deve riuscire un Tiro Salvezza di Tempra CD 20 o cadere prona.

\emph{\textbf{Corna.} Attacco con arma da mischia}: +19 a colpire, portata 3 m, un bersaglio.

\emph{Colpisce:} 32 (4d10 + 10) danni perforanti.

\emph{\textbf{Morso.} Attacco con arma da mischia}: +19 a colpire, portata 3 m, un bersaglio.

\emph{Colpisce:} 36 (4d12 + 10) danni perforanti. Se il bersaglio è una creatura, è afferrata (CD 20 per fuggire). Fino al termine dell'afferrare, il bersaglio è intralciato, e il tarrasque non può usare il morso contro un altro bersaglio.

\emph{\textbf{Inghiottire.}} Il tarrasque effettua una attacco di morso contro un bersaglio di taglia Grande o inferiore che sta afferrando. Se l'attacco colpisce, il bersaglio è inghiottito, e l'afferrare ha termine. Il bersaglio inghiottito è accecato e intralciato, ha copertura totale contro gli attacchi e altri effetti all'esterno del tarrasque, e subisce 56 (16d6) danni da acido all'inizio di ciascun turno del tarrasque.

Se il tarrasque subisce 60 o più danni in un singolo turno da una creatura al suo interno, il tarrasque deve riuscire un Tiro Salvezza su Tempra CD 30 al termine di quel turno o vomitare tutte le creature inghiottite, che cadono prone in uno spazio entro 3 metri dal tarrasque. Se il tarrasque muore, una creatura inghiottita non  più intralciata da esso e può uscire dal cadavere utilizzando 9 metri  di movimento, uscendo prona. 

\emph{\textbf{Presenza Spaventosa.}} Ogni creatura scelta dal tarrasque, che si trovi entro 36 metri da esso e consapevole della sua presenza, deve riuscire un Tiro Salvezza di Volontà CD 17 o restare spaventata per 1 minuto. Una creatura può ripetere il Tiro Salvezza al termine di ciascun suo turno, con -1d6 se il tarrasque è in linea di visuale, terminando l'effetto per sé, se lo riesce. Se il Tiro Salvezza della creatura ha successo o l'effetto ha termine per essa, la creatura è immune alla Presenza Spaventosa del tarrasque per le successive 24 ore.

\textbf{Azioni Aggiuntive}

Il tarrasque può effettuare 3 Azioni aggiuntive, scelte tra le opzioni seguenti. Può usare solo un'opzione leggendaria alla volta e solo al termine del turno di un'altra creatura. Il tarrasque recupera le azioni aggiuntive spese all'inizio del proprio turno.

\textbf{Attacco.} Il tarrasque effettua un attacco di artiglio o di coda. \textbf{Masticare (Costa 2 Azioni).} Il tarrasque effettua un attacco di morso o usa Inghiottire.

\textbf{Muoversi.} Il tarrasque si muove fino a metà del suo movimento.


\textbf{Ecologia}\\
Ambiente: Qualsiasi\\
Organizzazione: Solitario\\
Tesoro: Nessuno\\
\textbf{Descrizione}\\
Il leggendario tarrasque è fra i mostri più distruttivi del mondo. Fortunatamente, passa la maggior parte del suo tempo in una specie di profondo letargo in una sconosciuta caverna in un remoto angolo del mondo. Quando si risveglia, però, muoiono interi regni.\\
Pur non particolarmente intelligente, il tarrasque è abbastanza intelligente da capire alcune parole in Parlata delle Profondità (pur non potendo parlare). Allo stesso modo, la furia non è incontrollata: si concentra sulla creatura che l'ha danneggiato maggiormente ed è difficile distrarlo con l'inganno.\\


\medskip\index{Mostri - Testuggine Dragona}\textbf{Testuggine Dragona}

\emph{Mastodontica drago, neutrale}

\textbf{FORZA} +7

\textbf{DESTREZZA} +0

\textbf{COSTITUZIONE} +5

\textbf{INTELLIGENZA} +0

\textbf{SAGGEZZA} +1

\textbf{CARISMA} +1

\textbf{Iniziativa} +0 -- \textbf{Difesa} 29

\textbf{Punti Ferita} 341 (22d20 + 110) 

\textbf{Movimento} 6 m, nuoto 12 m

\textbf{Tiri Salvezza} Tempra +12, Riflessi +8, Volontà +9

\textbf{Sensi} scurovisione 18 m

\textbf{Linguaggi} Aquan, Draconico

\textbf{Sfida} 17 (18.000 PE)

\emph{\textbf{Anfibio.}} La testuggine dragona può respirare aria e acqua.

\textbf{Azioni}

\emph{\textbf{Multiattacco.}} Il drago può effettuare tre attacchi: uno con il morso e due con gli artigli. Può effettuare un attacco di coda al posto di due attacchi di artiglio.

\emph{\textbf{Artiglio.} Attacco con arma da mischia}: +13 a colpire, portata 3 m, un bersaglio.

\emph{Colpisce:} 16 (2d8 + 7) danni taglienti.

\emph{\textbf{Coda.} Attacco con arma da mischia}: +13 a colpire, portata 5 metri, un bersaglio.

\emph{Colpisce:} 26 (3d12 + 7) danni da botta. Se il bersaglio è una creatura, deve riuscire un Tiro Salvezza di Tempra CD 20 o venire spinta di 3 metri lontano dalla testuggine dragona e cadere prona.

\emph{\textbf{Morso.} Attacco con arma da mischia}: +13 a colpire, portata 5 metri, un bersaglio.

\emph{Colpisce:} 26 (3d12 + 7) danni perforanti.

\emph{\textbf{Soffio di Vapore (Ricarica 5-6).}} La testuggine dragona esala un vapore caldo in un cono di 18 metri. Ogni creatura in quell'area deve effettuare un Tiro Salvezza di Tempra CD 18 e subire 52 (15d6) danni da fuoco se fallisce il Tiro Salvezza, o la metà di questi danni se lo riesce. Trovarsi sott'acqua non dà resistenza contro questo tipo di danno.

\textbf{Ecologia}
Ambiente: Acquatico temperato\\
Organizzazione: Solitario\\
Tesoro: Doppio\\
\textbf{Descrizione}\\
Le testuggini dragone abitano nelle acque dolci e salate, dove si attestano tra i più grandi pericoli per i marinai e coloro che viaggiano per nave attraverso le rotte marine del mondo. Gli esperti marinai sanno quello che vogliono le testuggini dragone della zona e frequentemente fanno offerte in oro e magia per garantirsi un passaggio sicuro o evitano completamente l'area. Da parte sua, una testuggine dragona apprezza ed anche si aspetta tali pedaggi e regalie, e una testuggine dragona che si aspetta regali ma viene ignorata è davvero un nemico pericoloso.\\

Il colore del guscio di una testuggine dragona varia da individuo a individuo. Alcuni hanno gusci opachi marrone e rosso ruggine, mentre altri hanno carapaci di un intenso color verde-blu con riflessi argentei sulle punte rocciose. La colorazione di testa, coda e zampe è lievemente più pallida del guscio e comprende striature dorate lungo la cresta e le spine.\\
Le testuggini dragone reclamano enormi territori in mare aperto, che comprendono regioni che spesso superano i 75 km quadrati. Qui, queste bestie pericolose capovolgono le navi che non rispettano i loro territori, aggiungendo relitti sommersi ed i loro preziosi carichi ai loro nascondigli. Le testuggini dragone generalmente fanno le loro tane in profonde caverne accessibili solo attraverso l'acqua, e spesso non solo le decorano con le ricchezze trafugate dalle navi che hanno affondato, ma anche coi relitti di queste sfortunate imbarcazioni. La loro natura territoriale e la loro predilezione per questo tipo di tane le mettono in conflitto diretto con le altre razze sottomarine come Marinidi e Sahuagin.\\

I grandi pesci, come tonni, storioni ed anche squali sono compresi tra i cibi preferiti dalle testuggini dragone, ma essendo onnivore, qualche volta si alimentano anche di grandi campi subacquei di alghe marine. Certamente non disdegnano di integrare la loro dieta coi passeggeri delle navi che affondano, anche se tale pratica non è dovuta né a malvagità né a crudeltà. Le testuggini dragone hanno gusci del diametro di 5 metri, con gli arti che si estendono pochi metri più in là, e misurano 7 metri dalla punta del naso all'estremità della loro possente coda.\\

\medskip\index{Mostri - Troll}\textbf{Troll}

\emph{Grande gigante, caotico malvagio}

\textbf{FORZA} +4

\textbf{DESTREZZA} +1

\textbf{COSTITUZIONE} +5

\textbf{INTELLIGENZA} -2

\textbf{SAGGEZZA} -1

\textbf{CARISMA} -2

\textbf{Iniziativa} +1 -- \textbf{Difesa} 18

\textbf{Punti Ferita} 84 (8d10 + 40)

\textbf{Movimento} 9 m

\textbf{Tiri Salvezza}: Tempra +11, Riflessi +4, Volontà +3

\textbf{Competenze} Consapevolezza +2

\textbf{Sensi} scurovisione 18 m

\textbf{Linguaggi} Gigante

\textbf{Sfida} 5 (1.800 PE)

\emph{\textbf{Olfatto Affinato.}} Il troll ha +1d6 alle prove di Saggezza (Consapevolezza) basate sull'olfatto.

\emph{\textbf{Rigenerazione.}} Il troll recupera 10 punti ferita all'inizio del suo turno. Se il troll subisce danno da acido o da fuoco, questo tratto non funziona all'inizio del prossimo turno del troll. Il troll muore solo se inizia il suo turno a -5 punti ferita e non può rigenerarsi.

\textbf{Azioni}

\emph{\textbf{Multiattacco.}} Il troll può effettuare tre attacchi: uno con il morso e due con gli artigli.

\emph{\textbf{Artiglio.} Attacco con arma da mischia}: +7 a colpire, portata 1 m, un bersaglio.

\emph{Colpisce:} 11 (2d6 + 4) danni taglienti.

\emph{\textbf{Morso.} Attacco con arma da mischia}: +7 a colpire, portata 1 m, un bersaglio.

\emph{Colpisce:} 7 (1d6 + 4) danni perforanti.

\textbf{Ecologia}\\
Ambiente: Montagne Fredde\\
Organizzazione: Solitario o banda (2-4)\\
Tesoro: Standard\\
\textbf{Descrizione}\\
I troll possiedono artigli affilati ed incredibili capacità rigenerative che permettono loro di guarire quasi tutte le ferite. Sono gobbi, brutti ma fortissimi: combinata con i loro artigli, la loro forza gli permette di lacerare la carne a mani nude. I troll sono alti circa 4 metri, ma la loro postura li fa apparire più bassi. Un troll adulto pesa circa 500 kg.\\

l'appetito di un troll e le sue capacità rigenerative lo rendono un combattente indomito, che carica a testa bassa la creatura vivente più vicina ed attacca con tutta la sua furia. Solo il fuoco fa esitare un troll, ma perfino quello che per lui è un pericolo mortale non ferma la sua avanzata. Chi affronta i troll sa di dover localizzare e bruciare qualsiasi sua parte dopo un combattimento, perché perfino dal brandello più piccolo del suo corpo, con il tempo può rinascere un troll completo. Fortunatamente, solo le parti più grandi di un troll, come gli arti, ricrescono in questo modo.\\

Nonostante la loro ferocia, i troll sono straordinariamente teneri e gentili verso i loro piccoli. I troll femmina lavorano in gruppo, passando molto tempo ad insegnare ai cuccioli come cacciare e difendersi prima di mandarli a cercare un proprio territorio. Un troll maschio vive un'esistenza solitaria, incontrando brevemente le femmine solo per accoppiarsi. Tutti i troll trascorrono il loro tempo a cercare cibo, dato che devono consumarne enormi quantità ogni giorno o muoiono di fame. Per questo, la maggior parte dei troll si crea un proprio territorio di caccia che viene spesso difeso combattendo con i rivali. Simili scontri sono di solito non letali, ma i troll conoscono bene le proprie debolezze, sfruttandole per uccidere l'avversario nei periodi di magra.\\


\medskip\index{Mostri - Uomo Acquatico}\textbf{Uomo Acquatico}

\emph{Media umanoide (uomo acquatico), neutrale}

\textbf{FORZA} +0

\textbf{DESTREZZA} +1

\textbf{COSTITUZIONE} +1

\textbf{INTELLIGENZA} +0

\textbf{SAGGEZZA} +0

\textbf{CARISMA} +1

\textbf{Iniziativa} +1 -- \textbf{Difesa} 12

\textbf{Punti Ferita} 11 (2d8 + 2)

\textbf{Movimento} 3 m, nuoto 12 m

\textbf{Tiri Salvezza}:  Tempra +3, Riflessi +1, Volontà -1; +2 contro Ammaliamento

\textbf{Competenze} Consapevolezza +2

\textbf{Linguaggi} Aquan, Comune

\textbf{Sfida} 1/8 (25 PE)

\emph{\textbf{Anfibio.}} L'uomo acquatico può respirare aria e acqua.

\textbf{Azioni}

\emph{\textbf{Lancia.} Attacco con arma da mischia o a Distanza}: +2 a colpire, portata 1 m o gittata 6m, un bersaglio.

\emph{Colpisce:} 3 (1d6) danni perforanti, o 4 (1d8) danni perforanti se
usata con due mani per effettuare un attacco da mischia.

\textbf{Ecologia}\\
Ambiente: Oceani temperati\\
Organizzazione: Solitario, pattuglia (2-6), banda (6-10 più un tenete di 3° livello, compagnia (11-60 più 3 tenenti di 3° livello, 2 comandanti di 5° livello, 1 commodoro di 7° livello e 3-12 Calamari\\
Tesoro: Equipaggiamento da PNG (Tridente, Balestra Leggera con 10 Quadrelli, altro tesoro)\\
\textbf{Descrizione}\\
Fisicamente, gli Uomini Pesce somigliano ai loro antenati, con fronti espressive, pelle pallida, capelli scuri e occhi porpora. Hanno tre sottili branchie sul collo, ma possono passare per Umani per brevi periodi, se lo desiderano.\\


\medskip\index{Mostri - Uomo Albero (Treant)}\textbf{Uomo Albero (Treant)}

\emph{Enorme pianta, caotico buono}

\textbf{FORZA} +6

\textbf{DESTREZZA} -1

\textbf{COSTITUZIONE} +5

\textbf{INTELLIGENZA} +1

\textbf{SAGGEZZA} +3

\textbf{CARISMA} +1

\textbf{Iniziativa} +1 -- \textbf{Difesa} 21

\textbf{Punti Ferita} 138 (12d12 + 60)

\textbf{Movimento} 9 m

\textbf{Tiri Salvezza}: Tempra +13, Riflessi +3, Volontà +9

\textbf{Resistenze al Danno} da botta, perforante

\textbf{Vulnerabilità al Danno} fuoco

\textbf{Linguaggi} Comune, Druidico, Elfico, Silvano

\textbf{Sfida} 9 (5.000 PE)

\emph{\textbf{Falso Aspetto.}} Mentre l'uomo albero rimane immobile, è indistinguibile da un normale albero.

\emph{\textbf{Mostro d'Assedio.}} L'uomo albero infligge danni doppi agli oggetti e le strutture.

\textbf{Azioni}

\emph{\textbf{Multiattacco.}} L'uomo albero effettua due attacchi di schianto.

\emph{\textbf{Schianto.} Attacco con arma da mischia}: +10 a colpire, portata 1 m, un bersaglio.

\emph{Colpisce:} 16 (3d6 + 6) danni da botta.

\emph{\textbf{Sasso.} Attacco con arma a Distanza}: +10 a colpire, gittata 18m, un bersaglio.

\emph{Colpisce:} 28 (4d10 + 6) danni da botta.

\emph{\textbf{Animare Alberi (1/Giorno).}} L'uomo albero anima magicamente uno o due alberi visibili entro 18 metri da lui. Questi albeti hanno le stesse statistiche dell'ent, eccetto che hanno punteggio di Intelligenza e Carisma -3, non possono parlare, e hanno solo l'opzione di attacco Schianto. Un albero animato agisce come alleato dell'uomo albero. L'albero resta per 1 giorno o finché muore; finché l'uomo albero muore o si trova più di 36 metri lontano dall'albero, o finché l'uomo albero non effettua un'azione bonus per ritrasformarlo in un albero inanimato. Poi l'albero prenderà radici, se possibile.

\textbf{Ecologia}\\
Ambiente: Qualsiasi foresta\\
Organizzazione: Solitario o macchia (2-7)\\
Tesoro: Standard\\
\textbf{Descrizione}\\
I treant sono guardiani delle foreste ed ambasciatori degli alberi. Antichi quanto le foreste stesse, si vedono come genitori e pastori piuttosto che giardinieri: sono lenti e metodici, ma terrificanti quando costretti a combattere per difendere il loro gregge. Anche se raramente cercano la compagnia delle razze dalla vita breve ed hanno un'innata sfiducia verso i cambiamenti, mostrano tolleranza verso chi desidera imparare dai loro lunghi, lenti monologhi, specialmente coloro nei cui occhi leggono il desiderio di proteggere le regioni selvagge. Contro coloro che minacciano le loro foreste, specialmente i boscaioli che raccolgono legna o coloro che vorrebbero disboscare una foresta per costruire una strada o un forte, la rabbia dei treant si scatena rapida e devastante. Sono in grado di demolire ciò che gli altri costruiscono: un tratto che li aiuta durante i loro eccessi di furia.\\

I treant sono principalmente creature solitarie, ed un singolo individuo è spesso responsabile di un'intera foresta, ma a volte si raccolgono in gruppi detti boschetti per scambiarsi le ultime notizie e riprodursi.\\

In tempi di grave pericolo, tutti i boschetti di una regione si uniscono per una riunione della durata di mesi detta concilio, ma simili eventi sono molto rari, e fra i concili passano anche millenni.\\

Un tipico treant è alto 9 metri, con un tronco del diametro di 60 centimetri, e pesa circa 2.250 kg. I treant somigliano agli alberi più comuni dei territori dove vivono.\\



\medskip\index{Mostri - Uomo Magma (Magmin)}\textbf{Uomo Magma (Magmin)}

\emph{Piccola elementale, caotico neutrale}

\textbf{FORZA} -2

\textbf{DESTREZZA} +2

\textbf{COSTITUZIONE} +1

\textbf{INTELLIGENZA} -1

\textbf{SAGGEZZA} +0

\textbf{CARISMA} +0

\textbf{Iniziativa} +2 -- \textbf{Difesa} 15

\textbf{Punti Ferita} 9 (2d6 + 2)

\textbf{Movimento} 9 m

\textbf{Tiri Salvezza}: Tempra +6, Riflessi +4, Volontà +3

\textbf{Resistenze al Danno} da botta, perforante e tagliente di attacchi non magici

\textbf{Immunità ai Danni} fuoco

\textbf{Sensi} scurovisione 18 m

\textbf{Linguaggi} Ignan

\textbf{Sfida} 1/2 (100 PE)

\emph{\textbf{Illuminazione Incendiaria.}} Come azione bonus, l'uomo magma può accendere o spegnere le sue fiamme. Mentre la fiamma è accesa, l'uomo magma irradia luce intensa in un raggio di 3 metri e luce fioca per ulteriori 3 metri.

\emph{\textbf{Scoppio Mortale.}} Quando l'uomo magma muore, esplode in uno scoppio di fuoco e magma. Ogni creatura entro 3 metri da esso deve effettuare un Tiro Salvezza di Riflessi CD 11, subendo 7 (2d6) danni da fuoco se fallisce il Tiro Salvezza, o la metà di questi danni se lo riesce. Gli oggetti infiammabili che non siano indossati o trasportati e che si trovino nell'area, prendono fuoco.

\textbf{Azioni}

\emph{\textbf{Tocco.} Attacco con arma da mischia}: +4 a colpire, portata 1 m, un bersaglio.

\emph{Colpisce:} 7 (2d6) danni da fuoco. Se il bersaglio è una creatura o un oggetto infiammabile, questi prende fuoco. Fino a che una creatura effettua un'azione per estinguere la fiamma, la creatura subisce 3 (1d6) danni da fuoco al termine di ciascun suo turno.

\textbf{Ecologia}\\
Ambiente: Qualsiasi terreno (Piano del Fuoco)\\
Organizzazione: Solitario o banda (2-8)\\
Tesoro: Standard\\
\textbf{Descrizione}\\
Benché i magmin popolino il Piano del Fuoco, a volte scivolano nelle crepe elementali nel Piano Materiale. Queste crepe di solito si formano in luoghi di forte calore, come vulcani o fiumi sotterranei di magma, o in luoghi di forte e imprevedibile magia. Quest'ultimo scenario termina di solito in eventi più complessi, dato che i magmin tendono ad appiccare involontariamente fuoco agli oggetti infiammabili vicini.\\

Anche se non sono coraggiosi, questi piccoli esterni sono tuttavia temibili nemici delle creature senza resistenza al loro intenso calore. Il loro tocco incenerisce gli abiti, e le creature che colpiscono i loro corpi con l'acciaio corrono il rischio di ridurre in scorie le loro armi. La miglior difesa dei magmin in patria sul Piano del Fuoco è il loro numero. Gli insediamenti, costellati di laghi di magma e spruzzanti geyser di roccia fusa, brulicano di incredibili quantità di queste creature.\\

I magmin sono paranoici e diffidenti. Sempre spaventati dagli abitanti più grandi del Piano del Fuoco, i magmin sommergono gli intrusi con migliaia di domande, chiedendo dove vanno, da dove vengono, e che cosa fanno vicino ai loro preziosi laghi di magma. Se le risposte dei viaggiatori non sono soddisfacenti, i magmin tentano di sbarazzarsi il più rapidamente possibile delle creature. Chi si rifiuta di andarsene rischia di essere gettato in un lago di roccia liquida.\\

I magmin sono molto orgogliosi di come curano i loro laghi di magma. Ogni lago ha un diverso scopo: per farsi il bagno, per cucinare i pasti o per rilassarsi. I magmin inseriscono minerali e sali in questi laghi per adeguarli al loro scopo. I laghi per cucinare (a volte chiamati dagli stra-nieri "laghi assassini") sono più caldi degli altri, e quelli per lo svago sono di solito più scuri di quelli da bagno.\\

Alla maturità, i magmin sono alti 1,2 metri, la loro densa composizione li fa pesera 150 kg.\\


\medskip\index{Mostri - Unicorno}\textbf{Unicorno}

\emph{Grande celestiale, legale buono}

\textbf{FORZA} +4

\textbf{DESTREZZA} +2

\textbf{COSTITUZIONE} +2

\textbf{INTELLIGENZA} +0

\textbf{SAGGEZZA} +3

\textbf{CARISMA} +3

\textbf{Iniziativa} +2 -- \textbf{Difesa} 15

\textbf{Punti Ferita} 67 (9d10 + 18)

\textbf{Movimento} 15 m

\textbf{Tiri Salvezza}: Tempra +7, Riflessi +7, Volontà +6; +2 resistenza contro il Vuoto, Energia Negativa

\textbf{Immunità al Danno} veleno

\textbf{Immunità alle Condizioni} affascinato, avvelenato, paralizzato

\textbf{Sensi} scurovisione 18 m

\textbf{Linguaggi} Celestiale, Elfico, Silvano, telepatia 18 m

\textbf{Sfida} 5 (1.800 PE)

\emph{\textbf{Armi Magiche.}} Gli attacchi con armi dell'unicorno sono magici.

\emph{\textbf{Carica.}} Se l'unicorno si muove di almeno 6 metri in linea retta verso il bersaglio e lo colpisce con un attacco di corno durante lo stesso turno, il bersaglio subisce 9 (2d8) danni perforanti aggiuntivi. Se il bersaglio è una creatura, deve riuscire un Tiro Salvezza su Tempra CD 15 o cadere prono.

\emph{\textbf{Incantesimi Innati.}} La caratteristica da incantatore innato dell'unicorno è il Carisma (CD 14 per i Tiri Salvezza degli incantesimi). L'unicorno può lanciare in maniera innata i seguenti incantesimi, senza bisogno di componenti:

A volontà: \emph{arte del druido, individuazione del bene e male,} \emph{passare senza tracce}

1/giorno ciascuno: \emph{calmare emozioni, dissolvi il bene e il male,} \emph{intralciare}

\emph{\textbf{Resistenza alla Magia.}} L'unicorno ha +1d6 ai Tiri Salvezza contro incantesimi e altri effetti magici. 

\textbf{Azioni}

\emph{\textbf{Multiattacco.}} L'unicorno effettua due attacchi: uno con gli zoccoli e uno con il corno.

\emph{\textbf{Corno.} Attacco con arma da mischia}: +7 a colpire, portata 1 m, un bersaglio.

\emph{Colpisce:} 8 (1d8 + 4) danni perforanti.

\emph{\textbf{Zoccoli.} Attacco con arma da mischia}: +7 a colpire, portata 1 m, un bersaglio.

\emph{Colpisce:} 11 (2d6 + 4) danni da botta.

\emph{\textbf{Telestraporto (1/Giorno).}} L'unicorno può teletrasportare  magicamente sé stesso e fino a tre altre creature consenzienti  visibili entro 1 metro da esso, insieme a tutto  l'equipaggiamento che stanno indossando o trasportando, in un  luogo familiare all'unicorno, che si trova ad un massimo di 1,5 chilometri di distanza.

\emph{\textbf{Tocco Guaritore (3/Giorno).}} L'unicorno entra a contatto tramite il corno con un'altra creatura. Il bersaglio recupera magicamente 11 (2d8 + 2) punti ferita. Inoltre, il contatto rimuove tutte le malattie e neutralizza tutti i veleni che affliggono il bersaglio.

\textbf{Azioni Aggiuntive}

L'unicorno può effettuare 3 Azioni aggiuntive, scelte tra le opzioni seguenti. Può usare solo un'opzione leggendaria alla volta e solo al termine del turno di un'altra creatura. L'unicorno recupera le azioni aggiuntive spese all'inizio del proprio turno.

\textbf{Autoguarigione (Costa 3 Azioni).} L'unicorno recupera magicamente 11 (2d8 + 2) punti ferita.

\textbf{Scudo Scintillante (Costa 2 Azioni).} L'unicorno crea un campo magico scintillante che circonda lui o un'altra creatura visibile a lui entro 18 metri. Il bersaglio ottiene un bonus di +2 alla Difesa fino al termine del prossimo turno dell'unicorno.

\textbf{Zoccoli.} L'unicorno effettua un attacco con gli zoccoli.

\textbf{Ecologia}\\
Ambiente: Foreste Temperate\\
Organizzazione: Solitario, coppia o benedizione (3-6)\\
Tesoro: Nessuno\\
\textbf{Descrizione}\\
Gli unicorni sono fiere, intelligenti creature silvane che preferiscono rimanere isolate, apparendo solo per difendere le loro dimore dal male. Evitano tutte le creature tranne i folletti buoni, le donne umanoidi buone e gli animali nativi della loro foresta, ma potrebbero unirsi ad altre creature buone contro nemici comuni. Un tipico unicorno è lungo 2,4 metri, alto 1 metro al garrese e pesa 600 kg.\\

Le coppie di unicorni rimangono insieme per tutta la vita e dimorano in radure particolari o all'interno delle foreste che difendono. Permettono alle creature buone e neutrali di attraversarle, cacciare o di abitarvi, ma le creature malvagie o quelle che vorrebbero turbarne l'ecosistema, ad esempio cacciando per divertimento o abbattendone gli alberi per venderne il legname, vengono in fretta allontanati o uccisi. In alcune rare occasioni, gli unicorni il cui partner è stato ucciso prendono giovani donne di rara virtù come surrogati, permettendo loro di cavalcarli divenendo loro guardiani per tutta la vita. Se la donna si lega a qualcun altro, come un figlio o un amante, il legame con l'unicorno si scioglie amorevolmente, generando la leggenda che gli unicorni diventano amici solo delle vergini.\\

Il corno di un unicorno è la fonte dei suoi poteri, e per utilizzare le proprie capacità magiche su altre creature questi deve toccarle con esso. Le creature malvagie danno grande valore ai corni di unicorno come reagenti per pozioni di guarigione e per riti oscuri: un corno di unicorno in polvere vale 1.600 mo quando utilizzato per creare un oggetto magico di guarigione.\\


\subsection{Vampiri}

\medskip\index{Mostri - Vampiro}\textbf{Vampiro}

\emph{Media non morto (mutaforma), legale malvagio}

\textbf{FORZA} +4

\textbf{DESTREZZA} +4

\textbf{COSTITUZIONE} +4

\textbf{INTELLIGENZA} +3

\textbf{SAGGEZZA} +2

\textbf{CARISMA} +4

\textbf{Iniziativa} +4 -- \textbf{Difesa} 23

\textbf{Punti Ferita} 144 (17d8 + 68)

\textbf{Movimento} 9 m

\textbf{Tiri Salvezza} : Tempra +13, Riflessi +11, Volontà +12

\textbf{Competenze} Muoversi Silenziosamente / Nascondersi +9, Consapevolezza +17

\textbf{Immunità al Danno} da Vuoto; da botta, perforante e tagliente di attacchi non magici

\textbf{Sensi} scurovisione 36 m

\textbf{Linguaggi} le lingue che conosceva in vita

\textbf{Sfida} 13 (10.000 PE)

\emph{\textbf{Mutaforma.}} Se il vampiro non è sotto la luce del sole o immerso in acqua corrente, può usare la sua azione per trasformarsi in un Minuscolo pipistrello, una nube di foschia Media, o per tornare alla sua vera forma.

Mentre è in forma di pipistrello, il vampiro non può parlare, la sua velocità di passeggio è 1 metro e ha velocità di volo 9 metri. Le sue statistiche, a parte la taglia e la velocità, sono immutate. Qualsiasi equipaggiamento stia indossando si trasforma con esso, ma quello che stava trasportando viene fatto cadere a terra. Alla morte ritorna alla sua vera forma.

Mentre è in forma di foschia, il vampiro non può effettuare azioni, parlare o manipolare oggetti. È privo di peso, ha velocità di volo 6 metri, può fluttuare, e può entrare nello spazio di una creatura ostile e fermarsi lì. Inoltre, se in uno spazio vi passa dell'aria, la foschia può fare altrettanto senza stringersi, ma non può attraversare l'acqua. Ha +1d6 ai Tiri Salvezza su Tempra e Riflessi ed è immune a tutti i danni non magici, eccetto i danni subiti dalla luce del
sole.

\emph{\textbf{Debolezze del Vampiro.}} Il vampiro ha i seguenti difetti:

\emph{Danneggiato dall'Acqua Corrente.} Il vampiro subisce 20 danni da acido se termina il suo turno all'interno dell'acqua corrente.

\emph{Ipersensibilità alla Luce.} Il vampiro subisce 20 danni da Luce quando inizia il suo turno alla luce del sole. Mentre è alla luce del sole, ha -1d6 ai tiri di attacco e le prove di competenza. 

\emph{Paletto nel Cuore.} Se un'arma perforante fatta di legno viene conficcata nel cuore del vampiro mentre il vampiro è inabile nel suo luogo di riposo, il vampiro resta paralizzato finché il paletto non viene rimosso.

\emph{Proibizione.} Il vampiro non può entrare in un'abitazione senza invito da parte dei suoi occupanti.

\emph{\textbf{Fuga nella Foschia.}} Quando scende a 0 punti ferita al di fuori del suo luogo di riposo, il vampiro si trasforma in una nube di foschia (come per il tratto Mutaforma) invece di cadere privo di sensi, purché non sia esposto alla luce del sole o all'acqua corrente. Se non può trasformarsi, viene distrutto.

Mentre si trova a 0 punti ferita in questa forma, non può tornare alla sua forma di vampiro, e deve raggiungere il suo luogo di riposo entro 2 ore o venire distrutto. Una volta raggiunto il suo luogo di riposo, ritorna alla sua forma di vampiro. Resterà quindi paralizzato finché non avrà recuperato almeno 1 punto ferita. Dopo aver trascorso almeno 1 ora nel suo luogo di riposo a 0 punti ferita, il vampiro recupererà 1 punto ferita.

\emph{\textbf{Natura Non Morta.}} Il vampiro non ha bisogno di aria.

\emph{\textbf{Resistenza Leggendaria (3/Giorno).}} Se il vampiro fallisce un Tiro Salvezza, può scegliere invece di riuscire.

\emph{\textbf{Rigenerazione.}} Il vampiro recupera 20 punti ferita all'inizio del suo turno se possiede almeno 1 punto ferita e non è esposto alla luce del sole o l'acqua corrente. Se il vampiro subisce danno da Luce o danno dall'acqua sacra, questo tratto non funziona all'inizio del prossimo turno del vampiro.

\emph{\textbf{Scalare come Ragno.}} Il vampiro può scalare superfici difficili, compreso lo stare a testa in giù sul soffitto, senza bisogno di effettuare una prova di abilità.

\textbf{Azioni}

\emph{\textbf{Multiattacco.}} Il vampiro può effettuare due attacchi, ma solo uno di essi può essere un attacco con morso.

\emph{\textbf{Colpo Disarmato (Solo in Forma di Vampiro).} Attacco con arma da mischia}: +9 a colpire, portata 1 m, una creatura.

\emph{Colpisce:} 8 (1d8 + 4) danni da botta. Invece di infliggere danno, il vampiro può afferrare il bersaglio (CD per fuggire 18).

\emph{\textbf{Morso (Solo in Forma di Pipistrello o Vampiro).} Attacco con arma da mischia}: +9 a colpire, portata 1 m, una creatura consenziente o una creatura afferrata dal vampiro, inabile o intralciata.

\emph{Colpisce:} 7 (1d6 + 4) danni perforanti più 10 (3d6) danni da Vuoto. I punti ferita massimi del bersaglio sono ridotti di un ammontare pari al danno da Vuoto subito, e il vampiro recupera un numero di punti ferita pari a quell'ammontare. Questa riduzione permane fino alla nuova alba. Il bersaglio muore se questo effetto riduce i suoi punti ferita massimi a 0. Un umanoide ucciso in questo modo e poi sepolto nel terreno si rianima la notte seguente come progenie vampirica sotto il controllo del vampiro.

\emph{\textbf{Affascinare.}} Il vampiro prende a bersaglio un umanoide entro 9 metri che può vedere. Se il bersaglio può vedere il vampiro, deve effettuare un Tiro Salvezza di Volontà CD 17 contro questa magia o esserne affascinato. Il bersaglio affascinato considera il vampiro un amico fidato da ascoltare e proteggere. Sebbene il bersaglio non sia sotto il controllo del vampiro, prende le richieste e le azioni del vampiro nel modo più favorevole possibile, ed è un bersaglio consenziente dell'attacco con morso del vampiro.

Ogni volta che il vampiro o i compagni del vampiro fanno qualcosa di nocivo al bersaglio, questi può ripetere il Tiro Salvezza, terminando l'effetto su di sé in caso di successo. Altrimenti, l'effetto persiste 24 ore o finché il vampiro non viene distrutto, si trova su di un piano di esistenza diverso dal bersaglio, o effettua un'azione bonus per terminare l'effetto.

\emph{\textbf{Figli della Notte (1/Giorno).}} Il vampiro richiama magicamente 2d4 sciami di pipistrelli o ratti, purché il sole non sia sorto. Mentre è all'esterno, il vampiro può richiamare invece 3d6 lupi. Le creature richiamate arrivano in 1d4 round, agendo da alleati del vampiro e obbedendo ai suoi comandi. Le bestie restano per 1 ora, finché il vampiro non muore, o finché non le congeda con un'azione bonus.

\textbf{Azioni Aggiuntive}

Il vampiro può effettuare 3 Azioni aggiuntive, scelte tra le opzioni seguenti. Può usare solo un'opzione leggendaria alla volta e solo al termine del turno di un'altra creatura. Il vampiro recupera all'inizio del proprio turno le Azioni aggiuntive che ha speso.

\textbf{Colpo Disarmato.} Il vampiro effettua un colpo disarmato. \textbf{Morso (Costa 2 Azioni).} Il vampiro effettua un attacco con
morso.

\textbf{Muoversi.} Il vampiro si muove del suo movimento senza provocare attacchi di opportunità.

\textbf{Ecologia}
Ambiente: Qualsiasi\\
Organizzazione: Solitario o famiglia (vampiro più 2-8 Progenie)\\
Tesoro: Equipaggiamento da PNG (Anello della Protezione +2, Fascia della Seduzione +4, Mantello della Resistenza +3)\\
\textbf{Descrizione}\\
I vampiri sono creature umanoidi non morte che si nutrono del sangue dei viventi. Hanno un aspetto molto simile a quando erano in vita, diventando spesso più attraenti, sebbene alcuni appaiano invece duri e ferini.\\



\medskip\index{Mostri - Progenie Vampirica}\textbf{Progenie Vampirica}

\emph{Media non morto, neutrale malvagio}

\textbf{Iniziativa} +0 -- \textbf{Difesa} 18

\textbf{Punti Ferita} 82 (11d8 + 33)

\textbf{Movimento} 9 m

\textbf{Tiri Salvezza} Tempra +3, Riflessi +2, Volontà +5

\textbf{FORZA} +3

\textbf{DESTREZZA} +3

\textbf{COSTITUZIONE} +3

\textbf{INTELLIGENZA} +0

\textbf{SAGGEZZA} +0

\textbf{CARISMA} +1

\textbf{Competenze} Muoversi Silenziosamente / Nascondersi +6, Consapevolezza +3

\textbf{Resistenze ai Danni} da Vuoto; da botta, perforante e tagliente di attacchi non magici

\textbf{Sensi} scurovisione 18 m

\textbf{Linguaggi} le lingue che conosceva in vita 

\textbf{Sfida} 5 (1.800 PE)

\emph{\textbf{Debolezze della Progenie Vampirica.}} La Progenie Vampirica ha i seguenti difetti:

\emph{Danneggiato dall'Acqua Corrente.} La Progenie Vampirica subisce 20 danni da acido se termina il suo turno all'interno dell'acqua corrente.

\emph{Ipersensibilità alla Luce.} La Progenie Vampirica subisce 20 danni da Luce quando inizia il suo turno alla luce del sole. Mentre è alla luce del sole, ha -1d6 ai tiri di attacco e le prove di competenza.

\emph{Paletto nel Cuore.} La Progenie Vampirica è distrutto se un'arma perforante di legno gli viene conficcata nel cuore mentre è inabile all'interno del suo luogo di riposo.

\emph{Proibizione.} La Progenie Vampirica non può entrare in un'abitazione senza invito da parte dei suoi occupanti.

\emph{\textbf{Natura Non Morta.}} La Progenie Vampirica non ha bisogno di aria.

\emph{\textbf{Rigenerazione.}} La Progenie Vampirica recupera 10 punti ferita all'inizio del suo turno se possiede almeno 1 punto ferita e non è esposto alla luce del sole o l'acqua corrente. Se la Progenie Vampirica subisce danno da Luce o danno dall'acqua sacra, questo tratto non funziona all'inizio del prossimo turno del vampiro.

\emph{\textbf{Scalare come Ragno.}} La Progenie Vampirica può scalare superfici difficili, compreso lo stare a testa in giù sul soffitto, senza bisogno di effettuare una prova di abilità.

\textbf{Azioni}

\emph{\textbf{Multiattacco.}} La progenie vampirica può effettuare due attacchi, ma solo uno di essi può essere un attacco con morso.

\emph{\textbf{Artigli.} Attacco con arma da mischia}: +6 a colpire, portata 1 m, una creatura.

\emph{Colpisce:} 8 (2d4 + 3) danni taglienti. Invece di infliggere danno, il vampiro può afferrare il bersaglio (CD per fuggire 13). 

\emph{\textbf{Morso.} Attacco con arma da mischia}: +6 a colpire, portata 1 m, una creatura afferrata dal vampiro, inabile o intralciata.

\emph{Colpisce:} 6 (1d6 + 3) danni perforanti più 7 (2d6) danni da Vuoto. I punti ferita massimi del bersaglio sono ridotti di un ammontare pari al danno da Vuoto subito, e il vampiro recupera un numero di punti ferita pari a quell'ammontare. Questa riduzione permane fino alla nuova alba. Il bersaglio muore se questo effetto riduce i suoi punti ferita massimi a 0.

\textbf{Ecologia}\\
Ambiente: Qualsiasi\\
Organizzazione: Solitario, coppia, gruppo (3-6) o turba (7-12)\\
Tesoro: Standard\\
\textbf{Descrizione}\\
Un Vampiro può decidere di creare da una vittima una progenie vampirica anziché farne un vampiro completo solo quando usa la sua capacità creare progenie su una creatura umanoide. Questa decisione deve essere presa come azione gratuita appena un vampiro uccide una creatura appropriata usando risucchio di sangue o risucchio di energia. \\


\medskip\index{Mostri - Verme Purpureo}\textbf{Verme Purpureo}

\emph{Mastodontica mostruosità, disallineato}

\textbf{FORZA} +9

\textbf{DESTREZZA} -2

\textbf{COSTITUZIONE} +6

\textbf{INTELLIGENZA} -5

\textbf{SAGGEZZA} -1

\textbf{CARISMA} -3

\textbf{Iniziativa} -2 -- \textbf{Difesa} 26

\textbf{Punti Ferita} 247 (15d20 + 90)

\textbf{Movimento} 15 m, scavo 9 m

\textbf{Tiri Salvezza}: Tempra +17, Riflessi +8, Volontà +4

\textbf{Sensi} vista cieca 9 m, senso tellurico 18 m

\textbf{Linguaggi} -

\textbf{Sfida} 15 (13.000 PE)

\emph{\textbf{Scavatore di Tunnel.}} Il verme può scavare attraverso la roccia solida a metà della velocità di scavare e lascia un tunnel di 3 metri di diametro dietro di sè.

\textbf{Azioni}

\emph{\textbf{Multiattacco.}} Il verme effettua due attacchi: uno con il morso e uno con il pungiglione.

\emph{\textbf{Morso.} Attacco con arma da mischia}: +9 a colpire,
portata 3 m, un bersaglio.

\emph{Colpisce:} 22 (3d8 + 9) danni perforanti. Se il bersaglio è una creatura di taglia Grande, deve riuscire un Tiro Salvezza di Riflessi CD 19 o venire inghiottita dal verme. Mentre è inghiottita, la creatura è accecata e intralciata, ha copertura totale contro gli attacchi e altri effetti provenienti dall'esterno del verme, e subisce 21 (6d6) danni da acido all'inizio di ciascun turno del verme.

Se il verme subisce 30 o più danni in un singolo turno da una creatura al suo interno, il verme deve riuscire un Tiro Salvezza di Tempra CD 21 al termine del suo turno o vomitare tutte le creature inghiottite, che cadono prone in uno spazio entro 3 metri dal verme. Se il verme muore, una creatura inghiottita non risulta più intralciata da esso e può fuggire dal cadavere usando 6 metri di movimento, uscendo prona.

\emph{\textbf{Pungiglione.} Attacco con arma da mischia}: +9 a colpire, portata 3 m, una creatura.

\emph{Colpisce:} 19 (3d6 + 9) danni perforanti, e il bersaglio deve effettuare un Tiro Salvezza di Tempra CD 19, subendo 42 (12d6) danni da veleno se fallisce il Tiro Salvezza, o la metà di questi danni se lo riesce.

\textbf{Ecologia}\\
Ambiente: Qualsiasi sotterraneo\\
Organizzazione: Solitario\\
Tesoro: Accidentale\\
\textbf{Descrizione}\\
I vermi purpurei sono giganteschi necrofagi che abitano nelle regioni più profonde del mondo, mangiando qualsiasi materiale organico incontrino. Sono noti per inghiottire le loro prede intere. Non è insolito sentire di un gruppo di avventurieri scomparso all'interno delle fameliche fauci di un verme purpureo, gridando di terrore mentre i suoi membri sparivano uno alla volta.\\

Mentre vanno in cerca di creature viventi per divorarle, i vermi purpurei ingoiano anche un'enorme quantità di terra e minerali scavando nel sottosuolo. Le interiora di un verme purpureo possono contenere un considerevole numero di gemme e altri oggetti in grado di resistere all'acido corrosivo all'interno del suo esofago. In zone ricche di minerali preziosi, come quelle vicine alle miniere naniche, i tunnel naturali creati dagli scavi dei vermi purpurei sono spesso pieni di un notevole numero di pepite d'oro grezzo.\\

Un verme purpureo generalmente reclama una grande caverna sotterranea come sua tana, e anche se vi torna per riposare e digerire il cibo, passa la maggior parte del suo tempo in cerca di preda, scavando attraverso l'oscurità senza fine o scivolando lungo tunnel preesistenti alla costante ricerca di cibo per saziare la sua immensa fame. Sebbene quasi privi di intelletto, i vermi purpurei raramente sono stupidi. Sono diffusi come guardiani fra chi riesce a controllarli magicamente o hanno nel loro covo una stanza abbastanza grande da ospitarli.\\


\medskip\index{Mostri - Viverna}\textbf{Viverna}

\emph{Grande drago, disallineato}

\textbf{FORZA} +4

\textbf{DESTREZZA} +0

\textbf{COSTITUZIONE} +3

\textbf{INTELLIGENZA} -3

\textbf{SAGGEZZA} +1

\textbf{CARISMA} -2

\textbf{Iniziativa} +0 -- \textbf{Difesa} 16

\textbf{Punti Ferita} 110 (13d10 + 39)

\textbf{Movimento} 6 m, volo 24 m

\textbf{Tiri Salvezza}: Tempra +9, Riflessi +6, Volontà +8

\textbf{Competenze} Consapevolezza +4

\textbf{Sensi} scurovisione 18 m

\textbf{Linguaggi} -

\textbf{Sfida} 6 (2.300 PE)

\textbf{Azioni}

\emph{\textbf{Multiattacco.}} La viverna può effettuare due attacchi: uno con il morso e uno con il pungiglione. Mentre vola, può usare i suoi artigli al posto di uno degli altri attacchi.

\emph{\textbf{Artigli.} Attacco con arma da mischia}: +7 a colpire, portata 1 m, un bersaglio.

\emph{Colpisce:} 13 (2d8 + 4) danni taglienti.

\emph{\textbf{Morso.} Attacco con arma da mischia}: +7 a colpire, portata 3 m, una creatura.

\emph{Colpisce:} 11 (2d6 + 4) danni perforanti.

\emph{\textbf{Pungiglione.} Attacco con arma da mischia}: +7 a colpire, portata 3 m, una creatura.

\emph{Colpisce:} 11 (2d6 + 4) danni perforanti. Il bersaglio deve effettuare un Tiro Salvezza di Tempra CD 15, e subire 24 (7d6) danni da veleno se lo fallisce, o la metà di questi danni se lo riesce.

\textbf{Ecologia}\\
Ambiente: Colline temperate o calde\\
Organizzazione: Solitario, coppia o stormo (3-6)\\
Tesoro: standard\\
\textbf{Descrizione}\\
Le viverne sono rettili brutali e violenti imparentati con i draghi. Sono sempre aggressive ed impazienti e preferiscono raggiungere i loro scopi utilizzando la forza. Per questa ragione, i draghi guardano alle viverne con superiorità, considerando questi loro lontani parenti come selvaggi primitivi privi di stile ed intelligenza.\\

Nella maggior parte dei casi, questa generalizzazione è azzeccata. Anche se non certo di intelletto animale e capace di parola, la maggior parte delle viverne non si cura della diplomazia, preferendo combattere prima e discutere poi, solo se si trovano davanti ad un avversario che non possono sconfiggere o da cui non possono fuggire.\\

Le viverne sono creature territoriali. Pur cacciando occasionalmente prede più grandi in gruppi più estesi, sono creature solitarie il cui territorio di caccia si estende dai 160 ai 320 km quadrati. È noto che le viverne combattono spesso fra loro fino alla morte per le contese su un territorio ricco di prede.\\


Seppur costantemente affamate ed inclini ad attaccare, una viverna può essere resa amichevole attraverso un'attenta combinazione di lusinghe, intimidazione, cibo e tesoro, per farne un potente alleato. Spesso servono Giganti e Umanoidi Mostruosi come guardiani come guardiani, ed alcune tribù di Boggard e Lucertoloidi le usano come cavalcature, anche se tali accordi spesso risultano parecchio costosi in termini di cibo ed oro, poiché sono poche le viverne che accettano di servire a lungo creature simili come cavalcature.\\

Una viverna è lunga circa 4,8 metri e la coda rappresenta da sola circa metà della lunghezza. Una viverna pesa in media 1.000 kg.\\


\medskip\index{Mostri - Wight}\textbf{Wight}

\emph{Media non morto, neutrale malvagio}

\textbf{FORZA} +2

\textbf{DESTREZZA} +2

\textbf{COSTITUZIONE} +3

\textbf{INTELLIGENZA} +0

\textbf{SAGGEZZA} +1

\textbf{CARISMA} +2

\textbf{Iniziativa} +2 -- \textbf{Difesa} 16 (armatura borchiata)

\textbf{Punti Ferita} 45 (6d8 + 18)

\textbf{Movimento} 9 m

\textbf{Tiri Salvezza}: Tempra +3, Riflessi +2, Volontà +5

\textbf{Competenze} Muoversi Silenziosamente / Nascondersi +4, Consapevolezza +3

\textbf{Resistenze al Danno} da Vuoto; da botta, perforante e tagliente di attacchi non magici o che non siano argentati

\textbf{Immunità al Danno} veleno

\textbf{Immunità alle Condizioni} avvelenato, affaticamento

\textbf{Sensi} scurovisione 18 m

\textbf{Linguaggi} le lingue che conosceva in vita

\textbf{Sfida} 3 (700 PE)

\emph{\textbf{Natura Non Morta.}} Il wight non ha bisogno di aria, cibo, bevande o sonno.

\emph{\textbf{Sensibilità alla Luce}}. Mentre è alla luce del sole, il wight ha -1d6 ai tiri di attacco, oltre che alle prove di Saggezza (Consapevolezza) basate sulla vista.

\textbf{Azioni}

\emph{\textbf{Multiattacco.}} Il wight può effettuare due attacchi con la spada lungha o due attacchi con l'arco lungo. Può usare Risucchiare Vita al posto di uno dei suoi attacchi con la spada lungha.

\emph{\textbf{Risucchiare Vita.} Attacco con arma da mischia}: +4 a colpire, portata 1 m, una creatura.

\emph{Colpisce:} 5 (1d6 + 2) danni da Vuoto. Il bersaglio deve riuscire un Tiro Salvezza di Tempra CD 13 o vedere i suoi punti ferita massimi ridotti di un ammontare pari al danno subito. Questa riduzione perdura fino al sorgere della nuova alba. Il bersaglio muore se l'effetto riduce i suoi punti ferita massimi a 0.

Un umanoide ucciso da questo attacco si rianima 24 ore più tardi come zombi sotto il controllo del wight, a meno che l'umanoide non venga prima riportato in vita o il corpo sia distrutto. Il wight non può controllare più di dodici zombi alla volta.

\emph{\textbf{Spada Lunga.} Attacco con arma da mischia}: +4 a colpire, portata 1 m, un bersaglio.

\emph{Colpisce:} 6 (1d8 + 2) danni taglienti o 7 (1d10 + 2) danni taglienti se usata con due mani.

\emph{\textbf{Arco Lungo.} Attacco con arma a Distanza}: +4 a colpire, gittata 45m, un bersaglio.

\emph{Colpisce:} 6 (1d8 + 2) danni perforanti.

\textbf{Ecologia}\\
Ambiente: qualsiasi\\
Organizzazione: Solitario, coppia, gruppo (3-6) o branco (7-12)\\
Tesoro: Standard\\
\textbf{Descrizione}\\
I wight sono umanoidi risorti come non morti a causa della necromanzia, di una morte violenta o di una personalità estremamente malevola. In alcuni casi, un wight sorge quando uno spirito non morto si lega permanentemente ad un cadavere, spesso quello di un guerriero. Sono appena riconoscibili da chi li conosceva in vita: le loro carni sono corrotte dalla malvagità e dalla non morte, gli occhi ardono d'odio ed i denti divengono quelli di una bestia. In un certo senso, un wight è l'anello di congiunzione tra ghoul e spettri: un cadavere deforme che risucchia energia vitale col tocco.\\

Essendo non morti, i wight non hanno bisogno di respirare, così a volte si possono trovare sott'acqua, sebbene non siano nuotatori particolarmente abili a meno che non siano originati da creature nuotatrici quali elfi acquatici e marinidi. Sott'acqua i wight preferiscono le caverne dal soffitto basso dove le loro scarse capacità di nuoto non sono una limitazione.\\


\medskip\index{Mostri - Wraith}\textbf{Wraith}

\emph{Media non morto, neutrale malvagio}

\textbf{FORZA} -2

\textbf{DESTREZZA} +3

\textbf{COSTITUZIONE} +3

\textbf{INTELLIGENZA} +1

\textbf{SAGGEZZA} +2

\textbf{CARISMA} +2

\textbf{Iniziativa} +3 -- \textbf{Difesa} 16

\textbf{Punti Ferita} 67 (9d8 + 27)

\textbf{Movimento} 0 m, volo 18 m (fluttua)

\textbf{Tiri Salvezza}: Tempra +6, Riflessi +4, Volontà +6

\textbf{Resistenze al Danno} acido, freddo, fulmine, fuoco, tuono; da botta, perforante e tagliente di attacchi non magici o che non siano argentati

\textbf{Immunità al Danno} da Vuoto, veleno

\textbf{Immunità alle Condizioni} affascinato, afferrato, avvelenato, intralciato, paralizzato, pietrificato, prono, affaticamento

\textbf{Sensi} scurovisione 18 m 

\textbf{Linguaggi} le lingue
che conosceva in vita 

\textbf{Sfida} 5 (1.800 PE)

\emph{\textbf{Movimento Incorporeo.}} Il wraith può attraversare creature e oggetti come fossero terreno difficile. Subisce 5 (1d10) danni da forza se termina il proprio turno all'interno di un oggetto.

\emph{\textbf{Natura Non Morta.}} Il wraith non ha bisogno di aria, cibo, bevande o sonno.

\emph{\textbf{Sensibilità alla Luce}}. Mentre è alla luce del sole, il wraith ha -1d6 ai tiri di attacco, oltre che alle prove di Saggezza (Consapevolezza) basate sulla vista.

\textbf{Azioni}

\emph{\textbf{Risucchiare Vita.} Attacco con arma da mischia}: +6 a colpire, portata 1 m, una creatura.

\emph{Colpisce:} 21 (4d8 + 3) danni da Vuoto. Il bersaglio deve riuscire un Tiro Salvezza di Tempra CD 14 o vedere i suoi punti ferita massimi ridotti di un ammontare pari al danno subito. Questa riduzione perdura fino al sorgere della nuova alba. Il bersaglio muore se l'effetto riduce i suoi punti ferita massimi a 0.

\emph{\textbf{Creare Spettro.}} Il wraith prende a bersaglio un umanoide entro 3 metri da esso e che sia morto da non più di 1 minuto e per cause violente. Lo spirito del bersaglio si anima come spettro nello spazio del suo cadavere e nello spazio più vicino non occupato. Lo spettro è sotto ilcontrollo del wraith. Il wraith non può tenere più di sette  spettri alla volta sotto il suo controllo.

\textbf{Ecologia}\\
Ambiente: Qualsiasi\\
Organizzazione: Solitario, coppia, gruppo (3-6) o branco (7-12)\\
Tesoro: Nessuno\\
\textbf{Descrizione}\\
I wraith sono creature nate dal male e dall'oscurità. Detestano la luce e le creature viventi, avendo perduto la maggior parte del legame con la loro vita precedente.\\


\medskip\index{Mostri - Xorn}\textbf{Xorn}

\emph{Media elementale, neutrale}

\textbf{FORZA} +3

\textbf{DESTREZZA} +0

\textbf{COSTITUZIONE} +6

\textbf{INTELLIGENZA} +0

\textbf{SAGGEZZA} +0

\textbf{CARISMA} +0

\textbf{Iniziativa} +0 -- \textbf{Difesa} 22

\textbf{Punti Ferita} 73 (7d8 + 42)

\textbf{Movimento} 6 m, scavo 6 m

\textbf{Tiri Salvezza}: Tempra +8, Riflessi +2, Volontà +5

\textbf{Competenze} Muoversi Silenziosamente / Nascondersi +3, Consapevolezza +6

\textbf{Resistenze al Danno} perforante e tagliente di attacchi non magici che non siano di adamantio

\textbf{Sensi} scurovisione 18 m, senso tellurico 18 m

\textbf{Linguaggi} Terran

\textbf{Sfida} 5 (1.800 PE)

\emph{\textbf{Mimetismo di Pietra.}} Lo xorn ha +1d6 alle prove di Destrezza (Nascondersi) effettuate per nascondersi su terreno roccioso.

\emph{\textbf{Scorrere sulla Terra.}} Lo xorn può scavare attraversa la terra e la pietra non magiche e non lavorate. Quando lo fa, lo xorn non disturba il materiale che sposta.

\emph{\textbf{Senso del Tesoro.}} Lo xorn può individuare precisamente, con l'olfatto, la posizione di metalli e pietre preziose, come monete e gemme, entro 18 metri da esso.

\textbf{Azioni}

\emph{\textbf{Multiattacco.}} Lo xorn effettua tre attacchi di artiglio e un attacco di morso.

\emph{\textbf{Artiglio.} Attacco con arma da mischia}: +6 a colpire, portata 1 m, un bersaglio.

\emph{Colpisce:} 6 (1d6 + 3) danni taglienti.

\emph{\textbf{Morso.} Attacco con arma da mischia}: +6 a colpire, portata 1 m, un bersaglio.

\emph{Colpisce:} 13 (3d6 + 3) danni perforanti.

\textbf{Ecologia}\\
Ambiente: Qualsiasi (Piano della Terra)\\
Organizzazione: Solitario, coppia o gruppo (3-6)\\
Tesoro: Standard (solo metalli preziosi, gemme e gioielli e gemme magiche)\\
\textbf{Descrizione}
Strane creature larghe quanto alte, gli xorn hanno poco interesse verso i nativi del Piano Materiale, non fosse per le gemme ed i metalli preziosi che potrebbero avere con sé. Nascosti sotto la superficie del terreno per un tempo che ad un umano potrebbe sembrare lunghissimo, uno xorn può attendere mesi, perfino anni, per la preda ideale, per poi assalire chi porta con sé il suo cibo preferito, come una gemma particolare o un determinato tipo di argento. Gli avventurieri che si addentrano nelle regioni abitate dagli xorn portano spesso con sé piccole pepite di minerali o gemme e cristalli di scarso valore da utilizzare come tributo. Anche se il suo valore è solitamente direttamente proporzionale al suo sapore e all'appetibilità che esso può avere, la maggior parte degli xorn è piuttosto ingorda, e preferisce la quantità alla qualità.\\

Il tesoro che uno xorn porta con sé o nasconde nella sua tana consiste in uno spuntino che ha conservato per il giorno successivo. Offrire un gioiello o un metallo preziosi particolarmente deliziosi (e costosi) ad uno xorn può cementare un'alleanza temporanea. Dato che gli xorn possono attraversare la roccia con facilità sono ottime guide nelle regioni sotterranee.\\

Gli xorn non sono molto religiosi, ma quelli fra loro che trovano la fede sono solitamente Druidi (anche se è raro, se non improbabile, che gli xorn abbiano Compagni Animali, dato che non possono seguirli nella roccia, e scelgono invece il dominio della Terra). Bardi e Devoti xorn non sono sconosciuti: i Bardi scelgono di solito Intrattenere (canto), e gli Devoti hanno invariabilmente la Stirpe Elementale (terra).\\


\medskip\index{Mostri - Zombi}\textbf{Zombi}

\emph{Media non morto, neutrale malvagio}

\textbf{FORZA} +1

\textbf{DESTREZZA} -2

\textbf{COSTITUZIONE} +3

\textbf{INTELLIGENZA} -4

\textbf{SAGGEZZA} -2

\textbf{CARISMA} -3

\textbf{Iniziativa} -2 -- \textbf{Difesa} 9

\textbf{Punti Ferita} 22 (3d8 + 9)

\textbf{Movimento} 6 m

\textbf{Tiri Salvezza}  Tempra +0, Riflessi +0, Volontà +3

\textbf{Immunità al Danno} veleno

\textbf{Immunità alle Condizioni} avvelenato

\textbf{Sensi} scurovisione 18 m

\textbf{Linguaggi} comprende tutte le lingue che parlava in vita ma non può parlare

\textbf{Sfida} 1/4 (50 PE)

\emph{\textbf{Natura Non Morta.}} Lo zombi non ha bisogno di aria, cibo, bevande o sonno.

\emph{\textbf{Tempra dei Non Morti.}} Se il danno riduce lo zombi a 0 punti ferita, lo zombi deve effettuare un Tiro Salvezza di Tempra CD 5 + il danno subito, a meno che il danno non sia da Luce o un colpo critico. Se riesce, lo zombi scende invece a 1 punto ferita.

\textbf{Azioni}

\emph{\textbf{Schianto.} Attacco con arma da mischia}: +3 a colpire, portata 1 m, un bersaglio.

\emph{Colpisce:} 4 (1d6 + 1) danni da botta.

\textbf{Ecologia}\\
Ambiente: Qualsiasi\\
Organizzazione: Qualsiasi\\
Tesoro: Nessuno\\
\textbf{Descrizione}\\
Gli zombi sono i cadaveri animati di creature morte, costretti a muoversi da magie necromantiche come Animare Morti. Anche se gli zombi incontrati di norma sono lenti e robusti, altri possiedono tratti differenti, che permettono loro di diffondere una malattia o di muoversi più rapidi.\\

Gli zombi sono automi senza mente e non possono fare altro che seguire gli ordini. Se lasciati a loro stessi, attendono immobili o si spostano alla ricerca di creature viventi da massacrare e divorare. Gli zombi attaccano fino alla distruzione, senza curarsi della loro sicurezza.\\

Sebbene siano in grado di seguire gli ordini, gli zombi vengono spesso lasciati liberi con l'ordine di uccidere tutte le creature viventi. Spesso vengono incontrati in branchi che infestano le terre frequentate dai viventi, in cerca di preda. La maggior parte degli zombi viene creata attraverso Animare Morti. Simili zombi sono sempre standard, a meno che il creatore lanci anche Velocità o Rimuovi Paralisi per creare Zombi Rapidi o Contagio per creare Zombi Infetti.\\


\medskip\index{Mostri - Zombi Ogre}\textbf{Zombi Ogre}

\emph{Grande non morto, neutrale malvagio}

\textbf{FORZA} +4

\textbf{DESTREZZA} -2

\textbf{COSTITUZIONE} +4

\textbf{INTELLIGENZA} -4

\textbf{SAGGEZZA} -2

\textbf{CARISMA} -3

\textbf{Iniziativa} -2 -- \textbf{Difesa} 9

\textbf{Punti Ferita} 85 (9d10 + 36)

\textbf{Movimento} 9 m

\textbf{Tiri Salvezza}: Tempra +6, Riflessi +0, Volontà +3

\textbf{Immunità al Danno} veleno

\textbf{Immunità alle Condizioni} avvelenato

\textbf{Sensi} scurovisione 18 m

\textbf{Linguaggi} comprende Comune e Gigante ma non può parlare

\textbf{Sfida} 2 (450 PE)

\emph{\textbf{Natura Non Morta.}} Lo zombi non ha bisogno di aria, cibo, bevande o sonno.

\emph{\textbf{Tempra dei Non Morti.}} Se il danno riduce lo zombi a 0 punti ferita, lo zombi deve effettuare un Tiro Salvezza di Tempra CD 5 + il danno subito, a meno che il danno non sia da Luce o un colpo critico. Se riesce, lo zombi scende invece a 1 punto ferita.

\textbf{Azioni}

\emph{\textbf{Mazza Chiodata.} Attacco con arma da mischia}: +6 a colpire, portata 1 m, un bersaglio.

\emph{Colpisce:} 13 (2d8 + 4) danni da botta.


\subsection{Appendice A: Creature Varie}

Questa appendice contiene le statistiche di vari animali, parassiti e
altre creature. Le statistiche sono organizzate in ordine alfabetico.

\medskip\textbf{Albero Risvegliato}\index{Mostri - Albero Risvegliato}

L'albero risvegliato è un normale albero fornito dalla magia di capacità
senziente e mobilità.

\emph{Enorme pianta, disallineato}

\textbf{FORZA} +4

\textbf{DESTREZZA} -2

\textbf{COSTITUZIONE} +2

\textbf{INTELLIGENZA} +0

\textbf{SAGGEZZA} +0

\textbf{CARISMA} -2

\textbf{Iniziativa} +0 -- \textbf{Difesa} 14

\textbf{Punti Ferita} 59 (7d12 + 14)

\textbf{Movimento} 6 m

\textbf{Tiri Salvezza}: Tempra +6, Riflessi -1, Volontà +1

\textbf{Vulnerabilità al Danno} fuoco

\textbf{Resistenze al Danno} da botta, perforante

\textbf{Lingue} una lingua conosciuta dal suo creatore

\textbf{Sfida} 2 (450 PE)

\emph{\textbf{Falso Aspetto.}} Mentre l'albero rimane immobile, è indistinguibile da un normale albero.

\textbf{Azioni}

\emph{\textbf{Schianto.} Attacco con Arma da Mischia}: +6 a colpire, portata 3 m, un bersaglio.

\emph{Colpisce:} 14 (3d6 + 4) danni da botta.

\medskip\textbf{Alce}\index{Mostri - Alce}

\emph{Grande bestia, disallineato}

\textbf{FORZA} +3

\textbf{DESTREZZA} +0

\textbf{COSTITUZIONE} +1

\textbf{INTELLIGENZA} -4

\textbf{SAGGEZZA} +0

\textbf{CARISMA} -2

\textbf{Iniziativa} +0 -- \textbf{Difesa} 11

\textbf{Punti Ferita} 13 (2d10 + 2)

\textbf{Movimento} 15 m

\textbf{Tiri Salvezza}:  Tempra +4, Riflessi +1, Volontà +0

\textbf{Lingue} -

\textbf{Sfida} 1/4 (50 PE)

\emph{\textbf{Carica.}} Se l'alce si muove di almeno 6 metri diretto verso il bersaglio e lo colpisce con un attacco di rostro durante lo stesso turno, il bersaglio subisce 7 (2d6) danni da botta aggiuntivi. Se il bersaglio è una creatura, deve riuscire un Tiro Salvezza di Tempra
CD 13 o cadere prono.

\textbf{Azioni}

\emph{\textbf{Rostro.} Attacco con Arma da Mischia}: +5 a colpire, portata 1 m, un bersaglio.

\emph{Colpisce:} 6 (1d6 + 3) danni da botta.

\emph{\textbf{Zoccoli.} Attacco con Arma da Mischia}: +5 a colpire, portata 1 m, una creatura prona.

\emph{Colpisce:} 8 (2d4 + 3) danni da botta.

\medskip\textbf{Alce Gigante}\index{Mostri - Alce Gigante}

\emph{Enorme bestia, disallineato}

\textbf{FORZA} +4

\textbf{DESTREZZA} +3

\textbf{COSTITUZIONE} +2

\textbf{INTELLIGENZA} -2

\textbf{SAGGEZZA} +2

\textbf{CARISMA} +0

\textbf{Iniziativa} +3 -- \textbf{Difesa} 15

\textbf{Punti Ferita} 42 (5d12 + 10)

\textbf{Movimento} 18 m

\textbf{Tiri Salvezza}:  Tempra +8, Riflessi +7, Volontà +2

\textbf{Competenze} Consapevolezza +4

\textbf{Lingue} Alce Gigante, comprende il Comune, l'Elfico e il

Silvano ma non può parlarli

\textbf{Sfida} 2 (450 PE)

\emph{\textbf{Carica.}} Se l'alce si muove di almeno 6 metri diretto verso il bersaglio e lo colpisce con un attacco di rostro durante lo stesso turno, il bersaglio subisce 7 (2d6) danni da botta aggiuntivi. Se il bersaglio è una creatura, deve riuscire un Tiro Salvezza di Tempra CD 14 o cadere prono.

\textbf{Azioni}

\emph{\textbf{Rostro.} Attacco con Arma da Mischia}: +6 a colpire, portata 3 m, un bersaglio.

\emph{Colpisce:} 11 (2d6 + 4) danni perforanti.

\emph{\textbf{Zoccoli.} Attacco con Arma da Mischia}: +6 a colpire, portata 1 m, una creatura prona.

\emph{Colpisce:} 22 (4d4 + 4) danni da botta.

\medskip\textbf{Aquila}\index{Mostri - Aquila}

\emph{Piccola bestia, disallineato}

\textbf{FORZA} -2

\textbf{DESTREZZA} +2

\textbf{COSTITUZIONE} +0

\textbf{INTELLIGENZA} -4

\textbf{SAGGEZZA} +2

\textbf{CARISMA} -2

\textbf{Iniziativa} +2 -- \textbf{Difesa} 13

\textbf{Punti Ferita} 3 (1d6)

\textbf{Movimento} 3 m, volo 18 m

\textbf{Tiri Salvezza}: Tempra +3, Riflessi +4, Volontà +2

\textbf{Competenze} Consapevolezza +4

\textbf{Lingue} -

\textbf{Sfida} 0 (10 PE)

\emph{\textbf{Vista Affinata.}} L'aquila ha +1d6 nelle prove di Saggezza (Consapevolezza) basate sulla vista.

\textbf{Azioni}

\emph{\textbf{Speroni.} Attacco con Arma da Mischia}: +4 a colpire, portata 1 m, un bersaglio.

\emph{Colpisce:} 4 (1d4 + 2) danni taglienti.

\medskip\textbf{Aquila Gigante}\index{Mostri - Aquila Gigante}

L'aquila gigante è una nobile creatura che parla la propria lingua e comprende quella di altre razze.

\emph{Grande bestia, neutrale buono}

\textbf{FORZA} +3

\textbf{DESTREZZA} +3

\textbf{COSTITUZIONE} +1

\textbf{INTELLIGENZA} -1

\textbf{SAGGEZZA} +2

\textbf{CARISMA} +0

\textbf{Iniziativa} +3 -- \textbf{Difesa} 14

\textbf{Punti Ferita} 26 (4d10 + 4)

\textbf{Movimento} 3 m, volo 24 m

\textbf{Tiri Salvezza}: Tempra +5, Riflessi +7, Volontà +3

\textbf{Competenze} Consapevolezza +4

\textbf{Lingue} Aquila Gigante, comprende il Comune e l'Auran ma non può parlarli

\textbf{Sfida} 1 (200 PE)

\emph{\textbf{Vista Affinata.}} L'aquila ha +1d6 nelle prove di Saggezza (Consapevolezza) basate sulla vista.

\textbf{Azioni}

\emph{\textbf{Multiattacco.}} L'aquila effettua due attacchi: uno con il becco e uno con gli speroni.

\emph{\textbf{Becco.} Attacco con Arma da Mischia}: +5 a colpire, portata 1 m, un bersaglio.

\emph{Colpisce:} 6 (1d6 + 3) danni perforanti.

\emph{\textbf{Speroni.} Attacco con Arma da Mischia}: +5 a colpire, portata 1 m, un bersaglio.

\emph{Colpisce:} 10 (2d6 + 3) danni taglienti.

\medskip\textbf{Avvoltoio}\index{Mostri - Avvoltoio}

\emph{Media bestia, disallineato}

\textbf{FORZA} -2

\textbf{DESTREZZA} +0

\textbf{COSTITUZIONE} +1

\textbf{INTELLIGENZA} -4

\textbf{SAGGEZZA} +1

\textbf{CARISMA} -3

\textbf{Iniziativa} +0 -- \textbf{Difesa} 11

\textbf{Punti Ferita} 5 (1d8 + 1)

\textbf{Movimento} 3 m, volo 15 m

\textbf{Tiri Salvezza}: Tempra +6, Riflessi +3, Volontà +1; +4 contro malattie

\textbf{Competenze} Consapevolezza +3

\textbf{Lingue} -

\textbf{Sfida} 0 (10 PE)

\emph{\textbf{Olfatto e Vista Affinati.}} L'avvoltoio ha +1d6 nelle prove di Saggezza (Consapevolezza) basate su olfatto o vista.

\emph{\textbf{Tattiche di Branco.}} L'avvoltoio ha +1d6 al tiro di attacco contro una creatura se almeno uno degli alleati dell'avvoltoio si trova entro 1 metro dalla creatura e quell'alleato non è inabile.

\textbf{Azioni}

\emph{\textbf{Becco.} Attacco con Arma da Mischia}: +2 a colpire, portata 1 m, un bersaglio.

\emph{Colpisce:} 2 (1d4) danni perforanti.

\medskip\textbf{Avvoltoio Gigante}\index{Mostri - Avvoltoio Gigante}

L'avvoltoio gigante possiede un'intelligenza superiore e un'attitudine maligna.

\emph{Grande bestia, neutrale malvagio}

\textbf{FORZA} +2

\textbf{DESTREZZA} +0

\textbf{COSTITUZIONE} +2

\textbf{INTELLIGENZA} -2

\textbf{SAGGEZZA} +1

\textbf{CARISMA} -2

\textbf{Iniziativa} +0 -- \textbf{Difesa} 11

\textbf{Punti Ferita} 22 (3d10 + 6)

\textbf{Movimento} 3 m, volo 18 m

\textbf{Tiri Salvezza}: Tempra +10, Riflessi +6, Volontà +3; +4 contro malattie

\textbf{Competenze} Consapevolezza +3

\textbf{Lingue} comprende il Comune ma non può parlare

\textbf{Sfida} 1 (200 PE)

\emph{\textbf{Olfatto e Vista Affinati.}} L'avvoltoio ha +1d6 nelle prove di Saggezza (Consapevolezza) basate su olfatto o vista.

\emph{\textbf{Tattiche di Branco.}} L'avvoltoio ha +1d6 al tiro di attacco contro una creatura se almeno uno degli alleati dell'avvoltoio si trova entro 1 metro dalla creatura e quell'alleato non è inabile.

\textbf{Azioni}

\emph{\textbf{Multiattacco.}} L'avvoltoio effettua due attacchi: uno con il becco e uno con gli speroni.

\emph{\textbf{Becco.} Attacco con Arma da Mischia}: +4 a colpire, portata 1 m, un bersaglio.

\emph{Colpisce:} 7 (2d4 + 2) danni perforanti.

\emph{\textbf{Speroni.} Attacco con Arma da Mischia}: +4 a colpire, portata 1 m, un bersaglio.

\emph{Colpisce:} 9 (2d6 + 2) danni taglienti.

\medskip\textbf{Babbuino}\index{Mostri - Babbuino}

\emph{Piccola bestia, disallineato}

\textbf{FORZA} -1

\textbf{DESTREZZA} +2

\textbf{COSTITUZIONE} +0

\textbf{INTELLIGENZA} -3

\textbf{SAGGEZZA} +1

\textbf{CARISMA} -2

\textbf{Iniziativa} +2 -- \textbf{Difesa} 13

\textbf{Punti Ferita} 3 (1d6)

\textbf{Movimento} 9 m, scalata 9 m

\textbf{Tiri Salvezza}: Tempra +3, Riflessi +4, Volontà +1

\textbf{Lingue} -

\textbf{Sfida} 0 (10 PE)

\emph{\textbf{Tattiche di Branco.}} Il babbuino ha +1d6 al tiro di attacco contro una creatura se almeno uno degli alleati del babbuino si trova entro 1 metro dalla creatura e quell'alleato non è inabile.

\textbf{Azioni}

\emph{\textbf{Morso.} Attacco con Arma da Mischia}: +1 a colpire, portata 1 m, un bersaglio.

\emph{Colpisce:} 1 (1d4 - 1) danni perforanti.


\medskip\textbf{Balena Assassina (Orca)}\index{Mostri - Orca}

\emph{Enorme bestia, disallineato}

\textbf{FORZA} +4

\textbf{DESTREZZA} +0

\textbf{COSTITUZIONE} +1

\textbf{INTELLIGENZA} -4

\textbf{SAGGEZZA} +1

\textbf{CARISMA} -2

\textbf{Iniziativa} +0 -- \textbf{Difesa} 14

\textbf{Punti Ferita} 90 (12d12 + 12)

\textbf{Movimento} 0 m, nuoto 18 m

\textbf{Tiri Salvezza}: Tempra +9, Riflessi +8, Volontà +5

\textbf{Competenze} Consapevolezza +3

\textbf{Sensi} vista cieca 36 m

\textbf{Lingue} -

\textbf{Sfida} 3 (700 PE)

\emph{\textbf{Ecolocazione.}} La balena non può usare la vista cieca se assordata.

\emph{\textbf{Trattenere il Fiato.}} La balena può trattenere il fiato per 30 minuti

\emph{\textbf{Udito Affinato.}} La balena ha +1d6 alle prove di Saggezza (Consapevolezza) basate sull'udito.

\textbf{Azioni}

\emph{\textbf{Morso.} Attacco con Arma da Mischia}: +6 a colpire, portata 1 m, un bersaglio.

\emph{Colpisce:} 21 (5d6 + 4) danni perforanti.

\medskip\textbf{Becco d'Ascia}\index{Mostri - Becco d'Ascia}

Il becco d'ascia è un grosso e slanciato volatile privo di ali ma con potenti gambe, un becco a cuneo, e un pessimo carattere.

\emph{Grande bestia, disallineato}

\textbf{FORZA} +2

\textbf{DESTREZZA} +1

\textbf{COSTITUZIONE} +1

\textbf{INTELLIGENZA} -4

\textbf{SAGGEZZA} +0

\textbf{CARISMA} -3

\textbf{Iniziativa} +1 -- \textbf{Difesa} 12

\textbf{Punti Ferita} 19 (3d10 + 3)

\textbf{Movimento} 15 m

\textbf{Tiri Salvezza}: Tempra +3, Riflessi +1, Volontà +1

\textbf{Lingue} -

\textbf{Sfida} 1/4 (50 PE)

\textbf{Azioni}

\emph{\textbf{Becco.} Attacco con Arma da Mischia}: +4 a colpire, portata 1 m, un bersaglio.

\emph{Colpisce:} 6 (1d8 + 2) danni taglienti.

\medskip\textbf{Cammello}\index{Mostri - Cammello}

\emph{Grande bestia, disallineato}

\textbf{FORZA} +3

\textbf{DESTREZZA} -1

\textbf{COSTITUZIONE} +2

\textbf{INTELLIGENZA} -4

\textbf{SAGGEZZA} -1

\textbf{CARISMA} -3

\textbf{Iniziativa} -1 -- \textbf{Difesa} 10

\textbf{Punti Ferita} 15 (2d10 + 4)

\textbf{Movimento} 15 m

\textbf{Tiri Salvezza}: Tempra +5, Riflessi +6, Volontà +0 

\textbf{Lingue} -

\textbf{Sfida} 1/8 (25 PE)

\textbf{Azioni}

\emph{\textbf{Morso.} Attacco con Arma da Mischia}: +5 a colpire, portata 1 m, un bersaglio.

\emph{Colpisce:} 2 (1d4) danni da botta.

\medskip\textbf{Cane della Morte}\index{Mostri - Cane della Morte}

Il cane della morte è un orribile segugio a due teste che si aggira per pianure, deserti e sotterranei.

\emph{Media mostruosità, neutrale malvagio}

\textbf{FORZA} +2

\textbf{DESTREZZA} +2

\textbf{COSTITUZIONE} +2

\textbf{INTELLIGENZA} -4

\textbf{SAGGEZZA} +1

\textbf{CARISMA} -2

\textbf{Iniziativa} +2 -- \textbf{Difesa} 13

\textbf{Punti Ferita} 39 (6d8 + 12)

\textbf{Movimento} 12 m

\textbf{Tiri Salvezza}: Tempra +4, Riflessi +5, Volontà +2

\textbf{Competenze} Muoversi Silenziosamente / Nascondersi +4, Consapevolezza +5

\textbf{Sensi} visione al buio 36 m

\textbf{Lingue} -

\textbf{Sfida} 1 (200 PE)

\emph{\textbf{Bicefalo.}} Il cane ha +1d6 nelle prove di Saggezza (Consapevolezza) e nei Tiri Salvezza contro le condizioni accecato, affascinato, assordato, spaventato, stordito o svenuto.

\textbf{Azioni}

\emph{\textbf{Multiattacco.}} Il cane effettua due attacchi di morso. 

\emph{\textbf{Morso.} Attacco con Arma da Mischia}: +4 a colpire, portata 1 m, un bersaglio.

\emph{Colpisce:} 5 (1d6 + 2) danni perforanti. Se il bersaglio è una creatura, deve riuscire un Tiro Salvezza di Tempra CD 12 contro la malattia o restare avvelenato finché la malattia non viene curata. Dopo ogni 24 ore, la creatura deve ripetere il Tiro Salvezza, riducendo i suoi punti ferita massimi di 5 (1d10) in caso di fallimento. Questa riduzione perdura finché la malattia non viene curata. La creatura muore se la malattia riduce i suoi punti ferita massimi a 0.

\medskip\textbf{Cane Intermittente}\index{Mostri - Cane Intermittente}

Il cane intermittente deriva il nome dalla sua abilità di entrare e uscire dalla realtà, un talento che usa per attaccare ed evitare di essere attaccato.

\emph{Media fatato, legale buono}

\textbf{FORZA} +1

\textbf{DESTREZZA} +3

\textbf{COSTITUZIONE} +1

\textbf{INTELLIGENZA} +0

\textbf{SAGGEZZA} +1

\textbf{CARISMA} +0

\textbf{Iniziativa} +3 -- \textbf{Difesa} 14

\textbf{Punti Ferita} 22 (4d8 + 4)

\textbf{Vulnerabilità al Danno} ferro freddo

\textbf{Movimento} 12 m

\textbf{Tiri Salvezza}:  Tempra +5, Riflessi +5, Volontà +4

\textbf{Competenze} Muoversi Silenziosamente / Nascondersi +5, Consapevolezza +3

\textbf{Lingue} Cane Intermittente, comprende il Silvano ma non può parlarlo

\textbf{Sfida} 1/4 (50 PE)

\emph{\textbf{Udito e Olfatto Affinato.}} Il cane ha +1d6 nelle prove di Saggezza (Consapevolezza) basate su udito o olfatto.

\textbf{Azioni}

\emph{\textbf{Morso.} Attacco con Arma da Mischia}: +3 a colpire, portata 1 m, un bersaglio.

\emph{Colpisce:} 4 (1d6 + 1) danni perforanti.

\emph{\textbf{Teletrasporto (Ricarica 4-6).}} Il cane si teletrasporta magicamente, insieme a qualsiasi cosa stia indossando o trasportando, fino a 12 metri in uno spazio non occupato che possa vedere. Prima o dopo il teletrasporto, il cane può effettuare un attacco di morso.

\medskip\textbf{Caprone}\index{Mostri - Caprone}

\emph{Media bestia, disallineato}

\textbf{FORZA} +1

\textbf{DESTREZZA} +0

\textbf{COSTITUZIONE} +0

\textbf{INTELLIGENZA} -4

\textbf{SAGGEZZA} +0

\textbf{CARISMA} -3

\textbf{Iniziativa} +0 -- \textbf{Difesa} 11

\textbf{Punti Ferita} 4 (1d8)

\textbf{Movimento} 12 m

\textbf{Tiri Salvezza}: Tempra +1, Riflessi +1, Volontà +0 

\textbf{Lingue} -

\textbf{Sfida} 0 (10 PE)

\emph{\textbf{Carica.}} Se il caprone si muove di almeno 6 metri diretto verso il bersaglio e colpisce con un attacco di rostro durante lo stesso turno, il bersaglio subisce 2 (1d4) danni da botta aggiuntivi. Se il bersaglio è una creatura, deve riuscire un Tiro Salvezza di Tempra CD 10
o cadere prona.

\emph{\textbf{Piedi Saldi.}} Il caprone ha +1d6 ai Tiri Salvezza su Tempra e Riflessi effettuati contro effetti che lo farebbero cadere prono.

\textbf{Azioni}

\emph{\textbf{Rostro.} Attacco con Arma da Mischia}: +3 a colpire, portata 1 m, un bersaglio.

\emph{Colpisce:} 3 (1d4 + 1) danni da botta.

\medskip\textbf{Caprone Gigante}\index{Mostri - Caprone Gigante}

\emph{Grande bestia, disallineato}

\textbf{FORZA} +3

\textbf{DESTREZZA} +0

\textbf{COSTITUZIONE} +1

\textbf{INTELLIGENZA} -4

\textbf{SAGGEZZA} +1

\textbf{CARISMA} -2

\textbf{Iniziativa} +0 -- \textbf{Difesa} 12

\textbf{Punti Ferita} 19 (3d10 + 3)

\textbf{Movimento} 12 m

\textbf{Tiri Salvezza}: Tempra +4, Riflessi +1, Volontà +1 

\textbf{Lingue} -

\textbf{Sfida} 1/2 (100 PE)

\emph{\textbf{Carica.}} Se il caprone si muove di almeno 6 metri diretto verso il bersaglio e colpisce con un attacco di rostro durante lo stesso turno, il bersaglio subisce 5 (2d4) danni da botta aggiuntivi. Se il bersaglio è una creatura, deve riuscire un Tiro Salvezza di Tempra CD 13 o cadere prona.

\emph{\textbf{Piedi Saldi.}} Il caprone ha +1d6 ai Tiri Salvezza su Tempra e Riflessi effettuati contro effetti che lo farebbero cadere prono.

\textbf{Azioni}

\emph{\textbf{Rostro.} Attacco con Arma da Mischia}: +5 a colpire, portata 1 m, un bersaglio.

\emph{Colpisce:} 8 (2d4 + 3) danni da botta.

\medskip\textbf{Cavallo da Corsa}\index{Mostri - Cavallo da Corsa}

\emph{Grande bestia, disallineato}

\textbf{FORZA} +3

\textbf{DESTREZZA} +0

\textbf{COSTITUZIONE} +1

\textbf{INTELLIGENZA} -4

\textbf{SAGGEZZA} +0

\textbf{CARISMA} -2

\textbf{Iniziativa} +0 -- \textbf{Difesa} 11

\textbf{Punti Ferita} 13 (2d10 + 2)

\textbf{Movimento} 18 m

\textbf{Tiri Salvezza}: Tempra +3, Riflessi +1, Volontà +1 

\textbf{Lingue} -

\textbf{Sfida} 1/4 (50 PE)

\textbf{Azioni}

\emph{\textbf{Zoccoli.} Attacco con Arma da Mischia}: +5 a colpire, portata 1 m, un bersaglio.

\emph{Colpisce:} 8 (2d4 + 3) danni da botta.

\medskip\textbf{Cavallo da Guerra}\index{Mostri - Cavallo da Guerra}

\emph{Grande bestia, disallineato}

\textbf{FORZA} +4

\textbf{DESTREZZA} +1

\textbf{COSTITUZIONE} +1

\textbf{INTELLIGENZA} -4

\textbf{SAGGEZZA} +1

\textbf{CARISMA} -2

\textbf{Iniziativa} +1 -- \textbf{Difesa} 12 (più possibile bardatura)

\textbf{Punti Ferita} 19 (3d10 + 3)

\textbf{Movimento} 18 m

\textbf{Tiri Salvezza}:  Tempra +4, Riflessi +2, Volontà +1 

\textbf{Lingue} -

\textbf{Sfida} 1/2 (100 PE)

\emph{\textbf{Carica Travolgente.}} Se il cavallo si muove di almeno 6 metri diretto verso il bersaglio e lo colpisce con un attacco di zoccoli durante lo stesso turno, il bersaglio deve riuscire un Tiro Salvezza su Tempra CD 14 o cadere prono. Se il bersaglio è prono, il cavallo può effettuare un altro attacco di zoccoli contro di lui come azione bonus.

\textbf{Azioni}

\emph{\textbf{Zoccoli.} Attacco con Arma da Mischia}: +6 a colpire, portata 1 m, un bersaglio.

\emph{Colpisce:} 11 (2d6 + 4) danni da botta.

\medskip\textbf{Cavallo da Tiro}\index{Mostri - Cavallo da Tiro}

\emph{Grande bestia, disallineato}

\textbf{FORZA} +4

\textbf{DESTREZZA} +0

\textbf{COSTITUZIONE} +1

\textbf{INTELLIGENZA} -4

\textbf{SAGGEZZA} +0

\textbf{CARISMA} -2

\textbf{Iniziativa} +0 -- \textbf{Difesa} 11

\textbf{Punti Ferita} 19 (3d10 + 3)

\textbf{Movimento} 12 m

\textbf{Tiri Salvezza}:  Tempra +5, Riflessi +1, Volontà +2 

\textbf{Lingue} -

\textbf{Sfida} 1/4 (50 PE)

\textbf{Azioni}

\emph{\textbf{Zoccoli.} Attacco con Arma da Mischia}: +6 a colpire, portata 1 m, un bersaglio.

\emph{Colpisce:} 9 (2d4 + 4) danni da botta.

\medskip\textbf{Cavallo Marino Gigante}\index{Mostri - Cavallo Marino Gigante}

Il cavallo marino gigante viene spesso impiegato come cavalcatura dagli umanoidi acquatici.

\emph{Grande bestia, disallineato}

\textbf{FORZA} +1

\textbf{DESTREZZA} +2

\textbf{COSTITUZIONE} +0

\textbf{INTELLIGENZA} -4

\textbf{SAGGEZZA} +1

\textbf{CARISMA} -3

\textbf{Iniziativa} +2 -- \textbf{Difesa} 14

\textbf{Punti Ferita} 16 (3d10)

\textbf{Movimento} 0 m, nuoto 12 m

\textbf{Tiri Salvezza}: Tempra +2, Riflessi +3, Volontà +1

\textbf{Lingue} -

\textbf{Sfida} 1/2 (100 PE)

\emph{\textbf{Carica.}} Se il cavallo marino si muove di almeno 6 metri diretto verso il bersaglio e colpisce con un attacco di rostro durante lo stesso turno, il bersaglio subisce 7 (2d6) danni da botta aggiuntivi. Se il bersaglio è una creatura, deve riuscire un Tiro Salvezza su Tempra CD 11 o cadere prona.

\emph{\textbf{Respirare Acqua.}} Il cavallo marino può respirare solo sottacqua.

\textbf{Azioni}

\emph{\textbf{Rostro.} Attacco con Arma da Mischia}: +3 a colpire, portata 1 m, un bersaglio.

\emph{Colpisce:} 4 (1d6 + 1) danni da botta.

\medskip\textbf{Centopiedi Gigante}\index{Centopiedi Gigante}

\emph{Piccola bestia, disallineato}

\textbf{FORZA} -3

\textbf{DESTREZZA} +2

\textbf{COSTITUZIONE} +1

\textbf{INTELLIGENZA} -5

\textbf{SAGGEZZA} -2

\textbf{CARISMA} -4

\textbf{Iniziativa} +2 -- \textbf{Difesa} 14

\textbf{Punti Ferita} 4 (1d6 + 1)

\textbf{Movimento} 9 m, scalata 9 m

\textbf{Tiri Salvezza}: Tempra -2, Riflessi +3, Volontà -2 

\textbf{Sensi} vista cieca 9 m

\textbf{Lingue} -

\textbf{Sfida} 1/4 (50 PE)

\textbf{Azioni}

\emph{\textbf{Morso.} Attacco con Arma da Mischia}: +4 a colpire, portata 1 m, una creatura.

\emph{Colpisce:} 4 (1d4 + 2) danni perforanti e il bersaglio deve riuscire un Tiro Salvezza di Tempra CD 11 o subire 10 (3d6) danni da veleno. Se il danno da veleno riduce il bersaglio a 0 punti ferita, il bersaglio è stabile ma resta avvelenato per 1 ora, anche dopo aver recuperato i punti ferita, e mentre è avvelenato in questo modo resta paralizzato.

\medskip\textbf{Cervo}\index{Mostri - Cervo}

\emph{Media bestia, disallineato}

\textbf{FORZA} +0

\textbf{DESTREZZA} +3

\textbf{COSTITUZIONE} +0

\textbf{INTELLIGENZA} -4

\textbf{SAGGEZZA} +2

\textbf{CARISMA} -3

\textbf{Iniziativa} +3 -- \textbf{Difesa} 14

\textbf{Punti Ferita} 4 (1d8)

\textbf{Movimento} 15 m

\textbf{Tiri Salvezza}: Tempra +2, Riflessi +3, Volontà +2 

\textbf{Lingue} -

\textbf{Sfida} 0 (10 PE)

\textbf{Azioni}

\emph{\textbf{Morso.} Attacco con Arma da Mischia}: +2 a colpire, portata 1 m, un bersaglio.

\emph{Colpisce:} 2 (1d4) danni perforanti.

\medskip\textbf{Cinghiale}\index{Mostri - Cinghiale}

\emph{Media bestia, disallineato}

\textbf{FORZA} +1

\textbf{DESTREZZA} +0

\textbf{COSTITUZIONE} +1

\textbf{INTELLIGENZA} -4

\textbf{SAGGEZZA} -1

\textbf{CARISMA} -3

\textbf{Iniziativa} +0 -- \textbf{Difesa} 12

\textbf{Punti Ferita} 11 (2d8 + 2)

\textbf{Movimento} 12 m

\textbf{Tiri Salvezza}: Tempra +2, Riflessi +1, Volontà -1 

\textbf{Lingue} -

\textbf{Sfida} 1/4 (50 PE)

\emph{\textbf{Carica.}} Se il cinghiale si muove di almeno 6 metri diretto verso il bersaglio e colpisce con un attacco di zanna durante lo stesso turno, il bersaglio subisce 3 (1d6) danni taglienti aggiuntivi. Se il bersaglio è una creatura, deve riuscire un Tiro Salvezza di Tempra
CD 11 o cadere prono.

\emph{\textbf{Implacabile (Ricarica dopo un 1 ora).}} Se il cinghiale subisce 7 danni o meno che lo ridurrebbero a 0 punti ferita, scende invece a 1 punto ferita.

\textbf{Azioni}

\emph{\textbf{Zanna.} Attacco con Arma da Mischia}: +3 a colpire, portata 1 m, un bersaglio.

\emph{Colpisce:} 4 (1d6 + 1) danni taglienti.

\medskip\textbf{Cinghiale Gigante}\index{Mostri - Cinghiale Gigante}

\emph{Grande bestia, disallineato}

\textbf{FORZA} +3

\textbf{DESTREZZA} +0

\textbf{COSTITUZIONE} +3

\textbf{INTELLIGENZA} -4

\textbf{SAGGEZZA} -2

\textbf{CARISMA} -3

\textbf{Iniziativa} +0 -- \textbf{Difesa} 13

\textbf{Punti Ferita} 42 (5d10 + 15)

\textbf{Movimento} 12 m

\textbf{Tiri Salvezza}: Tempra +4, Riflessi +2, Volontà +0

\textbf{Lingue} -

\textbf{Sfida} 2 (450 PE)

\emph{\textbf{Carica.}} Se il cinghiale si muove di almeno 6 metri diretto verso il bersaglio e colpisce con un attacco di zanna durante lo stesso turno, il bersaglio subisce 7 (2d6) danni taglienti aggiuntivi. Se il bersaglio è una creatura, deve riuscire un Tiro Salvezza di Tempra CD 13 o cadere prono.

\emph{\textbf{Implacabile (Ricarica dopo un 1 ora).}} Se il cinghiale subisce 10 danni o meno che lo ridurrebbero a 0 punti ferita, scende invece a 1 punto ferita.

\textbf{Azioni}

\emph{\textbf{Zanna.} Attacco con Arma da Mischia}: +5 a colpire, portata 1 m, un bersaglio.

\emph{Colpisce:} 10 (2d6 + 3) danni taglienti.

\medskip\textbf{Coccodrillo}\index{Mostri - Coccodrillo}

\emph{Grande bestia, disallineato}

\textbf{FORZA} +2

\textbf{DESTREZZA} +0

\textbf{COSTITUZIONE} +1

\textbf{INTELLIGENZA} -4

\textbf{SAGGEZZA} +0

\textbf{CARISMA} -3

\textbf{Iniziativa} +0 -- \textbf{Difesa} 13

\textbf{Punti Ferita} 19 (3d10 + 3)

\textbf{Movimento} 6 m, nuoto 9 m

\textbf{Tiri Salvezza}: Tempra +6, Riflessi +4, Volontà +2 

\textbf{Competenze} Muoversi Silenziosamente / Nascondersi +2

\textbf{Lingue} -

\textbf{Sfida} 1/2 (100 PE)

\emph{\textbf{Trattenere il Fiato.}} Il coccodrillo può trattenere il fiato per 15 minuti.

\textbf{Azioni}

\emph{\textbf{Morso.} Attacco con Arma da Mischia}: +4 a colpire, portata 1 m, una creatura.

\emph{Colpisce:} 7 (1d10 + 2) danni perforanti, e il bersaglio è afferrato (CD 12 per fuggire). Fino al termine dell'afferrare, il bersaglio è intralciato, e il coccodrillo non può usare il morso contro un altro bersaglio.

\medskip\textbf{Coccodrillo Gigante}\index{Mostri - Coccodrillo Gigante}

\emph{Enorme bestia, disallineato}

\textbf{FORZA} +5

\textbf{DESTREZZA} -1

\textbf{COSTITUZIONE} +3

\textbf{INTELLIGENZA} -4

\textbf{SAGGEZZA} +0

\textbf{CARISMA} -2

\textbf{Iniziativa} -1 -- \textbf{Difesa} 15

\textbf{Punti Ferita} 85 (9d12 + 27)

\textbf{Movimento} 9 m, nuoto 15 m

\textbf{Tiri Salvezza}: Tempra +15, Riflessi +8, Volontà +8

\textbf{Competenze} Muoversi Silenziosamente / Nascondersi +5

\textbf{Lingue} -

\textbf{Sfida} 5 (1.800 PE)

\emph{\textbf{Trattenere il Fiato.}} Il coccodrillo può trattenere il fiato per 30 minuti.

\textbf{Azioni}

\emph{\textbf{Multiattacco.}} Il coccodrillo effettua due attacchi: uno con il morso e uno con la coda.

\emph{\textbf{Coda.} Attacco con Arma da Mischia}: +8 a colpire, portata 3 m, un bersaglio non afferrato dal coccodrillo.

\emph{Colpisce:} 14 (2d8 + 5) danni da botta. Se il bersaglio è una creatura, deve riuscire un Tiro Salvezza di Tempra CD 16 o cadere prono.

\emph{\textbf{Morso.} Attacco con Arma da Mischia}: +8 a colpire, portata 1 m, un bersaglio.

\emph{Colpisce:} 21 (3d10 + 5) danni perforanti, e il bersaglio è afferrato (CD 16 per fuggire). Fino al termine dell'afferrare, il bersaglio è intralciato, e il coccodrillo non può usare il morso contro un altro bersaglio.

\medskip\textbf{Corvo}\index{Mostri - Corvo}

\emph{Minuscola bestia, disallineato}

\textbf{FORZA} -4

\textbf{DESTREZZA} +2

\textbf{COSTITUZIONE} -1

\textbf{INTELLIGENZA} -4

\textbf{SAGGEZZA} +1

\textbf{CARISMA} -2

\textbf{Iniziativa} +2 -- \textbf{Difesa} 13

\textbf{Punti Ferita} 1 (1d4 - 1)

\textbf{Movimento} 3 m, volo 15 m

\textbf{Tiri Salvezza}: Tempra +1, Riflessi +4, Volontà +2

\textbf{Competenze} Consapevolezza +3

\textbf{Lingue} -

\textbf{Sfida} 0 (10 PE)

\emph{\textbf{Imitazione.}} Il corvo può imitare dei semplici suoni che ha udito, come il sussurro di una persona, il pianto di un bambino o il verso di un animale. Una creatura che ode il suono può identificarlo come imitazione riuscendo una prova di Saggezza (Sopravvivenza) CD 10.

\textbf{Azioni}

\emph{\textbf{Becco.} Attacco con Arma da Mischia}: +4 a colpire, portata 1 m, un bersaglio.

\emph{Colpisce:} 1 danno perforante.

\medskip\textbf{Donnola}\index{Mostri - Donnola}

\emph{Minuscola bestia, disallineato}

\textbf{FORZA} -4

\textbf{DESTREZZA} +3

\textbf{COSTITUZIONE} -1

\textbf{INTELLIGENZA} -4

\textbf{SAGGEZZA} +1

\textbf{CARISMA} -4

\textbf{Iniziativa} +3 -- \textbf{Difesa} 14

\textbf{Punti Ferita} 1 (1d4 - 1)

\textbf{Movimento} 9 m

\textbf{Tiri Salvezza}: Tempra +2, Riflessi +4, Volontà +1

\textbf{Competenze} Muoversi Silenziosamente / Nascondersi +5, Consapevolezza +3

\textbf{Lingue} -

\textbf{Sfida} 0 (10 PE)

\emph{\textbf{Udito e Olfatto Affinati.}} La donnola ha +1d6 nelle prove di Saggezza (Consapevolezza) basate su udito o olfatto.

\textbf{Azioni}

\emph{\textbf{Morso.} Attacco con Arma da Mischia}: +5 a colpire, portata 1 m, un bersaglio.

\emph{Colpisce:} 1 danno perforante.

\medskip\textbf{Donnola Gigante}\index{Mostri - Donnola Gigante}

\emph{Media bestia, disallineato}

\textbf{FORZA} +0

\textbf{DESTREZZA} +3

\textbf{COSTITUZIONE} +0

\textbf{INTELLIGENZA} -3

\textbf{SAGGEZZA} +1

\textbf{CARISMA} -3

\textbf{Iniziativa} +3 -- \textbf{Difesa} 14

\textbf{Punti Ferita} 9 (2d8)

\textbf{Movimento} 12 m

\textbf{Tiri Salvezza}:  Tempra +6, Riflessi +7, Volontà +2 

\textbf{Competenze} Muoversi Silenziosamente / Nascondersi +5, Consapevolezza +3

\textbf{Sensi} visione al buio 18 m

\textbf{Lingue} -

\textbf{Sfida} 1/8 (25 PE)

\emph{\textbf{Udito e Olfatto Affinati.}} La donnola ha +1d6 nelle prove di Saggezza (Consapevolezza) basate su udito o olfatto.

\textbf{Azioni}

\emph{\textbf{Morso.} Attacco con Arma da Mischia}: +5 a colpire, portata 1 m, un bersaglio.

\emph{Colpisce:} 5 (1d4 + 3) danni perforanti.

\medskip\textbf{Elefante}\index{Mostri - Elefante}

\emph{Enorme bestia, disallineato}

\textbf{FORZA} +6

\textbf{DESTREZZA} -1

\textbf{COSTITUZIONE} +3

\textbf{INTELLIGENZA} -4

\textbf{SAGGEZZA} +0

\textbf{CARISMA} -2

\textbf{Iniziativa} -1 -- \textbf{Difesa} 14

\textbf{Punti Ferita} 76 (8d12 + 24)

\textbf{Movimento} 12 m

\textbf{Tiri Salvezza}: Tempra +13, Riflessi +7, Volontà +6 

\textbf{Lingue} -

\textbf{Sfida} 4 (1.000 PE)

\emph{\textbf{Carica Travolgente.}} Se l'elefante si muove di almeno 6 metri diretto verso una creatura e la colpisce con un attacco di incornata durante lo stesso turno, il bersaglio deve riuscire un Tiro Salvezza su Tempra CD 12 o cadere prono. Se il bersaglio è prono, l'elefante può effettuare un attacco di pestone contro di lui come azione bonus.

\textbf{Azioni}

\emph{\textbf{Incornata.} Attacco con Arma da Mischia}: +8 a colpire, portata 1 m, un bersaglio.

\emph{Colpisce:} 19 (3d8 + 6) danni perforanti. 

\emph{\textbf{Pestone.} Attacco con Arma da Mischia}: +8 a colpire, portata 1 m, un bersaglio prono.

\emph{Colpisce:} 22 (3d10 + 6) danni da botta.

\medskip\textbf{Falco}\index{Mostri - Falco}

\emph{Minuscola bestia, disallineato}

\textbf{FORZA} -3

\textbf{DESTREZZA} +3

\textbf{COSTITUZIONE} -1

\textbf{INTELLIGENZA} -4

\textbf{SAGGEZZA} +2

\textbf{CARISMA} -2

\textbf{Iniziativa} +3 -- \textbf{Difesa} 14

\textbf{Punti Ferita} 1 (1d4 - 1)

\textbf{Movimento} 3 m, volo 18 m

\textbf{Tiri Salvezza}: Tempra +2, Riflessi +5, Volontà +2 

\textbf{Competenze} Consapevolezza +4

\textbf{Lingue} -

\textbf{Sfida} 0 (10 PE)

\emph{\textbf{Vista Affinata.}} Il falco ha +1d6 alle prove di Saggezza (Consapevolezza) basate sulla vista.

\textbf{Azioni}

\emph{\textbf{Speroni.} Attacco con Arma da Mischia}: +5 a colpire, portata 1 m, un bersaglio.

\emph{Colpisce:} 1 danno tagliente.

\medskip\textbf{Falco di Sangue}\index{Mostri - Falco di Sangue}

Dovendo il suo nome alle sue piume cremisi e alla sua natura aggressiva, il falco di sangue attacca senza timore usando il suo becco appuntito.

\emph{Piccola bestia, disallineato}

\textbf{FORZA} -2

\textbf{DESTREZZA} +2

\textbf{COSTITUZIONE} +0

\textbf{INTELLIGENZA} -4

\textbf{SAGGEZZA} +2

\textbf{CARISMA} -3

\textbf{Iniziativa} +2 -- \textbf{Difesa} 13

\textbf{Punti Ferita} 7 (2d6)

\textbf{Movimento} 3 m, volo 18 m

\textbf{Tiri Salvezza}: Tempra +3, Riflessi +6, Volontà +3 

\textbf{Competenze} Consapevolezza +4

\textbf{Lingue} -

\textbf{Sfida} 1/8 (25 PE)

\emph{\textbf{Tattiche di Branco.}} Il falco ha +1d6 ai tiri di attacco contro una creatura se almeno uno degli alleati del falco si trova entro 1 metro dalla creatura e quell'alleato non è inabile.

\emph{\textbf{Vista Affinata.}} Il falco ha +1d6 alle prove di Saggezza (Consapevolezza) basate sulla vista.

\textbf{Azioni}

\emph{\textbf{Becco.} Attacco con Arma da Mischia}: +4 a colpire, portata 1 m, un bersaglio.

\emph{Colpisce:} 4 (1d4 + 2) danni perforanti.

\medskip\textbf{Pirana}\index{Mostri - Pirana}

Il pirana è un pesce carnivoro dai denti affilati.

\emph{Minuscola bestia, disallineato}

\textbf{FORZA} -4

\textbf{DESTREZZA} +3

\textbf{COSTITUZIONE} -1

\textbf{INTELLIGENZA} -5

\textbf{SAGGEZZA} -2

\textbf{CARISMA} -4

\textbf{Iniziativa} +3 -- \textbf{Difesa} 14

\textbf{Punti Ferita} 1 (1d4 - 1)

\textbf{Movimento} 0 m, nuoto 12 m

\textbf{Tiri Salvezza}: Tempra -4, Riflessi +3, Volontà -2 

\textbf{Sensi} visione al buio 18 m

\textbf{Lingue} -

\textbf{Sfida} 0 (10 PE)

\emph{\textbf{Frenesia Sanguinaria.}} Il pirana ha +1d6 ai tiri di attacco in mischia contro qualsiasi creatura che non sia al massimo dei punti ferita.

\emph{\textbf{Respirare Acqua.}} Il pirana può respirare solo sottacqua. 

\textbf{Azioni}

\emph{\textbf{Morso.} Attacco con Arma da Mischia}: +5 a colpire, portata 1 m, un bersaglio.

\emph{Colpisce:} 1 danno perforante.

\medskip\textbf{Gatto}\index{Mostri - Gatto}

\emph{Minuscola bestia, disallineato}

\textbf{FORZA} -4

\textbf{DESTREZZA} +2

\textbf{COSTITUZIONE} +0

\textbf{INTELLIGENZA} -4

\textbf{SAGGEZZA} +1

\textbf{CARISMA} -2

\textbf{Iniziativa} +2 -- \textbf{Difesa} 13

\textbf{Punti Ferita} 2 (1d4)

\textbf{Movimento} 12 m, scalata 9 m

\textbf{Tiri Salvezza}:  Tempra +1, Riflessi +4, Volontà +1

\textbf{Competenze} Muoversi Silenziosamente / Nascondersi +4, Consapevolezza +3

\textbf{Lingue} -

\textbf{Sfida} 0 (10 PE)

\emph{\textbf{Olfatto Affinato.}} Il gatto ha +1d6 alle prove di Saggezza (Consapevolezza) basate sull'olfatto.

\textbf{Azioni}

\emph{\textbf{Artigli.} Attacco con Arma da Mischia}: +0 a colpire, portata 1 m, un bersaglio.

\emph{Colpisce:} 1 danno tagliente.

\medskip\textbf{Granchio Gigante}\index{Mostri - Granchio Gigante}

\emph{Media bestia, disallineato}

\textbf{FORZA} +1

\textbf{DESTREZZA} +2

\textbf{COSTITUZIONE} +0

\textbf{INTELLIGENZA} -5

\textbf{SAGGEZZA} -1

\textbf{CARISMA} -4

\textbf{Iniziativa} +2 -- \textbf{Difesa} 16

\textbf{Punti Ferita} 13 (3d8)

\textbf{Movimento} 9 m, nuoto 9 m

\textbf{Tiri Salvezza}: Tempra +5, Riflessi +2, Volontà +1

\textbf{Competenze} Muoversi Silenziosamente / Nascondersi +4

\textbf{Sensi} vista cieca 9 m

\textbf{Lingue} -

\textbf{Sfida} 1/8 (25 PE)

\emph{\textbf{Anfibio.}} Il granchio può respirare aria e acqua.

\textbf{Azioni}

\emph{\textbf{Artiglio (Chela).} Attacco con Arma da Mischia}: +3 a colpire, portata 1 m, un bersaglio.

\emph{Colpisce:} 4 (1d6 + 1) danni da botta e il bersaglio è afferrato (CD 11 per fuggire). Il granchio ha due chele, ciascuna delle quali può afferrare un solo bersaglio.

\medskip\textbf{Gufo}\index{Mostri - Gufo}

\emph{Minuscola bestia, disallineato}

\textbf{FORZA} -4

\textbf{DESTREZZA} +1

\textbf{COSTITUZIONE} -1

\textbf{INTELLIGENZA} -4

\textbf{SAGGEZZA} +1

\textbf{CARISMA} -2

\textbf{Iniziativa} +1 -- \textbf{Difesa} 12

\textbf{Punti Ferita} 1 (1d4 - 1)

\textbf{Movimento} 1 m, volo 18 m

\textbf{Tiri Salvezza}: Tempra +2, Riflessi +5, Volontà +2 

\textbf{Competenze} Muoversi Silenziosamente / Nascondersi +3, Consapevolezza +3

\textbf{Sensi} visione al buio 36 m

\textbf{Lingue} -

\textbf{Sfida} 0 (10 PE)

\emph{\textbf{Sorvolare.}} Il gufo non provoca attacchi di opportunità quando vola via dalla portata di un nemico.

\emph{\textbf{Udito e Vista Affinati.}} Il gufo ha +1d6 nelle prove di Saggezza (Consapevolezza) basate su udito o vista.

\textbf{Azioni}

\emph{\textbf{Speroni.} Attacco con Arma da Mischia}: +3 a colpire, portata 1 m, un bersaglio.

\emph{Colpisce:} 1 danno tagliente.

\medskip\textbf{Gufo Gigante}\index{Mostri - Gufo Gigante}

I gufi giganti sono creature intelligenti che proteggono i regni silvani.

\emph{Grande bestia, neutrale}

\textbf{FORZA} +1

\textbf{DESTREZZA} +2

\textbf{COSTITUZIONE} +1

\textbf{INTELLIGENZA} -1

\textbf{SAGGEZZA} +1

\textbf{CARISMA} +0

\textbf{Iniziativa} +2 -- \textbf{Difesa} 13

\textbf{Punti Ferita} 19 (3d10 + 3)

\textbf{Movimento} 1 m, volo 18 m

\textbf{Tiri Salvezza}: Tempra +1, Riflessi +4, Volontà +1 

\textbf{Competenze} Muoversi Silenziosamente / Nascondersi +4, Consapevolezza +5

\textbf{Sensi} visione al buio 36 m

\textbf{Lingue} Gufo Gigante, comprende Comune, Elfico e Silvano ma non può parlarli

\textbf{Sfida} 1/4 (50 PE)

\emph{\textbf{Sorvolare.}} Il gufo non provoca attacchi di opportunità quando vola via dalla portata di un nemico.

\emph{\textbf{Udito e Vista Affinati.}} Il gufo ha +1d6 nelle prove di Saggezza (Consapevolezza) basate su udito o vista.

\textbf{Azioni}

\emph{\textbf{Speroni.} Attacco con Arma da Mischia}: +3 a colpire, portata 1 m, un bersaglio.

\emph{Colpisce:} 8 (2d6 + 1) danni perforanti.

\medskip\textbf{Iena}\index{Mostri - Iena}

\emph{Media bestia, disallineato}

\textbf{FORZA} +0

\textbf{DESTREZZA} +1

\textbf{COSTITUZIONE} +1

\textbf{INTELLIGENZA} -4

\textbf{SAGGEZZA} +1

\textbf{CARISMA} -3

\textbf{Iniziativa} +1 -- \textbf{Difesa} 12

\textbf{Punti Ferita} 5 (1d8 + 1)

\textbf{Movimento} 15 m

\textbf{Tiri Salvezza}: Tempra +5, Riflessi +5, Volontà +1

\textbf{Competenze} Consapevolezza +3

\textbf{Lingue} -

\textbf{Sfida} 0 (10 PE)

\emph{\textbf{Tattiche di Branco.}} La iena ha +1d6 ai tiri di attacco contro una creatura se almeno uno degli alleati della iena si trova entro 1 metro dalla creatura e quell'alleato non è inabile.

\textbf{Azioni}

\emph{\textbf{Morso.} Attacco con Arma da Mischia}: +2 a colpire, portata 1 m, un bersaglio.

\emph{Colpisce:} 3 (1d6) danni perforanti.

\medskip\textbf{Iena Gigante}\index{Mostri - Iena Gigante}

\emph{Grande bestia, disallineato}

\textbf{FORZA} +3

\textbf{DESTREZZA} +2

\textbf{COSTITUZIONE} +2

\textbf{INTELLIGENZA} -4

\textbf{SAGGEZZA} +1

\textbf{CARISMA} -2

\textbf{Iniziativa} +2 -- \textbf{Difesa} 13

\textbf{Punti Ferita} 45 (6d10 + 12)

\textbf{Movimento} 15 m

\textbf{Tiri Salvezza}: Tempra +6, Riflessi +6, Volontà +2 

\textbf{Competenze} Consapevolezza +3

\textbf{Lingue} -

\textbf{Sfida} 1 (200 PE)

\emph{\textbf{Rabbia.}} Quando la iena riduce una creatura a 0 punti ferita con un attacco di mischia durante il proprio turno, la iena può svolgere un'azione bonus per muoversi fino a metà del suo movimento effettuare un attacco di morso.

\textbf{Azioni}

\emph{\textbf{Morso.} Attacco con Arma da Mischia}: +5 a colpire, portata 1 m, un bersaglio.

\emph{Colpisce:} 10 (2d6 + 3) danni perforanti.

\medskip\textbf{Leone}\index{Mostri - Leone}

\emph{Grande bestia, disallineato}

\textbf{FORZA} +3

\textbf{DESTREZZA} +2

\textbf{COSTITUZIONE} +1

\textbf{INTELLIGENZA} -4

\textbf{SAGGEZZA} +1

\textbf{CARISMA} -1

\textbf{Iniziativa} +2 -- \textbf{Difesa} 13

\textbf{Punti Ferita} 26 (4d10 + 4)

\textbf{Movimento} 15 m

\textbf{Tiri Salvezza}: Tempra +6, Riflessi +7, Volontà +2 

\textbf{Competenze} Muoversi Silenziosamente / Nascondersi +6, Consapevolezza +3

\textbf{Lingue} -

\textbf{Sfida} 1 (200 PE)

\emph{\textbf{Balzo.}} Se il leone si muove di almeno 6 metri diretto verso una creatura e la colpisce con un attacco di artiglio durante lo stesso turno, il bersaglio deve riuscire un Tiro Salvezza di Tempra CD 13 o cadere prono. Se il bersaglio è prono, il leone può effettuare un
attacco di morso come azione bonus.

\emph{\textbf{Olfatto Affinato.}} Il leone ha +1d6 alle prove di Saggezza (Consapevolezza) basate sull'olfatto.

\emph{\textbf{Salto con Rincorsa.}} Con 3 metri di rincorsa, il leone può saltare in lungo fino a 7 metri.

\emph{\textbf{Tattiche di Branco.}} Il leone ha +1d6 ai tiri di attacco contro una creatura se almeno uno degli alleati del leone si trova entro 1 metro dalla creatura e quell'alleato non è inabile.

\textbf{Azioni}

\emph{\textbf{Artiglio.} Attacco con Arma da Mischia}: +5 a colpire, portata 1 m, un bersaglio.

\emph{Colpisce:} 6 (1d6 + 3) danni taglienti. 

\emph{\textbf{Morso.} Attacco con Arma da Mischia}: +5 a colpire, portata 1 m, un bersaglio.

\emph{Colpisce:} 7 (1d8 + 3) danni perforanti.

\medskip\textbf{Lucertola}\index{Mostri - Lucertola}

\emph{Minuscola bestia, disallineato}

\textbf{FORZA} -4

\textbf{DESTREZZA} +0

\textbf{COSTITUZIONE} +0

\textbf{INTELLIGENZA} -5

\textbf{SAGGEZZA} -1

\textbf{CARISMA} -4

\textbf{Iniziativa} +0 -- \textbf{Difesa} 11

\textbf{Punti Ferita} 2 (1d4)

\textbf{Movimento} 6 m, scalata 6 m

\textbf{Tiri Salvezza}:  Tempra +1, Riflessi +4, Volontà +1 

\textbf{Sensi} visione al buio 9 m

\textbf{Lingue} -

\textbf{Sfida} 0 (10 PE)

\emph{\textbf{Scalare come Ragno.}} La lucertola può scalare superfici difficili, compreso lo stare a testa in giù sul soffitto, senza bisogno di effettuare una prova di abilità.

\textbf{Azioni}

\emph{\textbf{Morso.} Attacco con Arma da Mischia}: +0 a colpire, portata 1 m, un bersaglio.

\emph{Colpisce:} 1 danno perforante.

\medskip\textbf{Lucertola Gigante}\index{Mostri - Lucertola Gigante}

Le lucertole giganti sono temibili predatori e spesso vengono impiegate come cavalcature o animali da tiro da umanoidi rettiloidi e residenti del sottosuolo.

\emph{Grande bestia, disallineato}

\textbf{FORZA} +2

\textbf{DESTREZZA} +1

\textbf{COSTITUZIONE} +1

\textbf{INTELLIGENZA} -4

\textbf{SAGGEZZA} +0

\textbf{CARISMA} -3

\textbf{Iniziativa} +1 -- \textbf{Difesa} 13

\textbf{Punti Ferita} 19 (3d10 + 3)

\textbf{Movimento} 9 m, scalata 9 m

\textbf{Tiri Salvezza}: Tempra +11, Riflessi +8, Volontà +4 

\textbf{Sensi} visione al buio 9 m

\textbf{Lingue} -

\textbf{Sfida} 1/4 (50 PE)

\textbf{Azioni}

\emph{\textbf{Morso.} Attacco con Arma da Mischia}: +4 a colpire, portata 1 m, un bersaglio.

\emph{Colpisce:} 6 (1d8 + 2) danni perforanti.

\textbf{VARIANTE}

Alcune lucertole giganti possiedono uno o entrambi i seguenti tratti.

\emph{\textbf{Scalare come Ragno.}} La lucertola può scalare superfici difficili, compreso lo stare a testa in giù sul soffitto, senza bisogno di effettuare una prova di abilità. 

\emph{\textbf{Trattenere il Fiato.}} La lucertola può trattenere il fiato per 15 minuti. (Una lucertola con questo tratto possiede anche velocità di nuoto 9 metri).

\medskip\textbf{Lupo}\index{Mostri - Lupo}

\emph{Media bestia, disallineato}

\textbf{FORZA} +1

\textbf{DESTREZZA} +2

\textbf{COSTITUZIONE} +1

\textbf{INTELLIGENZA} -4

\textbf{SAGGEZZA} +1

\textbf{CARISMA} -2

\textbf{Iniziativa} +2 -- \textbf{Difesa} 14

\textbf{Punti Ferita} 11 (2d8 + 2)

\textbf{Movimento} 12 m

\textbf{Tiri Salvezza}: Tempra +5, Riflessi +5, Volontà +1 

\textbf{Competenze} Muoversi Silenziosamente / Nascondersi +4, Consapevolezza +3

\textbf{Lingue} -

\textbf{Sfida} 1/4 (50 PE)

\emph{\textbf{Udito e Olfatto Affinato.}} Il lupo ha +1d6 nelle prove di Saggezza (Consapevolezza) basate su udito o olfatto.

\emph{\textbf{Tattiche di Branco.}} Il lupo ha +1d6 ai tiri di attacco contro una creatura se almeno uno degli alleati del lupo si trova entro 1 metro dalla creatura e quell'alleato non è inabile.

\textbf{Azioni}

\emph{\textbf{Morso.} Attacco con Arma da Mischia}: +4 a colpire, portata 1 m, un bersaglio.

\emph{Colpisce:} 7 (2d4 + 2) danni perforanti. Se il bersaglio è una creatura, deve riuscire un Tiro Salvezza di Tempra CD 11 o cadere prona.

\medskip\textbf{Dinolupo (Metalupo)}\index{Mostri - Dinolupo (Metalupo}

\emph{Grande bestia, disallineato}

\textbf{FORZA} +3

\textbf{DESTREZZA} +2

\textbf{COSTITUZIONE} +2

\textbf{INTELLIGENZA} -2

\textbf{SAGGEZZA} +1

\textbf{CARISMA} -2

\textbf{Iniziativa} +2 -- \textbf{Difesa} 15

\textbf{Punti Ferita} 37 (5d10 + 10)

\textbf{Movimento} 15 m

\textbf{Tiri Salvezza}: Tempra +7, Riflessi +6, Volontà +2 

\textbf{Competenze} Muoversi Silenziosamente / Nascondersi +4, Consapevolezza +3

\textbf{Lingue} -

\textbf{Sfida} 1 (200 PE)

\emph{\textbf{Udito e Olfatto Affinato.}} Il lupo ha +1d6 nelle prove di Saggezza (Consapevolezza) basate su udito o olfatto.

\emph{\textbf{Tattiche di Branco.}} Il lupo ha +1d6 ai tiri di attacco contro una creatura se almeno uno degli alleati del lupo si trova entro 1 metro dalla creatura e quell'alleato non è inabile.

\textbf{Azioni}

\emph{\textbf{Morso.} Attacco con Arma da Mischia}: +5 a colpire, portata 1 m, un bersaglio.

\emph{Colpisce:} 10 (2d6 + 3) danni perforanti. Se il bersaglio è una creatura, deve riuscire un Tiro Salvezza di Tempra CD 13 o cadere prona.

\medskip\textbf{Lupo Invernale}\index{Mostri - Lupo Invernale}

I lupi invernali abitano nelle regioni artiche e sono creature malvagie e intelligenti dal manto bianco come la neve e gli occhi color del ghiaccio.

\emph{Grande mostruosità, neutrale malvagio}

\textbf{FORZA} +4

\textbf{DESTREZZA} +1

\textbf{COSTITUZIONE} +2

\textbf{INTELLIGENZA} -2

\textbf{SAGGEZZA} +1

\textbf{CARISMA} -1

\textbf{Iniziativa} +1 -- \textbf{Difesa} 15

\textbf{Punti Ferita} 75 (10d10 + 20)

\textbf{Movimento} 15 m

\textbf{Tiri Salvezza}: Tempra +9, Riflessi +6, Volontà +3 

\textbf{Competenze} Muoversi Silenziosamente / Nascondersi +3, Consapevolezza +5

\textbf{Immunità al Danno} freddo

\textbf{Lingue} Comune, Gigante, Lupo Invernale

\textbf{Sfida} 3 (700 PE)

\emph{\textbf{Camuffamento di Neve.}} Il lupo ha +1d6 alle prove di Destrezza (Nascondersi) effettuate per nascondersi su terreno innevato.

\emph{\textbf{Udito e Olfatto Affinato.}} Il lupo ha +1d6 nelle prove di Saggezza (Consapevolezza) basate su udito o olfatto.

\emph{\textbf{Tattiche di Branco.}} Il lupo ha +1d6 ai tiri di attacco contro una creatura se almeno uno degli alleati del lupo si trova entro 1 metro dalla creatura e quell'alleato non è inabile.

\textbf{Azioni}

\emph{\textbf{Morso.} Attacco con Arma da Mischia}: +6 a colpire, portata 1 m, un bersaglio.

\emph{Colpisce:} 11 (2d6 + 4) danni perforanti. Se il bersaglio è una creatura, deve riuscire un Tiro Salvezza di Tempra CD 14 o cadere prona.

\emph{\textbf{Soffio Gelido (Ricarica 5-6).}} Il lupo esala un'esplosione di vento gelido in un cono di 5 metri. Ogni creatura in quell'area deve effettuare un Tiro Salvezza di Riflessi CD 12, e subire 18 (4d8) danni da freddo se fallisce il Tiro Salvezza, o la metà di questi danni se lo riesce.

\medskip\textbf{Mammut}\index{Mostri - Mammut}

Il mammut è una creatura simile all'elefante dalla folta pelliccia e lunghe zanne.

\emph{Enorme bestia, disallineato}

\textbf{FORZA} +7

\textbf{DESTREZZA} -1

\textbf{COSTITUZIONE} +5

\textbf{INTELLIGENZA} -4

\textbf{SAGGEZZA} +0

\textbf{CARISMA} -2

\textbf{Iniziativa} -1 -- \textbf{Difesa} 16

\textbf{Punti Ferita} 126 (11d12 + 55)

\textbf{Movimento} 12 m

\textbf{Tiri Salvezza}: Tempra +14, Riflessi +10, Volontà +7 

\textbf{Lingue} -

\textbf{Sfida} 6 (2.300 PE)

\emph{\textbf{Carica Travolgente.}} Se il mammut si muove di almeno 6 metri diretto verso una creatura e la colpisce con un attacco di incornata durante lo stesso turno, il bersaglio deve riuscire un Tiro Salvezza su Tempra CD 18 o cadere prono. Se il bersaglio è prono, il mammut può effettuare un attacco di pestone contro di lui come azione bonus.

\textbf{Azioni}

\emph{\textbf{Incornata.} Attacco con Arma da Mischia}: +10 a colpire, portata 3 m, un bersaglio.

\emph{Colpisce:} 25 (4d8 + 7) danni perforanti. 

\emph{\textbf{Pestone.} Attacco con Arma da Mischia}: +10 a colpire, portata 1 m, una creatura prona.

\emph{Colpisce:} 29 (4d10 + 7) danni da botta.

\medskip\textbf{Mastino}\index{Mostri - Mastino}

\textbf{I} mastini sono impressionanti segugi apprezzati dagli umanoidi per la loro realtà e sensi affinati.

\emph{Media bestia, disallineato}

\textbf{FORZA} +1

\textbf{DESTREZZA} +2

\textbf{COSTITUZIONE} +1

\textbf{INTELLIGENZA} -4

\textbf{SAGGEZZA} +1

\textbf{CARISMA} -2

\textbf{Iniziativa} +2 -- \textbf{Difesa} 13

\textbf{Punti Ferita} 5 (1d8 + 1)

\textbf{Movimento} 12 m

\textbf{Tiri Salvezza}: Tempra +3, Riflessi +3, Volontà +1 

\textbf{Competenze} Consapevolezza +3, Sopravvivenza (Seguire Tracce) +3

\textbf{Lingue} -

\textbf{Sfida} 1/8 (25 PE)

\emph{\textbf{Udito e Olfatto Affinato.}} Il mastino ha +1d6 nelle prove di Saggezza (Consapevolezza) basate su udito o olfatto.

\textbf{Azioni}

\emph{\textbf{Morso.} Attacco con Arma da Mischia}: +3 a colpire, portata 1 m, un bersaglio.

\emph{Colpisce:} 4 (1d6 + 1) danni perforanti. Se il bersaglio è una creatura, deve riuscire un Tiro Salvezza di Tempra CD 11 o cadere prono.

\medskip\textbf{Mulo}\index{Mostri - Mulo}

\emph{Media bestia, disallineato}

\textbf{FORZA} +2

\textbf{DESTREZZA} +0

\textbf{COSTITUZIONE} +1

\textbf{INTELLIGENZA} -4

\textbf{SAGGEZZA} +0

\textbf{CARISMA} -3

\textbf{Iniziativa} +0 -- \textbf{Difesa} 11

\textbf{Punti Ferita} 11 (2d8 + 2)

\textbf{Movimento} 12 m

\textbf{Tiri Salvezza}: Tempra +3, Riflessi +1, Volontà +1 

\textbf{Lingue} -

\textbf{Sfida} 1/8 (25 PE)

\emph{\textbf{Bestia da Soma.}} Il mulo è considerato un animale Grande al fine di determinare la sua capacità di carico.

\emph{\textbf{Piedi Saldi.}} Il mulo ha +1d6 ai Tiri Salvezza su Tempra e Riflessi effettuati contro effetti che lo farebbero cadere prono.

\textbf{Azioni}

\emph{\textbf{Zoccoli.} Attacco con Arma da Mischia}: +2 a colpire, portata 1 m, un bersaglio.

\emph{Colpisce:} 4 (1d4 + 2) danni da botta.

\medskip\textbf{Orso Bruno}\index{Mostri - Orso Bruno}

\emph{Grande bestia, disallineato}

\textbf{FORZA} +4

\textbf{DESTREZZA} +0

\textbf{COSTITUZIONE} +3

\textbf{INTELLIGENZA} -4

\textbf{SAGGEZZA} +1

\textbf{CARISMA} -2

\textbf{Iniziativa} +0 -- \textbf{Difesa} 12

\textbf{Punti Ferita} 34 (4d10 + 12)

\textbf{Movimento} 12 m, scalata 9 m

\textbf{Tiri Salvezza}: Tempra +6, Riflessi +2, Volontà +3 

\textbf{Competenze} Consapevolezza +3

\textbf{Lingue} -

\textbf{Sfida} 1 (200 PE)

\emph{\textbf{Olfatto Affinato.}} L'orso ha +1d6 alle prove di Saggezza (Consapevolezza) basate sull'olfatto.

\textbf{Azioni}

\emph{\textbf{Multiattacco.}} L'orso effettua due attacchi: uno con il morso e uno con gli artigli.

\emph{\textbf{Artigli.} Attacco con Arma da Mischia}: +5 a colpire, portata 1 m, un bersaglio.

\emph{Colpisce:} 11 (2d6 + 4) danni taglienti.

\emph{\textbf{Morso.} Attacco con Arma da Mischia}: +5 a colpire, portata 1 m, un bersaglio.

\emph{Colpisce:} 8 (1d8 + 4) danni perforanti.

\medskip\textbf{Orso Nero}\index{Mostri - Orso Nero}

\emph{Media bestia, disallineato}

\textbf{FORZA} +2

\textbf{DESTREZZA} +0

\textbf{COSTITUZIONE} +2

\textbf{INTELLIGENZA} -4

\textbf{SAGGEZZA} +1

\textbf{CARISMA} -2

\textbf{Iniziativa} +0 -- \textbf{Difesa} 12

\textbf{Punti Ferita} 19 (3d8 + 6)

\textbf{Movimento} 12 m, scalata 9 m

\textbf{Tiri Salvezza}: Tempra +4, Riflessi +1, Volontà +1 

\textbf{Competenze} Consapevolezza +3

\textbf{Lingue} -

\textbf{Sfida} 1/2 (100 PE)

\emph{\textbf{Olfatto Affinato.}} L'orso ha +1d6 alle prove di Saggezza (Consapevolezza) basate sull'olfatto.

\textbf{Azioni}

\emph{\textbf{Multiattacco.}} L'orso nero effettua due attacchi: uno con il morso e uno con gli artigli.

\emph{\textbf{Artigli.} Attacco con Arma da Mischia}: +3 a colpire, portata 1 m, un bersaglio.

\emph{Colpisce:} 7 (2d4 + 2) danni taglienti.

\emph{\textbf{Morso.} Attacco con Arma da Mischia}: +3 a colpire, portata 1 m, un bersaglio.

\emph{Colpisce:} 5 (1d6 + 2) danni perforanti.

\medskip\textbf{Orso Polare}\index{Mostri - Orso Polare}

\emph{Grande bestia, disallineato}

\textbf{FORZA} +5

\textbf{DESTREZZA} +0

\textbf{COSTITUZIONE} +3

\textbf{INTELLIGENZA} -4

\textbf{SAGGEZZA} +1

\textbf{CARISMA} -2

\textbf{Iniziativa} +0 -- \textbf{Difesa} 13

\textbf{Punti Ferita} 42 (5d10 + 15)

\textbf{Movimento} 12 m, nuoto 9 m

\textbf{Tiri Salvezza}: Tempra +10, Riflessi +7, Volontà +4 

\textbf{Competenze} Consapevolezza +3

\textbf{Lingue} -

\textbf{Sfida} 2 (450 PE)

\emph{\textbf{Olfatto Affinato.}} L'orso ha +1d6 alle prove di Saggezza (Consapevolezza) basate sull'olfatto.

\textbf{Azioni}

\emph{\textbf{Multiattacco.}} L'orso effettua due attacchi: uno con il morso e uno con gli artigli.

\emph{\textbf{Artigli.} Attacco con Arma da Mischia}: +7 a colpire, portata 1 m, un bersaglio.

\emph{Colpisce:} 12 (2d6 + 5) danni taglienti.

\emph{\textbf{Morso.} Attacco con Arma da Mischia}: +7 a colpire, portata 1 m, un bersaglio.

\emph{Colpisce:} 9 (1d8 + 5) danni perforanti.

\textbf{VARIANTE: ORSO DELLE CAVERNE}\index{Mostri - Orso delle Caverne}

Alcuni orsi si sono adattati alla vita sottoterra. Costoro hanno le stesse statistiche degli orsi polari, ma con visione al buio 18 m.

\medskip\textbf{Pantera}\index{Mostri - Pantera}

\emph{Media bestia, disallineato}

\textbf{FORZA} +2

\textbf{DESTREZZA} +2

\textbf{COSTITUZIONE} +0

\textbf{INTELLIGENZA} -4

\textbf{SAGGEZZA} +2

\textbf{CARISMA} -2

\textbf{Iniziativa} +2 -- \textbf{Difesa} 13

\textbf{Punti Ferita} 13 (3d8)

\textbf{Movimento} 15 m, scalata 12 m

\textbf{Tiri Salvezza}: Tempra +3, Riflessi +5, Volontà +3 

\textbf{Competenze} Muoversi Silenziosamente / Nascondersi +6, Consapevolezza +4

\textbf{Lingue} -

\textbf{Sfida} 1/4 (50 PE)

\emph{\textbf{Balzo.}} Se la pantera si muove di almeno 6 metri diretta verso una creatura e la colpisce con un attacco di artiglio durante lo stesso turno, il bersaglio deve riuscire un Tiro Salvezza di Tempra CD 12 o cadere prono. Se il bersaglio è prono, la pantera può effettuare un  attacco di morso contro di esso come azione bonus.

\emph{\textbf{Olfatto Affinato.}} La pantera ha +1d6 alle prove di Saggezza (Consapevolezza) basate sull'olfatto.

\textbf{Azioni}

\emph{\textbf{Artiglio.} Attacco con Arma da Mischia}: +4 a colpire, portata 1 m, un bersaglio.

\emph{Colpisce:} 4 (1d4 + 2) danni taglienti.

\emph{\textbf{Morso.} Attacco con Arma da Mischia}: +4 a colpire, portata 1 m, un bersaglio.

\emph{Colpisce:} 5 (1d6 + 2) danni perforanti.


\medskip\textbf{Pony}\index{Mostri - Pony}

\emph{Media bestia, disallineato}

\textbf{FORZA} +2

\textbf{DESTREZZA} +0

\textbf{COSTITUZIONE} +1

\textbf{INTELLIGENZA} -4

\textbf{SAGGEZZA} +0

\textbf{CARISMA} -2

\textbf{Iniziativa} +0 -- \textbf{Difesa} 11

\textbf{Punti Ferita} 11 (2d8 + 2)

\textbf{Movimento} 12 m

\textbf{Tiri Salvezza}: Tempra +5, Riflessi +4, Volontà +0

\textbf{Lingue} -

\textbf{Sfida} 1/8 (25 PE)

\textbf{Azioni}

\emph{\textbf{Zoccoli.} Attacco con Arma da Mischia}: +4 a colpire, portata 1 m, un bersaglio.

\emph{Colpisce:} 7 (2d4 + 2) danni da botta.

\medskip\textbf{Ragno}\index{Mostri - Ragno}

\emph{Minuscola bestia, disallineato}

\textbf{FORZA} 2 (-5)

\textbf{DESTREZZA} +2

\textbf{COSTITUZIONE} -1

\textbf{INTELLIGENZA} -5

\textbf{SAGGEZZA} +0

\textbf{CARISMA} -4

\textbf{Iniziativa} +2 -- \textbf{Difesa} 13

\textbf{Punti Ferita} 1 (1d4 - 1)

\textbf{Movimento} 6 m, scalata 6 m

\textbf{Tiri Salvezza}: Tempra -4, Riflessi +2, Volontà -4 

\textbf{Competenze} Muoversi Silenziosamente / Nascondersi +4

\textbf{Sensi} visione al buio 9 m

\textbf{Lingue} -

\textbf{Sfida} 0 (10 PE)

\emph{\textbf{Camminare sulla Tela.}} Il ragno ignora le restrizioni al movimento provocate dalle ragnatele.

\emph{\textbf{Scalare come Ragno.}} Il ragno può scalare superfici difficili, compreso lo stare a testa in giù sul soffitto, senza bisogno  di effettuare una prova di abilità.

\emph{\textbf{Senso della Tela.}} Mentre è in contatto con una ragnatela, il ragno sa l'esatta posizione di qualsiasi altra creatura in contatto con la stessa ragnatela.

\textbf{Azioni}

\emph{\textbf{Morso.} Attacco con Arma da Mischia}: +4 a colpire, portata 1 m, una creatura.

\emph{Colpisce:} 1 danno perforante e il bersaglio deve riuscire un Tiro Salvezza su Tempra 9 o subire 2 (1d4) danni da veleno.

\medskip\textbf{Ragno Fase}\index{Mostri - Ragno Fase}

Il ragno fase possiede l'abilità magica di entrare ed uscire dal Piano Etereo. Sembra apparire dal nulla e scompare rapidamente dopo aver attaccato.

\emph{Grande mostruosità, disallineato}

\textbf{FORZA} +2

\textbf{DESTREZZA} +2

\textbf{COSTITUZIONE} +1

\textbf{INTELLIGENZA} -2

\textbf{SAGGEZZA} +0

\textbf{CARISMA} -2

\textbf{Iniziativa} +2 -- \textbf{Difesa} 15

\textbf{Punti Ferita} 32 (5d10 + 5)

\textbf{Movimento} 9 m, scalata 9 m

\textbf{Tiri Salvezza}: Tempra +8, Riflessi +8, Volontà +3 

\textbf{Competenze} Muoversi Silenziosamente / Nascondersi +6

\textbf{Sensi} visione al buio 18 m

\textbf{Lingue} -

\textbf{Sfida} 3 (700 PE)

\emph{\textbf{Camminare sulla Tela.}} Il ragno ignora le restrizioni al movimento provocate dalle ragnatele.

\emph{\textbf{Gita Eterea.}} Come azione bonus, il ragno può magicamente spostarsi dal Piano Materiale al Piano Etereo, o viceversa.

\emph{\textbf{Scalare come Ragno.}} Il ragno può scalare superfici difficili, compreso lo stare a testa in giù sul soffitto, senza bisogno di effettuare una prova di abilità.

\textbf{Azioni}

\emph{\textbf{Morso.} Attacco con Arma da Mischia}: +4 a colpire, portata 1 m, una creatura.

\emph{Colpisce:} 7 (1d10 + 2) danni perforanti e il bersaglio deve effettuare un Tiro Salvezza di Tempra CD 11, e subire 18 (4d8) danni da veleno se fallisce il Tiro Salvezza, o la metà di questo danno se lo riesce. Se il danno da veleno riduce il bersaglio a 0 punti ferita, il bersaglio è stabile ma avvelenato per 1 ora, anche dopo aver recuperato i punti ferita, e mentre è avvelenato in questo modo resta paralizzato.

\medskip\textbf{Ragno Gigante}\index{Mostri - Ragno Gigante}

\emph{Grande bestia, disallineato}

\textbf{FORZA} +2

\textbf{DESTREZZA} +3

\textbf{COSTITUZIONE} +1

\textbf{INTELLIGENZA} -4

\textbf{SAGGEZZA} +0

\textbf{CARISMA} -3

\textbf{Iniziativa} +2 -- \textbf{Difesa} 15

\textbf{Punti Ferita} 26 (4d10 + 4)

\textbf{Movimento} 9 m, scalata 9 m

\textbf{Tiri Salvezza}:  Tempra +4, Riflessi +4, Volontà +1 

\textbf{Competenze} Muoversi Silenziosamente / Nascondersi +7

\textbf{Sensi} vista cieca 3 m, visione al buio 18 m

\textbf{Lingue} -

\textbf{Sfida} 1 (200 PE)

\emph{\textbf{Camminare sulla Tela.}} Il ragno ignora le restrizioni al movimento provocate dalle ragnatele.

\emph{\textbf{Scalare come Ragno.}} Il ragno può scalare superfici difficili, compreso lo stare a testa in giù sul soffitto, senza bisogno di effettuare una prova di abilità.

\emph{\textbf{Senso della Tela.}} Mentre è in contatto con una ragnatela, il ragno sa l'esatta posizione di qualsiasi altra creatura in contatto con la stessa ragnatela. 

\textbf{Azioni}

\emph{\textbf{Morso.} Attacco con Arma da Mischia}: +5 a colpire, portata 1 m, una creatura.

\emph{Colpisce:} 7 (1d8 + 3) danni perforanti e il bersaglio deve effettuare un Tiro Salvezza di Tempra CD 11, e subire 9

(2d8) danni da veleno se fallisce il Tiro Salvezza, o la metà di questi danni se lo riesce. Se il danno da veleno riduce il bersaglio a 0 punti ferita, il bersaglio è stabile ma avvelenato per 1 ora, anche dopo aver recuperato i punti ferita, e mentre è avvelenato in questo modo resta paralizzato.

\emph{\textbf{Ragnatela (Ricarica 5-6).} Attacco con Arma a Gittata}: +5 a colpire, gittata 9m, una creatura.

\emph{Colpisce:} Il bersaglio è intralciato dalla ragnatela. Con un'azione, il bersaglio intralciato può effettuare una prova di Forza CD 12 e, in caso di successo, spezzare la tela. La ragnatela può essere anche attaccata e distrutta (CA 10; pf 5; vulnerabilità al danno da fuoco; immunità ai danni da botta, psichici e da veleno). 

\medskip\textbf{Ragno Lupo Gigante}\index{Mostri - Ragno Lupo Gigante}

Un ragno lupo gigante caccia le prede su terreno aperto o si nasconde in tane o fessure del terreno per tendere imboscate.

\emph{Media bestia, disallineato}

\textbf{FORZA} +1

\textbf{DESTREZZA} +3

\textbf{COSTITUZIONE} +1

\textbf{INTELLIGENZA} -4

\textbf{SAGGEZZA} +1

\textbf{CARISMA} -3

\textbf{Iniziativa} +3 -- \textbf{Difesa} 14

\textbf{Punti Ferita} 11 (2d8 + 2)

\textbf{Movimento} 12 m, scalata 12 m

\textbf{Tiri Salvezza}:  Tempra +2, Riflessi +4, Volontà +1 

\textbf{Competenze} Muoversi Silenziosamente / Nascondersi +7, Consapevolezza +3

\textbf{Sensi} vista cieca 3 m, visione al buio 18 m

\textbf{Lingue} -

\textbf{Sfida} 1/4 (50 PE)

\emph{\textbf{Camminare sulla Tela.}} Il ragno ignora le restrizioni al movimento provocate dalle ragnatele.

\emph{\textbf{Scalare come Ragno.}} Il ragno può scalare superfici difficili, compreso lo stare a testa in giù sul soffitto, senza bisogno di effettuare una prova di abilità.

\emph{\textbf{Senso della Tela.}} Mentre è in contatto con una ragnatela, il ragno sa l'esatta posizione di qualsiasi altra creatura in contatto con la stessa ragnatela.

\textbf{Azioni}

\emph{\textbf{Morso.} Attacco con Arma da Mischia}: +3 a colpire, portata 1 m, una creatura.

\emph{Colpisce:} 4 (1d6 + 1) danni perforanti e il bersaglio deve effettuare un Tiro Salvezza di Tempra CD 11, e subire 7 (2d6) danni da veleno se fallisce il Tiro Salvezza, o la metà di questi danni se lo riesce. Se il danno da veleno riduce il bersaglio a 0 punti ferita, il bersaglio è stabile ma avvelenato per 1 ora, anche dopo aver recuperato i punti ferita, e mentre è avvelenato in questo modo resta paralizzato.

\medskip\textbf{Rana}\index{Mostri - Rana}

\emph{Minuscola bestia, disallineato}

\textbf{FORZA} -5

\textbf{DESTREZZA} +1

\textbf{COSTITUZIONE} -1

\textbf{INTELLIGENZA} -5

\textbf{SAGGEZZA} -1

\textbf{CARISMA} -4

\textbf{Iniziativa} +1 -- \textbf{Difesa} 12

\textbf{Punti Ferita} 1 (1d4 - 1)

\textbf{Movimento} 6 m, nuoto 6 m

\textbf{Tiri Salvezza}:  Tempra -4, Riflessi +1, Volontà -2

\textbf{Competenze} Muoversi Silenziosamente / Nascondersi +3, Consapevolezza +1

\textbf{Sensi} visione al buio 9 m

\textbf{Lingue} -

\textbf{Sfida} 0 (0 PE)

\emph{\textbf{Anfibio.}} La rana può respirare aria e acqua.

\emph{\textbf{Salto da Fermo.}} Una rana può saltare in lungo fino a 3 metri e in alto fino a 1 metro, con o senza la rincorsa.

Una \textbf{rana} è sprovvista di attacchi. Si nutre di piccoli insetti e di solito vive in prossimità di acquitrini, dentro gli alberi o sottoterra.

\medskip\textbf{Rana Gigante}\index{Mostri - Rana Gigante}

\emph{Media bestia, disallineato}

\textbf{FORZA} +1

\textbf{DESTREZZA} +1

\textbf{COSTITUZIONE} +0

\textbf{INTELLIGENZA} -4

\textbf{SAGGEZZA} +0

\textbf{CARISMA} -4

\textbf{Iniziativa} +1 -- \textbf{Difesa} 12

\textbf{Punti Ferita} 18 (4d8)

\textbf{Movimento} 9 m, nuoto 9 m

\textbf{Tiri Salvezza}: Tempra +2, Riflessi +2, Volontà +0 

\textbf{Competenze} Muoversi Silenziosamente / Nascondersi +3, Consapevolezza +2

\textbf{Sensi} visione al buio 9 m

\textbf{Lingue} -

\textbf{Sfida} 1/4 (50 PE)

\emph{\textbf{Anfibio.}} La rana può respirare aria e acqua.

\emph{\textbf{Salto da Fermo.}} Una rana può saltare in lungo fino a 6 metri e in alto fino a 3 metri, con o senza la rincorsa.

\textbf{Azioni}

\emph{\textbf{Morso.} Attacco con Arma da Mischia}: +3 a colpire, portata 1 m, un bersaglio.

\emph{Colpisce:} 4 (1d6 + 1) danni perforanti e il bersaglio è afferrato (CD 11 per fuggire). Fino al termine dell'afferrare, il bersaglio è intralciato, e la rana non può usare il morso contro un altro bersaglio.

\emph{\textbf{Inghiottire.}} La rana effettua una attacco di morso contro un bersaglio di taglia Piccola o inferiore che sta afferrando. Se l'attacco colpisce, il bersaglio è inghiottito, e l'afferrare ha termine. Il bersaglio inghiottito è accecato e intralciato, ha copertura totale contro gli attacchi e altri effetti all'esterno della rana, e subisce 5 (2d4) danni da acido all'inizio di ciascun turno della rana. La rana può inghiottire solo un bersaglio alla volta. Se la rana muore, una creatura inghiottita non è più intralciata da essa e può uscire dal cadavere utilizzando 1 metro di movimento, uscendo prona.

\medskip\textbf{Ratto}\index{Mostri - Ratto}

\emph{Minuscola bestia, disallineato}

\textbf{FORZA} -4

\textbf{DESTREZZA} +0

\textbf{COSTITUZIONE} -1

\textbf{INTELLIGENZA} -4

\textbf{SAGGEZZA} +0

\textbf{CARISMA} -3

\textbf{Iniziativa} +0 -- \textbf{Difesa} 11

\textbf{Punti Ferita} 1 (1d4 - 1)

\textbf{Movimento} 6 m

\textbf{Tiri Salvezza}: Tempra -4, Riflessi +0, Volontà +0 

\textbf{Sensi} visione al buio 9 m

\textbf{Lingue} -

\textbf{Sfida} 0 (10 PE)

\emph{\textbf{Olfatto Affinato.}} Il ratto ha +1d6 alle prove di Saggezza (Consapevolezza) basate sull'olfatto.

\textbf{Azioni}

\emph{\textbf{Morso.} Attacco con Arma da Mischia}: +0 a colpire, portata 1 m, un bersaglio.

\emph{Colpisce:} 1 danno perforante.

\medskip\textbf{Ratto Gigante}\index{Mostri - Ratto Gigante}

\emph{Piccola bestia, disallineato}

\textbf{FORZA} -2

\textbf{DESTREZZA} +2

\textbf{COSTITUZIONE} +0

\textbf{INTELLIGENZA} -4

\textbf{SAGGEZZA} +0

\textbf{CARISMA} -3

\textbf{Iniziativa} +2 -- \textbf{Difesa} 13

\textbf{Punti Ferita} 7 (2d6)

\textbf{Movimento} 9 m

\textbf{Tiri Salvezza}: Tempra +3, Riflessi +5, Volontà +1 

\textbf{Sensi} visione al buio 18 m

\textbf{Lingue} -

\textbf{Sfida} 1/8 (25 PE)

\emph{\textbf{Olfatto Affinato.}} Il ratto ha +1d6 alle prove di Saggezza (Consapevolezza) basate sull'olfatto.

\emph{\textbf{Tattiche di Branco.}} Il ratto ha +1d6 al tiro di attacco contro una creatura se almeno uno degli alleati del ratto si trova entro 1 metro dalla creatura e quell'alleato non è inabile.

\textbf{Azioni}

\emph{\textbf{Morso.} Attacco con Arma da Mischia}: +4 a colpire, portata 1 m, un bersaglio.

\emph{Colpisce:} 4 (1d4 + 2) danni perforanti.

\textbf{VARIANTE: RATTO GIGANTE AMMALATO}\index{Mostri - Ratto Gigante ammalato}

Alcuni ratti giganti recano una terribile malattia che diffondono tramite il morso. Un ratto gigante ammalato ha grado di sfida 1/8 (25 PE) e la seguente azione invece del suo normale attacco di morso.

\emph{\textbf{Morso.} Attacco con Arma da Mischia}: +4 a colpire, portata 1 m, un bersaglio.

\emph{Colpisce:} 4 (1d4 + 2) danni perforanti. Se il bersaglio è una creatura, deve riuscire un Tiro Salvezza di Tempra CD 10 o contrarre una malattia. Fino a che la malattia non viene curata, il bersaglio non può recuperare punti ferita eccetto tramite metodi magici, e i punti ferita massimi del bersaglio diminuiscono di 3 (1d6) ogni 24 ore. Se i punti ferita massimi del bersaglio scendono a 0 come risultato della malattia, il bersaglio muore.

\medskip\textbf{Rinoceronte}\index{Mostri - Rinoceronte}

\emph{Grande bestia, disallineato}

\textbf{FORZA} +5

\textbf{DESTREZZA} -1

\textbf{COSTITUZIONE} +2

\textbf{INTELLIGENZA} -4

\textbf{SAGGEZZA} +1

\textbf{CARISMA} -2

\textbf{Iniziativa} -1 -- \textbf{Difesa} 12

\textbf{Punti Ferita} 45 (6d10 + 12)

\textbf{Movimento} 12 m

\textbf{Tiri Salvezza}: Tempra +10, Riflessi +4, Volontà +2

\textbf{Lingue} -

\textbf{Sfida} 2 (450 PE)

\emph{\textbf{Carica.}} Se il rinoceronte si muove di almeno 6 metri diretto verso un bersaglio e lo colpisce con un attacco di incornata durante lo stesso turno, il bersaglio subisce 9 (2d8) danni da botta aggiuntivi. Se il bersaglio è una creatura, deve riuscire un Tiro Salvezza su Tempra CD 15 o cadere prono.

\textbf{Azioni}

\emph{\textbf{Incornata.} Attacco con Arma da Mischia}: +7 a colpire, portata 1 m, un bersaglio.

\emph{Colpisce:} 14 (2d8 + 5) danni da botta.

\medskip\textbf{Rospo Gigante}\index{Mostri - Rospo Gigante}

\emph{Grande bestia, disallineato}

\textbf{FORZA} +2

\textbf{DESTREZZA} +1

\textbf{COSTITUZIONE} +1

\textbf{INTELLIGENZA} -4

\textbf{SAGGEZZA} +0

\textbf{CARISMA} -4

\textbf{Iniziativa} +1 -- \textbf{Difesa} 12

\textbf{Punti Ferita} 39 (6d10 + 6)

\textbf{Movimento} 6 m, nuoto 12 m

\textbf{Tiri Salvezza}: Tempra +6, Riflessi +6, Volontà +0

\textbf{Sensi} visione al buio 9 m

\textbf{Lingue} -

\textbf{Sfida} 1 (200 PE)

\emph{\textbf{Anfibio.}} Il rospo può respirare aria e acqua.

\emph{\textbf{Salto da Fermo.}} Un rospo può saltare in lungo fino a 6 metri e in alto fino a 3 metri, con o senza la rincorsa.

\textbf{Azioni}

\emph{\textbf{Morso.} Attacco con Arma da Mischia}: +4 a colpire, portata 1 m, un bersaglio.

\emph{Colpisce:} 7 (1d10 + 2) danni perforanti più 5 (1d10) danni da veleno, e il bersaglio è afferrato (CD 13 per fuggire). Fino al termine dell'afferrare, il bersaglio è intralciato, e il rospo non può usare il morso contro un altro bersaglio.

\emph{\textbf{Inghiottire.}} Il rospo effettua una attacco di morso contro un bersaglio di taglia Media o inferiore che sta afferrando. Se l'attacco colpisce, il bersaglio è inghiottito, e l'afferrare ha termine. Il bersaglio inghiottito è accecato e intralciato, ha copertura totale contro gli attacchi e altri effetti all'esterno della rana, e subisce 10 (3d6) danni da acido all'inizio di ciascun turno del rospo. Il rospo può inghiottire solo un bersaglio alla volta.

Se il rospo muore, una creatura inghiottita non è più intralciata da esso e può uscire dal cadavere utilizzando 1 metro di movimento, uscendo prono.

\medskip\textbf{Scarabeo di Fuoco Gigante}\index{Mostri - Scarabeo di Fuoco Gigante}

Uno scarabeo di fuoco gigante è una creatura notturna che possiede una coppia di ghiandole luminose capaci di emettere luce per 1d6 giorni dopo la morte dello scarabeo.

\emph{Piccola bestia, disallineato}

\textbf{FORZA} -1

\textbf{DESTREZZA} +0

\textbf{COSTITUZIONE} +1

\textbf{INTELLIGENZA} -5

\textbf{SAGGEZZA} -2

\textbf{CARISMA} -4

\textbf{Iniziativa} +0 -- \textbf{Difesa} 14

\textbf{Punti Ferita} 4 (1d6 + 1)

\textbf{Movimento} 9 m

\textbf{Tiri Salvezza}: Tempra +2, Riflessi +0, Volontà +0

\textbf{Sensi} vista cieca 9 m

\textbf{Lingue} -

\textbf{Sfida} 0 (10 PE)

\emph{\textbf{Illuminazione.}} Lo scarabeo irradia luce intensa in un raggio di 3 metri e luce fioca per ulteriori 3 metri.

\textbf{Azioni}

\emph{\textbf{Morso.} Attacco con Arma da Mischia}: +1 a colpire, portata 1 m, un bersaglio.

\emph{Colpisce:} 2 (1d6 - 1) danni taglienti.

\medskip\textbf{Sciacallo}\index{Mostri - Sciacallo}

\emph{Piccola bestia, disallineato}

\textbf{FORZA} -1

\textbf{DESTREZZA} +2

\textbf{COSTITUZIONE} +0

\textbf{INTELLIGENZA} -4

\textbf{SAGGEZZA} +1

\textbf{CARISMA} -2

\textbf{Iniziativa} +2 -- \textbf{Difesa} 13

\textbf{Punti Ferita} 3 (1d6)

\textbf{Movimento} 12 m

\textbf{Tiri Salvezza}: Tempra -1, Riflessi +3, Volontà +1

\textbf{Competenze} Consapevolezza +3

\textbf{Lingue} -

\textbf{Sfida} 0 (10 PE)

\emph{\textbf{Tattiche di Branco.}} Lo sciacallo ha +1d6 ai tiri di attacco contro una creatura se almeno uno degli alleati dello sciacallo si trova entro 1 metro dalla creatura e quell'alleato non è inabile.

\emph{\textbf{Udito e Olfatto Affinato.}} Lo sciacallo ha +1d6 nelle prove di Saggezza (Consapevolezza) basate su udito o olfatto.

\textbf{Azioni}

\emph{\textbf{Morso.} Attacco con Arma da Mischia}: +1 a colpire, portata 1 m, un bersaglio.

\emph{Colpisce:} 1 (1d4 - 1) danni perforanti.

\medskip\textbf{Sciami}\index{Mostri - Sciami}

Gli sciami presentati qui di seguito non sono dei normali o benigni raduni di piccole creature. Si formano invece come risultato di un'influenza esterna, spesso maligna. Anche i druidi non sono in grado di affascinare questi sciami, e la loro aggressività è quasi innaturale. 

\textbf{Sciame di Centopiedi}\index{Mostri - Sciame di Centopiedi}

\emph{Medio sciame di Minuscole bestie, disallineato}

\textbf{FORZA} -4

\textbf{DESTREZZA} +1

\textbf{COSTITUZIONE} +0

\textbf{INTELLIGENZA} -5

\textbf{SAGGEZZA} -2

\textbf{CARISMA} -5

\textbf{Iniziativa} +1 -- \textbf{Difesa} 13

\textbf{Punti Ferita} 22 (5d8)

\textbf{Movimento} 6 m, scalata 6 m

\textbf{Tiri Salvezza}: Tempra -1, Riflessi +3, Volontà +1

\textbf{Resistenze al Danno} da botta, perforante, tagliente

\textbf{Immunità alle Condizioni} affascinato, afferrato, intralciato,

paralizzato, pietrificato, prono, spaventato, stordito

\textbf{Sensi} vista cieca 3 m 

\textbf{Lingue} -

\textbf{Sfida} 1/2 (100 PE)

\emph{\textbf{Sciame.}} Lo sciame può occupare lo spazio di un'altra creatura e viceversa, e lo sciame può muoversi attraverso qualsiasi apertura grande abbastanza per un Minuscolo insetto. Lo sciame non può recuperare punti ferita né ottenere punti ferita temporanei.

\textbf{Azioni}

\emph{\textbf{Morsi.} Attacco con Arma da Mischia}: +3 a colpire, portata 0 m, un bersaglio nello spazio dello sciame. 

\emph{Colpisce:} 10 (4d4) danni perforanti, o 5 (2d4) danni perforanti se lo sciame è ha metà o meno dei suoi punti ferita. Una creatura ridotta a 0 punti ferita da uno sciame di centopiedi e stabile resta avvelenata per 1 ora, anche dopo aver recuperato i punti ferita, e rimane paralizzata dal veleno durante questo periodo.

\medskip\textbf{Sciame di Corvi}\index{Mostri - Sciame di Corvi}

\emph{Medio sciame di Minuscole bestie, disallineato}

\textbf{FORZA} -2

\textbf{DESTREZZA} +2

\textbf{COSTITUZIONE} -1

\textbf{INTELLIGENZA} -4

\textbf{SAGGEZZA} +1

\textbf{CARISMA} -2

\textbf{Iniziativa} +2 -- \textbf{Difesa} 13

\textbf{Punti Ferita} 24 (7d8 -- 7)

\textbf{Movimento} 3 m, volo 15 m

\textbf{Tiri Salvezza}: Tempra -1, Riflessi +3, Volontà +2

\textbf{Competenze} Consapevolezza +5

\textbf{Resistenze al Danno} da botta, perforante, tagliente \textbf{Immunità alle Condizioni} affascinato, afferrato, intralciato, paralizzato, pietrificato, prono, spaventato, stordito

\textbf{Lingue} -

\textbf{Sfida} 1/4 (50 PE)

\emph{\textbf{Sciame.}} Lo sciame può occupare lo spazio di un'altra creatura e viceversa, e lo sciame può muoversi attraverso qualsiasi apertura grande abbastanza per un Minuscolo corvo. Lo sciame non può recuperare punti ferita né ottenere punti ferita temporanei.

\textbf{Azioni}

\emph{\textbf{Becchi.} Attacco con Arma da Mischia}: +4 a colpire, portata 1 m, un bersaglio nello spazio dello sciame.

\emph{Colpisce:} 7 (2d6) danni perforanti, o 3 (1d6) danni perforanti se lo sciame è ha metà o meno dei suoi punti ferita.

\medskip\textbf{Sciame di Pirana}\index{Mostri - Sciame di Pirana}

\emph{Medio sciame di Minuscole bestie, disallineato}

\textbf{FORZA} +1

\textbf{DESTREZZA} +3

\textbf{COSTITUZIONE} -1

\textbf{INTELLIGENZA} -5

\textbf{SAGGEZZA} -2

\textbf{CARISMA} -4

\textbf{Iniziativa} +3 -- \textbf{Difesa} 14

\textbf{Punti Ferita} 28 (8d8 -- 8)

\textbf{Movimento} 0 m, nuoto 12 m

\textbf{Tiri Salvezza}: Tempra -3, Riflessi +4, Volontà -1

\textbf{Resistenze al Danno} da botta, perforante, tagliente

\textbf{Immunità alle Condizioni} affascinato, afferrato, intralciato, paralizzato, pietrificato, prono, spaventato, stordito

\textbf{Sensi} visione al buio 18 m

\textbf{Lingue} -

\textbf{Sfida} 1 (200 PE)

\emph{\textbf{Frenesia Sanguinaria.}} Lo sciame ha +1d6 ai tiri di attacco in mischia contro qualsiasi creatura che non sia al massimo dei punti ferita.

\emph{\textbf{Respirare Acqua.}} Lo sciame può respirare solo sottacqua.

\emph{\textbf{Sciame.}} Lo sciame può occupare lo spazio di un'altra creatura e viceversa, e lo sciame può muoversi attraverso qualsiasi apertura grande abbastanza per un Minuscolo pirana. Lo sciame non può recuperare punti ferita né ottenere punti ferita temporanei.

\textbf{Azioni}

\emph{\textbf{Morsi.} Attacco con Arma da Mischia}: +5 a colpire, portata 0 m, una creatura nello spazio dello sciame.

\emph{Colpisce:} 14 (4d6) danni perforanti, o 7 (2d6) danni perforanti se lo sciame è ha metà o meno dei suoi punti ferita.

\medskip\textbf{Sciame di Insetti}\index{Mostri - Sciame di Insetti}

\emph{Medio sciame di Minuscole bestie, disallineato}

\textbf{FORZA} -4

\textbf{DESTREZZA} +1

\textbf{COSTITUZIONE} +0

\textbf{INTELLIGENZA} -5

\textbf{SAGGEZZA} -2

\textbf{CARISMA} -5

\textbf{Iniziativa} +1 -- \textbf{Difesa} 13

\textbf{Punti Ferita} 22 (5d8)

\textbf{Movimento} 6 m, scalata 6 m

\textbf{Tiri Salvezza}: Tempra -3, Riflessi +2, Volontà -1

\textbf{Resistenze al Danno} da botta, perforante, tagliente

\textbf{Immunità alle Condizioni} affascinato, afferrato, intralciato, paralizzato, pietrificato, prono, spaventato, stordito

\textbf{Sensi} vista cieca 3 m

\textbf{Lingue} -

\textbf{Sfida} 1/2 (100 PE)

\emph{\textbf{Sciame.}} Lo sciame può occupare lo spazio di un'altra creatura e viceversa, e lo sciame può muoversi attraverso qualsiasi apertura grande abbastanza per un Minuscolo insetto. Lo sciame non può recuperare punti ferita né ottenere punti ferita temporanei.

\textbf{Azioni}

\emph{\textbf{Morsi.} Attacco con Arma da Mischia}: +3 a colpire, portata 0 m, un bersaglio nello spazio dello sciame.

\emph{Colpisce:} 10 (4d4) danni perforanti, o 5 (2d4) danni perforanti se lo sciame è ha metà o meno dei suoi punti ferita.

\medskip\textbf{Sciame di Pipistrelli}\index{Mostri - Sciame di Pipistrelli}

\emph{Medio sciame di Minuscole bestie, disallineato}

\textbf{FORZA} -3

\textbf{DESTREZZA} +2

\textbf{COSTITUZIONE} +0

\textbf{INTELLIGENZA} -4

\textbf{SAGGEZZA} +1

\textbf{CARISMA} -3

\textbf{Iniziativa} +2 -- \textbf{Difesa} 13

\textbf{Punti Ferita} 22 (5d8)

\textbf{Movimento} 0 m, volo 9 m

\textbf{Tiri Salvezza}: Tempra -2, Riflessi +4, Volontà +2

\textbf{Resistenze al Danno} da botta, perforante, tagliente

\textbf{Immunità alle Condizioni} affascinato, afferrato, intralciato, paralizzato, pietrificato, prono, spaventato, stordito

\textbf{Sensi} vista cieca 18 m

\textbf{Lingue} -

\textbf{Sfida} 1/4 (50 PE)

\emph{\textbf{Ecolocazione.}} Lo sciame non può usare la vista cieca se assordato.

\emph{\textbf{Sciame.}} Lo sciame può occupare lo spazio di un'altra creatura e viceversa, e lo sciame può muoversi attraverso qualsiasi apertura grande abbastanza per un Minuscolo pipistrello. Lo sciame non può recuperare punti ferita né ottenere punti ferita temporanei.

\emph{\textbf{Udito Affinato.}} Lo sciame ha +1d6 alle prove di Saggezza (Consapevolezza) basate sull'udito.

\textbf{Azioni}

\emph{\textbf{Morsi.} Attacco con Arma da Mischia}: +4 a colpire, portata 0 m, una creatura nello spazio dello sciame.

\emph{Colpisce:} 5 (2d4) danni perforanti, o 2 (1d4) danni perforanti se lo sciame è ha metà o meno dei suoi punti ferita.

\medskip\textbf{Sciame di Ragni}\index{Mostri - Sciame di Ragni}

\emph{Medio sciame di Minuscole bestie, disallineato}

\textbf{FORZA} -4

\textbf{DESTREZZA} +1

\textbf{COSTITUZIONE} +0

\textbf{INTELLIGENZA} -5

\textbf{SAGGEZZA} -2

\textbf{CARISMA} -5

\textbf{Iniziativa} +1 -- \textbf{Difesa} 13

\textbf{Punti Ferita} 22 (5d8)

\textbf{Movimento} 6 m, scalata 6 m

\textbf{Tiri Salvezza}: Tempra -3, Riflessi +2, Volontà -1

\textbf{Resistenze al Danno} da botta, perforante, tagliente

\textbf{Immunità alle Condizioni} affascinato, afferrato, intralciato, paralizzato, pietrificato, prono, spaventato, stordito

\textbf{Sensi} vista cieca 3 m

\textbf{Lingue} -

\textbf{Sfida} 1/2 (100 PE)

\emph{\textbf{Camminare sulla Tela.}} Lo sciame ignora le restrizioni al movimento provocate dalle ragnatele.

\emph{\textbf{Scalare come Ragno.}} Lo sciame può scalare superfici difficili, compreso lo stare a testa in giù sul soffitto, senza bisogno di effettuare una prova di abilità.

\emph{\textbf{Senso della Tela.}} Mentre è in contatto con una ragnatela, lo sciame sa l'esatta posizione di qualsiasi altra creatura in contatto con la stessa ragnatela.

\emph{\textbf{Sciame.}} Lo sciame può occupare lo spazio di un'altra creatura e viceversa, e lo sciame può muoversi attraverso qualsiasi apertura grande abbastanza per un Minuscolo insetto. Lo sciame non può recuperare punti ferita né ottenere punti ferita temporanei.

\textbf{Azioni}

\emph{\textbf{Morsi.} Attacco con Arma da Mischia}: +3 a colpire, portata 0 m, un bersaglio nello spazio dello sciame.

\emph{Colpisce:} 10 (4d4) danni perforanti, o 5 (2d4) danni perforanti se lo sciame è ha metà o meno dei suoi punti ferita.

\medskip\textbf{Sciame di Ratti}\index{Mostri - Sciame di Ratti}

\emph{Medio sciame di Minuscole bestie, disallineato}

\textbf{FORZA} -1

\textbf{DESTREZZA} +0

\textbf{COSTITUZIONE} -1

\textbf{INTELLIGENZA} -4

\textbf{SAGGEZZA} +0

\textbf{CARISMA} -4

\textbf{Iniziativa} +0 -- \textbf{Difesa} 11

\textbf{Punti Ferita} 24 (7d8 - 7)

\textbf{Movimento} 9 m

\textbf{Tiri Salvezza}: Tempra +0, Riflessi +1, Volontà +1

\textbf{Resistenze al Danno} da botta, perforante, tagliente

\textbf{Immunità alle Condizioni} affascinato, afferrato, intralciato, paralizzato, pietrificato, prono, spaventato, stordito

\textbf{Sensi} visione al buio 9 m
\textbf{Lingue} -

\textbf{Sfida} 1/4 (50 PE)

\emph{\textbf{Olfatto Affinato.}} Lo sciame ha +1d6 alle prove di Saggezza (Consapevolezza) basate sull'olfatto.

\emph{\textbf{Sciame.}} Lo sciame può occupare lo spazio di un'altra creatura e viceversa, e lo sciame può muoversi attraverso qualsiasi apertura grande abbastanza per un Minuscolo ratto. Lo sciame non può recuperare punti ferita né ottenere punti ferita temporanei.

\textbf{Azioni}

\emph{\textbf{Morsi.} Attacco con Arma da Mischia}: +2 a colpire, portata 0 m, un bersaglio nello spazio dello sciame.

\emph{Colpisce:} 7 (2d6) danni perforanti, o 3 (1d6) danni perforanti se lo sciame è ha metà o meno dei suoi punti ferita.

\medskip\textbf{Sciame di Scarabei}\index{Sciame di Scarabei}

\emph{Medio sciame di Minuscole bestie, disallineato}

\textbf{FORZA} -4

\textbf{DESTREZZA} +1

\textbf{COSTITUZIONE} +0

\textbf{INTELLIGENZA} -5

\textbf{SAGGEZZA} -2

\textbf{CARISMA} -5

\textbf{Iniziativa} +1 -- \textbf{Difesa} 13

\textbf{Punti Ferita} 22 (5d8)

\textbf{Movimento} 6 m, scalata 6 m, scavo 6 m

\textbf{Tiri Salvezza}: Tempra -3, Riflessi +2, Volontà -1

\textbf{Resistenze al Danno} da botta, perforante, tagliente

\textbf{Immunità alle Condizioni} affascinato, afferrato, intralciato, paralizzato, pietrificato, prono, spaventato, stordito

\textbf{Sensi} vista cieca 3 m

\textbf{Lingue} -

\textbf{Sfida} 1/2 (100 PE)

\emph{\textbf{Sciame.}} Lo sciame può occupare lo spazio di un'altra creatura e viceversa, e lo sciame può muoversi attraverso qualsiasi apertura grande abbastanza per un Minuscolo insetto. Lo sciame non può recuperare punti ferita né ottenere punti ferita temporanei.

\textbf{Azioni}

\emph{\textbf{Morsi.} Attacco con Arma da Mischia}: +3 a colpire, portata 0 m, un bersaglio nello spazio dello sciame.

\emph{Colpisce:} 10 (4d4) danni perforanti, o 5 (2d4) danni perforanti se lo sciame è ha metà o meno dei suoi punti ferita.

\medskip\textbf{Sciame di Serpenti Velenosi}\index{Sciame di Serpenti Velenosi}

\emph{Medio sciame di Minuscole bestie, disallineato}

\textbf{FORZA} -1

\textbf{DESTREZZA} +4

\textbf{COSTITUZIONE} +0

\textbf{INTELLIGENZA} -5

\textbf{SAGGEZZA} +0

\textbf{CARISMA} -4

\textbf{Iniziativa} +4 -- \textbf{Difesa} 15

\textbf{Punti Ferita} 36 (8d8)

\textbf{Movimento} 9 m, nuoto 9 m

\textbf{Tiri Salvezza}: Tempra +0, Riflessi +5, Volontà +1

\textbf{Resistenze al Danno} da botta, perforante, tagliente

\textbf{Immunità alle Condizioni} affascinato, afferrato, intralciato, paralizzato, pietrificato, prono, spaventato, stordito

\textbf{Sensi} vista cieca 3 m

\textbf{Lingue} -

\textbf{Sfida} 2 (450 PE)

\emph{\textbf{Sciame.}} Lo sciame può occupare lo spazio di un'altra creatura e viceversa, e lo sciame può muoversi attraverso qualsiasi apertura grande abbastanza per un Minuscolo serpente. Lo sciame non può recuperare punti ferita né ottenere punti ferita temporanei.

\textbf{Azioni}

\emph{\textbf{Morsi.} Attacco con Arma da Mischia}: +6 a colpire, portata 0 m, una creatura nello spazio dello sciame.

\emph{Colpisce:} 7 (2d6) danni perforanti, o 3 (1d6) danni perforanti se lo sciame è ha metà o meno dei suoi punti ferita, e il bersaglio deve effettuare un Tiro Salvezza di Tempra CD 10, e subire 14 (4d6) danni da veleno se fallisce il Tiro Salvezza, o la metà di questi danni se lo riesce.

\medskip\textbf{Sciame di Vespe}\index{Sciame di Serpenti Velenosi}

\emph{Medio sciame di Minuscole bestie, disallineato}

\textbf{FORZA} -4

\textbf{DESTREZZA} +1

\textbf{COSTITUZIONE} +0

\textbf{INTELLIGENZA} -5

\textbf{SAGGEZZA} -2

\textbf{CARISMA} -5

\textbf{Iniziativa} +1 -- \textbf{Difesa} 13

\textbf{Punti Ferita} 22 (5d8)

\textbf{Movimento} 1 m, volo 9 m

\textbf{Tiri Salvezza}: Tempra -3, Riflessi +2, Volontà -1

\textbf{Resistenze al Danno} da botta, perforante, tagliente

\textbf{Immunità alle Condizioni} affascinato, afferrato, intralciato, paralizzato, pietrificato, prono, spaventato, stordito

\textbf{Sensi} vista cieca 3 m

\textbf{Lingue} -

\textbf{Sfida} 1/2 (100 PE)

\emph{\textbf{Sciame.}} Lo sciame può occupare lo spazio di un'altra creatura e viceversa, e lo sciame può muoversi attraverso qualsiasi apertura grande abbastanza per un Minuscolo insetto. Lo sciame non può recuperare punti ferita né ottenere punti ferita temporanei.

\textbf{Azioni}

\emph{\textbf{Morsi.} Attacco con Arma da Mischia}: +3 a colpire, portata 0 m, un bersaglio nello spazio dello sciame.

\emph{Colpisce:} 10 (4d4) danni perforanti, o 5 (2d4) danni perforanti se lo sciame è ha metà o meno dei suoi punti ferita.

\medskip\textbf{Scimmione}\index{Mostri - Scimmione}

\emph{Media bestia, disallineato}

\textbf{FORZA} +3

\textbf{DESTREZZA} +2

\textbf{COSTITUZIONE} +2

\textbf{INTELLIGENZA} -2

\textbf{SAGGEZZA} +1

\textbf{CARISMA} -2

\textbf{Iniziativa} +2 -- \textbf{Difesa} 13

\textbf{Punti Ferita} 19 (3d8 + 6)

\textbf{Movimento} 9 m, scalata 9 m

\textbf{Tiri Salvezza}: Tempra +3, Riflessi +3, Volontà +2

\textbf{Competenze} Acrobatica +5, Consapevolezza +3

\textbf{Lingue} -

\textbf{Sfida} 1/2 (100 PE)

\textbf{Azioni}

\emph{\textbf{Multiattacco.}} Lo scimmione effettua due attacchi di pugno.

\emph{\textbf{Pugno.} Attacco con Arma da Mischia}: +5 a colpire, portata 1 m, un bersaglio.

\emph{Colpisce:} 6 (1d6 + 3) danni da botta.

\emph{\textbf{Sasso.} Attacco con Arma a Gittata}: +5 a colpire, gittata 8m, un bersaglio.

\emph{Colpisce:} 6 (1d6 + 3) danni da botta.

\medskip\textbf{Scimmione Gigante}\index{Mostri - Scimmione Gigante}

\emph{Enorme bestia, disallineato}

\textbf{FORZA} +6

\textbf{DESTREZZA} +2

\textbf{COSTITUZIONE} +4

\textbf{INTELLIGENZA} -2

\textbf{SAGGEZZA} +1

\textbf{CARISMA} -2

\textbf{Iniziativa} +2 -- \textbf{Difesa} 16

\textbf{Punti Ferita} 157 (15d12 + 60)

\textbf{Movimento} 12 m, scalata 12 m

\textbf{Tiri Salvezza}: Tempra +7, Riflessi +6, Volontà +4

\textbf{Competenze} Acrobatica +9, Consapevolezza +4

\textbf{Lingue} -

\textbf{Sfida} 7 (2.900 PE)

\textbf{Azioni}

\emph{\textbf{Multiattacco.}} Lo scimmione effettua due attacchi di pugno.

\emph{\textbf{Pugno.} Attacco con Arma da Mischia}: +9 a colpire, portata 3 m, un bersaglio.

\emph{Colpisce:} 22 (3d10 + 6) danni da botta.

\emph{\textbf{Sasso.} Attacco con Arma a Gittata}: +9 a colpire, gittata 15m, un bersaglio.

\emph{Colpisce:} 30 (7d6 + 6) danni da botta.

\medskip\textbf{Scorpione}\index{Mostri - Scorpione}

\emph{Minuscola bestia, disallineato}

\textbf{FORZA} -4

\textbf{DESTREZZA} +0

\textbf{COSTITUZIONE} -1

\textbf{INTELLIGENZA} -5

\textbf{SAGGEZZA} -1

\textbf{CARISMA} -4

\textbf{Iniziativa} +0 -- \textbf{Difesa} 12

\textbf{Punti Ferita} 1 (1d4 - 1)

\textbf{Movimento} 3 m

\textbf{Tiri Salvezza}: Tempra -3, Riflessi +2, Volontà -1

\textbf{Sensi} vista cieca 3 m

\textbf{Lingue} -

\textbf{Sfida} 0 (10 PE)

\textbf{Azioni}

\emph{\textbf{Pungiglione.} Attacco con Arma da Mischia}: +2 a colpire, portata 1 m, una creatura.

\emph{Colpisce:} 1 danno perforante e il bersaglio deve effettuare un Tiro Salvezza di Tempra CD 9, e subire 4 (1d8) danni da veleno se fallisce il Tiro Salvezza, o la metà di questi danni se lo riesce.

\medskip\textbf{Scorpione Gigante}\index{Mostri - Scorpione Gigante}

\emph{Grande bestia, disallineato}

\textbf{FORZA} +2

\textbf{DESTREZZA} +1

\textbf{COSTITUZIONE} +2

\textbf{INTELLIGENZA} -5

\textbf{SAGGEZZA} -1

\textbf{CARISMA} -4

\textbf{Iniziativa} +1 -- \textbf{Difesa} 17

\textbf{Punti Ferita} 52 (7d10 + 14)

\textbf{Movimento} 12 m

\textbf{Tiri Salvezza}: Tempra +7, Riflessi +1, Volontà +1

\textbf{Sensi} vista cieca 18 m

\textbf{Lingue} -

\textbf{Sfida} 3 (700 PE)

\textbf{Azioni}

\emph{\textbf{Multiattacco.}} Lo scorpione effettua tre attacchi: due con gli artigli e uno con il pungiglione.

\emph{\textbf{Artiglio.} Attacco con Arma da Mischia}: +4 a colpire, portata 1 m, un bersaglio.

\emph{Colpisce:} 6 (1d8 + 2) danni da botta e il bersaglio è afferrato (CD 12 per fuggire). Lo scorpione ha due artigli, ciascuno dei quali può afferrare solo un bersaglio.

\emph{\textbf{Pungiglione.} Attacco con Arma da Mischia}: +4 a colpire, portata 1 m, una creatura.

\emph{Colpisce:} 7 (1d10 + 2) danni perforanti e il bersaglio deve effettuare un Tiro Salvezza di Tempra CD 12, e subire 22 (4d10) danni da veleno se fallisce il Tiro Salvezza, o la metà di questi danni se lo riesce.

\medskip\textbf{Serpente Costrittore}\index{Mostri - Serpente Costrittore}

\emph{Grande bestia, disallineato}

\textbf{FORZA} +2

\textbf{DESTREZZA} +2

\textbf{COSTITUZIONE} +1

\textbf{INTELLIGENZA} -5

\textbf{SAGGEZZA} +0

\textbf{CARISMA} -4

\textbf{Iniziativa} +2 -- \textbf{Difesa} 13

\textbf{Punti Ferita} 13 (2d10 + 2)

\textbf{Movimento} 9 m, nuoto 9 m

\textbf{Tiri Salvezza}: Tempra +3, Riflessi +2, Volontà +0

\textbf{Sensi} vista cieca 3 m

\textbf{Lingue} -

\textbf{Sfida} 1/4 (50 PE)

\textbf{Azioni}

\emph{\textbf{Morso.} Attacco con Arma da Mischia}: +4 a colpire, portata 1 m, una creatura.

\emph{Colpisce:} 5 (1d6 + 2) danni perforanti.

\emph{\textbf{Stritolare.} Attacco con Arma da Mischia}: +4 a colpire, portata 1 m, una creatura.

\emph{Colpisce:} 6 (1d8 + 2) danni da botta, e il bersaglio è afferrato (CD 14 per fuggire). Fino al termine dell'afferrare, la creatura è intralciata, e il serpente non può stritolare un altro bersaglio.

\medskip\textbf{Serpente Costrittore Gigante}\index{Mostri - Serpente Costrittore Gigante}

\emph{Enorme bestia, disallineato}

\textbf{FORZA} +4

\textbf{DESTREZZA} +2

\textbf{COSTITUZIONE} +1

\textbf{INTELLIGENZA} -5

\textbf{SAGGEZZA} +0

\textbf{CARISMA} -4

\textbf{Iniziativa} +2 -- \textbf{Difesa} 13

\textbf{Punti Ferita} 60 (8d12 + 8)

\textbf{Movimento} 9 m, nuoto 9 m

\textbf{Tiri Salvezza}: Tempra +3, Riflessi +2, Volontà +0

\textbf{Competenze} Consapevolezza +2

\textbf{Sensi} vista cieca 3 m

\textbf{Lingue} -

\textbf{Sfida} 2 (450 PE)

\textbf{Azioni}

\emph{\textbf{Morso.} Attacco con Arma da Mischia}: +6 a colpire, portata 3 m, una creatura.

\emph{Colpisce:} 11 (2d6 + 4) danni perforanti.

\emph{\textbf{Stritolare.} Attacco con Arma da Mischia}: +6 a colpire, portata 1 m, una creatura.

\emph{Colpisce:} 13 (2d8 + 4) danni da botta, e il bersaglio è afferrato (CD 16 per fuggire). Fino al termine dell'afferrare, la creatura è intralciata, e il serpente non può stritolare un altro bersaglio.

\medskip\textbf{Serpente Velenoso}\index{Mostri - Serpente Velenoso}

\emph{Minuscola bestia, disallineato}

\textbf{FORZA} -4

\textbf{DESTREZZA} +3

\textbf{COSTITUZIONE} +0

\textbf{INTELLIGENZA} -5

\textbf{SAGGEZZA} +0

\textbf{CARISMA} -4

\textbf{Iniziativa} +3 -- \textbf{Difesa} 14

\textbf{Punti Ferita} 2 (1d4)

\textbf{Movimento} 9 m, nuoto 9 m

\textbf{Tiri Salvezza}: Tempra +1, Riflessi +4, Volontà +1

\textbf{Sensi} vista cieca 3 m

\textbf{Lingue} -

\textbf{Sfida} 1/8 (25 PE)

\textbf{Azioni}

\emph{\textbf{Morso.} Attacco con Arma da Mischia}: +5 a colpire, portata 1 m, un bersaglio.

\emph{Colpisce:} 1 danno perforante e il bersaglio deve effettuare un Tiro Salvezza di Tempra CD 10, e subire 5 (2d4) danni da veleno se fallisce il Tiro Salvezza, o la metà di questi danni se lo riesce.

\medskip\textbf{Serpente Velenoso Gigante}\index{Mostri - Serpente Velenoso Gigante}

\emph{Media bestia, disallineato}

\textbf{FORZA} +0

\textbf{DESTREZZA} +4

\textbf{COSTITUZIONE} +1

\textbf{INTELLIGENZA} -4

\textbf{SAGGEZZA} +0

\textbf{CARISMA} -4

\textbf{Iniziativa} +4 -- \textbf{Difesa} 15

\textbf{Punti Ferita} 11 (2d8 + 2)

\textbf{Movimento} 9 m, nuoto 9 m

\textbf{Tiri Salvezza}: Tempra +1, Riflessi +5, Volontà +2

\textbf{Competenze} Consapevolezza +2

\textbf{Sensi} vista cieca 3 m

\textbf{Lingue} -

\textbf{Sfida} 1/4 (50 PE)

\textbf{Azioni}

\emph{\textbf{Morso.} Attacco con Arma da Mischia}: +6 a colpire, portata 3 m, un bersaglio.

\emph{Colpisce:} 6 (1d4 + 4) danni perforanti e il bersaglio deve effettuare un Tiro Salvezza di Tempra CD 11, e subire 10 (3d6) danni da veleno se fallisce il Tiro Salvezza, o la metà di questi danni se lo riesce.

\medskip\textbf{Serpente Volante}\index{Mostri - Serpente Volante}

Un serpente volante è una serpe alata, dai colori intensi, rinvenuta in giungle remote.

\emph{Minuscola bestia, disallineato}

\textbf{FORZA} -3

\textbf{DESTREZZA} +4

\textbf{COSTITUZIONE} +0

\textbf{INTELLIGENZA} -4

\textbf{SAGGEZZA} +1

\textbf{CARISMA} -3

\textbf{Iniziativa} +4 -- \textbf{Difesa} 15

\textbf{Punti Ferita} 5 (2d4)

\textbf{Movimento} 9 m, nuoto 9 m, volo 18 m

\textbf{Tiri Salvezza}: Tempra -2, Riflessi +5, Volontà +1

\textbf{Sensi} vista cieca 3 m

\textbf{Lingue} -

\textbf{Sfida} 1/8 (25 PE)

\emph{\textbf{Sorvolare.}} Il serpente non provoca attacchi di opportunità quando vola via dalla portata di un nemico.

\textbf{Azioni}

\emph{\textbf{Morso.} Attacco con Arma da Mischia}: +6 a colpire, portata 1 m, un bersaglio.

\emph{Colpisce:} 1 danno perforante più 7 (3d4) danni da veleno.

\medskip\textbf{Squalo Cacciatore}\index{Mostri - Squalo Cacciatore}

Uno squalo cacciatore è lungo da 4 a 6 metri e di solito caccia in solitario nelle acque più profonde.

\emph{Grande bestia, disallineato}

\textbf{FORZA} +4

\textbf{DESTREZZA} +1

\textbf{COSTITUZIONE} +2

\textbf{INTELLIGENZA} -5

\textbf{SAGGEZZA} +0

\textbf{CARISMA} -3

\textbf{Iniziativa} +1 -- \textbf{Difesa} 13

\textbf{Punti Ferita} 45 (6d10 + 12)

\textbf{Movimento} 0 m, nuoto 12 m

\textbf{Tiri Salvezza}: Tempra +4, Riflessi +2, Volontà +0

\textbf{Competenze} Consapevolezza +2

\textbf{Sensi} vista cieca 9 m

\textbf{Lingue} -

\textbf{Sfida} 2 (450 PE)

\emph{\textbf{Frenesia Sanguinaria.}} Lo squalo ha +1d6 ai tiri di attacco in mischia contro qualsiasi creatura che non sia al massimo dei punti ferita.

\emph{\textbf{Respirare Acqua.}} Lo squalo può respirare solo sott'acqua.

\textbf{Azioni}

\emph{\textbf{Morso.} Attacco con Arma da Mischia}: +6 a colpire, portata 1 m, un bersaglio.

\emph{Colpisce:} 13 (2d8 + 4) danni perforanti.

\medskip\textbf{Squalo Corallino}\index{Mostri - Squalo Corallino}

Gli squali corallini sono lunghi da 2 a 3 metri e vivono nelle acque meno profonde e lungo le barriere coralline.

\emph{Media bestia, disallineato}

\textbf{FORZA} +2

\textbf{DESTREZZA} +1

\textbf{COSTITUZIONE} +1

\textbf{INTELLIGENZA} -5

\textbf{SAGGEZZA} +0

\textbf{CARISMA} -3

\textbf{Iniziativa} +1 -- \textbf{Difesa} 13

\textbf{Punti Ferita} 22 (4d8 + 4)

\textbf{Movimento} 0 m, nuoto 12 m

\textbf{Tiri Salvezza}: Tempra +2, Riflessi +2, Volontà +1

\textbf{Competenze} Consapevolezza +2

\textbf{Sensi} vista cieca 9 m

\textbf{Lingue} -

\textbf{Sfida} 1/2 (100 PE)

\emph{\textbf{Respirare Acqua.}} Lo squalo può respirare solo sottacqua.

\emph{\textbf{Tattiche di Branco.}} Lo squalo ha +1d6 al tiro di attacco contro una creatura se almeno uno degli alleati dello squalo si trova entro 1 metro dalla creatura e quell'alleato non è inabile.

\textbf{Azioni}

\emph{\textbf{Morso.} Attacco con Arma da Mischia}: +4 a colpire, portata 1 m, un bersaglio.

\emph{Colpisce:} 6 (1d8 + 2) danni perforanti.

\medskip\textbf{Squalo Gigante}\index{Mostri - Squalo Gigante}

Lo squalo gigante è lungo 9 metri e lo si incontra

normalmente solo negli oceani più profondi.

\emph{Enorme bestia, disallineato}

\textbf{FORZA} +6

\textbf{DESTREZZA} +0

\textbf{COSTITUZIONE} +5

\textbf{INTELLIGENZA} -5

\textbf{SAGGEZZA} +0

\textbf{CARISMA} -3

\textbf{Iniziativa} +0 -- \textbf{Difesa} 16

\textbf{Punti Ferita} 126 (11d12 + 55)

\textbf{Movimento} 0 m, nuoto 15 m

\textbf{Tiri Salvezza}: Tempra +7, Riflessi +2, Volontà +1

\textbf{Competenze} Consapevolezza +3

\textbf{Sensi} vista cieca 18 m

\textbf{Lingue} -

\textbf{Sfida} 5 (1.800 PE)

\emph{\textbf{Frenesia Sanguinaria.}} Lo squalo ha +1d6 ai tiri di attacco

in mischia contro qualsiasi creatura che non sia al massimo dei punti ferita.

\emph{\textbf{Respirare Acqua.}} Lo squalo può respirare solo sottacqua. 

\textbf{Azioni}

\emph{\textbf{Morso.} Attacco con Arma da Mischia}: +9 a colpire, portata 1 m, un bersaglio.

\emph{Colpisce:} 22 (3d10 + 6) danni perforanti.

\medskip\textbf{Strige}\index{Mostri - Strige}

Questo orrendo mostro sembra un incrocio tra un grosso pipistrello e una zanzara sovradimensionata. Le sue zampe terminano in lunghe pinze, e la sua lunga proboscide, simile ad un ago, fende l'aria mentre cerca di nutrirsi del sangue delle creature viventi. 

\emph{Minuscola bestia, disallineato}

\textbf{FORZA} -3

\textbf{DESTREZZA} +3

\textbf{COSTITUZIONE} +0

\textbf{INTELLIGENZA} -4

\textbf{SAGGEZZA} -1

\textbf{CARISMA} -2

\textbf{Iniziativa} +3 -- \textbf{Difesa} 15

\textbf{Punti Ferita} 2 (1d4)

\textbf{Movimento} 3 m, volo 12 m

\textbf{Tiri Salvezza}: Tempra -3, Riflessi +4, Volontà -1

\textbf{Sensi} visione al buio 18 m

\textbf{Lingue} -

\textbf{Sfida} 1/8 (25 PE)

\textbf{Azioni}

\emph{\textbf{Risucchio di Sangue.} Attacco con Arma da Mischia}: +5 a colpire, portata 1 m, una creatura.

\emph{Colpisce:} 5 (1d4 + 3) danni perforanti e lo strige si attacca al bersaglio. Mentre è attaccato, lo strige non attacca. Invece, all'inizio di ciascun turno dello strige, il bersaglio perde 5 (1d4 + 3) punti ferita a causa della perdita di sangue.

Lo strige può staccarsi spendendo 1 metro di movimento. Lo fa automaticamente dopo aver risucchiato 10 punti ferita dal bersaglio o alla morte del bersaglio. Una creatura, compreso il bersaglio, può usare la sua azione per staccare lo strige.

\medskip\textbf{Tasso}\index{Mostri - Tasso}

\emph{Minuscola bestia, disallineato}

\textbf{FORZA} -3

\textbf{DESTREZZA} +0

\textbf{COSTITUZIONE} +1

\textbf{INTELLIGENZA} -4

\textbf{SAGGEZZA} +1

\textbf{CARISMA} -3

\textbf{Iniziativa} +0 -- \textbf{Difesa} 11

\textbf{Punti Ferita} 3 (1d4 + 1)

\textbf{Movimento} 6 m, scavo 1 m

\textbf{Tiri Salvezza}: Tempra -3, Riflessi +1, Volontà +1

\textbf{Sensi} visione al buio 9 m

\textbf{Lingue} -

\textbf{Sfida} 0 (10 PE)

\emph{\textbf{Olfatto Affinato.}} Il tasso ha +1d6 alle prove di Saggezza (Consapevolezza) basate sull'olfatto.

\textbf{Azioni}

\emph{\textbf{Morso.} Attacco con Arma da Mischia}: +2 a colpire, portata 1 m, un bersaglio.

\emph{Colpisce:} 1 danno perforante.

\medskip\textbf{Tasso Gigante}\index{Mostri - Tasso Gigante}

\emph{Media bestia, disallineato}

\textbf{FORZA} +1

\textbf{DESTREZZA} +0

\textbf{COSTITUZIONE} +2

\textbf{INTELLIGENZA} -4

\textbf{SAGGEZZA} +1

\textbf{CARISMA} -3

\textbf{Iniziativa} +0 -- \textbf{Difesa} 11

\textbf{Punti Ferita} 13 (2d8 + 4)

\textbf{Movimento} 9 m, scavo 3 m

\textbf{Tiri Salvezza}: Tempra +2, Riflessi +1, Volontà +2

\textbf{Sensi} visione al buio 9 m

\textbf{Lingue} -

\textbf{Sfida} 1/4 (50 PE)

\emph{\textbf{Olfatto Affinato.}} Il tasso ha +1d6 alle prove di Saggezza (Consapevolezza) basate sull'olfatto.

\textbf{Azioni}

\emph{\textbf{Multiattacco.}} Il tasso effettua due attacchi: uno con il morso e uno con gli artigli.

\emph{\textbf{Artigli.} Attacco con Arma da Mischia}: +3 a colpire,  portata 1 m, un bersaglio.

\emph{Colpisce:} 6 (2d4 + 1) danni taglienti.

\emph{\textbf{Morso.} Attacco con Arma da Mischia}: +3 a colpire, portata 1 m, un bersaglio.

\emph{Colpisce:} 4 (1d6 + 1) danni perforanti.

\medskip\textbf{Tigre}\index{Mostri - Tigre}

\emph{Grande bestia, disallineato}

\textbf{FORZA} +3

\textbf{DESTREZZA} +2

\textbf{COSTITUZIONE} +2

\textbf{INTELLIGENZA} -4

\textbf{SAGGEZZA} +1

\textbf{CARISMA} -1

\textbf{Iniziativa} +2 -- \textbf{Difesa} 13

\textbf{Punti Ferita} 37 (5d10 + 10)

\textbf{Movimento} 12 m

\textbf{Tiri Salvezza}: Tempra +4, Riflessi +4, Volontà +2

\textbf{Competenze} Muoversi Silenziosamente / Nascondersi +6, Consapevolezza +3

\textbf{Sensi} visione al buio 18 m

\textbf{Lingue} -

\textbf{Sfida} 1 (200 PE)

\emph{\textbf{Balzo.}} Se la tigre si muove di almeno 6 metri diretta verso una creatura e la colpisce con un attacco di artiglio durante lo stesso turno, il bersaglio deve riuscire un Tiro Salvezza di Tempra CD 13 o cadere prono. Se il bersaglio è prono, la tigre può effettuare un attacco di morso contro di esso come azione bonus.

\emph{\textbf{Olfatto Affinato.}} La tigre ha +1d6 alle prove di Saggezza (Consapevolezza) basate sull'olfatto.

\textbf{Azioni}

\emph{\textbf{Artiglio.} Attacco con Arma da Mischia}: +5 a colpire, portata 1 m, un bersaglio.

\emph{Colpisce:} 7 (1d8 + 3) danni taglienti.

\emph{\textbf{Morso.} Attacco con Arma da Mischia}: +5 a colpire, portata 1 m, un bersaglio.

\emph{Colpisce:} 8 (1d10 + 3) danni perforanti.

\medskip\textbf{Tigre dai Denti a Sciabola}\index{Mostri - Tigre dai Denti a Sciabola}

\emph{Grande bestia, disallineato}

\textbf{FORZA} +4

\textbf{DESTREZZA} +2

\textbf{COSTITUZIONE} +2

\textbf{INTELLIGENZA} -4

\textbf{SAGGEZZA} +1

\textbf{CARISMA} -1

\textbf{Iniziativa} +2 -- \textbf{Difesa} 13

\textbf{Punti Ferita} 52 (7d10 + 14)

\textbf{Movimento} 12 m

\textbf{Tiri Salvezza}: Tempra +5, Riflessi +3, Volontà +2

\textbf{Competenze} Muoversi Silenziosamente / Nascondersi +6, Consapevolezza +3

\textbf{Lingue} -

\textbf{Sfida} 2 (450 PE)

\emph{\textbf{Balzo.}} Se la tigre si muove di almeno 6 metri diretta verso una creatura e la colpisce con un attacco di artiglio durante lo stesso turno, il bersaglio deve riuscire un Tiro Salvezza di Tempra CD 14 o cadere prono. Se il bersaglio è prono, la tigre può effettuare un attacco di morso contro di esso come azione bonus.

\emph{\textbf{Olfatto Affinato.}} La tigre ha +1d6 alle prove di Saggezza (Consapevolezza) basate sull'olfatto.

\textbf{Azioni}

\emph{\textbf{Artiglio.} Attacco con Arma da Mischia}: +6 a colpire, portata 1 m, un bersaglio.

\emph{Colpisce:} 12 (2d6 + 5) danni taglienti.

\emph{\textbf{Morso.} Attacco con Arma da Mischia}: +6 a colpire, portata 1 m, un bersaglio.

\emph{Colpisce:} 10 (1d10 + 5) danni perforanti.

\medskip\textbf{Topi}\\\index{Mostri - Topi}
\emph{Minuscola fatata}\\
\textbf{Forza}: -1\\
\textbf{Destrezza}: +4\\
\textbf{Costituzione}: +0\\
\textbf{Intelligenza}: +6\\
\textbf{Saggezza}: +2\\
\textbf{Carisma}: +6\\
\textbf{Difesa}: 17 -- \textbf{Iniziativa}: +15\\
\textbf{Punti Ferita}: 4 (1d10 - 1)\\
\textbf{Movimento}: 6 m\\
\textbf{Tiri Salvezza}: Tempra +20, Riflessi +30, Volontà +20 \\
\textbf{Sensi}: Senso tellurico 30 , Scurovisione 9 m, Visione del Vero 30 m\\
\textbf{Lingue}: tutte\\
\textbf{Sfida}: 0 (10 PE)\smallskip\\
\textbf{Immunità}: al danno delle armi con bonus magico inferiore a +6\\
\textbf{Immunità}: a qualsiasi magia la Topi non voglia essere influenzata\\
\emph{\textbf{E' la Topi}} La Topi ha +3d6 (oppure +18) ogni volta che deve tirare dei dadi o contare un valore.
Qualsiasi attacco effettuato dalla Topi e' considerato magico +5 e non e' resistibile.\\
\smallskip\textbf{Azioni}\\
\emph{\textbf{Musetto}} ogni creatura a scelta di Topi, entro 30 metri, subisce un Musetto. La creatura viene allontanata di 2d6 metri e subisce 3d6 danni\\
\emph{\textbf{Morso topetto} Attacco con Arma da Mischia}: +26 al colpire, portata 1 m, un bersaglio.\\
\emph{Colpisce:} 6 danno perforante.\\
\emph{\textbf{Graffiotto} fino a 8 Attacchi con Arma da Mischia}: colpisce automaticamente, portata 1 m, fino a 4 bersagli.\\
\emph{Colpisce:} 1 danno perforante, ignora ogni Resistenza, immunità o protezione.\\


\medskip\textbf{Vespa Gigante}\index{Mostri - Vespa Gigante}

\emph{Media bestia, disallineato}

\textbf{FORZA} +0

\textbf{DESTREZZA} +2

\textbf{COSTITUZIONE} +0

\textbf{INTELLIGENZA} -5

\textbf{SAGGEZZA} +0

\textbf{CARISMA} -4

\textbf{Iniziativa} +2 -- \textbf{Difesa} 13

\textbf{Punti Ferita} 13 (3d8)

\textbf{Movimento} 3 m, volo 15 m

\textbf{Tiri Salvezza}: Tempra +1, Riflessi +3, Volontà +0 

\textbf{Lingue} -

\textbf{Sfida} 1/2 (100 PE)

\textbf{Azioni}

\emph{\textbf{Pungiglione.} Attacco con Arma da Mischia}: +4 a colpire, portata 1 m, una creatura.

\emph{Colpisce:} 5 (1d6 + 2) danni perforanti e il bersaglio deve effettuare un Tiro Salvezza di Tempra CD 11, e subire 10 (3d6) danni da veleno se fallisce il Tiro Salvezza, o la metà di questi danni se lo riesce. Se il danno da veleno riduce il bersaglio a 0 punti ferita, il bersaglio è stabile ma avvelenato per 1 ora, anche dopo aver recuperato i punti ferita, e mentre è avvelenato in questo modo resta paralizzato.

\medskip\textbf{Worg}\index{Mostri - Worg}

I worg sono mostruosi predatori dall'aspetto simile ad un lupo che amano cacciare e divorare le creature più deboli di loro.

\emph{Grande mostruosità, neutrale malvagio}

\textbf{FORZA} +3

\textbf{DESTREZZA} +1

\textbf{COSTITUZIONE} +1

\textbf{INTELLIGENZA} -2

\textbf{SAGGEZZA} +0

\textbf{CARISMA} -1

\textbf{Iniziativa} +1 -- \textbf{Difesa} 14

\textbf{Punti Ferita} 26 (4d10 + 4)

\textbf{Movimento} 15 m

\textbf{Tiri Salvezza}: Tempra +3, Riflessi +2, Volontà +2 

\textbf{Competenze} Consapevolezza +4

\textbf{Sensi} visione al buio 18 m

\textbf{Lingue} Goblin, Worg

\textbf{Sfida} 1/2 (100 PE)

\emph{\textbf{Udito e Olfatto Affinato.}} Il worg ha +1d6 nelle prove di Saggezza (Consapevolezza) basate su udito o olfatto.

\textbf{Azioni}

\emph{\textbf{Morso.} Attacco con Arma da Mischia}: +5 a colpire, portata 1 m, un bersaglio.

\emph{Colpisce:} 10 (2d6 + 3) danni perforanti. Se il bersaglio è una creatura, deve riuscire un Tiro Salvezza di Tempra CD 13 o cadere prona.

\subsection{Appendice B: Personaggi Non Giocanti}\index{Mostri - Personaggi Non Giocanti}

Questa appendice contiene le statistiche di vari personaggi non giocanti (PNG) umanoidi che gli avventurieri possono incontrare nel corso di una campagna, da infimi popolani a potenti arcimaghi. Queste statistiche possono essere utilizzate per rappresentare PNG umani e non.

Personalizzare i PNG

Esistono molti semplici modi di personalizzare i PNG di questa appendice per l'uso nella tua campagna casalinga.

\emph{\textbf{Cambiare Incantesimi.}} Un modo per personalizzare un PNG incantatore è quello di rimpiazzare uno o più dei suoi incantesimi. Puoi sostituire qualsiasi incantesimo della lista di
incantesimi del PNG con un diverso incantesimo della stessa Difficoltà. Cambiare incantesimi in questo modo non modifica il grado di sfida del PNG.

\textbf{\emph{Cambiare Armi e Armatura}.} Puoi migliorare o peggiorare l'armatura del PNG o aggiungere o cambiare armi. Le modifiche alla Difesa e ai danni possono modificare il grado di sfida del PNG.

\emph{\textbf{Oggetti Magici}}. Più potente è un PNG, maggiori le probabilità che possieda uno o più
oggetti magici. Un mago, ad esempio, potrebbe avere una bacchetta o un bastone magico, oltre ad una o più pozioni e pergamene. Fornire un PNG di un potente oggetto magico capace di infliggere danni potrebbe modificarne il grado di sfida.

Alcuni oggetti magici di esempio sono descritti più avanti in questo documento.

\textbf{Combattenti}

I combattenti sono individui che si guadagnano da vivere mettendo la loro spada al servizio di un individuo o un ideale.

\medskip\textbf{Guardia}

Le guardie comprendono membri della ronda cittadina, sentinelle di una cittadella o città fortificata e le guardie del corpo di nobili e mercanti.

\emph{Media umanoide (qualsiasi razza), qualsiasi allineamento} 

\textbf{FORZA} +1

\textbf{DESTREZZA} +1

\textbf{COSTITUZIONE} +1

\textbf{INTELLIGENZA} +0

\textbf{SAGGEZZA} +0

\textbf{CARISMA} +0

\textbf{Iniziativa} +1 -- \textbf{Difesa} 17 (giaco di maglia, scudo)

\textbf{Punti Ferita} 11 (2d8 + 2)

\textbf{Movimento} 9 m

\textbf{Tiri Salvezza}: Tempra +3, Riflessi +1, Volontà +1 

\textbf{Competenze} Consapevolezza +2


\textbf{Lingue} una qualsiasi lingua (di solito il Comune)

\textbf{Sfida} 1/8 (25 PE)

\textbf{Azioni}

\emph{\textbf{Lancia.} Attacco con Arma da Mischia o a Gittata}: +3 a colpire, portata 1 m o gittata 6m, un bersaglio.

\emph{Colpisce:} 4 (1d6 + 1) danni perforanti o 5 (1d8 + 1) danni perforanti se impiegata con due mani per effettuare un attacco da mischia.

\medskip\textbf{Veterano}

Guerrieri sopravvissuti a lungo, guadagnandosi una grande fama di esperti e abili combattenti.

\emph{Media umanoide (qualsiasi razza), qualsiasi allineamento}

\textbf{FORZA} +3

\textbf{DESTREZZA} +1

\textbf{COSTITUZIONE} +2

\textbf{INTELLIGENZA} +0

\textbf{SAGGEZZA} +0

\textbf{CARISMA} +0

\textbf{Iniziativa} +1 -- \textbf{Difesa} 19 (armatura di strisce)

\textbf{Punti Ferita} 58 (9d8 + 18)

\textbf{Movimento} 9 m

\textbf{Tiri Salvezza}: Tempra +4, Riflessi +2, Volontà +3 

\textbf{Competenze} Acrobatica +5, Consapevolezza +2

\textbf{Lingue} una lingua qualsiasi (di solito il Comune)

\textbf{Sfida} 3 (700 PE)

\textbf{Azioni}

\emph{\textbf{Multiattacco.}} Il veterano effettua due attacchi con la spada lunga. Se ha estratto una spada corta, può effettuare anche un attacco con la spada corta.

\emph{\textbf{Spada Lunga.} Attacco con Arma da Mischia}: +5 a colpire, portata 1 m, un bersaglio.

\emph{Colpisce:} 7 (1d8 + 3) danni taglienti, o 8 (1d10 + 3) danni taglienti se usata con due mani.

\emph{\textbf{Spada Corta.} Attacco con Arma da Mischia}: +5 a colpire, portata 1 m, un bersaglio.

\emph{Colpisce:} 6 (1d6 + 3) danni perforanti.

\emph{\textbf{Balestra Pesante.} Attacco con Arma a Gittata}: +3 a colpire, gittata 30m, un bersaglio. \emph{Colpisce:} 6 (1d10 + 1) danni perforanti.

\medskip\textbf{Cavaliere}

I cavalieri sono combattenti che giurano fedeltà a sovrani, ordini religiosi, e nobili cause. L'allineamento del cavaliere determina fino a che punto è disposto ad onorare il suo giuramento.

\emph{Media umanoide (qualsiasi razza), qualsiasi allineamento}

\textbf{FORZA} +3

\textbf{DESTREZZA} +0

\textbf{COSTITUZIONE} +2

\textbf{INTELLIGENZA} +0

\textbf{SAGGEZZA} +0

\textbf{CARISMA} +2

\textbf{Iniziativa} +0 -- \textbf{Difesa} 20 (armatura di piastre)

\textbf{Punti Ferita} 52 (8d8 + 16)

\textbf{Movimento} 9 m

\textbf{Tiri Salvezza}: Tempra +4, Riflessi +1, Volontà +3

\textbf{Lingue} una qualsiasi lingua (di solito il Comune)

\textbf{Sfida} 3 (700 PE)

\emph{\textbf{Coraggioso.}} Il cavaliere ha +1d6 ai Tiri Salvezza contro l'essere spaventato.

\textbf{Azioni}

\emph{\textbf{Multiattacco.}} Il cavaliere effettua due attacchi da mischia.

\emph{\textbf{Spada Grossa.} Attacco con Arma da Mischia}: +5 a colpire, portata 1 m, un bersaglio.

\emph{Colpisce:} 10 (2d6 + 3) danni taglienti.

\emph{\textbf{Balestra Pesante.} Attacco con Arma a Gittata}: +2 a colpire, gittata 30m, un bersaglio.

\emph{Colpisce:} 5 (1d10) perforanti.

\emph{\textbf{Autorità (Ricarica dopo un 1 ora)}}. Per 1 minuto, il cavaliere può pronunciare un comando speciale o avvertimento ogni qualvolta una creatura non ostile entro 9 metri da lui, e che possa vedere, effettua un tiro di attacco o Tiro Salvezza. La creatura può sommare un d4 al suo tiro purchè possa udire e comprendere il cavaliere. Una creatura può beneficiare di un solo dado Autorità alla volta. Questo effetto termina se il cavaliere è inabile.

\textbf{Reazioni}

\emph{\textbf{Parata.}} Il cavaliere può aggiungere 2 alla sua Difesa contro un attacco da mischia che lo colpirebbe. Per farlo, il cavaliere deve vedere l'attaccante e star impugnando un'arma da mischia.

\medskip\textbf{Gladiatore}

Addestrati per intrattenere le folle, sono tra i combattenti più pericolosi in circolazione.

\emph{Media umanoide (qualsiasi razza), qualsiasi allineamento}
\textbf{FORZA} +4

\textbf{DESTREZZA} +2

\textbf{COSTITUZIONE} +3

\textbf{INTELLIGENZA} +0

\textbf{SAGGEZZA} +1

\textbf{CARISMA} +2

\textbf{Iniziativa} +2 -- \textbf{Difesa} 19 (armatura di cuoio borchiato, scudo)

\textbf{Punti Ferita} 112 (15d8 + 45)

\textbf{Movimento} 9 m

\textbf{Tiri Salvezza}: Tempra +5, Riflessi +5, Volontà +3 

\textbf{Competenze} Acrobatica +10, Intimidazione +5

\textbf{Lingue} una lingua qualsiasi (di solito il Comune)

\textbf{Sfida} 5 (1.800 PE)

\emph{\textbf{Bruto.}} Un'arma da mischia infligge un dado aggiuntivo di danno

quando un gladiatore colpisce con essa (già incluso nell'attacco).

\emph{\textbf{Coraggioso.}} Il gladiatore ha +1d6 ai Tiri Salvezza contro l'essere spaventato.

\textbf{Azioni}

\emph{\textbf{Multiattacco.}} Il gladiatore effettua tre attacchi da mischia o due attacchi a gittata.

\emph{\textbf{Lancia.} Attacco con Arma da Mischia o a Gittata}: +7 a colpire, portata 1 m o gittata 6m, un bersaglio.

\emph{Colpisce:} 11 (2d6 + 4) danni perforanti, o 13 (2d8 + 4) danni taglienti se usata con due mani.

\emph{\textbf{Botta di Scudo.} Attacco con Arma da Mischia}: +7 a colpire, portata 1 m, un bersaglio.

\emph{Colpisce:} 9 (2d4 + 4) danni da botta. Se il bersaglio è una creatura di taglia Media o inferiore, deve riuscire un Tiro Salvezza su Tempra CD 15 o cadere prono.

\textbf{Reazioni}

\emph{\textbf{Parata.}} Il gladiatore somma 3 alla sua Difesa contro un attacco da mischia che lo colpirebbe. Per farlo, il gladiatore deve vedere l'attaccante e impugnare un'arma da mischia.

\medskip\textbf{Cittadini}

In questa categoria rientrano quegli individui che si occupano di mandare avanti il mondo, svolgendo le mansioni necessarie affinché i campi vengano coltivati, le città amministrate, il cibo coltivato e
nuovi territori esplorati.

\medskip\textbf{Nobile}

I nobili comandano sulla popolazione, in virtù di un diritto di nascita o per le ricchezze accumulate. Tra costoro si annoverano anche i cortigiani che affollano le corti dei ricchi e dei potenti.

\emph{Media umanoide (qualsiasi razza), qualsiasi allineamento}

\textbf{FORZA} +0

\textbf{DESTREZZA} +1

\textbf{COSTITUZIONE} +0

\textbf{INTELLIGENZA} +1

\textbf{SAGGEZZA} +2

\textbf{CARISMA} +3

\textbf{Iniziativa} +1 -- \textbf{Difesa} 16 (pettorale)

\textbf{Punti Ferita} 9 (2d8)

\textbf{Movimento} 9 m

\textbf{Tiri Salvezza}: Tempra +1, Riflessi +1, Volontà +2 

\textbf{Competenze} Percepire Emozioni +4, Ingannare +5

\textbf{Lingue} due lingue qualsiasi

\textbf{Sfida} 1/8 (25 PE)

\textbf{Azioni}

\emph{\textbf{Stocco.} Attacco con Arma da Mischia}: +3 a colpire, portata 1 m, un bersaglio.

\emph{Colpisce:} 5 (1d8 + 1) danni perforanti.

\textbf{Reazioni}

\emph{\textbf{Parata.}} Il nobile somma 2 alla sua Difesa contro un attacco da mischia che lo colpirebbe. Per farlo, il nobile deve vedere

l'attaccante e impugnare un'arma da mischia.

\medskip\textbf{Popolano}

I popolani comprendono contadini, servi, schiavi, servitori, pellegrini, mercanti, artigiani ed eremiti.

\emph{Media umanoide (qualsiasi razza), qualsiasi allineamento}

\textbf{FORZA} +0

\textbf{DESTREZZA} +0

\textbf{COSTITUZIONE} +0

\textbf{INTELLIGENZA} +0

\textbf{SAGGEZZA} +0

\textbf{CARISMA} +0

\textbf{Iniziativa} +0 -- \textbf{Difesa} 11

\textbf{Punti Ferita} 4 (1d8)

\textbf{Movimento} 9 m

\textbf{Tiri Salvezza}: Tempra +0, Riflessi +0, Volontà +0 

\textbf{Lingue} una qualsiasi lingua (di solito il Comune)

\textbf{Sfida} 0 (10 PE)

\textbf{Azioni}

\emph{\textbf{Randello.} Attacco con Arma da Mischia}: +2 a colpire, portata 1 m, un bersaglio.

\emph{Colpisce:} 2 (1d4) danni da botta.

\medskip\textbf{Criminali}

I criminali sono individui che vivono al margine della legalità, procurandosi il pane svolgendo attività spesso considerate illecite e immorali.

\medskip\textbf{Picchiatore}

I picchiatori sono criminali spietati abili nell'intimidire e perpetrare atti di violenza. Lavorano per soldi e si fanno pochi scrupoli.

\emph{Media umanoide (qualsiasi razza), qualsiasi allineamento non buono}

\textbf{FORZA} +2

\textbf{DESTREZZA} +0

\textbf{COSTITUZIONE} +2

\textbf{INTELLIGENZA} +0

\textbf{SAGGEZZA} +0

\textbf{CARISMA} +0

\textbf{Iniziativa} +0 -- \textbf{Difesa} 12 (armatura di cuoio)

\textbf{Punti Ferita} 32 (5d8 + 10)

\textbf{Movimento} 9 m

\textbf{Tiri Salvezza}: Tempra +3, Riflessi +1, Volontà +0 

\textbf{Competenze} Intimidazione +2

\textbf{Lingue} una lingua qualsiasi (di solito il Comune)

\textbf{Sfida} 1/2 (100 PE)

\emph{\textbf{Tattiche di Branco.}} Il picchiatore ha +1d6 ai tiri di attacco contro una creatura se almeno uno degli alleati del picchiatore si trova entro 1 metro dalla creatura e quell'alleato non
è inabile.

\textbf{Azioni}

\emph{\textbf{Multiattacco.}} Il picchiatore effettua due attacchi da mischia.

\emph{\textbf{Mazza.} Attacco con Arma da Mischia}: +4 a colpire, portata 1 m, una creatura.

\emph{Colpisce:} 5 (1d6 + 2) danni da botta.

\emph{\textbf{Balestra Pesante.} Attacco con Arma a Gittata}: +2 a colpire, gittata 30m, un bersaglio. \emph{Colpisce:} 5 (1d10) danni perforanti.

\medskip\textbf{Bandito/Pirata}

Che siano uomini di strada o di mare (pirati) costoro guadagnano da vivere depredando il prossimo.

\emph{Media umanoide (qualsiasi razza), qualsiasi allineamento non legale}

\textbf{FORZA} +0

\textbf{DESTREZZA} +1

\textbf{COSTITUZIONE} +1

\textbf{INTELLIGENZA} +0

\textbf{SAGGEZZA} +0

\textbf{CARISMA} +0

\textbf{Iniziativa} +1 -- \textbf{Difesa} 13 (armatura di cuoio)

\textbf{Punti Ferita} 11 (2d8 + 2)

\textbf{Movimento} 9 m

\textbf{Tiri Salvezza}: Tempra +1, Riflessi +2, Volontà +1 

\textbf{Lingue} una qualsiasi lingua (di solito il Comune)

\textbf{Sfida} 1/8 (25 PE)

\textbf{Azioni}

\emph{\textbf{Scimitarra.} Attacco con Arma da Mischia}: +3 a colpire, portata 1 m, un bersaglio.

\emph{Colpisce:} 4 (1d6 + 1) danni taglienti.

\emph{\textbf{Balestra Leggera.} Attacco con Arma a Gittata}: +3 a colpire, gittata 24m, un bersaglio. \emph{Colpisce:} 5 (1d8 + 1) danni taglienti.

\medskip\textbf{Spia}

Una spia è un individuo addestramento nel reperire segreti per conto di qualcuno, o a volte per rivenderli al miglior offerente.

\emph{Media umanoide (qualsiasi razza), qualsiasi allineamento}

\textbf{FORZA} +0

\textbf{DESTREZZA} +2

\textbf{COSTITUZIONE} +0

\textbf{INTELLIGENZA} +1

\textbf{SAGGEZZA} +2

\textbf{CARISMA} +3

\textbf{Iniziativa} +2 -- \textbf{Difesa} 13

\textbf{Punti Ferita} 27 (6d8)

\textbf{Movimento} 9 m

\textbf{Tiri Salvezza}: Tempra +2, Riflessi +3, Volontà +3 

\textbf{Competenze} Muoversi Silenziosamente / Nascondersi +4, Percepire Emozioni +4, Investigazione +5, Consapevolezza +6, Ingannare +5, Mani di fata +4

\textbf{Lingue} due lingue qualsiasi

\textbf{Sfida} 1 (200 PE)

\emph{\textbf{Attacco Furtivo (1/Turno).}} La spia infligge 7 (2d6) danni aggiuntivi quando colpisce un bersaglio con un attacco con arma e ha +1d6 al tiro di attacco, o quando il bersaglio è entro 1 metro da un alleato dell'assassino che non è inabile e l'assassino non ha -1d6 al tiro di attacco.

\emph{\textbf{Azione Astuta.}} Durante ciascun suo turno, la spia può usare un'azione bonus per effettuare l'azione Ritirarsi, Nascondersi o Scattare.

\textbf{Azioni}

\emph{\textbf{Multiattacco.}} La spia effettua due attacchi da mischia.

\emph{\textbf{Spada Corta.} Attacco con Arma da Mischia}: +4 a colpire, portata 1 m, un bersaglio.

\emph{Colpisce:} 5 (1d6 + 2) danni perforanti.

\emph{\textbf{Balestrino.} Attacco con Arma a Gittata}: +4 a colpire, gittata 9m, un bersaglio. \emph{Colpisce:} 5 (1d6 + 2) danni perforanti.


\medskip\textbf{Capitano dei Banditi/Pirata}

Che viva in terra o in mare, è un individuo munito di una grande personalità che riesce a tenere in riga la marmaglia che risponde ai suoi ordini.

\emph{Media umanoide (qualsiasi razza), qualsiasi allineamento non legale}

\textbf{FORZA} +2

\textbf{DESTREZZA} +3

\textbf{COSTITUZIONE} +2

\textbf{INTELLIGENZA} +2

\textbf{SAGGEZZA} +0

\textbf{CARISMA} +2

\textbf{Iniziativa} +2 -- \textbf{Difesa} 16 (armatura di cuoio borchiato)

\textbf{Punti Ferita} 65 (10d8 + 8)

\textbf{Movimento} 9 m

\textbf{Tiri Salvezza}: Tempra +5, Riflessi +5, Volontà +3 

\textbf{Competenze} Acrobatica +4, Raggiro +4 

\textbf{Lingue} due lingue qualsiasi

\textbf{Sfida} 2 (450 PE)

\textbf{Azioni}

\emph{\textbf{Multiattacco.}} Il capitano effettua tre attacchi da mischia: due con la scimitarra e uno con il pugnale. Oppure il capitano effettua due attacchi a gittata con i pugnali.

\emph{\textbf{Scimitarra.} Attacco con Arma da Mischia}: +5 a colpire, portata 1 m, un bersaglio.

\emph{Colpisce:} 6 (1d6 + 3) danni taglienti.

\emph{\textbf{Pugnale.} Attacco con Arma da Mischia o a Gittata}: +5 a colpire, portata 1 m o gittata 6m, un bersaglio. \emph{Colpisce:} 5 (1d4 + 3) danni perforanti.

\textbf{Reazioni}

\emph{\textbf{Parata.}} Il capitano somma 2 alla sua Difesa contro un attacco da mischia che lo colpirebbe. Per farlo, il capitano deve vedere l'attaccante e impugnare un'arma da mischia.

\medskip\textbf{Assassino}

Solitari o membri di una gilda, gli assassini sono pagati per eliminare, spesso in modo silenzioso e discreto, rivali e nemici dei loro datori di lavoro.

\emph{Media umanoide (qualsiasi razza), qualsiasi allineamento non buono}

\textbf{FORZA} +0

\textbf{DESTREZZA} +3

\textbf{COSTITUZIONE} +2

\textbf{INTELLIGENZA} +1

\textbf{SAGGEZZA} +0

\textbf{CARISMA} +0

\textbf{Iniziativa} +3 -- \textbf{Difesa} 19 (armatura di cuoio borchiato)

\textbf{Punti Ferita} 78 (12d8 + 24)

\textbf{Movimento} 9 m

\textbf{Tiri Salvezza}: Tempra +4, Riflessi +6, Volontà +3 

\textbf{Competenze} Acrobazia +6, Muoversi Silenziosamente / Nascondersi +9, Consapevolezza +3, Raggiro +3


\textbf{Lingue} Gergo dei Ladri più due altre lingue

\textbf{Sfida} 8 (3.900 PE)

\emph{\textbf{Assassinare.}} Durante il suo primo turno, l'assassino ha +1d6 ai tiri di attacco contro le creature che non hanno ancora svolto nessun turno. Qualsiasi colpo che l'assassino mandi a segno contro una creatura sorpresa, è un colpo critico.

\emph{\textbf{Attacco Furtivo (1/Turno).}} L'assassino infligge 14 (4d6) danni aggiuntivi quando colpisce un bersaglio con un attacco con arma e ha +1d6 al tiro di attacco, o quando il bersaglio è entro 1 metro da un alleato dell'assassino che non è inabile e l'assassino non ha -1d6 al tiro di attacco.

\emph{\textbf{Evasione.}} Se l'assassino è vittima di un effetto che permette di effettuare un Tiro Salvezza di Riflessi per dimezzare i danni, l'assassino non prende danni se riesce il Tiro Salvezza, e solo la metà se lo fallisce.

\textbf{Azioni}

\emph{\textbf{Multiattacco.}} L'assassino effettua due attacchi con le spade corte.

\emph{\textbf{Spada Corta.} Attacco con Arma da Mischia}: +6 a colpire, portata 1 m, un bersaglio.

\emph{Colpisce:} 6 (1d6 + 3) danni perforanti, e il bersaglio deve effettuare un Tiro Salvezza di Tempra CD 15, subendo 24 (7d6) danni da veleno se fallisce il Tiro Salvezza, o la metà di questi danni se lo riesce.

\emph{\textbf{Balestra Leggera.} Attacco con Arma a Gittata}: +6 a colpire, gittata 24m, un bersaglio.

\emph{Colpisce:} 7 (1d8 + 3) danni perforanti, e il bersaglio deve effettuare un Tiro Salvezza di Tempra CD 15, subendo 24 (7d6) danni da veleno se fallisce il Tiro Salvezza, o la metà di questi danni se lo riesce.

\medskip\textbf{Mago}

Il mago trascorrono la vita nello studio e la pratica della magia.

\textbf{VARIANTE: FAMIGLI}

Qualsiasi incantatore che possa eseguire l'incantesimo \emph{trovare} \emph{famiglio} è probabile che abbia un famiglio. Il famiglio può essere una delle creature descritte nell'incantesimo (vedi le \emph{Regole Base}) o qualche altro mostro Minuscolo, come un artiglio strisciante, un diavoletto, uno pseudodrago o un demonietto.

\medskip\textbf{Mago Avventuriero}

Un Mago novizio, che ha superato con successo le sue prime avventure e ha iniziato a stabilire una reputazione come nobile o famigerato avventuriero.

\emph{Media umanoide (qualsiasi razza), qualsiasi malvagio}

\textbf{FORZA} -1

\textbf{DESTREZZA} +2

\textbf{COSTITUZIONE} +0

\textbf{INTELLIGENZA} +3

\textbf{SAGGEZZA} +1

\textbf{CARISMA} +0

\textbf{Iniziativa} +3 -- \textbf{Difesa} 13

\textbf{Punti Ferita} 22 (5d8)

\textbf{Movimento} 9 m

\textbf{Tiri Salvezza}: Tempra +0, Riflessi +3, Volontà +2 

\textbf{Competenze} Arcano +5, Storia +5

\textbf{Lingue} quattro lingue qualsiasi

\textbf{Sfida} 1 (200 PE)

\emph{\textbf{Incantesimi.}} Il mago ha CM 4. La sua abilità da incantatore è l'Intelligenza (+5 al colpire con attacchi con incantesimo). Il Mago ha preparato i seguenti incantesimi: Trucchetti (a volontà): 

\emph{luce, mano magica, stretta folgorante}

Difficoltà 16 (4 slot): \emph{charme su persone, dardo incantato}

Difficoltà 19 (3 slot): \emph{bloccare persona, passo velato}

\textbf{Azioni}

\emph{\textbf{Bastone.} Attacco con Arma da Mischia}: +1 a colpire, portata 1 m, un bersaglio.

\emph{Colpisce:} 3 (1d8 - 1) danni da botta.

\medskip\textbf{Grande Mago}

Un Mago che ha stabilito una discreta fama nel territorio e che attira intorno a sé studenti da ogni dove.

\emph{Media umanoide (qualsiasi razza), qualsiasi allineamento}

\textbf{FORZA} -1

\textbf{DESTREZZA} +2

\textbf{COSTITUZIONE} +0

\textbf{INTELLIGENZA} +3

\textbf{SAGGEZZA} +1

\textbf{CARISMA} +0

\textbf{Iniziativa} +3 -- \textbf{Difesa} 15 (18 con \emph{armatura del Mago})

\textbf{Punti Ferita} 40 (9d8)

\textbf{Movimento} 9 m

\textbf{Tiri Salvezza}: Tempra +1, Riflessi +4, Volontà +3 

\textbf{Competenze} Arcano +6, Storia +6

\textbf{Lingue} quattro lingue qualsiasi

\textbf{Sfida} 6 (2.300 PE)

\emph{\textbf{Incantesimi.}} Il mago ha CM 9. La sua abilità da incantatore è l'Intelligenza (+6 al colpire con attacchi con incantesimo). Il Mago ha preparato i seguenti incantesimi:

Trucchetti (a volontà): \emph{dardo infuocato, luce, mano magica,}
\emph{prestidigitazione}

Difficoltà 16 (4 slot): \emph{armatura del Mago, dardo incantato,}
\emph{individuare magia, scudo}

Difficoltà 19 (3 slot): \emph{passo velato, suggestione}

Difficoltà 21 (3 slot): \emph{controincantesimo, palla di fuoco, volare}

Difficoltà 23 (3 slot): \emph{invisibilità superiore, tempesta di ghiaccio}

Difficoltà 26 (1 slot): \emph{cono di freddo}

\textbf{Azioni}

\emph{\textbf{Pugnale.} Attacco con Arma da Mischia o a Gittata}: +5 a colpire, portata 1 m o gittata 6m, un bersaglio. \emph{Colpisce:} 4 (1d4 + 2) danni perforanti.

\medskip\textbf{ArciMago}

Un mago molto potente (e anche molto anziano) che studia i segreti del multiverso.

\emph{Media umanoide (qualsiasi razza), qualsiasi allineamento}

\textbf{FORZA} +0

\textbf{DESTREZZA} +2

\textbf{COSTITUZIONE} +1

\textbf{INTELLIGENZA} +5

\textbf{SAGGEZZA} +2

\textbf{CARISMA} +3

\textbf{Iniziativa} +5 -- \textbf{Difesa} 18 (21 con \emph{armatura del Mago})

\textbf{Punti Ferita} 99 (18d8 + 18)

\textbf{Movimento} 9 m

\textbf{Tiri Salvezza}: Tempra +8, Riflessi +10, Volontà +12 

\textbf{Competenze} Arcano +13, Storia +13

\textbf{Resistenze al Danno} danno degli incantesimi; da botta, perforante e tagliente non magico (da \emph{pelle di pietra})

\textbf{Lingue} sei lingue qualsiasi

\textbf{Sfida} 12 (8.400 PE)

\emph{\textbf{Incantesimi.}} Il mago ha CM 18. La sua abilità da incantatore è l'Intelligenza (+9 al colpire con attacchi con incantesimo).

L'arciMago può eseguire \emph{camuffare sé stesso} e \emph{invisibilità} a volontà e ha preparato i seguenti incantesimi: Trucchetti (a volontà): \emph{dardo infuocato, luce, mano magica,}
\emph{prestidigitazione, stretta folgorante}

Difficoltà 16 (4 slot): \emph{armatura magica*, dardo incantato,}
\emph{identificare, individuare magia}

Difficoltà 19 (3 slot): \emph{immagine speculare, individuazione dei}
\emph{pensieri, passo velato}

Difficoltà 21 (3 slot): \emph{controincantesimo, fulmine}

Difficoltà 23 (3 slot): \emph{esilio, pelle di pietra*, scudo di fuoco}

Difficoltà 26 (3 slot): \emph{cono di freddo, muro di forza, scrutare}

Difficoltà 29 (1 slot): \emph{globo di invulnerabilità}

Difficoltà 31 (1 slot): \emph{teletrasporto}

Difficoltà 34 (1 slot): \emph{vuoto mentale*}

Difficoltà 36 (1 slot): \emph{fermare il tempo}

L'arciMago esegue questi incantesimi su di sé prima del combattimento.

\textbf{Azioni}

\emph{\textbf{Pugnale.} Attacco con Arma da Mischia o a Gittata}: +6 a colpire, portata 1 m o gittata 6m, un bersaglio. \emph{Colpisce:} 4 (1d4 + 2) danni perforanti.


\medskip\textbf{Sacerdoti}

I sacerdoti sono devoti di una divinità o una fede che si prendono cura di impartire gli insegnamenti divini al loro gregge.

\medskip\textbf{Cultista}

I cultisti giurano fedeltà ai poteri oscuri, e nelle loro credenze e pratiche mostrano spesso segni di follia.

\emph{Media umanoide (qualsiasi razza), qualsiasi allineamento non buono}

\textbf{FORZA} +0

\textbf{DESTREZZA} +1

\textbf{COSTITUZIONE} +0

\textbf{INTELLIGENZA} +0

\textbf{SAGGEZZA} +0

\textbf{CARISMA} +0

\textbf{Iniziativa} +0- \textbf{Difesa} 13 (armatura di cuoio)

\textbf{Punti Ferita} 9 (2d8)

\textbf{Movimento} 9 m

\textbf{Tiri Salvezza}: Tempra +1, Riflessi +1, Volontà +2 

\textbf{Competenze} Raggiro +2, Religione +2

\textbf{Lingue} una qualsiasi lingua (di solito il Comune)

\textbf{Sfida} 1/8 (25 PE)

\emph{\textbf{Oscura Devozione.}} Il cultista ha +1d6 sui Tiri Salvezza contro l'essere affascinato o spaventato.

\textbf{Azioni}

\emph{\textbf{Scimitarra.} Attacco con Arma da Mischia}: +3 a colpire, portata 1 m, una creatura.

\emph{Colpisce:} 4 (1d6 + 1) danni taglienti.

\medskip\textbf{Accolito}

Gli accoliti sono membri di grado minore del clero, e di solito rispondono ad un sacerdote di rango superiore. Svolgono diverse funzioni in un tempio e gli viene conferita dalla loro divinità l'abilità di eseguire incantesimi minori.

\emph{Media umanoide (qualsiasi razza), qualsiasi allineamento}

\textbf{FORZA} +0

\textbf{DESTREZZA} +0

\textbf{COSTITUZIONE} +0

\textbf{INTELLIGENZA} +0

\textbf{SAGGEZZA} +2

\textbf{CARISMA} +0

\textbf{Iniziativa} +0 -- \textbf{Difesa} 11

\textbf{Punti Ferita} 9 (2d8)

\textbf{Movimento} 9 m

\textbf{Tiri Salvezza}: Tempra +0, Riflessi +0, Volontà +3 

\textbf{Competenze} Pronto Soccorso +4, Religione +2

\textbf{Lingue} una qualsiasi lingua (di solito il Comune)

\textbf{Sfida} 1/4 (50 PE)

\emph{\textbf{Incantesimi.}} L'accolito ha CM 1. La sua abilità da incantatore è la Saggezza (+4 al colpire con attacchi con incantesimo). L'accolito ha preparato i seguenti incantesimi: Trucchetti (a volontà): \emph{fiamma sacra, luce, taumaturgia} Difficoltà 16 (3 slot): \emph{benedizione}, \emph{cura ferite, santuario}

\medskip\textbf{Azioni}

\emph{\textbf{Randello.} Attacco con Arma da Mischia}: +2 a colpire, portata 1 m, un bersaglio.

\emph{Colpisce:} 2 (1d4) danni da botta.

\textbf{Fanatico del Culto}

Sono i capi di un culto, che usano il proprio carisma e i propri dogmi per influenzare i deboli di volontà.

\emph{Media umanoide (qualsiasi razza), qualsiasi allineamento non buono}

\textbf{FORZA} +0

\textbf{DESTREZZA} +2

\textbf{COSTITUZIONE} +1

\textbf{INTELLIGENZA} +0

\textbf{SAGGEZZA} +1

\textbf{CARISMA} +2

\textbf{Iniziativa} +2 -- \textbf{Difesa} 14 (armatura di cuoio)

\textbf{Punti Ferita} 33 (6d8 + 6)

\textbf{Movimento} 9 m

\textbf{Tiri Salvezza}: Tempra +2, Riflessi +2, Volontà +3 

\textbf{Competenze} Ingannare +4, Raggiro +4, Religione +2 

\textbf{Lingue} una qualsiasi lingua (di solito il Comune)

\textbf{Sfida} 2 (450 PE)

\emph{\textbf{Incantesimi.}} Il sacerdote ha CM 4. La sua abilità da incantatore è la Saggezza (+3 al colpire con attacchi con incantesimo). Il sacerdote ha preparato i seguenti incantesimi: Trucchetti (a volontà): \emph{fiamma sacra, luce, taumaturgia}

Difficoltà 16 (4 slot): \emph{comando, infliggi ferite, scudo della fede}

Difficoltà 19 (3 slot): \emph{arma spirituale, blocca persona}

\emph{\textbf{Oscura Devozione.}} Il cultista ha +1d6 sui Tiri Salvezza contro l'essere affascinato o spaventato.

\textbf{Azioni}

\emph{\textbf{Multiattacco.}} Il fanatico effettua due attacchi da mischia.

\emph{\textbf{Pugnale.} Attacco con Arma da Mischia o a Gittata}: +4 a colpire, portata 1 m o gittata 6m, una creatura. \emph{Colpisce:} 4 (1d4 + 2) danni perforanti.

\medskip\textbf{Gran Sacerdote}

Sono individui al comando di un tempio o altro luogo sacro e che hanno a loro disposizione diversi accoliti.

\emph{Media umanoide (qualsiasi razza), qualsiasi allineamento}

\textbf{FORZA} +0

\textbf{DESTREZZA} +0

\textbf{COSTITUZIONE} +1

\textbf{INTELLIGENZA} +1

\textbf{SAGGEZZA} +3

\textbf{CARISMA} +1

\textbf{Iniziativa} +1 -- \textbf{Difesa} 14 (giaco di maglia)

\textbf{Punti Ferita} 27 (5d8 + 5)

\textbf{Movimento} 7 m

\textbf{Tiri Salvezza}: Tempra +1, Riflessi +1, Volontà +4 

\textbf{Competenze} Pronto Soccorso +7, Ingannare +3, Religione +4

\textbf{Lingue} due lingue qualsiasi

\textbf{Sfida} 2 (450 PE)

\emph{\textbf{Eminenza Divina.}} Come azione bonus, il sacerdote può spendere uno slot incantesimo per far sì che il suo attacco con arma da mischia infligge 10 (3d6) danni da Luce aggiuntivi. Il beneficio dura fino al termine del turno.

\emph{\textbf{Incantesimi.}} Il sacerdote ha CM 5. La sua abilità da incantatore è la Saggezza (+5 al colpire con attacchi con incantesimo). Il sacerdote ha preparato i seguenti incantesimi: Trucchetti (a volontà): \emph{fiamma sacra, luce, taumaturgia}

Difficoltà 16 (4 slot): \emph{cura ferite, dardo tracciante, santuario}

Difficoltà 19 (3 slot): \emph{arma spirituale, ristorare inferiore}

Difficoltà 21 (2 slot): \emph{dissolvi magie}, \emph{guardiani spirituali}

\textbf{Azioni}

\emph{\textbf{Mazza.} Attacco con Arma da Mischia}: +2 a colpire, portata 1 m, un bersaglio.

\emph{Colpisce:} 3 (1d6) danni da botta.


\medskip\textbf{Selvaggi}

Questi individui vivono ai margini della civiltà, a volte entrandovi raramente in contatto. A disagio tra le mura e nelle terre civilizzate, si trovano nel loro ambiente quando possono muoversi tra le terre selvagge.

\medskip\textbf{Berserker}

Provenienti da terre selvagge, gli imprevedibili berserker si radunano in compagnie di guerra e sono sempre alla ricerca di conflitti in cui combattere.

\emph{Media umanoide (qualsiasi razza), qualsiasi allineamento caotico}

\textbf{FORZA} +3

\textbf{DESTREZZA} +1

\textbf{COSTITUZIONE} +3

\textbf{INTELLIGENZA} -1

\textbf{SAGGEZZA} +0

\textbf{CARISMA} -1

\textbf{Iniziativa} +1 -- \textbf{Difesa} 14 (armatura di pelle)

\textbf{Punti Ferita} 67 (9d8 + 27)

\textbf{Movimento} 9 m

\textbf{Tiri Salvezza}: Tempra +4, Riflessi +3, Volontà +2 

\textbf{Lingue} una qualsiasi lingua (di solito il Comune)

\textbf{Sfida} 2 (450 PE)

\emph{\textbf{Incauto.}} All'inizio del suo turno, il berserker può ottenere +1d6 su tutti i tiri di attacco con armi da mischia effettuati durante quel turno, ma i tiri di attacco contro di esso hanno
+1d6 fino all'inizio del suo prossimo turno.

\textbf{Azioni}

\emph{\textbf{Ascia Grossa.} Attacco con Arma da Mischia}: +5 a colpire, portata 1 m, un bersaglio.

\emph{Colpisce:} 9 (1d12 + 3) danni taglienti.

\textbf{Combattente Tribale}

Sono i difensori delle tribù che vivono ai margini della civiltà.

\emph{Media umanoide (qualsiasi razza), qualsiasi allineamento}

\textbf{FORZA} +1

\textbf{DESTREZZA} +0

\textbf{COSTITUZIONE} +1

\textbf{INTELLIGENZA} -1

\textbf{SAGGEZZA} +0

\textbf{CARISMA} -1

\textbf{Iniziativa} +0 -- \textbf{Difesa} 13 (armatura di pelle)

\textbf{Punti Ferita} 11 (2d8 + 2)

\textbf{Movimento} 9 m

\textbf{Tiri Salvezza}: Tempra +2, Riflessi +1, Volontà +1 

\textbf{Lingue} una qualsiasi lingua

\textbf{Sfida} 1/8 (25 PE)

\emph{\textbf{Tattiche di Branco.}} Il combattente tribale ha +1d6 ai tiri di attacco contro una creatura se almeno uno degli alleati del picchiatore si trova entro 1 metro dalla creatura e quell'alleato non è inabile.

\textbf{Azioni}

\emph{\textbf{Lancia.} Attacco con Arma da Mischia o a Gittata}: +3 a colpire, portata 1 m o gittata 6m, un bersaglio.

\emph{Colpisce:} 4 (1d6 + 1) danni perforanti, o 5 (1d8 + 1) danni perforanti se usata con due mani per effettuare un attacco da mischia.

\medskip\textbf{Druido}

I druidi proteggono il mondo naturale dai mostri e dall'avanzare della civiltà. Alcuni sono sciamani tribali che curano i malati, pregano agli spiriti animali e forniscono consigli spirituali.

\emph{Media umanoide (qualsiasi razza), qualsiasi allineamento}

\textbf{FORZA} +0

\textbf{DESTREZZA} +1

\textbf{COSTITUZIONE} +1

\textbf{INTELLIGENZA} +1

\textbf{SAGGEZZA} +2

\textbf{CARISMA} +0

\textbf{Iniziativa} +1 -- \textbf{Difesa} 12 (17 con \emph{pelle di corteccia}*)

\textbf{Punti Ferita} 27 (5d8 + 5)

\textbf{Movimento} 9 m

\textbf{Tiri Salvezza}: Tempra +1, Riflessi +2, Volontà +3 \\

\textbf{Competenze} Pronto Soccorso +4, Natura +3, Consapevolezza +4 

\textbf{Lingue} Druidico più due altre lingue

\textbf{Sfida} 2 (450 PE)

\emph{\textbf{Incantesimi.}} Il sacerdote ha CM 4. La sua abilità da incantatore è la Saggezza (+4 al colpire con attacchi con incantesimo). Il sacerdote ha preparato i seguenti incantesimi: Trucchetti (a volontà): \emph{arte druidica, bastone, produrre fiamma}

Difficoltà 16 (4 slot): \emph{intralciare, onda tonante, parlare con gli}
\emph{animali, passo veloce}

Difficoltà 19 (3 slot): \emph{animale messaggero, pelle di corteccia}

\textbf{Azioni}

\emph{\textbf{Bastone da Combattimento.} Attacco con Arma da Mischia}: +2 a colpire (+4 a colpire con \emph{bastone*}), portata 1 m o gittata 6m, un bersaglio.

\emph{Colpisce:} 3 (1d6) danni da botta, o 6 (1d8 + 2) danni da botta con \emph{bastone} o se impugnato con due mani.

\medskip\textbf{Esploratore}

Abili cacciatori e battitori di piste.

\emph{Media umanoide (qualsiasi razza), qualsiasi allineamento}

\textbf{FORZA} +0

\textbf{DESTREZZA} +2

\textbf{COSTITUZIONE} +1

\textbf{INTELLIGENZA} +0

\textbf{SAGGEZZA} +1

\textbf{CARISMA} +0

\textbf{Iniziativa} +2 -- \textbf{Difesa} 14 (armatura di cuoio)

\textbf{Punti Ferita} 16 (3d8 + 3)

\textbf{Movimento} 9 m

\textbf{Tiri Salvezza}: Tempra +1, Riflessi +2, Volontà +3

\textbf{Competenze} Muoversi Silenziosamente / Nascondersi +6, Natura +4, Consapevolezza +5, Sopravvivenza +5

\textbf{Lingue} una qualsiasi lingua (di solito Comune)

\textbf{Sfida} 1/2 (100 PE)

\emph{\textbf{Olfatto e Vista Affinati.}} L'esploratore ha +1d6 nelle prove di Saggezza (Consapevolezza) basate su olfatto o vista.

\textbf{Azioni}

\emph{\textbf{Multiattacco.}} L'esploratore effettua due attacchi da mischia o due attacchi a gittata.

\emph{\textbf{Spada Corta.} Attacco con Arma da Mischia}: +4 a colpire, portata 1 m, un bersaglio.

\emph{Colpisce:} 5 (1d6 + 2) danni perforanti.

\emph{\textbf{Arco Lungo.} Attacco con Arma da Mischia}: +4 a colpire, gittata 45m, un bersaglio.

\emph{Colpisce:} 6 (1d8 + 2) danni perforanti.


\end{multicols}

%{\scriptsize 
%	\printindex}
%\end{document}

\pagebreak


\section{Conversione Mostri}\index{Conversione Mostri}

\bigskip

Per aggiungere altri mostri a DBS vi invito a convertirli da Pathfinder o dalla 5ed del famoso gioco di ruolo. DBS è di base un sistema D20 fortemente modificato nelle dinamiche ma non nelle fondamenta dei valori numerici.

\medskip

\textbf{Conversione da Pathfinder}

Prendiamo ad esempio l'Orco comune da https://www.d20pfsrd.com/bestiary/monster-listings/humanoids/orcs/orc/ tralasciamo le parti descrittive e concentriamoci sui numeri e valori.

\bigskip

\textbf{Orc (grado di Sfida 1/3)} questo valore rimane il medesimo in DBS

\textbf{XP 135} questo valore non ha più senso

\textbf{Orc warrior 1} non ci interessa la classe, solo al più la razza.

\textbf{CE Medium humanoid} ci indica che la creatura è di taglia media, umanoide e malvagio, ai fini dei Tratti la creatura non è di livello tale da aver attirato l'attenzione di un Patrono.

\textbf{Init +0} è l'iniziativa, prendete il bonus alla Destrezza o Intelligenza

\textbf{Senses} darkvision 60 ft.; Perception -1, rimane uguale, si tengono gli stessi valori ed abilità. In questo caso 60 piedi indica che la distanza è di 20 metri

\textbf{Weakness} light sensitivity si cerca dove possibile l'equivalente in DBS, in questo caso fotofobia leggera oppure si applicano direttamente le penalità indicate.

\textbf{AC} 13, touch 10, flat-footed 13 (+3 armor) questa è la Difesa. 

\textbf{Competenza Armi}: è pari al BAB indicato

\textbf{Competenza Magia}: di base è metà del grado di Sfida. Utile solo se la creatura ha poteri magici.

\textbf{hp} 6 (1d10+1) rimane uguale

\textbf{Fort} +3, Ref +0, Will -1 si traducono in Tempra, Destrezza e Volontà. Il punteggio rimane uguale

\textbf{Speed} 30 ft. E' il movimento, in questo caso è 9 metri

\textbf{Melee} falchion +5 (2d4+4/18--20) è il mio Tiro per Colpire e danno. Rimane uguale

\textbf{Ranged} javelin +1 (1d6+3) è il Tiro per Colpire. Rimane uguale

\textbf{Str} 17, Dex 11, Con 12, Int 7, Wis 8, Cha 6. Devi prendere solo la parte bonus.

\textbf{Base Atk} +1; CMB +4; CMD 14 il primo valore determina la CA. Suggerisco di usare direttamente i bonus al colpire indicati nel melee.

\textbf{Feats} Weapon Focus (falchion) Arma Focalizzata. Il bonus dell'abilità è già calcolata nei valori di Melee

\textbf{Skills} Intimidate +2 rimane uguale. 

\textbf{Ferocity} (Ex): An orc remains conscious and can continue fighting even if its hit point total is below 0. It is still staggered and loses 1 hit point each round. A creature with ferocity still dies when its hit point total reaches a negative amount equal to its Constitution score. Si prende la abilità così come è.


\bigskip

\textbf{Conversione dalla 5e} del famoso Gioco di Ruolo:

- Aumentate la classe armatura di 1/2 del Grado di Sfida della creatura, arrotondate per eccesso.

- Per i Tiri Salvezza prendete il triplo del Grado di Sfida e divideteli tra Tempra, Riflessi e Volontà in base alle caratteristiche del mostro (oppure prendete l'equivalente da Pathfinder).


\pagebreak

\section{Condizioni}\index{Condizioni}

\begin{enfasiwide}{
La grandezza dell'uomo è nella decisione di essere più forte della sua condizione. (Albert Camus)}
\end{enfasiwide}\medskip

\begin{multicols}{2}
	
\label{condizioni}
	
\textbf{Abbagliato}:\index{Abbagliato} La creatura è incapace di vedere bene a causa di un'eccessiva stimolazione degli occhi. Una creatura abbagliata subisce penalità -1d6 al Tiro per Colpire e alle prove di Consapevolezza basate sulla vista.
	
\textbf{Accecato}:\index{Accecato} Il personaggio non riesce a vedere nulla. subisce penalità -2 alla maggior parte delle Competenze basate su Forza e Destrezza. 
	
Tutte le prove o le attività basate sulla visione (come ad esempio leggere, o eventuali prove di Consapevolezza basate sulla vista) falliscono automaticamente. Tutti gli avversari vengono considerati dotati di invisibilità nei confronti del personaggio accecato.
	
I personaggi accecati devono effettuare una prova di Acrobatica con DC 12 per muoversi più veloci della propria velocità dimezzata. Le creature che falliscono questa prova cadono a terra Prone. I personaggi che rimangono per lungo tempo accecati possono abituarsi ad alcune di queste penalità e iniziare a superarne alcune, a discrezione del Narratore.

Chi attacca una creatura per lei invisibile ha un -2d6 al Tiro per Colpire, la creatura invisibile che attacca una creatura che non la vede ha +1d6 al Tiro per Colpire
	
\textbf{Accovacciato}:\index{Accovacciato} Un personaggio accovacciato subisce penalità -2 alla Difesa e -1 al Tiro per Colpire, ha un -2 alle prove di Destrezza.
	
\textbf{Affascinato}:\index{Affascinato} Una creatura affascinata è soggiogata da un effetto soprannaturale o di un incantesimo di Charme. La creatura rimane in piedi o si siede tranquilla, senza effettuare alcuna azione se non prestare attenzione alla fonte del fascino, fintanto che dura l'effetto. L'effetto provoca penalità -4 alle Prove di Competenza richieste come reazione, come ad esempio le prove di Consapevolezza.
	
Qualsiasi potenziale minaccia, come ad esempio una creatura ostile in avvicinamento, consente alla creatura affascinata un nuovo Tiro Salvezza contro l'effetto del fascino. Qualsiasi minaccia palese, come ad esempio qualcuno che estrae un'arma, lancia un'incantesimo o punta un'arma a distanza verso la creatura affascinata, interrompe automaticamente l'effetto.
	
Un alleato della creatura affascinata può scuoterla per liberarla dall'effetto spendendo 2 Azioni.
	
L'affascinatore ha +1d6 su qualsiasi prova di caratteristica per interagire socialmente con la creatura.
	
\textbf{Affaticato}\index{Affaticato}: Un personaggio affaticato non può correre o Caricare e subisce una penalità -1 a Costituzione e Destrezza. Se compie qualsiasi cosa normalmente affaticante diventa Esausto.
	
Ci vogliono 8 ore di riposo totale per rimuovere la condizione di affaticato. Se un personaggio non dorme almeno 6 ore alla mattina è affaticato.
	
\textbf{Afferrato}\index{Afferrato}: Un personaggio afferrato non può muoversi, deve usare due Azioni per liberarsi (TS Tempra contrapposto).
	
Può attaccare con armi in mischia se adeguate (difficilmente potrà usare uno spadone, alabarda.. un pugnale o spada corta è più probabile).
	
\textbf{Amichevole}:\index{Amichevole} Una creatura amichevole non attaccherà il personaggio se non minacciata esplicitamente.

\textbf{Annegare/Trattenere il fiato}: \index{Annegare}\index{Trattenere il fiato} Qualsiasi personaggio può trattenere il fiato per un numero di round pari 6 round per il suo punteggio di Costituzione, con un minimo di 3 round. Per ogni Azione compiuta la durata restante diminuisce di 1 round. Trascorso questo periodo di tempo, il personaggio deve effettuare un Tiro Salvezza su Tempra con DC 12 ogni round per continuare a trattenere il fiato. Ogni round, la DC aumenta di 1.

\textbf{Assordato}:\index{Assordato}\index{Sordo} Un personaggio assordato non può ascoltare. Subisce penalità -2 alle prove di Iniziativa, fallisce automaticamente tutte le prove di Consapevolezza basate sul suono e si considera Distratto nel lancio degli incantesimi con componenti almeno verbali.
	
I personaggi che rimangono assordati per lunghi periodi di tempo, possono abituarsi a queste penalità e superarne alcune, a discrezione del Narratore.
	
\textbf{Avvelenato}\index{Avvelenato}: si considera avvelenato qualsiasi soggetto sotto l'influenza di un veleno o pozione, indipendentemente che questa stia già producendo gli effetti o li debba ancora produrre dato il tempo dell'insorgenza.
	
\textbf{Charmato}:\index{Charmato} una creatura charmata tratta il giocatore con un fidato amico ed alleato. Se la creatura viene minacciata o attaccata può fare un nuovo Tiro Salvezza su Volontà con un +2.
	
L'effetto di charme non permetto il controllo del target ma questo percepisce le tue parole nel modo più favorevole. Puoi anche dare ordini ma devi riuscire in una prova di Intimidire contro un Tiro Salvezza su Volontà.
	
Un target influenzato da charme non farà nulla di pericoloso per se stesso (tranne se convinto) o per altri soggetti che reputa amici.
	
\textbf{Confuso}: \index{Confuso}Una creatura confusa è mentalmente ottenebrata e non può agire normalmente. Una creatura confusa non riesce a distinguere un alleato da un nemico e considera tutti come nemici.
	
Gli alleati che vogliono utilizzare un'incantesimo a vantaggio della creatura confusa devono comunque toccarla con un attacco di contatto in mischia riuscito.
	
Se una creatura confusa è attaccata, attacca sempre l'ultima creatura che la ha attaccata, finché quella creatura non muore o esce dalla sua visuale.
	
Tirate un dado sulla tabella seguente all'inizio di ogni round della creatura confusa ad ogni round per vedere quello che la creatura fa in quel round.
	
\textbf{d100 Comportamento:}
	
01-25 Agisce normalmente
	
26-50 Non fa altro che balbettare in modo incoerente
	
51-75 Si infligge 1d8 + modificatore di Forza con l'arma che tiene in mano
	
76-100 Attacca la creatura più vicina (a tale scopo, un Famiglio conta come parte del soggetto stesso)
	
Una creatura confusa che non è in grado di eseguire l'azione indicata non farà altro che balbettare in modo incoerente. Gli aggressori non hanno alcun vantaggio speciale quando attaccano una creatura confusa. Qualsiasi creatura confusa che venga attaccata, attacca automaticamente a sua volta il suo aggressore al suo round successivo, fintanto che rimane confusa quando giunge il suo round.
	
\textbf{Dominato}:\index{Dominato} si è in grado di controllare le azioni di una qualsiasi creatura Umanoide mediante un legame telepatico con la mente del soggetto.
	
Se si ha un linguaggio in comune, si può generalmente costringere il soggetto ad eseguire i comandi entro i limiti delle sue capacità. Se non si condivide nessun linguaggio, si possono impartire solo comandi di base come "vieni qui", "vai li"', "combatti" o "stai fermo". Si è a conoscenza di ciò che il soggetto sta provando ma non si ricevono percezioni sensoriali dirette da lui, né si può comunicare con lui telepaticamente.
	
Una volta impartito un ordine alla creatura dominata, questa continua a tentare di eseguirlo con l'esclusione di tutte le altre attività ad eccezione di quelle necessarie per la sopravvivenza quotidiana (come mangiare, dormire e così via). Grazie a questo limitato spettro di attività, una prova di Consapevolezza con DC 15 (invece che DC 25) può determinare se il comportamento del soggetto è stato influenzato da un effetto di ammaliamento.
	
Concentrandosi completamente sull'incantesimo (2 Azioni), si possono ricevere percezioni sensoriali come vengono interpretate dalla mente del soggetto, anche se questo non può comunque comunicarle. Non si può in realtà vedere attraverso gli occhi del soggetto, quindi non è come se si fosse presenti, ma ci si può rendere conto di cosa sta succedendo.
	
Ovviamente ordini palesemente autodistruttivi non vengono eseguiti. Una volta stabilito il controllo, il raggio di azione entro il quale può essere mantenuto è illimitato purché entrambi i soggetti rimangano sullo stesso piano. Non c'è bisogno di vedere il soggetto per controllarlo. Se ogni giorno non si trascorre almeno 1 minuto a concentrarsi sull'incantesimo, il soggetto riceve un nuovo Tiro Salvezza per liberarsi dal controllo.
	
\textbf{Esausto}:\index{Esausto} Un personaggio esausto si muove a velocità dimezzata e subisce penalità -2 a Costituzione e Destrezza. Dopo 1 ora di completo riposo (o Ristorare Inferiore), un personaggio esausto diventa solo Affaticato. Un personaggio Affaticato diventa Esausto compiendo una azione che normalmente lo affaticherebbe.
	
\textbf{Inabile}\index{Inabile}: Una creatura inabile non può effettuare azioni o reazioni.
	
\textbf{Infermo}\index{Infermo}: Un personaggio infermo subisce penalità -2 a tutti i Tiri per Colpire e per i danni delle armi, ai Tiri Salvezza, alle prove di competenza e di caratteristica
	
\textbf{Impreparato}:\index{Impreparato} Un personaggio che non ha ancora agito in combattimento è impreparato, non potendo ancora reagire alla situazione. Un personaggio impreparato perde il suo bonus di Destrezza alla Difesa (se presente)
	
\textbf{In Lotta}:\index{Lotta} Una creatura in lotta è trattenuta da una creatura, da una trappola o da un effetto. Le creature in lotta non possono muoversi e subiscono penalità -2 a Destrezza. Una creatura in lotta subisce penalità -2 a Tiro per Colpire e Difesa. Inoltre, le creature in lotta non possono compiere azioni che richiedano due mani per essere effettuate.
	
\textbf{Incorporeo}:\index{Incorporeo} Le creature di questo tipo non possiedono un corpo fisico. Le creature incorporee possono essere colpite solo da armi magiche con un bonus di +2 o superiore. Dimezzano gli effetti di incantesimi che non specifichino di funzionare su creature incorporee. Le creature incorporee subiscono danno pieno da altri soggetti ed effetti incorporei, così come tutti gli effetti di forza.
	
Una creatura incorporea può entrare in un oggetto corporeo o passarvi attraverso,
	
Gli attacchi di una creatura incorporea passano attraverso (ignorano) armature non magiche e scudi, solo la naturale Destrezza e appunto armature/scudi magici offrono resistenza.
	
Le creature incorporee possono muoversi ed agire normalmente nell'acqua come nell'aria. Le creature incorporee non possono cadere e subire danni da caduta.
	
Le creature incorporee non possono effettuare attacchi per Sbilanciare o Lottare, né possono essere sbilanciate o afferrate.
	
Le creature incorporee non hanno peso, e non fanno scattare trappole attivate dal peso.
	
Una creatura incorporea si muove sempre silenziosamente e non può essere sentita con Consapevolezza a meno che non lo desideri. Non ha punteggio di Forza, e si applica il suo bonus di Destrezza agli attacchi in mischia ed a distanza
	
\textbf{Indifeso}:\index{Indifeso}\hypertarget{morente}{}  Un personaggio addormentato, Privo di Sensi, Morente o per qualche altro motivo completamente alla mercé dei suoi avversari, è considerato indifeso.

Una creatura svenuta è Inabile (vedi la condizione), non può muoversi o parlare, ed è inconsapevole di cosa accada nei dintorni.
La creatura lascia cadere qualsiasi cosa impugni e cade prona. La creatura fallisce automaticamente i tiri salvezza su Tempra e Riflessi.
Un personaggio indifeso viene considerato come se avesse Destrezza 0 e non si considerano bonus derivanti dallo scudo. Gli attacchi in mischia contro un personaggio indifeso ottengono bonus +1d6 (equivalente ad attaccare un personaggio Prono) oltre al fatto che non ha bonus da Destrezza alla Difesa. 
	
Gli attacchi a distanza, non ottengono alcun bonus particolare contro i bersagli indifesi, oltre al fatto che non c'è bonus alla Difesa dato dalla Destrezza.
	
Non può eseguire azioni.
	
\textbf{Colpo di Grazia}:\index{Colpo di Grazia} Come unica azione nel round, una creatura può utilizzare un'arma da mischia per infliggere un colpo di grazia ad un personaggio indifeso. Può anche usare un arco o una balestra, l'importante è che sia adiacente al bersaglio.
	
L'attaccante colpisce automaticamente ed infligge un colpo critico. Se il difensore sopravvive, deve superare un Tiro Salvezza su Tempra (DC 10 + danni inflitti) o muore.
	
Le creature immuni ai colpi critici, non subiscono danni critici, né devono superare un Tiro Salvezza su Tempra per evitare di essere uccisi da un colpo di grazia.
	
\textbf{Intralciato}:\index{Intralciato} Un personaggio intralciato ha difficoltà di movimento, ma può comunque provare a muoversi, a meno che i legami che lo intralciano non siano ancorati a un oggetto immobile o impugnati da una forza contrapposta.
	
Una creatura intralciata può muoversi a velocità dimezzata ma non può Correre o Caricare, e subisce penalità -2 ai Tiri per Colpire e penalità -2 alle prove di Destrezza.
	
Un personaggio intralciato che cerca di lanciare un incantesimo deve superare la Difficoltà dell'incantesimo di almeno 2.
	
\textbf{Immobilizzato}:\index{Immobilizzato} Una creatura immobilizzata è strettamente limitata nei movimenti e può compiere solo alcune azioni.
	
Una creatura immobilizzata non può muoversi ed è Impreparata. Una creatura immobilizzata può tentare sempre di liberarsi, solitamente attraverso una prova di Artista della Fuga o un Tiro Salvezza su riflessi.
	
Può compiere azioni verbali e mentali, ma non può utilizzare, di norma, gli incantesimi. Un personaggio immobilizzato che tenta di utilizzare gli incantesimi e usare una Capacità Magica si considera che sia Disturbato.
	
Se il soggetto è legato la prova è contro la prova di Uso Corde o Sopravvivenza di chi ha legato.
	
\textbf{Invisibile}:\index{Invisibile} Le creature invisibili non sono percepibili dalla vista.
Chi attacca una creatura per lei invisibile ha un -2d6 al Tiro per Colpire, la creatura invisibile che attacca una creatura che non la vede ha +1d6 al Tiro per Colpire.
	
\textbf{Morente} \index{Morente}: Un personaggio morente ha -1 Punti Ferita e si considera Indifeso per le penalità ed è prossimo alla morte. Ogni round perde 1 punto ferita finché muore o viene curato.
	
\textbf{Morto}:\index{Morto}\hypertarget{morto}{} L'anima del personaggio abbandona permanentemente il suo corpo. I personaggi morti non possono beneficiare delle cure normali o magiche, e non possono essere riportati in vita da un incantesimo. Solo un Patrono ha sufficiente potere per riportare l'anima nel corpo e riportare in vita la creatura. La Scuola di Necromanzia ha incantesimi per rianimare un corpo come non morto.
	
\textbf{Nauseato}:\index{Nauseato} Le creature nauseate soffrono di disturbi di stomaco.
Le creature nauseate non sono in grado di attaccare, utilizzare incantesimi, concentrarsi sugli incantesimi o fare qualsiasi altra cosa che richieda attenzione. La sola azione che un tale personaggio può compiere è una singola Azione di movimento per round.
	
\textbf{Paralizzato}: \index{Paralizzato}Un personaggio paralizzato è bloccato sul posto ed è incapace di muoversi od agire. Ha punteggi effettivi di Forza e Destrezza pari a -4, è Indifeso e può compiere azioni esclusivamente mentali.
	
Una creatura alata in volo, nel momento in cui viene paralizzata non può più battere le ali e precipita.
Un nuotatore paralizzato non può più Nuotare e potrebbe annegare.
	
Una creatura può attraversare una zona occupata da una creatura paralizzata (o morta), che sia un alleato o meno e si considera come terreno difficile.
	
\textbf{Pietrificato}: \index{Pietrificato}Un personaggio pietrificato è stato trasformato in pietra ed è privo di sensi ed Indifeso. Se un personaggio pietrificato si incrina o si rompe, ma i pezzi rotti sono uniti al corpo quando ritorna di carne, il personaggio non viene ferito o danneggiato. Se il corpo pietrificato del personaggio è incompleto quando viene ritrasformato in carne, il corpo rimane incompleto e potrebbe avere una qualche perdita permanente di punti ferita e/o altre menomazioni.
	
\textbf{Paura}:\index{Paura} Incantesimi, Oggetti Magici e certe creature possono influenzare i personaggi con paura. In molti casi, il personaggio deve effettuare un Tiro Salvezza su Volontà per resistere agli effetti, e un tiro fallito indica che il personaggio è scosso, spaventato o in preda al panico.
	
\textbf{Prono}\index{Prono}: chi è prono ha un -1d6 ad attaccare ed un -4 alla Difesa. Alzarsi da prono costa 2 Azioni.
L'unica opzione di movimento per una creatura prona è di strisciare, a meno che non si rialzi e ponga termina alla sua condizione.
Il personaggio può eseguire una prova di Acrobatica, se è pari o superiore a 15 ed entro 20 ti permette di dimezzare questi malus e costa una Azione alzarsi, se fai 20 o più annulli i malus e costa 1 Azione alzarsi.

\textbf{Punti Ferita Massimi}\index{Punti Ferita Massimi}: una creatura che subisca un attacco che abbassa i punti ferita massimo deve calare prima i punti ferita massimi attuali e poi diminuisce dello stesso ammontare i punti ferita attuali. Se i punti ferita massimi arrivano a 0 la creatura è morta.

	
\textbf{Scosso}:\index{Scosso}I personaggi che sono scossi subiscono penalità -2 ai Tiri per Colpire, ai Tiri Salvezza e alle prove.

Un personaggio scosso che viene nuovamente scosso diventa spaventato, mentre invece un personaggio scosso che viene spaventato cade in preda al panico. Un personaggio spaventato che viene scosso o spaventato cade in preda al panico.

\textbf{Spaventato}:\index{Spaventato} I personaggi spaventati sono anche scossi, e inoltre fuggono dalla fonte della loro paura il più velocemente possibile, anche se possono scegliere la direzione di fuga.
	
A parte ciò, una volta che sono fuori vista (o udito) dalla fonte della loro paura, possono agire normalmente. Se la durata della paura non è ancora arrivata al termine, qualora dovessero incontrare di nuovo la fonte della loro paura, cercherebbero nuovamente di fuggire.
	
I personaggi che non sono in grado di fuggire possono combattere (anche se continuano ad essere scossi).
	
\textbf{Stordito/Svenuto}:\index{Stordito}\index{Svenuto} si considera che sia Indifeso.
	
\textbf{In Preda al Panico}:\index{In Preda al Panico} I personaggi in preda al panico sono scossi e, inoltre, hanno una probabilità del 50\% di far cadere a terra qualsiasi cosa stanno tenendo in mano e di fuggire dalla fonte del loro terrore il più in fretta possibile seguendo un percorso di fuga completamente casuale.
	
I personaggi in preda al panico fuggono davanti a qualsiasi altro pericolo che possano trovarsi di fronte. A parte ciò, una volta che sono fuori vista (o udito) dalla fonte della loro paura, possono agire normalmente.
	
I personaggi in preda al panico prendono anche la condizione Accovacciato se non possono fuggire.
	
\textbf{Trattenere} il fiato: COS*6 round. Ogni Azione -1 round. Incantesimo -3 round. Poi Tiro salvezza su Tempra DC 12 +1 per round.
	
\textbf{Rotto}\index{Rotto}
	
La condizione rotto ha i seguenti effetti, a seconda dell'oggetto:
	
- Se l'oggetto è un'arma, tutti gli attacchi effettuati con l'oggetto subiscono penalità -2 al Tiro per Colpire e per i danni. Tali armi ottengono un Colpo Critico soltanto con un 4 volte 6 naturale ed infliggono al massimo solo 1 volta il danno in aggiunta.
	
- Se l'oggetto è un'armatura o uno scudo, il bonus che concede alla Difesa è dimezzato, arrotondando per difetto. L'armatura rotta raddoppia la penalità di armatura alla Prova sulle Competenze.
	
- Se l'oggetto è un attrezzo necessario per una Competenza, tutte le prove di Competenza effettuate con esso subiscono penalità -2.
	
- Se l'oggetto è una Bacchetta o un Bastone, utilizzate il doppio delle cariche necessarie ogni volta che viene usato.
	
- Se l'oggetto non rientra in nessuna delle precedenti categorie, la condizione rotto non ha effetto sul suo uso. Gli oggetti con condizione rotto, a prescindere dal tipo, valgono il 25\% del loro costo normale. Se l'oggetto è magico, può essere riparato soltanto con l'incantesimo Fabbricare utilizzata da un incantatore di livello uguale o superiore a quello che ha creato dell'oggetto.
	
\textbf{Sanguinante}\index{Sanguinante} Una creatura che sta subendo danni da sanguinamento subisce la quantità di danno indicata all'inizio del suo round. Il sanguinamento può essere interrotto superando una prova di Pronto Soccorso con DC 15 o con l'uso di un incantesimo che curi ferite.
	
Alcuni effetti di sanguinamento causano un danno di caratteristica o persino un risucchio di caratteristica. 
	
Se non indicato diversamente il danno massimo da sanguinamento, anche cumulato, è di 5 Punti Ferita a round.
	
\end{multicols}

\pagebreak

\section{Autore}\index{Autore}

\bigskip

\textsc{Autore ed Ideatore}: Andres Zanzani - azanzani@gmail.com

\bigskip
\textsc{Coautore}: Roberta Giorgini - madgiorgini@yahoo.it

\bigskip
\textsc{Contributi}: Federica Angeli - angelifdc@gmail.com

\bigskip

Playtesting: Fabrizio Bonetti, Emanuele Pezzi, Nicola Ricottone, Marco Valmori, Edoardo Zanzani, Isotta Zanzani, Federica Angeli

\bigskip

Un grazie speciale a tutta la mia famiglia che mi ha sopportato e supportato in questi anni disperati!

\bigskip

Powered by \Large\LaTeX\ \normalfont\& \Large\textbf{GitHub}

\bigskip

Andres Zanzani

\bigskip
\bigskip


\section{Scheda, Manuale e Schermo}\index{Scheda}\index{Manuale}\index{Schermo}

\label{scheda-e-manuale}
{\normalsize
	
	
Siete invitati a scaricare da GitHub \textbf{Dungeon Bell System} liberamente e senza vincoli se non quelli espressi dalla licenza OGL. Il link diretto per l'ultima versione compilata di DBS è \href{https://github.com/buzzqw/TUS/blob/master/DBS\%20-\%20Dungeon\%20Bell\%20System.pdf}{Manuale DBS in pdf} oppure l'url https://github.com/buzzqw/TUS/blob/master/DBS\%20-\%20Dungeon\%20Bell\%20System.pdf
	
\medskip
	
* Questo il link per la scheda:
\href{https://github.com/buzzqw/TUS/blob/master/DBS-scheda.pdf}{DBS-Scheda.pdf}

oppure l'url  https://github.com/buzzqw/TUS/blob/master/DBS-scheda.pdf
	
\medskip

* Questo il link per lo schermo del Narratore:
\href{https://github.com/buzzqw/TUS/blob/master/screen.pdf}{screen.pdf}

oppure l'url  https://github.com/buzzqw/TUS/blob/master/screen.pdf

\medskip
	
Questa versione non è propriamente la più indicata da usare in quanto contiene tutte le modifiche attive e non definitive apportate al sistema. Le versioni "stabili" le trovi su Versioni, scarica l'ultima versione "stabile" di \href{https://github.com/buzzqw/TUS/releases}{DBS}  (\url{https://github.com/buzzqw/TUS/releases}) puoi vedere tutte le versioni rilasciate.
	
\bigskip
	
\textbf{Questo il link al changelog} che viene aggiornato ad ogni commit significativo   \href{https://github.com/buzzqw/TUS/blob/master/changelog.md}{changelog.md}

oppure \url {https://github.com/buzzqw/TUS/blob/master/changelog.md}

	
\medskip
	
Qualsiasi segnalazione o consiglio che vogliate darmi è più che ben accetto! Aprite una issue su GitHub!

}


\pagebreak

\includepdf[pages={1,2},scale=0.95]{DBS-scheda.pdf}

\pagebreak

\section{Licenza}\index{Licenza} 

\medskip

\textbf{{\huge DBS - Dungeon Bell System}} è licenziato con la OGL 5.1
 \medskip

{\scriptsize 
Permission to copy, modify and distribute the files collectively known as the System Reference Document 5.1 ("SRD5") is granted solely through the use of the Open Gaming License, Version 1.0a. This material is being released using the Open Gaming License Version 1.0a and you should read and understand the terms of that License before using this material.

The text of the Open Gaming License itself is not Open Game Content. Instructions on using the License are provided within the License itself. The following items are designated Product Identity, as defined in Section 1(e) of the Open Game License Version 1.0a, and are subject to the Conditions set forth in Section 7 of the OGL, and are not Open Content: Dungeons \& Dragons, D\&D, Player's Handbook, Dungeon Master, Monster Manual, d20 System, Wizards of the Coast, d20 (when used as a trademark), Forgotten Realms, Faerûn, proper names (including those used in the names of Spells or items), places, Underdark, Red Wizard of Thay, the City of Union, Heroic Domains of Ysgard, EverChanging Chaos of Limbo, Windswept Depths of Pandemonium, Infinite Layers of the Abyss, Tarterian Depths of Carceri, Gray Waste of Hades, Bleak Eternity of Gehenna, Nine Hells of Baator, Infernal Battlefield of Acheron, Clockwork Nirvana of Mechanus, Peaceable Kingdoms of Arcadia, Seven Mounting Heavens of Celestia, Twin Paradises of Bytopia, Blessed Fields of Elysium, Wilderness of the Beastlands, Olympian Glades of Arborea, Concordant Domain of the Outlands, Sigil, Lady of Pain, Book of Exalted Deeds, Book of Vile Darkness, Beholder, gauth, Carrion Crawler, tanar'ri, baatezu, Displacer Beast, Githyanki, Githzerai, Mind Flayer, illithid, Umber Hulk, Yuan-ti.
All of the rest of the SRD5 is Open Game Content as described in Section 1(d) of the License. The terms of the Open Gaming License Version 1.0a are as follows:


\medskip

\textbf{OPEN GAME License Version 1.0a}

\medskip

The following text is the property of Wizards of the Coast, LLC. and is Copyright 2000 Wizards of the Coast, Inc ("Wizards"). All Rights Reserved.

\medskip

\textbf{1}. Definitions: (a)"Contributors" means the copyright and/or trademark owners who have contributed Open Game Content; (b)"Derivative Material" means copyrighted material including derivative works and translations (including into other computer languages), potation, modification, correction, addition, extension, upgrade, improvement, compilation, abridgment or other form in which an existing work may be recast, transformed or adapted; (c) "Distribute" means to reproduce, License, rent, lease, sell, broadcast, publicly display, transmit or otherwise distribute; (d)"Open Game Content" means the game mechanic and includes the methods, procedures, processes and routines to the extent such content does not embody the Product Identity and is an enhancement over the prior art and any additional content clearly identified as Open Game Content by the Contributor, and means any work covered by this License, including translations and derivative works under copyright law, but specifically excludes Product Identity. (e) "Product Identity" means product and product line names, logos and identifying marks including trade dress; artifacts; creatures characters; stories, storylines, plots, thematic elements, dialogue, incidents, language, artwork, symbols, designs, depictions, likenesses, formats, poses, concepts, themes and graphic, photographic and other visual or audio representations; names and descriptions of characters, Spells, enchantments, personalities, teams, personas, likenesses and Special abilities; places, locations, environments, creatures, Equipment, magical or supernatural Abilities or Effects, logos, symbols, or graphic designs; and any other trademark or registered trademark clearly identified as Product identity by the owner of the Product Identity, and which specifically excludes the OPEN Game Content; (f) "Trademark" means the logos, names, mark, sign, motto, designs that are used by a Contributor to Identify itself or its products or the associated products contributed to the Open Game License by the Contributor (g) "Use", "Used" or "Using" means to use, Distribute, copy, edit, format, modify, translate and otherwise create Derivative Material of Open Game Content. (h) "You" or "Your" means the licensee in terms of this agreement.

\medskip

\textbf{2}. The License: This License applies to any Open Game Content that contains a notice indicating that the Open Game Content may only be Used under and in terms of this License. You must affix such a notice to any Open Game Content that you Use. No terms may be added to or subtracted from this License except as described by the License itself. No other terms or Conditions may be applied to any Open Game Content distributed using this License.

\medskip

\textbf{3}.Offer and Acceptance: By Using the Open Game Content You indicate Your acceptance of the terms of this License.

\medskip

\textbf{4}. Grant and Consideration: In consideration for agreeing to use this License, the Contributors grant You a perpetual, worldwide, royalty-free, nonexclusive License with the exact terms of this License to Use, the Open Game Content.

\medskip

\textbf{5}.Representation of Authority to Contribute: If You are contributing original material as Open Game Content, You represent that Your Contributions are Your original Creation and/or You have sufficient rights to grant the rights conveyed by this License.

\medskip

\textbf{6}.Notice of License Copyright: You must update the COPYRIGHT NOTICE portion of this License to include the exact text of the COPYRIGHT NOTICE of any Open Game Content You are copying, modifying or distributing, and You must add the title, the copyright date, and the copyright holder's name to the COPYRIGHT NOTICE of any original Open Game Content you Distribute.

\medskip

\textbf{7}. Use of Product Identity: You agree not to Use any Product Identity, including as an indication as to compatibility, except as expressly licensed in another, independent Agreement with the owner of each element of that Product Identity. You agree not to indicate compatibility or co-adaptability with any Trademark or Registered Trademark in conjunction with a work containing Open Game Content except as expressly licensed in another, independent Agreement with the owner of such Trademark or Registered Trademark. The use of any Product Identity in Open Game Content does not constitute a Challenge to the ownership of that Product Identity. The owner of any Product Identity used in Open Game Content shall retain all rights, title and interest in and to that Product Identity.

\medskip

\textbf{8}. Identification: If you distribute Open Game Content You must clearly indicate which portions of the work that you are distributing are Open Game Content.

\medskip

\textbf{9}. Updating the License: Wizards or its designated Agents may publish updated versions of this License. You may use any authorized version of this License to copy, modify and distribute any Open Game Content originally distributed under any version of this License.

\medskip

\textbf{10}. Copy of this License: You MUST include a copy of this License with every copy of the Open Game Content You Distribute.

\medskip

\textbf{11}. Use of Contributor Credits: You may not market or advertise the Open Game Content using the name of any Contributor unless You have written permission from the Contributor to do so.

\medskip

\textbf{12}. Inability to Comply: If it is impossible for You to comply with any of the terms of this License with respect to some or all of the Open Game Content due to statute, judicial order, or governmental regulation then You may not Use any Open Game Material so affected.

\medskip

\textbf{13}. Termination: This License will terminate automatically if You fail to comply with all terms herein and fail to cure such breach within 30 days of becoming aware of the breach. All sublicenses shall survive the termination of this License.

\medskip

\textbf{14}. Reformation: If any provision of this License is held to be unenforceable, such provision shall be reformed only to the extent necessary to make it enforceable.

\medskip

\textbf{15}. COPYRIGHT NOTICE Open Game License v 1.0a Copyright 2000, Wizards of the Coast, LLC. System Reference Document 5.1 Copyright 2016, Wizards of the Coast, LLC.; Authors Mike Mearls, Jeremy Crawford, Chris Perkins, Rodney Thompson, Peter Lee, James Wyatt, Robert J. Schwalb, Bruce R. Cordell, Chris Sims, and Steve Townshend, based on original material by E. Gary Gygax and Dave Arneson.}

\bigskip

{\normalsize Le immagini incluse nel manuale sono licenziate per un uso libero o sono senza licenza. In caso di inclusione per errore di immagini protette da copyright vi invito a segnalarle per la rimozione.}

\pagebreak

{\scriptsize 
\printindex}
\end{document}
