\section{La Magia}\index{Magia}\index{Essenza}


\label{la-magia}
\begin{enfasi}{
Le parole sono, nella mia NON modesta opinione, la nostra massima ed inesauribile fonte di magia. In grado sia di infliggere dolore che di alleviarlo (Albus Silente)

\medskip

Non lascerai vivere colei che pratica la magia. (Libro dell'Esodo)(Sempre a seconda dei propri Tratti...)} \end{enfasi} \medskip


\begin{multicols}{2}

\lettrine[lines=2, lhang=0.33, loversize=0.25, findent=1.5em]{L}{a} magia permea i mondi di gioco e la sua forma più comune è quella di un incantesimo. Questo capitolo fornisce le regole per lanciare incantesimi. 

\medskip

\subsection{Cos'è un Incantesimo?}

Un incantesimo è un preciso effetto magico, una singola alterazione delle energie magiche che permeano il multiverso in una specifica, limitata, espressione. Nel lanciare un incantesimo, un personaggio tira attentamente i fili invisibili della magia pura che i Patroni hanno concesso e li ricuce in una trama particolare, li fa vibrare in un modo specifico, e poi li libera per scatenare l'effetto desiderato: tutto ciò, nella maggior parte dei casi, in pochi secondi.

Gli incantesimi possono essere strumenti versatili, armi o barriere protettive. Possono infliggere danni o ripararli, imporre o rimuovere condizioni, risucchiare l'energia vitale e ridare vita ai morti (se permesso!).

Nel corso della storia del multiverso sono stati creati innumerevoli migliaia di incantesimi, molti dei quali sono andati dimenticati. Alcuni possono ancora essere nascosti tra le pagine di libri degli incantesimi impolverati all'interno di antiche rovine o segregati nella mente di divinità morte. Oppure potrebbero un giorno essere reinventati da un personaggio che abbia ammassato sufficiente potere e capacità per farlo.

\subsection{Le caratteristiche degli incantesimi}

La descrizione di ciascun incantesimo inizia con un blocco di informazioni che comprende la Difficoltà, scuola di magia, tempo di lancio, gittata, componenti e durata dell'incantesimo. Il resto della descrizione ci informa dell'effetto dell'incantesimo. 

Quando un personaggio lancia qualsiasi incantesimo, si usano le seguenti regole base indipendentemente dall'effetto dell'incantesimo.

\medskip

\textbf{Tempo di Lancio}

La maggior parte degli incantesimi possono essere lanciati con due Azioni. Alcuni incantesimi richiedono un'Azione Immediata, una Reazione o molto più tempo per essere lanciati.

\textbf{Azione Immediata}

Un incantesimo lanciato con un'Azione Immediata è particolarmente rapido. Puoi usare un'Azione Immediata durante il tuo round per lanciare l'incantesimo che sia Immediato, purché tu non abbia già effettuato un'Azione Immediata durante il tuo round. Durante lo stesso round non puoi lanciare un altro incantesimo, a meno che non si tratti di un incantesimo a Difficoltà 12 (chiamati trucchetti).


\includegraphics[width=0.7\linewidth]{immagini/Hex32.jpg}
	
\textit{The Witchcraft Art of Jacques de Gheyn II}


\textbf{Reazioni}

Alcuni incantesimi possono essere lanciati come Reazioni. Questi incantesimi richiedono una frazione di secondo per essere creati e possono essere lanciati in risposta a un evento. Se un incantesimo può essere lanciato come reazione, la descrizione dell'incantesimo ti dice esattamente quando puoi farlo. Devi avere a disposizione una Reazione e non averla già usata.

\textbf{Tempo di Lancio Più Lungo}

Certi incantesimi richiedono più tempo per essere lanciati: minuti o addirittura ore. Quando lanci un incantesimo con tempo di lancio più lungo di due Azioni, devi spendere una azione ogni round per lanciare l'incantesimo, e nel farlo devi mantenere anche la concentrazione (vedi "Concentrazione" di seguito). Se la tua concentrazione viene infranta, l'incantesimo fallisce, ma non avrai speso lo slot incantesimo. Se vuoi tentare di lanciare l'incantesimo di nuovo, dovrai ricominciare da capo.

\begin{center}
	\includegraphics[width=0.8\linewidth]{immagini/infanticidalwitch.jpg}
	
	\textit{The Witchcraft Art of Jacques de Gheyn II}
\end{center}

\medskip

\textbf{Le Scuole di Magia}\index{Le Scuole di Magia}

Le accademie di magia raggruppano gli incantesimi in nove categorie dette scuole di magia. Gli studiosi applicano queste categorie a tutti gli incantesimi, credendo che tutta la magia funzioni essenzialmente allo stesso modo, che derivi da uno studio rigoroso e venga conferita da un Patrono.

Le scuole di magia aiutano a descrivere gli incantesimi; non hanno delle proprie regole, sebbene alcune regole possano fare riferimento a queste scuole.

\begin{itemize}
\item
\textit{Abiurazione} riguarda incantesimi di natura protettiva, sebbene ne contenga anche alcuni dall'uso aggressivo. Questi incantesimi creano barriere magiche, negano effetti dannosi, danneggiano i violatori, o bandiscono le creature in altri piani di esistenza.

\item
\textit{Ammaliamento} riguarda incantesimi che agiscono sulla mente altrui, influenzandone o controllandone il comportamento. Questi incantesimi possono far sì che i nemici considerino l'incantatore un amico, forzare creature a effettuare determinate azioni, o addirittura controllare un'altra creatura come fosse una marionetta.

\item
\textit{Cura} riguarda gli incantesimi che permettono di recuperare le energie fisiche, mentali ed annullare debolezze e veleni.

\item
\textit{Divinazione} riguarda incantesimi che rivelano informazioni nella forma di segreti da tempo dimenticati, visioni del futuro, la posizione di oggetti nascosti, la verità dietro le illusioni o immagini di persone e luoghi lontani.

\item
\textit{Evocazione} riguarda incantesimi che trasportano oggetti e creature da un luogo all'altro. Alcuni incantesimi richiamano creature o oggetti al fianco dell'incantatore, mentre altri permettono all'incantatore di teletrasportarsi da un luogo a un altro. Alcune evocazioni creano oggetti o effetti dal nulla. 

\begin{center}
	\includegraphics[width=0.8\linewidth]{immagini/Leonids-1833.jpg}
	
	\textit{The most famous depiction of the famous 1833 Leonids \textbf{Meteor Storm}.}
\end{center}

\textit{Illusione} riguarda incantesimi che ingannano i sensi e la mente altrui. Fanno vedere alle persone cose che non esistono, non gli fanno notare le cose che esistono, fanno udire rumori fasulli o ricordare cose che non sono mai accadute. Alcune illusioni creano immagini spettrali che chiunque può vedere, ma le illusioni più insidiose impiantano un'immagine direttamente nella mente di una creatura.

\item
\textit{Invocazione} riguarda incantesimi che manipolano l'energia magica per produrre un effetto desiderato. Molti incantesimi creano esplosioni di fuoco o fulmini.

\item
\textit{Necromanzia} riguarda incantesimi che manipolano le energie della vita e della morte. Questi incantesimi possono conferire una riserva aggiuntiva di forza vitale, risucchiare l'energia vitale da un'altra creatura, creare non morti o addirittura riportare in vita i morti (se concesso).

Creare non morti tramite l'uso di incantesimi di necromanzia come animare morti non è un'azione buona, e solo gli incantatori malvagi fanno frequentemente uso di questo incantesimo.

In DBS solo un Patrono ha sufficiente potere per poter riportare in vita un morto.

\item
\textit{Trasmutazione} riguarda incantesimi che cambiano le proprietà di una creatura, oggetto o ambiente. Possono trasformare un nemico in una creatura innocua, aumentare la forza di un alleato, far spostare un oggetto agli ordini dell'incantatore o potenziare le capacità di guarigione innate di una creatura per farle recuperare più in fretta da una ferita.

\item
\textit{Universale} alcuni incantesimi sono capisaldi della magia in se e come tali accessibili a tutte le scuole e maghi.

\end{itemize}

\begin{narratoresmall}{In DBS i giocatori hanno libero accesso a tutti gli incantesimi con quindi la possibilità di avere una lista estremamente variegata e potente. Per fare in modo che le scelte siano diverse tra i vari giocatori e tra le avventure, insistete sulle componenti, fate in modo che siano indispensabili al personaggio e siano usate. Fate anche in modo che gli incantesimi nuovi siano non facili da trovare come alla stregua di un tesoro.
}\end{narratoresmall}


\medskip

\textbf{Gittata}\index{Gittata Incantesimi}

Il bersaglio di un incantesimo deve essere nella gittata dell'incantesimo. Per un incantesimo come dardo incantato, il bersaglio è una creatura. Per un incantesimo come palla di fuoco, il bersaglio è il punto nello spazio da cui la sfera di fuoco esplode. La maggior parte degli incantesimi hanno una gittata espressa in metri. Alcuni incantesimi possono prendere a bersaglio solo una creatura (te compreso) con cui sei in contatto fisico. Altri incantesimi, come l'incantesimo scudo, agiscono solo su di te: questi incantesimi hanno come gittata personale.

Gli incantesimi che creano coni o linee di effetto che originano da te, hanno anch'essi gittata personale, a indicare che sei tu il punto di origine dell'effetto dell'incantesimo (vedi "Aree di Effetto" più avanti in questo capitolo).

Una volta lanciato l'incantesimo, i suoi effetti non sono più limitati dalla sua gittata, a meno che la descrizione dell'incantesimo non dica altrimenti.

\medskip

\textbf{Componenti}\index{Componenti Incantesimi}

Le componenti di un incantesimo sono i requisiti fisici che devi soddisfare per lanciarlo. La descrizione di ciascun incantesimo indica se richieda componenti verbali (V), somatiche (S) o materiali (M). Se non sei in grado di fornire una o più delle componenti dell'incantesimo, non potrai lanciarlo.

La maggior parte degli incantesimi richiede di intonare parole mistiche. Le parole, il ritmo, la cadenza e risonanza che mettono in moto i filamenti della magia. Di conseguenza, un personaggio imbavagliato o in un'area di silenzio, come quella creata dall'incantesimo silenzio, non può lanciare incantesimi con componenti verbali.

\medskip

\begin{center}
	\includegraphics[width=0.8\linewidth]{immagini/merlin.png}
	
	\textit{Merlin dictating his prophecies to his scribe. Robert de Boron's Merlin en prose (written ca 1200)}
\end{center}

\medskip

\textbf{Lanciare Incantesimi in Armatura}

Data la concentrazione mentale e i gesti precisi richiesti l'armatura distrae e sbilancia i flussi. La Difficoltà di lancio dell'incantesimo aumenta come indicato nella tabella delle armature.\\

\medskip

\textbf{Durata}\index{Durata Incantesimi}

La durata di un incantesimo è la lunghezza di tempo per cui esso persiste. La durata può essere espressa in round, minuti, ore o addirittura anni. Alcuni incantesimi specificano che i loro effetti durano finché l'incantesimo non viene dissolto o distrutto.

\begin{itemize}
	
\item
\textit{Istantanea}

Molti incantesimi sono istantanei. L'incantesimo ferisce, cura, crea o altera una creatura o un oggetto in modo che non possa essere dissolto, dato che la sua magia esiste solo per un istante.

\item

\textit{Concentrazione}

Alcuni incantesimi richiedono che tu mantenga la concentrazione per tenerne la magia attiva. Se perdi la concentrazione, l'incantesimo avrà fine. Se un incantesimo deve essere mantenuto tramite concentrazione, la cosa è indicata alla voce Durata, e l'incantesimo specifica quanto a lungo vi potrai mantenere la concentrazione. Puoi terminare la concentrazione in qualsiasi momento (senza usare un'azione).

Normali attività, come muoversi e attaccare, non interferiscono con la concentrazione. Mantenere la concentrazione costa 1 Azione a round.
\end{itemize}

\medskip

\textbf{Somatica (S)}

La gestualità del lancio di un incantesimo può includere un gesticolare forzato o intricate serie di gesti. Se un incantesimo richiede una componente somatica, l'incantatore deve essere libero di usare almeno una mano per eseguire questi gesti.

\medskip

\textbf{Materiale (M)}

Lanciare taluni incantesimi richiede oggetti particolari,specificati tra parentesi alla voce componenti. Un personaggio può usare una borsa dei componenti o un focus di incantamento al posto delle componenti specificate da un incantesimo. Se la componente ha un costo indicato, il personaggio deve procurarsi quella specifica componente prima di poter lanciare l'incantesimo.

Se un incantesimo indica che la componente materiale viene consumata dall'incantesimo, l'incantatore deve fornire questa componente per ogni lancio dell'incantesimo.
Un incantatore deve avere una mano libera per avere accesso a queste componenti, ma può essere la stessa mano utilizzata per eseguire le componenti somatiche.

\medskip

\textbf{Lanciare un altro incantesimo che richiede concentrazione}

Perdi la concentrazione su di un incantesimo se lanci un altro incantesimo che richieda concentrazione. Non ti puoi concentrare su due incantesimi alla volta. 


\medskip

\textbf{Essere inabile o ucciso}

Perdi la concentrazione su di un incantesimo se svieni o muori. Se scendi sotto gli zero PF perdi uno slot di incantesimi da lanciare per quel giorno.

\medskip

Il Narratore può anche decidere che certi fenomeni ambientali, come un'onda che si abbatte su di te mentre sei sul ponte di una nave in mezzo a una tempesta, una forte grandinata... siano situazioni di disturbo per lanciare un incantesimo.

\medskip

\textbf{Bersagli}\index{Bersagli Incantesimi}

Un normale incantesimo richiede che tu scelga uno o più bersagli che siano affetti dalla sua magia. La descrizione dell'incantesimo ti dice se l'incantesimo prende a bersaglio creature, oggetti o un punto di origine per generare un'area di effetto (descritta di seguito).A meno che l'incantesimo non abbia un effetto percepibile, una creatura potrebbe non capire mai di essere stata bersaglio di un incantesimo. Un effetto come un fulmine crepitante è palese, ma un effetto più subdolo, come il tentativo di leggere i pensieri di una creatura, di solito non viene notato, a meno che l'incantesimo non dica altrimenti.

Lanciare un incantesimo è una azione che non passa inosservata. Una prova di Nascondersi a difficoltà 13 oppure lanciare l'incantesimo come se si fosse Distratto (+5 alla Difficoltà) permettono di celare il lancio, se non avviene proprio davanti al bersaglio.

\textbf{Traiettoria Sgombra Verso il Bersaglio}

Per prendere come bersaglio una cosa, devi avere la traiettoria sgombera verso di essa, e quindi questa non può trovarsi dietro una copertura totale. Se piazzi un'area di effetto in un punto che non puoi vedere e un'ostruzione, come un muro, si trova tra di te e quel punto, il punto di origine si crea dal lato più vicino dell'ostruzione.

\textbf{Prendere Te Stesso Come Bersaglio} 

Se un incantesimo prende come bersaglio una creatura a tua scelta, puoi scegliere anche te stesso, a meno che la creatura non debba essere ostile o sia specificato che non possa essere tu. Se ti trovi nell'area di effetto di un incantesimo lanciato da te, anche tu ne sarai influenzato.

\medskip

\textbf{Aree di Effetto}\index{Aree di Effetto Incantesimi}

Incantesimi come mani brucianti e cono di freddo coprono un'area, permettendogli di colpire più creature alla volta.

La descrizione di un incantesimo specifica la sua area di effetto, che di solito rientra in una di queste cinque forme: cilindro, cono, cubo, linea o sfera. Ogni area di effetto ha un punto di origine, un luogo da cui erutta l'energia dell'incantesimo. Le regole per ciascuna forma specificano come posizionare il suo punto di origine. Di solito il punto di origine è un punto nello spazio, ma alcuni incantesimi hanno un'area la cui origine è una creatura o un oggetto.

\begin{itemize}
\item
\textit{Cilindro}

Il punto di origine di un cilindro è il centro di un cerchio di specifico raggio, come indicato nella descrizione dell'incantesimo. Il cerchio deve essere sul pavimento o all'altezza dell'effetto dell'incantesimo. L'energia in un cilindro si espande in linee dritte dal punto di origine al perimetro del cerchio, formando la base del cilindro. L'effetto dell'incantesimo parte poi dal basso verso l'alto o dall'alto verso il basso, fino a una distanza uguale all'altezza del cilindro.Il punto di origine del cilindro è incluso nella sua area di effetto.

\item

\textit{Cono}

Un cono si estende in una direzione a tua scelta dal suo punto di origine. L'ampiezza di un cono in un dato punto della sua lunghezza è uguale alla distanza di quel punto dal punto di origine. L'area di effetto di un cono specifica la sua lunghezza massima. Il punto di origine del cono non è incluso nella sua area di effetto, a meno che tu non decida altrimenti.

\item
\textit{Cubo}

Selezioni il punto di origine di un cubo, su cui si piazzerà una delle facce dell'effetto cubico. La taglia del cubo viene espressa come lunghezza di ciascun suo spigolo. Il punto di origine del cubo non è incluso nella sua area di effetto, a meno che tu non decida altrimenti.

\item
\textit{Linea}

Una linea si estende dal suo punto di origine in un percorso dritto per tutta la sua lunghezza e copre un'area definita dalla sua larghezza. Il punto di origine della linea non è incluso nella sua area di effetto, a meno che tu non decida altrimenti.

\item
\textit{Sfera}

Selezioni il punto di origine di una sfera, e la sfera si estenderà da quel punto. La taglia della sfera è espressa come raggio in metri che si estende da quel punto.
Il punto di origine della sfera è incluso nella sua area di effetto.

\end{itemize}


\begin{center}
	\includegraphics[width=0.6\linewidth]{immagini/3dforme.png}
	\textit{Cono, Sfera, Cilindro, Cubo}
\end{center}

\textbf{Combinare Effetti Magici}\\
Gli effetti di incantesimi diversi si sommano fino a che la loro durata si sovrappone. Tuttavia, gli effetti dello stesso incantesimo lanciato più volte sullo stesso bersaglio non si combinano. Sarà invece l'incantesimo più potente fra quelli lanciati (per esempio, quello con il bonus più alto) ad applicarsi finché le durate si sovrappongono.\\

\subsection{Regole di base...}

\begin{itemize}

\item
Il mago al lancio del suo primo incantesimo sceglie se utilizzare come modificatore alla prova di Competenza Magica l'Intelligenza oppure se e' un Devoto puo' scegliere la Caratteristica indicata dal Patrono.

Una volta fatta la scelta non e' più possibile cambiarla. Questo modificatore viene chiamato \emph{modificatore di caratteristica da incantatore}.
\item 
Ogni volta che il mago acquisisce un punto in Competenza Magica puo' apprendere due nuovi incantesimi che abbia a disposizione nel suo Tomo. Questi possono avere Difficoltà massima pari a 12 + Competenza Magica + modificatore di caratteristica da incantatore + eventuale bonus di scuola preferita dato dal Patrono (se Seguace o Devoto).
\item
Il personaggio quando assegna il primo punto di Competenza Magica apprende due incantesimi + metà del modificatore di caratteristica/2 (arrotondato per difetto)
\item
Ogni volta che il mago acquisisce un punto in Competenza Magica puo' rinunciare ad apprendere un incantesimo con Difficoltà pari o superiore a 16 per apprendere due Trucchetti (Difficoltà 12).
\item 
Ogni volta che il mago acquisisce un punto in Competenza Magica è possibile dimenticare un incantesimo appreso e sostituirlo con un altro a disposizione nel Tomo.
\item 
Per poter lanciare un incantesimo il mago deve superare la Difficoltà dell'incantesimo con la prova di Competenza Magica.

La prova si effettua con 3d6+Competenza Magica+modificatore di caratteristica da incantatore ed eventuali bonus dati da Abilità e dal Patrono.
\item 
Un incantatore può formulare nel giorno un numero di magie pari a Competenza Magica/2 (arrotondato per difetto, con un minimo di 1)+modificatore di caratteristica da incantatore. Questo numero viene anche chiamato slot incantesimo.
\item
Un Devoto aggiunge +4 alla Prova di Magia nelle scuole preferite dal Patrono e puo' sostituire le forme di energia degli incantesimi con quelle preferite dal Patrono
\item
Gli incantesimi con Difficoltà 12 (detti anche trucchetti) non contano nel numero delle magie lanciate nel giorno. Va comunque fatta la prova di competenza finché Competenza Magica non e' maggiore di 4.
\item
Rispettivamente a Competenza Magica 9, 12, 14 al giocatore non e' piu' richiesto fare una prova di magia per lanciare incantesimi con Difficoltà 16, 19 e 21.
\item
Se nel lancio di un incantesimo ottiene almeno un critico (due volte 6 nel lancio dei dadi) non si computa questa magia nel numero di incantesimi lanciabili al giorno.
\item
Se sei un Seguace ottieni +2 alle prove di magia nella scuola preferita dal Patrono
\end{itemize}


\subsubsection{Il Tomo della Magia}\index{Il Tomo della Magia}

Se i Patroni garantiscono l'accesso alla fonte della magia è solo l'applicazione di antichi riti e formule che permette di manifestare questa energia grezza in una forma ed espressione che chiamiamo incantesimo.

Ogni usufruitore di magia ha un proprio \textbf{Tomo} degli incantesimi, non pensate solo a un grosso tomo antico rilegato in pelle, le diverse culture hanno sviluppato nel tempo la capacità di iscrivere le rune degli incantesimi in carte, bastoni, lastre di pietra, tatuaggi... fate la vostra scelta quando create il personaggio.
Questa scelta non vi impedirà di copiare incantesimi da \textbf{Tomi} fatti diversamente (foglie di tabacco, liquidi della conoscenza..) per voi sarà sempre facile (Arcana DC 12) capire se si e' di fronte ad un Tomo di qualche tipo, semplicemente il vostro "metodo" di studio (o tradizione) vi ha insegnato un sistema diverso.

Il personaggio con Competenza Magica 1 partirà con 2 + modificatore da caratteristica per incantesimi di magie a sua scelta (oppure dei trucchetti) trascritti nel suo "tomo" e ogni altro incantesimi che vorrà imparare dovrà trovarlo e trascriverlo sul suo libro.

\includegraphics[width=0.8\linewidth]{immagini/spellbook.jpg}

Ogni incantesimi occupa un numero di pagine nel tomo pari alla propria (Difficoltà-10)/2 (arrotondando per eccesso), copiare una pagina di incantesimo porta via 1 ora di lavoro e 10 mo di preziosi inchiostri.

Un Tomo (libro) di incantesimi costa 10 mo per pagina che puo' custodire.

Un incantatore puo' copiare sul suo tomo incantesimi fino ad una Difficoltà pari a 14 + Competenza Magica + modificatore di caratteristica da incantatore + eventuale bonus di scuola preferita dato dal Patrono (se Seguace o Devoto). Se l'incantesimo e' di piu' alta Difficoltà il mago deve fare una prova di magia a quel grado di Difficoltà e solo se passa la prova potrà copiare l'incantesimo nel tomo.

Se la prova fallisce non riuscirà a copiare quell'incantesimo fino al prossimo punto di Competenza Magica acquisito. Se la prova fallisce con un risultato inferiore alla metà della Difficoltà dell'incantesimo, accadranno brutte cose al Tomo e 1d4 incantesimi verranno cancellati dal tomo stesso.

La sorgente di nuovi incantesimi puo' essere un altro tomo, bastone, pergamena.. insomma qualsiasi cosa che il precedente mago usasse per custodire gli incantesimi. Un oggetto magico (bastone magico, anello, verga..bacchetta..) non e' idoneo quale fonte da cui copiare l'incantesimo che contiene, si deve copiare dall'equivalente tomo di un altro mago.

Durante le avventure il vostro mago potrà copiare tanti e numerosi incantesimi sul suo tomo ma non potrà apprenderli immediatamente. Il personaggio quando acquisirà un nuovo punto di Competenza Magica potrà dimenticare un incantesimo appreso per sostituirlo con un incantesimo presente nel suo tomo.

\medskip

\begin{tcolorbox}[width=0.43\textwidth,title = Scegliere gli Incantesimi]
Avete letto bene, gli incantesimi non si apprendono da soli, non si scelgono da una lista. Ogni incantesimo e' un'arte preziosa che si impara e si deve trovare.	
	
I personaggi dovranno intraprendere perigliose avventure, pagare mercenari, cercare i tomi antichi e svelare i segreti piu' oscuri e dimenticati per poter imparare nuovi incantesimi.
	
Ogni incantesimo e' alla stregua di un oggetto magico, un vero tesoro da cercare!
\end{tcolorbox}

\medskip

Tramite un rito magico difficile e costoso potrà per un solo giorno sostituire 1 incantesimo con altro presente nel tomo. Questo rito, della durata di 1 ora per Difficoltà e dal costo di 10 mo sempre Difficoltà dell'incantesimo, permetterà per le 24 ore successive di sostituire un incantesimo appreso con uno presente nel proprio tomo e apprendibile data la Difficoltà.

\medskip

\begin{narratoresmall}
	Gli incantesimi diventano oggetti e premi magici a tutti gi effetti. Sfruttate la sete di conoscenza e potere dei personaggi per costruire avventure interessanti che possano ruotare attorno tomi antichi e leggendari incantesimi perduti.
\end{narratoresmall}

\subsubsection{Tiro per Colpire con le Magie}\index{Tiro per Colpire con Incantesimi}

Quando l'incantesimo ti dice di fare un Tiro per Colpire se e' un "\textbf{attacco a distanza con incantesimo}" devi effettuare un Tiro per Colpire contro la Difesa dell'avversario. Questo Tiro per Colpire e' effettuato con 3d6+\textbf{Competenza Magica}+Destrezza più Abilità e modificatori vari.\\

\medskip

Se il Tiro \textbf{per Colpire con incantesimo e' in mischia} eseguirai un Tiro per Colpire con il bonus di Forza (3d6+\textbf{Competenza Magica}+Forza) ed Abilità e modificatore vari contro la Difesa dell'avversario.

\medskip

Quando la magia e' ad area non e' necessario effettuare un Tiro per Colpire se non per difficili e specificate aree, ovvero si mira in una area ben circoscritta o si vuole evitare di colpire qualcuno con un incantesimo ad area.

\medskip

\begin{tcolorbox}[width=0.85\linewidth,title=Personaggi Gish (Guerrieri Maghi)]
Usando un unica Competenza Attiva per i Tiri per Colpire e usare la Magia e' piu' facile costruire personaggi che usino la magia per combattere, ma non che combattano anche con la magia...

Per quanto il mago sarà bravo a colpire ed esperto in incantesimi, con un arma in mano i suoi colpi non avranno mai la precisione che con la magia.

Distribuite con attenzione i punti delle Competenze Attive, un punticino in Competenza Armi si rileverà sempre molto utile\end{tcolorbox}


\subsubsection{L'esplosione del 6 nella Magia}\index{Esplosione del 6 nella Magia}

Anche nella prova di Competenza Magica i 6 esplodono, i 6 tirati nella prova di magia vengono ritirati, e ritirati ancora nel caso, e sommati alla prova di magia. Per le prove di Competenza Magica l'uno non viene conteggiato, conta 0.

Tenete traccia di quanti critici (due 6 tirati) fate, potrebbero servire per ottenere effetti "speciali" nell'incantesimo! Ricordate che se effettuate un critico l'incantesimo non si conta tra gli incantesimi lanciati nel giorno.


\begin{tcolorbox}[width=0.85\linewidth,title = La Prova di Magia] DBS e' strutturato perché sia difficile lanciare gli incantesimi, specialmente quelli di massima Difficoltà, mediamente un tiro di 12 o piu' sui dadi e' necessario per passare la prova a lanciare la magia. Si presume però che il personaggio che intraprenda la strada della magia continui a perseguirla, detta diversamente con valori di Competenza Magica bassa non aspettatevi di poter lanciare facilmente incantesimi di Difficoltà elevata!	
\end{tcolorbox}

\subsubsection{Tentare la sorte con la Magia}\index{Tentare la sorte con la Magia}

Anche nella prova di competenza magica puoi Tentare la Sorte, ovvero rinunci ad un +4 di bonus (solo tra Competenza Magica, da modificatore di caratteristica da incantatore e da bonus concesso da Patrono) e aggiungi un d6 in più nel tiro della prova.

\subsubsection{Riuscire e Fallire nella prova di Magia}\index{Riuscire e Fallire nella prova di Magia}

Per capire se si riesce nella formulazione dell'incantesimo si deve innanzitutto superare, con una prova di Competenza Magica (3d6 + Competenza Magica + caratteristica da incantatore + bonus concesso dal Patrono + Abilità...) la Difficoltà indicata nell'incantesimo stesso.

Se non si riesce a superare la Difficoltà l'incantesimo non si attiva e non sortisce effetto (anche se il Narratore potrebbe descrivere l'eventuale incantesimo scaturito tirando tre volte uno con i dadi...).

Anche se la prova fallisce l'incantesimo si conta tra quelli lanciati e i componenti eventuali sono consumati.

\begin{narratoresmall}
Concedete un +1 alla prova di magia quando il giocatore declama con perizia e trasporto il lancio dell'incatesimo. Se dice "\textit{Lancio una palla di fuoco}" non otterrà vantaggi ma se con trasporto declama "\textit{Per tutti i fuochi infernali possa Nedraf trascinarvi all'inferno con le sue sacre fiamme. Bruciate indegni. Palla di Fuoco!}" allora un bonus di +1 e' piu' che doveroso.	
\end{narratoresmall}	


\subsubsection{Riuscire molto bene nella prova di Magia* - Opzionale}\index{Riuscire molto bene nella prova di Magia - Opzionale}

Questa opzione puo' essere applicata se non si vuole lasciare al fato dei dadi la possibilità o meno di fare dei critici, bensì all'abilità del mago.

Se il Narratore lo permette, da stabilire ad inizio campagna, si ottiene un critico per ogni 50\% in più che si ottiene nella prova di magia.

Es. 6,5,6. Ritiro i sei e ottengo 5,5. Sommo 9 di Competenza Magica e 3 di caratteristica da incantatore per un totale di 39. Non conto quindi 1 critico perché ho fatto due volte 6, ma dato che volevo lanciare l'incantesimo Cura Ferite Leggere che ha Difficoltà 19 si avrà questo effetto:

1d8 + Saggezza + (39-16)/(16/2) = 1d8 + Saggezza + 2d6 aggiuntivi (ovvero due critici)


\subsubsection{Riuscire nel fallimento della prova di Magia* - Opzionale}\index{Riuscire nel fallimento della prova di Magia - Opzionale}

Può essere molto frustrante mancare magari di un solo punto una prova di magia e vedere vanificato l'incantesimo, ancor di piu' se si erano anche ottenuti critici.

Queste regole opzionali, sceglietene una sola quando si parte con la campagna, permettono al Narratore di stabilire se una prova di magia fallita può comunque essere sufficiente per ottenere un qualche risultato.

\begin{enumerate}

\item
\textbf{I 6 comandano}\index{I 6 comandano}: con questa regola se nella prova di magia fallita si sono ottenuti almeno 2 volte sei (quindi successo critico) il giocatore potrà lanciare un incantesimo di Difficoltà inferiore purché la Difficoltà sia superata.

Es: Tups fa 6,2,6. Ritira i sei ed ottiene 2,1, somma 2 come modificatore di caratteristica e 3 per la Competenza Magica per un totale di 22. Non potrà quindi lanciare Blocca Persona Avanzato  come aveva sperato (Difficoltà 23) ma potrà comunque lanciare ad esempio l'incantesimo Mura di Vento (Difficoltà 21). L'incantesimo così formulato consumerà tutte e 3 le Azioni.

\item
Fare che ogni personaggio con Competenza Magica 1 e che sia Seguace o Devoto abbia gratuitamente l'Abilità  \hyperlink{ilpatronoelamiamagia}{Il Patrono e' la mia magia}. 

\end{enumerate}

\begin{tcolorbox}[width=0.85\linewidth,title = Fallire gli Incantesimi]{
Può capitare ed anche spesso di fallire nella prova di magia quando si lanciano incantesimi della massima Difficoltà conosciuta. Provate con incantesimi a minore Difficoltà. Ricordate anche le possibilità offerta da Alterare la Magia ovvero Magia Efficace, Aumentare il tempo di lancio, Magie Collaborative e Circolo di Potere possono concedere un grosso vantaggio.

Infine un Seguace ottiene un +2 alla prova di Competenza Magia per la scuola preferita dal Patrono ed il Devono ottiene un +4 alla stessa prova. 

Questi modificatori, compreso l'Abilità Magie Potenti, sono tutti cumulativi.
}\end{tcolorbox}

\subsubsection{Resistere all'incantesimo (Tiro Salvezza)}\index{Resistere all'Incantesimo}\index{Tiro Salvezza Incantesimi}

\label{resistere-allessenza-tiro-salvezza}

Una volta che la prova di magia è superata e quindi la magia liberata, anche in base alla descrizione e note dell'incantesimo, è possibile resistere all'effetto della magia.

Il Tiro Salvezza in base a quanto richiesto dalla magia e' pari alla prova di magia effettuata dal mago. Una prova di magia particolarmente efficace renderà altrettanto difficile resistergli e in base all'incantesimo usato ed eventuali critici ottenuti nella prova potranno manifestarsi risultati ancora migliori. Nella descrizione dell'incantesimo e' descritto cosa succede in caso di critico nella prova di magia.

Se il Tiro Salvezza riesce o fallisce di più di 10 (\textbf{successo critico}\index{Successo Critico Incantesimi} o \textbf{fallimento critico}\index{Fallimento Critico Incantesimi}) il Narratore potrà decidere di applicare svantaggi o vantaggi al risultato finale.\index{Più di 10}.
E' anche possibile che nella descrizione dell'incantesimo sia riportato cosa succede in casa di successo o fallimento critico. 

Per i mostri o comunque per un lancio di incantesimi dato da abilità naturali o oggetti, se non specificato la DC del Tiro Salvezza e' pari alla Difficoltà dell'incantesimo+Intelligenza.

\medskip

\includegraphics[width=0.85\linewidth]{immagini/donnaserpente.png}
\textit{Henry Justice Ford}

\subsubsection{Resistenza alla Magia}\index{Resistenza alla Magia}

Una creatura potrebbe avere una naturale resistenza alle magie.

Il valore di RM (Resistenza Magia) indica tale resistenza e più è alta più la creatura è immune alla magia, che lo voglia o meno.

Ogni qual volta la creatura è influenzata direttamente da un incantesimo o effetto magico deve effettuare una prova di RM, ovvero tirare 3d6+Carisma sommare il valore di RM e se è superiore alla prova di magia effettuata dall'incantatore l'incantesimo non ha effetto.

In caso di essenze scaturite da oggetti (anelli, bastoni, pozioni) la prova di RM deve superare la Difficoltà dell'incantesimo.

\subsubsection{Distrazioni - Problemi nel lancio dell'incantesimo}\index{Distrazioni - Problemi nel lancio dell'incantesimo}

Se l'incantatore viene è severamente distratto, impedito, disturbato, è sotto attacco mentre cerca di lanciare un incantesimo la prova di magia questa deve riuscire di almeno +5 rispetto alla Difficoltà dell'incantesimo, altrimenti la "Distrazione" e' stata tale da impedire il buon esito della magia.
Un incantesimo fallito per distrazione non si conta nel novero degli incantesimi lanciati.

Un mago che lancia incantesimi mentre e' in combattimento di mischia ha un -4 alla Difesa.

\subsubsection{Conservare la magia}\index{Conservare la magia}

Il mago può lanciare l'incantesimo (solitamente 2 Azioni) e trattenerlo nel suo pugno. Per farlo deve riuscire a lanciare l'incantesimo dopo di che lo puo' trattenere fino ad 1 round per punto Competenza Magica posseduto.

Trattenere un incantesimo equivale a rimanere concentrato (costo 1 Azione per round).
Per lanciare l'incantesimo conservato e' sufficiente tirare l'iniziativa ed usare 1 Azione. Non e' possibile lanciare ulteriori incantesimi con difficoltà 10 o piu' finché si conserva un'incantesimo e nel round in cui si manifesta l'incantesimo trattenuto non e' possibile lanciare incantesimi con Difficoltà 10 o piu'.

\subsubsection{Influenzati da più Magie}\index{Influenzati da più Magie}

Quando un personaggio è influenzato da \textbf{due o più effetti magici} che danno lo stesso tipo di bonus, malus o danno nello stesso segmento di iniziativa (protezione verso fuoco, bonus alla Difesa o TS... , multiple palle di acido), si tiene conto solo di quella che ha la Difficoltà maggiore.

Un personaggio che prende 2 Palle di Fuoco nel medesimo segmento di Iniziativa fara' il Tiro Salvezza solo per quella con la difficoltà più alta.

Se prende una Palla di Fuoco in due tempi diversi del medesimo round fara' due Tiri Salvezza distinti prendendo il danno relativo.

\subsubsection{Alterare le Magie}\index{Alterare le Magie}

Il mago può modificare la sua prova di magia tramite diverse possibilità.

\begin{itemize}
	\item
	\textbf{Magia efficace}\index{Magia efficace}: sacrificando Punti Ferita puo' aumentare il successo nel lancio dell'incantesimo. Ottiene un +1 alla prova di Competenza Magica ogni 4 punti ferita sacrificati
	\item
	\textbf{Magia eterea}\index{Magia eterea}: aumentando di 3 la Difficoltà dell'incantesimo le proprie magie hanno pieno effetto su creature eteree o incorporee
	\item
	\textbf{Magia pietosa}: aumentando di 3 la Difficoltà di lancio le magie infliggono danni temporanei. 
	Le magie che infliggono danni di un tipo particolare (come da fuoco) infliggono danni temporanei dello stesso tipo.
	\item
	\textbf{Aumentare il tempo}\index{Aumentare il tempo} di lancio da 2 Azioni a 1 round diminuisce la Difficoltà dell'incantesimo di 1.
	\item
	\textbf{Magie collaborative}\index{Magie collaborative}: un altro mago, sacrificando uno slot incantesimi e impiegando le stesso tempo di lancio di incantesimo del compagno, puo' concedere +4 alla prova di magia del compagno.
	\item
	\textbf{Circolo del Potere}\index{Circolo del Potere}: piu' maghi che siano tutti Devoti o Seguaci dello stesso Patrono possono collaborare affinché uno di loro riesca meglio nel lancio di un incantesimo.
	Ogni mago sacrifica uno slot di incantesimo per concedere +1d6 nella prova di magia ad un collega nel lancio del suo incantesimo (massimo 3 maghi ovvero +3d6). Il tempo di lancio di un incantesimo nel Circolo di Potere diviene almeno 1 turno.
	\item
	\textbf{Modifiche lievi}\index{Modifiche lievi agli incantesimi} alla manifestazione dell'incantesimo possono essere concordati con il Narratore per una Difficoltà aggiuntiva. 
\end{itemize}

\subsubsection{Tentare Incantesimi con impedimenti}\index{Tentare Incantesimi con impedimenti} \index{Impedimenti}

Il lancio di un incantesimo e' vincolato a gesti e parole particolari e unici. Quando il personaggio si trova in una situazione in cui non puo' gesticolare o parlare allora può tentare di lanciare l'incantesimo comunque anche se diventa molto piu' difficile.

La Difficoltà di lancio aumenta di 5 se non puo' gesticolare e di 7 se non puo' parlare. Se non puo' ne parlare ne gesticolare la Difficoltà aumenta di 15 e lanciare un incantesimo richiede 1 round minimo.

\subsubsection{Definizioni obiettivi degli incantesimi}\index{Obiettivi degli incantesimi}

Negli incantesimi sotto elencati troverete spesso i riferimenti alle tipologie di soggetti ed obiettivi influenzabili nonché a diverse tipologie di energia ed elementi. 

\item Le \textbf{Creature} \textbf{Naturali} sono Insetti, Rettili, Bestie, Umanoidi, Piante, Creature acquatiche, Monstrusità, Melme.

\item Le \textbf{Creature} \textbf{Magiche} sono: Immondi, Fatati, Spiriti, Non morti, Giganti, Celestiali, Elementali, Costrutti, Aberrazioni (tutto ciò che e' alieno o innaturale) e Draghi.
Se una Creatura Naturale ha poteri magici allora si considera anche come Creatura Magica. Una descrizione piu' completa di questi "categorie" la trovate nel Capitolo delle Mostruosità.

\item \textbf{Energia} comprende: Fuoco, Luce, Suono, Elettricità, Energia Positiva, Energia Negativa, Freddo, Vuoto.

Il danno causato da \textbf{Luce}\index{Luce} e' per metà da fuoco e per metà da energia positiva, ovvero una resistenza al fuoco od all'energia positiva si applica solo su metà del danno causato dall'attacco.

Il danno causato da \textbf{Vuoto}\index{Vuoto} e' per metà da freddo e per metà da energia negativa, eventuali protezioni si applicano alle rispettive metà del danno.

L'energia negativa danneggia\index{Energia Negativa} i viventi e cura i non morti, l'energia positiva\index{Energia Positiva} danneggia i non morti ma non cura i viventi (a discrezione del Narratore l'esposizione potrebbe equivalere ad un incantesimo di Ristorare Inferiore), vedi anche descrizioni dei Piani.

\end{multicols}

\pagebreak

\begin{multicols}{2}
	
\begin{narratore}
Volete un sistema piu' tradizionale e piu' facile ?

\medskip

Opzione 1) Lasciate tutto come e'. Semplicemente non fate eseguire le prove di magia per verificare il successo nel lancio di incantesimi. L'eventuale esplosione dell'incantesimo la avrete usando 2 slot di incantesimo invece che 1.


Opzione 2) Fate comunque tirare 3d6 non modificati, ma solo per vedere se si genera un critico
\end{narratore}	

\medskip

\begin{tcolorbox}[title = Più effetti speciali!]
Gli incantesimi elencati sono quelli della 5ed piu' alcune mie proposte ed altre rivisitazioni. Se avete suggerimenti per il Narratore per gestire critici non previsti parlatene con lui! Lo spirito di collaborazione deve essere sempre costruttivo.	
\end{tcolorbox}


\subsection{La lista degli incantesimi}

\medskip\textbf{Aiuto}\index{Incantesimi - Aiuto}\\
\textbf{Scuola}: Abiurazione\\
\textbf{Difficoltà}: 19\\
\textbf{Tempo di Lancio}: 2 Azioni\\
\textbf{Gittata}: 9 metri\\
\textbf{Componenti}: V, S, M (una sottile striscia di tessuto bianco)\\
\textbf{Durata}: 8 ore\\
Il tuo incantesimo aumenta la robustezza e risolutezza dei tuoi alleati. Scegli fino a tre creature a gittata. Per la durata, i punti ferita massimi e i punti ferita attuali di ciascun bersaglio aumentano di 5.\\
\textbf{Per ogni Critico ottenuto} nella prova di magia i punti ferita del bersaglio aumentano di ulteriori 5 punti

\medskip\textbf{Allarme}\index{Incantesimi - Allarme}\\
\textbf{Scuola}: Abiurazione\\
\textbf{Difficoltà}: 16\\
\textbf{Tempo di Lancio}: 1 minuto\\
\textbf{Gittata}: 9 metri\\
\textbf{Componenti}: V, S, M (una campanella e un pezzo di pregiato filo d'argento)\\
\textbf{Durata}: 8 ore\\
Predisponi un allarme contro intrusioni indesiderate. Scegli una porta, una finestra o un'area a gittata che non sia più grande di un cubo di 6 metri di spigolo. Fino al termine dell'incantesimo, sarai avvertito da un allarme ogni volta che una creatura di taglia Minuscola o superiore entri in contatto o acceda all'area protetta. Quando lanci l'incantesimo, puoi indicare delle creature che non faranno scattare l'allarme Scegli anche se l'allarme è udibile o solo mentale. Un allarme mentale, qualora ti trovi entro 1,5 chilometri dall'area protetta, ti avverte con un rumore nella tua mente. Il rumore è in grado di svegliarti se stai dormendo. Un allarme udibile produce il suono di una campanella per 10 secondi, udibile entro 18 metri.

\medskip\textbf{Allucinazione Mortale}\index{Incantesimi - Allucinazione Mortale}\\
\textbf{Scuola}: Illusione\\
\textbf{Difficoltà}: 23\\
\textbf{Tempo di Lancio}: 2 Azioni\\
\textbf{Gittata}: 36 metri\\
\textbf{Componenti}: V, S\\
\textbf{Durata}: Istantanea\\
Attingi agli incubi di una creatura a gittata e che puoi vedere, e crei una manifestazione illusoria delle sue più insite paure, visibile solo per quella creatura. Il bersaglio deve effettuare un Tiro Salvezza su Volontà.\\
Se fallisce il Tiro Salvezza, il bersaglio è spaventato per 1 minuto e subisce 4d10 di danno psichico. \\
\textbf{Per ogni Critico ottenuto} nella prova di magia il danno aumenta di 1d10

\medskip\textbf{Alterare Sé Stesso}\index{Incantesimi - Alterare Sé Stesso}\\
\textbf{Scuola}: Trasmutazione\\
\textbf{Difficoltà}: 19\\
\textbf{Tempo di Lancio}: 2 Azioni\\
\textbf{Gittata}: Personale\\
\textbf{Componenti}: V, S\\
\textbf{Durata}: 1 ora\\
Assumi una forma diversa. Quando lanci questo incantesimo, scegli una della seguenti opzioni, l'effetto della quale permane per la durata dell'incantesimo. Per la durata dell'incantesimo puoi terminare un'opzione per ottenere i benefici di un'altra.\\
Adattamento Acquatico. Adatti il tuo corpo a un ambiente acquatico, sviluppando branchie e dita palmate. Puoi respirare sott'acqua e ottieni velocità di nuoto pari alla tua velocità di passeggio.\\
\textit{Armi Naturali}. Sviluppi degli artigli, zanne, spuntoni, corna o una diversa arma naturale a tua scelta. I tuoi colpi senz'armi infliggono 1d6 danni da botta, perforanti o taglienti, come appropriato all'arma naturale scelta, con la quale sei competente. Infine, l'arma naturale è magica e ricevi un bonus di +1 ai Tiri per Colpire e danno effettuati quando la usi.\\
\textit{Cambio di Aspetto}. Trasformi il tuo aspetto. Decidi il tuo aspetto esteriore, compresa l'altezza, il peso, i lineamenti facciali, il suono della tua voce, la lunghezza dei capelli, il colorito e qualsiasi peculiarità tu desideri. Puoi apparire come membro di un'altra razza, sebbene nessuna delle tue statistiche cambi. Inoltre non puoi apparire come una creatura di taglia diversa dalla tua, e la tua forma base resta la medesima; se sei bipede, non puoi usare questo incantesimo per diventare quadrupede, per esempio.\\
In qualsiasi momento della durata dell'incantesimo, puoi usare due Azioni per cambiare nuovamente di aspetto in questo modo.\\

\medskip\textbf{Amicizia con gli Animali}\index{Incantesimi - Amicizia con gli Animali}\\
\textbf{Scuola}: Ammaliamento\\
\textbf{Difficoltà}: 16\\
\textbf{Tempo di Lancio}: 2 Azioni\\
\textbf{Gittata}: 9 metri\\
\textbf{Componenti}: V, S, M (un po' di cibo)\\
\textbf{Durata}: 24 ore\\
Questo incantesimo ti permette di convincere una bestia naturale che non vuoi arrecargli danno. Scegli una bestia a gittata che puoi vedere. Questa deve vederti e udirti. Se l'Intelligenza della bestia è 4 o più, l'incantesimo fallisce. Altrimenti, la bestia deve superare un Tiro Salvezza su Volontà o restare affascinata da te per la durata dell'incantesimo. Se tu o uno dei tuoi compagni danneggiate il bersaglio, l'incantesimo ha termine.\\
\textbf{Per ogni critico ottenuto} nella prova di magia puoi agire su di una bestia aggiuntiva. 

\medskip\textbf{Anatema}\index{Incantesimi - Anatema}\\
\textbf{Scuola}: Ammaliamento\\
\textbf{Difficoltà}: 16\\
\textbf{Tempo di Lancio}: 1 minuto\\
\textbf{Gittata}: 9 metri\\
\textbf{Componenti}: V, S, M (un goccio di sangue)\\
\textbf{Durata}: 1 minuto\\
Fino a tre creature di tua scelta che puoi vedere, e che sono a gittata, devono effettuare un Tiro Salvezza su Volontà. Ogni bersaglio che fallisca questo Tiro Salvezza ed effettua un tiro per colpire o un Tiro Salvezza prima del termine dell'incantesimo, deve tirare un d4 e sottrarre il numero così ottenuto dal tiro per colpire o Tiro Salvezza.\\
\textbf{Per ogni Critico ottenuto} nella prova di magia puoi prendere come bersaglio una creatura aggiuntiva.

\medskip\textbf{Animale Messaggero}\index{Incantesimi - Animale Messaggero}\\
\textbf{Scuola}: Ammaliamento\\
\textbf{Difficoltà}: 19\\
\textbf{Tempo di Lancio}: 2 Azioni\\
\textbf{Gittata}: 9 metri\\
\textbf{Componenti}: V, S, M (un poco di cibo)\\
\textbf{Durata}: 24 ore\\
Tramite questo incantesimo, usi un animale per consegnare un messaggio. Scegli una bestia Minuscola a gittata e che puoi vedere, come uno scoiattolo, una ghiandaia o un pipistrello. Specifichi un luogo, che devi aver visitato in passato, e un destinatario che corrisponda a una descrizione generica, come "un uomo o una donna che vesta l'uniforme della guardia cittadina" o "un nano dai capelli rossi che indossa una fedora". Pronuncia anche un messaggio di massimo venticinque parole. La bestia bersaglio viaggia per la durata dell'incantesimo verso il luogo specificato, coprendo circa 75 chilometri in 24 ore per un messaggero volante, o 40 chilometri per gli altri animali. Quando il messaggero arriva a destinazione, consegna il messaggio alla creatura da te descritta, replicando il suono della tua voce. Il messaggero parla solo a una creatura corrispondente alla descrizione da te fornita. Se il messaggero non riesce a raggiungere la destinazione prima del termine dell'incantesimo, il messaggio è perduto, e la bestia ritorna verso il punto in cui hai lanciato l'incantesimo.\\
\textbf{Per ogni Critico ottenuto} nella prova di magia la durata dell'incantesimo aumenta di 8 ore

\medskip\textbf{Animare Morti}\index{Incantesimi - Animare Morti}\\
\textbf{Scuola}: Necromanzia\\
\textbf{Difficoltà}: 21\\
\textbf{Tempo di Lancio}: 1 minuto\\
\textbf{Gittata}: 3 metri\\
\textbf{Componenti}: V, S, M (una goccia di sangue, un pezzo di carne e un pizzico di polvere d'ossa)\\
\textbf{Durata}: Istantanea\\
Questo incantesimo crea un servitore non morto. Scegli una pila di ossa o un cadavere di un umanoide Medio o Piccolo a gittata. Il tuo incantesimo imbeve il bersaglio di una nefanda parvenza di vita, rianimandolo come creatura non morta. Il bersaglio diventa uno scheletro se scegli le ossa o uno zombi se scegli un cadavere. Durante ciascun tuo round, puoi usare un'Azione per comandare mentalmente qualsiasi creatura da te creata con questo incantesimo che si trovi entro 18 metri da te (se controlli più creature, puoi comandarle tutte o solo alcune di loro allo stesso momento, inviando lo stesso comando a tutte). Decidi quale azione la creatura svolgerà e dove si muoverà durante il suo prossimo round, oppure inviale un comando generale, come quello di stare di guardia a una particolare stanza o corridoio. Se non invii alcun comando, la creatura si limita a difendersi dalle creature ostili. Una volta ricevuto un ordine, la creatura continuerà a svolgerlo fino al suo compimento. La creatura è sotto il tuo controllo per 24 ore, dopodiché smetterà di eseguire i comandi che le impartirai. Per mantenere il controllo sulla creatura per altre 24 ore, devi lanciare di nuovo questo incantesimo su di essa prima del termine dell'attuale periodo di 24 ore. Questo impiego dell'incantesimo riafferma il tuo controllo su di un massimo di quattro creature che hai animato con questo incantesimo, piuttosto che animarne una nuova.\\
\textbf{Per ogni Critico ottenuto} nella prova di magia animi o riaffermi il controllo su due creature non morte. Ciascuna di queste creature deve provenire da un cadavere o pila di ossa differenti.

\medskip\textbf{Animare Oggetti}\index{Incantesimi - Animare Oggetti}\\
\textbf{Scuola}: Trasmutazione\\
\textbf{Difficoltà}: 26\\
\textbf{Tempo di Lancio}: 1 minuto\\
\textbf{Gittata}: 36 metri\\
\textbf{Componenti}: V, S\\
\textbf{Durata}: Concentrazione, massimo 1 minuto\\
Gli oggetti prendono vita al tuo comando. Scegli fino a dieci oggetti non magici a gittata e che non siano indossati o trasportati. I bersagli Medi contano come due oggetti, i bersagli Grandi contano come quattro oggetti, i bersagli Enormi contano come otto oggetti. Non puoi animare oggetti di taglia più grossa di Enorme. Ogni bersaglio si anima e diventa una creatura sotto il tuo controllo fino al termine dell'incantesimo o finché non viene ridotto a 0 punti ferita.\\
Con un'Azione puoi comandare mentalmente qualsiasi creatura che hai generato con questo incantesimo e che si trovi entro 150 metri da te (se controlli più creature, puoi comandarne solo alcune o tutte allo stesso tempo, impartendo lo stesso comando a ciascuna). Decidi tu quale azione intraprenderà la creatura e dove si muoverà durante il suo round successivo, o puoi emettere un comando generico,come quello di fare la guardia a una particolare stanza o corridoio. Se non impartisci comandi, la creatura si limiterà a difendersi dalle creature ostili. Una volta dato un ordine, la creatura continuerà a seguirlo finché non avrà completato il suo compito.
\bigskip

\end{multicols}

\textbf{Statistiche degli Oggetti Animati}
\bigskip

\begin{tabular}{llllll}
Taglia		&Punti Ferita	&Difesa	&CA, danni					&Forza	&Destrezza\\ 
\toprule
Minuscola 	&20 			&18		&8, {1d4+4} 	&-3 		&4\\
Piccola 	&25 			&16 	&6, {1d8+2} 	&-2 		&2\\
Media 		&40 			&13 	&5, {2d6+1} 	&0 			&1\\
Grande 		&50 			&10 	&6, {2d10+2}	&2 			&0\\
Enorme 		&80 			&10 	&8, {2d12+4}	&4 			&-2\\
\end{tabular}

\bigskip

\begin{multicols}{2}

Un oggetto animato è un costrutto con Difesa, punti ferita, attacchi, Forza e Destrezza in base alla sua taglia. Il suo punteggio di Intelligenza e Saggezza è -3, mentre Carisma e' -4.\\
Ha movimento 9 metri; se l'oggetto è privo di gambe o altre appendici che può usare per muoversi ha movimento 0, ha invece movimento volare 9 metri e può fluttuare. 
\\Se l'oggetto è ancorato a una superficie o un oggetto più grosso, come una catena attaccata al muro, la sua velocità è 0.\\
Possiede vista cieca con un raggio di 9 metri ed è cieco oltre questa distanza.\\
Quando l'oggetto animato scende a 0 punti ferita, ritorna alla sua normale forma di oggetto, e tutti i danni in eccesso vengono inflitti alla sua forma originale.\\
Se comandi a un oggetto di attaccare, questo può effettuare un singolo attacco da mischia contro una creatura entro 1 metri da esso. Effettua un attacco on CA e danni determinati dalla taglia (vedi tabella). Il Narratore potrebbe determinare che a seconda della sua forma, un oggetto potrebbe invece infliggere danni taglienti o perforanti.\\
\textbf{Per ogni Critico ottenuto} nella prova di magia puoi animare due oggetti aggiuntivi.

\medskip\textbf{Anti-Individuazione}\index{Incantesimi - Anti-Individuazione}\\
\textbf{Scuola}: Abiurazione\\
\textbf{Difficoltà}: 21\\
\textbf{Tempo di Lancio}: 2 Azioni\\
\textbf{Gittata}: Contatto\\
\textbf{Componenti}: V, S, M (un pizzico di polvere di diamante del valore di 25 mo sparsa sul bersaglio, che l'incantesimo consuma)\\
\textbf{Durata}: 8 ore\\
Per la durata, nascondi il bersaglio con cui sei stato in contatto dalla magia di divinazione. Il bersaglio può essere una creatura consenziente o un luogo o un oggetto che occupi uno spazio equivalente a un cubo non superiore ai 3 metri di spigolo. Il bersaglio non può divenire bersaglio di alcuna magia di divinazione o essere percepito tramite sensi di scrutamento magici.

\medskip\textbf{Antipatia/Simpatia}\index{Incantesimi - Antipatia/Simpatia}\\
\textbf{Scuola}: Ammaliamento\\
\textbf{Difficoltà}: 34\\
\textbf{Tempo di Lancio}: 1 ora\\
\textbf{Gittata}: 18 metri\\
\textbf{Componenti}: V, S, M (o un pezzo di allume immerso nell'aceto per l'effetto antipatia o un goccio di miele per l'effetto simpatia)\\
\textbf{Durata}: 10 giorni\\
Questo incantesimo attrae o repelle delle creature di tua scelta. Prendi un bersaglio a gittata, che sia un oggetto Enorme o più piccolo o una creatura o un'area non più grande di un cubo di 60 metri di spigolo. Poi specifica una specie di creature intelligenti, come i draghi rossi, i goblin o i vampiri. Investi il bersaglio di un'aura che attrae o respinge le creature specificate per la durata. Scegli antipatia o simpatia come effetto dell'aura.\\
Antipatia. L'ammaliamento fa sì che le creature del tipo da te indicato provino un forte impulso a lasciare l'area ed evitare il bersaglio. Quando una creatura del genere può vedere il bersaglio o si avvicina entro 18 metri da esso, la creatura deve superare un Tiro Salvezza su Volontà o diventare spaventata. La creatura rimane spaventata finché può vedere il bersaglio o resta entro 18 metri da esso. Mentre è spaventata dal bersaglio, la creatura deve impiegare il suo movimento per muoversi verso il posto sicuro più vicino dal quale non possa più vedere il bersaglio. Se la creatura si muove più di 18 metri lontano dal bersaglio e non può vederlo, la creatura non è più spaventata, ma torna a essere spaventata se torna a vedere il bersaglio o si muove entro 18 metri da esso.\\
Simpatia. L'ammaliamento fa sì che le creature specificate provino un forte impulso ad avvicinarsi al bersaglio se si trovano entro 18 metri da esso o possono vederlo. Quando una simile creatura può vedere il bersaglio o si avvicina entro 18 metri da esso, la creatura deve superare un Tiro Salvezza su Volontà o usare il suo movimento durante ciascun round per entrare nell'area, o muoversi a portata del bersaglio. Quando la creatura l'avrà fatto, non potrà più volontariamente muoversi lontano dal bersaglio. Se il bersaglio danneggia o altrimenti nuoce alla creatura soggetta, questa può effettuare un Tiri Salvezza su Volontà per terminare l'effetto, come descritto di seguito.\\
Terminare l'Effetto. Se una creatura soggetta termina il suo round mentre si trova più lontana di 18 metri dal bersaglio o non può vederlo, la creatura effettua un Tiro Salvezza su Volontà. Se supera il Tiro Salvezza, la creatura non è più soggetta al bersaglio e riconosce la sensazione di ripugnanza o attrazione come magica. Inoltre, una creatura soggetta all'incantesimo, ha diritto a un altro Tiro Salvezza su Volontà ogni 24 ore di durata dell'incantesimo. Una creatura che supera il Tiro Salvezza contro questo effetto è immune a esso per 1 minuto, dopodiché può subirlo nuovamente.

\medskip\textbf{Arma Magica}\index{Incantesimi - Arma Magica}\\
\textbf{Scuola}: Trasmutazione\\
\textbf{Difficoltà}: 19\\
\textbf{Tempo di Lancio}: 1 Azione Immediata\\
\textbf{Gittata}: Contatto\\
\textbf{Componenti}: V, S\\
\textbf{Durata}: 10 minuti\\
Lanci l'incantesimo a contatto di un'arma non magica. Fino al termine dell'incantesimo, l'arma diventa un'arma magica con un bonus di +1 ai Tiri per Colpire e di danno.\\
\textbf{Per ogni Critico ottenuto} nella prova di magia puoi il bonus aumenta a +1.

\medskip\textbf{Arma Spirituale}\index{Incantesimi - Arma Spirituale}\\
\textbf{Scuola}: Invocazione\\
\textbf{Difficoltà}: 19\\
\textbf{Tempo di Lancio}: 2 Azioni\\
\textbf{Gittata}: 18 metri\\
\textbf{Componenti}: V, S\\
\textbf{Durata}: 1 minuto\\
In un punto nella gittata, crei un'arma spettrale fluttuante, che resta per la durata o finché non lanci di nuovo questo incantesimo. Quando lanci l'incantesimo, puoi effettuare un attacco da incantesimo in mischia contro una creatura entro 1 metro dall'arma. Se colpisci, il bersaglio subisce danni da forza pari a 1d8 + il tuo valore di caratteristica da incantatore. Durante il tuo round, con due Azioni, puoi spostare l'arma di 6 metri e ripetere l'attacco contro una creatura entro 1 metro dall'arma. L'arma può assumere qualsiasi forma tu voglia, magari affine al Patrono.\\
\textbf{Per ogni Critico ottenuto} nella prova di magia il danno aumenta di 2.

\medskip\textbf{Armatura Magica}\index{Incantesimi - Arma Magica}\\
\textbf{Scuola}: Abiurazione\\
\textbf{Difficoltà}: 16\\
\textbf{Tempo di Lancio}: 2 Azioni\\
\textbf{Gittata}: Contatto\\
\textbf{Componenti}: V, S, M (un pezzo di cuoio lavorato)\\
\textbf{Durata}: 8 ore
Lanci l'incantesimo a contatto di una creatura consenziente che non indossa un'armatura. Una forza magica protettiva circonda il bersaglio fino al termine dell'incantesimo. La Difesa del bersaglio diventa 13 + Destrezza. L'incantesimo termina se il bersaglio indossa un'armatura o interrompe l'incantesimo con un'azione.


\medskip\textbf{Artificio Druidico}\index{Trucchetto - Artificio Druidico}\\
\textbf{Scuola}: Universale\\
\textbf{Difficoltà}: 12\\
\textbf{Tempo di Lancio}: 2 Azioni\\
\textbf{Gittata}: 9 metri\\
\textbf{Componenti}: V, S\\
\textbf{Durata}: Istantanea\\
Sussurrando agli spiriti della natura, crei, a gittata, uno dei seguenti effetti:
\begin{itemize}
\item
Crei un minuscolo e innocuo effetto sensoriale che predice quale clima ci sarà nel luogo in cui ti trovi per le prossime 24 ore. L'effetto potrebbe manifestarsi come una sfera dorata per i cieli limpidi, una nube per la pioggia, fiocchi di neve per la neve, e così via. L'effetto persiste per 1 round.
\item 
Fai immediatamente sbocciare un fiore, un seme o simile pianta.
\item 
Crei un istantaneo e innocuo effetto sensoriale, come foglie che cadono, uno sbuffo di vento, il suono di un piccolo animale, o il lieve tanfo di una puzzola. L'effetto deve entrare in un cubo di 1 metro.
\item Accendi o spegni istantaneamente una candela, una torcia o un piccolo falò.
\end{itemize}

\medskip\textbf{Aura Magica dell'Arcanista}\index{Incantesimi - Aura Magica dell'Arcanista}\\
\textbf{Scuola}: Illusione\\
\textbf{Difficoltà}: 19\\
\textbf{Tempo di Lancio}: 2 Azioni\\
\textbf{Gittata}: Contatto\\
\textbf{Componenti}: V, S, M (un piccolo quadretto di seta)\\
\textbf{Durata}: 24 ore\\
Poni un'illusione su di una creatura od oggetto con cui sei in contatto, così che gli incantesimi di divinazione rivelino false informazioni su di esso. Il bersaglio può essere una creatura consenziente o un oggetto che non sia trasportato o indossato da un'altra creatura. Quando lanci questo incantesimo, scegli uno o entrambi i seguenti effetti. L'effetto permane per la durata. Se esegui questo incantesimo sulla stessa creatura od oggetto ogni giorno per 30 giorni, piazzando ogni volta lo stesso effetto, l'illusione permarrà finché non viene dissolta.\\
\textit{Aura Falsa}. Cambi il modo in cui il bersaglio risulta a incantesimi ed effetti magici, come individuazione del magico, che individuano le aure magiche. Puoi far apparire magico un oggetto normale, non magico un oggetto magico, o cambiare l'aura magica dell'oggetto così che sembri appartenere a una scuola di magia di tua scelta. Quando impieghi questo effetto su di un oggetto, puoi far sì che la falsa magia sia apparente a qualsiasi creatura che lo manipoli.\\ \textit{Mascherare}. Cambi il modo in cui il bersaglio risulta a incantesimi ed effetti magici che individuano il tipo di creatura o Tratti, come l'attivazione dell'incantesimo simbolo. Scegli un tipo di creatura o Tratto, e gli altri incantesimi ed effetti magici considereranno il bersaglio come fosse una creatura di quel tipo o di quel Tratto, e non più di quello originale.

\medskip\textbf{Aura Sacra}\index{Incantesimi - Aura Sacra}\\
\textbf{Scuola}: Abiurazione\\
\textbf{Difficoltà}: 34\\
\textbf{Tempo di Lancio}: 2 Azioni\\
\textbf{Gittata}: Personale\\
\textbf{Componenti}: V, S, M (un minuscolo reliquario del valore di almeno 1.000 mo contenente una reliquia sacra, come un pezzo di tessuto dell'abito di un santo o un frammento di pergamena di un testo religioso)\\
\textbf{Durata}: Concentrazione, 1 minuto\\
Irradi da te luce divina che si raccoglie in una debole luminosità con raggio di 9 metri intorno a te. Quando lanci l'incantesimo, le creature da te scelte in questo raggio emanano luce fioca con un raggio di 1 metro e hanno {+2d6} a tutti i Tiri Salvezza, mentre le altre creature hanno {-2d6} sui Tiri per Colpire contro di loro fino al termine dell'incantesimo. Inoltre, quando un demone o non morto colpisce una creatura bersaglio con un attacco in mischia, l'aura risplende di una luce intensa e deve superare un Tiro Salvezza su Tempra o restare accecato fino al termine dell'incantesimo.

\medskip\textbf{Bacche Benefiche}\index{Incantesimi - Bacche Benefiche}\\
\textbf{Scuola}: Trasmutazione\\
\textbf{Difficoltà}: 19\\
\textbf{Tempo di Lancio}: 2 Azioni\\
\textbf{Gittata}: Contatto\\
\textbf{Componenti}: V, S, M (un rametto di vischio)\\
\textbf{Durata}: Istantanea\\
Incanti fino a 2d4 bacche nella tua mano che vengono infuse di magia per la durata. Una creatura può usare 1 Azione Immediata per mangiare una bacca. Mangiare una bacca ripristina 1 punto ferita, e la bacca inoltre provvede nutrimento sufficiente per alimentare una creatura per un giorno. Solo la prima bacca e' efficace nel giorno.\\
Le bacche perdono la loro efficacia se non vengono consumate entro 72 ore dal lancio dell'incantesimo. \\
\textbf{Per ogni Critico ottenuto} nella prova di magia le bacche durano un giorno in più oppure incanti una bacca in più (fino ad un massimo totale di 8).

\medskip\textbf{Bagliore Solare}\index{Incantesimi - Bagliore Solare}\\
\textbf{Scuola}: Invocazione\\
\textbf{Difficoltà}: 29\\
\textbf{Tempo di Lancio}: 2 Azioni\\
\textbf{Gittata}: Personale (linea di 18 metri)\\
\textbf{Componenti}: V, S, M (una lente di ingrandimento)\\
\textbf{Durata}: Concentrazione, massimo 1 minuto\\
Una fascio di luce brillante esplode dalla tua mano in una linea larga 1 metri e lunga 18 metri. Ogni creatura sulla linea deve effettuare un Tiro Salvezza su Tempra. Se fallisce il Tiro Salvezza, la creatura subisce 6d8 danni da Luce e rimane accecata fino al tuo prossimo round. Se supera il Tiro Salvezza, subisce la metà dei danni e non è accecata. I non morti e le melme hanno -1d6 su questo Tiro Salvezza. Puoi creare una nuova linea di luminosità con un'azione durante qualsiasi tuo round fino al termine dell'incantesimo.\\
Per la durata, una particella di luce brillante risplende nella tua mano. Produce luce in un raggio di 9 metri e penombra per ulteriori 9 metri. Questa luce è considerata luce solare.

\medskip\textbf{Banchetto degli Eroi}\index{Incantesimi - Banchetto degli Eroi}\\
\textbf{Scuola}: Evocazione\\
\textbf{Difficoltà}: 29\\
\textbf{Tempo di Lancio}: 10 minuti\\
\textbf{Gittata}: 9 metri\\
\textbf{Componenti}: V, S, M (una ciotola incrostata di gemme del valore di almeno 1.000 mo, che l'incantesimo consuma)\\
\textbf{Durata}: Istantanea\\
Crei un magnifico banchetto, comprensivo di cibi e bevande prelibate. Il banchetto viene consumato in 1 ora e scompare al termine di questo periodo, ma gli effetti benefici non si faranno sentire fino al termine dell'ora. Fino ad altre dodici creature possono
partecipare al banchetto. Una creatura che partecipi al banchetto ottiene diversi benefici. La creatura viene guarita da tutte le malattie e i veleni, diventa immune al veleno e all'essere
spaventata, e ha +2d6 su tutti i Tiri Salvezza su Volontà. I suoi punti ferita massimi aumentano di 2d10, e guarisce lo stesso quantitativo di punti ferita attuali. Questi benefici durano 24 ore. 

\medskip\textbf{Barriera di Lame}\index{Incantesimi - Barriera di Lame}\\
\textbf{Scuola}: Invocazione\\
\textbf{Difficoltà}: 29\\
\textbf{Tempo di Lancio}: 2 Azioni\\
\textbf{Gittata}: 18 metri\\
\textbf{Componenti}: V, S\\
\textbf{Durata}: 10 minuti \\
Crei un muro verticale di lame rotanti fatte di energia magica, affilate come rasoi. Il muro compare a gittata e resta per la durata. Puoi creare un muro diritto lungo fino a 30 metri, alto 6 metri e spesso 1 metro, o un muro circolare di 18 metri massimo di diametro, alto 6 metri e spesso 1 metro. Il muro fornisce tre quarti di copertura alle creature dietro di esso, e il suo spazio è terreno difficile. \\
Quando una creatura entra per la prima volta in un round nell'area del muro o comincia il suo round lì, la creatura deve effettuare un Tiro Salvezza su Riflessi. Se la creatura fallisce il Tiro Salvezza subisce 6d10 danni taglienti, o la metà se lo supera.\\
Un incantatore che è ad una distanza di un metro dalla Barriera di Lame si considera distratto.

\medskip\textbf{Beffa Crudele}\index{Trucchetto - Beffa Crudele}\\
\textbf{Scuola}: Ammaliamento\\
\textbf{Difficoltà}: 12\\
\textbf{Tempo di Lancio}: 2 Azioni\\
\textbf{Gittata}: 18 metri\\
\textbf{Componenti}: V\\
\textbf{Durata}: Istantanea\\
Scateni una serie di insulti avvolti da una subdola malia contro una creatura a gittata e che puoi vedere. Se il bersaglio ti può udire (sebbene non è necessario che ti comprenda), deve superare un Tiro Salvezza su Volontà o subire 1d4 danni e avere -1d6 al prossimo tiro per colpire che effettuerà prima del termine del suo prossimo round.\\
Il danno dell'incantesimo aumenta di 1d4 quando raggiungi CM 5, CM 11 e CM 17.

\medskip\textbf{Benedici Acqua}\index{Incantesimi - Benedici Acqua}\\
\textbf{Scuola}: Invocazione\\
\textbf{Difficoltà}: 19\\
\textbf{Tempo di Lancio}: 10 Minuti\\
\textbf{Gittata}: Tocco\\
\textbf{Componenti}: V, S, M (25 monete d'oro in offerta alla chiesa)\\
\textbf{Durata}: Istantanea\\
Benedici fino ad un litro di liquido, sufficiente a creare 5 boccette di acqua benedetta.\\
\textbf{Per ogni Critico ottenuto} nella prova benedici un litro di liquido in piu'.\\

\medskip\textbf{Benedizione}\index{Incantesimi - Benedizione}\\
\textbf{Scuola}: Invocazione\\
\textbf{Difficoltà}: 16\\
\textbf{Tempo di Lancio}: 2 Azioni\\
\textbf{Gittata}: 9 metri\\
\textbf{Componenti}: V, S, M (uno spruzzo di acqua benedetta)\\
\textbf{Durata}: 1 minuto\\
Benedici fino a tre creature a gittata, scelte da te. I bersagli guadagnano +1 ai Tiri Salvezza e Tiro per Colpire.\\
Piu' benedizioni, anche da Patroni diversi non si sommano.\\
\textbf{Per ogni Critico ottenuto} nella prova puoi aggiungere una creatura come bersaglio.

\medskip\textbf{Benedizione Superiore}\index{Incantesimi - Benedizione Superiore}\\
\textbf{Scuola}: Invocazione\\
\textbf{Difficoltà}: 19\\
\textbf{Tempo di Lancio}: 1 Minuto\\
\textbf{Gittata}: 18 metri\\
\textbf{Componenti}: V, S, M (uno spruzzo di acqua benedetta, 10 monete d'oro)\\
\textbf{Durata}: 1 ora\\
Benedici una creatura a tua scelta. La creatura entro la durata puo' aggiungere 1d6 ad un tiro prima di sapere se la prova (TC/TS/Check..) ha avuto successo o meno. Questo bonus può essere usato 2 volte nell'ora.\\
\textbf{Per ogni Critico ottenuto} nella prova puoi aggiungere una creatura come bersaglio o aggiungere un ora alla durata.\\

\medskip\textbf{Benedizione Suprema}\index{Incantesimi - Benedizione Suprema}\\
\textbf{Scuola}: Invocazione\\
\textbf{Difficoltà}: 21\\
\textbf{Tempo di Lancio}: 1 Reazione\\
\textbf{Gittata}: 27 metri\\
\textbf{Componenti}: V, S, M (uno spruzzo di acqua benedetta, 25 monete d'oro)\\
\textbf{Durata}: Istantanea\\
Benedici una creatura a tua scelta. La creatura puo' ritirare due dadi di una singola prova prima di sapere se la prova ha avuto successo o meno. La creatura sceglie se prendere i nuovi tiri ottenuti o tenere i vecchi.\\
\textbf{Per ogni Critico ottenuto} nella prova la creatura prende un +1 di bonus alla prova.\\

\medskip\textbf{Blocca Mostri}\index{Incantesimi - Blocca Mostri}\\
\textbf{Scuola}: Ammaliamento\\
\textbf{Difficoltà}: 26\\
\textbf{Tempo di Lancio}: 2 Azioni\\
\textbf{Gittata}: 27 metri\\
\textbf{Componenti}: V, S, M (un piccolo pezzo dritto di ferro)\\
\textbf{Durata}: 1 minuto\\
Scegli una creatura a gittata e che puoi vedere. Il bersaglio deve superare un Tiro Salvezza su Volontà, o restare paralizzato per la durata. Questo incantesimo non ha effetto su non morti o costrutti. Al termine di ciascun suo round, il bersaglio può effettuare un altro Tiro Salvezza su Volontà. Se lo supera, per quel bersaglio l'incantesimo ha termine.\\
\textbf{Per ogni Critico ottenuto} nella prova di magia puoi aggiungere una creatura come bersaglio purché siano entro 9 metri l'una dall'altra.

\medskip\textbf{Blocca Persona}\index{Incantesimi - Blocca Persona}\\
\textbf{Scuola}: Ammaliamento\\
\textbf{Difficoltà}: 19\\
\textbf{Tempo di Lancio}: 2 Azioni\\
\textbf{Gittata}: 18 metri\\
\textbf{Componenti}: V, S, M (un piccolo pezzo dritto di ferro)\\
\textbf{Durata}: 1 minuto\\
Scegli un umanoide a gittata e che puoi vedere. L'incantesimo non ha effetto su creature con CR 4 o piu'. Il bersaglio deve superare un Tiro Salvezza su Volontà o restare paralizzato per la durata.\\
\textbf{Per ogni Critico ottenuto} nella prova di magia puoi aggiungere una creatura come bersaglio purché siano entro 9 metri l'una dall'altra.

\medskip\textbf{Blocca Persona Avanzato}\index{Incantesimi - Blocca Persona Avanzato}\\
\textbf{Scuola}: Ammaliamento\\
\textbf{Difficoltà}: 23\\
\textbf{Tempo di Lancio}: 2 Azioni\\
\textbf{Gittata}: 18 metri, raggio 6 metri\\
\textbf{Componenti}: V, S, M (un piccolo pezzo dritto di argento)\\
\textbf{Durata}: 1 minuto\\
Blocchi fino a 2d4 CR (o livelli) di creature entro 18 metri da te in un raggio di 6 metri. Si incomincia bloccando le creature dal CR piu' basso e sottraendo ai 2d4 tirati il CR, procedi finche' non hai piu' punti per bloccare le creature. I bersagli deve superare un Tiro Salvezza su Volontà o restare paralizzati per la durata.\\
\textbf{Per ogni Critico ottenuto} nella prova di magia puoi aggiungere 2 punti ai 2d4 tirati.

\medskip\textbf{Bocca Magica}\index{Incantesimi - Bocca Magica}\\
\textbf{Scuola}: Illusione\\
\textbf{Difficoltà}: 19\\
\textbf{Tempo di Lancio}: 1 minuto\\
\textbf{Gittata}: 9 metri\\
\textbf{Componenti}: V, S, M (un piccolo pezzo di favo e polvere di giada del valore di almeno 10 mo, che l'incantesimo consuma)\\
\textbf{Durata}: Fino a che dissolto\\
Impianti un messaggio in un oggetto a gittata, messaggio che viene pronunciato quando si soddisfa la condizione di attivazione. Scegli un oggetto che puoi vedere e che non sia indossato o trasportato da un'altra creatura. Poi pronuncia il messaggio, che deve essere di 25 parole o meno, ma può essere distribuito in un periodo di massimo 10 minuti. Infine, determina la circostanza che attiverà l'incantesimo, affinché questo trasmetta il tuo messaggio.\\
Quando la circostanza si manifesta, una bocca magica appare sull'oggetto e recita il messaggio con la tua voce e allo stesso volume con cui l'hai pronunciato. Se l'oggetto da te scelto ha una bocca o qualcosa che assomiglia a una bocca (per esempio, la bocca di una statua), la bocca magica appare così che le parole sembrino provenire dalla bocca dell'oggetto. Quando lanci questo incantesimo, puoi far sì che l'incantesimo termini dopo aver trasmesso il suo messaggio, o che perduri e ripeta il messaggio ogni volta che la condizione si attiva.\\
La circostanza di attivazione può essere generica o dettagliata quanto desideri, ma deve essere basata su condizioni visibili o udibili che avvengono entro 9 metri dall'oggetto. Per esempio, potresti istruire la bocca di parlare quando una qualsiasi creatura si avvicina entro 9 metri dall'oggetto o quando una campanella d'argento suona entro 9 metri da esso.

\medskip\textbf{Caduta Morbida}\index{Incantesimi - Caduta Morbida}\\
\textbf{Scuola}: Trasmutazione\\
\textbf{Difficoltà}: 16\\
\textbf{Tempo di Lancio}: 1 reazione, che effettui quando tu o una creatura entro 18 metri da te cadete\\
\textbf{Gittata}: 18 metri\\
\textbf{Componenti}: V, M (una piccola piuma o un pezzo di piuma)\\
\textbf{Durata}: 1 minuto\\
Scegli fino a cinque creature a gittata. La velocità di discesa di una creatura che cade diminuisce a 18 metri per round fino al termine dell'incantesimo. Se la creatura atterra prima del termine dell'incantesimo, non subisce danni da caduta e può atterrare sui suoi piedi; per quella creatura l'incantesimo ha termine.\\

\medskip\textbf{Calmare Emozioni}\index{Incantesimi - Calmare Emozioni}\\
\textbf{Scuola}: Ammaliamento\\
\textbf{Difficoltà}: 19\\
\textbf{Tempo di Lancio}: 2 Azioni\\
\textbf{Gittata}: 18 metri\\
\textbf{Componenti}: V, S\\
\textbf{Durata}: Concentrazione, massimo 1 minuto\\
Tenti di sopprimere le forti emozioni in un gruppo di persone. Ogni umanoide in una sfera di 6 metri di raggio centrata su di un punto a gittata da te scelto, deve effettuare un Tiro Salvezza su Volontà; se lo desidera, una creatura può scegliere di fallire questo Tiro Salvezza. Se una creatura fallisce il Tiro Salvezza, scegli uno di questi due effetti. \\
\textit{Placare}. Puoi sopprimere qualsiasi effetto che renda il bersaglio affascinato o spaventato. Quando questo incantesimo termina, gli effetti soppressi riprendono, purché la loro durata non sia nel frattempo esaurita.\\
\textit{Indifferenza}. Puoi rendere un bersaglio indifferente nei confronti di una creatura di tua scelta, verso la quale è ostile. Questa indifferenza termina se il bersaglio viene attaccato o danneggiato da un incantesimo o se vede uno dei suoi amici venir danneggiato. Quando l'incantesimo termina, la creatura diventa di nuovo ostile, a meno che il Narratore non determini diversamente.

\medskip\textbf{Camminare sull'Acqua}\index{Incantesimi - Camminare sull'Acqua}\\
\textbf{Scuola}: Trasmutazione\\
\textbf{Difficoltà}: 21\\
\textbf{Tempo di Lancio}: 2 Azioni\\
\textbf{Gittata}: 9 metri\\
\textbf{Componenti}: V, S, M (un pezzo di sughero)\\
\textbf{Durata}: 1 ora\\
Questo incantesimo conferisce la capacità di muoversi attraverso superfici liquide (come acqua, acido, fango, neve, sabbie mobili o lava) come se fossero innocuo terreno solido (le creature che attraversano la lava fusa possono comunque subire danni dal calore). Fino a dieci creature consenzienti a gittata, e che puoi vedere, ricevono questa capacità per tutta la durata. Se il tuo bersaglio è immerso in un liquido, l'incantesimo riporta il bersaglio in superficie del liquido a una velocità di 9 metri per round. 

\medskip\textbf{Camminare nel Vento}\index{Incantesimi - Camminare nel Vento}\\
\textbf{Scuola}: Trasmutazione\\
\textbf{Difficoltà}: 29\\
\textbf{Tempo di Lancio}: 1 minuto\\
\textbf{Gittata}: 9 metri\\
\textbf{Componenti}: V, S, M (fuoco e Acqua Benedetta)\\
\textbf{Durata}: 8 ore\\
Per la durata, tu e fino ad altre dieci creature consenzienti a gittata, che puoi vedere, assumete forma gassosa, diventando nubi. Mentre è in forma di nube, una creatura ha velocità di volo 90 metri e ha resistenza ai danni dalle armi non magiche. Ritornare alla forma normale richiede 1 minuto, durante il quale la creatura è inabile e non può muoversi. Fino al termine dell'incantesimo, una creatura può tornare alla forma di nube, che richiede una trasformazione di un minuto. Se una creatura è in forma di nube e sta volando quando l'effetto ha termine, la creatura scende 18 metri per round al minuto finché non atterra, al sicuro. Se non riesce ad atterrare dopo 1 minuto, la creatura cadrà per la distanza rimanente.

\medskip\textbf{Charme su Persone}\index{Incantesimi - Charme su Persone}\\
\textbf{Scuola}: Ammaliamento\\
\textbf{Difficoltà}: 16\\
\textbf{Tempo di Lancio}: 2 Azioni\\
\textbf{Gittata}: 9 metri\\
\textbf{Componenti}: V, S\\
\textbf{Durata}: 1 ora\\
Cerchi di affascinare un umanoide a gittata e che puoi vedere. Egli deve effettuare un Tiro Salvezza su Volontà, e avrà +1d6 se sta combattendo contro di te o i tuoi alleati. Se fallisce il Tiro Salvezza, è affascinato da te fino al termine dell'incantesimo o finché tu o i tuoi alleati non gli facciate qualcosa di nocivo. La creatura affascinata ti considera un amichevole conoscente. Quando l'incantesimo termina, la creatura è consapevole di essere stata affascinata da te. \\
\textbf{Per ogni Critico ottenuto} nella prova di magia puoi puoi aggiungere una creatura come bersaglio. Quando lanci l'incantesimo, le creature bersaglio devono trovarsi entro 9 metri l'una dall'altra.

\medskip\textbf{Campo Anti-Magia}\index{Incantesimi - Campo Anti-Magia}\\
\textbf{Scuola}: Abiurazione\\
\textbf{Difficoltà}: 34\\
\textbf{Tempo di Lancio}: 2 Azioni\\
\textbf{Gittata}: Personale (sfera di 3 metri di raggio)\\
\textbf{Componenti}: V, S, M (un pizzico di ferro in polvere o lime di ferro)\\
\textbf{Durata}: Concentrazione, massimo 1 ora\\
Vieni circondato da una sfera invisibile di anti-magia di 3 metri di raggio. Quest'area è separata dall'energia magica che permea il multiverso. All'interno della sfera non si possono lanciare incantesimi, le creature richiamate scompaiono e anche gli oggetti magici diventano normali. Fino al termine dell'incantesimo, la sfera si muove con te, centrata su di te. Gli incantesimi e altri effetti magici, eccetto quelli creati da un artefatto o Patrono, sono soppressi all'interno della sfera né vi possono penetrare. Uno slot speso per lanciare un incantesimo soppresso è consumato. Mentre un effetto è soppresso, non funziona, ma il tempo che trascorre soppresso è conteggiato per la sua durata. 
\\\textit{Effetti con Bersaglio}. Incantesimi e altri effetti magici, come dardo incantato e charme su persone, che prendono come bersaglio una creatura o un oggetto all'interno della sfera non hanno effetto su quel bersaglio.
\\\textit{Aree di Magia}. L'area di un altro incantesimo o effetto magico, come palla di fuoco, non può estendersi all'interno della sfera. Se la sfera si sovrappone a un'area di magia, la parte di quell'area coperta dalla sfera viene soppressa. Per esempio, le fiamme generate da un muro di fuoco vengono soppresse all'interno della sfera, creando un buco nel muro se la sovrapposizione è sufficientemente grande. Incantesimi. Qualsiasi incantesimo o altro effetto magico attivo su di una creatura od oggetto all'interno della sfera viene soppresso finché la creatura o l'oggetto si trovano all'interno della sfera.\\
\textit{Oggetti Magici}. Le proprietà e poteri degli oggetti magici vengono soppressi dalla sfera. Per esempio, una spada lunga +1 all'interno della sfera funziona come una spada lunga non magica. Le proprietà e i poteri delle armi magiche vengono soppressi se sono usati contro un bersaglio all'interno della sfera o impugnate da un attaccante dentro la sfera. Se un'arma magica o munizione magica lascia interamente la sfera (per esempio, se tiri una freccia magica o scagli una lancia magica a un bersaglio al di fuori della sfera), la magia dell'oggetto non è più soppressa non appena esce dalla sfera.
\\\textit{Magia di Viaggio}. Il teletrasporto e il viaggio planare non funzionano all'interno della sfera, che la sfera sia il punto di destinazione o di partenza di questo viaggio magico. All'interno della sfera, un portale verso un altro luogo, mondo, o piano di esistenza, così come uno spazio extradimensionale come quello creato dall'incantesimo trucco della corda, resta chiuso.
\\\textit{Creature e Oggetti}. All'interno della sfera, una creatura o oggetto evocati o creati dalla magia svaniscono temporaneamente dall'esistenza. La creatura od oggetto riappare istantaneamente una volta che lo spazio occupato da essa non si trova più all'interno della sfera.
\\\textit{Dissolvi magie}. Gli incantesimi e gli effetti magici come dissolvi magie non hanno effetto sulla sfera. Allo stesso modo, le sfere create da altri incantesimi campo antimagia non si annullano vicendevolmente. 

\medskip\textbf{Camuffare Sé Stesso}\index{Incantesimi - Camuffare Sé Stesso}\\
\textbf{Scuola}: Illusione\\
\textbf{Difficoltà}: 16\\
\textbf{Tempo di Lancio}: 2 Azioni\\
\textbf{Gittata}: Personale\\
\textbf{Componenti}: V, S\\
\textbf{Durata}: 1 ora\\
Cambi il tuo aspetto, assieme a quello dei tuoi abiti, armatura, armi e altri oggetti che indossi, fino al termine dell'incantesimo o finché non impieghi un'azione per interrompere l'incantesimo. Puoi apparire 30 centimetri più basso o più alto, magro, grasso o una via di mezzo. Non puoi modificare la tua conformazione fisica, quindi devi adottare una forma che abbia la medesima distribuzione di arti. Per tutto il resto, l'illusione è limitata solo dalla tua fantasia.\\
I cambi apportati da questo incantesimo non sono in grado di sostenere un'ispezione fisica. Per esempio, se usi questo incantesimo per aggiungere un cappello al tuo abbigliamento, gli oggetti attraversano il cappello, e chiunque lo tocchi non avvertirebbe nulla e finirebbe per toccarti la testa e i capelli. Se usi questo incantesimo per apparire più magro di quello che sei, la mano di una persona che provasse a toccarti rimbalzerebbe su di te, mentre alla vista sembrerebbe fermarsi a mezz'aria. Per distinguere il tuo camuffamento, una creatura può usare 2 Azioni per ispezionare il tuo aspetto e deve superare una prova di Consapevolezza+4 contro la DC del Tiro Salvezza dell'incantesimo. 

\medskip\textbf{Capanna}\index{Incantesimi - Capanna}\\
\textbf{Scuola}: Invocazione\\
\textbf{Difficoltà}: 21\\
\textbf{Tempo di Lancio}: 1 minuto\\
\textbf{Gittata}: Personale (semisfera di 3 metri di raggio)\\
\textbf{Componenti}: V, S, M (una piccola biglia di cristallo)\\
\textbf{Durata}: 8 ore\\
Una cupola di forza immobile del raggio di 3 metri si forma intorno e sopra di te, restando stazionaria per la durata. L'incantesimo termina se lasci l'area. Nove creature di taglia Media o inferiore possono entrare nella cupola insieme a te. L'incantesimo fallisce se l'area include una creatura più grande o più di dieci creature. Le creature e gli oggetti all'interno della cupola, quando lanci questo incantesimo, la possono attraversare liberamente. Tutte le altre creature e oggetti sono proibiti dall'attraversarla. Incantesimi e altri effetti magici non possono estendersi oltre la cupola o attraversarla (non puoi lanciare una palla di fuoco all'esterno della Capanna). L'atmosfera all'interno dello spazio è confortevole e asciutta, quale che sia il clima all'esterno.\\
Fino al termine dell'incantesimo, puoi comandare che l'interno diventi illuminato fioco o buio. La cupola è opaca dall'esterno, di qualsiasi colore tu scelga, ma è trasparente dall'interno. 

\medskip\textbf{Caratteristica Potenziata}\index{Incantesimi - Caratteristica Potenziata}\\
\textbf{Scuola}: Trasmutazione\\
\textbf{Difficoltà}: 19\\
\textbf{Tempo di Lancio}: 2 Azioni\\
\textbf{Gittata}: Contatto\\
\textbf{Componenti}: V, S, M (pelo o piuma di una bestia)\\
\textbf{Durata}: massimo 10 minuti\\
Conferisci un potenziamento magico a una creatura con cui sei in contatto. Scegli uno degli effetti seguenti; il bersaglio ottiene quell'effetto fino al termine dell'incantesimo.\\
\textit{Astuzia della Volpe}. Il bersaglio ha +1d6 alle prove di Intelligenza e Forza\\
\textit{Forza del Toro}. Il bersaglio ha +1d6 alle prove di Forza, e la sua capacità di Ingombro raddoppia.\\
\textit{Grazia del Energia Luminosa}. Il bersaglio ha +1d6 alle prove di Destrezza. Inoltre, qualora non sia inabile, non subisce danni dalle cadute di 6 metri o meno.\\
\textit{Resistenza dell'Orso}. Il bersaglio ha +1d6 alle prove di Costituzione. Ottiene anche 2d6 punti ferita temporanei, che vengono persi alla fine dell'incantesimo.\\
\textit{Saggezza del Gufo}. Il bersaglio ha +1d6 alle prove di Saggezza. \\
\textit{Splendore dell'Aquila}. Il bersaglio ha +1d6 alle prove di Carisma.\\
\textbf{Per ogni Critico ottenuto} nella prova di magia puoi prendere come bersaglio un'ulteriore creatura

\medskip\textbf{Carne in Pietra - Pietra in Carne}\index{Incantesimi - Carne in Pietra}\\
\textbf{Scuola}: Trasmutazione\\
\textbf{Difficoltà}: 29\\
\textbf{Tempo di Lancio}: 2 Azioni\\
\textbf{Gittata}: 18 metri\\
\textbf{Componenti}: V, S, M (un pizzico di lime, acqua e terra)\\
\textbf{Durata}: Permanente\\
Cerchi di trasformare in pietra una creatura a gittata che puoi vedere. Se il corpo del bersaglio è fatto di carne, la creatura deve effettuare un Tiro Salvezza su Tempra. Se fallisce il Tiro Salvezza, è intralciata e la sua carne comincia a indurirsi. Se supera il Tiro Salvezza, la creatura non subisce l'incantesimo. Una creatura intralciata da questo incantesimo deve effettuare un altro Tiro Salvezza su Tempra al termine di ciascun suo round. Se supera il Tiro Salvezza con successo per tre volte, l'incantesimo termina. Se fallisce il Tiro Salvezza per tre volte, viene trasformata in pietra e resta vittima della condizione pietrificato per la durata. I successi e i fallimenti non devono essere continuativi; tenere traccia di entrambi finché il bersaglio non ne ottiene tre di un tipo.\\
Se la creatura viene danneggiata fisicamente mentre è pietrificata, soffre di deformità simili ai danni arrecati alla pietra, se ritorna al suo stato originale. Se mantieni la tua concentrazione su questo incantesimo per la sua intera possibile durata, la creatura è trasformata in pietra finché l'effetto non viene rimosso.\\
L'incantesimo \emph{Pietra in Carne} fa tornare una creatura di carne purché non sia stata trasformata da piu' di un anno.\\


\medskip\textbf{Catena di Fulmini}\index{Incantesimi - Catena di Fulmini}\\
\textbf{Scuola}: Invocazione\\
\textbf{Difficoltà}: 29\\
\textbf{Tempo di Lancio}: 2 Azioni\\
\textbf{Gittata}: 45 metri\\
\textbf{Componenti}: V, S, M (un po' di pelliccia; un pezzo d'ambra, vetro o una verga di cristallo; e tre spille d'argento)\\
\textbf{Durata}: Istantanea\\
Crei una saetta di luce che colpisce un bersaglio a gittata che puoi vedere, scelto da te. Da questo si genera una ulteriore saetta che colpisce il piu' vicino bersaglio entro 6 metri. Il processo continua finche' non sono state colpite 7 bersagli o non c'e' piu' nessun nuovo avversario a distanza. Un bersaglio può essere una creatura o oggetto almeno di taglia media e può essere bersaglio di una sola saetta. Un bersaglio deve effettuare un Tiro Salvezza su Riflessi. Il bersaglio subisce 10d8 danni da fulmine se fallisce il Tiro Salvezza, o la metà di questi danni se lo supera.\\
\textbf{Per ogni Critico ottenuto} nella prova di magia la saetta si protende su un ulteriore bersaglio.

\medskip\textbf{Cecità/Sordità}\index{Incantesimi - Cecità/Sordità}\\
\textbf{Scuola}: Necromanzia\\
\textbf{Difficoltà}: 19\\
\textbf{Tempo di Lancio}: 2 Azioni\\
\textbf{Gittata}: 9 metri\\
\textbf{Componenti}: V\\
\textbf{Durata}: 1 minuto, Concentrazione\\
Puoi accecare o assordare un nemico. Scegli una creatura a gittata e che puoi vedere. Il bersaglio deve effettuare un Tiro Salvezza su Tempra. Se lo fallisce, il bersaglio è accecato o assordato (a tua scelta) per la durata.\\

\medskip\textbf{Cecità/Sordità Avanzata}\index{Incantesimi - Cecità/Sordità}\\
\textbf{Scuola}: Necromanzia\\
\textbf{Difficoltà}: 21\\
\textbf{Tempo di Lancio}: 2 Azioni\\
\textbf{Gittata}: 9 metri\\
\textbf{Componenti}: V,S,M (del cerume oppure un pezzo di stoffa nera)\\
\textbf{Durata}: 1 minuto\\
Puoi accecare o assordare un nemico. Scegli una creatura a gittata e che puoi vedere. Il bersaglio deve effettuare un Tiro Salvezza su Tempra. Se lo fallisce, il bersaglio è accecato o assordato (a tua scelta) per la durata.\\
\textbf{Per ogni Critico ottenuto} nella prova di magia puoi prendere come bersaglio una creatura aggiuntiva.

\medskip\textbf{Celare}\index{Incantesimi - Celare}\\
\textbf{Scuola}: Trasmutazione\\
\textbf{Difficoltà}: 31\\
\textbf{Tempo di Lancio}: 2 Azioni\\
\textbf{Gittata}: Contatto\\
\textbf{Componenti}: V, S, M (una polvere composta da polvere di diamante, smeraldo, rubino e zaffiro del valore di almeno 50.000 mo, che l'incantesimo consuma)\\
\textbf{Durata}: Fino a che dissolto \\
Tramite questo incantesimo, una creatura consenziente o un oggetto può essere nascosto, impossibile da individuare per la durata. Eseguendo questo incantesimo ed entrando in contatto con un bersaglio, questo diventa invisibile e non può essere preso come bersaglio dagli incantesimi di divinazione, né percepito da sensori di scrutamento creati da incantesimi di divinazione.\\
Se il bersaglio è una creatura, cade in uno stato di animazione sospesa. Per lui il tempo cessa di scorrere, e non invecchia. \\
Puoi predisporre una condizione per cui l'incantesimo termini anticipatamente. La condizione può essere qualsiasi cosa tu voglia, ma deve avvenire o essere visibile entro 1,5 chilometri dal bersaglio. Esempi includono "al prossimo giudizio dei Patroni" o "quando il tarrasque si risveglia". Questo incantesimo termina anche qualora il bersaglio subisca danni.


\medskip\textbf{Cerchio Magico}\index{Incantesimi - Cerchio Magico}\\
\textbf{Scuola}: Abiurazione\\
\textbf{Difficoltà}: 21\\
\textbf{Tempo di Lancio}: 1 minuto\\
\textbf{Gittata}: 3 metri\\
\textbf{Componenti}: V, S, M (Acqua Benedetta o argento e ferro in polvere del valore di almeno 100 mo, che l'incantesimo consuma)\\
\textbf{Durata}: 1 ora\\
Crei un cilindro di energia magica di 3 metri di raggio e alto 6 metri, centrato su di un punto del terreno a gittata e che puoi vedere. Rune luminose compaiono dovunque il cilindro si intersechi con il pavimento o altra superficie.\\
Scegli uno o più dei seguenti tipi di creature: celestiali, elementali, fatati, demoni o non morti. Il circolo influisce su di una creatura del tipo scelto nei modi seguenti:\\
\begin{itemize}
	\item 
La creatura non può entrare consapevolmente nel cilindro tramite alcun mezzo non magico. Se la creatura prova a usare il teletrasporto o il viaggio tra i piani per farlo, deve prima superare un Tiro Salvezza su Volontà.
	\item 
La creatura ha -1d6 ai Tiri per Colpire contro i bersagli all'interno del cilindro.
	\item 
I bersagli all'interno del cilindro non possono essere affascinati, spaventati o posseduti dalla creatura. Quando lanci questo incantesimo, puoi decidere che la magia operi in direzione opposta, impedendo a una creatura del tipo specificato di lasciare il cilindro e proteggendo i bersagli all'esterno.
\end{itemize}
\textbf{Per ogni Critico ottenuto} nella prova di magia puoi aumentare la durata di 1 ora.

\medskip\textbf{Cerchio di Morte}\index{Incantesimi - Cerchio di Morte}\\
\textbf{Scuola}: Invocazione\\
\textbf{Difficoltà}: 29\\
\textbf{Tempo di Lancio}: 2 Azioni\\
\textbf{Gittata}: 45 metri\\
\textbf{Componenti}: V, S, M (una perla nera ridotta in polvere del valore di almeno 500 mo)\\
\textbf{Durata}: Istantanea\\
Una sfera di energia negativa del raggio di 18 metri, erutta in un punto a gittata. Ogni creatura in quell'area deve effettuare un Tiro Salvezza su Tempra. Un bersaglio subisce 8d6 danni da Vuoto se fallisce il Tiro Salvezza, o la metà di questi danni se lo supera. \\
\textbf{Per ogni Critico ottenuto} nella prova il danno aumenta di 2d6.

\medskip\textbf{Cerchio di Teletrasporto}\index{Incantesimi - Cerchio di Teletrasporto}\\
\textbf{Scuola}: Evocazione\\
\textbf{Difficoltà}: 26\\
\textbf{Tempo di Lancio}: 1 minuto\\
\textbf{Gittata}: 3 metri\\
\textbf{Componenti}: V, M (gessi e inchiostri rari infusi di gemme preziose del valore di almeno 50 mo, che l'incantesimo consuma)\\
\textbf{Durata}: 1 round\\
Mentre lanci l'incantesimo, tracci un cerchio di 3 metri di diametro sul pavimento, inscritto con sigilli che collegano il posto in cui ti trovi a un cerchio di teletrasporto permanente di tua scelta, di cui conosci la sequenza dei sigilli e che si trovi sullo stesso piano di esistenza in cui ti trovi tu. Un portale luminoso si apre all'interno del cerchio tracciato da te e resta aperto fino al termine del tuo prossimo round. Qualsiasi creatura che entri nel portale, riappare istantaneamente entro 1 metro dal cerchio di destinazione o nello spazio non
occupato più vicino, se non può comparire entro 1 metro da esso.\\
Molti grandi templi, gilde, e altri luoghi importanti possiedono dei cerchi di teletrasporto permanenti, incisi da qualche parte nelle loro prossimità. Ciascuno di questi cerchi possiede una sequenza di sigilli unica: una serie di rune magiche disposte seguendo una trama precisa.\\ Quando ottieni la capacità di lanciare questo incantesimo, apprendi le sequenze di sigilli di
due destinazioni sul Piano Materiale, determinate dal Narratore. Nel corso delle tue avventure puoi imparare nuove sequenze di sigilli. Puoi mandare a memoria una sequenza di sigilli dopo averla studiata per almeno 1 minuto.\\
Puoi creare un cerchio di teletrasporto permanente eseguendo questo incantesimo nello stesso luogo ogni giorno per un anno. Non devi usare il cerchio di teletrasporto quando lanci l'incantesimo in questo modo.

\medskip\textbf{Chiaroveggenza}\index{Incantesimi - Chiaroveggenza}\\
\textbf{Scuola}: Divinazione\\
\textbf{Difficoltà}: 21\\
\textbf{Tempo di Lancio}: 10 minuti\\
\textbf{Gittata}: 1,5 chilometri\\
\textbf{Componenti}: V, S, M (un focus del valore di almeno 100 mo, che sia un corno ingioiellato per udire o un occhio di vetro per guardare)\\
\textbf{Durata}: Concentrazione, massimo 10 minuti\\
Crei un sensore invisibile in un luogo a te familiare e che sia a gittata (un luogo che hai già visitato o visto precedentemente) o in un luogo ovvio ma che non ti è familiare (come dietro una porta o un angolo, o in mezzo un boschetto di alberi). Il sensore rimane sul posto per la durata, e non può essere attaccato né altrimenti vi si può interagire. Quando lanci questo incantesimo, scegli se vedere o udire. Puoi usare il senso scelto tramite il sensore, come ti trovassi nel suo spazio. Con due azioni, puoi passare da udire a sentire e viceversa. Una creatura che può vedere il sensore (una creatura munita di vedere invisibilità o di visione del vero) lo percepisce come un orbe intangibile e luminoso delle dimensioni del tuo pugno.

\medskip\textbf{Clone}\index{Incantesimi - Clone}\\
\textbf{Scuola}: Necromanzia\\
\textbf{Difficoltà}: 34\\
\textbf{Gittata}: Contatto
\textbf{Componenti}: V, S, M (un diamante del valore di almeno 1.000 mo e almeno 16 centimetri cubi di carne della creatura che deve essere clonata, che l'incantesimo consuma, e un recipiente da almeno 2.000 mo di valore che abbia un coperchio sigillabile e sia grande a sufficienza da contenere una creatura Media, come una grossa urna, una bara, una fossa piena di fango nel terreno o un contenitore di cristallo pieno di acqua salata)\\
\textbf{Durata}: Istantanea\\
Questo incantesimo produce il duplicato inerte di una creatura vivente come salvaguardia dalla morte. Questo clone si forma all'interno di un recipiente sigillato e raggiunge la massima dimensione e maturità dopo 120 giorni; puoi anche decidere che il clone sia una versione più giovane della stessa creatura. Rimane inerte e sopravvive all'infinito, purché il recipiente resti indisturbato.\\
In qualsiasi momento dopo che il clone è maturato, se la creatura originale muore, la sua anima si trasferisce nel clone, purché l'anima sia libera e consenziente a tornare. Il clone è fisicamente identico all'originale e ha la stessa personalità, ricordi e caratteristiche, ma nulla dell'equipaggiamento dell'originale. I resti fisici della creatura originale, se esistono ancora, divengono inerti e non possono essere riportati alla vita, dato che l'anima della creatura è altrove. \\
\textbf{Questo incantesimo non e' selezionabile se i Patroni sono attivi}

\medskip\textbf{Colpo Accurato}\index{Trucchetto - Colpo Accurato}\\
\textbf{Scuola}: Divinazione\\
\textbf{Difficoltà}: 12\\
\textbf{Tempo di Lancio}: 2 Azioni\\
\textbf{Gittata}: 9 metri\\
\textbf{Componenti}: S\\
\textbf{Durata}: 1 round\\
Allunghi la mano e punti il dito verso un bersaglio a gittata. La tua magia ti conferisce una breve comprensione delle difese del bersaglio. Durante il tuo prossimo round, purché questo incantesimo non sia terminato, ottieni +1d6 al primo tiro per colpire contro quel bersaglio.

\medskip\textbf{Colpo Infuocato}\index{Incantesimi - Colpo Infuocato}\\
\textbf{Scuola}: Invocazione\\
\textbf{Difficoltà}: 26\\
\textbf{Tempo di Lancio}: 2 Azioni\\
\textbf{Gittata}: 18 metri\\
\textbf{Componenti}: V, S, M (pizzico di zolfo)\\
\textbf{Durata}: Istantanea\\
Una colonna verticale di fuoco divino scende dal cielo e si abbatte sul luogo da te specificato. Ogni creatura in un cilindro di 3 metri di raggio e alto 12 metri centrato su di un punto a gittata deve effettuare un Tiro Salvezza su Riflessi. Una creatura subisce 8d6 danni da Luce se fallisce il Tiro Salvezza, o la metà di questi danni se lo supera.\\
\textbf{Per ogni Critico ottenuto} nella prova di magia il danno Luce aumenta di 2d6.

\medskip\textbf{Comando}\index{Incantesimi - Comando}\\
\textbf{Scuola}: Ammaliamento\\
\textbf{Difficoltà}: 16\\
\textbf{Tempo di Lancio}: 2 Azioni\\
\textbf{Gittata}: 18 metri\\
\textbf{Componenti}: V\\
\textbf{Durata}: 1 round\\
Pronunci un comando di una parola verso una creatura a gittata e che puoi vedere. Il bersaglio deve superare un Tiro Salvezza su Volontà o eseguire il comando entro il suo prossimo round. L'incantesimo non ha effetto se il bersaglio è non morto, se non capisce la tua lingua, o se il tuo comando gli recherebbe danni. Seguono alcuni tipici comandi e i loro effetti. Puoi dare comandi diversi da quelli descritti qui, e in quel caso il Narratore determinerà il comportamento del bersaglio. Se il bersaglio non può eseguire il tuo comando, l'incantesimo ha fine.
\begin{itemize}
	\item 
\textit{Avvicinati}. Il bersaglio si muove verso di te per il tragitto più breve e diretto, terminando il suo round se si avvicina a 1 metro da te.
	\item 
\textit{Fermo}. Il bersaglio non si muove e poi termina il suo round. Una creatura volante resta sul posto, purché le sia possibile. Se deve muoversi per restare in aria, vola la distanza minima necessaria per farlo.
	\item 
	\textit{Getta}. Il bersaglio getta qualsiasi cosa stia tenendo in mano e poi termina il suo round. 	
	\item 
	\textit{Scappa}. Il bersaglio spende il suo round a muoversi lontano da te con il mezzo più veloce a sua disposizione.
	\item \textit{Striscia}. Il bersaglio si getta prono e poi termina il suo round.
\end{itemize}

\textbf{Per ogni Critico ottenuto} nella prova di magia puoi agire su di un'ulteriore creatura. Nel momento in cui lanci l'incantesimo, le creature bersaglio devono trovarsi entro 9 metri l'una da l'altra ed eseguono il medesimo comando.

\medskip\textbf{Comprensione dei Linguaggi}\index{Incantesimi - Comprensione dei Linguaggi}\\
\textbf{Scuola}: Divinazione\\
\textbf{Difficoltà}: 16\\
\textbf{Tempo di Lancio}: 2 Azioni\\
\textbf{Gittata}: Personale\\
\textbf{Componenti}: V, S, M (un pizzico di sale e fuliggine)\\
\textbf{Durata}: 1 ora\\
Per la durata, capisci il significato letterale di qualsiasi linguaggio parlato che ascolti.

\medskip\textbf{Comprensione degli Scritti}\index{Incantesimi - Comprensione dei Scritti}\\
\textbf{Scuola}: Divinazione\\
\textbf{Difficoltà}: 19\\
\textbf{Tempo di Lancio}: 2 Azioni\\
\textbf{Gittata}: Personale\\
\textbf{Componenti}: V, S, M (un pizzico di argento e inchiostro secco)\\
\textbf{Durata}: 1 ora\\
Per la durata comprendi qualsiasi linguaggio scritto non magico che vedi. Devi essere a contatto con la superficie su cui le parole sono scritte. Per leggere una pagina di testo impieghi 1 minuto. Questo incantesimo non decodifica i messaggi segreti in un testo o glifo, come un sigillo arcano, che non faccia parte di un linguaggio scritto.

\medskip\textbf{Compulsione}\index{Incantesimi - Compulsione}\\
\textbf{Scuola}: Ammaliamento\\
\textbf{Difficoltà}: 23\\
\textbf{Tempo di Lancio}: 2 Azioni\\
\textbf{Gittata}: 9 metri\\
\textbf{Componenti}: V, S\\
\textbf{Durata}: Concentrazione, massimo 1 minuto\\
Le creature di tua scelta entro la gittata, che puoi vedere e che ti possono sentire, devono effettuare un Tiro Salvezza su Volontà. Un bersaglio supera automaticamente il Tiro Salvezza se non può essere affascinato. Fino al termine dell'incantesimo, puoi usare un'Azione durante ciascun tuo round per indicare una direzione orizzontale rispetto a te. Ogni bersaglio soggetto all'incantesimo deve usare quanto più possibile del suo movimento, durante il suo prossimo round, per muoversi in quella direzione. Il bersaglio non può effettuare nessuna azione prima di muoversi. Dopo essersi mosso in questo modo, il bersaglio può effettuare un altro Tiro Salvezza su Volontà per tentare di terminare l'effetto.\\
Un bersaglio non può essere obbligato a muoversi dentro un pericolo palesemente letale, come fiamme o pozzi, ma per muoversi nella direzione indicata potrà provocare attacchi di opportunità.

\medskip\textbf{Comunione}\index{Incantesimi - Comunione}\\
\textbf{Scuola}: Divinazione\\
\textbf{Difficoltà}: 26\\
\textbf{Tempo di Lancio}: 1 minuto\\
\textbf{Gittata}: Personale\\
\textbf{Componenti}: V, S, M (incenso e una fiala di Acqua Benedetta o blasfema)\\
\textbf{Durata}: 1 minuto\\
Comunichi con il tuo Patrono e gli poni fino a tre domande a cui si può dare risposta con un sì o un no. Devi porre le domande prima della fine dell'incantesimo. Riceverai la risposta corretta a ciascuna domanda. Le creature divine non sono necessariamente onniscienti, quindi potresti ricevere "non è chiaro" come risposta a una domanda che riguarda informazioni non pertinenti alle conoscenze del Patrono. Nel caso in cui una risposta di una parola potrebbe essere fuorviante o contraria agli interessi del Patrono, il Narratore potrebbe invece dare una breve frase come risposta.\\
Se lanci l'incantesimo due o più volte prima che sia sorta la nuova alba c'è una probabilità cumulativa del 25\% che per ogni lancio dopo il primo tu non ottenga alcuna risposta. Il Narratore effettua questo tiro in segreto.\\
\textbf{Questo incantesimo non e' selezionabile se i Patroni non sono attivi}

\medskip\textbf{Comunione con la Natura}\index{Incantesimi - Comunione con la Natura}\\
\textbf{Scuola}: Divinazione\\
\textbf{Difficoltà}: 26\\
\textbf{Tempo di Lancio}: 1 minuto\\
\textbf{Gittata}: Personale\\
\textbf{Componenti}: V, S\\
\textbf{Durata}: Istantanea\\
Per un istante diventi tutt'uno con la natura e ottieni informazioni sul territorio circostante. In ambienti esterni, l'incantesimo ti fornisce informazioni sul territorio entro 5 chilometri da te. In grotte e altri ambienti naturali sotterranei, il raggio è limitato a 100 metri. L'incantesimo non funziona nei luoghi in cui la natura è stata soppiantata da costruzioni, come in sotterranei e paesi.\\
Apprendi immediatamente informazioni su un massimo di tre argomenti a tua scelta su uno dei seguenti soggetti, in relazione all'area:
\begin{itemize}
	\item 
	terreno e corpi d'acqua
	\item 
	piante, minerali, animali e popolazioni prevalenti
	\item 
  potenti celestiali, elementali, fatati, demoni o non morti
	\item 
  influenze da altri piani di esistenza
	\item
  edifici
\end{itemize}

\medskip\textbf{Confusione}\index{Incantesimi - Confusione}\\
\textbf{Scuola}: Ammaliamento\\
\textbf{Difficoltà}: 23\\
\textbf{Tempo di Lancio}: 2 Azioni\\
\textbf{Gittata}: 27 metri\\
\textbf{Componenti}: V, S, M (tre gusci di noce)\\
\textbf{Durata}: 1 minuto\\
Questo incantesimo assale e piega la mente delle creature, generando illusioni e provocando azioni incontrollate. Quando lanci questo incantesimo ogni creatura, in una sfera di 3 metri di raggio centrata su di un punto da te scelto entro la gittata, deve superare un Tiro Salvezza su Volontà o subirne gli effetti. Un bersaglio soggetto all'incantesimo non può effettuare reazioni e deve tirare un d10 all'inizio di ciascun suo round per determinare il proprio comportamento per quel round. 

\medskip

\begin{tabularx}{0.45\textwidth}{lX}
	\hline 
d10 & Comportamento\\ 
1 & La creatura usa tutto il suo movimento per muoversi in una direzione casuale. Per determinare la direzione, tira un d8 assegnando a ciascuna faccia un punto cardinale. La
creatura non effettuerà nessuna azione in questo round. \\
2-6 & La creatura non può muoversi né attaccare in questo round.\\
7-8 & La creatura usa la sua 2 Azioni (e nessun'altra) per effettuare un attacco da mischia contro una creatura determinata a caso entro la sua portata. Se non c'è nessuna creatura a portata, per questo round la creatura non farà nulla.\\
9-10 & La creatura può agire e muoversi normalmente.\\
\end{tabularx} 

\medskip

Al termine di ciascun suo round, un bersaglio soggetto all'incantesimo può effettuare un Tiro Salvezza su Volontà. Se lo supera, per lui l'effetto ha termine. \\
\textbf{Per ogni Critico ottenuto} nella prova di magia il raggio della sfera aumenta di 1 metro.

\medskip\textbf{Cono di Freddo}\index{Incantesimi - Cono di Freddo}\\
\textbf{Scuola}: Invocazione\\
\textbf{Difficoltà}: 26\\
\textbf{Tempo di Lancio}: 2 Azioni\\
\textbf{Gittata}: Personale (cono di 18 metri)\\
\textbf{Componenti}: V, S, M (un piccolo cristallo o cono di vetro)\\
\textbf{Durata}: Istantanea\\
Un'esplosione di aria fredda erutta dalle tue mani. Ogni creatura in un cono di 18 metri deve effettuare un Tiro Salvezza su Tempra. Una creatura subisce 8d8 danni da freddo se fallisce il Tiro Salvezza, o la metà di questi danni se lo supera. Una creatura uccisa da questo incantesimo diventa una statua di ghiaccio fino a quando disgela.\\
\textbf{Per ogni Critico ottenuto} nella prova di magia il danno aumenta di 1d8

\medskip\textbf{Conoscenza delle Leggende}\index{Incantesimi - Conoscenza delle Leggende}\\
\textbf{Scuola}: Divinazione\\
\textbf{Difficoltà}: 26\\
\textbf{Tempo di Lancio}: 10 minuti\\
\textbf{Gittata}: Personale\\
\textbf{Componenti}: V, S, M (incenso del valore di almeno 250 mo, che l'incantesimo consuma, e quattro strisce d'avorio del valore di almeno 50 mo)\\
\textbf{Durata}: Istantanea\\
Nomina o descrivi una persona, luogo od oggetto. L'incantesimo ti porta alla mente un breve riassunto delle conoscenze più importanti sull'argomento da te nominato. Se la cosa da te nominata non ha alcuna rilevanza leggendaria, non ottieni alcuna informazione. Maggiori informazioni hai sull'argomento, più precise e dettagliate saranno le informazioni che riceverai. L'informazione che riceverai sarà accurata, ma celata magari in linguaggio metaforico.

\medskip\textbf{Contagio}\index{Incantesimi - Contagio}\\
\textbf{Scuola}: Necromanzia\\
\textbf{Difficoltà}: 26\\
\textbf{Tempo di Lancio}: 2 Azioni\\
\textbf{Gittata}: Contatto\\
\textbf{Componenti}: V, S\\
\textbf{Durata}: 7 giorni\\
Tramite il contatto puoi infliggere malattie. Effettua un attacco da mischia contro una creatura a portata. Se colpisci, infetti la creatura con una malattia a tua scelta tra quelle descritte di seguito. Al termine di ciascun round del bersaglio, esso deve effettuare un Tiro Salvezza su Tempra. Dopo aver fallito tre di questi Tiri Salvezza, gli effetti della malattia permangono per la durata, e la creatura non effettua più Tiri Salvezza. Dopo aver superato tre di questi Tiri Salvezza, la creatura recupera dalla malattia, e l'incantesimo ha termine. \\
Dato che questo incantesimo induce nel suo bersaglio una malattia naturale, qualsiasi effetto che rimuova le malattie o migliori gli effetti delle malattie si applica a essa.\\
\begin{itemize}
	\item 
	\textit{Carne Putrida}. La pelle della creatura marcisce. La creatura ha -1d6 alle prove di Carisma e ogni danno e' raddoppiato.
\item 
	\textit{Debolezza Accecante}. Il dolore attanaglia la mente della creatura mentre i suoi occhi diventano bianco latte. La creatura ha -1d6 alle prove di Saggezza e ai Tiri Salvezza su Volontà, ed è accecata.
\item 
  \textit{Febbre Lurida}. Una febbre devastante sconvolge il corpo della creatura. La creatura ha -1d6 alle prove di Forza e ai Tiri Salvezza su Tempra, e ai Tiri per Colpire che usano la Forza.
\item 
\textit{Fitte}. La creatura è sopraffatta dai tremiti. La creatura ha -1d6 alle prove di Destrezza e ai Tiri Salvezza su Destrezza, e ai Tiri per Colpire che usano l'Destrezza.
\item 
\textit{Fuoco Mentale}. La mente della creatura è preda della febbre. La creatura ha -1d6 alle prove di Intelligenza e ai Tiri Salvezza su Intelligenza, e si comporta come se in combattimento fosse sotto l'effetto dell'incantesimo confusione.
\item 
\textit{Morte Melmosa}. La creatura inizia a sanguinare incessantemente. La creatura ha -1d6 alle prove di Costituzione e ai Tiri Salvezza su Tempra. Inoltre, ogni qualvolta la creatura subisce danni, è stordita fino alla fine del suo prossimo round.
\end{itemize}

\medskip\textbf{Contingenza}\index{Incantesimi - Contingenza}\\
\textbf{Scuola}: Invocazione\\
\textbf{Difficoltà}: 29\\
\textbf{Tempo di Lancio}: 10 minuti\\
\textbf{Gittata}: Personale\\
\textbf{Componenti}: V, S, M (una statuetta raffigurante te stesso scolpita in avorio e decorata con gemme del valore di almeno 1.500 mo)\\
\textbf{Durata}: 10 giorni\\
Scegli un incantesimo di Difficoltà 23 o più basso che puoi lanciare, che abbia il tempo di lancio di 2 Azioni e che può avere te come bersaglio. Lanci quell'incantesimo (detto incantesimo contingente) come parte del lancio di contingenza, spendendo gli slot incantesimo di entrambi, ma senza che l'incantesimo contingente abbia effetto. Avrà invece effetto quando si avvererà una determinata circostanza. Descrivi questa circostanza mentre lanci i due incantesimi. Per esempio, contingenza lanciato assieme a respirare sott'acqua potrebbe stipulare che respirare sott'acqua entra in azione quando sei immerso nell'acqua o simile liquido.\\
l'incantesimo contingente ha effetto immediatamente dopo che la circostanza si verifica per la prima volta, che tu lo voglia o no, e poi contingenza termina. L'incantesimo contingente agisce solo su di te, anche se normalmente può prendere come bersaglio anche altri. Puoi usare un solo incantesimo contingenza alla volta. Se lanci di nuovo questo incantesimo, l'effetto di un altro incantesimo contingenza su di te avrà termine. Inoltre, contingenza per te ha termine se la componente materiale non dovesse più trovarsi sulla tua persona.


\medskip\textbf{Controincantesimo}\index{Incantesimi - Controincantesimo}\\
\textbf{Scuola}: Abiurazione\\
\textbf{Difficoltà}: 21\\
\textbf{Tempo di Lancio}: 1 reazione, che effettui quando vedi una creatura entro 18 metri da te lanciare un incantesimo\\
\textbf{Gittata}: 18 metri\\
\textbf{Componenti}: S \\
\textbf{Durata}: Istantanea\\
Cerchi di interrompere una creatura nell'atto di lanciare un incantesimo. Se la creatura sta lanciando un incantesimo di Difficoltà 21 o più basso, l'incantesimo fallisce e non ha effetto. 


\medskip\textbf{Controllare Acqua}\index{Incantesimi - Controllare Acqua}\\
\textbf{Scuola}: Trasmutazione\\
\textbf{Difficoltà}: 23\\
\textbf{Tempo di Lancio}: 2 Azioni\\
\textbf{Gittata}: 90 metri\\
\textbf{Componenti}: V, S, M (un goccio d'acqua e un pizzico di polvere)\\
\textbf{Durata}: Concentrazione, massimo 10 minuti\\
Fino al termine dell'incantesimo, controlli qualsiasi acqua libera all'interno dell'area che hai scelto fino a un cubo di 30 metri di spigolo. Quando lanci questo incantesimo puoi scegliere qualsiasi tra i seguenti effetti. Come azione, durante il tuo round, puoi ripetere lo stesso effetto o sceglierne uno diverso.\\
\begin{itemize}
\item 
\textit{Allagamento}. Fai sì che il livello di tutta l'acqua nell'area aumenti fino a 6 metri. Se l'area include una costa, l'acqua inonda la terraferma. Se scegli un'area all'interno di un grosso corpo d'acqua, crei invece un'onda alta 6 metri che viaggia da un lato all'altro dell'area prima di infrangersi. Qualsiasi veicolo di taglia Enorme o inferiore sul percorso dell'onda viene trasportato dall'altro lato. Qualsiasi veicolo di taglia Enorme o inferiore colpito dall'acqua ha una percentuale del 25\% di cappottarsi.\\
Il livello dell'acqua resta elevato fino al termine dell'incantesimo o finché non scegli un effetto diverso. Se questo effetto ha prodotto un'onda, l'onda si ripete all'inizio del tuo round successivo, finché perdura l'effetto di allagamento.\\
\item 
\textit{Dividere le Acque}. Fai sì che l'acqua nell'area si sposti a lato per creare un varco. Il varco si estende per l'area dell'incantesimo, e l'acqua divisa forma un muro su entrambi i lati del varco. Il varco resta fino al termine dell'incantesimo o finché non scegli un effetto diverso. L'acqua tornerà poi lentamente a riempire il varco nel corso del round successivo, fino a che non sarà risalita al suo normale livello.
\item 
\textit{Ridirigere il Flusso}. Fai sì che l'acqua corrente nell'area si muova in una direzione a tua scelta, anche se l'acqua deve superare degli ostacoli, risalire muri o dirigersi verso altre direzioni improbabili. L'acqua nell'area si muove secondo le tue indicazioni, ma una volta giunta oltre l'area dell'incantesimo, riprende il suo flusso in base alle condizioni del terreno. L'acqua continua a muoversi nella direzione da te scelta fino al termine dell'incantesimo o finché non scegli un effetto diverso.
\item 
\textit{Turbine}. Questo effetto richiede un corpo d'acqua che copra un quadrato di 15 metri di lato e abbia una profondità di 7 metri. Fai sì che si formi un turbine al centro dell'area. Il turbine produce un vortice largo 1 metro alla base, largo fino a 15 metri in cima e alto 7
metri. Qualsiasi creatura od oggetto nell'acqua e che si trovi entro 7 metri dal vortice viene trascinato 3 metri verso di esso. Una creatura può nuotare per allontanarsi dal vortice effettuando una prova di Destrezza (Atletica) contro la DC del Tiro Salvezza dell'incantesimo.
Quando una creatura entra nel vortice per la prima volta durante un round o inizia lì il suo round, deve effettuare un Tiro Salvezza su Tempra. Se lo fallisce, la creatura subisce 2d8 danni da botta e viene catturata dal vortice fino al termine dell'incantesimo. Se supera il Tiro Salvezza, la creatura subisce la metà di questi danni, e non è catturata dal vortice. Una creatura catturata dal vortice può usare 3 Azioni per cercare di nuotare via dal vortice come descritto sopra, ma ha -1d6 alle prove di Destrezza (Atletica) per farlo. La prima volta durante ciascun round in un cui un oggetto entra nel vortice, l'oggetto subisce 2d8 danni da botta; questo danno viene subito ogni round in cui l'oggetto rimane nel vortice.
\end{itemize}


\medskip\textbf{Controllare Tempo Atmosferico}\index{Incantesimi - Controllare Tempo Atmosferico}\\
\textbf{Scuola}: Trasmutazione\\
\textbf{Difficoltà}: 34\\
\textbf{Tempo di Lancio}: 10 minuti\\
\textbf{Gittata}: Personale (raggio di 1,5 chilometri)\\
\textbf{Componenti}: V, S, M (incenso bruciato e po' di terra e legno mescolati nell'acqua)\\
\textbf{Durata}: Concentrazione, massimo 8 ore \\
Per la durata, assumi il controllo del clima entro 7,5 chilometri da te. Per lanciare questo incantesimo devi essere all'esterno. Muoversi in un posto dove non hai la visuale aperta verso il cielo, termina l'incantesimo anticipatamente. Quando lanci questo incantesimo, cambia le attuali condizioni climatiche, determinate dal Narratore in base alla stagione e la latitudine. Puoi modificare le precipitazioni, la temperatura e il vento. Ci vogliono 1d4 x 10 minuti perché la nuova condizione prenda effetto. Una volta che la condizione avrà preso effetto, potrai cambiarla di nuovo. Quando l'incantesimo termina, il clima tornerà gradualmente alla norma.\\
Quando cambi le condizioni climatiche, trova l'attuale condizione sulla seguente tabella e cambiala di uno stadio, verso l'alto o il basso. Quando cambi il vento, puoi cambiarne anche la direzione. \\
\medskip
\textit{Precipitazione}
\begin{itemize}
	\item 
1 Limpido
	\item 
2 Qualche nuvola
	\item 
3 Coperto o foschia a terra
	\item 
4 Pioggia, grandine o neve
	\item 
5 Pioggia torrenziale, grandinata pesante, tormenta
\end{itemize}

\textit{Temperatura}

\begin{itemize}
 \item 
1 Caldo insopportabile
	\item 
2 Caldo
	\item 
3 Tiepido
	\item 
4 Fresco
	\item 
5 Freddo
	\item 
6 Freddo polare
\end{itemize}

\textit{Vento}
\begin{itemize}
	\item 
1 Calmo
	\item 
2 Vento moderato
	\item 
3 Vento moderato
	\item 
4 Fortunale
	\item 
5 Tempesta
\end{itemize}

\medskip\textbf{Costrizione}\index{Incantesimi - Costrizione}\\
\textbf{Scuola}: Ammaliamento\\
\textbf{Difficoltà}: 26\\
\textbf{Tempo di Lancio}: 1 minuto\\
\textbf{Gittata}: 18 metri\\
\textbf{Componenti}: V\\
\textbf{Durata}: 30 giorni\\
Imponi un comando magico a una creatura a gittata che puoi vedere, obbligandolo ad adempiere un determinato compito o vietandole di svolgere un'azione o corso d'attività deciso da te. Se la creatura ti può capire, deve superare un Tiro Salvezza su Volontà o restare affascinata da te per la durata. Mentre la creatura è affascinata da te, subisce 3d10 danni ogni volta che agisce in maniera direttamente contraria alle tue istruzioni, ma non più di una volta al giorno. Una creatura che non ti può capire ignora gli effetti di questo incantesimo. Puoi dare qualsiasi comando di tua scelta, tranne un'attività che provocherebbe morte certa. Dovessi tu pronunciare un comando suicida, l'incantesimo avrebbe termine.\\
Puoi terminare l'incantesimo usando un'azione. Anche rimuovi maledizione, ristorare superiore o desiderio vi pongono termine.\\
\textbf{Se ottini almeno due Critici} nella prova di magia la durata è 1 anno. Se ottieni 3 Critici l'incantesimo dura finché non viene terminato da uno degli incantesimi sopra menzionati.

\medskip\textbf{Creare Cibo e Acqua}\index{Incantesimi - Creare Cibo e Acqua}\\
\textbf{Scuola}: Evocazione\\
\textbf{Difficoltà}: 21\\
\textbf{Tempo di Lancio}: 2 Azioni\\
\textbf{Gittata}: 9 metri\\
\textbf{Componenti}: V, S\\
\textbf{Durata}: Istantanea\\
Crei 22,5 chili di cibo e 120 litri d'acqua sul terreno o in contenitori a gittata, sufficienti a sostenere fino a quindici umanoidi o cinque cavalcature per 24 ore. Il cibo è blando ma nutriente, e marcisce dopo 24 ore se non viene consumato, come anche l'acqua.

\medskip\textbf{Creare Birra}\index{Incantesimi - Creare Birra}\\
\textbf{Scuola}: Evocazione\\
\textbf{Difficoltà}: 12\\
\textbf{Tempo di Lancio}: 2 Azioni o piu'\\
\textbf{Gittata}: 9 metri\\
\textbf{Componenti}: V, S, M (lievito di birra, malto, acqua)\\
\textbf{Durata}: 1 ora\\
Crei 1 litro di birra. La qualita' e tipologia di birra dipende dal lievito, malto e acqua usata.
Maggiore e' il tempo di lancio dell'incantesimo piu' viene alta la gradazione alcolica, con un tempo di lancio di due azioni la gradazione e' di 4.3, se 1 azione e' analcolica, ogni azione spesa aumenta la gradazione di 0.3 vol fino ad un massimo di 12.5 vol.
Dopo un ora la birra svanisce, se consumata dopo un ora terminano anche eventuali effetti alcolici della stessa sulle persone che l'hanno bevuta.\\
\textbf{Per ogni Critico ottenuto} nella prova di magia aumenti di un litro o di un ora la durata.

\medskip\textbf{Creare o Distruggere Acqua}\index{Incantesimi - Creare o Distruggere Acqua}\\
\textbf{Scuola}: Trasmutazione\\
\textbf{Difficoltà}: 16\\
\textbf{Tempo di Lancio}: 2 Azioni\\
\textbf{Gittata}: 9 metri\\
\textbf{Componenti}: V, S, M (un goccio d'acqua per creare acqua o qualche granello di sale per distruggerla)\\
\textbf{Durata}: Istantanea\\
Crei o distruggi l'acqua.\\
\textit{Creare Acqua}. Crei fino a 40 litri di acqua limpida dalle tue mani che spruzzano fino a 9 metri. In alternativa l'acqua cade come pioggia in un cubo di 9 metri di spigolo che si trovi entro la gittata, estinguendo le fiamme esposte nell'area.\\
L'incantesimo non puo' essere usato su fiamme magiche.\\
\textit{Distruggere Acqua}. Distruggi fino a 40 litri di acqua in un contenitore aperto a gittata. In alternativa, puoi distruggere la nebbia in un cubo di 9 metri di spigolo entro la gittata.\\
\textbf{Per ogni Critico ottenuto} nella prova di magia crei o distruggi ulteriori 40 litri d'acqua, o le dimensioni del cubo aumentano di 1 metro di spigolo in caso di nebbia.\\
L'acqua e' potabile e disseta se bevuta entro un round dalla creazione.

\medskip\textbf{Creare Non Morti}\index{Incantesimi - Creare Non Morti}\\
\textbf{Scuola}: Necromanzia\\
\textbf{Difficoltà}: 29\\
\textbf{Tempo di Lancio}: 2 Azioni\\
\textbf{Gittata}: 3 metri\\
\textbf{Componenti}: V, S, M (un vaso di terracotta pieno di terra di cimitero, un vaso di terracotta pieno di acqua salmastra, e un onice nero del valore di 50 mo per ogni cadavere)\\
\textbf{Durata}: Istantanea\\
Puoi lanciare questo incantesimo solo di notte. Scegli fino a tre cadaveri di umanoidi Medi o Piccoli a gittata. Ogni cadavere diventa un ghoul sotto il tuo controllo (il Narratore possiede le statistiche di gioco di queste creature). Durante il tuo round, con due Azioni, puoi comandare mentalmente una qualsiasi creatura da te animata con questo incantesimo, se la creatura si trova entro 36 metri da te (se controlli più creature, puoi comandarle tutte o solo una nello stesso momento impartendo lo stesso comando). Decidi tu quale azione effettuerà la creatura e dove si muoverà durante il suo prossimo round, oppure puoi impartire un comando generico, come quello di fare la guardia a una specifica stanza o corridoio. Se non impartisci comandi, le creature si limiteranno a difendersi dalle creature ostili. Una volta ricevuto un comando, la creatura continuerà a eseguirlo finche il compito sarà completo. La creatura è sotto il tuo controllo per 24 ore, dopodiché smetterà di rispondere ai comandi che gli impartisci. Per mantenere il controllo della creatura per altre 24 ore, devi lanciare questo incantesimo sulla creatura prima che l'attuale periodo di 24 ore abbia termine. Questo impiego dell'incantesimo riasserisce il tuo controllo su di un massimo di tre creature che hai animato con questo incantesimo, anziché animarne di nuove.\\
\textbf{Se ottieni un Critico} nella prova di magia puoi rianimare o riasserire il controllo su quattro ghoul. Con due Critici puoi animare o riasserire il controllo su cinque
ghoul o due ghast o wight. Con tre Critici puoi animare o riasserire il controllo su sei ghoul, tre ghast o wight, o due mummie. 

\medskip\textbf{Creazione}\index{Incantesimi - Creazione}\\
\textbf{Scuola}: Illusione\\
\textbf{Difficoltà}: 26\\
\textbf{Tempo di Lancio}: 1 minuto\\
\textbf{Gittata}: 9 metri\\
\textbf{Componenti}: V, S, M (un minuscolo pezzo di materiale dello stesso tipo di oggetto che intendi creare) \\
\textbf{Durata}: Speciale\\
Afferri pezzi di materia d'ombra dal piano delle Ombre per creare, a gittata, oggetti non viventi di materia vegetale: beni morbidi, corda, legno o qualcosa di simile. Puoi usare questo incantesimo anche per creare oggetti minerali come pietra, cristallo o metallo. L'oggetto creato non può essere più grosso di un cubo di 1 metro di spigolo, e l'oggetto deve essere di una forma e materiale che hai già visto in passato.\\
La durata dipende dal materiale dell'oggetto. Se l'oggetto è composto da più materiali, usa la durata più breve.
\medskip
Tabella Materiale - Durato
\medskip

\begin{tabularx}{0.45\textwidth}{lX}
	\hline 
Materia vegetale &1 giorno\\
Pietra o cristallo &12 ore\\
Metalli preziosi &1 ora\\
Gemme &10 minuti\\
Adamantio o mithril &1 minuto\\
\end{tabularx} 
\medskip

Usare qualsiasi materiale creato da questo incantesimo come componente materiale di un altro incantesimo farà fallire il nuovo incantesimo.\\
\textbf{Per ogni Critico ottenuto} nella prova di magia il cubo aumenta di 1 metro di spigolo

\medskip\textbf{Crescita di Spuntoni}\index{Incantesimi - Crescita di Spuntoni}\\
\textbf{Scuola}: Trasmutazione\\
\textbf{Difficoltà}: 19\\
\textbf{Tempo di Lancio}: 2 Azioni\\
\textbf{Gittata}: 45 metri\\
\textbf{Componenti}: V, S, M (sette spine affilate o sette ramoscelli, ciascuno di esse appuntito ad un'estremità)\\
\textbf{Durata}: 10 minuti\\
Il terreno in un raggio di 6 metri centrato su di un punto a gittata si contorce e genera spuntoni e spine molto acuminate. Per la durata, l'area diventa terreno difficile. Quando una creatura entra o si muove all'interno dell'area, subisce 2d4 danni per ogni 1 metro percorsi.
La trasformazione del terreno è talmente ben camuffata da sembrare naturale. Qualsiasi creatura che non abbia visto l'area al momento del lancio dell'incantesimo deve effettuare una prova di Consapevolezza contro la DC del Tiro Salvezza dell'incantesimo, per riconoscere il pericolo rappresentato dal terreno prima di entrarvi. 

\medskip\textbf{Crescita Vegetale}\index{Incantesimi - Crescita Vegetale}\\
\textbf{Scuola}: Trasmutazione\\
\textbf{Difficoltà}: 21\\
\textbf{Tempo di Lancio}: 2 Azioni o 8 ore\\
\textbf{Gittata}: 45 metri\\
\textbf{Componenti}: V, S\\
\textbf{Durata}: Istantanea\\
Questo incantesimo incanala vitalità nei vegetali entro una specifica area. Esistono due usi possibili per questo incantesimo, che conferiscono benefici immediati o a lungo termine. Se lanci questo incantesimo impiegando 1 azione, scegli un punto a gittata. Tutte i vegetali normali in un raggio di 30 metri centrato su quel punto diventano densi e folti. Una creatura che attraversa l'area quadruplica il costo del suo movimento.\\
Puoi escludere dai suoi effetti una o più aree di qualsiasi dimensione all'interno dell'area dell'incantesimo.\\
Se lanci questo incantesimo nel corso di 8 ore, nutri la terra. Tutti i vegetali in un raggio di 750 metri centrato su di un punto a gittata diventano super produttivi per 1 anno. I vegetali producono il doppio del normale ammontare di cibo al momento del raccolto.

\medskip\textbf{Cuoco Invisibile}\index{Incantesimi - Cuoco Invisibile}\\
\textbf{Scuola}: Evocazione\\
\textbf{Difficoltà}: 16\\
\textbf{Tempo di Lancio}: 2 Azioni\\
\textbf{Gittata}: 18 metri\\
\textbf{Componenti}: V, S, M (un mestolo di legno e qualche goccia di olio di oliva, il cibo che si vuole cucinato)\\
\textbf{Durata}: 2 ore\\
Questo incantesimo crea una forza quasi invisibile solo delimitata da una leggera aura (di colore a tua scelta) capace e competente nel cucinare. Assieme al cuoco si manifesta anche un set di pentole e padelle nonché stoviglie ed un piccolo fornello da campo.\\
In base agli ingredienti a disposizione o erbe e verdure nel raggio di 100 metri (il cuoco non va a caccia) il cuoco cucinera' al meglio degli ingredienti preparando delle ottime vivande fino a 4 persone. L'incantesimo non crea cibo o acqua, questo deve essere a disposizione al momento del lancio dell'incantesimo. \\
Una volta a disposizione gli ingredienti entro le due ore il cuoco invisibile preparera' da mangiare. E' possibile anche affrettare l'esecuzione ma a discapito della qualita'.\\
Nessuna delle pentole, stoviglie o fuoco puo' essere usato fuorché dal cuoco invisibile.

\medskip\textbf{Cura Ferite Leggere}\index{Incantesimi - Cura Ferite Leggere}\\
\textbf{Scuola}: Cura\\
\textbf{Difficoltà}: 16\\
\textbf{Tempo di Lancio}: 2 Azioni\\
\textbf{Gittata}: Contatto\\
\textbf{Componenti}: V, S\\
\textbf{Durata}: Istantanea\\
Una creatura in mischia con te recupera un numero di punti ferita uguale a 1d8 + Saggezza. Questo incantesimo se usato su un non morto, tiro per colpire con incantesimo, lo danneggia dello stesso ammontare.\\
Questo incantesimo se non esplicitato diversamente non puo' essere usato su animali o piante.\\
\textbf{Per ogni Critico ottenuto} nella prova di magia curi 1d6 PF in più.\\
Se incantatore e creatura curata sono entrambi Seguaci dello stesso Patrono l'incantesimo cura 1d8 in piu'.\\
Se incantatore e creatura curata sono entrambi Devoti dello stesso Patrono ogni valore sul dado pari a 1,2,3 sara' considerato 4.\\

\medskip\textbf{Cura Ferite Serie}\index{Incantesimi - Cura Ferite Serie}\\
\textbf{Scuola}: Cura\\
\textbf{Difficoltà}: 21\\
\textbf{Tempo di Lancio}: 2 Azioni\\
\textbf{Gittata}: Contatto\\
\textbf{Componenti}: V, S\\
\textbf{Durata}: Istantanea\\
Una creatura in mischia con te recupera un numero di punti ferita uguale a 3d8 + 2*Saggezza. Questo incantesimo se usato su un non morto, tiro per colpire con incantesimo, lo danneggia dello stesso ammontare.\\
Questo incantesimo se non esplicitato diversamente non puo' essere usato su animali o piante.\\
\textbf{Per ogni Critico ottenuto} nella prova di magia curi 1d6 PF in più.\\
Se incantatore e creatura curata sono entrambi Seguaci dello stesso Patrono l'incantesimo cura 1d8 in piu'.\\
Se incantatore e creatura curata sono entrambi Devoti dello stesso Patrono ogni valore sul dado pari a 1,2,3 sara' considerato 4.\\

\medskip\textbf{Cura Ferite Critiche}\index{Incantesimi - Cura Ferite Critiche }\\
\textbf{Scuola}: Cura\\
\textbf{Difficoltà}: 20 \\
\textbf{Tempo di Lancio}: 2 Azioni\\
\textbf{Gittata}: Contatto\\
\textbf{Componenti}: V, S\\
\textbf{Durata}: Istantanea\\
Una creatura in mischia con te recupera un numero di punti ferita uguale a 5d8 + 3*Saggezza. Questo incantesimo se usato su un non morto, tiro per colpire con incantesimo, lo danneggia dello stesso ammontare.\\
Questo incantesimo se non esplicitato diversamente non puo' essere usato su animali o piante.\\
\textbf{Per ogni Critico ottenuto} nella prova di magia curi 1d6 PF in più.\\
Se incantatore e creatura curata sono entrambi Seguaci dello stesso Patrono l'incantesimo cura 1d8 in piu'.\\
Se incantatore e creatura curata sono entrambi Devoti dello stesso Patrono ogni valore sul dado pari a 1,2,3 sara' considerato 4.\\

\medskip\textbf{Cura Ferite di Massa}\index{Incantesimi - Cura Ferite di Massa}\\
\textbf{Scuola}: Cura\\
Come le Cura Ferite ma curi fino a 4 creature.\\
La Difficoltà aumenta di 6 rispetto al Cura Ferite selezionato.\\
\textbf{Per ogni Critico ottenuto} nella prova curi una creatura in piu'.\\
Se incantatore e creatura curata sono entrambi Seguaci dello stesso Patrono l'incantesimo cura 1d8 in piu'.\\
Questo incantesimo se non esplicitato diversamente non puo' essere usato su animali o piante.\\
Se incantatore e creatura curata sono entrambi Devoti dello stesso Patrono ogni valore sul dado pari a 1,2,3 sara' considerato 4.\\

\medskip\textbf{Dardo di Fuoco}\index{Incantesimi - Dardo di Fuoco}\\
\textbf{Scuola}: Invocazione\\
\textbf{Difficoltà}: 16\\
\textbf{Tempo di Lancio}: 2 Azioni\\
\textbf{Gittata}: 36 metri\\
\textbf{Componenti}: V, S\\
\textbf{Durata}: Istantanea\\
Scagli una scintilla infuocata a una creatura od oggetto a gittata. Effettua un attacco a distanza con incantesimo contro il bersaglio. Se colpisci, il bersaglio subisce 1d10 danni da fuoco. Un oggetto infiammabile colpito da questo incantesimo prende fuoco, se non è indossato o trasportato.\\
Il danno dell'incantesimo aumenta di 1d8 quando raggiungi CM 5, CM 11 e CM 17.

\medskip\textbf{Dardo Tracciante}\index{Incantesimi - Dardo Tracciante}\\
\textbf{Scuola}: Invocazione\\
\textbf{Difficoltà}: 16\\
\textbf{Tempo di Lancio}: 2 Azioni\\
\textbf{Gittata}: 36 metri\\
\textbf{Componenti}: V, S\\
\textbf{Durata}: 1 round\\
Un lampo di luce viaggia verso una creatura a gittata, scelta da te. Effettua un attacco a distanza con incantesimo contro il bersaglio. Se colpisci, il bersaglio subisce 4d6 danni da Luce, e il prossimo tiro per colpire effettuato contro di lui prima del termine del tuo
prossimo round ha +1d6 al TC, grazie alla mistica luce fioca che continuerà a brillare intorno al bersaglio fino ad allora.\\
\textbf{Per ogni Critico ottenuto} nella prova di magia il danno aumenta di 1d6.

\medskip\textbf{Danza Irresistibile}\index{Incantesimi - Danza Irresistibile}\\
\textbf{Scuola}: Ammaliamento\\
\textbf{Difficoltà}: 34\\
\textbf{Tempo di Lancio}: 2 Azioni\\
\textbf{Gittata}: 9 metri\\
\textbf{Componenti}: V\\
\textbf{Durata}: 1 minuto\\
Scegli una creatura a gittata e che puoi vedere. Il bersaglio comincia un comico balletto sul posto: agitando le gambe, battendo i piedi e saltellando per la durata. Le creature che non possono essere affascinate sono immuni a questo incantesimo.\\
Una creatura che balla deve usare 2 Azioni di Movimento per ballare senza lasciare il suo spazio e ha -1d6 ai Tiri Salvezza su Destrezza e i Tiri per Colpire. Mentre il bersaglio è soggetto a questo incantesimo, le altre creature hanno +1d6 ai Tiri per Colpire contro di lui. Spendendo 2 Azioni la creatura che balla puo' affettuare un nuovo Tiro Salvezza su Volontà per recuperare il controllo di se stessa. Se lo supera, l'incantesimo ha fine.

\medskip\textbf{Dardo Incantato}\index{Incantesimi - Dardo Incantato}\\
\textbf{Scuola}: Invocazione\\
\textbf{Difficoltà}: 16\\
\textbf{Tempo di Lancio}: 2 Azioni\\
\textbf{Gittata}: 36 metri\\
\textbf{Componenti}: V, S\\
\textbf{Durata}: 1 Turno\\
Crei un dardo luminoso di forza magica. Lanciare uno o più dardi gia' evocati costa 1 Azione e puo' cumularsi con il lancio di incantesimi. Il dardo colpisce una creatura a gittata che puoi vedere, scelta da te. Un dardo infligge 1d4 + 1 danni da forza al suo bersaglio e li puoi dirigere perché colpiscano una o più creature.\\
Crei un dardo aggiuntivo quando raggiungi CM 3, CM 5 e CM 7\\.
\textbf{Per ogni Critico ottenuto} nella prova di magia l'incantesimo crea un dardo aggiuntivo

\medskip\textbf{Deflagrazione Occulta}\index{Trucchetto - Deflagrazione Occulta}\\
\textbf{Scuola}: Invocazione\\
\textbf{Difficoltà}: 12\\
\textbf{Tempo di Lancio}: 2 Azioni\\
\textbf{Gittata}: 36 metri\\
\textbf{Componenti}: V, S\\
\textbf{Durata}: Istantanea\\
Un fascio di energia crepitante si dirige verso una creatura a gittata. Effettua un attacco a distanza con incantesimo contro il bersaglio. Se colpisci, il bersaglio subisce 1d10 danni da forza.\\
Il danno dell'incantesimo aumenta di 1d8 quando raggiungi CM 5, CM 11 e CM 17.

\medskip\textbf{Desiderio}\index{Incantesimi - Desiderio}\\
\textbf{Scuola}: Evocazione\\
\textbf{Difficoltà}: 36\\
\textbf{Tempo di Lancio}: 2 Azioni\\
\textbf{Gittata}: Personale\\
\textbf{Componenti}: V\\
\textbf{Durata}: Istantanea\\
Desiderio è il più potente incantesimo che una creatura mortale possa lanciare. Semplicemente parlando ad alta voce, puoi modificare le stesse fondamenta della realtà a seconda dei tuoi bisogni. \\
L'uso basilare di questo incantesimo è quello di riprodurre l'effetto di qualsiasi altro incantesimo con Difficoltà 28 o meno. Non devi soddisfare nessuno dei requisiti dell'incantesimo, comprese le componenti materiali costose. L'incantesimo ha semplicemente effetto.\\
In alternativa, puoi creare uno dei seguenti effetti a tua scelta:
\begin{itemize}
	\item 
Crei un oggetto del valore massimo di 25.000 mo, che non sia un oggetto magico. L'oggetto non può avere dimensioni superiori ai 90 metri in qualsiasi dimensione, e compare in uno spazio non occupato sul terreno.
	\item 
Permetti fino a venti creature che puoi vedere di recuperare tutti i punti ferita, e termini tutti gli effetti su di loro descritti dall'incantesimo ristorare superiore. 
	\item 
Conferisci a un massimo di dieci creature che puoi vedere la resistenza a un tipo di danno a tua scelta.
	\item 
Conferisci a un massimo di dieci creature che puoi vedere l'immunità a un singolo incantesimo o altro effetto magico per 8 ore. Per esempio, potresti rendere te e tutti tuoi compagni immuni all'attacco risucchia vita del lich.
	\item 
Annulli un qualsiasi evento recente obbligando a ritirare qualsiasi tiro effettuato nell'ultimo round (compreso il tuo ultimo round). La realtà si rimodella per assecondare il nuovo risultato. Puoi far sì che il nuovo tiro abbia +2d6 o -2d6, puoi scegliere se usare il tiro originale o il nuovo tiro. Potresti anche riuscire a ottenere altro, oltre gli obiettivi negli esempi di cui sopra.\\
\end{itemize}
\medskip
Definisci i tuoi desideri quanto più possibile al Narratore. Il Narratore ha grande spazio di
manovra nel decidere cosa accada in questi casi; maggiore il desiderio, più grosse le probabilità che qualcosa vada storto. L'incantesimo potrebbe semplicemente fallire, l'effetto desiderato manifestarsi solo in parte, oppure potresti subire delle conseguenze impreviste, in base a come hai proferito il desiderio. Lo stress del lanciare questo incantesimo per creare qualsiasi effetto che non sia riprodurre un altro incantesimo ti indebolisce.\\
Dopo averne retto lo stress, ogni volta che lancerai un incantesimo, fino a che non avrai terminato una notte di riposo, subirai 1d10 danni da Vuoto per Difficoltà/3 dell'incantesimo. Questo danno non può essere ridotto o diminuito in alcun modo. Inoltre, la tua Costituzione scende a -3, se non è già a -3 o meno, per 2d4 giorni.\\
Per ciascun giorno che trascorri a riposare e non svolgere altro che un'attività leggera, il tuo tempo di recupero rimanente diminuisce di 2 giorni. Infine, c'è una probabilità del 33 percento che tu non sia mai più in grado di lanciare desiderio a causa dello stress sofferto per il lancio dell'incantesimo.

\medskip\textbf{Destriero Fantasma}\index{Incantesimi - Destriero Fantasma}\\
\textbf{Scuola}: Illusione\\
\textbf{Difficoltà}: 21\\
\textbf{Tempo di Lancio}: 1 minuto\\
\textbf{Gittata}: 9 metri\\
\textbf{Componenti}: V, S\\
\textbf{Durata}: 1 ora\\
Una creatura quasi reale simile a un cavallo di taglia Grande, appare sul terreno in uno spazio non occupato di tua scelta e a gittata. Decidi tu l'aspetto della creatura, e questa compare equipaggiata di sella, morso e briglia. Qualsiasi equipaggiamento creato dall'incantesimo svanisce in una nuvola di fumo se viene portato a più di 3 metri di distanza dal destriero. Per la durata, tu o una creatura di tua scelta potete cavalcare il destriero. La creatura usa le statistiche del cavallo da corsa, eccetto che ha velocità 30 metri e può percorrere 15 chilometri in un'ora, o 20 chilometri ad andatura veloce. Quando l'incantesimo termina, il destriero inizia gradualmente a svanire, dando al cavallerizzo 1 minuto per smontare di sella. L'incantesimo termina se usi un'azione per interromperlo o se il destriero subisce danni.

\medskip\textbf{Disco Fluttuante}\index{Incantesimi - Disco Fluttuante}\\
\textbf{Scuola}: Evocazione\\
\textbf{Difficoltà}: 16\\
\textbf{Tempo di Lancio}: 2 Azioni\\
\textbf{Gittata}: 9 metri\\
\textbf{Componenti}: V, S, M (una goccia di mercurio)\\
\textbf{Durata}: 1 ora\\
Questo incantesimo crea un piano di forza orizzontale, perfettamente circolare, di 1 metro di diametro e 2,5 centimetri di spessore che fluttua a 1 metro da terra, in uno spazio non occupato di tua scelta a gittata e che puoi vedere. Il disco rimane attivo per la durata, e può sostenere 250 chili. Se gli viene poggiato sopra un peso superiore, l'incantesimo termina e tutto quello che vi si trova sopra cade a terra. Finché ti trovi entro 6 metri da esso, il disco è immobile. Se ti muovi più di 6 metri lontano da esso, il disco ti segue in modo da rimanere sempre a 6 metri da te. Può muoversi attraverso terreno disomogeneo, su e giù per le scale, pendenze e simili, ma non può superare cambi di altitudine di 3 o più metri. Per esempio, il disco non può attraversare un fossato profondo 3 metri, né potrebbe lasciare il fossato se fosse creato in fondo a esso. Il disco puo' essere afferrato dall'incantatore e spostato manualmente. Se ti allontani più di 30 metri dal disco (di solito perché non riesce ad aggirare un ostacolo nel seguirti) l'incantesimo termina.

\medskip\textbf{Disintegrazione}\index{Incantesimi - Disintegrazione}\\
\textbf{Scuola}: Trasmutazione\\
\textbf{Difficoltà}: 29\\
\textbf{Tempo di Lancio}: 2 Azioni\\
\textbf{Gittata}: 18 metri\\
\textbf{Componenti}: V, S, M (una calamita e un pizzico di polvere)\\
\textbf{Durata}: Istantanea\\
Un sottile raggio verde parte dal tuo dito puntato verso un bersaglio a gittata e che puoi vedere. Il bersaglio può essere una creatura, un oggetto o una creazione di forza magica, come un muro creato da muro di forza. Una creatura bersaglio di questo incantesimo deve effettuare un Tiro Salvezza su Riflessi. Il bersaglio subisce 10d6 + 40 danni da forza se fallisce il Tiro Salvezza. Se questo danno riduce il bersaglio a 0 punti ferita, questi è disintegrato. Una creatura disintegrata e tutto quello che indossa e trasporta, eccetto gli oggetti magici, viene ridotta a una pila di sottile polvere grigia. La creatura può essere riportata in vita solo tramite l'intervento di un Patrono\\
Questo incantesimo disintegra automaticamente gli oggetti non magici o una creazione di forza magica di taglia Grande o più piccola. Se il bersaglio è un oggetto non magico o una creazione di forza di taglia Enorme o più grossa, questo incantesimo disintegra una porzione di essa pari a un cubo di 3 metri di spigolo. Gli oggetti magici ignorano questo incantesimo.\\
\textbf{Per ogni Critico ottenuto} nella prova di magia danno aumenta di 3d6.

\medskip\textbf{Dissolvi il Bene e il Male}\index{Incantesimi - Dissolvi il Bene e il Male}\\
\textbf{Scuola}: Abiurazione\\
\textbf{Difficoltà}: 26\\
\textbf{Tempo di Lancio}: 2 Azioni\\
\textbf{Gittata}: Personale\\
\textbf{Componenti}: V, S, M (Acqua Benedetta o argento e ferro in polvere)\\
\textbf{Durata}: Concentrazione, 1 minuto \\
Un'energia luminosa ti circonda e ti protegge da fatati, non morti e creature originarie di luoghi al di là del Piano Materiale. Per la durata, i celestiali, elementali, fatati, demoni e non morti hanno -1d6 ai Tiri per Colpire contro di te. Puoi terminare l'incantesimo anticipatamente usando una delle seguenti funzioni speciali.\\
\textit{Spezzare Ammaliamento}. Con un'azione, puoi entrare in contatto con una creatura affascinata, spaventata o posseduta da un celestiale, elementale, fatato, demoni o non morto. La creatura con cui sei in contatto non è più affascinata, spaventata o posseduta da queste creature.\\
\textit{Congedo}. Con un'azione, effettua un attacco da mischia contro un celestiale, elementale, fatato, demone o non morto nella tua portata. Se lo colpisci, puoi cercare di rimandare la creatura al suo piano di origine. La creatura deve superare un Tiro Salvezza su Volontà o venire rispedita sul suo piano nativo (se non vi si trova già). Se non si trovano sul loro piano nativo, i non morti vengono rispediti nel Mondo delle Ombre e i fatati nel Primo Mondo.

\medskip\textbf{Dissolvi Magie}\index{Incantesimi - Dissolvi Magie}\\
\textbf{Scuola}: Abiurazione\\
\textbf{Difficoltà}: 21\\
\textbf{Tempo di Lancio}: 2 Azioni\\
\textbf{Gittata}: 36 metri\\
\textbf{Componenti}: V, S\\
\textbf{Durata}: Istantanea\\
Scegli una creatura, oggetto o effetto magico a gittata. Qualsiasi incantesimo di Difficoltà 18 o più basso sul bersaglio ha fine. \\
\textbf{Per ogni Critico ottenuto} nella prova di magia la Difficoltà dispellabile aumenta di 2.

\medskip\textbf{Dito}\index{Incantesimi - Dito}\\
\textbf{Scuola}: Ammaliamento\\
\textbf{Difficoltà}: 12\\
\textbf{Tempo di Lancio}: 1 Azione Immediata\\
\textbf{Gittata}: 18 metri\\
\textbf{Componenti}: S\\
\textbf{Durata}: 3 round\\
Fai il dito (o pernacchia o gesto dell'ombrello) all'avversario che deve poterlo vedere (o sentire)\\
Questo deve fare un Tiro Salvezza su Volontà, se riesce non succede nulla.
Se fallisce il TS di 5 o piu' viene umiliato, per i prossimi 3 round ha una penalità di un 1d6 ai Tiri per Colpire, TS ed alle prove di Competenza.\\
Se fallisce il TS di 3 o 4, viene mortificato, per i prossimi 3 round ha una penalità di 1d6 ai Tiri per Colpire e Competenza.\\
Se fallisce il TS di 2 o 1, e' punito, per i prossimi 3 round ha una penalità di 2 ai Tiri per Colpire.\\
\textbf{Per ogni Critico ottenuto} nella prova di magia puoi influenzare una altra creatura che possa vedere il dito.\\

\medskip\textbf{Dito della Morte}\index{Incantesimi - Dito della Morte}\\
\textbf{Scuola}: Necromanzia\\
\textbf{Difficoltà}: 29\\
\textbf{Tempo di Lancio}: 2 Azioni\\
\textbf{Gittata}: 18 metri\\
\textbf{Componenti}: V, S\\
\textbf{Durata}: Istantanea\\
Invii una scarica di energia negativa a una creatura a gittata e che puoi vedere, provocandole profondo dolore. Il bersaglio deve effettuare un Tiro Salvezza su Tempra. Il bersaglio subisce 7d8 + 30 danni da Vuoto se fallisce il Tiro Salvezza, o la metà di questi danni se lo supera.\\
Un umanoide ucciso da questo incantesimo si rianima come zombi sotto il tuo comando permanente all'inizio del tuo prossimo round, e seguirà i tuoi ordini verbali al meglio delle sue capacità.

\medskip\textbf{Divinazione}\index{Incantesimi - Divinazione}\\
\textbf{Scuola}: Divinazione\\
\textbf{Difficoltà}: 29\\
\textbf{Tempo di Lancio}: 2 Azioni\\
\textbf{Gittata}: Personale\\
\textbf{Componenti}: V, S, M (incenso e un'offerta sacrificale appropriata alla tua religione, il cui valore complessivo sia di 25 mo, che saranno consumati dall'incantesimo)\\
\textbf{Durata}: Istantanea\\
La tua magia e un'offerta votiva ti mettono in comunicazione con un Patrono o il servitore di un Patrono. Gli puoi porre una singola domanda in merito a uno specifico obiettivo, evento o attività che debba verificarsi entro 7 giorni. Il Narratore dà una risposta veritiera. La replica potrebbe essere una breve frase, una rima criptica o un presagio. \\
L'incantesimo non tiene conto di ogni possibile circostanza che possa modificare il risultato, come il lancio di ulteriori incantesimi o la perdita o l'arrivo di un alleato.\\
Se lanci l'incantesimo due o più volte prima di aver terminato il giorno lungo, c'è una probabilità cumulativa del 25\% che per ogni lancio dopo il primo tu ottenga una lettura erronea. Il Narratore effettua questo tiro in segreto. 

\medskip\textbf{Dominare Bestie}\index{Incantesimi - Dominare Bestie}\\
\textbf{Scuola}: Ammaliamento\\
\textbf{Difficoltà}: 23\\
\textbf{Tempo di Lancio}: 2 Azioni\\
\textbf{Gittata}: 18 metri\\
\textbf{Componenti}: V, S\\
\textbf{Durata}: Concentrazione, massimo 1 minuto\\
Cerchi di affascinare una bestia a gittata che puoi vedere. Essa deve superare un Tiro Salvezza su Volontà o restare affascinata per la durata, ricevendo +1d6 al tiro se tu o i tuoi alleati la state combattendo.\\
Mentre la bestia è affascinata, finché voi due vi trovate sullo stesso piano di esistenza mantieni un collegamento telepatico con essa. Puoi usare questo collegamento telepatico per inviare comandi alla creatura mentre sei cosciente (richiede 1 azione), a cui essa obbedirà al suo meglio. Puoi specificare un corso d'azione semplice e generico, come "Attacca quella creatura", "Corri laggiù", o "Prendi quell'oggetto". Se la creatura completa l'ordine e non riceve ulteriori indicazioni da te, si difenderà e preserverà al meglio delle sue capacità.\\
Puoi impiegare 2 tue azioni per assumere il totale e preciso controllo del bersaglio. Fino al termine del tuo prossimo round, il bersaglio effettuerà solo le azioni decise da te, e non farà nulla che tu non gli permetta di fare. Durante questo periodo, puoi anche far usare una reazione al bersaglio, ma ciò richiede l'uso della tua reazione.\\
Ogni volta che il bersaglio subisce danni, effettua un nuovo Tiro Salvezza su Volontà contro l'incantesimo. Se supera il Tiro Salvezza, l'incantesimo termina.\\
\textbf{Per ogni Critico ottenuto} nella prova di magia la durata raddoppia fino ad un massimo di 8 ore.

\medskip\textbf{Dominare Mostri}\index{Incantesimi - Dominare Mostri}\\
\textbf{Scuola}: Ammaliamento\\
\textbf{Difficoltà}: 34\\
\textbf{Tempo di Lancio}: 2 Azioni\\
\textbf{Gittata}: 18 metri\\
\textbf{Componenti}: V, S\\
\textbf{Durata}: Concentrazione, massimo 1 ora\\
Cerchi di affascinare una creatura a gittata che puoi vedere. Essa deve superare un Tiro Salvezza su Volontà o restare affascinata per la durata, ricevendo +1d6 al tiro se tu o i tuoi alleati la state combattendo.\\
Mentre la creatura è affascinata, finché voi due vi trovate sullo stesso piano di esistenza mantieni un collegamento telepatico con essa. Puoi usare questo collegamento telepatico per inviare comandi alla creatura mentre sei cosciente (richiede 1 azione), a cui essa obbedirà al suo meglio. Puoi specificare un corso d'azione semplice e generico, come "Attacca quella creatura", "Corri laggiù", o "Prendi quell'oggetto". Se la creatura completa l'ordine e non riceve ulteriori indicazioni da te, si difenderà e preserverà al meglio delle sue capacità.\\
Puoi impiegare due tua Azioni per assumere il totale e preciso controllo del bersaglio. Fino al termine del tuo prossimo round la creatura effettuerà solo le azioni decise da te, e non farà nulla che tu non le permetta di fare. Durante questo periodo, puoi anche far usare una reazione alla creatura, ma ciò richiede l'uso della tua reazione. Ogni volta che il bersaglio subisce danni, effettua un nuovo Tiro Salvezza su Volontà contro l'incantesimo. Se supera il Tiro Salvezza, l'incantesimo termina.\\
\textbf{Per ogni Critico ottenuto} nella prova di magia la durata raddoppia fino ad un massimo di 8 ore.

\medskip\textbf{Dominare Persone}\index{Incantesimi - Dominare Persone}\\
\textbf{Scuola}: Ammaliamento\\
\textbf{Difficoltà}: 26\\
\textbf{Tempo di Lancio}: 2 Azioni\\
\textbf{Gittata}: 18 metri\\
\textbf{Componenti}: V, S\\
\textbf{Durata}: Concentrazione, massimo 1 minuto\\
Cerchi di affascinare un umanoide a gittata che puoi vedere. Esso deve superare un Tiro Salvezza su Volontà o restare affascinato per la durata, ricevendo +1d6 al tiro se tu o i tuoi alleati lo state combattendo.\\
Mentre il bersaglio è affascinato, finché voi due vi trovate sullo stesso piano di esistenza mantieni un collegamento telepatico con esso. Puoi usare questo collegamento telepatico per inviare comandi al bersaglio mentre sei cosciente (richiede 1 azione), a cui esso obbedirà al suo meglio. Puoi specificare un corso d'azione semplice e generico, come "Attacca quella creatura", "Corri laggiù", o "Prendi quell'oggetto". Se il bersaglio completa l'ordine e non riceve ulteriori indicazioni da te, si difenderà e preserverà al meglio delle sue capacità.\\
Puoi impiegare 2 Azioni per assumere il totale e preciso controllo del bersaglio. Fino al termine del tuo prossimo round, il bersaglio effettuerà solo le azioni decise da te, e non farà nulla che tu non gli permetta di fare. Durante questo periodo, puoi anche far usare una reazione al bersaglio, ma ciò richiede l'uso della tua reazione. Ogni volta che il bersaglio subisce danni, effettua un nuovo Tiro Salvezza su Volontà contro l'incantesimo. Se supera il Tiro Salvezza, l'incantesimo termina.\\
\textbf{Per ogni Critico ottenuto} nella prova di magia la durata raddoppia fino ad un massimo di 8 ore.

\medskip\textbf{Eroismo}\index{Incantesimi - Eroismo}\\
\textbf{Scuola}: Ammaliamento\\
\textbf{Difficoltà}: 16\\
\textbf{Tempo di Lancio}: 2 Azioni\\
\textbf{Gittata}: Contatto\\
\textbf{Componenti}: V, S\\
\textbf{Durata}: 1 minuto\\
Una creatura consenziente con cui sei in contatto vene infusa di coraggio. Fino al termine dell'incantesimo, la creatura è immune all'essere spaventata e, all'inizio di ciascun suo round, ottiene punti ferita temporanei pari al tuo valore di Intelligenza o modificatore da incantesimo. Quando l'incantesimo ha termine, il bersaglio perde tutti i punti ferita temporanei rimanenti derivati da questo incantesimo.

\medskip\textbf{Esilio}\index{Incantesimi - Esilio}\\
\textbf{Scuola}: Abiurazione\\
\textbf{Difficoltà}: 23\\
\textbf{Tempo di Lancio}: 2 Azioni\\
\textbf{Gittata}: 18 metri\\
\textbf{Componenti}: V, S, M (un oggetto disprezzato dal bersaglio)\\
\textbf{Durata}: 1 minuto\\
Cerchi di spedire una creatura a gittata e che puoi vedere in un altro piano di esistenza. Il bersaglio deve superare un Tiro Salvezza su Volontà o venire esiliato. Se il bersaglio è natio del piano di esistenza in cui ti trovi, esili il bersaglio in un semipiano innocuo. Mentre è lì, il bersaglio è inabile. Il bersaglio rimane lì fino al termine dell'incantesimo, quando riapparirà nello spazio che aveva lasciato o nello spazio non occupato più vicino, se il suo spazio originale adesso è occupato. Se il bersaglio è natio di un diverso piano di esistenza da quello in cui ti trovi, il bersaglio svanisce emettendo un lieve scoppio, tornando al suo piano natio. Se l'incantesimo termina prima che sia trascorso 1 minuto, il bersaglio riappare nello spazio che aveva lasciato o nello spazio non occupato più vicino, se il suo spazio originale è occupato.\\
\textbf{Per ogni Critico ottenuto} nella prova di magia puoi influenzare un altra creatura

\medskip\textbf{Esplosione Solare}\index{Incantesimi - Esplosione Solare}\\
\textbf{Scuola}: Invocazione\\
\textbf{Difficoltà}: 34\\
\textbf{Tempo di Lancio}: 2 Azioni\\
\textbf{Gittata}: 45 metri\\
\textbf{Componenti}: V, S, M (fuoco e un pezzo di pietra di sole)\\
\textbf{Durata}: Istantanea\\
Un'intensa luce solare illumina in un raggio di 18 metri centrato su di un punto a gittata, scelto da te. Tutte le creature all'interno della luce devono effettuare un Tiro Salvezza su Tempra. Se fallisce il Tiro Salvezza, una creatura subisce 12d6 danni da Luce e resta accecata per 1 minuto. Se lo supera, subisce la metà dei danni e non resta accecata dall'incantesimo. Non morti e melme hanno -2d6 a questo Tiro Salvezza. Una creatura accecata da questo incantesimo effettua un altro Tiro Salvezza su Tempra alla fine di ciascun suo round. Se supera il Tiro Salvezza, non è più accecata.\\
Nella sua area, questo incantesimo dissolve qualsiasi oscurità generata da un incantesimo. 

\medskip\textbf{Estasiare}\index{Incantesimi - Estasiare}\\
\textbf{Scuola}: Ammaliamento\\
\textbf{Difficoltà}: 19\\
\textbf{Tempo di Lancio}: 2 Azioni\\
\textbf{Gittata}: Personale\\
\textbf{Componenti}: V, S\\
\textbf{Durata}: 1 minuto\\
Intessi una serie di parole svianti, facendo sì che delle creature di tua scelta entro la gittata, che puoi vedere e possano sentirti, effettuino un Tiro Salvezza su Volontà. Qualsiasi creatura che non può restare affascinata supera il Tiro Salvezza automaticamente, e se tu o i tuoi compagni state combattendo una creatura, questa ha +1d6 al Tiro Salvezza. Se fallisce il Tiro Salvezza, il bersaglio ha -1d6 sulle prove di Consapevolezza effettuate per percepire una qualsiasi creatura diversa da te fino al termine dell'incantesimo o finché il bersaglio non può più sentirti.
L'incantesimo termina se sei reso inabile o non puoi più parlare.

\medskip\textbf{Evoca Animali}\index{Incantesimi - Evoca Animali}\\
\textbf{Scuola}: Evocazione\\
\textbf{Difficoltà}: 21\\
\textbf{Tempo di Lancio}: 2 Azioni\\
\textbf{Gittata}: 18 metri\\
\textbf{Componenti}: V, S\\
\textbf{Durata}: 1 ora\\
Evochi spiriti fatati che assumono l'aspetto di bestie e compaiono in spazi non occupati a gittata e che puoi vedere. Scegli una delle seguenti opzioni per determinare ciò che appare:
\begin{itemize}
\item
Una bestia di grado di sfida 2 o inferiore
\item
Due bestie di grado di sfida 1 o inferiore
\item
Quattro bestie di grado di sfida 1/2 o inferiore
\item
Otto bestie di grado di sfida 1/4 o inferiore
\end{itemize}
\medskip
Ogni bestia è considerata anche un fatato, e sparisce quando scende a 0 punti ferita o quando l'incantesimo termina. \\
Le creature evocate sono amichevoli verso di te e i tuoi compagni. Tirare l'iniziativa per le creature evocate come gruppo, che agisce durante il proprio round. Esse obbediscono a qualsiasi comando verbale che gli viene dato (senza bisogno che tu compia azioni). Se non dai comandi alle bestie, si difenderanno dalle creature ostili, ma non compiranno altre azioni.\\
\textbf{Per ogni Critico ottenuto} nella prova di magia appariranno due bestie in piu'

\medskip\textbf{Evoca Creature Boschive}\index{Incantesimi - Evoca Creature Boschive}\\
\textbf{Scuola}: Evocazione\\
\textbf{Difficoltà}: 23\\
\textbf{Tempo di Lancio}: 2 Azioni\\
\textbf{Gittata}: 18 metri\\
\textbf{Componenti}: V, S, M (una bacca di agrifoglio per creature convocata)\\
\textbf{Durata}: 1 ora \\
Evochi spiriti fatati che compaiono in spazi non occupati a gittata e che puoi vedere. Scegli una delle seguenti opzioni per determinare ciò che appare:
\begin{itemize}
\item Un fatato di grado di sfida 2 o inferiore
\item Due fatati di grado di sfida 1 o inferiore
\item Quattro fatati di grado di sfida 1/2 o inferiore
\item Otto fatati di grado di sfida 1/4 o inferiore
\end{itemize}
\medskip
Una creatura evocata sparisce quando scende a 0 punti ferita o quando l'incantesimo termina. Le creature evocate sono amichevoli verso di te e i tuoi compagni. Tirare l'iniziativa per le creature evocate come gruppo, che agisce durante il proprio round. Esse obbediscono a qualsiasi comando verbale che gli viene dato (senza bisogno che tu compia azioni). Se non dai comandi ai fatati, si difenderanno dalle creature ostili, ma non compiranno altre azioni.\\
\textbf{Per ogni Critico ottenuto} nella prova di magia appariranno due creature in piu'

\medskip\textbf{Evoca Elementale}\index{Incantesimi - Evoca Elementale}\\
\textbf{Scuola}: Evocazione\\
\textbf{Difficoltà}: 26\\
\textbf{Tempo di Lancio}: 1 minuto\\
\textbf{Gittata}: 27 metri\\
\textbf{Componenti}: V, S, M (incenso bruciato per l'aria, argilla malleabile per la terra, zolfo e fosforo per il fuoco, o acqua e sabbia per l'acqua) \\
\textbf{Durata}: 1 ora\\
Evochi un servitore elementale. Scegli un'area a gittata composta di acqua, aria, fuoco o terra e che riempia un cubo di 3 metri di spigolo. Un elementale di grado di sfida 5 o minore appropriato all'area da te scelta compare in uno spazio non occupato entro 3 metri da essa. L'elementale sparisce quando scende a 0 punti ferita o l'incantesimo termina.\\
L'elementale è amichevole verso di te e i tuoi compagni per la durata dell'incantesimo. Tira l'iniziativa per l'elementale, che agisce durante il proprio round. Obbedisce a qualsiasi comando verbale che gli viene dato (se il comando e' complesso consumi delle azioni). Se non dai comandi all'elementale, si difenderà dalle creature ostili, ma non compirà altre azioni.\\
\textbf{Per ogni Critico ottenuto} nella prova di magia il grado di sfida dell'elementale evocato aumenta di 1

\medskip\textbf{Evoca Elementali Minori}\index{Incantesimi - Evoca Elementali Minori}\\
\textbf{Scuola}: Evocazione\\
\textbf{Difficoltà}: 23\\
\textbf{Tempo di Lancio}: 1 minuto\\
\textbf{Gittata}: 27 metri\\
\textbf{Componenti}: V, S\\
\textbf{Durata}: 1 ora\\
Evochi degli elementali che compariranno in spazi non occupati a gittata e che puoi vedere. Scegli una della seguenti opzioni per decidere cosa appare:
\begin{itemize}
\item Un elementale di grado di sfida 2 o meno
\item Due elementali di grado di sfida 1 o meno
\item Quattro elementali di grado di sfida 1/2 o meno
\item Otto elementali di grado di sfida 1/4 o meno
\end{itemize}
\medskip
Un elementale evocato sparisce quando scende a 0 punti ferita o l'incantesimo termina. Un elementale evocato è amichevole verso di te e i tuoi compagni. Tirare l'iniziativa per gli elementali evocati come gruppo, che agisce durante il proprio round. Essi obbediscono a qualsiasi comando verbale che gli viene dato (se il comando e' complesso consumi delle azioni). Se non dai comandi agli elementali, si difenderanno dalle creature ostili, ma non compiranno altre azioni.\\
\textbf{Per ogni Critico ottenuto} nella prova di magia appariranno due Elementali in piu'.

\medskip\textbf{Evocazioni Istantanee}\index{Incantesimi - Evocazioni Istantanee}\\
\textbf{Scuola}: Evocazione\\
\textbf{Difficoltà}: 29\\
\textbf{Tempo di Lancio}: 1 minuto\\
\textbf{Gittata}: Contatto\\
\textbf{Componenti}: V, S, M (uno zaffiro del valore di 1.000 mo)\\
\textbf{Durata}: Fino a che dissolto \\
Entri a contatto con un oggetto del peso di 5 chili o meno e la cui dimensione più grossa non superi i 180 centimetri. L'incantesimo lascia un marchio sulla superficie dell'oggetto e ne incide invisibilmente il nome sullo zaffiro usato come componente materiale. Ogni volta che lanci questo incantesimo, devi usare uno zaffiro diverso.\\
In qualsiasi momento successivo, puoi usare 2 Azioni per pronunciare il nome dell'oggetto e frantumare lo zaffiro. L'oggetto appare istantaneamente nella tua mano quale che sia la distanza fisica o planare che vi separa, e l'incantesimo ha termine.\\
Se un'altra creatura sta impugnando o trasportando l'oggetto, frantumare lo zaffiro non trasporterà l'oggetto da te, ma invece apprenderai chi sia la creatura che ne è in possesso e indicativamente dove si trovi in questo momento.\\
Dissolvi magie, o un effetto simile applicato con successo allo zaffiro, termina l'effetto dell'incantesimo. 

\medskip\textbf{Fabbricare}\index{Incantesimi - Fabbricare}\\
\textbf{Scuola}: Trasmutazione\\
\textbf{Difficoltà}: 23\\
\textbf{Tempo di Lancio}: 10 minuti\\
\textbf{Gittata}: 36 metri\\
\textbf{Componenti}: V, S\\
\textbf{Durata}: Istantanea\\
Converti le materie prime in prodotti finiti dello stesso materiale. Per esempio, puoi fabbricare un piccolo ponte di legno da un cumulo di alberi, una corda da un mucchio di canapa, e abiti dal lino o la lana. Scegli le materie prima che puoi vedere a gittata. Puoi fabbricare un oggetto di taglia Grande o inferiore (contenuto in un cubo di 3 metri di spigolo, o otto cubi connessi di 1 metro di spigolo) data una sufficiente quantità di materie prime. Se stai lavorando con il metallo, la pietra o altre sostanze minerali, l'oggetto fabbricato non può essere più grande di taglia Media (contenuto in un singolo cubo di 1 metro di spigolo). La qualità degli oggetti creati da questo incantesimo è commisurata alla qualità delle materie prime.\\
Tramite questo incantesimo non si possono creare o trasmutare creature od oggetti magici. Inoltre non puoi usarlo per creare oggetti che normalmente richiedono un alto livello di lavorazione, come i gioielli, le armi, il vetro o le armature, a meno che tu non abbia la competenza con il tipo di strumenti da artigiano utilizzati per costruire questi oggetti.

\medskip\textbf{Faro di Speranza}\index{Incantesimi - Faro di Speranza}\\
\textbf{Scuola}: Abiurazione\\
\textbf{Difficoltà}: 21\\
\textbf{Tempo di Lancio}: 2 Azioni\\
\textbf{Gittata}: 9 metri\\
\textbf{Componenti}: V, S\\
\textbf{Durata}: 1 minuto, Concentrazione\\
Questo incantesimo conferisce speranza e vitalità. Scegli fino a 6 creature a gittata. Per la durata, ciascun bersaglio ha +1d6 ai Tiri Salvezza su Volontà e da ogni dado di cura ottiene +1 PF curato.

\medskip\textbf{Fatale}\index{Incantesimi - Fatale}\\
\textbf{Scuola}: Illusione\\
\textbf{Difficoltà}: 36\\
\textbf{Tempo di Lancio}: 2 Azioni\\
\textbf{Gittata}: 36 metri\\
\textbf{Componenti}: V, S\\
\textbf{Durata}: Concentrazione, massimo 1 minuto\\
Attingendo alle paure più intime di un gruppo di creature, crei delle creature illusorie nella loro mente, visibili solo a loro. Ogni creatura in una sfera di 9 metri di raggio centrata su di un punto a tua scelta nella gittata, deve effettuare un Tiro Salvezza su Volontà. Se fallisce il Tiro Salvezza, la creatura diventa spaventata per la durata. L'illusione affonda nelle paure più intime della creatura, manifestando i suoi incubi peggiori come una implacabile minaccia. Alla fine di ciascun round della creatura spaventata, questa deve superare un Tiro Salvezza su Volontà o subire 4d10 danni. Se supera il Tiro Salvezza, per quella creatura l'incantesimo ha termine.

\medskip\textbf{Favore Divino}\index{Incantesimi - Favore Divino}\\
\textbf{Scuola}: Invocazione\\
\textbf{Difficoltà}: 16\\
\textbf{Tempo di Lancio}: 1 Azione Immediata\\
\textbf{Gittata}: Personale\\
\textbf{Componenti}: V, S\\
\textbf{Durata}: 1 minuto\\
Le tue preghiere potenziano te e la tua arma. Fino al termine dell'incantesimo, quando colpisce, la tua arma infligge 1d4 danni da Luce aggiuntivi.

\medskip\textbf{Ferire}\index{Incantesimi - Ferire}\\
\textbf{Scuola}: Necromanzia\\
\textbf{Difficoltà}: 29\\
\textbf{Tempo di Lancio}: 2 Azioni\\
\textbf{Gittata}: 18 metri\\
\textbf{Componenti}: V, S\\
\textbf{Durata}: Istantanea\\
Scateni una malattia virulenta su di una creatura a gittata che puoi vedere. Il bersaglio deve effettuare un Tiro Salvezza su Tempra. Il bersaglio subisce 14d6 danni da Vuoto se fallisce il Tiro Salvezza, o la metà di questi danni se lo supera. il danno non può ridurre i punti ferita del bersaglio sotto l'1. Se il bersaglio fallisce il Tiro Salvezza, i suoi punti ferita massimi sono ridotti per 1 ora di un ammontare uguale al danno da Vuoto subito. Qualsiasi effetto che rimuova una malattia permette ai punti ferita massimi del personaggio di poter tornare al valore normale prima che trascorra quel tempo.

\medskip\textbf{Fermare il Tempo}\index{Incantesimi - Fermare il Tempo}\\
\textbf{Scuola}: Trasmutazione\\
\textbf{Difficoltà}: 36\\
\textbf{Tempo di Lancio}: 2 Azioni\\
\textbf{Gittata}: Personale\\
\textbf{Componenti}: V\\
\textbf{Durata}: Istantanea\\
Fermi brevemente il flusso del tempo per tutti, tranne che per te. Il tempo non scorre per le altre creature, mentre tu effettui 1d4 + 1 round di fila, durante i quali puoi effettuare azioni e muoverti come sempre. Questo incantesimo termina se una delle azioni che usi durante questo periodo, o qualsiasi effetto che crei durante questo periodo, ha effetto su di una creatura diversa da te o su di un oggetto indossato o trasportato da qualcuno che non sia tu. Inoltre, l'incantesimo termina se ti muovi in un posto lontano più di 300 metri da quello in cui lo hai lanciato.

\medskip\textbf{Fiamma Perenne}\index{Incantesimi - Fiamma Perenne}\\
\textbf{Scuola}: Invocazione\\
\textbf{Difficoltà}: 19\\
\textbf{Tempo di Lancio}: 2 Azioni\\
\textbf{Gittata}: Contatto\\
\textbf{Componenti}: V, S, M (polvere di rubino del valore di 50 mo, che l'incantesimo consuma)\\ \textbf{Durata}: Fino a che dissolto\\
Una luminosità simile a quella prodotta da una fiaccola si sprigiona da un oggetto con cui sei in contatto. L'effetto sembra quello di una normale fiamma, ma non produce calore né necessita ossigeno. Una fiamma perpetua può essere celata o nascosta ma non può essere smorzata né spenta.

\medskip\textbf{Fiamma Sacra}\index{Trucchetto - Fiamma Sacra}\\
\textbf{Scuola}: Invocazione\\
\textbf{Difficoltà}: 12\\
\textbf{Tempo di Lancio}: 2 Azioni\\
\textbf{Gittata}: 18 metri\\
\textbf{Componenti}: V, S\\
\textbf{Durata}: Istantanea\\
Una luminosità simile a quella prodotta da una fiaccola discende su di una creatura a gittata che puoi vedere. Il bersaglio deve superare un Tiro Salvezza su Riflessi o subire 1d8 danni da Luce. Il bersaglio non riceve il beneficio della copertura per questo Tiro Salvezza.\\
Il danno dell'incantesimo aumenta di 1d8 quando raggiungi CM 5, CM 11 e CM 17.

\medskip\textbf{Fiotto Acido}\index{Trucchetto - Fiotto Acido}\\
\textbf{Scuola}: Evocazione\\
\textbf{Difficoltà}: 12\\
\textbf{Tempo di Lancio}: 2 Azioni\\
\textbf{Gittata}: 18 metri\\
\textbf{Componenti}: V, S\\
\textbf{Durata}: Istantanea\\
Scagli una bolla di acido. Scegli una creatura a gittata o due creature a gittata che siano entro 1 metro l'una dall'altra. Il bersaglio deve superare un Tiro Salvezza su Riflessi o subire 1d6 danni da acido.\\
Il danno dell'incantesimo aumenta di 1d8 quando raggiungi CM 5, CM 11 e CM 17.

\medskip\textbf{Folata di Vento}\index{Incantesimi - Folata di Vento}\\
\textbf{Scuola}: Invocazione\\
\textbf{Difficoltà}: 19\\
\textbf{Tempo di Lancio}: 2 Azioni\\
\textbf{Gittata}: Personale (linea di 18 metri)\\
\textbf{Componenti}: V, S, M (un seme di legume)\\
\textbf{Durata}: Concentrazione, massimo 1 minuto\\
Una linea di forte vento lunga 18 metri e larga 3 metri esplode partendo da te in una direzione a tua scelta, per la durata dell'incantesimo. Ogni creatura che inizia il suo round dentro la linea deve superare un Tiro Salvezza su Tempra o venire spinta lontano da te di 4 metri, seguendo la direzione della linea.\\
Qualsiasi creatura sulla linea deve spendere il doppio del movimento per avvicinarsi a te.\\
La folata disperde gas o vapori, estingue candele, torce e simili fiamme non protette nell'area. Le fiamme protette, come quelle della lanterne, si agitano, e hanno una probabilità del 50\% di estinguersi. Come 1 Azione durante ciascun tuo round, prima del termine dell'incantesimo, puoi cambiare la direzione in cui la linea si proietta da te.\\
Un arma da lancio che attraversa una folata di vento ha il 50\% di mancare il bersaglio.

\medskip\textbf{Fondersi nella Pietra}\index{Incantesimi - Fondersi nella Pietra}\\
\textbf{Scuola}: Trasmutazione\\
\textbf{Difficoltà}: 21\\
\textbf{Tempo di Lancio}: 2 Azioni\\
\textbf{Gittata}: Contatto\\
\textbf{Componenti}: V, S\\
\textbf{Durata}: 8 ore\\
Entri in un oggetto o superficie di pietra grossi abbastanza da contenere tutto il tuo corpo, fondendoti con la pietra assieme a tutto l'equipaggiamento che trasporti per la durata. Usando il tuo movimento, entri nella pietra in un punto con cui sei in contatto. Non resta nulla della tua presenza che rimanga visibile o altrimenti possa essere individuato da sensi non magici. Mentre sei fuso con la pietra, non puoi vedere ciò che avviene all'esterno, e qualsiasi prova di Consapevolezza che effettui per ascoltare i suoni prodotti fuori da essa è fatta con -1d6. Resti consapevole del passare del tempo e puoi lanciare incantesimi su di te mentre sei fuso con la pietra. Puoi usare il tuo movimento per lasciare la pietra e ricomparire nel punto in cui vi sei entrato, terminando così l'incantesimo. Altrimenti non puoi muoverti.\\
I danni minori alla pietra non ti danneggiano, ma la sua parziale distruzione o cambio di forma (di modo che tu non vi entri più) ti espellono da essa e ti infliggono 6d6 danni da botta. La completa distruzione della pietra (o la sua trasmutazione in un'altra sostanza) ti fa espellere e ti infligge 50 danni da botta. Se vieni espulso, cadi prono in uno spazio non occupato, nel punto più vicino a quello in cui sei entrato nella pietra.

\medskip\textbf{Forma Eterea}\index{Incantesimi - Forma Eterea}\\
\textbf{Scuola}: Trasmutazione\\
\textbf{Difficoltà}: 31\\
\textbf{Tempo di Lancio}: 2 Azioni\\
\textbf{Gittata}: Personale\\
\textbf{Componenti}: V, S\\
\textbf{Durata}: Massimo 8 ore\\
Entri nelle regioni di confine del Piano Etereo, nell'area che si sovrappone al tuo piano attuale. Resti sul Confine Etereo per la durata o finché non usi un'azione per interrompere l'incantesimo. Se ti muovi verso l'alto o il basso, il costo del movimento è raddoppiato, se ti muovi invece orizzontalmente il movimento e' raddoppiato per azione di movimento. Puoi vedere e udire il piano da cui provieni, ma tutto quello che si trova lì ti appare grigio, e non puoi vedere a più di 18 metri di distanza.\\
Mentre sei sul Piano Etereo, può interagire solo con altre creature su quel piano. Le creature che non sono sul Piano Etereo non ti possono percepire né interagire con te, a meno che una capacità speciale o la magia gli fornisca la possibilità di farlo.\\
Ignori tutti gli oggetti e gli effetti che non sono sul Piano etereo, potendo così attraversare gli oggetti che percepisci sul piano da cui provieni. Quando l'incantesimo termina, ritorni immediatamente al piano da cui provieni nel punto che occupi attualmente. Se quando accade occupi lo stesso spazio di un oggetto solido o di una creatura, vieni immediatamente spostato nel più vicino spazio non occupato che puoi occupare e subisci 6 danni da forza per ogni metro di cui vieni spostato (o sua frazione). Questo incantesimo non ha effetto se lo esegui mentre sei già nel Piano Etereo o su di un piano che non vi confina, come uno dei Piani Esterni.\\
\textbf{Per ogni Critico ottenuto} nella prova di magia puoi portare con te un altra creatura.

\medskip\textbf{Forma Gassosa}\index{Incantesimi - Forma Gassosa}\\
\textbf{Scuola}: Trasmutazione\\
\textbf{Difficoltà}: 21\\
\textbf{Tempo di Lancio}: 2 Azioni\\
\textbf{Gittata}: Contatto\\
\textbf{Componenti}: V, S, M (un pezzo di garza e un filo di fumo)\\
\textbf{Durata}: Concentrazione, massimo 1 ora\\
Trasformi una creatura consenziente insieme a tutto ciò che sta indossando e trasportando, in una nube vaporosa per la durata. L'incantesimo termina se la creatura scende a 0 punti ferita. Le creature incorporee ignorano questo effetto. Mentre è in questa forma, l'unico metodo di movimento del bersaglio è una velocità di volo 3 metri. Il bersaglio può entrare e occupare lo spazio di un'altra creatura. Il bersaglio ha resistenza ai danni non magici, e ha +1d6 ai Tiri Salvezza su Tempra e Riflessi. Il bersaglio può attraversare piccoli buchi, strettoie, e anche semplici fori, sebbene consideri i liquidi come superfici solide. Il bersaglio non può cadere e resta fluttuante nell'aria anche se stordito o altrimenti reso inabile.\\
Mentre è nella forma di una nube vaporosa, il bersaglio non può parlare né manipolare oggetti, e qualsiasi oggetto stesse indossando o trasportando non può essere gettato, usato o altrimenti impiegato. Il bersaglio non può attaccare né lanciare incantesimi. 

\medskip\textbf{Forme Animali}\index{Incantesimi - Forme Animali}\\
\textbf{Scuola}: Trasmutazione\\
\textbf{Difficoltà}: 34\\
\textbf{Tempo di Lancio}: 2 Azioni\\
\textbf{Gittata}: 9 metri\\
\textbf{Componenti}: V, S\\
\textbf{Durata}: 24 ore\\
Trasformi magicamente altre creature in bestie. Scegli un qualsiasi numero di creature consenzienti a gittata e che puoi vedere. Trasformi ciascun bersaglio nella forma di una bestia di taglia Grande o minore con un grado di sfida 4 o inferiore. Nei turni successivi, puoi usare 2 Azioni per trasformare le creature soggette in nuove forme.\\
La trasformazione permane per ciascun bersaglio per la durata dell'incantesimo, o finché quel bersaglio scende a 0 punti ferita o muore. Puoi scegliere una forma diversa per ciascun bersaglio. Le statistiche di gioco del bersaglio sono rimpiazzate dalle statistiche della bestia scelta, a eccezione dei Tratti e dei punteggi di Intelligenza, Saggezza e Carisma che restano quelli del
bersaglio. Il bersaglio assume i punti ferita della sua nuova forma e, quando ritorna alla sua forma normale, ritorna al numero di punti ferita che aveva prima di trasformarsi. Se si ritrasforma perché è sceso a 0 punti ferita, il danno in eccesso viene applicato alla forma originale. Purché il danno in eccesso non riduca la forma normale della creatura a 0 punti ferita, essa non è priva di sensi. La creatura è limitata nelle azioni che può svolgere dalla natura della sua nuova forma, e non può parlare né lanciare incantesimi.\\
L'equipaggiamento del bersaglio si fonde nella nuova forma. Il bersaglio non può attivare, impugnare o in altro modo beneficiare del suo equipaggiamento.

\medskip\textbf{Frantumare}\index{Incantesimi - Frantumare}\\
\textbf{Scuola}: Invocazione\\
\textbf{Difficoltà}: 19\\
\textbf{Tempo di Lancio}: 2 Azioni\\
\textbf{Gittata}: 18 metri\\
\textbf{Componenti}: V, S, M (un frammento di metallo)\\
\textbf{Durata}: Istantanea\\
Un forte rombo, molto intenso, erutta da un punto a gittata di tua scelta. Ogni creatura in una sfera di 3 metri di raggio centrata su quel punto deve effettuare un Tiro Salvezza su Tempra. Una creatura subisce 3d8 danni da tuono se fallisce il Tiro Salvezza, o la metà di questi danni se lo supera. Una creatura composta di materiale inorganico, come pietra, cristallo o metallo, ha -1d6 sul Tiro Salvezza. Un oggetto non magico che non è indossato né trasportato subisce anch'esso danni se si trova nell'area dell'incantesimo.\\
\textbf{Per ogni Critico ottenuto} nella prova di magia il danno aumenta di 1d8.

\medskip\textbf{Freccia Acida}\index{Incantesimi - Freccia Acida}\\
\textbf{Scuola}: Invocazione\\
\textbf{Difficoltà}: 19\\
\textbf{Tempo di Lancio}: 2 Azioni\\
\textbf{Gittata}: 27 metri\\
\textbf{Componenti}: V, S, M (una foglia di rabarbaro in polvere e uno stomaco di pitone)\\
\textbf{Durata}: Istantanea\\
Una freccia verde luminosa saetta verso un bersaglio a gittata ed esplode con uno spruzzo d'acido. Effettua un attacco a distanza con incantesimo contro il bersaglio. Se colpisci, il bersaglio subisce immediatamente 4d4 danni da acido e 2d4 danni da acido al termine del suo prossimo round. Se manchi, la freccia spruzza il bersaglio di acido infliggendo la metà dei danni iniziali e non arrecando danni al termine del prossimo round del bersaglio.\\
\textbf{Per ogni Critico ottenuto} nella prova di magia il danno aumenta di 1d4.

\medskip\textbf{Fulmine}\index{Incantesimi - Fulmine}\\
\textbf{Scuola}: Invocazione\\
\textbf{Difficoltà}: 21\\
\textbf{Tempo di Lancio}: 2 Azioni\\
\textbf{Gittata}: Personale (linea di 30 metri)\\
\textbf{Componenti}: V, S, M (un pezzo di pelliccia e una verga d'ambra, cristallo o vetro)\\
\textbf{Durata}: Istantanea\\
Esplodi un fulmine che forma una linea lunga 30 metri e larga 1 metro che parte da dove ti trovi in una direzione scelta da te. Ogni creatura sulla linea deve superare un Tiro Salvezza su Riflessi. La creatura subisce 8d6 danni da fulmine se fallisce il Tiro Salvezza, o la metà di questi danni se lo supera.\\
Il fulmine incendia gli oggetti infiammabili nell'area che non sono indossati o trasportati.\\
Il fulmine se lanciato contro della pietra lavorata rimbalza con un angolo di 90 gradi rispetto all'incidenza di origine. Un fulmine lanciato in acqua crea una sfera di 3 metri di raggio di elettricità nel punto in cui entra.\\
\textbf{Per ogni Critico ottenuto} nella prova di magia il danno aumenta di 1d6.\\
\textbf{Successo/Fallimento Critico}: In caso si fallimento critico il danno raddoppia, in caso di successo critico il danno viene ulteriormente dimezzato

\medskip\textbf{Fuorviare}\index{Incantesimi - Fuorviare}\\
\textbf{Scuola}: Illusione\\
\textbf{Difficoltà}: 26\\
\textbf{Tempo di Lancio}: 2 Azioni\\
\textbf{Gittata}: Personale\\
\textbf{Componenti}: S\\
\textbf{Durata}: 1 ora\\
Diventi invisibile nello stesso momento in cui un tuo doppione illusorio compare nel posto in cui ti trovi. Il doppione resta per la durata dell'incantesimo, ma l'invisibilità termina se attacchi o lanci un incantesimo. Puoi usare 2 Azioni per far muovere il doppione illusorio fino al doppio della tua velocità e fargli compiere un gesto, parlare e comportarsi in qualsiasi maniera tu voglia.\\
Puoi vedere attraverso i suoi occhi e udire tramite le sue orecchie come se fossi nello spazio in cui si trova lui. Durante ciascun tuo round, con un'Azione, puoi passare dall'usare i suoi sensi all'usare i tuoi, o viceversa. Mentre stai usando i suoi sensi, sei accecato e assordato riguardo i tuoi dintorni. 

\medskip\textbf{Gabbia di Forza}\index{Incantesimi - Gabbia di Forza}\\
\textbf{Scuola}: Invocazione\\
\textbf{Difficoltà}: 28\
\textbf{Tempo di Lancio}: 2 Azioni\\
\textbf{Gittata}: 30 metri\\
\textbf{Componenti}: V, S, M (polvere di rubino del valore di 1.500 mo)\\
\textbf{Durata}: 1 ora\\
Una prigione cubica, immobile e invisibile, composta di forza magica compare intorno a un'area a gittata da te scelta. La prigione può essere una gabbia o una scatola solida, a tua scelta. Una prigione nella forma di una gabbia può avere 6 metri di lato ed essere composta da sbarre di 1,5 centimetri separate di 1,5 centimetri tra di loro. Una prigione a forma di scatola può avere 3 metri di lato, creando una barriera solida che impedisce a qualsiasi materia di attraversarla e bloccando qualsiasi incantesimo lanciato dall'interno o l'esterno dell'area. Quando lanci questo incantesimo, qualsiasi creatura che è completamente all'interno della gabbia, è intrappolata. Le creature solo parzialmente nell'area della gabbia, o quelle troppo grosse per entrarvi, vengono spinte via dal centro dell'area finché non ne sono completamente fuori.\\
Una creatura all'interno della gabbia non può lasciarla tramite mezzi non magici. Se la creatura prova a usare il teletrasporto o il viaggio interplanare per lasciare la gabbia, deve prima effettuare un Tiro Salvezza su Volontà. Se lo supera, la creatura può usare quella magia per sfuggire alla gabbia. Se lo fallisce, la creatura non può uscire dalla gabbia e spreca l'uso dell'incantesimo o dell'effetto. La gabbia si estende anche sul Piano Etereo, bloccando così il viaggio etereo.\\
Questo incantesimo non può essere dissolto da dissolvi magie.

\medskip\textbf{Giara Magica}\index{Incantesimi - Giara Magica}\\
\textbf{Scuola}: Necromanzia\\
\textbf{Difficoltà}: 29\\
\textbf{Tempo di Lancio}: 1 minuto\\
\textbf{Gittata}: Personale\\
\textbf{Componenti}: V, S, M (una gemma, cristallo, reliquario o qualche altro contenitore ornamentale del valore di almeno 500 mo)\\
\textbf{Durata}: Finché a che dissolto\\
Il tuo corpo entra in uno stato catatonico mentre la tua anima lo abbandona ed entra nel contenitore da te usato come componente materiale. Mentre la tua anima occupa il contenitore, sei consapevole dei tuoi dintorni come se fossi nello spazio del contenitore. Non puoi muoverti né usare reazioni. L'unica azione che puoi effettuare è quella di proiettare la tua anima fino a 30 metri di distanza, fuori dal contenitore, ritornando al tuo corpo vivente (e terminando l'incantesimo) o cercando di possedere un corpo umanoide.\\
Puoi tentare di possedere qualsiasi umanoide entro 30 metri da te e che tu possa vedere (le creature protette dagli incantesimi protezione dal bene e dal male o cerchio magico non possono essere possedute). Il bersaglio deve effettuare un Tiro Salvezza su Volontà e, se lo fallisce, la tua anima entra nel corpo del bersaglio, mentre l'anima del bersaglio resta intrappolata nel contenitore. Se lo supera, il bersaglio resiste ai tuoi tentativi di possederlo, e non puoi tentare di possederlo nuovamente prima che siano trascorse 24 ore.\\
Una volta che possiedi il corpo di una creatura, lo puoi controllare. Le tue statistiche di gioco sono rimpiazzate dalle statistiche della creatura, a eccezione dei tuoi Tratti e dei tuoi punteggi di Intelligenza, Saggezza e Carisma. Mantieni i benefici forniti dalle Abilità. Se il bersaglio possiede delle Abilità non puoi usarne nessuna.\\
Nel frattempo, l'anima della creatura posseduta può percepire i dintorni del contenitore usando i propri sensi, ma non può muoversi né effettuare alcuna azione.\\
Mentre possiedi un corpo, puoi usare 2 Azioni per ritornare dal corpo ospite al contenitore, se ti trovi entro 30 metri da esso, riportando l'anima della creatura ospite nel suo corpo. Se il corpo ospite muore mentre sei al suo interno, la creatura muore, e tu devi effettuare un Tiro Salvezza su Volontà contro la tua DC dei Tiri Salvezza degli incantesimi. Se lo superi, ritorni al contenitore, se si trova entro 30 metri da te. Altrimenti, morirai.\\
Se il contenitore viene distrutto o l'incantesimo termina, la tua anima ritorna immediatamente al tuo corpo. Se il tuo corpo è più di 30 metri lontano o se è morto mentre cerchi di farvi ritorno, morirà anche la tua anima. Se l'anima di un'altra creatura è nel contenitore quando viene distrutto, l'anima della creatura ritorna al suo corpo, se il corpo è vivo e si trova entro 30 metri. Altrimenti, la creatura muore. Quando l'incantesimo termina, il contenitore viene distrutto.

\medskip\textbf{Glifo di Interdizione}\index{Incantesimi - Glifo di Interdizione}\\
\textbf{Scuola}: Abiurazione\\
\textbf{Difficoltà}: 21\\
\textbf{Tempo di Lancio}: 2 Azioni\\
\textbf{Gittata}: Contatto\\
\textbf{Componenti}: V. S, M (incenso e diamante in polvere del valore di almeno 200 mo, che l'incantesimo consuma)\\
\textbf{Durata}: Fino a che dissolto o attivato \\
Quando lanci questo incantesimo, inscrivi un glifo che danneggia altre creature su di una superficie (come un tavolo o una sezione di pavimento o muro) o all'internodi un oggetto che può essere chiuso (come un libro, una pergamena o un forziere) per celare il glifo. Se scegli una superficie, il glifo può coprire un'area di superficie non maggiore di 3 metri di diametro. Se scegli un oggetto, quell'oggetto deve restare al suo posto; se l'oggetto viene spostato più di 3 metri dal punto in cui è stato lanciato l'incantesimo, il glifo è spezzato, e l'incantesimo termina senza essere stato attivato.\\
Il glifo è quasi invisibile e può essere trovato con una prova di Intelligenza (Indagare) contro la DC del Tiro Salvezza dei tuoi incantesimi. Decidi tu cosa attivi il glifo al momento del lancio dell'incantesimo.\\
Per i glifi inscritti su di una superficie, l'attivazione tipica comprende entrare in contatto o stare sopra il glifo, rimuovere un altro oggetto che copra il glifo, avvicinarsi a una certa distanza dal glifo, o manipolare l'oggetto su cui è inscritto il glifo. Per i glifi inscritti su di un oggetto, l'attivazione tipica comprende aprire l'oggetto, avvicinarsi a una certa distanza dall'oggetto, o vedere o leggere il glifo. Una volta che il glifo è stato attivato, l'incantesimo ha termine.\\
Puoi definire meglio l'attivazione così che l'incantesimo si attivi solo in determinate circostanze o secondo certe peculiarità fisiche (come l'altezza o il peso), specie di creatura (per esempio, l'interdizione potrebbe agire contro le aberrazioni o gli elfi oscuri), o specifici Tratti. Puoi anche predisporre condizioni per evitare che il glifo venga attivato, come la pronuncia di una parola d'ordine.\\
Quando inscrivi il glifo scegli rune esplosive o glifo incantesimo.
\medskip
\begin{itemize}
\item
\textit{Glifo Incantesimo}. Puoi inserire un incantesimo preparato di Difficoltà 18 o inferiore nel glifo lanciandolo come parte della creazione del glifo. L'incantesimo deve prendere come bersaglio una singola creatura o un'area. L'incantesimo che viene inserito non ha effetto immediato se lanciato in questo modo. Quando il glifo è attivato, l'incantesimo inserito viene lanciato. Se l'incantesimo ha un bersaglio, prende come bersaglio la creatura che ha attivato il glifo. Se l'incantesimo agisce su di un'area, l'area è incentrata su quella creatura. Se l'incantesimo evoca creature ostili o crea oggetti o trappole nocive, questi appaiono quanto più vicino possibile all'intruso e lo attaccano. Se l'incantesimo richiede concentrazione, questa è mantenuta fino al termine della sua normale durata.
\item
\textit{Rune Esplosive}. Quando attivato, il glifo erutta energia magica in una sfera di raggio 6 metri centrata sul glifo. La sfera si propaga intorno agli angoli. Ogni creatura nell'area deve effettuare un Tiro Salvezza su Riflessi. Una creatura subisce 5d8 danni da acido, fulmine, fuoco, freddo o tuono se fallisce il Tiro Salvezza (a tua scelta quando crei il glifo), o la metà di questi danni se supera il Tiro Salvezza.
\end{itemize}
\medskip
\textbf{Per ogni Critico ottenuto} nella prova di il danno del glifo rune esplosive aumenta di 1d8.

\medskip\textbf{Globo di Invulnerabilità}\index{Incantesimi - Globo di Invulnerabilità}\\
\textbf{Scuola}: Abiurazione\\
\textbf{Difficoltà}: 29\\
\textbf{Tempo di Lancio}: 2 Azioni\\
\textbf{Gittata}: Personale (raggio di 3 metri)\\
\textbf{Componenti}: V. S, M (una pallina di vetro o di cristallo che si frantuma quando l'incantesimo termina) \\
\textbf{Durata}: Concentrazione, massimo 1 minuto\\
Una barriera immobile e lievemente scintillante si erge in un raggio di 3 metri intorno a te e vi rimane per la durata.\\
Qualsiasi incantesimo di Difficoltà 23 (ad esclusione di risultati superiori grazie a critici) o più basso lanciato dall'esterno della barriera non può agire sulle creature o gli oggetti al suo interno. Questi incantesimi possono prendere come bersaglio creature e oggetti all'interno della barriera, ma non avranno effetto su di essi. Allo stesso modo, l'area all'interno della barriera viene esclusa dalle aree di effetto di questi incantesimi.\\
\textbf{Per ogni Critico ottenuto} nella prova di magia puoi bloccare un livello superiore di Difficoltà.

\medskip\textbf{Guarigione}\index{Incantesimi - Guarigione}\\
\textbf{Scuola}: Cura\\
\textbf{Difficoltà}: 29\\
\textbf{Tempo di Lancio}: 2 Azioni\\
\textbf{Gittata}: 18 metri\\
\textbf{Componenti}: V, S\\
\textbf{Durata}: Istantanea\\
Scegli una creatura a gittata e che puoi vedere. un'ondata di energia positiva travolge la creatura, facendole recuperare 70 punti ferita. L'incantesimo pone anche termine a qualsiasi cecità, sordità e malattia (anche magica) che affligga il bersaglio. Questo incantesimo causa 50 PF di danno ad un non morto.\\
\textbf{Per ogni Critico ottenuto} nella prova di magia l'ammontare guarito aumenta di 10.
Se incantatore e creatura curata sono entrambi Seguaci dello stesso Patrono l'incantesimo cura 90 PF.\\
Se incantatore e creatura curata sono entrambi Devoti dello stesso Patrono l'incantesimo riporta a pieno di PF.\\

\medskip\textbf{Guarigione di Massa}\index{Incantesimi - Guarigione di Massa}\\
\textbf{Scuola}: Cura\\
\textbf{Difficoltà}: 36\\
\textbf{Tempo di Lancio}: 2 Azioni\\
\textbf{Gittata}: 18 metri\\
\textbf{Componenti}: V, S\\
\textbf{Durata}: Istantanea\\
Un effluvio di energia guaritrice scorre da te verso le creature ferite che ti circondano. Ripristini fino a 700 punti ferita, divisi come preferisci tra qualsiasi creatura a gittata e che puoi vedere (con un massimo di 70 pf a creatura). Le creature guarite da questo incantesimo sono curate anche di tutte le malattie e da qualsiasi effetto che le renda accecate o assordate. Questo incantesimo puo' infliggere fino a 120 PF di danno ad un non morto. TS su Tempra per annullare l'effetto.
Se l'incantatore e creatura curata sono entrambi Seguaci dello stesso Patrono la cura assegnata aumenta di 20\%\\
Se l'incantatore e creatura curata sono entrambi Devoti dello stesso Patrono la cura assegnata aumenta di 50\%\\

\medskip\textbf{Guida}\index{Incantesimi - Guida}\\
\textbf{Scuola}: Divinazione\\
\textbf{Difficoltà}: 10\
\textbf{Tempo di Lancio}: 2 Azioni\\
\textbf{Gittata}: Contatto\\
\textbf{Componenti}: V, S\\
\textbf{Durata}: Concentrazione, massimo 1 minuto\\
Lanci l'incantesimo a contatto di una creatura consenziente. Una volta, prima che l'incantesimo termini, il bersaglio può tirare un d4 e sommare il risultato tirato a una prova di caratteristica a sua scelta. Può tirare il dado prima o dopo aver effettuato la prova di caratteristica. L'incantesimo ha poi termine. 

\medskip\textbf{Guscio Anti-Vita}\index{Incantesimi - Guscio Anti-Vita}\\
\textbf{Scuola}: Abiurazione\\
\textbf{Difficoltà}: 26\\
\textbf{Tempo di Lancio}: 2 Azioni\\
\textbf{Gittata}: Personale (raggio di 3 metri)\\
\textbf{Componenti}: V, S\\
\textbf{Durata}: Concentrazione, massimo 1 ora\\
Una barriera luminosa si estende fino a un raggio di 3 metri intorno a te, muovendosi con te e rimanendo centrata su di te, tenendo distanti le creature che non siano non morti o costrutti. La barriera permane per la durata. \\
La barriera impedisce a una creatura soggetta di attraversarla in alcun modo. Una creatura soggetta può lanciare incantesimi o effettuare attacchi con armi a distanza o con portata attraverso la barriera. Se ti muovi in modo che una creatura soggetta venga forzata ad attraversare la barriera, l'incantesimo termina.

\medskip\textbf{Identificare}\index{Incantesimi - Identificare}\\
\textbf{Scuola}: Divinazione\\
\textbf{Difficoltà}: 16\\
\textbf{Tempo di Lancio}: 1 minuto\\
\textbf{Gittata}: Contatto\\
\textbf{Componenti}: V, S, M (una perla del valore di almeno 100 mo e una piuma di gufo)\\ \textbf{Durata}: Istantanea\\
Scegli un oggetto con cui devi restare a contatto per tutto il lancio dell'incantesimo. Se si tratta di un oggetto magico o altro oggetto imbevuto di magia effettua una prova di Arcana a DC 25 con un +10 di bonus, se riesci ne apprendi le proprietà e come usarle e quante cariche abbia, se ne ha. \\
Apprendi se degli incantesimi stiano agendo sull'oggetto e cosa siano. Se l'oggetto è stato creato da un incantesimo, apprendi quale incantesimo lo abbia creato. Se invece durante l'esecuzione resti a contatto con una creatura, apprendi se degli incantesimi stiano agendo su di essa e quali siano.

\medskip\textbf{Illusione Minore}\index{Incantesimi - Illusione Minore}\\
\textbf{Scuola}: Illusione\\
\textbf{Difficoltà}: 10\
\textbf{Tempo di Lancio}: 2 Azioni\\
\textbf{Gittata}: 9 metri\\
\textbf{Componenti}: S, M (un pezzo di vello)\\
\textbf{Durata}: 1 minuto\\
Crei l'immagine di un oggetto o un suono a gittata per la durata dell'incantesimo. L'illusione ha termine se la interrompi con un'azione o lanci di nuovo questo incantesimo.\\
Se crei un suono, il suo volume può variare da quello di un bisbiglio a un urlo. Può essere la tua voce, la voce di qualcun altro, il ruggito di un leone, un battito di tamburi, o qualsiasi altro suono tu scelga. Il suono continua incessante per tutta la durata, oppure puoi produrre suoni diversi in momenti diversi prima del termine dell'incantesimo.\\
Se crei l'immagine di un oggetto (come una sedia, un'impronta fangosa o un piccolo forziere) non può essere più grande di un cubo di 1 metro di spigolo. L'immagine non può produrre suoni, luci, odori o qualsiasi altro effetto sensoriale. L'interazione fisica con l'oggetto lo rivela come illusione, perché le cose lo possono attraversare.\\
Una creatura che usa 3 Azioni per esaminare il suono o l'immagine può determinare che si tratta di un'illusione con una prova riuscita di Intelligenza (Indagare) contro la DC del Tiro Salvezza del tuo incantesimo. Se una creatura riconosce l'illusione per quello che è, per lei l'illusione sbiadisce. 

\medskip\textbf{Illusione Programmata}\index{Incantesimi - Illusione Programmata}\\
\textbf{Scuola}: Illusione\\
\textbf{Difficoltà}: 29\\
\textbf{Tempo di Lancio}: 2 Azioni\\
\textbf{Gittata}: 36 metri\\
\textbf{Componenti}: V, S, M (un pezzo di vello e polvere di giada del valore di almeno 25 mo)\\
\textbf{Durata}: Fino a che dissolto\\
Crei, a gittata, l'illusione di un oggetto, creatura o qualche altro fenomeno visibile che si attiva quando viene soddisfatta una specifica condizione. Fino ad allora l'illusione è impercettibile. Non può essere più grande di un cubo di 9 metri di spigolo, e decidi tu quando lanci l'incantesimo, come si comporti l'illusione e che suoni produca. L'esibizione programmata può durare fino a 5 minuti. Quando occorrono le condizioni da te specificate, l'illusione si manifesta e si comporta nel modo da te descritto. Una volta che l'illusione ha terminato la sua esibizione, scompare e rimane dormiente per 10 minuti. Dopo questo periodo, l'illusione può essere attivata di nuovo.\\
La condizione di attivazione può essere generica o dettagliata quanto vuoi, sebbene debba essere basata su condizioni visibili o udibili che avvengano entro 9 metri dall'area. Per esempio, potresti creare un'illusione di te stesso che appare e avverta chi tenti di aprire una porta munita di trappola, oppure potresti predisporre l'illusione perché si attivi solo quando una creatura pronunci la parola o la frase giusta.\\
L'interazione fisica con l'immagine la rivela come illusione, dato che le cose le passano attraverso. Una creatura che usi 3 Azioni per esaminare l'immagine può determinare che è un'illusione con una prova riuscita di Intelligenza (Indagare) contro la DC del Tiro Salvezza dell'incantesimo. Se una creatura riconosce l'illusione per quello che è, essa può vedere attraverso l'immagine, e qualsiasi suono prodotto dall'immagine le suona artefatto.

\medskip\textbf{Immagine Maggiore}\index{Incantesimi - Immagine Maggiore}\\
\textbf{Scuola}: Illusione\\
\textbf{Difficoltà}: 21\\
\textbf{Tempo di Lancio}: 2 Azioni\\
\textbf{Gittata}: 36 metri\\
\textbf{Componenti}: V, S, M (un pezzo di vello)\\
\textbf{Durata}: Concentrazione, massimo 10 minuti\\
Crei l'immagine di un oggetto, una creatura o qualche altro fenomeno visibile non più grande di un cubo di 6 metri di spigolo. L'immagine appare in un punto a gittata che puoi vedere e vi rimane per la durata dell'incantesimo. L'immagine sembra completamente reale, e comprende suoni, odori e la temperatura appropriata alla cosa raffigurata. Non puoi generare calore o freddo sufficiente a provocare danni, né un suono abbastanza forte da infliggere danno da tuono o assordare una creatura, o un odore che possa far star male una creatura (come il fetore di un troglodita). Finché resti a gittata dell'illusione, puoi usare un'azione per far muovere l'immagine in qualsiasi altro punto a gittata.\\
Quando l'immagine cambia posizione, puoi alterarne l'aspetto così che i suoi movimenti appaiano naturali. Per esempio, se crei l'immagine di una creatura e la muovi, puoi alterare l'immagine in modo che sembri camminare. Allo stesso modo, puoi impiegare l'illusione per produrre suoni diversi in momenti diversi, fino a farle portare avanti una conversazione.\\
L'interazione fisica con l'immagine la rivela come illusione, dato che le cose vi passano attraverso. Una creatura che usa 3 Azioni per esaminare l'immagine può determinare che si tratta di un'illusione con una prova riuscita di Intelligenza (Indagare) contro la DC del Tiro Salvezza del tuo incantesimo. Se una creatura riconosce l'illusione per quello che è, la creatura può vedervi attraverso, e per quella creatura tutte le altre qualità sensoriali svaniscono.\\
\textbf{Se ottieni un critico} l'incantesimo dura finché non viene dissolto, senza richiedere la tua concentrazione.
	
\medskip\textbf{Immagine Proiettata}\index{Incantesimi - Immagine Proiettata}\\
\textbf{Scuola}: Illusione\\
\textbf{Difficoltà}: 31\\
\textbf{Tempo di Lancio}: 2 Azioni\\
\textbf{Gittata}: 750 chilometri\\
\textbf{Componenti}: V, S, M (una tua piccola riproduzione fatta di materiali del valore almeno di 5 mo)\\
\textbf{Durata}: 1 giorno\\
Crei una copia illusoria di te stesso che permane per la durata. La copia può apparire in qualsiasi luogo entro la gittata che tu abbia già visto, ignorando qualsiasi ostacolo frapposto. L'illusione riproduce il tuo aspetto e i tuoi rumori ma è intangibile. Se l'illusione subisce danni, scompare, e l'incantesimo ha termine.\\
Puoi usare 2 Azioni per far muovere questa illusione fino al doppio della tua velocità e farle compiere un gesto, parlare e comportarsi in qualsiasi maniera tu voglia. Imita alla perfezione i tuoi comportamenti.\\
Puoi vedere attraverso i suoi occhi e udire tramite le sue orecchie come se fossi nello spazio in cui essa si trova. Durante ciascun tuo round, con un'Azione, puoi passare dall'usare i suoi sensi all'usare i tuoi, o viceversa. Mentre stai usando i suoi sensi, sei accecato e assordato riguardo i tuoi dintorni.\\
L'interazione fisica con l'immagine la rivela come illusione, dato che le cose le passano attraverso. Una creatura che usi 3 Azioni per esaminare l'immagine può determinare che è un'illusione con una prova riuscita di Consapevolezza contro la DC del Tiro Salvezza dell'incantesimo. Se una creatura riconosce l'illusione per quello che è, essa può vedere attraverso l'immagine, e qualsiasi suono prodotto dall'immagine le suona artefatto.

\medskip\textbf{Immagine Silenziosa}\index{Incantesimi - Immagine Silenziosa}\\
\textbf{Scuola}: Illusione\\
\textbf{Difficoltà}: 16\\
\textbf{Tempo di Lancio}: 2 Azioni\\
\textbf{Gittata}: 36 metri\\
\textbf{Componenti}: V, S, M (un pezzo di vello)\\
\textbf{Durata}: Concentrazione, massimo 10 minuti\\
Crei l'immagine di un oggetto, una creatura o qualche altro fenomeno visibile non più grande di un cubo di 4 metri di spigolo. L'immagine appare in un punto a gittata che puoi vedere e resta per la durata dell'incantesimo. L'immagine è puramente visiva; non è accompagnata da suoni, odori o altri effetti sensoriali. Puoi usare un'azione per far muovere l'immagine in qualsiasi altro punto a gittata. Quando l'immagine cambia posizione, puoi alterarne l'aspetto così che i suoi movimenti appaiano naturali. Per esempio, se crei l'immagine di una creatura e la muovi, puoi alterare l'immagine in modo che sembri camminare.\\
L'interazione fisica con l'immagine la rivela come illusione, dato che le cose vi passano attraverso. Una creatura che usa 3 Azioni per esaminare l'immagine può determinare che si tratta di un'illusione con una prova di Consapevolezza contro la DC del Tiro Salvezza del tuo incantesimo. Se una creatura riconosce l'illusione per quello che è, la creatura può vedervi attraverso.

\medskip\textbf{Immagine Speculare}\index{Incantesimi - Immagine Speculare}\\
\textbf{Scuola}: Illusione\\
\textbf{Difficoltà}: 19\\
\textbf{Tempo di Lancio}: 2 Azioni\\
\textbf{Gittata}: Personale\\
\textbf{Componenti}: V, S\\
\textbf{Durata}: 1 minuto\\
Nel tuo spazio compaiono 2d4 duplicati illusori di te stesso. Fino al termine dell'incantesimo, i duplicati si muovono con te e imitano le tue azioni, scambiandosi di posto in modo da rendere impossibile determinare quale sia l'immagine reale. Puoi usare 2 Azioni per congedare i duplicati illusori.\\
Ogni volta che una creatura ti prende in realtà ha colpito una immagine illusoria.
Se una creatura fa piu' attacchi a turno può disperdere una immagine per ogni attacco andato a buon fine. Se vieni colpito da un incantesimo ad area tutte le immagini svaniscono.\\
Una creatura che non può vedere, o si affida a sensi diversi dalla vista (come la vista cieca), o che può distinguere le illusioni come false (come la visione del vero), ignora gli effetti di questo incantesimo. 

\medskip\textbf{Imprigionare}\index{Incantesimi - Imprigionare}\\
\textbf{Scuola}: Abiurazione\\
\textbf{Difficoltà}: 36\\
\textbf{Tempo di Lancio}: 2 Azioni\\
\textbf{Gittata}: 9 metri\\
\textbf{Componenti}: V, S, M (una raffigurazione su vello o una statuetta incisa con le fattezze del bersaglio, e una componente speciale che varia a seconda della versione che scegli dell'incantesimo, del valore di almeno 500 mo per Dado Ferita del bersaglio)\\
\textbf{Durata}: Fino a dissolvimento\\
Crei dei vincoli magici per bloccare una creatura a gittata e che puoi vedere. Il bersaglio deve superare un Tiro Salvezza su Volontà o essere avvinto dall'incantesimo; se lo supera, è immune all'incantesimo qualora lo lanci di nuovo. Mentre è soggetta a questo incantesimo, la creatura non ha bisogno di respirare, mangiare o bere e non invecchia. Gli incantesimi di divinazione non possono localizzare né percepire il bersaglio.\\
Quando lanci questo incantesimo, scegli una delle seguenti forme di prigionia.
\medskip
\begin{itemize}
\item
\textit{Incatenamento}. Catene pesanti, ben saldate al terreno, tengono il bersaglio ancorato. Il bersaglio è intralciato fino al termine dell'incantesimo, e non può muoversi né essere mosso in alcun modo fino ad allora. La componente speciale per questa versione dell'incantesimo è una catenella di metallo prezioso. 
\item
\textit{Isolamento Minimo}. Il bersaglio rimpicciolisce fino a 2,5 centimetri di altezza ed è imprigionato in una gemma o simile oggetto. La luce può attraversare normalmente la gemma (permettendo al bersaglio di vedere all'esterno e ad altre creature di vedere dentro), ma null'altro può attraversarla, neppure tramite teletrasporto o viaggio planare. La gemma non può essere tagliata né infranta finché l'incantesimo rimane in atto. La componente speciale per questa versione dell'incantesimo è una grande gemma trasparente, come il corindone, il diamante o il rubino.
\item
\textit{Prigione Confinata}. L'incantesimo trasporta il bersaglio in un minuscolo semipiano interdetto al teletrasporto e al viaggio planare. Il semipiano può essere un labirinto, una gabbia, una torre, o qualsiasi altra struttura chiusa scelta da te. La componente speciale per questa versione dell'incantesimo è una rappresentazione in miniatura della prigione fatta di giada.
\item
\textit{Sepoltura}. Il bersaglio viene sepolto nelle profondità della terra in una sfera di forza magica grande a sufficienza da contenere il bersaglio. Nulla può attraversare la sfera, né alcuna creatura può teletrasportarsi o usare il viaggio planare per entrarvi o uscire. La componente speciale per questa versione dell'incantesimo è una piccola sfera di mithril. 
\item
\textit{Sopore}. Il bersaglio cade addormentato e non può essere risvegliato. La componente speciale per questa versione dell'incantesimo consiste di rare erbe soporifere.
\end{itemize}
\medskip
\textit{Terminare l'incantesimo}. Durante il lancio dell'incantesimo, in qualsiasi delle sue versioni, puoi specificare una condizione che possa porre fine all'incantesimo e liberare il bersaglio. La condizione può essere tanto specifica o elaborata quanto desideri, ma il Narratore deve concordare che la condizione sia ragionevole e possa avverarsi. Le condizioni possono essere basate sul nome, l'identità o il Patrono di una creatura, ma comunque basate su azioni o qualità percepibili e non su cose intangibili come il livello, le Abilità o i punti ferita.\\
Un incantesimo dissolvi magie può porre fine all'incantesimo solo se lanciato come incantesimo a Difficolà 30, che prenda come bersaglio la prigione o la componente materiale usata per crearla.\\
Puoi usare una particolare componente speciale per creare solo una prigione alla volta. Se lanci l'incantesimo di nuovo usando la stessa componente, il bersaglio del primo lancio dell'incantesimo viene immediatamente liberato dal suo vincolo.

\medskip\textbf{Inaridire}\index{Incantesimi - Inaridire}\\
\textbf{Scuola}: Necromanzia\\
\textbf{Difficoltà}: 23\\
\textbf{Tempo di Lancio}: 2 Azioni\\
\textbf{Gittata}: 9 metri\\
\textbf{Componenti}: V, S\\
\textbf{Durata}: Istantanea\\
Energia necromantica avvolge una creatura di tua scelta a gittata e che puoi vedere, deprivandola di linfa e vitalità. Il bersaglio deve effettuare un Tiro Salvezza su Tempra. Se fallisce il Tiro Salvezza, il bersaglio subisce 8d8 danni da Vuoto, o la metà di questi danni se supera il Tiro Salvezza. L'incantesimo non ha effetto su non morti o costrutti.\\
Se il bersaglio è un vegetale non magico che non sia anche una creatura, come un albero o un cespuglio, non effettua alcun Tiro Salvezza, avvizzisce e muore all'istante.\\
\textbf{Per ogni Critico ottenuto} nella prova di magia il danno aumenta di 1d8.

\medskip\textbf{Individuazione del Bene e del Male}\index{Incantesimi - Individuazione del Bene e del Male}\\
\textbf{Scuola}: Divinazione\\
\textbf{Difficoltà}: 16\\
\textbf{Tempo di Lancio}: 2 Azioni\\
\textbf{Gittata}: Personale\\
\textbf{Componenti}: V, S\\
\textbf{Durata}: 1 minuto\\
Per la durata, apprendi se entro 9 metri da te si trova un'aberrazione, celestiale, elementale, fatato, demone o non morto, e la sua posizione. Allo stesso modo, apprendi se entro 9 metri da te si trovi un luogo o oggetto che sia stato consacrato o dissacrato magicamente.\\
l'incantesimo può penetrare la maggior parte delle barriere, ma è bloccato da 30 centimetri di pietra, 2,5 centimetri di metallo comune, un sottile foglio di piombo o 1 metro di legno o terra. 
\textbf{Nota}: questo incantesimo non ha effetto sulle creature che seguono i Tratti.

\medskip\textbf{Individuazione del Magico}\index{Incantesimi - Individuazione del Magico}\\
\textbf{Scuola}: Divinazione\\
\textbf{Difficoltà}: 16\\
\textbf{Tempo di Lancio}: 2 Azioni\\
\textbf{Gittata}: Personale\\
\textbf{Componenti}: V, S\\
\textbf{Durata}: 1 minuto\\
Per la durata, percepisci la presenza della magia entro 9 metri da te. Puoi usare 1 Azione per vedere una flebile aura che si estende intorno a qualsiasi creatura o oggetto visibile nell'area che rechi magia. Con due Azioni ne apprendi anche la scuola di magia, se ce l'ha.\\
L'incantesimo può penetrare la maggior parte delle barriere, ma è bloccato da 30 centimetri di pietra, 2,5 centimetri di metallo comune, un sottile foglio di piombo o 1 metro di legno o terra.

\medskip\textbf{Individuazione delle Malattie e dei Veleni}\index{Incantesimi - Individuazione delle Malattie e dei Veleni}\\
\textbf{Scuola}: Divinazione\\
\textbf{Difficoltà}: 13\
\textbf{Tempo di Lancio}: 2 Azioni\\
\textbf{Gittata}: Personale\\
\textbf{Componenti}: V, S, M (una foglia di tasso)\\
\textbf{Durata}: 1 minuto\\
Per la durata, percepisci la presenza e posizione di veleni, creature velenose e malattie entro 9 metri da te. Inoltre riesci a identificare il tipo di veleno, creatura velenosa o malattia. L'incantesimo può penetrare la maggior parte delle barriere, ma è bloccato da 30 centimetri di pietra, 2,5 centimetri di metallo comune, un sottile foglio di piombo o 1 metro di legno o terra.

\medskip\textbf{Individuazione dei Pensieri}\index{Incantesimi - Individuazione dei Pensieri}\\
\textbf{Scuola}: Divinazione\\
\textbf{Difficoltà}: 19\\
\textbf{Tempo di Lancio}: 2 Azioni\\
\textbf{Gittata}: Personale\\
\textbf{Componenti}: V, S, M (un pezzo di rame)\\
\textbf{Durata}: 1 minuto\\
Per la durata, puoi leggere i pensieri di certe creature. Quando lanci questo incantesimo e con altre due Azioni in ciascun round successivo sino al termine dell'incantesimo, puoi concentrare la tua mente su qualsiasi creatura che tu possa vedere e si trovi entro 9 metri da te. Se la creatura che hai scelto ha un punteggio di Intelligenza -3 o meno o non parla nessun linguaggio, la creatura ignora l'effetto.\\
Inizialmente, apprendi solo i pensieri di superficie della creatura: quelli più ricorrenti. Con un'azione, puoi o spostare la tua attenzione sui pensieri di un'altra creatura o tentare di sondare più a fondo la mente della stessa creatura. Se sondi più a fondo, il bersaglio deve effettuare un Tiro Salvezza su Volontà. Se lo fallisce, ottieni una percezione dei suoi ragionamenti (se ve ne sono), del suo stato emotivo, e di ogni cosa abbia prevalenza nei suoi pensieri (come una preoccupazione, l'amore, o l'odio). Se supera il Tiro Salvezza, l'incantesimo termina. A ogni modo, il bersaglio sa che stai sondando la sua mente e, a meno che non sposti la tua attenzione verso la mente di un'altra creatura, nel suo round la creatura può usare la 2 Azioni per effettuare una prova di Intelligenza contesa dalla tua prova di Intelligenza; se la vince, l'incantesimo termina.\\
Le domande poste verbalmente alla creatura bersaglio, ovviamente, modellano il corso dei suoi pensieri, cosicché questo incantesimo risulta particolarmente efficace negli interrogatori.\\
Puoi anche usare questo incantesimo per individuare la presenza di creature pensanti che non puoi vedere. Quando lanci questo incantesimo o con 2 Azioni nella sua durata, puoi cercare pensieri entro 9 metri da te. L'incantesimo può penetrare le barriere, ma è bloccato da 60 centimetri di pietra, 5 centimetri di metallo che non sia il piombo, o un sottile foglio di piombo. Non puoi individuare una creatura con Intelligenza -3 o meno, o una creatura che non parla alcun linguaggio. Una volta individuata in questo modo la presenza di una creatura, puoi leggerne i pensieri per la durata dell'incantesimo finché resta nella gittata, come descritto sopra, anche se non puoi vederla.
Mentre hai attivo questo incantesimo per il lancio di altri incantesimo risulterai Distratto.

\medskip\textbf{Infliggi Ferite}\index{Incantesimi - Infliggi Ferite}\\
\textbf{Scuola}: Necromanzia\\
\textbf{Difficoltà}: 13 \\
\textbf{Tempo di Lancio}: 2 Azioni\\
\textbf{Gittata}: Contatto\\
\textbf{Componenti}: V, S\\
\textbf{Durata}: Istantanea\\
Effettua un attacco in mischia con incantesimo contro una creatura a portata. Se colpisci, il bersaglio subisce 3d10 danni da Vuoto.\\
\textbf{Per ogni Critico ottenuto} nella prova di magia il danno aumenta di 1d8.

\medskip\textbf{Ingrandire/Ridurre}\index{Incantesimi - Ingrandire/Ridurre}\\
\textbf{Scuola}: Trasmutazione\\
\textbf{Difficoltà}: 19\\
\textbf{Tempo di Lancio}: 2 Azioni\\
\textbf{Gittata}: 9 metri\\
\textbf{Componenti}: V, S, M (un pizzico di ferro in polvere)\\
\textbf{Durata}: 1 minuto\\
Fai sì che una creatura od oggetto a gittata e che puoi vedere ingrandisca o rimpicciolisca per la durata dell'incantesimo. Scegli una creatura o un oggetto che non sia né indossato né trasportato. Se il bersaglio non è consenziente, può effettuare un Tiro Salvezza su Tempra, se lo supera, l'incantesimo non ha effetto. Se il bersaglio è una creatura, tutto ciò che sta indossando e trasportando cambia taglia assieme a essa. Qualsiasi oggetto lasciato cadere da una creatura soggetta a questo incantesimo ritorna subito alla sua taglia normale.\\
\medskip
\begin{itemize}
\item
\textit{Ingrandire}. La taglia del bersaglio raddoppia in tutte le dimensioni, e il suo peso è moltiplicato per otto. Questa crescita aumenta la sua taglia di una categoria: da Media a Grande, per esempio. Se non c'è spazio sufficiente perché il bersaglio raddoppi la sua taglia, la creatura od oggetto assume la taglia più grossa possibile permessagli dallo spazio disponibile. Fino al termine dell'incantesimo, il bersaglio ha +1d6 alle prove di Forza e ai Tiri Salvezza su Tempra. Le armi del bersaglio crescono per raggiungere la nuova taglia. Mentre queste armi sono ingrandite, gli attacchi del bersaglio con esse faranno una categoria di danno ulteriore. 
\item
\textit{Ridurre}. La taglia del bersaglio si dimezza in tutte le dimensioni, e il suo peso è ridotto a un ottavo. Questa crescita diminuisce la sua taglia di una categoria: da Media a Piccola, per esempio. Fino al termine dell'incantesimo, il bersaglio ha -1d6 alle prove di Forza e ai Tiri Salvezza su Tempra. Le armi del bersaglio rimpiccioliscono per raggiungere la nuova taglia. Mentre queste armi sono rimpicciolite, gli attacchi del bersaglio con esse faranno una categoria di danno inferiore (ma senza ridurre il danno dell'arma a meno di 1).\\
\textbf{Per ogni due Critici ottenuti} nella prova di magia la creatura aumenta di un altra taglia.

\end{itemize}

\medskip\textbf{Insetto Gigante}\index{Incantesimi - Insetto Gigante}\\
\textbf{Scuola}: Trasmutazione\\
\textbf{Difficoltà}: 23\\
\textbf{Tempo di Lancio}: 2 Azioni\\
\textbf{Gittata}: 9 metri\\
\textbf{Componenti}: V, S\\
\textbf{Durata}: 10 minuti\\
Per la durata dell'incantesimo, trasformi fino a dieci centopiedi, tre ragni, cinque vespe o uno scorpione a gittata, in versioni giganti della loro forma naturale. Un centopiedi diventa un centopiedi gigante, un ragno diventa un ragno gigante, una vespa diventa una vespa gigante e uno scorpione diventa uno scorpione gigante. Ogni creatura obbedisce ai tuoi comandi vocali e, in combattimento, agisce in ciascun round durante il tuo round. Il Narratore possiede le statistiche di queste creature, e sarà sempre Il Narratore a risolvere le loro azioni e i loro movimenti. Una creatura resta nella sua forma gigante per la durata, finché non scende a 0 punti ferita, o finché non usi un'azione per interrompere l'effetto su di essa.\\
Il Narratore può permetterti di scegliere bersagli differenti. Per esempio, se trasformi un'ape, la sua versione gigante potrebbe avere le stesse statistiche della vespa gigante.

\medskip\textbf{Interdizione alla Morte}\index{Incantesimi - Interdizione alla Morte}\\
\textbf{Scuola}: Abiurazione\\
\textbf{Difficoltà}: 23\\
\textbf{Tempo di Lancio}: 2 Azioni\\
\textbf{Gittata}: Contatto\\
\textbf{Componenti}: V, S\\
\textbf{Durata}: 8 ore\\
Lanci l'incantesimo a contatto con una creatura. Conferisci al bersaglio protezione dalla morte. La prima volta che il bersaglio dovesse scendere a 0 punti ferita in seguito al danno subito, il bersaglio scende invece a 1 punto ferita e l'incantesimo ha fine. Se l'incantesimo è ancora attivo quando il bersaglio è vittima di un effetto che lo ucciderebbe all'istante senza infliggere danni,quell'effetto viene invece negato sul bersaglio e l'incantesimo ha fine.

\medskip\textbf{Intermittenza}\index{Incantesimi - Intermittenza}\\
\textbf{Scuola}: Trasmutazione\\
\textbf{Difficoltà}: 21\\
\textbf{Tempo di Lancio}: 2 Azioni\\
\textbf{Gittata}: Personale\\
\textbf{Componenti}: V, S\\
\textbf{Durata}: 1 minuto\\
Tira un 1d6 alla fine di ciascun tuo round per la durata di questo incantesimo. Se ottieni un numero dispari svanisci dal tuo attuale piano di esistenza e riappari sul Piano Etereo (l'incantesimo fallisce e il lancio è sprecato qualora tu fossi già su quel piano). All'inizio del tuo prossimo round, e quando l'incantesimo termina, qualora tu fossi sul Piano Etereo, ritorni in uno spazio non occupato di tua scelta e che puoi vedere, entro 3 metri dallo spazio da cui sei svanito. Se nessuno spazio non occupato è disponibile entro questa gittata, compari nello spazio non occupato più vicino (determinato casualmente se è disponibile più di uno spazio). Puoi interrompere l'incantesimo con un'azione.\\
Mentre sei sul Piano Etereo, puoi vedere e udire il piano da cui provieni, che percepisci in sfumature di grigio, ma non puoi comunque percepire nulla che si trovi a più di 18 metri di distanza. Puoi interagire solo con creature che si trovano sul Piano Etereo. Le creature che non si trovano lì non possono né percepirti né interagire con te, a meno che non siano provviste della capacità di farlo.

\medskip\textbf{Intimorire Infernale}\index{Incantesimi - Intimorire Infernale}\\
\textbf{Scuola}: Evocazione\\
\textbf{Difficoltà}: 16\\
\textbf{Tempo di Lancio}: 1 reazione, che puoi effettuare in risposta al danno arrecatoti da una creatura entro 18 metri da te che puoi vedere\\
\textbf{Gittata}: 18 metri\\
\textbf{Componenti}: V, S\\
\textbf{Durata}: Istantanea\\
Punti il dito, e la creatura che ti ha danneggiato viene momentaneamente avvolta da fiamme diaboliche. La creatura deve effettuare un Tiro Salvezza su Riflessi. Subisce 2d10 danni da fuoco se fallisce il Tiro Salvezza, o la metà di questi danni se lo supera.\\
\textbf{Per ogni Critico ottenuto} nella prova di magia i danno aumenta di 1d8

\medskip\textbf{Intralciare}\index{Incantesimi - Intralciare}\\
\textbf{Scuola}: Invocazione\\
\textbf{Difficoltà}: 16\\
\textbf{Tempo di Lancio}: 2 Azioni\\
\textbf{Gittata}: 27 metri\\
\textbf{Componenti}: V, S\\
\textbf{Durata}: 1 minuto\\
Rampicanti e rami stritolanti spuntano dal terreno in un quadrato di 6 metri di lato a partire da un punto a gittata. Per la durata, questi vegetali trasformano il terreno nell'area in terreno difficile.\\
Una creatura nell'area nel momento in cui lanci questo incantesimo deve superare un Tiro Salvezza su Tempra o restare intralciata da questi vegetali fino al termine dell'incantesimo. Una creatura intralciata dai vegetali può usare le sue azioni per effettuare una prova di Forza contro la DC del Tiro Salvezza dell'incantesimo. Se la supera, si libera. Quando l'incantesimo ha termine, i vegetali evocati svaniscono.

\medskip\textbf{Inversione della Gravità}\index{Incantesimi - Inversione della Gravità}\\
\textbf{Scuola}: Trasmutazione\\
\textbf{Difficoltà}: 31\\
\textbf{Tempo di Lancio}: 2 Azioni\\
\textbf{Gittata}: 30 metri\\
\textbf{Componenti}: V, S, M (una calamita e un fil di ferro)\\
\textbf{Durata}: Concentrazione, massimo 1 minuto 
Questo incantesimo inverte la gravità in un cilindro di raggio 15 metri, alto 30 metri, centrato in un punto a gittata. Quando lanci questo incantesimo, tutte le creature e gli oggetti che non sono in qualche modo ancorati al terreno cadono verso l'alto e raggiungono la cima dell'area. Una creatura può tentare un Tiro Salvezza su Riflessi per afferrare un oggetto fisso a portata, per evitare di cadere in questo modo, in caso lo superi.\\
Se lungo questa caduta si incontra un oggetto solido (il soffitto), gli oggetti e le creature che cadono vi impattano come accadrebbe durante una normale caduta. Se un oggetto o creatura raggiunge la cima dell'area senza colpire nulla, rimane lì, oscillando lievemente, per la durata.\\
Al termine della durata, gli oggetti e le creature colpite ricadono verso il basso.

\medskip\textbf{Inviare}\index{Incantesimi - Inviare}\\
\textbf{Scuola}: Invocazione\\
\textbf{Difficoltà}: 21\\
\textbf{Tempo di Lancio}: 2 Azioni\\
\textbf{Gittata}: Illimitata\\
\textbf{Componenti}: V, S, M (un piccolo pezzo di cavo di rame)\\
\textbf{Durata}: 1 round\\
Invii un breve messaggio di 25 parole o meno a una creatura con cui sei familiare. La creatura sente il messaggio nella sua mente, ti riconosce come mittente, e può risponderti in modo simile. L'incantesimo permette a creature con un punteggio di Intelligenza almeno di 1 di comprendere il significato del tuo messaggio anche se non comprende la tua lingua.\\
Puoi inviare il messaggio attraverso qualsiasi distanza e anche su altri piani di esistenza, ma se il bersaglio è su di un piano diverso dal tuo, c'è una probabilità del 5\% che il messaggio non arrivi.

\medskip\textbf{Invisibilità}\index{Incantesimi - Invisibilità}\\
\textbf{Scuola}: Illusione\\
\textbf{Difficoltà}: 19\\
\textbf{Tempo di Lancio}: 2 Azioni\\
\textbf{Gittata}: Contatto\\
\textbf{Componenti}: V, S, M (un ciglio avvolto nella gomma arabica)\\
\textbf{Durata}: 1 ora \\
Lanci l'incantesimo a contatto di una creatura. Il bersaglio diventa invisibile fino alla fine dell'incantesimo. Qualsiasi cosa il bersaglio stia indossando o trasportando diventa invisibile finché resta sul bersaglio. L'incantesimo ha fine per il bersaglio che attacca o esegue un incantesimo.\\
\textbf{Per ogni Critico ottenuto} nella prova di magia puoi scegliere un'ulteriore creatura bersaglio.

\medskip\textbf{Invisibilità Superiore}\index{Incantesimi - Invisibilità Superiore}\\
\textbf{Scuola}: Illusione\\
\textbf{Difficoltà}: 23\\
\textbf{Tempo di Lancio}: 2 Azioni\\
\textbf{Gittata}: Contatto\\
\textbf{Componenti}: V, S\\
\textbf{Durata}: 1 minuto\\
Lanci l'incantesimo a contatto di una creatura. Il bersaglio diventa invisibile fino alla fine dell'incantesimo. Qualsiasi cosa indossata o trasportata dal bersaglio diventa invisibile finché resta addosso al bersaglio.\\
Eseguire incantesimi o azioni di attacco non ti fa diventare visibile.

\medskip\textbf{Invocare il Fulmine}\index{Incantesimi - Invocare il Fulmine}\\
\textbf{Scuola}: Evocazione\\
\textbf{Difficoltà}: 21\\
\textbf{Tempo di Lancio}: 1 round\\
\textbf{Gittata}: 36 metri\\
\textbf{Componenti}: V, S\\
\textbf{Durata}: Concentrazione, massimo 10 minuti\\
Una nube di tempesta compare nella forma di un cilindro alto 3 metri con un raggio di 18 metri, centrato su di un punto che puoi vedere, 30 metri sopra di te. L'incantesimo fallisce automaticamente se non puoi vedere il punto nell'aria dove apparirà la nube di tempesta (per esempio, se sei in una stanza che non può accogliere la nube). Quando lanci l'incantesimo, scegli un punto che puoi vedere entro la gittata. Un fulmine si abbatterà dalla nuvola su quel punto. Ogni creatura entro 1 metro da quel punto deve effettuare un Tiro Salvezza su Riflessi. Una creatura subisce 3d10 danni da fulmine se fallisce il Tiro Salvezza, o la metà di questi danni se lo supera. Durante ciascun tuo round fino al termine dell'incantesimo, puoi usare due Azioni per richiamare un altro fulmine in questo modo, prendendo come bersaglio lo stesso punto o uno diverso.\\
Se quando lanci questo incantesimo ti trovi all'esterno in condizioni di tempesta, l'incantesimo ti fornisce il controllo della tempesta esistente invece di crearne una nuova. Sotto queste condizioni, il danno dell'incantesimo aumenta di 1d10. \\
\textbf{Per ogni Critico ottenuto} nella prova di magia il danno aumenta di 1d8

\medskip\textbf{Labirinto}\index{Incantesimi - Labirinto}\\
\textbf{Scuola}: Evocazione\\
\textbf{Difficoltà}: 34\\
\textbf{Tempo di Lancio}: 2 Azioni\\
\textbf{Gittata}: 18 metri\\
\textbf{Componenti}: V, S\\
\textbf{Durata}: Concentrazione, massimo 10 minuti\\
Bandisci una creatura a gittata e che puoi vedere in un semipiano labirintico. Il bersaglio rimane lì per la durata dell'incantesimo o finché non fugge dal labirinto. Il bersaglio può impiegare 3 Azioni per tentare di fuggire. Quando lo fa, effettua una prova di Intelligenza DC 25. Se la supera, fugge, e l'incantesimo termina (un minotauro o un demone goristro riescono automaticamente).\\
Quando l'incantesimo termina, il bersaglio riappare nello spazio che aveva lasciato o, se quello spazio è occupato, nel più vicino spazio non occupato. 

\medskip\textbf{Lama Infuocata}\index{Incantesimi - Lama Infuocata}\\
\textbf{Scuola}: Invocazione\\
\textbf{Difficoltà}: 19\\
\textbf{Tempo di Lancio}: 1 Azione Immediata\\
\textbf{Gittata}: Personale\\
\textbf{Componenti}: V, S, M (una foglia di sommacco)\\
\textbf{Durata}: Concentrazione, massimo 10 minuti \\
Crei nella tua mano una lama infuocata. La lama è simile in dimensioni e forma a una scimitarra, e rimane per la durata. Se lasci andare la lama, questa sparisce, ma ne puoi creare un'altra con un'Azione. Puoi usare 2 Azioni per effettuare un attacco in mischia con la lama infuocata. Se colpisci, il bersaglio subisce 3d6 danni da fuoco. La lama infuocata emana luce intensa in un raggio di 3 metri e luce fioca per ulteriori 3 metri.\\
\textbf{Per ogni due Critici ottenuti} nella prova di magia il danno aumenta di 1d6.

\medskip\textbf{Legame Telepatico}\index{Incantesimi - Legame Telepatico}\\
\textbf{Scuola}: Divinazione\\
\textbf{Difficoltà}: 26\\
\textbf{Tempo di Lancio}: 2 Azioni\\
\textbf{Gittata}: 9 metri\\
\textbf{Componenti}: V, S, M (pezzi di gusci d'uovo da due differenti specie di creature)\\
\textbf{Durata}: 1 ora\\
Stabilisci un collegamento telepatico tra un massimo di otto creature consenzienti a gittata di tua scelta, collegando psichicamente ciascuna creatura alle altre per la durata dell'incantesimo. Le creature con punteggio di Intelligenza -3 o meno ignorano questo incantesimo. Fino al termine dell'incantesimo, i bersagli possono comunicare telepaticamente tramite questo legame, che condividano o meno un linguaggio comune. La comunicazione è possibile a qualsiasi distanza, ma non può estendersi su differenti piani di esistenza.

\medskip\hypertarget{lentezza}{\textbf{Lentezza}}\index{Incantesimi - Lentezza}\\
\textbf{Scuola}: Trasmutazione\\
\textbf{Difficoltà}: 21\\
\textbf{Tempo di Lancio}: 2 Azioni\\
\textbf{Gittata}: 36 metri\\
\textbf{Componenti}: V, S, M (una goccia di melassa) \\
\textbf{Durata}: 1 minuto, Concentrazione\\
Modifichi lo scorrere del tempo intorno a un massimo di sei creature di tua scelta in un cubo di 12 metri di spigolo a gittata. Ciascun bersaglio deve superare un Tiro Salvezza su Volontà o subire gli effetti dell'incantesimo per la sua durata.\\
La velocità di un bersaglio soggetto all'incantesimo è dimezzata, questi subisce una penalità di -2 alla Difesa e ai Tiri Salvezza su Destrezza, e non può usare reazioni. Durante il suo round, può usare un'Azione o un'Azione Immediata, ma non entrambe. Quali che siano le capacità o gli oggetti magici della creatura, durante il suo round questa non può effettuare più di un attacco in mischia o a distanza.\\
Se la creatura tenta di lanciare un incantesimo con tempo di lancio di 2 azioni, tira un 1d6. Con 4 o più, l'incantesimo non avrà effetto fino al prossimo round della creatura, e la creatura dovrà usare 3 Azioni in quel round per completare l'incantesimo. Se non potrà farlo, l'incantesimo viene sprecato.\\
Una creatura sotto l'effetto di questo incantesimo effettua un altro Tiro Salvezza su Volontà al termine del suo round. Se supera questo Tiro Salvezza, l'effetto ha termine.\\

\medskip\textbf{Levitazione}\index{Incantesimi - Levitazione}\\
\textbf{Scuola}: Trasmutazione\\
\textbf{Difficoltà}: 19\\
\textbf{Tempo di Lancio}: 2 Azioni\\
\textbf{Gittata}: 18 metri\\
\textbf{Componenti}: V, S, M (o un piccolo laccio di cuoio oppure un pezzo di cavo d'oro piegato a forma di tazza con un lungo stelo alla fine)\\
\textbf{Durata}: 10 minuti \\
Una creatura o oggetto a gittata che puoi vedere, scelto da te, si alza verticalmente fino a 6 metri e rimane sospeso per la durata dell'incantesimo. L'incantesimo può levitare un bersaglio pesante fino a 250 chili. Una creatura non consenziente che superi un Tiro Salvezza su Tempra ignora l'effetto.\\
Il bersaglio può muoversi solo spingendo o tirando verso un oggetto fisso o superficie a portata (per esempio un muro o un soffitto). Durante il tuo round puoi cambiare l'altitudine del bersaglio fino a 6 metri in entrambe le direzioni. Se sei tu il bersaglio, ti puoi muovere verso l'alto o il basso come parte del tuo movimento. Altrimenti puoi usare 1 Azione per muovere il bersaglio, che deve rimanere nella gittata dell'incantesimo. Quando l'incantesimo termina, qualora sia ancora in aria, il bersaglio fluttua dolcemente a terra.\\
Mentre sei sotto l'influenza di questo incantesimo sei considerato Distratto nel lancio di incantesimi.

\medskip\textbf{Libertà di Movimento}\index{Incantesimi - Libertà di Movimento}\\
\textbf{Scuola}: Abiurazione\\
\textbf{Difficoltà}: 23\\
\textbf{Tempo di Lancio}: 2 Azioni\\
\textbf{Gittata}: Contatto\\
\textbf{Componenti}: V, S, M (una striscia di cuoio, avvolta intorno a un braccio o simile appendice)\\
\textbf{Durata}: 1 ora\\
Lanci l'incantesimo a contatto di una creatura consenziente. Per la sua durata, il movimento del bersaglio ignora il terreno difficile, mentre gli incantesimi o altri effetti magici non possono ridurre la sua velocità né far sì che il bersaglio sia paralizzato o intralciato.\\
Il bersaglio può usare due Azioni per liberarsi automaticamente da qualsiasi restrizione non magica, come manette o una creatura da cui è afferrato. Infine, trovarsi sott'acqua non comporta penalità al movimento o gli attacchi del bersaglio. 

\medskip\textbf{Lingue}\index{Incantesimi - Lingue}\\
\textbf{Scuola}: Divinazione\\
\textbf{Difficoltà}: 21\\
\textbf{Tempo di Lancio}: 2 Azioni\\
\textbf{Gittata}: Contatto\\
\textbf{Componenti}: V, M (un piccolo modello di argilla di una ziggurat)\\
\textbf{Durata}: 1 ora\\
Questo incantesimo conferisce alla creatura con cui sei stato in contatto al momento del lancio dell'incantesimo la capacità di comprendere qualsiasi linguaggio parlata che ode. Inoltre, quando il bersaglio parla, qualsiasi creatura che conosca almeno un linguaggio e può udire il bersaglio, comprende ciò che dice.

\medskip\textbf{Localizza Animali e Piante}\index{Incantesimi - Localizza Animali e Piante}\\
\textbf{Scuola}: Divinazione\\
\textbf{Difficoltà}: 19\\
\textbf{Tempo di Lancio}: 2 Azioni\\
\textbf{Gittata}: Personale\\
\textbf{Componenti}: V, S, M (un pezzo di pelo di un segugio) \\
\textbf{Durata}: Istantanea\\
Descrivi o nomina uno specifico tipo di bestia o vegetale. Concentrandoti sulla voce della natura nei tuoi dintorni, apprendi la direzione e la distanza dalla più vicina creatura o vegetale di quella specie, se ce ne sono entro 7,5 chilometri.

\medskip\textbf{Localizza Creatura}\index{Incantesimi - Localizza Creatura}\\
\textbf{Scuola}: Divinazione\\
\textbf{Difficoltà}: 23\\
\textbf{Tempo di Lancio}: 2 Azioni\\
\textbf{Gittata}: Personale\\
\textbf{Componenti}: V, S, M (un pezzo di pelliccia di segugio)\\
\textbf{Durata}: Concentrazione, massimo 1 ora\\
Descrivi o nomina una creatura che ti è familiare. Percepisci la direzione della posizione della creatura, purché quella creatura si trovi entro 300 metri da te. Se la creatura si muove, conosci anche la direzione del suo movimento.\\
L'incantesimo può localizzare una specifica creatura a te nota, o la più vicina creatura di una specie (come umano o unicorno), purché tu abbia visto una simile creatura da vicino (entro 9 metri) almeno una volta. Se la creatura che descrivi o nomini ha una forma diversa, per esempio è sotto gli effetti dell'incantesimo metamorfosi, questo incantesimo non sarà in grado di localizzare la creatura.\\
Questo incantesimo non può localizzare una creatura se un flusso di acqua corrente largo almeno 3 metri blocca un percorso diretto tra te e la creatura.

\medskip\textbf{Localizza Oggetto}\index{Incantesimi - Localizza Oggetto}\\
\textbf{Scuola}: Divinazione\\
\textbf{Difficoltà}: 19\\
\textbf{Tempo di Lancio}: 2 Azioni\\
\textbf{Gittata}: Personale\\
\textbf{Componenti}: V, S, M (un ramoscello biforcuto)\\
\textbf{Durata}: Concentrazione, massimo 10 minuti \\
Descrivi o nomina un oggetto che ti è familiare. Percepisci la direzione della posizione dell'oggetto, purché quell'oggetto si trovi entro 300 metri da te. Se l'oggetto si muove, conosci anche la direzione del suo movimento.\\
l'incantesimo può localizzare uno specifico oggetto a te noto, purché tu lo abbia visto da vicino (entro 9 metri) almeno una volta. In alternativa, l'incantesimo può localizzare l'oggetto più vicino di un particolare tipo, come certi tipi di abbigliamento, gioielleria, mobili, attrezzi o armi.\\
Questo incantesimo non può localizzare un oggetto se qualsiasi spessore di piombo, anche un foglio sottile, blocca un percorso diretto tra di te e l'oggetto. 

\medskip\textbf{Loquacità}\index{Incantesimi - Loquacità}\\
\textbf{Scuola}: Trasmutazione\\
\textbf{Difficoltà}: 34\\
\textbf{Tempo di Lancio}: 2 Azioni\\
\textbf{Gittata}: Personale\\
\textbf{Componenti}: V\\
\textbf{Durata}: 1 ora\\
Fino al termine dell'incantesimo, quando effettui una prova di Carisma puoi rimpiazzare il numero tirato con 15. Inoltre, non importa quello che dici, la magia o l'analisi che determina se stai dicendo la verità indicherà sempre che sei onesto.

\medskip\textbf{Luce}\index{Incantesimi - Luce}\\
\textbf{Scuola}: Invocazione\\
\textbf{Difficoltà}: 16\\
\textbf{Tempo di Lancio}: 2 Azioni\\
\textbf{Gittata}: Contatto\\
\textbf{Componenti}: V, M (una lucciola o del muschio fosforescente)\\
\textbf{Durata}: 1 ora +1 turno per Competenza Magica\\
Lanci l'incantesimo a contatto di un oggetto che non sia più grosso di 3 metri in qualsiasi direzione. Fino al termine dell'incantesimo, l'oggetto irradia una luce intensa in un raggio di 6 metri e penombra per ulteriori 6 metri. La luce può essere di qualsiasi colore tu voglia. Coprire completamente l'oggetto con qualcosa di opaco blocca la luce. Se un oggetto bersaglio è tenuto o indossato da una creatura ostile, quella creatura deve superare un Tiro Salvezza su Riflessi per evitare l'incantesimo. Una creatura colpita dall'incantesimo non si considera accecata.\\
\textbf{Per ogni due Critici ottenuti} nella prova di magia la durata aumenta di 1 ora.

\medskip\textbf{Luce Diurna}\index{Incantesimi - Luce Diurna}\\
\textbf{Scuola}: Invocazione\\
\textbf{Difficoltà}: 21\\
\textbf{Tempo di Lancio}: 2 Azioni\\
\textbf{Gittata}: 18 metri\\
\textbf{Componenti}: V, S\\
\textbf{Durata}: 1 ora\\
Una sfera di luce con raggio 18 metri si espande da un punto a tua scelta entro la gittata. La sfera irradia luce intensa e luca fioca per ulteriori 18 metri. Se scegli un punto su di un oggetto che stai reggendo o che non è indossato o trasportato, la luce si irradia dall'oggetto e si muove con esso. Coprire completamente un oggetto con qualcosa di opaco, come un vaso o un elmo, blocca la luce. Se qualsiasi parte dell'area di questo incantesimo si sovrappone con l'area di oscurità creata da un incantesimo di Difficoltà 18 o più basso, l'incantesimo che ha creato l'oscurità viene dissolto.\\
La luce creata si considera luce solare.

\medskip\textbf{Luci Danzanti}\index{Trucchetto - Luci Danzanti}\\
\textbf{Scuola}: Invocazione\\
\textbf{Difficoltà}: 12\\
\textbf{Tempo di Lancio}: 2 Azioni\\
\textbf{Gittata}: 36 metri\\
\textbf{Componenti}: V, S, M (un pezzo di fosforo o legno stregato, o un lombrico)\\
\textbf{Durata}: 1 minuto\\
Crei, a gittata, fino a quattro luci delle dimensioni di una torcia, facendole apparire come torce, lanterne o sfere luminose che fluttuano nell'aria per la durata dell'incantesimo. Puoi anche combinare le quattro luci in un'unica forma luminosa vagamente umanoide di taglia Media. Qualsiasi forma scegli, ciascuna luce emette una luce fioca in un raggio di 3 metri. Come 1 Azione di movimento durante il tuo round, puoi spostare le luci fino a 18 metri in un nuovo punto a gittata.\\
Una luce deve trovarsi entro 6 metri da un'altra luce creata con questo incantesimo, e le luci svaniscono se eccedono la gittata dell'incantesimo. 

\medskip\textbf{Luminescenza}\index{Incantesimi - Luminescenza}\\
\textbf{Scuola}: Invocazione\\
\textbf{Difficoltà}: 16\\
\textbf{Tempo di Lancio}: 2 Azioni\\
\textbf{Gittata}: 18 metri\\
\textbf{Componenti}: V\\
\textbf{Durata}: 1 minuto \\
Tutti gli oggetti in un cubo di 6 metri di spigolo a gittata vengono circondati da una luce blu, verde o viola (a tua scelta). Qualsiasi creatura nell'area quando l'incantesimo viene lanciato, viene anch'essa circondata dalla luce se fallisce un Tiro Salvezza su Riflessi. Per la durata dell'incantesimo, gli oggetti e le creature soggette emettono una luce fioca con raggio di 3 metri. Qualsiasi tiro per colpire contro una creatura od oggetto soggetto ha +1d6 se l'attaccante può vederlo, e la creatura od oggetto non può beneficiare dell'invisibilità.

\medskip\textbf{Mani Brucianti}\index{Incantesimi - Mani Brucianti}\\
\textbf{Scuola}: Invocazione\\
\textbf{Difficoltà}: 16\\
\textbf{Tempo di Lancio}: 2 Azioni\\
\textbf{Gittata}: Personale (cono di 4 metri)\\
\textbf{Componenti}: V, S\\
\textbf{Durata}: Istantanea\\
Mentre tieni le mani con i pollici che si toccano e le dita tese, un sottile fiotto di fiamme parte da ciascuna delle punta delle tue dita. Ogni creatura in un cono di 4 metri deve effettuare un Tiro Salvezza su Riflessi. Una creatura subisce 3d6 danni da fuoco se fallisce il Tiro Salvezza, o la metà se lo supera. Il fuoco incendia gli oggetti infiammabili nell'area che non siano indossati o trasportati.\\
\textbf{Per ogni Critico ottenuto} nella prova di magia il danno aumenta di 1d6

\medskip\textbf{Mano Arcana}\index{Incantesimi - Mano Arcana}\\
\textbf{Scuola}: Invocazione\\
\textbf{Difficoltà}: 26\\
\textbf{Tempo di Lancio}: 2 Azioni\\
\textbf{Gittata}: 36 metri\\
\textbf{Componenti}: V, S, M (un guscio d'uovo e un guanto di pelle di serpente)\\
\textbf{Durata}: Concentrazione, 1 minuto\\
Crei una mano Grande, composta di energia trasparente e luminosa, in uno spazio non occupato a gittata e che puoi vedere. La mano permane per la durata dell'incantesimo, e si muove al tuo comando, imitando i movimenti della tua mano.\\
La mano è un oggetto che ha Difesa 25 e punti ferita uguali ai tuoi punti ferita massimi. Ha Forza 8 e Destrezza 0 . La mano non riempie il suo spazio.
Quando lanci l'incantesimo e come 2 Azioni durante i tuoi round successivi, puoi muovere la mano fino a 18 metri e poi generare uno dei seguenti effetti. 
\medskip
\begin{itemize}
\item
\textit{Mano Afferrante}. La mano cerca di afferrare una creatura di taglia Enorme o più piccola che si trovi entro 1 metro da essa. Per risolvere l'azione di lottare usi la Forza della mano. Se il bersaglio è di taglia Media o inferiore, hai +1d6 alla prova. Mentre la mano tiene afferrato il bersaglio, puoi usare un'Azione per fare stritolare il bersaglio dalla mano. Quando lo fai, il bersaglio subisce danni da botta pari a 2d6 + il tuo valore di Intelligenza o Saggezza
\item
\textit{Mano di Forza}. La mano cerca di spingere una creatura di 1 metro in una direzione a tua scelta. Effettua una prova di Forza della mano contesa dalla prova di Forza del bersaglio. Se il bersaglio è di taglia Media o inferiore, hai +1d6 alla prova. Se vinci la contesa, la mano spinge il bersaglio di 1 metro più 1 metro moltiplicato per il valore di Intelligenza o Saggezza (minimo 1 metro). La mano si muove assieme al bersaglio per restare entro 1 metro da lui.\\
\item
\textit{Mano Frapposta}. La mano si frappone tra di te e una creatura di tua scelta finché non le dai un comando diverso. La mano si muove di modo da restare tra di te e il bersaglio, fornendoti metà copertura contro il bersaglio. Il bersaglio non può muoversi attraverso lo spazio della mano se il suo punteggio di Forza è uguale o inferiore al punteggio di Forza della mano. Se il suo punteggio di Forza è superiore al punteggio di Forza della mano, il bersaglio può muoversi attraverso lo spazio della mano, ma considera quello spazio come fosse terreno difficile. \\
\item
\textit{Pugno Serrato}. La mano colpisce una creatura o un oggetto entro 1 metro da essa. Effettua un attacco in mischia con incantesimo usando la mano, il TC e' basato sul CM e Destrezza. Se colpisci, il bersaglio subisce 4d8 danni da forza.
\end{itemize}
\medskip
\textbf{Per ogni Critico ottenuto} nella prova di magia il danno dell'opzione pugno serrato aumenta di 1d8 e il danno dell'opzione mano afferrante aumenta di 1d6.

\medskip\textbf{Mano Magica}\index{Trucchetto - Mano Magica}\\
\textbf{Scuola}: Evocazione\\
\textbf{Difficoltà}: 12\\
\textbf{Tempo di Lancio}: 2 Azioni\\
\textbf{Gittata}: 9 metri\\
\textbf{Componenti}: V, S\\
\textbf{Durata}: 1 minuto\\
Una mano spettrale fluttuante compare in un punto a gittata, scelto da te. La mano resta per la durata dell'incantesimo o finché non viene interrotta con un'azione. La mano svanisce se si dovesse trovare a più di 9 metri da te o se lanci nuovamente l'incantesimo.\\
Puoi usare 2 Azioni per controllare la mano. Puoi usare la mano per manipolare un oggetto, aprire una porta o un contenitore non chiusi a chiave, inserire o recuperare un oggetto da un contenitore aperto, o versare fuori i contenuti di una fiala. Puoi muovere la mano di 9 metri ogni volta che la usi.\\
La mano non può attaccare, attivare oggetti magici o trasportare più di 5 chili.

\medskip\textbf{Messaggio}\index{Trucchetto - Messaggio}\\
\textbf{Scuola}: Trasmutazione\\
\textbf{Difficoltà}: 12\\
\textbf{Tempo di Lancio}: 2 Azioni\\
\textbf{Gittata}: 36 metri\\
\textbf{Componenti}: V, S, M (un piccolo pezzo di cavo di rame)\\
\textbf{Durata}: 1 round\\
Punti il dito verso una creatura a gittata e sussurri un messaggio breve. Il bersaglio (e solo il bersaglio) ode il messaggio e può replicare con un sussurro che solo tu puoi udire.\\
Puoi lanciare questo incantesimo anche attraverso oggetti solidi, se sei familiare col bersaglio e sai che questi si trova dietro la barriera. Il silenzio magico, 30 centimetri di pietra, 2,5 centimetri di metallo normale, un sottile foglio di piombo o 1 metro di legno bloccano l'incantesimo. L'incantesimo non deve seguire una linea retta, e può liberamente aggirare gli angoli o attraversare gli spiragli.

\medskip\textbf{Metamorfosi}\index{Incantesimi - Metamorfosi}\\
\textbf{Scuola}: Trasmutazione\\
\textbf{Difficoltà}: 23\\
\textbf{Tempo di Lancio}: 2 Azioni\\
\textbf{Gittata}: 18 metri\\
\textbf{Componenti}: V, S, M (un bozzolo di bruco)\\
\textbf{Durata}: 1 ora \\
Questo incantesimo trasforma una creatura a gittata, che puoi vedere, in una nuova forma. Una creatura non consenziente deve superare un Tiro Salvezza su Volontà per evitare l'effetto. I mutaforma superano automaticamente il Tiro Salvezza. L'incantesimo non ha effetto su di un bersaglio con 0 punti ferita. \\
La trasformazione permane per la durata dell'incantesimo o finché il bersaglio non scende a 0 punti ferita o muore. La nuova forma può essere quella di qualsiasi bestia il cui grado di sfida sia uguale o più basso di quello del bersaglio (o del livello del bersaglio, se questi non ha un grado di sfida). Le statistiche di gioco del bersaglio, compresi i punteggi delle caratteristiche mentali, vengono rimpiazzate dalle statistiche della bestia scelta. Egli mantiene però i suoi Tratti e personalità.\\
Il bersaglio mantiene i medesimi punti ferita e ne recupera 1d12 punti ferita nella sua nuova forma. Quando ritorna alla sua forma normale, la creatura ritorna al numero di punti ferita che aveva prima di trasformarsi. Se però si ritrasforma perché ridotto a 0 punti ferita, qualsiasi danno in eccesso si ripercuote sulla sua normale forma. Purché il danno in eccesso non riduca la forma normale della creatura a 0 punti ferita, ella non cade priva di sensi.\\
La creatura è limitata nelle azioni che può svolgere dalla natura della sua nuova forma, e non può dialogare, lanciare incantesimi, o effettuare qualsiasi altra azione che richieda mani o di parlare. L'equipaggiamento del bersaglio si fonde nella nuova forma. La creatura non può attivare, usare, impugnare o beneficiare in alcun modo del suo equipaggiamento. 

\medskip\textbf{Metamorfosi Pura}\index{Incantesimi - Metamorfosi Pura}\\
\textbf{Scuola}: Trasmutazione\\
\textbf{Difficoltà}: 36\\
\textbf{Tempo di Lancio}: 2 Azioni\\
\textbf{Gittata}: 9 metri\\
\textbf{Componenti}: V, S, M (un goccio di mercurio, un mucchietto di gomma arabica, e uno sbuffo di fumo) \\
\textbf{Durata}: 1 ora \\
Scegli una creatura od oggetto non magico a gittata e che puoi vedere. L'incantesimo non ha effetto su di un bersaglio con 0 punti ferita. Trasformi la creatura in una creatura diversa, la creatura in un oggetto, o l'oggetto in una creatura (l'oggetto non deve essere indossato né trasportato da un'altra creatura). La trasformazione permane per la durata dell'incantesimo o finché il bersaglio non scende a 0 punti ferita o muore. Se ti concentri su questo incantesimo per l'intera durata, la trasformazione diventa permanente.\\
I mutaforma ignorano questo incantesimo. Una creatura non consenziente può effettuare un Tiro Salvezza su Volontà e, se lo supera, ignora l'effetto di questo incantesimo.\\
\medskip\begin{itemize}
\item
\textit{Creatura in Creatura}. Se trasformi una creatura in un'altra specie di creatura, la nuova forma può essere quella di qualsiasi specie tu voglia, il cui grado di sfida sia pari o inferiore a quello del bersaglio (o del suo livello, se il bersaglio non ha un grado di sfida). Le statistiche di gioco del bersaglio, compresi i punteggi delle caratteristiche mentali, vengono rimpiazzate dalle statistiche della nuova forma. Egli mantiene però il suoi Tratti e personalità. Il bersaglio mantiene i medesimi punti ferita e ne recupera 1d12 punti ferita nella sua nuova forma.\\
Quando ritorna alla sua forma normale, la creatura ritorna al numero di punti ferita che aveva prima di trasformarsi. Se però si ritrasforma perché ridotta a 0 punti ferita, qualsiasi danno in eccesso si ripercuote sulla sua normale forma. Purché il danno in eccesso non riduca la forma normale della creatura a 0 punti ferita, ella non cade priva di sensi. La creatura è limitata nelle azioni che può svolgere dalla natura della sua nuova forma, e non può dialogare, lanciare incantesimi, o effettuare qualsiasi altra azione che richieda mani o di parlare, a meno che la nuova forma non sia capace di svolgere queste azioni. L'equipaggiamento del bersaglio si fonde nella nuova forma. La creatura non può attivare, usare, impugnare o beneficiare in alcun modo del suo equipaggiamento. Oggetto in Creatura. Puoi trasformare un oggetto in un qualsiasi tipo di creatura, purché la taglia della creatura non sia maggiore della taglia dell'oggetto e il grado di sfida della creatura sia 9 o meno. La creatura è amichevole verso di te e i tuoi compagni. Essa agisce durante i tuoi turni. Decidi tu quali azioni essa compirà e come si muove. Il Narratore possiede le statistiche della creatura e risolverà tutte le sue azioni e i suoi movimenti.\\
Se l'incantesimo diventa permanente, perdi il controllo della creatura. A seconda di come l'hai trattata, potrebbe restare amichevole nei tuoi confronti.\\
\item
\textit{Creatura in Oggetto}. Se trasformi una creatura in un oggetto, essa si trasforma assieme a qualsiasi cosa stia indossando o trasportando. Le statistiche della creatura diventano quelle dell'oggetto, e, dopo che l'incantesimo termina e la creatura ritorna alla sua forma normale, questa non ha più ricordi del tempo trascorso in forma di oggetto.
\end{itemize}

\medskip\textbf{Miraggio Arcano}\index{Incantesimi - Miraggio Arcano}\\
\textbf{Scuola}: Illusione\\
\textbf{Difficoltà}: 31\\
\textbf{Tempo di Lancio}: 10 minuti\\
\textbf{Gittata}: Vista\\
\textbf{Componenti}: V, S\\
\textbf{Durata}: 10 giorni\\
Fai sì che un pezzo di terreno a gittata, in un'area quadrata fino a 1,5 chilometri, appaia, risuoni e odori come qualche altro tipo di terreno. La conformazione generale del terreno rimane tuttavia la stessa. Campi aperti o una strada possono essere trasformati in un acquitrino, colline, un crepaccio o qualche altro tipo di terreno difficile o invalicabile. Un laghetto può essere trasformato in una radura erbosa, un precipizio in una lieve pendenza, un burrone cosparso di rocce in una strada ampia e liscia.\\
Allo stesso modo, puoi modificare l'aspetto delle strutture, o aggiungerne dove non ve ne sono. L'incantesimo non camuffa, occulta né aggiunge creature.\\
L'illusione comprende elementi uditivi, visivi, tattili e olfattivi, così da poter trasformare un terreno sgombro in terreno difficile (o viceversa) o impedire altrimenti il movimento nell'area. Qualsiasi pezzo di terreno illusorio (come una pietra o un bastone), che venga rimosso dall'area dell'incantesimo, svanisce immediatamente. Le creature con visione del vero possono vedere oltre l'illusione e distinguere la vera forma del terreno; tuttavia, gli altri elementi dell'illusione rimangono, così, sebbene la creatura sia consapevole della presenza dell'illusione, vi può comunque interagire fisicamente. 

\medskip\textbf{Modificare Memoria}\index{Incantesimi - Modificare Memoria}\\
\textbf{Scuola}: Ammaliamento\\
\textbf{Difficoltà}: 26\\
\textbf{Tempo di Lancio}: 2 Azioni\\
\textbf{Gittata}: 9 metri\\
\textbf{Componenti}: V, S\\
\textbf{Durata}: Concentrazione, massimo 1 minuto\\
Tenti di rimodellare i ricordi di un'altra creatura. Una creatura che puoi vedere deve effettuare un Tiro Salvezza su Volontà. Se la stai combattendo, la creatura ha +1d6 sul Tiro Salvezza. Se fallisce il Tiro Salvezza, il bersaglio diventa affascinato da te per la durata dell'incantesimo. Il bersaglio affascinato è inabile e inconsapevole dell'ambiente circostante, sebbene sia ancora in grado di udirti. Se subisce danni o diviene bersaglio di un altro incantesimo, questo incantesimo termina, e nessuno dei ricordi del bersaglio viene modificato.\\
Mentre il bersaglio resta affascinato da questo incantesimo, puoi agire sui ricordi del bersaglio in merito a un evento che abbia vissuto nelle ultime 24 ore e che non sia durato più di 10 minuti. Puoi eliminare permanentemente tutti i ricordi dell'evento, permettere al bersaglio di ricordare l'evento con perfetta chiarezza e dettagli particolareggiati, modificare il ricordo dei dettagli dell'evento, o creare il ricordo di un altro evento. Devi poter parlare al bersaglio per descrivere il modo in cui i suoi ricordi saranno colpiti, e questi deve essere in grado di comprendere il tuo linguaggio, affinché i ricordi modificati si instaurino nella sua memoria. Se l'incantesimo termina prima che tu abbia finito di descrivere i ricordi modificati, la memoria della creatura non viene alterata. Altrimenti, i ricordi modificati si instaurano al termine dell'incantesimo.\\
Una memoria modificata non influisce necessariamente sul comportamento della creatura, in particolare se i suoi ricordi contraddicono le inclinazioni naturali, i Tratti o la fede della creatura. Una memoria modificata in modo illogico, come impiantare il ricordo di quanto la creatura ami immergersi nell'acido, viene rimossa, come fosse un brutto sogno. Il Narratore può giudicare un ricordo modificato troppo insensato perché abbia alcun effetto su di una creatura. Un incantesimo rimuovi maledizione o ristorare superiore lanciato sul bersaglio ne ripristina i veri ricordi.\\
\textbf{Per ogni critico ottenuto} nella prova di magia puoi alterare i ricordi di un bersaglio riguardo un evento svoltosi fino a 7 giorni prima, 30 giorni prima, 1 anno prima o qualsiasi punto nel passato della creatura.

\medskip\textbf{Movimenti del Ragno}\index{Incantesimi - Movimenti del Ragno}\\
\textbf{Scuola}: Trasmutazione\\
\textbf{Difficoltà}: 19\\
\textbf{Tempo di Lancio}: 2 Azioni\\
\textbf{Gittata}: Contatto\\
\textbf{Componenti}: V, S, M (una goccia di bitume e un ragno)\\
\textbf{Durata}: 10 minuti \\
Lanci l'incantesimo a contatto di una creatura consenziente. Fino al termine dell'incantesimo, la creatura ottiene la capacità di spostarsi verso l'alto, il basso e lungo superfici verticali o stando a testa in giù sul soffitto, tenendo le mani libere. Il bersaglio ottiene anche velocità di scalata pari alla sua velocità di passeggio.\\
La creatura soggetta all'incantesimo si considera Distratta nel lancio di altri incantesimi.\\

\medskip\textbf{Muovere il Terreno}\index{Incantesimi - Muovere il Terreno}\\
\textbf{Scuola}: Trasmutazione\\
\textbf{Difficoltà}: 29\\
\textbf{Tempo di Lancio}: 2 Azioni\\
\textbf{Gittata}: 36 metri\\
\textbf{Componenti}: V, S, M (un badile di ferro e un piccola borsa contenente un misto di tipi di terreno - argilla, concime e sabbia)\\
\textbf{Durata}: Concentrazione, massimo 2 ore\\
Scegli un'area sul terreno a gittata, non più grande di 12 metri di lato. Per la durata, puoi rimodellare terriccio, sabbia o argilla nell'area in qualsiasi modo tu voglia. Puoi innalzare o abbassare l'altitudine dell'area, creare o riempire un fossato, erigere o abbassare un muro, o formare un pilastro. La portata di questi cambiamenti non può eccedere metà della dimensione più grossa dell'area. Così, se operi su di un quadrato di 12 metri di lato, puoi creare un pilastro alto 6 metri, innalzare o abbassare l'altitudine del terreno di 6 metri, scavare un fossato profondo 6 metri, e così via. Ci vogliono 10 minuti per completare questi mutamenti. Al termine di ogni 10minuti trascorsi a concentrarsi sull'incantesimo, puoi scegliere una nuova area di terreno su cui operare.\\
Dato che la trasformazione del terreno avviene lentamente, le creature nell'area di solito non possono restare intrappolate o ferite dal movimento del terreno. L'incantesimo non può manipolare la pietra naturale o le costruzioni in pietra. Le rocce e le strutture si muovono per adattarsi al nuovo terreno. Se il modo in cui modelli il terreno renderebbe una struttura instabile, questa potrebbe crollare. Allo stesso modo, questo incantesimo non influenza direttamente la crescita dei vegetali. La terra smossa trasporta con sé qualsiasi vegetale presente.

\medskip\textbf{Muro di Forza}\index{Incantesimi - Muro di Forza}\\
\textbf{Scuola}: Invocazione\\
\textbf{Difficoltà}: 26\\
\textbf{Tempo di Lancio}: 2 Azioni\\
\textbf{Gittata}: 36 metri\\
\textbf{Componenti}: V, S, M (un pizzico di polvere prodotta frantumando una gemma trasparente)\\
\textbf{Durata}: 10 minuti\\
Un invisibile muro di forza si forma in un punto a gittata scelto da te. Il muro appare in qualsiasi orientamento da te desiderato, come una barriera orizzontale o verticale oppure angolata. Può fluttuare nell'aria o appoggiarsi su di una superficie solida. Puoi darle la forma di una cupola semisferica o di una sfera con un raggio massimo di 3 metri, oppure darle l'aspetto di una superficie piana composta da un massimo di dieci pannelli di 3 metri per 3 metri. Ogni pannello deve essere contiguo a un altro pannello. In qualsiasi forma, il muro ha uno spessore di 75 centimetri e resta per tutta la durata dell'incantesimo. Se il muro taglia uno spazio di una creatura, quando compare, la creatura viene spinta da un lato del muro (a tua discrezione). Nulla può attraversare fisicamente il muro. È immune a tutti i danni e non può essere dissolto da dissolvi magie. Tuttavia, il muro è distrutto all'istante dall'incantesimo disintegrazione. Il muro si estende anche sul Piano Etereo, impedendo ai viaggiatori eterei di attraversarlo.

\medskip\textbf{Muro di Fuoco}\index{Incantesimi - Muro di Fuoco}\\
\textbf{Scuola}: Invocazione\\
\textbf{Difficoltà}: 23\\
\textbf{Tempo di Lancio}: 2 Azioni\\
\textbf{Gittata}: 36 metri\\
\textbf{Componenti}: V, S, M (un piccolo pezzo di fosforo)\\
\textbf{Durata}: 1 minuto\\
Crei un muro di fuoco su di una superficie solida a gittata. Puoi creare un muro lungo fino a 18 metri, alto fino a 6 metri e spesso 30 centimetri, o un muro circolare di 6 metri di diametro, 6 metri di altezza e 30 centimetri di spessore. Il muro è opaco e rimane per la durata dell'incantesimo. \\
Quando il muro appare, ogni creatura nella sua area deve effettuare un Tiro Salvezza su Riflessi. Una creatura subisce 5d8 danni da fuoco se fallisce il Tiro Salvezza, o la metà se lo supera. Un lato del muro, selezionato da te quando lanci questo incantesimo, infligge 5d8 danni da fuoco a ciascuna creatura che termini il suo round entro 3 metri da quel lato o all'interno del muro. Una creatura subisce lo stesso danno quando entra nel muro per la prima volta durante un round. L'altro lato del muro non infligge danni.\\
\textbf{Per ogni critico ottenuto} nella prova di magia il danno aumenta di 1d8.

\medskip\textbf{Muro di Ghiaccio}\index{Incantesimi - Muro di Ghiaccio}\\
\textbf{Scuola}: Invocazione\\
\textbf{Difficoltà}: 29\\
\textbf{Tempo di Lancio}: 2 Azioni\\
\textbf{Gittata}: 36 metri\\
\textbf{Componenti}: V, S, M (un piccolo pezzo di quarzo)\\
\textbf{Durata}: 10 minuti\\
Crei un muro di ghiaccio su di una superficie solida a gittata. Puoi creare una cupola semisferica o una sfera con un raggio massimo di 3 metri, o puoi creare una superficie piana composta di un massimo di dieci panelli quadrati di 3 metri di lato. Ogni pannello deve essere contiguo ad almeno un altro pannello. In ogni forma, il muro è spesso 30 centimetri e rimane per la durata dell'incantesimo. \\
Se, quando compare, il muro attraversa lo spazio di una creatura, la creatura viene spinta da una parte del muro (a tua scelta) e deve effettuare un Tiro Salvezza su Riflessi. Se fallisce il Tiro Salvezza, la creatura subisce 10d6 danni da freddo, o la metà di questi danni se lo supera.\\
Il muro è un oggetto che può essere danneggiato e sfondato. Ogni sezione di 3 metri ha CA 12 e 30 punti ferita, ed è vulnerabile al danno da fuoco. Ridurre una sezione di 3 metri a 0 punti ferita la distrugge e lascia nello spazio che era occupato dal muro una brezza di vento gelido. Una creatura che si muova attraverso questa brezza di vento gelido per la prima volta in un round, deve effettuare un Tiro Salvezza su Tempra. Se lo fallisce, la creatura subisce 5d6 danni da freddo, o la metà di questi danni se lo supera.\\
\textbf{Per ogni critico ottenuto} nella prova di magia il danno aumentano di 1d8.

\medskip\textbf{Muro di Pietra}\index{Incantesimi - Muro di Pietra}\\
\textbf{Scuola}: Invocazione\\
\textbf{Difficoltà}: 26\\
\textbf{Tempo di Lancio}: 2 Azioni\\
\textbf{Gittata}: 36 metri\\
\textbf{Componenti}: V, S, M (un piccolo blocco di granito)\\
\textbf{Durata}: 10 minuti\\
Un muro di pietra solida non magico si forma in un punto a gittata, scelto da te. Il muro è spesso 15 centimetri ed è composto da 10 pannelli di 3 per 3 metri. Ogni pannello deve essere contiguo ad almeno un altro pannello. In alternativa, puoi creare pannelli 3 x 6 metri di soli 7,5 centimetri di spessore.\\
Se, quando compare, il muro attraversa lo spazio di una creatura, la creatura viene spinta da una parte del muro (a tua scelta). Se la creatura fosse circondata da tutte le parti dal muro (o dal muro e un'altra superficie solida), la creatura può effettuare un Tiro Salvezza su Riflessi. Se lo supera, può usare la sua reazione per muoversi della sua velocità in modo da non essere più intrappolata nel muro.\\
Il muro può aver qualsiasi forma tu desideri, sebbene non possa occupare lo stesso spazio di una creatura od oggetto. Il muro può anche non essere verticale o poggiare su di un piano. Deve, tuttavia, fondersi con ed essere sostenuto da pietra già esistente. Quindi, puoi usare questo incantesimo per creare un ponte su di un baratro o creare un rampa.\\
Se crei un muro non verticale del genere, più lungo di 6 metri, devi dimezzare le dimensioni di ciascun pannello per creare dei supporti. Puoi modellare rozzamente la pietra per creare merlature, spalti e così via. Il muro è un oggetto fatto di pietra che può essere danneggiato e sfondato. Ogni pannello ha Difesa CA 15, Durezza 15 e 15 punti ferita ogni 2,5 centimetri di spessore. Ridurre un pannello a 0 punti ferita lo distrugge e potrebbe far crollare i pannelli connessi, a discrezione del Narratore. Se mantieni la concentrazione su questo incantesimo per la sua intera durata, il muro diventa permanente e non può essere dissolto. Altrimenti, il muro sparisce al termine dell'incantesimo.

\medskip\textbf{Muro Prismatico}\index{Incantesimi - Muro Prismatico}\\
\textbf{Scuola}: Abiurazione\\
\textbf{Difficoltà}: 36\\
\textbf{Tempo di Lancio}: 2 Azioni\\
\textbf{Gittata}: 18 metri\\
\textbf{Componenti}: V, S\\
\textbf{Durata}: 10 minuti\\
Un piano di luci brillanti e multicolore forma un muro verticale opaco, largo fino a 27 metri, alto 9 metri e spesso 2,5 centimetri, centrato su di un punto a gittata e che puoi vedere. In alternativa, puoi modellare il muro in una sfera, fino a 9 metri di diametro, centrata su di un punto a gittata di tua scelta. Il muro resta fisso sul posto per la durata dell'incantesimo. Se posizioni il muro in modo che attraversi lo spazio occupato da una creatura, l'incantesimo fallisce e lo slot incantesimo sono sprecati. Il muro irradia luce intensa fino a una gittata di 30 metri e luce fioca per ulteriori 30 metri. Tu e le creature indicate da te al momento del lancio dell'incantesimo potete attraversare e restare vicini al muro senza pericolo. Se un'altra creatura che può vedere il muro si muove entro 6 metri da esso o inizia lì il suo round, deve superare un Tiro Salvezza su Tempra o restare accecata per 1 minuto. Il muro consiste di sette strati, ognuno di un diverso colore. Quando una creatura cerca di immergersi o attraversare il muro, lo fa uno strato alla volta, attraverso tutti gli strati del muro. Mentre si immerge oattraversa ciascuno strato, la creatura deve superare un Tiro Salvezza su Riflessi o subire le proprietà di ciascuno strato, uno alla volta, come descritto di seguito.\\
Il muro può essere distrutto, uno strato alla volta, in ordine dal rosso al violetto, in un modo specifico per ogni strato. Una volta che uno strato è distrutto, lo sarà per la durata dell'incantesimo. Una verga di cancellazione distrugge un muro prismatico, ma un campo anti-magia non ha effetto su di esso.
\medskip
\begin{itemize}
\item
\textit{1. Rosso}. Il bersaglio subisce 10d6 danni da fuoco se fallisce il Tiro Salvezza, o la metà di questi danni se lo supera. Finché questo strato esiste, gli attacchi a distanza non magici non possono attraversare il muro. Lo strato può essere distrutto infliggendogli 25 danni da freddo.
\item
\textit{2. Arancio}. Il bersaglio subisce 10d6 danni da acido se fallisce il Tiro Salvezza, o la metà di questi danni se lo supera. Finché questo strato esiste, gli attacchi a distanza magici non possono attraversare il muro. Lo strato può essere distrutto da un forte vento. 3. Giallo. Il bersaglio subisce 10d6 danni da fulmine se fallisce il Tiro Salvezza, o la metà di questi danni se lo supera. Questo strato può essere distrutto infliggendogli 60 danni di forza.
\item
\textit{4. Verde}. Il bersaglio subisce 10d6 danni da veleno se fallisce il Tiro Salvezza, o la metà di questi danni se lo supera. Un incantesimo passapareti, o un altro incantesimo di pari Difficoltà o più alto che può aprire un portale su di una superficie solida, distrugge questo strato.
\item
\textit{5. Blu}. Il bersaglio subisce 10d6 danni da freddo se fallisce il Tiro Salvezza, o la metà di questi danni se lo supera. Lo strato può essere distrutto infliggendogli almeno 25 danni da fuoco.
\item
\textit{6. Indaco}. Se fallisce il Tiro Salvezza, il bersaglio è intralciato. Deve poi effettuare un Tiro Salvezza su Tempra all'inizio di ciascun suo round. Se supera il Tiro Salvezza tre volte,l'incantesimo termina. Se fallisce il Tiro Salvezza tre volte, viene permanentemente trasformato in pietra e diventa vittima della condizione pietrificato. I successi e i fallimenti non devono essere consecutivi; tieni traccia di entrambi finché il bersaglio non ne ha ottenuti tre dello stesso tipo. Finché questo strato esiste, non si possono lanciare incantesimi attraverso il muro. Lo strato viene distrutto dalla luce intensa emanata dall'incantesimo luce diurna o da un simile incantesimo di Difficoltà più alta.
\item
\textit{7. Violetto}. Se fallisce il Tiro Salvezza, il bersaglio è accecato. Deve poi effettuare un Tiro Salvezza su Volontà all'inizio del tuo prossimo round. Se supera il Tiro Salvezza, la cecità termina. Se fallisce il Tiro Salvezza, la creatura viene trasportata su di un altro piano di esistenza a scelta del Narratore e non è più accecata (di solito, una creatura che non è sul suo piano natio, viene esiliata su di esso, mentre le altre creature sono di solito gettate nei pianiAstrale o Etereo). Questo strato è distrutto dall'incantesimo dissolvi magie o da un incantesimo simile di pari Difficoltà o più alto che possa porre fine a incantesimi ed effetti magici.
\end{itemize}

\medskip\textbf{Muro di Spine}\index{Incantesimi - Muro di Spine}\\
\textbf{Scuola}: Evocazione\\
\textbf{Difficoltà}: 29\\
\textbf{Tempo di Lancio}: 2 Azioni\\
\textbf{Gittata}: 36 metri\\
\textbf{Componenti}: V, S, M (una manciata di spine)\\
\textbf{Durata}: massimo 10 minuti\\
Crei un muro di cespugli robusti, malleabili e impigliati, ricolmi di spine appuntite. Il muro compare a gittata su di una superficie solida e rimane per la durata dell'incantesimo. Il muro che puoi creare può essere lungo fino a 18 metri, alto fino a 3 metri, e spesso fino a 1 metro o un circolo che abbia un diametro di 6 metri e sia alto fino a 6 metri e spesso 1 metro. Il muro blocca la linea di visuale.\\
Quando il muro compare, ogni creatura nella sua area deve effettuare un Tiro Salvezza su Riflessi. Se fallisce il Tiro Salvezza, una creatura subisce 7d8 danni perforanti, o la metà di questi danni se lo supera. Una creatura può muoversi attraverso il muro, seppure in maniera lenta e dolorosa. Per ogni 1 metro che la creatura si muove attraverso il muro, deve spendere 6 metri di movimento. Inoltre, la prima volta che una creatura entra nel muro durante un round o vi termina il suo round dentro, la creatura deve effettuare un Tiro Salvezza su Riflessi. Subisce 7d8 danni taglienti se fallisce il Tiro Salvezza, o la metà di questi danni se lo supera.\\
\textbf{Per ogni critico ottenuto} nella prova di magia il danno aumenta di 1d8.

\medskip\textbf{Muro di Vento}\index{Incantesimi - Muro di Vento}\\
\textbf{Scuola}: Invocazione\\
\textbf{Difficoltà}: 21\\
\textbf{Tempo di Lancio}: 2 Azioni\\
\textbf{Gittata}: 36 metri\\
\textbf{Componenti}: V, S, M (un minuscolo ventaglio e una piuma di origini esotiche)\\
\textbf{Durata}: 1 minuto\\
Un muro di forte vento si leva dal terreno in un punto a gittata di tua scelta. Puoi creare un muro lungo fino a 15 metri, alto 4 metri e spesso 30 centimetri. Puoi modellare il muro in qualsiasi maniera desideri purché componga un percorso continuo sul terreno. Il muro rimane per la durata dell'incantesimo. Quando il muro appare, ogni creatura all'interno della sua area deve effettuare un Tiro Salvezza su Tempra. Una creatura subisce 3d8 danni da botta se fallisce il Tiro Salvezza, o la metà di questi danni se lo supera. Il forte vento tiene lontana foschia, fumo e altri gas. Le creature volanti di taglia Piccola o minore non possono attraversare il muro. I materiali leggeri trascinati nel muro volano verso l'alto. Frecce, quadrelli e altre munizioni normali vengono deviati e mancano automaticamente il bersaglio (i macigni scagliati dai giganti e dalle macchine d'assedio, e munizioni simili, ne ignorano invece gli effetti). Le creature in forma gassosa non possono attraversarlo.

\medskip\textbf{Nube Incendiaria}\index{Incantesimi - Nube Incendiaria}\\
\textbf{Scuola}: Evocazione\\
\textbf{Difficoltà}: 34\\
\textbf{Tempo di Lancio}: 2 Azioni\\
\textbf{Gittata}: 45 metri\\
\textbf{Componenti}: V, S\\
\textbf{Durata}: 1 minuto\\
Una nube di fumo turbinante attraversata da lapilli incandescenti si forma in una sfera di 6 metri di raggio centrata su di un punto a gittata. La nube si propaga intorno agli angoli ed è in penombra. Rimane per la durata dell'incantesimo o finché un vento di velocità moderata o superiore (almeno 15 chilometri all'ora) non la disperde.\\
Quando la nube appare, ogni creatura al suo interno deve effettuare un Tiro Salvezza su Riflessi. Una creatura subisce 10d8 danni da fuoco se fallisce il Tiro Salvezza, e la metà di questi danni se lo supera. Una creatura deve effettuare il Tiro Salvezza anche quando entra per la prima volta nell'area o termina lì il suo round.\\
All'inizio di ciascun tuo round, la nube si muove di 3 metri lontano da te in una direzione a tua scelta. \\

\medskip\textbf{Nube Maleodorante}\index{Incantesimi - Nube Maleodorante}\\
\textbf{Scuola}: Evocazione\\
\textbf{Difficoltà}: 21\\
\textbf{Tempo di Lancio}: 2 Azioni\\
\textbf{Gittata}: 27 metri\\
\textbf{Componenti}: V, S, M (un uovo marcio o foglie di cavolo puzzolente)\\
\textbf{Durata}: 1 ora\\
Crei, in un punto a gittata, una sfera di 6 metri di raggio composta di un gas giallo e nauseabondo. La nube si propaga dietro gli angoli e la sua area è in penombra. La nube permane nell'aria per la durata. Ogni creatura che si trovi completamente all'interno della nube all'inizio del proprio round, deve effettuare un Tiro Salvezza su Tempra contro il veleno. Se il Tiro Salvezza fallisce, la creatura spende la 2 Azioni di quel round a vomitare e barcollare. Le creature che non hanno bisogno di respirare o che sono immuni al veleno superano automaticamente il Tiro Salvezza.\\
Un vento moderato (almeno 15 chilometri all'ora) disperde la nube dopo 4 round. Un vento forte (almeno 30 chilometri all'ora) lo disperde dopo 1 round.

\medskip\textbf{Nube Mortale}\index{Incantesimi - Nube Mortale}\\
\textbf{Scuola}: Evocazione\\
\textbf{Difficoltà}: 26\\
\textbf{Tempo di Lancio}: 2 Azioni\\
\textbf{Gittata}: 36 metri\\
\textbf{Componenti}: V, S\\
\textbf{Durata}: 10 minuti \\
Crei una sfera di 6 metri di raggio formata da una nebbia velenosa giallo-verde centrata in un punto a gittata di tua scelta. La nebbia si propaga dietro gli angoli. Rimane per la durata dell'incantesimo o finché un forte vento non disperde la nebbia, terminando l'incantesimo. La sua area è in penombra. Quando una creatura entra nell'area dell'incantesimo per la prima volta in un round o inizia lì il suo round, quella creatura deve effettuare un Tiro Salvezza su Tempra. La creatura subisce 5d8 danni da veleno se fallisce il Tiro Salvezza, o la metà di questi danni se lo supera. Le creature ne sono soggette anche se trattengono il respiro o non hanno bisogno di respirare. La nebbia si allontana di 3 metri da te all'inizio di ogni tuo round, spostandosi lungo la superficie del terreno. I vapori, essendo più pesanti dell'aria, tendono a scendere verso il basso, arrivando addirittura a insinuarsi nelle aperture.\\
\textbf{Per ogni critico ottenuto} nella prova di magia il danno aumenta di 1d8.

\medskip\textbf{Nube di Nebbia}\index{Incantesimi - Nube di Nebbia}\\
\textbf{Scuola}: Evocazione\\
\textbf{Difficoltà}: 16\\
\textbf{Tempo di Lancio}: 2 Azioni\\
\textbf{Gittata}: 36 metri\\
\textbf{Componenti}: V, S\\
\textbf{Durata}: 1 ora\\
Crei una sfera di foschia del raggio di 6 metri centrata su di un punto a gittata. La sfera si propaga intorno agli angoli, e la sua area è in penombra. Rimane per la durata dell'incantesimo o finché un vento di velocità moderata o superiore (almeno 15 chilometri all'ora) non la disperde.\\
\textbf{Per ogni critico ottenuto} nella prova di magia il raggio della foschia aumenta di 6 metri.

\medskip\textbf{Occhio Arcano}\index{Incantesimi - Occhio Arcano}\\
\textbf{Scuola}: Divinazione\\
\textbf{Difficoltà}: 23\\
\textbf{Tempo di Lancio}: 2 Azioni\\
\textbf{Gittata}: 9 metri\\
\textbf{Componenti}: V, S, M (un pezzo di manto di pipistrello)\\
\textbf{Durata}: Concentrazione, massimo 1 ora\\
Crei a gittata un occhio magico e invisibile, che fluttua nell'aria per la durata dell'incantesimo.\\
Ricevi mentalmente le informazioni visive dall'occhio, che ha vista normale e scurovisione fino a 9 metri. L'occhio può guardare in tutte le direzioni. Con un'azione di movimento, puoi spostare l'occhio di 9 metri in qualsiasi direzione. Non c'è limite a quanto lontano possa spostarsi l'occhio, ma non può entrare in un altro piano di esistenza. Una barriera solida blocca il movimento dell'occhio, ma questo può attraversare un'apertura di una grandezza minima di 2,5 centimetri di diametro.

\medskip\textbf{Onda Tonante}\index{Incantesimi - Onda Tonante}\\
\textbf{Scuola}: Invocazione\\
\textbf{Difficoltà}: 16\\
\textbf{Tempo di Lancio}: 2 Azioni\\
\textbf{Gittata}: Personale (cubo di 4 metri di spigolo)\\
\textbf{Componenti}: V, S\\
\textbf{Durata}: Istantanea\\
un'onda di forza tonante si proietta da te. Ogni creatura in un cubo di 4 metri di spigolo che origina da te deve effettuare un Tiro Salvezza su Tempra. Se fallisce il Tiro Salvezza, una creatura subisce 2d8 danni da tuono e viene allontana 3 metri da te. Se supera il Tiro Salvezza, la creatura subisce la metà dei danni e non viene allontanata. Inoltre, gli oggetti non ancorati che sono totalmente all'interno dell'area vengono spinti 3 metri lontano da te dall'effetto dell'incantesimo, e l'incantesimo produce un rimbombo tonante udibile fino a 90 metri.\\
\textbf{Per ogni critico ottenuto} nella prova di magia il danno aumenta di 1d8.

\medskip\textbf{Oscurità}\index{Incantesimi - Oscurità}\\
\textbf{Scuola}: Invocazione\\
\textbf{Difficoltà}: 16\\
\textbf{Tempo di Lancio}: 2 Azioni\\
\textbf{Gittata}: 18 metri\\
\textbf{Componenti}: V, M (pelo di pipistrello e un pizzico di bitume o un pezzo di carbone)\\
\textbf{Durata}: 10 minuti\\
L'oscurità magica si propaga da un punto a gittata, scelto da te, per riempire una sfera di 4 metri di raggio per la durata dell'incantesimo. L'oscurità si propaga intorno agli angoli. Una creatura con scurovisione non può vedere in questa oscurità, e la luce non magica non può illuminarla.\\
Se il punto che hai scelto è su di un oggetto che stai trasportando o uno che non è indossato o trasportato, l'oscurità emana dall'oggetto e si muove con esso. Coprire completamente la fonte dell'oscurità con un oggetto opaco, come un vaso o un elmo, blocca l'oscurità.\\
Se qualsiasi parte dell'area di questo incantesimo si sovrappone con l'area di luce creata da un incantesimo con difficoltà 13 o più basso, l'incantesimo che ha creato la luce viene dissolto.

\medskip

\begin{enfasi}{
		Mi sparpaglio in giro per evitare incantesimi ad area (detta da un giocatore per evitare una Palla di Fuoco)
}\end{enfasi}

\medskip\textbf{Palla di Fuoco}\index{Incantesimi - Palla di Fuoco}\\
\textbf{Scuola}: Invocazione\\
\textbf{Difficoltà}: 21\\
\textbf{Tempo di Lancio}: 2 Azioni\\
\textbf{Gittata}: 45 metri\\
\textbf{Componenti}: V, S, M (una minuscola palla di guano di pipistrello e zolfo)\\
\textbf{Durata}: Istantanea\\
Un fascio di luce gialla parte dal tuo dito puntato verso un punto a gittata scelto da te, e poi esplode con un boato roboante e si trasforma in lingua di fiamme.\\
Ogni creatura in una sfera di 6 metri di raggio centrata in quel punto deve effettuare un Tiro Salvezza su Riflessi. Una creatura subisce 8d6 danni da fuoco se fallisce il Tiro Salvezza, o la metà di questi danni se lo supera.\\
Il fuoco si propaga ed occupa tutto il volume disponibile entro i 6 metri dal punto di esplosione. Il fuoco incendia gli oggetti infiammabili nell'area che non sono indossati o trasportati.\\
\textbf{Per ogni critico ottenuto} nella prova di magia il danno base aumenta di 1d6.\\
\textbf{Successo/Fallimento Critico}: In caso si fallimento critico il danno raddoppia, in caso di successo critico il danno viene ulteriormente dimezzato

\medskip\textbf{Palla di Fuoco Ritardata}\index{Incantesimi - Palla di Fuoco Ritardata}\\
\textbf{Scuola}: Invocazione\\
\textbf{Difficoltà}: 31\\
\textbf{Tempo di Lancio}: 2 Azioni\\
\textbf{Gittata}: 45 metri\\
\textbf{Componenti}: V, S, M (una grossa palla di guano di pipistrello e zolfo)\\
\textbf{Durata}: Concentrazione, 1 minuto\\
Un fascio di luce gialla parte dal tuo dito puntato, per condensarsi per la durata dell'incantesimo nella forma di una pallina luminosa in un punto a gittata, scelto da te. Quando l'incantesimo termina, o perché la tua concentrazione è spezzata o perché decidi tu di porgli fine, la pallina esplode con un boato sommesso e si trasforma in un getto di fiamme che si propaga dietro gli angoli. Ogni creatura in una sfera di 6 metri di raggio centrata in quel punto deve effettuare un Tiro Salvezza su Riflessi. Una creatura subisce danni da fuoco pari al danno totale accumulato se fallisce il Tiro Salvezza, o la metà di questi danni se lo supera. Il danno base dell'incantesimo è 12d6. Se al termine del tuo round la pallina non è ancora detonata, il danno aumenta di 1d6.\\
Se la pallina luminosa viene toccata prima che l'incantesimo abbia avuto fine, la creatura che la tocca deve effettuare un Tiro Salvezza su Riflessi. Se fallisce il Tiro Salvezza, l'incantesimo termina immediatamente, facendo eruttare fiamme dalla pallina. Se supera il Tiro Salvezza, la creatura può lanciare la pallina fino a 12 metri di distanza. Quando colpisce una creatura od oggetto solido, l'incantesimo ha fine e la pallina esplode.\\
Il fuoco danneggia gli oggetti nell'area e incendia gli oggetti infiammabili che non sono indossati o trasportati.\\
\textbf{Per ogni critico ottenuto} nella prova di magia il danno aumento di 1d6.\\
\textbf{Successo/Fallimento Critico}: In caso si fallimento critico il danno raddoppia, in caso di successo critico il danno viene ulteriormente dimezzato.

\medskip\textbf{Parlare con gli Animali}\index{Incantesimi - Parlare con gli Animali}\\
\textbf{Scuola}: Divinazione\\
\textbf{Difficoltà}: 16\\
\textbf{Tempo di Lancio}: 2 Azioni\\
\textbf{Gittata}: Personale\\
\textbf{Componenti}: V, S\\
\textbf{Durata}: 10 minuti\\
Per la durata dell'incantesimo, ottieni la capacità di comprendere e comunicare verbalmente con le bestie. Il sapere e la consapevolezza di molte bestie sono limitati dal loro intelletto ma, come minimo, le bestie possono fornirti informazioni riguardo luoghi e mostri nelle vicinanze, compresi quelli che possono percepire o hanno percepito nei giorni passati. A discrezione del Narratore potresti riuscire a convincere una bestia a farti un piccolo favore.

\medskip\textbf{Parlare con i Morti}\index{Incantesimi - Parlare con i Morti}\\
\textbf{Scuola}: Necromanzia\\
\textbf{Difficoltà}: 21\\
\textbf{Tempo di Lancio}: 2 Azioni\\
\textbf{Gittata}: 3 metri\\
\textbf{Componenti}: V, S, M (incenso acceso)\\
\textbf{Durata}: 10 minuti\\
Conferisci un'apparenza di vita e Intelligenza a un cadavere a gittata, scelto da te, permettendogli di rispondere alle domande che gli poni. Il cadavere deve avere ancora una bocca e non può essere non morto. L'incantesimo fallisce se il cadavere è già stato bersaglio di questo incantesimo negli ultimi 10 giorni. Fino al termine dell'incantesimo, puoi porre al cadavere fino a cinque domande. Il cadavere conosce solo quello che già sapeva in vita, compresi i linguaggi parlati. Le risposte sono di solito brevi, criptiche o ripetitive, e il cadavere non è sotto nessun obbligo a darti risposte veritiere se gli sei ostile o ti riconosce come suo nemico. Questo incantesimo non riporta l'anima della creatura nel corpo, ma solo lo spirito che lo muove. Di conseguenza, il cadavere non può apprendere nuove informazioni, non capisce nulla di quello che è successo da quando è morto, e non può fare valutazioni su eventi futuri.

\medskip\textbf{Parlare con le Piante}\index{Incantesimi - Parlare con le Piante}\\
\textbf{Scuola}: Trasmutazione\\
\textbf{Difficoltà}: 21\\
\textbf{Tempo di Lancio}: 2 Azioni\\
\textbf{Gittata}: Personale (raggio di 9 metri)\\
\textbf{Componenti}: V, S\\
\textbf{Durata}: 10 minuti\\
Infondi i vegetali entro 9 metri da te di capacità senziente e di limitata mobilità, dandole la capacità di comunicare con te ed eseguire dei semplici comandi. Puoi interrogare i vegetali in merito a eventi avvenuti nell'ultimo giorno nell'area dell'incantesimo, ottenendo informazioni sulle creature di passaggio, il clima e altro. Puoi anche trasformare il terreno difficile prodotto dalla crescita dei vegetali (come cespugli e fitto sottobosco) in terreno ordinario per la durata dell'incantesimo.\\
Oppure puoi trasformare del terreno normale in cui siano presenti dei vegetali in terreno difficile, che rimane per la durata dell'incantesimo facendo sì, per esempio, che rampicanti e rami rallentino gli inseguitori. \\
A discrezione del Narratore i vegetali potrebbero svolgere anche altri compiti per tuo conto. L'incantesimo non permette ai vegetali di sradicarsi e muoversi, ma possono muovere liberamente rami, steli e gambi. Se una creatura vegetale si trova nell'area, puoi comunicare con essa come se parlaste lo stesso linguaggio, ma non ottieni alcuna capacità magica per influenzarla.\\
Questo incantesimo può far sì che i vegetali creati dall'incantesimo intralciare rilascino una creatura intralciata. 

\medskip\textbf{Parola Divina}\index{Incantesimi - Parola Divina}\\
\textbf{Scuola}: Invocazione\\
\textbf{Difficoltà}: 31\\
\textbf{Tempo di Lancio}: 1 Azione Immediata\\
\textbf{Gittata}: 9 metri\\
\textbf{Componenti}: V\\
\textbf{Durata}: Istantanea\\
Pronunci una parola divina, infusa del potere del tuo Patrono. Scegli un qualsiasi numero di creature a gittata e che puoi vedere. Ogni creatura che può udirti deve effettuare un Tiro Salvezza su Volontà. Se fallisce il Tiro Salvezza, la creatura subisce un effetto in base ai suoi attuali punti ferita:
\medskip
\begin{itemize}
\item	
0 punti ferita o meno: assordata per 1 minuto
\item	
40 punti ferita o meno: assordata e accecata per 10 minuti
\item	
30 punti ferita o meno: accecata, assordata e stordita per 1 ora
\item	
20 punti ferita o meno: uccisa all'istante
\end{itemize}
\medskip
Quali che siano i suoi attuali punti ferita, un celestiale, elementale, fatato o demone che fallisca il Tiro Salvezza è obbligato a tornare al suo piano di origine (se non vi si trova già) e non può tornare sul tuo attuale piano prima che siano passate 24 ore, a meno dell'uso dell'incantesimo desiderio.

\medskip\textbf{Parola Guaritrice}\index{Incantesimi - Parola Guaritrice}\\
\textbf{Scuola}: Cura\\
\textbf{Difficoltà}: 16\\
\textbf{Tempo di Lancio}: 1 Azione Immediata\\
\textbf{Gittata}: 18 metri\\
\textbf{Componenti}: V\\
\textbf{Durata}: Istantanea\\
Una creatura a gittata che puoi vedere, scelta da te, recupera punti ferita pari a 1d4 + il tuo valore di caratteristica da incantatore. Questo incantesimo causa lo stesso ammontare di danno su un non morto.\\
\textbf{Per ogni critico ottenuto} nella prova di magia la cura aumenta di 1d4.\\
Se incantatore e creatura curata sono entrambi Seguaci dello stesso Patrono l'incantesimo cura 1d4 in piu'.\\
Se incantatore e creatura curata sono entrambi Devoti dello stesso Patrono l'incantesimo cura 2d4 in piu'.\\

\medskip\textbf{Parola Guaritrice di Massa}\index{Incantesimi - Parola Guaritrice di Massa}\\
\textbf{Scuola}: Cura\\
\textbf{Difficoltà}: 21\\
\textbf{Tempo di Lancio}: 1 Azione Immediata\\
\textbf{Gittata}: 18 metri\\
\textbf{Componenti}: V\\
\textbf{Durata}: Istantanea\\
Mentre pronunci parole di cura, fino a sei creature a gittata che puoi vedere, scelte da te, recuperano punti ferita pari a 1d4 + il tuo modificatore di caratteristica da incantatore. Questo incantesimo causa lo stesso ammontare di danno sui non morti.\\
\textbf{Per ogni critico ottenuto} nella prova di magia la cura aumenta di 1d4.\\
Se incantatore e creatura curata sono entrambi Seguaci dello stesso Patrono l'incantesimo cura 1d4 in piu'.\\
Se incantatore e creatura curata sono entrambi Devoti dello stesso Patrono l'incantesimo cura 2d4 in piu'.\\
 
\medskip\textbf{Parola del Potere Stordire}\index{Incantesimi - Parola del Potere Stordire}\\
\textbf{Scuola}: Ammaliamento\\
\textbf{Difficoltà}: 34\\
\textbf{Tempo di Lancio}: 1 Azione Immediata\\
\textbf{Gittata}: 18 metri\\
\textbf{Componenti}: V\\
\textbf{Durata}: 1 minuti\\
Pronunci una parola di potere che può travolgere la mente di una creatura a gittata e che puoi vedere, lasciandola confusa. Se il bersaglio ha 150 punti ferita o meno, è stordito. Altrimenti, l'incantesimo non ha effetto.\\

\medskip\textbf{Parola del Potere Uccidere}\index{Incantesimi - Parola del Potere Uccidere}\\
\textbf{Scuola}: Ammaliamento\\
\textbf{Difficoltà}: 36\\
\textbf{Tempo di Lancio}: 1 Azione Immediata\\
\textbf{Gittata}: 18 metri\\
\textbf{Componenti}: V\\
\textbf{Durata}: Istantanea\\
Pronunci una parola di potere che costringe a morire all'istante una creatura a gittata che puoi vedere. Se la creatura che scegli ha 100 punti ferita o meno, muore. Altrimenti, l'incantesimo non ha effetto. 

\medskip\textbf{Parola del Ritiro}\index{Incantesimi - Parola del Ritiro}\\
\textbf{Scuola}: Evocazione\\
\textbf{Difficoltà}: 29\\
\textbf{Tempo di Lancio}: 2 Azioni\\
\textbf{Gittata}: 1 metro\\
\textbf{Componenti}: V\\
\textbf{Durata}: Istantanea\\
Te e fino a cinque creature consenzienti entro 1 metro da te vi teletrasportate istantaneamente in un luogo sicuro indicato precedentemente, detto santuario. Tu e tutte le creature teletrasportate con te, riapparite nello spazio non occupato più vicino al punto indicato quando hai preparato questo santuario (vedi sotto). Se lanci questo incantesimo senza aver prima preparato un santuario, l'incantesimo non ha effetto.\\
Devi indicare un santuario, che sia dedicato o fortemente collegato al tuo Patrono. Se tenti di lanciare l'incantesimo perche' ti porti in un'area che non sia dedicata dal tuo Patrono, l'incantesimo non ha effetto.

\medskip\textbf{Passapareti}\index{Incantesimi - Passapareti}\\
\textbf{Scuola}: Trasmutazione\\
\textbf{Difficoltà}: 26\\
\textbf{Tempo di Lancio}: 2 Azioni\\
\textbf{Gittata}: 9 metri\\
\textbf{Componenti}: V, S, M (un pizzico di semi di sesamo)\\
\textbf{Durata}: 1 ora\\
Per la durata dell'incantesimo, compare un passaggio in un punto a gittata che puoi vedere, su di una superficie di legno, muro o pietra (come una parete, un soffitto o un pavimento) scelta da te. Scegli le dimensioni dell'apertura: al massimo larga 1 metro, alta 2,4 metri e profonda 6 metri. Il passaggio non crea instabilità nella struttura che lo circonda.\\
Quando l'apertura sparisce, qualsiasi creatura od oggetto ancora nel passaggio creato dall'incantesimo viene espulso al sicuro nello spazio non occupato più vicino alla superficie su cui hai lanciato l'incantesimo.

\medskip\textbf{Passare Senza Tracce}\index{Incantesimi - Passare Senza Tracce}\\
\textbf{Scuola}: Abiurazione\\
\textbf{Difficoltà}: 19\\
\textbf{Tempo di Lancio}: 2 Azioni\\
\textbf{Gittata}: Personale\\
\textbf{Componenti}: V, S, M (ceneri di una foglia di vischio bruciata e un ramoscello di abete rosso)\\
\textbf{Durata}: Concentrazione, 1 ora
Per la durata dell'incantesimo le tue tracce non possono essere seguite eccetto che da mezzi magici. La creatura che riceve questo bonus non lascia tracce né altri segni del suo passaggio.
\textbf{Per ogni critico ottenuto} nella prova di magia puoi includere un altra creature nei benefici dell'incantesimo.\\


\medskip\textbf{Passo Velato}\index{Incantesimi - Passo Velato}\\
\textbf{Scuola}: Evocazione\\
\textbf{Difficoltà}: 19\\
\textbf{Tempo di Lancio}: 1 Azione Immediata\\
\textbf{Gittata}: Personale\\
\textbf{Componenti}: V\\
\textbf{Durata}: Istantanea\\
Avvolto rapidamente da una foschia argentata, ti teletrasporti di massimo 9 metri in uno spazio non occupato che puoi vedere.

\medskip\textbf{Passo Veloce}\index{Incantesimi - Passo Veloce}\\
\textbf{Scuola}: Trasmutazione\\
\textbf{Difficoltà}: 16\\
\textbf{Tempo di Lancio}: 2 Azioni\\
\textbf{Gittata}: Contatto\\
\textbf{Componenti}: V, S, M (un pizzico di terra)\\
\textbf{Durata}: 1 ora\\
La velocità di una creatura aumenta di 3 metri fino al termine dell'incantesimo. \\
\textbf{Per ogni critico ottenuto} nella prova di magia puoi prendere come bersaglio un'ulteriore creatura.

\medskip\textbf{Paura}\index{Incantesimi - Paura}\\
\textbf{Scuola}: Illusione\\
\textbf{Difficoltà}: 21\\
\textbf{Tempo di Lancio}: 2 Azioni\\
\textbf{Gittata}: Personale (cono di 9 metri)\\
\textbf{Componenti}: V, S, M (una piuma bianca o il cuore di una gallina)\\
\textbf{Durata}: 1 minuto\\
Proietti un'immagine illusoria delle peggiori paure di una creatura. Ogni creatura in un cono di 9 metri deve superare un Tiro Salvezza su Volontà o far cadere qualsiasi cosa stia impugnando e restare spaventata per la durata dell'incantesimo.\\
Mentre è spaventata da questo incantesimo, una creatura deve, durante ciascun suo round, effettuare l'azione Scattare e muoversi lontano da te tramite il tragitto più sicuro, a meno che non abbia spazio per muoversi. Se la creatura termina il suo round in un posto dove non ha linea di visuale su di te, può effettuare un Tiro Salvezza su Volontà. Se lo supera, l'incantesimo, per quella creatura, ha termine. 

\medskip\textbf{Pelle di Corteccia}\index{Incantesimi - Pelle di Corteccia}\\
\textbf{Scuola}: Trasmutazione\\
\textbf{Difficoltà}: 19\\
\textbf{Tempo di Lancio}: 2 Azioni\\
\textbf{Gittata}: Contatto\\
\textbf{Componenti}: V, S, M (una manciata di corteccia di quercia)\\
\textbf{Durata}: 1 ora\\
La pelle del bersaglio con cui sei in contatto al momento del lancio dell'incantesimo diventa ruvida e dall'aspetto simile alla corteccia fino al termine dell'incantesimo, e la Difesa del bersaglio non può essere inferiore a 16, quale che sia l'armatura che stia indossando.

\medskip\textbf{Pelle di Pietra}\index{Incantesimi - Pelle di Pietra}\\
\textbf{Scuola}: Abiurazione\\
\textbf{Difficoltà}: 23\\
\textbf{Tempo di Lancio}: 2 Azioni\\
\textbf{Gittata}: Contatto\\
\textbf{Componenti}: V, S, M (polvere di diamante del valore di 100 mo, che l'incantesimo consuma)\\
\textbf{Durata}: 1 ora\\
Lanci l'incantesimo a contatto di una creatura consenziente, la cui pelle si tramuta in una sostanza dura come la pietra. Tira 1d4+metà del valore di CA, la somma risultante e' le volte che un attacco con arma di mischia o distanza viene annullato (indipendentemente che di colpisca o meno).\\

\medskip\textbf{Piaga degli Insetti}\index{Incantesimi - Piaga degli Insetti}\\
\textbf{Scuola}: Evocazione\\
\textbf{Difficoltà}: 26\\
\textbf{Tempo di Lancio}: 2 Azioni\\
\textbf{Gittata}: 90 metri\\
\textbf{Componenti}: V, S, M (qualche granello di zucchero, qualche chicco di grano, e una passata di lardo)\\
\textbf{Durata}: 10 minuti\\
Uno sciame di locuste affamate riempie una sfera di 6 metri di raggio centrata in un punto a gittata scelto da te. La sfera si propaga intorno agli angoli. La sfera rimane per la durata dell'incantesimo, e la sua area è in penombra. L'area della sfera è terreno difficile.\\
Quando l'area appare, ogni creatura al suo interno deve effettuare un Tiro Salvezza su Tempra. Una creatura subisce 4d10 danni se fallisce il Tiro Salvezza, o la metà di questi danni se lo supera. Una creatura deve effettuare questo Tiro Salvezza anche quando entra per la prima volta nell'area dell'incantesimo durante un round o se termina il proprio round al suo interno.\\
\textbf{Per ogni critico ottenuto} nella prova di magia il danno aumenta di 1d8.

\medskip\textbf{Porta Dimensionale}\index{Incantesimi - Porta Dimensionale}\\
\textbf{Scuola}: Evocazione\\
\textbf{Difficoltà}: 23\\
\textbf{Tempo di Lancio}: 2 Azioni\\
\textbf{Gittata}: 150 metri\\
\textbf{Componenti}: V\\
\textbf{Durata}: Istantanea\\
Ti teletrasporti dalla tua attuale posizione in qualsiasi altro posto a gittata. Arrivi esattamente nel posto desiderato. Può essere un luogo che puoi vedere, uno che puoi visualizzare, o uno che puoi descrivere indicando distanza e direzione, come "30 metri verso il basso" o "90 metri in alto a nordovest con un angolo di 45 gradi."\\
Puoi portare con te oggetti il cui peso non ecceda la tua capacità di ingombro. Puoi portare con te anche una creatura consenziente della tua taglia o più piccola con equipaggiamento fino al limite della sua capacità di carico. La creatura deve essere entro 1 metro da te quando lanci questo incantesimo. \\
Se dovessi arrivare in un posto già occupato da un oggetto o creatura, tu e la creatura che viaggia con te subite ciascuno 4d6 danni da forza, e l'incantesimo non riesce a teletrasportarvi.


\medskip\textbf{Preghiera di Guarigione}\index{Incantesimi - Preghiera di Guarigione}\\
\textbf{Scuola}: Cura\\
\textbf{Difficoltà}: 19\\
\textbf{Tempo di Lancio}: 10 minuti\\
\textbf{Gittata}: 9 metri\\
\textbf{Componenti}: V\\
\textbf{Durata}: Istantanea\\
Fino a sei creature a gittata che puoi vedere, scelte da te, recuperano ciascuna punti ferita pari a 2d8 + il tuo modificatore di caratteristica da incantatore. Questo incantesimo causa lo stesso ammontare di danno sui non morti.\\
\textbf{Per ogni critico ottenuto} nella prova di magia la cura aumento di aumenta di 1d8.

\medskip\textbf{Presagio}\index{Incantesimi - Presagio}\\
\textbf{Scuola}: Divinazione\\
\textbf{Difficoltà}: 19\\
\textbf{Tempo di Lancio}: 1 minuto\\
\textbf{Gittata}: Personale\\
\textbf{Componenti}: V, S, M (dei bastoncini, ossa o simili oggetti marchiati appositamente e del valore di almeno 25 mo)\\
\textbf{Durata}: Istantanea\\
Gettando bastoncini intarsiati con gemme, facendo rotolare ossa di drago, impilando carte elaborate o impiegando qualche altro strumento di divinazione, ricevi un presagio da un'entità ultraterrena riguardo il risultato di uno specifico corso di azione che intendi intraprendere nei prossimi 30 minuti. Il Narratore sceglie tra i seguenti presagi:
\medskip
\begin{itemize}
\item 
Prosperità, per i risultati positivi
\item 
Calamità, per i risultati negativi
\item 
Prosperità e calamità, per i risultati sia positivi che negativi
\item 
Nulla, per i risultati che non sono né particolarmente positivi né negativi
\end{itemize}
\medskip
L'incantesimo non tiene conto di ogni possibile circostanza che possa modificare il risultato, come il lancio di ulteriori incantesimi o la perdita o l'arrivo di un alleato. Se lanci l'incantesimo due o più volte prima che sia sorto il nuovo sole, c'è una probabilità cumulativa del 25\% che per ogni lancio dopo il primo tu ottenga una lettura erronea. Il Narratore effettua questo tiro in segreto.

\medskip\textbf{Prestidigitazione}\index{Trucchetto - Prestidigitazione}\\
\textbf{Scuola}: Universale\\
\textbf{Difficoltà}: 12\\
\textbf{Tempo di Lancio}: 2 Azioni\\
\textbf{Gittata}: 3 metri\\
\textbf{Componenti}: V, S\\
\textbf{Durata}: Massimo 1 ora\\
Questo incantesimo è un trucco magico minore che gli incantatori novizi impiegano per fare pratica. Crei a gittata uno dei seguenti effetti magici:
\medskip
\begin{itemize}
\item
Crei un effetto sensoriale innocuo e istantaneo come una pioggia di scintille, un soffio di vento, una debole nota musicale o uno strano odore.
\item
Illumini o spegni istantaneamente una candela, una torcia o piccolo fuoco da campo.
\item
Ripulisci o insozzi istantaneamente un oggetto non più grosso di 0,03 metri cubi.
\item
Raffreddi, riscaldi o insapori per 1 ora fino a 0,03 metri cubi di materiale non vivente.
\item
Fai comparire per 1 ora un colore, un piccolo segno o un simbolo su di un oggetto o una superficie.
\item
Crei un ninnolo non magico o un'immagine illusoria che entri nella tua mano e che resta fino al termine del tuo prossimo round.
\end{itemize}
\medskip
Se lanci questo incantesimo più volte, puoi tenere attivi fino a tre effetti non istantanei alla volta, e puoi interrompere uno di questi effetti con un'azione.

\medskip\textbf{Previsione}\index{Incantesimi - Previsione}\\
\textbf{Scuola}: Divinazione\\
\textbf{Difficoltà}: 36\\
\textbf{Tempo di Lancio}: 1 minuto\\
\textbf{Gittata}: Contatto\\
\textbf{Componenti}: V, S, M (una piuma di colibrì)\\
\textbf{Durata}: 8 ore\\
Lanci l'incantesimo a contatto di una creatura consenziente per conferirle una limitata capacità di vedere nell'immediato futuro. Per la durata, il bersaglio non può essere sorpreso e ha +1d6 sui Tiri per Colpire, prove di caratteristica e Tiri Salvezza. Inoltre, sempre per la durata, le altre creature hanno -1d6 sui Tiri per Colpire contro il bersaglio. L'incantesimo ha immediatamente termine se lo lanci di nuovo prima che la sua durata abbia fine.

\medskip\textbf{Produrre Fiamma}\index{Trucchetto - Produrre Fiamma}\\
\textbf{Scuola}: Evocazione\\
\textbf{Difficoltà}: 12\\
\textbf{Tempo di Lancio}: 2 Azioni\\
\textbf{Gittata}: Personale\\
\textbf{Componenti}: V, S\\
\textbf{Durata}: 10 minuti\\
Una fiammella compare nella tua mano. La fiammella resta lì per la durata dell'incantesimo e non danneggia né te né il tuo equipaggiamento. La fiamma produce luce intensa nel raggio di 3 metri e luce fioca per ulteriori 3 metri. L'incantesimo termina se lo interrompi con un'azione o se lo lanci di nuovo.\\
Puoi usare la fiamma anche per attaccare, sebbene farlo ponga termine all'incantesimo. Quando lanci questo incantesimo, o con un'azione in un round successivo, puoi scagliare la fiamma a una creatura entro 9 metri da te. Effettua un attacco a distanza con incantesimo. Se colpisci, il bersaglio subisce 1d8 danni da fuoco.\\
Il danno dell'incantesimo aumenta di 1d8 quando raggiungi CM 5, CM 11 e CM 17.

\medskip\textbf{Proibizione}\index{Incantesimi - Proibizione}\\
\textbf{Scuola}: Abiurazione\\
\textbf{Difficoltà}: 29\\
\textbf{Tempo di Lancio}: 10 minuti\\
\textbf{Gittata}: Contatto\\
\textbf{Componenti}: V, S, M (uno spruzzo di Acqua Benedetta, incensi rari, e un rubino in polvere del valore di 1.000 mo)\\
\textbf{Durata}: 1 giorno\\
Crei una interdizione al viaggio magico che protegge fino a 4.000 metri quadri di pavimento, fino a un'altezza di 9 metri dal suolo. Per la durata dell'incantesimo, le creature non possono teletrasportarsi nell'area o usare passaggi, come quello creato dall'incantesimo portale, per entrare nell'area. L'incantesimo protegge l'area dal viaggio planare, e quindi impedisce alle creature di accedere all'area tramite il Piano Astrale, il Piano Etereo, le Lande Fatate o il Mondo delle Ombre, o l'incantesimo spostamento planare.\\
Inoltre, l'incantesimo danneggia i tipi di creatura scelti da te durante il lancio. Scegli uno o più dei seguenti: celestiali, elementali, fatati, demoni e non morti. Quando una creatura selezionata entra nell'area dell'incantesimo per la prima volta in un round o inizia qui il suo round, la creatura subisce 5d10 danni da Luce o da Vuoto (a tua scelta, quando lanci l'incantesimo). \\
Quando lanci questo incantesimo, puoi stabilire una parola d'ordine. Una creatura che pronuncia la parola d'ordine mentre entra nell'area dell'incantesimo, non subisce danni da esso.\\
L'area dell'incantesimo non può sovrapporsi all'area di un altro incantesimo proibizione. Se esegui proibizione ogni giorno per 30 giorni nello stesso posto, l'incantesimo durerà finché non viene dissolto, e le componenti materiali saranno consumate durante l'ultimo lancio.

\medskip\textbf{Proiezione Astrale}\index{Incantesimi - Proiezione Astrale}\\
\textbf{Scuola}: Necromanzia\\
\textbf{Difficoltà}: 36\\
\textbf{Tempo di Lancio}: 2 Azioni\\
\textbf{Gittata}: 3 metri\\
\textbf{Componenti}: V, S, M (per ogni creatura soggetta a questo incantesimo, devi fornire un giacinto del valore di almeno 1.000 mo e un lingotto d'argento elegantemente scolpito del valore di almeno 100 mo, tutti i quali sono consumati dall'incantesimo)\\
\textbf{Durata}: Speciale\\
Tu e fino ad altre otto creature consenzienti a gittata proiettate i vostri corpi astrali nel Piano Astrale (l'incantesimo fallisce e il lancio è sprecato qualora vi trovaste già in quel piano). Il corpo materiale che ti lasci alle spalle è privo di sensi e in uno stato di animazione sospesa; non ha bisogno di cibo né di acqua e non invecchia.\\
Il tuo corpo astrale assomiglia in tutto e per tutto alla tua forma mortale, replicando le tue statistiche di gioco e i tuoi oggetti. La principale differenza è l'aggiunta di un cordone argenteo che si estende dalle scapole per 30 centimetri dietro di te, divenendo poi invisibile. Il cordone è la tua connessione al tuo corpo materiale. Finché questa connessione resterà intatta, potrai tornare a casa. Se il cordone viene tagliato (un avvenimento che accade solo quando uno specifico effetto lo indica) la tua anima e corpo vengono separati, uccidendoti all'istante.\\
La tua forma astrale può viaggiare liberamente per il Piano Astrale e attraversare i portali che da lì conducono ad altri piani. Se entri in un nuovo piano o ritorni al piano su cui eri al momento del lancio dell'incantesimo, il tuo corpo e i tuoi oggetti vengono trasportati lungo il cordone argenteo, permettendoti di rientrare nel tuo corpo al momento dell'ingresso nel nuovo piano. La tua forma astrale è una incarnazione separata. Qualsiasi danno o altro effetto che si applica a essa, non ha effetto sul tuo corpo fisico, né vi compare al tuo ritorno.\\
L'incantesimo ha termine per te e i tuoi compagni quando userai un'azione per interromperlo. Quando l'incantesimo termina, la creatura su cui agisce torna al proprio corpo fisico, e si risveglia. L'incantesimo potrebbe anche avere una fine anticipata per te o uno dei tuoi compagni. Un incantesimo dissolvi magie usato con successo sul corpo astrale o fisico termina l'incantesimo per quella creatura. Se il corpo originale della creatura o la sua forma astrale scende a 0 punti ferita, per quella creatura l'incantesimo ha termine. Se l'incantesimo ha termine e il cordone argenteo è intatto, il cordone trascina indietro al suo corpo la forma astrale della creatura, ponendo fine al suo stato di animazione sospesa.\\
Se vieni riportato al tuo corpo prematuramente, i tuoi compagni devono restare nella loro forma astrale e trovare per proprio conto la via di ritorno ai loro corpi, di solito scendendo a 0 punti ferita.

\medskip\textbf{Protezione dal Bene e dal Male}\index{Incantesimi - Protezione dal Bene e dal Male}\\
\textbf{Scuola}: Abiurazione\\
\textbf{Difficoltà}: 16\\
\textbf{Tempo di Lancio}: 2 Azioni\\
\textbf{Gittata}: Contatto\\
\textbf{Componenti}: V, S, M (Acqua Benedetta o argento e ferro in polvere, che l'incantesimo consuma)\\
\textbf{Durata}: 10 minuti\\
Fino al termine dell'incantesimo, una creatura consenziente in contatto con te al momento dell'esecuzione è protetta da certi tipi di creature: aberrazioni, celestiali, elementali, fatati, demoni e non morti.\\
La protezione conferisce diversi benefici. Le creature di quei tipi hanno -1d6 ai Tiri per Colpire contro il bersaglio. Il bersaglio non può essere affascinato, spaventato o posseduto da loro. Se il bersaglio è già affascinato, spaventato o posseduto da una simile creatura, il bersaglio ha +1d6 su qualsiasi nuovo Tiro Salvezza contro l'effetto in questione.\\
\textbf{Questo incantesimo non e' usabile se si usano Tratti}

\medskip\textbf{Protezione dall'Energia}\index{Incantesimi - Protezione dall'Energia}\\
\textbf{Scuola}: Abiurazione\\
\textbf{Difficoltà}: 21\\
\textbf{Tempo di Lancio}: 2 Azioni\\
\textbf{Gittata}: Contatto\\
\textbf{Componenti}: V, S\\
\textbf{Durata}: 10 minuti\\
Lanci l'incantesimo a contatto di una creatura consenziente. Per la durata dell'incantesimo, il bersaglio ha resistenza a un tipo di danno scelto da te: acido, freddo, fuoco, fulmine o tuono. Puoi sacrificare tutta la durata dell'incantesimo, terminandolo, per annullare completamente il danno subito da una fonte di energia.\\
\textbf{Per ogni critico ottenuto} nella prova di magia puoi influenzare un altra persona o raddoppiare la durata.\

\medskip\textbf{Protezione dai Veleni}\index{Incantesimi - Protezione dai Veleni}\\
\textbf{Scuola}: Abiurazione\\
\textbf{Difficoltà}: 19\\
\textbf{Tempo di Lancio}: 2 Azioni\\
\textbf{Gittata}: Contatto\\
\textbf{Componenti}: V, S\\
\textbf{Durata}: 1 ora\\
Per la durata dell'incantesimo, il bersaglio ha +1d6 ai Tiri Salvezza contro l'essere avvelenato, e ha resistenza al danno da veleno.

\medskip\textbf{Punizione Marchiante}\index{Incantesimi - Punizione Marchiante}\\
\textbf{Scuola}: Invocazione\\
\textbf{Difficoltà}: 19\\
\textbf{Tempo di Lancio}: 1 Azione Immediata\\
\textbf{Gittata}: Personale\\
\textbf{Componenti}: V\\
\textbf{Durata}: 1 minuto\\
La prossima volta che colpisci una creatura con un attacco in mischia con arma nella durata dell'incantesimo, l'arma riluce di un bagliore astrale mentre colpisci. L'attacco infligge 1d6 danni da Luce aggiuntivi al bersaglio, che diventa visibile qualora sia invisibile ed emette luce fioca in un raggio di 1 metro. Inoltre il bersaglio non può diventare invisibile fino al termine dell'incantesimo. \\
\textbf{Per ogni critico ottenuto} nella prova di magia il danno aggiuntivo aumenta di 1d6.

\medskip\textbf{Purificare Cibo e Bevande}\index{Incantesimi - Purificare Cibo e Bevande}\\
\textbf{Scuola}: Trasmutazione\\
\textbf{Difficoltà}: 16\\
\textbf{Tempo di Lancio}: 2 Azioni\\
\textbf{Gittata}: 3 metri\\
\textbf{Componenti}: V, S\\
\textbf{Durata}: Istantanea\\
Tutti i cibi e le bevande non magiche in una sfera di 1 metri di raggio, centrata in un punto a gittata di tua scelta, vengono purificati e liberati da veleni e malattie. 

\medskip\textbf{Raggio di Gelo}\index{Trucchetto - Raggio di Gelo}\\
\textbf{Scuola}: Invocazione\\
\textbf{Difficoltà}: 12\\
\textbf{Tempo di Lancio}: 2 Azioni\\
\textbf{Gittata}: 18 metri\\
\textbf{Componenti}: V, S\\
\textbf{Durata}: Istantanea\\
Un fascio gelato di luce azzurra colpisce una creatura a gittata. Effettua un attacco a distanza con incantesimo contro il bersaglio. Se colpisci, egli subisce 1d8 danni da freddo, e la sua velocità è ridotta di 3 metri fino all'inizio del tuo prossimo round. \\
Il danno dell'incantesimo aumenta di 1d8 quando raggiungi CM 5, CM 11 e CM 17.

\medskip\textbf{Raggio di Affaticamento}\index{Incantesimi - Raggio di Affaticamento}\\
\textbf{Scuola}: Necromanzia\\
\textbf{Difficoltà}: 19\\
\textbf{Tempo di Lancio}: 2 Azioni\\
\textbf{Gittata}: 18 metri\\
\textbf{Componenti}: V, S\\
\textbf{Durata}: 1 minuto\\
Un fascio nero di energia debilitante parte dal tuo dito diretto contro una creatura a gittata. Effettua un attacco a distanza con incantesimo contro il bersaglio. Se colpisci, il bersaglio infliggerà la metà dei danni con gli attacchi con arma che usano la Forza fino al termine dell'incantesimo.\\ 

\medskip\textbf{Raggio Rovente}\index{Incantesimi - Raggio Rovente}\\
\textbf{Scuola}: Invocazione\\
\textbf{Difficoltà}: 19\\
\textbf{Tempo di Lancio}: 2 Azioni\\
\textbf{Gittata}: 36 metri\\
\textbf{Componenti}: V, S\\
\textbf{Durata}: Istantanea\\
Crei tre raggi di fuoco e li proietti verso tre bersagli a gittata. Puoi proiettarli contro lo stesso bersaglio o bersagli diversi.\\
Effettua un attacco a distanza con incantesimo per ciascun raggio. Se colpisci, il bersaglio subisce 2d6 danni da fuoco.\\
\textbf{Per ogni critico ottenuto} nella prova di magia crei un raggio aggiuntivo.

\medskip\textbf{Ragnatela}\index{Incantesimi - Ragnatela}\\
\textbf{Scuola}: Evocazione\\
\textbf{Difficoltà}: 19\\
\textbf{Tempo di Lancio}: 2 Azioni\\
\textbf{Gittata}: 18 metri\\
\textbf{Componenti}: V, S, M (un pezzo di tela di ragno)\\
\textbf{Durata}: 1 ora\\
Evochi una spessa massa di tela densa e appiccicosa in un punto a gittata, scelto da te. Per la durata, la ragnatela riempie un cubo di 6 metri di spigolo da quel punto. La ragnatela è terreno difficile e rende quell'area oscurata leggermente.\\
Se la tela non è ancorate tra due masse solide (come pareti o alberi) o stesa lungo un pavimento, parete o soffitto, la ragnatela evocata crolla su se stessa, e l'incantesimo termina all'inizio del tuo prossimo round. Le tele distese su di una superficie piatta hanno una profondità di 1 metro.\\
Ogni creatura che inizia il suo round nella ragnatela o che vi entra durante il proprio round deve effettuare un Tiro Salvezza su Riflessi. Se lo fallisce, la creatura è intralciata finché rimane nella ragnatela o finché non si libera.\\
Una creatura intralciata dalle ragnatele può usare 2 Azioni per effettuare una prova di Forza contro la DC del Tiro Salvezza dell'incantesimo. Se la supera, non è più intralciata.\\
Le ragnatele sono infiammabili. Qualsiasi cubo di 1 metro di spigolo di ragnatela che venga esposto al fuoco, brucia in 1 round, infliggendo 2d4 danni da fuoco a qualsiasi creatura che inizi il suo round in mezzo al fuoco.\\

\medskip\textbf{Randello Incantato}\index{Trucchetto - Randello Incantato}\\
\textbf{Scuola}: Trasmutazione\\
\textbf{Difficoltà}: 12\\
\textbf{Tempo di Lancio}: 1 Azione Immediata\\
\textbf{Gittata}: Contatto\\
\textbf{Componenti}: V, S, M (vischio, una foglia di quadrifoglio, e una randello o bastone da combattimento)\\
\textbf{Durata}: 1 minuto\\
Il legno di un randello o bastone da combattimento che stai impugnando viene infuso del potere della natura. Per la durata dell'incantesimo, usando quell'arma puoi usare la tua caratteristica da incantatore al posto della Forza per i Tiri per Colpire e danno da mischia, e il dado di danno dell'arma diventa un d8. L'arma diventa anche magica, se già non lo è. L'incantesimo ha termine se lo lanci di nuovo o se lasci l'arma.

\medskip\textbf{Reggia Meravigliosa}\index{Incantesimi - Reggia Meravigliosa}\\
\textbf{Scuola}: Evocazione\\
\textbf{Difficoltà}: 31\\
\textbf{Tempo di Lancio}: 1 minuto\\
\textbf{Gittata}: 90 metri\\
\textbf{Componenti}: V, S, M (un portale in miniatura scolpito in avorio, un piccolo pezzo di marmo lucido, e un minuscolo cucchiaio d'argento, ciascuno di questi oggetti deve essere almeno del valore di 5 mo)\\
\textbf{Durata}: 24 ore\\
Entro la gittata, evochi un'abitazione extradimensionale che rimane per la durata dell'incantesimo. Scegli dove è posizionato il suo portone d'ingresso. Il portone d'ingresso emette una lieve luminosità ed è largo 1 metri per 3 metri di altezza. Tu e tutte le creature da te indicate quando hai lanciato l'incantesimo potete entrare nell'abitazione extradimensionale, fino a quando il portone resta aperto. Puoi aprire o chiudere il portone se ti trovi entro 9 metri da esso. Mentre è chiuso, il portone è invisibile.\\
Oltre il portone si trova un magnifico ingresso, oltre il quale si dipanano numerose stanze. L'atmosfera è pulita, fresca e accogliente. Puoi creare quanti piani desideri, ma lo spazio non può eccedere 50 cubi ognuno di 3 metri di spigolo. Il luogo è ammobiliato e decorato come preferisci. Contiene cibo sufficiente a soddisfare un banchetto di 9 portate per 100 persone. Uno staff di 100 servitori quasi trasparenti è al servizio di chiunque vi faccia ingresso. Sta a te decidere l'aspetto visivo di questi servitori e il loro abbigliamento. Essi obbediscono assolutamente ai tuoi ordini. Ogni servitore può svolgere qualsiasi compito un normale servitore umano possa svolgere, ma non possono attaccare o effettuare alcuna azione che potrebbe arrecare direttamente danno a un'altra creatura. I servitori possono quindi raccogliere oggetti, pulire, riparare, ripiegare vestiti, accendere fuochi, servire cibi, versare vini e così via. I servitori possono recarsi in qualsiasi punto della dimora, ma non possono uscirne. I mobili e gli altri oggetti creati da questo incantesimo diventano fumo quando vengono portati fuori dalla dimora. Quando l'incantesimo termina, qualsiasi creatura all'interno dello spazio extradimensionale viene espulsa nello spazio aperto più vicino all'uscita.\\
\textbf{Nota}: l'incantesimo lanciato per un anno tutti i giorni sempre nello stesso luogo diventa permanente.

\medskip\textbf{Regressione Mentale}\index{Incantesimi - Regressione Mentale}\\
\textbf{Scuola}: Ammaliamento\\
\textbf{Difficoltà}: 34\\
\textbf{Tempo di Lancio}: 2 Azioni\\
\textbf{Gittata}: 45 metri\\
\textbf{Componenti}: V, S, M (una manciata di sfere di argilla, cristallo, vetro o minerali)\\
\textbf{Durata}: Istantanea\\
Assalti la mente di una creatura a gittata e che puoi vedere, cercando di frammentarne l'intelletto e la personalità. Il bersaglio subisce 4d6 danni e deve effettuare un Tiro Salvezza su Volontà. Se fallisce il Tiro Salvezza, i punteggi di Intelligenza e Carisma della creatura scendono a -4. La creatura non può lanciare incantesimi, attivare oggetti magici, comprendere linguaggi, o comunicare in alcun modo comprensibile. La creatura può, tuttavia, identificare i suoi amici, seguirli e anche proteggerli. Dopo 30 giorni, la creatura può ripetere il Tiro Salvezza contro l'incantesimo. Se lo supera, l'incantesimo ha termine se fallisce l'effetto e' permanete.\\ 
l'incantesimo può essere terminato entro i 30 giorni da ristorare superiore, guarigione o desiderio.

\medskip\textbf{Reincarnazione}\index{Incantesimi - Reincarnazione}\\
\textbf{Scuola}: Trasmutazione\\
\textbf{Difficoltà}: 26\\
\textbf{Tempo di Lancio}: 1 ora\\
\textbf{Gittata}: Contatto\\
\textbf{Componenti}: V, S, M (oli e unguenti rari del valore di almeno 1.000 mo, che l'incantesimo consuma)\\
\textbf{Durata}: Istantanea\\
Entri a contatto con un umanoide morto o un frammento di umanoide morto. Purché la creatura non sia morta da più di 10 giorni, l'incantesimo gli forma un nuovo corpo adulto e poi ne richiama l'anima affinché entri nel corpo. Se l'anima del bersaglio non è libera o consenziente a farlo, l'incantesimo fallisce.\\
La magia modella un nuovo corpo, che probabilmente provocherà un cambio di razza alla creatura. Il Narratore tira un d10 e consulta la seguente tabella per determinare quale forma assuma la creatura una volta riportata in vita, oppure sarà Il Narratore a scegliere la forma.\\

\medskip
\begin{tabular}{ll}
\textbf{d100} &\textbf{Razza}\\
\toprule
0 & Lupo/Aquila/Volpe/Lince (tirate 1d4)\\
1&Nano\\
2&Elfo\\
3&Mezzelfo\\
4&Mezzorco\\
5&Cinghiale/Tasso/Cane/Ratto (tirate 1d4)\\
6&Nibali\\
7&Diversi\\
8&Orso/Gufo/Procione/Gatto (tirate 1d4)\\
9&Umano\\
10&Stessa razza precedente\\
\end{tabular}

La creatura reincarnata ricorda la sua vita e le sue esperienze passate. Mantiene le capacità che aveva nella sua forma originale se e' in grado di applicarle.\\
\textbf{Questo incantesimo non non e' disponibile se non ai Devoti e Seguaci di Shayalia od Efrem}\\
\textit{Nota}: un Devoto o Seguace di Shayalia od Efrem reincanerà la creatura sempre in un animale, pero' potendo scegliere il tipo.\\
Non e' possibile reincarnarsi in uno gnomo se non si era prima uno gnomo.

\medskip\textbf{Resistenza}\index{Trucchetto - Resistenza}\\
\textbf{Scuola}: Abiurazione\\
\textbf{Difficoltà}: 12\\
\textbf{Tempo di Lancio}: 2 Azioni\\
\textbf{Gittata}: Contatto\\
\textbf{Componenti}: V, S, M (un mantello in miniatura)\\
\textbf{Durata}: Concentrazione, 1 minuto\\
Lanci l'incantesimo a contatto con una creatura consenziente. Una volta prima del termine dell'incantesimo, il bersaglio può tirare un d4 e sommare il risultato ottenuto a un Tiro Salvezza a sua scelta. Può tirare il dado prima o dopo aver effettuato il Tiro Salvezza. Poi l'incantesimo termina.

\medskip\textbf{Respirare Sott'Acqua}\index{Incantesimi - Respirare Sott'Acqua}\\
\textbf{Scuola}: Trasmutazione\\
\textbf{Difficoltà}: 21\\
\textbf{Tempo di Lancio}: 2 Azioni\\
\textbf{Gittata}: 9 metri\\
\textbf{Componenti}: V, S, M (una cannuccia o una pagliuzza)\\
\textbf{Durata}: 24 ore\\
Questo incantesimo consente a un massimo di dieci creature consenzienti a gittata e che puoi vedere, di respirare sott'acqua fino al termine dell'incantesimo. Le creature soggette mantengono anche il loro normale metodo di respirazione.\\
\textbf{Per ogni critico ottenuto} nella prova di magia puoi scegliere una creatura aggiuntiva.

\medskip\textbf{Resurrezione}\index{Incantesimi - Resurrezione}\\
\textbf{Scuola}: Necromanzia\\
\textbf{Difficoltà}: 31\\
\textbf{Tempo di Lancio}: 1 ora\\
\textbf{Gittata}: Contatto\\
\textbf{Componenti}: V, S, M (un diamante del valore di almeno 1.000 mo, che l'incantesimo consuma)\\
\textbf{Durata}: Istantanea\\
Lanci l'incantesimo a contatto di una creatura morta da non più di un secolo, che non è morta di vecchiaia e che non sia non morta. Se la sua anima è libera e consenziente, il bersaglio ritornerà in vita con tutti i suoi punti ferita.\\
Questo incantesimo neutralizza tutti i veleni e cura le normali malattie che affliggevano la creatura quando è morta. Tuttavia non rimuove malattie magiche, maledizioni e simili; se questi effetti non sono rimossi prima del lancio dell'incantesimo, affliggeranno il bersaglio al suo ritorno in vita.\\
Questo incantesimo chiude tutte le ferite mortali e ripristina qualsiasi parte del corpo mancante. Tornare dalla morte è un'ordalia. Il bersaglio subisce una penalità di -4 a tutti i Tiri per Colpire, Tiri Salvezza e prove di caratteristica. Ogni volta che il bersaglio termina una notte di riposo la penalità viene ridotta di 1 finché non scompare.\\
Lanciare questo incantesimo per riportare in vita una creatura che è morta da un anno o più ti sfianca. Fino al termine di una notte di riposo, non potrai più lanciare incantesimi e avrai -1d6 su tutti i Tiri per Colpire, prove di caratteristica e Tiri Salvezza.\\
\textbf{Questo incantesimo non dovrebbe essere disponibile. Solo un Patrono puo' riportare in vita.}

\medskip\textbf{Resurrezione Pura}\index{Incantesimi - Resurrezione Pura}\\
\textbf{Scuola}: Trasmutazione\\
\textbf{Difficoltà}: 36\\
\textbf{Tempo di Lancio}: 1 ora\\
\textbf{Gittata}: Contatto\\
\textbf{Componenti}: V, S, M (un po' di Acqua Benedetta e diamanti del valore di 25.000 mo, che l'incantesimo consuma)\\
\textbf{Durata}: Istantanea\\
Lanci l'incantesimo a contatto di una creatura morta da non più di 200 anni e che sia morta per qualsiasi motivo ma non di vecchiaia. Se la sua anima è libera e consenziente, la creatura ritornerà in vita con tutti i suoi punti ferita. \\
Questo incantesimo chiude tutte le ferite, neutralizza qualsiasi veleno, cura tutte le malattie e rimuove qualsiasi maledizione che affliggeva la creatura quando è morta. L'incantesimo rimpiazza gli organi e gli arti danneggiati.\\
l'incantesimo può fornire anche un nuovo corpo se l'originale non esiste più, in qual caso devi pronunciare il nome della creatura. La creatura riapparirà poi in uno spazio non occupato di tua scelta, entro 3 metri da te. \\
\textbf{Questo incantesimo non dovrebbe essere disponibile. Solo un Patrono puo' riportare in vita.}

\medskip\textbf{Rianimare Morti}\index{Incantesimi - Rianimare Morti}\\
\textbf{Scuola}: Necromanzia\\
\textbf{Difficoltà}: 26\\
\textbf{Tempo di Lancio}: 1 ora
\textbf{Gittata}: Contatto
\textbf{Componenti}: V, S, M (una diamante del valore di almeno 500 mo, che l'incantesimo consuma)\\
\textbf{Durata}: Istantanea\\
Riporti in vita una creatura morta, purché questa non sia morta da più di 10 giorni. Se l'anima della creatura è sia consenziente che libera di riunirsi al corpo, la creatura torna in vita con 1 punto ferita.\\
Questo incantesimo neutralizza anche qualsiasi veleno e cura le malattie non magiche che affliggevano la creatura al momento della morte. Questo incantesimo, tuttavia, non rimuove le malattie magiche, maledizioni o simili effetti; se questi non vengono rimossi prima del lancio dell'incantesimo, riprenderanno a manifestarsi quando la creatura torna in vita. L'incantesimo non può riportare in vita una creatura non morta.\\
Questo incantesimo richiude tutte le ferite mortali, ma non ripristina le parti del corpo mancanti. Se la creatura è priva di parti del corpo o organi fondamentali per la sopravvivenza (la testa, per esempio) l'incantesimo fallisce automaticamente.\\
Tornare dalla morte è un'ordalia. Il bersaglio subisce una penalità di -4 a tutti i Tiri per Colpire, Tiri Salvezza e prove di caratteristica. Ogni volta che il bersaglio termina una notte di riposo la penalità viene ridotta di 1 finché non scompare.\\
\textbf{Questo incantesimo non dovrebbe essere disponibile. Solo un Patrono puo' riportare in vita.}

\medskip\textbf{Rigenerazione}\index{Incantesimi - Rigenerazione}\\
\textbf{Scuola}: Trasmutazione\\
\textbf{Difficoltà}: 31\\
\textbf{Tempo di Lancio}: 1 minuto\\
\textbf{Gittata}: Contatto\\
\textbf{Componenti}: V, S, M (un rosario e Acqua Benedetta)\\
\textbf{Durata}: 1 ora\\
Lanci l'incantesimo a contatto di una creatura per stimolare la sua capacità di guarigione naturale. Il bersaglio recupera 4d8 + 15 punti ferita. Per la durata dell'incantesimo, il bersaglio recupera 1 punto ferita all'inizio di ciascun suo round (6 punti ferita al minuto). Le membra recise del corpo del bersaglio (dita, gambe, code e così via), se ne ha, vengono ripristinate in 2 minuti. Se hai la parte recisa e la tieni appoggiata al moncherino, l'incantesimo fa sì che l'arto si ricucia istantaneamente col moncherino.\\
\textbf{Per ogni critico ottenuto} nella prova di magia raddoppi i punti ferita recuperati per round.

\medskip\textbf{Rimuovi Malattia}\index{Incantesimi - Rimuovi Malattia}\\
\textbf{Scuola}: Cura\\
\textbf{Difficoltà}: 23\\
\textbf{Tempo di Lancio}: 1 turno\\
\textbf{Gittata}: Contatto\\
\textbf{Componenti}: V, S\\
\textbf{Durata}: Istantanea\\
Puoi porre fine a una malattia anche magica la cui Difficolta' (se magica o DC se naturale) sia inferiore alla prova di magia ottenuta lanciando questo incantesimo.\\

\medskip\textbf{Rimuovi Maledizione}\index{Incantesimi - Rimuovi Maledizione}\\
\textbf{Scuola}: Abiurazione\\
\textbf{Difficoltà}: 21\\
\textbf{Tempo di Lancio}: 2 Azioni\\
\textbf{Gittata}: Contatto\\
\textbf{Componenti}: V, S\\
\textbf{Durata}: Istantanea\\
Puoi terminare una maledizione la cui Difficolta' sia inferiore alla tua prova di magia ottenuta con il lancio di questo incantesimo.\\
Se l'oggetto è un oggetto magico maledetto, la maledizione resta, ma l'incantesimo permette di rimuovere l'oggetto e gettarlo.

\medskip\textbf{Rimuovi Veleno}\index{Incantesimi - Rimuovi Veleno}\\
\textbf{Scuola}: Cura\\
\textbf{Difficoltà}: 21\\
\textbf{Tempo di Lancio}: 1 round\\
\textbf{Gittata}: Contatto\\
\textbf{Componenti}: V, S\\
\textbf{Durata}: Istantanea\\
L'obiettivo oggetto dell'incantesimo non e' piu' avvelenato. Puoi terminare un avvelenamento la cui Difficolta' sia inferiore alla tua prova di magia ottenuta con il lancio di questo incantesimo.

\medskip\textbf{Rinascita}\index{Incantesimi - Rinascita}\\
\textbf{Scuola}: Cura\\
\textbf{Difficoltà}: 21\\
\textbf{Tempo di Lancio}: 2 Azioni\\
\textbf{Gittata}: Contatto\\
\textbf{Componenti}: V, S, M (diamante del valore di 300 mo, che l'incantesimo consuma)\\
\textbf{Durata}: Istantanea\\
Una creatura morta nell'ultimo minuto e con cui sei in contatto, ritorna in vita con 1 punto ferita. Questo incantesimo non può riportare in vita le persone morte di vecchiaia, né può ripristinare le parti del corpo mancanti.\\
\textbf{Nota}: a discrezione del Narratore questo potrebbe essere l'unico incantesimo concesso per riportare in vita una creatura, altrimenti vale la regola che solo un Patrono puo' riportare in vita.

\medskip\textbf{Riparare}\index{Trucchetto - Riparare}\\
\textbf{Scuola}: Trasmutazione\\
\textbf{Difficoltà}: 12\\
\textbf{Tempo di Lancio}: 1 minuto\\
\textbf{Gittata}: Contatto\\
\textbf{Componenti}: V, S, M (due calamite)\\
\textbf{Durata}: Istantanea\\
Questo incantesimo ripara una singola rottura o spaccatura in un oggetto con cui sei a contatto, come una catenella spezzata, due metà di una chiave rotta, un mantello lacerato, o un otre che perde. Purché la rottura o la spaccatura non sia più grande di 30 centimetri in qualsiasi dimensione, sei in grado di ripararle, senza lasciare traccia dei danni subiti. Questo incantesimo può riparare fisicamente un oggetto magico o un costrutto, ma non è in grado di ripristinare le funzioni magiche di questi oggetti.

\medskip\textbf{Riposo Inviolato}\index{Incantesimi - Riposo Inviolato}\\
\textbf{Scuola}: Necromanzia\\
\textbf{Difficoltà}: 19\\
\textbf{Tempo di Lancio}: 2 Azioni\\
\textbf{Gittata}: Contatto\\
\textbf{Componenti}: V, S, M (un pizzico di sale e un pezzo di rame posto su ciascun occhio del cadavere, che devono restare lì per la durata)\\
\textbf{Durata}: 10 giorni\\
Entri a contatto con un cadavere o altri resti. Per la durata, il bersaglio è protetto dalla putrefazione e non può diventare non morto. \\
\textbf{Per ogni critico ottenuto} nella prova di magia raddoppi la durata fino ad un massimo di un anno.

\medskip\textbf{Risata Incontenibile}\index{Incantesimi - Risata Incontenibile}\\
\textbf{Scuola}: Ammaliamento\\
\textbf{Difficoltà}: 16\\
\textbf{Tempo di Lancio}: 2 Azioni\\
\textbf{Gittata}: 9 metri\\
\textbf{Componenti}: V, S, M (piccole torte e una piuma che viene agitata nell'aria)\\
\textbf{Durata}: 1 minuto 
Una creatura a gittata di tua scelta e che puoi vedere percepisce tutto come tremendamente ilare e divertente, scoppiando in fragorose risate finché è soggetta a questo incantesimo. Il bersaglio deve superare un Tiro Salvezza su Volontà o cadere prono, restando inabile e incapace di rialzarsi per la durata. Le creature con un punteggio di Intelligenza -2 o meno, ignorano l'effetto.\\
Al termine di ciascun suo round, e ogni volta che subisce danni, il bersaglio può effettuare un altro Tiro Salvezza su Volontà. Il bersaglio ha +1d6 al Tiro Salvezza se questo è stato provocato dai danni. Se lo supera, l'incantesimo termina.

\medskip\textbf{Riscaldare il Metallo}\index{Incantesimi - Riscaldare il Metallo}\\
\textbf{Scuola}: Trasmutazione\\
\textbf{Difficoltà}: 19\\
\textbf{Tempo di Lancio}: 2 Azioni\\
\textbf{Gittata}: 18 metri\\
\textbf{Componenti}: V, S, M (un pezzo di ferro e una fiamma)\\
\textbf{Durata}: 1 minuto\\
Scegli un manufatto di metallo, come un'arma di metallo o un'armatura di metallo media o pesante, a gittata e che puoi vedere. Fai sì che l'oggetto risplenda di rosso per il calore. Qualsiasi creatura in contatto fisico con l'oggetto subisce 2d8 danni da fuoco quando lanci questo incantesimo. Fino al termine dell'incantesimo, puoi usare 2 Azioni per infliggere di nuovo questo danno nei tuoi turni successivi.\\
Se una creatura sta impugnando o indossando l'oggetto e subisce danno da esso, la creatura deve superare un Tiro Salvezza su Tempra o gettare l'oggetto se ne è in grado. Se non getta l'oggetto, ha -1d6 ai Tiri per Colpire e le prove di caratteristica fino all'inizio del suo prossimo round.\\
\textbf{Per ogni critico ottenuto} nella prova di magia il danno aumenta di 1d8.

\medskip\textbf{Ristorare Inferiore}\index{Incantesimi - Ristorare Inferiore}\\
\textbf{Scuola}: Cura\\
\textbf{Difficoltà}: 19\\
\textbf{Tempo di Lancio}: 2 Azioni\\
\textbf{Gittata}: Contatto\\
\textbf{Componenti}: V, S\\
\textbf{Durata}: Istantanea\\
Puoi porre fine a una malattia non magica o condizione che affligge una creatura con cui sei a contatto. La condizione può essere accecato, assordato o paralizzato. Da Esausto puo' portare ad Affaticato. Se la condizione e' stata causata da una magia la prova di magia di lancio di questo incantesimo deve superare la DC/Difficoltà che ha originato l'effetto.\\
Puoi recuperare 1 punto di Caratteristica perso non permanentemente.

\medskip\textbf{Ristorare Superiore}\index{Incantesimi - Ristorare Superiore}\\
\textbf{Scuola}: Cura\\
\textbf{Difficoltà}: 26\\
\textbf{Tempo di Lancio}: 2 Azioni\\
\textbf{Gittata}: Contatto\\
\textbf{Componenti}: V, S, M (polvere di diamante del valore di almeno 100 mo, che l'incantesimo consuma)\\
\textbf{Durata}: Istantanea\\
Imbevi una creatura a contatto di energia positiva per annullare un effetto debilitante. 
Se la condizione e' stata causata da una magia la prova di magia di lancio di questo incantesimo deve superare la DC/Difficoltà che ha originato l'effetto.\\
Rimuovi la condizione di Esausto o terminare uno dei seguenti effetti che affliggono il bersaglio: 
\medskip
\begin{itemize}
\item
Un effetto che ha affascinato il bersaglio.
\item
Fai recuperare 1d2 punti ad una statistica al bersaglio. Anche punti permanenti.
\item
Un effetto che riduce i punti ferita massimi del bersaglio.
\end{itemize}

\medskip\textbf{Risveglio}\index{Incantesimi - Risveglio}\\
\textbf{Scuola}: Trasmutazione\\
\textbf{Difficoltà}: 26\\
\textbf{Tempo di Lancio}: 8 ore\\
\textbf{Gittata}: Contatto\\
\textbf{Componenti}: V, S, M (un'agata del valore di almeno 1.000 mo, che l'incantesimo consuma)\\
\textbf{Durata}: Istantanea\\
Dopo aver trascorso il tempo di lancio a disegnare tracciati magici con una gemma preziosa, entri a contatto con una bestia o vegetale Enorme o di taglia inferiore. Il bersaglio deve essere privo di punteggio di Intelligenza o avere Intelligenza -2 o meno. Il bersaglio ottiene Intelligenza 0. Il bersaglio ottiene anche la capacità di parlare un linguaggio che conosci. Se il bersaglio è un vegetale, ottiene la capacità di muovere i suoi arti, radici, liane, rampicanti e così via, e ottiene sensi simili a quelli di un umano. Il Narratore sceglierà le statistiche appropriate al tipo di vegetale risvegliato, come le statistiche per il cespuglio risvegliato o l'albero risvegliato.\\
La bestia o vegetale risvegliato è affascinato da te per 30 giorni o finché tu o i tuoi compagni non gli arrecherete danno. Quando la condizione affascinato termina, la creatura risvegliata sceglie se rimanerti amichevole, in base a come l'hai trattata mentre era affascinata.\\
\textbf{Per ogni critico ottenuto} nella prova di magia raddoppi la durata della fascinazione fino ad un massimo di 1 anno.

\medskip\textbf{Ritirata Rapida}\index{Incantesimi - Ritirata Rapida}\\
\textbf{Scuola}: Trasmutazione\\
\textbf{Difficoltà}: 16\\
\textbf{Tempo di Lancio}: 1 Azione Immediata\\
\textbf{Gittata}: Personale\\
\textbf{Componenti}: V, S\\
\textbf{Durata}: Concentrazione, 1 minuto\\
Questo incantesimo ti permette di muoverti a un'andatura incredibile. Quando lanci questo incantesimo guadagni un Azione di Movimento bonus.\\
\textbf{Per ogni critico ottenuto} nella prova di magia la durata aumenta di 1 minuto.

\medskip\textbf{Saltare}\index{Incantesimi - Saltare}\\
\textbf{Scuola}: Trasmutazione\\
\textbf{Difficoltà}: 16\\
\textbf{Tempo di Lancio}: 2 Azioni\\
\textbf{Gittata}: Contatto\\
\textbf{Componenti}: V, S, M (la zampa posteriore di una cavalletta)\\
\textbf{Durata}: 1 minuto\\
La distanza di salto della creatura con cui sei in contatto al momento del lancio è triplicata fino al termine dell'incantesimo.\\

\medskip\textbf{Santificare}\index{Incantesimi - Santificare}\\
\textbf{Scuola}: Invocazione\\
\textbf{Difficoltà}: 26\\
\textbf{Tempo di Lancio}: 24 ore\\
\textbf{Gittata}: Contatto\\
\textbf{Componenti}: V, S, M (erbe, oli e incensi del valore di almeno 1.000 mo, che l'incantesimo consuma)\\
\textbf{Durata}: Fino a che dissolto\\
Infondi l'area circostante a un punto con cui sei in contatto del potere del tuo Patrono. L'area può avere un raggio massimo di 18 metri, e l'incantesimo fallisce se include un'area già sotto l'effetto di un incantesimo santificare. L'area soggetta all'incantesimo genera i seguenti effetti.
\textit{Per prima cosa}, celestiali, elementali, fatati, demoni e non morti non possono entrare nell'area, né una simile creatura può affascinare, spaventare o possederne altre al suo interno. Qualsiasi creatura affascinata, spaventata o posseduta da una creatura simile non è più affascinata,spaventata o posseduta dal momento in cui entra in quest'area. Puoi escludere uno o più tipi di queste creature da questo effetto.\\
\textit{Seconda cosa}, puoi vincolare un effetto ulteriore all'area. Scegli l'effetto dalla lista seguente, o scegline uno presentatoti dal Narratore. Alcuni di questi effetti si applicano alle creature nell'area; puoi decidere se gli effetti si applichino a tutte le creature, le creature Devote o Seguaci di specifica Patrono, o le creature di un tipo specifico, come orchi o troll. Quando una creatura soggetta all'incantesimo entra in quest'area per la prima volta durante un round o inizia il suo round qui, deve effettuare un Tiro Salvezza su Volontà. Se lo supera, la creatura ignora l'effetto aggiuntivo finché non lascia l'area.\\
\medskip
\begin{itemize}
\item
\textit{Coraggio}. Le creature soggette non possono essere spaventate mentre restano in quest'area. Interferenza Extradimensionale. Le creature soggette non possono muoversi o viaggiare usando il teletrasporto o altri mezzi extradimensionali o interplanari.
\item
\textit{Lingue}. Le creature soggette possono comunicare con qualsiasi altra creatura nell'area, anche se non condividono un linguaggio comune. 
\item
\textit{Luce Diurna}. Luce intensa riempie l'area. L'oscurità magica creata da incantesimi di piu' bassa Difficoltà di quella usata per lanciare questo incantesimo non possono estinguere la luce.
\item
\textit{Oscurità}. L'oscurità riempie l'area. La luce normale, e anche la luce magica creata da incantesimi di Difficoltà più bassa di quella usata per lanciare questo incantesimo, non possono illuminare l'area. 
\item
\textit{Paura}. Le creature soggette sono spaventate mentre restano in quest'area.
\item
\textit{Protezione dall'Energia}. Le creature soggette ricevono resistenza a un tipo di danno a tua scelta (a eccezione dei danni da botta, perforanti o taglienti), finché restano nell'area.
\item
\textit{Riposo Inviolato}. I corpi morti seppelliti nell'area non possono essere trasformati in non morti. 
\item
\textit{Silenzio}. Nessun suono può emanare dall'interno dell'area, e nessun suono può entrarvi.
\item
\textit{Vulnerabilità all'Energia}. Le creature soggette ricevono vulnerabilità a un tipo di danno a tua scelta (a eccezione dei danni da botta, perforanti o taglienti), finché restano nell'area.
\end{itemize}

\medskip\textbf{Santuario}\index{Incantesimi - Santuario}\\
\textbf{Scuola}: Abiurazione\\
\textbf{Difficoltà}: 16\\
\textbf{Tempo di Lancio}: 1 Azione Immediata\\
\textbf{Gittata}: 9 metri\\
\textbf{Componenti}: V, S, M (un piccolo specchio d'argento)\\
\textbf{Durata}: 1 minuto\\
Proteggi una creatura a gittata dagli attacchi. Fino al termine dell'incantesimo, qualsiasi creatura che prenda come bersaglio la creatura protetta con un attacco o incantesimo dannoso deve prima effettuare un Tiro Salvezza su Volontà. Se fallisce il Tiro Salvezza, l'attaccante deve scegliere un nuovo bersaglio o perdere l'attacco o l'incantesimo. Questo incantesimo non protegge la creatura protetta dagli effetti ad area, come lo scoppio di una palla di fuoco. Se la creatura protetta effettua un attacco o lancia un incantesimo che agisce su creature nemiche, l'incantesimo termina.

\medskip\textbf{Santuario Privato}\index{Incantesimi - Santuario Privato}\\
\textbf{Scuola}: Abiurazione\\
\textbf{Difficoltà}: 23\\
\textbf{Tempo di Lancio}: 10 minuti\\
\textbf{Gittata}: 36 metri\\
\textbf{Componenti}: V, S, M (un sottile foglio di piombo, un pezzo di vetro opaco, un batuffolo di cotone o tessuto, e crisolito in polvere)\\
\textbf{Durata}: 24 ore \\
Proteggi con la magia un'area. L'area è un cubo che può essere piccolo fino a 1 metro di spigolo o grande fino a 30 metri di spigolo. L'incantesimo agisce fino al termine della durata o finché non usi un'azione per interromperlo.\\
Quando lanci l'incantesimo, decidi che tipo di protezione questo fornisce, scegliendo una o più delle seguenti proprietà:\\
\medskip
\begin{itemize}
\item
Il suono non può attraversare il perimetro dell'area protetta.
\item
Il perimetro dell'area protetta appare buio e nebbioso, impedendo di vedervi attraverso (anche
alla scurovisione).
\item
Sensori creati da incantesimi di divinazione non possono apparire all'interno dell'area protetta o attraversare la sua barriera perimetrale.
\item
Le creature nell'area non possono essere bersaglio di incantesimi di divinazione.
\item
Nulla può teletrasportarsi dentro o fuori dell'area protetta.
\item
All'interno dell'area protetta, il viaggio planare è interdetto.
\end{itemize}
Lanciare questo incantesimo sullo stesso punto ogni giorno per un anno, rende l'effetto permanente.\\
\textbf{Per ogni critico ottenuto} nella prova di magia puoi aumentare le dimensioni del cubo di 10 metri di spigolo.

\medskip\textbf{Scagliare Maledizione}\index{Incantesimi - Scagliare Maledizione}\\
\textbf{Scuola}: Necromanzia\\
\textbf{Difficoltà}: 21\\
\textbf{Tempo di Lancio}: 2 Azioni\\
\textbf{Gittata}: Contatto\\
\textbf{Componenti}: V, S\\
\textbf{Durata}: 1 minuto\\
Una creatura con cui sei a contatto deve superare un Tiro Salvezza su Volontà o restare maledetta per la durata dell'incantesimo. Quando lanci questo incantesimo, scegli la natura della maledizione tra le seguenti opzioni:
\medskip
\begin{itemize}
\item
Scegli un punteggio di caratteristica. Mentre è maledetto, il bersaglio ha -1d6 alle prove di
caratteristica e i Tiri Salvezza basati eventualmente su quel punteggio di caratteristica.
\item
Mentre è maledetto, il bersaglio ha -1d6 ai Tiri per Colpire contro di te.
\item
Mentre è maledetto, il bersaglio deve effettuare un Tiro Salvezza su Volontà all'inizio di ciascun suo round. Se lo fallisce, spreca l'azione di quel suo round senza fare nulla.
\item
Mentre il bersaglio è maledetto, i tuoi attacchi e incantesimi infliggono 1d8 danni da Vuoto aggiuntivi contro di lui.
\end{itemize}
\medskip
L'incantesimo rimuovi maledizione (vedi descrizione) termina questo effetto. A discrezione del Narratore, puoi scegliere una maledizione dall'effetto diverso, ma non dovrebbe essere comunque più potente di quelle descritte qui sopra. Il Narratore detiene il giudizio finale sull'effetto di una maledizione.\\
\textbf{Se ottieni un critico} la durata della maledizione e' un giorno. Se ottieni 3 critici la durata e' permanente.

\medskip\textbf{Scassinare}\index{Incantesimi - Scassinare}\\
\textbf{Scuola}: Trasmutazione\\
\textbf{Difficoltà}: 19\\
\textbf{Tempo di Lancio}: 2 Azioni\\
\textbf{Gittata}: 18 metri\\
\textbf{Componenti}: V\\
\textbf{Durata}: Istantanea\\
Scegli un oggetto a gittata e che puoi vedere. L'oggetto può essere una porta, scatola, delle manette, una serratura o un altro oggetto che possieda un metodo comune o magico per prevenirne l'accesso.\\
Un bersaglio che è chiuso da una serratura comune o che è bloccato o sbarrato viene aperto, sbloccato o liberato. Se l'oggetto ha più serrature, solo una di queste viene aperta.\\
Se scegli un bersaglio che è tenuto chiuso con serratura arcana, quell'incantesimo resta soppresso per 10 minuti, durante i quali il bersaglio può essere aperto come di norma. Quando lanci questo incantesimo, un sonoro bussare, udibile fino a 90 metri di distanza, emana dall'oggetto bersaglio.\\
\textbf{Per ogni critico ottenuto} nella prova di magia puoi aprire un altro lucchetto/serratura.

\medskip\textbf{Sciame di Meteore}\index{Incantesimi - Sciame di Meteore}\\
\textbf{Scuola}: Invocazione\\
\textbf{Difficoltà}: 36\\
\textbf{Tempo di Lancio}: 2 Azioni\\
\textbf{Gittata}: 1,5 chilometri\\
\textbf{Componenti}: V, S\\
\textbf{Durata}: Istantanea\\
Sfere incandescenti di fuoco si schiantano a terra in quattro punti differenti a gittata e che puoi vedere. Ogni creatura, in una sfera di 2 metri di raggio centrata sul punto scelto da te, deve effettuare un Tiro Salvezza su Riflessi. La sfera si propaga intorno agli angoli. Una creatura subisce 20d6 danni da fuoco e 20d6 danni da botta se fallisce il Tiro Salvezza, o la metà di
questi danni se lo supera. Una creatura nell'area di più di uno scoppio infuocato ne subisce gli effetti solo una volta.\\
\textbf{Successo/Fallimento Critico}: In caso si fallimento critico il danno raddoppia, in caso di successo critico il danno viene ulteriormente dimezzato\\
\textbf{Ogni 3 critici ottenuti} nella prova di magia scegli un altro punto di impatto.

\medskip\textbf{Scolpire Pietra}\index{Incantesimi - Scolpire Pietra}\\
\textbf{Scuola}: Trasmutazione\\
\textbf{Difficoltà}: 23\\
\textbf{Tempo di Lancio}: 2 Azioni\\
\textbf{Gittata}: Contatto\\
\textbf{Componenti}: V, S, M (argilla malleabile, che deve essere lavorata per ottenere una vaga forma dell'oggetto di pietra)\\
\textbf{Durata}: Istantanea\\
Scolpisci in qualsiasi forma che si presti ai tuoi scopi un oggetto di pietra di taglia Media o inferiore o una sezione di pietra non più grossa di 1 metro in qualsiasi direzione, con cui sei in contatto.\\
Così, per esempio, potresti scolpire una grossa pietra in un'arma, idolo o feretro, o creare un piccolo passaggio attraverso il muro, purché il muro sia spesso meno di 1 metro. Potresti anche modellare una porta di pietra o la sua cornice per sigillare la porta. L'oggetto che crei può avere fino a due cardini e un chiavistello, ma è impossibile creare meccanismi più complessi.

\medskip\textbf{Scopri il Percorso}\index{Incantesimi - Scopri il Percorso}\\
\textbf{Scuola}: Divinazione\\
\textbf{Difficoltà}: 29\\
\textbf{Tempo di Lancio}: 1 minuto\\
\textbf{Gittata}: Personale\\
\textbf{Componenti}: V, S, M (degli attrezzi da divinazione - dei bastoncini d'avorio, ossa, carte, denti o rune incise - del valore di almeno 100 mo e un oggetto dal luogo che desideri trovare)\\
\textbf{Durata}: 1 giorno\\
Questo incantesimo ti permette di trovare la rotta fisica più breve e diretta verso uno specifico luogo fisso con cui hai familiarità ed è sullo stesso piano di esistenza. Se indichi una destinazione su di un altro piano di esistenza, una destinazione che si muove (come una fortezza mobile) o una destinazione non specifica (come "la tana di un drago verde"), l'incantesimo fallisce.\\
Per la durata dell'incantesimo, finché sei nello stesso piano di esistenza della destinazione, saprai quanto è distante e in che direzione si trovi. Mentre sei in viaggio verso di essa, ogni volta che ti si presenterà la possibilità di scegliere tra percorsi diversi, determinerai automaticamente qual è la via più breve e la rotta più diretta (ma non necessariamente la più sicura) per raggiungere la destinazione.\\
\textbf{Per ogni critico} ottenuto nella prova di magia l'incantesimo dura 8 ore in piu'.

\medskip\textbf{Scopri Trappole}\index{Incantesimi - Scopri Trappole}\\
\textbf{Scuola}: Divinazione\\
\textbf{Difficoltà}: 19\\
\textbf{Tempo di Lancio}: 2 Azioni\\
\textbf{Gittata}: 36 metri\\
\textbf{Componenti}: V, S\\
\textbf{Durata}: 1 ora\\
Per la durata dell'incantesimo avverti la presenza di qualsiasi trappola a gittata che sia nella tua linea di visuale. Una trappola, ai fini di questo incantesimo, comprende qualsiasi cosa che sia in grado di infliggere un effetto improvviso o inaspettato che tu possa considerare dannoso o indesiderabile, e che è stato espressamente inteso come tale dal suo creatore. Di conseguenza, l'incantesimo percepirebbe un'area sotto l'incantesimo allarme, un glifo di interdizione, o una botola meccanica, ma non rivelerebbe una debolezza naturale del pavimento, un soffitto instabile o una buca nascosta.\\
La trappola viene evidenziata alla tua vista con un segnale viola.

\medskip\textbf{Scrigno Segreto}\index{Incantesimi - Scrigno Segreto}\\
\textbf{Scuola}: Evocazione\\
\textbf{Difficoltà}: 23\\
\textbf{Tempo di Lancio}: 2 Azioni\\
\textbf{Gittata}: Contatto\\
\textbf{Componenti}: V, S, M (un forziere lavorato, 1 metro x 50 cm x 50 cm, costruito con rari materiali del valore di almeno 5.000 mo, e una sua replica Minuscola fatta degli stessi materiali e del valore di almeno 50) \\
\textbf{Durata}: Istantanea\\
Nascondi un forziere e tutti i suoi contenuti sul Piano Etereo. Quando lanci questo incantesimo devi essere in contatto con il forziere e la replica in miniatura che serve da componente materiale. Il forziere può contenere fino a 0,25 metri cubi di materiale non vivente (1 x metro x 50 centimetri x 50 centimetri). Mentre il forziere rimane sul Piano Etereo, puoi usare un'azione per entrare in contatto con la replica e richiamare il forziere. Esso riapparirà in uno spazio non occupato sul terreno entro 1 metro da te. Puoi rispedire il forziere nel Piano Etereo, usando un'azione ed entrando in contatto sia col forziere che con la replica.\\
Dopo 60 giorni, c'è una percentuale cumulativa del 5\% al giorno che l'effetto dell'incantesimo abbia termine. \\
L'effetto termina se l'incantesimo viene lanciato nuovamente, se la replica del forziere viene distrutta, o se decidi di terminare l'incantesimo con un'azione. Se l'incantesimo termina e il forziere si trova sul Piano Etereo, viene irrimediabilmente perduto.

\medskip\textbf{Scritto Illusorio}\index{Incantesimi - Scritto Illusorio}\\
\textbf{Scuola}: Illusione\\
\textbf{Difficoltà}: 16\\
\textbf{Tempo di Lancio}: 1 minuto\\
\textbf{Gittata}: Contatto\\
\textbf{Componenti}: S, M (un inchiostro a base di piombo del valore di almeno 10 mo, che l'incantesimo consuma)\\
\textbf{Durata}: 10 giorni\\
Scrivi su di una pergamena, un pezzo di carta o qualche altro materiale adatto a scrivere e lo infondi di una potente illusione che permane per la durata dell'incantesimo.\\
Per te e qualsiasi creatura da te indicata al lancio dell'incantesimo, la scritta appare normale, con la tua grafia, e trasmette qualsiasi significato volevi comunicare quando hai vergato il testo. Per tutti gli altri, la scritta appare come se fosse redatta in una scrittura ignota o magica, che risulta incomprensibile. In alternativa, puoi far sì che la scritta sembri un messaggio totalmente diverso, in una grafia e linguaggio differente, sebbene debba essere un linguaggio a te conosciuto.\\
In caso l'incantesimo venisse dissolto, sia la scritta originale che l'illusione svaniscono. Una creatura con visione del vero può leggere il messaggio nascosto.

\medskip\textbf{Scrutare}\index{Incantesimi - Scrutare}\\
\textbf{Scuola}: Divinazione\\
\textbf{Difficoltà}: 26\\
\textbf{Tempo di Lancio}: 10 minuti\\
\textbf{Gittata}: Personale\\
\textbf{Componenti}: V, S, M (un focus del valore di almeno 1.000 mo, come una sfera di cristallo, un specchio d'argento o una fonte ricolma di Acqua Benedetta)\\
\textbf{Durata}: Concentrazione, massimo 10 minuti\\
Puoi vedere e udire una particolare creatura a tua scelta che si trovi sul tuo stesso piano di esistenza. Il bersaglio deve effettuare un Tiro Salvezza su Volontà, modificato da quanto bene conosci il bersaglio e la tua connessione fisica a esso. Se il bersaglio sa che stai lanciando l'incantesimo, può fallire volontariamente il Tiro Salvezza, in caso desiderasse essere osservato da
te.
\medskip
\begin{tabular}{ll}
\toprule
\textbf{Conoscenza} & \textbf{Mod. al Tiro Salvezza}\\
Ne hai sentito parlare &+5\\
Hai incontrato il bersaglio &+0\\
Conosci bene il bersaglio &-5\\
\end{tabular}

\medskip

\begin{tabular}{ll}
	\toprule
\textbf{Connessione} & \textbf{Mod. Tiro Salvezza}\\
Descrizione o immagine &-2\\
Proprietà o indumento & -4\\
Parte del corpo (capelli...)&-10\\
\end{tabular}
\medskip

Se supera il Tiro Salvezza, il bersaglio ignora gli effetti dell'incantesimo, e non potrai usare di nuovo questo incantesimo contro di lui prima che siano passate 24 ore.\\
Se il Tiro Salvezza fallisce, l'incantesimo crea un sensore invisibile entro 3 metri dal bersaglio. Tramite il sensore puoi udire e vedere come se fossi sul posto. Il sensore si muove assieme al bersaglio, rimanendo entro 3 metri da lui per la durata dell'incantesimo. Una creatura che può vedere oggetti invisibili vede il sensore come una sfera luminosa delle dimensioni all'incirca di un pugno.\\
Invece di prendere come bersaglio una creatura, puoi scegliere come bersaglio dell'incantesimo un luogo che hai già visto in passato. Quando scegli questa opzione, il sensore compare in quel luogo ma non si muove. 

\medskip\textbf{Scudo}\index{Incantesimi - Scudo}\\
\textbf{Scuola}: Abiurazione\\
\textbf{Difficoltà}: 16\\
\textbf{Tempo di Lancio}: 1 reazione, che effettui quando sei colpito da un attacco o bersaglio dell'incantesimo dardo incantato\\
\textbf{Gittata}: Personale\\
\textbf{Componenti}: V, S\\
\textbf{Durata}: 1 round\\
Compare una barriera di forza magica invisibile a proteggerti. Fino all'inizio del tuo prossimo round hai un bonus di +5 alla Difesa compreso l'attacco innescante, e non subisci danni da dardo incantato.

\medskip\textbf{Scudo della Fede}\index{Incantesimi - Scudo della Fede}\\
\textbf{Scuola}: Abiurazione\\
\textbf{Difficoltà}: 16\\
\textbf{Tempo di Lancio}: 1 Azione Immediata\\
\textbf{Gittata}: 18 metri\\
\textbf{Componenti}: V, S, M (una piccola pergamena con su scritto un frammento di testo sacro)\\
\textbf{Durata}: 10 minuti\\
Compare un campo scintillante che circonda una creatura a gittata, scelta da te, conferendole un bonus di +2 alla Difesa per la durata dell'incantesimo.

\medskip\textbf{Scudo di Fuoco}\index{Incantesimi - Scudo di Fuoco}\\
\textbf{Scuola}: Invocazione\\
\textbf{Difficoltà}: 23\\
\textbf{Tempo di Lancio}: 2 Azioni\\
\textbf{Gittata}: Personale\\
\textbf{Componenti}: V, S, M (un po' di fosforo o una lucciola) \\
\textbf{Durata}: 10 minuti\\
Fiamme sottili e vaporose avvolgono il tuo corpo per la durata dell'incantesimo, emettendo luce intensa in un raggio di 3 metri e luce fioca per ulteriori 3 metri. Puoi terminare l'incantesimo in anticipo, usando un'azione per interromperlo.\\
Le fiamme ti forniscono uno scudo caldo o uno scudo freddo, a tua scelta. Lo scudo caldo ti conferisce resistenza al danno da freddo, mentre lo scudo freddo ti fornisce resistenza al danno da caldo.\\
Inoltre, ogni qualvolta una creatura entro 1 metro da te ti colpisce con un attacco in mischia, lo scudo erutta fiamme. L'attaccante subisce 2d8 danni da fuoco da uno scudo caldo, o 2d8 danni da freddo da uno scudo freddo.

\medskip\textbf{Scurovisione}\index{Incantesimi - Scurovisione}\\
\textbf{Scuola}: Trasmutazione\\
\textbf{Difficoltà}: 19\\
\textbf{Tempo di Lancio}: 2 Azioni\\
\textbf{Gittata}: Contatto\\
\textbf{Componenti}: V, S, M (o un pizzico di carota o di agata secca)\\
\textbf{Durata}: 8 ore\\
Una creatura consenziente con cui sei in contatto ottiene la capacità di vedere al buio. Per la durata dell'incantesimo, quella creatura ha scurovisione fino a una gittata di 18 metri.

\medskip\textbf{Segugio Fedele}\index{Incantesimi - Segugio Fedele}\\
\textbf{Scuola}: Evocazione\\
\textbf{Difficoltà}: 23\\
\textbf{Tempo di Lancio}: 2 Azioni\\
\textbf{Gittata}: 9 metri\\
\textbf{Componenti}: V, S, M (un minuscolo fischietto d'argento, e un pezzo d'osso, e un filo)\\
\textbf{Durata}: 8 ore\\
Puoi evocare un cane da guardia fantasma in uno spazio non occupato a gittata e che puoi vedere, dove rimarrà per la durata dell'incantesimo, finché non viene congedato con un'azione, o finché non si allontanerà più di 30 metri da te.\\
Il segugio è invisibile a tutte le creature eccetto che a te e non può essere danneggiato. Quando una creatura di taglia Piccola o superiore si avvicina entro 9 metri da esso senza aver prima pronunciato la parola d'ordine da te specificata quando hai lanciato l'incantesimo, il segugio inizia ad abbaiare a grande volume. Il segugio vede le creature invisibili e può vedere nel Piano Etereo. Esso ignora le illusioni. All'inizio di ciascun tuo round, il segugio tenta di mordere una creatura entro 1 metro da esso e che ti sia ostile. Il bonus di attacco del segugio è uguale al tuo modificatore di caratteristica da incantatore + CM. Se colpisce, infligge 2d8 danni perforanti.

\medskip\textbf{Sembrare}\index{Incantesimi - Sembrare}\\
\textbf{Scuola}: Illusione\\
\textbf{Difficoltà}: 26\\
\textbf{Tempo di Lancio}: 2 Azioni\\
\textbf{Gittata}: 9 metri\\
\textbf{Componenti}: V, S\\
\textbf{Durata}: 8 ore\\
Questo incantesimo ti permette di cambiare l'aspetto di un qualsiasi numero di creature a gittata e che puoi vedere. Fornisci a ciascun bersaglio un nuovo aspetto illusorio. Una creatura non consenziente può effettuare un Tiro Salvezza su Volontà e, se lo supera, ignora l'incantesimo.\\
L'incantesimo camuffa l'aspetto fisico oltre che gli abiti, le armature, le armi e l'equipaggiamento. Puoi far sembrare ciascuna creatura 30 centimetri più bassa o più alta, sembrare magra, grassa o una via di mezzo. Non puoi cambiare la conformazione del corpo del bersaglio, e quindi devi scegliere una forma che abbia la stessa distribuzione basilare di arti. \\
Per tutto il resto, l'illusione è limitata solo dalla tua fantasia. L'incantesimo permane per la sua durata, a meno che tu non usi una azione per interromperlo prima. I cambi apportati da questo incantesimo non sono in grado di sostenere un'ispezione fisica. Per esempio, se usi questo incantesimo per aggiungere un cappello all'abbigliamento di una creatura, gli oggetti attraversano il cappello, e chiunque lo tocchi non avvertirebbe nulla e finirebbe per toccare la testa e i capelli della creatura.
Se usi questo incantesimo per apparire più magro di quello che sei, la mano di una persona che provasse a toccarti rimbalzerebbe su di te, mentre alla vista sembrerebbe fermarsi a mezz'aria. Una creatura può usare 2 Azioni per ispezionare un bersaglio ed effettuare una prova di Consapevolezza contro la DC del Tiro Salvezza dell'incantesimo, se impiega 3 Azione ha +1d6 di bonus. Se la riesce, capisce che il bersaglio è camuffato.

\medskip\textbf{Semipiano}\index{Incantesimi - Semipiano}\\
\textbf{Scuola}: Evocazione\\
\textbf{Difficoltà}: 34\\
\textbf{Tempo di Lancio}: 2 Azioni\\
\textbf{Gittata}: 18 metri\\
\textbf{Componenti}: S\\
\textbf{Durata}: 1 ora\\
Crei una porta d'ombra su di una superficie piana a gittata e che puoi vedere. La porta è grande abbastanza da permettere il passaggio senza problemi a una creatura Media. Quando viene aperta, la porta conduce a un semipiano che appare come una stanza vuota di 9 metri in ciascuna dimensione, fatta di legno e pietra. Quando l'incantesimo termina, la porta scompare, e qualsiasi creatura od oggetto all'interno del semipiano rimane intrappolato lì, mentre la porta scompare anche dall'altro lato.\\
Ogni volta che esegui questo incantesimo, crei un nuovo semipiano, oppure permetti alla porta d'ombra di connettersi a un semipiano creato da un precedente lancio dell'incantesimo oppure aumenti di altri 9 metri in ciascuna dimensione un semipiano conosciuto creato da te precedentemente. \\
Inoltre, se conosci la natura e i contenuti di un semipiano creato dal lancio di questo incantesimo da parte di un'altra creatura, puoi far sì che la porta d'ombra si colleghi invece a quel semipiano.

\medskip\textbf{Serratura Arcana}\index{Incantesimi - Serratura Arcana}\\
\textbf{Scuola}: Abiurazione\\
\textbf{Difficoltà}: 19\\
\textbf{Tempo di Lancio}: 2 Azioni\\
\textbf{Gittata}: Contatto\\
\textbf{Componenti}: V, S, M (polvere d'oro del valore di almeno 25 mo, che viene consumata dall'incantesimo) \\
\textbf{Durata}: Fino a che dissolto\\
Lanci l'incantesimo a contatto di una porta, finestra, portale, forziere o altro ingresso chiuso, e questo diventa chiuso a chiave per la durata. Tu e le creature che hai indicato, quando hai lanciato questo incantesimo, potete aprire l'oggetto normalmente. Puoi anche predisporre una parola d'ordine che, quando pronunciata entro 1 metro dall'oggetto, sopprime l'incantesimo per 1 minuto. Altrimenti l'apertura è invalicabile fino a che non viene distrutta ol'incantesimo è dissolto o soppresso. Lanciare scassinare sull'oggetto sopprime serratura arcana per 10 minuti.\\
Mentre è soggetto a questo incantesimo, l'oggetto è più difficile da distruggere o aprire a forza; la DC per romperlo o scassinare una serratura su di esso aumenta di 10.

\medskip\textbf{Servitore Invisibile}\index{Incantesimi - Servitore Invisibile}\\
\textbf{Scuola}: Evocazione\\
\textbf{Difficoltà}: 16\\
\textbf{Tempo di Lancio}: 2 Azioni\\
\textbf{Gittata}: 18 metri\\
\textbf{Componenti}: V, S, M (un pezzo di corda e un pezzo di legno)\\
\textbf{Durata}: 1 ora\\
Questo incantesimo crea una forza quasi invisibile solo delimitata da una leggera aura (di colore a tua scelta) che svolge dei semplici compiti al tuo comando, fino al termine dell'incantesimo. Il servitore si forma in uno spazio sul terreno non occupato, entro la gittata. Ha Difesa 10, 1 punto ferita, Forza 0 e non può attaccare. Se scende a 0 punti ferita, l'incantesimo ha termine.\\
Come Azione Immediata, durante ciascun tuo round, puoi comandare mentalmente il servitore di muoversi fino a 4 metri e interagire con un oggetto. Il servitore può svolgere dei semplici compiti alla stregua di un servitore umano, come raccogliere cose, pulire, riparare, piegare abiti, accendere fuochi, servire il cibo e versare il vino. Una volta impartito il comando, il servitore svolgerà il compito al meglio delle sue capacità finché non l'avrà completato, e poi aspetterà il tuo prossimo comando. \\
Se comandi al servitore di svolgere un compito che lo farà muovere a più di 18 metri da te, l'incantesimo termina.

\medskip\textbf{Sfera Congelante}\index{Incantesimi - Sfera Congelante}\\
\textbf{Scuola}: Invocazione\\
\textbf{Difficoltà}: 29\\
\textbf{Tempo di Lancio}: 2 Azioni\\
\textbf{Gittata}: 90 metri\\
\textbf{Componenti}: V, S, M (una piccola sfera di cristallo)\\
\textbf{Durata}: Istantanea\\
Un globo gelido di energia fredda parte dalla punta delle tue dita verso un punto di tua scelta a gittata, dove esplode in una sfera di 18 metri di raggio. Ogni creatura nell'area deve effettuare un Tiro Salvezza su Tempra. Se fallisce il Tiro Salvezza, una creatura subisce 10d6 danni da freddo. Se lo supera, subisce la metà di questi danni.\\
Se il globo colpisce un corpo d'acqua o un liquido composto principalmente d'acqua (escluse però le creature a base d'acqua), congela il liquido fino a una profondità di 15 centimetri in un'area quadrata di 9 metri di lato. Il ghiaccio dura 1 minuto. Le creature che stavano nuotando sulla superficie dell'acqua congelata restano intrappolate nel ghiaccio. Una creatura intrappolata può usare due azioni per effettuare una prova di Forza contro la DC del Tiro Salvezza dell'incantesimo, al fine di liberarsi.\\
Se lo desideri, dopo aver completato l'incantesimo, puoi trattenerti dallo sparare il globo. Un piccolo globo, circa delle dimensioni di una pietra da fionda, freddo al contatto, appare nella tua mano. In qualsiasi momento, tu, o una creatura a cui hai dato il globo, potete lanciare il globo (fino a una gittata di 12 metri). Questo si frantumerà all'impatto, con lo stesso effetto del normale lancio dell'incantesimo. Puoi anche poggiare il globo a terra senza che si frantumi. Dopo 1 minuto, se il globo non è già stato frantumato, esploderà.\\
\textbf{Per ogni critico ottenuto} nella prova di magia il danno aumenta di 1d6

\medskip\textbf{Sfera Elastica}\index{Incantesimi - Sfera Elastica}\\
\textbf{Scuola}: Invocazione\\
\textbf{Difficoltà}: 23\\
\textbf{Tempo di Lancio}: 2 Azioni\\
\textbf{Gittata}: 90 metri\\
\textbf{Componenti}: V, S, M (un pezzo semisferico di cristallo trasparente e un pezzo semisferico corrispondente di gomma arabica)\\
\textbf{Durata}: Concentrazione, massimo 1 minuto\\
Una sfera di energia luminosa avvolge una creatura od oggetto di taglia Grande o inferiore a gittata. Una creatura non consenziente deve effettuare un Tiro Salvezza su Riflessi. Se lo fallisce, la creatura è avvolta dall'incantesimo per la sua durata.\\
Nulla (né oggetti fisici, né energia, né altri effetti di incantesimi) può attraversare questa barriera, in entrata o uscita, sebbene una creatura all'interno della sfera possa respirare senza problemi. La sfera è immune a tutti i danni, e una creatura al suo interno non può essere danneggiata da attacchi o effetti originanti dall'esterno, né una creatura all'interno della sfera può danneggiare nulla che si trovi all'esterno. La sfera è priva di peso e grande giusto a sufficienza per contenere la creatura o l'oggetto al suo interno. Una creatura avvolta può usare 1 Azione per spingere contro le pareti della sfera e quindi farla rotolare fino alla metà della velocità della creatura. Allo stesso modo, il globo può essere raccolto e mosso da altre creature.\\
Un incantesimo disintegrazione che prenda come bersaglio il globo lo distrugge senza danneggiare nulla al suo interno.

\medskip\textbf{Sfera Infuocata}\index{Incantesimi - Sfera Infuocata}\\
\textbf{Scuola}: Evocazione\\
\textbf{Difficoltà}: 19\\
\textbf{Tempo di Lancio}: 2 Azioni\\
\textbf{Gittata}: 18 metri\\
\textbf{Componenti}: V, S, M (un po' di sego, un pizzico di zolfo, e una manciata di ferro in polvere)\\
\textbf{Durata}: 1 minuto\\
Per la durata dell'incantesimo compare una sfera di 1 metro di diametro in uno spazio a gittata, scelto da te. Qualsiasi creatura che termini il suo round entro 1 metro dalla sfera deve effettuare un Tiro Salvezza su Riflessi. La creatura subisce 2d6 danni da fuoco se fallisce il Tiro Salvezza, o la metà di questi danni se lo supera.\\
Con un'azione puoi spostare la sfera di 9 metri. Se fai schiantare la sfera contro una creatura, la creatura deve effettuare un Tiro Salvezza contro il danno della sfera, e la sfera smetterà di muoversi per quel round.
Quando muovi la sfera, la puoi spostare oltre barriere alte fino a 1 metro, e farle saltare spazi larghi fino a 3 metri. La sfera incendia gli oggetti infiammabili non indossati o trasportati, e irradia una luce intensa in un raggio di 6 metri e una luce fioca per ulteriori 6 metri.\\
Mentre hai questo incantesimo attivo sei Distratto nel lancio di altri incantesimi.\\
\textbf{Per ogni critico ottenuto} nella prova di magia il danno aumenta di 1d6.\\

\medskip\textbf{Sfocatura}\index{Incantesimi - Sfocatura}\\
\textbf{Scuola}: Illusione\\
\textbf{Difficoltà}: 19\\
\textbf{Tempo di Lancio}: 2 Azioni\\
\textbf{Gittata}: Personale\\
\textbf{Componenti}: V\\
\textbf{Durata}: 1 minuto \\
Il tuo corpo diventa sfocato, indistinto e tremolante a chiunque ti veda. Per la durata dell'incantesimo, tutte le creature hanno ha -1d6 ai Tiri per Colpire contro di te. Gli attaccanti che non si affidano alla vista sono immuni a questo effetto, per esempio se hanno vista cieca o sono in grado di distinguere le illusioni, come per visione del vero.

\medskip\textbf{Sguardo Penetrante}\index{Incantesimi - Sguardo Penetrante}\\
\textbf{Scuola}: Necromanzia\\
\textbf{Difficoltà}: 29\\
\textbf{Tempo di Lancio}: 2 Azioni\\
\textbf{Gittata}: Personale\\
\textbf{Componenti}: V, S\\
\textbf{Durata}: Concentrazione, massimo 1 minuto\\
Per la durata dell'incantesimo, i tuoi occhi si tramutano in un vuoto nero infuso di terribile potere. Una creatura a tua scelta entro 18 metri da te e che puoi vedere, deve superare un Tiro Salvezza su Volontà o, per la durata, subire uno dei seguenti effetti di tua scelta. Durante ciascun tuo round, fino al termine dell'incantesimo, puoi usare due Azioni per prendere come bersaglio un'altra creatura, ma non puoi prendere di nuovo come bersaglio una creatura che abbia superato un Tiro Salvezza contro questo lancio di sguardo penetrante.\\
\medskip
\begin{itemize}
\item
\textit{Addormentato}. Il bersaglio cade privo di sensi. Si risveglia qualora subisca qualsiasi ammontare di danno o se un'altra creatura usa 2 Azioni per scuoterlo dal sonno.
\item
\textit{Ammalato}. Il bersaglio ha -1d6 ai Tiri per Colpire e le prove di caratteristica. Al termine di ciascun suo round, può effettuare un altro Tiro Salvezza su Volontà. Se lo supera, l'effetto ha termine.
\item
\textit{Impanicato}. Il bersaglio è spaventato da te. Durante ciascun suo round, la creatura spaventata deve effettuare usare due Azioni di Movimento e muoversi lontano da te tramite il tragitto più breve e sicuro possibile, a meno che non abbia spazio per muoversi. Se il bersaglio si muove in un luogo lontano almeno 18 metri da te, dove non ti possa vedere, questo effetto ha termine.
\end{itemize}

\medskip\textbf{Silenzio}\index{Incantesimi - Silenzio}\\
\textbf{Scuola}: Illusione\\
\textbf{Difficoltà}: 19\\
\textbf{Tempo di Lancio}: 2 Azioni\\
\textbf{Gittata}: 36 metri\\
\textbf{Componenti}: V, S\\
\textbf{Durata}: 10 minuti\\
Per la durata dell'incantesimo, nessun suono può essere creato all'interno o attraversare una sfera di 6 metri di raggio centrata su di un punto a gittata, scelto da te. Qualsiasi creatura o oggetto che si trovi completamente all'interno della sfera è immune al danno da tuono, e le creature che sono completamente al suo interno sono assordate. È impossibile lanciare un incantesimo che comprende una componente verbale mentre si è al suo interno.

\medskip\textbf{Simbolo}\index{Incantesimi - Simbolo}
\textbf{Scuola}: Abiurazione\\
\textbf{Difficoltà}: 31\\
\textbf{Tempo di Lancio}: 2 Azioni\\
\textbf{Gittata}: Contatto\\
\textbf{Componenti}: V, S, M (mercurio, fosforo e diamante e opale in polvere con un valore totale di almeno 1.000 mo, che l'incantesimo consuma)\\
\textbf{Durata}: Fino a che dissolto o attivato\\
Quando lanci questo incantesimo, inscrivi un glifo dannoso su di una superficie (come una sezione di pavimento, muro o un tavolo) o all'interno di un oggetto che può essere chiuso per nascondere il glifo (come un libro, una pergamena o un forziere). Se scegli una superficie, il glifo può coprire un'area di superficie non maggiore di 3 metri di diametro. Se scegli un oggetto,quell'oggetto deve restare al suo posto; se l'oggetto viene spostato più di 3 metri dal punto in cui è stato lanciato l'incantesimo, il glifo è spezzato, e l'incantesimo termina senza essere stato attivato.\\
Il glifo è quasi invisibile e può essere trovato con una prova di Sopravvivenza contro la DC del Tiro Salvezza dei tuoi incantesimi.\\
Decidi tu cosa attivi il glifo al momento del lancio dell'incantesimo.\\
Per i glifi inscritti su di una superficie, l'attivazione tipica comprende entrare in contatto o stare sopra il glifo, rimuovere un altro oggetto che copra il glifo, avvicinarsi a una certa distanza dal glifo, o manipolare l'oggetto su cui è inscritto il glifo.\\
Per i glifi inscritti su di un oggetto, l'attivazione tipica comprende aprire l'oggetto, avvicinarsi a una certa distanza dall'oggetto, o vedere o leggere il glifo.\\
Puoi definire meglio l'attivazione così che l'incantesimo si attivi solo in determinate circostanze o secondo certe peculiarità fisiche (come l'altezza o il peso) o specie di creatura (per esempio, la protezione potrebbe agire contro le megere o i mutaforma). Puoi anche predisporre condizioni per evitare che il glifo venga attivato, come la pronuncia di una parola d'ordine.\\
Quando inscrivi il glifo scegli una delle opzioni seguenti come suo effetto. Una volta attivato, il glifo riluce, riempiendo una sfera di 18 metri di raggio di luce fioca per 10 minuti, dopo i quali l'incantesimo termina. Ogni creatura nella sfera quando il glifo si attiva diventabersaglio del suo effetto, così come una creatura cheentri per la prima volta nella sfera durante un round o termine lì il suo round.\\
\medskip
\begin{itemize}
\item
\textit{Demenza}. Ogni bersaglio deve effettuare un Tiro Salvezza su Volontà. Se fallisce il Tiro Salvezza, il bersaglio diventa demente per 1 minuto. Una creatura demente non può effettuare azioni, non comprende quello che gli altri le dicono, non può leggere, e parla solo farfugliando. Il Narratore ne controlla i movimenti, che risultano erratici.\\
\item
\textit{Discordia}. Ogni bersaglio deve effettuare un Tiro Salvezza su Tempra. Se lo fallisce, il bersaglio inizia a bisticciare e discutere con un'altra creatura per 1 minuto. In questo periodo, è incapace di effettuare qualsiasi comunicazione significativa e ha -1d6 ai Tiri per Colpire e le prove di caratteristica. Dolore. Ogni bersaglio deve effettuare un Tiro Salvezza su Tempra. Se lo fallisce, il bersaglio diventa inabile a causa del dolore lacerante.
\item
\textit{Morte}. Ogni bersaglio deve effettuare un Tiro Salvezza su Tempra, subendo 10d10 danni da Vuoto se lo fallisce, o la metà di questi danni se lo supera. Paura. Ogni bersaglio deve effettuare un Tiro Salvezza su Volontà e, se lo fallisce, restare spaventato per 1 minuto. Mentre è spaventato, il bersaglio getta qualsiasi cosa stesse tenendo e deve muoversi almeno 9 metri lontano dal glifo durante ciascuno suo round, se in grado.
\item\
\textit{Sfiducia}. Ogni bersaglio deve effettuare un Tiro Salvezza su Volontà. Se fallisce il Tiro Salvezza, il bersaglio è sopraffatto dalla disperazione per 1 minuto. Durante questo periodo, non può attaccare o prendere come bersaglio nessuna creatura con capacità, incantesimi o altri effetti magici nocivi.
\item\
\textit{Sonno}. Ogni bersaglio deve effettuare un Tiro Salvezza su Volontà, e cadere privo di sensi per 10 minuti se lo fallisce. Una creatura si risveglia se subisce danni o se qualcuno usa un'azione per risvegliarla. 
\item
\textit{Stordimento}. Ogni bersaglio deve effettuare un Tiro Salvezza su Volontà, e restare stordito per 1 minuto se lo fallisce.
\end{itemize}

\medskip\textbf{Simulacro}\index{Incantesimi - Simulacro}\\
\textbf{Scuola}: Illusione\\
\textbf{Difficoltà}: 31\\
\textbf{Tempo di Lancio}: 12 ore\\
\textbf{Gittata}: Contatto\\
\textbf{Componenti}: V, S, M (neve o ghiaccio in quantità per creare una copia a dimensioni reali della creatura duplicata; un po' di capelli, unghie o altro pezzo del corpo di quella creatura da piazzare in mezzo alla neve o al ghiaccio; e un rubino in polvere del valore di 1.500 mo, sparso sopra il duplicato e consumato dall'incantesimo)\\
\textbf{Durata}: Fino a che dissolto\\
Modelli un duplicato illusorio di una bestia o umanoide che resti a gittata per l'intero tempo di lancio dell'incantesimo. Il duplicato è una creatura, in parte reale e formata di ghiaccio o neve, che può effettuare azioni e interagire come una normale creatura. Sembra essere identica all'originale, ma ha la metà dei punti ferita massimi di quella creatura e si presenta priva di equipaggiamento. Altrimenti, l'illusione usa tutte le statistiche della creatura che duplica.\\
Il simulacro è amichevole verso di te e le creature da te indicate. Obbedisce ai comandi da te pronunciati, muovendosi e agendo in accordo ai tuoi desideri e agendo durante il tuo round in combattimento. Il simulacro è privo della capacità di apprendere o diventare più potente, e quindi non accresce mai di livello o nelle caratteristiche, né può recuperare gli slot incantesimi spesi.\\
Se il simulacro è danneggiato, puoi ripararlo in un laboratorio alchemico, usando erbe rare e minerali del valore di 100 mo per punto ferita recuperato. Il simulacro rimane finché non scende a 0 punti ferita, a quel punto si ritrasforma in neve e si scioglie all'istante. Se lanci di nuovo questo incantesimo, qualsiasi duplicato da te creato con questo incantesimo attualmente attivo viene immediatamente distrutto.

\medskip\textbf{Sogno}\index{Incantesimi - Sogno}\\
\textbf{Scuola}: Illusione\\
\textbf{Difficoltà}: 26\\
\textbf{Tempo di Lancio}: 2 Azioni\\
\textbf{Gittata}: Speciale\\
\textbf{Componenti}: V, S, M (una manciata di sabbia, una punta di inchiostro, e una penna per scrivere presa da un volatile addormentato)\\
\textbf{Durata}: 8 ore\\
Questo incantesimo modella i sogni di una creatura. Scegli una creatura a te nota come bersaglio dell'incantesimo. Il bersaglio deve trovarsi sul tuo stesso piano di esistenza. Le creature che non dormono non possono essere soggette a questo incantesimo. Tu o una creatura consenziente con cui sei a contatto entrate in uno stato di trance, agendo da messaggero. Mentre è in trance, il messaggero è consapevole di ciò che lo circonda, ma non può effettuare azioni o muoversi.\\
Per la durata dell'incantesimo, se il bersaglio è addormentato, il messaggero appare nei sogni del bersaglio e può conversare con lui finché questi rimane addormentato. Il messaggero può anche modellare l'ambiente del sogno, creando terreni, oggetti e altre immagini. Il messaggero può emergere dalla trance in qualsiasi momento, terminando anticipatamente l'effetto dell'incantesimo. Al risveglio, il bersaglio ricorda perfettamente il suo sogno. Se il bersaglio è sveglio quando lanci l'incantesimo, il messaggero ne viene a conoscenza e può porre fine alla trance (e all'incantesimo) o aspettare che il bersaglio si addormenti. A quel punto il messaggero potrà comparire nei sogni del bersaglio.\\
Puoi fare apparire il messaggero al bersaglio con un aspetto mostruoso e terrificante. Se lo fai, il messaggero può consegnare un messaggio di al massimo dieci parole e poi il bersaglio deve effettuare un Tiro Salvezza su Volontà. Se fallisce il Tiro Salvezza, gli echi della spaventosa mostruosità generano un incubo per la durata del sonno del bersaglio che gli impedisce di ottenere qualsiasi beneficio da quel riposo. Inoltre, quando il bersaglio si sveglia, subisce 3d6 danni.\\
Se possiedi una ciocca di capelli, delle unghie tagliate, o simile porzione del corpo del bersaglio, egli effettuerà il suo Tiro Salvezza con -1d6.

\medskip\textbf{Sonno}\index{Incantesimi - Sonno}\\
\textbf{Scuola}: Ammaliamento\\
\textbf{Difficoltà}: 16\\
\textbf{Tempo di Lancio}: 2 Azioni\\
\textbf{Gittata}: 27 metri\\
\textbf{Componenti}: V, S, M (un pizzico di sabbia, petali di rosa o un grillo)\\
\textbf{Durata}: 1 minuto\\
Questo incantesimo pone le creature in un torpore magico. Tira 5d8; il totale è il numero di punti ferita di creature su cui l'incantesimo può agire. Le creature, entro 6 metri dal punto a gittata scelto da te, sono influenzate in ordine ascendente di punti ferita (ignorando le creature svenute).\\
A partire dalla creatura con il numero più basso di punti ferita attuali, ogni creatura soggetta a questo incantesimo perde i sensi fino al termine dell'incantesimo, chi dorme subisce danni, o qualcuno usa un'azione per scuotere o prendere a schiaffi l'addormentato. Sottrarre i punti ferita di ciascuna creatura dal totale prima di considerare la creatura con il valore di punti ferita più basso successiva. I punti ferita di una creatura devono essere uguali o inferiori al totale rimanente perché l'effetto agisca su di essa. Non morti e creature che non possono essere affascinate non sono influenzate da questo incantesimo.\\
\textbf{Per ogni critico ottenuto} nella prova di magia tira 2d8 pf aggiuntivi.

\medskip\textbf{Spada Arcana}\index{Incantesimi - Spada Arcana}\\
\textbf{Scuola}: Invocazione\\
\textbf{Difficoltà}: 31\\
\textbf{Tempo di Lancio}: 2 Azioni\\
\textbf{Gittata}: 18 metri\\
\textbf{Componenti}: V, S, M (una spada di platino in miniatura con l'impugnatura e il pomello di rame e zinco, del valore di 250 mo)\\
\textbf{Durata}: Concentrazione, massimo 1 minuto \\
Per la durata dell'incantesimo, crei a gittata un piano di forza a forma di spada fluttuante. Quando la spada appare, effettui un attacco in mischia con modificatore CM + modificatore da incantesimo contro un bersaglio scelto da te entro 1 metro dalla spada. Se colpisci, il bersaglio subisce 3d10 danni da forza. Fino al termine dell'incantesimo,puoi usare un'azione ogni tuo round per muovere la spada di 6 metri in un punto che puoi vedere e ripetere questo attacco contro lo stesso bersaglio o uno differente.

\medskip\textbf{Spostamento Planare}\index{Incantesimi - Spostamento Planare}\\
\textbf{Scuola}: Evocazione\\
\textbf{Difficoltà}: 31\\
\textbf{Tempo di Lancio}: 2 Azioni\\
\textbf{Gittata}: Contatto\\
\textbf{Componenti}: V, S, M (una verga di metallo biforcuta del valore di almeno 250 mo, sintonizzata verso uno specifico piano di esistenza)\\
\textbf{Durata}: Istantanea\\
Tu e un massimo di altre otto creature consenzienti, che si stringono le mani per formare un cerchio, venite trasportati su di un diverso piano di esistenza. Puoi specificare una destinazione bersaglio in termini generici, e riapparirai all'interno o in prossimità di quella destinazione, a discrezione del Narratore.\\
In alternativa, se conosci la sequenza di sigilli di un cerchio di teletrasporto verso un altro piano di esistenza, l'incantesimo può condurti a quel cerchio. Se il cerchio di teletrasporto è troppo piccolo per contenere tutte le creature che trasporti con te, esse riappariranno nello spazio non occupato più vicino possibile al cerchio.\\
Puoi usare questo incantesimo per bandire una creatura non consenziente in un altro piano. Scegli una creatura a portata ed effettua un attacco in mischia con incantesimo contro di essa. Se colpisci, la creatura deve effettuare un Tiro Salvezza su Volontà. Se la creatura fallisce il Tiro Salvezza, viene trasportata in un luogo casuale sul piano di esistenza da te specificato. Una creatura così trasportata dovrà trovare per proprio conto il modo di tornare sul tuo attuale piano di esistenza.

\medskip\textbf{Spruzzo Colorato}\index{Incantesimi - Spruzzo Colorato}\\
\textbf{Scuola}: Illusione\\
\textbf{Difficoltà}: 16\\
\textbf{Tempo di Lancio}: 2 Azioni\\
\textbf{Gittata}: Personale (cono di 4 metri)\\
\textbf{Componenti}: V, S, M (un pizzico di polvere o sabbia che sia colorata di rosso, giallo e blu)\\
\textbf{Durata}: 1 round\\
Dalla tua mano si sprigiona una raffica di luci colorate e abbaglianti. Tira 6d10; il totale è l'ammontare di punti ferita di creature su cui questo incantesimo agisce. Le creature, in un cono di 4 metri che origina da te, sono soggette in ordine ascendente dei loro attuali punti ferita (ignorando le creature prive di sensi e le creature che non possono vedere).\\
A partire dalla creatura che ha il minor numero di punti ferita attuali, ciascuna creatura soggetta a questo incantesimo resta accecata fino al termine dell'incantesimo. Sottrarre i punti ferita di ciascuna creatura dal totale prima di passare alla creatura col totale più basso di punti ferita successiva. I punti ferita di una creatura devono essere uguali o minori del totale rimanente perché l'incantesimo agisca su di essa. \\
\textbf{Per ogni critico ottenuto} nella prova di magia tira 2d10 aggiuntivi.

\medskip\textbf{Spruzzo Prismatico}\index{Incantesimi - Spruzzo Prismatico}\\
\textbf{Scuola}: Invocazione\\
\textbf{Difficoltà}: 31\\
\textbf{Tempo di Lancio}: 2 Azioni\\
\textbf{Gittata}: Personale (cono di 18 metri)\\
\textbf{Componenti}: V, S\\
\textbf{Durata}: Istantanea\\
Otto raggi di luce multicolore partono dalla tua mano. Ogni raggio è di un diverso colore e ha un potere e uno scopo diverso. Ogni creatura in un cono di 18 metri deve effettuare un Tiro Salvezza su Riflessi. Per ogni bersaglio, tirare un d8 per determinare quale sia il colore del raggio che lo ha colpito.\\
\medskip
\begin{itemize}
\item
\textit{1. Rosso}. Il bersaglio subisce 10d6 danni da fuoco se fallisce il Tiro Salvezza, o la metà di questi danni se lo supera.
\item
\textit{2. Arancio}. Il bersaglio subisce 10d6 danni da acido se fallisce il Tiro Salvezza, o la metà di questi danni se lo supera.
\item
\textit{3. Giallo}. Il bersaglio subisce 10d6 danni da fulmine se fallisce il Tiro Salvezza, o la metà di questi danni se lo supera.
\item
\textit{4. Verde}. Il bersaglio subisce 10d6 danni da veleno se fallisce il Tiro Salvezza, o la metà di questi danni se lo supera.
\item
\textit{5. Blu}. Il bersaglio subisce 10d6 danni da freddo se fallisce il Tiro Salvezza, o la metà di questi danni se lo supera.
\item
\textit{6. Indaco}. Se fallisce il Tiro Salvezza, il bersaglio è intralciato. Deve poi effettuare un Tiro Salvezza su Tempra all'inizio di ciascun suo round. Se supera il Tiro Salvezza tre volte, l'incantesimo termina. Se fallisce il Tiro Salvezza tre volte, viene permanentemente trasformato in pietra e diventa vittima della condizione pietrificato. I successi e i fallimenti non devono essere consecutivi; tieni traccia di entrambi finché il bersaglio non ne ha ottenuti tre dello stesso tipo.
\item
\textit{7. Violetto}. Se fallisce il Tiro Salvezza, il bersaglio è accecato. Deve poi effettuare un Tiro Salvezza su Volontà all'inizio del tuo prossimo round. Se supera il Tiro Salvezza, la cecità termina. Se fallisce il Tiro Salvezza, la creatura viene trasportata su di un altro piano di esistenza a scelta del Narratore e non è più accecata (di solito, una creatura che non è sul suo piano natio, viene esiliata su di esso, mentre le altre creature sono di solito portate nei piani Astrale o Etereo).
\item
\textit{8. Speciale}. Il bersaglio è colpito da due raggi. Tira altre due volte, ritirando gli 8.
\end{itemize}

\medskip\textbf{Spruzzo Velenoso}\index{Trucchetto - Spruzzo Velenoso}\\
\textbf{Scuola}: Evocazione\\
\textbf{Difficoltà}: 12\\
\textbf{Tempo di Lancio}: 2 Azioni\\
\textbf{Gittata}: 3 metri\\
\textbf{Componenti}: V, S\\
\textbf{Durata}: Istantanea\\
Stendi la mano verso una creatura a gittata e che puoi vedere, e proietti una nube di gas velenoso dal tuo palmo. La creatura deve superare un Tiro Salvezza su Tempra o subire 1d12 danni da veleno. \\
Il danno dell'incantesimo aumenta di 1d8 quando raggiungi CM 5, CM 11 e CM 17.

\medskip\textbf{Stretta Folgorante}\index{Trucchetto - Stretta Folgorante}\\
\textbf{Scuola}: Invocazione\\
\textbf{Difficoltà}: 12\\
\textbf{Tempo di Lancio}: 2 Azioni\\
\textbf{Gittata}: Contatto\\
\textbf{Componenti}: V, S\\
\textbf{Durata}: Istantanea\\
Dalle tue mani saettano lampi che infliggono una scossa a una creatura con cui provi a entrare in contatto. Effettua un attacco in mischia con incantesimo contro il bersaglio. Hai +1d6 sul tiro per colpire se il bersaglio sta indossando un'armatura fatta di metallo.Se colpisci, il bersaglio subisce 1d8 danni da fulmine, e non può effettuare reazioni fino all'inizio del suo prossimo round.\\
Il danno dell'incantesimo aumenta di 1d8 quando raggiungi CM 5, CM 11 e CM 17.

\medskip\textbf{Suggestione}\index{Incantesimi - Suggestione}\\
\textbf{Scuola}: Ammaliamento\\
\textbf{Difficoltà}: 19\\
\textbf{Tempo di Lancio}: 2 Azioni\\
\textbf{Gittata}: 9 metri\\
\textbf{Componenti}: V, M (la lingua di un serpente e un pezzo di favo o un goccio di olio dolce)\\
\textbf{Durata}: 8 ore \\
Suggerisci un corso di attività (limitato a una o due frasi) e influenzi magicamente una creatura a gittata e che puoi vedere e udire e ti possa capire, scelta da te. Le creature che non possono essere affascinate sono immuni a questo effetto. La suggestione deve essere pronunciata in modo che il corso d'azione suoni ragionevole. Chiedere a una creatura di pugnalarsi,gettarsi su una lancia, darsi fuoco, o fare qualche altro atto palesemente dannoso nega automaticamente gli effetti dell'incantesimo.\\
Il bersaglio deve effettuare un Tiro Salvezza su Volontà. Se fallisce il Tiro Salvezza, esso segue il corso d'azione da te descritto al meglio delle sue capacità. Il corso d'azione suggerito può proseguire per l'intera durata dell'incantesimo. Se l'attività suggerita può essere completata in un tempo più breve,l'incantesimo ha termine quando il soggetto termina di fare ciò che gli è stato chiesto.\\
Puoi anche specificare condizioni che attiveranno un'attività speciale per la durata dell'incantesimo. Per esempio, potresti suggerire a un cavaliere di cedere il suo cavallo da guerra al primo mendicante che incontri. Se la condizione non viene soddisfatta prima del termine dell'incantesimo, l'attività non verrà svolta. Se tu o uno qualsiasi dei tuoi compagni danneggia il bersaglio, l'incantesimo ha termine.\\

\medskip\textbf{Suggestione di Massa}\index{Incantesimi - Suggestione di Massa}\\
\textbf{Scuola}: Ammaliamento\\
\textbf{Difficoltà}: 29\\
\textbf{Tempo di Lancio}: 2 Azioni\\
\textbf{Gittata}: 18 metri\\
\textbf{Componenti}: V, M (la lingua di un serpente e un pezzo di favo o un goccio di olio dolce)\\
\textbf{Durata}: 24 ore\\
Suggerisci un corso di attività (limitato a una o due frasi) e influenzi magicamente fino a dodici creature a gittata che puoi vedere e udire e ti possano capire, scelte da te. Le creature che non possono essere affascinate sono immuni a questo effetto. La suggestione deve essere pronunciata in modo che il corso d'azione suoni ragionevole. Chiedere a una creatura di pugnalarsi, gettarsi su di una lancia, darsi fuoco, o fare qualche altro atto palesemente dannoso nega automaticamente gli effetti dell'incantesimo.\\
Ogni bersaglio deve effettuare un Tiro Salvezza su Volontà. Se fallisce il Tiro Salvezza, esso segue il corso d'azione da te descritto al meglio delle sue capacità. Il corso d'azione suggerito può proseguire per l'intera durata dell'incantesimo. Se l'attività suggerita può essere completata in un tempo più breve, l'incantesimo ha termine quando il soggetto termina di fare ciò che gli è stato chiesto.\\
Puoi anche specificare condizioni che attiveranno un'attività speciale per la durata dell'incantesimo. Per esempio, potresti suggerire a un gruppo di soldati di cedere tutti i loro soldi al primo mendicante che incontrino. Se la condizione non viene soddisfatta prima del termine dell'incantesimo, l'attività non verrà svolta. Se tu o uno qualsiasi dei tuoi compagni danneggia una creatura soggetta a questo incantesimo, per quella creatura l'incantesimo ha termine.\\
\textbf{Per ogni critico ottenuto} nella prova di magia aggiungi un giorno alla durata.

\medskip\textbf{Taumaturgia}\index{Trucchetto - Taumaturgia}\\
\textbf{Scuola}: Universale\\
\textbf{Difficoltà}: 12\\
\textbf{Tempo di Lancio}: 2 Azioni\\
\textbf{Gittata}: 9 metri\\
\textbf{Componenti}: V\\
\textbf{Durata}: Massimo 1 minuto\\
Manifesti a gittata una trucco minore, un segno di potere soprannaturale. Crei a gittata uno dei seguenti effetti magici:
\medskip
\begin{itemize}
\item
La tua voce risuona tre volte più forte del normale per 1 minuto.
\item
Fai sì che le fiamme tremolino, si intensifichino, affievoliscano o cambino colore per 1 minuto.
\item
Provochi innocui tremori sul terreno per 1 minuto. 
\item
Crei un rumore istantaneo, come un rombo di tuono, il verso di un corvo, o un sussurro inquietante, che origina da un punto a gittata scelto da te.
\item
Fai sì che una porta o una finestra non chiusa a chiave si spalanchi o si chiuda di colpo.
\item
Modifichi l'aspetto dei tuoi occhi per 1 minuto.
\end{itemize}
\medskip
Se lanci questo incantesimo più volte, puoi tenere attivi fino a tre effetti da un minuto alla volta, e puoi interrompere questi effetti con un'azione.

\medskip\textbf{Telecinesi}\index{Incantesimi - Telecinesi}\\
\textbf{Scuola}: Trasmutazione\\
\textbf{Difficoltà}: 26\\
\textbf{Tempo di Lancio}: 2 Azioni\\
\textbf{Gittata}: 18 metri\\
\textbf{Componenti}: V, S\\
\textbf{Durata}: Concentrazione, massimo 10 minuti \\
Ottieni la capacità di muovere o manipolare creature o oggetti tramite il pensiero. Quando lanci questo incantesimo, e come 2 Azioni durante ciascun round, puoi esercitare la tua volontà su di una creatura od oggetto a gittata e che puoi vedere, provocando l'effetto appropriato tra quelli seguenti. Puoi agire round dopo round sempre sullo stesso bersaglio, o sceglierne uno nuovo ogni volta. Se cambi bersaglio, il bersaglio precedente non è più soggetto all'incantesimo.
\textit{Creatura}. Puoi tentare di muovere una creatura di taglia Enorme o più piccola. Effettua una prova di caratteristica usando la tua caratteristica da incantatore contesa da una prova di Forza della creatura. Se vinci la contesa, muovi la creatura di 9 metri in qualsiasi direzione, compreso verso l'alto, ma senza eccedere la gittata dell'incantesimo. Fino al termine del tuo prossimo round, la creatura è intralciata dalla tua presa telecinetica. Una creatura sollevata in alta, resta sospesa a mezz'aria.\\
Nei round successivi, puoi usare 2 Azioni per tentare di mantenere la tua presa telecinetica sulla creatura ripetendo la contesa.
\textit{Oggetto}. Puoi tentare di muovere un oggetto che pesa fino a 500 chili. Se l'oggetto non è indossato o trasportato, lo sposti automaticamente di 9 metri in qualsiasi direzione, ma senza superare la gittata dell'incantesimo.\\
Se l'oggetto è indossato o trasportato da una creatura, devi effettuare una prova di caratteristica con la tua caratteristica da incantatore contesa dalla prova di Forza della creatura. Se vinci la contesa, trascini via l'oggetto da quella creatura e lo muovi di 9 metri in una qualsiasi direzione, senza però superare la gittata dell'incantesimo.\\
Puoi esercitare un controllo preciso sugli oggetti tramite la tua presa telecinetica, riuscendo così a manipolare un attrezzo semplice, aprire una porta o un contenitore,inserire o recuperare un oggetto da un contenitore aperto, o versare del materiale in una fiala.

\medskip\textbf{Teletrasporto}\index{Incantesimi - Teletrasporto}\\
\textbf{Scuola}: Evocazione\\
\textbf{Difficoltà}: 31\\
\textbf{Tempo di Lancio}: 2 Azioni\\
\textbf{Gittata}: 3 metri\\
\textbf{Componenti}: V\\
\textbf{Durata}: Istantanea\\
Questo incantesimo teletrasporta istantaneamente te e altre otto creature consenzienti (oppure un singolo oggetto) a gittata e che puoi vedere, scelte da te, in una destinazione di tua scelta. Se il bersaglio è un oggetto, deve poter entrare in un cubo di 3 metri di spigolo, e non può essere tenuto o trasportato da una creatura non consenziente.\\
La destinazione che scegli ti deve essere nota, e deve essere sullo stesso piano di esistenza in cui ti trovi. La tua familiarità con la destinazione determina se vi riesce ad arrivare.\\
Il DM tira un d100 e consulta la tabella.
\end{multicols}
\medskip
\begin{tabular}{lllll}
\toprule
d100 				&	Errore		&	Area Simile	&Fuori Bersaglio&Sul Bersaglio\\
Cerchio permanente	&-	&-	&-	&01-100\\
Oggetto Associato	&-	&-	&-	&01-100\\
Molto Familiare		&01-05	&06-13	&14-24	&25-100\\
Visto per caso		&01-33	&34-43	&44-53	&54-100\\
Visto una volta		&01-43	&44-53	&54-73	&74-100\\
Descrizione			&01-43	&44-53	&54-73	&74-100\\
Falsa Destinazione	&01-50	&51-100	&-	&-\\
\end{tabular}
\medskip
\begin{multicols}{2}

\textit{Cerchio permanente} indica un cerchio di teletrasporto permanente di cui conosci la sequenza dei sigilli.\\
\textit{Oggetto associato} indica che possiedi uno ggetto preso negli ultimi sei mesi dalla destinazione desiderata, come il libro della biblioteca di un mago, biancheria della suite reale, o un pezzo di marmo della tomba segreta di un lich.\\
\textit{Molto familiare} è un luogo in cui sei stato molto spesso, un posto che hai studiato attentamente, o un posto che puoi vedere quando lanci l'incantesimo.\\
\textit{Visto casualmente} è un posto che hai visto più di una volta ma con cui non sei molto familiare. \\
\textit{Visto una volta} è un posto che hai visto una volta sola, magari tramite la magia.\\ \textit{Descrizione} è un luogo la cui posizione e aspetto conosci solo tramite la descrizione di qualcun altro, magari una mappa.\\
\textit{Falsa destinazione} è un posto che non esiste. Magari hai cercato di scrutare il nascondiglio di un nemico ma hai invece visto un'illusione, oppure stai cercando di teletrasportarti in un posto familiare che non esiste più. \\
\textit{Sul Bersaglio}. Tu e il tuo gruppo (o l'oggetto bersaglio) apparite dove desideri.\\
\textit{Fuori Bersaglio}. Tu e il tuo gruppo (o l'oggetto bersaglio) apparite a una distanza casuale dalla destinazione in una direzione casuale. La distanza fuori bersaglio è 1d10 x 1d10 percento della distanza viaggiata. Per esempio, se hai provato a viaggiare per 180 chilometri, atterri fuori bersaglio e tiri 5 e 3 su due d10, allora saresti fuori bersaglio del 15\%, ovvero 27 chilometri. Il Narratore determina la direzione fuori bersaglio casualmente, tirando un d8 e indicando l'1 come nord, il 2 come nordest, il 3 come est e così via seguendo le direzioni della bussola. Se ti stai teletrasportando in una città costiera e finisci 27 chilometri al largo in mare, potresti essere nei guai!\\
\textit{Area Simile}. Tu e il tuo gruppo (o l'oggetto bersaglio) finite in un'area diversa che è visualmente o tematicamente simile all'area bersaglio. Per esempio, se sei diretto al tuo laboratorio personale, potresti finire nel laboratorio di un altro mago o in un negozio di oggetti alchemici che possiede molti degli attrezzi e strumenti del tuo laboratorio. In genere, compari nel luogo simile più vicino, ma dato che l'incantesimo non ha limiti di gittata, potresti finire praticamente dovunque sullo stesso piano.\\
\textit{Errore}. L'imprevedibile magia dell'incantesimo provoca un viaggio difficile. Ogni creatura teletrasportata (o l'oggetto bersaglio) subisce 3d10 danni da forza, e il Narratore ritira sulla tabella per vedere dove finiscano (possono capitare più errori, che infliggono danni ogni
volta).

\medskip\textbf{Tempesta di Fuoco}\index{Incantesimi - Tempesta di Fuoco}\\
\textbf{Scuola}: Invocazione\\
\textbf{Difficoltà}: 31\\
\textbf{Tempo di Lancio}: 2 Azioni\\
\textbf{Gittata}: 45 metri\\
\textbf{Componenti}: V, S\\
\textbf{Durata}: Istantanea\\
Una tempesta composta di fiamme ruggenti compare in un punto a gittata, scelto da te. L'area della tempesta consiste di un massimo di dieci cubi di 3 metri di spigolo, che puoi disporre come preferisci. Ogni cubo deve avere almeno una faccia adiacente a quella di un altro cubo. Ogni creatura nell'area deve effettuare un Tiro Salvezza su Riflessi. Se lo fallisce subisce 7d10 danni da fuoco, o la metà di questi danni se lo supera. Il fuoco danneggia gli oggetti nell'area e incendia gli oggetti infiammabili che non sono indossati o trasportati. Se lo desideri, la vita vegetale nell'area resta illesa dagli effetti di questo incantesimo. \\
\textbf{Successo/Fallimento Critico}: In caso si fallimento critico il danno raddoppia, in caso di successo critico il danno viene ulteriormente dimezzato

\medskip\textbf{Tempesta di Ghiaccio}\index{Incantesimi - Tempesta di Ghiaccio}\\
\textbf{Scuola}: Invocazione\\
\textbf{Difficoltà}: 23\\
\textbf{Tempo di Lancio}: 2 Azioni\\
\textbf{Gittata}: 90 metri\\
\textbf{Componenti}: V, S, M (un pizzico di polvere e alcune gocce d'acqua)\\
\textbf{Durata}: Istantanea\\
Una grandinata di ghiaccio si abbatte a terra in un cilindro di 6 metri di raggio e 12 metri di altezza centrato su di un punto a gittata. Ogni creatura nel cilindro deve effettuare un Tiro Salvezza su Riflessi. La creatura subisce 2d8 danni da botta e 4d6 danni da freddo se fallisce il Tiro Salvezza, o la metà se lo supera. La grandine trasforma l'area di effetto della tempesta in terreno difficile fino al termine del tuo prossimo round.
\textbf{Per ogni critico ottenuto} nella prova di magia il danno aumento di 1d8.\\
\textbf{Successo/Fallimento Critico}: In caso si fallimento critico il danno raddoppia, in caso di successo critico il danno viene ulteriormente dimezzato

\medskip\textbf{Tempesta di Nevischio}\index{Incantesimi - Tempesta di Nevischio}\\
\textbf{Scuola}: Evocazione\\
\textbf{Difficoltà}: 21\\
\textbf{Tempo di Lancio}: 2 Azioni\\
\textbf{Gittata}: 45 metri\\
\textbf{Componenti}: V, S, M (un pizzico di polvere e qualche goccia d'acqua)\\
\textbf{Durata}: 1 minuto\\
Fino al termine dell'incantesimo, pioggia gelida e nevischio si abbattono in un cilindro alto 6 metri e del raggio di 12 metri centrato in un punto da te scelto a gittata. L'area è in penombra, mentre le fiamme esposte vengono spente. Il terreno nell'area è coperto di ghiaccio scivoloso, rendendolo terreno difficile. Quando una creatura entra nell'area dell'incantesimo per la prima volta durante un round o inizia il suo round lì, deve effettuare un Tiro Salvezza su Riflessi. Se lo fallisce, cade prona. Se una creatura nell'area dell'incantesimo si sta concentrando, deve superare un Tiro Salvezza su Tempra contro la DC del Tiro Salvezza dell'incantesimo o perdere la concentrazione. 

\medskip\textbf{Tentacoli Neri}\index{Incantesimi - Tentacoli Neri}\\
\textbf{Scuola}: Evocazione\\
\textbf{Difficoltà}: 23\\
\textbf{Tempo di Lancio}: 2 Azioni\\
\textbf{Gittata}: 27 metri\\
\textbf{Componenti}: V, S, M (un pezzo di tentacolo di una piovra gigante o di un calamaro gigante)\\
\textbf{Durata}: 1 minuto\\
Viscidi tentacoli d'ebano riempiono un quadrato di 6 metri di lato sul terreno, a gittata e che puoi vedere. Per la durata dell'incantesimo, questi tentacoli trasformano l'area in terreno difficile.\\
Quando una creatura entra nell'area soggetta per la prima volta in un round o comincia qui il suo round, la creatura deve superare un Tiro Salvezza su Riflessi o subire 3d6 danni da botta e rimanere intralciata dai tentacoli fino al termine dell'incantesimo. Una creatura che inizia il suo round nell'area ed è già intralciata dai tentacoli, subisce 3d6 danni da botta. Una creatura intralciata dai tentacoli può usare 2 Azioni per effettuare una prova di Forza o Destrezza (a sua scelta) contro la DC del Tiro Salvezza dell'incantesimo, se la supera, si libera.

\medskip\textbf{Terremoto}\index{Incantesimi - Terremoto}\\
\textbf{Scuola}: Invocazione\\
\textbf{Difficoltà}: 34\\
\textbf{Tempo di Lancio}: 2 Azioni\\
\textbf{Gittata}: 150 metri\\
\textbf{Componenti}: V, S, M (un pizzico di terriccio, un pezzo di pietra e un grumo di argilla)\\
\textbf{Durata}: Concentrazione, massimo 1 minuto\\
Provochi un disturbo sismico in un punto sul terreno a gittata e che puoi vedere. Per la durata, un intenso tremore scuote il terreno in un cerchio di 30 metri di raggio centrato su quel punto e scuote le creature e le strutture in quell'area che sono a contatto del terreno.Il terreno nell'area diventa terreno difficile. Ogni creatura a terra che si sta concentrando deve effettuare un Tiro Salvezza su Tempra. Se lo fallisce, la sua concentrazione è infranta.\\
Quando lanci questo incantesimo e alla fine di ogni round che hai speso a concentrarti su di esso, ogni creatura nell'area che si trovi a terra deve effettuare un Tiro Salvezza su Riflessi. Se lo fallisce, la creatura cade prona.\\
Questo incantesimo ha effetti aggiuntivi a seconda del tipo di terreno nell'area, a discrezione del Narratore. Fenditure. All'inizio del round successivo a quello in cui hai lanciato l'incantesimo si aprono delle fenditure per tutta l'area dell'incantesimo. Un totale di 1d6 fenditure si aprono in punti scelti dal Narratore. Ognuna di esse è profonda 1d10 x 3 metri, larga 3 metri e si estende da un lato dell'area dell'incantesimo all'altro. Una creatura che si trova sul punto in cui si apre una fenditura deve superare un Tiro Salvezza su Riflessi o cadervi dentro. Una creatura che riesca il Tiro Salvezza si sposta sul bordo della fenditura, nel momento in cui questa si apre.\\
Una fenditura che si apre sotto una struttura la fa crollare immediatamente (vedi sotto). Strutture. Il tremore infligge 50 danni da botta a qualsiasi struttura in contatto col terreno nell'area quando lanci l'incantesimo e alla fine di ciascuno dei tuoi turni fino al termine dell'incantesimo. Se una struttura scende a 0 punti ferita, crolla e potrebbe danneggia le creature vicine. Una creatura distante dalla struttura metà della altezza o meno della struttura, deve effettuare un Tiro Salvezza su Riflessi. Se lo fallisce, la creatura subisce 5d6 danni da botta, cade prona ed è sommersa dalle macerie. Dovrà poi impiegare un'azione riuscendo una prova di Destrezza (Atletica) DC 20 per liberarsi. Il Narratore può modificare verso l'alto o il basso la DC, a seconda della natura delle macerie. Se supera il Tiro Salvezza, la creatura subisce solo la metà dei danni e non cade né resta sepolta.

\medskip\textbf{Terreno Illusorio}\index{Incantesimi - Terreno Illusorio}\\
\textbf{Scuola}: Illusione\\
\textbf{Difficoltà}: 23\\
\textbf{Tempo di Lancio}: 10 minuti\\
\textbf{Gittata}: 90 metri\\
\textbf{Componenti}: V, S, M (una pietra, un rametto e un pezzo di pianta verde)\\
\textbf{Durata}: 24 ore \\
Fai sì che un pezzo di terreno naturale a gittata, in un cubo di 45 metri di spigolo, appaia, risuoni e odori come qualche altro tipo di terreno naturale. Di conseguenza, campi aperti o una strada possono essere trasformati in un acquitrino, colline, un crepaccio o qualche altro tipo di terreno difficile o invalicabile. Un laghetto può essere trasformato in una radura erbosa, un precipizio in una lieve pendenza, un burrone cosparso di rocce in una strada ampia e liscia. Le strutture edificate, l'equipaggiamento e le creature all'interno dell'area non mutano d'aspetto.\\
Le peculiarità tattili del terreno sono immutate, così che le creature che entrano nell'area è probabile che svelino l'illusione. Se al contatto la differenza non è ovvia, una creatura che esamina con cautela l'illusione può tentare una prova di Intelligenza (Indagare) contro la DC del Tiro Salvezza dei tuoi incantesimi per dubitare di essa. Una creatura che riconosca l'illusione per quello che è, la percepisce come una vaga immagine sovrapposta al terreno.

\medskip\textbf{Tocco Gelido}\index{Trucchetto - Tocco Gelido}\\
\textbf{Scuola}: Necromanzia\\
\textbf{Difficoltà}: 12\\
\textbf{Tempo di Lancio}: 2 Azioni\\
\textbf{Gittata}: 36 metri\\
\textbf{Componenti}: V, S\\
\textbf{Durata}: 1 round\\
Crei una scheletrica mano spettrale nello spazio di una creatura a gittata. Effettua un attacco a distanza con incantesimo contro la creatura, per aggredirla con il gelo della morte. Se colpisci, il bersaglio subisce 1d8 danni da Vuoto, e non può recuperare punti ferita fino all'inizio del tuo prossimo round. Fino ad allora, la mano resterà serrata sul bersaglio. Se colpisci un bersaglio non morto, esso avrà anche -1d6 ai Tiri per Colpire contro di te fino alla fine del suo prossimo round.\\
Il danno dell'incantesimo aumenta di 1d8 quando raggiungi CM 5, CM 11 e CM 17.

\medskip\textbf{Tocco Vampirico}\index{Incantesimi - Tocco Vampirico}\\
\textbf{Scuola}: Necromanzia\\
\textbf{Difficoltà}: 21\\
\textbf{Tempo di Lancio}: 2 Azioni\\
\textbf{Gittata}: Personale\\
\textbf{Componenti}: V, S\\
\textbf{Durata}: 1 minuto \\
Il contatto con la tua mano avvolta dall'ombra può risucchiare la forza vitale altrui per curare le tue ferite. Effettua un attacco in mischia con incantesimo contro una creatura a portata. Se colpisci, il bersaglio subisce 3d6 danni da Vuoto, e tu recuperi un numero di punti ferita pari alla metà del danno da Vuoto che hai inflitto. Fino al termine dell'incantesimo, puoi effettuare ogni round di nuovo questo attacco come tua azione di attacco.\\
Mentre hai questo incantesimo attivo sei considerato Distratto per il lancio di altri incantesimi.\\
\textbf{Per ogni critico ottenuto} nella prova di magia il danno aumento di 1d6.\\

\medskip\textbf{Trama Ipnotica}\index{Incantesimi - Trama Ipnotica}\\
\textbf{Scuola}: Illusione\\
\textbf{Difficoltà}: 21\\
\textbf{Tempo di Lancio}: 2 Azioni\\
\textbf{Gittata}: 36 metri\\
\textbf{Componenti}: S, M (un bastoncino luminoso di incenso o una fiala di cristallo piena di materiale fosforescente)\\
\textbf{Durata}: 1 minuto\\
Crei a gittata una trama contorta di colori che si muove nell'aria all'interno di un cubo di 9 metri di spigolo. La trama appare per un momento e poi svanisce. Ogni creatura nell'area che veda la trama deve effettuare un Tiro Salvezza su Volontà. Se fallisce il Tiro Salvezza, una creatura rimane affascinata per la durata. Mentre è affascinata da questo incantesimo, la creatura è inabile e ha velocità 0. L'incantesimo termina per la creatura soggetta, qualora questa subisca danni o se qualcuno usa un'azione per scuoterla dal suo stato confusionale.

\medskip\textbf{Trasformazione}\index{Incantesimi - Trasformazione}\\
\textbf{Scuola}: Trasmutazione\\
\textbf{Difficoltà}: 36\\
\textbf{Tempo di Lancio}: 2 Azioni\\
\textbf{Gittata}: Personale\\
\textbf{Componenti}: V, S, M (un cerchietto di giada del valore di almeno 1.500 mo, che devi poggiare sulla tua testa prima di lanciare l'incantesimo)\\
\textbf{Durata}: 1 ora\\
Per la durata assumi la forma di una creatura differente. La nuova forma può essere quella di qualsiasi creatura il cui grado di sfida sia pari o inferiore alla tua CM. La creatura non può essere un costrutto o un non morto, e devi averla vista almeno una volta. Ti trasformi in un esemplare medio di quella creatura, uno senza Abilità specifiche. Puoi restare nella forma assunta fino al termine dell'incantesimo. Ti ritrasformi automaticamente se cadi privo di sensi, scendi a 0 punti ferita o muori. Le tue statistiche di gioco sono rimpiazzate dalle statistiche della creatura scelta, fatta accezione per i tuoi Tratti, e dei tuoi punteggi di Intelligenza, Saggezza e Carisma. Mantieni tutte le tue competenze nelle abilità e i Tiri Salvezza, oltre a ottenere quelle della creatura. Se la creatura possiede le tue stesse competenze e il bonus indicato nelle sue statistiche è più alto del tuo, usa il bonus della creatura al posto del tuo. Non puoi usare nessuna azione aggiuntiva o azione da tana della nuova forma.\\
Quando ti trasformi, assumi i punti ferita e i Dadi Vita della creatura. Quando ritorni alla tua forma normale, ritorni al numero di punti ferita che avevi prima di trasformarti. Tuttavia, se ti ritrasformi perché sei stato ridotto a 0 punti ferita, tutto il danno in eccesso viene riportato alla tua forma originale. A meno che il danno in eccesso non riduca la tua forma normale a 0 punti ferita, non cadrai privo di sensi. \\
Mantieni tutti i benefici di qualsiasi Abilità possedessi, razza, o altra fonte e puoi usarli se la nuova forma è fisicamente capace di farne uso. Tuttavia, non puoi usare nessuno dei tuoi sensi speciali, come la scurovisione, a meno che la nuova forma non possieda anch'essa lo stesso senso. Puoi parlare solo se la creatura è normalmente in grado di parlare.\\
Quando ti trasformi scegli se il tuo equipaggiamento cade a terra nel tuo spazio, si fonde con la nuova forma o sia indossato da essa. L'equipaggiamento indossato funziona come di norma, ma sta al Narratore decidere se sia comodo per la nuova forma indossare un simile pezzo di equipaggiamento, in base alla taglia e le dimensioni della creatura. Il tuo equipaggiamento non cambia dimensioni né si adatta alla nuova forma, e qualsiasi equipaggiamento che la nuova forma non può indossare deve essere fatto cadere a terra o fondersi con la nuova forma. L'equipaggiamento che si fonde è inefficace.\\
Nella durata dell'incantesimo, puoi usare due azioni per assumere una forma diversa seguendo le stesse restrizioni e regole della forma originale, con una eccezione: se la tua nuova forma ha più punti ferita della forma attuale, i tuoi punti ferita restano al livello attuale.

\medskip\textbf{Traslazione Arborea}\index{Incantesimi - Traslazione Arborea}\\
\textbf{Scuola}: Evocazione\\
\textbf{Difficoltà}: 26\\
\textbf{Tempo di Lancio}: 2 Azioni\\
\textbf{Gittata}: Personale\\
\textbf{Componenti}: V, S\\
\textbf{Durata}: massimo 1 minuto\\
Ottieni la capacità di entrare in un albero e muoverti dal suo interno all'interno di un altro albero della stessa specie entro 150 metri. Entrambi gli alberi devono essere vivi e almeno della tua stessa taglia. Devi usare 1 metro di movimento per entrare nell'albero. Apprendi istantaneamente la posizione di tutti gli altri alberi della stessa specie entro 150 metri e, come parte del movimento impiegato per entrare nell'albero, puoi passare in uno degli altri alberi o uscire dall'albero in cui sei entrato. Riappari in un punto a tua scelta entro 1 metri dall'albero di destinazione, utilizzando altri 1 Azione di movimento. Se non ti rimane movimento da usare, riappari entro 1 metro dall'albero in cui sei entrato.\\
Per la durata dell'incantesimo puoi usare questa capacità di trasporto una volta per round. Devi terminare ogni round al di fuori di un albero.

\medskip\textbf{Trasporto Vegetale}\index{Incantesimi - Trasporto Vegetale}\\
\textbf{Scuola}: Evocazione\\
\textbf{Difficoltà}: 29\\
\textbf{Tempo di Lancio}: 2 Azioni\\
\textbf{Gittata}: 3 metri\\
\textbf{Componenti}: V, S\\
\textbf{Durata}: 1 round\\
Questo incantesimo crea un legame magico tra un vegetale inanimato di taglia Grande o maggiore a gittata e un altro vegetale, a qualsiasi distanza, sullo stesso piano di esistenza. Devi aver visto o essere entrato in contatto almeno una volta con il vegetale di destinazione. Per la durata dell'incantesimo, qualsiasi creatura può entrare nel vegetale bersaglio e uscire dal vegetale di destinazione usando 1 azione di movimento.

\medskip\textbf{Trova Cavalcatura}\index{Incantesimi - Trova Cavalcatura}\\
\textbf{Scuola}: Evocazione\\
\textbf{Difficoltà}: 19\\
\textbf{Tempo di Lancio}: 10 minuti\\
\textbf{Gittata}: 9 metri\\
\textbf{Componenti}: V, S\\
\textbf{Durata}: Istantanea\\
Evochi uno spirito che assume la forma di una cavalcatura insolitamente intelligente, forte e leale, stabilendo un legame duraturo con esso. Apparendo in uno spazio a gittata, non occupato, il destriero assume la forma di tua scelta, come quella di un cavallo da guerra, un pony, un cammello, un alce o un mastino (il Narratore potrebbe darti la possibilità di evocare come destrieri anche altri tipi di animali). Il destriero ha le statistiche della forma scelta, sebbene sia di tipo celestiale, fatato o demone (a tua scelta) invece del suo normale tipo. Inoltre, se il tuo destriero ha Intelligenza -3 o meno, la sua Intelligenza diventa -2, e ottiene la capacità di comprendere un linguaggio a tua scelta tra quelli che sei in grado di parlare. Il tuo destriero serve da cavalcatura, sia in combattimento che fuori da esso, e possiedi un legame istintivo con esso, che vi permette di combattere come foste un unico insieme. Mentre sei in groppa alla tua cavalcatura, puoi far sì che qualsiasi incantesimo che lanci e che prenda come bersaglio solo te, prenda come bersaglio anche il tuo destriero.\\
Quando il destriero scende a 0 punti ferita, scompare, non lasciandosi dietro alcuna forma fisica. puoi congedare il destriero in qualsiasi momento con un'azione, facendolo sparire. In entrambi i casi, lanciare di nuovo questo incantesimo evoca lo stesso destriero, ripristinato al massimo dei suoi punti ferita. Mentre il tuo destriero si trova entro 1,5 chilometri da te, puoi comunicare telepaticamente con esso.\\
Non puoi avere più di un destriero legato da questo incantesimo alla volta. Con un'azione, puoi liberare il destriero da questo legame in qualsiasi momento, facendolo sparire.

\medskip\textbf{Trucco della Corda}\index{Incantesimi - Trucco della Corda}\\
\textbf{Scuola}: Trasmutazione\\
\textbf{Difficoltà}: 19\\
\textbf{Tempo di Lancio}: 1 minuto\\
\textbf{Gittata}: Contatto\\
\textbf{Componenti}: V, S, M (estratto di grano in polvere e un laccio di pergamena)\\
\textbf{Durata}: 1 ora\\
Entri a contatto con un pezzo di corda lungo fino a 18 metri. Un'estremità della corda si leva nell'aria finché la corda non pende perpendicolare al terreno. All'estremità opposta della corda, un'entrata invisibile si apre su di uno spazio extradimensionale che resta fino al termine dell'incantesimo \\
Lo spazio extradimensionale può essere raggiunto arrampicandosi fino alla cima della corda. Lo spazio può contenere fino a otto creature di taglia Media o inferiore. La corda può essere trascinata nello spazio, facendola sparire dalla vista di chi è fuori di esso.\\
Attacchi e incantesimi non possono attraversare l'ingresso in entrata o uscita dallo spazio extradimensionale, ma chi si trova al suo interno può vedere fuori come se vedesse attraverso una finestra di 1 x 1 metro centrata sulla corda.\\
Qualsiasi cosa si trovi nello spazio extradimensionale ne cade fuori al termine dell'incantesimo

\medskip\textbf{Unto}\index{Incantesimi - Unto}\\
\textbf{Scuola}: Evocazione\\
\textbf{Difficoltà}: 16\\
\textbf{Tempo di Lancio}: 2 Azioni\\
\textbf{Gittata}: 18 metri\\
\textbf{Componenti}: V, S, M (un pezzo di cotenna di maiale o burro o unto topetto)\\
\textbf{Durata}: 1 minuto\\
Grasso scivoloso ricopre il terreno in un quadrato di 3 metri di lato, centrato su di un punto a gittata, e lo trasforma in terreno difficile per la durata dell'incantesimo\\
Quando compare il grasso, ciascun bersaglio che si trova in piedi nell'area deve superare un Tiro Salvezza su Riflessi o cadere prono. Una creatura che entra nell'area o termina il suo round lì, deve superare un Tiro Salvezza su Riflessi o cadere prona.\\

\medskip\textbf{Vedere Invisibilità}\index{Incantesimi - Vedere Invisibilità}\\
\textbf{Scuola}: Divinazione\\
\textbf{Difficoltà}: 19\\
\textbf{Tempo di Lancio}: 2 Azioni\\
\textbf{Gittata}: Personale\\
\textbf{Componenti}: V, S, M (un pizzico di talco e una manciata di polvere d'argento)\\
\textbf{Durata}: 1 ora\\
Per la durata dell'incantesimo, vedi le creature e gli oggetti invisibili come se fossero visibili, e inoltre puoi vedere nel Piano Etereo. Le creature e gli oggetti eterei ti appaiono spettrali e trasparenti.

\medskip\textbf{Velocità}\index{Incantesimi - Velocità}\\
\textbf{Scuola}: Trasmutazione\\
\textbf{Difficoltà}: 21\\
\textbf{Tempo di Lancio}: 2 Azioni\\
\textbf{Gittata}: 9 metri\\
\textbf{Componenti}: V, S, M (una grattata di radice di liquirizia)\\
\textbf{Durata}: 1 minuto\\
Scegli una creatura consenziente a gittata e che puoi vedere. Fino al termine dell'incantesimo, la velocità del bersaglio è raddoppiata, ottiene un bonus di +2 alla Difesa, ha +1d6 ai Tiri Salvezza su Destrezza, e ottiene un'Azione aggiuntiva durante ciascun suo round.\\
Quest'azione può essere impiegata solo per effettuare un Azione di Attacco, di Movimento oppure o Usare un Oggetto.\\
Questo incantesimo contrasta ed e' contrastato da \hyperlink{lentezza}{Lentezza}.\\
Quando l'incantesimo termina, il bersaglio non può muoversi o effettuare azioni fino al suo prossimo round, mentre è pervaso da un'improvvisa sonnolenza.

\medskip\textbf{Vigilanza e Interdizione}\index{Incantesimi - Vigilanza e Interdizione}\\
\textbf{Scuola}: Abiurazione\\
\textbf{Difficoltà}: 29\\
\textbf{Tempo di Lancio}: 10 minuti\\
\textbf{Gittata}: Contatto\\
\textbf{Componenti}: V, S, M (incenso bruciato, un piccolo misurino di zolfo e olio, un laccio legato, un piccolo ammontare di sangue di colosso di terra, e una piccola verga d'argento del valore di almeno 10 mo)\\
\textbf{Durata}: 24 ore\\
Crei una interdizione che protegge fino a 225 metri quadri di pavimento (un'area quadrata di 15 metri di lato, o cento quadrati di 1 metro di lato o venticinque quadrati di 3 metri di lato). L'area interdetta può essere alta fino a 6 metri, e modellata come preferisci. Puoi interdire diversi piani di una roccaforte dividendo l'area tra di essi, purché tu possa camminare ininterrottamente in ogni area adiacente, mentre lanci l'incantesimo\\
Quando lanci questo incantesimo, puoi specificare gli individui che ignorano qualcuno o tutti gli effetti di questo incantesimo. Puoi anche specificare una parola d'ordine che, pronunciata ad alta voce, rende chi la proferisce immune a questi effetti.\\
Vigilanza e interdizione crea i seguenti effetti all'interno dell'area interdetta.\\
\textit{Corridoi}. La nebbia riempie tutti i corridoi interdetti, rendendoli oscurati pesantemente. Inoltre, a ogni intersezione o biforcazione del passaggio che offre una scelta di direzione, c'è una probabilità del 50\% che una creatura, escluso te, creda di stare andando nella direzione opposta a quella che ha scelto.\\
\textit{Porte}. Tutte le porte nell'area interdetta sono chiuse magicamente, come se fossero sigillate dall'incantesimo serratura arcana. Inoltre, puoi coprire fino a dieci porte con un'illusione (equivalente della funzione oggetto illusorio dell'incantesimo illusione minore) per farle sembrare delle semplici sezioni di muro.\\
\textit{Scale}. Ragnatele ricoprono da cima a fondo tutte le scale nell'area interdetta, come per l'incantesimo ragnatela. Questi fili ricrescono in 10 minuti se vengono bruciati o strappati mentre vigilanza e interdizione resta attivo.\\
Altri Incantesimi in Effetto. Puoi piazzare uno dei seguenti effetti magici di tua scelta all'interno dell'area interdetta dell'edificio
\medskip
\begin{itemize}
\item
Piazza luci danzanti in quattro corridoi. Puoi indicare un semplice programma che le luci ripeteranno per la durata di vigilanza e interdizione.
\item
Piazza bocca magica in due posti.
\item
Piazza nube maleodorante in due posti. I vapori appaiono nel posto da te indicato; ritornano entro 10 minuti se dispersi dal vento mentre vigilanza e interdizione è ancora attivo.
\item
Piazza una folata di vento costante in un corridoio o stanza.
\item
Piazza una suggestione in un luogo. Seleziona un'area quadrata di 1 metro di lato, e qualsiasi creatura che entra o passa attraverso quell'area riceve mentalmente la suggestione.
\end{itemize}
\medskip
L'intera area interdetta irradia magia. Un incantesimo dissolvi magie lanciato contro uno specifico effetto, se riesce, rimuove solo quell'effetto Puoi creare una struttura perennemente vigilata e interdetta lanciandovi questo incantesimo ogni giorno per un anno.\\
\textbf{Se effettui tre critici} la durata e' permanente.

\medskip\textbf{Vincolo di Interdizione}\index{Incantesimi - Vincolo di Interdizione}\\
\textbf{Scuola}: Abiurazione\\
\textbf{Difficoltà}: 19\\
\textbf{Tempo di Lancio}: 2 Azioni\\
\textbf{Gittata}: Contatto\\
\textbf{Componenti}: V, S, M (una coppia di anelli di platino del valore di 50 mo l'uno, che tu e il bersaglio dovete indossare per la durata)\\
\textbf{Durata}: 1 ora\\
Lanci l'incantesimo a contatto di una creatura che vuoi proteggere. Crei una connessione mistica tra di te e il bersaglio fino al termine dell'incantesimo. Finché il bersaglio è entro 18 metri da te, ottiene un bonus di +1 alla Difesa e ai Tiri Salvezza e ha resistenza a tutti i danni. Inoltre, ogni volta che il bersaglio subisce danni, tu ne subisci la stessa quantità. L'incantesimo ha fine se scendi a 0 punti ferita o tu e il bersaglio vi allontanate più di 18 metri. Ha fine anche se lo lanci di nuovo sulla stessa creatura su cui è già in atto. Puoi interrompere l'incantesimo con un'azione.

\medskip\textbf{Visione del Vero}\index{Incantesimi - Visione del Vero}\\
\textbf{Scuola}: Divinazione\\
\textbf{Difficoltà}: 29\\
\textbf{Tempo di Lancio}: 2 Azioni\\
\textbf{Gittata}: Contatto\\
\textbf{Componenti}: V, S, M (un unguento per gli occhi che costa 25 mo; fatto di funghi in polvere, zafferano e grasso; viene consumato dall'incantesimo)\\
\textbf{Durata}: 1 ora\\
Lanci l'incantesimo a contatto di una creatura consenziente. Il bersaglio riceve la capacità di vedere le cose come sono realmente. Per la durata dell'incantesimo, la creatura ha visione del vero, nota porte segrete nascoste dalla magia, e può vedere nel Piano Etereo, fino a una gittata di 36 metri.

\medskip\textbf{Vita Falsata}\index{Incantesimi - Vita Falsata}\\
\textbf{Scuola}: Necromanzia\\
\textbf{Difficoltà}: 16\\
\textbf{Tempo di Lancio}: 2 Azioni\\
\textbf{Gittata}: Personale\\
\textbf{Componenti}: V, S, M (un piccolo ammontare di alcool o spirito distillato)\\
\textbf{Durata}: 1 ora\\
Potenziandoti con una parvenza necromantica di vitalità, ottieni 1d4 + 4 punti ferita temporanei per la durata.\\
\textbf{Per ogni critico ottenuto} nella prova di magia ottieni 5 punti ferita temporanei.

\medskip\textbf{Volare}\index{Incantesimi - Volare}\\
\textbf{Scuola}: Trasmutazione\\
\textbf{Difficoltà}: 21\\
\textbf{Tempo di Lancio}: 2 Azioni\\
\textbf{Gittata}: Contatto\\
\textbf{Componenti}: V, S, M (una piuma dell'ala di qualsiasi volatile)\\
\textbf{Durata}: 10 minuti \\
Lanci l'incantesimo a contatto di una creatura consenziente. Per la durata dell'incantesimo, il bersaglio ottiene velocità di volo 18 metri. Quando l'incantesimo ha fine, qualora sia ancora in aria, il bersaglio cade, a meno che non riesca a frenare la discesa.\\
Lanciare un incantesimo mentre si vola e' piu' complesso, si e' Distratti. La Difficoltà aumenta di 5.\\
\textbf{Per ogni critico ottenuto} nella prova di magia puoi prendere come bersaglio un'ulteriore creatura oppure raddoppiare la durata.

\medskip\textbf{Vuoto Mentale}\index{Incantesimi - Vuoto Mentale}\\
\textbf{Scuola}: Abiurazione\\
\textbf{Difficoltà}: 34\\
\textbf{Tempo di Lancio}: 2 Azioni\\
\textbf{Gittata}: Contatto\\
\textbf{Componenti}: V, S\\
\textbf{Durata}: 24 ore\\
Fino al termine dell'incantesimo, una creatura consenziente con cui sei in contatto durante il lancio è immune al danno psichico, qualsiasi effetto che ne percepirebbe le emozioni o leggerebbe i pensieri, incantesimi di divinazione e la condizione affascinato. l'incantesimo nega anche gli incantesimi desiderio e altri incantesimi o effetti di simili potenza impiegati per
influenzare la mente del bersaglio o per ottenere informazioni su di esso.\\
\textbf{Per ogni critico ottenuto} nella prova di magia la durata raddoppia. Se ottieni tre critici la durata e' permanente.

\medskip\textbf{Zona di Verità}\index{Incantesimi - Zona di Verità}\\
\textbf{Scuola}: Ammaliamento\\
\textbf{Difficoltà}: 19\\
\textbf{Tempo di Lancio}: 2 Azioni\\
\textbf{Gittata}: 18 metri\\
\textbf{Componenti}: V, S\\
\textbf{Durata}: 10 minuti\\
Crei una zona magica che protegge contro i raggiri in una sfera di 4 metri di raggio centrata su di un punto a gittata di tua scelta. Fino al termine dell'incantesimo, una creatura che entra nell'area dell'incantesimo per la prima volta durante un round, o inizia il suo round al suo interno, deve effettuare un Tiro Salvezza su Volontà. Se fallisce il Tiro Salvezza, la creatura non può pronunciare bugie deliberatamente mentre è nel raggio dell'incantesimo. Sei a conoscenza se una creatura ha superato o fallito il Tiro Salvezza. Una creatura soggetta all'incantesimo ne è consapevole e può quindi evitare di rispondere a domande a cui risponderebbe normalmente con una bugia. Questa creatura può dare risposte elusive purché rimanga entro i confini della verità.

\end{multicols}

\pagebreak

\begin{multicols}{3}
	
	
	\subsection{Lista degli Incantesimi divisi per Difficoltà}
	
	\textbf{Legenda}: Nome Incantesimo (Scuola di Magia di appartenenza)\\
	
	\subsubsection{Trucchetti - Difficoltà 12}
	Artificio Druidico (Universale)\\
	Beffa Crudele (Ammaliamento)\\
	Colpo Accurato (Divinazione)\\
	Creare Birra (Evocazione)\\
	Deflagrazione Occulta (Invocazione)\\
	Dito (Ammaliamento)\\
	Fiamma Sacra (Invocazione)\\
	Fiotto Acido (Evocazione)\\
	Luci Danzanti (Invocazione)\\
	Mano Magica (Evocazione)\\
	Messaggio (Trasmutazione)\\
	Prestidigitazione (Universale)\\
	Produrre Fiamma (Evocazione)\\
	Raggio di Gelo (Invocazione)\\
	Randello Incantato (Trasmutazione)\\
	Resistenza (Abiurazione)\\
	Riparare (Trasmutazione)\\
	Spruzzo Velenoso (Evocazione)\\
	Stretta Folgorante (Invocazione)\\
	Taumaturgia (Universale)\\
	Tocco Gelido (Necromanzia)\\
	
	\subsubsection{Difficoltà 16}
	Allarme (Abiurazione)\\
	Amicizia con gli Animali (Ammaliamento)\\
	Anatema (Ammaliamento)\\
	Armatura Magica (Abiurazione)\\
	Benedizione (Invocazione)\\
	Caduta Morbida (Trasmutazione)\\
	Charme su Persone (Ammaliamento)\\
	Camuffare Sé Stesso (Illusione)\\
	Comando (Ammaliamento)\\
	Comprensione dei Linguaggi (Divinazione)\\
	Creare o Distruggere Acqua (Trasmutazione)\\
	Cuoco Invisibile (Evocazione)\\
	Cura Ferite Leggere (Cura)\\
	Dardo di Fuoco (Invocazione)\\
	Dardo Tracciante (Invocazione)\\
	Dardo Incantato (Invocazione)\\
	Disco Fluttuante (Evocazione)\\
	Eroismo (Ammaliamento)\\
	Favore Divino (Invocazione)\\
	Guida (Divinazione)\\
	Identificare (Divinazione)\\
	Illusione Minore (Illusione)\\
	Immagine Silenziosa (Illusione)\\
	Individuazione del Bene e del Male (Divinazione)\\
	Individuazione del Magico (Divinazione)\\
	Intimorire Infernale (Evocazione)\\
	Intralciare (Invocazione)\\
	Luce (Invocazione)\\
	Luminescenza (Invocazione)\\
	Mani Brucianti (Invocazione)\\
	Nube di Nebbia (Evocazione)\\
	Onda Tonante (Invocazione)\\
	Oscurità (Invocazione)\\
	Parlare con gli Animali (Divinazione)\\
	Parola Guaritrice (Cura)\\
	Passo Veloce (Trasmutazione)\\
	Protezione dal Bene e dal Male (Abiurazione)\\
	Purificare Cibo e Bevande (Trasmutazione)\\
	Risata Incontenibile (Ammaliamento)\\
	Ritirata Rapida (Trasmutazione)\\
	Saltare (Trasmutazione)\\
	Santuario (Abiurazione)\\
	Scritto Illusorio (Illusione)\\
	Scudo (Abiurazione)\\
	Scudo della Fede (Abiurazione)\\
	Servitore Invisibile (Evocazione)\\
	Sonno (Ammaliamento)\\
	Spruzzo Colorato (Illusione)\\
	Trova Famiglio (Evocazione)\\
	Unto (Evocazione)\\
	Vita Falsata (Necromanzia)\\
	
	\subsubsection{Difficoltà 19}
	Aiuto (Abiurazione)\\
	Alterare Sé Stesso (Trasmutazione)\\
	Animale Messaggero (Ammaliamento)\\
	Arma Magica (Trasmutazione)\\
	Arma Spirituale (Invocazione)\\
	Aura Magica dell'Arcanista (Illusione)\\
	Bacche Benefiche (Trasmutazione)\\
	Benedici Acqua (Invocazione)\\
	Benedizione Superiore (Invocazione)\\
	Blocca Persona (Ammaliamento)\\
	Bocca Magica (Illusione)\\
	Calmare Emozioni (Ammaliamento)\\
	Caratteristica Potenziata (Trasmutazione)\\
	Cecità/Sordità (Necromanzia)\\
	Comprensione degli Scritti (Divinazione)\\
	Crescita di Spuntoni (Trasmutazione)\\
	Estasiare (Ammaliamento)\\
	Fiamma Perenne (Invocazione)\\
	Folata di Vento (Invocazione)\\
	Frantumare (Invocazione)\\
	Freccia Acida (Invocazione)\\
	Immagine Speculare (Illusione)\\
	Individuazione delle Malattie e dei Veleni (Divinazione)\\
	Individuazione dei Pensieri (Divinazione)\\
	Infliggi Ferite (Necromanzia)\\
	Ingrandire/Ridurre (Trasmutazione)\\
	Invisibilità (Illusione)\\
	Lama Infuocata (Invocazione)\\
	Levitazione (Trasmutazione)\\
	Localizza Animali e Piante (Divinazione)\\
	Localizza Oggetto (Divinazione)\\
	Movimenti del Ragno (Trasmutazione)\\
	Passare Senza Tracce (Abiurazione)\\
	Passo Velato (Evocazione)\\
	Pelle di Corteccia (Trasmutazione)\\
	Preghiera di Guarigione (Cura)\\
	Presagio (Divinazione)\\
	Protezione dai Veleni (Abiurazione)\\
	Punizione Marchiante (Invocazione)\\
	Raggio di Affaticamento (Necromanzia)\\
	Raggio Rovente (Invocazione)\\
	Ragnatela (Evocazione)\\
	Riposo Inviolato (Necromanzia)\\
	Riscaldare il Metallo (Trasmutazione)\\
	Ristorare Inferiore (Cura)\\
	Scassinare (Trasmutazione)\\
	Scopri Trappole (Divinazione)\\
	Scurovisione (Trasmutazione)\\
	Serratura Arcana (Abiurazione)\\
	Sfera Infuocata (Evocazione)\\
	Sfocatura (Illusione)\\
	Silenzio (Illusione)\\
	Suggestione (Ammaliamento)\\
	Trova Cavalcatura (Evocazione)\\
	Trucco della Corda (Trasmutazione)\\
	Vedere Invisibilità (Divinazione)\\
	Vincolo di Interdizione (Abiurazione)\\
	Zona di Verità (Ammaliamento)\\
	
	\subsubsection{Difficoltà 21}
	Animare Morti (Necromanzia)\\
	Anti-Individuazione (Abiurazione)\\
	Benedizione Suprema (Invocazione)\\
	Camminare sull'Acqua (Trasmutazione)\\
	Capanna (Invocazione)\\
	Cecità/Sordità Avanzata (Necromanzia)\\
	Cerchio Magico (Abiurazione)\\
	Chiaroveggenza (Divinazione)\\
	Controincantesimo (Abiurazione)\\
	Creare Cibo e Acqua (Evocazione)\\
	Crescita Vegetale (Trasmutazione)\\
	Cura Ferite Serie (Cura)\\
	Cura Ferite di Massa (Cura)\\
	Destriero Fantasma (Illusione)\\
	Dissolvi Magie (Abiurazione)\\
	Evoca Animali (Evocazione)\\
	Faro di Speranza (Abiurazione)\\
	Fondersi nella Pietra (Trasmutazione)\\
	Forma Gassosa (Trasmutazione)\\
	Fulmine (Invocazione)\\
	Glifo di Interdizione (Abiurazione)\\
	Immagine Maggiore (Illusione)\\
	Intermittenza (Trasmutazione)\\
	Inviare (Invocazione)\\
	Invocare il Fulmine (Evocazione)\\
	Lentezza (Trasmutazione)\\
	Lingue (Divinazione)\\
	Luce Diurna (Invocazione)\\
	Muro di Vento (Invocazione)\\
	Nube Maleodorante (Evocazione)\\
	Palla di Fuoco (Invocazione)\\
	Parlare con i Morti (Necromanzia)\\
	Parlare con le Piante (Trasmutazione)\\
	Parola Guaritrice di Massa (Cura)\\
	Paura (Illusione)\\
	Protezione dall'Energia (Abiurazione)\\
	Respirare Sott'Acqua (Trasmutazione)\\
	Rimuovi Maledizione (Abiurazione)\\
	Rimuovi Veleno (Cura)\\
	Rinascita (Cura)\\
	Scagliare Maledizione (Necromanzia)\\
	Tempesta di Nevischio (Evocazione)\\
	Tocco Vampirico (Necromanzia)\\
	Trama Ipnotica (Illusione)\\
	Velocità (Trasmutazione)\\
	Volare (Trasmutazione)\\
	
	\subsubsection{Difficoltà 23}
	Allucinazione Mortale (Illusione)\\
	Blocca Persona Avanzato (Ammaliamento)\\
	Compulsione (Ammaliamento)\\
	Confusione (Ammaliamento)\\
	Controllare Acqua (Trasmutazione)\\
	Dominare Bestie (Ammaliamento)\\
	Esilio (Abiurazione)\\
	Evoca Creature Boschive (Evocazione)\\
	Evoca Elementali Minori (Evocazione)\\
	Fabbricare (Trasmutazione)\\
	Inaridire (Necromanzia)\\
	Insetto Gigante (Trasmutazione)\\
	Interdizione alla Morte (Abiurazione)\\
	Invisibilità Superiore (Illusione)\\
	Libertà di Movimento (Abiurazione)\\
	Localizza Creatura (Divinazione)\\
	Metamorfosi (Trasmutazione)\\
	Muro di Fuoco (Invocazione)\\
	Occhio Arcano (Divinazione)\\
	Pelle di Pietra (Abiurazione)\\
	Porta Dimensionale (Evocazione)\\
	Rimuovi Malattia (Cura)\\
	Santuario Privato (Abiurazione)\\
	Scolpire Pietra (Trasmutazione)\\
	Scrigno Segreto (Evocazione)\\
	Scudo di Fuoco (Invocazione)\\
	Segugio Fedele (Evocazione)\\
	Sfera Elastica (Invocazione)\\
	Tempesta di Ghiaccio (Invocazione)\\
	Tentacoli Neri (Evocazione)\\
	Terreno Illusorio (Illusione)\\
	
	\subsubsection{Difficoltà 26}
	Animare Oggetti (Trasmutazione)\\
	Blocca Mostri (Ammaliamento)\\
	Cerchio di Teletrasporto (Evocazione)\\
	Colpo Infuocato (Invocazione)\\
	Comunione (Divinazione)\\
	Comunione con la Natura (Divinazione)\\
	Cono di Freddo (Invocazione)\\
	Conoscenza delle Leggende (Divinazione)\\
	Contagio (Necromanzia)\\
	Costrizione (Ammaliamento)\\
	Creazione (Illusione)\\
	Cura Ferite Critiche (Cura)\\
	Dissolvi il Bene e il Male (Abiurazione)\\
	Dominare Persone (Ammaliamento)\\
	Evoca Elementale (Evocazione)\\
	Fuorviare (Illusione)\\
	Guscio Anti-Vita (Abiurazione)\\
	Legame Telepatico (Divinazione)\\
	Mano Arcana (Invocazione)\\
	Modificare Memoria (Ammaliamento)\\
	Muro di Forza (Invocazione)\\
	Muro di Pietra (Invocazione)\\
	Nube Mortale (Evocazione)\\
	Passapareti (Trasmutazione)\\
	Piaga degli Insetti (Evocazione)\\
	Reincarnazione (Trasmutazione)\\
	Rianimare Morti (Necromanzia)\\
	Ristorare Superiore (Cura)\\
	Risveglio (Trasmutazione)\\
	Santificare (Invocazione)\\
	Scrutare (Divinazione)\\
	Sembrare (Illusione)\\
	Sogno (Illusione)\\
	Telecinesi (Trasmutazione)\\
	Traslazione Arborea (Evocazione)\\
	
	\subsubsection{Difficoltà 29}
	Alleato Planare (Evocazione)\\
	Bagliore Solare (Invocazione)\\
	Banchetto degli Eroi (Evocazione)\\
	Barriera di Lame (Invocazione)\\
	Camminare nel Vento (Trasmutazione)\\
	Carne in Pietra - Pietra in Carne (Trasmutazione)\\
	Catena di Fulmini (Invocazione)\\
	Cerchio di Morte (Invocazione)\\
	Contingenza (Invocazione)\\
	Creare Non Morti (Necromanzia)\\
	Disintegrazione (Trasmutazione)\\
	Dito della Morte (Necromanzia)\\
	Divinazione (Divinazione)\\
	Evocazioni Istantanee (Evocazione)\\
	Ferire (Necromanzia)\\
	Giara Magica (Necromanzia)\\
	Globo di Invulnerabilità (Abiurazione)\\
	Guarigione (Cura)\\
	Illusione Programmata (Illusione)\\
	Muovere il Terreno (Trasmutazione)\\
	Muro di Ghiaccio (Invocazione)\\
	Muro di Spine (Evocazione)\\
	Parola del Ritiro (Evocazione)\\
	Proibizione (Abiurazione)\\
	Scopri il Percorso (Divinazione)\\
	Sfera Congelante (Invocazione)\\
	Sguardo Penetrante (Necromanzia)\\
	Suggestione di Massa (Ammaliamento)\\
	Trasporto Vegetale (Evocazione)\\
	Vigilanza e Interdizione (Abiurazione)\\
	Visione del Vero (Divinazione)\\
	
	\subsubsection{Difficoltà 31}
	Celare (Trasmutazione)\\
	Evoca Celestiali (Evocazione)\\
	Forma Eterea (Trasmutazione)\\
	Immagine Proiettata (Illusione)\\
	Inversione della Gravità (Trasmutazione)\\
	Miraggio Arcano (Illusione)\\
	Palla di Fuoco Ritardata (Invocazione)\\
	Parola Divina (Invocazione)\\
	Reggia Meravigliosa (Evocazione)\\
	Resurrezione (Necromanzia)\\
	Rigenerazione (Trasmutazione)\\
	Simbolo (Abiurazione)\\
	Simulacro (Illusione)\\
	Spada Arcana (Invocazione)\\
	Spostamento Planare (Evocazione)\\
	Spruzzo Prismatico (Invocazione)\\
	Teletrasporto (Evocazione)\\
	Tempesta di Fuoco (Invocazione)\\
	
	\subsubsection{Difficoltà 34}
	Antipatia/Simpatia (Ammaliamento)\\
	Aura Sacra (Abiurazione)\\
	Campo Anti-Magia (Abiurazione)\\
	Clone (Necromanzia)\\
	Controllare Tempo Atmosferico (Trasmutazione)\\
	Danza Irresistibile (Ammaliamento)\\
	Dominare Mostri (Ammaliamento)\\
	Esplosione Solare (Invocazione)\\
	Forme Animali (Trasmutazione)\\
	Gabbia di Forza (Invocazione)\\
	Labirinto (Evocazione)\\
	Loquacità (Trasmutazione)\\
	Nube Incendiaria (Evocazione)\\
	Parola del Potere Stordire (Ammaliamento)\\
	Regressione Mentale (Ammaliamento)\\
	Semipiano (Evocazione)\\
	Terremoto (Invocazione)\\
	Vuoto Mentale (Abiurazione)\\
	
	\subsubsection{Difficoltà 36}
	Desiderio (Evocazione)\\
	Fatale (Illusione)\\
	Fermare il Tempo (Trasmutazione)\\
	Guarigione di Massa (Cura)\\
	Imprigionare (Abiurazione)\\
	Metamorfosi Pura (Trasmutazione)\\
	Muro Prismatico (Abiurazione)\\
	Parola del Potere Uccidere (Ammaliamento)\\
	Portale (Evocazione)\\
	Previsione (Divinazione)\\
	Proiezione Astrale (Necromanzia)\\
	Resurrezione Pura (Trasmutazione)\\
	Sciame di Meteore (Invocazione)\\
	Trasformazione (Trasmutazione)\\
	
	\subsection{Lista degli Incantesimi divisi per Scuola}
	
	\textbf{Legenda}: Nome Incantesimo (Difficoltà di lancio)\\
	
	\subsubsection{Abiurazione}
	Aiuto (19)\\
	Allarme	(10)\\
	Anti-Individuazione (21)\\
	Armatura Magica (16)\\
	Aura Sacra (34)\\
	Campo Anti-Magia (34)\\
	Cerchio Magico (21)\\
	Controincantesimo (21)\\
	Dissolvi il Bene e il Male (26)\\
	Dissolvi Magie (21)\\
	Esilio (23)\\
	Faro di Speranza (21)\\
	Glifo di Interdizione (21)\\
	Globo di Invulnerabilità (29)\\
	Guscio Anti-Vita (26)\\
	Imprigionare (36)\\
	Interdizione alla Morte (23)\\
	Libertà di Movimento (23)\\
	Muro Prismatico (36)\\
	Passare Senza Tracce (19)\\
	Pelle di Pietra (23)\\
	Proibizione (29)\\
	Protezione dal Bene e dal Male (16)\\
	Protezione dall'Energia (21)\\
	Protezione dai Veleni (19)\\
	Resistenza (12)\\
	Rimuovi Maledizione (21)\\
	Santuario (16)\\
	Santuario Privato (23)\\
	Scudo (16)\\
	Scudo della Fede (16)\\
	Serratura Arcana (19)\\
	Simbolo (31)\\
	Vigilanza e Interdizione (29)\\
	Vincolo di Interdizione (19)\\
	Vuoto Mentale (34)\\
	
	\subsubsection{Ammaliamento}
	Amicizia con gli Animali (10)	\\
	Anatema (16)\\
	Animale Messaggero (19)\\
	Antipatia/Simpatia (34)\\
	Beffa Crudele (12)\\
	Blocca Mostri (26)\\
	Blocca Persona (19)\\
	Blocca Persona Avanzato (23)\\
	Calmare Emozioni (19)\\
	Charme su Persone (16)\\
	Comando (16)\\
	Compulsione (23)\\
	Confusione (23)\\
	Costrizione (26)\\
	Danza Irresistibile (34)\\
	Dito (12)\\
	Dominare Bestie (23)\\
	Dominare Mostri (34)\\
	Dominare Persone (26)\\
	Eroismo (16)\\
	Estasiare (19)\\
	Modificare Memoria (26)\\
	Parola del Potere Stordire (34)\\
	Parola del Potere Uccidere (36)\\
	Regressione Mentale (34)\\
	Risata Incontenibile (16)\\
	Sonno (16)\\
	Suggestione (19)\\
	Suggestione di Massa (29)\\
	Zona di Verità (19)\\
	
	\subsubsection{Cura}
Cura Ferite Critiche (26)\\
Cura Ferite di Massa (variabile)\\
Cura Ferite Leggere (16)\\
Cura Ferite Serie (21)\\
Guarigione (29)\\
Guarigione di Massa (36)\\
Parola Guaritrice (16)\\
Parola Guaritrice di Massa (21)\\
Preghiera di Guarigione (19)\\
Rimuovi Malattia (23)\\
Rimuovi Veleno (21)\\
Rinascita (21)\\
Ristorare Inferiore (19)\\
Ristorare Superiore (26)\\	
	
	\subsubsection{Divinazione}
	Chiaroveggenza (21)\\
	Colpo Accurato (12)\\
	Comprensione dei Linguaggi (16)\\
	Comprensione degli Scritti (19)\\
	Comunione (26)\\
	Comunione con la Natura (26)\\
	Conoscenza delle Leggende (26)\\
	Divinazione (29)\\
	Guida (16)\\
	Identificare (16)\\
	Individuazione del Bene e del Male (16)\\
	Individuazione del Magico (16)\\
	Individuazione delle Malattie e dei Veleni (19)\\
	Individuazione dei Pensieri (19)\\
	Legame Telepatico (26)\\
	Lingue (21)\\
	Localizza Animali e Piante (19)\\
	Localizza Creatura (23)\\
	Localizza Oggetto (19)\\
	Occhio Arcano (23)\\
	Parlare con gli Animali (16)\\
	Presagio (19)\\
	Previsione (36)\\
	Scopri il Percorso (29)\\
	Scopri Trappole (19)\\
	Scrutare (26)\\
	Vedere Invisibilità (19)\\
	Visione del Vero (29)\\
	
	\subsubsection{Evocazione}
	Alleato Planare (29)\\
	Cerchio di Teletrasporto (26)\\
	Creare Cibo e Acqua (21)\\
	Creare Birra (12)\\
	Cuoco Invisibile (16)\\
	Desiderio (36)\\
	Disco Fluttuante (16)\\
	Evoca Animali (21)\\
	Evoca Celestiali (31)\\
	Evoca Creature Boschive (23)\\
	Evoca Elementale (26)\\
	Evoca Elementali Minori (23)\\
	Evocazioni Istantanee (29)\\
	Fiotto Acido (12)\\
	Intimorire Infernale (16)\\
	Invocare il Fulmine (21)\\
	Labirinto (34)\\
	Mano Magica (12)\\
	Muro di Spine (29)\\
	Nube Incendiaria (34)\\
	Nube Maleodorante (21)\\
	Nube Mortale (26)\\
	Nube di Nebbia (16)\\
	Parola del Ritiro (29)\\
	Passo Velato (19)\\
	Piaga degli Insetti (26)\\
	Porta Dimensionale (23)\\
	Portale (36)\\
	Produrre Fiamma (12)\\
	Ragnatela (19)\\
	Reggia Meravigliosa (31)\\
	Scrigno Segreto (23)\\
	Segugio Fedele (23)\\
	Semipiano (34)\\
	Servitore Invisibile (16)\\
	Sfera Infuocata (19)\\
	Spostamento Planare (31)\\
	Spruzzo Velenoso (12)\\
	Teletrasporto (31)\\
	Tempesta di Nevischio (21)\\
	Tentacoli Neri (23)\\
	Traslazione Arborea (26)\\
	Trasporto Vegetale (29)\\
	Trova Cavalcatura (19)\\
	Trova Famiglio (16)\\
	Unto (16)\\
	
	\subsubsection{Illusione}	
	Allucinazione Mortale (23)\\
	Aura Magica dell'Arcanista (19)\\
	Bocca Magica (19)\\
	Camuffare Sé Stesso (16)\\
	Creazione (26)\\
	Destriero Fantasma (21)\\
	Fatale (36)\\
	Fuorviare (26)\\
	Illusione Minore (16)\\
	Illusione Programmata (29)\\
	Immagine Maggiore (21)\\
	Immagine Proiettata (31)\\
	Immagine Silenziosa (16)\\
	Immagine Speculare (19)\\
	Invisibilità (19)\\
	Invisibilità Superiore (23)\\
	Miraggio Arcano (31)\\
	Paura (21)\\
	Scritto Illusorio (16)\\
	Sembrare (26)\\
	Sfocatura (19)\\
	Silenzio (19)\\
	Simulacro (31)\\
	Sogno (26)\\
	Spruzzo Colorato (16)\\
	Terreno Illusorio (23)\\
	Trama Ipnotica (21)\\
	
	\subsubsection{Invocazione}	
	Arma Spirituale (19)\\
	Bagliore Solare (29)\\
	Banchetto degli Eroi (29)\\
	Barriera di Lame (29)\\
	Benedici Acqua (19)\\
	Benedizione (16)\\
	Benedizione Superiore (19)\\
	Benedizione Suprema (21)\\
	Capanna (21)\\
	Catena di Fulmini (29)\\
	Cerchio di Morte (29)\\
	Colpo Infuocato (26)\\
	Cono di Freddo (26)\\
	Contingenza (29)\\
	Dardo di Fuoco (16)\\
	Dardo Tracciante (16)\\
	Dardo Incantato (16)\\
	Deflagrazione Occulta (12)\\
	Esplosione Solare (34)\\
	Favore Divino (16)\\
	Fiamma Perenne (19)\\
	Fiamma Sacra (12)\\
	Folata di Vento (19)\\
	Frantumare (19)\\
	Freccia Acida (19)\\
	Fulmine (21)\\
	Gabbia di Forza (34)\\
	Intralciare (16)\\
	Inviare (21)\\
	Lama Infuocata (19)\\
	Luce (16)\\
	Luce Diurna (21)\\
	Luci Danzanti (12)\\
	Luminescenza (16)\\
	Mani Brucianti (16)\\
	Mano Arcana (26)\\
	Muro di Forza (26)\\
	Muro di Fuoco (23)\\
	Muro di Ghiaccio (29)\\
	Muro di Pietra (26)\\
	Muro di Vento (21)\\
	Onda Tonante (16)\\
	Oscurità (16)\\
	Palla di Fuoco (21)\\
	Palla di Fuoco Ritardata (31)\\
	Parola Divina (31)\\
	Punizione Marchiante (19)\\
	Raggio di Gelo (12)\\
	Raggio Rovente (19)\\
	Santificare (26)\\
	Sciame di Meteore (36)\\
	Scudo di Fuoco (23)\\
	Sfera Congelante (29)\\
	Sfera Elastica (23)\\
	Spada Arcana (31)\\
	Spruzzo Prismatico (31)\\
	Stretta Folgorante (12)\\
	Tempesta di Fuoco (31)\\
	Tempesta di Ghiaccio (23)\\
	Terremoto (34)\\
	
	\subsubsection{Necromanzia}
	Animare Morti (21)\\
	Cecità/Sordità (19)\\
	Cecità/Sordità Avanzata (21)\\
	Clone (34)\\
	Contagio (26)\\
	Creare Non Morti (29)\\
	Dito della Morte (29)\\
	Ferire (29)\\
	Giara Magica (29)\\
	Inaridire (23)\\
	Infliggi Ferite (19)\\
	Parlare con i Morti (21)\\
	Proiezione Astrale (36)\\
	Raggio di Affaticamento (19)\\
	Resurrezione (31)\\
	Rianimare Morti (26)\\
	Rinascita (21)\\
	Riposo Inviolato (19)\\
	Scagliare Maledizione (21)\\
	Sguardo Penetrante (29)\\
	Tocco Gelido (12)\\
	Tocco Vampirico (21)\\
	Vita Falsata (16)\\
	
	\subsubsection{Trasmutazione}
	Alterare Sé Stesso (13)		\\
	Animare Oggetti (26)\\
	Arma Magica (19)\\
	Bacche Benefiche (19)\\
	Caduta Morbida (16)\\
	Camminare sull'Acqua (21)\\
	Camminare nel Vento (29)\\
	Caratteristica Potenziata (19)\\
	Carne in Pietra - Pietra in Carne (29)\\
	Celare (31)\\
	Controllare Acqua (23)\\
	Controllare Tempo Atmosferico (34)\\
	Creare o Distruggere Acqua (16)\\
	Crescita di Spuntoni (19)\\
	Crescita Vegetale (21)\\
	Disintegrazione (29)\\
	Fabbricare (23)\\
	Fermare il Tempo (36)\\
	Fondersi nella Pietra (21)\\
	Forma Eterea (31)\\
	Forma Gassosa (21)\\
	Forme Animali (34)\\
	Ingrandire/Ridurre (19)\\
	Insetto Gigante (23)\\
	Intermittenza (21)\\
	Inversione della Gravità (31)\\
	Lentezza (21)\\
	Levitazione (19)\\
	Loquacità (34)\\
	Messaggio (12)\\
	Metamorfosi (23)\\
	Metamorfosi Pura (36)\\
	Movimenti del Ragno (19)\\
	Muovere il Terreno (29)\\
	Parlare con le Piante (21)\\
	Passapareti (26)\\
	Passo Veloce (16)\\
	Pelle di Corteccia (19)\\
	Purificare Cibo e Bevande (16)\\
	Randello Incantato (12)\\
	Reincarnazione (26)\\
	Respirare Sott'Acqua (21)\\
	Resurrezione Pura (36)\\
	Rigenerazione (31)\\
	Riparare (12)\\
	Riscaldare il Metallo (19)\\
	Risveglio (26)\\
	Ritirata Rapida (16)\\
	Saltare (16)\\
	Scassinare (19)\\
	Scolpire Pietra (23)\\
	Scurovisione (19)\\
	Telecinesi (26)\\
	Trasformazione (36)\\
	Trucco della Corda (19)\\
	Velocità (21)\\
	Volare (21)\\
	
	\subsubsection{Universale}
	Artificio Druidico (12)\\
	Prestidigitazione (12)\\
	Taumaturgia (12)\\
	
	
	
\end{multicols}

\pagebreak

\subsection{Incantesimi antichi e perduti}

Gli incantesimi qui presenti sono stati persi nella storia e solo leggende rimandano alla loro esistenza.\\
Questi incantesimi sono i più rari da poter trovare e non possono essere scelti alla creazione del personaggio.

\begin{multicols}{2}

\medskip\textbf{Alleato Planare}\index{Incantesimi - Alleato Planare}\\
\textbf{Scuola}: Evocazione\\
\textbf{Difficoltà}: 29\\
\textbf{Tempo di Lancio}: 10 minuti\\
\textbf{Gittata}: 18 metri\\
\textbf{Componenti}: V, S\\
\textbf{Durata}: Istantanea\\
Supplichi un'entità ultraterrena perché ti conceda aiuto. L'essere ti deve essere noto: un dio, un primordiale, un principe dei demoni, o qualche altra creatura di grande potere. Quell'entità invia un celestiale, elementale o demone a essa leale perché ti aiuti, facendo comparire la creatura in uno spazio non occupato a gittata. Se conosci il nome di una specifica creatura, puoi pronunciarne il nome quando lanci questo incantesimo per richiedere l'aiuto di quella creatura, sebbene tu possa comunque riceverne un'altra (a discrezione del Narratore).\\
Quando la creatura appare, non è sotto l'obbligo di agire in alcun modo particolare. Puoi chiedere alla creatura di svolgere un servizio in cambio di una ricompensa, ma essa non è obbligata a soddisfarti. Il compito richiesto potrebbe essere facile ("portaci in volo oltre il baratro" o "aiutaci a combattere questa battaglia") o complesso ("spia i nostri nemici" o "proteggici durante la nostra esplorazione del sotterraneo"). Devi essere in grado di comunicare con la creatura per patteggiare i suoi servigi. La ricompensa può assumere diverse forme. Un celestiale potrebbe chiedere una considerevole donazione di oro od oggetti magici a un tempio alleato, mentre un demone potrebbe richiedere un sacrificio umano o il dono di un tesoro. Alcune creature potrebbero scambiare i loro servigi per una missione che dovrai intraprendere per conto loro. Come regola generale, un compito che può essere misurato in minuti richiede una ricompensa di 100 mo al minuto. Un compito misurato in ore, richiede 1.000 mo all'ora. Un compito misurato in giorni (massimo 10 giorni) richiede 10.000 mo al giorno. Il Narratore può modificare queste ricompense in base alle circostanze nelle quali si è lanciato l'incantesimo Se il compito è allineato alla morale della creatura, la richiesta di pagamento potrebbe essere dimezzata o addirittura annullata. I compiti non pericolosi di solito chiedono solo la metà di quanto suggerito come pagamento, mentre i compiti molto pericolosi possono richiedere donazioni superiori. È raro che queste creature accettino compiti che sembrino suicida.\\
Dopo che la creatura ha completato il compito, o quando il periodo di servizio concordato è terminato, la creatura tornerà al suo piano natio dopo averti fatto rapporto, se appropriato al compito svolto e se possibile. Se non sei in grado di concordare un prezzo per i servigi della creatura, la creatura tornerà immediatamente al suo piano natio. Una creatura arruolata per unirsi al tuo gruppo è considerata come un suo membro, e riceve una quota piena delle ricompense in punti esperienza.

\medskip\textbf{Bagliore Lunare}\index{Incantesimi - Bagliore Lunare}\\
\textbf{Scuola}: Invocazione\\
\textbf{Difficoltà}: 19\\
\textbf{Tempo di Lancio}: 2 Azioni\\
\textbf{Gittata}: 36 metri\\
\textbf{Componenti}: V, S, M (diversi semi di bella di notte e un pezzo di felpato opalescente)\\
\textbf{Durata}: Concentrazione, massimo 1 minuto\\
Un fascio argenteo di luce pallida risplende in un cilindro di raggio 1 metro, alto 12 metri centrato in un punto a gittata. Fino al termine dell'incantesimo, una luce fioca riempie il cilindro. \\
Quando una creatura entra nell'area dell'incantesimo per la prima volta durante un round o inizia qui il suo round, è avvolta da fiamme spettrali che provocano un dolore terribile, e deve effettuare un Tiro Salvezza su Tempra. Se fallisce il Tiro Salvezza subisce 2d10 danni da Luce, o la metà di questi danni se lo supera. Un mutaforma effettua il Tiro Salvezza con -1d6. Se lo fallisce ritorna immediatamente alla sua forma originale e non può assumere una forma diversa finché non esce dalla luce dell'incantesimo.\\
Durante ciascun tuo round dopo aver lanciato l'incantesimo, puoi usare un'azione per muovere il
fascio di 18 metri in qualsiasi direzione. \\
\textbf{Per ogni Critico ottenuto} nella prova di magia il danno aumenta di 1d10

\medskip\textbf{Contattare Altri Piani}\index{Incantesimi - Contattare Altri Piani}\\
\textbf{Scuola}: Divinazione\\
\textbf{Difficoltà}: 26\\
\textbf{Tempo di Lancio}: 1 minuto\\
\textbf{Gittata}: Personale\\
\textbf{Componenti}: V\\
\textbf{Durata}: 1 minuto\\
Contatti mentalmente un semidio, lo spirito di un saggio da tempo defunto, o qualche altra misteriosa entità di un altro piano. Contattare l'Intelligenza extraplanare può affaticare o addirittura spezzare la tua mente. Quando lanci questo incantesimo, effettua un Tiro Salvezza su Volontà con DC 15. Se lo fallisci, subisci 6d6 danni e resti demente fino all'alba del giorno dopo. Mentre sei demente, non puoi effettuare azioni, non puoi capire quello che dicono le altre creature, non puoi leggere, e parli solo farneticando. L'incantesimo ristorare superiore può porre fine a questo effetto. Se superi il Tiro Salvezza, puoi porre all'entità fino a cinque domande. Devi porre le domande prima del termine dell'incantesimo. Il Narratore risponderà a ciascuna domanda con una parola: "sì", "no", "forse", "mai", "irrilevante" o "confuso" (se l'entità non conosce la risposta alla domanda). Se una risposta di una parola potrebbe risultare fuorviante, il Narratore potrebbe invece dare come risposta una breve frase.

\medskip\textbf{Evoca Celestiali}\index{Incantesimi - Evoca Celestiali}\\
\textbf{Scuola}: Evocazione\\
\textbf{Difficoltà}: 31\\
\textbf{Tempo di Lancio}: 1 minuto\\
\textbf{Gittata}: 27 metri\\
\textbf{Componenti}: V, S\\
\textbf{Durata}: 10 minuti\\
Evochi un celestiale di grado di sfida 4 o inferiore, che appare in uno spazio non occupato a gittata e che puoi vedere. Il celestiale sparisce quando scende a 0 punti ferita o l'incantesimo termina. Il celestiale è amichevole verso di te e i tuoi compagni per la durata dell'incantesimo. Tira l'iniziativa per il celestiale, che agisce durante il proprio round. Obbedisce a qualsiasi comando verbale che gli viene dato (senza bisogno che tu compia azioni), purché non violi i suoi Tratti. Se non dai comandi al celestiale, si difenderà dalle creature ostili, ma non compirà altre azioni.\\
\textbf{Per ogni Critico ottenuto} nella prova di magia aumenti di uno il CR della creatura evocata.
	
\medskip\textbf{Evoca Folletto}\index{Incantesimi - Evoca Folletto}\\
\textbf{Scuola}: Evocazione\\
\textbf{Difficoltà}: 29\\
\textbf{Tempo di Lancio}: 1 minuto\\
\textbf{Gittata}: 27 metri\\
\textbf{Componenti}: V, S\\
\textbf{Durata}: 1 ora \\
Evochi uno spirito fatato di grado di sfida 6 o inferiore, o uno spirito fatato che assuma la forma di una bestia di grado di sfida 6 o inferiore. Esso compare in uno spazio non occupato a gittata e che puoi vedere. La creatura fatata sparisce quando scende a 0 punti ferita o quando l'incantesimo termina.\\
La creatura fatata è amichevole verso di te e i tuoi compagni. Tirare l'iniziativa per la creatura fatata, che agisce durante i propri turni. Essa obbedisce a qualsiasi comando verbale che gli viene dato (senza bisogno che tu compia azioni), purché non violi i suoi Tratti. Se non dai comandi, si difenderà dalle creature ostili, ma non compirà altre azioni.\\
\textbf{Per ogni Critico ottenuto} nella prova di magia aumenti di 1 il CR della creatura evocata.

\medskip\textbf{Guardiano della Fede}\index{Incantesimi - Guardiano della Fede}\\
\textbf{Scuola}: Evocazione\\
\textbf{Difficoltà}: 23\\
\textbf{Tempo di Lancio}: 2 Azioni\\
\textbf{Gittata}: 9 metri\\
\textbf{Componenti}: V\\
\textbf{Durata}: 8 ore\\
Un guardiano spettrale Grande compare per la durata e fluttua in uno spazio non occupato a gittata e che puoi vedere, scelto da te. Il guardiano occupa quello spazio ed è indistinguibile eccetto per una spada luminosa e uno scudo con il simbolo del tuo Patrono.\\
Qualsiasi creatura a te ostile che entri in uno spazio entro 3 metri dal guardiano per la prima volta durante un round, deve effettuare un Tiro Salvezza su Riflessi. La creatura subisce 20 danni da Luce/Vuoto se fallisce il Tiro Salvezza, o la metà di questi danni se lo supera. Il guardiano svanisce dopo aver inflitto un totale di 60 danni.

\medskip\textbf{Guardiani Spirituali}\index{Incantesimi - Guardiani Spirituali}\\
\textbf{Scuola}: Evocazione\\
\textbf{Difficoltà}: 21\\
\textbf{Tempo di Lancio}: 2 Azioni\\
\textbf{Gittata}: Personale (raggio di 4 metri)\\
\textbf{Componenti}: V, S, M (un simbolo sacro)\\
\textbf{Durata}: Concentrazione, massimo 10 minuti\\
Richiami degli spiriti che ti proteggano. Per la durata dell'incantesimo, essi fluttueranno intorno a te a una distanza di 4 metri. Se sei buono o neutrale, la forma spettrale sarà angelica o fatata (a tua scelta). Se sei malvagio, avranno un aspetto demone. Quando lanci questo incantesimo, puoi designare un qualsiasi numero di creature che ne siano immuni. La velocità di una creatura soggetta viene dimezzata all'interno dell'area, e quando una creatura entra nell'area per la prima volta durante un round o inizia il suo round lì, deve effettuare un Tiro Salvezza su Volontà. Se fallisce il Tiro Salvezza subisce 3d8 danni da Luce (se sei buono o neutrale) o 3d8 danni da Vuoto (se sei malvagio), o la metà di questi danni se lo supera.\\
\textbf{Per ogni Critico ottenuto} nella prova di magia il danno aumenta di 1d8 \\

\medskip\textbf{Legame Planare}\index{Incantesimi - Legame Planare}\\
\textbf{Scuola}: Abiurazione\\
\textbf{Difficoltà}: 26\\
\textbf{Tempo di Lancio}: 1 ora\\
\textbf{Gittata}: 18 metri\\
\textbf{Componenti}: V, S, M (un gioiello del valore di almeno 1.000 mo, che l'incantesimo consuma)\\
\textbf{Durata}: 24 ore\\
Con questo incantesimo, cerchi di vincolare un celestiale, elementale, fatato o demone al tuo servizio. La creatura deve restare nella gittata per l'intero lancio dell'incantesimo. (Di solito, la creatura viene prima evocata al centro di un cerchio magico invertito per tenerla intrappolata mentre questo incantesimo viene lanciato). Al completamento del lancio, il bersaglio deve effettuare un Tiro Salvezza su Volontà. Se fallisce il Tiro Salvezza, è vincolato al tuo servizio per la durata. Se la creatura è stata evocata o creata da un altro incantesimo, la durata di quell'incantesimo viene estesa per corrispondere alla durata di questo incantesimo. Una creatura vincolata deve eseguire le tue istruzioni al meglio delle sue capacità. Potresti comandare la creatura di accompagnarti nel corso di un'avventura, di proteggere un luogo o di consegnare un messaggio. La creatura obbedisce le tue istruzioni alla lettera, ma se ti è ostile, cercherà di distorcere le tue parole ai suoi fini. Se la creatura adempie completamente alle tue istruzioni prima del termine dell'incantesimo, qualora vi troviate sullo stesso piano di esistenza ritornerà da te per comunicarti l'avvenuto. Se vi trovate su piani di esistenza diversi, ritornerà nel luogo dove l'hai vincolata e rimarrà lì fino al termine dell'incantesimo.\\
\textbf{Per ogni Critico ottenuto} nella prova di magia raddoppi la permanenza della creatura

\medskip\textbf{Marchio del Cacciatore}\index{Incantesimi - Marchio del Cacciatore}\\
\textbf{Scuola}: Divinazione\\
\textbf{Difficoltà}: 16\\
\textbf{Tempo di Lancio}: 2 Azioni\\
\textbf{Gittata}: 27 metri\\
\textbf{Componenti}: V \\
\textbf{Durata}: Concentrazione, massimo 1 ora\\
Scegli una creatura a gittata che puoi vedere. La creatura è misticamente marchiata come tua preda. Fino al termine dell’incantesimo, infliggi 1d6 danni aggiuntivi al bersaglio ogni volta che lo colpisci con un attacco con arma, e hai +1d6  alle prove di Consapevolezza o Sopravvivenza per trovarlo.\\
Se il bersaglio scende a 0 punti ferita prima del termine dell’incantesimo, puoi usare un’azione immediata durante il tuo prossimo turno per marchiare una nuova creatura.\\
\textbf{Per ogni Critico ottenuto} nella prova di magia puoi mantenere la concentrazione sull’incantesimo un altra ora\\

\medskip\textbf{Portale}\index{Incantesimi - Portale}\\
\textbf{Scuola}: Evocazione\\
\textbf{Difficoltà}: 36\\
\textbf{Tempo di Lancio}: 2 Azioni\\
\textbf{Gittata}: 18 metri\\
\textbf{Componenti}: V, S, M (un diamante del valore di almeno 5.000 mo)\\
\textbf{Durata}: Concentrazione, massimo 1 minuto\\
Evochi in uno spazio non occupato a gittata che puoi vedere un portale collegato a un posto preciso su di un diverso piano di esistenza. Il portale è un'apertura circolare creata da te, da 1 a 6 metri di diametro. Puoi orientare il portale in qualsiasi direzione desideri. Il portale resta per la durata.\\
Il portale ha un fronte e un dietro su entrambi i piani in cui compare. Il viaggio attraverso il portale è possibile solo muovendosi dal fronte. Qualsiasi cosa lo faccia viene istantaneamente trasportata nell'altro piano, comparendo nello spazio non occupato più vicino al portale.\\
Divinità e altri sovrani planari possono impedire ai portali creati da incantesimi di aprirsi in loro presenza o in qualsiasi punto dei loro domini. Quando lanci questo incantesimo, puoi pronunciare il nome di una specifica creatura (lo pseudonimo, titolo o soprannome non funzionano). Se quella creatura si trova su di un piano diverso dal tuo, il portale si apre in prossimità della creatura nominata e attira la creatura attraverso di sé, verso lo spazio non occupato più vicino dal tuo lato del portale. Non detieni alcun potere speciale sulla creatura, ed essa è libera di agire come il Narratore ritiene appropriato. Potrebbe andarsene, attaccarti o aiutarti.

\medskip\textbf{Salvare i Morenti}\index{Trucchetto - Salvare i Morenti}\\
\textbf{Scuola}: Necromanzia\\
\textbf{Difficoltà}: 12\\
\textbf{Tempo di Lancio}: 2 Azioni\\
\textbf{Gittata}: Contatto\\
\textbf{Componenti}: V, S, M (un offerta al tuo Patrono di almeno 5 mo, che l'incantesimo consuma)\\	
\textbf{Durata}: Istantanea\\
Una creatura a 0 punti ferita, con cui sei a contatto, torna a 1 PF. L'incantesimo non ha effetto su non morti o costrutti.

\medskip\textbf{Tempesta di Vendetta}\index{Incantesimi - Tempesta di Vendetta}\\
\textbf{Scuola}: Evocazione\\
\textbf{Difficoltà}: 36\\
\textbf{Tempo di Lancio}: 2 Azioni\\
\textbf{Gittata}: Vista\\
\textbf{Componenti}: V, S\\
\textbf{Durata}: Concentrazione, massimo 1 minuto\\
Si forma una ribollente nube di tempesta, centrata in un punto che puoi vedere e che si propaga in un raggio di 110 metri. L'area è illuminata da fulmini, vi riecheggiano tuoni e venti forti la spazzano. Quando la nube compare, ogni creatura sotto di essa (ovvero non più di 1.500 metri sotto la nube) deve effettuare un Tiro Salvezza su Tempra. Se fallisce il Tiro Salvezza, la creatura subisce 2d6 danni da tuono e resta assordata per 5 minuti.\\
Ogni round in cui mantieni la concentrazione su questo incantesimo, la tempesta, durante il tuo round, produce ulteriori effetti.\\
\textit{Round 2}. Pioggia acida cade dalla nube. Ogni creatura e oggetto sotto la nube subiscono 1d6 danni da acido.\\
\textit{Round 3}. Richiami sei fulmini dalla nube per colpire sei creature o oggetti di tua scelta, che si trovino sotto la nube. Una specifica creatura od oggetto non può essere colpita da più di un fulmine. Una creatura colpita deve effettuare un Tiro Salvezza su Riflessi. La creatura subisce 10d6 danni da fulmine se fallisce il Tiro Salvezza, o la metà di questi danni se lo supera. \\
\textit{Round 4}. La nube produce una fitta grandinata. Ogni creatura sotto la nube subisce 2d6 danni da botta.\\
\textit{Round 5-10}. Folate di vento e pioggia gelida si abbattono sull'area sotto la nube. L'area diventa terreno difficile ed è in penombra. Ogni creatura nell'area subisce 1d6 danni da freddo. Nell'area diventa impossibile effettuare attacchi con armi a distanza. Il vento e la pioggia sono considerati una distrazione grave ai fini del mantenere la concentrazione sugli incantesimi.\\ Infine, folate di forte vento (che va dai 30 ai 75 chilometri all'ora) disperdono automaticamente nebbia, foschia e simili fenomeni nell'area, che siano naturali o magici.

\medskip\textbf{Trova Famiglio}\index{Incantesimi - Trova Famiglio}\\
\textbf{Scuola}: Evocazione\\
\textbf{Difficoltà}: 16\\
\textbf{Tempo di Lancio}: 1 ora\\
\textbf{Gittata}: 3 metri\\
\textbf{Componenti}: V, S, M (10 mo di carbone, incenso e erbe che devono essere consumate dal fuoco in un braciere d'ottone)\\
\textbf{Durata}: Istantanea\\
Ottieni il servizio di un famiglio, uno spirito che assume una forma animale di tua scelta: cavalluccio marino, corvo, donnola, falco, gatto, granchio, gufo, lucertola, pesce (frizzo), piovra, pipistrello, ragno, rana (rospo), ratto o serpente velenoso. Apparendo in uno spazio a gittata, non occupato, il famiglio ha le statistiche della forma scelta, sebbene sia di tipo celestiale, fatato o demone (a tua scelta) invece di una bestia. Il tuo famiglio agisce in maniera indipendente da te, ma ubbidisce sempre ai tuoi comandi. In combattimento, tira la propria iniziativa e agisce durante il proprio round. Un famiglio non può attaccare, ma può svolgere le altre azioni come di norma. 
Non puoi avere più di un famiglio alla volta. \\
\textbf{Verifica Abilità Famiglio} per le capacità del famiglio. Devi avere l'Abilita' Famiglio.

\subsection{Nuovi incantesimi}

In accordo con il Narratore il giocatore e' invitato a creare o proporre nuovi incantesimi.\\
Per praticità includo la tabella di conversione del Livello dell'Incantesimo, se usate Pathfinder o la 5e del Gioco di Ruolo Fantasy.\\

\medskip

\textbf{Tabella conversione Livello Incantesimo - Difficoltà}

\medskip

\begin{tabular}{ll}
\textbf{Livello} & \textbf{Difficoltà}\\
0	& 12\\
1	& 16\\
2	& 19\\
3	& 21\\
4	& 23\\
5	& 26\\
6	& 29\\
7	& 31\\
8	& 34\\
9	& 36\\
\end{tabular}
\end{multicols}
