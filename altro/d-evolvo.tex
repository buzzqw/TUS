\documentclass[12pt,a4paper]{book}
\usepackage[utf8]{inputenc}
\usepackage[T1]{fontenc}
\usepackage{amsmath}
\usepackage{amsfonts}
\usepackage{amssymb}
\usepackage{makeidx}
\usepackage{graphicx}
\begin{document}
	
5 livelli di difficolta', ogni livello di difficolta' significa ottenere un risultato positivo in piu'

il risultato e' fare 6 o piu' con il tiro del dado

caratteristica  + dado esperienza >= 6 risultato positivo, altrimenti negativo

\medskip

\begin{tabular}{ll}
	\hline
esperienza & dado esperienza\\
nulla		&	0	\\
scarsa		&	1d4	\\
normale		&   1d6	\\
buona		&	1d8	\\
ottima		&	1d10\\	
eccellente	&	1d12\\
\end{tabular}

\medskip

\begin{tabular}{ll}
	\hline
difficolta' della prova & prove da superare\\
	Normale          			& 1\\
	Difficile        			& 2\\
	Molto difficile 	 		& 3\\
	Estremamente difficile      & 4\\
	Quasi impossibile			& 5\\
	Leggendaria      			& 6\\
\end{tabular}

\medskip

se ho esperienza "normale" e la prova e' "difficile" devo fare 2 prove.\\

Tiro 1d6+modificatore caratteristica due volte e tutti e due devono fare 6 o piu'\\

i modificatori vari possono andare da +- 2\\

la magia e' basata sugli elementi e liste

aria, acqua, terra, fuoco, vita, entropia, spazio , tempo (queste due sono liste avanzate...)
	
il sistema	si basa su un blocco di competenze che assumono un certo grado di esperienza\\

la magia e' una competenza, solo che esplicito l'elemento che imparo\\
ogni elemento mi apre la strada a piu' liste di incantesimo con una difficolta' crescente che posso apprendere solo se ho quel livello di competenza nella lista (es. Lista Fuoco, livello esp. 3, ha certi incantesimi...)\\

la difesa funziona come una competenza, esprima una difficolta' da 1 2 3.. o piu' successi da passare con la proprio competenza di arma\\

un tiro contrapposto e' la somma dei valori dei tiri ottenuti confrontata con l'avversario\\

un tiro salvezza viene esplicitato nell'incantesimo solitamente devi ottenere un numero di successi pari max al livello di potere dell'incantesimo (lista Fuoco, esp 3 puo' richiedere 3 tiri riusciti su qualcosa)\\

i punti ferita sono pari alla caratteristica, quindi bassi\\

l'arma fa 1, 2 , 3 danni a seconda della differenza tra quanti successi ho nella prova al colpire e quanto hai difesa tu\\

la razza da la classe, ovvero da un set di competenze note di base\\

le caratteristiche:

corpo, mente, spirito\\


Atletica
Consapevolezza
Intimidire
Combattimento corpo a corpo
Sesto Senso

affinità Animale
Armi da gittata
Armi da Mischia
Criminalità
Furtività
Galateo
cavalcare
Sopravvivenza

Magia

Conoscenza: xxx
Medicina



	
\end{document}