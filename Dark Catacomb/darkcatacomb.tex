\documentclass[12pt,a4paper,twoside,openany,twocolumn]{book}
\usepackage[utf8]{inputenc}
\usepackage[T1]{fontenc}
\usepackage[left=2.00cm, right=2.00cm, top=2.00cm, bottom=2.00cm]{geometry}
\usepackage{graphicx}
\usepackage{hyperref}
\usepackage{xcolor}

\begin{document}

Introduzione\\
Razza\\
Caratteristiche\\
rami base\\
rami avanzati\\
Punti Fortuna\\
Punti Fato\\
Karma\\
Competenze\\
Costruiamo il personaggio\\
Regole per le competenze\\
Combattimento\\
Nascondigli e coperture\\
liste armi ed armature\\
abilita' ???\\
magia\\
incantesimi\\
equipaggiamento\\
veleni e droghe\\
movimento\\
oggetti magici\\
mostruario\\
Condizioni\\
Scheda\\

Settings v1:  in un futuro non troppo distante inquinamento, guerre, cambiamento climatico con carestie ed inondazioni hanno reso la terra oramai inabitabile. poi improvvisamente vengono scoperte leghe metalliche e cristalline che permettono di imbrigliare e conservare l'energia a livelli neanche immaginabili.
Questo progresso enfatizzo ancora di piu' la divisione sociale. Le poche corporazioni che avevano i nuovi materiali non fecero altro che arricchirsi e depauperare chiunque avesse risorse da vendere.
Piccole enclavi vivevano nel lusso ed in un ambiente idilliaco mentre il 98\% della popolazione si arrabattava come poteva e cercava di resistere ad un pianeta che piu' non lo voleva.
Ormai sull'orlo dell'estinzione la OXF Corp dichiaro' di aver migliorato il proprio materiale in grado di purificare l'aria, rendendo capace un solo cristallo di poter generare l'aria per una intera casa, e non una sola persona.
Durante lo show in mondo visione dove veniva mostrato come veniva aggiornato il "cristallo d'aria" avvenne quello che tutti poi chiamarono la Frattura. 
Il cristallo d'aria incomincio' a risonare, ad emettere un sordo suono e generare onde armoniche sempre con maggiore intensita'. Mentra la terra intorno incominciava a tremare la sede della OXF Corp si spezzo in due e da qualche laboratorio segreto in profondita' una luce intensa e pura sali' alta nel cielo. Per diversi minuti fu solo il panico a dominare gli animi finche' in mezzo a quella luce che si andava dissipando apparve una figura a mezzaria. I lineamenti erano umani, ma non era umana. La carnagione era dorata, i capelli come argento, le mani affusolate ed i lineamenti delicati, le orecchie stranamente lunghe ed a punta.
Il volto basso, gli occhi chiusi quasi fosse in profonda meditazione od addirittura morta.
Un attimo prima che la luce scomparisse del tutto quella creatura emisi una intensissima luce dorata per poi scomparire.
Quello che avvenne dopo fu la vera e propria "Frattura".

Il cielo sembro' aprirsi lasciando intravedere le profonde e lontane stelle, la spaccatura sotto la OXF Corp divenne un crepaccio senza fine.
Entrame le oscurita', diverse eppure simili, una difronte all'altra incominciarono a pulsare e migliaia di esseri dall'una e dall'altra parte incominciarono a camminare sulla terra.

Questo fenomeno, la Fratuttura, non fu un caso isolato, ovunque nel mondo dove ci fosse stato un cristallo purificatore avvenne una Frattura, magari di potenza minore, ma appastanza per distruggere diversi quartieri e richiamare altre creature dal cielo e dalle viscere della terra.

Cio' che usciva dal cielo non erano angeli, come cio' che usciva dalla terra non erano demoni...


Setting v2: 


Razze: umani, nani, elfi, mezz'orchi, mezz'elfi, ... trovare qualche razza interessante e particolare. no scurovisione

Caratteristiche: Mente, Corpo, Volonta'. tiri 2d10 per il valore, qualsiasi valore oltre 12 e' 12, qualsiasi valore sotto 4 e' 4.
Le caratteristiche non danno bonus ai tiri, ma sono prerequisiti per avanzamenti di rami, competenza magica ed armi (non puoi avere comp. armi/magica superiore al valore della statistica), scelta ed uso incantesimi. A seconda del ramo prendi dei bonus o malus (solo rami molto avanzati, sempre a fare il mago aumenti mente)


Kismet: ogni giocatore indica come il suo personaggio morira'.

statistica fortuna: si usa per abbassare il tiro fatto. recuperi 1 quando fai un fallimento critico. dichiari prima e spendi prima.
Ha un valore di 1d4 per sessione.

classi/rami: non serve perdersi, bastano 12 rami base, 8 intermedie, 6 avanzate, 4 elite
in una ramo si rimane finche' non si soddisfano i requisiti dei rami superiori
quindi le abilita' aumentano sempre, anche rimanendo nello stesso ramo
i punteggi aumentano con gli errori: se tiri 19/20 (sui 2d10) oltre al fallimento o fallimento critico segni con un pallino la competenza, quando c'e' downtime fai un check per ogni pallino e se fai 2 (ovvero 1/1) aumenti di 1 punto. massimo si segnano 5 pallini, poi si cancellano fatte le prove.

iniziativa: 2d10 0 arma leggera, +2 arma media, +6 arma 2 mani. Incantesimi se hanno solo verbale o somatico o materiali +0, con due componenti +2, con tre componenti +6 con ognuno tira la sua. i mostri la tirano in base alla loro taglia se piccoli, medi grandi...

azioni: rimanere con le 3 azioni. rivedere i costi per premiare azioni "smart". 

competenze di base: ogni ramo modifica/o meno il punteggio di Combattimento o Magia.

combattimento: chi attacca non tira nulla, dichiara solo che arma usa e se usa manovre. chi difende tira sotto il suo valore di combattimento, questo valore di combattimento puo' avere svantaggi dati da abilita' dell'attaccante (rami combattenti avanzati). se tira sotto allora para/evita il colpo, se tira sopra viene preso.
Si presume che l'attacco colpisce sempre se la difesa non funziona bene. Questo significa che le classi combattenti non aumentano l'attacco ma solo la difesa. alcuni rami avanzati di combattimento danno delle penalita' alla difesa avversaria.
il danno e' 1d6 per armi piccole, 1d8 armi medie, 1d10 armi grandi. 

armi: armi piccole 1d6, armi medie 1d8, armi grandi 2 mani 1d10. non sommi valore forza. arma leggera ha requisito forza 6, media 9, grande 11, altrimenti svantaggio nell'uso

una difesa particolarmente riuscita (almeno -6) puo' dare svantaggio al tiro successivo, un -9 potrebbe fare tirare con un solo dado
quando difendi  a seconda di quanto bene difendi, ovvero se hai un margine di -3,-6,-9.. rispetto alla tua prova ottieni dei bonus, azioni movimento, penalita' all'attacco successivo

le manovre funzionano con lo stesso principio. quando effettuo una manovra e l'altro riesce nel tiro di combattimento allora la manovra non riesce e io devo fare un tiro di combattimento, se fallisco prendo gli effetti della manovra

armature: riducono il danno. leggere riducono di 2, medie riducono di 3, pesanti riducono di 4. danno una penalita' alle prove di competenza. Non si fa magia con armatura se non solo a contatto
scudo: fai la prova di difesa con +1 (da 1 bonus). penalita' alle prove di competenza (da 1 penalita')

modificatori alle prove: un bonus e' +1, due bonus tiri 1d4, 3 bonus tiri 1d6 (non sono provisti oltre 3 bonus cumulativi). le penalita' funzionano in maniera simmetrica. bonus e malus si annullano in maniera reciproca.

vantaggio: tiri tre dadi e prendi 1 dado piu' basso piu' un altro a tua scelta   |  svantaggio: tiri tre dadi e prendi 1 dado con il punteggio piu' alto piu' un altro a tua scelta | attenzione al valore 0 zero.

magia: deve essere lanciare piu' difficile lanciare incantesimi e consumare "risorse" (pf/energia/stamina..)

certi rami magici possono privilegiare certe scuole di magia. lavorare su liste e ridurre e tanto. l'idea di base e' per esempio crea fuoco puo' diventare a seconda del punteggio del tiro altre cose, il valore influenza la distanza, AoE, danno, se e' un raggio o esplosione e che raggio...  Pochi incantesimi ma che si evolvono

lanciare un incantesimo: tirare a secondo dall'incantesimo su capacita' magica e avere un punteggio minimo di  mente, corpo o volonta'.  Ogni punteggio -3 rispetto alla prova, es. capacita' magica 13 e tiro 8, potenzi un fattore dell'incantesimo (distanza, AoE, danno, tipo di effetto..). Gli incantesimi hanno un punteggio minimo di mente/corpo/volonta' per essere tirati ed anche di capacita' magica
Ogni incantesimo ha degli attributi, danno, distanza, aoe, durata ed un punteggio minimo di competenza magica e corpo/mente/volonta'. se il tiro riesce bene puoi potenziare un attributo presente ma non darlo/aggiungerlo se questo e' assente. se lancio l'incantesimo crea fiamma, inc. base difficolta' 1, se faccio un ottimo tiro potro' potenziare la durata e aoe (l'area di luce che fa) ed il danno, ma non posso aggiungere distanza perche' e' un attributo assente.
ci sara' poi l'incatesimo globo di fuoco, la versione base della palla di fuoco, questa ha piu' attributi ma ad esempio non ha durata, o meglio l'istantanea non puo' essere migliorata.
valutare di aggiungere attributi assenti quando il tiro e' veramente buono, ovvero tiri veramente basso, direi almeno un -6 per aggiungere un attributo a livello base (3 metri)

se l'incantesimo riesce nel lancio non c'e' TS. alcuni incantesimi possono avere una prova di difesa per essere evitati/dimezzati l'effetto

questo implica che non ci saranno mai incantesimi super potenti in tutto..

quanti incantesimi lanciare:  a piacere, ma ogni volta che lanci lo stesso il punteggio minimo richiesta di capacita' magica dell'incantesimo aumenta di 1. quindi tiro l'incantesimo cura, richiede corpo 1 e capacita' magica 11 - la prova cmq la faccio sul mio punteggio di capacita' magica e devo fare piu' basso, lo ritiro e richiede corpo 2 e capacita' magica 12, arrivo ad un certo punto che non posso piu' soddisfare i requisiti richiesti anche se le prove le ho sempre superate

quando lanci un incantesimo e fallisci la prova non succede nulla, ma l'incantesimo l'hai lanciato e quindi i requisiti minimi aumentano di 1
quando lanci un incantesimo e fallisci con 19-20 la prova succedono cose  brutte
quando lanci l'incantesimo e fai veramente basso 2, non aumentano i requisiti per il lancio successivo.


mostri:
se il mostro tira la sua difesa c'e' il rischio che riesca sempre ad alti livelli. considerare abilita' che abbassano la difesa, l'idea di base e' che se un mostro ha difesa 16 o piu' deve essere di un livello tale da dover affrontare pg con rami che danno penalita' alla difesa


--------------------------------------------------------------------------------



- no classi, albero con dipendenze. ogni ramo ti da delle abilita' , come i feat

- la professione iniziale ti da dei punteggi di competenza base, solitamente vanno da 8 a 10.

- check/prove: per fare una prova tiri il 2d10 e devi fare meno della statistica o competenza. se e' collegata alla professione tiri con vantaggio

- fare 2 o 20  successo critico o fallimento  critico

- il giocatore dichiara cio' che fa e' solo il master a stabilire se serve una prova. 

- i rami presi possono aggiungere "competenze professionali". aumentano pf, competenze armi, competenze magica, competenze professionali. Per passare da un ramo ad un altro devi avere un valore minimo in certi punteggi

- il tempo e' un fattore, tabelle random per incontri basati su tempo trascorso, si computa il tempo reale.

- no livelli, quando una prova fallisce tirando 19 segni un pallino, se fai 00 aumenti direttamente di 1. quando c'e' riposo e tempo fai una prova per ogni pallino, se tiri 2 allora alzi di 1 il punteggio


- punti ferita: a seconda della razza ed ogni ramo che prendi lo aumenta di un valore dipendente dal ramo scelto, ma l'aumento e' sempre poco (1-4)

- PF: corpo + ramo/rami. quindi da 4 a 12/13. Rami combattenti aggiungono 3, 
magici 1.. molto magici se avanzati 0, rami ibridi 2/1

- PF: recupero. ogni notte recuperi 1/2 corpo (o corpo?)


- mostri:

\end{document}