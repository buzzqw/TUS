\documentclass[12pt,a4paper,twoside,openany,twocolumn]{book}
\usepackage[utf8]{inputenc}
\usepackage[T1]{fontenc}
\usepackage[left=2.00cm, right=2.00cm, top=2.00cm, bottom=2.00cm]{geometry}
\usepackage{graphicx}
\usepackage{hyperref}
\usepackage{xcolor}

\begin{document}

Introduzione\\
Razza\\
Caratteristiche\\
rami base\\
rami avanzati\\
Punti Fortuna\\
Punti Fato\\
Karma\\
Competenze\\
Costruiamo il personaggio\\
Regole per le competenze\\
Combattimento\\
Nascondigli e coperture\\
liste armi ed armature\\
abilita' ???\\
magia\\
incantesimi\\
equipaggiamento\\
veleni e droghe\\
movimento\\
oggetti magici\\
mostruario\\
Condizioni\\
Scheda\\

Settings v1:  in un futuro non troppo distante inquinamento, guerre, cambiamento climatico con carestie ed inondazioni hanno reso la terra oramai inabitabile. poi improvvisamente vengono scoperte leghe metalliche e cristalline che permettono di imbrigliare e conservare l'energia a livelli neanche immaginabili.
Questo progresso enfatizzo ancora di piu' la divisione sociale. Le poche corporazioni che avevano i nuovi materiali non fecero altro che arricchirsi e depauperare chiunque avesse risorse da vendere.
Piccole enclavi vivevano nel lusso ed in un ambiente idilliaco mentre il 98\% della popolazione si arrabattava come poteva e cercava di resistere ad un pianeta che piu' non lo voleva.
Ormai sull'orlo dell'estinzione la OXF Corp dichiaro' di aver migliorato il proprio materiale in grado di purificare l'aria, rendendo capace un solo cristallo di poter generare l'aria per una intera casa, e non una sola persona.
Durante lo show in mondo visione dove veniva mostrato come veniva aggiornato il "cristallo d'aria" avvenne quello che tutti poi chiamarono la Frattura. 
Il cristallo d'aria incomincio' a risonare, ad emettere un sordo suono e generare onde armoniche sempre con maggiore intensita'. Mentra la terra intorno incominciava a tremare la sede della OXF Corp si spezzo in due e da qualche laboratorio segreto in profondita' una luce intensa e pura sali' alta nel cielo. Per diversi minuti fu solo il panico a dominare gli animi finche' in mezzo a quella luce che si andava dissipando apparve una figura a mezzaria. I lineamenti erano umani, ma non era umana. La carnagione era dorata, i capelli come argento, le mani affusolate ed i lineamenti delicati, le orecchie stranamente lunghe ed a punta.
Il volto basso, gli occhi chiusi quasi fosse in profonda meditazione od addirittura morta.
Un attimo prima che la luce scomparisse del tutto quella creatura emisi una intensissima luce dorata per poi scomparire.
Quello che avvenne dopo fu la vera e propria "Frattura".

Il cielo sembro' aprirsi lasciando intravedere le profonde e lontane stelle, la spaccatura sotto la OXF Corp divenne un crepaccio senza fine.
Entrame le oscurita', diverse eppure simili, una difronte all'altra incominciarono a pulsare e migliaia di esseri dall'una e dall'altra parte incominciarono a camminare sulla terra.

Questo fenomeno, la Fratuttura, non fu un caso isolato, ovunque nel mondo dove ci fosse stato un cristallo purificatore avvenne una Frattura, magari di potenza minore, ma appastanza per distruggere diversi quartieri e richiamare altre creature dal cielo e dalle viscere della terra.

Cio' che usciva dal cielo non erano angeli, come cio' che usciva dalla terra non erano demoni...

passano secoli..civilta' decade tranne per piccole citta' mantenute da corporazioni


Setting v2: dark e dangerous.. ma come?


Razze: umani, nani, elfi, mezz'orchi, mezz'elfi, ... trovare qualche razza interessante e particolare. no scurovisione

Caratteristiche: Mente, Corpo, Volonta'. tiri 2d10 per il valore, qualsiasi valore oltre 12 e' 12, qualsiasi valore sotto 4 e' 4.
Le caratteristiche non danno bonus ai tiri, ma sono prerequisiti per avanzamenti di rami, competenza magica ed armi (non puoi avere comp. armi/magica superiore al valore della statistica), scelta ed uso incantesimi. A seconda del ramo prendi dei bonus o malus (solo rami molto avanzati, sempre a fare il mago aumenti mente)


Kismet: ogni giocatore tira su una tabella casuale che indica come morira' il personaggio. dove, come, quando

statistica fortuna: si usa per abbassare il tiro fatto. recuperi 1 quando fai un fallimento critico. dichiari prima e spendi prima.
Ha un valore di 1d4 per sessione.

classi/rami: non serve perdersi, bastano 12 rami base, 8 intermedie, 6 avanzate, 4 elite
in una ramo si rimane finche' non si soddisfano i requisiti dei rami superiori
quindi le abilita' aumentano sempre, anche rimanendo nello stesso ramo
i punteggi aumentano con gli errori: se tiri 19/20 (sui 2d10) oltre al fallimento o fallimento critico segni con un pallino la competenza, quando c'e' downtime fai un check per ogni pallino e se fai 2 (ovvero 1/1) aumenti di 1 punto. massimo si segnano 5 pallini, poi si cancellano fatte le prove.
i rami presi possono aggiungere "competenze professionali". aumentano pf, competenze armi, competenze magica Per passare da un ramo ad un altro devi avere un valore minimo in certi punteggi

rami: 
per passare da un ramo all'altro perdi 1 punto in ogni competenza ed investi 4 mesi di tempo
quando prendi un ramo la prima volta ha 10 punti da distribuire tra tutte le competenze della classe
scegli 1 tra le competenze, che non sia quella relativa ad armi o incantamenti e su questa hai vantaggio nei tiri, su questa competenza non metti punti, parte a 4


preparare una piramide ? su excell...

agitatore
mendicante
barcaiolo
guardia del corpo
cacciatore di taglie
intrattenitore
baro
bullo
cacciatore
apprendista prete
mercenario
milizia
minatore
nobile
fuorilegge
commerciante
contafrottole
bracconiere
guardia a cavallo
soldato
ladro
razziatore
apprendista mago


assassino
sgherro
capitano della guardia
ciarlatano
esploratore
cavaliere errante
bandito
capitano mercenario
mercante
menestrello
capo bandito
prete
scolaro
spia
soldato veterano
mago

ombra
condottiero
arcimago
vecchio lupo
arciprete
bardo
eroe






agitatore
armi piccole, intimidire, schivare, ingannare, conoscenza dei bassifondi

mendicante
armi piccole, valutare, ingannare, osservare, conoscenza dei bassifondi

barcaiolo
armi piccole, orientamento, riparare,  mercanteggiare, nuotare

guardia del corpo
armi medie, medicina, armi da tiro, intimidire, armi a due mani

cacciatore di taglie
lame piccole, valutare, armi da tiro, osservare, conoscenza dei bassifondi

Intrattenitore
armi piccole, diplomazia, storia, rissa, camuffarsi

Baro
armi piccole, osservare, ingannare, intimidire, mani di fata

bullo
rissa, intimidire, cavalcare, intimidire,  osservare

Cacciatore
armi da tiro, furtivita', nuotare,  osservare, sopravvivenza

Apprendista prete
armi piccole, cavalcare, incantamento, medicina, intimidire

MERCENArio
armi da tiro, armi medie, armi a due mani, armi piccole, conoscenza dei bassifondi

milizia
armi da tiro, armi grandi, armi medie, intimidire, cavalcare

minatore
armi medie, sopravvivenza, atletica , valutare, orientamento

nobile
linguaggi, medicina, intimidire,  diplomazia, storia

fuorilegge
armi piccole, armi medie, medicina, mani di fata, intimidire

commerciante
mercanteggiare, conoscenza dei bassifondi, valutare, armi piccole, cavalcare

contafrottole
armi piccole, storia, valutare, ingannare, conoscenza dei bassifondi

bracconiere
armi medie, armi da tiro, medicina, furtivita', sopravvivenza

guardia a cavallo
armi da tiro, armi medie, orientamento, armi a due mani, cavalcare

soldato
armi da tiro, armi piccole, armi medie, armi a due mani, intimidire

ladro
armi piccole, ingannare, conoscenza dei bassifondi, mani di fata, osservare

razziatore
armi medie, atletica, intimidire, valutare,  mercanteggiare

apprendista mago
armi piccole, intimidire,  storia, linguaggi, incantamento


rami avanzati almeno 3 competenze a 10
ASSASSINo, corpo 12, conoscenza bassifondi 12, armi piccole 12
armi piccole , armi da tiro , conoscenza dei bassifondi  camuffarsi , armi medie , furtivita' 

SGHERRO, corpo 12, intimidire 12, armi piccole 12
armi piccole , armi da tiro, rissa , intimidire , armi a due mani , conoscenza dei bassifondi ,

capitano della guardia, volonta' 12, armi medie 12, armi a due mani 12
armi da tiro , armi medie ,cavalcare , intimidire , armi a due mani, conoscenza dei bassifondi 

ciarlatano, mente 12, ingannare 12, valutare 12
valutare , cavalcare , osservare , ingannare , intimidire , conoscenza dei bassifondi 

esploratore, corpo 12, osservare 12, orientamento 12
lame piccole, atletica ,  nuotare , armi da tiro , orientamento , osservare 

cavaliere errante, volonta' 12, cavalcare 12, armi medie 12
armi da tiro, armi a due mani , diplomazia , medicina , cavalcare , armi medie 

bandito
armi medie , schivare , intimidire ,armi da tiro  , camuffarsi , cavalcare 

capitano mercenario
armi piccole , medicina , sopravvivenza , intimidire , armi da tiro , armi a due mani 

mercante
armi piccole , ingannare , mercanteggiare , valutare , osservare , cavalcare 

menestrello
armi piccole ,storia , ingannare , riparare , camuffarsi ,  mani di fata 

capo bandito
armi piccole , cavalcare , nuotare , armi da tiro , intimidire , armi medie 

prete
armi piccole , diplomazia , cavalcare ,  storia , incantamento , linguaggi 

scolaro
rissa, valutare  , riparare , storia , linguaggi , medicina 

spia
armi piccole , atletica , schivare ,  camuffarsi , ingannare , mani di fata 

soldato veterano
rissa , armi piccole  ,sopravvivenza , indimidire , armi da tiro , armi medie 

mago
armi piccole , intimidire , medicina , diplomazia , storia , incantamento 


incantesimi
Allarme
Antimagia
Armatura
Bandire
Protezione
Soffio
Nascondere (da blur ad invisibile)
Bruciare
Trovare
Dominare
Incantare
Impaurire
Piovere
Leggero
Aggiustare
Galleggiare
Comandare
Armeggiare
Guarire
Illusione
Fulmine
lucchetto
Moltiplicare
Notte
Sentiero
Leggere
Vedere
Servo
Scudo
Silenzio
Sonno
Parlare
Apri
Muro



Tratti
Attivo
Avventuroso
Affettuoso
Paura
Ambizioso
Ansioso
Argomentativo
Attento
Sconcertato
Autoritario
Coraggioso
Brillante
Bullo
Calma
Capace
Attento
Premuroso
Carismatico
Affascinante
Infantile
Intelligente
Goffo
Cuore freddo
Compassionevole
Competitivo
Presuntuoso
Interessato
Fiducioso
Coscienzioso
Premuroso
Cooperativa
Coraggioso
Vigliaccamente
Critico
Crudele
Curioso
Audace
Affidabile
Determinato
Disonesto
Irrispettoso
Sognatore
Desideroso
Di buon carattere
Efficiente
Energico
Entusiasta
Giusto
Fedele
Irrequieto
Feroce
Folle
Amichevole
Divertente
Generoso
Gentile
Cupola
Avido
Brontolone
Laborioso
Contento
Duro
Odiato
Speranzoso
Senza speranza
Umoristico
Ignorante
Fantasioso
Immaturo
Impaziente
Scortese
Impulsivo
Indipendente
Insistente
Intelligente
Geloso
Gioviale
Pigro
Capo
Logico
Solitario
Amabile
Amorevole
Leale
Fortunato
Maturo
Lunatico
Misterioso
Nervoso
Rumoroso
Obbediente
Antipatico
Attento
Ottimista
Tranquillo
Persistente
Pessimista
Esigente
Piacevole
Educato
Orgoglioso
Perplesso
Presto
Tranquillo
Affidabile
Rispettoso
Responsabile
Irrequieto
Chiassoso
Maleducato
Sarcastico
Segreto
Sicuro di sé
Egoista
Autosufficiente
Sensibile
Timido
Sciocco
Sincero
Abile
Furbo
Accorto
Subdolo
Snob
Socievole
Avaro
Rigoroso
Testardo
Studioso
Dolce
Talentuoso
Loquace
Premuroso
Senza pensieri
Timido
Fidarsi
Affidabile
Scortese
Utile
Versatile
Di buon cuore
Saggio
Spiritoso
Preoccupato


il ramo iniziale ti da dei punteggi di competenza base
per fare una prova tiri il 2d10 e devi fare meno della statistica o competenza. se e' collegata al ramo con competenza preferita tiri con vantaggio

punti ferita: a seconda della razza ed ogni ramo che prendi lo aumenta di un valore dipendente dal ramo scelto, ma l'aumento e' sempre poco (1-4)
pf base: corpo + ramo/rami. quindi da 4 a 12/13 a salire. 
PF:  reggi un certo numero di colpi. un colpo critico (ovvero difesa fallita di almeno 6) causa +2 ferite, un arma 2 mani causa 3 ferite, una arma media causa 2 ferita, un arma piccola causa 1 ferita.
parti con un numero di Ferite pari a Corpo/4. i rami possono aumentare le ferite sostenibili. 
Rami combattenti aggiungono 3, magici 1.. molto magici se avanzati 0, rami ibridi 2/1
recupero PF: ogni notte recuperi 2 (o corpo/8 ???)

azioni in round: in un round si possono usare fino a 10 punti azione. attaccare con arma a 2 mani costa 8 punti azione, arma a 2 mani costa 6 punti azione, attaccare con un arma leggera costa 4 punti azione. gli incantesimi solitamente costano da 4 punti azione in su con l'incantesimo tipico che costa 6 p.a. e quello veloce 4. valutare azioni incantesimi in base a componenti: 1 comp +4, 2 comp +6, 3 comp +8, spostarsi costa in base alla distanza coperta
i mostri in base alla loro taglia e cosa fanno
per altre azioni vedi obss e traduci in p.a.

distanze: mischia, vicino, medio, lontano, molto lontano. spostarsi da mischia a vicino costa 2 pa, vicino a medio 3 pa, medio lontano 4 pa, lontano molto lontano 10 pa.
armi piccole/medie: mischia
armi a due mani: vicino
armi da tiro: medio, lontano, molto lontano

iniziativa: e' pari ai punti azione usati. chi ne usa di meno incomincia per primo. in caso di parita' si controlla mente, volonta', corpo in quest'ordine. se ritarti "consumi" punti azione ad aspettare

competenze di base: ogni ramo modifica/o meno il punteggio di Combattimento o Magia.

combattimento: chi attacca non tira nulla, dichiara solo che arma usa e se usa manovre. chi difende tira sotto il suo valore di combattimento, questo valore di combattimento puo' avere svantaggi dati da abilita' dell'attaccante (rami combattenti avanzati). se tira sotto allora para/evita il colpo, se tira sopra viene preso.
Si presume che l'attacco colpisce sempre se la difesa non funziona bene. Questo significa che le classi combattenti non aumentano l'attacco ma solo la difesa. alcuni rami avanzati di combattimento danno delle penalita' alla difesa avversaria.
il danno e' 1d6 per armi piccole, 1d8 armi medie, 1d10 armi grandi. 
combattere con un arma che non conosci da svantaggio


armi: armi piccole 1d6, armi medie 1d8, armi grandi 2 mani 1d10. non sommi valore forza. arma piccola ha requisito forza 6, media 9, grande 12, altrimenti svantaggio nell'uso

una difesa particolarmente riuscita (almeno -6) puo' dare svantaggio al tiro successivo, un -9 potrebbe fare tirare con un solo dado
quando difendi  a seconda di quanto bene difendi, ovvero se hai un margine di -3,-6,-9.. rispetto alla tua prova ottieni dei bonus, azioni movimento, penalita' all'attacco successivo

le manovre funzionano con lo stesso principio. quando effettuo una manovra e l'altro riesce nel tiro di combattimento allora la manovra non riesce e io devo fare un tiro di combattimento, se fallisco prendo gli effetti della manovra

armature: riducono il danno. leggere riducono di 2, medie riducono di 3, pesanti riducono di 4. danno una penalita' alle prove di competenza. Non si fa magia con armatura se non solo a contatto. requisito di corpo: 8, 12, 
scudo: fai la prova di difesa con +1 (da 1 bonus). penalita' alle prove di competenza (da 1 penalita')

modificatori alle prove: un bonus e' +1, due bonus tiri 1d4, 3 bonus tiri 1d6 (non sono provisti oltre 3 bonus cumulativi). le penalita' funzionano in maniera simmetrica. bonus e malus si annullano in maniera reciproca. es. colpire da invisibili da 3 bonus

vantaggio: tiri tre dadi e prendi 1 dado piu' basso piu' un altro a tua scelta   |  svantaggio: tiri tre dadi e prendi 1 dado con il punteggio piu' alto piu' un altro a tua scelta | attenzione al valore 0 zero.

magia: deve essere lanciare piu' difficile lanciare incantesimi e consumare "risorse" (pf/energia/stamina..)

certi rami magici possono privilegiare certe scuole di magia. lavorare su liste e ridurre e tanto. l'idea di base e' per esempio crea fuoco puo' diventare a seconda del punteggio del tiro altre cose, il valore influenza la distanza, AoE, danno, se e' un raggio o esplosione e che raggio...  Pochi incantesimi ma che si evolvono

lanciare un incantesimo: tirare a secondo dall'incantesimo su capacita' magica e avere un punteggio minimo di  mente, corpo o volonta'.  Ogni punteggio -3 rispetto alla prova, es. capacita' magica 13 e tiro 8, potenzi un fattore dell'incantesimo (distanza, AoE, danno, tipo di effetto..). Gli incantesimi hanno un punteggio minimo di mente/corpo/volonta' per essere tirati ed anche di capacita' magica
Ogni incantesimo ha degli attributi, danno, distanza, aoe, durata ed un punteggio minimo di competenza magica e corpo/mente/volonta'. se il tiro riesce bene puoi potenziare un attributo presente ma non darlo/aggiungerlo se questo e' assente. se lancio l'incantesimo crea fiamma, inc. base difficolta' 1, se faccio un ottimo tiro potro' potenziare la durata e aoe (l'area di luce che fa) ed il danno, ma non posso aggiungere distanza perche' e' un attributo assente.
ci sara' poi l'incatesimo globo di fuoco, la versione base della palla di fuoco, questa ha piu' attributi ma ad esempio non ha durata, o meglio l'istantanea non puo' essere migliorata.
valutare di aggiungere attributi assenti quando il tiro e' veramente buono, ovvero tiri veramente basso, direi almeno un -6 per aggiungere un attributo a livello base (3 metri)
si possono spendere risorse aggiuntive per potenziare l'incantesimo, ovvero abbassare il risultato del dado
scuola di magia: raggruppa gli incantesimi per tipo
ci si puo' specializzare in una scuola di magia: hai +3 alla prova, ma -3 a tutte le altre scuole

se l'incantesimo riesce nel lancio non c'e' TS. alcuni incantesimi possono avere una prova di difesa per essere evitati/dimezzati l'effetto

questo implica che non ci saranno mai incantesimi super potenti in tutto..

quanti incantesimi lanciare:  a piacere, ma ogni volta che lanci lo stesso il punteggio minimo richiesta di capacita' magica dell'incantesimo aumenta di 1. quindi tiro l'incantesimo cura, richiede corpo 1 e capacita' magica 11 - la prova cmq la faccio sul mio punteggio di capacita' magica e devo fare piu' basso, lo ritiro e richiede corpo 2 e capacita' magica 12, arrivo ad un certo punto che non posso piu' soddisfare i requisiti richiesti anche se le prove le ho sempre superate

quando lanci un incantesimo e fallisci la prova non succede nulla, ma l'incantesimo l'hai lanciato e quindi i requisiti minimi aumentano di 1
quando lanci un incantesimo e fallisci con 19-20 la prova succedono cose  brutte
quando lanci l'incantesimo e fai veramente basso 2, non aumentano i requisiti per il lancio successivo.


mostri:
se il mostro tira la sua difesa c'e' il rischio che riesca sempre ad alti livelli. considerare abilita' che abbassano la difesa, l'idea di base e' che se un mostro ha difesa  o piu' deve essere di un livello tale da dover affrontare pg con rami che danno penalita' alla difesa


--------------------------------------------------------------------------------



- fare 2 o 20  successo critico o fallimento  critico

- il giocatore dichiara cio' che fa e' solo il master a stabilire se serve una prova. 

- il tempo e' un fattore, tabelle random per incontri basati su tempo trascorso, si computa il tempo reale.



\end{document}