\documentclass[12pt,a4paper,twoside,openany]{book}
\usepackage{quoting}
\usepackage{tcolorbox}
\usepackage{tikz}
\usetikzlibrary{shadows}
\usepackage{multicol}
\usepackage{tocloft}
\usepackage{lmodern}
\usepackage{caption}
\usepackage[utf8]{inputenc}
\usepackage[T1]{fontenc}
\usepackage{setspace}
\usepackage[a4paper]{geometry}
\geometry{verbose,tmargin=2cm,bmargin=2cm,lmargin=2cm,rmargin=2cm}  %std
\setcounter{secnumdepth}{-1}
\usepackage{booktabs}
\usepackage{url}
\usepackage[italian]{babel}
\usepackage{setspace}
\usepackage{graphicx}
\usepackage{amssymb}
\usepackage{makeidx}
%\usepackage[allfiguresdraft]{draftfigure}
%\usepackage{slashbox}
\usepackage{multirow}
\usepackage{titlesec}
\usepackage[unicode=true, bookmarks=true,
pdftitle={Dark Catacomb - DKC},pdfauthor={Andres Zanzani},
breaklinks=false,pdfborder={0 0 1},backref=section,colorlinks=false]
{hyperref}
\hypersetup{colorlinks=true,linkcolor=blue,pdfcreator={LaTeX}}
\usepackage{bookmark}
\usepackage{yfonts}
\usepackage{auncial}
\usepackage{ragged2e}
\usepackage{ulem}


\usepackage{fontspec}
\setmainfont[Path=./, BoldItalicFont=Soutane Bold Italic.ttf, ItalicFont=Soutane Italic.ttf, BoldFont=Soutane Bold.ttf, Ligatures=TeX, Scale=0.94]{Soutane Regular.ttf} 


\usepackage{wrapfig}
\usepackage{fancyhdr}
\usepackage{tcolorbox}
\tcbuselibrary{skins}
\tcbset{colback=brown!10, fonttitle=\scshape}
\usepackage{imakeidx}
\usepackage{cancel}

\def\CountIndexOccurrences#1{%
	\expandafter\newcount\csname #1\endcsname%
	\expandafter\newcount\csname #1\endcsname%
	\def\indexentry##1##2{\expandafter\advance\csname #1\endcsname 1}%
	\IfFileExists{#1.idx}{\input{#1.idx}}{}%
}
\CountIndexOccurrences{OBSS}
\CountIndexOccurrences{Incantesimi}
\CountIndexOccurrences{Mostruario}
\CountIndexOccurrences{OggettiMagici}
\def\TotalBox#1{\vfill%
	\fbox{Ci sono \expandafter\the\csname #1\endcsname\ voci in questo indice}\par}
\makeindex[columns=3, title=Indice Analitico, intoc=true]
\makeindex[columns=3, name=Incantesimi, title=Lista degli Incantesimi, intoc=true]
\makeindex[columns=3, name=Mostruario, title=Lista dei Mostri, intoc=true]
\makeindex[columns=3, name=OggettiMagici, title=Lista degli Oggetti Magici, intoc=true]
\usetikzlibrary{shapes.misc,calc}
\definecolor{lightgray}{gray}{0.95}
\usetikzlibrary{shapes.misc,calc}
\definecolor{lightgray}{gray}{0.95}
\usepackage{fancyhdr}
\pagestyle{fancy}
\fancyhf{} 
\fancyhead[LE,RO]{\leftmark}
\fancyhead[RE,LO]{}
\fancyfoot[C]{\thepage}
\renewcommand{\sectionmark}[1]{\markboth{#1}{}}
\usepackage{xltabular}
\usepackage{tabularx}
\usepackage{pdfpages}
\usepackage{hyperref}
\usepackage{tikz}
\usepackage[absolute,overlay]{textpos}
\usepackage{etoolbox}
\usepackage{soul}
\raggedbottom
\usepackage{array}
\newcolumntype{L}[1]{>{\raggedright\let\newline\\\arraybackslash\hspace{0pt}}m{#1}}
\newcolumntype{k}[1]{>{\centering\let\newline\\\arraybackslash\hspace{0pt}}m{#1}}
\newcolumntype{R}[1]{>{\raggedleft\let\newline\\\arraybackslash\hspace{0pt}}m{#1}}
\newcolumntype{D}[1]{>{\centering}m{#1}}
\newcolumntype{M}[1]{>{\centering\arraybackslash}m{#1}}
\titleformat{\section}{\filcenter\huge\bfseries\accanthis}{\thesection}{1em}\textsc{}
\titleformat{\subsection}{\Large\bfseries\accanthis}{\thesubsection}{1em}\textsc{}
\titleformat{\subsubsection}{\normalsize\bfseries\accanthis}{\thesubsubsection}{1em}\textsc{}
\def\changemargin#1#2{\list{}{\rightmargin#2\leftmargin#1}\item[]}
\let\endchangemargin=\endlist
\setcounter{tocdepth}{3}
\newtcolorbox{narratore}{
	enhanced, % enable advanced settings
	%left = 3mm,
	%width=0.45\textwidth,
	left = 9mm, % pushes text away from the left edge by 10mm
	sharp corners, % disables rounded corners
	rounded corners = southeast, % "round" the bottom right corner
	arc is angular, % make the "round" corner an angle
	arc = 3mm, % controls corner cut
	boxrule=0.6pt, % sets box line thickness
	underlay={%
		\path[fill=black] ([yshift=3mm]interior.south east)--++(-0.4,-0.1)--++(0.1,-0.2); % triangle
		\path[draw=black,shorten <=-0.05mm,shorten >=-0.05mm] ([yshift=3mm]interior.south east)--++(-0.4,-0.1)--++(0.1,-0.2); % triangle edge
		\path[fill=gray!50!black,draw=none] (interior.south west) rectangle node[brown!10]{\Huge\bfseries ?!} ([xshift=8mm]interior.north west);
	},
	drop fuzzy shadow }

\newtcolorbox{enfasi}{
	enhanced,
	arc=5pt,
	boxrule=0.3pt
} 

\usepackage{zref-savepos,graphicx}
\newcommand{\filltopageendgraphics}[2][]{%\filltopageendgraphics[width=.5\linewidth]{image-a}
	\par
	\zsaveposy{top-\thepage}% Mark (baseline of) top of image
	\vfill
	\zsaveposy{bottom-\thepage}% Mark (baseline of) bottom of image
	\smash{\includegraphics[keepaspectratio=true,height=\dimexpr\zposy{top-\thepage}sp-\zposy{bottom-\thepage}sp\relax,#1]{#2}}%
	\par
}


\usepackage{accanthis}
\usepackage[framemethod=TikZ]{mdframed}


\begin{document}
	
\def \versione {0.01}
\thispagestyle{empty}
 
{\Huge \begin{center}
		Dark Catacomb
\end{center}}

\vfill
\begin{center}
	\Large{\color{black} Fantasy Adventure Game}
\end{center}

\pagebreak

	
\bigskip
Non temere l'ignoto, affrontalo con rispetto.
	
	\vspace{\fill}
\begin{center}\textbf{\versione} - \today\end{center}
\thispagestyle{empty}


\newpage~\thispagestyle{empty}%%\newpage~\thispagestyle{empty}


\newcommand{\riga}{\rule{\textwidth}{0.4pt}}


{\Huge \begin{center} Dark Catacomb\end{center}}

\bigskip

\begin{center}{\LARGE Manuale per Giocatore e Narratore}\\ \end{center}

{\large \begin{center} Guida e Regole per il Gioco di Ruolo Fantasy \end{center}}

\begin{center}di \end{center}

{\LARGE \begin{center} Andres Zanzani \end{center}}

\vspace{2cm}


\vfill

\begin{mdframed}[roundcorner=10pt]

\medskip

\textbf{Playtesting}: ...to be done...

\bigskip

\begin{flushleft}\textbf{Condizioni d'uso}: Dark Catacomb, DKC, è un marchio registrato di Andres Zanzani (azanzani@gmail.com).
\end{flushleft}

\vspace{0.5cm}


\medskip

\end{mdframed}

%}%
%}

\pagebreak

\setcounter{page}{1}

\begin{multicols}{2}
\tableofcontents{}

\end{multicols}

\vfill

\begin{changemargin}{0.3cm}{0.3cm}\begin{tcolorbox}
"Nel mezzo del cammin di nostra vita\\
mi ritrovai per una selva oscura\\
ché la diritta via era smarrita.\\
Ahi quanto a dir qual era è cosa dura\\
esta selva selvaggia e aspra e forte\\
che nel pensier rinova la paura!\\
\end{tcolorbox}\end{changemargin}

\pagebreak

\section{Introduzione}
Razza\\
\sout{Caratteristiche}\\
\sout{rami base}\\
\sout{rami avanzati}\\
\sout{Punti Fato}\\
Karma\\
\sout{Competenze}\\
Costruiamo il personaggio\\
Regole per le competenze\\
Combattimento\\
Nascondigli e coperture\\
liste armi ed armature\\
abilita' ???\\
magia\\
incantesimi\\
equipaggiamento\\
veleni e droghe\\
movimento\\
oggetti magici\\
mostruario\\
Condizioni\\
Scheda\\

\pagebreak

\section{Le Stirpi di Dark Catacomb}\index{Le Stirpi di Dark Catacomb}

\begin{multicols}{2}
	
\end{multicols}



\pagebreak

\section{Caratteristiche}\index{Caratteristiche}

\begin{multicols}{2}

\subsection{Le Caratteristiche del Personaggio}\index{Le Caratteristiche del Personaggio}

Le Caratteristiche\index{caratteristiche} di un personaggio servono a comprendere quando possa essere forte e robusto, ma anche atletico se non intelligente e di buon senso. Rappresentano le potenzialità su cui le Competenze costruiscono l'esperienza.

Queste Caratteristiche sono \textbf{Corpo} \index{Corpo}, \textbf{Mente} \index{Mente} e \textbf{Volontà} \index{Volontà} e \textbf{Vitalità}\index{Vitalità}

\textbf{Corpo} rappresenta tutte le caratteristiche fisiche, quindi forza, resistenza, capacità atletiche. Corpo influenzerà tutte le prove basate sul fisico del personaggio.

\textbf{Mente} rappresenta la capacità di ragionamento, la memoria, la rapidità di pensiero e l'arguzia. Mente influenza tutte le prove in cui il personaggio deve ragionare, ricordare.

\textbf{Volontà} rappresenta il buon senso ma anche il saper resistere a shock emotivi. Volontà viene usato quando si gestiscono animali e si deve fare un lavoro che richieda impegno e dedizione.

\textbf{Vitalità} rappresenta l'energia vitale del personaggio e la sua capacità di resistere a colpi od incantesimi.

\subsection{Come stabilire le Caratteristiche del personaggio}\index{Come stabilire le Caratteristiche del personaggio}

Il giocatore tira 2d10 e somma il risultato, questo tiro e computo lo esegue per ogni Caratteristica tranne Vitalità.

Se il valore sommato di una Caratteristica è inferiore a 6 segnerai 6 nella Caratteristica. Se il valore sommato di una Caratteristica è superiore a 12 segnerai comunque 12 nella Caratteristica.

Il punteggio di \textbf{Vitalità} è pari al punteggio di Corpo aumentato dai punti indicati dal Ramo scelto.\index{Vitalità iniziale}

\subsubsection{I modificatori alla prove}\index{Modificatori alla prove}

Ogni Prova di Competenza viene modificata dal punteggio della Caratteristica connessa. 
Il modificatore alla prove di Competenza è pari al punteggio di Caratteristica -10. Nelle Competenze è indicata la Caratteristica che la modifica. Questo modificatore viene applicato sia che abbia un valore positivo o negativo al valore della Competenza.

\end{multicols}

\section{Punti Chaos}\index{Punti Chaos}\index{Fortuna del Principiante}

\begin{multicols}{2}
	

\begin{changemargin}{0.3cm}{0.3cm}\begin{enfasi}{Se il destino è contro di noi, peggio per lui. (motto del 1º Reggimento Carabinieri Paracadutisti "Tuscania")}\end{enfasi}\end{changemargin}

In un mondo non facile ne amichevole il Chaos domina il destino. Ogni personaggio può ricorrere ai punti Chaos per influenzare la sua prova o anche quella di un compagno o di un avversario!

Ogni personaggio ha tre segnalini ed e' libero di consumarne fino a tre alla volta. Ognuno di questo conta come un bonus o penalità, quindi prenderne 1 conta come un +1 (o -1), due conta come un +1d4 (o -1d4), prenderne tre da Vantaggio (o Svantaggio) alla prova. 

Quanti Punti Chaos si vogliano usare si dichiara prima del tiro, una volta dichiarato l'ammontare di Punti Chaos non é possibile utilizzarne di più o di meno.

Ogni volta che al personaggio nel tiro dei dadi esca un doppio 0 recupera un punto Chaos.

I punti Chaos vengono azzerati e reimpostati a 3 ad ogni sessione di gioco.

\end{multicols}

\pagebreak

\section{Le Competenze e le Prove}\index{Competenze}\index{Prove}

\begin{multicols}{2}

Ogni personaggio può seguire uno o più Rami ovvero un insieme di competenze e capacità professionali.

Queste Competenze quando apprese faranno parte del bagaglio culturale, conoscitivo e pratico del Personaggio. Il Personaggio usando le Competenze ne migliorerà l'uso.

Il Personaggio in base a quello che viene dichiarato effettuerà una Prova per capire se riesce e come nell'intento. 

\end{multicols}

\subsection{Le Competenze}

\begin{tabular*}{0.93\linewidth}{@{\extracolsep{\fill}}lll}
\textbf{Corpo} & \textbf{Mente} & \textbf{Volontà}\\
\toprule
Atletica				& Arcana					& Artigianato			\\	
Arrampicarsi			& Conoscenza *				& Cavalcare				\\
Artista della fuga		& Erboristeria				& Diplomazia			\\
Armi piccole 			& Disattivare congegni		& Osservare	\\
Armi medie 				& Falsificare				& Mani di fata\\
Armi a due mani			& Incantamento				& Gestire animali\\
Intimidire		 		& Raggirare					& Furtività\\
Nuotare					& Intrattenere				& Orientamento\\
Rissa					& Mercanteggiare			& Percepire Emozioni \\ 
Usare corda		 		& Natura					& Seguire tracce\\
						& Pronto soccorso			& Sopravvivenza\\
						& Tradizioni locali			& \\

\end{tabular*}\\

La \textbf{Conoscenza} va esplicitata su quale argomento verte: Dungeon, Legge, Lingue, Piani, Occulto, Architettura ed Ingegneria, Nobiltà ed Araldica, Miti e Leggende, Religione, Storia, Geografia ...\\

Alcune Competenze hanno una importanza peculiare nel sistema: \textbf{Incantamento} e le varie \textbf{Armi}, la prima permette di lanciare incantesimi e ne aiuta a determinare l'efficacia, la seconda indica la Competenza del personaggio con le varie tipologie di armi e quanto è capace di usarle.

\begin{multicols}{2}
	
\subsection{Le Prove}\index{Effettuare una Prova}

Ogni Competenza ha un valore numerico che ne stabilisce il grado di capacità nell'uso, più e' alto maggiore sarà la facilità con cui supero le prove.

Il valore di Caratteristica-10 è usato come modificatore alla Prova assumendo quindi valori sia positivi che negativi, anche se negativo viene comunque chiamato bonus nel manuale.

Per verificare l'esito di una Prova di Competenza è necessario sommare il valore di Competenza con il bonus di Caratteristica e sottrarre il risultato della somma di 2d10.

Per \textbf{Punteggio di Competenza}\index{Punteggio di Competenza} (PC) (e non valore di Competenza) si intende il valore già sommato del bonus di caratteristica e del valore di Competenza.\index{Punteggio di Competenza}\index{PC}

Si definisce \textbf{Margine di Successo}\index{Margine di Successo} (MS) il valore di differenza tra il Punteggio di Competenza (PC) con quanto tirato con i dadi.\index{Margine di Successo}\index{MS}

Il Margine di Successo (MS) può assumere valori negativi o positivi. Mentre in una prova di Competenza o Caratteristica il successo della stessa è solo nel MS positivo, con una prova contrapposta non è detto che un valore negativo sia un insuccesso, dipende da quanto fatto dal contendente

La Prova d'Armi viene effettuata sia per Difendersi che per Attaccare. Nel manuale troverete la Prova d'Armi per difendersi come \textbf{PDAD} e quella per Attaccare come \textbf{PDAA}.
 
Le \textbf{Prova d'Armi}, sia PDDA che PDAD \index{Prova d'Armi}, come per la altre Prove si effettuano sommando del valore di Competenza con il bonus di Caratteristica (solitamente Corpo) e al risultato si sottrae quello del tiro di 2d10.
Le singole PDAD e PDAA hanno il loro Margine di Successo.

\subsubsection{I modificatori alla Prova}\index{Modificatori alla Prova}

L'Arbitro può decidere la presenza di modificatori alla Prova in base alla situazione in cui si svolge la Prova.

Qualora ci sia una \textbf{penalità} (ho fretta, è buio, corro, l'avversario è a cavallo ed io sono appiedato..) la difficoltà della Prova aumenta, ovvero devo \textbf{aumentare il valore della Prova} effettuata della penalità presente.

Se invece ho un \textbf{bonus} allora la difficoltà della Prova diminuisce, ovvero devo \textbf{diminuire il valore della Prova} effettuata del bonus presente.

Un Bonus sarà un valore positivo che sommo ai 2d10 tirati nella Prova.\index{Applicare il Bonus}. Una Penalità è un valore negativo che sottraggo al Punteggio dei Competenza.

I \textbf{modificatori alla Prova si cumulano} tra di loro se omogenei, tutti positivi o tutti negativi, e si annullano o scalano a vicenda se di tipo opposto (bonus e penalità).

Es. Devo scalare una parete. Ho un Bonus perché sono presenti degli appigli, ho una penalità perché sta piovendo, ho due bonus perché posso aiutarmi con una corda, ho una penalità perché è buio. La differenza totale tra bonus e penalità é di 1 Bonus.

I \textbf{modificatori}, bonus o penalità, alla Prove assumono il valore di \textbf{1}, qualora ci sia un solo bonus o penalità, \textbf{1d4} qualora ci siano due modificatori omogenei attivi. Nel caso i modificatori siano tre o più, ovviamente di tipo omogeneo, si ha il cosiddetto \textbf{\textit{Vantaggio}} oppure \textbf{\textit{Svantaggio}}.

In caso di \textbf{Vantaggio} tiro 3d10 per effettuare la Prova e scarto quello con il valore più alto, poi sommo gli altri due dadi per verificare l'esito della Prova.\index{Vantaggio}

In caso di \textbf{Svantaggio}\index{Svantaggio} tiro 3d10 e scarto quello con valore più basso, poi sommo gli altri due dadi per verificare l'esito della Prova. 

\subsection{Migliorare le Competenze}\index{Migliorare le Competenze}\hypertarget{Migliorare le Competenze}{} \label{Migliorare le Competenze}

Ogni qual volta vi effettua una Prova su una Competenza e questa ha come risultato dei dadi \textbf{19 o 20} si mette un segno vicino alla Competenza. Si possono avere fino a cinque segni vicino ad una singola Competenza.

Quando il personaggio ha tempo di riflettere su quanto accaduto, sulle prove che ha fatto e come queste sono riuscite o fallite, può tirare 2d10 e se la somma dei dadi è superiore al suo \textbf{punteggio di Competenza} allora quel punteggio aumenta di 1.

Una volta fatta questa particolare prova si cancellano tre segni dalla Competenza.	

\subsection{Competenze ed i loro ambiti di utilizzo}\label{competenzeambitidiutilizzo}

Sono descritte sommariamente le Competenze ed i loro ambiti di utilizzo. Sono indicazioni di massima su cosa usare le competenze. Viene anche indicato il numero di Punti Azioni (PA) necessarie per svolgere la prova tipica, ovvio che usi più complessi richiedono più tempo e Punti Azioni (PA).

I PA necessari alla prova possono variare a seconda della capacità del personaggio e della complessità della prova.

In ogni caso ricordate sempre di valutare con attenzione come il giocatore dichiara di svolgere le azioni per capirne la durata ed effetti. 

La Competenze con un \textbf{*} subiscono le penalità dovute all'armatura indossata.\\

\textbf{Atletica* (Corpo)}: Questa competenza serve per mantenere l'equilibrio su superfici strette o precarie, per tuffarsi, rotolare, fare capriole, salti mortali, superare degli ostacoli nonché cadere e non farsi male. 

\textbf{Arcana (Mente)}: Con questa competenza si è esperti di magia e di incantesimi, di oggetti magici è si è grado di identificare gli incantesimi che vengono lanciati. 

\textbf{Arrampicarsi* (Corpo)}: Con questa competenza si possono scalare superfici verticali, dalle mura cittadine alle pareti rocciose. E' legata all'Azione di movimento. Con 8 punti il movimento è solo dimezzato.

\textbf{Artigianato (Mente)}: E' necessario specificare la tipologia di Artigianato in cui si è competente. Si è competente, ma non a livello di Professione, in una forma di artigianato.

\textbf{Artista della fuga (Corpo)}: Con questa competenza ci si può liberare da legacci e manette.

\textbf{Cavalcare (Volontà)}: Con questa competenza è possibile cavalcare in maniera professionale e dare comandi alla propria cavalcatura. 

\textbf{Osservare (Volontà)}: per cercare, accorgersi, notare. E' un qualcosa di attivo.

\textbf{Conoscenza dei Dungeon (Mente)}: Con questa competenza si hanno conoscenze di Aberrazioni, melme, caverne, esplorazioni sotterranee.

\textbf{Conoscenze di Geografia (Mente)}: Con questa competenza si hanno conoscenze sul clima, popolazione, terreni, territori, nazioni e confini.

\textbf{Conoscenza Lingue/Linguaggi (Mente)}: Con 1 punto sai parlare una lingua, con 3 punti la sai anche scrivere. Un buon punteggio di Lingue aiuta a comprendere lingue non note ed a farsi comprendere. Viene usata anche per comprendere testi complessi

\textbf{Conoscenze Occulte (Mente)}: Con questa competenza si è esperti di occulto, creature immondi. 

\textbf{Conoscenze Religione (Mente)}: Con questa competenza si hanno conoscenze su Patroni, mitologia, Celestiali, Non Morti, simboli sacri, tradizione ecclesiastica, feste e ricorrenze liturgiche. 

\textbf{Conoscenze di Storia (Mente)}: Con questa competenza si hanno conoscenze di Storia quali guerre, migrazioni, colonie, fondazioni di città, accadimenti importanti..

\textbf{Diplomazia (Volontà)}: Con questa competenza si possono risolvere diverbi, e raccogliere preziose informazioni e dicerie dalle persone. La competenza è anche usata per negoziare in modo efficace con la giusta etichetta e condotta adatta alla situazione controversa. 

\textbf{Disattivare congegni (Mente)}: Con questa competenza si possono disarmare Trappole e aprire serrature, sabotare congegni meccanici semplici, come le catapulte, le ruote di un carro o le porte.

\textbf{Erboristeria (Mente)}: Con questa competenza si hanno conoscenze di come riconoscere e preparare pozioni e veleni naturali. Il punteggio si applica alle prove per distillare pozioni.

\textbf{Falsificare (Mente)}: Con questa competenza si sa falsificare oggetti d'arte, mappe, firme... 1 Minuto

\textbf{Gestire animali (Volontà)}: Con questa competenza è possibile addestrare e ammansire animali.

\textbf{Intimidire (Corpo)}: Intimidire si basa sull'approccio fisico per convincere l'interessato. 

\textbf{Raggirare (Mente)}: La competenza Ingannare può essere usata per Raggirare (dicendo quindi fandonie) o Persuadere (adattando la verità) al fine di convincere delle proprie parole l'interessato.

\textbf{Intrattenere (Mente)}: Con questa competenza si è esperti in una espressione artistica, dal canto alla recitazione, dal ballo a suonare strumenti musicali. E' necessario specificare la forma di intrattenimento.

\textbf{Mani di fata* (Volontà)}: Con questa competenza si può borseggiare, estrarre un'arma nascosta, oppure compiere altre azioni senza essere notati. 

\textbf{Furtività (Volontà)}: Con questa competenza si è in grado di muoversi senza causare rumore oppure di passare inosservati stando fermi. 

\textbf{Natura (Mente)}: Con questa competenza si hanno conoscenze di Animali, Fatati, stagioni e cicli, tempo atmosferico, vegetali. 

\textbf{Nuotare* (Corpo)}: Con questa competenza si è in grado di nuotare, anche in acque tempestose. Senza competenza si sa stare a galla in acqua placide. Legata all'Azione di movimento.

\textbf{Orientamento (Volontà)}: Con questa competenza si ha il senso della direzione e orientamento rendendo impossibile perdersi indipendentemente dall'ambiente in cui ci si trova. 

\textbf{Percepire Emozioni (Volontà)}: Con questa competenza si può capire se qualcuno sta mentendo o si possono intuire le sue vere intenzioni.

\textbf{Pronto soccorso (Mente)}: Con questa competenza si possono curare le ferite e le malattie. Costo variabile.

\textbf{Seguire tracce (Volontà)}: Con questa competenza si sa seguire le tracce lasciate nell'ambiente. 

\textbf{Sopravvivenza (Volontà)}: Con questa competenza si può sopravvivere e orientarsi nelle terre selvagge. La competenza è usata anche per cercare attivamente trappole e fosse. .

\textbf{Tradizioni locali (Mente)}: Con questa competenza si hanno conoscenze degli abitanti (più noti), costumi, leggende, leggi, personalità, tradizioni. E' necessario specificare una regione geografica dove è applicabile la conoscenza. 

\textbf{Usare corda (Corpo)}: Con questa competenza si è esperti in legacci e nodi per fissare e bloccare oggetti o persone. 

\textbf{Mercanteggiare (Mente)}: Con questa competenza si sa stimare il valore monetario di un oggetto.

\subsubsection{Esempi Prove Competenza}\label{esempiprovecompetenza}\hypertarget{esempiprovecompetenze}{}\index{Esempi prove Competenza}

\textbf{Prove atipiche}\index{Prove atipiche}. Il giocatore è invitato a trovare usi, soluzioni, approcci che esulino dalle più ovvie prove. Siate creativi e descrivete al Narratore la meravigliosa azione che volete fare e quali risultati sperate di ottenere! Sarà lui a stabilire in base alla vostra descrizione dell'azione cosa provare e se hai dei bonus o penalità.

\medskip

Per \textbf{riconoscere un oggetto magico}\index{Riconoscere oggetto magico} e le sue capacità è necessaria una prova di \textbf{Arcana} per avere indicazioni di massima sui poteri e ambiti di utilizzo, con un MS di almeno 6 puoi apprenderne i dettagli, bonus magici e cariche. \textbf{10 minuti}. Con punteggio Arcana 6 costa 5 minuti, con 12 costa 1 minuto, con Arcana 18 costa 10 PA.

\medskip

\textbf{Riconoscere un incantesimo}\index{Riconoscere un incantesimo} mentre viene lanciato è una prova di \textbf{Arcana} Costa una \textbf{Reazione}. Se fatto assieme al lancio di un Controincantesimo non costa Reazione.

\medskip

Per \textbf{riconoscere un mostro}, una creatura particolare si effettua una prova di Conoscenza. Controlla il capitolo \hyperlink{riconoscereimostri}{Riconoscere i Mostri} nel Mostruario (pag. \pageref{riconoscereimostri})

\medskip

\textbf{Atletica}\index{Atletica} \textit{Penalità dovute all'armatura}

Una prova di Atletica riuscita permette al personaggio di dimezzare il danno quando cade da meno di 9 metri (\textbf{Reazione}).

\medskip

\textbf{Arrampicarsi/Scalare} \index{Arrampicarsi}\index{Scalare} \textit{Penalità dovuta all'Armatura.}

\medskip

Usare una corda\index{Arrampicarsi su una corta}\index{Salire su una corda}, scalare od arrampicarsi equivale a muoversi in un \textbf{terreno doppiamente difficile}. Se la prova di Arrampicarsi riesce si sale di 30 cm per PA speso.

In caso di fallimento della prova si consumano i PA senza spostarsi. Se la prova fallisce (MS negativo) di 6 o più perdi la presa e cadi. I modificatori indicati nella tabella si sommano.\\

\begin{tabularx}{0.45\textwidth}{Xl}
	\textbf{Esempio di Superficie} & Mod.\\
	\toprule
	Movimento solo dimezzato & -1\\
	Superficie scivolosa&-1\\
	Parete grezza con appigli, mattoni sporgenti&-1\\
	Una corda senza nodi&-2\\
	Una parete con appigli &+2\\
	Un muro/parete con pochissimi appigli&-3\\
	Ti puoi appoggiare a 2 pareti opposte&+2\\
	Ti puoi appoggiare a 2 pareti angolari&+1\\
	Puoi usare una corda&+2\\
\end{tabularx}\\

i modificatori indicati sono sulla prova effettuata dal giocatore, se positivo è un bonus (alzi il Punteggio di Competenza).

\medskip

Per \textbf{identificare una pozione o veleno naturale}\index{Identificare Veleno}\index{Erboristeria} \index{Identificare Pozione}è necessario una prova di \textbf{Erboristeria}.

Costa 5 minuti. Se il MS è +3 impieghi 4 minuti, se +6 impieghi 3 minuti, con -9 impieghi 2 minuti, +12 impieghi 1 minuto, +15 impieghi 1 round.

Se la prova fallisce ed il MS è -6 ha assunto la pozione consumandone una dose.

\medskip

\textbf{Intimidire}\index{Intimidire}. Il personaggio usa \textbf{6 PA} ed effettua una prova di Intimidire, l'avversario può contrapporre una prova di Intimidire o di Corpo. Chi ottiene il MS migliore intimidisce l'avversario.

Chi è intimidito ha 1 Penalità al PDA fino alla fine del round successivo.

\medskip

\textbf{Ammansire un animale} è una prova di \textbf{Gestire Animali}. Tempo richiesto 10 minuti. Per ogni MS il tempo si riduce di 1 minuto.

\medskip

\textbf{Furtività} \index{Furtività} \textit{Penalità dovuta all'Armatura.}

La prova di Furtività va effettuata solo se c'è qualcuno che può sentire/vedere. Si confrontano i MS delle prove di Furtività e Osservare per capire se si è stati percepiti. Muoversi in maniera Furtiva equivale a muoversi su terreno difficile e quindo ci vogliono 2 PA per spostarsi di 1.5 metri.

\medskip

\textbf{Nuotare}\index{Nuotare} \textit{Penalità dovuta all'Armatura}

In acque calme basta una prova riuscita di nuotare, se le acque sono mosse il MS deve essere di almeno 3, e di 6 se molto mosse e 9 se tempestose. La prova è necessaria sia per stare a galla o nuotare. Nuotare in acqua si considera \textbf{terreno difficile}.

\medskip

\textbf{Pronto Soccorso}\hypertarget{prontosoccorso}{}\label{prontosoccorso}\index{Pronto Soccorso}. Una prova riuscita fa recuperare 1d4 Vitalità se fatta entro 1 minuto dal termine dello scontro.

Concede 1 Bonus ad una prova di Caratteristica contro un veleno se non ha ancora fatto effetto. Costo \textbf{2 minuti}. Con MS di +6 costa 1 minuto. Con MS +9 costa 3 round, con MS +12 costa 1 round.

Una prova riuscita riduce di 1 i danni da \hyperlink{sanguinamento}{\textbf{Sanguinamento}}. Se la prova riesce con MS +3 riduce di 2 punti, con MS +6 riduce di 3 punti.

Un trattamento di almeno 8 ore permette di recuperare al paziente il doppio di Corpo in punti Vitalità. Se effettuato durante le ore di riposo chi prende cura risulterà Affaticato.

\medskip

\textbf{Saltare}\index{Tabella Saltare} \textit{Penalità dovuta all'Armatura.} \textbf{4 PA}\\

Con una prova di Atletica è possibile saltare in lungo 3 metri. Per ogni MS salti 30 cm in più.

La \textbf{distanza saltata in alto} è pari a 90cm + 10cm per MS.

In un \textbf{salto in lungo} la punta più alta del salto è pari ad un 1/4 della lunghezza saltata. Se esegui un salto in lungo di 6 metri a metà salto sei in alto di 1.5 metri.

Scendere da meno di 1m non usa PA. Se non si ha almeno 3 metri di rincorsa si salta la metà.

Vitalità persa per caduta (pag. \pageref{cadute}): 3x altezza caduta (in metri). Prova di Atletica per dimezzare il danno se cadi da meno di 9 metri.

\medskip

\textbf{Sopravvivenza}\index{Sopravvivenza}

\smallskip

\textbf{Inseguire una creatura}:

\begin{tabular}{ll}
	Situazioni & Mod.\\
	\toprule
	Se il terreno è molto morbido& +2\\
	Se il terreno è morbido& +1\\
	Se il terreno è stabile& 0\\
	Se il terreno è duro& -2\\
	Ogni 6 creature inseguite& +1\\
	Ogni 24 ore passate & -1\\
	Visibilità scarsa&-1\\
	Ogni ora di pioggia&-1 \\
	Cerca di occultare le traccie& -1\\
\end{tabular}\\

Un Modificatore negativo è una penalità (abbassa il risultato dei dadi), un modificatore positivo è un bonus (alza il risultato del Punteggio di Competenza).

Sopravvivenza può essere usata al posto di \textbf{Disattivare Congegni} con Svantaggio.

Una prova di Sopravvivenza per foraggiare cibo procura viveri per una persona aggiuntiva ogni 3 di MS.

\medskip

La prova di \textbf{Mercanteggiare}\index{Mercanteggiare} serve per abbassare il prezzo di una merce e per valutare un oggetto. Oggetti molto rari richiedono un MS di almeno +3 per essere valutati.


\end{multicols}

\pagebreak


\section{I Rami}\index{Rami}

\begin{multicols}{2}

I \textbf{Rami} sono la professione del personaggio, è l'insieme delle competenze che il personaggio conosce. In altri sistema di gioco il Ramo sarebbe l'equivalente della classe.

\textbf{Ogni Ramo conferisce al personaggio dei punti Vitalità}, da sommare a Corpo per stabilirne il valore iniziale, e delle Competenze.

I Rami di base ed avanzati concedono 6 Competenze.

Mentre i \textbf{Rami di base} possono essere presi come prima professione da chiunque, i\textbf{ Rami avanzati} possono essere appresi solo a patto di soddisfare i requisiti indicati.

\subsection{Rami Base}

Nella tabella sottostante sono indicati alcuni Rami base di esempio. il personaggio è invitato a crearsi Rami con Competenze più affini alla sua storia.
Sono indicati il nome del Ramo e la Vitalità. Le \textbf{competenze prendono un punteggio} pari alla punteggio indicato dalla prima colonna. Ad esempio un Apprendista ha punteggio 6 in Conoscenza mentre un Bandito ha 4 punti in Furtività.

\end{multicols}

\begin{tabular}{|l|l|l|l|l|}\hline

&\multicolumn{4}{c}{\textbf{Rami}}\\\hline

&\textbf{Apprendista}&\textbf{Bandito}&\textbf{Tagliaborse}&\textbf{Baro}\\\hline
\textbf{Vitalità}&2&9&4&6\\\hline
6&Conoscenza &Armi piccole&Mani di fata&Raggirare\\
5&Incantamento&Intimidire&Furtività&Per. Emozioni\\
4&Pronto soccorso&Furtività&Armi piccole&Armi piccole\\
3&Intimidire&Armi medie&Sopravvivenza&Mani di fata\\
2&Armi piccole&Mani di fata&Pronto Soccorso&Rissa\\
1&Tradizioni Locali&Pronto soccorso&Atletica&Intimidire\\\hline

&\textbf{Tirapiedi}&\textbf{Cacciatore}&\textbf{Segugio}&\textbf{Commerciante}\\\hline
\textbf{Vitalità}&6&4&7&2\\\hline		
6&Rissa &Natura&Armi da tiro&Mercanteggiare\\
5&Intimidire&Sopravvivenza&Intimidire&Raggirare\\
4&Furtività&Armi da tiro&Seguire tracce&Per. Emozioni\\
3&Osservare&Furtività&Osservare&Intrattenere\\
2&Cavalcare&Erboristeria&Mercanteggiare&Armi piccole\\
1&Mani di fata&Osservare&Armi piccole&Cavalcare\\\hline

&\textbf{Contafrottole}&\textbf{Guardia del corpo}&\textbf{Intrattenitore}&\textbf{Mendicante}\\\hline
\textbf{Vitalità}&3&6&3&3\\\hline		
6&Raggirare			&Armi a due mani&Intrattenere	&Osservare\\
5&Percepire Emozioni&Orientamento	&Travestimento	&Sopravvivenza\\
4&Storia			&Pronto soccorso&Storia			&Mercanteggiare\\
3&Armi piccole		&Osservare		&Armi piccole	&Armi piccole\\
2&Mercanteggiare	&Conoscenza		&Raggirare		&Furtività\\
1&Conoscenza		&Intimidire		&Diplomazia		&Mani di fata\\\hline

&\textbf{Mercenario}&\textbf{Milizia cittadina}&\textbf{Minatore}&\textbf{Nobile}\\\hline
\textbf{Vitalità}&16&9&4&2\\\hline		
6&Armi medie		&Armi medie		&Conoscenza	caverne	&Diplomazia\\
5&Armi a due mani	&Tradizioni locali&Sopravvivenza	&Conoscenza\\
4&Sopravvivenza		&Intimidire		&Armi medie			&Linguaggi\\
3&Armi da tiro		&Cavalcare		&Orientamento		&Storia\\
2&Armi piccole		&Rissa			&Mercanteggiare		&Cavalcare\\
1&Intimidire		&Armi da tiro	&Atletica			&Intimidire\\\hline
							
\end{tabular}	\\

La Vitalità che concede un Ramo si calcola sommando i punteggi delle Armi. Il valore minimo di Vitalità è 2.

\begin{multicols}{2}	

\subsection{Rami Avanzati}

I \textbf{Rami avanzati}\index{Rami Avanzati} hanno un prerequisito di Caratteristica e di Competenza a certi punteggi.
La Vitalità che concede un Ramo avanzato è pari a 3 per il numero di Armi che fa conoscere. In valore minimo di Vitalità che un Ramo avanzato concede è +3.
La Vitalità concessa da un Ramo avanzato si somma con quella già posseduta dal personaggio.

I Rami avanzati aumentano il punteggio delle Competenze di 1, se sono Competenze già note, altrimenti impostano ad 1 il valore se non erano note.

Alcuni esempi di Rami avanzati:\medskip

\textbf{Assassino}:\\
\textit{Requisito}: Corpo 12, Sopravvivenza 12, Armi piccole 12\\
\textit{Competenze concesse}: Armi da tiro, Sopravvivenza, Travestimento, Armi medie , Furtività , Osservare\\
\textit{Vitalità}: +6\\

\textbf{Sgherro}:\\
\textit{Requisito}: Corpo 12, Intimidire 12, Armi piccole 12\\
\textit{Competenze concesse}: Armi da tiro, Rissa, Armi a due mani, Sopravvivenza, Mani di fata, Intimidire\\
\textit{Vitalità}: +9\\

\textbf{Capitano della Guardia}:
\textit{Requisito}: Volontà 12, Armi medie 12, Tradizioni Locali 12
\textit{Competenze concesse}: Armi da tiro, Cavalcare, Intimidire, Sopravvivenza, Storia, Pronto soccorso\\
\textit{Vitalità}: +3\\

\textbf{Esploratore}:\\
\textit{Requisito}: Corpo 12, Sopravvivenza 12, Orientamento 12\\
\textit{Competenze concesse}: Geografia, Armi piccole, Atletica, Nuotare, Cavalcare, Linguaggi\\
\textit{Vitalità}: +3\\

\textbf{Cavaliere errante}:\\
\textit{Requisito}: Volontà 12, Cavalcare 12, Armi medie 12\\
\textit{Competenze concesse}: Armi da tiro, Armi a due mani, Diplomazia, Pronto soccorso, Linguaggi, Storia\\
\textit{Vitalità}: +6\\

\textbf{Menestrello}:\\
\textit{Requisito}: Mente 12, Intrattenere 12, Conoscenza 12\\
\textit{Competenze concesse}: Armi piccole, Raggirare, Storia, Travestimento, Diplomazia, Incantamento\\
\textit{Vitalità}: +3\\

\textbf{Incantatore}:
\textit{Requisito}: Mente 12, Incantamento 12, Conoscenza 12
\textit{Competenze concesse}: Armi medie, Conoscenza, Storia, Linguaggi, Diplomazia, Erboristeria 
\textit{Vitalità}: +3\\

\textbf{Monaco}:\\
\textit{Requisito}: Corpo 12, Volontà 12, Pronto soccorso 12\\
\textit{Competenze concesse}: Rissa, Diplomazia, Conoscenza, Storia, Linguaggi, Pronto soccorso\\
\textit{Vitalità}: +3\\


\subsection{Avanzare nel Ramo}\index{Avanzare nel Ramo}

Ogni volta che la\textbf{somma dei punti assegnati} grazie al \hyperlink{Migliorare le Competenze}{Migliorare le Competenze} arriva a 3 la Vitalità aumenta di 3 per i Rami base.
In caso di \textbf{Rami avanzati} la somma dei punti assegnati deve arrivare a 5 per poter aumentare di 5 la Vitalità.

\subsection{Prendere un nuovo Ramo}\index{Prendere un nuovo Ramo}

Alla creazione del Personaggio si scegli un Ramo e si apprendono Competenze e si imposta il valore di Vitalità a Corpo + il valore di Vitalità dato dal Ramo.

Il personaggio può decidere di acquisire più \textbf{Rami base} e quindi conoscere più Competenze. Per poter intraprendere un nuovo Ramo il personaggio deve trovare qualcuno che possa insegnarglielo e pagare 1000 monete d'oro. Tutte le Competenze del suo Ramo precedente diminuiscono di 1.
Il fatto di prendere un nuovo Ramo non fa aumentare la Vitalità, rimane valida la regole dei 3 punti distribuiti prima di aumentare la Vitalità.

Per prendere un \textbf{Ramo avanzato} si devono soddisfare i requisiti indicati, trovare qualcuno che possa insegnarlo, pagare 5000 monete d'oro. In questo caso non c'è una diminuzione delle Competenze originari ed i punti Vitalità non si acquisiscono.


\end{multicols}

\pagebreak

\section{Combattimento}\index{Combattimento}\label{Combattimento}\hypertarget{Combattimento}{}

\begin{multicols}{2}
	
Il combattimento è diviso in 2 fasi:\index{Combattimento}
\begin{itemize}
	\item verifica delle Azioni
	\item verifica Iniziativa
	\item risoluzione delle Azioni (movimento, attacco, azione varie..)
\end{itemize}

L'unità base di tempo nelle scene di combattimento è il Round ovvero una unità di tempo di 10 secondi.

Ogni personaggio può eseguire diverse Azioni nel round ed ognuna di queste costa dei Punti Azione. \textbf{Chi meno ne esegue più è veloce nell'eseguirle}.

All'inizio di ogni round il giocatore dichiara quanti PA userà. Non è necessario che dichiari cosa andrà ad eseguire.

La verifica dell'Iniziativa\index{Iniziativa} consiste nel tirare 2d10 sottrarre Corpo o Mente ed aggiungere i PA che si intendono usare.

\textbf{Il personaggio od avversario che ha un valore dell'Iniziativa più basso incomincia per primo.}

A parità di Iniziativa chi usa meno Punti Azione (PA) incomincia per primo, se i PA usati sono uguali parte per primo l'avversario.
A parità di PA usati tra i personaggi questi si mettono d'accordo tra loro per agire in che ordine.

Il personaggio potrebbe anche usare meno PA di quanti dichiarati, ma agirà in ogni caso nell'ordine stabilito dal valore dell'Iniziativa.

\textbf{Un personaggio non può dichiarare un certo numero di PA e poi usarne di più.}

\subsubsection{Il Tempo (Round, Minuti e Turni)}\index{Round}\label{iltempo}

Un \textbf{round} dura 10 secondi circa, è un lasso di tempo sufficiente per agire, correre, parlare.. combattere. Un Minuto sono 6 round ed un Turno dura 10 Minuti (o 60 round).

I round si usano nelle scene di combattimento o dove la tensione deve rimanere costantemente alta ed ad ogni Azione corrisponde un evolversi della situazione.

\subsubsection{Tempo di riattivazione Oggetti ed Abilita'}\index{Tempo di attivazione Oggetti ed Abilità}\label{temporiattivazioneoggetti}

Se non specificato diversamente un oggetto o Abilità che prevede un certo numero di usi al giorno \textit{"es. una volta al giorno"} si "ricarica" all'alba successiva l'uso.

\subsection{Punti Azioni nel Round}\index{Azioni nel Round}\index{Azione}\label{azioninelround}

Nella tabella sottostante sono indicate le Azioni principali e relativi Punti Azione che usano, sono linee guida da seguire. Nel capitolo dedicato al combattimento vengono elencate altre Azioni ed i loro costi relativi in Punti Azioni.

Le Azioni scelte possono essere eseguite nell'ordine preferito.

Una Azione non può essere interrotta\index{Interrompere Azioni}\index{Azioni, Interrompere} da un altra Azione, ma può essere seguita da una Azione di Reazione o da una Azione Immediata, se nel proprio round.

Se un personaggio vuole fare più attacchi spostandosi nel campo di battaglia può, ad esempio, usare 4 PA per eseguire un attacco, usare 2 PA per muoversi di 3 metri ed usare gli ultimi 4 PA (nel caso ne avesse dichiarati 10) per un ultimo attacco.

E' possibile \textbf{ritardare} una o più Azioni\index{Ritardare Azioni} per aspettare lo svolgersi delle scene. Il personaggio che ritarda una sua Azione si considera che abbia "sprecato" PA per aspettare fino a quel segmento di iniziativa e potrà usare solo i PA che rimangono fino alla fine del round.

Un giocatore che dichiara di aspettare una certa situazione per poter agire equivale ad eseguire una o più \textbf{Azioni Preparate}\index{Azioni Preparate}. In questo caso il personaggio (o nemico) agisce dopo l'Azione con solo i PA rimasti per avere aspettato fino a quel segmento di iniziativa.

Se il personaggio ha già usato tutte i PA allora potrà agire fuori dalla sua iniziativa solo tramite una Reazione, se a disposizione. L'Azione di Reazione arriva sempre dopo l'Azione scatenante.

\bigskip

\end{multicols}

\textbf{Tabella: Azioni per Round}\index{Tabella delle Azioni per Round}

\section{DOVE SONO ARRIVATO con la revisione}

\begin{tabularx}{0.95\textwidth}{Xc}
\textbf{Cosa si fa}  & \textbf{Punti Azioni}\\
\toprule
Attaccare con Rissa					& 4\\
Attaccare con Arma piccola			& 4\\
Attaccare con Arma media/da Tiro	& 5\\
Attaccare con Arma a due mani		& 8\\
Lanciare un'Incantesimo *1			& *\\
Muoversi *2							& 1 PA per 1.5 metri\\
Scatto *3							& 1 PA per 3 metri\\
Alzarsi da prono					& 4\\
Aiutare qualcuno					& 5\\
Scambiare un dialogo con qualcuno *4	& variabile\\
Scambiare poche battute con qualcuno *5& 0\\
Prendere qualcosa nello zaino		& 8\\
Prendere qualcosa dalla cintura o di pronto & 4\\
Usare un oggetto tenuto in mano		& 2\\
Bere una pozione tenuta alla cintura& 4\\
Estrarre/Rinfoderare l'arma			& 3\\
Imbracciare lo scudo				& 3\\
Usare un oggetto magico				& 6\\
Eseguire prova su una competenza *6	& 6\\
Sfondare una porta a spallate/calci	& 5\\
Forzare porta con piede di porco	& 6\\
Nascondersi							& 4\\
Concentrarsi su un Incantesimo		& 4\\
Salire o scendere dalla cavalcatura	& 4\\
Azione \textbf{I}mmediata - Azione \textbf{R}eazione& I - R\\
Bere una pozione tenuta in mano& I\\
Gettare un oggetto tenuto in mano& R\\
Gettarsi a terra prono& R\\
Riconoscere un Incantesimo& R\\
\end{tabularx}

\medskip

Per Attacco si intende sia l'uso di armi in mischia che l'uso di armi da lancio o tiro come archi, balestre o pugnali da lancio. Nel caso di armi da lancio ogni lancio/tiro conta come un attacco.

Qualora il personaggio esegua una Azione di Attacco e Lanciare un incantesimo si considera Distratto nell'eseguire la Prova di Magia.

\begin{multicols}{2}
	
\textbf{Lanciare un Incantesimo *1}: a seconda del potere dell'Incantesimo sono necessari più PA. Nella descrizione dell'incantesimo è indicato il numero di PA necessari. 

\textbf{Muoversi *2}: per ogni PA usato ci si può muovere fino a 2 metri.

\textbf{Scatto *3}: per ogni PA usato ci si può muovere fino a 3 metri ma si incorre nella penalità di aver \textbf{corso}.

\textbf{Scambiare un dialogo con qualcuno *4}: Un dialogo può essere di pochi secondi se non di minuti. Il Narratore valuterà quanto questo dura.

\textbf{Scambiare poche battute con qualcuno *5}: Finché sono veramente poche battute o uno sguardo non consuma PA, se questo diventa più articolato allora utilizza dei PA. L'obiettivo è non interrompere il flusso delle Azioni con un fitto dialogo ma comunque permettere l'interazione tra i personaggi.

\textbf{Eseguire prova su una competenza*6}: se fruttano una frazione del round costano 4 PA, altrimenti 8 PA o più. Controllate negli Esempi Prove Competenza i costi riportati.

Una Azione "\textbf{Reazione (R)}" \index{Azione Reazione}può essere eseguita liberamente anche fuori dal proprio round. Questa Azione è solitamente dovuta ad Abilità o situazioni particolari. Se non indicato diversamente una Azione di Reazione accade immediatamente dopo la causa che la scatena.

Una Azione "\textbf{Immediata (I)}" \index{Azione Immediata}può essere eseguita liberamente nel proprio round, primo o dopo la propria Azione. Una Azione Immediata è solitamente concessa da particolari Abilità.

E' possibile se non descritto diversamente eseguire solo una Azione Immediata ed una Azione di Reazione per round.

\medskip

Questo \textbf{elenco non è completo}, prendetelo come linee guida per stabilire il peso delle decisioni ed azioni dei giocatori. Una Azione dura circa 3 secondi.

L'\textbf{ordine} con cui si eseguono le Azioni non è importante se non per correlazione logica e fisica. Il Movimento può essere tra altre Azioni (movimento, attacco/incantesimi/altra azione, movimento).


\end{multicols}

\subsection{Movimento}\index{Movimento}\label{movimento}


\begin{changemargin}{0.3cm}{0.3cm}\begin{enfasi}{"Un mobile più lento non può essere raggiunto da uno più rapido; giacché quello che segue deve arrivare al punto che occupava quello che è seguito e dove questo non è più (quando il secondo arriva); in tal modo il primo conserva sempre un vantaggio sul secondo" (Paradosso di Zenone)}
\end{enfasi}\end{changemargin}

\begin{multicols}{2}

Il movimento di un personaggio è dato dalla sua taglia e razza e da ciò che porta, dai pesi, ingombri ma anche magie ed oggetti magici.

Il Movimento scritto nella razza del personaggio è l'indicazione di quanti metri può fare usando 10 PA.

Una creatura o personaggio potrebbe anche decidere di spostarsi più velocemente del solito ovvero correndo (Azione di Scatto).\index{Correre}

Non è possibile spostarsi anche solo di 1 metro se non si spendono PA.

Queste precisazioni hanno senso e vanno usate quando si tratta di combattere ed il dislocamento sul territorio, mappa, è fondamentale. Durante gli spostamenti normali, mentre si cavalca o cammina liberi senza pericoli, si usa la normale gestione del movimento orario.

Quando si parla di "\textbf{quadretto}" \index{Quadretto}per indicare una distanza si intende un quadretto di mappa di 1.5 metri x 1.5 metri.

Nel caso di spostamento diagonale\index{Movimento diagonale}\index{Spostarsi di lato} si conta una distanza di 2 quadretti, in caso di arrotondamenti sull'ultimo quadretto si fa per eccesso.

\textbf{Se ci si sposta in terreno "difficile"}, si consumano il doppio di PA per spostarsi di 1.5 metri.

\subsection{Distanza}\index{Distanza}\label{distanza}

Per \textbf{distanza di Mischia} \index{Distanza di Mischia} \index{Mischia}si intende una distanza che permette il combattimento corpo a corpo (1.5 metri attorno al personaggio). Nei mostri questa distanza è indicata dalla portata, per le armi da lancio è chiamata gittata.

Se non indicata nell'avversario/mostro la distanza di mischia/tocco aumenta di 1.5 metri per ogni taglia oltre la media.\index{Taglia e distanza di mischia}
Alcune Armi a due mani hanno una portata di 3 metri.

A distanza di mischia una creatura di dimensioni medie può avere al massimo 8 creature medie.

\end{multicols}

\pagebreak

\subsection{Vita e Morte}\index{Morire}\label{morire}

\begin{changemargin}{0.3cm}{0.3cm}\begin{enfasi}{Chi non conosce la morte, non conosce la vita. (Grand Hotel, film 1932)

\medskip

The worthy GM never purposely kills players' PCs. He presents opportunities for the rash and unthinking players to do that all on their own (Gary Gygax)}\end{enfasi}\end{changemargin}\medskip

\begin{multicols}{2}

Quando un personaggio raggiunge 0 (zero) di Vitalità si considera svenuto, ovvero Inabile a fare qualsiasi cosa. Una Cura magica (Incantesimo, Pozione..) lo porterà cosciente ed al valore Vitalità curato. Una prova di Pronto Soccorso, 10 PA lo porterà ad 1 di Vitalità. Dopo un ora se non è successo qualcosa a mutare la situazione il personaggio può fare prova di Corpo, se riesce torna a 1 di Vitalità, se fallisce va a -1 di Vitalità e diventa morente.

Un personaggio morente ha Vitalità negativa (-1 o meno) ed è svenuto e \hyperlink{morente}{prossimo alla morte}. Continuerà a perdere 1 punto di Vitalità a round fiche il valore non raggiungerà -10 e morità, se non viene curato.

Una magia (incantesimo o pozione) di Cura, di qualsiasi potere lo porterà a 1 di Vitalità, successive cure funzioneranno normalmente.

Una prova di \hyperlink{prontosoccorso}{Pronto Soccorso} con Svantaggio, 10 PA, porterà il personaggio a 0 di Vitalità, ovvero svenuto. 

Una successiva prova di Pronto Soccorso, 10 PA, potrà portarlo a 1 di Vitalità ed una cura magica lo curerà dell'ammontare dichiarato.

Un personaggio morente che subisce ulteriore danno, nemici che infieriscono sul corpo od incantesimi diretti a lui od ad area, continua a sottrarre Vitalità. 

Le Condizioni \index{Condizioni mentali}di tipo mentale quali Affascinato, Confuso ma non Dominato, terminano quando il personaggio diventa morente.

Quando un personaggio torna a Vitalità 1 dopo essere andato a Vitalità negativi ha Svantaggio a tutte le Prove finché non riposa una notte.

Un personaggio morto non può beneficiare delle cure normali o magiche. Solo incantesimi molto un potenti possono riportarlo in vita.

\subsubsection{Recupero punti Caratteristica}\index{Recupero punti caratteristica}\label{recuperopunticcaratteristica}

Eventuali punti Caratteristica persi si recuperano al ritmo di 1 punto al giorno, se non indicati come perdita permanente.

\subsubsection{Recupero Vitalità naturale}\index{Recupero Vitalità naturale}\label{recuperopuntiferitanaturale} 

Riposare 8 ore fa recuperare il punteggio di Corpo in Vitalità

\subsubsection{Recupero Vitalità non letale}\index{Recupero Vitalità non letale}\index{Perdita Vitalità non letale}\label{recuperopuntiferitanonletali}\hypertarget{recuperopuntiferitanonletali}{}

Ogni ora si recupera 1 punto Vitalità.

\subsection{Tiro per Colpire e Difesa}\index{Tiro per Colpire}\index{Difesa}\label{tiropercolpireedifesa}

\begin{changemargin}{0.3cm}{0.3cm}\begin{enfasi}{Applica sempre la giusta forza, mai troppa mai troppo poca. (Kano Jigoro)}\end{enfasi}\end{changemargin}\medskip

Ogni qual volta una creatura \textbf{decida di Attaccare} deve effettuare una Prova d'Arme d'attacco (PDAA), ovvero somma il suo punteggio di Competenza d'Armi, nell'arma che sta usando, con il punteggio di Corpo e sottrae la somma di 2d10. La differenza è il Margine di Sicurezza (MS) ottenuto.

Ogni qual volta una creatura \textbf{vuole difendersi} deve effettuare una Prova d'Arme di difesa (PDAD),  ovvero somma il suo punteggio di Competenza d'Armi, nell'arma che sta usando per difendersi, con il punteggio di Corpo e sottrae la somma di 2d10. La differenza è il Margine di Sicurezza (MS) ottenuto.

L'attaccante confronta il suo MS con quello di chi si difende, se superiore o uguale avrà colpito e causerà i danni alla Vitalità.

Il linea di massima le Armi piccole causano 1d6 di danno, le Armi medie 1d8, le Armi a due mani 1d10, Rissa causa 1d4, controllate questi valori nell'equipaggiamento. Questi numeri sono da aumentare con il modificatore dato da Corpo.

Al danno causato dall'attacco si sottrae l'eventuale protezione data dall'armatura, il danno rimanete (minimo 1) si sottrae ai punti Vitalità.

Al danno causato dall'avversario sottrarrà la protezione dell'armatura ed il rimanente alla Vitalità.

Se i modificatori e circostanze portano il danno inflitto ad essere negativo o zero comunque farai 1 danno a Vitalità.

Ci sono situazioni che possono avvantaggiare la difesa quali coperture, nascondigli, come fosse, porte, compagni di taglia molto più grande della propria. Consultate i paragrafi relativi ai \hyperlink{coperture}{Nascondigli e Coperture} per capire il vantaggio che possono dare.

\subsection{Sfortuna e Fortuna nella prova d'Armi per difenderti}

Se nella prova d'Armi di difesa \textbf{tiri due} 1 avrai sicuramente evitato il colpo indipendentemente dal risultato finale ed la tua prossima Prova d'Armi di Attacco avrà Vantaggio.

Se nella prova d'Armi di difesa \textbf{tiri due} 10 sei stato colpito e l'avversario avrà Vantaggio nel danno da applicare.

\subsection{Armi da tiro}\index{Attacchi multipli armi da tiro}\label{armidatiro}\index{Armi da Tiro}\index{Armi da Lancio}

Le armi da tiro sono tutte le armi con una gittata, ovvero che possono essere lanciate o lanciano dei proiettili. Le principali armi da lancio sono gli archi, balestre, fionde ma anche pugnali, giavellotti, lance qualora siano gettate.

Il bonus al danno dato da Corpo si applica in automatico per le fionde, pugnali, giavellotti..ovvero con tutte le armi che vengono lanciate "a mano", gli archi e le balestre non lo applicano mai.

\textbf{I proiettili lanciati da Archi, Fionde, Balestre magiche non si considerano magici.\\
In caso di proiettili magici questi sommano il loro bonus magico al Tiro per Colpire ed al danno}

In ogni arma da tiro è segnata la gittata ovvero entro che distanza è possibile tirare il proiettile senza penalità. Ogni arma da tiro può colpire entro tre volte la gittata indicata.

Se l'obiettivo è entro la gittata indicata non si hanno penalità al colpire, se il target è tra il primo e secondo hai una Penalità alla PDAA di 1, Se il target è tra il secondo è terzo incremento la penalità al colpire è di -2.

Un pugnale tirato entro 6 metri non ha penalità, ma tirato tra i 6 ed i 12 metri ha Penalità -1, a distanza tra 12 e 18 metri ha penalità -2 al PDAA, oltre non può essere tirato.

\subsection{Arma Lunga} \index{Arma Lunga}\label{armalunga}

Alcune armi a due mani hanno l'attributo di Arma lunga. L'arma lunga da diritto a colpire più lontano ovvero a 3 metri. Causa 1 penalità alla prova d'Armi per difendersi. Questo bonus rimane valido finché l'avversario non entra in distanza della propria mischia.

Se l'avversario riduce la distanza a meno di 3 metri non ha più la Penalità alla PDAD.

\subsection{Carica} \index{Carica}\label{carica}

l'Azione di carica costa 6 PA più quanto necessario per coprire la distanza.

Chi carica ha Vantaggio nella PDAA ma avrà Svantaggio all'attacco dell'avversario entro la fine del round successivo.

\subsubsection{Preparare una arma lunga/da controcarica contro una carica} \index{Preparare una arma lunga contro una carica}\label{prepararearmalungacontrocarica}

E' una Reazione.

\subsubsection{Carica con Arma da Controcarica} \index{Controcarica}\label{caricaarmadacontrocarica}

una Carica effettuata con successo con un arma da controcarica causa 2 danni a Vitalità in più.

\subsection{Attacchi con armi a spargimento} \index{Armi a spargimento}\index{Acqua santa}\index{Olio Incediato}\label{attacchiarmidaspargimento}\hypertarget{spargimento}{}

sono armi a spargimenti quelle che "spargono" il loro contenuto dove cadono, ad esempio olio incendiato/Acqua santa... Una arma a spargimento ha una gittata di 6 metri\index{Lanciare Armi a Spargimento}\index{Gittata armi a spargimento}.

In caso la difesa riesca nella prova d'Armi tira un d8 e consulta questo schemino per capire dove la boccia è caduta:

\medskip

\begin{tabularx}{0.30\textwidth}{ccc}
1& 2& 3\\
4 &\textbf{X}& 5\\
6 &7 &8\\
&\textbf{0}&\\
\end{tabularx}

\smallskip

\textbf{X} si considera il bersaglio dell'oggetto tirato. \textbf{0} il punto di origine del lancio.

Tira 2d4 per determinare lungo la direzione indicata dal d8 precedente a quanti 1.5 metri è caduto distante dal bersaglio, ovvero contate i metri dal target.

Ad esempio con il tiro del d8 faccio 5 e poi tirando 2d6 faccio 4, significa che la boccetta è caduta a destra del bersaglio a 6 metri.

E' anche possibile che ci si sia tirati sui piedi la boccetta (es faccio 7 e poi 6.. potrei averla tirata addosso ad un compagno o dietro di me!).


\subsection{Impreparato -- Colti di Sorpresa}\index{Impreparato}\index{Sorpresa}\label{coltidisorpresa}

se i personaggi vengono colti di sorpresa, ovvero non si aspettano di essere attaccati, si deve considerare questo primo round come round di sorpresa. Quando si è sorpresi si ha Svantaggio alla prova d'Armi per difendersi.

Non potrai reagire, non userai Azioni o Reazioni se non esplicitamente permesse; dal round successivo potrai dichiarare l'iniziativa ed agire normalmente. Le medesime considerazioni valgono per gli avversari.

Per valutare se un personaggio è sorpreso effettuate una prova di Corpo, se la prova riesce allora il personaggio non è sorpreso, altrimenti lo è.

\subsection{Magia in combattimento}\index{Magia in combattimento}\label{magiaincombattimento}

l'incantatore che lancia una magia mentre è in combattimento (ha un avversario in mischia o viene bersagliato da distanza) si considera Distratto.

\subsection{Bonus e Penalità in Combattimento}

il combattente impone 1 Penalità quando alla Prova d'Armi di Difesa (PDAD)

\begin{itemize}

\item ti fiancheggia, è in posizione sopraelevata, ti attacca alla schiena, arma lunga, sei abbagliato, sei intralciato, sei afferrato, combatti con luce fioca, 

\item 
il combattente impone 2 Penalità quando:
\subitem sei prono, sei ristretto, sei spaventato, ti difendi con un tipo di arma non conosciuta, 

\item 
il combattente impone Svantaggio quando:
\subitem è invisibile, è in carica, sei sorpreso

\end{itemize}

il combattente ha un Bonus nella prova d'Armi per difendersi quando:

\begin{itemize}
	
\item 
ha copertura, combatte da più in alto

\end{itemize}

\subsubsection{Aiutare un altro}\index{Aiutare}\label{aiutare}

si può aiutare un compagno ad attaccare o a difendersi negli scontri in mischia, distraendo o interferendo con l'avversario. 

Si esegue una prova d'Armi e se si riesce si concede un bonus alla prova d'Armi per difendersi o come penalità alla prova d'Armi di difesa dell'avversario.

\subsubsection{Tiri Mirati}\index{Tiri Mirati}\label{tirimirati}\index{Mirare a parti specifiche}

Dark Catacomb non prevede la possibilità di effettuare tiri mirati con qualsiasi arma o incantesimo, tranne se questo lo specifica.

Quando si colpisce il bersaglio lo si colpisce genericamente, senza possibilità di specificare se alla testa, gamba o altro, medesimo concetto vale in caso di colpi ad oggetti, es. se miri ad un cardine di una porta colpisci tutta la porta. Questo non impedisce all'Arbitro di valutare conseguenze adeguate.

\subsubsection{Danno non letale}\index{Danno non letale}\label{dannononletale}

il danno non letale è una forma di danno causato da armi particolari o quando volutamente lo scopo è fare svenire il nemico e non ucciderlo.

Il danno non letale si tratta come il danno alla Vitalità ma va segnato a parte nella scheda.

\subsubsection{Danno non letale con arma non idonea} \index{Danno non letale con arma non idonea}\label{dannononletalearmanonidonea}

se vuoi fare danno non letale con un'arma non predisposta al danno non letale la prova d'Armi attacco (PDAA) ha Svantaggio.

\subsubsection{Senza Competenza}\index{Senza Competenza}\label{senzacompetenza}

usare una tipologia di arma senza l'adeguata competenza, ovvero non avere Armi a due Mani mentre si vuole usare uno Spadone, causa Svntaggio alla prova d'Arma per attaccare.

\subsubsection{Lanciare armi} \index{Lanciare armi}\label{lanciarearmi}

una spada o comunque un arma non fatta per essere lanciata, senza Gittata, può comunque essere scagliata contro l'avversario con Svantaggio.

Il danno a Vitalità causato dall'arma viene dimezzato.

\subsubsection{Fiancheggiare} \index{Fiancheggiare}\label{fiancheggiare}

se due personaggi sono attorno allo stesso bersaglio ma non sono a fianco tra loro prendono un Bonus nella prova d'armi d'attacco.

Al massimo ci possono essere 4 personaggi attorno ad una creatura di taglia media che prendono il bonus di fiancheggiare.

Se tirando una ipotetica riga che collega i due personaggi questa attraversa in pieno il quadretto dell'avversario allora c'è la situazione di fiancheggiamento.

\bigskip

Esempio di fiancheggiamento\index{Esempi di Fiancheggiamento}

\medskip

\begin{tabularx}{0.45\textwidth}{lll}
\toprule
A &  G &  D\\
B & \textbf{X}  &  E\\
C &  H &  F\\
\end{tabularx}

\bigskip

In questo schema il fiancheggiamento è preso dalle coppie: A-F, B-E, C-D, G-H

\bigskip

Se la creatura può fronteggiare più creature contemporaneamente queste non godranno del bonus di fiancheggiamento.


\subsubsection{Prendere la Mira (cecchino)} \index{Prendere la Mira (cecchino)}\label{cecchino}

se dedichi 5 PA a prendere la mira ottieni un Bonus alla Prova d'Armi per Attaccare.

\subsubsection{Usare un'arma da lancio mirando ad un avversario impegnato in combattimento} \index{Usare un'arma da lancio mirando ad un avversario impegnato in combattimento}\label{usarearmalancioinmischia}

non è facile prendere la mira corretta e non colpire il proprio compagno. Hai Svantaggio alla Prova d'Armi per Attaccare. Se la prova di attacco ha un MS di -6 o meno hai colpito il compagno.

\subsubsection{Usare un'arma da lancio sotto minaccia} \index{Usare un'arma da lancio sotto minaccia}\label{usarearmalanciosottominaccia}

usare un'arma da lancio come arco, balestra o pugnale (che si vuole lanciare) mentre si è minacciati in mischia concede all'avversario Vantaggio nella prova d'Armi per difendersi.

\subsubsection{Difesa totale} \index{Difesa totale}\label{difesatotale}

costa 8 PA, non puoi eseguire nessun attacco con armi o lancio di incantesimo, guadagni Vantaggio alla prova d'Armi per difenderti.

\subsubsection{Disingaggiare} \index{Disingaggiare}\label{disingaggiare}

costa 2 PA per 1.5 metri che ti sposti. Non causi attacchi di opportunità.\index{Fare un passo}

\subsection{Manovre Opzionali in Combattimento}\label{azioniopzionaliincombattimento}

Queste Azioni di combattimento sono a discrezione dell'Arbitro che può concederle o meno.

\subsubsection{Disarmare*}\index{Disarmare}\label{disarmare}

entrambi eseguite una prova d'Armi, chi riesce con il MS maggiore disarma l'avversario. Costa 4 PA

\subsubsection{Finta*} \index{Finta}\label{finta}

entrambi eseguite una prova d'Armi, chi riesce con il MS maggiore ha Vantaggio nella prova d'Armi successiva per difendersi. Costa 4 PA

\subsubsection{Spingere un avversario*} \index{Spingere un avversario}\label{spingereavversario}\hypertarget{spingereavversario}{}

eseguite entrambi una prova di Corpo con un Bonus per ogni taglia di differenza maggiore.

Chi vince la prova con il margine maggiore puo' spingere l'avversario fino a 30 cm per valore di margine di differenza. Costa 4 PA

\subsubsection{Afferrare un avversario*}\index{Afferrare un avversario}\label{afferrareunavversario}

eseguite entrambi una prova di Corpo con un Bonus per ogni taglia di differenza maggiore.

Costa 4 PA fare e mantenere e liberarsi dalla presa. Si considera che chi afferra è anche afferrato ed abbia almeno una mano occupata nell'afferrare.

Muovere una creatura afferrata richiede \hyperlink{spingereavversario}{Spingere un avversario}.

Ogni contendente può attaccare l'altro afferrato con un arma piccola usabile con una mano o con pugni e calci.

\subsubsection{Fare cadere un avversario*} \index{Fare cadere un avversari}\label{farecadereavversario}

eseguite entrambi una prova di Corpo con un Bonus per ogni gamba/zampa di differenza maggiore.

Chi ha il MS maggiore fa cadere prono l'avversario. Costa 4 PA


\subsection{Cavalcature}\index{Combattimento a cavallo}\index{Cavallo}\label{cavalcature}

\begin{changemargin}{0.3cm}{0.3cm}\begin{enfasi}{
	- E ti puoi trovare un'altra moglie!
	
	- Ah, questo sì. ma il guaio è che mi ha portato via il fucile e il cavallo! Peccato, era così bella, io mi ci ero affezionato. Le davo qualche frustata, ma lei non ci faceva caso.
	
	- Chi, tua moglie?
	
	- No, la mia cavalla. A trovare un'altra moglie si fa presto, ma una cavalla come quella non la ritrovo più. (Ombre rosse, film 1939)}\end{enfasi}\end{changemargin}\medskip

Una cavalcatura ha le sue Azioni e di norma sono usate per spostarsi o per reagire ed ubbidire ai tuoi comandi.

Una cavalcatura agisce nel tuo round e sei tu a decidere quando esegue le sue Azioni rispetto alle tue. Non tira l'iniziativa, usa la tua.

Per fare muovere o attaccare una cavalcatura devi usare i tuoi Punti Azione.

Gli attacchi verso un personaggio a cavallo (o cavalcatura in genere) se non dichiarati diversamente mirano al cavaliere e non al cavallo.

Nella descrizione della Cavalcatura è indicato quanti metri fa per PA usato (solitamente 3 o più).

\subsubsection{Situazioni e regole}\label{cavallosituazioniregole}

\begin{itemize}
\item
Ogni qual volta la cavalcatura è colpita il cavaliere deve effettuare una prova di Cavalcare o essere disarcionato dalla cavalcatura.

Se la cavalcatura è da "guerra" (addestrata al combattimento) la prova 2 bonus.

\item
Combattere da posizione sopraelevata concede una Penalità alla prova d'Armi per difendersi della creatura.

\item
Salire o Scendere dalla cavalcatura costa 4 PA se si ha la competenza Cavalcare, altrimenti 8 PA.

\item
Se una magia o situazione sposta (bruscamente) la cavalcatura contro la tua volontà devi effettuare una prova di Cavalcare o venire disarcionato
\end{itemize}


\subsubsection{Essere disarcionato}\label{esseredisarcionato}

Se vieni disarcionato esegui una prova di Corpo. Se riesci cadi in piedi, se fallisci cadi prono e se il fallimento è di 5 o più subisci 1d6 di danno per la caduta a Vitalità.


\subsubsection{Controllare una Cavalcatura}\label{controllocavalcatura}

Mentre sei in sella, hai due scelte:

\begin{itemize}
\item puoi dare ordini alla tua cavalcatura
\item permettergli di agire da sola.
\end{itemize}

Cavalcature particolarmente intelligenti tendono a privilegiare l'autonomia di azione piuttosto che essere comandati.

Puoi controllare una cavalcatura solo se questa è stata addestrata ad accettare un cavaliere. Si presume che cavalli addestrati, muli e simili creature abbiano ricevuto tale addestramento.

L'iniziativa di una cavalcatura controllata cambia per corrispondere a quella di chi la cavalca. Si muove secondo le tue indicazioni e ha solo due opzioni di Azione: Muoversi, Attaccare.

Fare eseguire una Azione ad una cavalcatura costa l'equivalente Azione al cavaliere.

Se la cavalcatura è intelligente avere un cavaliere non restringe le azioni che la cavalcatura può effettuare e questa si muove e agisce come desidera. Potrebbe fuggire dal combattimento, lanciarsi all'attacco e divorare un nemico ferito gravemente, o agire in qualche altro modo contro la tua volontà.

\end{multicols}


\pagebreak



\begin{multicols}{2}
	
\end{multicols}

\pagebreak

Settings v1:  in un futuro non troppo distante inquinamento, guerre, cambiamento climatico con carestie ed inondazioni hanno reso la terra oramai inabitabile. poi improvvisamente vengono scoperte leghe metalliche e cristalline che permettono di imbrigliare e conservare l'energia a livelli neanche immaginabili.
Questo progresso enfatizzo ancora di piu' la divisione sociale. Le poche corporazioni che avevano i nuovi materiali non fecero altro che arricchirsi e depauperare chiunque avesse risorse da vendere.
Piccole enclavi vivevano nel lusso ed in un ambiente idilliaco mentre il 98\% della popolazione si arrabattava come poteva e cercava di resistere ad un pianeta che piu' non lo voleva.
Ormai sull'orlo dell'estinzione la OXF Corp dichiaro' di aver migliorato il proprio materiale in grado di purificare l'aria, rendendo capace un solo cristallo di poter generare l'aria per una intera casa, e non una sola persona.
Durante lo show in mondo visione dove veniva mostrato come veniva aggiornato il "cristallo d'aria" avvenne quello che tutti poi chiamarono la Frattura. 
Il cristallo d'aria incomincio' a risonare, ad emettere un sordo suono e generare onde armoniche sempre con maggiore intensita'. Mentra la terra intorno incominciava a tremare la sede della OXF Corp si spezzo in due e da qualche laboratorio segreto in profondita' una luce intensa e pura sali' alta nel cielo. Per diversi minuti fu solo il panico a dominare gli animi finche' in mezzo a quella luce che si andava dissipando apparve una figura a mezzaria. I lineamenti erano umani, ma non era umana. La carnagione era dorata, i capelli come argento, le mani affusolate ed i lineamenti delicati, le orecchie stranamente lunghe ed a punta.
Il volto basso, gli occhi chiusi quasi fosse in profonda meditazione od addirittura morta.
Un attimo prima che la luce scomparisse del tutto quella creatura emisi una intensissima luce dorata per poi scomparire.
Quello che avvenne dopo fu la vera e propria "Frattura".

Il cielo sembro' aprirsi lasciando intravedere le profonde e lontane stelle, la spaccatura sotto la OXF Corp divenne un crepaccio senza fine.
Entrame le oscurita', diverse eppure simili, una difronte all'altra incominciarono a pulsare e migliaia di esseri dall'una e dall'altra parte incominciarono a camminare sulla terra.

Questo fenomeno, la Fratuttura, non fu un caso isolato, ovunque nel mondo dove ci fosse stato un cristallo purificatore avvenne una Frattura, magari di potenza minore, ma appastanza per distruggere diversi quartieri e richiamare altre creature dal cielo e dalle viscere della terra.

Cio' che usciva dal cielo non erano angeli, come cio' che usciva dalla terra non erano demoni...

passano secoli..civilta' decade tranne per piccole citta' mantenute da corporazioni


Setting v2: dark e dangerous.. ma come?
la superfice e'  tossica e perrennemente in penombra, abitata predatori di ogni specie, il grande obelisco sparge.., le poche ricchezze e tesori rimasti, se non una fugace salvezza risiede nelle profondita', nelle oscure catacombe di antiche civilta' oramai scomparse.

competenza armi= Competenza combattimento


Razze: umani, nani, elfi, mezz'orchi, mezz'elfi, ... trovare qualche razza interessante e particolare. no scurovisione  

La scurovisione ha una portata espressa in metri. 

Entro quei metri: 
- le aree di luce fioca diventano di luce intensa
- le aree di oscurità diventano di luce fioca

Nelle aree modifice da scurovisione, NON si leggono i colori.  è tutto in scale di grigio. questo è un dettaglio importante perché è impossibile distinguere una chiazza di sangue da una di olio, et simili. è per questro motivo che le razze con scurovisione non amano comunque vivere al buio

Caratteristiche: Mente, Corpo, Volonta'. tiri 2d10. qualsiasi valore oltre 13 e' 13, qualsiasi valore sotto 4 e' 4.
Le caratteristiche danno un bonus alla prove pari al valore -10 , quandi la caratteristica e' superiore a 10. sono prerequisiti per avanzamenti di rami, competenza magica ed armi (non puoi avere comp. armi/magica superiore al valore della statistica), scelta ed uso incantesimi. A seconda del ramo prendi dei bonus o malus (solo rami molto avanzati, sempre a fare il mago aumenti mente)
quando si fanno prove di statistiche: di base si cerca di fare meno prove possibili e mai chiamate dal giocatore, si preferiesce fare prove di competenza. le prove di statistiche sono l'ultima ratio se non c'e' altro.


equipaggiamento: creare una lista essenziale di oggetti e relativo ingombro

ingombro: a slot. non vado a stabilire quanti slot tiene lo zaino, il sacco, la giacca, la cintura.. ma quanti slot puo' portare la creatura. il valore di slot e' uguale a corpo. si conta cio' che si porta, non cio' che si indossa (armatura)
armatura pesante 8, media 6, leggera 4
scudo 4
arma a due mani 6, arma media 4, arma da tiro 4, arma leggera 1
ogni 1000 monete = 1 ingrombro

armi magiche: possono fare piu' male e dare malus alla prova di comp combattimento per evitare. +1 al danno, +1 danno e 1 penalita' prova di combattimento (per difesa), +2 danni e 2 penalita'(questa e' veramente forte)
armature magica: assorbono piu' danno e danno bonus alla prova di combattimento. +1 assorbito, +1 bonus a comp. combattimento, +1 ass e +1 comp, 
scudo: +1 comp combattimento, +1 bonus comp combattimento
altri oggetti magici: fare una lista e ridurre e di tanto

Kismet: ogni giocatore tira su una tabella casuale che indica come morira' il personaggio. dove, come, quando (????)

statistica fortuna: si usa per abbassare il tiro fatto. recuperi 1 quando fai un fallimento critico. dichiari prima e spendi prima.
Ha un valore di 1d4 per sessione.

classi/rami: non serve perdersi, bastano 12 rami base, 8 intermedie, 6 avanzate, 4 elite
in una ramo si rimane finche' non si soddisfano i requisiti dei rami superiori
quindi le abilita' aumentano sempre, anche rimanendo nello stesso ramo
i punteggi aumentano con gli errori: se tiri 19/20 (sui 2d10) oltre al fallimento o fallimento critico segni con un pallino la competenza

arrivati a 5 al primo downtime fai una prova con svantaggio ma al contrario: tiri 3d10, scarti il piu' alto ed il risultato della somma deve essere superiore al valore della competenza. se riesci a fare 3 prove con successo alzi di 1 il punteggio della competenza.

i rami presi aggiungono "competenze professionali". aumentano resistenza, competenze armi, competenze magica Per passare da un ramo ad un altro devi avere un valore minimo in certi punteggi

per passare da un ramo ad un ramo diverso perdi 1 punto in ogni competenza che non sia armi e magia ed investi 500mo per tuo punteggio piu' alto competenza. del nuovo ramo prendi i resistenza e le nuove compenteze che non hai su cui distribuisci 6 punti tra tutte

Ogni volta che la somma dei 3 punteggi scelti alla presa del ramo diventano multipli di 3, tranne comp armi e comp magica, a parte la prima assegnazione o presa del ramo, prendi nuovamente i punti resistenza, comp. combattimento e comp. magica

la prima volta che prendi un ramo: scegli 3 competenze, saranno le competenze chiave per capire quando puoi riprendere il ramo.
la prima volta che prendi un ramo: scegli 1 tra le competenze, la specializzazione, che non sia quella relativa ad armi o incantamenti e su questa hai vantaggio nei tiri, su questa competenza non metti punti, parte a 4
sulle altre competenze hai 8 punti da distribuire tra tutte le competenze della classe, nessuna puo' essere sopra i 6 alla partenza

non c'e' il passaggio di livello, semplicemente le competenze (compreso armi e magica) aumentano con l'uso. la resistenza si prende ogni volta che riprendo lo stesso ramo (multiplo di 3 nella somma delle competenze preferite) oppure prende un nuovo ramo

lista compentenze

rissa non serve vare una prova di comp. armi ma direttamente di rissa, lo svantaggio di rissa e' che fa poco danno
resistenza=vitalita'



competenze:													modificatore
armi piccole, armi medie, armi a 2 mani, armi da tiro		corpo
atletica													corpo
Travestimento													mente
cavalcare													volonta'	
conoscenza dei bassifondi									mente
Conoscenza														mente			
diplomazia													volonta'		
furtivita'													mente	
incantamento												mente
indimidire 													dipende		
raggirare													volonta'
linguaggi													mente
borseggiare												corpo				
Pronto soccorso													mente	
mercanteggiare 												volonta'		
nuotare 													corpo
orientamento												mente		
osservare													volonta'
rissa 														corpo		
sopravvivenza												volonta'
storia														mente	
Mercanteggiare													mente			




considerando che ogni 3 punti si riprende il ramo (barcaiolo IV..) non serve esagerare con i punteggi di comp armi, magia, resistenza.

la resistenza di base e' 4 per rami magici o per chi ha 1 arma e basta, 6 per chi ha 2 armi, 8 per combattenti puri ovvero per chi' ha almeno 3 armi. 
armi e' pari a armi note +2

comp armi / combattimento e' pari a numero di armi +1 alla partenza (e modificatore di corpo)

incatamento se presente parte con un valore di +3




incantesimi:

ispirarsi a nome verbo, ma fai te gli accoppiamenti e stabilisci te cosa succede per ogni critico 
i livelli (II, III..) aggiuntivi hanno prerequisiti di consocenza magica maggiore e statistica maggiore, causano una maggiore perdita di resistenza ma garantiscono maggiore risultato, il tempo di lancio (iniziativa) e' piu' lenta piu' e' potente l'incantesimo.

Attacco Elementale I, II, III, IV, V
Creare Elemento I, II, III
Creare Corpo  ???
Creare Mente  ???
Muovere Elemento I, II
Muovere Corpo (feather fall, levitate, volare..)
Muovere Elemento
Proteggere Corpo
Proteggere Elemento 
Proteggere Mente
Distruggo Elemento (piccole cose.. poi piu' grandi)
Distruggo Corpo (maledizione, penalita'..)
Ripara Corpo
Ripara Elemento
Conoscere Corpo
Conoscere elemento
Conoscere Mente
Ripara Mente
Alterare Mente (charmi, compulsion..)
Distruggo Mente (come attacco)
Riparare Spirito, da Mercanteggiare se serve, se ci sono mostri effetti contro lo spirito...


Creo		mente (bonus a prove mentali), elemento
Muovo		corpo, elemento, mente (possessione ?)
Trasformo	
Altero
Proteggo	corpo, mente, elemento, spirito
Distruggo	corpo, mente, elemento, spirito
Attacco		elemento, spirito
Riparare	corpo, mente, elemento, spirito
Conoscere	corpo, mente, elemento, spirito

Corpo
Mente
Elemento
Spirito, cio' che riguarda la non vita e l'anima

Alterare
”Rotto il grimaldello? Lucchetto difficile ed antipatico ? Osserva come si apre al mio umile tocco”: Alte-
rare - Materia
”Potenti spiriti guerrieri infondete coraggio ai compagni”: Alterare - Spirito
”Dalle somme biblioteche io chiamo il Silenzio!”: Alterare - Mente / Alterare - Energia. O non fai sentire il
suono, o lo fai diminuire.
”Per i grandi mammuth lanosi, il freddo non mi fa nulla”: Alterare - Corpo. Per avere resistenza al freddo.
”Che Re Gorilla ti dia la forza di un esercito”: Alterare - Corpo
”Dal deserto delle 10 ombre chiamo il miraggio del muro”: Alterare - Mente.
Attaccare
”Possa tu bruciare delle fiamme dell’inferno”: Attaccare - Fuoco
”Fragili sottili deboli, una ad una rompo le tue ossa”: Attaccare - Corpo / Distruzione - Corpo
”Emicrania ? e’ solo l’inizio”: Attaccare - Corpo
”Che il fuoco della fucina scaldi la tua arma”: Attaccare - Materia
Creare
”Rotto il grimaldello? Ho il migliore dei setti regni”: Creare - Materia
”Nessuna mela è più buona di quella che puoi creare tu”: Creare - Materia
”Oh piccola torcia esplodi di luce in questa tetra caverna”: Creare - Energia. In questo caso non può esserci
danno non essendo il Verbo Attaccare.
”A me tomo delle idee! Illuminami il pensiero”: Creare - Mente. In questo caso può essere usato per ritirare
una prova con un bonus. Solo Conoscenza - Mente può dare la soluzione.
”Al più pavido degli eroi concedo il coraggio del leone”: Creare - Spirito. Può essere dato solo a chi il coraggio
non lo ha, altrimenti e’ Alterare.
”Sommi sapienti aiutatemi a chiamare Colui che Striscia nell’Oscurità”: Creare - Spirito. Non si evocano crea-
ture reale, ma si evocano simulacri, per questo si usa Spirito.
Riparare
”Rotto il grimaldello? Una mia carezza è più efficace di un fabbro” : Riparare - Materia
”Nessuna mela è troppo marcia, riempila del tuo amore e mangiala”: Riparare - Materia
”Possano le tue ferite rinsaldarsi, possa il tuo cuore riposare. Possano le mani della somma guaritrice placare
le tue sofferenze”: Riparare - Corpo
”Libera il cuore dalla paura! Che il maleficio delle immonde creature scompaia”: Riparare - Spirito. In questo
caso si porta alla condizione originaria, togliendo l’effetto di paura
Distruggere
”Rotto il grimaldello? Peggio per il lucchetto. Il mio tocco è quello dei millenni”: Distruggere - Materia
”Osserva il vuoto, perditi dentro, scompari nel nulla. Cosa hai fatto?”: Distruggere - Mente. Può fare perdere
Azioni
”Trema, annaspa, striscia, muori. Dal tuo posto non ti sposti”: Distruggere - Corpo. Per fermare il movimento
”Il coraggio non si da a chi non lo ha. Piccolo codardo, scappa dalla mamma”: Distruggere - Spirito. Può
essere usato per diminuire o annullare un Essenza che conferisce coraggio. ”Fiat Tenebris! Ogni luce muoia”:
Distruggere - Energia
Conoscere
”Rotto il Grimaldello? Ogni lucchetto ha un difetto ed un punto debole. Dimmi quale è il tuo e ti libererò
dal giogo della chiusura”: Conoscere - Materia
”Grande Rutte, concedimi il tuo sguardo di mille avventure. Come affrontare il mio avversario”: Conoscere -
Corpo.
”Dal labirinto chiamo il Minotauro. Ti ordino di dirmi la strada più breve per le Sale di Mazurdas”: Conoscere
- Materia
”I tuoi pensieri nei miei pensieri. I tuoi pensieri sono i mei pensieri”: Conoscere - Mente
”Possa il sommo curatore aiutarmi a capire che veleno di affligge”: Conoscere - Corpo
”Se non fuoco, se non fulmine, immonda creatura quale è il tuo punto debole”: Conoscere - Spirito



allineamento:  Legge, Chaos, neutrale


per fare una prova tiri il 2d10 e devi fare meno della statistica o competenza. 


punti ferita/Vitalita': a seconda della razza ed ogni ramo che prendi lo aumenta di un valore dipendente dal ramo scelto, ma l'aumento e' sempre poco (1-4)
resistenza base: corpo + ramo/rami. quindi da 4 a 12/13 a salire. 
Vitalita':  reggi un certo numero di colpi. un colpo critico (ovvero difesa fallita di almeno 6) causa +2 ferite
parti con un numero di Ferite pari a Corpo. i rami possono aumentare le ferite sostenibili. 
recupero resistenza: ogni notte recuperi 2 (o corpo/2 ???)

morte: arrivi a 0, svenuto. se sei a -1, metti un check su ferito, ogni round fai una prova di corpo se fallisci perdi -1, se riesci tre check di fila ti stabilizzi e torni a 0. arrivato a -10 sei morto.

condizioni: prendere da obss e ridurre sensibilmente  e semplificare bonus e malus, aggiungere condizioni solo se necessario

azioni in round: in un round si possono usare fino a 10 punti azione. attaccare con arma a 2 mani costa 8 punti azione, arma a 2 mani costa 6 punti azione, attaccare con un arma leggera costa 4 punti azione. gli incantesimi solitamente costano da 4 punti azione in su in base  al livello (II, III; IV..) ogni livello in piu' aumenta di 1 l'iniziativa, spostarsi costala meta' della in base alla distanza coperta
i mostri in base alla loro taglia e cosa fanno
per altre azioni vedi obss e traduci in p.a.

distanze: a metri

iniziativa: e' pari ai punti azione usati. chi ne usa di meno incomincia per primo. in caso di parita' si controlla mente, volonta', corpo in quest'ordine. se ritarti "consumi" punti azione ad aspettare

competenze di base: ogni ramo modifica/o meno il punteggio di Combattimento o Magia.

combattimento: chi attacca non tira nulla, dichiara solo che arma usa e se usa manovre. chi difende tira sotto il suo valore di combattimento, questo valore di combattimento puo' avere svantaggi dati da abilita' dell'attaccante (rami combattenti avanzati). se tira sotto allora para/evita il colpo, se tira sopra viene preso.
Si presume che l'attacco colpisce sempre se la difesa non funziona bene. Questo significa che le classi combattenti non aumentano l'attacco ma solo la difesa. alcuni rami avanzati di combattimento danno delle penalita' alla difesa avversaria.
il danno e' 1d6 per armi piccole, 1d8 armi medie, 1d10 armi a 2 mani, rissa 2 danni.
combattere con un arma che non conosci da svantaggio


armi: armi piccole 1d6, armi medie 1d8, armi grandi 2 mani 1d10 +bonus corpo. arma piccola ha requisito forza 6, media 9, grande 12, altrimenti svantaggio nell'uso

una difesa particolarmente riuscita (almeno -6) puo' dare svantaggio al tiro successivo, un -9 potrebbe fare tirare con un solo dado
quando difendi  a seconda di quanto bene difendi, ovvero se hai un margine di -3,-6,-9.. rispetto alla tua prova ottieni dei bonus, azioni movimento, penalita' all'attacco successivo

le manovre funzionano con lo stesso principio. quando effettuo una manovra e l'altro riesce nel tiro di combattimento allora la manovra non riesce e io devo fare un tiro di combattimento, se fallisco prendo gli effetti della manovra

armature: riducono il danno. leggere riducono di 2, medie riducono di 3, pesanti riducono di 4. danno una penalita' alle prove di competenza. Non si fa magia con armatura se non solo a contatto. requisito di corpo: 8, 12, 
scudo: fai la prova di difesa con +1 (da 1 bonus). non segno che da penalita' ma premo sull'ingombro 4


magia: deve essere lanciare piu' difficile lanciare incantesimi e consumare "risorse" (resistenza/energia/stamina..)

certi rami magici possono privilegiare certe scuole di magia. lavorare su liste e ridurre e tanto. l'idea di base e' per esempio crea fuoco puo' diventare a seconda del punteggio del tiro altre cose, il valore influenza la distanza, AoE, danno, se e' un raggio o esplosione e che raggio...  Pochi incantesimi ma che si evolvono

lanciare un incantesimo: tirare a secondo dall'incantesimo su capacita' magica e avere un punteggio minimo di  mente, corpo o volonta'.  Ogni punteggio -3 rispetto alla prova, es. capacita' magica 13 e tiro 8, potenzi un fattore dell'incantesimo (distanza, AoE, danno, tipo di effetto..). Gli incantesimi hanno un punteggio minimo di mente/corpo/volonta' per essere tirati ed anche di capacita' magica
Ogni incantesimo ha degli attributi, danno, distanza, aoe, durata ed un punteggio minimo di competenza magica e corpo/mente/volonta'. se il tiro riesce bene puoi potenziare un attributo presente ma non darlo/aggiungerlo se questo e' assente. se lancio l'incantesimo crea fiamma, inc. base difficolta' 1, se faccio un ottimo tiro potro' potenziare la durata e aoe (l'area di luce che fa) ed il danno, ma non posso aggiungere distanza perche' e' un attributo assente.
ci sara' poi l'incantesimo globo di fuoco, la versione base della palla di fuoco, questa ha piu' attributi ma ad esempio non ha durata, o meglio l'istantanea non puo' essere migliorata.
Mercanteggiare di aggiungere attributi assenti quando il tiro e' veramente buono, ovvero tiri veramente basso, direi almeno un -6 per aggiungere un attributo a livello base (3 metri)
si possono spendere risorse aggiuntive per potenziare l'incantesimo, ovvero abbassare il risultato del dado
scuola di magia: raggruppa gli incantesimi per tipo
ci si puo' specializzare in un "verbo" di magia: hai +3 alla prova, ma -3 a tutte le altre scuole

lanciare incantesimi mentre si combatte o si e' stato colpito: si puo' fare ma il primo critico si ignora

se l'incantesimo riesce nel lancio non c'e' TS. alcuni incantesimi possono avere una prova di difesa per essere evitati/dimezzati l'effetto

questo implica che non ci saranno mai incantesimi super potenti in tutto..

quanti incantesimi lanciare:  a piacere, ma ogni volta che lanci lo stesso il costo in vitalita' aumenta. 

quando lanci un incantesimo e fallisci la prova non succede nulla, ma l'incantesimo l'hai lanciato e quindi perdi la vitalita' 
quando lanci un incantesimo e fallisci con 19-20 la prova succedono cose  brutte
quando lanci l'incantesimo e fai veramente basso 2, non perdi vitalita'


mostri:
se il mostro tira la sua difesa c'e' il rischio che riesca sempre ad alti livelli. considerare abilita' che abbassano la difesa, l'idea di base e' che se un mostro ha difesa  o piu' deve essere di un livello tale da dover affrontare pg con rami che danno penalita' alla difesa
fare una lista minima di 20 mostri tipici e classici, verificare bx per compatibilita'


--------------------------------------------------------------------------------



- fare 2 o 20  successo critico o fallimento  critico

- il giocatore dichiara cio' che fa e' solo il master a stabilire se serve una prova. 

- il tempo e' un fattore, tabelle random per incontri basati su tempo trascorso, si computa il tempo reale.

Lower HP all around.
Less access to healing.
More basic fantasy races as PCs (elves, dwarfs, etc). Less "modern" ones (dragonborn, tiefling).
More dungeons.
No "at will" spell casting.
Lack of a universal skill system.
More poison, less disease.
THACO, or some variant where lower AC is better.
rulings over rules
player skill
zero to hero player progression


\end{document}