\documentclass[12pt,a4paper,twoside,openany]{book}
\usepackage{quoting}
\usepackage{tcolorbox}
\usepackage{tikz}
\usetikzlibrary{shadows}
\usepackage{multicol}
\usepackage{tocloft}
\usepackage{lmodern}
\usepackage{caption}
\usepackage[utf8]{inputenc}
\usepackage[T1]{fontenc}
\usepackage{setspace}
\usepackage[a4paper]{geometry}
\geometry{verbose,tmargin=2cm,bmargin=2cm,lmargin=2cm,rmargin=2cm}  %std
\setcounter{secnumdepth}{-1}
\usepackage{booktabs}
\usepackage{url}
\usepackage[italian]{babel}
\usepackage{setspace}
\usepackage{graphicx}
\usepackage{amssymb}
\usepackage{makeidx}
%\usepackage[allfiguresdraft]{draftfigure}  %senza figure, deve rimanere alla riga 24
\usepackage{multirow}
\usepackage{titlesec}
\usepackage[unicode=true, bookmarks=true,
pdftitle={Dark Catacomb - DKC},pdfauthor={Andres Zanzani},
breaklinks=false,pdfborder={0 0 1},backref=section,colorlinks=false]
{hyperref}
\hypersetup{colorlinks=true,linkcolor=blue,pdfcreator={LaTeX}}
\usepackage{bookmark}
\usepackage{yfonts}
\usepackage{auncial}
\usepackage{ragged2e}


\usepackage{fontspec}
\setmainfont[Path=./, BoldItalicFont=Soutane Bold Italic.ttf, ItalicFont=Soutane Italic.ttf, BoldFont=Soutane Bold.ttf, Ligatures=TeX, Scale=0.94]{Soutane Regular.ttf} 


\usepackage{wrapfig}
\usepackage{fancyhdr}
\usepackage{tcolorbox}
\tcbuselibrary{skins}
\tcbset{colback=brown!10, fonttitle=\scshape}
\usepackage{imakeidx}
\usepackage{cancel}

\def\CountIndexOccurrences#1{%
	\expandafter\newcount\csname #1\endcsname%
	\expandafter\newcount\csname #1\endcsname%
	\def\indexentry##1##2{\expandafter\advance\csname #1\endcsname 1}%
	\IfFileExists{#1.idx}{\input{#1.idx}}{}%
}
\CountIndexOccurrences{OBSS}
\CountIndexOccurrences{Incantesimi}
\CountIndexOccurrences{Mostruario}
\CountIndexOccurrences{OggettiMagici}
\def\TotalBox#1{\vfill%
	\fbox{Ci sono \expandafter\the\csname #1\endcsname\ voci in questo indice}\par}
\makeindex[columns=3, title=Indice Analitico, intoc=true]
\makeindex[columns=3, name=Incantesimi, title=Lista degli Incantesimi, intoc=true]
\makeindex[columns=3, name=Mostruario, title=Lista dei Mostri, intoc=true]
\makeindex[columns=3, name=OggettiMagici, title=Lista degli Oggetti Magici, intoc=true]
\usetikzlibrary{shapes.misc,calc}
\definecolor{lightgray}{gray}{0.95}
\usetikzlibrary{shapes.misc,calc}
\definecolor{lightgray}{gray}{0.95}
\usepackage{fancyhdr}
\pagestyle{fancy}
\fancyhf{} 
\fancyhead[LE,RO]{\leftmark}
\fancyhead[RE,LO]{}
\fancyfoot[C]{\thepage}
\renewcommand{\sectionmark}[1]{\markboth{#1}{}}
\usepackage{xltabular}
\usepackage{tabularx}
\usepackage{pdfpages}
\usepackage{hyperref}
\usepackage{tikz}
\usepackage[absolute,overlay]{textpos}
\usepackage{etoolbox}
\usepackage{soul}
\raggedbottom
\usepackage{array}
\newcolumntype{L}[1]{>{\raggedright\let\newline\\\arraybackslash\hspace{0pt}}m{#1}}
\newcolumntype{k}[1]{>{\centering\let\newline\\\arraybackslash\hspace{0pt}}m{#1}}
\newcolumntype{R}[1]{>{\raggedleft\let\newline\\\arraybackslash\hspace{0pt}}m{#1}}
\newcolumntype{D}[1]{>{\centering}m{#1}}
\newcolumntype{M}[1]{>{\centering\arraybackslash}m{#1}}
\titleformat{\section}{\filcenter\huge\bfseries\accanthis}{\thesection}{1em}\textsc{}
\titleformat{\subsection}{\Large\bfseries\accanthis}{\thesubsection}{1em}\textsc{}
\titleformat{\subsubsection}{\normalsize\bfseries\accanthis}{\thesubsubsection}{1em}\textsc{}
\def\changemargin#1#2{\list{}{\rightmargin#2\leftmargin#1}\item[]}
\let\endchangemargin=\endlist
\setcounter{tocdepth}{3}
\newtcolorbox{narratore}{
	enhanced, % enable advanced settings
	%left = 3mm,
	%width=0.45\textwidth,
	left = 9mm, % pushes text away from the left edge by 10mm
	sharp corners, % disables rounded corners
	rounded corners = southeast, % "round" the bottom right corner
	arc is angular, % make the "round" corner an angle
	arc = 3mm, % controls corner cut
	boxrule=0.6pt, % sets box line thickness
	underlay={%
		\path[fill=black] ([yshift=3mm]interior.south east)--++(-0.4,-0.1)--++(0.1,-0.2); % triangle
		\path[draw=black,shorten <=-0.05mm,shorten >=-0.05mm] ([yshift=3mm]interior.south east)--++(-0.4,-0.1)--++(0.1,-0.2); % triangle edge
		\path[fill=gray!50!black,draw=none] (interior.south west) rectangle node[brown!10]{\Huge\bfseries ?!} ([xshift=8mm]interior.north west);
	},
	drop fuzzy shadow }

\newtcolorbox{enfasi}{
	enhanced,
	arc=5pt,
	boxrule=0.3pt
} 

\usepackage{zref-savepos,graphicx}
\newcommand{\filltopageendgraphics}[2][]{%\filltopageendgraphics[width=.5\linewidth]{image-a}
	\par
	\zsaveposy{top-\thepage}% Mark (baseline of) top of image
	\vfill
	\zsaveposy{bottom-\thepage}% Mark (baseline of) bottom of image
	\smash{\includegraphics[keepaspectratio=true,height=\dimexpr\zposy{top-\thepage}sp-\zposy{bottom-\thepage}sp\relax,#1]{#2}}%
	\par
}


\usepackage{accanthis}
\usepackage[framemethod=TikZ]{mdframed}


\begin{document}
	
\def \versione {0.01}
\thispagestyle{empty}
 
{\Huge \begin{center}
		Dark Catacomb
\end{center}}

\vfill
\begin{center}
	\Large{\color{black} Fantasy Adventure Game}
\end{center}

\pagebreak

	
\bigskip
Non temere l'ignoto, affrontalo con rispetto.
	
	\vspace{\fill}
\begin{center}\textbf{\versione} - \today\end{center}
\thispagestyle{empty}


\newpage~\thispagestyle{empty}%%\newpage~\thispagestyle{empty}


\newcommand{\riga}{\rule{\textwidth}{0.4pt}}


{\Huge \begin{center} Dark Catacomb\end{center}}

\bigskip

\begin{center}{\LARGE Manuale per Giocatore e Narratore}\\ \end{center}

{\large \begin{center} Guida e Regole per il Gioco di Ruolo Fantasy \end{center}}

\begin{center}di \end{center}

{\LARGE \begin{center} Andres Zanzani \end{center}}

\vspace{2cm}


\vfill

\begin{mdframed}[roundcorner=10pt]

\medskip

\textbf{Playtesting}: ...to be done...

\bigskip

\begin{flushleft}\textbf{Condizioni d'uso}: Dark Catakomb, DKC, è un marchio registrato di Andres Zanzani (azanzani@gmail.com).
\end{flushleft}

\vspace{0.5cm}


\medskip

\end{mdframed}

%}%
%}

\pagebreak

\setcounter{page}{1}

\begin{multicols}{2}
{\small \tableofcontents{}}

\end{multicols}

\vfill

\begin{changemargin}{0.3cm}{0.3cm}\begin{tcolorbox}
"Nel mezzo del cammin di nostra vita\\
mi ritrovai per una selva oscura\\
ché la diritta via era smarrita.\\
Ahi quanto a dir qual era è cosa dura\\
esta selva selvaggia e aspra e forte\\
che nel pensier rinova la paura!\\
\end{tcolorbox}\end{changemargin}

\pagebreak

\section{Introduzione}
Razza\\
Caratteristiche\\
rami base\\
rami avanzati\\
Punti Fortuna\\
Punti Fato\\
Karma\\
Competenze\\
Costruiamo il personaggio\\
Regole per le competenze\\
Combattimento\\
Nascondigli e coperture\\
liste armi ed armature\\
abilita' ???\\
magia\\
incantesimi\\
equipaggiamento\\
veleni e droghe\\
movimento\\
oggetti magici\\
mostruario\\
Condizioni\\
Scheda\\

Settings v1:  in un futuro non troppo distante inquinamento, guerre, cambiamento climatico con carestie ed inondazioni hanno reso la terra oramai inabitabile. poi improvvisamente vengono scoperte leghe metalliche e cristalline che permettono di imbrigliare e conservare l'energia a livelli neanche immaginabili.
Questo progresso enfatizzo ancora di piu' la divisione sociale. Le poche corporazioni che avevano i nuovi materiali non fecero altro che arricchirsi e depauperare chiunque avesse risorse da vendere.
Piccole enclavi vivevano nel lusso ed in un ambiente idilliaco mentre il 98\% della popolazione si arrabattava come poteva e cercava di resistere ad un pianeta che piu' non lo voleva.
Ormai sull'orlo dell'estinzione la OXF Corp dichiaro' di aver migliorato il proprio materiale in grado di purificare l'aria, rendendo capace un solo cristallo di poter generare l'aria per una intera casa, e non una sola persona.
Durante lo show in mondo visione dove veniva mostrato come veniva aggiornato il "cristallo d'aria" avvenne quello che tutti poi chiamarono la Frattura. 
Il cristallo d'aria incomincio' a risonare, ad emettere un sordo suono e generare onde armoniche sempre con maggiore intensita'. Mentra la terra intorno incominciava a tremare la sede della OXF Corp si spezzo in due e da qualche laboratorio segreto in profondita' una luce intensa e pura sali' alta nel cielo. Per diversi minuti fu solo il panico a dominare gli animi finche' in mezzo a quella luce che si andava dissipando apparve una figura a mezzaria. I lineamenti erano umani, ma non era umana. La carnagione era dorata, i capelli come argento, le mani affusolate ed i lineamenti delicati, le orecchie stranamente lunghe ed a punta.
Il volto basso, gli occhi chiusi quasi fosse in profonda meditazione od addirittura morta.
Un attimo prima che la luce scomparisse del tutto quella creatura emisi una intensissima luce dorata per poi scomparire.
Quello che avvenne dopo fu la vera e propria "Frattura".

Il cielo sembro' aprirsi lasciando intravedere le profonde e lontane stelle, la spaccatura sotto la OXF Corp divenne un crepaccio senza fine.
Entrame le oscurita', diverse eppure simili, una difronte all'altra incominciarono a pulsare e migliaia di esseri dall'una e dall'altra parte incominciarono a camminare sulla terra.

Questo fenomeno, la Fratuttura, non fu un caso isolato, ovunque nel mondo dove ci fosse stato un cristallo purificatore avvenne una Frattura, magari di potenza minore, ma appastanza per distruggere diversi quartieri e richiamare altre creature dal cielo e dalle viscere della terra.

Cio' che usciva dal cielo non erano angeli, come cio' che usciva dalla terra non erano demoni...

passano secoli..civilta' decade tranne per piccole citta' mantenute da corporazioni


Setting v2: dark e dangerous.. ma come?
la superfice e'  tossica e perrennemente in penombra, abitata predatori di ogni specie, il grande obelisco sparge.., le poche ricchezze e tesori rimasti, se non una fugace salvezza risiede nelle profondita', nelle oscure catacombe di antiche civilta' oramai scomparse.

competenza armi= Competenza combattimento


Razze: umani, nani, elfi, mezz'orchi, mezz'elfi, ... trovare qualche razza interessante e particolare. no scurovisione  

La scurovisione ha una portata espressa in metri. 

Entro quei metri: 
- le aree di luce fioca diventano di luce intensa
- le aree di oscurità diventano di luce fioca

Nelle aree modifice da scurovisione, NON si leggono i colori.  è tutto in scale di grigio. questo è un dettaglio importante perché è impossibile distinguere una chiazza di sangue da una di olio, et simili. è per questro motivo che le razze con scurovisione non amano comunque vivere al buio

Caratteristiche: Mente, Corpo, Volonta'. tiri 2d10. qualsiasi valore oltre 13 e' 13, qualsiasi valore sotto 4 e' 4.
Le caratteristiche danno un bonus alla prove pari al valore -10 , quandi la caratteristica e' superiore a 10. sono prerequisiti per avanzamenti di rami, competenza magica ed armi (non puoi avere comp. armi/magica superiore al valore della statistica), scelta ed uso incantesimi. A seconda del ramo prendi dei bonus o malus (solo rami molto avanzati, sempre a fare il mago aumenti mente)
quando si fanno prove di statistiche: di base si cerca di fare meno prove possibili e mai chiamate dal giocatore, si preferiesce fare prove di competenza. le prove di statistiche sono l'ultima ratio se non c'e' altro.


equipaggiamento: creare una lista essenziale di oggetti e relativo ingombro

ingombro: a slot. non vado a stabilire quanti slot tiene lo zaino, il sacco, la giacca, la cintura.. ma quanti slot puo' portare la creatura. il valore di slot e' uguale a corpo. si conta cio' che si porta, non cio' che si indossa (armatura)
armatura pesante 8, media 6, leggera 4
scudo 4
arma a due mani 6, arma media 4, arma da tiro 4, arma leggera 1
ogni 1000 monete = 1 ingrombro

armi magiche: possono fare piu' male e dare malus alla prova di comp combattimento per evitare. +1 al danno, +1 danno e 1 penalita' prova di combattimento (per difesa), +2 danni e 2 penalita'(questa e' veramente forte)
armature magica: assorbono piu' danno e danno bonus alla prova di combattimento. +1 assorbito, +1 bonus a comp. combattimento, +1 ass e +1 comp, 
scudo: +1 comp combattimento, +1 bonus comp combattimento
altri oggetti magici: fare una lista e ridurre e di tanto

Kismet: ogni giocatore tira su una tabella casuale che indica come morira' il personaggio. dove, come, quando (????)

statistica fortuna: si usa per abbassare il tiro fatto. recuperi 1 quando fai un fallimento critico. dichiari prima e spendi prima.
Ha un valore di 1d4 per sessione.

classi/rami: non serve perdersi, bastano 12 rami base, 8 intermedie, 6 avanzate, 4 elite
in una ramo si rimane finche' non si soddisfano i requisiti dei rami superiori
quindi le abilita' aumentano sempre, anche rimanendo nello stesso ramo
i punteggi aumentano con gli errori: se tiri 19/20 (sui 2d10) oltre al fallimento o fallimento critico segni con un pallino la competenza

arrivati a 5 al primo downtime fai una prova con svantaggio ma al contrario: tiri 3d10, scarti il piu' alto ed il risultato della somma deve essere superiore al valore della competenza. se riesci a fare 3 prove con successo alzi di 1 il punteggio della competenza.

i rami presi aggiungono "competenze professionali". aumentano resistenza, competenze armi, competenze magica Per passare da un ramo ad un altro devi avere un valore minimo in certi punteggi

per passare da un ramo ad un ramo diverso perdi 1 punto in ogni competenza che non sia armi e magia ed investi 500mo per tuo punteggio piu' alto competenza. del nuovo ramo prendi i resistenza e le nuove compenteze che non hai su cui distribuisci 6 punti tra tutte

Ogni volta che la somma dei 3 punteggi scelti alla presa del ramo diventano multipli di 3, tranne comp armi e comp magica, a parte la prima assegnazione o presa del ramo, prendi nuovamente i punti resistenza, comp. combattimento e comp. magica

la prima volta che prendi un ramo: scegli 3 competenze, saranno le competenze chiave per capire quando puoi riprendere il ramo.
la prima volta che prendi un ramo: scegli 1 tra le competenze, la specializzazione, che non sia quella relativa ad armi o incantamenti e su questa hai vantaggio nei tiri, su questa competenza non metti punti, parte a 4
sulle altre competenze hai 8 punti da distribuire tra tutte le competenze della classe, nessuna puo' essere sopra i 6 alla partenza

non c'e' il passaggio di livello, semplicemente le competenze (compreso armi e magica) aumentano con l'uso. la resistenza si prende ogni volta che riprendo lo stesso ramo (multiplo di 3 nella somma delle competenze preferite) oppure prende un nuovo ramo

lista compentenze

rissa non serve vare una prova di comp. armi ma direttamente di rissa, lo svantaggio di rissa e' che fa poco danno
resistenza=vitalita'



competenze:													modificatore
armi piccole, armi medie, armi a 2 mani, armi da tiro		corpo
atletica													corpo
camuffarsi													mente
cavalcare													volonta'	
conoscenza dei bassifondi									mente
cultura														mente			
diplomazia													volonta'		
furtivita'													mente	
incantamento												mente
indimidire 													dipende		
ingannare													volonta'
linguaggi													mente
mani di fata												corpo				
medicina													mente	
mercanteggiare 												volonta'		
nuotare 													corpo
orientamento												mente		
osservare													volonta'
rissa 														corpo		
sopravvivenza												volonta'
storia														mente	
valutare													mente			




considerando che ogni 3 punti si riprende il ramo (barcaiolo IV..) non serve esagerare con i punteggi di comp armi, magia, resistenza.

la resistenza di base e' 4 per rami magici o per chi ha 1 arma e basta, 6 per chi ha 2 armi, 8 per combattenti puri ovvero per chi' ha almeno 3 armi. 
armi e' pari a armi note +2

comp armi / combattimento e' pari a numero di armi +1 alla partenza (e modificatore di corpo)

incatamento se presente parte con un valore di +3

rivoluzionario  4 resistenza, 2 comp armi
armi piccole, intimidire, osservare, storia, conoscenza dei bassifondi

mendicante 4 resistenza, 2 comp armi
armi piccole, valutare, ingannare, osservare, conoscenza dei bassifondi

barcaiolo 4 resistenza, 2 comp armi
cultura, orientamento, armi a due mani,  mercanteggiare, nuotare

guardia del corpo 6 resistenza, 3 comp armi
armi medie, medicina, armi da tiro, intimidire, osservare

cacciatore di taglie 6 resistenza, 3 comp armi
armi piccole, valutare, armi da tiro, osservare, conoscenza dei bassifondi

Intrattenitore 4 resistenza, 2 comp armi
armi piccole, diplomazia, storia, cultura, camuffarsi

Baro  6 resistenza, 3 comp armi
armi piccole, osservare, ingannare, rissa, mani di fata

bullo 4 resistenza, 2 comp armi
rissa, intimidire, cavalcare, intimidire,  osservare

Cacciatore 4 resistenza, 2 comp armi
armi da tiro, furtivita', nuotare,  osservare, sopravvivenza

Apprendista prete 4 resistenza, 2 comp armi
armi piccole, cultura, incantamento, medicina, intimidire

mercenario 8 resistenza, 4 comp armi
armi da tiro, armi medie, armi a due mani, armi piccole, conoscenza dei bassifondi

milizia cittadina   8 resistenza, 4 comp armi
armi da tiro, armi a due mani, armi medie, intimidire, cavalcare

minatore  4 resistenza, 2 comp armi
armi medie, sopravvivenza, atletica , valutare, orientamento

nobile 4 resistenza, 2 comp armi
linguaggi, cultura, intimidire,  diplomazia, storia, valutare, cavalcare

bandito  6 resistenza, 3 comp armi
armi piccole, armi medie, medicina, mani di fata, intimidire

commerciante  4 resistenza, 2 comp armi
mercanteggiare, conoscenza dei bassifondi, valutare, armi piccole, cavalcare

contafrottole  4 resistenza, 2 comp armi
armi piccole, storia, valutare, ingannare, conoscenza dei bassifondi

bracconiere  6 resistenza, 3 comp armi
armi medie, armi da tiro, medicina, furtivita', sopravvivenza

guardia a cavallo 8 resistenza, 4 comp armi
armi da tiro, armi medie, orientamento, armi a due mani, cavalcare

soldato 8 resistenza, 4 comp armi
armi da tiro, armi piccole, armi medie, armi a due mani, intimidire

ladro  4 resistenza, 2 comp armi
armi piccole, furtivita', conoscenza dei bassifondi, mani di fata, osservare

razziatore  4 resistenza, 2 comp armi
armi medie, atletica, intimidire, valutare,  mercanteggiare

apprendista mago  4 resistenza, 2 comp armi
armi piccole, cultura,  storia, linguaggi, incantamento


rami avanzati hanno prerequisiti, i rami avanzati si ripreondono ogni 5 punti distribuiti (e non ogni 3)


assassino, corpo 12, conoscenza bassifondi 12, armi piccole 12
armi da tiro , conoscenza dei bassifondi  camuffarsi , armi medie , furtivita' , osservare, 5 resistenza

sgherro, corpo 12, intimidire 12, armi piccole 12
armi da tiro, rissa , armi a due mani , conoscenza dei bassifondi , valutare, mani  di fata, intimidire,  6 resistenza

capitano della guardia, volonta' 12, armi medie 12, armi a due mani 12
armi da tiro , cavalcare , intimidire , conoscenza dei bassifondi , storia, medicina, 4 resistenza

ciarlatano, mente 12, ingannare 12, valutare 12
cavalcare , osservare ,  intimidire , conoscenza dei bassifondi, storia, furtivita,  3 resistenza

esploratore, corpo 12, osservare 12, orientamento 12
lame piccole, atletica ,  nuotare , armi da tiro ,  cavalcare, cultura,  linguaggi,  5 resistenza

cavaliere errante, volonta' 12, cavalcare 12, armi medie 12
armi da tiro, armi a due mani , diplomazia , medicina , linguaggi, storia,  5 resistenza

capitano mercenario, corpo 12, armi medie 12, intimidire1 12
armi piccole , medicina , sopravvivenza , armi da tiro , armi a due mani , valutare,  6 resistenza

mercante mente 12, valutare 12, mercanteggiare 12
armi piccole , persuadere  ,  osservare , cavalcare , cultura, armi medie,  5 resistenza

menestrello mente 12, ingannare 12, conoscenza bassi fondi 12
armi piccole , ingannare , cultura , camuffarsi , diplomazia, storia,  4 resistenza

capo bandito corpo 12, armi piccole 12m, cavalcare 12
armi da tiro , intimidire , armi medie, sopravvivenza, valutare, conoscenza dei bassifondi, 5 resistenza

prete volonta' volonta' 12, storia 12, diplomazia 12
armi medie, cavalcare ,  cultura , incantamento , linguaggi , intimidire,  4 resistenza

monaco corpo 12, volonta' 12, medicina 12
rissa, diplomazia  , cultura , storia , linguaggi , medicina ,  4 resistenza

spia mente 12, armi piccole 12, camuffarsi 12
atletica , armi da tiro ,  intimidire , ingannare , mani di fata, cultura,  4 resistenza

soldato veterano corpo 12, cavalcare 12, armi medie 12
rissa , armi piccole  ,sopravvivenza , indimidire , armi da tiro , atletica,  6 resistenza

mago mente 12, cultura 12, intimidire 12
armi medie , cultura , diplomazia , storia , incantamento ,valutare,  4 resistenza


incantesimi:

ispirarsi a nome verbo, ma fai te gli accoppiamenti e stabilisci te cosa succede per ogni critico 
i livelli (II, III..) aggiuntivi hanno prerequisiti di consocenza magica maggiore e statistica maggiore, causano una maggiore perdita di resistenza ma garantiscono maggiore risultato, il tempo di lancio (iniziativa) e' piu' lenta piu' e' potente l'incantesimo.

Attacco Elementale I, II, III, IV, V
Creare Elemento I, II, III
Creare Corpo  ???
Creare Mente  ???
Muovere Elemento I, II
Muovere Corpo (feather fall, levitate, volare..)
Muovere Elemento
Proteggere Corpo
Proteggere Elemento 
Proteggere Mente
Distruggo Elemento (piccole cose.. poi piu' grandi)
Distruggo Corpo (maledizione, penalita'..)
Ripara Corpo
Ripara Elemento
Conoscere Corpo
Conoscere elemento
Conoscere Mente
Ripara Mente
Alterare Mente (charmi, compulsion..)
Distruggo Mente (come attacco)
Riparare Spirito, da valutare se serve, se ci sono mostri effetti contro lo spirito...


Creo		mente (bonus a prove mentali), elemento
Muovo		corpo, elemento, mente (possessione ?)
Trasformo	
Altero
Proteggo	corpo, mente, elemento, spirito
Distruggo	corpo, mente, elemento, spirito
Attacco		elemento, spirito
Riparare	corpo, mente, elemento, spirito
Conoscere	corpo, mente, elemento, spirito

Corpo
Mente
Elemento
Spirito, cio' che riguarda la non vita e l'anima

Alterare
”Rotto il grimaldello? Lucchetto difficile ed antipatico ? Osserva come si apre al mio umile tocco”: Alte-
rare - Materia
”Potenti spiriti guerrieri infondete coraggio ai compagni”: Alterare - Spirito
”Dalle somme biblioteche io chiamo il Silenzio!”: Alterare - Mente / Alterare - Energia. O non fai sentire il
suono, o lo fai diminuire.
”Per i grandi mammuth lanosi, il freddo non mi fa nulla”: Alterare - Corpo. Per avere resistenza al freddo.
”Che Re Gorilla ti dia la forza di un esercito”: Alterare - Corpo
”Dal deserto delle 10 ombre chiamo il miraggio del muro”: Alterare - Mente.
Attaccare
”Possa tu bruciare delle fiamme dell’inferno”: Attaccare - Fuoco
”Fragili sottili deboli, una ad una rompo le tue ossa”: Attaccare - Corpo / Distruzione - Corpo
”Emicrania ? e’ solo l’inizio”: Attaccare - Corpo
”Che il fuoco della fucina scaldi la tua arma”: Attaccare - Materia
Creare
”Rotto il grimaldello? Ho il migliore dei setti regni”: Creare - Materia
”Nessuna mela è più buona di quella che puoi creare tu”: Creare - Materia
”Oh piccola torcia esplodi di luce in questa tetra caverna”: Creare - Energia. In questo caso non può esserci
danno non essendo il Verbo Attaccare.
”A me tomo delle idee! Illuminami il pensiero”: Creare - Mente. In questo caso può essere usato per ritirare
una prova con un bonus. Solo Conoscenza - Mente può dare la soluzione.
”Al più pavido degli eroi concedo il coraggio del leone”: Creare - Spirito. Può essere dato solo a chi il coraggio
non lo ha, altrimenti e’ Alterare.
”Sommi sapienti aiutatemi a chiamare Colui che Striscia nell’Oscurità”: Creare - Spirito. Non si evocano crea-
ture reale, ma si evocano simulacri, per questo si usa Spirito.
Riparare
”Rotto il grimaldello? Una mia carezza è più efficace di un fabbro” : Riparare - Materia
”Nessuna mela è troppo marcia, riempila del tuo amore e mangiala”: Riparare - Materia
”Possano le tue ferite rinsaldarsi, possa il tuo cuore riposare. Possano le mani della somma guaritrice placare
le tue sofferenze”: Riparare - Corpo
”Libera il cuore dalla paura! Che il maleficio delle immonde creature scompaia”: Riparare - Spirito. In questo
caso si porta alla condizione originaria, togliendo l’effetto di paura
Distruggere
”Rotto il grimaldello? Peggio per il lucchetto. Il mio tocco è quello dei millenni”: Distruggere - Materia
”Osserva il vuoto, perditi dentro, scompari nel nulla. Cosa hai fatto?”: Distruggere - Mente. Può fare perdere
Azioni
”Trema, annaspa, striscia, muori. Dal tuo posto non ti sposti”: Distruggere - Corpo. Per fermare il movimento
”Il coraggio non si da a chi non lo ha. Piccolo codardo, scappa dalla mamma”: Distruggere - Spirito. Può
essere usato per diminuire o annullare un Essenza che conferisce coraggio. ”Fiat Tenebris! Ogni luce muoia”:
Distruggere - Energia
Conoscere
”Rotto il Grimaldello? Ogni lucchetto ha un difetto ed un punto debole. Dimmi quale è il tuo e ti libererò
dal giogo della chiusura”: Conoscere - Materia
”Grande Rutte, concedimi il tuo sguardo di mille avventure. Come affrontare il mio avversario”: Conoscere -
Corpo.
”Dal labirinto chiamo il Minotauro. Ti ordino di dirmi la strada più breve per le Sale di Mazurdas”: Conoscere
- Materia
”I tuoi pensieri nei miei pensieri. I tuoi pensieri sono i mei pensieri”: Conoscere - Mente
”Possa il sommo curatore aiutarmi a capire che veleno di affligge”: Conoscere - Corpo
”Se non fuoco, se non fulmine, immonda creatura quale è il tuo punto debole”: Conoscere - Spirito



allineamento:  Legge, Chaos, neutrale


per fare una prova tiri il 2d10 e devi fare meno della statistica o competenza. 


punti ferita/Vitalita': a seconda della razza ed ogni ramo che prendi lo aumenta di un valore dipendente dal ramo scelto, ma l'aumento e' sempre poco (1-4)
resistenza base: corpo + ramo/rami. quindi da 4 a 12/13 a salire. 
Vitalita':  reggi un certo numero di colpi. un colpo critico (ovvero difesa fallita di almeno 6) causa +2 ferite
parti con un numero di Ferite pari a Corpo. i rami possono aumentare le ferite sostenibili. 
recupero resistenza: ogni notte recuperi 2 (o corpo/2 ???)

morte: arrivi a 0, svenuto. se sei a -1, metti un check su ferito, ogni round fai una prova di corpo se fallisci perdi -1, se riesci tre check di fila ti stabilizzi e torni a 0. arrivato a -10 sei morto.

condizioni: prendere da obss e ridurre sensibilmente  e semplificare bonus e malus, aggiungere condizioni solo se necessario

azioni in round: in un round si possono usare fino a 10 punti azione. attaccare con arma a 2 mani costa 8 punti azione, arma a 2 mani costa 6 punti azione, attaccare con un arma leggera costa 4 punti azione. gli incantesimi solitamente costano da 4 punti azione in su in base  al livello (II, III; IV..) ogni livello in piu' aumenta di 1 l'iniziativa, spostarsi costala meta' della in base alla distanza coperta
i mostri in base alla loro taglia e cosa fanno
per altre azioni vedi obss e traduci in p.a.

distanze: a metri

iniziativa: e' pari ai punti azione usati. chi ne usa di meno incomincia per primo. in caso di parita' si controlla mente, volonta', corpo in quest'ordine. se ritarti "consumi" punti azione ad aspettare

competenze di base: ogni ramo modifica/o meno il punteggio di Combattimento o Magia.

combattimento: chi attacca non tira nulla, dichiara solo che arma usa e se usa manovre. chi difende tira sotto il suo valore di combattimento, questo valore di combattimento puo' avere svantaggi dati da abilita' dell'attaccante (rami combattenti avanzati). se tira sotto allora para/evita il colpo, se tira sopra viene preso.
Si presume che l'attacco colpisce sempre se la difesa non funziona bene. Questo significa che le classi combattenti non aumentano l'attacco ma solo la difesa. alcuni rami avanzati di combattimento danno delle penalita' alla difesa avversaria.
il danno e' 1d6 per armi piccole, 1d8 armi medie, 1d10 armi a 2 mani, rissa 2 danni.
combattere con un arma che non conosci da svantaggio


armi: armi piccole 1d6, armi medie 1d8, armi grandi 2 mani 1d10 +bonus corpo. arma piccola ha requisito forza 6, media 9, grande 12, altrimenti svantaggio nell'uso

una difesa particolarmente riuscita (almeno -6) puo' dare svantaggio al tiro successivo, un -9 potrebbe fare tirare con un solo dado
quando difendi  a seconda di quanto bene difendi, ovvero se hai un margine di -3,-6,-9.. rispetto alla tua prova ottieni dei bonus, azioni movimento, penalita' all'attacco successivo

le manovre funzionano con lo stesso principio. quando effettuo una manovra e l'altro riesce nel tiro di combattimento allora la manovra non riesce e io devo fare un tiro di combattimento, se fallisco prendo gli effetti della manovra

armature: riducono il danno. leggere riducono di 2, medie riducono di 3, pesanti riducono di 4. danno una penalita' alle prove di competenza. Non si fa magia con armatura se non solo a contatto. requisito di corpo: 8, 12, 
scudo: fai la prova di difesa con +1 (da 1 bonus). non segno che da penalita' ma premo sull'ingombro 4

modificatori alle prove: un bonus e' +1, due bonus tiri 1d4, 3 bonus tiri 1d6 (non sono provisti oltre 3 bonus cumulativi). le penalita' funzionano in maniera simmetrica. bonus e malus si annullano in maniera reciproca. es. colpire da invisibili da 3 bonus

vantaggio: tiri tre dadi e scarti il piu' alto e sommi i rimanenti  |  svantaggio: tiri tre dadi e scarti il piu' basso e sommi i rimanenti | attenzione al valore 0 zero.

magia: deve essere lanciare piu' difficile lanciare incantesimi e consumare "risorse" (resistenza/energia/stamina..)

certi rami magici possono privilegiare certe scuole di magia. lavorare su liste e ridurre e tanto. l'idea di base e' per esempio crea fuoco puo' diventare a seconda del punteggio del tiro altre cose, il valore influenza la distanza, AoE, danno, se e' un raggio o esplosione e che raggio...  Pochi incantesimi ma che si evolvono

lanciare un incantesimo: tirare a secondo dall'incantesimo su capacita' magica e avere un punteggio minimo di  mente, corpo o volonta'.  Ogni punteggio -3 rispetto alla prova, es. capacita' magica 13 e tiro 8, potenzi un fattore dell'incantesimo (distanza, AoE, danno, tipo di effetto..). Gli incantesimi hanno un punteggio minimo di mente/corpo/volonta' per essere tirati ed anche di capacita' magica
Ogni incantesimo ha degli attributi, danno, distanza, aoe, durata ed un punteggio minimo di competenza magica e corpo/mente/volonta'. se il tiro riesce bene puoi potenziare un attributo presente ma non darlo/aggiungerlo se questo e' assente. se lancio l'incantesimo crea fiamma, inc. base difficolta' 1, se faccio un ottimo tiro potro' potenziare la durata e aoe (l'area di luce che fa) ed il danno, ma non posso aggiungere distanza perche' e' un attributo assente.
ci sara' poi l'incatesimo globo di fuoco, la versione base della palla di fuoco, questa ha piu' attributi ma ad esempio non ha durata, o meglio l'istantanea non puo' essere migliorata.
valutare di aggiungere attributi assenti quando il tiro e' veramente buono, ovvero tiri veramente basso, direi almeno un -6 per aggiungere un attributo a livello base (3 metri)
si possono spendere risorse aggiuntive per potenziare l'incantesimo, ovvero abbassare il risultato del dado
scuola di magia: raggruppa gli incantesimi per tipo
ci si puo' specializzare in un "verbo" di magia: hai +3 alla prova, ma -3 a tutte le altre scuole

lanciare incantesimi mentre si combatte o si e' stato colpito: si puo' fare ma il primo critico si ignora

se l'incantesimo riesce nel lancio non c'e' TS. alcuni incantesimi possono avere una prova di difesa per essere evitati/dimezzati l'effetto

questo implica che non ci saranno mai incantesimi super potenti in tutto..

quanti incantesimi lanciare:  a piacere, ma ogni volta che lanci lo stesso il costo in vitalita' aumenta. 

quando lanci un incantesimo e fallisci la prova non succede nulla, ma l'incantesimo l'hai lanciato e quindi perdi la vitalita' 
quando lanci un incantesimo e fallisci con 19-20 la prova succedono cose  brutte
quando lanci l'incantesimo e fai veramente basso 2, non perdi vitalita'


mostri:
se il mostro tira la sua difesa c'e' il rischio che riesca sempre ad alti livelli. considerare abilita' che abbassano la difesa, l'idea di base e' che se un mostro ha difesa  o piu' deve essere di un livello tale da dover affrontare pg con rami che danno penalita' alla difesa
fare una lista minima di 20 mostri tipici e classici, verificare bx per compatibilita'


--------------------------------------------------------------------------------



- fare 2 o 20  successo critico o fallimento  critico

- il giocatore dichiara cio' che fa e' solo il master a stabilire se serve una prova. 

- il tempo e' un fattore, tabelle random per incontri basati su tempo trascorso, si computa il tempo reale.



\end{document}