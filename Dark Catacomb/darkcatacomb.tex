\documentclass[12pt,a4paper,twoside,openany]{book}
\usepackage{quoting}
\usepackage{tcolorbox}
\usepackage{tikz}
\usetikzlibrary{shadows}
\usepackage{multicol}
\usepackage{tocloft}
\usepackage{lmodern}
\usepackage{caption}
\usepackage[utf8]{inputenc}
\usepackage[T1]{fontenc}
\usepackage{setspace}
\usepackage[a4paper]{geometry}
\geometry{verbose,tmargin=2cm,bmargin=2cm,lmargin=2cm,rmargin=2cm}  %std
\setcounter{secnumdepth}{-1}
\usepackage{booktabs}
\usepackage{url}
\usepackage[italian]{babel}
\usepackage{setspace}
\usepackage{graphicx}
\usepackage{amssymb}
\usepackage{makeidx}
%\usepackage[allfiguresdraft]{draftfigure}
%\usepackage{slashbox}
\usepackage{multirow}
\usepackage{titlesec}
\usepackage[unicode=true, bookmarks=true,
pdftitle={Dark Catacomb - DKC},pdfauthor={Andres Zanzani},
breaklinks=false,pdfborder={0 0 1},backref=section,colorlinks=false]
{hyperref}
\hypersetup{colorlinks=true,linkcolor=blue,pdfcreator={LaTeX}}
\usepackage{bookmark}
\usepackage{yfonts}
\usepackage{auncial}
\usepackage{ragged2e}
\usepackage{ulem}


\usepackage{fontspec}
\setmainfont[Path=./, BoldItalicFont=Soutane Bold Italic.ttf, ItalicFont=Soutane Italic.ttf, BoldFont=Soutane Bold.ttf, Ligatures=TeX, Scale=0.94]{Soutane Regular.ttf} 


\usepackage{wrapfig}
\usepackage{fancyhdr}
\usepackage{tcolorbox}
\tcbuselibrary{skins}
\tcbset{colback=brown!10, fonttitle=\scshape}
\usepackage{imakeidx}
\usepackage{cancel}

\def\CountIndexOccurrences#1{%
	\expandafter\newcount\csname #1\endcsname%
	\expandafter\newcount\csname #1\endcsname%
	\def\indexentry##1##2{\expandafter\advance\csname #1\endcsname 1}%
	\IfFileExists{#1.idx}{\input{#1.idx}}{}%
}
\CountIndexOccurrences{OBSS}
\CountIndexOccurrences{Incantesimi}
\CountIndexOccurrences{Mostruario}
\CountIndexOccurrences{OggettiMagici}
\def\TotalBox#1{\vfill%
	\fbox{Ci sono \expandafter\the\csname #1\endcsname\ voci in questo indice}\par}
\makeindex[columns=3, title=Indice Analitico, intoc=true]
\makeindex[columns=3, name=Incantesimi, title=Lista degli Incantesimi, intoc=true]
\makeindex[columns=3, name=Mostruario, title=Lista dei Mostri, intoc=true]
\makeindex[columns=3, name=OggettiMagici, title=Lista degli Oggetti Magici, intoc=true]
\usetikzlibrary{shapes.misc,calc}
\definecolor{lightgray}{gray}{0.95}
\usetikzlibrary{shapes.misc,calc}
\definecolor{lightgray}{gray}{0.95}
\usepackage{fancyhdr}
\pagestyle{fancy}
\fancyhf{} 
\fancyhead[LE,RO]{\leftmark}
\fancyhead[RE,LO]{}
\fancyfoot[C]{\thepage}
\renewcommand{\sectionmark}[1]{\markboth{#1}{}}
\usepackage{xltabular}
\usepackage{tabularx}
\usepackage{pdfpages}
\usepackage{hyperref}
\usepackage{tikz}
\usepackage[absolute,overlay]{textpos}
\usepackage{etoolbox}
\usepackage{soul}
\raggedbottom
\usepackage{array}
\newcolumntype{L}[1]{>{\raggedright\let\newline\\\arraybackslash\hspace{0pt}}m{#1}}
\newcolumntype{k}[1]{>{\centering\let\newline\\\arraybackslash\hspace{0pt}}m{#1}}
\newcolumntype{R}[1]{>{\raggedleft\let\newline\\\arraybackslash\hspace{0pt}}m{#1}}
\newcolumntype{D}[1]{>{\centering}m{#1}}
\newcolumntype{M}[1]{>{\centering\arraybackslash}m{#1}}
\titleformat{\section}{\filcenter\huge\bfseries\accanthis}{\thesection}{1em}\textsc{}
\titleformat{\subsection}{\Large\bfseries\accanthis}{\thesubsection}{1em}\textsc{}
\titleformat{\subsubsection}{\normalsize\bfseries\accanthis}{\thesubsubsection}{1em}\textsc{}
\def\changemargin#1#2{\list{}{\rightmargin#2\leftmargin#1}\item[]}
\let\endchangemargin=\endlist
\setcounter{tocdepth}{3}
\newtcolorbox{narratore}{
	enhanced, % enable advanced settings
	%left = 3mm,
	%width=0.45\textwidth,
	left = 9mm, % pushes text away from the left edge by 10mm
	sharp corners, % disables rounded corners
	rounded corners = southeast, % "round" the bottom right corner
	arc is angular, % make the "round" corner an angle
	arc = 3mm, % controls corner cut
	boxrule=0.6pt, % sets box line thickness
	underlay={%
		\path[fill=black] ([yshift=3mm]interior.south east)--++(-0.4,-0.1)--++(0.1,-0.2); % triangle
		\path[draw=black,shorten <=-0.05mm,shorten >=-0.05mm] ([yshift=3mm]interior.south east)--++(-0.4,-0.1)--++(0.1,-0.2); % triangle edge
		\path[fill=gray!50!black,draw=none] (interior.south west) rectangle node[brown!10]{\Huge\bfseries ?!} ([xshift=8mm]interior.north west);
	},
	drop fuzzy shadow }

\newtcolorbox{enfasi}{
	enhanced,
	arc=5pt,
	boxrule=0.3pt
} 

\usepackage{zref-savepos,graphicx}
\newcommand{\filltopageendgraphics}[2][]{%\filltopageendgraphics[width=.5\linewidth]{image-a}
	\par
	\zsaveposy{top-\thepage}% Mark (baseline of) top of image
	\vfill
	\zsaveposy{bottom-\thepage}% Mark (baseline of) bottom of image
	\smash{\includegraphics[keepaspectratio=true,height=\dimexpr\zposy{top-\thepage}sp-\zposy{bottom-\thepage}sp\relax,#1]{#2}}%
	\par
}


\usepackage{accanthis}
\usepackage[framemethod=TikZ]{mdframed}


\begin{document}
	
\def \versione {0.01}
\thispagestyle{empty}
 
{\Huge \begin{center}
		Dark Catacomb
\end{center}}

\vfill
\begin{center}
	\Large{\color{black} Fantasy Adventure Game}
\end{center}

\pagebreak

	
\bigskip
Non temere l'ignoto, affrontalo con rispetto.
	
	\vspace{\fill}
\begin{center}\textbf{\versione} - \today\end{center}
\thispagestyle{empty}


\newpage~\thispagestyle{empty}%%\newpage~\thispagestyle{empty}


\newcommand{\riga}{\rule{\textwidth}{0.4pt}}


{\Huge \begin{center} Dark Catacomb\end{center}}

\bigskip

\begin{center}{\LARGE Manuale per Giocatore e Arbitro}\\ \end{center}

{\large \begin{center} Guida e Regole per il Gioco di Ruolo Fantasy \end{center}}

\begin{center}di \end{center}

{\LARGE \begin{center} Andres Zanzani \end{center}}

\vspace{2cm}


\vfill

\begin{mdframed}[roundcorner=10pt]

\medskip

\textbf{Playtesting}: ...to be done...

\bigskip

\begin{flushleft}\textbf{Condizioni d'uso}: Dark Catacomb, DKC, è un marchio registrato di Andres Zanzani (azanzani@gmail.com).
\end{flushleft}

\vspace{0.5cm}


\medskip

\end{mdframed}

%}%
%}

\pagebreak

{\LARGE \begin{center}
		Io sono l'Alfa e l'Omega, dice il Signore Dio, Colui che è, che era e che viene, l'Onnipotente!
\end{center}}

\vspace{15cm}

Dark Catacomb e' un gioco di ruolo che tocca diversi temi religiosi. Se l'argomento ti da fastidio, mi dispiace. Cambia gioco.

\pagebreak

\setcounter{page}{1}

\begin{multicols}{2}
\tableofcontents{}

\end{multicols}

\vfill

\begin{changemargin}{0.3cm}{0.3cm}\begin{tcolorbox}
Vuolsi così colà dove si puote
ciò che si vuole, e più non dimandare\\
\end{tcolorbox}\end{changemargin}

\pagebreak

%pietra bianca
%100 atti unici di altruismo disinteressato
%10 demoni uccisi

%10 angeli uccisi
%100 anime raccolte
%100 7 peccati, la grande meretrice

%superbia: radicata convinzione della propria superiorità, reale o presunta, che si traduce in atteggiamento di altezzoso distacco o anche di ostentato disprezzo verso gli altri, nonché di disprezzo di norme, leggi, rispetto altrui;

%avarizia, derivante più precisamente dall'etimologia latina avaritia, collegata all'avidità della fame:[9] cupidigia, avidità, costante senso di insoddisfazione per ciò che si ha già e bisogno sfrenato di ottenere sempre di più;

%lussuria: incontrollata sensualità, irrefrenabile desiderio del piacere sessuale fine a se stesso, concupiscenza, carnalità, divinizzazione del sesso sempre maggiore, che può andare dalla fornicazione sino all'adulterio, e agli atti più estremi e perversi;

%invidia: in relazione a un bene o una qualità posseduta da un altro, si prova dispiacere e astio per non avere noi quel bene e a volte un risentimento tale da desiderare il male di colui che ha quel bene o qualità;

%gola: nel suo senso concreto, è l'irrefrenabile bramosia di ingurgitare cibi o bevande senza fermarsi al limite della sazietà imposto dal corpo, ma proseguire nella consumazione per puro piacere e ingordigia. Nel suo senso astratto, "goloso" è chi abusa di una determinata cosa, andando al di là del limite imposto dalla natura umana.

%ira: alterazione dello stato emotivo che manifesta in modo violento un'avversione profonda e vendicativa verso qualcosa o qualcuno;

%accidia: torpore malinconico, inerzia nel vivere e nel compiere opere di bene, pigrizia, indolenza, infingardaggine, svogliatezza, abulia.



%https://thealexandrian.net/wordpress/17308/roleplaying-games/hexcrawl
%http://osrsimulacrum.blogspot.com/2020/05/making-wilderness-play-meaningful-system.html
%http://www.innomedimaria.it/apocalisse/apocalisse.htm
%https://www.maranatha.it/Bibbia/8-Apocalisse/73-ApocalissePage.htm


\section{Introduzione}
Razza\\
\sout{Caratteristiche}\\
\sout{rami base}\\
\sout{rami avanzati}\\
\sout{Punti Fato}\\
Karma\\
\sout{Competenze}\\
Costruiamo il personaggio\\
Regole per le competenze\\
Combattimento\\
Nascondigli e coperture\\
liste armi ed armature\\
abilita' ???\\
magia\\
incantesimi\\
equipaggiamento\\
veleni e droghe\\
movimento\\
oggetti magici\\
mostruario\\
Condizioni\\
Scheda\\

\pagebreak

\section{Dark Catacomb - L'Ambientazione}


%https://www.gliscritti.it/dchiesa/bibbia_cei08/nt73-libro_dell_apocalisse.htm

\begin{narratore}
L'Ambientazione di Dark Catacomb si inspira a piene mani dall'Apocalisse di Giovanni e non vuole offendere nessun credente. Se quanto nell'ambientazione vi incuriosisce potete leggere il Libro dell'Apocalisse di San Giovanni.
	
La  vera rivelazione sarà leggere i Dottori della Chiesa.
	
Personalmente ho trovato di grande interesse personale e formativo Le Confessione di Sant'Agostino e Gli esercizi spirituali di Sant'Ignazio di Loyola.
	
Vi suggerisco di nutrire la vostra spiritualità perché, se ancora non ne avete percezione, ha sempre bisogno di ispirazione e illuminazione.
	
\end{narratore}	

\subsection{La Storia}

\begin{multicols}{2}
	


\subsubsection{Quella che già conosci}


Carissimi figlioli vi lascio queste poche pagine preziose, perché rare, affinché non dimentichiate chi eravate, da dove venivate e dove siete destinati ad andare.

Anche se tutti voi conoscete a grandi linee quello che è successo è opportuno chiarire e comprendere il perché noi siamo qui, perché non siamo andati oltri.

Circa 400 anni fa c'è stata l'Apocalisse. Non quella ecologica e climatica, non quella atomica della guerra, non quella razziale ma quella che il Signore Dio nostro aveva profetizzato a Giovanni. La vera Apocalisse.

I risultati li vediamo ancora e probabilmente per sempre.\\

Inizialmente grandine e fuoco mescolati a veleno scrosciarono sulla terra bruciando un terzo del pianeta, un terzo degli alberi e tutta l'erba verde.

Cadde poi una grande montagna infuocata (probabilmente un meteorite) nel mare ed un terzo dei mari divenne velenoso, un terzo della vita marina morì ed un terzo di tutte le navi andò distrutto.

Un altra grande stella fiammeggiante cadde dal cielo e colpì un terzo dei fiumi e delle sorgenti d'acqua. Queste acque divennero velenose e molti uomini morirono per averle bevute.

Alla quarta tromba angelica un terzo del sole, della luna e degli astri fu colpito e la penombra divenne eterna.

Poi fu la volta del pozzo dell'Abisso. L'angelo aprì il pozzo e da questo uscirono prima un fumo nero come di fornace e poi cavallette che erano destinate a causare grande sofferenza e dolore ma solo agli uomini, e senza ucciderli.

Queste cavallette grandi come cavalli avevano capelli lunghi e denti da leone, il torace come corazze di ferro ed il rombo delle loro ali come quello degli aeroplani.

Il loro re era l’angelo dell’Abisso, che in ebraico si chiama Abaddon, in greco Sterminatore.

Al sesto squillo di tromba vennero liberati i 4 Cavalieri, si proprio quelli, i Cavalieri dell'Apocalisse ed un altro terzo dell'umanità venne ucciso.

La civiltà, la società, la cultura l'umanità stessa era ormai una vaga ombra di quello che era all'inizio dell'Apocalisse.

Ma il peggio doveva ancora venire.

A questo punto un \textbf{enorme drago rosso}, Satana, con sette teste e dieci corna e sulle teste sette diademi apparì nel cielo. La sua coda trascinava un terzo delle stelle del cielo e le precipitava sulla terra. 

Dal mare salì una \textbf{Bestia} anch'essa con dieci corna e sette teste, sulle corna dieci diademi e su ciascuna testa un titolo blasfemo. Questa creatura immonda era simile a una pantera, con le zampe come quelle di un orso e la bocca come quella di un leone. 

Il Satana le diede la sua forza, il suo trono e il suo grande potere. 

E gli stupidi umani presi per l'ammirazione e la bramosia del potere incominciarono ad adorare Satana perché aveva dato il potere alla Bestia ed alla Bestia perché proferiva parole blasfeme e d'orgoglio contro Dio.
Queste persone vennero marchiate con il nome del Satana e della Bestia, perché su di loro si scatenasse l'ira di Dio.

E gli angeli agirono di nuovo. Versarono la prima coppa dell'ira di Dio e si formò una pioggia cattiva e maligna su gli adoratori della Bestia.

Il secondo angelo versò la sua coppa nel mare e quel poco che c'era nelle acque morì.

Il terzo angelo versò la sua coppa nei fiumi e nelle sorgenti 

Il quarto angelo versò la sua coppa sul sole e gli uomini bruciarono per il terribile calore e bestemmiarono il nome di Dio invece di pentirsi per rendergli gloria.

Il quinto angelo versò la sua coppa sul trono della Bestia; e il suo regno fu avvolto dalle tenebre. 

Il sesto angelo prosciugò fiumi e corsi d'acqua.

Ed intanto dalla bocca del Satana della Bestia uscirono gli spiriti dei demoni e andarono per le nazioni del mondo a radunare i regnanti nel luogo che si chiama Armaghedòn.

E le nazioni fecero la guerra tra di loro mentre la Bestia gioiva di come aveva offuscato i loro pensieri.

Quando poi il settimo angelo versò la sua coppa nell’aria  seguirono fulmini, tempeste ed un grande terremoto.
Ogni isola scomparve e i monti si dileguarono.

Solo i credenti, i puri di spirito e cuore, coloro che avevano ricevuto la \textbf{pietra bianca con inciso il loro nuovo nome} dalle mani di un angelo poterono andare alla \textbf{città eterna Gerusalemme}.

Una città scesa dal cielo splendente come una gemma preziosissima, come pietra di diaspro cristallino.  
La città è cinta da grandi ed alte mura con dodici porte: sopra queste porte ci sono dodici angeli.
Le mura della città poggiano su dodici basamenti con inciso i nomi dei dodici apostoli dell’Agnello.


\subsubsection{La storia nuova}

Come avrai capito caro figlio c'è qualcosa che non torna.

Tutta l'umanità doveva essere giudicata e quindi salvata o uccisa. Salvata nella città eterna o uccisa dalle piaghe o dalla Bestia.

Eppure, non troppo numerosi, ma siamo ancora qui. In una terra in perenne tramonto, con un sole stanco e debole ed un cielo privo di firmamento.

Le nostre nuove navi solcano i mari e raccolgono il pesce che ancora c'è.

Le nostre città non esistono più, non esiste più quella che una volta era chiamata industria o la tecnologia che ancora raramente troviamo.

La terra asciutta e brulla reclama lavoro e acqua. La civiltà è tornata indietro di più di mille anni in un periodo di ignoranza e barbarie.

Perché ci siamo salvati ? perché non siamo stati giudicati?

Per un \textit{disguido}. 

Sappiamo che almeno uno degli angeli che portava le pietre bianche non giunse mai a destinazione e disperse le sue pietre della salvezza sul mondo.

10, 100, 1000 ? forse un milione?  non lo sappiamo quanti  dei nostri avi non ricevettero la pietra, fatto sta che noi siamo i loro discendenti. Cerchiamo di sopravvivere in quella che per noi è una \textbf{Oscura Catacomba}\index{Dark Catacomb}.

Camminiamo sopra i resti dei nostri avi, abbiamo sottoterra intere città spogliate di vita e piene di creature che umane non sono più.

Quello che era il \textbf{nostro} mondo ora non lo è più. Demoni e angeli continuano a combattere e le loro discendenze umane portano guerra anche tra noi.

Non esiste più il concetto di nazione, di popolo. So che una volta tutti noi potevamo parlare insieme con una sola lingua, tutti potevano sapere qualsiasi cosa in qualsiasi momento.
Ecco.. non più. Consegnare una missiva a pochi giorni di cavallo, se sei tra i fortunati che sanno scrivere costa almeno 3 gemme grezze.

Siamo rimasti l'ombra di ciò che eravamo, ma forse questo è anche la nostra salvezza.

Insieme si stanno ricostruendo villaggi, la terra con il duro lavoro della schiena e delle braccia produce ancora qualche frutto.
Molte nostre comunità cercano la pace e la democrazia, molte altre ubbidiscono al gioco di un demone o di un padrone umano.

Sono tornati i mestieri di una volta e non ci sono accorgimenti o tecnologie che possono farlo al tuo posto. Ancora mi chiedo come si facesse una volta senza un bravo calzolaio. 

Concetti come la razza purtroppo esistono ancora, l'ignoranza becera e meschina è insita in noi ed il seme di Satana attecchisce vivace tra i bruti e gli stupidi.

\subsection{La nuova civilta'}

Città da milioni di abitanti sono state vaporizzate dalle piaghe, malattie, veleni e dalle guerre dei popoli. Quello che sopravvive sono piccoli e pochi grandi paesi dove una economia di sussistenza o poco più permette alle persone di sopravvivere.

I paesi spesso non superano i 200 abitanti e le città più grandi i 20000.

Si è tornati a quello che era il medioevo.

Ogni tanto qualche reperto antico viene ancora trovato, molto spesso sono cumuli di ruggine o apparecchi che nessuno più sa usare.

Il denaro è stato sostituito da un bene di più pratico valore le gemme.

Gli insediamenti sono costruiti o sfruttando i materiali e resti di antiche città ma molto più spesso dove qualche risorgiva d'acqua permette di coltivare i campi.

I terremoti hanno raso al suolo ogni manufatto umano o lo hanno portato sotto terra. Immensi e profondissimi crepacci hanno inghiottito intere regioni. Cosa sia sopravvissuto o rimasto sul fondo non è dato saperlo.

\subsubsection{I nuovi abitanti}

Per millenni abbiamo creduto di essere soli o addirittura qualcuno era convinto che discendessimo da creature venute dal cielo.

Adesso sappiamo che non siamo soli, non lo siamo mai stati.

Nel corso dei secoli recenti demoni e angeli hanno giaciuto con le nostre donne e nostri uomini, ne è discesa una stirpe di creature dal destino segnato.

I nefilim angelici sono solitamente alti quasi due metri, esadattili (sempre sei dita per mano), i maschi tendono ad avere la barba rossa e le donne una folta capigliatura nera come la pece.

I nefilim demoniaci sono solitamente alti oltre i due metri, con numerosa corna che amano ingioiellare ed ali da pipistrello (anche se molti preferiscono dire da Drago).

\subsubsection{La maledizione del 33}

Non è dato sapere il perché ed il calcolo non è neanche preciso dato che non cè più l'alternanza del giorno e della notte o delle stagioni, ma allo scoccare di ogni 33 anni di vita succede qualcosa di infausto.

Il primo accadimento è l'infertilità, maschile e femminile, nessuna oltre i 33 anni riesce a rimanere incinta e nessun maschio oltre i 33 anni sempre essere più fertile.

Il secondo accadimento è che a 66 anni si muore, tutti. Semplicemente non ci si sveglia più. Il cuore cessa di battere e si muore per un pò.
I primi tempi, ed ancora ad essere onesti, è scioccante vedere chi si ama, i propri amici morire così senza una causa apparente.

Ed è per questo che il compleanno dei 66 anni viene adesso festeggiata anche più della nascita, con una festa che dura per giorni, con tutti propri cari, parenti ed amici, fino ad arrivare all'ultimo saluto quando il festeggiato va a riposare per un ultima volta come tutti noi lo conoscevamo.

Il problema è ai compimento dei 99 anni, quando le forze demoniache reclamano il corpo e come un morto vivente il defunto risorge.

Adesso che questi fatti sono risaputi in quasi tutte le famiglie si procede con la cremazione dei corpi, eppure in remoti villaggi e quando il dolore per il distacco è troppo forte ed i corpi vengono seppelliti in semplici sudari di stoffa, ebbene accade quello che ben vi potete immaginare.

Il trauma peggiore di tutti però è stato vedere gli avi, coloro che erano morti da tanti e tanti anni risorgere. L'Apocalisse ha fatto risorgere i morti e chi aveva la benedizione dell'Agnello è assunto in cielo, ma tutti gli altri, e vi assicuro che i loro numeri sono incalcolabili sono risorti come dannati, come spiriti affamati della poca vita rimasta.

Immense città finite sottoterra per gli immensi terremoti e fratture sono ora popolate di non morti e forse di qualche sopravvissuto, forse. Di certo tutti i loro tesori ed averi sono rimasti tutti li, pronti per il primo o forse secondo avventuriero.

I saggi sono convinti che sia una maledizione del grande Drago, di Satana, che potendo comandare le creature rimaste sulla terra che non sono ascese nella città eterna si diverte a martoriarci. 

Solo i nefilim sono immuni a tutte e tre le maledizioni.

\subsection{Vivere e salvarsi}

Carissimi figlioli, dopo quando detto sembra ridicolo parlare di vivere e salvarsi, eppure una flebile speranza c'è sempre.

E' vero che a 33 anni non potrei avere più una discendenza, cosa pericolosissima in un mondo dove siamo rimasti veramente pochi a vivere, ed è altrettanto vero che a 66 anni morirai, ma c'e' sempre una speranza. SEMPRE.

Grazie alle visioni guidate dagli angeli e purtroppo dai demoni abbiamo imparato che è possibile salvarsi, evitare l'infausta maledizione del 33 seppure dovendo scegliere di lasciare questo regno.

Ci sono stati lasciate dalle potenze angeliche queste possibilità di redenzione:\\

- compiere 100 distinte azioni di puro altruismo\\
- uccidere 100 demoni\\
- trovare una delle pietre bianche ed incidere il proprio nuovo nome

chi adempie ad almeno una di queste missioni potrà ascendere alla Città Eterna e salvarsi.\\

Se invece vuoi seguire i dettami di Satana queste sono le azioni che ti porteranno a diventare un Demone

- compiere 100 distinte azioni di pura malvagità
- uccidere 100 anime non marchiate con il nome della Bestia o di Satana
- uccidere 100 angeli
- vivere l'intera vita ubbidendo solo ai sette peccati capitali

chi adempie ad almeno una di queste missioni potrà recarsi alla pozza di fuoco e zolfo dove Satana comanda le sue legioni e chiedere di diventare un Demone.

Si possono definire una salvezza queste scelte ? Non sta a me deciderlo ma al cuore delle persone che vorranno intraprenderlo che vorranno lasciare questo inferno sulla terra per il Paradiso oppure per governarlo come empio Demone.

Molti altri intraprendono la vita dell'avventuriero alla ricerca dei tesori che giacciono incustoditi sopra e sotto il suolo.


\subsection{Le Stirpi di Dark Catacomb}\index{Le Stirpi di Dark Catacomb}


Le stirpi presenti in Dark Catacomb sono 2, gli Umani ed i Nefilim.

I nefilim sono frutti degli incroci \textit{proibiti} tra umani ed angeli o demoni.

Solo per il fatto di avere sangue angelico o demoniaco non significa che siano buoni o malvagi a priori, a differenza dei loro progenitori ultraterreni i nefilim hanno un anima e come tale sono dotati di libero arbitrio.

I nefilim angelici sono solitamente alti quasi due metri, esadattili (sempre sei dita per mano), i maschi tendono ad avere la barba rossa e le donne una folta capigliatura nera come la pece.

I nefilim demoniaci sono solitamente alti oltre i due metri, con numerosa corna che amano ingioiellare ed ali da pipistrello (anche se molti preferiscono dire da Drago).

Come però già scritto un nefilim sente il richiamo del sangue e deve percorrere non uno ma tutte le scelte del percorso angelico o demoniaco che vorrà intraprendere.

A differenza degli umani un nefilim non è soggetto alla maledizione del 33, il suo percorso di vita può arrivare oltre i 250 anni.

Gli umani invece possono decidere di vivere come meglio credono i 66 anni che gli sono concessi.

Entrambe le stirpi concedono un +1 ad una Caratteristica a propria scelta alla creazione del personaggio, fino ad un massimo di 12.
	
\end{multicols}



\pagebreak

\section{Caratteristiche}\index{Caratteristiche}

\begin{multicols}{2}

\subsection{Le Caratteristiche del Personaggio}\index{Le Caratteristiche del Personaggio}

Le Caratteristiche\index{caratteristiche} di un personaggio servono a comprendere quando possa essere forte e robusto, ma anche atletico se non intelligente e di buon senso. Rappresentano le potenzialità su cui le Competenze costruiscono l'esperienza.

Queste Caratteristiche sono \textbf{Corpo} \index{Corpo}, \textbf{Mente} \index{Mente} e \textbf{Volontà} \index{Volontà} e \textbf{Vigore}\index{Vigore}

\textbf{Corpo} rappresenta tutte le caratteristiche fisiche, quindi forza, resistenza, capacità atletiche. Corpo influenzerà tutte le prove basate sul fisico del personaggio.

\textbf{Mente} rappresenta la capacità di ragionamento, la memoria, la rapidità di pensiero e l'arguzia. Mente influenza tutte le prove in cui il personaggio deve ragionare, ricordare.

\textbf{Volontà} rappresenta il buon senso ma anche il saper resistere a shock emotivi. Volontà viene usato quando si gestiscono animali e si deve fare un lavoro che richieda impegno e dedizione.

\textbf{Vigore} rappresenta l'energia vitale del personaggio e la sua capacità di resistere a colpi od incantesimi.

\subsection{Come stabilire le Caratteristiche del personaggio}\index{Come stabilire le Caratteristiche del personaggio}

Il giocatore tira 2d10 e somma il risultato, questo tiro e computo lo esegue per ogni Caratteristica tranne Vigore.

Se il valore sommato di una Caratteristica è inferiore a 6 segnerai 6 nella Caratteristica. Se il valore sommato di una Caratteristica è superiore a 12 segnerai comunque 12 nella Caratteristica.

Il punteggio di \textbf{Vigore} è pari al punteggio di Corpo aumentato dai punti indicati dal Ramo scelto.\index{Vigore iniziale}

\subsubsection{I modificatori alla prove}\index{Modificatori alla prove}

Ogni Prova di Competenza viene modificata dal punteggio della Caratteristica connessa. 
Il modificatore alla prove di Competenza è pari al punteggio di Caratteristica -10. Nelle Competenze è indicata la Caratteristica che la modifica. Questo modificatore viene applicato sia che abbia un valore positivo o negativo al valore della Competenza.

\end{multicols}

\section{Punti Chaos}\index{Punti Chaos}\index{Fortuna del Principiante}

\begin{multicols}{2}
	

\begin{changemargin}{0.3cm}{0.3cm}\begin{enfasi}{Se il destino è contro di noi, peggio per lui. (motto del 1º Reggimento Carabinieri Paracadutisti "Tuscania")}\end{enfasi}\end{changemargin}

In un mondo non facile ne amichevole il Chaos domina il destino. Ogni personaggio può ricorrere ai punti Chaos per influenzare la sua prova o anche quella di un compagno o di un avversario!

Ogni personaggio ha tre segnalini ed e' libero di consumarne fino a tre alla volta. Ognuno di questo conta come un bonus o penalità, quindi prenderne 1 conta come un +1 (o -1), due conta come un +1d4 (o -1d4), prenderne tre da Vantaggio (o Svantaggio) alla prova. 

Quanti Punti Chaos si vogliano usare si dichiara prima del tiro, una volta dichiarato l'ammontare di Punti Chaos non é possibile utilizzarne di più o di meno.

Ogni volta che al personaggio nel tiro dei dadi esca un doppio 0 recupera un punto Chaos.

I punti Chaos vengono azzerati e reimpostati a 3 ad ogni sessione di gioco.

\end{multicols}

\pagebreak

\section{Le Competenze e le Prove}\index{Competenze}\index{Prove}

\begin{multicols}{2}

Ogni personaggio può seguire uno o più Rami ovvero un insieme di competenze e capacità professionali.

Queste Competenze quando apprese faranno parte del bagaglio culturale, conoscitivo e pratico del Personaggio. Il Personaggio usando le Competenze ne migliorerà l'uso.

Il Personaggio in base a quello che viene dichiarato effettuerà una Prova per capire se riesce e come nell'intento. 

\end{multicols}

\subsection{Le Competenze}

\begin{tabular*}{0.93\linewidth}{@{\extracolsep{\fill}}lll}
\textbf{Corpo} & \textbf{Mente} & \textbf{Volontà}\\
\toprule
Atletica				& Arcana					& Artigianato			\\	
Arrampicarsi			& Conoscenza *				& Cavalcare				\\
Artista della fuga		& Erboristeria				& Diplomazia			\\
Armi piccole 			& Disattivare congegni		& Osservare	\\
Armi medie 				& Falsificare				& Mani di fata\\
Armi a due mani			& Incantamento				& Gestire animali\\
Intimidire		 		& Raggirare					& Furtività\\
Nuotare					& Intrattenere				& Orientamento\\
Rissa					& Mercanteggiare			& Percepire Emozioni \\ 
Usare corda		 		& Natura					& Seguire tracce\\
						& Pronto soccorso			& Sopravvivenza\\
						& Tradizioni locali			& \\

\end{tabular*}\\

La \textbf{Conoscenza} va esplicitata su quale argomento verte: Dungeon, Legge, Lingue, Piani, Occulto, Architettura ed Ingegneria, Nobiltà ed Araldica, Miti e Leggende, Religione, Storia, Geografia ...\\

Alcune Competenze hanno una importanza peculiare nel sistema: \textbf{Incantamento} e le varie \textbf{Armi}, la prima permette di lanciare incantesimi e ne aiuta a determinare l'efficacia, la seconda indica la Competenza del personaggio con le varie tipologie di armi e quanto è capace di usarle.

\begin{multicols}{2}
	
\subsection{Le Prove}\index{Effettuare una Prova}

Ogni Competenza ha un valore numerico che ne stabilisce il grado di capacità nell'uso, più e' alto maggiore sarà la facilità con cui supero le prove.

Il valore di Caratteristica-10 è usato come modificatore alla Prova assumendo quindi valori sia positivi che negativi, anche se negativo viene comunque chiamato bonus nel manuale.

Per verificare l'esito di una Prova di Competenza è necessario sommare il valore di Competenza con il bonus di Caratteristica e sottrarre il risultato della somma di 2d10.

Per \textbf{Punteggio di Competenza}\index{Punteggio di Competenza} (PC) (e non valore di Competenza) si intende il valore già sommato del bonus di caratteristica e del valore di Competenza.\index{Punteggio di Competenza}\index{PC}

Si definisce \textbf{Margine di Successo}\index{Margine di Successo} (MS) il valore di differenza tra il Punteggio di Competenza (PC) con quanto tirato con i dadi.\index{Margine di Successo}\index{MS}

Il Margine di Successo (MS) può assumere valori negativi o positivi. Mentre in una prova di Competenza o Caratteristica il successo della stessa è solo nel MS positivo, con una prova contrapposta non è detto che un valore negativo sia un insuccesso, dipende da quanto fatto dal contendente

La Prova d'Armi viene effettuata sia per Difendersi che per Attaccare. Nel manuale troverete la Prova d'Armi per difendersi come \textbf{PDAD} e quella per Attaccare come \textbf{PDAA}.
 
Le \textbf{Prova d'Armi}, sia PDDA che PDAD \index{Prova d'Armi}, come per la altre Prove si effettuano sommando del valore di Competenza con il bonus di Caratteristica (solitamente Corpo) e al risultato si sottrae quello del tiro di 2d10.
Le singole PDAD e PDAA hanno il loro Margine di Successo.

\subsubsection{I modificatori alla Prova}\index{Modificatori alla Prova}

L'Arbitro può decidere la presenza di modificatori alla Prova in base alla situazione in cui si svolge la Prova.

Qualora ci sia una \textbf{penalità} (ho fretta, è buio, corro, l'avversario è a cavallo ed io sono appiedato..) la difficoltà della Prova aumenta, ovvero devo \textbf{aumentare il valore della Prova} effettuata della penalità presente.

Se invece ho un \textbf{bonus} allora la difficoltà della Prova diminuisce, ovvero devo \textbf{diminuire il valore della Prova} effettuata del bonus presente.

Un Bonus sarà un valore positivo che sommo ai 2d10 tirati nella Prova.\index{Applicare il Bonus}. Una Penalità è un valore negativo che sottraggo al Punteggio dei Competenza.

I \textbf{modificatori alla Prova si cumulano} tra di loro se omogenei, tutti positivi o tutti negativi, e si annullano o scalano a vicenda se di tipo opposto (bonus e penalità).

Es. Devo scalare una parete. Ho un Bonus perché sono presenti degli appigli, ho una penalità perché sta piovendo, ho due bonus perché posso aiutarmi con una corda, ho una penalità perché è buio. La differenza totale tra bonus e penalità é di 1 Bonus.

I \textbf{modificatori}, bonus o penalità, alla Prove assumono il valore di \textbf{1}, qualora ci sia un solo bonus o penalità, \textbf{1d4} qualora ci siano due modificatori omogenei attivi. Nel caso i modificatori siano tre o più, ovviamente di tipo omogeneo, si ha il cosiddetto \textbf{\textit{Vantaggio}} oppure \textbf{\textit{Svantaggio}}.

In caso di \textbf{Vantaggio} tiro 3d10 per effettuare la Prova e scarto quello con il valore più alto, poi sommo gli altri due dadi per verificare l'esito della Prova.\index{Vantaggio}

In caso di \textbf{Svantaggio}\index{Svantaggio} tiro 3d10 e scarto quello con valore più basso, poi sommo gli altri due dadi per verificare l'esito della Prova. 

\subsection{Migliorare le Competenze}\index{Migliorare le Competenze}\hypertarget{Migliorare le Competenze}{} \label{Migliorare le Competenze}

Ogni qual volta vi effettua una Prova su una Competenza e questa ha come risultato dei dadi \textbf{19 o 20} si mette un segno vicino alla Competenza. Si possono avere fino a cinque segni vicino ad una singola Competenza.

Quando il personaggio ha tempo di riflettere su quanto accaduto, sulle prove che ha fatto e come queste sono riuscite o fallite, può tirare 2d10 e se la somma dei dadi è superiore al suo \textbf{punteggio di Competenza} allora quel punteggio aumenta di 1.

Una volta fatta questa particolare prova si cancellano tre segni dalla Competenza.	

\subsection{Competenze ed i loro ambiti di utilizzo}\label{competenzeambitidiutilizzo}

Sono descritte sommariamente le Competenze ed i loro ambiti di utilizzo. Sono indicazioni di massima su cosa usare le competenze. Viene anche indicato il numero di Punti Azioni (PA) necessarie per svolgere la prova tipica, ovvio che usi più complessi richiedono più tempo e Punti Azioni (PA).

I PA necessari alla prova possono variare a seconda della capacità del personaggio e della complessità della prova.

In ogni caso ricordate sempre di valutare con attenzione come il giocatore dichiara di svolgere le azioni per capirne la durata ed effetti. 

La Competenze con un \textbf{*} subiscono le penalità dovute all'armatura indossata.\\

\textbf{Atletica* (Corpo)}: Questa competenza serve per mantenere l'equilibrio su superfici strette o precarie, per tuffarsi, rotolare, fare capriole, salti mortali, superare degli ostacoli nonché cadere e non farsi male. 

\textbf{Arcana (Mente)}: Con questa competenza si è esperti di magia e di incantesimi, di oggetti magici è si è grado di identificare gli incantesimi che vengono lanciati. 

\textbf{Arrampicarsi* (Corpo)}: Con questa competenza si possono scalare superfici verticali, dalle mura cittadine alle pareti rocciose. E' legata all'Azione di movimento. Con 8 punti il movimento è solo dimezzato.

\textbf{Artigianato (Mente)}: E' necessario specificare la tipologia di Artigianato in cui si è competente. Si è competente, ma non a livello di Professione, in una forma di artigianato.

\textbf{Artista della fuga (Corpo)}: Con questa competenza ci si può liberare da legacci e manette.

\textbf{Cavalcare (Volontà)}: Con questa competenza è possibile cavalcare in maniera professionale e dare comandi alla propria cavalcatura. 

\textbf{Osservare (Volontà)}: per cercare, accorgersi, notare. E' un qualcosa di attivo.

\textbf{Conoscenza dei Dungeon (Mente)}: Con questa competenza si hanno conoscenze di Aberrazioni, melme, caverne, esplorazioni sotterranee.

\textbf{Conoscenze di Geografia (Mente)}: Con questa competenza si hanno conoscenze sul clima, popolazione, terreni, territori, nazioni e confini.

\textbf{Conoscenza Lingue/Linguaggi (Mente)}: Con 1 punto sai parlare una lingua, con 3 punti la sai anche scrivere. Un buon punteggio di Lingue aiuta a comprendere lingue non note ed a farsi comprendere. Viene usata anche per comprendere testi complessi

\textbf{Conoscenze Occulte (Mente)}: Con questa competenza si è esperti di occulto, creature immondi. 

\textbf{Conoscenze Religione (Mente)}: Con questa competenza si hanno conoscenze su Patroni, mitologia, Celestiali, Non Morti, simboli sacri, tradizione ecclesiastica, feste e ricorrenze liturgiche. 

\textbf{Conoscenze di Storia (Mente)}: Con questa competenza si hanno conoscenze di Storia quali guerre, migrazioni, colonie, fondazioni di città, accadimenti importanti..

\textbf{Diplomazia (Volontà)}: Con questa competenza si possono risolvere diverbi, e raccogliere preziose informazioni e dicerie dalle persone. La competenza è anche usata per negoziare in modo efficace con la giusta etichetta e condotta adatta alla situazione controversa. 

\textbf{Disattivare congegni (Mente)}: Con questa competenza si possono disarmare Trappole e aprire serrature, sabotare congegni meccanici semplici, come le catapulte, le ruote di un carro o le porte.

\textbf{Erboristeria (Mente)}: Con questa competenza si hanno conoscenze di come riconoscere e preparare pozioni e veleni naturali. Il punteggio si applica alle prove per distillare pozioni.

\textbf{Falsificare (Mente)}: Con questa competenza si sa falsificare oggetti d'arte, mappe, firme... 1 Minuto

\textbf{Gestire animali (Volontà)}: Con questa competenza è possibile addestrare e ammansire animali.

\textbf{Intimidire (Corpo)}: Intimidire si basa sull'approccio fisico per convincere l'interessato. 

\textbf{Raggirare (Mente)}: La competenza Ingannare può essere usata per Raggirare (dicendo quindi fandonie) o Persuadere (adattando la verità) al fine di convincere delle proprie parole l'interessato.

\textbf{Intrattenere (Mente)}: Con questa competenza si è esperti in una espressione artistica, dal canto alla recitazione, dal ballo a suonare strumenti musicali. E' necessario specificare la forma di intrattenimento.

\textbf{Mani di fata* (Volontà)}: Con questa competenza si può borseggiare, estrarre un'arma nascosta, oppure compiere altre azioni senza essere notati. 

\textbf{Furtività (Volontà)}: Con questa competenza si è in grado di muoversi senza causare rumore oppure di passare inosservati stando fermi. 

\textbf{Natura (Mente)}: Con questa competenza si hanno conoscenze di Animali, Fatati, stagioni e cicli, tempo atmosferico, vegetali. 

\textbf{Nuotare* (Corpo)}: Con questa competenza si è in grado di nuotare, anche in acque tempestose. Senza competenza si sa stare a galla in acqua placide. Legata all'Azione di movimento.

\textbf{Orientamento (Volontà)}: Con questa competenza si ha il senso della direzione e orientamento rendendo impossibile perdersi indipendentemente dall'ambiente in cui ci si trova. 

\textbf{Percepire Emozioni (Volontà)}: Con questa competenza si può capire se qualcuno sta mentendo o si possono intuire le sue vere intenzioni.

\textbf{Pronto soccorso (Mente)}: Con questa competenza si possono curare le ferite e le malattie. Costo variabile.

\textbf{Seguire tracce (Volontà)}: Con questa competenza si sa seguire le tracce lasciate nell'ambiente. 

\textbf{Sopravvivenza (Volontà)}: Con questa competenza si può sopravvivere e orientarsi nelle terre selvagge. La competenza è usata anche per cercare attivamente trappole e fosse. .

\textbf{Tradizioni locali (Mente)}: Con questa competenza si hanno conoscenze degli abitanti (più noti), costumi, leggende, leggi, personalità, tradizioni. E' necessario specificare una regione geografica dove è applicabile la conoscenza. 

\textbf{Usare corda (Corpo)}: Con questa competenza si è esperti in legacci e nodi per fissare e bloccare oggetti o persone. 

\textbf{Mercanteggiare (Mente)}: Con questa competenza si sa stimare il valore monetario di un oggetto.

\subsubsection{Esempi Prove Competenza}\label{esempiprovecompetenza}\hypertarget{esempiprovecompetenze}{}\index{Esempi prove Competenza}

\textbf{Prove atipiche}\index{Prove atipiche}. Il giocatore è invitato a trovare usi, soluzioni, approcci che esulino dalle più ovvie prove. Siate creativi e descrivete al Arbitro la meravigliosa azione che volete fare e quali risultati sperate di ottenere! Sarà lui a stabilire in base alla vostra descrizione dell'azione cosa provare e se hai dei bonus o penalità.

\medskip

Per \textbf{riconoscere un oggetto magico}\index{Riconoscere oggetto magico} e le sue capacità è necessaria una prova di \textbf{Arcana} per avere indicazioni di massima sui poteri e ambiti di utilizzo, con un MS di almeno 6 puoi apprenderne i dettagli, bonus magici e cariche. \textbf{10 minuti}. Con punteggio Arcana 6 costa 5 minuti, con 12 costa 1 minuto, con Arcana 18 costa 10 PA.

\medskip

\textbf{Riconoscere un incantesimo}\index{Riconoscere un incantesimo} mentre viene lanciato è una prova di \textbf{Arcana} Costa una \textbf{Reazione}. Se fatto assieme al lancio di un Controincantesimo non costa Reazione.

\medskip

Per \textbf{riconoscere un mostro}, una creatura particolare si effettua una prova di Conoscenza. Controlla il capitolo \hyperlink{riconoscereimostri}{Riconoscere i Mostri} nel Mostruario (pag. \pageref{riconoscereimostri})

\medskip

\textbf{Atletica}\index{Atletica} \textit{Penalità dovute all'armatura}

Una prova di Atletica riuscita permette al personaggio di dimezzare il danno quando cade da meno di 9 metri (\textbf{Reazione}).

\medskip

\textbf{Arrampicarsi/Scalare} \index{Arrampicarsi}\index{Scalare} \textit{Penalità dovuta all'Armatura.}

\medskip

Usare una corda\index{Arrampicarsi su una corta}\index{Salire su una corda}, scalare od arrampicarsi equivale a muoversi in un \textbf{terreno doppiamente difficile}. Se la prova di Arrampicarsi riesce si sale di 30 cm per PA speso.

In caso di fallimento della prova si consumano i PA senza spostarsi. Se la prova fallisce (MS negativo) di 6 o più perdi la presa e cadi. I modificatori indicati nella tabella si sommano.\\

\begin{tabularx}{0.45\textwidth}{Xl}
	\textbf{Esempio di Superficie} & Mod.\\
	\toprule
	Movimento solo dimezzato & -1\\
	Superficie scivolosa&-1\\
	Parete grezza con appigli, mattoni sporgenti&-1\\
	Una corda senza nodi&-2\\
	Una parete con appigli &+2\\
	Un muro/parete con pochissimi appigli&-3\\
	Ti puoi appoggiare a 2 pareti opposte&+2\\
	Ti puoi appoggiare a 2 pareti angolari&+1\\
	Puoi usare una corda&+2\\
\end{tabularx}\\

i modificatori indicati sono sulla prova effettuata dal giocatore, se positivo è un bonus (alzi il Punteggio di Competenza).

\medskip

Per \textbf{identificare una pozione o veleno naturale}\index{Identificare Veleno}\index{Erboristeria} \index{Identificare Pozione}è necessario una prova di \textbf{Erboristeria}.

Costa 5 minuti. Se il MS è +3 impieghi 4 minuti, se +6 impieghi 3 minuti, con -9 impieghi 2 minuti, +12 impieghi 1 minuto, +15 impieghi 1 round.

Se la prova fallisce ed il MS è -6 ha assunto la pozione consumandone una dose.

\medskip

\textbf{Intimidire}\index{Intimidire}. Il personaggio usa \textbf{6 PA} ed effettua una prova di Intimidire, l'avversario può contrapporre una prova di Intimidire o di Corpo. Chi ottiene il MS migliore intimidisce l'avversario.

Chi è intimidito ha 1 Penalità al PDA fino alla fine del round successivo.

\medskip

\textbf{Ammansire un animale} è una prova di \textbf{Gestire Animali}. Tempo richiesto 10 minuti. Per ogni MS il tempo si riduce di 1 minuto.

\medskip

\textbf{Furtività} \index{Furtività} \textit{Penalità dovuta all'Armatura.}

La prova di Furtività va effettuata solo se c'è qualcuno che può sentire/vedere. Si confrontano i MS delle prove di Furtività e Osservare per capire se si è stati percepiti. Muoversi in maniera Furtiva equivale a muoversi su terreno difficile e quindo ci vogliono 2 PA per spostarsi di 1.5 metri.

\medskip

\textbf{Nuotare}\index{Nuotare} \textit{Penalità dovuta all'Armatura}

In acque calme basta una prova riuscita di nuotare, se le acque sono mosse il MS deve essere di almeno 3, e di 6 se molto mosse e 9 se tempestose. La prova è necessaria sia per stare a galla o nuotare. Nuotare in acqua si considera \textbf{terreno difficile}.

\medskip

\textbf{Pronto Soccorso}\hypertarget{prontosoccorso}{}\label{prontosoccorso}\index{Pronto Soccorso}. Una prova riuscita fa recuperare 1d4 Vigore se fatta entro 1 minuto dal termine dello scontro.

Concede 1 Bonus ad una prova di Caratteristica contro un veleno se non ha ancora fatto effetto. Costo \textbf{2 minuti}. Con MS di +6 costa 1 minuto. Con MS +9 costa 3 round, con MS +12 costa 1 round.

Una prova riuscita riduce di 1 i danni da \hyperlink{sanguinamento}{\textbf{Sanguinamento}}. Se la prova riesce con MS +3 riduce di 2 punti, con MS +6 riduce di 3 punti.

Un trattamento di almeno 8 ore permette di recuperare al paziente il doppio di Corpo in punti Vigore. Se effettuato durante le ore di riposo chi prende cura risulterà Affaticato.

\medskip

\textbf{Saltare}\index{Tabella Saltare} \textit{Penalità dovuta all'Armatura.} \textbf{4 PA}\\

Con una prova di Atletica è possibile saltare in lungo 3 metri. Per ogni MS salti 30 cm in più.

La \textbf{distanza saltata in alto} è pari a 90cm + 10cm per MS.

In un \textbf{salto in lungo} la punta più alta del salto è pari ad un 1/4 della lunghezza saltata. Se esegui un salto in lungo di 6 metri a metà salto sei in alto di 1.5 metri.

Scendere da meno di 1m non usa PA. Se non si ha almeno 3 metri di rincorsa si salta la metà.

Vigore perso per caduta (pag. \pageref{cadute}): 3x altezza caduta (in metri). Prova di Atletica per dimezzare il danno se cadi da meno di 9 metri.

\medskip

\textbf{Sopravvivenza}\index{Sopravvivenza}

\smallskip

\textbf{Inseguire una creatura}:

\begin{tabular}{ll}
	Situazioni & Mod.\\
	\toprule
	Se il terreno è molto morbido& +2\\
	Se il terreno è morbido& +1\\
	Se il terreno è stabile& 0\\
	Se il terreno è duro& -2\\
	Ogni 6 creature inseguite& +1\\
	Ogni 24 ore passate & -1\\
	Visibilità scarsa&-1\\
	Ogni ora di pioggia&-1 \\
	Cerca di occultare le traccie& -1\\
\end{tabular}\\

Un Modificatore negativo è una penalità (abbassa il risultato dei dadi), un modificatore positivo è un bonus (alza il risultato del Punteggio di Competenza).

Sopravvivenza può essere usata al posto di \textbf{Disattivare Congegni} con Svantaggio.

Una prova di Sopravvivenza per foraggiare cibo procura viveri per una persona aggiuntiva ogni 3 di MS.

\medskip

La prova di \textbf{Mercanteggiare}\index{Mercanteggiare} serve per abbassare il prezzo di una merce e per valutare un oggetto. Oggetti molto rari richiedono un MS di almeno +3 per essere valutati.


\end{multicols}

\pagebreak


\section{I Rami}\index{Rami}

\begin{multicols}{2}

I \textbf{Rami} sono la professione del personaggio, è l'insieme delle competenze che il personaggio conosce. In altri sistema di gioco il Ramo sarebbe l'equivalente della classe.

\textbf{Ogni Ramo conferisce al personaggio dei punti Vigore}, da sommare a Corpo per stabilirne il valore iniziale, e delle Competenze.

I Rami di base ed avanzati concedono 6 Competenze.

Mentre i \textbf{Rami di base} possono essere presi come prima professione da chiunque, i\textbf{ Rami avanzati} possono essere appresi solo a patto di soddisfare i requisiti indicati.

\subsection{Rami Base}

Nella tabella sottostante sono indicati alcuni Rami base di esempio. il personaggio è invitato a crearsi Rami con Competenze più affini alla sua storia.
Sono indicati il nome del Ramo e il Vigore. Le \textbf{competenze prendono un punteggio} pari alla punteggio indicato dalla prima colonna. Ad esempio un Apprendista ha punteggio 6 in Conoscenza mentre un Bandito ha 4 punti in Furtività.

\end{multicols}

\begin{tabular}{|l|l|l|l|l|}\hline

&\multicolumn{4}{c}{\textbf{Rami}}\\\hline

&\textbf{Apprendista}&\textbf{Bandito}&\textbf{Tagliaborse}&\textbf{Baro}\\\hline
\textbf{Vigore}&2&9&4&6\\\hline
6&Conoscenza &Armi piccole&Mani di fata&Raggirare\\
5&Incantamento&Intimidire&Furtività&Per. Emozioni\\
4&Pronto soccorso&Furtività&Armi piccole&Armi piccole\\
3&Intimidire&Armi medie&Sopravvivenza&Mani di fata\\
2&Armi piccole&Mani di fata&Pronto Soccorso&Rissa\\
1&Tradizioni Locali&Pronto soccorso&Atletica&Intimidire\\\hline

&\textbf{Tirapiedi}&\textbf{Cacciatore}&\textbf{Segugio}&\textbf{Commerciante}\\\hline
\textbf{Vigore}&6&4&7&2\\\hline		
6&Rissa &Natura&Armi da tiro&Mercanteggiare\\
5&Intimidire&Sopravvivenza&Intimidire&Raggirare\\
4&Furtività&Armi da tiro&Seguire tracce&Per. Emozioni\\
3&Osservare&Furtività&Osservare&Intrattenere\\
2&Cavalcare&Erboristeria&Mercanteggiare&Armi piccole\\
1&Mani di fata&Osservare&Armi piccole&Cavalcare\\\hline

&\textbf{Contafrottole}&\textbf{Guardia del corpo}&\textbf{Intrattenitore}&\textbf{Mendicante}\\\hline
\textbf{Vigore}&3&6&3&3\\\hline		
6&Raggirare			&Armi a due mani&Intrattenere	&Osservare\\
5&Percepire Emozioni&Orientamento	&Travestimento	&Sopravvivenza\\
4&Storia			&Pronto soccorso&Storia			&Mercanteggiare\\
3&Armi piccole		&Osservare		&Armi piccole	&Armi piccole\\
2&Mercanteggiare	&Conoscenza		&Raggirare		&Furtività\\
1&Conoscenza		&Intimidire		&Diplomazia		&Mani di fata\\\hline

&\textbf{Mercenario}&\textbf{Milizia cittadina}&\textbf{Minatore}&\textbf{Nobile}\\\hline
\textbf{Vigore}&16&9&4&2\\\hline		
6&Armi medie		&Armi medie		&Conoscenza	caverne	&Diplomazia\\
5&Armi a due mani	&Tradizioni locali&Sopravvivenza	&Conoscenza\\
4&Sopravvivenza		&Intimidire		&Armi medie			&Linguaggi\\
3&Armi da tiro		&Cavalcare		&Orientamento		&Storia\\
2&Armi piccole		&Rissa			&Mercanteggiare		&Cavalcare\\
1&Intimidire		&Armi da tiro	&Atletica			&Intimidire\\\hline
							
\end{tabular}	\\

Il Vigore che concede un Ramo si calcola sommando i punteggi delle Armi. Il valore minimo di Vigore è 2.

\begin{multicols}{2}	

\subsection{Rami Avanzati}

I \textbf{Rami avanzati}\index{Rami Avanzati} hanno un prerequisito di Caratteristica e di Competenza a certi punteggi.
Il Vigore che concede un Ramo avanzato è pari a 3 per il numero di Armi che fa conoscere. In valore minimo di Vigore che un Ramo avanzato concede è +3.
Il Vigore concessa da un Ramo avanzato si somma con quella già posseduta dal personaggio.

I Rami avanzati aumentano il punteggio delle Competenze di 1, se sono Competenze già note, altrimenti impostano ad 1 il valore se non erano note.

Alcuni esempi di Rami avanzati:\medskip

\textbf{Assassino}:\\
\textit{Requisito}: Corpo 12, Sopravvivenza 12, Armi piccole 12\\
\textit{Competenze concesse}: Armi da tiro, Sopravvivenza, Travestimento, Armi medie , Furtività , Osservare\\
\textit{Vigore}: +6\\

\textbf{Sgherro}:\\
\textit{Requisito}: Corpo 12, Intimidire 12, Armi piccole 12\\
\textit{Competenze concesse}: Armi da tiro, Rissa, Armi a due mani, Sopravvivenza, Mani di fata, Intimidire\\
\textit{Vigore}: +9\\

\textbf{Capitano della Guardia}:
\textit{Requisito}: Volontà 12, Armi medie 12, Tradizioni Locali 12
\textit{Competenze concesse}: Armi da tiro, Cavalcare, Intimidire, Sopravvivenza, Storia, Pronto soccorso\\
\textit{Vigore}: +3\\

\textbf{Esploratore}:\\
\textit{Requisito}: Corpo 12, Sopravvivenza 12, Orientamento 12\\
\textit{Competenze concesse}: Geografia, Armi piccole, Atletica, Nuotare, Cavalcare, Linguaggi\\
\textit{Vigore}: +3\\

\textbf{Cavaliere errante}:\\
\textit{Requisito}: Volontà 12, Cavalcare 12, Armi medie 12\\
\textit{Competenze concesse}: Armi da tiro, Armi a due mani, Diplomazia, Pronto soccorso, Linguaggi, Storia\\
\textit{Vigore}: +6\\

\textbf{Menestrello}:\\
\textit{Requisito}: Mente 12, Intrattenere 12, Conoscenza 12\\
\textit{Competenze concesse}: Armi piccole, Raggirare, Storia, Travestimento, Diplomazia, Incantamento\\
\textit{Vigore}: +3\\

\textbf{Incantatore}:
\textit{Requisito}: Mente 12, Incantamento 12, Conoscenza 12
\textit{Competenze concesse}: Armi medie, Conoscenza, Storia, Linguaggi, Diplomazia, Erboristeria 
\textit{Vigore}: +3\\

\textbf{Monaco}:\\
\textit{Requisito}: Corpo 12, Volontà 12, Pronto soccorso 12\\
\textit{Competenze concesse}: Rissa, Diplomazia, Conoscenza, Storia, Linguaggi, Pronto soccorso\\
\textit{Vigore}: +3\\


\subsection{Avanzare nel Ramo}\index{Avanzare nel Ramo}

Ogni volta che la\textbf{somma dei punti assegnati} grazie al \hyperlink{Migliorare le Competenze}{Migliorare le Competenze} arriva a 3 il Vigore aumenta di 3 per i Rami base.
In caso di \textbf{Rami avanzati} la somma dei punti assegnati deve arrivare a 5 per poter aumentare di 5 il Vigore.

\subsection{Prendere un nuovo Ramo}\index{Prendere un nuovo Ramo}

Alla creazione del Personaggio si scegli un Ramo e si apprendono Competenze e si imposta il valore di Vigore a Corpo + il valore di Vigore dato dal Ramo.

Il personaggio può decidere di acquisire più \textbf{Rami base} e quindi conoscere più Competenze. Per poter intraprendere un nuovo Ramo il personaggio deve trovare qualcuno che possa insegnarglielo e pagare 1000 monete d'oro. Tutte le Competenze del suo Ramo precedente diminuiscono di 1.
Il fatto di prendere un nuovo Ramo non fa aumentare il Vigore, rimane valida la regole dei 3 punti distribuiti prima di aumentare il Vigore.

Per prendere un \textbf{Ramo avanzato} si devono soddisfare i requisiti indicati, trovare qualcuno che possa insegnarlo, pagare 5000 monete d'oro. In questo caso non c'è una diminuzione delle Competenze originari ed i punti Vigore non si acquisiscono.


\end{multicols}

\pagebreak

\section{Combattimento}\index{Combattimento}\label{Combattimento}\hypertarget{Combattimento}{}

\begin{multicols}{2}
	
Il combattimento è diviso in 2 fasi:\index{Combattimento}
\begin{itemize}
	\item verifica delle Azioni
	\item verifica Iniziativa
	\item risoluzione delle Azioni (movimento, attacco, azione varie..)
\end{itemize}

L'unità base di tempo nelle scene di combattimento è il Round ovvero una unità di tempo di 10 secondi.

Ogni personaggio può eseguire diverse Azioni nel round ed ognuna di queste costa dei Punti Azione. \textbf{Chi meno ne esegue più è veloce nell'eseguirle}.

All'inizio di ogni round il giocatore dichiara quanti PA userà. Non è necessario che dichiari cosa andrà ad eseguire.

La verifica dell'Iniziativa\index{Iniziativa} consiste nel tirare 2d10 sottrarre Corpo o Mente ed aggiungere i PA che si intendono usare.

\textbf{Il personaggio od avversario che ha un valore dell'Iniziativa più basso incomincia per primo.}

A parità di Iniziativa chi usa meno Punti Azione (PA) incomincia per primo, se i PA usati sono uguali parte per primo l'avversario.
A parità di PA usati tra i personaggi questi si mettono d'accordo tra loro per agire in che ordine.

Il personaggio potrebbe anche usare meno PA di quanti dichiarati, ma agirà in ogni caso nell'ordine stabilito dal valore dell'Iniziativa.

\textbf{Un personaggio non può dichiarare un certo numero di PA e poi usarne di più.}

\subsubsection{Il Tempo (Round, Minuti e Turni)}\index{Round}\label{iltempo}

Un \textbf{round} dura 10 secondi circa, è un lasso di tempo sufficiente per agire, correre, parlare.. combattere. Un Minuto sono 6 round ed un Turno dura 10 Minuti (o 60 round).

I round si usano nelle scene di combattimento o dove la tensione deve rimanere costantemente alta ed ad ogni Azione corrisponde un evolversi della situazione.

\subsubsection{Tempo di riattivazione Oggetti ed Abilita'}\index{Tempo di attivazione Oggetti ed Abilità}\label{temporiattivazioneoggetti}

Se non specificato diversamente un oggetto o Abilità che prevede un certo numero di usi al giorno \textit{"es. una volta al giorno"} si "ricarica" all'alba successiva l'uso.

\subsection{Punti Azioni nel Round}\index{Azioni nel Round}\index{Azione}\label{azioninelround}

Nella tabella sottostante sono indicate le Azioni principali e relativi Punti Azione che usano, sono linee guida da seguire. Nel capitolo dedicato al combattimento vengono elencate altre Azioni ed i loro costi relativi in Punti Azioni.

Le Azioni scelte possono essere eseguite nell'ordine preferito.

Una Azione non può essere interrotta\index{Interrompere Azioni}\index{Azioni, Interrompere} da un altra Azione, ma può essere seguita da una Azione di Reazione o da una Azione Immediata, se nel proprio round.

Se un personaggio vuole fare più attacchi spostandosi nel campo di battaglia può, ad esempio, usare 4 PA per eseguire un attacco, usare 2 PA per muoversi di 3 metri ed usare gli ultimi 4 PA (nel caso ne avesse dichiarati 10) per un ultimo attacco.

E' possibile \textbf{ritardare} una o più Azioni\index{Ritardare Azioni} per aspettare lo svolgersi delle scene. Il personaggio che ritarda una sua Azione si considera che abbia "sprecato" PA per aspettare fino a quel segmento di iniziativa e potrà usare solo i PA che rimangono fino alla fine del round.

Un giocatore che dichiara di aspettare una certa situazione per poter agire equivale ad eseguire una o più \textbf{Azioni Preparate}\index{Azioni Preparate}. In questo caso il personaggio (o nemico) agisce dopo l'Azione con solo i PA rimasti per avere aspettato fino a quel segmento di iniziativa.

Se il personaggio ha già usato tutte i PA allora potrà agire fuori dalla sua iniziativa solo tramite una Reazione, se a disposizione. L'Azione di Reazione arriva sempre dopo l'Azione scatenante.

\bigskip

\end{multicols}

\textbf{Tabella: Azioni per Round}\index{Tabella delle Azioni per Round}



\begin{tabularx}{0.95\textwidth}{Xc}
\textbf{Cosa si fa}  & \textbf{Punti Azioni}\\
\toprule
Attaccare con Rissa					& 4\\
Attaccare con Arma piccola			& 4\\
Attaccare con Arma media/da Tiro	& 5\\
Attaccare con Arma a due mani		& 8\\
Lanciare un'Incantesimo *1			& *\\
Muoversi *2							& 1 PA per 1.5 metri\\
Scatto *3							& 1 PA per 3 metri\\
Alzarsi da prono					& 4\\
Aiutare qualcuno					& 5\\
Scambiare un dialogo con qualcuno *4	& variabile\\
Scambiare poche battute con qualcuno *5& 0\\
Prendere qualcosa nello zaino		& 8\\
Prendere qualcosa dalla cintura o di pronto & 4\\
Usare un oggetto tenuto in mano		& 2\\
Bere una pozione tenuta alla cintura& 4\\
Estrarre/Rinfoderare l'arma			& 3\\
Imbracciare lo scudo				& 3\\
Usare un oggetto magico				& 6\\
Eseguire prova su una competenza *6	& 6\\
Sfondare una porta a spallate/calci	& 5\\
Forzare porta con piede di porco	& 6\\
Nascondersi							& 4\\
Concentrarsi su un Incantesimo		& 4\\
Salire o scendere dalla cavalcatura	& 4\\
Azione \textbf{I}mmediata - Azione \textbf{R}eazione& I - R\\
Bere una pozione tenuta in mano& I\\
Gettare un oggetto tenuto in mano& R\\
Gettarsi a terra prono& R\\
Riconoscere un Incantesimo& R\\
\end{tabularx}

\medskip

Per Attacco si intende sia l'uso di armi in mischia che l'uso di armi da lancio o tiro come archi, balestre o pugnali da lancio. Nel caso di armi da lancio ogni lancio/tiro conta come un attacco.

Qualora il personaggio esegua una Azione di Attacco e Lanciare un incantesimo si considera Distratto nell'eseguire la Prova di Magia.

\begin{multicols}{2}
	
\textbf{Lanciare un Incantesimo *1}: a seconda del potere dell'Incantesimo sono necessari più PA. Nella descrizione dell'incantesimo è indicato il numero di PA necessari. 

\textbf{Muoversi *2}: per ogni PA usato ci si può muovere fino a 2 metri.

\textbf{Scatto *3}: per ogni PA usato ci si può muovere fino a 3 metri ma si incorre nella penalità di aver \textbf{corso}.

\textbf{Scambiare un dialogo con qualcuno *4}: Un dialogo può essere di pochi secondi se non di minuti. Il Arbitro valuterà quanto questo dura.

\textbf{Scambiare poche battute con qualcuno *5}: Finché sono veramente poche battute o uno sguardo non consuma PA, se questo diventa più articolato allora utilizza dei PA. L'obiettivo è non interrompere il flusso delle Azioni con un fitto dialogo ma comunque permettere l'interazione tra i personaggi.

\textbf{Eseguire prova su una competenza*6}: se fruttano una frazione del round costano 4 PA, altrimenti 8 PA o più. Controllate negli Esempi Prove Competenza i costi riportati.

Una Azione "\textbf{Reazione (R)}" \index{Azione Reazione}può essere eseguita liberamente anche fuori dal proprio round. Questa Azione è solitamente dovuta ad Abilità o situazioni particolari. Se non indicato diversamente una Azione di Reazione accade immediatamente dopo la causa che la scatena.

Una Azione "\textbf{Immediata (I)}" \index{Azione Immediata}può essere eseguita liberamente nel proprio round, primo o dopo la propria Azione. Una Azione Immediata è solitamente concessa da particolari Abilità.

E' possibile se non descritto diversamente eseguire solo una Azione Immediata ed una Azione di Reazione per round.

\medskip

Questo \textbf{elenco non è completo}, prendetelo come linee guida per stabilire il peso delle decisioni ed azioni dei giocatori. Una Azione dura circa 3 secondi.

L'\textbf{ordine} con cui si eseguono le Azioni non è importante se non per correlazione logica e fisica. Il Movimento può essere tra altre Azioni (movimento, attacco/incantesimi/altra azione, movimento).


\end{multicols}

\subsection{Movimento}\index{Movimento}\label{movimento}


\begin{changemargin}{0.3cm}{0.3cm}\begin{enfasi}{"Un mobile più lento non può essere raggiunto da uno più rapido; giacché quello che segue deve arrivare al punto che occupava quello che è seguito e dove questo non è più (quando il secondo arriva); in tal modo il primo conserva sempre un vantaggio sul secondo" (Paradosso di Zenone)}
\end{enfasi}\end{changemargin}

\begin{multicols}{2}

Il movimento di un personaggio è dato dalla sua taglia e razza e da ciò che porta, dai pesi, ingombri ma anche magie ed oggetti magici.

Il Movimento scritto nella razza del personaggio è l'indicazione di quanti metri può fare usando 10 PA.

Una creatura o personaggio potrebbe anche decidere di spostarsi più velocemente del solito ovvero correndo (Azione di Scatto).\index{Correre}

Non è possibile spostarsi anche solo di 1 metro se non si spendono PA.

Queste precisazioni hanno senso e vanno usate quando si tratta di combattere ed il dislocamento sul territorio, mappa, è fondamentale. Durante gli spostamenti normali, mentre si cavalca o cammina liberi senza pericoli, si usa la normale gestione del movimento orario.

Quando si parla di "\textbf{quadretto}" \index{Quadretto}per indicare una distanza si intende un quadretto di mappa di 1.5 metri x 1.5 metri.

Nel caso di spostamento diagonale\index{Movimento diagonale}\index{Spostarsi di lato} si conta una distanza di 2 quadretti, in caso di arrotondamenti sull'ultimo quadretto si fa per eccesso.

\textbf{Se ci si sposta in terreno "difficile"}, si consumano il doppio di PA per spostarsi di 1.5 metri.

\subsection{Distanza}\index{Distanza}\label{distanza}

Per \textbf{distanza di Mischia} \index{Distanza di Mischia} \index{Mischia}si intende una distanza che permette il combattimento corpo a corpo (1.5 metri attorno al personaggio). Nei mostri questa distanza è indicata dalla portata, per le armi da lancio è chiamata gittata.

Se non indicata nell'avversario/mostro la distanza di mischia/tocco aumenta di 1.5 metri per ogni taglia oltre la media.\index{Taglia e distanza di mischia}
Alcune Armi a due mani hanno una portata di 3 metri.

A distanza di mischia una creatura di dimensioni medie può avere al massimo 8 creature medie.

\end{multicols}

\pagebreak

\subsection{Vita e Morte}\index{Morire}\label{morire}

\begin{changemargin}{0.3cm}{0.3cm}\begin{enfasi}{Chi non conosce la morte, non conosce la vita. (Grand Hotel, film 1932)

\medskip

The worthy GM never purposely kills players' PCs. He presents opportunities for the rash and unthinking players to do that all on their own (Gary Gygax)}\end{enfasi}\end{changemargin}\medskip

\begin{multicols}{2}

Quando un personaggio raggiunge 0 (zero) di Vigore si considera svenuto, ovvero Inabile a fare qualsiasi cosa. Una Cura magica (Incantesimo, Pozione..) lo porterà cosciente ed al valore Vigore curato. Una prova di Pronto Soccorso, 10 PA lo porterà ad 1 di Vigore. Dopo un ora se non è successo qualcosa a mutare la situazione il personaggio può fare prova di Corpo, se riesce torna a 1 di Vigore, se fallisce va a -1 di Vigore e diventa morente.

Un personaggio morente ha Vigore negativa (-1 o meno) ed è svenuto e \hyperlink{morente}{prossimo alla morte}. Continuerà a perdere 1 punto di Vigore a round fiche il valore non raggiungerà -10 e morità, se non viene curato.

Una magia (incantesimo o pozione) di Cura, di qualsiasi potere lo porterà a 1 di Vigore, successive cure funzioneranno normalmente.

Una prova di \hyperlink{prontosoccorso}{Pronto Soccorso} con Svantaggio, 10 PA, porterà il personaggio a 0 di Vigore, ovvero svenuto. 

Una successiva prova di Pronto Soccorso, 10 PA, potrà portarlo a 1 di Vigore ed una cura magica lo curerà dell'ammontare dichiarato.

Un personaggio morente che subisce ulteriore danno, nemici che infieriscono sul corpo od incantesimi diretti a lui od ad area, continua a sottrarre Vigore. 

Le Condizioni \index{Condizioni mentali}di tipo mentale quali Affascinato, Confuso ma non Dominato, terminano quando il personaggio diventa morente.

Quando un personaggio torna a Vigore 1 dopo essere andato a Vigore negativi ha Svantaggio a tutte le Prove finché non riposa una notte.

Un personaggio morto non può beneficiare delle cure normali o magiche. Solo incantesimi molto un potenti possono riportarlo in vita.

\subsubsection{Recupero punti Caratteristica}\index{Recupero punti caratteristica}\label{recuperopunticcaratteristica}

Eventuali punti Caratteristica persi si recuperano al ritmo di 1 punto al giorno, se non indicati come perdita permanente.

\subsubsection{Recupero Vigore naturale}\index{Recupero Vigore naturale}\label{recuperopuntiferitanaturale} 

Riposare 8 ore fa recuperare il punteggio di Corpo in Vigore

\subsubsection{Recupero Vigore non letale}\index{Recupero Vigore non letale}\index{Perdita Vigore non letale}\label{recuperopuntiferitanonletali}\hypertarget{recuperopuntiferitanonletali}{}

Ogni ora si recupera 1 punto Vigore.

\subsection{Tiro per Colpire e Difesa}\index{Tiro per Colpire}\index{Difesa}\label{tiropercolpireedifesa}

\begin{changemargin}{0.3cm}{0.3cm}\begin{enfasi}{Applica sempre la giusta forza, mai troppa mai troppo poca. (Kano Jigoro)}\end{enfasi}\end{changemargin}\medskip

Ogni qual volta una creatura \textbf{decida di Attaccare} deve effettuare una Prova d'Arme d'attacco (PDAA), ovvero somma il suo punteggio di Competenza d'Armi, nell'arma che sta usando, con il punteggio di Corpo e sottrae la somma di 2d10. La differenza è il Margine di Sicurezza (MS) ottenuto.

Ogni qual volta una creatura \textbf{vuole difendersi} deve effettuare una Prova d'Arme di difesa (PDAD),  ovvero somma il suo punteggio di Competenza d'Armi, nell'arma che sta usando per difendersi, con il punteggio di Corpo e sottrae la somma di 2d10. La differenza è il Margine di Sicurezza (MS) ottenuto.

L'attaccante confronta il suo MS con quello di chi si difende, se superiore o uguale avrà colpito e causerà i danni al Vigore.

Il linea di massima le Armi piccole causano 1d6 di danno, le Armi medie 1d8, le Armi a due mani 1d10, Rissa causa 1d4, controllate questi valori nell'equipaggiamento. Questi numeri sono da aumentare con il modificatore dato da Corpo.

Al danno causato dall'attacco si sottrae l'eventuale protezione data dall'armatura, il danno rimanete (minimo 1) si sottrae ai punti Vigore.

Al danno causato dall'avversario sottrarrà la protezione dell'armatura ed il rimanente al Vigore.

Se i modificatori e circostanze portano il danno inflitto ad essere negativo o zero comunque farai 1 danno a Vigore.

Ci sono situazioni che possono avvantaggiare la difesa quali coperture, nascondigli, come fosse, porte, compagni di taglia molto più grande della propria. Consultate i paragrafi relativi ai \hyperlink{coperture}{Nascondigli e Coperture} per capire il vantaggio che possono dare.

\subsection{Sfortuna e Fortuna nella prova d'Armi per difenderti}

Se nella prova d'Armi di difesa \textbf{tiri due} 1 avrai sicuramente evitato il colpo indipendentemente dal risultato finale ed la tua prossima Prova d'Armi di Attacco avrà Vantaggio.

Se nella prova d'Armi di difesa \textbf{tiri due} 0 sei stato colpito e l'avversario avrà Vantaggio nel danno da applicare.

\subsection{Armi da tiro}\index{Attacchi multipli armi da tiro}\label{armidatiro}\index{Armi da Tiro}\index{Armi da Lancio}

Le armi da tiro sono tutte le armi con una gittata, ovvero che possono essere lanciate o lanciano dei proiettili. Le principali armi da lancio sono gli archi, balestre, fionde ma anche pugnali, giavellotti, lance qualora siano gettate.

Il bonus al danno dato da Corpo si applica in automatico per le fionde, pugnali, giavellotti..ovvero con tutte le armi che vengono lanciate "a mano", gli archi e le balestre non lo applicano mai.

\textbf{I proiettili lanciati da Archi, Fionde, Balestre magiche non si considerano magici.\\
In caso di proiettili magici questi sommano il loro bonus magico al Tiro per Colpire ed al danno}

In ogni arma da tiro è segnata la gittata ovvero entro che distanza è possibile tirare il proiettile senza penalità. Ogni arma da tiro può colpire entro tre volte la gittata indicata.

Se l'obiettivo è entro la gittata indicata non si hanno penalità al colpire, se il target è tra il primo e secondo hai una Penalità alla PDAA di 1, Se il target è tra il secondo è terzo incremento la penalità al colpire è di -2.

Un pugnale tirato entro 6 metri non ha penalità, ma tirato tra i 6 ed i 12 metri ha Penalità -1, a distanza tra 12 e 18 metri ha penalità -2 al PDAA, oltre non può essere tirato.

\subsection{Arma Lunga} \index{Arma Lunga}\label{armalunga}

Alcune armi a due mani hanno l'attributo di Arma lunga. L'arma lunga da diritto a colpire più lontano ovvero a 3 metri. Causa 1 penalità alla prova d'Armi per difendersi. Questo bonus rimane valido finché l'avversario non entra in distanza della propria mischia.

Se l'avversario riduce la distanza a meno di 3 metri non ha più la Penalità alla PDAD.

\subsection{Carica} \index{Carica}\label{carica}

l'Azione di carica costa 6 PA più quanto necessario per coprire la distanza.

Chi carica ha Vantaggio nella PDAA ma avrà Svantaggio all'attacco dell'avversario entro la fine del round successivo.

\subsubsection{Preparare una arma lunga/da controcarica contro una carica} \index{Preparare una arma lunga contro una carica}\label{prepararearmalungacontrocarica}

Alcune armi, con l'attributo \textbf{Controcarica}, sono particolarmente efficaci per fermare una carica. E' una Reazione sollevare l'arma per preparare la controcarica. Il danno causato da queste armi è +2 in risposta ad una carica.

\subsubsection{Carica con Arma da Controcarica} \index{Controcarica}\label{caricaarmadacontrocarica}

una Carica effettuata con successo con un arma da controcarica causa 2 danni a Vigore in più.

\subsection{Attacchi con armi a spargimento} \index{Armi a spargimento}\index{Acqua santa}\index{Olio Incediato}\label{attacchiarmidaspargimento}\hypertarget{spargimento}{}

sono armi a spargimenti quelle che "spargono" il loro contenuto dove cadono, ad esempio olio incendiato/Acqua santa... Una arma a spargimento ha una gittata di 6 metri\index{Lanciare Armi a Spargimento}\index{Gittata armi a spargimento}.

In caso la difesa riesca nella prova d'Armi tira un d8 e consulta questo schemino per capire dove la boccia è caduta:

\medskip

\begin{tabularx}{0.30\textwidth}{ccc}
1& 2& 3\\
4 &\textbf{X}& 5\\
6 &7 &8\\
&\textbf{0}&\\
\end{tabularx}

\smallskip

\textbf{X} si considera il bersaglio dell'oggetto tirato. \textbf{0} il punto di origine del lancio.

Tira 2d4 per determinare lungo la direzione indicata dal d8 precedente a quanti 1.5 metri è caduto distante dal bersaglio, ovvero contate i metri dal target.

Ad esempio con il tiro del d8 faccio 5 e poi tirando 2d6 faccio 4, significa che la boccetta è caduta a destra del bersaglio a 6 metri.

E' anche possibile che ci si sia tirati sui piedi la boccetta (es faccio 7 e poi 6.. potrei averla tirata addosso ad un compagno o dietro di me!).


\subsection{Impreparato -- Colti di Sorpresa}\index{Impreparato}\index{Sorpresa}\label{coltidisorpresa}

se i personaggi vengono colti di sorpresa, ovvero non si aspettano di essere attaccati, si deve considerare questo primo round come round di sorpresa. Quando si è sorpresi si ha Svantaggio alla prova d'Armi per difendersi.

Non potrai reagire, non userai Azioni o Reazioni se non esplicitamente permesse; dal round successivo potrai dichiarare l'iniziativa ed agire normalmente. Le medesime considerazioni valgono per gli avversari.

Per valutare se un personaggio è sorpreso effettuate una prova di Corpo, se la prova riesce allora il personaggio non è sorpreso, altrimenti lo è.

\subsection{Magia in combattimento}\index{Magia in combattimento}\label{magiaincombattimento}

l'incantatore che lancia una magia mentre è in combattimento (ha un avversario in mischia o viene bersagliato da distanza) si considera Distratto.

\subsection{Bonus e Penalità in Combattimento}

il combattente impone 1 Penalità quando alla Prova d'Armi di Difesa (PDAD)

\begin{itemize}

\item ti fiancheggia, è in posizione sopraelevata, ti attacca alla schiena, arma lunga, sei abbagliato, sei intralciato, sei afferrato, combatti con luce fioca, 

\item 
il combattente impone 2 Penalità quando:
\subitem sei prono, sei ristretto, sei spaventato, ti difendi con un tipo di arma non conosciuta, 

\item 
il combattente impone Svantaggio quando:
\subitem è invisibile, è in carica, sei sorpreso

\end{itemize}

il combattente ha un Bonus nella prova d'Armi per difendersi quando:

\begin{itemize}
	
\item 
ha copertura, combatte da più in alto

\end{itemize}

\subsubsection{Quando colpisci molto bene...}

quando la differenza tra il MS di chi attacca e quello di chi difende è tra i 3 ed i 5 l'attacco causa 1 punto di danno a Vigore in più.

Se la differenza è tra 6 e 8 causa 2 danni in più.

Se la differenza è 9 o più l'attaccante ha Vantaggio nel dado dell'arma, ovvero tira due dadi per il danno l'arma e sceglie il quello che preferisce.

\subsubsection{Quando ti difendi molto bene...}

quando la differenza tra il MS di chi si difende e quello di che attacca è tra i 3 ed i 5 ottieni 1 Bonus alla prima PDAA effettuata entro la fine del round successivo.

Se la differenza è tra 6 o 8 ottieni due Bonus alla prima PDAA effettuata entro la fine del round successivo.

Se la differenza è 9 o più ottieni Vantaggio alla prima PDAA entro la fine del round successivo.

\subsubsection{Aiutare un altro}\index{Aiutare}\label{aiutare}

si può aiutare un compagno ad attaccare o a difendersi negli scontri in mischia, distraendo o interferendo con l'avversario. 

Si esegue una prova d'Armi e se si riesce si concede un bonus alla prova d'Armi per difendersi o come penalità alla prova d'Armi di difesa dell'avversario.

\subsubsection{Tiri Mirati}\index{Tiri Mirati}\label{tirimirati}\index{Mirare a parti specifiche}

Dark Catacomb non prevede la possibilità di effettuare tiri mirati con qualsiasi arma o incantesimo, tranne se questo lo specifica.

Quando si colpisce il bersaglio lo si colpisce genericamente, senza possibilità di specificare se alla testa, gamba o altro, medesimo concetto vale in caso di colpi ad oggetti, es. se miri ad un cardine di una porta colpisci tutta la porta. Questo non impedisce all'Arbitro di valutare conseguenze adeguate.

\subsubsection{Danno non letale}\index{Danno non letale}\label{dannononletale}

il danno non letale è una forma di danno causato da armi particolari o quando volutamente lo scopo è fare svenire il nemico e non ucciderlo.

Il danno non letale si tratta come il danno al Vigore ma va segnato a parte nella scheda.

\subsubsection{Danno non letale con arma non idonea} \index{Danno non letale con arma non idonea}\label{dannononletalearmanonidonea}

se vuoi fare danno non letale con un'arma non predisposta al danno non letale la prova d'Armi attacco (PDAA) ha Svantaggio.

\subsubsection{Senza Competenza}\index{Senza Competenza}\label{senzacompetenza}

usare una tipologia di arma senza l'adeguata competenza, ovvero non avere Armi a due Mani mentre si vuole usare uno Spadone, causa Svntaggio alla prova d'Arma per attaccare.

\subsubsection{Lanciare armi} \index{Lanciare armi}\label{lanciarearmi}

una spada o comunque un arma non fatta per essere lanciata, senza Gittata, può comunque essere scagliata contro l'avversario con Svantaggio.

Il danno a Vigore causato dall'arma viene dimezzato.

\subsubsection{Fiancheggiare} \index{Fiancheggiare}\label{fiancheggiare}

se due personaggi sono attorno allo stesso bersaglio ma non sono a fianco tra loro prendono un Bonus nella prova d'armi d'attacco.

Al massimo ci possono essere 4 personaggi attorno ad una creatura di taglia media che prendono il bonus di fiancheggiare.

Se tirando una ipotetica riga che collega i due personaggi questa attraversa in pieno il quadretto dell'avversario allora c'è la situazione di fiancheggiamento.

\bigskip

Esempio di fiancheggiamento\index{Esempi di Fiancheggiamento}

\medskip

\begin{tabularx}{0.45\textwidth}{lll}
\toprule
A &  G &  D\\
B & \textbf{X}  &  E\\
C &  H &  F\\
\end{tabularx}

\bigskip

In questo schema il fiancheggiamento è preso dalle coppie: A-F, B-E, C-D, G-H

\bigskip

Se la creatura può fronteggiare più creature contemporaneamente queste non godranno del bonus di fiancheggiamento.


\subsubsection{Prendere la Mira (cecchino)} \index{Prendere la Mira (cecchino)}\label{cecchino}

se dedichi 5 PA a prendere la mira ottieni un Bonus alla Prova d'Armi per Attaccare.

\subsubsection{Usare un'arma da lancio mirando ad un avversario impegnato in combattimento} \index{Usare un'arma da lancio mirando ad un avversario impegnato in combattimento}\label{usarearmalancioinmischia}

non è facile prendere la mira corretta e non colpire il proprio compagno. Hai Svantaggio alla Prova d'Armi per Attaccare. Se la prova di attacco ha un MS di -6 o meno hai colpito il compagno.

\subsubsection{Usare un'arma da lancio sotto minaccia} \index{Usare un'arma da lancio sotto minaccia}\label{usarearmalanciosottominaccia}

usare un'arma da lancio come arco, balestra o pugnale (che si vuole lanciare) mentre si è minacciati in mischia concede all'avversario Vantaggio nella prova d'Armi per difendersi.

\subsubsection{Difesa totale} \index{Difesa totale}\label{difesatotale}

costa 8 PA, non puoi eseguire nessun attacco con armi o lancio di incantesimo, guadagni Vantaggio alla prova d'Armi per difenderti.

\subsubsection{Disingaggiare} \index{Disingaggiare}\label{disingaggiare}

costa 2 PA per 1.5 metri che ti sposti. Non causi attacchi di opportunità.\index{Fare un passo}

\subsection{Manovre Opzionali in Combattimento}\label{azioniopzionaliincombattimento}

Queste Azioni di combattimento sono a discrezione dell'Arbitro che può concederle o meno.

Le Prove confrontano il Margine di Successo dei contendenti tra loro per stabilire chi riesce nella manovra.

\subsubsection{Disarmare*}\index{Disarmare}\label{disarmare}

entrambi eseguite una prova d'Armi, chi riesce con il MS maggiore disarma l'avversario. Costa 6 PA

\subsubsection{Finta*} \index{Finta}\label{finta}

entrambi eseguite una prova d'Armi, chi riesce con il MS maggiore ha Vantaggio nella prova d'Armi successiva per difendersi. Costa 6 PA

\subsubsection{Spingere un avversario*} \index{Spingere un avversario}\label{spingereavversario}\hypertarget{spingereavversario}{}

eseguite entrambi una prova di Corpo con un Bonus per ogni taglia di differenza maggiore.

Chi vince la prova con il margine maggiore puo' spingere l'avversario fino a 30 cm punteggio di margine di differenza. Costa 6 PA

\subsubsection{Afferrare un avversario*}\index{Afferrare un avversario}\label{afferrareunavversario}

eseguite entrambi una prova di Corpo, chi ha taglia maggiore ha un Bonus per taglia di differenza.

Costa 6 PA fare e mantenere e liberarsi dalla presa. Si considera che chi afferra è anche afferrato ed abbia almeno una mano occupata nell'afferrare.

Muovere una creatura afferrata richiede \hyperlink{spingereavversario}{Spingere un avversario}.

Ogni contendente può attaccare l'altro afferrato con un Arma piccola o con Rissa.

\subsubsection{Fare cadere un avversario*} \index{Fare cadere un avversari}\label{farecadereavversario}

eseguite entrambi una prova di Corpo. Chi ha più zampe/gambe dell'altro ha un Bonus.

Chi ha il MS maggiore fa cadere prono l'avversario. Costa 6 PA



\subsection{Cavalcature}\index{Combattimento a cavallo}\index{Cavallo}\label{cavalcature}

\begin{changemargin}{0.3cm}{0.3cm}\begin{enfasi}{
	- E ti puoi trovare un'altra moglie!
	
	- Ah, questo sì. ma il guaio è che mi ha portato via il fucile e il cavallo! Peccato, era così bella, io mi ci ero affezionato. Le davo qualche frustata, ma lei non ci faceva caso.
	
	- Chi, tua moglie?
	
	- No, la mia cavalla. A trovare un'altra moglie si fa presto, ma una cavalla come quella non la ritrovo più. (Ombre rosse, film 1939)}\end{enfasi}\end{changemargin}\medskip

Una cavalcatura ha le sue Azioni e di norma sono usate per spostarsi o per reagire ed ubbidire ai tuoi comandi.

Una cavalcatura agisce nel tuo round e sei tu a decidere quando esegue le sue Azioni rispetto alle tue. Non tira l'iniziativa, usa la tua.

Per fare muovere o attaccare una cavalcatura devi usare i tuoi Punti Azione.

Gli attacchi verso un personaggio a cavallo (o cavalcatura in genere) se non dichiarati diversamente mirano al cavaliere e non al cavallo.

Nella descrizione della Cavalcatura è indicato quanti metri fa per PA usato (solitamente 3 o più).

\subsubsection{Situazioni e regole}\label{cavallosituazioniregole}

\begin{itemize}
\item
Ogni qual volta la cavalcatura è colpita il cavaliere deve effettuare una prova di Cavalcare o essere disarcionato dalla cavalcatura.

Se la cavalcatura è da "guerra" (addestrata al combattimento) la prova 2 Bonus.

\item
Combattere da posizione sopraelevata concede una Penalità alla prova d'Armi per difendersi della creatura.

\item
Salire o Scendere dalla cavalcatura costa 4 PA se si ha la competenza Cavalcare, altrimenti 8 PA.

\item
Se una magia o situazione sposta (bruscamente) la cavalcatura contro la tua volontà devi effettuare una prova di Cavalcare o venire disarcionato
\end{itemize}


\subsubsection{Essere disarcionato}\label{esseredisarcionato}

Se vieni disarcionato esegui una prova di Corpo. Se riesci cadi in piedi, se fallisci cadi prono e se il fallimento è di 5 o più subisci 1d6 di danno per la caduta a Vigore.


\subsubsection{Controllare una Cavalcatura}\label{controllocavalcatura}

Mentre sei in sella, hai due scelte:

\begin{itemize}
\item puoi dare ordini alla tua cavalcatura
\item permettergli di agire da sola.
\end{itemize}

Cavalcature particolarmente intelligenti tendono a privilegiare l'autonomia di azione piuttosto che essere comandati.

Puoi controllare una cavalcatura solo se questa è stata addestrata ad accettare un cavaliere. Si presume che cavalli addestrati, muli e simili creature abbiano ricevuto tale addestramento.

L'iniziativa di una cavalcatura controllata cambia per corrispondere a quella di chi la cavalca. Si muove secondo le tue indicazioni e ha solo due opzioni di Azione: Muoversi, Attaccare.

Fare eseguire una Azione ad una cavalcatura costa l'equivalente Azione al cavaliere.

Se la cavalcatura è intelligente avere un cavaliere non restringe le azioni che la cavalcatura può effettuare e questa si muove e agisce come desidera. Potrebbe fuggire dal combattimento, lanciarsi all'attacco e divorare un nemico ferito gravemente, o agire in qualche altro modo contro la tua volontà.

\end{multicols}


\pagebreak

\section{Equipaggiamento}\hypertarget{equipaggiamento}{}\label{equipaggiamento}

\subsection{Ricchezza e Denaro}\index{Ricchezza e Denaro}
\begin{changemargin}{0.3cm}{0.3cm}\begin{enfasi}{Sono pronto ad andare. Ho uno zaino! (Morgan Grimes, Chuck, Serie TV)
}\end{enfasi}\end{changemargin}

\begin{multicols}{2}
	
\label{ricchezza-e-denaro}

In un mondo sull'orlo del collasso poche merci hanno un vero valore e sicuramente il denaro non è tra questi.

I beni che valgono sono quelli che ti permettono di vivere un giorno in più, quelle che possono darti sicurezza, proteggerti, sfamarti.

Certo, ci sono poi le armi della grande guerra. E non parlo delle armi umane ma le reliquie dei combattimenti tra angeli e demoni. Gli oggetti importanti sono pochi e preziosi.

Per tutti gli altri che sono sopravvissuti sono presentati i pochi attrezzi e strumenti che ancora possono essere trovati e prodotti da una civiltà che è già caduta, morta e sepolta.

\subsubsection{Monete e Gemme}\index{Monete e Gemme}

Mi dispiace, ma le monete di oro, argento non valgono più nulla, forse qualche antichissimo pezzo di carta, quelli che una volta chiamavano banconota, ha un valore ma solo storico.

Le gemme sono la vera moneta di scambio se non hai una gallina o una preziosa mucca. Le gemme, specialmente le più preziose possono, a volte, anche salvarti la vita specialmente se usate per corrompere quelle creature dalla pelle rossa, corna allungate e ali ossute.

Le gemme si dividono in base alla loro qualità e valore. nel gradino più basso c'è la generica \textbf{gemma grezza} (sigla GG), una pietra semipreziosa, di non specificata tipologia e non lavorata. Ogni tipologia di gemma ha una sigla per indicarne brevemente la categoria a cui appartiene

\textbf{Gemme di Bassa Qualità, GR}: agata; azzurrite; quarzo blu; ematite; lapislazzuli; malachite; ossidiana; rodocrosite; occhio di tigre; turchese; perla di fiume (irregolare). Una gemma di bassa qualità vale circa \textbf{10 gemme grezze}.


\textbf{Gemme Semi Preziose, GE}: eliotropio, corniola; calcedonio; crisoprasio; citrino; diaspro; lunaria; onice; crisolito; cristallo di roccia (quarzo chiaro); sardonice; quarzo rosato, affumicato o rosa di stella. Una gemma semi preziosa vale circa \textbf{10 gemme di bassa qualità}.

\textbf{Gemme di Media Qualità, GA}: ambra; ametista; crisoberillo; corallo; granato rosso o verde-marrone; giada; perla bianca, dorata, rosa o argentata; spinello rosso, marrone-rosso o verde scuro; tormalina. Ognuna di queste vale circa \textbf{10 gemme semi preziose}.

\textbf{Gemme di Alta Qualità, GO}: alessandrite; acquamarina; granato viola; perla nera; spinello blu scuro; topazio giallo oro.  Ognuna di queste vale circa \textbf{10 gemme di media qualità}.

\textbf{Gemme Preziose, GP}: opale bianco, nero, o di fuoco; zaffiro blu; corindone giallo fuoco o vermiglio; zaffiro a stella blu o nero. Ognuna di queste vale circa \textbf{10 gemme di alta qualità}.

\textbf{Gemme Eccezionali; GT}: smeraldo, verde brillante cristallino, diamante, rubino, giada pura.

E non provate a pagare con i minerali creati prima della guerra, tipo con lo zircone, sarete considerati alla stregua di falsari.

\subsubsection{Ricchezza iniziale}\index{Richezza iniziale}

Solitamente un personaggio appena creato ha 2d6 gemme grezze (GG) come unico suo tesoro.

\subsubsection{Altre Ricchezze - Merci di scambio}\index{Altre Ricchezze}

I mercanti di solito scambiano merci anche senza l'uso di gemme.
Per farsi un'idea delle transazioni commerciali, alcune merci di scambio sono descritte nella tabella. Ricordate che un opale può essere anche bellissimo ma non puoi prepararci del pane per sfamarti.

\medskip

\textbf{Tabella: Esempi altre ricchezze}\index{Tabella Esempi altre ricchezze}

\medskip


\begin{tabular}{ll}
\textbf{Costo} & \textbf{Oggetto}\\
\toprule
1 GG & Frumento (0.5 kg)\\
2 GG & Farina (0.5 kg) o pollo (1)\\
1 GA & Ferro (0.5 kg)\\
5 GA & Tabacco o rame (0.5 kg)\\
1 GO & Cannella (0.5 kg) o capra \\
2 GO & Zenzero o pepe (0.5 kg) o pecora (1)\\
3 GO & Maiale (1) \\
4 GO & Lino (1 m\textsuperscript{2}\\
5 GO & Sale o argento (0.5 kg) \\
10 GP& Seta (1 m) o mucca (1)\\
15 GO& Zafferano(0.5 kg)/bue (1)\\
3 GP&Chiodi di garofano (1kg)\\
\end{tabular}

\medskip

Consultate anche il capitolo sull'Ingombro in Movimento e Trasporto.

\end{multicols}

\pagebreak

\section{Equipaggiamento - Armi}\index{Equipaggiamento}\index{Armi}\label{equipaggiamentoarmi}
\hypertarget{equipaggiamento.armi}{}

\label{equipaggiamento---armi}
\begin{changemargin}{0.3cm}{0.3cm}\begin{enfasi}{
Questo è il mio fucile. Ce ne sono tanti come lui, ma questo è il mio. Il mio fucile è il mio migliore amico, è la mia vita. Io debbo dominarlo come domino la mia vita. Senza di me il mio fucile non è niente; senza il mio fucile io sono niente. Debbo saper colpire il bersaglio, debbo sparare meglio del mio nemico che cerca di ammazzare me, debbo sparare io prima che lui spari a me e lo farò. Al cospetto di Dio giuro su questo credo: il mio fucile e me stesso siamo i difensori della patria, siamo i dominatori dei nostri nemici, siamo i salvatori della nostra vita e così sia, finché non ci sarà più nemico ma solo pace, amen. (Full Metal Jacket, Film 1987)

\medskip

La spada davvero buona è quella che rimane nel suo fodero. (Sanjuro)}\end{enfasi}\end{changemargin}

\medskip

Se per i più lo scopo è sopravvivere molti altri devono imbracciare le armi per poter difendersi e difendere ciò che è a loro caro.

La tabella presenta il nome dell'arma, il suo costo in gemme grezze, il danno ed il tipo di danno (se da Taglio, Botta o Punta), la gittata, la tipologia di Arma e le caratteristiche speciali che può avere. Vedi anche \hyperref[sec:capacita-di-carico-e-trasporto-ingombro]{Capacità di Carico e Trasporto.}

Ricordo che usare un'Arma senza l'adeguata competenza da Svantaggio al PDAA.

\medskip

\textbf{Tabella: Lista della Armi}\index{Tabella Lista della Armi}

%\begin{tabularx}{lllll}
\begin{xltabular}{0.99\textwidth}{lllX}
\textbf{Arma}&\textbf{Costo}&\textbf{Dim./Danno} & \textbf{Gittata, Lista, Speciale}\\
\toprule
Alabarda& 10 & G/1d10 P/T& \textbf{Armi a due mani} Controcarica, Arma lunga \\
Arco Corto& 30 & M/1d6 P& 15 metri, \textbf{Armi da tiro}\\
Arco Lungo& 75 & G/Frecce& 20 metri, \textbf{Armi da tiro}\\
Ascia Martello& 16 & M/1d6 T/B& \textbf{Armi medie}\\
Ascia ad una mano& 6  & M/1d6 T& 6 metri, \textbf{Armi piccole}\\
Ascia da battaglia& 10 & M/1d10 T&\textbf{Armi a due mani}\\
Balestra ad una mano& 200& M/Dardi& 6 metri, \textbf{Armi da tiro}\\
Balestra pesante& 150 & G/Dardi& 30 metri \textbf{Armi da tiro}\\
Bastone& 3& M/1d6 B& \textbf{Armi medie}, Arma lunga\\
Brandistocco& 10 & M/2d4 P/T& \textbf{Armi a due mani}, Controcarica, Arma lunga\\
Catena chiodata& 55 & G/2d4 P& 3 metri, \textbf{Armi medie}, Arma lunga\\
Falce& 18 & G/2d4 P/T& \textbf{Armi a due mani}, Arma lunga\\
Falcetto& 6& P/1d6 T& \textbf{Armi piccole}\\
Falcione in asta& 12 & G/1d10 P/T& \textbf{Armi a due mani}, Controcarica, Arma lunga\\
Fionda& -& P/1d4 B& 10 metri, \textbf{Armi da tiro}\\
Flagello Pesante& 20 & M/1d10 B& \textbf{Armi a due mani}\\
Flagello& 8& M/1d8 B& \textbf{Armi medie}\\
Frusta& 1& M/1d3 T& \textbf{Armi medie}, Arma lunga\\
Giavellotto& 1& P/1d6P& 12 metri,  \textbf{Armi piccole} \textbf{Armi da tiro}\\
Grosso randello& 2& M/1d8 B&\textbf{Armi a due mani}\\
Lancia da fante& 5& M/1d8 P&3 metri, \textbf{Armi medie}, Arma lunga, Controcarica\\
Lancia& 10 & G/1d8 P&\textbf{Arma a due mani}, Arma lunga, Controcarica\\
Maglio da guerra& 15& G/1d10 B& \textbf{Armi a due mani}\\
Manganello& 1& P/1d6 B& \textbf{Armi piccole}, non letale\\
Martello da guerra& 8& M/1d8 B/P& 6 metri, \textbf{Armi medie}\\
Mazza Pesante& 5& M/1d8 B/T& \textbf{Armi medie}\\
Mazza chiodata& 10& M 1d8 B/P& \textbf{Armi medie}\\
Picca Leggera& 4& M/1d4 P&\textbf{Armi semplici}\\
Picca Pesante& 8& G/1d6 P&\textbf{Armi a due mani}, Arma lunga\\
Pugnale& 2& P/1d4 P& 6 metri, \textbf{Armi piccole}, \textbf{Armi da tiro}\\
Rissa& note*& P/1d4 B&\\
Randello& 1& P/1d6 B& \textbf{Armi piccole}\\
Scimitarra& 30 & M/1d6 T&\textbf{Armi piccole}, \textbf{Armi medie}\\
Spada Corta& 10 & P/1d6 P&\textbf{Armi piccole}\\
Spada Lunga& 20 & M/1d8 T&\textbf{Armi medie}\\
Spada Bastarda& 35 & M/1d8 T&\textbf{Armi medie}, \textbf{Armi armi a due mani}\\
Spadone a due mani& 50 & G/2d6 T&\textbf{Armi a due mani}\\
Stocco& 40 & P/1d6 P& \textbf{Armi piccole}\\
Tridente& 15 & M/1d6 P/T& 3 metri, \textbf{Armi da tiro}, \textbf{Armi medie}, \textbf{Armi a due mani}, Arma Lunga, Controcarica\\
\end{xltabular}

\medskip

Un \textbf{Arma} Piccola ha \textbf{Ingombro} 1, una Arma Media ha Ingombro 2, un Arma Grande ha Ingombro 4, un Arma Enorme ha Ingombro 8.\index{Ingombro Armi}\index{Ingombro Armi}

\medskip

\textbf{Tabella: Lista dei proiettili - Archi - Armi da tiro - Fionde}\index{Tabella Lista dei proiettili - Archi - Armi da tiro - Fionde}

\begin{tabular}{lcc}
\textbf{Nome Proiettile}& \textbf{Numero di colpi/Costo (mo)} & \textbf{Danno / Tipo}\\
\toprule
Dardi per balestra & 6/1 GG & 1d6 P\\
Frecce per arco& 20/1 GG & 1d6 P\\
Biglie di Marmo (fionde)& 15/1 GG & 1d4 B\\
Sasso (fionde)& -& 1d3 B\\
\end{tabular}

\medskip

Una \textbf{Faretra} (piena o vuota) di Proiettili ha \textbf{Ingombro} 2.\index{Ingombro Proiettili}\\

\begin{multicols}{2}
	
\subsubsection{Armi magiche}	

Solo le reliquie o le armi da loro derivate possono essere considerate magiche.

Il Bonus magico indicato nella armi si applica alla Prova d'Armi per attaccare (PDAA) e si applica al danno inflitto a Vigore.

Non è possibile acquistare armi magiche devono essere "trovate".

\textbf{Un proiettile non acquisisce proprietà magiche perché il suo lanciatore è magico.}

\bigskip



\textbf{Balestra}\index{Balestre}\index{Ricarica Balestra}
Una balestra richiede 4 PA per essere ricaricata. Una balestra leggera od a una mano richiede 2 PA per essere ricaricata.

\textbf{Gittata}\index{Gittata}
La distanza indicata è quella senza penalità alla prova d'arme per colpire. Ogni arma a distanza può colpire entro tre volte la distanza indicata.

Se l'obiettivo è entro la gittata indicata non si hanno penalità al colpire, se il target è tra il primo e secondo hai una Penalità alla PDAA di 1, Se il target è tra il secondo è terzo incremento la penalità al colpire è di -2.

Un pugnale tirato entro 6 metri non ha penalità, ma tirato tra i 6 ed i 12 metri ha Penalità -1, a distanza tra 12 e 18 metri ha penalità -2 al PDAA, oltre non può essere tirato.

Un giavellotto tirato entro 12 metri non ha penalità, ma tirato entro 24 metri ha un 1 di Penalità al PDAA, a distanza tra 24 e 36 metri un -2 al PDAA, oltre non può essere tirato.

Una \textbf{Freccia o Dardo che colpisce si considera distrutta}, se manca si considera che abbia un 50\% (4-5-6 su un d6) di probabilità che sia ancora integra.

Una Freccia/Dardo/Sasso magico somma i suoi bonus a quelli del lanciatore per determinare il PDAA ed il Danno.

\medskip

Un arma media se usata a due mani causa +2 al danno a Vigore.


Le Armi hanno indicato una \textbf{Tipologia di danno}\index{Tipologia di danno}, ovvero T/B/P.
Queste lettere stanno ad indicare se il danno è di tipo Taglio, Botta o da Perforazione. Questa caratteristica può essere importante perché determinate creature possono essere immuni o subire meno danno da un particolare tipo di ferita (es uno scheletro contro un'arma da penetrazione o un cubo gelatinoso contro un arma da taglio..).

Un arma può essere usata per causare un tipo di danno diverso (da taglio a perforazione o botta) riducendo di una categoria il dado di danno (es. Spada Lunga per fare danno da botta causa 1d6).

\medskip

\textbf{Armi Improvvisate}\index{Armi Improvvisate}

Talvolta oggetti che non sono stati creati per essere armi possono avere una certa efficacia in combattimento. Dal momento che non si tratta di oggetti pensati per questo utilizzo, la creatura che attacca con uno di essi subisce 2 Penalità al PDAA. Un'arma improvvisata di piccole dimensioni (bottiglia) fa 1d3 di danno, di medie dimensioni (la gamba di una sedia) da 1d6, di grandi dimensioni (la gamba di un tavolo) fa 1d8 di danno.

Un'arma da lancio improvvisata ha una gittata 3 metri.

\medskip

\textbf{Lanciare armi}\index{Lanciare armi}

Una spada o comunque un arma non fatta per essere lanciata può comunque essere scagliata contro l'avversario. Il PDAA ha 2 Penalità e l'arma fa una categoria di danno inferiore (la spada lunga fa 1d6, una spada corta 1d4..). La gittata è 3 metri.

\medskip

\textbf{Usare un'Arma senza l'adeguata competenza comporta Svantaggio nel PDAA}.

\end{multicols}

\pagebreak

\section{Equipaggiamento - Armature e Scudi} \index{Armature}\index{Scudi}\hypertarget{equipaggiamento.armature.scudi}{}\label{equipaggiamentoarmature}

\label{equipaggiamento---armature-e-scudi}

\begin{changemargin}{0.3cm}{0.3cm}\begin{enfasi}{
Armatura (s.f.). Abito che si indossa se il proprio sarto è un fabbro. (Ambrose Bierce)

\medskip

Armatura Fantozzi: banderuola 4 venti in funzione di pennacchio, pauroso elmo vichingo con visibilità azzerata, sospensorio in bronzo sottratto alla statua di Pipino il Breve e, ai piedi, ferroni da stiro a carbonella di piombo fuso. Peso complessivo armatura Fantozzi: 4 quintali, 32 chili e 7 etti e mezzo. (Superfantozzi, Film)} \end{enfasi}\end{changemargin}\medskip

Le armature aiutano ad assorbire il danno dei colpi e penalizzano la Prova di Magia e le Prove di competenza. A seconda dell'armatura può essere richiesto un valore di Corpo minimo.

Le Penalità Competenze è la penalità che si applica alle prove di competenza influenzate dal peso ed Ingombro dell'armatura. Armature diverse, specifiche o magiche hanno punteggio diversi, questa tabella serve come linea guida per il Arbitro.\index{Penalità armatura}

\subsubsection{Tabella Armature}\index{Tabella Armature}

\label{tabella-armature}
\begin{tabular}{llllll}
%\begin{xltabular}{0.95\textwidth}{lXXXXXXX}
\textbf{Armatura} & \textbf{Costo (GG)} & \textbf{Riduzione} & \textbf{Penalità} & \textbf{Corpo} &\textbf{Ingombro}\\
\toprule		%GG		rid pen  corpo	ingombro
Imbottita  		& 5		& 1		& 0	 & 4  &	2\\
Cuoio 			& 10	& 1d2 	& 1	 & 5  &	2\\
Cuoio rinforzato& 40	& 1d3 	& 1  & 6  & 2\\
Giaco di Maglia & 100 	& 1d4 	& 2  & 9  & 4\\
Scaglie			& 150 	& 1d6 	& 3  & 10 & 4\\
Anelli 			& 250  	& 1d6+1 & 4  & 12 & 6\\
Pettorale  		& 400  	& 1d8 	& 5  & 13 & 7\\
Mezza armatura  & 1200 	& 1d10 	& 6  & 14 & 10\\
Completa		& 1500 	& 1d10+1& 6  & 15 & 10\\
\end{tabular}

\begin{multicols}{2}

\textbf{Costo}: è per un armatura di taglia media espresso in gemme grezze.

\textbf{Riduzione}: è di quando il danno a Vigore viene ridotto. Il giocatore tira il dado segnato e diminuisce il danno di quell'ammontare con un \textbf{minimo danno subito di 1}.

\textbf{Penalità Comp.}: è il valore da sottrarre alle prove di Competenza dato dal peso ed Ingombro dell'armatura.

\textbf{Corpo}: è il requisito minimo di punteggio di Corpo per portarla senza penalità, con un punteggio inferiore muoversi costa 1 PA in più 1.5 metri per differenza tra il punteggio di Corpo e quello necessario.

\medskip

\begin{changemargin}{0.3cm}{0.3cm}\begin{narratore} Quando conteggiate l'Ingombro dato dall'armatura e scudo \textbf{indossato} dovete dividerlo per due.

L'Ingombro di armatura e scudi è da intendersi quando è "caricata nello zaino", ovvero trasportata ma non indossata.\end{narratore}\end{changemargin}

\subsubsection{Descrizione delle Armature}


\textit{Imbottita}. Le armature imbottite consistono di strati di tessuto e imbottitura cuciti insieme.

\textit{Cuoio}. Il corpetto e le protezioni delle spalle di questa armatura sono fatte di cuoio indurito dopo essere stato bollito nell'olio. Il resto dell'armatura è composto di
materiali più morbidi e flessibili.

\textit{Cuoio Rinforzato}. Fatta di cuoio duro ma flessibile, l'armatura di cuoio rinforzato è arricchita da rivetti o spuntoni.

\textit{Giaco di Maglia}. Composto di anelli metallici intrecciati tra di loro, un giaco di maglia viene indossato sopra strati di abiti o cuoio. Questo tipo di armatura offre una protezione modesta alla parte superiore del corpo, mentre il rumore degli anelli che strusciano fra di loro viene attutito dagli altri strati.

\textit{Scaglie}. Quest'armatura consiste in una cotta e gambali (a volte anche di una gonna separata) di cuoio coperti da pezzi di metallo sovrapposti, in maniera simile alle scaglie di un pesce. L'armatura è completa di guanti.

\textit{Anelli}. Quest'armatura è un'armatura di cuoio con dei pesanti anelli cuciti sopra. Gli anelli servono a rinforzare l'armatura contro i colpi di spada e d'ascia. L'armatura è completa di guanti.

\textit{Pettorale}. Questa armatura consiste di un corpetto di metallo indossato su di uno strato di cuoio. Sebbene lasci braccia e gambe relativamente scoperte, l'armatura fornisce una buona protezione agli organi vitali del personaggio, senza procurargli grande ingombro.

\textit{Mezza Armatura}. La mezza armatura di piastre consiste di piastre di metallo sagomate che coprono gran parte del corpo del personaggio. Non comprende protezioni per le gambe oltre a dei semplici schinieri legati con lacci di cuoio.

\textit{Completa}. Quest'armatura consiste di piastre di metallo sagomate a incastro che coprono l'intero corpo. Un'armatura di piastre comprende guanti, stivali di cuoio pesanti, un elmo con visiera, e uno spesso strato di imbottitura sotto l'armatura. Fibbie e lacci distribuiscono il peso dell'armatura su tutto il corpo.


\subsubsection{Regole base per l'utilizzo dell'armatura}

\textbf{Dormire in Armatura}: se si dorme in un'armatura media o pesante, il giorno seguente si è automaticamente \hyperlink{affaticato}{Affaticati}.

Dormire in un'armatura con ingombro 2 o meno non provoca Affaticamento.

\textbf{Peso}: il peso indicato si riferisce alla versione per personaggi di taglia Media. Le armature adattate per personaggi di taglia Piccola pesano la metà, mentre per quelli di taglia Grande pesano il doppio.

\textbf{Armature magiche}\index{Armature magiche}\index{Scudi magici}

Un armatura magica o scudo magico non solo protegge meglio ma è anche più leggera e affine alla magia. Una armatura magica ha un più alto valore di Riduzione e peso ed ingombro minore.

\subsubsection{Gli Scudi}

Gli \textbf{Scudi} \index{Scudi}permettono di aumentare la prova di PDAD. 

Gli Scudi possono essere di tipo Leggero, Medio, Pesante.

\end{multicols}

\subsubsection{Tabella Scudi}\index{Tabella Scudi}

\label{tabella-scudi}

\begin{tabular}{lllll}
\textbf{Scudi} & \textbf{Costo} & \textbf{PDAD} & \textbf{Tipo} & \textbf{Ingombro}\\
\toprule
Scudo leggero di legno 		& 3 GG   &  1		& L&1\\
Scudo leggero di metallo 	& 20 GG  &  1d2 	& L&1\\
Scudo medio legno			& 10 GG  &  1d2+1	& M&2\\
Scudo medio metallo 		& 30 GG  &  1d4  	& M&2\\
Scudo pesante di legno 		& 30 GG  &  1d4+1  	& P&4\\
Scudo pesante di metallo	& 80 GG  &  1d6  	& P&4\\
\end{tabular}

\begin{multicols}{2}

\textbf{PDAD}: è il valore che si aggiunge alla prova di PDAD.

\textbf{Tipo}: indica la taglia dello scudo. \textbf{L}eggero, \textbf{M}edio, \textbf{P}esante.

Uno \textbf{Scudo} Leggero ha \textbf{Ingombro} 1, uno Scudo Medio ha Ingombro 2, uno Scudo Pesante ha Ingombro 4.\index{Imgombro per Scudi}

Il valore di Ingombro si sottrae alla prove di Competenza.

Uno scudo può essere usato come \textbf{arma improvvisata}. Il PDAA ha Svantaggio. Uno scudo piccolo fa 2 di danno (B/T), uno scudo medio fa 1d4 di danno (B/T), uno scudo pesante fa 1d6 di danno (B/T).

Imbracciare uno scudo occupa una mano/braccio.

\subsubsection{Indossare e Togliere Armature}\index{Indossare e Togliere Armature}

Indossare e togliere armature è una operazione che richiede tempo ed attenzione, farlo in fretta non aiuta ed anzi tende a peggiorare la protezione data dall'armatura.

\end{multicols}

\textbf{Tabella: Tempi per indossare e togliere l'armatura}\index{Tabella Tempi per indossare e togliere l'armatura}

\begin{tabular}{llll}
\textbf{Tipo di Armatura}& \textbf{Indossare} & \textbf{Indossare in fretta} & \textbf{Togliere}\\
\toprule
Scudo& 1 azione & - & 1 azione\\
Imbottita, Cuoio, Cuoio rinforzato  & 1 minuto& 3 round  & - \\
Giaco di Maglia& 1 minuto& 5 round  & 5 round\\
Scaglie, Anelli, Pettorale& 4 minuti & 1 minuto{*}  & 1 minuto\\
Completa  & 4 minuti{*}{*}& 4 minuti{*}& 1d4+1 minuti\\
\end{tabular}

\bigskip

\begin{multicols}{2}

{*} Se qualcuno aiuta, il tempo si dimezza. Un singolo personaggio che non sta facendo altro può aiutare uno o due personaggi adiacenti a lui. Due personaggi non possono aiutarsi l'un l'altro a indossare un'armatura contemporaneamente.

{*}{*} Bisogna essere aiutati per indossare questa armatura. Senza aiuto è possibile indossarla solo in fretta.

\textbf{Indossare un'armatura in fretta} implica penalità di -2 alla Riduzione.


\end{multicols}




\section{Merci e Servizi}\index{Merci}\index{Servizi}


\subsection{Ricchezza e Denaro}\index{Ricchezza e Denaro}


\begin{changemargin}{0.3cm}{0.3cm}\begin{enfasi}{
- Doc... c'è soltanto bisogno di un pochino di plutonio.

\medskip

- Ah, sono certo che nell'85 il plutonio si compra nella drogheria sotto casa, ma nel '55 la faccenda è molto più complicata! (Ritorno al futuro, Film 1985)}
\end{enfasi}\end{changemargin}\medskip

\begin{multicols}{2}

\subsubsection{Vendere Tesori}

Nei sotterranei che esplorerai avrai qualche opportunità di trovare tesori, equipaggiamento, armi, armature e altro ancora. Di solito, potrai vendere tesori e ninnoli quando raggiungerai un paese o altro insediamento, purché tu riesca a trovare acquirenti e mercanti interessati al tuo bottino.

\medskip

\textbf{Armi, Armature e Altro Equipaggiamento }

Come regola generale, le armi, le armature e il resto dell’equipaggiamento non danneggiato quando viene venduto costa la metà. È difficile che le armi e le armature utilizzate dai mostri siano in condizioni ottimali per la vendita. Tranne se reliquie.

\medskip

\textbf{Oggetti Magici}

La vendita di oggetti magici è un problema. Trovare qualcuno che voglia comprare una pozione o pergamena non comporta grandi difficoltà, ma la maggior parte degli oggetti sono fuori della portata delle tasche di chiunque salvo dei nobili più ricchi. Il valore della magia trasale la vile gemms e dovrebbe essere sempre trattato con riguardo.

\medskip

\textbf{Merci}

Sulle terre di confine, la maggior parte delle transazioni avvengono tramite baratto. Come le gemme e gli oggetti d’arte, le merci - lingotti di ferro, sacchi di sale, bestiame e così via - possono essere scambiate come gemme grezze al loro pieno valore.


\section{DOVE SONO ARRIVATO con la revisione}


\subsubsection{Equipaggiamento da Avventura}\index{Equipaggiamento avventura}\index{Cose da comprare}

Questo è un breve e non esaustivo elenco di equipaggiamento che i vostri personaggi potrebbero essere interessati a comprare. L'elenco non è certo esaustivo o completo ma potrà fornirvi linee guida sui prezzi.

Come Arbitro usate sempre il buon senso nelle richieste valutate bene la tipologia di richiesta, la necessità dell'oggetto, il luogo dove si compra e come lo si compra.

In base alla tipologia di compagna potrebbero essere disponibili ulteriori oggetti quali armi da fuoco o alchemici.

\medskip

{\small
\begin{tabularx}{0.42\textwidth}{lll}
\textbf{Oggetto}    & \textbf{Costo} & \textbf{Ing.}\\
Abaco&2 mo&L\\
Abito da Monaco & 5 mo& 1\\
Abito da artigiano& 1 mo& 1\\
Abito da contadino& 1 ma& 1\\
Abito da cortigiano& 30 mo  & 1\\
Abito da esploratore  & 10 mo& 1\\
Abito da intrattenitore & 3 mo& 1\\
Abito da nobile & 75 mo  & 2\\
Abito da studioso & 5 mo& 1\\
Abito da viaggiatore  & 1 mo& 2\\
Abito invernale & 8 mo& 2\\
Abito reale & 200 mo & 3\\
Acciarino e pietra focaia & 1 mo&\\
Acido Intenso (ampolla) & 10 mo  & L \\
Acqua santa (ampolla) & 25 mo& L\\
Ago da cucito & 5 ma &- \\
Agrifoglio e vischio  & -  & -\\
Amo da pesca  & 1 ma& - \\
Ampolla (vuota)& 3 mr& L \\
Anello con sigillo  & 5 mo& - \\
Anello per veleno & +20 mo&-\\
Antitossina (boccetta)  & 50 mo  & L\\
Ariete portatile  & 10 mo& 3 \\
Arnesi da artigiano& 5 mo& 2\\
Arnesi da scasso  & 30 mo& 1\\
Asta (3 m)  & 5 mr& 2\\
Attrezzi da scalatore & 80 mo& 1\\
Banchetto (a persona) & 10 mo  & -\\
Bandoliera & 3 mo & L\\
Barca a remi& 50 mo  & 12\\
Barcone & 3000 mo  & -\\
Barile (vuoto)& 2 mo& 4\\
Bastone & 2 mo& 1\\
Bilancia da mercante  & 2 mo& 1\\
Birra Boccale& 5 mr& L\\
Birra Caraffa& 2 ma& L\\
Boccale di ceramica & 2 mr& L\\
Boccetta di inchiostro o pozione  & 1 mo& L \\
Borsa&5 ma&L\\
Borsa da cintura (vuota) & 1 mo& L\\
Borsa per Componenti &25 mo&L\\
Borsa del guaritore& 50 mo  & 1\\
\end{tabularx}

\begin{tabularx}{0.42\textwidth}{lll}
\textbf{Oggetto}    & \textbf{Costo} & \textbf{Ingombro}\\
Bottiglia di vetro  & 2 mo& L \\
Brocca di ceramica  (5lt) & 2 mr& L\\
Campanella  & 1 mo& - \\
Candela & 1 mr& -\\
Canna da pesca & 1 mo&1\\
Cannocchiale  & 1000 mo  & 1 \\
Caraffa di ceramica & 2 mr& L\\
Carne (1 pezzo) & 3 ma& L\\
Carretto  & 15 mo  & 10\\
Carro & 35 mo& 20\\
Carrozza  & 300 mo & -\\
Carrucola e paranco & 20 mo& 2 \\
Carta (foglio)& 4 ma& -\\
Cassa (vuota) & 2 mo& 3 \\
Catena (3 m)  & 30 mo & 1\\
Ceralacca& 1 mo& -\\
Cerata&5 ma&1\\
Cesto (vuoto) & 4 ma& 1 \\
Chiodo da rocciatore& 1 ma&L\\
Clessidra& 25 mo  & -\\
Coperta invernale & 5 ma& 1 \\
Corda di canapa (15 m)& 1 mo& 1\\
Corda di canapa grossa (15 m)& 2 mo& 2 \\
Corda di seta (15 m)& 10 mo  & L\\
Cote per affilare & 2 mr& L \\
Custodia per Dardi o Frecce  & 1 mo& 1 \\
Custodia per mappe o pergamene  & 1 mo& 1 \\
Fischietto  & 8 ma& - \\
Formaggio (1 pezzo)& 1 ma& \\
Forziere & 5 mo&4\\
Fuoco dell'alchimista (ampolla)& 20 mo& L\\
Galea & 30k mo  & -\\
Gancio di metallo & 1 mo& L\\
Gessetto, (1 pezzo) & 1 mr& -  \\
Giaciglio& 1 ma& 1 \\
Inchiostro (boccetta da 30 g)& 8 mo& - \\
Laboratorio da alchimista& 200 mo  & 5\\
Lanterna comune& 1 mo& 2 \\
Lanterna a lente sporgente  & 12 mo  & 1 \\
Lanterna schermabile& 7 mo& 1 \\
Legna da ardere (per giorno)& 1 mr& 4 \\
Lente d'ingrandimento & 100 mo & -\\
Locanda Buona (dormire) & 2 mo& -\\
Locanda Normale (dormire)& 5 ma& -\\
Locanda Scadente (dormire) & 2 ma& -\\
Maglio& 1 mo& 2 \\
Manette & 15 mo  & L \\
Martello& 5 ma& 1  \\
Morso e briglie & 2 mo&1\\
Nave a vela & 10k mo & -\\
Nave da guerra  & 25k mo  & -\\
Nave lunga  & 10k mo & -\\
Olio da lanterna& 1 ma& 1 \\
Orologio ad acqua & 1000 mo & -\\
Otre  & 1 mo& 2 \\
Pala o badile & 2 mo& 1 \\
Pane (a pagnotta) & 2 mr& -\\
Pasti (al giorno) Buono & 5 ma&-\\
Pasti (al giorno) Normale& 3 ma&-\\
Pasti (al giorno) Scadente  & 1 ma&-\\
Pennino & 1 ma& - \\
Pentola di ferro  & 8 ma& 1 \\
Pergamena (Foglio)  & 2 ma& - \\
Piccone da minatore & 3 mo& 2 \\
Piede di porco& 2 mo& 1 \\
Pozione di Cura & 50 mo & L\\
Pozione di Cura potenziata & 125 mo & L\\
\end{tabularx}


\begin{tabularx}{0.42\textwidth}{lll}
\textbf{Oggetto}    & \textbf{Costo} & \textbf{Ingombro}\\
Profumo & 5 mo & L\\
Rampino & 1 mo& 1 \\
Razioni da viaggio (al giorno)  & 5 ma& 1 \\
Remo  & 2 mo& 2\\
Rete da pesca (2,25 m)& 4 mo& 1 \\
Sacche da sella & 4 mo& 2\\
Sacco (vuoto) & 1 ma& L \\
Sapone (per 0,5 kg) & 5 ma& - \\
Scala a pioli (3 m) & 2 ma& 3 \\
Secchio (vuoto)& 5 ma& L\\
Sella Da galoppo  & 30 mo& 2\\
Sella Militare  & 50 mo  & 3\\
Sella da carico & 15 mo  & 2\\
Sella esotica& 40 mo& 3\\
Serratura/lucchetto Buona & 80 mo  & -\\
Serratura/lucchetto Media & 40 mo&  \\
Serratura/lucchetto Semplice& 20 mo  & -\\
Serratura/lucchetto Superiore& 150 mo  & - \\
Sfere Metalliche (100) & 3 mo & 1\\
Simbolo sacro d'argento & 25 mo& L\\
Simbolo sacro di legno  & 1 mo& L\\
Slitta& 20 mo  & 3 \\
Specchio piccolo di metallo & 10 mo  & L\\
Stallaggio (al giorno)  & 5 ma& -\\
Strumento musicale comune& 5 mo& 2\\
Tagliola& 5mo&3\\
Tela (per mq)& 1 ma& L \\
Tenda & 10 mo  & 3 \\
Torcia& 1 ma& 1\\
Tribolo (20) & 1 ma& L \\
Trucchi per il camuffamento & 50 mo& L\\
Vanga o Badile & 1 mo&1\\
Veste da Devoto & 5 mo& 1\\
Vino Buono (bottiglia) & 10 mo& 1\\
Vino Comune (caraffa)  & 2 ma& 1\\
Zaino & 2 mo& 1 \\
\end{tabularx}}


\begin{changemargin}{0.3cm}{0.3cm}\begin{enfasi}{
Ogni tecnologia sufficientemente avanzata è indistinguibile dalla magia. (Arthur C. Clarke, da Profiles of the Future)
}\end{enfasi}\end{changemargin}

\textbf{Acido Intenso}. Con un’azione, puoi spargere il contenuto di questa fiala su di una creatura entro 1 metri da te o lanciare la fiala fino a 6 metri, fracassandola all’impatto. In entrambi i casi, effettua un Tiro per Colpire a distanza contro la creatura o l’oggetto, trattando l’acido come un’arma improvvisata (-1d6 Tiro per Colpire). Se colpisci, il bersaglio subisce 2d6 danni da acido.

\textbf{Acqua santa}. Con un’azione, puoi spargere il contenuto di questa ampolla su di una creatura entro 1 metri da te o lanciare l’ampolla fino a 6 metri, fracassandola all’impatto. In entrambi i casi, effettua un Tiro per Colpire a distanza contro la creatura o l’oggetto, trattando l’Acqua santa come un’arma improvvisata. Se colpisci, e il bersaglio è un immondo o un non morto, subisce 2d4 danni da energia positiva.

\textbf{Ampolla (vuota)}: piccola anfora in vetro o ceramica con collo sottile.

\textbf{Anello con Sigillo:} cerchietto di metallo, generalmente pregiato, con un'incisione atta ad imprimere sigilli su ceralacca.

\textbf{Anello per Veleno:} +20 mo, rispetto a costo anello, questo anello ha un piccolo scompartimento sotto la gemma, di solito utilizzato per contenere del veleno. Aprirlo e chiuderlo richiede un'azione; farlo senza essere notati richiede una prova di Mani di Fata con DC 20.

\textbf{Antitossina}. Una creatura che beve da questa fiala di liquido ottiene +1d6 sui Tiri Salvezza contro il veleno per 1 ora. Non conferisce alcun bonus ai non morti e ai costrutti.

\textbf{Ariete Portatile}. Puoi usare un ariete portatile per abbattere le porte. Nel farlo, ottieni un bonus di +1d6 alle prove di Forza. Un altro personaggio può aiutarti con l’uso dell’ariete, dandoti +2 sulla prova.

\textbf{Attrezzatura da Pesca}. Questo kit comprende un’asta di legno, filo di seta, taglierino di legno, ami d’acciaio, peso di piombo, esche di velluto e un retino.

\textbf{Bandoliera}. Questa cintura specializzata per contenere piccoli oggetti quali pozioni o pergamene si porta al collo. Estrarre un oggetto da essa costa 1 Azione, come se fosse alla cintura.

\textbf{Biglie di Metallo}. Con un’azione, puoi spargere una singola borsa di queste minuscole biglie di metallo per coprire un’area piana quadrata di 3 metri di lato. Una creatura che attraversa l’area coperta deve superare un Tiro Salvezza su Riflessi con DC 12 o cadere prona. Una creatura che attraversa l’area a metà velocità non deve effettuare il Tiro Salvezza.

\textbf{Bilancia da Mercante}. Una bilancia da mercante include un piccolo bilanciere, un piatto, e un assortimento di pesi fino a 1 chilo. Con essa, puoi misurare il peso esatto di piccoli oggetti, come metalli preziosi o merci, per aiutarti a determinarne il valore.

\textbf{Borsa dei Componenti}. Una borsa dei componenti è un piccolo borsello da cinta di cuoio impermeabile munito di compartimenti contenenti tutte le componenti materiali e altri oggetti speciali di cui hai bisogno per lanciare i tuoi incantesimi, eccetto per quelle componenti che hanno un costo specifico o sono materiali non comuni (come indicato nella descrizione dell’incantesimo).

\textbf{Borsello}. Un borsello di tessuto o cuoio può contenere, tra le altre cose, fino a 20 proiettili da fionda o 50 aghi da cerbottana. Un borsello diviso in compartimenti per contenere componenti per incantesimi viene detto borsa dei componenti.

\textbf{Candela}. Per 1 ora di tempo reale di gioco, una candela proietta luce in un raggio di 1,5 metri e luce fioca per ulteriori 1 metro.

\textbf{Cerata}. E' un mantello trattato per essere idrorepellente, ti permette di rimanere asciutto anche sotto la pioggia.

\textbf{Cannocchiale}. Gli oggetti osservati tramite un cannocchiale sono ingranditi al doppio delle loro dimensioni.

\textbf{Carrucola e Paranco}. Una serie di leve collegate da un cavo e un gancio per attaccarsi ad oggetti, carrucola e paranco ti permettono di tirare su fino a quattro volte il
peso che puoi normalmente sollevare.

\textbf{Catena}. Una catena ha 15 Punti Ferita e durezza 6. Può essere spezzata superando una prova di Forza con DC 24.

\textbf{Chiodi da rocciatore}. Se ne deve usare 1 almeno ogni 6 metri per fissare la corda alla parete.

\textbf{Colonia di Scarafaggi Necrofagi}: 3 mo, questa giara di vetro contiene scarafaggi necrofagi carnivori. Gli scarafaggi devono essere nutriti con almeno 125 grammi di carne al giorno oppure muoiono. Quando rilasciati su un organismo morto, ne divorano le carni in 1d4 giorni, lasciando solo le ossa. Gli scarafaggi necrofagi mangiano soltanto la carne morta e non possono danneggiare le creature viventi. Una volta rilasciati, gli scarafaggi non possono essere rimessi nella giara.

\textbf{Corda}. Una corda, che sia fatta di canapa o seta, ha 2 Punti Ferita e può essere spezzata superando una prova di Forza con DC 19. La versione grossa ha 4 Punti Ferita, DC 22.

\textbf{Corda di Seta} (15 m): 10 mo, questa corda ha 4 Punti Ferita e può essere spezzata con una prova di Forza con DC 24

\textbf{Faretra}. Una faretra può contenere fino a 12 frecce\index{Faretra}.


\textbf{Fuoco Alchemico}. Questo fluido appiccicoso si incendia quando entra a contatto con l’aria. Con due azioni, puoi lanciare questa ampolla fino a 6 metri, fracassandola all’impatto. Effettua un Tiro per Colpire a distanza contro la creatura o l’oggetto, trattando il fuoco alchemico come un’arma improvvisata. Se colpisci, il bersaglio subisce 1d6 danni da fuoco all’inizio di ciascun suo round. Una creatura può porre fine a questi danni spendendo due Aziono e superando una prova di Destrezza con DC 12. Se la prova riesce le fiamme si estinguono.

\textbf{Borsa da Guaritore}. Questo kit è una borsa di cuoio contenente bende, unguenti e stecche. Il kit può essere usato dieci volte. Concede un +2 alle prove di pronto soccorso.

\textbf{Kit da Pranzo}. 4 mo. Questa piccola scatola di latta contiene una ciotola e delle semplici posate. Le due parti della scatola possono essere staccate, e un lato impiegato come pentola per cucinare e l’altro come piatto o contenitore

\textbf{Kit da Scalatore}. 8 mo. Un kit da scalatore comprende chiodi speciali, punte per stivali, guanti e un’imbracatura. Puoi ancorarti usando il kit da scalatore con un’azione; quando lo fai, non puoi cadere per più di 7 metri dal punto in cui ti sei ancorato, e non puoi arrampicarti a più di 7 metri di distanza dal punto a cui ti sei ancorato senza prima disfare l’ancora.

\textbf{Lanterna}. Una Lanterna proietta luce intensa in un raggio di 3 metri e luce fioca per ulteriori 6 metri. Una volta accesa, brucia per 3 ore di tempo reale di gioco con un’ampolla (0,5 litri) d’olio.


\textbf{Lanterna a Lente Sporgente}. Una lanterna a lente sporgente proietta luce in un cono di 3 metri e luce fioca per ulteriori 9 metri. Una volta accesa, brucia per 3 di tempo reale di gioco ore con un’ampolla (0,5 litri) d’olio.

\textbf{Lanterna Schermabile}. Una lanterna schermabile proietta luce in un raggio di 6 metri e luce fioca per ulteriori 6 metri. Una volta accesa, brucia per 1 ora di tempo reale di gioco con un’ampolla (0,5 litri) d’olio. Con un’azione, puoi abbassare la schermatura, riducendo la luce a fioca con un raggio di 1 metri.

\textbf{Lente Ingranditrice}. Questa lente permette di dare un’occhiata più ravvicinata agli oggetti piccoli. È anche un utile sostituto per pietra focaia e acciarino nell’accendere un fuoco. Appiccare un fuoco con la lente ingranditrice richiede almeno una luce di intensità pari a quella solare, legna da accendere, e circa 5 minuti di tempo perché il legno prenda fuoco. Una lente ingranditrice fornisce aiuto (+1d6) in qualsiasi prova effettuata per valutare o analizzare un oggetto piccolo o molto dettagliato. 

\textbf{Lente del Cacciatore:} 100 mo, questa complessa lente viene posta su un occhio e occupa lo slot occhi quando è in uso. Quando la si utilizza con un attacco a distanza, si riduce di 1d6 le penalità da attacchi a distanza. Gli oggetti entro 9 metri diventano difficili da vedere, e si subisce penalità -1d6 alle prove di Consapevolezza basate sulla vista e Tiri per Colpire.

\textbf{Manette}. Questi strumenti di metallo possono imprigionare una creatura Piccola o Media. Per liberarsi dalle manette bisogna superare una prova di Destrezza con DC 24. Per romperle bisogna superare una prova di Forza con DC 24. Ogni set di manette è fornito di una chiave. Senza la chiave, una creatura può usare Artista della Fuga o Disattivare Congegni per aprire la serratura superando una prova DC 18. Le manette hanno 15 Punti Ferita e Durezza 2

\textbf{Olio}. Di solito si compra in un’ampolla d’argilla che contiene 0,5 litri. Con un’azione, puoi spargere l’olio in questa ampolla su di una creatura entro 1 metri da te o lanciarla fino a 6 metri, fracassandola all’impatto. In entrambi i casi, effettua un Tiro per Colpire a distanza contro la creatura o l’oggetto, trattando l’olio come un’arma improvvisata. Se colpisci, il bersaglio è ricoperto d’olio. Se il bersaglio subisse qualsiasi entità di danno da fuoco prima che l’olio si asciughi (dopo 1 minuto), il bersaglio subisce altri 1d6 danni da fuoco dall’olio infiammato per round. Se infiammato, l’olio brucia per 2 round e infligge 1d6 danni da fuoco a qualsiasi creatura che entri nell’area o termini il suo round dentro di essa. Una creatura può subire questo danno solo una volta per round. Puoi anche versare un’ampolla d’olio sul pavimento per coprire un quadrato di 1 metro di lato, purché la superficie sia piana.

\textbf{Piede di Porco}. Utilizzare un piede di porco dà +1d6 alle prove di Forza ogni volta si possa applicare la leva del piede di porco.

\textbf{Pozione di Cura}. Questa pozione generica di cura consente di recuperare 1d8+1 Punti Ferita.

\textbf{Pozione di Cura potenziata}. Questa pozione generica di cura consente di recuperare 3d8+3 Punti Ferita.

\begin{changemargin}{0.3cm}{0.3cm}\begin{narratore} Per quanto sia personalmente contrario all'acquisto di oggetti magici da parte dei personaggi le Pozioni di Cura devono essere disponibili.
\end{narratore}\end{changemargin}

\textbf{Razioni}. Le razioni consistono di cibo secco adatto a lunghi viaggi, e includono carne secca, frutta secca, gallette e noci.

\textbf{Scatola con l’Esca}. Questo piccolo contenitore contiene pietra, acciarino ed esca (di solito uno straccio secco imbevuto d’olio) impiegati per appiccare un fuoco. Utilizzarlo per accendere una torcia (o qualsiasi altro oggetto facilmente incendiabile) richiede due azioni. Accendere qualsiasi altro fuoco richiede 1 minuto.

\textbf{Scatola per Mappe o Pergamene}. Questa scatola cilindrica di cuoio può contenere, arrotolati, fino a dieci pezzi di carta o cinque fogli di pergamena.

\textbf{Faretra per Quadrelli da Balestra}. Questa scatola di legno contiene fino a 12 quadrelli per balestra.

\textbf{Serratura}. Insieme alla serratura viene fornita una chiave. Senza la chiave, una creatura può scassinare questa serratura superando una prova di Disattivare Congegni con DC 17. Il Arbitro può decidere che per cifre maggiori sono disponibili serrature di qualità migliore.

\textbf{Simbolo Sacro}. Un simbolo sacro è la raffigurazione di un Patrono. Potrebbe essere un amuleto che raffigura il simbolo di un Patrono, lo stesso simbolo accuratamente inciso o intrecciato su di un emblema o scudo, o una minuscola scatola contenente una reliquia sacra.

\textbf{Tappi per Orecchie} 3 mr, fatti di cotone o sughero cerato, i tappi per orecchie concedono Bonus +2 al Tiro Salvezza contro gli effetti che richiedono l'udito ma infliggono penalità -4 alle prove di Consapevolezza basate sull'udito.

\textbf{Tenda}. Un semplice riparo portabile di tela, una tenda può contenere due persone. Ci vogliono circa 20 minuti per montare una tenda.

\textbf{Torcia}. Una torcia brucia per \textbf{1 ora di tempo di gioco reale}, fornendo luce in un raggio di 3 metri e luce fioca per ulteriori 6 metri. Se effettui un Tiro per Colpire con una torcia accesa, arma improvvisata, e colpisci, infliggi 14d di danno più 1 danno da fuoco aggiuntivo. \index{Torcia}

\textbf{Trappola da Caccia}. 12 mo, 2. Usi due azioni per disporre questa trappola, formata da un anello d’acciaio seghettato, che scatta quando una creatura calpesta la piastra metallica al centro di essa. La trappola è fissata tramite una catena pesante a un oggetto immobile, come un albero o uno spuntone conficcato nel terreno. Una creatura che calpesti la piastra deve superare un Tiro Salvezza su Riflessi con DC 15 o subire 1d4 danni perforanti e interrompere il movimento. Una creatura può usare 2 azioni per superare una prova di Forza con DC 15, e se la riesce si libera o libera un’altra creatura a portata. Ogni tentativo fallito infligge 1 danno perforante alla creatura intrappolata.


\textbf{Tribolo}. Con un’azione, puoi spargere una singola borsa di questi minuscoli triboli per coprire un’area quadrata di 1 metri di lato. Una creatura che attraversa l’area coperta deve superare un Tiro Salvezza su Riflessi con DC 17 o subire 1 danno perforante. Finché la creatura non recupera almeno 1 punto ferita, la sua velocità a piedi è diminuita di 3 metri. Una creatura che attraversa l’area a metà velocità non deve effettuare il Tiro 

\textbf{Veleno Base}. Puoi usare il veleno in questa fiala per coprire un’arma tagliente o perforante o fino a tre pezzi di munizioni. Applicare il veleno necessita un’azione. Una creatura colpita da un’arma o munizione avvelenata deve superare un Tiro Salvezza su tempra con DC 12 o subire 1d4 danni da veleno.
Una volta applicato, il veleno mantiene la sua efficacia per 1 minuto prima di seccarsi.

\subsubsection{Dotazioni di base}
Se il personaggio sceglie di acquistare il suo equipaggiamento di partenza, può acquistare una dotazione al prezzo indicato, che generalmente è più conveniente rispetto all'acquisto dei singoli oggetti separati.

\textbf{Dotazione da Avventuriero (18 mo)}. Include uno zaino, un piede di porco, un martello, 10 chiodi da rocciatore, 10 torce, un acciarino e pietra focaia, 10 razioni giornaliere e un otre. La dotazione include anche 15 metri di corda di canapa legata allo zaino.

\textbf{Dotazione da Cacciatore (24 mo)}: contiene acciarino e pietra focaia, una borsa da cintura, una corda 18m, un giaciglio, una cerata, un otre, una pentola di ferro, razioni da viaggio (5 giorni), torce (10) e uno zaino.

\textbf{Dotazione da Diplomatico (57 mo)}. Include un forziere, 2 custodie per mappe e pergamene, un abito pregiato, una boccetta di inchiostro, un pennino, una lanterna, 2 ampolle di olio, 5 fogli di carta, una fiala di profumo, cera per sigillo e sapone.


\textbf{Dotazione da Devoto (30 mo)}: contiene acciarino e pietra focaia, una borsa da cintura, una Borsa per Componenti di Incantesimi, candele (10), corda 18m, un giaciglio, una pentola di ferro, un otre, razioni da viaggio (per 5 giorni), sapone, un simbolo sacro di legno, un testo sacro economico, torce (10) e uno zaino.

\textbf{Dotazione da Esploratore (15 mo)}. Include uno zaino, un giaciglio, una gavetta, un acciarino e pietra focaia, 10 torce, 10 razioni giornaliere e un otre. La dotazione include anche 15 metri di corda di canapa legata allo zaino.

\textbf{Dotazione da Esploratore di caverne (24 mo)}: contiene un insieme di attrezzi di base per esplorare rovine e città abbandonate include 2 candele, un gessetto, un martello e 4 Chiodi da Rocciatore, 18 metri di corda, una lanterna schermabile con 5 ampolle d'olio, 2 sacchi, 2 torce, razioni da viaggio (per 3 giorni)

\textbf{Dotazione da Intrattenitore (60 mo}). Include uno zaino, un giaciglio, 2 costumi, 5 candele, 5 razioni giornaliere, un otre e trucchi per il camuffamento.

\textbf{Dotazione da Scassinatore (24 mo)}. Include uno zaino, un sacchetto con 1000 sfere metalliche, 3 metri di spago, una campanella, 5 candele, un piede di porco, un martello, 10 chiodi da rocciatore, una lanterna schermabile, 2 ampolle di olio, 5 razioni giornaliere, un acciarino e pietra focaia e un otre. La dotazione include anche 15 metri di corda di canapa legata allo zaino.

\textbf{Dotazione da Studioso (60 mo)}. Include uno zaino, un libro di studio, una boccetta d'inchiostro, un pennino, 10 fogli di pergamena, un sacchetto di sabbia e un coltellino.


\end{multicols}

\subsubsection{Capienza dei Contenitori}

\begin{tabularx}{0.95\textwidth}{lXl|lXl}
\textbf{Oggetto}&\textbf{Capienza}&\textbf{CdC}&\textbf{Oggetto}&\textbf{Capienza}&\textbf{CdC}\\
\toprule
Borsa&1 cubo con spigolo di 30 cm/3 kg di equipaggiamento&1&Barile&160 litri liquidi, 4 cubi con spigolo di 30 cm&35\\
Boccale&0,5 litri&L&Bottiglia&1 litro di liquido&L\\
Secchio&12 litri liquidi, 1 cubo con spigolo di 25 cm&3&Canestro&2 cubi con spigolo di 30 cm/20 kg di equipaggiamento&5\\
Sacco&1 cubo con spigolo di 30 cm/15 kg di equipaggiamento&3&Forziere&12 cubi con spigolo di 30 cm/150 kg di equipaggiamento&35\\
Fiala&120 ml di liquidi&L&Otre&2 litri liquidi&1\\
Caraffa&4 litri liquidi&2&Zaino&2 cubi con spigolo di 30 cm/30 kg di equipaggiamento&6\\
\end{tabularx}


\medskip

Esiste anche la versione dello Zaino Perfetto (+100 mo) che concede un +1 al valore di Ingombro trasportabile.

\begin{multicols}{2}

\subsubsection{Strumenti}

L'elenco degli strumenti presentati aiuta i personaggi ad eseguire le prove legate alle loro professioni.

Le prove legate alle professioni sono solitamente legate alla Saggezza.

Ad esempio una prova di "Calligrafia" si risolve con una prova di Saggezza, se il personaggio ha a disposizione gli strumenti idonei ("\textit{Scorte da Calligrafo}") ottiene un bonus di +2 alla prova.

Se il personaggio deve effettuare una prova su quella che è la sua professione questa sarà fatta con un bonus pari a metà del livello del personaggio, se ha a disposizione anche gli strumenti la prova un ulteriore bonus di +2.

\end{multicols}

\medskip

\begin{tabularx}{0.95\textwidth}{llX|llX}
\textbf{Oggetto}&\textbf{Costo}&\textbf{Ing.}&\textbf{Oggetto}&\textbf{Costo}&\textbf{Ing.}\\
\toprule
Arnesi da Scasso/da Falsario&25 mo&1&Borsa da Erborista&5 mo&1\\
Dadi&1 ma&-&Mazzo di Carte&5 ma&-\\
Scacchi dei Draghi&1 mo&1&Tre Draghi al Buio&1 mo&-\\
Sostanze da Avvelenatore&50 mo&1&Scorte da Alchimista&50 mo&2\\
Scorte da Calligrafo&10 mo&1&Scorte da Mescitore&20 mo&2\\
Strumenti da Calzolaio&5 mo&2&Strumenti da Cartografo&15 mo&2\\
Strumenti da Conciatore&5 mo&2&Strumenti da Costruttore&10 mo&2\\
Strumenti da Fabbro&20 mo&3&Strumenti da Falegname&8 mo&2\\
Strumenti da Gioielliere&25 mo&1&Strumenti da Intagliatore&1 mo&2\\
Strumenti da Inventore&50 mo&2&Strumenti da Pittore&10 mo&1\\
Strumenti da Soffiatore&30 mo&2&Strumenti da Tessitore&1 mo&2\\
Strumenti da Vasaio&10 mo&2&Utensili da Cuoco&1 mo&2\\
Strumenti da Navigatore&25 mo&2&Ciaramella&2 mo&1\\
Cornamusa&30 mo&1&Corno&3 mo&L\\
Dulcimer&25 mo&2&Flauto&2 mo&0L\\
Flauto di Pan&12 mo&L&Lira&30 mo&L\\
Liuto&35 mo&1&Tamburo&6 mo&1\\
Viola&30 mo&1&Trucchi per il Camuffamento&25 mo&1\\
\end{tabularx}

\begin{multicols}{2}


\subsubsection{Cavalcature e Veicoli}

Una buona cavalcatura può consentire a un personaggio di attraversare rapidamente un territorio selvaggio, ma il suo scopo primario è trasportare l'equipaggiamento che altrimenti rallenterebbe il suo padrone.

La tabella "Cavalcature e Altri Animali" indica la velocità e la capacità di trasporto base di ogni animale. Un animale che tira una biga, un carretto, un carro, una carrozza o una slitta può spostare un peso pari a cinque volte la sua capacità di trasporto, incluso il peso del veicolo. Se più animali tirano lo stesso veicolo, possono sommare assieme la loro capacità di trasporto.

Nei mondi di fantasy esistono altre cavalcature oltre a quelle elencate in questa sezione, ma si tratta di cavalcature rare che normalmente non sono disponibili per l'acquisto, come certe cavalcature volanti (pegasi, grifoni, ippogrifi e altri animali simili) o perfino alcune cavalcature acquatiche (come per esempio i cavallucci marini giganti).

Per entrare in possesso di una cavalcatura del genere spesso è necessario rubare un uovo e crescere la creatura di persona, stipulare un patto con una potente entità o negoziare con la cavalcatura stessa.

\textbf{Bardatura}. Una bardatura è un'armatura concepita per proteggere la testa, il collo, il petto e il corpo di un animale. Ogni tipo di armatura elencato nella tabella "Armature" di questo capitolo può essere acquistata come bardatura. Il costo è pari al quadruplo dell'armatura equivalente fabbricata per gli umanoidi, mentre il peso è pari al doppio.



\textbf{Sella}. Un cavalcatore può agganciarsi a una sella militare per rimanere al suo posto su una cavalcatura attiva, nel corso di una battaglia. Una sella militare conferisce vantaggio alle prove che il personaggio effettua per rimanere in sella. E' necessaria una sella esotica per cavalcare una creatura acquatica o volante.


\textbf{Imbarcazioni a Remi}. I barconi e le barche a remi sono solitamente usati sui laghi e sui fiumi. Se un'imbarcazione segue la corrente, si aggiunge la velocità della corrente (solitamente 4,5 km all'ora) alla sua velocità. In genere non è possibile remare controcorrente se la corrente ha un'intensità rilevante, ma è possibile far risalire un corso d'acqua a queste imbarcazioni portandole a riva e facendole trainare da una o più bestie da soma. Una barca a remi pesa 50 kg (Ingombro 10) qualora gli avventurieri debbano trasportarla via terra.

\subsubsection{Cavalcature e Altri Animali}


\begin{tabularx}{0.42\textwidth}{lllll}
\toprule
\textbf{Cavalcatura}&\textbf{Costo}&\textbf{Mov.}&\textbf{Carico}&\textbf{Km/h}\\
&(\textbf{mo})&&kg&\\
Asino o Mulo&8&12 m&210&6km\\
Cammello&50&15 m&240&8km\\
Cavallo da Galoppo&75&18 m&240&12km\\
Cavallo da Guerra&400&18 m&270&9km\\
Cavallo da Tiro&50&12 m&270&6km\\
Elefante&200&12 m&660&6km\\
Mastino&25&12 m&97,5&6km\\
Pony&30&12 m&112,5&6km\\
\end{tabularx}

\bigskip

\textbf{Finimenti e Veicoli da Tiro}\\
\begin{tabularx}{0.45\textwidth}{llX}
\toprule
\textbf{Oggetto}&\textbf{Costo}&\textbf{Peso}\\
Bardatura&x4&x2\\
Biga&250 mo&50 kg\\
Bisacce&4 mo&4 kg\\
Carretto&15 mo&100 kg\\
Carro&35 mo&200 kg\\
Carrozza&100 mo&300 kg\\
Morso e Briglie&2 mo&0,5 kg\\
Nutrimento (al giorno)&5 mr&15 kg\\
\end{tabularx}

\bigskip

\textbf{Sella}\\
\begin{tabularx}{0.45\textwidth}{llX}
\toprule
\textbf{Oggetto}&\textbf{Costo}&\textbf{Peso}\\
Da Carico&5 mo&7,5 kg\\
Da Galoppo&10 mo&12,5 kg\\
Esotica&60 mo&20 kg\\
Militare&20 mo&15 kg\\
Slitta&20 mo&150 kg\\
Stallaggio (al giorno)&5 ma&\\
\end{tabularx}

\bigskip

\textbf{Imbarcazioni}\\
\begin{tabularx}{0.45\textwidth}{llX}
\toprule
\textbf{Oggetto}&\textbf{Costo}&\textbf{Velocità}\\
Barca a Remi&50 mo&2,25 km orari\\
Barcone&3000 mo&1,5 km orari\\
Galea&30000 mo&6 km orari\\
Nave a Vela&10000 mo&3 km orari\\
Nave da Guerra&25000 mo&3,75 km orari\\
Nave Lunga&10000 mo&4,5 km orari\\
\end{tabularx}

\subsubsection{Servizi}


Gli avventurieri possono pagare i personaggi non giocanti affinché li aiutino o agiscano in loro vece nelle circostanze più disparate. La maggior parte di questi gregari è dotata di abilità pressoché ordinarie, mentre altri hanno padroneggiato un'arte o un mestiere e alcuni si sono specializzati in qualche abilità da avventuriero.

Altri gregari comuni includono i numerosi abitanti di un tipico paese o città che gli avventurieri possono ingaggiare per svolgere un compito specifico. Per esempio, un incantatore potrebbe pagare un falegname per farsi costruire un pregiato scrigno (e la sua replica in miniatura) da usare per un incantesimo.
Un guerriero potrebbe commissionare a un fabbro la forgiatura di una spada speciale.

\medskip

\textbf{Servizi}

\bigskip

\begin{tabularx}{0.45\textwidth}{Xl}
\textbf{Servizio}&\textbf{Costo}\\
\toprule
Carrozza all'interno di una città&5 mr/1 km\\
Carrozza tra due paesi&1 ma/1 km\\
Gregario Abile&2 mo al giorno\\
Gregario Inesperto&5 ma al giorno\\
Messaggero&5 mr/1,5 km\\
Passaggio in nave&1 ma/1,5 km\\
Pedaggio stradale o di ingresso&5 mr/5 ma\\
\end{tabularx}


\subsubsection{Servizi Magici}

\textbf{Livello Incantesimo x Livello Incantesimo ×100 mo}

Questo è il costo per avere un incantatore che manipola la magia. Questo costo presuppone che si possa andare dall’incantatore e chiedergli di manipolare una certa magia a proprio piacimento (solitamente gli servono almeno 8 ore per prepararsi). Se si vuole portare da qualche parte l’incantatore per fargli usare la magia è necessario negoziare con lui, e la risposta di base è "no".

Se l’incantesimo a ha conseguenze pericolose, l’incantatore deve ricevere delle prove certe che il personaggio ha la possibilità di pagare e che non mancherà di farlo nel caso queste conseguenze si verifichino (sempre che accetti di lanciare l’incantesimo richiesto, cosa nient’affatto sicura). Quando si tratta di incantesimi che trasportano il personaggio e l’incantatore lungo una distanza, è necessario pagare l’incantesimo due volte anche se il personaggio non desidera tornare indietro con l’incantatore.

Non tutti i villaggi e i paesi hanno un incantatore abbastanza capace a manipolare la magia. Come regola generale, è necessario spostarsi almeno in un piccolo paese per essere abbastanza sicuri di trovare un incantatore. In un piccolo paese si potrebbe trovare un incantatore in grado di lanciare incantesimi a livello 2, in un grande paese quelli a livello 3, una piccola città per quelli a livello 5, in una grande città per quelli di livello 6, in una metropoli per quelli di livello 8. Nemmeno in una metropoli si è certi di trovare un incantatore capaci di lanciare magie con livello 9 o piu’.


\subsubsection{Oggetti e Sostanze Speciali}\index{Sostanze Speciali}

\textbf{Antiemetico} 25 mo,  questo liquido verde dolce e saporito crea un senso di calore e conforto. Lo sciroppo protegge lo stomaco e lo rende più resistente. Per 1 ora dopo averlo bevuto si ottiene Bonus +4 ai Tiri Salvezza per resistere agli effetti che rendono Nauseati o contro i veleni da Ingestione.

\textbf{Antibiotico} (fiala) 50 mo, bevendo una fiala di questo liquido bianco latte dal pessimo sapore si ottiene Bonus +4 ai Tiri Salvezza contro le Malattie, effettuati nell'ora successiva. Se già infetti, si possono effettuare due Tiro Salvezza per resistere alla Malattia in quella determinata giornata (senza il bonus +4) e tenere il risultato migliore. Monodose. 

\textbf{Antitossina} (boccetta) 50 mo, se si beve l'antitossina, si ottiene Bonus +4 a tutti i Tiri Salvezza su Tempra contro Veleni per 1 ora. Monodose. 

\textbf{Bastone del Fumo} 20, questo bastone di legno trattato con procedimento alchemico crea istantaneamente un denso fumo opaco quando viene infiammato. Il fumo riempie un cubo con spigolo di 3 metri (distanza di mischia), tranne che il fumo viene dissipato in 1 round da un vento moderato o più intenso. Il bastone si consuma in 1 round e il fumo si dissolve poi naturalmente. Tutte le creature nell'area influenzata hanno copertura totale.

\textbf{Caffettone dell'Alchimista} 1 mo, molto amata dai giovani si tratta di una polvere cristallina bruna. Mischiata con l'acqua crea una bevanda amara che cura gli effetti della sbornia. Monodose. Lavoro DC 15

\textbf{Borsa dell'Impedimento} 50 mo, questa borsa di cuoio rotonda è piena di melassa, resina o altra sostanza appiccicosa. Quando si scaglia la borsa contro una creatura (come attacco di contatto a distanza con gittata 3 metri), la borsa si apre e la sostanza contenuta invischia ed Intralcia la vittima, diventando resistente ed elastica con l'esposizione all'aria.

La sostanza non agisce su creature di taglia Enorme o superiore. Una creatura volante non viene appiccicata al suolo, ma deve effettuare un Tiro Salvezza su Riflessi con DC 15 o perde la capacità di Volare (sempre che usi le ali per farlo), cadendo a terra. La borsa dell'impedimento non funziona sott'acqua.

\textbf{Fermasangue} 25 mo, questa sostanza rosa e appiccicosa aiuta a curare le ferite. Utilizzarne una dose concede Bonus +4 alle prove di Pronto Soccorso. 6 Usi.

\textbf{Fiasco Alcalino} 15 mo, questo fiasco di liquidi caustici reagisce con gli acidi naturali delle melme. E' possibile lanciare un fiasco alcalino come arma a spargimento con gittata 3 metri. Contro le creature non melme un fiasco alcalino funziona come un'Ampolla d'acido. Contro le melme e altre creature acide il fiasco alcalino infligge i danni raddoppiati indicati da Ampolla d'Acido. 

\textbf{Fumogeno} 25 mo, questa piccola sfera di argilla contiene due sostanze alchemiche separate da una sottile barriera. Quando si rompe la sfera, le sostanze si uniscono e riempiono un area di mischia con una nuvola di fumo nerastro e innocuo. Il fumogeno funziona come un bastone del fumo, ma il fumo rimane per 1 round prima di disperdersi. E' possibile lanciare un fumogeno come attacco di contatto con gittata 3 metri.

\textbf{Fuoco dell'Alchimista} 20 mo, si può lanciare un'ampolla di fuoco dell'alchimista come arma a spargimento. Si consideri l'attacco come un attacco di contatto a distanza, con gittata 3 metri.

Il colpo diretto provoca 1d6 danni da fuoco. Tutte le creature entro raggio di mischia dal punto in cui è caduta l'ampolla subiscono 1 danno da fuoco come effetto dello spargimento. Nel round successivo al colpo diretto la vittima subisce 1d6 danni da fuoco aggiuntivi. La vittima può sfruttare 1 Azione per tentare di spegnere le fiamme prima di subire questi danni aggiuntivi. Occorre superare un Tiro Salvezza su Riflessi con DC 15 per spegnere le fiamme. Usare 2 Azioni dà al personaggio bonus +2 al Tiro Salvezza. Tuffarsi in acqua o smorzare le fiamme con mezzi magici spegne automaticamente le fiamme.

\textbf{Gesso per Calchi:} 5 ma, questa polvere bianca e secca, mischiata con l'acqua, si addensa nel giro di un'ora per creare un materiale solido. Può essere utilizzato per creare un calco di un'orma o di un bassorilievo, riempire buchi o crepe nei muri o (se applicato ad una copertura di stoffa) per fermare un osso rotto. Il gesso indurito ha Durezza 1 e 5 Punti Ferita ogni 2.5 centimetri di spessore. Un vaso di 2 kg di gesso può coprire un raggio di mischia per la profondità di 2.5 centimetri, creare cinque ingessature per l'avambraccio o il polpaccio di una creatura di taglia Media o due ingessature complete per braccio o gamba. Monodose.

\textbf{Ghiaccio Liquido} (fiala) 40 mo, detto anche "ghiaccio dell'alchimista", questo fluido blu cristallino inizia ad evaporare appena tolto dal contenitore. Nei successivi 1d6 round è possibile utilizzarlo per congelare un liquido o coprire un oggetto con un sottile strato di ghiaccio. E' possibile anche lanciare il ghiaccio liquido come arma a spargimento. Un colpo diretto infligge 1d6 danni da freddo, mentre le creature entro raggio di mischia subiscono 1 danno da freddo per lo spargimento. La confezione contiene 3 dosi.

\textbf{Grasso Alchemico} 5 mo, ogni vaso di questa sostanza nerastra può coprire una creatura Media o due Piccole. Coprendosi di grasso alchemico si ottiene Bonus +4 alle prove di lotta e per sfuggire alle prese. L'effetto dura 4 ore o finché si lava via il grasso.

\textbf{Individua Luce} 1 mo, questa piastra di metallo grande quanto una mano è coperta da una crema trasparente sensibile alla luce. Se esposta alla luce, la crema si scurisce e diviene opaca a seconda di quanta luce sia presente. La luce intensa la fa scurire in 1 round, quella normale in 3 round, quella fioca in 10 round.
La piastra viene venduta avvolta in un panno pesante per evitare esposizioni accidentali. 

\textbf{Individua Luce avanzata} 50 mo, questa piastra di metallo simile alla piastra Individua luce è grande circa 50cm*50 cm. Se esposta alla luce imprime su di essa l'immagine dell'ambiente circostante entro 3 metri.

\textbf{Pietra del Tuono} 30 mo, si può scagliare questa pietra con un attacco a distanza con gittata 6 metri. Quando colpisce una superficie dura (o è colpita con forza), crea un rumore assordante che equivale a un attacco sonoro. Le creature presenti entro una distanza di 3 metri devono effettuare un Tiro Salvezza su Tempra con DC 15 o restano Assordate per 1 ora. Monouso.

\textbf{Polvere Lampo} 50 mo, questa polvere argentea brucia ed esplode quasi istantaneamente se esposta al fuoco, frizionandola o lanciandola con forza contro una superficie (1 Azione). Le creature entro raggio 3 metri sono Accecate per 1 round (Tempra DC 13 nega). La confezione contiene 3 dosi. 

\textbf{Proteggilama} 40 mo, questa resina trasparente protegge un'arma dagli attacchi di Melme, Rugginofagi ed effetti che corrodono o sciolgono le armi, rendendola immune a tali attacchi per 24 ore. Un vasetto può coprire un'arma a due mani, due armi ad una mano o leggere o 50 munizioni. Applicarla richiede 2 Azioni. La confezione contiene 3 dosi.

\textbf{Solvente Universale} (fiala) 20 mo, questa gelatina viola ribollente divora gli adesivi. Ogni fiala può coprire un raggio di mischia. Distrugge i normali adesivi (come la pece, la resina o la colla) in 1 round, ma richiede 1d4+1 round per dissolvere adesivi più potenti (borse dell'impedimento, ragnatele, ecc.). Non ha effetti sugli adesivi magici.

\textbf{Tizzone Ardente} 1 mo, la sostanza alchemica sulla punta di questo piccolo bastone di legno si infiamma quando viene sfregata contro una superficie ruvida. Creare una fiamma con un tizzone ardente è molto più rapido che crearla con acciarino, pietra focaia (o lente d'ingrandimento) e esca. Accendere una torcia con un tizzone ardente costa 1 Azione (invece che 2 Azioni) e per accendere qualsiasi altro fuoco occorre almeno 3 Azioni.

\subsubsection{Attrezzature Alchemiche}

\textbf{Carta Reagente} 1 mo, questo pezzo di carta può aiutare a identificare i liquidi. Il suo colore cambia a seconda di tratti come acidità, salinità e magia. Consumare un foglio conferisce Bonus +2 alle prove di Lavoro (alchimia) o Arcano per identificare Pozioni o altri liquidi.

\textbf{Inchiostro Esplosivo} (fiala) 40 mo, questo inchiostro infuso alchemicamente aiuta ad assicurarsi che un messaggio segreto venga distrutto dopo essere stato letto. Se la luce colpisce l'inchiostro dopo che quest'ultimo si è asciugato, le sostanze chimiche lo fanno bruciare spontaneamente nel giro di 1 minuto
Questa combustione è di piccole dimensioni: non è abbastanza significativa da dar fuoco ad altro che alla carta. L'inchiostro usato su altri materiali come pietra o legno semplicemente svanisce, non lasciando alcuna traccia della scrittura
Una fiala di questo inchiostro ne contiene abbastanza da scrivere 10 brevi messaggi di non più di 50 parole ciascuno.

\textbf{Olio dei Liutai} 50 mo, quest'olio dorato profuma di legno antico. Quando lo si applica sulla cassa di uno strumento musicale di legno ne migliora la qualità del suono. Per 1 ora, chiunque suoni lo strumento ottiene Bonus +2 alla prova di Intrattenere appropriata.

\textbf{Pastiglia dell'Usignolo} 50 mo, questa caramella ricoperta di miele è fatta di reagenti calmanti. Se mangiata, ha bisogno di 1 round per iniziare ad avere effetto, dopodiché conferisce Bonus +2 alle prove di Intrattenere (canto) per 1 ora.

\textbf{Pietre di Via} 50 mo, questi piccoli sassolini bianchi sono trattati alchemicamente in modo che emanino una luce soffusa quando attivati sfregandoli gli uni contro gli altri. La luminescenza è fioca, appena sufficiente a illuminare la pietra. La durata è di 8 ore.

\textbf{Polvere Tracciante} 30 mo, quando sparsa per terra, questa sottilissima polvere blu chiaro rivela le tracce di qualsiasi creatura o individuo che sia passato nell'area nelle ultime 48 ore.
La polvere fornisce anche Bonus +8 alle prove di Sopravvivenza per individuare le tracce. Una singola applicazione può coprire un'area di 3 metri. La polvere tracciante viene venduta in piccole borse di cuoio che contengono 10 applicazioni ciascuna.

\subsubsection{Rimedi Alchemici}\index{Rimedi Alchemici}

\label{rimedi-alchemici}

\textbf{Aiuto Gassato} 25 mo, questo pacchetto è pieno di foglie dai bordi spinosi e ha un odore pungente quasi abbastanza forte da far lacrimare gli occhi. Mentre si masticano le foglie, si ignorano gli effetti dell'essere affaticati o esausti. Le foglie durano per 10 round, dopodiché ne rimane solo un mucchietto di poltiglia.
Quando l'effetto dell'aiuto gassato si esaurisce, si aumenta di 1 grado il livello di affaticamento. Un pacchetto basta per 1 sola volta.

\textbf{Balsamo Anti-veleno} 15 mo, questo balsamo alle erbe può essere applicato direttamente sulla pelle per prevenire gli effetti dei Veleni a contatto. Se una creatura tocca un veleno a contatto, ma applica su di sé il balsamo entro 1 round dal contatto, effettua il Tiro Salvezza due volte e tiene il risultato migliore. Monouso.

\textbf{Balsamo Coagulante} 5 ma, applicare questo balsamo alle erbe su una ferita cura 1 danno, non è possibile suare più di due dosi al giorno sullo stesso paziente. La confezione è per 3 usi.

\textbf{Amaro Fortificante} 20 mo, questo liquido alcolico genera una piacevole sensazione di calore quando ingerito. Per l'ora successiva, si ottiene Bonus +2 ai Tiri Salvezza contro Paura. Usare più dosi nell'arco delle stesse 24 ore rende Nauseati per 1 ora. La confezione è per 3 usi.


\subsubsection{Lo Zaino Standard\texorpdfstring{\huge{\textregistered}}{\textregistered}} \index{Zaino Standard}

Lo Zaino Standard\textregistered \space è una lista di oggetti che ho segnato nel tempo andando ad aggiungere ogni cosa che nel corso delle avventure mi era servito.
Prendetela come spunto per capire che oggetti avere dietro, non segnateveli tutti altrimenti il Arbitro incomincerà seriamente a guardare le regole dell'Ingombro!

Questo il contenuto dello zaino dell'avventuriero: cintura, 3 candele, 6 torce, esca e acciarino, 7 razioni secche, fiasca d'acqua, materasso arrotolato, cerata, tenda, 18 metri corda, rete, specchio di metallo, piede di porco, bussola, 3 olio da lanterna, inchiostro, gesso, carboncino, uncino, vanga, amo da pesca, stracci, cavo di metallo 2m, fischietto, 6 fiale da pozione vuota, biglie di marmo, campanella in ottone, 1kg di farina in sacchetto, 3 zeppe, catena di metallo 12 metri, 2 manette, 8 chiodi da rocciatore, martello, carrucola, rampino.



\end{multicols}

\pagebreak

Settings v1:  in un futuro non troppo distante inquinamento, guerre, cambiamento climatico con carestie ed inondazioni hanno reso la terra oramai inabitabile. poi improvvisamente vengono scoperte leghe metalliche e cristalline che permettono di imbrigliare e conservare l'energia a livelli neanche immaginabili.
Questo progresso enfatizzo ancora di piu' la divisione sociale. Le poche corporazioni che avevano i nuovi materiali non fecero altro che arricchirsi e depauperare chiunque avesse risorse da vendere.
Piccole enclavi vivevano nel lusso ed in un ambiente idilliaco mentre il 98\% della popolazione si arrabattava come poteva e cercava di resistere ad un pianeta che piu' non lo voleva.
Ormai sull'orlo dell'estinzione la OXF Corp dichiaro' di aver migliorato il proprio materiale in grado di purificare l'aria, rendendo capace un solo cristallo di poter generare l'aria per una intera casa, e non una sola persona.
Durante lo show in mondo visione dove veniva mostrato come veniva aggiornato il "cristallo d'aria" avvenne quello che tutti poi chiamarono la Frattura. 
Il cristallo d'aria incomincio' a risonare, ad emettere un sordo suono e generare onde armoniche sempre con maggiore intensita'. Mentra la terra intorno incominciava a tremare la sede della OXF Corp si spezzo in due e da qualche laboratorio segreto in profondita' una luce intensa e pura sali' alta nel cielo. Per diversi minuti fu solo il panico a dominare gli animi finche' in mezzo a quella luce che si andava dissipando apparve una figura a mezzaria. I lineamenti erano umani, ma non era umana. La carnagione era dorata, i capelli come argento, le mani affusolate ed i lineamenti delicati, le orecchie stranamente lunghe ed a punta.
Il volto basso, gli occhi chiusi quasi fosse in profonda meditazione od addirittura morta.
Un attimo prima che la luce scomparisse del tutto quella creatura emisi una intensissima luce dorata per poi scomparire.
Quello che avvenne dopo fu la vera e propria "Frattura".

Il cielo sembro' aprirsi lasciando intravedere le profonde e lontane stelle, la spaccatura sotto la OXF Corp divenne un crepaccio senza fine.
Entrame le oscurita', diverse eppure simili, una difronte all'altra incominciarono a pulsare e migliaia di esseri dall'una e dall'altra parte incominciarono a camminare sulla terra.

Questo fenomeno, la Fratuttura, non fu un caso isolato, ovunque nel mondo dove ci fosse stato un cristallo purificatore avvenne una Frattura, magari di potenza minore, ma appastanza per distruggere diversi quartieri e richiamare altre creature dal cielo e dalle viscere della terra.

Cio' che usciva dal cielo non erano angeli, come cio' che usciva dalla terra non erano demoni...

passano secoli..civilta' decade tranne per piccole citta' mantenute da corporazioni


Setting v2: dark e dangerous.. ma come?
la superfice e'  tossica e perrennemente in penombra, abitata predatori di ogni specie, il grande obelisco sparge.., le poche ricchezze e tesori rimasti, se non una fugace salvezza risiede nelle profondita', nelle oscure catacombe di antiche civilta' oramai scomparse.


Razze: umani, nani, elfi, mezz'orchi, mezz'elfi, ... trovare qualche razza interessante e particolare. no scurovisione  

La scurovisione ha una portata espressa in metri. 

Entro quei metri: 
- le aree di luce fioca diventano di luce intensa
- le aree di oscurità diventano di luce fioca

Nelle aree modifice da scurovisione, NON si leggono i colori.  è tutto in scale di grigio. questo è un dettaglio importante perché è impossibile distinguere una chiazza di sangue da una di olio, et simili. è per questro motivo che le razze con scurovisione non amano comunque vivere al buio

equipaggiamento: creare una lista essenziale di oggetti e relativo ingombro

ingombro: a slot. non vado a stabilire quanti slot tiene lo zaino, il sacco, la giacca, la cintura.. ma quanti slot puo' portare la creatura. il valore di slot e' uguale a corpo. si conta cio' che si porta, non cio' che si indossa (armatura)
armatura pesante 8, media 6, leggera 4
scudo 4
arma a due mani 6, arma media 4, arma da tiro 4, arma leggera 1
ogni 1000 monete = 1 ingrombro

armi magiche: possono fare piu' male e dare malus alla prova di comp combattimento per evitare. +1 al danno, +1 danno e 1 penalita' prova di combattimento (per difesa), +2 danni e 2 penalita'(questa e' veramente forte)
armature magica: assorbono piu' danno e danno bonus alla prova di combattimento. +1 assorbito, +1 bonus a comp. combattimento, +1 ass e +1 comp, 
scudo: +1 comp combattimento, +1 bonus comp combattimento
altri oggetti magici: fare una lista e ridurre e di tanto


ispirarsi a nome verbo, ma fai te gli accoppiamenti e stabilisci te cosa succede per ogni critico 
i livelli (II, III..) aggiuntivi hanno prerequisiti di consocenza magica maggiore e statistica maggiore, causano una maggiore perdita di resistenza ma garantiscono maggiore risultato, il tempo di lancio (iniziativa) e' piu' lenta piu' e' potente l'incantesimo.

Attacco Elementale I, II, III, IV, V
Creare Elemento I, II, III
Creare Corpo  ???
Creare Mente  ???
Muovere Elemento I, II
Muovere Corpo (feather fall, levitate, volare..)
Muovere Elemento
Proteggere Corpo
Proteggere Elemento 
Proteggere Mente
Distruggo Elemento (piccole cose.. poi piu' grandi)
Distruggo Corpo (maledizione, penalita'..)
Ripara Corpo
Ripara Elemento
Conoscere Corpo
Conoscere elemento
Conoscere Mente
Ripara Mente
Alterare Mente (charmi, compulsion..)
Distruggo Mente (come attacco)
Riparare Spirito, da Mercanteggiare se serve, se ci sono mostri effetti contro lo spirito...


Creomente (bonus a prove mentali), elemento
Muovocorpo, elemento, mente (possessione ?)
Trasformo
Altero
Proteggocorpo, mente, elemento, spirito
Distruggocorpo, mente, elemento, spirito
Attaccoelemento, spirito
Ripararecorpo, mente, elemento, spirito
Conoscerecorpo, mente, elemento, spirito

Corpo
Mente
Elemento
Spirito, cio' che riguarda la non vita e l'anima

Alterare
”Rotto il grimaldello? Lucchetto difficile ed antipatico ? Osserva come si apre al mio umile tocco”: Alte-
rare - Materia
”Potenti spiriti guerrieri infondete coraggio ai compagni”: Alterare - Spirito
”Dalle somme biblioteche io chiamo il Silenzio!”: Alterare - Mente / Alterare - Energia. O non fai sentire il
suono, o lo fai diminuire.
”Per i grandi mammuth lanosi, il freddo non mi fa nulla”: Alterare - Corpo. Per avere resistenza al freddo.
”Che Re Gorilla ti dia la forza di un esercito”: Alterare - Corpo
”Dal deserto delle 10 ombre chiamo il miraggio del muro”: Alterare - Mente.
Attaccare
”Possa tu bruciare delle fiamme dell’inferno”: Attaccare - Fuoco
”Fragili sottili deboli, una ad una rompo le tue ossa”: Attaccare - Corpo / Distruzione - Corpo
”Emicrania ? e’ solo l’inizio”: Attaccare - Corpo
”Che il fuoco della fucina scaldi la tua arma”: Attaccare - Materia
Creare
”Rotto il grimaldello? Ho il migliore dei setti regni”: Creare - Materia
”Nessuna mela è più buona di quella che puoi creare tu”: Creare - Materia
”Oh piccola torcia esplodi di luce in questa tetra caverna”: Creare - Energia. In questo caso non può esserci
danno non essendo il Verbo Attaccare.
”A me tomo delle idee! Illuminami il pensiero”: Creare - Mente. In questo caso può essere usato per ritirare
una prova con un bonus. Solo Conoscenza - Mente può dare la soluzione.
”Al più pavido degli eroi concedo il coraggio del leone”: Creare - Spirito. Può essere dato solo a chi il coraggio
non lo ha, altrimenti e’ Alterare.
”Sommi sapienti aiutatemi a chiamare Colui che Striscia nell’Oscurità”: Creare - Spirito. Non si evocano crea-
ture reale, ma si evocano simulacri, per questo si usa Spirito.
Riparare
”Rotto il grimaldello? Una mia carezza è più efficace di un fabbro” : Riparare - Materia
”Nessuna mela è troppo marcia, riempila del tuo amore e mangiala”: Riparare - Materia
”Possano le tue ferite rinsaldarsi, possa il tuo cuore riposare. Possano le mani della somma guaritrice placare
le tue sofferenze”: Riparare - Corpo
”Libera il cuore dalla paura! Che il maleficio delle immonde creature scompaia”: Riparare - Spirito. In questo
caso si porta alla condizione originaria, togliendo l’effetto di paura
Distruggere
”Rotto il grimaldello? Peggio per il lucchetto. Il mio tocco è quello dei millenni”: Distruggere - Materia
”Osserva il vuoto, perditi dentro, scompari nel nulla. Cosa hai fatto?”: Distruggere - Mente. Può fare perdere
Azioni
”Trema, annaspa, striscia, muori. Dal tuo posto non ti sposti”: Distruggere - Corpo. Per fermare il movimento
”Il coraggio non si da a chi non lo ha. Piccolo codardo, scappa dalla mamma”: Distruggere - Spirito. Può
essere usato per diminuire o annullare un Essenza che conferisce coraggio. ”Fiat Tenebris! Ogni luce muoia”:
Distruggere - Energia
Conoscere
”Rotto il Grimaldello? Ogni lucchetto ha un difetto ed un punto debole. Dimmi quale è il tuo e ti libererò
dal giogo della chiusura”: Conoscere - Materia
”Grande Rutte, concedimi il tuo sguardo di mille avventure. Come affrontare il mio avversario”: Conoscere -
Corpo.
”Dal labirinto chiamo il Minotauro. Ti ordino di dirmi la strada più breve per le Sale di Mazurdas”: Conoscere
- Materia
”I tuoi pensieri nei miei pensieri. I tuoi pensieri sono i mei pensieri”: Conoscere - Mente
”Possa il sommo curatore aiutarmi a capire che veleno di affligge”: Conoscere - Corpo
”Se non fuoco, se non fulmine, immonda creatura quale è il tuo punto debole”: Conoscere - Spirito



allineamento:  Legge, Chaos, neutrale


per fare una prova tiri il 2d10 e devi fare meno della statistica o competenza. 


punti ferita/Vitalita': a seconda della razza ed ogni ramo che prendi lo aumenta di un valore dipendente dal ramo scelto, ma l'aumento e' sempre poco (1-4)
resistenza base: corpo + ramo/rami. quindi da 4 a 12/13 a salire. 
Vitalita':  reggi un certo numero di colpi. un colpo critico (ovvero difesa fallita di almeno 6) causa +2 ferite
parti con un numero di Ferite pari a Corpo. i rami possono aumentare le ferite sostenibili. 
recupero resistenza: ogni notte recuperi 2 (o corpo/2 ???)

morte: arrivi a 0, svenuto. se sei a -1, metti un check su ferito, ogni round fai una prova di corpo se fallisci perdi -1, se riesci tre check di fila ti stabilizzi e torni a 0. arrivato a -10 sei morto.

condizioni: prendere da obss e ridurre sensibilmente  e semplificare bonus e malus, aggiungere condizioni solo se necessario

azioni in round: in un round si possono usare fino a 10 punti azione. attaccare con arma a 2 mani costa 8 punti azione, arma a 2 mani costa 6 punti azione, attaccare con un arma leggera costa 4 punti azione. gli incantesimi solitamente costano da 4 punti azione in su in base  al livello (II, III; IV..) ogni livello in piu' aumenta di 1 l'iniziativa, spostarsi costala meta' della in base alla distanza coperta
i mostri in base alla loro taglia e cosa fanno
per altre azioni vedi obss e traduci in p.a.

distanze: a metri

iniziativa: e' pari ai punti azione usati. chi ne usa di meno incomincia per primo. in caso di parita' si controlla mente, volonta', corpo in quest'ordine. se ritarti "consumi" punti azione ad aspettare

competenze di base: ogni ramo modifica/o meno il punteggio di Combattimento o Magia.

combattimento: chi attacca non tira nulla, dichiara solo che arma usa e se usa manovre. chi difende tira sotto il suo valore di combattimento, questo valore di combattimento puo' avere svantaggi dati da abilita' dell'attaccante (rami combattenti avanzati). se tira sotto allora para/evita il colpo, se tira sopra viene preso.
Si presume che l'attacco colpisce sempre se la difesa non funziona bene. Questo significa che le classi combattenti non aumentano l'attacco ma solo la difesa. alcuni rami avanzati di combattimento danno delle penalita' alla difesa avversaria.
il danno e' 1d6 per armi piccole, 1d8 armi medie, 1d10 armi a 2 mani, rissa 2 danni.
combattere con un arma che non conosci da svantaggio


armi: armi piccole 1d6, armi medie 1d8, armi grandi 2 mani 1d10 +bonus corpo. arma piccola ha requisito forza 6, media 9, grande 12, altrimenti svantaggio nell'uso

una difesa particolarmente riuscita (almeno -6) puo' dare svantaggio al tiro successivo, un -9 potrebbe fare tirare con un solo dado
quando difendi  a seconda di quanto bene difendi, ovvero se hai un margine di -3,-6,-9.. rispetto alla tua prova ottieni dei bonus, azioni movimento, penalita' all'attacco successivo

le manovre funzionano con lo stesso principio. quando effettuo una manovra e l'altro riesce nel tiro di combattimento allora la manovra non riesce e io devo fare un tiro di combattimento, se fallisco prendo gli effetti della manovra

armature: riducono il danno. leggere riducono di 2, medie riducono di 3, pesanti riducono di 4. danno una penalita' alle prove di competenza. Non si fa magia con armatura se non solo a contatto. requisito di corpo: 8, 12, 
scudo: fai la prova di difesa con +1 (da 1 bonus). non segno che da penalita' ma premo sull'ingombro 4


magia: deve essere lanciare piu' difficile lanciare incantesimi e consumare "risorse" (resistenza/energia/stamina..)

certi rami magici possono privilegiare certe scuole di magia. lavorare su liste e ridurre e tanto. l'idea di base e' per esempio crea fuoco puo' diventare a seconda del punteggio del tiro altre cose, il valore influenza la distanza, AoE, danno, se e' un raggio o esplosione e che raggio...  Pochi incantesimi ma che si evolvono

lanciare un incantesimo: tirare a secondo dall'incantesimo su capacita' magica e avere un punteggio minimo di  mente, corpo o volonta'.  Ogni punteggio -3 rispetto alla prova, es. capacita' magica 13 e tiro 8, potenzi un fattore dell'incantesimo (distanza, AoE, danno, tipo di effetto..). Gli incantesimi hanno un punteggio minimo di mente/corpo/volonta' per essere tirati ed anche di capacita' magica
Ogni incantesimo ha degli attributi, danno, distanza, aoe, durata ed un punteggio minimo di competenza magica e corpo/mente/volonta'. se il tiro riesce bene puoi potenziare un attributo presente ma non darlo/aggiungerlo se questo e' assente. se lancio l'incantesimo crea fiamma, inc. base difficolta' 1, se faccio un ottimo tiro potro' potenziare la durata e aoe (l'area di luce che fa) ed il danno, ma non posso aggiungere distanza perche' e' un attributo assente.
ci sara' poi l'incantesimo globo di fuoco, la versione base della palla di fuoco, questa ha piu' attributi ma ad esempio non ha durata, o meglio l'istantanea non puo' essere migliorata.
Mercanteggiare di aggiungere attributi assenti quando il tiro e' veramente buono, ovvero tiri veramente basso, direi almeno un -6 per aggiungere un attributo a livello base (3 metri)
si possono spendere risorse aggiuntive per potenziare l'incantesimo, ovvero abbassare il risultato del dado
scuola di magia: raggruppa gli incantesimi per tipo
ci si puo' specializzare in un "verbo" di magia: hai +3 alla prova, ma -3 a tutte le altre scuole

lanciare incantesimi mentre si combatte o si e' stato colpito: si puo' fare ma il primo critico si ignora

se l'incantesimo riesce nel lancio non c'e' TS. alcuni incantesimi possono avere una prova di difesa per essere evitati/dimezzati l'effetto

questo implica che non ci saranno mai incantesimi super potenti in tutto..

quanti incantesimi lanciare:  a piacere, ma ogni volta che lanci lo stesso il costo in vitalita' aumenta. 

quando lanci un incantesimo e fallisci la prova non succede nulla, ma l'incantesimo l'hai lanciato e quindi perdi la vitalita' 
quando lanci un incantesimo e fallisci con 19-20 la prova succedono cose  brutte
quando lanci l'incantesimo e fai veramente basso 2, non perdi vitalita'


mostri:
se il mostro tira la sua difesa c'e' il rischio che riesca sempre ad alti livelli. considerare abilita' che abbassano la difesa, l'idea di base e' che se un mostro ha difesa  o piu' deve essere di un livello tale da dover affrontare pg con rami che danno penalita' alla difesa
fare una lista minima di 20 mostri tipici e classici, verificare bx per compatibilita'


--------------------------------------------------------------------------------



- fare 2 o 20  successo critico o fallimento  critico

- il giocatore dichiara cio' che fa e' solo il master a stabilire se serve una prova. 

- il tempo e' un fattore, tabelle random per incontri basati su tempo trascorso, si computa il tempo reale.

Lower HP all around.
Less access to healing.
More dungeons.
No "at will" spell casting.
Lack of a universal skill system.
More poison, less disease.
player skill
zero to hero player progression





\end{document}

