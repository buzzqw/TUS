\documentclass[12pt,a4paper,twoside,openany,twocolumn]{book}
\usepackage[utf8]{inputenc}
\usepackage[T1]{fontenc}
\usepackage[left=2.00cm, right=2.00cm, top=2.00cm, bottom=2.00cm]{geometry}
\usepackage{graphicx}
\usepackage{hyperref}
\usepackage{xcolor}

\begin{document}

Introduzione\\
Razza\\
Caratteristiche\\
rami base\\
rami avanzati\\
Punti Fortuna\\
Punti Fato\\
Karma\\
Competenze\\
Costruiamo il personaggio\\
Regole per le competenze\\
Combattimento\\
Nascondigli e coperture\\
liste armi ed armature\\
abilita' ???\\
magia\\
incantesimi\\
equipaggiamento\\
veleni e droghe\\
movimento\\
oggetti magici\\
mostruario\\
Condizioni\\
Scheda\\

Settings: ???

Razze: umani, nani, elfi, mezz'orchi, mezz'elfi, ... trovare qualche razza interessante e particolare

Caratteristiche: Mente, Corpo, Volonta'. tiri 2d10 per il valore, qualsiasi valore oltre 12 e' 12, qualsiasi valore sotto 4 e' 4.
Le caratteristiche non danno bonus ai tiri, ma sono prerequisiti per avanzamenti di rami, competenza magica ed armi (non puoi avere comp. armi/magica superiore al x3 della statistica), scelta ed uso incantesimi

Kismet: ogni giocatore indica come il suo personaggio morira'.

punti fortuna: come i chaos in obss

classi/rami: non serve perdersi, bastano 12 rami base, 8 intermedie, 6 avanzate, 4 elite
in una ramo si rimane finche' non si soddisfano i requisiti dei rami superiori
quindi le abilita' aumentano sempre, anche rimanendo nello stesso ramo
i punteggi aumentano con gli errori: se tiri 19/20 (sui 2d10) oltre al fallimento o fallimento critico segni con un pallino la competenza, quando c'e' downtime fai un check per ogni pallino e se fai 2 (ovvero 1/1) aumenti di 1 punto


--------------------------------------------------------------------------------



- una difesa particolarmente riuscita (almeno -6) puo' dare svantaggio al tiro successivo, un -9 potrebbe fare tirare con un solo dado

- magia: deve essere lanciare piu' difficile lanciare incantesimi e consumare "risorse" (pf/energia/stamina..)

- quando qualcuno attacca si e' presi se la prova di difesa e' oltre il valore di difesa. si presume che l'attacco colpisce sempre se la difesa non funziona bene. Questo significa che le classi combattenti non aumentano l'attacco ma solo la difesa. alcuni rami avanzati di combattimento danno delle penalita' alla difesa avversaria.

- se il mostro tira la sua difesa c'e' il rischio che riesca sempre ad alti livelli. considerare abilita' che abbassano la difesa, l'idea di base e' che se un mostro ha difesa 16 o piu' deve essere di un livello tale da dover affrontare pg con rami che danno penalita' alla difesa

- statistica fortuna si usa per abbassare il tiro fatto. recuperi 1 quando fai un fallimento critico. dichiari prima e spendi prima.

- il giocatore tira un dado per lanciare incantesimi, piu' riesce (tiro basso) piu' l'incantesimo riesce meglio. puo' non tirare dato per avere un risultato fisso di base poco potente. chiamato effetto minimo nella descrizione dell'incantesimo

- se l'incantesimo riesce nel lancio non c'e' TS. detta diversamente non c'e' mai TS

- le statistiche sono corpo, mente, volonta'

- le statistiche vanno da 4 a 12. 2d6 + razziale. max 13. Nessuna statistica sotto il 3. Seconda del ramo prendi dei bonus o malus (solo rami molto avanzati, sempre a fare il mago aumenti mente)

- le prove: si tira 2d10 > valore statistica o competenza,  bisogna tirare piu' basso del valore della statistica

- vantaggio: tiri tre dadi e prendi i due peggiori  |  svantaggio: tiri due dadi e prendi i due migliori | attenzione al valore 0 zero

- bonus/malus 1d4, 1d6, 1d8 (normale, buono, alto). Puoi avere un solo bonus o malus, se ne hai piu' di uno tiri piu' dadi ma prendi il valore piu' alto/basso, come avere vantaggio o svantaggio

- no classi, albero con dipendenze. ogni ramo ti da delle abilita' , come i feat

- la professione iniziale ti da dei punteggi di competenza base, solitamente vanno da 8 a 10.

- check/prove: per fare una prova tiri il 2d10 e devi fare meno della statistica o competenza. se e' collegata alla professione tiri con vantaggio

- fare 2 o 20  successo critico o fallimento  critico

- il giocatore dichiara cio' che fa e' solo il master a stabilire se serve una prova. 

- i rami presi possono aggiungere "competenze professionali". aumentano pf, competenze armi, competenze magica, competenze professionali. Per passare da un ramo ad un altro devi avere un valore minimo in certi punteggi

- razze no scurovisione

- il tempo e' un fattore, tabelle random per incontri basati su tempo trascorso, si computa il tempo reale.

- no livelli, quando una prova fallisce tirando 19 segni un pallino, se fai 00 aumenti direttamente di 1. quando c'e' riposo e tempo fai una prova per ogni pallino, se tiri 2 allora alzi di 1 il punteggio

- armi: armi piccole 1d6, armi medie 1d8, armi grandi 2 mani 1d10. non sommi valore forza. arma leggera ha requisito forza 6, media 9, grande 11. ogni arma ha un requisito per l'uso altrimenti svantaggio nell'uso

- sistema 3 azioni.

- iniziativa: 2d10 - 2 arma leggera, +2 arma media, +6 arma 2 mani

- azioni personalizzate e prove particolari: solo se possono fallire sono prove. fare un elenco con manovre e relative prove e cosa significa fallire. Se sono manovre di attacco questa si considera di successo se non si passa una prova di difesa

- armatura: assorbe il danno e penalizza certe prove, ovvero tiro sotto alla prova richiesta. attenzione che in questa maniera tutti disarmano, fanno cadere.... Impostare 

- scudo assorbe il danno e penalizza certe prove. valutare scudo che effetti alternativi puo' avere.

- punti ferita: a seconda della razza ed ogni ramo che prendi lo aumenta di un valore dipendente dal ramo scelto, ma l'aumento e' sempre poco (1-4)

- certi rami magici possono privilegiare certe scuole di magia. lavorare su liste e ridurre e tanto. l'idea di base e' per esempio crea fuoco puo' diventare a seconda del punteggio del tiro altre cose, il valore influenza la distanza, AoE, danno, se e' un raggio o esplosione e che raggio...  Pochi incantesimi ma che si evolvono

- lanciare un incantesimo: tirare a secondo dall'incantesimo su capacita' magica e avere un punteggio minimo di  mente, corpo o volonta'.  Ogni punteggio -3 rispetto alla prova, es. capacita' magica 13 e tiro 8, potenzi un fattore dell'incantesimo (distanza, AoE, danno, tipo di effetto..). Gli incantesimi hanno un punteggio minimo di mente/corpo/volonta' per essere tirati ed anche di capacita' magica
Ogni incantesimo ha degli attributi, danno, distanza, aoe, durata ed un punteggio minimo di competenza magica e corpo/mente/volonta'. se il tiro riesce bene puoi potenziare un attributo presente ma non darlo/aggiungerlo se questo e' assente. se lancio l'incantesimo crea fiamma, inc. base difficolta' 1, se faccio un ottimo tiro potro' potenziare la durata e aoe (l'area di luce che fa) ed il danno, ma non posso aggiungere distanza perche' e' un attributo assente.
ci sara' poi l'incatesimo globo di fuoco, la versione base della palla di fuoco, questa ha piu' attributi ma ad esempio non ha durata, o meglio l'istantanea non puo' essere migliorata.
valutare di aggiungere attributi assenti quando il tiro e' veramente alto, ovvero tiri veramente basso, direi almeno un -6 per aggiungere un attributo a livello base (3 metri)

- questo implica che non ci saranno mai incantesimi super potenti in tutto..

- PF: corpo + ramo/rami. quindi da 4 a 12/13. Rami combattenti aggiungono 3, 
magici 1.. molto magici se avanzati 0, rami ibridi 2/1

- PF: recupero. ogni notte recuperi 1/2 corpo (o corpo?)

- quanti incantesimi lanciare:  a piacere, ma ogni volta che lanci lo stesso il punteggio minimo richiesta di capacita' magica dell'incantesimo aumenta di 1. quindi tiro l'incantesimo cura, richiede corpo 1 e capacita' magica 11 - la prova cmq la faccio sul mio punteggio di capacita' magica e devo fare piu' basso, lo ritiro e richiede corpo 2 e capacita' magica 12, arrivo ad un certo punto che non posso piu' soddisfare i requisiti richiesti anche se le prove le ho sempre superate

- quando lanci un incantesimo e fallisci la prova non succede nulla, ma l'incantesimo l'hai lanciato e quindi di requisiti minimi aumentano di 1

- quando lanci un incantesimo e fallisci con 19-20 la prova succedono cose  brutte

- quando lanci un incantesimo e riesci con 2 la prova ha successo, non aumenti il requisiti ed ottieni quello che si chiama "effetto massimo" ed e' descritto nella scheda dell'incantesimo

- quando difendi  a seconda di quanto bene difendi, ovvero se hai un margine di -3,-6,-9..., rispetto alla tua prova ottieni dei bonus, azioni movimento, penalita' all'attacco successivo

- mostri:

\end{document}