\documentclass[12pt,a4paper,twoside,openany]{book}
\usepackage{quoting}
\usepackage{tcolorbox}
\usepackage{tikz}
\usetikzlibrary{shadows}
\usepackage{multicol}
\usepackage{tocloft}
\usepackage{lmodern}
\usepackage{caption}
\usepackage[utf8]{inputenc}
\usepackage[T1]{fontenc}
\usepackage{setspace}
\usepackage[a4paper]{geometry}
\geometry{verbose,tmargin=2cm,bmargin=2cm,lmargin=2cm,rmargin=2cm}  %std
\setcounter{secnumdepth}{-1}
\usepackage{booktabs}
\usepackage{url}
\usepackage[italian]{babel}
\usepackage{setspace}
\usepackage{graphicx}
\usepackage{amssymb}
\usepackage{makeidx}
%\usepackage[allfiguresdraft]{draftfigure}
%\usepackage{slashbox}
\usepackage{multirow}
\usepackage{titlesec}
\usepackage[unicode=true, bookmarks=true,
pdftitle={Dark Catacomb - DKC},pdfauthor={Andres Zanzani},
breaklinks=false,pdfborder={0 0 1},backref=section,colorlinks=false]
{hyperref}
\hypersetup{colorlinks=true,linkcolor=blue,pdfcreator={LaTeX}}
\usepackage{bookmark}
\usepackage{yfonts}
\usepackage{auncial}
\usepackage{ragged2e}
\usepackage{ulem}


\usepackage{fontspec}
\setmainfont[Path=./, BoldItalicFont=Soutane Bold Italic.ttf, ItalicFont=Soutane Italic.ttf, BoldFont=Soutane Bold.ttf, Ligatures=TeX, Scale=0.94]{Soutane Regular.ttf} 


\usepackage{wrapfig}
\usepackage{fancyhdr}
\usepackage{tcolorbox}
\tcbuselibrary{skins}
\tcbset{colback=brown!10, fonttitle=\scshape}
\usepackage{imakeidx}
\usepackage{cancel}

\def\CountIndexOccurrences#1{%
	\expandafter\newcount\csname #1\endcsname%
	\expandafter\newcount\csname #1\endcsname%
	\def\indexentry##1##2{\expandafter\advance\csname #1\endcsname 1}%
	\IfFileExists{#1.idx}{\input{#1.idx}}{}%
}
\CountIndexOccurrences{OBSS}
\CountIndexOccurrences{Incantesimi}
\CountIndexOccurrences{Mostruario}
\CountIndexOccurrences{OggettiMagici}
\def\TotalBox#1{\vfill%
	\fbox{Ci sono \expandafter\the\csname #1\endcsname\ voci in questo indice}\par}
\makeindex[columns=3, title=Indice Analitico, intoc=true]
\makeindex[columns=3, name=Incantesimi, title=Lista degli Incantesimi, intoc=true]
\makeindex[columns=3, name=Mostruario, title=Lista dei Mostri, intoc=true]
\makeindex[columns=3, name=OggettiMagici, title=Lista degli Oggetti Magici, intoc=true]
\usetikzlibrary{shapes.misc,calc}
\definecolor{lightgray}{gray}{0.95}
\usetikzlibrary{shapes.misc,calc}
\definecolor{lightgray}{gray}{0.95}
\usepackage{fancyhdr}
\pagestyle{fancy}
\fancyhf{} 
\fancyhead[LE,RO]{\leftmark}
\fancyhead[RE,LO]{}
\fancyfoot[C]{\thepage}
\renewcommand{\sectionmark}[1]{\markboth{#1}{}}
\usepackage{xltabular}
\usepackage{tabularx}
\usepackage{pdfpages}
\usepackage{hyperref}
\usepackage{tikz}
\usepackage[absolute,overlay]{textpos}
\usepackage{etoolbox}
\usepackage{soul}
\raggedbottom
\usepackage{array}
\newcolumntype{L}[1]{>{\raggedright\let\newline\\\arraybackslash\hspace{0pt}}m{#1}}
\newcolumntype{k}[1]{>{\centering\let\newline\\\arraybackslash\hspace{0pt}}m{#1}}
\newcolumntype{R}[1]{>{\raggedleft\let\newline\\\arraybackslash\hspace{0pt}}m{#1}}
\newcolumntype{D}[1]{>{\centering}m{#1}}
\newcolumntype{M}[1]{>{\centering\arraybackslash}m{#1}}
\titleformat{\section}{\filcenter\huge\bfseries\accanthis}{\thesection}{1em}\textsc{}
\titleformat{\subsection}{\Large\bfseries\accanthis}{\thesubsection}{1em}\textsc{}
\titleformat{\subsubsection}{\normalsize\bfseries\accanthis}{\thesubsubsection}{1em}\textsc{}
\def\changemargin#1#2{\list{}{\rightmargin#2\leftmargin#1}\item[]}
\let\endchangemargin=\endlist
\setcounter{tocdepth}{3}
\newtcolorbox{narratore}{
	enhanced, % enable advanced settings
	%left = 3mm,
	%width=0.45\textwidth,
	left = 9mm, % pushes text away from the left edge by 10mm
	sharp corners, % disables rounded corners
	rounded corners = southeast, % "round" the bottom right corner
	arc is angular, % make the "round" corner an angle
	arc = 3mm, % controls corner cut
	boxrule=0.6pt, % sets box line thickness
	underlay={%
		\path[fill=black] ([yshift=3mm]interior.south east)--++(-0.4,-0.1)--++(0.1,-0.2); % triangle
		\path[draw=black,shorten <=-0.05mm,shorten >=-0.05mm] ([yshift=3mm]interior.south east)--++(-0.4,-0.1)--++(0.1,-0.2); % triangle edge
		\path[fill=gray!50!black,draw=none] (interior.south west) rectangle node[brown!10]{\Huge\bfseries ?!} ([xshift=8mm]interior.north west);
	},
	drop fuzzy shadow }

\newtcolorbox{enfasi}{
	enhanced,
	arc=5pt,
	boxrule=0.3pt
} 

\usepackage{zref-savepos,graphicx}
\newcommand{\filltopageendgraphics}[2][]{%\filltopageendgraphics[width=.5\linewidth]{image-a}
	\par
	\zsaveposy{top-\thepage}% Mark (baseline of) top of image
	\vfill
	\zsaveposy{bottom-\thepage}% Mark (baseline of) bottom of image
	\smash{\includegraphics[keepaspectratio=true,height=\dimexpr\zposy{top-\thepage}sp-\zposy{bottom-\thepage}sp\relax,#1]{#2}}%
	\par
}


\usepackage{accanthis}
\usepackage[framemethod=TikZ]{mdframed}


\begin{document}
	
\def \versione {0.01}
\thispagestyle{empty}
 
{\Huge \begin{center}
		Dark Catacomb
\end{center}}

\vfill
\begin{center}
	\Large{\color{black} Fantasy Adventure Game}
\end{center}

\pagebreak

	
\bigskip
Non temere l'ignoto, affrontalo con rispetto.
	
	\vspace{\fill}
\begin{center}\textbf{\versione} - \today\end{center}
\thispagestyle{empty}


\newpage~\thispagestyle{empty}%%\newpage~\thispagestyle{empty}


\newcommand{\riga}{\rule{\textwidth}{0.4pt}}


{\Huge \begin{center} Dark Catacomb\end{center}}

\bigskip

\begin{center}{\LARGE Manuale per Giocatore e Arbitro}\\ \end{center}

{\large \begin{center} Guida e Regole per il Gioco di Ruolo Fantasy \end{center}}

\begin{center}di \end{center}

{\LARGE \begin{center} Andres Zanzani \end{center}}

\vspace{2cm}


\vfill

\begin{mdframed}[roundcorner=10pt]

\medskip

\textbf{Playtesting}: ...to be done...

\bigskip

\begin{flushleft}\textbf{Condizioni d'uso}: Dark Catacomb, DKC, è un marchio registrato di Andres Zanzani (azanzani@gmail.com).
\end{flushleft}

\vspace{0.5cm}


\medskip

\end{mdframed}

%}%
%}

\pagebreak

{\LARGE \begin{center}
		Io sono l'Alfa e l'Omega, dice il Signore Dio, Colui che è, che era e che viene, l'Onnipotente!
\end{center}}

\vspace{15cm}

Dark Catacomb e' un gioco di ruolo che tocca diversi temi religiosi. Se l'argomento ti da fastidio, mi dispiace. Cambia gioco.

\pagebreak

\setcounter{page}{1}

\begin{multicols}{2}
\tableofcontents{}

\end{multicols}

\vfill

\begin{changemargin}{0.3cm}{0.3cm}\begin{tcolorbox}
Vuolsi così colà dove si puote
ciò che si vuole, e più non dimandare\\
\end{tcolorbox}\end{changemargin}

\pagebreak

%pietra bianca
%100 atti unici di altruismo disinteressato
%10 demoni uccisi

%10 angeli uccisi
%100 anime raccolte
%100 7 peccati, la grande meretrice

%superbia: radicata convinzione della propria superiorità, reale o presunta, che si traduce in atteggiamento di altezzoso distacco o anche di ostentato disprezzo verso gli altri, nonché di disprezzo di norme, leggi, rispetto altrui;

%avarizia, derivante più precisamente dall'etimologia latina avaritia, collegata all'avidità della fame:[9] cupidigia, avidità, costante senso di insoddisfazione per ciò che si ha già e bisogno sfrenato di ottenere sempre di più;

%lussuria: incontrollata sensualità, irrefrenabile desiderio del piacere sessuale fine a se stesso, concupiscenza, carnalità, divinizzazione del sesso sempre maggiore, che può andare dalla fornicazione sino all'adulterio, e agli atti più estremi e perversi;

%invidia: in relazione a un bene o una qualità posseduta da un altro, si prova dispiacere e astio per non avere noi quel bene e a volte un risentimento tale da desiderare il male di colui che ha quel bene o qualità;

%gola: nel suo senso concreto, è l'irrefrenabile bramosia di ingurgitare cibi o bevande senza fermarsi al limite della sazietà imposto dal corpo, ma proseguire nella consumazione per puro piacere e ingordigia. Nel suo senso astratto, "goloso" è chi abusa di una determinata cosa, andando al di là del limite imposto dalla natura umana.

%ira: alterazione dello stato emotivo che manifesta in modo violento un'avversione profonda e vendicativa verso qualcosa o qualcuno;

%accidia: torpore malinconico, inerzia nel vivere e nel compiere opere di bene, pigrizia, indolenza, infingardaggine, svogliatezza, abulia.



%https://thealexandrian.net/wordpress/17308/roleplaying-games/hexcrawl
%http://osrsimulacrum.blogspot.com/2020/05/making-wilderness-play-meaningful-system.html
%http://www.innomedimaria.it/apocalisse/apocalisse.htm
%https://www.maranatha.it/Bibbia/8-Apocalisse/73-ApocalissePage.htm


\section{Introduzione}
Razza\\
\sout{Caratteristiche}\\
\sout{rami base}\\
\sout{rami avanzati}\\
\sout{Punti Chaos}\\
\sout{Competenze}\\
Costruiamo il personaggio\\
\sout{Regole per le competenze}\\
\sout{Combattimento}\\
\sout{Nascondigli e coperture}\\
\sout{liste armi ed armature}\\
\sout{magia}\\
incantesimi\\
\sout{equipaggiamento}\\
veleni e droghe\\
\sout{movimento}\\
oggetti magici\\
mostruario\\
Condizioni\\
Scheda\\

\pagebreak

\section{Dark Catacomb - L'Ambientazione}


%https://www.gliscritti.it/dchiesa/bibbia_cei08/nt73-libro_dell_apocalisse.htm

\begin{narratore}
L'Ambientazione di Dark Catacomb si inspira a piene mani dall'Apocalisse di San Giovanni e non vuole offendere nessun credente. Se quanto nell'ambientazione vi incuriosisce potete leggere per intero il Libro dell'Apocalisse di San Giovanni, non sono molte pagine!
	
La  vera rivelazione sarà leggere i Dottori della Chiesa.
	
Personalmente ho trovato di grande interesse personale e formativo Le Confessione di Sant'Agostino e Gli esercizi spirituali di Sant'Ignazio di Loyola.
	
Vi suggerisco di nutrire la vostra spiritualità perché, se ancora non ne avete percezione, ha sempre bisogno di ispirazione e illuminazione.
	
\end{narratore}	

\subsection{La Storia}

\begin{multicols}{2}
	


\subsubsection{Quella che già conosci}

Carissimi figlioli vi lascio queste poche pagine preziose, perché rare, affinché non dimentichiate chi eravate, da dove venivate e dove siete destinati ad andare.

Anche se tutti voi conoscete a grandi linee quello che è successo è opportuno chiarire e comprendere il perché noi siamo qui, perché non siamo andati oltri.

Circa 400 anni fa c'è stata l'Apocalisse. Non quella ecologica e climatica, non quella delle cryptovalute, non quella razziale ma quella che il Signore Dio nostro aveva profetizzato a San Giovanni. La vera Apocalisse.

I risultati li vediamo ancora e probabilmente per sempre.\\

Inizialmente grandine e fuoco mescolati a sostanze velenose caddero sulla terra bruciando un terzo del pianeta, un terzo degli alberi e tutta l'erba verde.

Cadde poi una grande meteorite nell'oceano ed un terzo di tutti i mari divennero tossici, un terzo della vita marina morì ed un terzo di tutte le navi andò distrutto.

Un altra grande stella fiammeggiante cadde dal cielo e colpì un terzo dei fiumi e delle sorgenti d'acqua. Anche queste acque divennero tossiche e molti uomini morirono per averle bevute.

Alla quarta tromba angelica un terzo del sole, della luna e delle stelle fu colpito e la penombra divenne eterna.

Poi fu la volta del pozzo dell'Abisso. L'angelo aprì il pozzo e da questo uscirono prima un fumo nero denso come di fornace e poi cavallette che dovevano causare molta sofferenza e dolore ma solo agli uomini, e senza ucciderli.

Queste cavallette grandi come cavalli avevano capelli lunghi e denti da leone, il torace come corazze di ferro ed il rombo delle loro ali come quello degli aeroplani.

Il loro re era l’angelo dell’Abisso, che in ebraico si chiama Abaddon, in greco Sterminatore.

Al sesto squillo di tromba vennero liberati i 4 Cavalieri, si proprio quelli, i Cavalieri dell'Apocalisse ed un altro terzo dell'umanità venne ucciso.

La civiltà, la società, la cultura l'umanità stessa era ormai una vaga ombra di quello che era all'inizio dell'Apocalisse.

Ma il peggio doveva ancora venire.

A questo punto un \textbf{enorme drago rosso}, Satana, con sette teste e dieci corna e sulle teste sette diademi apparì nel cielo. La sua coda trascinava un terzo delle stelle del cielo e le precipitava sulla terra. 

Dal mare salì una \textbf{Bestia} anch'essa con dieci corna e sette teste, sulle corna dieci diademi e su ciascuna testa un titolo blasfemo. Questa creatura immonda era simile a una pantera, con le zampe come quelle di un orso e la bocca come quella di un leone. 

Satana diede la sua forza, il suo trono e il suo grande potere alla Bestia. 

Gli stupidi umani presi per l'ammirazione e la bramosia del potere incominciarono ad adorare Satana perché aveva dato il potere alla Bestia ed alla Bestia perché proferiva parole blasfeme e d'orgoglio contro Dio che stava causando la distruzione della terra.
.
Chiunque adorasse Satana e la Bestia venne marchiato con il 666 o il nome di queste immonde creature affinchè su di loro si scatenasse l'ira di Dio.

Mentre gli angeli avvelenavano e prosciugavano le acque e le sorgenti altre potenze facevano brillare il sole di una luce e calore così intenso che molti uomini perirono per il gran caldo. Ancora più furiosi gli uomini bestemmiavano Dio invece di rendergli gloria.

Dalla bocca del Satana della Bestia continuavano ad uscire gli spiriti dei demoni. Molti di questi e andarono per le nazioni del mondo a radunare i governanti nel luogo che si chiama Armaghedòn.

E le nazioni fecero la guerra tra loro mentre la Bestia gioiva di come aveva offuscato i loro pensieri. E fu l'olocausto nucleare. Del poco che rimaneva rimase ancora di meno, qualsiasi cosa fosse.

Quando poi il settimo angelo versò la sua coppa nell’aria seguirono fulmini, tempeste ed un terremoto così forte che l'intero globo ne fu martoriato.

Solo i credenti, i puri di spirito e cuore, coloro che avevano ricevuto la \textbf{pietra bianca con inciso il loro nuovo nome} dalle mani di un angelo poterono andare alla \textbf{Città Eterna}.

Una città scesa dal cielo splendente come una gemma preziosissima, come pietra di diaspro cristallino. Una città cinta da grandi ed alte mura con dodici porte: sopra queste porte ci sono dodici angeli.
Le mura della città poggiano su dodici basamenti con inciso i nomi dei dodici apostoli dell’Agnello.


\subsubsection{La storia nuova}

Come avrai capito caro figlio c'è qualcosa che non torna.

Tutta l'umanità doveva essere giudicata e quindi salvata o uccisa. Salvata nella Città Eterna o uccisa dalla guerra, dalle piaghe o dalla Bestia.

Eppure, non troppo numerosi, ma siamo ancora qui. In una terra dal cielo tinto di rosso come fosse un perenne tramonto, con un sole stanco e debole ed un cielo privo di firmamento.

Le nostre nuove navi solcano i mari e raccolgono il pesce che ancora c'è.

Le nostre città non esistono più, non esiste più quella che una volta era chiamata industria o tecnologia.

L'inverno atomico è passato e le radizioni, se mai potessimo misurarle, non sono più un problema.

La terra asciutta e brulla reclama lavoro e acqua. La civiltà è tornata indietro di più di mille anni in un periodo di ignoranza e barbarie.

Quello che ci è rimasto è un mondo diverso dalla geografia riscritta, dalla natura distorta.

Perché ci siamo salvati ? perché non siamo stati giudicati?

Per un \textit{disguido}. 

Sappiamo che almeno uno degli angeli che portava le pietre bianche non giunse mai a destinazione e disperse le sue pietre della salvezza sul mondo.

100, 1000, 10000 ? forse un milione?  non lo sappiamo quanti dei nostri avi non ricevettero la pietra.

Non potevano essere uccisi dalla Bestia e dalle sue armate perché puri ma non potevano neanche accedere alla Città Eterna perché non avevo la pietra con il loro nuovo nome.

Noi tutti siamo i loro discendenti. Cerchiamo di sopravvivere in quella che per noi è una \textbf{Oscura Catacomba}\index{Dark Catacomb}.

Camminiamo sopra i resti dei nostri avi, abbiamo sottoterra intere città spogliate di vita e piene di creature che umane non sono più.

Quello che era il \textbf{nostro} mondo ora non lo è più. Demoni e angeli continuano a combattere e le loro discendenze umane portano guerra anche tra noi.

Non esiste più il concetto di nazione, di popolo. So che una volta tutti noi potevamo parlare insieme con una sola lingua, tutti potevano sapere qualsiasi cosa in qualsiasi momento.
Ecco.. non più. Consegnare una missiva a pochi giorni di cavallo è diventato un lavoro anche pericoloso.

Siamo rimasti l'ombra di ciò che eravamo, ma forse questo è anche la nostra salvezza. 

Insieme si stanno ricostruendo villaggi, la terra con il duro lavoro della schiena e delle braccia produce ancora qualche frutto.
Molte nostre comunità cercano la pace e la democrazia, molte altre ubbidiscono al gioco di un demone o di un padrone umano.

Sono tornati i mestieri di una volta e non ci sono accorgimenti o tecnologie che possono lavorare al tuo posto. Ancora mi chiedo come si facesse una volta senza un bravo calzolaio. 

Concetti come la razza purtroppo esistono ancora, l'ignoranza becera e meschina è insita in noi ed il seme di Satana attecchisce vivace tra i bruti e gli stupidi.

Mi piace pensare che il Signore Dio nostro ci abbia voluto dare un altra possibilità per essere salvati.

\subsection{La nuova civilta'}

Città da milioni di abitanti sono state distrutte dalle piaghe, malattie, elementi tossici e dalle guerre dei popoli. 

Quello che sopravvive sono piccoli e pochi grandi insediamenti dove una economia di base o poco più permette alle persone di sopravvivere.

I paesi spesso non superano i 200 abitanti e le città più grandi i 20000.

Si è tornati a quello che era il medioevo.

Ogni tanto qualche reperto antico viene ancora trovato, molto spesso sono cumuli di ruggine o apparecchi che nessuno più sa usare.

Il denaro è stato sostituito da un bene di più pratico valore le gemme.

Gli insediamenti sono costruiti o sfruttando i materiali e resti di antiche città. Si è tornati a costruire attorno alle risorgive d'acqua ed ai fiumi che permettono di coltivare i campi.

I terremoti hanno raso al suolo ogni manufatto umano o hanno portato nel sottosuolo intere città.

Immensi e profondissimi crepacci hanno inghiottito intere regioni. Orde di non morti abitano lussuosi edifici alla ricerca di qualcosa di vivo di cui nutrirsi.

La maggior parte del territorio rimane inesplorato ma non disabitato. Fortificazioni ed gruppi di risorti perlustrano e saccheggiano ciò che rimane.

La natura ha ripreso vita e si è diffusa ma è mutata. Mentre le piante si sono mostrate più resistenti ai cambiamenti gli animali sono regrediti ad uno stato più selvaggio ed aggressivo.

La Bestia si è divertita a creare e popolare il mondo dei mostri che sfidavamo alcuni giochi da tavolo fatti da ragazzi.

\subsection{Quel che rimane}

La Terra, se ancora così la si vuole ancora chiamare è diventata una desolata zona di guerra.

Non pensate però che angeli e demoni si diano battaglia ovunque e dappertutto. Satana ha reclamato la vittoria sulla Terra ma non per questo governa tutto.

La Città Eterna, manifestazione del potere divino, è inespugnabile alle truppe della Bestia ma non per questo l'assedio si ferma.

I Demoni sono creature preziose e rare e Satana sa che deve moderarne l'uso. La Bestia vorrebbe dare a ferro e fuoco ogni angolo del pianeta ma non può usare le truppe del Drago.

Allo stesso modo gli angeli hanno lo scopo di proteggere la Città Eterna e cantare le lodi al Signore e non si immischiano in altre questioni.

Quello che la Bestia può è comandare i risorti e usarli per causare ancora più morti e rigenerare così le sue truppe.

Un avventuriero difficilmente incontrerà un Demone od un Angelo nella sua vita eppure non finirà di incontrare le truppe della Bestia e disgraziati che per qualche tozzo di pane sono disposti ad uccidere ed ingraziarsi le lodi di Satana.

Le zone forse più interessanti solo le profonde ed oscure catacombe che una volta chiamavamo città.
Città purtroppo troppo tecnologiche per risultare ancora pratiche ed abitabili e raramente dotate di funzionali acquedotti o campi agricoli che possano sostenere la vita.

\subsubsection{I Nuovi abitanti}

Per millenni abbiamo creduto di essere soli o addirittura qualcuno era convinto che discendessimo da creature venute dallo spazio.

Adesso sappiamo che non siamo soli, non lo siamo mai stati.

Nel corso dei secoli recenti demoni e angeli hanno giaciuto con le nostre donne e nostri uomini, ne è discesa una stirpe di creature dal destino segnato.

I nefilim sono creature figlie di due mondi, in parte umane in parte angeliche o demoniache.

Angeli e Demoni che non potevano procreare con la loro stirpe hanno scoperto il piacere della carne, o forse dovere, di poter  procreare con gli umani.

\subsubsection{Gli altri abitanti}

Gli altri abitanti sono i risorti. Come nel peggiore degli incubi coloro che non sono stati richiamati al Signore nei giorni dell'Apocalisse sono diventati preda dei demoni.

I morti e risorti sono esseri dal potere non sempre uguale. Coloro che erano più affini a Satana ed alla Bestia hanno sviluppato poteri maggiori, capacità che definiremo ultraterrene se non magiche. La quasi totalità varia tra scheletri, zombi e creature simili sempre affamate.

E mai, mai, mai farsi uccidere da un risorto! La tua anima, il tuo bene più prezioso, verrà altrimenti catturato ed offerto ai risorti di più alto grado se non ai demoni direttamente.


\subsubsection{La maledizione del 33}

Per uno sciocca scherzo di Satana allo scoccare di ogni 33 anni di vita succede qualcosa di infausto. 

Il primo accadimento è l'infertilità, maschile e femminile, nessuna oltre i 33 anni riesce a rimanere incinta e nessun maschio oltre i 33 anni ad essere più fertile.

Il secondo accadimento è che a 66 anni si muore, tutti. Semplicemente non ci si sveglia più. Il cuore cessa di battere e si muore per un pò.
E' scioccante vedere chi si ama, i propri amici morire così senza una causa apparente.

Ed è per questo che il compleanno dei 66 anni viene festeggiato anche più della nascita, con una festa che dura per giorni, con tutti propri cari, parenti ed amici, fino ad arrivare all'ultimo saluto quando il festeggiato va a riposare per un ultima volta come tutti noi lo conoscevamo.

Il problema è ai compimento dei 99 anni, od ai 33 anni dopo la morte, quando le forze demoniache reclamano il corpo e come un morto vivente il defunto risorge.

Questi fatti sono risaputi ed in quasi tutte le famiglie si procede con la cremazione dei corpi dei defunti. Accade in remoti villaggi e quando il dolore per il distacco è troppo forte che i corpi vengano seppelliti in semplici sudari di stoffa. Quello succede dopo 33 anni potete ben immaginarlo.

Il trauma peggiore di tutti però è stato vedere gli avi, coloro che erano morti da tanti e tanti anni risorgere. L'Apocalisse ha fatto risorgere i morti e chi aveva la benedizione dell'Agnello è assunto in cielo, ma tutti gli altri, e vi assicuro che i loro numeri sono incalcolabili sono risorti come dannati, come spiriti affamati della poca vita rimasta.

Immense città finite sottoterra per gli immensi terremoti e fratture sono ora popolate di non morti e forse di qualche sopravvissuto, forse. Di certo tutti i loro tesori ed averi sono rimasti tutti li, pronti per il primo o forse secondo saccheggiaatore.

I saggi sono convinti che sia una maledizione del grande Drago, di Satana, per beffeggiarsi dell'età in cui è morto il Figlio di Dio. 
Satana ha il pieno potere delle le creature rimaste sulla terra che non sono ascese nella Città Eterna si diverte a martoriarci. 

Solo i Nefilim sono immuni a tutte e tre le maledizioni.

\subsection{Vivere e salvarsi}

Carissimi figlioli, dopo quando detto sembra ridicolo parlare di vivere e salvarsi, eppure una flebile speranza c'è sempre.

E' vero che a 33 anni non potrai avere più una discendenza, cosa pericolosissima in un mondo dove siamo rimasti veramente pochi a vivere, ed è altrettanto vero che a 66 anni morirai, ma c'è sempre una speranza. SEMPRE.

Grazie alle visioni indotte dagli angeli e purtroppo dai demoni abbiamo imparato che è possibile salvarsi, evitare l'infausta maledizione del 33 seppure dovendo scegliere di lasciare questo regno.

Ci sono stati lasciate dalle potenze angeliche queste possibilità di redenzione:

\begin{itemize}

\item compiere 100 distinte azioni di pura bontà
\item uccidere 100 demoni
\item trovare una delle pietre bianche ed incidere il proprio nuovo nome

\end{itemize}

chi adempie ad almeno una di queste missioni potrà ascendere alla Città Eterna e salvarsi.\\

Se invece vuoi seguire i dettami di Satana queste sono le azioni che ti porteranno a diventare un Demone

\begin{itemize}

\item compiere 100 distinte azioni di pura malvagità
\item uccidere 100 anime non marchiate con il nome della Bestia o di Satana
\item uccidere 100 angeli
\item vivere l'intera vita ubbidendo solo ai sette peccati capitali

\end{itemize}

chi adempie ad almeno una di queste missioni potrà recarsi alla pozza di fuoco e zolfo dove Satana comanda le sue legioni e chiedere di diventare un Demone.

Si possono definire una salvezza queste scelte ? Non sta a me deciderlo ma al cuore delle persone che vorranno intraprenderlo che vorranno lasciare questo inferno sulla terra per il Paradiso oppure per governarlo come empio Demone.

Molti altri intraprendono la vita dell'avventuriero alla ricerca dei tesori che giacciono incustoditi sopra e sotto il suolo sapendo che prima o poi Satana reclamerà la loro anima e la Bestia la loro vita.


\subsection{Le Stirpi di Dark Catacomb}\index{Le Stirpi di Dark Catacomb}


Le stirpi presenti in Dark Catacomb sono 2, gli Umani ed i Nefilim.

I nefilim sono frutti degli incroci (\textit{proibiti}?) tra umani ed angeli o demoni.

Solo per il fatto di avere sangue angelico o demoniaco non significa che siano buoni o malvagi a priori, a differenza dei loro progenitori ultraterreni i nefilim hanno un anima e come tale sono dotati di libero arbitrio.

I nefilim angelici sono solitamente di bell'aspetto, esadattili (sempre sei dita per mano), i maschi tendono ad avere la barba rossa e le donne una folta capigliatura nera come la pece.

I nefilim demoniaci sono solitamente alti oltre i due metri, con numerosa corna che amano ingioiellare ed ali da pipistrello (anche se molti preferiscono dire da Drago).

Un nefilim sente il richiamo del sangue e deve percorrere non uno ma tutte le scelte del percorso angelico o demoniaco che vorrà intraprendere.

A differenza degli umani un nefilim non è soggetto alla maledizione del 33, il suo percorso di vita può arrivare oltre i 250 anni. I nefilim possono però procreare solo con umani, generando solo semplici umani.

Gli umani invece possono decidere di vivere come meglio credono i 66 anni che gli sono concessi.

Entrambe le stirpi concedono un +1 ad una Caratteristica a propria scelta alla creazione del personaggio, fino ad un massimo di 12.
	
\end{multicols}



\pagebreak

\section{Caratteristiche}\index{Caratteristiche}

\begin{multicols}{2}

\subsection{Le Caratteristiche del Personaggio}\index{Le Caratteristiche del Personaggio}

Le Caratteristiche\index{caratteristiche} di un personaggio servono a comprendere quando possa essere forte e robusto, ma anche atletico se non intelligente e di buon senso. Rappresentano le potenzialità su cui le Competenze costruiscono l'esperienza.

Queste Caratteristiche sono \textbf{Corpo} \index{Corpo}, \textbf{Mente} \index{Mente} e \textbf{Volontà} \index{Volontà} e \textbf{Vigore}\index{Vigore}

\textbf{Corpo} rappresenta tutte le caratteristiche fisiche, quindi forza, resistenza, capacità atletiche. Corpo influenzerà tutte le prove basate sul fisico del personaggio.

\textbf{Mente} rappresenta la capacità di ragionamento, la memoria, la rapidità di pensiero e l'arguzia. Mente influenza tutte le prove in cui il personaggio deve ragionare, ricordare.

\textbf{Volontà} rappresenta il buon senso ma anche il saper resistere a shock emotivi. Volontà viene usato quando si gestiscono animali e si deve fare un lavoro che richieda impegno e dedizione.

\textbf{Vigore} rappresenta l'energia vitale del personaggio e la sua capacità di resistere a colpi od alla magia.

\subsection{Come stabilire le Caratteristiche del personaggio}\index{Come stabilire le Caratteristiche del personaggio}

Il giocatore tira 2d10 e somma il risultato, questo tiro e computo lo esegue per ogni Caratteristica tranne Vigore.

Se il valore sommato di una Caratteristica è inferiore a 6 segnerai 6 nella Caratteristica. Se il valore sommato di una Caratteristica è superiore a 12 segnerai comunque 12 nella Caratteristica.

Il punteggio di \textbf{Vigore} è pari al punteggio di Corpo aumentato dai punti indicati dal Ramo scelto.\index{Vigore iniziale}

\subsubsection{I modificatori alla prove}\index{Modificatori alla prove}

Ogni Prova di Competenza viene modificata dal punteggio della Caratteristica connessa. 
Il modificatore alla prove di Competenza è pari al punteggio di Caratteristica -10. Nelle Competenze è indicata la Caratteristica che la modifica. Questo modificatore viene applicato sia che abbia un valore positivo o negativo al valore della Competenza.

\end{multicols}

\section{Punti Chaos}\index{Punti Chaos}\index{Fortuna del Principiante}

\begin{multicols}{2}
	

\begin{changemargin}{0.3cm}{0.3cm}\begin{enfasi}{Se il destino è contro di noi, peggio per lui. (motto del 1º Reggimento Carabinieri Paracadutisti "Tuscania")}\end{enfasi}\end{changemargin}

In un mondo non facile ne amichevole il Chaos domina il destino. Ogni personaggio può ricorrere ai punti Chaos per influenzare la sua prova o anche quella di un compagno o di un avversario!

Ogni personaggio ha tre segnalini ed e' libero di consumarne fino a tre alla volta. Ognuno di questo conta come un bonus o penalità, quindi prenderne 1 conta come un +1 (o -1), due conta come un +1d4 (o -1d4), prenderne tre da Vantaggio (o Svantaggio) alla prova. 

Quanti Punti Chaos si vogliano usare si dichiara prima del tiro, una volta dichiarato l'ammontare di Punti Chaos non é possibile utilizzarne di più o di meno.

Ogni volta che il personaggio nel tiro dei dadi esca un doppio 0 recupera un punto Chaos.

I punti Chaos vengono azzerati e reimpostati a 3 ad ogni sessione di gioco.

\end{multicols}

\pagebreak

\section{Le Competenze e le Prove}\index{Competenze}\index{Prove}

\begin{multicols}{2}

Ogni personaggio può seguire uno o più Rami ovvero un insieme di competenze e capacità professionali.

Queste Competenze quando apprese faranno parte del bagaglio culturale, conoscitivo e pratico del Personaggio. Il Personaggio usando le Competenze ne migliorerà l'uso.

Il Personaggio in base a quello che viene dichiarato effettuerà una Prova per capire se riesce e come nell'intento. 

\end{multicols}

\subsection{Le Competenze}

\begin{tabular*}{0.93\linewidth}{@{\extracolsep{\fill}}lll}
\textbf{Corpo} & \textbf{Mente} & \textbf{Volontà}\\
\toprule
Atletica				& Occulto					& Artigianato			\\	
Arrampicarsi			& Conoscenza *				& Cavalcare				\\
Artista della fuga		& Erboristeria				& Diplomazia			\\
Armi piccole 			& Disattivare congegni		& Osservare	\\
Armi medie 				& Falsificare				& Mani di fata\\
Armi a due mani			& Incantamento				& Gestire animali\\
Intimidire		 		& Raggirare					& Furtività\\
Nuotare					& Intrattenere				& Orientamento\\
Rissa					& Mercanteggiare			& Percepire Emozioni \\ 
Usare corda		 		& Natura					& Seguire tracce\\
						& Pronto soccorso			& Sopravvivenza\\
						& Tradizioni locali			& \\

\end{tabular*}\\

La \textbf{Conoscenza} va esplicitata su quale argomento verte: Dungeon, Legge, Lingue, Bibliche, Architettura ed Ingegneria, Miti e Leggende, Altre Religioni, Storia, Geografia ...\\

Alcune Competenze hanno una importanza peculiare nel sistema: \textbf{Incantamento} e le varie \textbf{Armi}, la prima permette di lanciare incantesimi e ne aiuta a determinare l'efficacia, la seconda indica la Competenza del personaggio con le varie tipologie di armi e quanto è capace di usarle.

\begin{multicols}{2}
	
\subsection{Le Prove}\index{Effettuare una Prova}

Ogni Competenza ha un valore numerico che ne stabilisce il grado di capacità nell'uso, più e' alto maggiore sarà la facilità con cui supero le prove.

Il valore di Caratteristica-10 è usato come modificatore alla Prova assumendo quindi valori sia positivi che negativi, anche se negativo viene comunque chiamato bonus nel manuale.

Per verificare l'esito di una Prova di Competenza è necessario sommare il valore di Competenza con il bonus di Caratteristica e sottrarre il risultato della somma di 2d10.

Per \textbf{Punteggio di Competenza}\index{Punteggio di Competenza} (PC) (e non valore di Competenza) si intende il valore già sommato del bonus di Caratteristica e del valore di Competenza.\index{Punteggio di Competenza}\index{PC}

Si definisce \textbf{Margine di Successo}\index{Margine di Successo} (MS) il valore di differenza tra il Punteggio di Competenza (PC) con quanto tirato con i dadi.\index{Margine di Successo}\index{MS}

Il Margine di Successo (MS) può assumere valori negativi o positivi. Mentre in una prova di Competenza o Caratteristica il successo della stessa è solo nel MS positivo, con una prova contrapposta non è detto che un valore negativo sia un insuccesso, dipende da quanto fatto dal contendente

La Prova di Competenza d'Armi viene effettuata sia per Difendersi che per Attaccare. Nel manuale troverete la Prova d'Armi per difendersi come \textbf{PDAD} e quella per Attaccare come \textbf{PDAA}.
 
Le \textbf{Prova d'Armi}, sia PDAD che PDAA \index{Prova d'Armi}, come per la altre Prove si effettuano sommando del valore di Competenza con il bonus di Caratteristica (solitamente Corpo) e al risultato si sottrae quello del tiro di 2d10.
Anche per PDAD e PDAA si può calcolare il Margine di Successo.

\subsubsection{I modificatori alla Prova}\index{Modificatori alla Prova}

L'Arbitro può decidere la presenza di modificatori alla Prova in base alla situazione in cui si svolge la Prova.

Qualora ci sia una \textbf{penalità} (ho fretta, è buio, corro, l'avversario è a cavallo ed io sono appiedato..) la difficoltà della Prova aumenta, ovvero devo \textbf{aumentare il valore della Prova} effettuata della penalità presente.

Se invece ho un \textbf{bonus} allora la difficoltà della Prova diminuisce, ovvero devo \textbf{diminuire il valore della Prova} effettuata del bonus presente.

Un Bonus sarà un valore positivo che sommo ai 2d10 tirati nella Prova.\index{Applicare il Bonus}. Una Penalità è un valore negativo che sottraggo al Punteggio dei Competenza.

I \textbf{modificatori alla Prova si cumulano} tra di loro se omogenei, tutti positivi o tutti negativi, e si annullano o scalano a vicenda se di tipo opposto (bonus e penalità).

Es. Devo scalare una parete. Ho un Bonus perché sono presenti degli appigli, ho una Penalità perché sta piovendo, ho due Bonus perché posso aiutarmi con una corda, ho una Penalità perché è buio. La differenza totale tra bonus e Penalità é di 1 Bonus.

I \textbf{modificatori}, Bonus o Penalità, alla Prove assumono il valore di \textbf{1}, qualora ci sia un solo Bonus o Penalità, \textbf{1d4} qualora ci siano due modificatori omogenei attivi. Nel caso i modificatori siano tre o più, ovviamente di tipo omogeneo, si ha il cosiddetto \textbf{\textit{Vantaggio}} oppure \textbf{\textit{Svantaggio}}.

In caso di \textbf{Vantaggio} tiro 3d10 per effettuare la Prova e scarto quello con il valore più alto, poi sommo gli altri due dadi per verificare l'esito della Prova.\index{Vantaggio}

In caso di \textbf{Svantaggio}\index{Svantaggio} tiro 3d10 e scarto quello con valore più basso, poi sommo gli altri due dadi per verificare l'esito della Prova. 

\subsection{Prove Contrapposte}

Viene definita una Prova Contrapposta quando la Prova viene eseguita in contrapposizione da due contendenti per valutare chi ha successo.

Il raffronto che viene eseguito è solo sul Margine di Successo, chi ha quello maggiormente positivo ha successo.

\subsection{Migliorare le Competenze}\index{Migliorare le Competenze}\hypertarget{Migliorare le Competenze}{} \label{Migliorare le Competenze}

Ogni qual volta vi effettua una Prova su una Competenza e questa ha come risultato dei dadi \textbf{19 o 20} si mette un segno vicino alla Competenza. Si possono avere fino a cinque segni vicino ad una singola Competenza.

Quando il personaggio ha tempo di riflettere su quanto accaduto, sulle prove che ha fatto e come queste sono riuscite o fallite, può tirare 2d10 e se la somma dei dadi è superiore al suo \textbf{punteggio di Competenza} allora quel punteggio aumenta di 1.

Una volta fatta questa tiro si cancellano tre segni dalla Competenza.	

\subsection{Competenze ed i loro ambiti di utilizzo}\label{competenzeambitidiutilizzo}

Sono descritte sommariamente le Competenze ed i loro ambiti di utilizzo. Sono indicazioni di massima su cosa usare le competenze. Viene anche indicato il numero di Punti Azioni (PA) necessarie per svolgere la prova tipica, ovvio che usi più complessi richiedono più tempo e Punti Azioni (PA).

I PA necessari alla prova possono variare a seconda della capacità del personaggio e della complessità della prova.

In ogni caso ricordate sempre di valutare con attenzione come il giocatore dichiara di svolgere le azioni per capirne la durata ed effetti. 

La Competenze con un \textbf{*} subiscono le penalità dovute all'armatura indossata.\\

\textbf{Atletica* (Corpo)}: Questa competenza serve per mantenere l'equilibrio su superfici strette o precarie, per tuffarsi, rotolare, fare capriole, salti mortali, superare degli ostacoli nonché cadere e non farsi male. 

\textbf{Arrampicarsi* (Corpo)}: Con questa competenza si possono scalare superfici verticali, dalle mura cittadine alle pareti rocciose. E' legata all'Azione di movimento. Con 8 punti il movimento è solo dimezzato.

\textbf{Artigianato (Volontà)}: E' necessario specificare la tipologia di Artigianato in cui si è competente.

\textbf{Artista della fuga (Corpo)}: Con questa competenza ci si può liberare da legacci e manette.

\textbf{Cavalcare (Volontà)}: Con questa competenza è possibile cavalcare in maniera professionale e dare comandi alla propria cavalcatura. 

\textbf{Osservare (Volontà)}: per cercare, accorgersi, notare. E' un qualcosa di attivo.

\textbf{Conoscenza dei Dungeon (Mente)}: Con questa competenza si hanno conoscenze di Aberrazioni, melme, caverne, esplorazioni sotterranee.

\textbf{Conoscenze di Geografia (Mente)}: Con questa competenza si hanno conoscenze sul clima, popolazione, terreni, territori, nazioni e confini pre e post Apocalisse.

\textbf{Conoscenza Lingue/Linguaggi (Mente)}: Con 1 punto sai parlare una lingua, con 3 punti la sai anche scrivere. Un buon punteggio di Lingue aiuta a comprendere lingue non note ed a farsi comprendere. Viene usata anche per comprendere testi complessi

\textbf{Conoscenze Bibliche (Mente)}: Con questa competenza si è esperti di occulto, creature immonde e celestiali. 

\textbf{Conoscenze Altre Religione (Mente)}: Con questa competenza si hanno conoscenze sulle religioni che governano la Terra come simboli sacri, tradizione ecclesiastica, feste e ricorrenze liturgiche. 

\textbf{Conoscenze di Storia (Mente)}: Con questa competenza si hanno conoscenze di Storia quali guerre, migrazioni, colonie, fondazioni di città, accadimenti importanti..

\textbf{Diplomazia (Volontà)}: Con questa competenza si possono risolvere diverbi, e raccogliere preziose informazioni e dicerie dalle persone. La competenza è anche usata per negoziare in modo efficace con la giusta etichetta e condotta adatta alla situazione controversa. 

\textbf{Disattivare congegni (Mente)}: Con questa competenza si possono disarmare Trappole e aprire serrature, sabotare congegni meccanici semplici, come le catapulte, le ruote di un carro o le porte.

\textbf{Erboristeria (Mente)}: Con questa competenza si hanno conoscenze di come riconoscere e preparare pozioni e veleni naturali. Il punteggio si applica alle prove per distillare pozioni.

\textbf{Falsificare (Mente)}: Con questa competenza si sa falsificare oggetti d'arte, mappe, firme... 1 Minuto

\textbf{Gestire animali (Volontà)}: Con questa competenza è possibile addestrare e ammansire animali.

\textbf{Intimidire (Corpo)}: Intimidire si basa sull'approccio fisico per convincere l'interessato. 

\textbf{Intrattenere (Mente)}: Con questa competenza si è esperti in una espressione artistica, dal canto alla recitazione, dal ballo a suonare strumenti musicali. E' necessario specificare la forma di intrattenimento.

\textbf{Mani di fata* (Volontà)}: Con questa competenza si può borseggiare, estrarre un'arma nascosta, oppure compiere altre azioni senza essere notati. 

\textbf{Furtività (Volontà)}: Con questa competenza si è in grado di muoversi senza causare rumore oppure di passare inosservati stando fermi. 

\textbf{Natura (Mente)}: Con questa competenza si hanno conoscenze di Animali, Fatati, stagioni e cicli, tempo atmosferico, vegetali. 

\textbf{Nuotare* (Corpo)}: Con questa competenza si è in grado di nuotare, anche in acque tempestose. Senza competenza si sa stare a galla in acqua placide. Legata all'Azione di movimento.

\textbf{Occulto (Mente)}: Con questa competenza si è esperti di magia e di incantesimi, di oggetti magici è si è grado di identificare gli incantesimi che vengono lanciati. 

\textbf{Orientamento (Volontà)}: Con questa competenza si ha il senso della direzione e orientamento rendendo impossibile perdersi indipendentemente dall'ambiente in cui ci si trova. 

\textbf{Percepire Emozioni (Volontà)}: Con questa competenza si può capire se qualcuno sta mentendo o si possono intuire le sue vere intenzioni.

\textbf{Pronto soccorso (Mente)}: Con questa competenza si possono curare le ferite e le malattie. Costo variabile.

\textbf{Raggirare (Mente)}: La competenza Ingannare può essere usata per Raggirare (dicendo quindi fandonie) o Persuadere (adattando la verità) al fine di convincere delle proprie parole l'interessato.

\textbf{Seguire tracce (Volontà)}: Con questa competenza si sa seguire le tracce lasciate nell'ambiente. 

\textbf{Sopravvivenza (Volontà)}: Con questa competenza si può sopravvivere e orientarsi nelle terre selvagge. La competenza è usata anche per cercare attivamente trappole e fosse.

\textbf{Tradizioni locali (Mente)}: Con questa competenza si hanno conoscenze degli abitanti (più noti), costumi, leggende, leggi, personalità, tradizioni. E' necessario specificare una regione geografica dove è applicabile la conoscenza. 

\textbf{Usare corda (Corpo)}: Con questa competenza si è esperti in legacci e nodi per fissare e bloccare oggetti o persone. 

\textbf{Mercanteggiare (Mente)}: Con questa competenza si sa stimare e contrattare il valore monetario di un oggetto.

\subsubsection{Esempi Prove Competenza}\label{esempiprovecompetenza}\hypertarget{esempiprovecompetenze}{}\index{Esempi prove Competenza}

\textbf{Prove atipiche}\index{Prove atipiche}. Il giocatore è invitato a trovare usi, soluzioni, approcci che esulino dalle più ovvie prove. Siate creativi e descrivete al Arbitro la meravigliosa azione che volete fare e quali risultati sperate di ottenere! Sarà lui a stabilire in base alla vostra descrizione dell'azione cosa provare e se hai dei bonus o penalità.

\medskip

Per \textbf{riconoscere un oggetto magico}\index{Riconoscere oggetto magico} e le sue capacità è necessaria una prova di \textbf{Arcana} per avere indicazioni di massima sui poteri e ambiti di utilizzo, con un MS di almeno 6 puoi apprenderne i dettagli, bonus magici e cariche. \textbf{10 minuti}. Con punteggio Arcana 6 costa 5 minuti, con 12 costa 1 minuto, con Arcana 18 costa 10 PA.

\medskip

\textbf{Riconoscere un incantesimo}\index{Riconoscere un incantesimo} mentre viene lanciato è una prova di \textbf{Arcana} Costa una \textbf{Reazione}. Se fatto assieme al lancio di un Controincantesimo non costa Reazione.

\medskip

Per \textbf{riconoscere un mostro}, una creatura particolare si effettua una prova di Conoscenza. Controlla il capitolo \hyperlink{riconoscereimostri}{Riconoscere i Mostri} nel Mostruario (pag. \pageref{riconoscereimostri})

\medskip

\textbf{Atletica}\index{Atletica} \textit{Penalità dovute all'armatura}

Una prova di Atletica riuscita permette al personaggio di dimezzare il danno quando cade da meno di 9 metri (\textbf{Reazione}).

\medskip

\textbf{Arrampicarsi/Scalare} \index{Arrampicarsi}\index{Scalare} \textit{Penalità dovuta all'Armatura.}

\medskip

Usare una corda\index{Arrampicarsi su una corta}\index{Salire su una corda}, scalare od arrampicarsi equivale a muoversi in un \textbf{terreno doppiamente difficile}. Se la prova di Arrampicarsi riesce si sale di 30 cm per PA speso.

In caso di fallimento della prova si consumano i PA senza spostarsi. Se la prova fallisce (MS negativo) di 6 o più perdi la presa e cadi. I modificatori indicati nella tabella si sommano.\\

\begin{tabularx}{0.45\textwidth}{Xl}
	\textbf{Esempio di Superficie} & Mod.\\
	\toprule
	Movimento solo dimezzato & -1\\
	Superficie scivolosa&-1\\
	Parete grezza con appigli, mattoni sporgenti&-1\\
	Una corda senza nodi&-2\\
	Una parete con appigli &+2\\
	Un muro/parete con pochissimi appigli&-3\\
	Ti puoi appoggiare a 2 pareti opposte&+2\\
	Ti puoi appoggiare a 2 pareti angolari&+1\\
	Puoi usare una corda&+2\\
\end{tabularx}\\

i modificatori indicati sono sulla prova effettuata dal giocatore, se positivo è un Bonus.

\medskip

Per \textbf{identificare una pozione o veleno naturale}\index{Identificare Veleno}\index{Erboristeria} \index{Identificare Pozione}è necessario una prova di \textbf{Erboristeria}.

Costa 5 minuti. Se il MS è +3 impieghi 4 minuti, se +6 impieghi 3 minuti, con -9 impieghi 2 minuti, +12 impieghi 1 minuto, +15 impieghi 1 round.

Se la prova fallisce ed il MS è -6 ha assunto la pozione consumandone una dose.

\medskip

\textbf{Intimidire}\index{Intimidire}. Il personaggio usa \textbf{6 PA} ed effettua una prova di Intimidire, l'avversario può contrapporre una prova di Intimidire o di Corpo. Chi ottiene il MS migliore intimidisce l'avversario.

Chi è intimidito ha 1 Penalità al PDA fino alla fine del round successivo.

\medskip

\textbf{Ammansire un animale} è una prova di \textbf{Gestire Animali}. Tempo richiesto 10 minuti. Per ogni MS il tempo si riduce di 1 minuto.

\medskip

\textbf{Furtività} \index{Furtività} \textit{Penalità dovuta all'Armatura.}

La prova di Furtività va effettuata solo se c'è qualcuno che può sentire/vedere. Si confrontano i MS delle prove di Furtività e Osservare per capire se si è stati percepiti. Muoversi in maniera Furtiva equivale a muoversi su terreno difficile e quindo ci vogliono 2 PA per spostarsi di 1.5 metri.

\medskip

\textbf{Nuotare}\index{Nuotare} \textit{Penalità dovuta all'Armatura}

In acque calme basta una prova riuscita di nuotare, se le acque sono mosse il MS deve essere di almeno 3, e di 6 se molto mosse e 9 se tempestose. La prova è necessaria sia per stare a galla o nuotare. Nuotare in acqua si considera \textbf{terreno difficile}.

\medskip

\textbf{Pronto Soccorso}\hypertarget{prontosoccorso}{}\label{prontosoccorso}\index{Pronto Soccorso}. Una prova riuscita fa recuperare 1d4 Vigore se fatta entro 1 minuto dal termine dello scontro.

Concede 1 Bonus ad una prova di Caratteristica contro un veleno se non ha ancora fatto effetto. Costo \textbf{2 minuti}. Con MS di +6 costa 1 minuto. Con MS +9 costa 3 round, con MS +12 costa 1 round.

Una prova riuscita riduce di 1 i danni da \hyperlink{sanguinamento}{\textbf{Sanguinamento}}. Se la prova riesce con MS +3 riduce di 2 punti, con MS +6 riduce di 3 punti.

Un trattamento di almeno 8 ore permette di recuperare al paziente il doppio di Corpo in punti Vigore. Se effettuato durante le ore di riposo chi prende cura risulterà Affaticato.

\medskip

\textbf{Saltare}\index{Tabella Saltare} \textit{Penalità dovuta all'Armatura.} \textbf{4 PA}\\

Con una prova di Atletica è possibile saltare in lungo 3 metri. Per ogni MS salti 30 cm in più.

La \textbf{distanza saltata in alto} è pari a 90cm + 10cm per MS.

In un \textbf{salto in lungo} la punta più alta del salto è pari ad un 1/4 della lunghezza saltata. Se esegui un salto in lungo di 6 metri a metà salto sei in alto di 1.5 metri.

Scendere da meno di 1m non usa PA. Se non si ha almeno 3 metri di rincorsa si salta la metà.

Vigore perso per caduta (pag. \pageref{cadute}): 3x altezza caduta (in metri). Prova di Atletica per dimezzare il danno se cadi da meno di 9 metri.

\medskip

\textbf{Sopravvivenza}\index{Sopravvivenza}

\smallskip

\textbf{Inseguire una creatura}:

\begin{tabular}{ll}
	Situazioni & Mod.\\
	\toprule
	Se il terreno è molto morbido& +2\\
	Se il terreno è morbido& +1\\
	Se il terreno è stabile& 0\\
	Se il terreno è duro& -2\\
	Ogni 6 creature inseguite& +1\\
	Ogni 24 ore passate & -1\\
	Visibilità scarsa&-1\\
	Ogni ora di pioggia&-1 \\
	Cerca di occultare le traccie& -1\\
\end{tabular}\\

Un Modificatore negativo è una penalità (abbassa il risultato dei dadi), un modificatore positivo è un bonus (alza il risultato del Punteggio di Competenza).

Sopravvivenza può essere usata al posto di \textbf{Disattivare Congegni} con Svantaggio.

Una prova di Sopravvivenza per foraggiare cibo procura viveri per una persona aggiuntiva ogni 3 di MS.

\medskip

La prova di \textbf{Mercanteggiare}\index{Mercanteggiare} serve per abbassare il prezzo di una merce e per valutare un oggetto. Oggetti molto rari richiedono un MS di almeno +3 per essere valutati.


\end{multicols}

\pagebreak


\section{I Rami}\index{Rami}

\begin{multicols}{2}

I \textbf{Rami} sono la professione del personaggio, è l'insieme delle competenze che il personaggio conosce. In altri sistema di gioco il Ramo sarebbe l'equivalente della classe.

\textbf{Ogni Ramo conferisce al personaggio dei punti Vigore}, da sommare a Corpo per stabilirne il valore iniziale, e delle Competenze.

I Rami di base ed avanzati concedono 6 Competenze.

Mentre i \textbf{Rami di base} possono essere presi come prima professione da chiunque, i\textbf{ Rami avanzati} possono essere appresi solo a patto di soddisfare i requisiti indicati.

\subsection{Rami Base}

Nella tabella sottostante sono indicati alcuni Rami base di esempio. il personaggio è invitato a crearsi Rami con Competenze più affini alla sua storia.
Sono indicati il nome del Ramo e il Vigore. Le \textbf{competenze prendono un punteggio} pari alla punteggio indicato dalla prima colonna. Ad esempio un Apprendista ha punteggio 6 in Conoscenza mentre un Bandito ha 4 punti in Furtività.

\end{multicols}

\begin{tabular}{|l|l|l|l|l|}\hline

&\multicolumn{4}{c}{\textbf{Rami}}\\\hline

&\textbf{Apprendista}&\textbf{Bandito}&\textbf{Tagliaborse}&\textbf{Baro}\\\hline
\textbf{Vigore}&2&9&4&6\\\hline
6&Conoscenza &Armi piccole&Mani di fata&Raggirare\\
5&Incantamento&Intimidire&Furtività&Per. Emozioni\\
4&Pronto soccorso&Furtività&Armi piccole&Armi piccole\\
3&Intimidire&Armi medie&Sopravvivenza&Mani di fata\\
2&Armi piccole&Mani di fata&Pronto Soccorso&Rissa\\
1&Tradizioni Locali&Pronto soccorso&Atletica&Intimidire\\\hline

&\textbf{Tirapiedi}&\textbf{Cacciatore}&\textbf{Segugio}&\textbf{Commerciante}\\\hline
\textbf{Vigore}&6&4&7&2\\\hline		
6&Rissa &Natura&Armi da tiro&Mercanteggiare\\
5&Intimidire&Sopravvivenza&Intimidire&Raggirare\\
4&Furtività&Armi da tiro&Seguire tracce&Per. Emozioni\\
3&Osservare&Furtività&Osservare&Intrattenere\\
2&Cavalcare&Erboristeria&Mercanteggiare&Armi piccole\\
1&Mani di fata&Osservare&Armi piccole&Cavalcare\\\hline

&\textbf{Contafrottole}&\textbf{Guardia del corpo}&\textbf{Intrattenitore}&\textbf{Mendicante}\\\hline
\textbf{Vigore}&3&6&3&3\\\hline		
6&Raggirare			&Armi a due mani&Intrattenere	&Osservare\\
5&Percepire Emozioni&Orientamento	&Travestimento	&Sopravvivenza\\
4&Storia			&Pronto soccorso&Storia			&Mercanteggiare\\
3&Armi piccole		&Osservare		&Armi piccole	&Armi piccole\\
2&Mercanteggiare	&Conoscenza		&Raggirare		&Furtività\\
1&Conoscenza		&Intimidire		&Diplomazia		&Mani di fata\\\hline

&\textbf{Mercenario}&\textbf{Milizia cittadina}&\textbf{Minatore}&\textbf{Nobile}\\\hline
\textbf{Vigore}&16&9&4&2\\\hline		
6&Armi medie		&Armi medie		&Conoscenza	caverne	&Diplomazia\\
5&Armi a due mani	&Tradizioni locali&Sopravvivenza	&Conoscenza\\
4&Sopravvivenza		&Intimidire		&Armi medie			&Linguaggi\\
3&Armi da tiro		&Cavalcare		&Orientamento		&Storia\\
2&Armi piccole		&Rissa			&Mercanteggiare		&Cavalcare\\
1&Intimidire		&Armi da tiro	&Atletica			&Intimidire\\\hline
							
\end{tabular}	\\

Il Vigore che concede un Ramo si calcola sommando i punteggi delle Armi. Il valore minimo di Vigore è 2.

\begin{multicols}{2}	

\subsection{Rami Avanzati}

I \textbf{Rami avanzati}\index{Rami Avanzati} hanno un prerequisito di Caratteristica e di Competenza a certi punteggi.
Il Vigore che concede un Ramo avanzato è pari a 3 per il numero di Armi che fa conoscere. In valore minimo di Vigore che un Ramo avanzato concede è +3.
Il Vigore concessa da un Ramo avanzato si somma con quella già posseduta dal personaggio.

I Rami avanzati aumentano il punteggio delle Competenze di 1, se sono Competenze già note, altrimenti impostano ad 1 il valore se non erano note.

Alcuni esempi di Rami avanzati:\medskip

\textbf{Assassino}:\\
\textit{Requisito}: Corpo 12, Sopravvivenza 12, Armi piccole 12\\
\textit{Competenze concesse}: Armi da tiro, Sopravvivenza, Travestimento, Armi medie , Furtività , Osservare\\
\textit{Vigore}: +6\\

\textbf{Sgherro}:\\
\textit{Requisito}: Corpo 12, Intimidire 12, Armi piccole 12\\
\textit{Competenze concesse}: Armi da tiro, Rissa, Armi a due mani, Sopravvivenza, Mani di fata, Intimidire\\
\textit{Vigore}: +9\\

\textbf{Capitano della Guardia}:
\textit{Requisito}: Volontà 12, Armi medie 12, Tradizioni Locali 12
\textit{Competenze concesse}: Armi da tiro, Cavalcare, Intimidire, Sopravvivenza, Storia, Pronto soccorso\\
\textit{Vigore}: +3\\

\textbf{Esploratore}:\\
\textit{Requisito}: Corpo 12, Sopravvivenza 12, Orientamento 12\\
\textit{Competenze concesse}: Geografia, Armi piccole, Atletica, Nuotare, Cavalcare, Linguaggi\\
\textit{Vigore}: +3\\

\textbf{Cavaliere errante}:\\
\textit{Requisito}: Volontà 12, Cavalcare 12, Armi medie 12\\
\textit{Competenze concesse}: Armi da tiro, Armi a due mani, Diplomazia, Pronto soccorso, Linguaggi, Storia\\
\textit{Vigore}: +6\\

\textbf{Menestrello}:\\
\textit{Requisito}: Mente 12, Intrattenere 12, Conoscenza 12\\
\textit{Competenze concesse}: Armi piccole, Raggirare, Storia, Travestimento, Diplomazia, Incantamento\\
\textit{Vigore}: +3\\

\textbf{Incantatore}:
\textit{Requisito}: Mente 12, Incantamento 12, Conoscenza 12
\textit{Competenze concesse}: Armi medie, Conoscenza, Storia, Linguaggi, Diplomazia, Erboristeria 
\textit{Vigore}: +3\\

\textbf{Monaco}:\\
\textit{Requisito}: Corpo 12, Volontà 12, Pronto soccorso 12\\
\textit{Competenze concesse}: Rissa, Diplomazia, Conoscenza, Storia, Linguaggi, Pronto soccorso\\
\textit{Vigore}: +3\\


\subsection{Avanzare nel Ramo}\index{Avanzare nel Ramo}

Ogni volta che la\textbf{somma dei punti assegnati} grazie al \hyperlink{Migliorare le Competenze}{Migliorare le Competenze} arriva a 3 il Vigore aumenta di 3 per i Rami base.
In caso di \textbf{Rami avanzati} la somma dei punti assegnati deve arrivare a 5 per poter aumentare di 5 il Vigore.

\subsection{Prendere un nuovo Ramo}\index{Prendere un nuovo Ramo}

Alla creazione del Personaggio si scegli un Ramo e si apprendono Competenze e si imposta il valore di Vigore a Corpo + il valore di Vigore dato dal Ramo.

Il personaggio può decidere di acquisire più \textbf{Rami base} e quindi conoscere più Competenze. Per poter intraprendere un nuovo Ramo il personaggio deve trovare qualcuno che possa insegnarglielo e pagare 1000 monete d'oro. Tutte le Competenze del suo Ramo precedente diminuiscono di 1.
Il fatto di prendere un nuovo Ramo non fa aumentare il Vigore, rimane valida la regole dei 3 punti distribuiti prima di aumentare il Vigore.

Per prendere un \textbf{Ramo avanzato} si devono soddisfare i requisiti indicati, trovare qualcuno che possa insegnarlo, pagare 5000 monete d'oro. In questo caso non c'è una diminuzione delle Competenze originari ed i punti Vigore non si acquisiscono.


\end{multicols}

\pagebreak

\section{Combattimento}\index{Combattimento}\label{Combattimento}\hypertarget{Combattimento}{}

\begin{multicols}{2}
	
Il combattimento è diviso in 2 fasi:\index{Combattimento}
\begin{itemize}
	\item verifica delle Azioni
	\item verifica Iniziativa
	\item risoluzione delle Azioni (movimento, attacco, azione varie..)
\end{itemize}

L'unità base di tempo nelle scene di combattimento è il Round ovvero una unità di tempo di 10 secondi.

Ogni personaggio può eseguire diverse Azioni nel round ed ognuna di queste costa dei Punti Azione. \textbf{Chi meno ne esegue più è veloce nell'eseguirle}.

All'inizio di ogni round il giocatore dichiara quanti PA userà. Non è necessario che dichiari cosa andrà ad eseguire.

La verifica dell'Iniziativa\index{Iniziativa} consiste nel tirare 2d10 sottrarre Corpo o Mente ed aggiungere i PA che si intendono usare.

\textbf{Il personaggio od avversario che ha un valore dell'Iniziativa più basso incomincia per primo.}

A parità di Iniziativa chi usa meno Punti Azione (PA) incomincia per primo, se i PA usati sono uguali parte per primo l'avversario.
A parità di PA usati tra i personaggi questi si mettono d'accordo tra loro per agire in che ordine.

Il personaggio potrebbe anche usare meno PA di quanti dichiarati, ma agirà in ogni caso nell'ordine stabilito dal valore dell'Iniziativa.

\textbf{Un personaggio non può dichiarare un certo numero di PA e poi usarne di più.}

\subsubsection{Il Tempo (Round, Minuti e Turni)}\index{Round}\label{iltempo}

Un \textbf{round} dura 10 secondi circa, è un lasso di tempo sufficiente per agire, correre, parlare.. combattere. Un Minuto sono 6 round ed un Turno dura 10 Minuti (o 60 round).

I round si usano nelle scene di combattimento o dove la tensione deve rimanere costantemente alta ed ad ogni Azione corrisponde un evolversi della situazione.

\subsubsection{Tempo di riattivazione Oggetti ed Abilita'}\index{Tempo di attivazione Oggetti ed Abilità}\label{temporiattivazioneoggetti}

Se non specificato diversamente un oggetto o Abilità che prevede un certo numero di usi al giorno \textit{"es. una volta al giorno"} si "ricarica" all'alba successiva l'uso.

\subsection{Punti Azioni nel Round}\index{Azioni nel Round}\index{Azione}\label{azioninelround}

Nella tabella sottostante sono indicate le Azioni principali e relativi Punti Azione che usano, sono linee guida da seguire. Nel capitolo dedicato al combattimento vengono elencate altre Azioni ed i loro costi relativi in Punti Azioni.

Le Azioni scelte possono essere eseguite nell'ordine preferito.

Una Azione non può essere interrotta\index{Interrompere Azioni}\index{Azioni, Interrompere} da un altra Azione, ma può essere seguita da una Azione di Reazione o da una Azione Immediata, se nel proprio round.

Se un personaggio vuole fare più attacchi spostandosi nel campo di battaglia può, ad esempio, usare 4 PA per eseguire un attacco, usare 2 PA per muoversi di 3 metri ed usare gli ultimi 4 PA (nel caso ne avesse dichiarati 10) per un ultimo attacco.

E' possibile \textbf{ritardare} una o più Azioni\index{Ritardare Azioni} per aspettare lo svolgersi delle scene. Il personaggio che ritarda una sua Azione si considera che abbia "sprecato" PA per aspettare fino a quel segmento di iniziativa e potrà usare solo i PA che rimangono fino alla fine del round.

Un giocatore che dichiara di aspettare una certa situazione per poter agire equivale ad eseguire una o più \textbf{Azioni Preparate}\index{Azioni Preparate}. In questo caso il personaggio (o nemico) agisce dopo l'Azione con solo i PA rimasti per avere aspettato fino a quel segmento di iniziativa.

Se il personaggio ha già usato tutte i PA allora potrà agire fuori dalla sua iniziativa solo tramite una Reazione, se a disposizione. L'Azione di Reazione arriva sempre dopo l'Azione scatenante.

\bigskip

\end{multicols}

\textbf{Tabella: Azioni per Round}\index{Tabella delle Azioni per Round}



\begin{tabularx}{0.95\textwidth}{Xc}
\textbf{Cosa si fa}  & \textbf{Punti Azioni}\\
\toprule
Attaccare con Rissa					& 4\\
Attaccare con Arma piccola			& 4\\
Attaccare con Arma media/da Tiro	& 5\\
Attaccare con Arma a due mani		& 8\\
Lanciare un'Incantesimo *1			& *\\
Muoversi *2							& 1 PA per 1.5 metri\\
Scatto *3							& 1 PA per 3 metri\\
Alzarsi da prono					& 4\\
Aiutare qualcuno					& 5\\
Scambiare un dialogo con qualcuno *4	& variabile\\
Scambiare poche battute con qualcuno *5& 0\\
Prendere qualcosa nello zaino		& 8\\
Prendere qualcosa dalla cintura o di pronto & 4\\
Usare un oggetto tenuto in mano		& 2\\
Bere una pozione tenuta alla cintura& 4\\
Estrarre/Rinfoderare l'arma			& 3\\
Imbracciare lo scudo				& 3\\
Usare un oggetto magico				& 6\\
Eseguire prova su una competenza *6	& 6\\
Sfondare una porta a spallate/calci	& 5\\
Forzare porta con piede di porco	& 6\\
Nascondersi							& 4\\
Concentrarsi su un Incantesimo		& 4\\
Salire o scendere dalla cavalcatura	& 4\\
Azione \textbf{I}mmediata - Azione \textbf{R}eazione& I - R\\
Bere una pozione tenuta in mano& I\\
Gettare un oggetto tenuto in mano& R\\
Gettarsi a terra prono& R\\
Riconoscere un Incantesimo& R\\
\end{tabularx}

\medskip

Per Attacco si intende sia l'uso di armi in mischia che l'uso di armi da lancio o tiro come archi, balestre o pugnali da lancio. Nel caso di armi da lancio ogni lancio/tiro conta come un attacco.

Qualora il personaggio esegua una Azione di Attacco e Lanciare un incantesimo si considera Distratto nell'eseguire la Prova di Incantamento.

\begin{multicols}{2}
	
\textbf{Attaccare con armi}: ogni attacco effettuato somma il PA usati. Attaccare due volte con un arma piccola costa 8 PA, attaccare a due mani con un Arma Piccola ed una Media costa 9 PA.

\textbf{Lanciare un Incantesimo *1}: a seconda del potere dell'Incantesimo sono necessari più PA. Nella descrizione dell'incantesimo è indicato il numero di PA necessari. 

\textbf{Muoversi *2}: per ogni PA usato ci si può muovere fino a 1.5 metri.

\textbf{Scatto *3}: per ogni PA usato ci si può muovere fino a 3 metri ma si incorre nella penalità di aver \textbf{corso}.

\textbf{Scambiare un dialogo con qualcuno *4}: Un dialogo può essere di pochi secondi se non di minuti. L'Arbitro valuterà quanto questo dura.

\textbf{Scambiare poche battute con qualcuno *5}: Finché sono veramente poche battute o uno sguardo non consuma PA, se questo diventa più articolato allora utilizza dei PA. L'obiettivo è non interrompere il flusso delle Azioni con un fitto dialogo ma comunque permettere l'interazione tra i personaggi.

\textbf{Eseguire prova su una competenza*6}: se fruttano una frazione del round costano 4 PA, altrimenti 8 PA o più. Controllate negli Esempi Prove Competenza i costi riportati.

Una Azione "\textbf{Reazione (R)}" \index{Azione Reazione}può essere eseguita liberamente anche fuori dal proprio round. Questa Azione è solitamente dovuta ad Abilità o situazioni particolari. Se non indicato diversamente una Azione di Reazione accade immediatamente dopo la causa che la scatena.

Una Azione "\textbf{Immediata (I)}" \index{Azione Immediata}può essere eseguita liberamente nel proprio round, primo o dopo la propria Azione. Una Azione Immediata è solitamente concessa da particolari Abilità.

E' possibile se non descritto diversamente eseguire solo una Azione Immediata ed una Azione di Reazione per round.

\medskip

Questo \textbf{elenco non è completo}, prendetelo come linee guida per stabilire il peso delle decisioni ed azioni dei giocatori. Una Azione dura circa 3 secondi.

L'\textbf{ordine} con cui si eseguono le Azioni non è importante se non per correlazione logica e fisica. Il Movimento può essere tra altre Azioni (movimento, attacco/incantesimi/altra azione, movimento).


\end{multicols}

\subsection{Movimento}\index{Movimento}\label{movimento}


\begin{changemargin}{0.3cm}{0.3cm}\begin{enfasi}{"Un mobile più lento non può essere raggiunto da uno più rapido; giacché quello che segue deve arrivare al punto che occupava quello che è seguito e dove questo non è più (quando il secondo arriva); in tal modo il primo conserva sempre un vantaggio sul secondo" (Paradosso di Zenone)}
\end{enfasi}\end{changemargin}

\begin{multicols}{2}

Il movimento di un personaggio è dato dalla sua taglia e razza e da ciò che porta, dai pesi, ingombri ma anche magie ed oggetti magici.

Umani e Neflim possono muoversi di 1,5m per PA usati, altre creature possono compiere distanze diverse per PA usato.

Una creatura o personaggio potrebbe anche decidere di spostarsi più velocemente del solito ovvero correndo (Azione di Scatto).\index{Correre}

Non è possibile spostarsi anche solo di 1 metro se non si spendono PA.

Queste precisazioni hanno senso e vanno usate quando si tratta di combattere ed il dislocamento sul territorio, mappa, è fondamentale. Durante gli spostamenti normali, mentre si cavalca o cammina liberi senza pericoli, si usa la normale gestione del movimento orario.

Quando si parla di "\textbf{quadretto}" \index{Quadretto}per indicare una distanza si intende un quadretto di mappa di 1.5 metri x 1.5 metri.

Nel caso di spostamento diagonale\index{Movimento diagonale}\index{Spostarsi di lato} si conta una distanza di 2 quadretti, in caso di arrotondamenti sull'ultimo quadretto si fa per eccesso.

\textbf{Se ci si sposta in terreno "difficile"}, si consumano il doppio di PA per spostarsi di 1.5 metri.

\subsection{Distanza}\index{Distanza}\label{distanza}

Per \textbf{distanza di Mischia} \index{Distanza di Mischia} \index{Mischia}si intende una distanza che permette il combattimento corpo a corpo (1.5 metri attorno al personaggio). Nei mostri questa distanza è indicata dalla portata, per le armi da lancio è chiamata gittata.

Se non indicata nell'avversario/mostro la distanza di mischia/tocco aumenta di 1.5 metri per ogni taglia oltre la media.\index{Taglia e distanza di mischia}
Alcune Armi a due mani hanno una portata di 3 metri.

A distanza di mischia una creatura di dimensioni medie può avere al massimo 8 creature medie.

\end{multicols}

\pagebreak

\subsection{Vita e Morte}\index{Morire}\label{morire}

\begin{changemargin}{0.3cm}{0.3cm}\begin{enfasi}{Chi non conosce la morte, non conosce la vita. (Grand Hotel, film 1932)

\medskip

The worthy GM never purposely kills players' PCs. He presents opportunities for the rash and unthinking players to do that all on their own (Gary Gygax)}\end{enfasi}\end{changemargin}\medskip

\begin{multicols}{2}

Quando un personaggio raggiunge 0 (zero) di Vigore si considera svenuto, ovvero Inabile a fare qualsiasi cosa. Una Cura magica (Incantesimo, Pozione..) lo porterà cosciente ed al valore Vigore curato. Una prova di Pronto Soccorso, 10 PA lo porterà ad 1 di Vigore. Dopo un ora se non è successo qualcosa a mutare la situazione il personaggio può fare prova di Corpo, se riesce torna a 1 di Vigore, se fallisce va a -1 di Vigore e diventa morente.

Un personaggio morente ha Vigore negativa (-1 o meno) ed è svenuto e \hyperlink{morente}{prossimo alla morte}. Continuerà a perdere 1 punto di Vigore a round fiche il valore non raggiungerà -10 e morità, se non viene curato.

Una magia (incantesimo o pozione) di Cura, di qualsiasi potere lo porterà a 1 di Vigore, successive cure funzioneranno normalmente.

Una prova di \hyperlink{prontosoccorso}{Pronto Soccorso} con Svantaggio, 10 PA, porterà il personaggio a 0 di Vigore, ovvero svenuto. 

Una successiva prova di Pronto Soccorso, 10 PA, potrà portarlo a 1 di Vigore ed una cura magica lo curerà dell'ammontare dichiarato.

Un personaggio morente che subisce ulteriore danno, nemici che infieriscono sul corpo od incantesimi diretti a lui od ad area, continua a sottrarre Vigore. 

Le Condizioni \index{Condizioni mentali}di tipo mentale quali Affascinato, Confuso ma non Dominato, terminano quando il personaggio diventa morente.

Quando un personaggio torna a Vigore 1 dopo essere andato a Vigore negativi ha Svantaggio a tutte le Prove finché non riposa una notte.

Un personaggio morto non può beneficiare delle cure normali o magiche. Solo incantesimi molto un potenti possono riportarlo in vita.

\subsubsection{Recupero punti Caratteristica}\index{Recupero punti caratteristica}\label{recuperopunticcaratteristica}

Eventuali punti Caratteristica persi si recuperano al ritmo di 1 punto al giorno, se non indicati come perdita permanente.

\subsubsection{Recupero Vigore naturale}\index{Recupero Vigore naturale}\label{recuperopuntiferitanaturale} 

Riposare 8 ore fa recuperare il punteggio di Corpo in Vigore

\subsubsection{Recupero Vigore non letale}\index{Recupero Vigore non letale}\index{Perdita Vigore non letale}\label{recuperopuntiferitanonletali}\hypertarget{recuperopuntiferitanonletali}{}

Ogni ora si recupera 1 punto Vigore.

\subsection{Tiro per Colpire e Difesa}\index{Tiro per Colpire}\index{Difesa}\label{tiropercolpireedifesa}

\begin{changemargin}{0.3cm}{0.3cm}\begin{enfasi}{Applica sempre la giusta forza, mai troppa mai troppo poca. (Kano Jigoro)}\end{enfasi}\end{changemargin}\medskip

Ogni qual volta una creatura \textbf{decida di Attaccare} deve effettuare una Prova d'Arma d'attacco (PDAA), ovvero somma il suo punteggio di Competenza d'Armi, nell'arma che sta usando, con il modificatore di Corpo e sottrae la somma di 2d10. La differenza è il Margine di Successo (MS) ottenuto.

Ogni qual volta una creatura \textbf{vuole difendersi} deve effettuare una Prova d'Arma di difesa (PDAD), ovvero somma il suo punteggio di Competenza d'Armi, nell'arma che sta usando per difendersi, con il modificatore di Corpo e sottrae la somma di 2d10. La differenza è il Margine di Successo (MS) ottenuto.

L'attaccante confronta il suo MS con quello di chi si difende, se superiore o uguale avrà colpito e causerà i danni al Vigore.

Il linea di massima le Armi piccole causano 1d6 di danno, le Armi medie 1d8, le Armi a due mani 1d10, Rissa causa 1d4, controllate questi valori nell'equipaggiamento. Questi numeri sono da aumentare con il modificatore dato da Corpo.

Al danno causato dall'attacco si sottrae l'eventuale protezione data dall'armatura, il danno rimanente (con un minimo 1) si sottrae ai punti Vigore.

Se i modificatori e circostanze portano il danno inflitto ad essere negativo o zero comunque farai 1 danno a Vigore.

Ci sono situazioni che possono avvantaggiare la difesa quali coperture, nascondigli, come fosse, porte, compagni di taglia molto più grande della propria. Consultate i paragrafi relativi ai \hyperlink{coperture}{Nascondigli e Coperture} per capire il vantaggio che possono dare.

\subsection{Sfortuna e Fortuna nella prova d'Armi per difenderti}

Se nella prova d'Armi di difesa \textbf{tiri due} 1 avrai sicuramente evitato il colpo indipendentemente dal risultato finale ed la tua prossima Prova d'Armi di Attacco avrà Vantaggio.

Se nella prova d'Armi di difesa \textbf{tiri due} 0 sei stato colpito e l'avversario avrà Vantaggio nel danno da applicare.

\subsection{Armi da tiro}\index{Attacchi multipli armi da tiro}\label{armidatiro}\index{Armi da Tiro}\index{Armi da Lancio}

Le armi da tiro sono tutte le armi con una gittata, ovvero che possono essere lanciate o lanciano dei proiettili. Le principali armi da lancio sono gli archi, balestre, fionde ma anche pugnali, giavellotti, lance qualora siano gettate.

Il bonus al danno dato da Corpo si applica in automatico per le fionde, pugnali, giavellotti..ovvero con tutte le armi che vengono lanciate "a mano", gli archi e le balestre non lo applicano mai.

\textbf{I proiettili lanciati da Archi, Fionde, Balestre magiche non si considerano magici.\\
In caso di proiettili magici questi sommano il loro bonus magico al Tiro per Colpire ed al danno}

In ogni arma da tiro è segnata la gittata ovvero entro che distanza è possibile tirare il proiettile senza penalità. Ogni arma da tiro può colpire entro tre volte la gittata indicata.

Se l'obiettivo è entro la gittata indicata non si hanno penalità al colpire, se il target è tra il primo e secondo hai una Penalità alla PDAA di 1, Se il target è tra il secondo è terzo incremento la penalità al colpire è di -2.

Un pugnale tirato entro 6 metri non ha penalità, ma tirato tra i 6 ed i 12 metri ha Penalità -1, a distanza tra 12 e 18 metri ha penalità -2 al PDAA, oltre non può essere tirato.

\subsection{Arma Lunga} \index{Arma Lunga}\label{armalunga}

Alcune armi a due mani hanno l'attributo di Arma lunga. L'arma lunga da diritto a colpire più lontano ovvero a 3 metri. Causa 1 penalità alla prova d'Armi per difendersi. Questo bonus rimane valido finché l'avversario non entra in distanza della propria mischia.

Se l'avversario riduce la distanza a meno di 3 metri o combatte anche lui da tre metri, non ha più la Penalità alla PDAD.

\subsection{Carica} \index{Carica}\label{carica}

l'Azione di carica costa 6 PA più quanto necessario per coprire la distanza.

Chi carica ha Vantaggio nella PDAA ma avrà Svantaggio al PDAD dell'attacco dell'avversario entro la fine del round successivo.

\subsubsection{Preparare una arma lunga/da controcarica contro una carica} \index{Preparare una arma lunga contro una carica}\label{prepararearmalungacontrocarica}

Alcune armi, con l'attributo \textbf{Controcarica}, sono particolarmente efficaci per fermare una carica. E' una Reazione sollevare l'arma per preparare la controcarica. Il danno causato da queste armi è +2 in risposta ad una carica.

\subsubsection{Carica con Arma da Controcarica} \index{Controcarica}\label{caricaarmadacontrocarica}

una Carica effettuata con successo con un arma da controcarica causa 2 danni a Vigore in più.

\subsection{Attaccare con due armi}

Ogni attacco usa i suoi Punti Azione e si gestisce come un normale attacco.

Una delle due armi può essere usato come se fosse uno scudo leggero di metallo per avere un Bonus alla PDAD ma in questo caso non puoi usarlo come arma per attaccare.


\subsection{Attacchi con armi a spargimento} \index{Armi a spargimento}\index{Acqua santa}\index{Olio Incediato}\label{attacchiarmidaspargimento}\hypertarget{spargimento}{}

sono armi a spargimenti quelle che "spargono" il loro contenuto dove cadono, ad esempio olio incendiato/Acqua santa... Una arma a spargimento ha una gittata di 6 metri\index{Lanciare Armi a Spargimento}\index{Gittata armi a spargimento}.

In caso la difesa riesca nella prova d'Armi tira un d8 e consulta questo schemino per capire dove la boccia è caduta:

\medskip

\begin{tabularx}{0.30\textwidth}{ccc}
1& 2& 3\\
4 &\textbf{X}& 5\\
6 &7 &8\\
&\textbf{0}&\\
\end{tabularx}

\smallskip

\textbf{X} si considera il bersaglio dell'oggetto tirato. \textbf{0} il punto di origine del lancio.

Tira 2d4 per determinare lungo la direzione indicata dal d8 precedente a quanti 1.5 metri è caduto distante dal bersaglio, ovvero contate i metri dal target.

Ad esempio con il tiro del d8 faccio 5 e poi tirando 2d6 faccio 4, significa che la boccetta è caduta a destra del bersaglio a 6 metri.

E' anche possibile che ci si sia tirati sui piedi la boccetta (es faccio 7 e poi 6.. potrei averla tirata addosso ad un compagno o dietro di me!).


\subsection{Impreparato -- Colti di Sorpresa}\index{Impreparato}\index{Sorpresa}\label{coltidisorpresa}

se i personaggi vengono colti di sorpresa, ovvero non si aspettano di essere attaccati, si deve considerare questo primo round come round di sorpresa. Quando si è sorpresi si ha Svantaggio alla prova d'Armi per difendersi.

Non potrai reagire, non userai Azioni o Reazioni se non esplicitamente permesse; dal round successivo potrai dichiarare l'iniziativa ed agire normalmente. Le medesime considerazioni valgono per gli avversari.

Per valutare se un personaggio è sorpreso effettuate una prova di Corpo, se la prova riesce allora il personaggio non è sorpreso, altrimenti lo è.

\subsection{Magia in combattimento}\index{Magia in combattimento}\label{magiaincombattimento}

l'incantatore che lancia una magia mentre è in combattimento (ha un avversario in mischia o viene bersagliato da distanza) si considera Distratto.

\subsection{Bonus e Penalità in Combattimento}

l'\textbf{Attaccante} impone 1 Penalità quando alla Prova d'Armi di Difesa (PDAD)

\begin{itemize}

\item ti fiancheggia, è in posizione sopraelevata, ti attacca alla schiena, arma lunga, sei abbagliato, sei intralciato, sei afferrato, combatti con luce fioca, 

\item 
il attaccante impone 2 Penalità quando:
\subitem sei prono, sei ristretto, sei spaventato, ti difendi con un tipo di arma non conosciuta, 

\item 
il attaccante impone Svantaggio quando:
\subitem è invisibile, è in carica, ti ha sorpreso

\end{itemize}

il difensore ha un Bonus nella prova d'Armi per difendersi quando:

\begin{itemize}
	
\item 
ha copertura, combatte da più in alto

\end{itemize}


l'Attaccante ha una Penalità nella prova d'Armi per Attaccare (PDAA) quando:

\begin{itemize}

		\item 
l'attaccante ha Svantaggio se usa un arma improvvisata, usa un arma senza la competenza, usa uno scudo per attaccare.
	
\end{itemize}

Chi ha corso nel round ha 2 Penalità al PDAD ed al PDAA


\subsubsection{Quando colpisci molto bene...}

quando la differenza tra il MS di chi attacca e quello di chi difende è tra i 3 ed i 5 l'attacco causa 1 punto di danno a Vigore in più.

Se la differenza è tra 6 e 8 causa 2 danni in più.

Se la differenza è 9 o più l'attaccante ha Vantaggio nel dado dell'arma, ovvero tira due dadi per il danno l'arma e sceglie il quello che preferisce.

\subsubsection{Quando ti difendi molto bene...}

quando la differenza tra il MS di chi si difende e quello di che attacca è tra i 3 ed i 5 ottieni 1 Bonus alla prima PDAA effettuata entro la fine del round successivo.

Se la differenza è tra 6 o 8 ottieni due Bonus alla prima PDAA effettuata entro la fine del round successivo.

Se la differenza è 9 o più ottieni Vantaggio alla prima PDAA entro la fine del round successivo.

\subsubsection{Aiutare un altro}\index{Aiutare}\label{aiutare}

si può aiutare un compagno ad attaccare o a difendersi negli scontri in mischia, distraendo o interferendo con l'avversario. 

Si esegue una prova d'Armi e se si riesce si concede un bonus alla prova d'Armi per difendersi o come penalità alla prova d'Armi di difesa dell'avversario.

\subsubsection{Tiri Mirati}\index{Tiri Mirati}\label{tirimirati}\index{Mirare a parti specifiche}

Dark Catacomb non prevede la possibilità di effettuare tiri mirati con qualsiasi arma o incantesimo, tranne se questo lo specifica.

Quando si colpisce il bersaglio lo si colpisce genericamente, senza possibilità di specificare se alla testa, gamba o altro, medesimo concetto vale in caso di colpi ad oggetti, es. se miri ad un cardine di una porta colpisci tutta la porta. Questo non impedisce all'Arbitro di valutare conseguenze adeguate.

\subsubsection{Danno non letale}\index{Danno non letale}\label{dannononletale}

il danno non letale è una forma di danno causato da armi particolari o quando volutamente lo scopo è fare svenire il nemico e non ucciderlo.

Il danno non letale si tratta come il danno al Vigore ma va segnato a parte nella scheda.

\subsubsection{Danno non letale con arma non idonea} \index{Danno non letale con arma non idonea}\label{dannononletalearmanonidonea}

se vuoi fare danno non letale con un'arma non predisposta al danno non letale la prova d'Armi attacco (PDAA) ha Svantaggio.

\subsubsection{Senza Competenza}\index{Senza Competenza}\label{senzacompetenza}

usare una tipologia di arma senza l'adeguata competenza, ovvero non avere Armi a due Mani mentre si vuole usare uno Spadone, causa Svantaggio alla prova d'Arma per attaccare.

\subsubsection{Lanciare armi} \index{Lanciare armi}\label{lanciarearmi}

una spada o comunque un arma non fatta per essere lanciata, senza Gittata, può comunque essere scagliata contro l'avversario con Svantaggio.

Il danno a Vigore causato dall'arma viene dimezzato.

\subsubsection{Fiancheggiare} \index{Fiancheggiare}\label{fiancheggiare}

se due personaggi sono attorno allo stesso bersaglio ma non sono a fianco tra loro prendono un Bonus nella prova d'armi d'attacco.

Al massimo ci possono essere 4 personaggi attorno ad una creatura di taglia media che prendono il bonus di fiancheggiare.

Se tirando una ipotetica riga che collega i due personaggi questa attraversa in pieno il quadretto dell'avversario allora c'è la situazione di fiancheggiamento.

\bigskip

Esempio di fiancheggiamento\index{Esempi di Fiancheggiamento}

\medskip

\begin{tabularx}{0.45\textwidth}{lll}
\toprule
A &  G &  D\\
B & \textbf{X}  &  E\\
C &  H &  F\\
\end{tabularx}

\bigskip

In questo schema il fiancheggiamento è preso dalle coppie: A-F, B-E, C-D, G-H

\bigskip

Se la creatura può fronteggiare più creature contemporaneamente queste non godranno del bonus di fiancheggiamento.


\subsubsection{Prendere la Mira (cecchino)} \index{Prendere la Mira (cecchino)}\label{cecchino}

se dedichi 5 PA a prendere la mira ottieni un Bonus alla Prova d'Armi per Attaccare.

\subsubsection{Usare un'arma da lancio mirando ad un avversario impegnato in combattimento} \index{Usare un'arma da lancio mirando ad un avversario impegnato in combattimento}\label{usarearmalancioinmischia}

non è facile prendere la mira corretta e non colpire il proprio compagno. Hai Svantaggio alla Prova d'Armi per Attaccare. Se la prova di attacco ha un MS di -6 o meno hai colpito il compagno.

\subsubsection{Usare un'arma da lancio sotto minaccia} \index{Usare un'arma da lancio sotto minaccia}\label{usarearmalanciosottominaccia}

usare un'arma da lancio come arco, balestra o pugnale (che si vuole lanciare) mentre si è minacciati in mischia concede all'avversario Vantaggio nella prova d'Armi per difendersi.

\subsubsection{Difesa totale} \index{Difesa totale}\label{difesatotale}

costa 8 PA, non puoi eseguire nessun attacco con armi o lancio di incantesimo, guadagni Vantaggio alla prova d'Armi per difenderti.

\subsubsection{Disingaggiare} \index{Disingaggiare}\label{disingaggiare}

costa 2 PA per 1.5 metri che ti sposti. Non causi attacchi di opportunità.\index{Fare un passo}

\subsection{Manovre Opzionali in Combattimento}\label{azioniopzionaliincombattimento}

Queste Azioni di combattimento sono a discrezione dell'Arbitro che può concederle o meno.

Le Prove confrontano il Margine di Successo dei contendenti tra loro per stabilire chi riesce nella manovra.

\subsubsection{Disarmare*}\index{Disarmare}\label{disarmare}

entrambi eseguite una prova d'Armi, chi riesce con il MS maggiore disarma l'avversario. Costa 6 PA

\subsubsection{Finta*} \index{Finta}\label{finta}

entrambi eseguite una prova d'Armi, chi riesce con il MS maggiore ha Vantaggio nella prova d'Armi successiva per difendersi. Costa 6 PA

\subsubsection{Spingere un avversario*} \index{Spingere un avversario}\label{spingereavversario}\hypertarget{spingereavversario}{}

eseguite entrambi una prova di Corpo con un Bonus per ogni taglia di differenza maggiore.

Chi vince la prova con il margine maggiore puo' spingere l'avversario fino a 30 cm punteggio di margine di differenza. Costa 6 PA

\subsubsection{Afferrare un avversario*}\index{Afferrare un avversario}\label{afferrareunavversario}

eseguite entrambi una prova di Corpo, chi ha taglia maggiore ha un Bonus per taglia di differenza.

Costa 6 PA fare e mantenere e liberarsi dalla presa. Si considera che chi afferra è anche afferrato ed abbia almeno una mano occupata nell'afferrare.

Muovere una creatura afferrata richiede \hyperlink{spingereavversario}{Spingere un avversario}.

Ogni contendente può attaccare l'altro afferrato con un Arma piccola o con Rissa.

\subsubsection{Fare cadere un avversario*} \index{Fare cadere un avversari}\label{farecadereavversario}

eseguite entrambi una prova di Corpo. Chi ha più zampe/gambe dell'altro ha un Bonus.

Chi ha il MS maggiore fa cadere prono l'avversario. Costa 6 PA

\subsection{Cavalcature}\index{Combattimento a cavallo}\index{Cavallo}\label{cavalcature}

\begin{changemargin}{0.3cm}{0.3cm}\begin{enfasi}{
E così vidi nella visione i cavalli e i loro cavalieri: questi avevano corazze di fuoco, di giacinto, di zolfo; le teste dei cavalli erano come teste di leoni e dalla loro bocca uscivano fuoco, fumo e zolfo. Da questo triplice flagello, dal fuoco, dal fumo e dallo zolfo che uscivano dalla loro bocca, fu ucciso un terzo dell’umanità. La potenza dei cavalli infatti sta nella loro bocca e nelle loro code, perché le loro code sono simili a serpenti, hanno teste e con esse fanno del male. AP libro 9}\end{enfasi}\end{changemargin}\medskip

Una cavalcatura ha le sue Azioni e di norma sono usate per spostarsi o per reagire ed ubbidire ai tuoi comandi.

Una cavalcatura agisce nel tuo round e sei tu a decidere quando esegue le sue Azioni rispetto alle tue. Non tira l'iniziativa, usa la tua.

Per fare muovere o attaccare una cavalcatura devi usare i tuoi Punti Azione.

Gli attacchi verso un personaggio a cavallo (o cavalcatura in genere) se non dichiarati diversamente mirano al cavaliere e non al cavallo.

Nella descrizione della Cavalcatura è indicato quanti metri fa per PA usato (solitamente 3 o più).

\subsubsection{Situazioni e regole}\label{cavallosituazioniregole}

\begin{itemize}
\item
Ogni qual volta la cavalcatura è colpita il cavaliere deve effettuare una prova di Cavalcare o essere disarcionato dalla cavalcatura.

Se la cavalcatura è da "guerra" (addestrata al combattimento) la prova 2 Bonus.

\item
Combattere da posizione sopraelevata concede una Penalità alla prova d'Armi per difendersi della creatura.

\item
Salire o Scendere dalla cavalcatura costa 4 PA se si ha la competenza Cavalcare, altrimenti 8 PA.

\item
Se una magia o situazione sposta (bruscamente) la cavalcatura contro la tua volontà devi effettuare una prova di Cavalcare o venire disarcionato
\end{itemize}


\subsubsection{Essere disarcionato}\label{esseredisarcionato}

Se vieni disarcionato esegui una prova di Corpo. Se riesci cadi in piedi, se fallisci cadi prono e se il fallimento è di 5 o più subisci 1d6 di danno per la caduta a Vigore.


\subsubsection{Controllare una Cavalcatura}\label{controllocavalcatura}

Mentre sei in sella, hai due scelte:

\begin{itemize}
\item puoi dare ordini alla tua cavalcatura
\item permettergli di agire da sola.
\end{itemize}

Cavalcature particolarmente intelligenti tendono a privilegiare l'autonomia di azione piuttosto che essere comandati.

Puoi controllare una cavalcatura solo se questa è stata addestrata ad accettare un cavaliere. Si presume che cavalli addestrati, muli e simili creature abbiano ricevuto tale addestramento.

L'iniziativa di una cavalcatura controllata cambia per corrispondere a quella di chi la cavalca. Si muove secondo le tue indicazioni e ha solo due opzioni di Azione: Muoversi, Attaccare.

Fare eseguire una Azione ad una cavalcatura costa l'equivalente Azione al cavaliere.

Se la cavalcatura è intelligente avere un cavaliere non restringe le azioni che la cavalcatura può effettuare e questa si muove e agisce come desidera. Potrebbe fuggire dal combattimento, lanciarsi all'attacco e divorare un nemico ferito gravemente, o agire in qualche altro modo contro la tua volontà.

\end{multicols}

\pagebreak

\pagebreak

\section{Nascondigli e coperture} \index{Nascondigli}\index{Copertura}\hypertarget{coperture}{}

\begin{changemargin}{0.3cm}{0.3cm}\begin{enfasi} Dove c'è molta luce, l'ombra è più nera. (Johann Wolfgang von Goethe) \end{enfasi}\end{changemargin}\medskip

\begin{multicols}{2}
	
Non sempre l'avversario si palesa davanti a noi, spesso questo può essere nascosto se non addirittura invisibile.
	
Potrebbe essere nascosto dietro un muretto o dei barili, se non dietro un muscoloso e gigantesco famiglio.

E se fosse alle nostre spalle e neanche l'abbiamo visto ?
	
\subsection{La Copertura}\index{Copertura}\label{copertura}
	
Se l'obiettivo è noto che ci sia ma è occultato in qualche maniera allora si dice che ha "copertura".
	
\begin{itemize}
	\item
Se l'obiettivo ha più della metà (ma non totale) della superficie "visibile" allora la copertura si definisce \textbf{leggera}, ovvero ha 1 Bonus alla PDAD. Può essere il caso di una creatura dietro un altra creatura della medesima taglia o di 1 taglia più grande.
		
Può essere il caso di un arciere in piedi dietro un muretto di 1 metro.
		
\item
Se l'obiettivo ha meno della metà (ma almeno un terzo) della superficie "visibile" allora la copertura si definisce \textbf{media}, ovvero 2 Bonus alla PDAD. Può essere il caso di una creatura dietro un altra creatura di 2 taglie più grande.
		
Può essere il caso di un nemico armato di balestra che si sporge quel tanto per tenere appoggiata la balestra al muretto e sparare (spalle, braccia e testa visibili).
		
\item
Se l'obiettivo si sa dove è ma si nasconde completamente affacciandosi solo per controllare i personaggi o tirare una freccia ogni tanto, dietro ad un muro, finestra, porta, tavolo, una creatura più grande di lui (almeno 3 taglie).. allora la copertura si definisce \textbf{completa}, ovvero ha Vantaggio alla PDAD.

Questo può essere anche il caso di una creatura completamente occultata dalla oscurità, in cui si ipotizza la presenza per suoni, tracce, magie lanciate o proiettili sparati.
		
Chiaramente una creatura che non si sa dove è non può essere colpito normalmente...
		
	\end{itemize}
	
Il Bonus di copertura, ridotta di 1, si applica anche alle Prove di Corpo contro Incantesimi che abbiano un effetto ad area (es. Sfere di Fuoco che esplodano intorno..).
	
\subsection{Invisibilita'}\index{Invisibilità} \hypertarget{invisibilita}{}\label{invisibilita}
	
Se un avversario è invisibile o non si sa dove è si seguono le regole della Invisibilità.
	

Anche se si è invisibili non è detto che non si possa essere percepiti diversamente attraverso altri sensi, come l'olfatto, l'udito o il tatto. 	

Una creatura accecata, che combatte contro una creatura invisibile o che combatte nell'oscurità più completa può effettuare una prova di Osservare. Se riesce con un MS di +6 ha capito dove è la creatura, se entro 6 metri.
	
Se la creatura invisibile ha attaccato in mischia e non si è spostata si considera \textbf{automaticamente trovata}.
	
Chi attacca una creatura per lei \textbf{invisibile ma trovata} ha 3 Penalità al PDAD, la creatura che attacca colui che non la vede ha 3 Bonus al PDAA.
	
\bigskip
	
Se un personaggio invisibile raccoglie un oggetto visibile, l'oggetto resta visibile. Una creatura invisibile può raccogliere un piccolo oggetto visibile e nasconderselo addosso (mettendolo in una tasca o sotto il mantello, chiudendolo nel pugno) e renderlo effettivamente invisibile.
	
Qualcuno potrebbe spargere su un oggetto invisibile della farina per tenere traccia almeno della sua posizione (finché la farina non cade del tutto o viene soffiata via).
	
Le creature invisibili lasciano impronte. Le loro tracce possono essere seguite senza problemi. Impronte su sabbia, fango o altre superfici soffici possono dare ai nemici indicazioni sulla posizione della creatura invisibile rendendola individuata.
	
Una creatura invisibile nell'acqua muove il liquido, rivelando la propria posizione. La creatura invisibile rimane comunque difficile da colpire e gode dei benefici di una copertura leggera (+2 alla Difesa).
	
Una torcia accesa invisibile emana comunque luce (così come un oggetto invisibile soggetto ad una magia di luce).
	
Le creature invisibili non possono utilizzare gli attacchi con lo sguardo. L'invisibilità non influisce sull'essere obiettivo di un incantesimo di Divinazione.
	
\end{multicols}



\pagebreak

\section{Equipaggiamento}\hypertarget{equipaggiamento}{}\label{equipaggiamento}

\begin{multicols}{2}

\subsection{Ricchezza e Denaro}\index{Ricchezza e Denaro}


\begin{changemargin}{0.3cm}{0.3cm}\begin{enfasi}{
«Guai, guai, la grande città,\\
tutta ammantata di lino puro,\\
di porpora e di scarlatto,\\
adorna d’oro,\\
di pietre preziose e di perle!\\
In un’ora sola\\
tanta ricchezza è andata perduta!» (AP libro 18)
}\end{enfasi}\end{changemargin}


	
\label{ricchezza-e-denaro}

In un mondo sull'orlo del collasso poche merci hanno un vero valore e sicuramente il denaro non è tra questi.

I beni che valgono sono quelli che ti permettono di vivere un giorno in più, quelle che possono darti sicurezza, proteggerti o sfamarti.

Certo, ci sono poi le armi della grande guerra. E non parlo delle armi umane ma le reliquie dei combattimenti tra angeli e demoni

Per chi è sopravvissuto sono presentati i pochi attrezzi e strumenti che ancora possono essere trovati e prodotti da una civiltà che è già caduta, morta e sepolta ma non ancora sconfitta.

\subsubsection{Monete e Gemme}\index{Monete e Gemme}

Mi dispiace, ma oro ed argento non valgono più nulla, forse qualche antichissimo pezzo di carta, quelli che una volta chiamavano banconota, ha un valore ma solo storico.

Le gemme sono la vera moneta di scambio se non hai una gallina o una preziosa mucca. Le gemme, specialmente le più preziose possono, a volte, anche salvarti la vita specialmente se usate per corrompere quelle creature dalla pelle rossa, corna allungate e ali ossute.

Le gemme si dividono in base alla loro qualità e valore. nel gradino più basso c'è la \textbf{gemma di bassa qualità} (sigla GR)

Ogni tipologia di gemma ha una sigla per indicarne brevemente la categoria a cui appartiene

\textbf{Gemme di Bassa Qualità, GR}: agata; azzurrite; quarzo blu; ematite; lapislazzuli; malachite; ossidiana; rodocrosite; occhio di tigre; turchese; perla di fiume (irregolare).

\textbf{Gemme Semi Preziosa, GA}: eliotropio, corniola; calcedonio; crisoprasio; citrino; diaspro; lunaria; onice; crisolito; cristallo di roccia (quarzo chiaro); sardonice; quarzo rosato, affumicato o rosa di stella. Una gemma semi preziosa vale circa \textbf{10 gemme di bassa qualità}.

\textbf{Gemme di Media Qualità, GO}: ambra; ametista; crisoberillo; corallo; granato rosso o verde-marrone; giada; perla bianca, dorata, rosa o argentata; spinello rosso, marrone-rosso o verde scuro; tormalina. Ognuna di queste vale circa \textbf{10 gemme semi preziose}.

\textbf{Gemme di Alta Qualità, GP}: alessandrite; acquamarina; granato viola; perla nera; spinello blu scuro; topazio giallo oro.  Ognuna di queste vale circa \textbf{10 gemme di media qualità}.

\textbf{Gemme Preziose, GMP}: opale bianco, nero, o di fuoco; zaffiro blu; corindone giallo fuoco o vermiglio; zaffiro a stella blu o nero. Ognuna di queste vale circa \textbf{10 gemme di alta qualità}.

\textbf{Gemme Eccezionali; GT}: smeraldo, verde brillante cristallino, diamante, rubino, giada pura. Il valore può variare dalle 10 alle 1000 Gemme Preziose.

E non provate a pagare con i minerali creati dagli umani prima della guerra, tipo con lo zircone, sarete considerati alla stregua di falsari.

\end{multicols}

\begin{center}
	\begin{tabular}{llllll}
	
\textbf{Gemma} & \textbf{GR}& \textbf{GA} &\textbf{GO} &\textbf{GP} &\textbf{GMP} \\
	\toprule
Bassa Qualità	& 1		 &1/10	&1/100	&1/1000	&1/10000 \\
Semi Preziose	& 10	 &1		&1/10	&1/100	&1/1000 \\
Media Qualità	& 100	 &10	&1		&1/10	&1/100	 \\
Alta Qualità 	& 1000   &100  	&10		&1	 	&1/10	\\
Preziose	 	& 10000  &1000 	&100  	&10		&1	\\

\end{tabular}

\end{center}
\bigskip

\begin{multicols}{2}

\subsubsection{Ricchezza iniziale}\index{Richezza iniziale}

Solitamente un personaggio appena creato ha 2d6 gemme grezze (GG) come unico suo tesoro.

\subsubsection{Altre Ricchezze - Merci di scambio}\index{Altre Ricchezze}

I mercanti di solito scambiano merci anche senza l'uso di gemme.
Per farsi un'idea delle transazioni commerciali, alcune merci di scambio sono descritte nella tabella. Ricordate che un opale può essere anche bellissimo ma non puoi prepararci del pane per sfamarti.

\medskip

\textbf{Tabella: Esempi altre ricchezze}\index{Tabella Esempi altre ricchezze}

\medskip


\begin{tabular}{ll}
\textbf{Costo} & \textbf{Oggetto}\\
\toprule
5 GR & Frumento (0.5 kg)\\
10 GR & Farina (0.5 kg) o pollo (1)\\
5 GA & Tabacco o rame (0.5 kg)\\
7 GA & Ferro (0.5 kg)\\
1 GO & Cannella (0.5 kg) o capra \\
2 GO & Zenzero o pepe (0.5 kg) o pecora (1)\\
3 GO & Maiale (1) \\
4 GO & Lino (1 m\textsuperscript{2}\\
5 GO & Sale o argento (0.5 kg) \\
1 GP& Seta (1 m) o mucca (1)\\
15 GP& Zafferano(0.5 kg)/bue (1)\\
3 GP&Chiodi di garofano (1kg)\\
\end{tabular}

\medskip

Consultate anche il capitolo sull'Ingombro in Movimento e Trasporto.

\end{multicols}

\section{Equipaggiamento - Armi}\index{Equipaggiamento}\index{Armi}\label{equipaggiamentoarmi}
\hypertarget{equipaggiamento.armi}{}

\label{equipaggiamento---armi}
\begin{changemargin}{0.3cm}{0.3cm}\begin{enfasi}{
E vidi, quando l’Agnello sciolse il primo dei sette sigilli, e udii il primo dei quattro esseri viventi che diceva come con voce di tuono: «Vieni». 

E vidi: ecco, un cavallo bianco. Colui che lo cavalcava aveva un arco; gli fu data una corona ed egli uscì vittorioso per vincere ancora.\\

Quando l’Agnello aprì il secondo sigillo, udii il secondo essere vivente che diceva: «Vieni». 4Allora uscì un altro cavallo, rosso fuoco. A colui che lo cavalcava fu dato potere di togliere la pace dalla terra e di far sì che si sgozzassero a vicenda, e gli fu consegnata una grande spada.\\

Quando l’Agnello aprì il terzo sigillo, udii il terzo essere vivente che diceva: «Vieni». E vidi: ecco, un cavallo nero. Colui che lo cavalcava aveva una bilancia in mano.\\

Quando l’Agnello aprì il quarto sigillo, udii la voce del quarto essere vivente che diceva: «Vieni». E vidi: ecco, un cavallo verde. Colui che lo cavalcava si chiamava Morte e gli inferi lo seguivano. Fu dato loro potere sopra un quarto della terra, per sterminare con la spada, con la fame, con la peste e con le fiere della terra. (AP libro 6)
}\end{enfasi}\end{changemargin}

\begin{multicols}{2}
	

Se per i più lo scopo è sopravvivere molti altri devono imbracciare le armi per poter difendersi e difendere ciò che è a loro caro.

La tabella presenta il nome dell'arma, il suo costo in Gemme di Media Qualità (GO), il danno ed il tipo di danno (se da Taglio, Botta o Punta), la gittata, la tipologia di Arma e le caratteristiche speciali che può avere. Vedi anche \hyperref[sec:capacita-di-carico-e-trasporto-ingombro]{Capacità di Carico e Trasporto.}

Ricordo che usare un'Arma senza l'adeguata competenza da Svantaggio al PDAA.

\end{multicols}

\textbf{Tabella: Lista della Armi}\index{Tabella Lista della Armi}

%\begin{tabularx}{lllll}
\begin{xltabular}{0.99\textwidth}{lllX}
\textbf{Arma}&\textbf{Costo}&\textbf{Dim./Danno} & \textbf{Gittata, Lista, Speciale}\\
\toprule
Alabarda& 10 & G/1d10 P/T& \textbf{Armi a due mani} Controcarica, Arma lunga \\
Arco Corto& 30 & M/1d6 P& 15 metri, \textbf{Armi da tiro}\\
Arco Lungo& 75 & G/Frecce& 20 metri, \textbf{Armi da tiro}\\
Ascia Martello& 16 & M/1d6 T/B& \textbf{Armi medie}\\
Ascia ad una mano& 6  & M/1d6 T& 6 metri, \textbf{Armi piccole}\\
Ascia da battaglia& 10 & M/1d10 T&\textbf{Armi a due mani}\\
Balestra ad una mano& 200& M/Dardi& 6 metri, \textbf{Armi da tiro}\\
Balestra pesante& 150 & G/Dardi& 30 metri \textbf{Armi da tiro}\\
Bastone& 3& M/1d6 B& \textbf{Armi medie}, Arma lunga\\
Brandistocco& 10 & M/2d4 P/T& \textbf{Armi a due mani}, Controcarica, Arma lunga\\
Catena chiodata& 55 & G/2d4 P& 3 metri, \textbf{Armi medie}, Arma lunga\\
Falce& 18 & G/2d4 P/T& \textbf{Armi a due mani}, Arma lunga\\
Falcetto& 6& P/1d6 T& \textbf{Armi piccole}\\
Falcione in asta& 12 & G/1d10 P/T& \textbf{Armi a due mani}, Controcarica, Arma lunga\\
Fionda& -& P/1d4 B& 10 metri, \textbf{Armi da tiro}\\
Flagello Pesante& 20 & M/1d10 B& \textbf{Armi a due mani}\\
Flagello& 8& M/1d8 B& \textbf{Armi medie}\\
Frusta& 1& M/1d3 T& \textbf{Armi medie}, Arma lunga\\
Giavellotto& 1& P/1d6P& 12 metri,  \textbf{Armi piccole} \textbf{Armi da tiro}\\
Grosso randello& 2& M/1d8 B&\textbf{Armi a due mani}\\
Lancia da fante& 5& M/1d8 P&3 metri, \textbf{Armi medie}, Arma lunga, Controcarica\\
Lancia& 10 & G/1d8 P&\textbf{Arma a due mani}, Arma lunga, Controcarica\\
Maglio da guerra& 15& G/1d10 B& \textbf{Armi a due mani}\\
Manganello& 1& P/1d6 B& \textbf{Armi piccole}, non letale\\
Martello da guerra& 8& M/1d8 B/P& 6 metri, \textbf{Armi medie}\\
Mazza Pesante& 5& M/1d8 B/T& \textbf{Armi medie}\\
Mazza chiodata& 10& M 1d8 B/P& \textbf{Armi medie}\\
Picca Leggera& 4& M/1d4 P&\textbf{Armi semplici}\\
Picca Pesante& 8& G/1d6 P&\textbf{Armi a due mani}, Arma lunga\\
Pugnale& 2& P/1d4 P& 6 metri, \textbf{Armi piccole}, \textbf{Armi da tiro}\\
Rissa& note*& P/1d4 B&\\
Randello& 1& P/1d6 B& \textbf{Armi piccole}\\
Scimitarra& 30 & M/1d6 T&\textbf{Armi piccole}, \textbf{Armi medie}\\
Spada Corta& 10 & P/1d6 P&\textbf{Armi piccole}\\
Spada Lunga& 20 & M/1d8 T&\textbf{Armi medie}\\
Spada Bastarda& 35 & M/1d8 T&\textbf{Armi medie}, \textbf{Armi armi a due mani}\\
Spadone a due mani& 50 & G/2d6 T&\textbf{Armi a due mani}\\
Stocco& 40 & P/1d6 P& \textbf{Armi piccole}\\
Tridente& 15 & M/1d6 P/T& 3 metri, \textbf{Armi da tiro}, \textbf{Armi medie}, \textbf{Armi a due mani}, Arma Lunga, Controcarica\\
\end{xltabular}

\medskip

Un \textbf{Arma} Piccola ha \textbf{Ingombro} 1, una Arma Media ha Ingombro 2, un Arma Grande ha Ingombro 4, un Arma Enorme ha Ingombro 8.\index{Ingombro Armi}\index{Ingombro Armi}

\medskip

\textbf{Tabella: Lista dei proiettili - Archi - Armi da tiro - Fionde}\index{Tabella Lista dei proiettili - Archi - Armi da tiro - Fionde}

\begin{tabular}{lcc}
\textbf{Nome Proiettile}& \textbf{Numero di colpi/Costo (mo)} & \textbf{Danno / Tipo}\\
\toprule
Dardi per balestra & 6/1 GA & 1d6 P\\
Frecce per arco& 20/1 GA & 1d6 P\\
Biglie di Marmo (fionde)& 15/1 GA & 1d4 B\\
Sasso (fionde)& -& 1d3 B\\
\end{tabular}

\medskip

Una \textbf{Faretra} (piena o vuota) di Proiettili ha \textbf{Ingombro} 2.\index{Ingombro Proiettili}\\

\begin{multicols}{2}
	
\subsubsection{Armi magiche}	

Solo le reliquie o le armi da loro derivate possono essere considerate magiche.

Il Bonus magico indicato nella armi si applica alla Prova d'Armi per attaccare (PDAA) e si applica al danno inflitto a Vigore.

Non è possibile acquistare armi magiche devono essere "trovate".

\textbf{Un proiettile non acquisisce proprietà magiche perché il suo lanciatore è magico.}

\bigskip



\textbf{Balestra}\index{Balestre}\index{Ricarica Balestra}
Una balestra richiede 4 PA per essere ricaricata. Una balestra leggera od a una mano richiede 2 PA per essere ricaricata.

\textbf{Gittata}\index{Gittata}
La distanza indicata è quella senza penalità alla prova d'arme per colpire. Ogni arma a distanza può colpire entro tre volte la distanza indicata.

Se l'obiettivo è entro la gittata indicata non si hanno penalità al colpire, se il target è tra il primo e secondo hai una Penalità alla PDAA di 1, Se il target è tra il secondo è terzo incremento la penalità al colpire è di -2.

Un pugnale tirato entro 6 metri non ha penalità, ma tirato tra i 6 ed i 12 metri ha Penalità -1, a distanza tra 12 e 18 metri ha penalità -2 al PDAA, oltre non può essere tirato.

Un giavellotto tirato entro 12 metri non ha penalità, ma tirato entro 24 metri ha un 1 di Penalità al PDAA, a distanza tra 24 e 36 metri un -2 al PDAA, oltre non può essere tirato.

Una \textbf{Freccia o Dardo che colpisce si considera distrutta}, se manca si considera che abbia un 50\% (4-5-6 su un d6) di probabilità che sia ancora integra.

Una Freccia/Dardo/Sasso magico somma i suoi bonus a quelli del lanciatore per determinare il PDAA ed il Danno.

\medskip

Un arma media se usata a due mani causa +2 al danno a Vigore.


Le Armi hanno indicato una \textbf{Tipologia di danno}\index{Tipologia di danno}, ovvero T/B/P.
Queste lettere stanno ad indicare se il danno è di tipo Taglio, Botta o da Perforazione. Questa caratteristica può essere importante perché determinate creature possono essere immuni o subire meno danno da un particolare tipo di ferita (es uno scheletro contro un'arma da penetrazione o un cubo gelatinoso contro un arma da taglio..).

Un arma può essere usata per causare un tipo di danno diverso (da taglio a perforazione o botta) riducendo di una categoria il dado di danno (es. Spada Lunga per fare danno da botta causa 1d6).

\medskip

\textbf{Armi Improvvisate}\index{Armi Improvvisate}

Talvolta oggetti che non sono stati creati per essere armi possono avere una certa efficacia in combattimento. Dal momento che non si tratta di oggetti pensati per questo utilizzo, la creatura che attacca con uno di essi subisce 2 Penalità al PDAA. Un'arma improvvisata di piccole dimensioni (bottiglia) fa 1d3 di danno, di medie dimensioni (la gamba di una sedia) da 1d6, di grandi dimensioni (la gamba di un tavolo) fa 1d8 di danno.

Un'arma da lancio improvvisata ha una gittata 3 metri.

\medskip

\textbf{Lanciare armi}\index{Lanciare armi}

Una spada o comunque un arma non fatta per essere lanciata può comunque essere scagliata contro l'avversario. Il PDAA ha 2 Penalità e l'arma fa una categoria di danno inferiore (la spada lunga fa 1d6, una spada corta 1d4..). La gittata è 3 metri.

\medskip

\textbf{Usare un'Arma senza l'adeguata competenza comporta Svantaggio nel PDAA}.

\subsubsection{Le armi antiche}

E' possibile trovare ancora delle armi antiche funzionanti, armi che dopo 400 anni ancora possono essere usate.

La maggior parte delle armi da fuoco dopo un lasso di tempo così lungo richiedono pezzi di ricambio ed una continua manutenzione.

Le armi che potrete trovare funzionanti sono i revolver, gli shotgun, i fucili semi automatici ed i fucili automatici.

I revolver, a differenza delle pistole \textit{moderne} hanno meno pezzi che si possono rompere, non hanno pezzi di plastica o gomma che si sbriciolano e richiedono meno manutenzione per funzionare.

\textbf{Revolver}:\\
\textbf{Punti Azione}: 3 per un singolo colpo sparato\\
\textbf{Caricatore}: 6 proiettili\\
\textbf{Gittata}: 9 metri\\
\textbf{Danno}: 1d10 a proiettile danni a Vigore\\
\textbf{Regole}: e' necessario una PDAA di Armi da Tiro per proiettile usato. Chi si difende deve fare una Prova di Corpo per colpo, se il MS è superiore l'avversario ha mancato il bersaglio.\\

\textbf{Shotgun}:\\
\textbf{Punti Azione}: 4 per un singolo colpo sparato\\
\textbf{Caricatore}: 4 proiettili\\
\textbf{Gittata}: 6 metri\\
\textbf{Danno}: 2d8 a proiettile danni a Vigore\\
\textbf{Regole}: e' necessario una PDAA di Armi da Tiro per proiettile usato. Chi si difende deve fare una Prova di Corpo per colpo, se il MS è superiore l'avversario ha mancato il bersaglio.\\

\textbf{Fucile Semi-automatico}:\\
\textbf{Punti Azione}: 4 per un 3 colpi sparati\\
\textbf{Caricatore}: 21 proiettili\\
\textbf{Gittata}: 18 metri\\
\textbf{Danno}: 1d8 a proiettile danni a Vigore\\
\textbf{Regole}: e' necessario una PDAA di Armi da Tiro per gruppo di 3 proiettili usati. Chi si difende deve fare una Prova di Corpo per gruppo di proiettili, se il MS è superiore l'avversario ha mancato il bersaglio con tutti e 3 i colpi.\\


\textbf{Fucile Automatico}:\\
\textbf{Punti Azione}: 1 per un 1 colpo sparato\\
\textbf{Caricatore}: 30 proiettili\\
\textbf{Gittata}: 12 metri\\
\textbf{Danno}: 1d6 a proiettile danni a Vigore\\
\textbf{Regole}: si esegue un unico PDAA in Armi da Tiro indipendentemente dal numero di proiettili sparati. Chi si difende deve fare una Prova di Corpo. Va a segno un proiettile per MS di chi attacca.

\bigskip

Ogni arma usa dei proiettili diversi. Non puoi utilizzare i proiettili del revolver su un fucile semi automatico o quelli del fucile automatico su uno shotgun o fucile semi automatico.

L'armatura e scudo funzionano anche contro i colpi delle armi da fuoco.

\begin{center}
	\textbf{Proiettili}
\end{center}

I proiettili sono la cosa in assoluto più difficile da trovarsi. Nessun proiettile veniva costruito con l'idea di essere sparato 400 anni dopo la sua creazione.

La polvere da sparo si è inumidita, ha perso la carica, la camica di metallo si è corrosa con il tempo.. ci sono tantissimo fattori che rendono i proiettili estremamente rari, quasi e più delle armi magiche.

\end{multicols}

\pagebreak

\section{Equipaggiamento - Armature e Scudi} \index{Armature}\index{Scudi}\hypertarget{equipaggiamento.armature.scudi}{}\label{equipaggiamentoarmature}

\label{equipaggiamento---armature-e-scudi}

\begin{changemargin}{0.3cm}{0.3cm}\begin{enfasi}{
E così vidi nella visione i cavalli e i loro cavalieri: questi avevano corazze di fuoco, di giacinto, di zolfo. (AP libro 9)} \end{enfasi}\end{changemargin}\medskip

Le armature aiutano ad assorbire il danno dei colpi e penalizzano la Prova di Incantamento e le Prove di competenza. A seconda dell'armatura può essere richiesto un valore di Corpo minimo.

Le Penalità Competenze è la penalità che si applica alle prove di competenza influenzate dal peso ed Ingombro dell'armatura. Armature diverse, specifiche o magiche hanno punteggio diversi, questa tabella serve come linea guida per l'Arbitro.\index{Penalità armatura}

\subsubsection{Tabella Armature}\index{Tabella Armature}

\label{tabella-armature}
\begin{tabular}{llllll}
%\begin{xltabular}{0.95\textwidth}{lXXXXXXX}
\textbf{Armatura} & \textbf{Costo (GO)} & \textbf{Riduzione} & \textbf{Penalità} & \textbf{Corpo} &\textbf{Ingombro}\\
\toprule		%GG		rid pen  corpo	ingombro
Imbottita  		& 5		& 1		& 0	 & 4  &	2\\
Cuoio 			& 10	& 1d2 	& 1	 & 5  &	2\\
Cuoio rinforzato& 40	& 1d3 	& 1  & 6  & 2\\
Giaco di Maglia & 100 	& 1d4 	& 2  & 9  & 4\\
Scaglie			& 150 	& 1d6 	& 2  & 10 & 4\\
Anelli 			& 250  	& 1d6+1 & 2+1  & 12 & 6\\
Pettorale  		& 400  	& 1d8 	& 2+1  & 13 & 7\\
Mezza armatura  & 800 	& 1d10 	& 3  & 14 & 10\\
Completa		& 1200 	& 1d10+1& 3  & 15 & 10\\
\end{tabular}

\begin{multicols}{2}

\textbf{Costo}: è per un armatura di taglia media espresso in gemme grezze.

\textbf{Riduzione}: è di quando il danno a Vigore viene ridotto. Il giocatore tira il dado segnato e diminuisce il danno di quell'ammontare con un \textbf{minimo danno subito di 1}.

\textbf{Penalità Comp.}: è la Penalità alle Prove di Competenza dato l'ingombro e peso dell'armatura. Quando il valore è segnato +1 significa, per esempio 2+1, che ha 1d4 di penalità ed un ulteriore 1 di Penalità. Quanto le Penalità sono 3 hai Svantaggio alle Prove di Competenza.

\textbf{Corpo}: è il requisito minimo di punteggio di Corpo ridurre le penalità. Con un punteggio inferiore muoversi costa 1 PA in più 1.5 metri per differenza tra il punteggio di Corpo e quello necessario. Con un punteggio pari o superiore a quello indicato le Penalità alle Competenze si riducono di 1.

\medskip

\begin{changemargin}{0.3cm}{0.3cm}\begin{narratore} Quando conteggiate l'Ingombro dato dall'armatura e scudo \textbf{indossato} dovete dividerlo per due.

L'Ingombro di armatura e scudi è da intendersi quando è "caricata nello zaino", ovvero trasportata ma non indossata.\end{narratore}\end{changemargin}

\subsubsection{Descrizione delle Armature}


\textit{Imbottita}. Le armature imbottite consistono di strati di tessuto e imbottitura cuciti insieme.

\textit{Cuoio}. Il corpetto e le protezioni delle spalle di questa armatura sono fatte di cuoio indurito dopo essere stato bollito nell'olio. Il resto dell'armatura è composto di
materiali più morbidi e flessibili.

\textit{Cuoio Rinforzato}. Fatta di cuoio duro ma flessibile, l'armatura di cuoio rinforzato è arricchita da rivetti o spuntoni.

\textit{Giaco di Maglia}. Composto di anelli metallici intrecciati tra di loro, un giaco di maglia viene indossato sopra strati di abiti o cuoio. Questo tipo di armatura offre una protezione modesta alla parte superiore del corpo, mentre il rumore degli anelli che strusciano fra di loro viene attutito dagli altri strati.

\textit{Scaglie}. Quest'armatura consiste in una cotta e gambali (a volte anche di una gonna separata) di cuoio coperti da pezzi di metallo sovrapposti, in maniera simile alle scaglie di un pesce. L'armatura è completa di guanti.

\textit{Anelli}. Quest'armatura è un'armatura di cuoio con dei pesanti anelli cuciti sopra. Gli anelli servono a rinforzare l'armatura contro i colpi di spada e d'ascia. L'armatura è completa di guanti.

\textit{Pettorale}. Questa armatura consiste di un corpetto di metallo indossato su di uno strato di cuoio. Sebbene lasci braccia e gambe relativamente scoperte, l'armatura fornisce una buona protezione agli organi vitali del personaggio, senza procurargli grande ingombro.

\textit{Mezza Armatura}. La mezza armatura di piastre consiste di piastre di metallo sagomate che coprono gran parte del corpo del personaggio. Non comprende protezioni per le gambe oltre a dei semplici schinieri legati con lacci di cuoio.

\textit{Completa}. Quest'armatura consiste di piastre di metallo sagomate a incastro che coprono l'intero corpo. Un'armatura di piastre comprende guanti, stivali di cuoio pesanti, un elmo con visiera, e uno spesso strato di imbottitura sotto l'armatura. Fibbie e lacci distribuiscono il peso dell'armatura su tutto il corpo.


\subsubsection{Regole base per l'utilizzo dell'armatura}

\textbf{Dormire in Armatura}: se si dorme in un'armatura media o pesante, il giorno seguente si è automaticamente \hyperlink{affaticato}{Affaticati}.

Dormire in un'armatura con ingombro 2 o meno non provoca Affaticamento.

\textbf{Peso}: il peso indicato si riferisce alla versione per personaggi di taglia Media. Le armature adattate per personaggi di taglia Piccola pesano la metà, mentre per quelli di taglia Grande pesano il doppio.

\textbf{Armature magiche}\index{Armature magiche}\index{Scudi magici}

Un armatura magica o scudo magico non solo protegge meglio ma è anche più leggera e affine alla magia. Una armatura magica ha un più alto valore di Riduzione e peso ed ingombro minore.

\subsubsection{Gli Scudi}

Gli \textbf{Scudi} \index{Scudi}permettono di aumentare la prova di PDAD. 

Gli Scudi possono essere di tipo Leggero, Medio, Pesante.

\end{multicols}

\subsubsection{Tabella Scudi}\index{Tabella Scudi}

\label{tabella-scudi}

\begin{center}
	\begin{tabular}{llll}
\textbf{Scudi} & \textbf{Costo} & \textbf{PDAD} & \textbf{Tipo}\\
\toprule
Scudo leggero di legno 		& 3 GO   &  1	& L\\
Scudo leggero di metallo 	& 20 GO  &  1 	& L\\
Scudo medio legno			& 10 GO  &  2	& M\\
Scudo medio metallo 		& 30 GO  &  2  	& M\\
Scudo pesante di legno 		& 30 GO  &  2+1 & P\\
Scudo pesante di metallo	& 80 GO  &  2+1 & P\\
\end{tabular}

\end{center}

\begin{multicols}{2}

\medskip

\textbf{PDAD}: è il Bonus che si aggiunge alla prova di PDAD. Come per le Armature quando è indicato 2+1 significa avere +1d4 ed un ulteriore +1 alla prova.

\textbf{Tipo}: indica la taglia dello scudo. \textbf{L}eggero, \textbf{M}edio, \textbf{P}esante.

Uno \textbf{Scudo} Leggero ha \textbf{Ingombro} 1, uno Scudo Medio ha Ingombro 2, uno Scudo Pesante ha Ingombro 4.\index{Imgombro per Scudi}

Il Bonus a PDAD si sottrae alla Prove di Competenza influenzate dall'armatura.

Uno scudo può essere usato come \textbf{arma improvvisata}. Il PDAA ha Svantaggio. Uno scudo piccolo fa 2 di danno (B/T), uno scudo medio fa 1d4 di danno (B/T), uno scudo pesante fa 1d6 di danno (B/T).

Imbracciare uno scudo occupa una mano/braccio.

\subsubsection{Indossare e Togliere Armature}\index{Indossare e Togliere Armature}

Indossare e togliere armature è una operazione che richiede tempo ed attenzione, farlo in fretta non aiuta ed anzi tende a peggiorare la protezione data dall'armatura.

\end{multicols}

\textbf{Tabella: Tempi per indossare e togliere l'armatura}\index{Tabella Tempi per indossare e togliere l'armatura}

\begin{tabular}{llll}
\textbf{Tipo di Armatura}& \textbf{Indossare} & \textbf{Indossare in fretta} & \textbf{Togliere}\\
\toprule
Scudo& 1 azione & - & 1 azione\\
Imbottita, Cuoio, Cuoio rinforzato  & 1 minuto& 3 round  & - \\
Giaco di Maglia& 1 minuto& 5 round  & 5 round\\
Scaglie, Anelli, Pettorale& 4 minuti & 1 minuto{*}  & 1 minuto\\
Completa  & 4 minuti{*}{*}& 4 minuti{*}& 1d4+1 minuti\\
\end{tabular}

\bigskip

\begin{multicols}{2}

{*} Se qualcuno aiuta, il tempo si dimezza. Un singolo personaggio che non sta facendo altro può aiutare uno o due personaggi adiacenti a lui. Due personaggi non possono aiutarsi l'un l'altro a indossare un'armatura contemporaneamente.

{*}{*} Bisogna essere aiutati per indossare questa armatura. Senza aiuto è possibile indossarla solo in fretta.

\textbf{Indossare un'armatura in fretta} implica penalità di -2 alla Riduzione.


\end{multicols}

\pagebreak

\section{Merci e Servizi}\index{Merci}\index{Servizi}


\subsection{Ricchezza e Denaro}\index{Ricchezza e Denaro}


\begin{changemargin}{0.3cm}{0.3cm}\begin{enfasi}{
Anche i mercanti della terra piangono e si lamentano su di essa, perché nessuno compera più le loro merci: i loro carichi d’oro, d’argento e di pietre preziose, di perle, di lino, di porpora, di seta e di scarlatto; legni profumati di ogni specie, oggetti d’avorio, di legno, di bronzo, di ferro, di marmo; cinnamomo, amomo, profumi, unguento, incenso, vino, olio, fior di farina, frumento, bestiame, greggi, cavalli, carri, schiavi e vite umane. (AP libro 18)
}
\end{enfasi}\end{changemargin}\medskip

\begin{multicols}{2}

\subsubsection{Vendere Tesori}

Nei sotterranei che esplorerai avrai qualche opportunità di trovare tesori, equipaggiamento, armi, armature e altro ancora. Di solito, potrai vendere tesori e ninnoli quando raggiungerai un paese o altro insediamento, purché tu riesca a trovare acquirenti e mercanti interessati al tuo bottino.

\medskip

\textbf{Armi, Armature e Altro Equipaggiamento }

Come regola generale, le armi, le armature e il resto dell’equipaggiamento non danneggiato quando viene venduto costa la metà. È difficile che le armi e le armature utilizzate dai mostri siano in condizioni ottimali per la vendita. Tranne se reliquie.

\medskip

\textbf{Oggetti Magici}

La vendita di oggetti magici è un problema. Trovare qualcuno che voglia comprare una pozione o pergamena non comporta grandi difficoltà, ma la maggior parte degli oggetti sono fuori della portata delle tasche di chiunque salvo dei nobili più ricchi. Il valore della magia trasale la vile gemms e dovrebbe essere sempre trattato con riguardo.

\medskip

\textbf{Merci}

Sulle terre di confine, la maggior parte delle transazioni avvengono tramite baratto. Come le gemme e gli oggetti d’arte, le merci - lingotti di ferro, sacchi di sale, bestiame e così via - possono essere scambiate come gemme grezze al loro pieno valore.


\subsubsection{Equipaggiamento da Avventura}\index{Equipaggiamento avventura}\index{Cose da comprare}

Questo è un breve e non esaustivo elenco di equipaggiamento che i vostri personaggi potrebbero essere interessati a comprare. L'elenco non è certo esaustivo o completo ma potrà fornirvi linee guida sui prezzi.

Come Arbitro usate sempre il buon senso nelle richieste valutate bene la tipologia di richiesta, la necessità dell'oggetto, il luogo dove si compra e come lo si compra.

In base alla tipologia di compagna potrebbero essere disponibili ulteriori oggetti quali armi da fuoco o alchemici.

\medskip

{\small
\begin{tabularx}{0.42\textwidth}{lll}
\textbf{Oggetto}    & \textbf{Costo} & \textbf{Ing.}\\
Abaco&2 GO&L\\
Abito da Monaco & 5 GO& 1\\
Abito da artigiano& 1 GO& 1\\
Abito da contadino& 1 GA& 1\\
Abito da esploratore  & 10 GO& 1\\
Abito da intrattenitore & 3 GO& 1\\
Abito da ricco & 75 GO  & 2\\
Abito da studioso & 5 GO& 1\\
Abito da viaggiatore  & 1 GO& 2\\
Abito invernale & 8 GO& 2\\
Acciarino e pietra focaia & 1 GO&\\
Acido Intenso (ampolla) & 10 GO  & L \\
Acqua santa (ampolla) & 25 GO& L\\
Ago da cucito & 5 GA &- \\
Amo da pesca  & 1 GA& - \\
Ampolla (vuota)& 3 GR& L \\
Anello con sigillo  & 5 GO& - \\
Anello per veleno & +20 GO&-\\
Antitossina (boccetta)  & 50 GO  & L\\
Ariete portatile  & 10 GO& 3 \\
Arnesi da artigiano& 5 GO& 2\\
Arnesi da scasso  & 30 GO& 1\\
Asta (3 m)  & 5 GR& 2\\
Attrezzi da scalatore & 80 GO& 1\\
Bandoliera & 3 GO & L\\
Barca a remi &  50 GO  & 12\\
Barcone & 3000 GO  & -\\
Barile (vuoto)& 2 GO& 4\\
Bastone & 2 GO& 1\\
Bilancia da mercante  & 2 GO& 1\\
Birra Boccale& 5 GR& L\\
Birra Caraffa& 2 GA& L\\
Boccale di ceramica & 2 GR& L\\
Boccetta di inchiostro o pozione  & 1 GO& L \\
Borsa&5 GA&L\\
Borsa da cintura (vuota) & 1 GO& L\\
Borsa per Componenti &25 GO&L\\
Borsa del guaritore& 50 GO  & 1\\
Bottiglia di vetro  & 2 GO& L \\
Brocca di ceramica  (5lt) & 2 GR& L\\
Campanella  & 1 GO& - \\
Candela & 1 GR& -\\
Canna da pesca & 1 GO&1\\
Cannocchiale  & 1000 GO  & 1 \\
Caraffa di ceramica & 2 GR& L\\
Carne (1 pezzo) & 3 GA& L\\
Carretto  & 15 GO  & 10\\
Carro & 35 GO& 20\\
Carrozza  & 300 GO & -\\
Carrucola e paranco & 20 GO& 2 \\
Carta (foglio)& 4 GA& -\\
Cassa (vuota) & 2 GO& 3 \\
Catena (3 m)  & 30 GO & 1\\
Ceralacca& 1 GO& -\\
\end{tabularx}

\begin{tabularx}{0.42\textwidth}{lll}
\textbf{Oggetto}    & \textbf{Costo} & \textbf{Ingombro}\\
Cerata&5 GA&1\\
Cesto (vuoto) & 4 GA& 1 \\
Chiodo da rocciatore& 1 GA&L\\
Clessidra& 25 GO  & -\\
Coperta invernale & 5 GA& 1 \\
Corda di canapa (15 m)& 1 GO& 1\\
Corda di canapa grossa (15 m)& 2 GO& 2 \\
Cote per affilare & 2 GR& L \\
Custodia per Dardi o Frecce  & 1 GO& 1 \\
Custodia per mappe o pergamene  & 1 GO& 1 \\
Fischietto  & 8 GA& - \\
Formaggio (1 pezzo)& 1 GA& \\
Forziere & 5 GO&4\\
Fuoco dell'alchimista (ampolla)& 20 GO& L\\
Galea & 30k GO  & -\\
Gancio di metallo & 1 GO& L\\
Gessetto, (1 pezzo) & 1 GR& -  \\
Giaciglio& 1 GA& 1 \\
Inchiostro (boccetta da 30 g)& 8 GO& - \\
Lanterna comune& 1 GO& 2 \\
Lanterna a lente sporgente  & 12 GO  & 1 \\
Lanterna schermabile& 7 GO& 1 \\
Legna da ardere (per giorno)& 1 GR& 4 \\
Lente d'ingrandimento & 100 GO & -\\
Locanda Buona (dormire) & 2 GO& -\\
Locanda Normale (dormire)& 5 GA& -\\
Locanda Scadente (dormire) & 2 GA& -\\
Maglio& 1 GO& 2 \\
Manette & 15 GO  & L \\
Martello& 5 GA& 1  \\
Morso e briglie & 2 GO&1\\
Nave a vela & 10k GO & -\\
Nave da guerra  & 25k GO  & -\\
Nave lunga  & 10k GO & -\\
Olio da lanterna& 1 GA& 1 \\
Orologio ad acqua & 1000 GO & -\\
Otre  & 1 GO& 2 \\
Pala o badile & 2 GO& 1 \\
Pane (a pagnotta) & 2 GR& -\\
Pasti (al giorno) Buono & 5 GA&-\\
Pasti (al giorno) Normale& 3 GA&-\\
Pasti (al giorno) Scadente  & 1 GA&-\\
Pennino & 1 GA& - \\
Pentola di ferro  & 1 GO& 1 \\
Pergamena (Foglio)  & 2 GA& - \\
Piccone da minatore & 3 GO& 2 \\
Piede di porco& 2 GO& 1 \\
Pozione di Cura & 50 GO & L\\
Pozione di Cura potenziata & 125 GO & L\\
Profumo & 5 GO & L\\
Rampino & 1 GO& 1 \\
Razioni da viaggio (al giorno)  & 5 GA& 1 \\
Remo  & 2 GO& 2\\
Rete da pesca (2,25 m)& 4 GO& 1 \\

\end{tabularx}

\begin{tabularx}{0.42\textwidth}{lll}
Sacche da sella & 4 GO& 2\\
Sacco (vuoto) & 1 GA& L \\
Sapone (per 0,5 kg) & 5 GA& - \\
Scala a pioli (3 m) & 2 GA& 3 \\
Secchio (vuoto)& 5 GA& L\\
Sella Da galoppo  & 30 GO& 2\\
Sella Militare  & 50 GO  & 3\\
Sella da carico & 15 GO  & 2\\
Serratura/lucchetto Buona & 80 GO  & -\\
Serratura/lucchetto Media & 40 GO&  \\
Serratura/lucchetto Semplice& 20 GO  & -\\
Serratura/lucchetto Superiore& 150 GO  & - \\
Sfere Metalliche (100) & 10 GO & 1\\
Simbolo religioso di legno  & 1 GA& L\\
Slitta& 20 GO  & 3 \\
Specchio piccolo di metallo & 10 GO  & L\\
Stallaggio (al giorno)  & 5 GA& -\\
Strumento musicale comune& 5 GO& 2\\
Tagliola& 5GO&3\\
Tela (per mq)& 1 GA& L \\
Tenda & 10 GO  & 3 \\
Torcia& 1 GA& 1\\
Tribolo (20) & 1 GA& L \\
Trucchi per il camuffamento & 50 GO& L\\
Vanga o Badile & 1 GO&1\\
Veste da Devoto & 5 GO& 1\\
Vino Buono (bottiglia) & 10 GO& 1\\
Vino Comune (caraffa)  & 2 GA& 1\\
Zaino & 2 GO& 1 \\
\end{tabularx}}


\begin{changemargin}{0.3cm}{0.3cm}\begin{enfasi}{
Tu dici: Sono ricco, mi sono arricchito, non ho bisogno di nulla. Ma non sai di essere un infelice, un miserabile, un povero, cieco e nudo (AP libro 3)
}\end{enfasi}\end{changemargin}

\textbf{Acido Intenso}. Con un’azione, puoi spargere il contenuto di questa fiala su di una creatura entro 1 metri da te o lanciare la fiala fino a 6 metri, fracassandola all’impatto. In entrambi i casi, effettua un Tiro per Colpire a distanza contro la creatura o l’oggetto, trattando l’acido come un’arma improvvisata (Svantaggio al PDAA). Se colpisci, il bersaglio subisce 2d6 danni da acido.

\textbf{Acqua santa}. Con un’azione, puoi spargere il contenuto di questa ampolla su di una creatura entro 1 metri da te o lanciare l’ampolla fino a 6 metri, fracassandola all’impatto. In entrambi i casi, effettua un Tiro per Colpire a distanza contro la creatura o l’oggetto, trattando l’Acqua santa come un’arma improvvisata. Se colpisci, e il bersaglio è un immondo o un non morto, subisce 2d4 danni da energia positiva.

\textbf{Ampolla (vuota)}: piccola anfora in vetro o ceramica con collo sottile.

\textbf{Anello con Sigillo:} cerchietto di metallo, generalmente pregiato, con un'incisione atta ad imprimere sigilli su ceralacca.

\textbf{Anello per Veleno:} +20 GO, rispetto a costo anello, questo anello ha un piccolo scompartimento sotto la gemma, di solito utilizzato per contenere del veleno. Aprirlo e chiuderlo richiede un'azione; farlo senza essere notati richiede una prova di Mani di Fata.

\textbf{Antitossina}. Una creatura che beve da questa fiala di liquido ottiene +1d6 sui Tiri Salvezza contro il veleno per 1 ora. Non conferisce alcun bonus ai non morti e ai costrutti.

\textbf{Ariete Portatile}. Puoi usare un ariete portatile per abbattere le porte. Nel farlo, ottieni +3 bonus di prove di Corpo. Un altro personaggio può aiutarti con l’uso dell’ariete, dandoti +2 sulla prova.

\textbf{Attrezzatura da Pesca}. Questo kit comprende un’asta di legno, filo di seta, taglierino di legno, ami d’acciaio, peso di piombo, esche di velluto e un retino.

\textbf{Bandoliera}. Questa cintura specializzata per contenere piccoli oggetti quali pozioni o pergamene si porta al collo. Estrarre un oggetto da essa costa 1 PA, come se fosse alla cintura.

\textbf{Biglie di Metallo}. Con un’azione, puoi spargere una singola borsa di queste minuscole biglie di metallo per coprire un’area piana quadrata di 3 metri di lato. Una creatura che attraversa l’area coperta deve superare una Prova su Corpo o cadere prona. Una creatura che attraversa l’area a metà velocità non deve effettuare la Prova di Corpo.

\textbf{Bilancia da Mercante}. Una bilancia da mercante include un piccolo bilanciere, un piatto, e un assortimento di pesi fino a 1 chilo. Con essa, puoi misurare il peso esatto di piccoli oggetti, come metalli preziosi o merci, per aiutarti a determinarne il valore.

\textbf{Borsa dei Componenti}. Una borsa dei componenti è un piccolo borsello da cinta di cuoio impermeabile munito di compartimenti contenenti tutte le componenti materiali e altri oggetti speciali di cui hai bisogno per lanciare i tuoi incantesimi, eccetto per quelle componenti che hanno un costo specifico o sono materiali non comuni (come indicato nella descrizione dell’incantesimo).

\textbf{Borsello}. Un borsello di tessuto o cuoio può contenere, tra le altre cose, fino a 20 proiettili da fionda o 50 aghi da cerbottana. Un borsello diviso in compartimenti per contenere componenti per incantesimi viene detto borsa dei componenti.

\textbf{Candela}. Per 1 ora di tempo reale di gioco, una candela proietta luce in un raggio di 1,5 metri e luce fioca per ulteriori 1.5 metri.

\textbf{Cerata}. E' un mantello trattato per essere idrorepellente, ti permette di rimanere asciutto anche sotto la pioggia.

\textbf{Cannocchiale}. Gli oggetti osservati tramite un cannocchiale sono ingranditi al doppio delle loro dimensioni.

\textbf{Carrucola e Paranco}. Una serie di leve collegate da un cavo e un gancio per attaccarsi ad oggetti, carrucola e paranco ti permettono di tirare su fino a quattro volte il
peso che puoi normalmente sollevare.

\textbf{Catena}. Una catena ha 15 di Vigore e PDAD 16. Può essere spezzata superando una prova di Corpo con MS +12.

\textbf{Chiodi da rocciatore}. Se ne deve usare 1 almeno ogni 6 metri per fissare la corda alla parete.

\textbf{Corda}. Una corda, che sia fatta di canapa o seta, ha 2 di Vigore e può essere spezzata superando una prova di Corpo con MS +6. La versione grossa ha 4 di Vigore e MS +9 per spezzarla.

\textbf{Faretra}. Una faretra può contenere fino a 12 frecce\index{Faretra}.

\textbf{Fuoco Alchemico}. Questo fluido appiccicoso si incendia quando entra a contatto con l’aria. Con due azioni, puoi lanciare questa ampolla fino a 6 metri, fracassandola all’impatto. Effettua un PDAA a distanza contro la creatura o l’oggetto, trattando il fuoco alchemico come un’arma improvvisata (Svantaggio). Se colpisci, il bersaglio subisce 1d6 danni da fuoco a Vigore all’inizio di ciascun suo round. Una creatura può porre fine a questi danni spendendo 8 PA e superando una prova di Corpo. Se la prova riesce le fiamme si estinguono.

\textbf{Borsa da Guaritore}. Questo kit è una borsa di cuoio contenente bende, unguenti e stecche. Il kit può essere usato dieci volte. Concede un 2 Bonus alle prove di pronto soccorso.

\textbf{Kit da Pranzo}. 4 GO. Questa piccola scatola di latta contiene una ciotola e delle semplici posate. Le due parti della scatola possono essere staccate, e un lato impiegato come pentola per cucinare e l’altro come piatto o contenitore

\textbf{Kit da Scalatore}. 8 GO. Un kit da scalatore comprende chiodi speciali, punte per stivali, guanti e un’imbracatura. Puoi ancorarti usando il kit da scalatore con un’azione; quando lo fai, non puoi cadere per più di 6 metri dal punto in cui ti sei ancorato, e non puoi arrampicarti a più di 6 metri di distanza dal punto a cui ti sei ancorato senza prima disfare l’ancora.

\textbf{Lanterna}. Una Lanterna proietta luce intensa in un raggio di 3 metri e luce fioca per ulteriori 6 metri. Una volta accesa, brucia per 3 ore di tempo reale di gioco con un’ampolla (0,5 litri) d’olio.

\textbf{Lanterna a Lente Sporgente}. Una lanterna a lente sporgente proietta luce in un cono di 3 metri e luce fioca per ulteriori 9 metri. Una volta accesa, brucia per 3 di tempo reale di gioco ore con un’ampolla (0,5 litri) d’olio.

\textbf{Lanterna Schermabile}. Una lanterna schermabile proietta luce in un raggio di 6 metri e luce fioca per ulteriori 6 metri. Una volta accesa, brucia per 1 ora di tempo reale di gioco con un’ampolla (0,5 litri) d’olio. Con un’azione, puoi abbassare la schermatura, riducendo la luce a fioca con un raggio di 1 metri.

\textbf{Lente Ingranditrice}. Questa lente permette di dare un’occhiata più ravvicinata agli oggetti piccoli. È anche un utile sostituto per pietra focaia e acciarino nell’accendere un fuoco. Appiccare un fuoco con la lente ingranditrice richiede almeno una luce di intensità pari a quella solare, legna da accendere, e circa 5 minuti di tempo perché il legno prenda fuoco. Una lente ingranditrice fornisce aiuto (2 Bonus) in qualsiasi prova effettuata per valutare o analizzare un oggetto piccolo o molto dettagliato. 

\textbf{Lente del Cacciatore:} 100 GO, questa complessa lente viene posta su un occhio e occupa lo slot occhi quando è in uso. Quando la si utilizza con un attacco a distanza, si riduce di 2 le penalità da attacchi a distanza. Gli oggetti entro 9 metri diventano difficili da vedere, e si subisce penalità 2 alle prove di Osservare basate sulla vista e PDAA.

\textbf{Manette}. Questi strumenti di metallo possono imprigionare una creatura Piccola o Media. Per liberarsi dalle manette bisogna superare una prova di Corpo con MS +6. Per romperle bisogna superare una prova di Corpo con MS +9. Ogni set di manette è fornito di una chiave. Senza la chiave, una creatura può usare Artista della Fuga o Disattivare Congegni per aprire la serratura. Le manette hanno 15 di Vigore e PDAD 12

\textbf{Olio}. Di solito si compra in un’ampolla d’argilla che contiene 0,5 litri. Con un’azione, puoi spargere l’olio in questa ampolla su di una creatura entro 1 metri da te o lanciarla fino a 6 metri, fracassandola all’impatto. In entrambi i casi, effettua un PDAA a distanza contro la creatura o l’oggetto, trattando l’olio come un’arma improvvisata. Se colpisci, il bersaglio è ricoperto d’olio. Se il bersaglio subisse qualsiasi entità di danno da fuoco prima che l’olio si asciughi (dopo 1 minuto), il bersaglio subisce altri 1d6 danni da fuoco dall’olio infiammato per round. Se infiammato, l’olio brucia per 2 round e infligge 1d6 danni da fuoco a qualsiasi creatura che entri nell’area o termini il suo round dentro di essa. Una creatura può subire questo danno solo una volta per round. Puoi anche versare un’ampolla d’olio sul pavimento per coprire un quadrato di 1 metro di lato, purché la superficie sia piana.

\textbf{Piede di Porco}. Utilizzare un piede di porco dà 2 Bonus alle prove di Corpo ogni volta si possa applicare la leva del piede di porco.

\textbf{Pozione di Cura}. Questa pozione generica di cura consente di recuperare 1d8+1 di Vigore.

\textbf{Pozione di Cura potenziata}. Questa pozione generica di cura consente di recuperare 3d8+3 di Vogore.

\begin{changemargin}{0.3cm}{0.3cm}\begin{narratore} Per quanto sia assolutamente contrario all'acquisto di oggetti magici da parte dei personaggi le Pozioni di Cura devono essere disponibili.
\end{narratore}\end{changemargin}

\textbf{Razioni}. Le razioni consistono di cibo secco adatto a lunghi viaggi, e includono carne secca, frutta secca, gallette e noci.

\textbf{Scatola con l’Esca}. Questo piccolo contenitore contiene pietra, acciarino ed esca (di solito uno straccio secco imbevuto d’olio) impiegati per appiccare un fuoco. Utilizzarlo per accendere una torcia (o qualsiasi altro oggetto facilmente incendiabile) richiede due azioni. Accendere qualsiasi altro fuoco richiede 1 minuto.

\textbf{Scatola per Mappe o Pergamene}. Questa scatola cilindrica di cuoio può contenere, arrotolati, fino a dieci pezzi di carta o cinque fogli di pergamena.

\textbf{Faretra per Quadrelli da Balestra}. Questa scatola di legno contiene fino a 12 quadrelli per balestra.

\textbf{Serratura}. Insieme alla serratura viene fornita una chiave. Senza la chiave, una creatura può scassinare questa serratura superando una prova di Disattivare Congegni. L'Arbitro può decidere che per costi maggiori sono disponibili serrature di qualità migliore che richiedano un MS di +3 od oltre.

\textbf{Tappi per Orecchie} 3 GR, fatti di cotone o sughero cerato, i tappi per orecchie concedono 1 Bonus alle Prove contro gli effetti che richiedono l'udito ma infliggono penalità -4 alle prove di Osservare basate sull'udito.

\textbf{Tenda}. Un semplice riparo portabile di tela, una tenda può contenere due persone. Ci vogliono circa 20 minuti per montare una tenda.

\textbf{Torcia}. Una torcia brucia per \textbf{1 ora di tempo di gioco reale}, fornendo luce in un raggio di 3 metri e luce fioca per ulteriori 6 metri. Se effettui un PDAA con una torcia accesa, arma improvvisata, e colpisci, infliggi 1d4 di danno più 1 danno da fuoco aggiuntivo a vigore. \index{Torcia}

\textbf{Trappola da Caccia}. 12 GO, 2. Usi due azioni per disporre questa trappola, formata da un anello d’acciaio seghettato, che scatta quando una creatura calpesta la piastra metallica al centro di essa. La trappola è fissata tramite una catena pesante a un oggetto immobile, come un albero o uno spuntone conficcato nel terreno. Una creatura che calpesti la piastra deve superare una Prova di Corpo con MS +3 o subire 1d4 danni perforanti e interrompere il movimento. Una creatura può usare 6 PA per superare una Prova di Corpo con MS +3, e se la riesce si libera o libera un’altra creatura a portata. Ogni tentativo fallito infligge 1 danno perforante alla creatura intrappolata.

\textbf{Tribolo}. Con un’azione, puoi spargere una singola borsa di questi minuscoli triboli per coprire un’area quadrata di 1 metri di lato. Una creatura che attraversa l’area coperta deve superare una Prova di Corpo o subire 1 danno perforante. Finché la creatura non recupera almeno 1 danno a Vigore, il terreno si considera difficile e muoversi di 1.5 metri costa 2 PA. Una creatura che attraversa l’area a metà velocità non deve effettuare la Prova.

\textbf{Veleno Base}. Puoi usare il veleno in questa fiala per coprire un’arma tagliente o perforante o fino a tre pezzi di munizioni. Applicare il veleno necessita di 4 PA. Una creatura colpita da un’arma o munizione avvelenata deve superare una Prova di Corpo o subire 1d4 danni da veleno a Vigore.
Una volta applicato, il veleno mantiene la sua efficacia per 1 minuto prima di seccarsi.

\subsubsection{Dotazioni di base}
Se il personaggio sceglie di acquistare il suo equipaggiamento di partenza, può acquistare una dotazione al prezzo indicato, che generalmente è più conveniente rispetto all'acquisto dei singoli oggetti separati.

\textbf{Dotazione da Avventuriero (18 GO)}. Include uno zaino, un piede di porco, un martello, 10 chiodi da rocciatore, 10 torce, un acciarino e pietra focaia, 10 razioni giornaliere e un otre. La dotazione include anche 15 metri di corda di canapa legata allo zaino.

\textbf{Dotazione da Cacciatore (24 GO)}: contiene acciarino e pietra focaia, una borsa da cintura, una corda 18m, un giaciglio, una cerata, un otre, una pentola di ferro, razioni da viaggio (5 giorni), torce (10) e uno zaino.

\textbf{Dotazione da Diplomatico (57 GO)}. Include un forziere, 2 custodie per mappe e pergamene, un abito pregiato, una boccetta di inchiostro, un pennino, una lanterna, 2 ampolle di olio, 5 fogli di carta, una fiala di profumo, cera per sigillo e sapone.


\textbf{Dotazione da Devoto (30 GO)}: contiene acciarino e pietra focaia, una borsa da cintura, una Borsa per Componenti di Incantesimi, candele (10), corda 18m, un giaciglio, una pentola di ferro, un otre, razioni da viaggio (per 5 giorni), sapone, un simbolo sacro di legno, un testo sacro economico, torce (10) e uno zaino.

\textbf{Dotazione da Esploratore (15 GO)}. Include uno zaino, un giaciglio, una gavetta, un acciarino e pietra focaia, 10 torce, 10 razioni giornaliere e un otre. La dotazione include anche 15 metri di corda di canapa legata allo zaino.

\textbf{Dotazione da Esploratore di caverne (24 GO)}: contiene un insieme di attrezzi di base per esplorare rovine e città abbandonate include 2 candele, un gessetto, un martello e 4 Chiodi da Rocciatore, 18 metri di corda, una lanterna schermabile con 5 ampolle d'olio, 2 sacchi, 2 torce, razioni da viaggio (per 3 giorni)

\textbf{Dotazione da Intrattenitore (60 GO}). Include uno zaino, un giaciglio, 2 costumi, 5 candele, 5 razioni giornaliere, un otre e trucchi per il camuffamento.

\textbf{Dotazione da Scassinatore (24 GO)}. Include uno zaino, un sacchetto con 1000 sfere metalliche, 3 metri di spago, una campanella, 5 candele, un piede di porco, un martello, 10 chiodi da rocciatore, una lanterna schermabile, 2 ampolle di olio, 5 razioni giornaliere, un acciarino e pietra focaia e un otre. La dotazione include anche 15 metri di corda di canapa legata allo zaino.

\textbf{Dotazione da Studioso (60 GO)}. Include uno zaino, un libro di studio, una boccetta d'inchiostro, un pennino, 10 fogli di pergamena, un sacchetto di sabbia e un coltellino.


\end{multicols}

\subsubsection{Capienza dei Contenitori}

\begin{tabularx}{0.95\textwidth}{lXl|lXl}
\textbf{Oggetto}&\textbf{Capienza}&\textbf{CdC}&\textbf{Oggetto}&\textbf{Capienza}&\textbf{CdC}\\
\toprule
Borsa&1 cubo con spigolo di 30 cm/3 kg di equipaggiamento&1&Barile&160 litri liquidi, 4 cubi con spigolo di 30 cm&35\\
Boccale&0,5 litri&L&Bottiglia&1 litro di liquido&L\\
Secchio&12 litri liquidi, 1 cubo con spigolo di 25 cm&3&Canestro&2 cubi con spigolo di 30 cm/20 kg di equipaggiamento&5\\
Sacco&1 cubo con spigolo di 30 cm/15 kg di equipaggiamento&3&Forziere&12 cubi con spigolo di 30 cm/150 kg di equipaggiamento&35\\
Fiala&120 ml di liquidi&L&Otre&2 litri liquidi&1\\
Caraffa&4 litri liquidi&2&Zaino&2 cubi con spigolo di 30 cm/30 kg di equipaggiamento&6\\
\end{tabularx}


\begin{multicols}{2}

\subsubsection{Strumenti}

L'elenco degli strumenti presentati aiuta i personaggi ad eseguire le prove legate alle loro Competenze.

Ad esempio una prova di "Calligrafia" si risolve con una prova di Mente se il personaggio ha a disposizione gli strumenti idonei ("\textit{Scorte da Calligrafo}") ottiene 2 bonus alla Prova.

\end{multicols}

\medskip

\begin{tabularx}{0.95\textwidth}{llX|llX}
\textbf{Oggetto}&\textbf{Costo}&\textbf{Ing.}&\textbf{Oggetto}&\textbf{Costo}&\textbf{Ing.}\\
\toprule
Arnesi da Scasso&25 GO&1&Borsa da Erborista&5 GO&1\\
Arnesi da Falsario&25 GO&1&Strumenti da Gioielliere&150 GO&2\\
Dadi&1 GA&-&Mazzo di Carte&5 GA&-\\
Scacchi dei Draghi&1 GO&1&Tre Draghi al Buio&1 GO&-\\
Sostanze da Avvelenatore&50 GO&1&Scorte da Alchimista&50 GO&2\\
Scorte da Calligrafo&10 GO&1&Scorte da Mescitore&20 GO&2\\
Strumenti da Calzolaio&5 GO&2&Strumenti da Cartografo&15 GO&2\\
Strumenti da Conciatore&5 GO&2&Strumenti da Costruttore&10 GO&2\\
Strumenti da Fabbro&20 GO&3&Strumenti da Falegname&8 GO&2\\
Strumenti da Gioielliere&25 GO&1&Strumenti da Intagliatore&1 GO&2\\
Strumenti da Inventore&50 GO&2&Strumenti da Pittore&10 GO&1\\
Strumenti da Soffiatore&30 GO&2&Strumenti da Tessitore&1 GO&2\\
Strumenti da Vasaio&10 GO&2&Utensili da Cuoco&1 GO&2\\
Strumenti da Navigatore&25 GO&2&Ciaramella&2 GO&1\\
Cornamusa&30 GO&1&Corno&3 GO&L\\
Dulcimer&25 GO&2&Flauto&2 GO&0L\\
Flauto di Pan&12 GO&L&Lira&30 GO&L\\
Liuto&35 GO&1&Tamburo&6 GO&1\\
Viola&30 GO&1&Trucchi per il Camuffamento&25 GO&1\\
\end{tabularx}

\begin{multicols}{2}

\subsubsection{Cavalcature e Veicoli}

Una buona cavalcatura può consentire a un personaggio di attraversare rapidamente un territorio selvaggio, ma il suo scopo primario è trasportare l'equipaggiamento che altrimenti rallenterebbe il suo padrone.

La tabella "Cavalcature e Altri Animali" indica i piedi (30cm) per PA usata e la capacità di trasporto base di ogni animale. Un animale che tira una biga, un carretto, un carro, una carrozza o una slitta può spostare un peso pari a cinque volte la sua capacità di trasporto, incluso il peso del veicolo. Se più animali tirano lo stesso veicolo, possono sommare assieme la loro capacità di trasporto.

In Dark Catacomb esistono altre cavalcature oltre a quelle elencate in questa sezione, ma si tratta di cavalcature rare che normalmente non sono disponibili per l'acquisto, come certe cavalcature volanti (pegasi, grifoni, ippogrifi e altri animali simili) o perfino alcune cavalcature acquatiche (come per esempio i cavallucci marini giganti) oppure demoniache o angeliche.

Per entrare in possesso di una cavalcatura del genere spesso è necessario rubare un uovo e crescere la creatura di persona, stipulare un patto con una potente entità o negoziare con la cavalcatura stessa.

\textbf{Bardatura}. Una bardatura è un'armatura concepita per proteggere la testa, il collo, il petto e il corpo di un animale. Ogni tipo di armatura elencato nella tabella "Armature" di questo capitolo può essere acquistata come bardatura. Il costo è pari al quadruplo dell'armatura equivalente fabbricata per gli umanoidi, mentre il peso è pari al doppio.

\textbf{Sella}. Un cavalcatore può agganciarsi a una sella militare per rimanere al suo posto su una cavalcatura attiva, nel corso di una battaglia. Una sella militare conferisce vantaggio alle prove che il personaggio effettua per rimanere in sella. E' necessaria una sella esotica per cavalcare una creatura acquatica o volante.

\textbf{Imbarcazioni a Remi}. I barconi e le barche a remi sono solitamente usati sui laghi e sui fiumi. Se un'imbarcazione segue la corrente, si aggiunge la velocità della corrente (solitamente 4,5 km all'ora) alla sua velocità. In genere non è possibile remare controcorrente se la corrente ha un'intensità rilevante, ma è possibile far risalire un corso d'acqua a queste imbarcazioni portandole a riva e facendole trainare da una o più bestie da soma. Una barca a remi pesa 50 kg (Ingombro 10) qualora gli avventurieri debbano trasportarla via terra.

\end{multicols}

\subsubsection{Cavalcature e Altri Animali}

\begin{center}
	\begin{tabular}{lllll}
\toprule
\textbf{Cavalcatura}&\textbf{Costo (GO)}&\textbf{Piedi x PA} &\textbf{Carico} &\textbf{Km/h}\\
Asino o Mulo&8&4&210&6km\\
Cammello&50&5&240&8km\\
Cavallo da Galoppo&75&6&240&12km\\
Cavallo da Guerra&400&6&270&9km\\
Cavallo da Tiro&50&4&270&6km\\
Elefante&200&4&660&6km\\
Mastino&25&4&97,5&6km\\
Pony&30&4&112,5&6km\\
\end{tabular}
\end{center}

\bigskip

\begin{multicols}{2}

\textbf{Finimenti e Veicoli da Tiro}\\
\begin{tabularx}{0.45\textwidth}{lll}
\toprule
\textbf{Oggetto}&\textbf{Costo}&\textbf{Peso}\\
Bardatura&x4&x2\\
Biga&250 GO&50 kg\\
Bisacce&4 GO&4 kg\\
Carretto&15 GO&100 kg\\
Carro&35 GO&200 kg\\
Carrozza&100 GO&300 kg\\
Morso e Briglie&2 GO&0,5 kg\\
Nutrimento (x giorno)&5 GR&15 kg\\
\end{tabularx}

\bigskip

\textbf{Sella}\\
\begin{tabularx}{0.45\textwidth}{lll}
\toprule
\textbf{Oggetto}&\textbf{Costo}&\textbf{Peso}\\
Da Carico&5 GO&7,5 kg\\
Da Galoppo&10 GO&12,5 kg\\
Esotica&60 GO&20 kg\\
Militare&20 GO&15 kg\\
Slitta&20 GO&150 kg\\
Stallaggio (x giorno)&5 GA&\\
\end{tabularx}

\end{multicols}

\bigskip

\begin{center}

\textbf{Imbarcazioni}\medskip

\begin{tabular}{lll}
\toprule
\textbf{Oggetto}&\textbf{Costo}&\textbf{Velocità}\\
Barca a Remi&50 GO&2,25 km orari\\
Barcone&3000 GO&1,5 km orari\\
Galea&30000 GO&6 km orari\\
Nave a Vela&10000 GO&3 km orari\\
Nave da Guerra&25000 GO&3,75 km orari\\
Nave Lunga&10000 GO&4,5 km orari\\
\end{tabular}
\end{center}

\begin{multicols}{2}


\subsubsection{Servizi}


Gli avventurieri possono pagare i personaggi non giocanti affinché li aiutino o agiscano in loro vece nelle circostanze più disparate. La maggior parte di questi gregari è dotata di abilità pressoché ordinarie, mentre altri hanno padroneggiato un'arte o un mestiere e alcuni si sono specializzati in qualche abilità da avventuriero.

Altri gregari comuni includono i numerosi abitanti di un tipico paese o città che gli avventurieri possono ingaggiare per svolgere un compito specifico. Per esempio, un incantatore potrebbe pagare un falegname per farsi costruire un pregiato scrigno (e la sua replica in miniatura) da usare per un incantesimo.
Un guerriero potrebbe commissionare a un fabbro la forgiatura di una spada speciale.

\medskip

\textbf{Servizi}

\bigskip

\begin{tabularx}{0.45\textwidth}{Xl}
\textbf{Servizio}&\textbf{Costo}\\
\toprule
Carrozza all'interno di una città&5 GR/1 km\\
Carrozza tra due paesi&1 GA/1 km\\
Gregario Abile&2 GO al giorno\\
Gregario Inesperto&5 GA al giorno\\
Messaggero&5 GR/1,5 km\\
Passaggio in nave&1 GA/1,5 km\\
Pedaggio stradale o di ingresso&5 GR/5 GA\\
\end{tabularx}


\subsubsection{Servizio magici}


La magia è rara e difficilmente una strega vorrà mettersi in mostra.
I personaggi dovranno effettuare una eccellente prova di convincimento per qualsiasi magia abbiano bisogno.

\subsubsection{Oggetti e Sostanze Speciali}\index{Sostanze Speciali}

\textbf{Antiemetico} 25 GO,  questo liquido verde dolce e saporito crea un senso di calore e conforto. Lo sciroppo protegge lo stomaco e lo rende più resistente. Per 1 ora dopo averlo bevuto si ottiene 3 Bonus alle Prove su Corpo per resistere agli effetti che rendono Nauseati o contro i veleni da Ingestione.

\textbf{Antibiotico} (fiala) 50 GO, bevendo una fiala di questo liquido bianco latte dal pessimo sapore si ottiene 3 Bonus alle Prova di Corpo contro le Malattie, effettuati nell'ora successiva. Se già infetti, si possono effettuare una altra Prova di Corpo con 3 Bonus. Monodose. 

\textbf{Antitossina} (boccetta) 50 GO, se si beve l'antitossina, si ottiene 3 Bonus alle prove di Corpo contro Veleni per 1 ora. Monodose. 

\textbf{Bastone del Fumo} 20 GO, questo bastone di legno trattato con procedimento alchemico crea istantaneamente un denso fumo opaco quando viene infiammato. Il fumo riempie un cubo con spigolo di 3 metri (distanza di mischia), tranne che il fumo viene dissipato in 1 round da un vento moderato o più intenso. Il bastone si consuma in 1 round e il fumo si dissolve poi naturalmente. Tutte le creature nell'area influenzata hanno copertura totale.

\textbf{Caffettone dell'Alchimista} 1 GO, molto amata dai giovani si tratta di una polvere cristallina bruna. Mischiata con l'acqua crea una bevanda amara che cura gli effetti della sbornia. Monodose.

\textbf{Borsa dell'Impedimento} 50 GO, questa borsa di cuoio rotonda è piena di melassa, resina o altra sostanza appiccicosa. Quando si scaglia la borsa contro una creatura (come attacco di contatto a distanza con gittata 3 metri), la borsa si apre e la sostanza contenuta invischia ed Intralcia la vittima, diventando resistente ed elastica con l'esposizione all'aria.

La sostanza non agisce su creature di taglia Enorme o superiore. Una creatura volante non viene appiccicata al suolo, ma deve effettuare una Prova di Corpo o perde la capacità di Volare (sempre che usi le ali per farlo), cadendo a terra. La borsa dell'impedimento non funziona sott'acqua.

\textbf{Fermasangue} 25 GO, questa sostanza rosa e appiccicosa aiuta a curare le ferite. Utilizzarne una dose concede 3 Bonus alle prove di Pronto Soccorso. 6 Usi.

\textbf{Fiasco Alcalino} 15 GO, questo fiasco di liquidi caustici reagisce con gli acidi naturali delle melme. E' possibile lanciare un fiasco alcalino come arma a spargimento con gittata 3 metri. Contro le creature non melme un fiasco alcalino funziona come un'Ampolla d'acido. Contro le melme e altre creature acide il fiasco alcalino infligge i danni raddoppiati indicati da Ampolla d'Acido. 

\textbf{Fumogeno} 25 GO, questa piccola sfera di argilla contiene due sostanze alchemiche separate da una sottile barriera. Quando si rompe la sfera, le sostanze si uniscono e riempiono un area di mischia con una nuvola di fumo nerastro e innocuo. Il fumogeno funziona come un bastone del fumo, ma il fumo rimane per 1 round prima di disperdersi. E' possibile lanciare un fumogeno come attacco di contatto con gittata 3 metri.

\textbf{Fuoco dell'Alchimista} 20 GO, si può lanciare un'ampolla di fuoco dell'alchimista come arma a spargimento. Si consideri l'attacco come un attacco di contatto a distanza, con gittata 3 metri.

Il colpo diretto provoca 1d6 danni da fuoco a Vigore. Tutte le creature entro raggio di mischia dal punto in cui è caduta l'ampolla subiscono 1 danno da fuoco come effetto dello spargimento. Nel round successivo al colpo diretto la vittima subisce 1d6 danni da fuoco aggiuntivi. La vittima può sfruttare 4 PA per tentare di spegnere le fiamme prima di subire questi danni aggiuntivi. Occorre superare una Prova di Corpo per spegnere le fiamme. Usare 6 PA dà al personaggio 2 bonus alla Prova. Tuffarsi in acqua o smorzare le fiamme con mezzi magici spegne automaticamente le fiamme.

\textbf{Gesso per Calchi:} 5 GA, questa polvere bianca e secca, mischiata con l'acqua, si addensa nel giro di un'ora per creare un materiale solido. Può essere utilizzato per creare un calco di un'orma o di un bassorilievo, riempire buchi o crepe nei muri o (se applicato ad una copertura di stoffa) per fermare un osso rotto. Il gesso indurito ha Vigore 6 e PDAD 11. Un vaso di 2 kg di gesso può coprire un raggio di mischia per la profondità di 2.5 centimetri, creare cinque ingessature per l'avambraccio o il polpaccio di una creatura di taglia Media o due ingessature complete per braccio o gamba. Monodose.

\textbf{Ghiaccio Liquido} (fiala) 40 GO, detto anche "ghiaccio dell'alchimista", questo fluido blu cristallino inizia ad evaporare appena tolto dal contenitore. Nei successivi 1d6 round è possibile utilizzarlo per congelare un liquido o coprire un oggetto con un sottile strato di ghiaccio. E' possibile anche lanciare il ghiaccio liquido come arma a spargimento. Un colpo diretto infligge 1d6 danni da freddo a Vigore, mentre le creature entro raggio di mischia subiscono 1 danno da freddo per lo spargimento. La confezione contiene 3 dosi.

\textbf{Grasso Alchemico} 5 GO, ogni vaso di questa sostanza nerastra può coprire una creatura Media o due Piccole. Coprendosi di grasso alchemico si ottiene 2 Bonus alle prove di lotta e per sfuggire alle prese. L'effetto dura 4 ore o finché si lava via il grasso.

\textbf{Individua Luce} 1 GO, questa piastra di metallo grande quanto una mano è coperta da una crema trasparente sensibile alla luce. Se esposta alla luce, la crema si scurisce e diviene opaca a seconda di quanta luce sia presente. La luce intensa la fa scurire in 1 round, quella normale in 3 round, quella fioca in 10 round.
La piastra viene venduta avvolta in un panno pesante per evitare esposizioni accidentali. 

\textbf{Individua Luce avanzata} 50 GO, questa piastra di metallo simile alla piastra Individua luce è grande circa 50cm*50 cm. Se esposta alla luce imprime su di essa l'immagine dell'ambiente circostante entro 3 metri.

\textbf{Pietra del Tuono} 30 GO, si può scagliare questa pietra con un attacco a distanza con gittata 6 metri. Quando colpisce una superficie dura (o è colpita con forza), crea un rumore assordante che equivale a un attacco sonoro. Le creature presenti entro una distanza di 3 metri devono effettuare una Prova di Corpo o restano Assordate per 1 ora. Monouso.

\textbf{Polvere Lampo} 50 GO, questa polvere argentea brucia ed esplode quasi istantaneamente se esposta al fuoco, frizionandola o lanciandola con forza contro una superficie (4 PA). Le creature entro raggio 3 metri sono Accecate per 1 round (Prova di Corpo nega). La confezione contiene 3 dosi. 

\textbf{Proteggilama} 40 GO, questa resina trasparente protegge un'arma dagli attacchi di Melme, Rugginofagi ed effetti che corrodono o sciolgono le armi, rendendola immune a tali attacchi per 24 ore. Un vasetto può coprire un'arma a due mani, due armi ad una mano o leggere o 50 munizioni. Applicarla richiede 6 PA. La confezione contiene 3 dosi.

\textbf{Solvente Universale} (fiala) 20 GO, questa gelatina viola ribollente divora gli adesivi. Ogni fiala può coprire un raggio di mischia. Distrugge i normali adesivi (come la pece, la resina o la colla) in 1 round, ma richiede 1d4+1 round per dissolvere adesivi più potenti (borse dell'impedimento, ragnatele, ecc.). Non ha effetti sugli adesivi magici.

\textbf{Tizzone Ardente} 1 GO, la sostanza alchemica sulla punta di questo piccolo bastone di legno si infiamma quando viene sfregata contro una superficie ruvida. Creare una fiamma con un tizzone ardente è molto più rapido che crearla con acciarino, pietra focaia (o lente d'ingrandimento) e esca. Accendere una torcia con un tizzone ardente costa 4 PA e per accendere qualsiasi altro fuoco occorre almeno 10 PA.

\subsubsection{Attrezzature Alchemiche}

\textbf{Carta Reagente} 1 GO, questo pezzo di carta può aiutare a identificare i liquidi. Il suo colore cambia a seconda di tratti come acidità, salinità e magia. Consumare un foglio conferisce 1 Bonus alle prove di Lavoro (alchimia) od Occulto per identificare Pozioni o altri liquidi.

\textbf{Inchiostro Esplosivo} (fiala) 40 GO, questo inchiostro infuso alchemicamente aiuta ad assicurarsi che un messaggio segreto venga distrutto dopo essere stato letto. Se la luce colpisce l'inchiostro dopo che quest'ultimo si è asciugato, le sostanze chimiche lo fanno bruciare spontaneamente nel giro di 1 minuto

Questa combustione è di piccole dimensioni: non è abbastanza significativa da dar fuoco ad altro che alla carta. L'inchiostro usato su altri materiali come pietra o legno semplicemente svanisce, non lasciando alcuna traccia della scrittura
Una fiala di questo inchiostro ne contiene abbastanza da scrivere 10 brevi messaggi di non più di 50 parole ciascuno.

\textbf{Olio dei Liutai} 50 GO, quest'olio dorato profuma di legno antico. Quando lo si applica sulla cassa di uno strumento musicale di legno ne migliora la qualità del suono. Per 1 ora, chiunque suoni lo strumento ottiene 1 Bonus alla prova di Intrattenere appropriata.

\textbf{Pastiglia dell'Usignolo} 50 GO, questa caramella ricoperta di miele è fatta di reagenti calmanti. Se mangiata, ha bisogno di 1 round per iniziare ad avere effetto, dopodiché conferisce 1 Bonus alle prove di Intrattenere (canto) per 1 ora.

\textbf{Pietre di Via} 50 GO, questi piccoli sassolini bianchi sono trattati alchemicamente in modo che emanino una luce soffusa quando attivati sfregandoli gli uni contro gli altri. La luminescenza è fioca, appena sufficiente a illuminare la pietra. La durata è di 8 ore.

\textbf{Polvere Tracciante} 30 GO, quando sparsa per terra, questa sottilissima polvere blu chiaro rivela le tracce di qualsiasi creatura o individuo che sia passato nell'area nelle ultime 48 ore.
La polvere fornisce anche 3 Bonus alle prove di Sopravvivenza per individuare le tracce. Una singola applicazione può coprire un'area di 3 metri. La polvere tracciante viene venduta in piccole borse di cuoio che contengono 10 applicazioni ciascuna.

\subsubsection{Rimedi Alchemici}\index{Rimedi Alchemici}

\label{rimedi-alchemici}

\textbf{Aiuto Gassato} 25 GO, questo pacchetto è pieno di foglie dai bordi spinosi e ha un odore pungente quasi abbastanza forte da far lacrimare gli occhi. Mentre si masticano le foglie, si ignorano gli effetti dell'essere affaticati o esausti. Le foglie durano per 10 round, dopodiché ne rimane solo un mucchietto di poltiglia.
Quando l'effetto dell'aiuto gassato si esaurisce, si aumenta di 1 grado il livello di affaticamento. Un pacchetto basta per 1 sola volta.

\textbf{Balsamo Anti-veleno} 15 GO, questo balsamo alle erbe può essere applicato direttamente sulla pelle per prevenire gli effetti dei Veleni a contatto. Se una creatura tocca un veleno a contatto, ma applica su di sé il balsamo entro 1 round dal contatto, effettua la Prova di Corpo con 3 Bonus. Monouso.

\textbf{Balsamo Coagulante} 5 GA, applicare questo balsamo alle erbe su una ferita cura 1 danno a Vigore, non è possibile usare più di due dosi al giorno sullo stesso paziente. La confezione è per 3 usi.

\textbf{Amaro Fortificante} 20 GO, questo liquido alcolico genera una piacevole sensazione di calore quando ingerito. Per l'ora successiva, si ottiene 2 Bonus alle Prove di Mente contro Paura. Usare più dosi nell'arco delle stesse 24 ore rende Nauseati per 1 ora. La confezione è per 3 usi.


\subsubsection{Lo Zaino Standard\texorpdfstring{\huge{\textregistered}}{\textregistered}} \index{Zaino Standard}

Lo Zaino Standard\textregistered \space è una lista di oggetti che ho segnato nel tempo andando ad aggiungere ogni cosa che nel corso delle avventure mi era servito.
Prendetela come spunto per capire che oggetti avere dietro, non segnateveli tutti altrimenti l'Arbitro incomincerà seriamente a guardare le regole dell'Ingombro!

Questo il contenuto dello zaino dell'avventuriero: cintura, 3 candele, 6 torce, esca e acciarino, 7 razioni secche, fiasca d'acqua, materasso arrotolato, cerata, tenda, 18 metri corda, rete, specchio di metallo, piede di porco, bussola, 3 olio da lanterna, inchiostro, gesso, carboncino, uncino, vanga, amo da pesca, stracci, cavo di metallo 2m, fischietto, 6 fiale da pozione vuota, biglie di marmo, campanella in ottone, 1kg di farina in sacchetto, 3 zeppe, catena di metallo 12 metri, 2 manette, 8 chiodi da rocciatore, martello, carrucola, rampino.


\end{multicols}

\pagebreak

\section{La Magia}

\begin{changemargin}{0.3cm}{0.3cm}\begin{enfasi}{
Non lascerai vivere colei che pratica la magia. (Libro dell'Esodo 22,17-18)\\

"Non si trovi in mezzo a te chi fa passare il figlio o la figlia nel fuoco, chi usa la divinazione, chi fa presagi, chi pratica la magia, chi fa incantesimi, chi consulta gli spiriti, chi evoca i morti. Infatti chiunque fa queste cose è in abominio al Signore; e a motivo di queste abominazioni il Signore tuo Dio sta per scacciare quelle nazioni davanti a te. (Deuteronomio 18:10-12) 
} \end{enfasi}\end{changemargin}

\begin{multicols}{2}
	


La magia è sempre stata insieme a noi, occultata e protetta da sguardi curiosi.

La magia è il dono della Madre alle sue figlie per le loro figlie.
La discendenza magica è quasi esclusivamente matrilineare, ovvero la figlia di una strega sarà a sua volta una strega.

Ci sono rari casi in cui può saltare una o due generazioni ma mai di più.

Stregoni di genere maschile sono estremamente rari.

La magia si manifesta in incantesimi grazie ad una sintesi di canto e gesti che canalizzano l'energia dalla Sorgente.

\subsection{Le caratteristiche degli incantesimi}\index{Le caratteristiche degli incantesimi}\label{caratteristicheincantesimi}

La descrizione di ciascun incantesimo inizia con un blocco di informazioni che comprende il nome dell'incantesimo, il livello, il Verbo di Magia, i Punti Azione di lancio, il Costo, gittata e durata dell'incantesimo, la descrizione dell'incantesimo e gli effetti aggiuntivi che si possono avere per Margine di Successo.

Quando un personaggio lancia qualsiasi incantesimo, si usano le seguenti regole base indipendentemente dall'effetto dell'incantesimo.

\subsubsection{Tempo di Lancio}\index{Tempo di Lancio Incantesimi}\label{magietempodilancio}\index{Incantesimi, Azioni per lanciare}

La maggior parte degli incantesimi possono essere lanciati con 6 Punti Azioni. Alcuni incantesimi richiedono un'Azione Immediata, una Azione di Reazione o molto più tempo per essere lanciati.

\textbf{Azione Immediata}

Un incantesimo lanciato con un'Azione Immediata è particolarmente rapido. Puoi usare un'Azione Immediata durante il tuo round per lanciare l'incantesimo che sia Immediato, purché tu non abbia già effettuato un'Azione Immediata durante il tuo round.

\textbf{Reazioni}

Alcuni incantesimi possono essere lanciati come Reazioni. Questi incantesimi richiedono una frazione di secondo per essere creati e possono essere lanciati in risposta a un evento. Se un incantesimo può essere lanciato come reazione, la descrizione dell'incantesimo ti dice esattamente quando puoi farlo. Devi avere a disposizione una Azione di Reazione e non averla già usata.

\textbf{Tempo di Lancio Più Lungo}

Certi incantesimi richiedono più tempo per essere lanciati: minuti o addirittura ore. Quando lanci un incantesimo con tempo di lancio più lungo di 10 Punti Azioni ogni round successivo al primo si considera usato nel lancio dell'incantesimo. Per quei round è come se dovessi mantenere la Concentrazione.

Nel round finale, quando il tempo di lancio è esaurito, tiri 2d10 per determinare il segmento di lancio dell'incantesimo.

\subsection{I Fondamenti della Magia}

\subsubsection{I Verbi}

La magia viene organizzata in Verbi di Potere che la strega apprende con l'esperienza e l'uso.

I Verbi sono:\\


- \textbf{Attaccare} (Volontà)\\
- \textbf{Creare} (Corpo)\\
- \textbf{Muovere} (Mente)\\
- \textbf{Proteggere} (Mente)\\
- \textbf{Distruggere} (Mente)\\
- \textbf{Riparare} (Volontà)\\
- \textbf{Alterare} (Corpo)\\

\textit{Attaccare}: ciò che causa danno al Vigore dell'obiettivo tramite il Nome.\\
\textit{Creare}: ciò che permette di creare il Nome\\
\textit{Muovere}: ciò che consente di spostare il Nome\\
\textit{Proteggere}: ciò che permette di proteggere, ridurre, schermare il Nome\\
\textit{Distruggere}: ciò che causa danno non al Vigore dell'obiettivo tramite il Nome\\
\textit{Riparare}: ciò che rinsalda, rinforza, rigenera il Vigore del Nome\\
\textit{Alterare}: ciò che altera e modifica il Nome

\subsubsection{I Nomi}

La Strega applica i Verbi ai Nomi:\\


- \textbf{Elemento} \\
- \textbf{Corpo}\\
- \textbf{Mente}\\
- \textbf{Spirito}\\


\textit{Elemento}: può essere espresso come Fuoco, Suono, Elettricità, Energia Positiva, Energia Negativa, Freddo.

- Energia Positiva: fa danno al Vigore dei Diavoli e di chi è marchiato con il nome della Bestia o di Satana

- Energia Negativa: fa danno al Vigore di tutte le creature viventi\\

\textit{Corpo}: ciò che riguarda il corpo fisico di una creatura.\\

\textit{Mente}: ciò che agisce sulla mente di una creatura\\

\textit{Spirito}: ciò che agisce sullo spirito e anima di una creatura. Questo Nome è accessibile solo ai Nefilim

Gli Incantesimi vengono poi codificati abbinando Verbi e Nomi.

\subsubsection{Gittata}\index{Gittata}\label{magiegittata}\index{Incantesimi, Gittata}

Il bersaglio di un incantesimo deve essere nella gittata dell'incantesimo. Per un incantesimo come Attacco Elementale, il bersaglio è una creatura. Se l'area di effetto è una sfera il bersaglio è il punto nello spazio da cui la sfera di fuoco esplode. La maggior parte degli incantesimi hanno una gittata espressa in metri. Alcuni incantesimi possono prendere a bersaglio solo una creatura (te compreso) con cui sei in contatto fisico. Altri incantesimi, come un incantesimo di protezione, agiscono solo su di te: questi incantesimi hanno come gittata personale. Un incantesimo che ha come area di effetto "un alleato" può essere lanciato anche su se stesso.

Gli incantesimi che creano coni o linee di effetto che originano da te, hanno anch'essi gittata personale, a indicare che sei tu il punto di origine dell'effetto dell'incantesimo (vedi "Aree di Effetto" più avanti in questo capitolo).

\subsubsection{Lanciare Incantesimi in Armatura}\index{Lanciare Incantesimi in Armatura}\label{magielanciareincantesimiinarmatura}\index{Incantesimi, in Armatura}

Data la concentrazione mentale e i gesti precisi richiesti l'armatura distrae e sbilancia i flussi. La Prova di Incantamento subisce le Penalità di Competenza indicate dalla armatura.

La strego aumentando di 1 il PA di lancio può ridurre di uno la Penalità data dall'armatura.

\subsubsection{Durata}\index{Durata Incantesimi}\label{magiedurata}\index{Incantesimi, Durata}

La durata di un incantesimo è la lunghezza di tempo per cui esso persiste. La durata può essere espressa in round, minuti, ore o addirittura anni. Alcuni incantesimi specificano che i loro effetti durano finché l'incantesimo non viene dissolto o distrutto. Un incantesimo può essere interrotto dal proprio incantatore come azione immediata.\index{Interrompere un proprio incantesimo}

Qualora un margine di successo raddoppi la durata si intente sempre riferita alla durata iniziale. Es. se la durata è 2 ore dopo il primo raddoppio diventa 4 ore, con il secondo diventa di 6 ore e poi 8 ore..\index{Successo critico magico sulla durata}

\begin{itemize}
	
\item
\textit{Istantanea}
	
Molti incantesimi sono istantanei. L'incantesimo ferisce, cura, crea o altera una creatura o un oggetto in modo che non possa essere dissolto, dato che la sua magia esiste solo per un istante.
	
\item
	
\textit{Concentrazione}\index{Concentrazione}\index{Incantesimi, Durata Concentrazione}
	
Alcuni incantesimi richiedono che tu mantenga la concentrazione per tenerne la magia attiva. Se non puoi mantenere la concentrazione, l'incantesimo avrà fine. Se un incantesimo deve essere mantenuto tramite concentrazione, la cosa è indicata alla voce Durata, l'incantesimo specifica quanto a lungo vi potrai mantenere la concentrazione. Puoi terminare la concentrazione in qualsiasi momento usando una Reazione.
	
Normali attività, come muoversi e attaccare, non interferiscono con la concentrazione. Mantenere la concentrazione costa 2 Punti Azione a round.

\end{itemize}

\subsubsection{Formulare un Incantesimo}

Per lanciare correttamente un incantesimo è necessario gesticolare e cantare in maniera particolare.

Un incantatore deve avere una mano libera per poter formulare l'incantesimo.

\subsubsection{Essere inabile o ucciso}\index{Essere inabile o ucciso per la Magia}\label{magieessereucciso}\index{Incantesimi, Inabile}

Se scendi sotto gli zero punti di Vigore tutti gli incantesimi su cui stai tenendo la concentrazione vengono interrotti.

\subsubsection{Bersagli}\index{Bersagli}\label{magiebersagli}\index{Incantesimi, Bersagli}

Un normale incantesimo richiede che tu scelga uno o più bersagli che siano affetti dalla sua magia. La descrizione dell'incantesimo ti dice se l'incantesimo prende a bersaglio creature, oggetti o un punto di origine per generare un'area di effetto (descritta di seguito). A meno che l'incantesimo non abbia un effetto percepibile, una creatura potrebbe non capire mai di essere stata bersaglio di un incantesimo. Un effetto come un fulmine crepitante è palese, ma un effetto più subdolo, come il tentativo di leggere i pensieri di una creatura, di solito non viene notato, a meno che l'incantesimo non dica altrimenti.

Lanciare un incantesimo è una azione che non passa inosservata. Solo un Margine di Successo di 3 o più nella Prova di Incantamento può celare la formulazione, se non avviene proprio davanti all'osservatore.

\textbf{Traiettoria Sgombra Verso il Bersaglio}\index{Incantesimi, vedere bersaglio}

\textbf{Per prendere come bersaglio una creatura od oggetto}, devi vederlo ed avere la traiettoria sgombera verso di essa, e quindi questa \textbf{non può trovarsi dietro una copertura completa}. Se piazzi un'area di effetto in un punto che non puoi vedere e un'ostruzione, come un muro, si trova tra di te e quel punto, il punto di origine si crea dal tuo lato più vicino dell'ostruzione (una Palla di Fuoco dietro una porta chiusa esplode al contatto con la porta dalla tua parte e non si manifesta oltre la porta).\index{Magia vedere il bersaglio}

\textbf{Prendere Te Stesso Come Bersaglio}\index{Se stesso come bersaglio}\index{Incantesimi, se stesso come bersaglio}

Se un incantesimo prende come bersaglio una creatura a tua scelta od un alleato, puoi scegliere anche te stesso, a meno che la creatura non debba essere ostile o sia specificato che non possa essere tu. Se ti trovi nell'area di effetto di un incantesimo lanciato da te, anche tu ne sarai influenzato.

%\begin{center}
%	\includegraphics[width=0.6\linewidth]{immagini/tarothanged.png}
%
%	\textit{Tarocchi - L'Impiccato}
%\end{center}

\subsubsection{Aree di Effetto}\index{Area di Effetto incantesimi}\label{magieareedieffetto}\index{Incantesimi, Area di effetto}

Diversi incantesimi coprono un'area, permettendogli di colpire più creature alla volta.

La descrizione di un incantesimo specifica la sua area di effetto, che di solito rientra in una di queste cinque forme: cilindro, cono, cubo, linea o sfera. Ogni area di effetto ha un punto di origine, un luogo da cui si manifesta l'energia dell'incantesimo. Le regole per ciascuna forma specificano come posizionare il suo punto di origine. Di solito il punto di origine è un punto nello spazio, ma alcuni incantesimi hanno un'area la cui origine è una creatura o un oggetto.

\begin{itemize}
\item
\textit{\textbf{Cilindro}}: il punto di origine di un cilindro è il centro di un cerchio di specifico raggio, come indicato nella descrizione dell'incantesimo. Il cerchio deve essere sul pavimento o all'altezza dell'effetto dell'incantesimo. L'energia in un cilindro si espande in linee dritte dal punto di origine al perimetro del cerchio, formando la base del cilindro. L'effetto dell'incantesimo parte poi dal basso verso l'alto o dall'alto verso il basso, fino a una distanza uguale all'altezza del cilindro.Il punto di origine del cilindro è incluso nella sua area di effetto.
	
\item
\textit{\textbf{Cono}}: un cono si estende in una direzione a tua scelta dal suo punto di origine. Il diametro del un cono in un dato punto della sua lunghezza è uguale alla distanza di quel punto dal punto di origine. L'area di effetto di un cono specifica la sua lunghezza massima. Il punto di origine del cono non è incluso nella sua area di effetto, a meno che tu non decida altrimenti.
	
\item
\textit{\textbf{Cubo}}: selezioni il punto di origine di un angolo del cubo. Le dimensioni del cubo vengono espresse come lunghezza di ciascun suo spigolo. Il punto di origine del cubo non è incluso nella sua area di effetto, a meno che tu non decida altrimenti.
	
\item
\textit{\textbf{Linea}}: una linea si estende dal suo punto di origine in un percorso dritto per tutta la sua lunghezza e copre un'area definita dalla sua larghezza. Il punto di origine della linea non è incluso nella sua area di effetto, a meno che tu non decida altrimenti.
	
\item
\textit{\textbf{Sfera}}: selezioni il punto di origine di una sfera e la sfera si estenderà da quel punto fino ad incontrare un ostacolo insormontabile o la sua dimensione espresse nel raggio. La misura della sfera è indicata come raggio in metri che si estende da quel punto. Il punto di origine della sfera è incluso nella sua area di effetto.
	
Una sfera di fuoco che viene generata in una stanza di 9x9 m ne prenderà una buona parte e in una stanza di 6x6 m la riempirà tutta. In una stanza di 3x3 m se ha modo di uscire da una porta od una finestra continuerà la sua esplosione fino ad arrivare ai 6 metri di raggio. sfera di fuoco in un corridoio di 3x3 m lo saturerà per 3 metri avanti e indietro dal punto di origine.
	
\end{itemize}

\subsubsection{Rarità degli Incantesimi}\index{Rarità degli Incantesimi}\label{magieraritaincantesimi}\index{Incantesimi, Rarita' incantesimi}

Su alcuni Incantesimi è indicata la Rarità ovvero quanto è probabile trovare questo incantesimo o quanto può essere conosciuto. La rarità dipende non solo dal livello stesso dell'incantesimo, ovviamente gli incantesimi più potenti sono anche i più rari, ma anche da quanto normalmente sono diffusi e conosciuti nella lista. Il Narratore userà questa scala per valutare cosa può essere trovato più facilmente: Comune (70\%) - Non Comune (23\%) - Raro (4\%) - Molto Raro (2\%) - Leggendario (1\%), (1-70,71-93,94-97,98-99,100)

\subsubsection{Combinare Effetti Magici}\index{Combinare Effetti Magici}\label{magiecombinareeffettimagici}\index{Incantesimi, Combinare effetti}

Gli effetti di incantesimi diversi si sommano fino a che la loro durata si sovrappone. Tuttavia, gli effetti dello stesso incantesimo o che danno lo stesso bonus lanciato più volte sullo stesso bersaglio non si combinano. Sarà invece l'incantesimo più potente fra quelli lanciati, quello che ha avuto un MS migliore), ad applicarsi finché le durate si sovrappongono.

In caso di incantesimi istantanei gli effetti agiscono singolarmente se agiscono nel medesimo segmento di iniziativa. Es. Se vengo colpito da un attacco magico a segmento di iniziativa 4 e poi da un altro attacco magico a segmento di iniziativa 8 farò due distinti Prove di Corpo e relativa gestione del danno, se fossero nel medesimo segmento di iniziativa subirei solo quello che ha avuto più Margine di Successo.

\subsection{Regole di base}\index{Regole Base per la Magia}\label{magieregoledibase}\index{Incantesimi, Regole base}


\begin{itemize}

\item 
Quando una Strega vuole lanciare un incantesimo può scegliere se eseguire una Prova di Incantamento oppure prendere il risultato minimo.

Se esegue una Prova di Incantamento l'incantesimo riesce purché il Margine di Successo sia entro -3.

Se la Prova di Incantamento riesce con un margine di +3 o più può scegliere un effetto critico per ogni multiplo di 3 ottenuto nel successo della prova (con un MS di +3 hai 1 effetto, con +6 hai 2 successi, con +12 hai 4 successi...)

A seconda del Verbo utilizzato nell'incantesimo alla Prova di Incantamento usa il modificatore di Caratteristica indicato dal Verbo.

Se non esegue alcuna prova utilizzerà gli effetti descritti dall'incantesimo senza applicare alcun effetto di successo.

\textbf{Esempio}: Lucia vuole lanciare un Attacco Elementale e decide di fare una Prova di Incantamento per sperare in un risultato migliore.

Esegue una Prova di Incantamento (ha punteggio 5) e somma il modificatore di +2.
Lucia tira 2d10 ed ottiene 9, la Prova ha avuto un Margine di Successo di -2. Per poco non ha invece fallito l'incantesimo!

\item 

Ogni qual volta il suo punteggio di Incantamento raggiunge un multiplo di 3 impara ad usare un nuovo Verbo ed acquisisce 1 nuovo incantesimo di quel Verbo

\item 
Ogni incantesimo ha un Costo descritto nell'incantesimo stesso. Se non si paga in anticipo l'incantesimo non funziona correttamente.\\
Ogni volta che la Strega ottiene un effetto critico perde 1 punto Vigore\\
Dopo 8 ore di riposo la Prova di Incantamento torna normale

\item 
Ogni qual volta la Strega tira due 1 nella Prova di Incantamento non perde il punto Vigore e perde 1 Penalità alla Prova di Incantamento

\item 
Ogni 6 punti Vigore sacrificati può aggiungere 1 effetto critico all'incantesimo

\item 
La Strega può scegliere di usare diversi effetti critici per incantesimo (uno da 1 critico, uno da 2 critici ed uno da 3 critici) purché li abbia a disposizione

\item 
Una Strega può conoscere quanti incantesimi vuole

\item 
Se una Armatura o Scudo può fornire protezione da un Incantesimo viene indicato nella descrizione dell'incantesimo stesso, altrimenti si considera che non fornisca alcuna protezione

\end{itemize}

\begin{changemargin}{0.3cm}{0.3cm}\begin{narratore}
Concedete 1 Bonus quando il personaggio declama con perizia e trasporto il lancio dell'incantesimo. Se dice "\textit{Lancio una sfera di fuoco}" non otterrà vantaggi ma se con trasporto declama "\textit{Per il sacro fuoco dell'Eterno Altare! Possano gli Arcangeli distruggervi con le sacre fiamme. Bruciate indegni. Sfera di Fuoco!}" allora 1 Bonus alla Prova di Incantamento, ridurre il danno a Vigore o anche non aumentare la successiva Prova può starci.
\end{narratore}\end{changemargin}


\subsubsection{Fallimento Critico nella Prova di Incantamento}\index{Fallimento Critico nella Prova di Incantamento}\label{magiefallimentocriticonellaprovadimagia}\index{Incantesimi, Fallimento Prova di Incantamento}

Se nella Prova di Incantamento hai tirato due 0 hai avuto un fallimento critico. Indipendentemente dal Margine di Successo l'incantesimo fallisce.

Tira 2d10 e consulta la seguente tabella.

%\end{multicols}

\textbf{Tabella: Effetti Fallimento Critico magico}\index{Tabella Effetti Fallimento Critico Prova di Magia}

\medskip
{\small
	\begin{tabularx}{0.45\textwidth}{lX}
		\hline
		2 & Aumenti la condizione di Affaticato di 2 gradi\\
		3 & Per 1 giorno non sei più in grado di chiamare energia dalla Sorgente. Non puoi lanciare incantesimi se non facendo una Prova di Incantamento con un MS di +3\\
		4 & Manifesti una modifica corporea minore\\
		5 & Vieni investito da una roboante colonna di Fiamme. In un raggio di 3 metri intorno a te compreso, chiunque deve fare una Prova di Corpo per dimezzare o subire 1d6 a Vigore\\
		6 & Per 3 round sei sotto l'influenza dell'incantesimo Confusione\\
		7 & Sei paralizzato per 3 round\\
		8-9 & Vieni teletrasportato entro 3d10 metri in una direzione casuale\\
		10-11 & Diventi Invisibile ed incapace di parlare per 6 round\\
		12-13 &  Solo tu vieni avvolto da una cortina di oscurità magica impenetrabile per 6 round\\
		14-15 & Non riesci a parlare bene, sei balbuziente. Ogni lancio di incantesimi ti costringe a fare una Prova di Incantamento. Durata 3 round\\
		16 & Il prossimo incantesimo che lanci ha effetti se minimizzati\\
		17 & Il battito del tuo cuore è come il battito di un tamburo, si può sentire entro 50 metri\\
		18 & Ogni oggetto che tieni in mano ti cade a terra. Se non hai oggetti in mano capita ad una creatura a caso entro 6 metri.\\
		19 & Una incudine cade su una creatura a caso entro 6 metri da te. 3d6 di danno a Vigore, Prova di Corpo per dimezzare il danno.\\
		20 & Tutte le creature, escluso te, nel raggio di 6 metri da te subiscono 1d10 danni a Vigore.
\end{tabularx}}

\subsubsection{Imparare un nuovo incantesimo}

Una Strega che trova un incantesimo che non conosce deve farsi insegnare a formularlo. E' necessario conoscere il Verbo e riuscire in una Prova di Incantamento.

\begin{narratore}
In Dark Catacomb non esistono i libri di magia o le pergamene di incantesimo. Quando si trova qualcuna che conosce un nuovo incantesimo lo si deve imparare subito o tornare quando si può imparare
\end{narratore}

\subsubsection{Prova per Colpire con le Magie}\index{Tiro per Colpire con Incantesimi}\label{magietiropercolpireconlemagie}\index{Incantesimi, Tiro per Colpire}

Quando l'incantesimo dice che è necessario colpire l'avversario, la Strega deve effettuare una Prova di Incantamento, con modificatore di Caratteristica in base al Verbo usato opposta alla PDAD dell'avversario.

\medskip

Quando la magia è ad area non è necessario effettuare una Prova di Incantamento se non per difficili e specificate aree, ovvero si mira in una area ben circoscritta e si vuole evitare di colpire qualcuno con un incantesimo ad area.

\subsection{Resistere all'incantesimo}

Diversi incantesimi prevedono che l'obiettivo della magia possa fare una Prova di Corpo, o comunque di qualche Caratteristica per modificare gli effetti della magia.

Nella descrizione dell'incantesimo è indicato cosa fare.

\subsubsection{Distratto - Problemi nel lancio dell'incantesimo}\index{Distratto - Problemi nel lancio dell'incantesimo}\index{Distratto}\label{magiedistratto}

Se la Strega è severamente \textbf{Distratta}, cerca di nascondere il lancio della magia, è impedita, disturbata, è sanguinante, minacciata, è sotto attacco mentre cerca di lanciare un incantesimo deve effettuare la Prova di Incantamento con 2 Penalità.

\subsubsection{Concentrazione}\index{Colpito mentre concentrato}\index{Concentrazione}\label{magieconcentrazione}

Perdi la concentrazione su di un incantesimo se lanci un altro incantesimo che richieda concentrazione. Non ti puoi concentrare su due incantesimi alla volta. Interrompere la concentrazione costa una Reazione.

Se vieni colpito\index{Colpito mentre tieni la concentrazione} mentre sei concentrato su un incantesimo devi effettuare una Prova di Corpo o perdi la concentrazione.

Mantenere la Concentrazione costa 2 PA a round.\index{Costo PA concentrarsi}

\subsubsection{Influenzati da più Magie}\index{Influenzati da più Magie}\label{magieinfluenzatodapiumagie}

Quando un personaggio è influenzato da \textbf{due o più effetti magici} che danno lo stesso tipo di bonus, penalità o danno nello stesso segmento di iniziativa (protezione verso fuoco, bonus alla Prova d'Arme per Difendersi... , multiple palle di acido), si tiene conto solo di quella che ha Margine di Successo maggiore

Un personaggio che prende 2 Sfere di Fuoco nel medesimo segmento di Iniziativa farà la Prova di Corpo solo per quella con il Margine di Successo più alto, indipendentemente che sia quella con il danno maggiore. Se prende una Sfera di Fuoco in due tempi diversi del medesimo round farà due Prove di Corpo distinti prendendo il danno relativo.

\subsubsection{Definizioni obiettivi degli incantesimi}\index{Obiettivi degli incantesimi}\label{magiedefinizioniobiettivi}

Negli incantesimi sotto elencati troverete spesso i riferimenti alle tipologie di soggetti ed obiettivi influenzabili nonché a diverse tipologie di energia ed elementi.

- Le \textbf{Creature} \textbf{Naturali} sono Insetti, Rettili, Bestie, Umanoidi, Piante, Creature acquatiche, Mostruosità, Melme.

- Le \textbf{Creature} \textbf{Magiche} sono: Immondi (Demoni), Fatati, Spiriti, Non morti, Giganti, Celestiali, Elementali, Costrutti, Aberrazioni (tutto ciò che è alieno o innaturale) ed i  Draghi.
Se una Creatura Naturale ha poteri magici allora si considera anche come Creatura Magica. Una descrizione più completa di questi "categorie" la trovate nel Capitolo del Mostruario.

- \textbf{Elemento} comprende: Fuoco, Suono, Elettricità, Energia Positiva, Energia Negativa, Freddo.

\end{multicols}

\pagebreak

\section{Elenco Incantesimi}

\begin{multicols}{2}

%il danno subito a vigore e' solitamente pari ad un 1/3 del massimo danno causato

\subsection{Attaccare}

\textbf{Attacco Elementale I}\\
\textbf{Prerequisiti}: Volontà 4, Incantamento 3
\textbf{Verbo}: Attaccare (Volontà)\\
\textbf{Nome}: variabile\\
\textbf{Punti Azione}: 4\\
\textbf{Costo}: 2 Punti Vigore
\textbf{Gittata}: 9 metri\\
\textbf{Area di Effetto}: 1 bersaglio\\
\textbf{Durata}: istantanea\\
\textbf{Rarità}: Comune\\
\textbf{Descrizione}: focalizzi la magia della sorgente in un lampo energetico. Il personaggio può scegliere l'Elemento da usare.

Il bersaglio subisce 1d6 danni a Vigore.


\textbf{Effetti Critici}:\\
\textbf{Ogni 1 successo critico}: puoi influenzare una persona in più che non ne sia già influenzata\\
\textbf{Ogni 1 successo critico}: puoi aumentare la gittata di 9 metri\\
\textbf{Ogni 2 successo critico}: crei un altro lampo di energia che può colpire chi preferisci\\
\textbf{Ogni 2 successo critico}: aumenti di 1 il danno a Vigore causato per ogni dado di danno\\
\textbf{Con 3 successo critico}: il dado di danno a Vigore diventa 1d8\\

\textbf{Attacco Elementale II}\\
\textbf{Prerequisiti}: Volontà 6, Incantamento 6
\textbf{Verbo}: Attaccare (Volontà)\\
\textbf{Nome}: variabile\\
\textbf{Punti Azione}: 6\\
\textbf{Costo}: 4 Punti Vigore
\textbf{Gittata}: 18 metri\\
\textbf{Area di Effetto}: 1 bersaglio\\
\textbf{Durata}: istantanea\\
\textbf{Rarità}: Comune\\
\textbf{Descrizione}: focalizzi la magia della sorgente in un lampo energetico. Il personaggio può scegliere l'Elemento da usare.

Il bersaglio subisce 2d6 danni a Vigore.


\textbf{Effetti Critici}:\\
\textbf{Ogni 1 successo critico}: puoi influenzare una persona in più che non ne sia già influenzata\\
\textbf{Ogni 1 successo critico}: puoi aumentare la gittata di 9 metri\\
\textbf{Ogni 2 successo critico}: crei un altro lampo di energia che può colpire chi preferisci\\
\textbf{Ogni 2 successo critico}: aumenti di 1 il danno a Vigore causato per ogni dado di danno\\
\textbf{Con 3 successo critico}: il dado di danno a Vigore diventa 2d8\\

\textbf{Attacco Elementale III}\\
\textbf{Prerequisiti}: Volontà 8, Incantamento 9
\textbf{Verbo}: Attaccare (Volontà)\\
\textbf{Nome}: variabile\\
\textbf{Punti Azione}: 8\\
\textbf{Costo}: 6 Punti Vigore
\textbf{Gittata}: 18 metri\\
\textbf{Area di Effetto}: 2 bersaglio\\
\textbf{Durata}: istantanea\\
\textbf{Rarità}: Comune\\
\textbf{Descrizione}: focalizzi la magia della sorgente in un lampo energetico. Il personaggio può scegliere l'Elemento da usare.

I bersagli, distinti, subiscono 3d6 danni a Vigore.


\textbf{Effetti Critici}:\\
\textbf{Ogni 1 successo critico}: puoi influenzare una persona in più \\
\textbf{Ogni 1 successo critico}: puoi aumentare la gittata di 9 metri\\
\textbf{Ogni 2 successo critico}: crei un altro lampo di energia che può colpire chi preferisci\\
\textbf{Ogni 2 successo critico}: aumenti di 1 il danno a Vigore causato per ogni dado di danno\\
\textbf{Con 3 successo critico}: il dado di danno a Vigore diventa 3d8\\



\textbf{Attacco Elementale IV}\\
\textbf{Prerequisiti}: Volontà 10, Incantamento 12
\textbf{Verbo}: Attaccare (Volontà)\\
\textbf{Nome}: variabile\\
\textbf{Punti Azione}: 9\\
\textbf{Costo}: 8 Punti Vigore
\textbf{Gittata}: 9 metri\\
\textbf{Area di Effetto}: 2 bersaglio\\
\textbf{Durata}: istantanea\\
\textbf{Rarità}: Comune\\
\textbf{Descrizione}: focalizzi la magia della sorgente in un lampo energetico. Il personaggio può scegliere l'Elemento da usare.

I bersagli, distinti, subiscono 4d6 danni a Vigore.


\textbf{Effetti Critici}:\\
\textbf{Ogni 1 successo critico}: puoi influenzare una persona in più \\
\textbf{Ogni 1 successo critico}: puoi aumentare la gittata di 9 metri\\
\textbf{Ogni 2 successo critico}: crei un altro lampo di energia che può colpire chi preferisci\\
\textbf{Ogni 2 successo critico}: aumenti di 1 il danno a Vigore causato per ogni dado di danno\\
\textbf{Con 3 successo critico}: il dado di danno a Vigore diventa 4d8\\

\subsection{Creare}

\textbf{Crea Elemento I}\\
\textbf{Prerequisiti}: Corpo 4, Incantamento 3
\textbf{Verbo}: Creare (Corpo)\\
\textbf{Nome}: variabile\\
\textbf{Punti Azione}: 4\\
\textbf{Costo}: 2 Punti Vigore
\textbf{Gittata}: 9 metri\\
\textbf{Area di Effetto}: 1 cubo entro 10x10cm\\
\textbf{Durata}: 1 round\\
\textbf{Rarità}: Comune\\
\textbf{Descrizione}: crei un oggetto di dimensione entro 10cm x 10cm di legno, acqua, ferro, ghiaccio.
L'oggetto la cui forma è a tua scelta anche se grezza, appare entro 9 metri da te. 
L'oggetto non può avere parti mobili ed è unica come forma (non possono essere più oggetti). L'oggetto risponde alle normali leggi della fisica.

L'oggetto scompare alla fine dell'incantesimo o quando annullato con una Reazione.


\textbf{Effetti Critici}:\\
\textbf{Ogni 1 successo critico}: puoi creare un volume di 10x10cm in più od aumentare a 20x20cm il cubo iniziale\\
\textbf{Ogni 1 successo critico}: puoi aumentare la gittata di 9 metri\\
\textbf{Ogni 2 successo critico}: l'oggetto dura un round in più\\
\textbf{Con 3 successo critico}: l'oggetto puù essere creato anche con fuoco o da elettricità (1d6 di danno a contatto).\\


\textbf{Crea Elemento II}\\
\textbf{Prerequisiti}: Corpo 6, Incantamento 6
\textbf{Verbo}: Creare (Corpo)\\
\textbf{Nome}: variabile\\
\textbf{Punti Azione}: 6\\
\textbf{Costo}: 4 Punti Vigore
\textbf{Gittata}: 9 metri\\
\textbf{Area di Effetto}: 1 cubo entro 30x30cm\\
\textbf{Durata}: 2 round\\
\textbf{Rarità}: Comune\\
\textbf{Descrizione}: crei un oggetto di dimensione entro 30cm x 30cm di legno, acqua, ferro, ghiaccio.
L'oggetto la cui forma è a tua scelta anche se grezza, appare entro 9 metri da te. 
L'oggetto non può avere parti mobili ed è unica come forma (non possono essere più oggetti). L'oggetto risponde alle normali leggi della fisica.

L'oggetto scompare alla fine dell'incantesimo o quando annullato con una Reazione.


\textbf{Effetti Critici}:\\
\textbf{Ogni 1 successo critico}: puoi creare un volume di 20x20cm in più od aumentare a 40x40cm il cubo iniziale\\
\textbf{Ogni 1 successo critico}: puoi aumentare la gittata di 9 metri\\
\textbf{Ogni 2 successo critico}: l'oggetto dura un round in più\\
\textbf{Con 3 successo critico}: l'oggetto può essere creato anche con fuoco o da elettricità (2d6 di danno a contatto).\\

\textbf{Crea Elemento III}\\
\textbf{Prerequisiti}: Corpo 8, Incantamento 9
\textbf{Verbo}: Creare (Corpo)\\
\textbf{Nome}: variabile\\
\textbf{Punti Azione}: 8\\
\textbf{Costo}: 6 Punti Vigore
\textbf{Gittata}: 9 metri\\
\textbf{Area di Effetto}: 1 cubo entro 50x50cm\\
\textbf{Durata}: 3 round\\
\textbf{Rarità}: Comune\\
\textbf{Descrizione}: crei fino a due oggetti entro volume di 50cm x 50cm di legno, acqua, ferro, ghiaccio.
L'oggetto la cui forma è a tua scelta anche se grezza, appare entro 9 metri da te. 
L'oggetto può avere 2 parti mobili. L'oggetto risponde alle normali leggi della fisica.

L'oggetto scompare alla fine dell'incantesimo o quando annullato con una Reazione.

\textbf{Effetti Critici}:\\
\textbf{Ogni 1 successo critico}: puoi creare un volume di 30x30cm in più od aumentare a 60x60cm il cubo iniziale\\
\textbf{Ogni 1 successo critico}: puoi aumentare la gittata di 9 metri\\
\textbf{Ogni 2 successo critico}: l'oggetto dura un round in più\\
\textbf{Con 3 successo critico}: l'oggetto può essere creato anche con fuoco o da elettricità (3d6 di danno a contatto).\\


\textbf{Crea Elemento IV}\\
\textbf{Prerequisiti}: Corpo 10, Incantamento 13
\textbf{Verbo}: Creare (Corpo)\\
\textbf{Nome}: variabile\\
\textbf{Punti Azione}: 9\\
\textbf{Costo}: 8 Punti Vigore
\textbf{Gittata}: 9 metri\\
\textbf{Area di Effetto}: 1 cubo entro 100x100cm\\
\textbf{Durata}: 4 round\\
\textbf{Rarità}: Comune\\
\textbf{Descrizione}: crei uno o più oggetti entro un volume di dimensione entro 100cm x 100cm di legno, acqua, ferro, ghiaccio.
Gli oggetti la cui forma è a tua scelta anche se grezza, appaiono entro 9 metri da te. 
L'oggetto può avere fino a 3 parti mobili. L'oggetto risponde alle normali leggi della fisica.

L'oggetto scompare alla fine dell'incantesimo o quando annullato con una Reazione.

\textbf{Effetti Critici}:\\
\textbf{Ogni 1 successo critico}: puoi creare un volume di 50x50cm in più od aumentare a 120x120cm il cubo iniziale\\
\textbf{Ogni 1 successo critico}: puoi aumentare la gittata di 9 metri\\
\textbf{Ogni 2 successo critico}: l'oggetto dura un round in più\\
\textbf{Con 3 successo critico}: l'oggetto può essere creato anche con fuoco o da elettricità (3d6 di danno a contatto).\\


\textbf{Colpo di Genio I}\\
\textbf{Prerequisiti}: Corpo 6, Incantamento 6
\textbf{Verbo}: Creare (Corpo)\\
\textbf{Nome}: Mente\\
\textbf{Punti Azione}: 6\\
\textbf{Costo}: 4 Punti Vigore
\textbf{Gittata}: personale\\
\textbf{Area di Effetto}: te stesso\\
\textbf{Durata}: 1 round\\
\textbf{Rarità}: Comune\\
\textbf{Descrizione}: un lampo di genio ti illumina. 

Ottieni 2 Bonus alla prossima Prova di Competenza entro la fine del round successivo.


\textbf{Effetti Critici}:\\
\textbf{Ogni 1 successo critico}: la durata diviene 2 round\\


\textbf{Colpo di Genio II}\\
\textbf{Prerequisiti}: Corpo 10, Incantamento 12
\textbf{Verbo}: Creare (Corpo)\\
\textbf{Nome}: Mente\\
\textbf{Punti Azione}: 8\\
\textbf{Costo}: 8 Punti Vigore
\textbf{Gittata}: 9 metri\\
\textbf{Area di Effetto}: una creatura\\
\textbf{Durata}: 1 round\\
\textbf{Rarità}: Comune\\
\textbf{Descrizione}: una creatura a tua scelta ha un'ottima intuizione e ottiene 2 Bonus alla prossima Prova di Competenza.


\textbf{Effetti Critici}:\\
\textbf{Ogni 1 successo critico}: la durata diviene 2 round\\
\textbf{Ogni 1 successo critico}: puoi aumentare la gittata di 9 metri\\
\textbf{Ogni 2 successo critico}: puoi influenzare una creatura in più\\




\end{multicols}

\pagebreak

\section{Movimento e Trasporto}\index{Trasporto}\index{Movimento}

\label{movimento-e-trasporto}

\begin{changemargin}{0.3cm}{0.3cm}\begin{enfasi}{
Quando non puoi più correre, cammina veloce; quando non puoi più camminare veloce, cammina; quando non puoi più camminare, usa il bastone; però non trattenerti mai. (Madre Teresa di Calcutta)}
\end{enfasi}\end{changemargin}\medskip


\begin{multicols}{2}
	
Il movimento si può distinguere in base a quale situazione si applica.
	
	\medskip
	
	\begin{itemize}
		\item Tattico, quando si combatte, si usano le distanze precise, griglia ed i quadretti di 1,5 metro di lato
		\item Locale, per esplorare una zona, misurato in metri al minuto.
		\item Via Terra, per muoversi da un posto all'altro, misurato in km all'ora o al giorno.
	\end{itemize}
	
	\subsection{Tipi di Movimento}\label{tipodimovimento}
	
	Quando si muovono nelle differenti situazioni di movimento (Tattico, Locale Via Terra), le creature generalmente camminano o corrono.
	
	\textbf{Camminare}:\index{Camminare} Camminare rappresenta un movimento non affrettato ma deciso di circa 4 km all'ora per un umano senza Ingombro.
	
	\textbf{Correre}\index{Correre}: Significa muoversi di circa 12 km all'ora per un umano.
	
	Il personaggio che corre ha 2 Penalità al PDAA ed al PDAD.
	
	Correre come azione di movimento raddoppia la velocità di movimento e non la triplica. Solo in situazioni di non combattimento la corsa triplica il movimento (movimento locale, via terra)
	
	\subsection{Tabella: Movimento e Distanza e Velocita': a Piedi}\index{Movimento a Piedi}\index{Tabella Movimento e Distanza e Velocità : a Piedi}
	
	Questa tabella mostra i valori base di movimento a terra in situazioni di non combattimento.
	
	\bigskip
	
	\begin{tabularx}{0.43\textwidth}{lccc}
		\multirow{2}*{Tipo di movimento} &
		\multicolumn{3}{c}{Movimento}  \\
		\cmidrule(lr){2-4} & 6m  & 9m & 12m  \\
		\midrule
		\multicolumn{4}{c}{\textbf{Movimento Tattico)}}\\
		Camminare  & 6m & 9m & 12m  \\
		Correre (x2) & 12m  & 18m  & 24m  \\
		\multicolumn{4}{c}{\textbf{Un minuto (locale)}} \\
		Camminare & 36m  & 54m  & 72m \\
		Correre (x3) & 108m & 162m & 216m \\
		\multicolumn{4}{c}{\textbf{Un'ora (via terra)}} \\
		Camminare  & 3km  & 4km  & 6km  \\
		Correre (x3) & 9km  & 12km & 18km \\
		\multicolumn{4}{c}{\textbf{Un giorno (via terra)}}  \\
		Camminare  & 24km & 32km & 54km \\
	\end{tabularx}
	
	
\subsection{Movimento Tattico}\index{Movimento Tattico}\label{movimentotattico}
	
Durante un combattimento si utilizza il Movimento tattico.
Le distanze vengono misurate in quadretti da un 1,5 metri, il movimento è gestito tramite i Punti Azione dedicata al movimento.
	
Con 1 PA il personaggio può percorrere 1.5 metri. Può spostarsi più volte nel round consumando i relativi Punti Azione.
	
Può anche usare un PA per effettuare uno scatto ovvero spostandosi fino a 3 metri per PA usato. Incappa così però nelle penalità per chi corre.
		
\subsubsection{Movimento Ostacolato}\index{Terreno difficile}\label{terrenodifficile}
	
Terreno difficile, ostacoli o scarsa visibilità possono impedire i movimenti. Quando il movimento è ostacolato ci si muove a metà della velocità, sono necessarie 2 Punti Azione per coprire 1.5 metri.
	
Se esiste più di una condizione particolare, aggiungere tra loro tutti i costi aggiuntivi applicabili, ovvero se un terreno è difficile e ci si muove a carponi significa che ogni 1.5 costa 3 PA.

Ogni condizione di maggiore difficoltà aumenta il costa in PA per 1.5m di 1.
	
In alcune situazioni il movimento è talmente ostacolato\index{Movimento quasi impossibile} che la distanza percorribile e' minima. In tal caso si possono utilizzare tutte e 10 i PA per muoversi di solo 1.5 metri in qualsiasi direzione.
	
Non applicate questa regola per attraversare terreni impraticabili o per muoversi quando non è possibile farlo in alcun modo.
	
Non si può \textbf{Scattare} (Correre) o \textbf{Caricare} \index{Carica su Terreno difficile} \index{Scatto su terreno difficile}agevolmente attraverso un \textbf{percorso che ostacola il movimento}, ovvero terreno difficile. Il giocatore può tentare una Prova di Acrobatica con un MS di +3 per riuscire a caricare o correre
	
\textbf{Muoversi da prono}\index{Muoversi da prono}\index{Muoversi a carponi}, Nuotare o Strisciare\index{Strisciare} è considerato terreno difficile.
	
Il terreno dove sono presenti dei corpi di creature si considera difficile\index{Muoversi su corpi}.
	
\subsubsection{Attraverso i nemici}\index{Attraverso i nemici}\index{Attraversare quadretti occupati}\label{attraversonemici}
	
\index{Attraversare nemici}\index{Movimento attraverso}Un personaggio può \textbf{attraversare} ma non sostare in \textbf{una zona occupata} da un compagno.

Per attraversare il terreno dove c'è una creatura ostile è necessario eseguire una Prova di Corpo contrapposta a quello dell'avversario al quale si vuole \textbf{attraversare} il terreno. Se si fallisce la prova si torna nel quadretto più prossimo che non sia della creatura che si voleva attraversare od occupato. Attraversare uno spazio occupato da una creatura avversario costa, oltre ai PA per movimento fatto 1 PA in più per creatura attraversata.

	
\subsubsection{Scambiarsi di posto}\index{Scambiarsi di posto}\index{Scambiarsi di posto}\label{scambiarsidiposto}
	
Un personaggio a contatto con un altra creatura può usare 2 PA per \textbf{scambiarsi di posto}, se la creatura è ostile è necessario una Prova di Corpo contrapposta per riuscire a scambiarsi. Per ogni \textbf{taglia di differenza}, chi ha quella maggiore, prende 1 Bonus 
	
\subsection{Movimento Locale}\index{Movimento Locale}\label{movimentolocale}
	
I personaggi che esplorano una zona usano il movimento locale, misurato in metri al minuto.
	
In queste situazione non è fondamentale misurare la distanza in maniera precisa ma appena la situazione diventa "problematica" o richiede attenzione la mappa si converte in movimento tattico, quadrettata e misurata.
	
\medskip
	
\begin{itemize}
\item
Camminare: Un personaggio può camminare senza problemi in scala locale per 8 ore al giorno.
\item
Correre: Un personaggio può Correre per un numero di minuti pari al triplo del proprio punteggio di Costituzione su scala locale senza bisogno di riposarsi (minimo un round).
\end{itemize}


\subsection{Movimento Via Terra}\index{Movimento Via Terra}\label{movimentoviaterra}

I personaggi che percorrono lunghe distanze usano il movimento via terra. Il movimento via terra è misurato in ore o giorni. Un giorno rappresenta 8 ore di tempo di viaggio reale. Per imbarcazioni a remi, un giorno significa remare per 10 ore. Per navi a vela rappresenta 24 ore di movimento.

\textbf{Camminare}\index{Camminare}\label{camminare}

Si può camminare per 8 ore in un giorno di viaggio senza problemi.

Camminare più a lungo può sfinire (vedi Marcia forzata, sotto).

\textbf{Andare Veloci}\index{Andare Veloci}\label{andareveloci}

Si può andare veloci (movimento*2) per 1 ora senza problemi. Andare veloci per una seconda ora compresa tra due cicli di sonno provoca 1 Danno a Vigore Non Letale, e ogni ora aggiuntiva provoca il doppio dei danni subiti nell'ora precedente. Un personaggio che subisce Danni Non Letali da andatura veloce è considerato Affaticato.

Un personaggio Affaticato non può Correre o Caricare.

\textbf{Correre}\index{Correre}\label{correre}

Non è possibile Correre per un lungo periodo di tempo. Tentativi di Correre e riposarsi a cicli funzionano come Andare Veloci.

\textbf{Terreno}\index{Terreno}\label{terreno}

Il terreno su cui si viaggia influenza quanta distanza viene percorsa in un'ora o in un giorno (vedi Tabella: Terreno e Movimento Via Terra). Una strada maestra è una strada principale, dritta e lastricata. Una strada comune è solitamente un cammino impervio. Un sentiero è come una strada comune tranne per il fatto che permette di viaggiare solo in fila indiana e non avvantaggia un gruppo che viaggia con veicoli. Un terreno libero è una zona selvaggia senza sentieri segnati.

\bigskip

\textbf{Opzionale - Tabella: Terreno e Movimento Via Terra}\index{Opzionale - Tabella Terreno e Movimento Via Terra}

Nella tabella sono indicati i moltiplicatori per la distanza percorsa.

\medskip

\begin{tabularx}{0.45\textwidth}{XXXX}
\textbf{Terreno}  & \textbf{Strada maestra} & \textbf{Strada comune} & \textbf{Sentiero non battuto}\\
\toprule
Brughiera & x1  & x1 & x3/4\\
Collina & x1  & x3/4 & x1/2\\
Deserto Sabbioso  & x1  & x1/2 & x1/2\\
Foresta & x1  & x3/4 & x1/2\\
Giungla & x1  & x3/4 & x1/4\\
Montagna  & x3/4  & x3/4 & x1/2\\
Palude  & x1  & \texttimes 3/4 & \texttimes 1/2\\
Pianura & x1  & \texttimes 3/4 & \texttimes 1/2\\
Tundra Ghiacciata & x1  & \texttimes 3/4 & \texttimes 3/4\\
\end{tabularx}

\bigskip

\textbf{Marcia Forzata}\index{Marcia Forzata}\label{marciaforzata}

In un giorno di cammino normale, si può camminare per 8 ore. Il resto del giorno viene sfruttato per fare e disfare il campo, riposarsi e mangiare.

Se si cammina di più di 8 ore è necessario effettuare un Tiro Salvezza su Tempra a difficoltà 11 +1 per ogni giorno consecutivo di marcia forzata o si diventa Affaticati. Il Tiro Salvezza viene effettuato ogni 2 ore oltre le 8 di cammino.

La marcia forzata può essere tenuta per un numero di giorni pari al valore di Costituzione+1 prima di incorrere nell'Affaticamento indipendentemente dall'esito del Tiro Salvezza.


\textbf{Movimento in sella}\index{Movimento in sella}\label{movimentoacavallo}

Una cavalcatura che porta un cavaliere può muoversi con andatura veloce. Tuttavia, i danni che subisce sono danni normali invece che non letali. Può anche essere costretta a una marcia forzata, ma le sue prove di Corpo falliscono automaticamente e di nuovo i danni che subisce sono danni normali. Anche le cavalcature sono considerate Affaticate quando subiscono danni da andatura veloce o marcia forzata.

\end{multicols}

%\medskip
%\begin{center}
%\includegraphics[height=0.3\linewidth]{immagini/carretto.png}
%\end{center}

\subsection{Tabella: Cavalcature e Veicoli}\index{Cavalcature}\index{Veicoli}\index{Tabella Cavalcature e Veicoli}\index{Cavallo movimento}\index{Movimento al giorno su cavallo}

\medskip

\label{tabella-cavalcature-e-veicoli}\index{Cane}\index{Pony}\index{Carretto}\index{Zattera}\index{Barca}\index{Nave}

\begin{tabularx}{0.95\textwidth}{lXXX}
\textbf{Cavalcatura o Veicolo} & \textbf{Ingombro trasportato} & \textbf{All'ora} & \textbf{Al giorno}\\
\toprule
Cane da Galoppo*  & 4 	& 6km & 36km  \\
Cavallo Leggero*  	& 10  & 8km & 48km  \\
Cavallo Pesante* 	& 25  & 7km & 42km  \\
Pony* 				& 8		& 5km & 30km  \\
\textbf{Imbarcazione} &  &  & \\
\toprule
Zattera o Chiatta (pertica o rimorchio)  & 1000kg & 0.75km & 7.5km \\
Barcone (a Remi)**  & 2000kg & 1.5km  & 15km  \\
Barca a Remi**  & 1000kg & 2.25km & 22.5km  \\
Nave a Vela (vele)  & 4000kg & 3km  & 72km  \\
Nave da Guerra (vele e remi) & 8000kg & 3.75km & 90km  \\
Nave Lunga (vele e remi)  & 1200kg  & 4.5km  & 108km \\
Galea (remi e vele) & 15000kg & 6km  & 144km \\
\end{tabularx}

\begin{multicols}{2}

\bigskip

\textbf{*I quadrupedi}, come i cavalli, possono portare carichi superiori rispetto ai personaggi (x4). Vedi Capacità di Trasporto per maggiori informazioni.

Una cavalcatura può portare in groppa una creatura solo se di taglia inferiore alla sua. Il movimento al giorno si intende per 6 ore di cavalcata, oltre queste ore la cavalcatura si sfianca richiedendo un giorno intero di riposo.\index{Ore di cavalcata al giorno}

**Zattere, chiatte e barconi sono usati su laghi e fiumi. Se seguono la corrente, sommare la velocità della corrente (di solito 4,5 km/h) alla velocità dell'imbarcazione. Oltre a essere spinta con i remi per 10 ore, l'imbarcazione può anche essere trasportata dalla corrente per altre 14 ore, se qualcuno è in grado di guidarla, e quindi si aggiungono altri 63 km alla distanza giornaliera percorsa. Queste imbarcazioni non possono essere spinte a remi contro una corrente molto forte, ma possono essere tirate controcorrente da animali da soma sulla riva.

\textbf{Bardature da Cavalcatura}\index{Bardature da Cavalcatura}\index{Armature da Cavallo}

Una cavalcatura può essere bardata con un armatura. Genericamente un armatura leggera conferirà 1 Bonus alla PDAD , una armatura Media concederà 2 Bonus alla PDAD riducendo il movimento del un 25\%, una armatura Pesante darà 3 Bonus alla PDAD abbassando il movimento del 33\%.

\subsection{Fuga e Inseguimento}\index{Fuga}\index{Inseguimento}\label{fugainseguimento}

Nel movimento round per round è impossibile per un personaggio lento sfuggire ad un personaggio veloce senza qualche tipo di aiuto. Allo stesso modo, non è un problema per un personaggio veloce sfuggire ad uno più lento.

Quando la velocità per PA dei due personaggi coinvolti è uguale effettuate tre Prove contrapposte di Corpo, che ne vince di più riesce a fare perdere le proprie tracce o agguantare il fuggitivo.

\subsection{Capacita' di Carico e Trasporto: Ingombro}\index{Capacità di Carico}\index{Ingombro}

\label{sec:capacita-di-carico-e-trasporto-ingombro}

\subsubsection{Peso e Ingombro}\index{Ingombro}\index{Peso}

Portare tesori, pezzi di drago, armature complete per non parlare di armi sproporzionate o arieti da sfondamento, carrucole e paranco, rendono difficile il movimento.

Quando valutate il peso trasportato ragionate anche sull'ingombro!
Portare un rotolo di 12 metri x 6 metri di seta non è una attività fisica impegnativa, saranno pochi chili, ma l'ingombro è tale da non poter permettere ulteriore carico.

Ci possono essere oggetti leggeri ma estremamente ingombranti (tronchi cavi, tappeti di seta appunto..) oppure piccoli ma pesantissimi (sfere di mercurio, vestiti intessuti d'oro), per tutti questi oggetti il valore del peso deve essere ragionato anche in funzione dell'ingombro.

Ogni oggetto ha un proprio valore di Ingombro, in linea di massima \textbf{ogni 3 kg si ha 1 come fattore di Ingombro}. Questo valore può diventare anche 5Kg se l'oggetto è facilmente trasportabile. I valori di Ingombro degli oggetti si sommano tra di loro per dare il carico totale portato da confrontare con la Capacità di Carico della creatura.\index{Kili ed Ingombro}

Gli oggetti con scarso peso e volume hanno ingombro \textbf{Leggero} (L). Questi oggetti contano come 1 di Ingombro ogni 10. Ogni 500 monete si ha 1 di Ingombro.\index{Ingombro delle monete}


\subsubsection{Capacita' di Carico}\label{capacitadicarico}\index{Capacita' di Carico}

La Capacità di Carico di una creatura è data dalla somma di fattori quali Taglia e Corpo.

La Taglia di una creatura concede un bonus alla \textbf{CdC} (Capacità di Carico) pari a 9 se Piccola, 16 se Media, 25 se Grande. L'Ingombro di una creatura se trasportata di peso è pari alla metà della sua Capacità di Carico data dalla taglia più il suo ingombro.\index{Ingombro Creature trasportate}

Quando la CdC totale viene superata allora muoversi diventa problematico.  Si diviene appesantiti e si usano 2 PA per 1.5 m. 

Se la CdC viene doppiata allora non ci si può più muovere per l'ingombro dei pesi portati.

\textit{Ricordate che l'armatura e scudo indossati hanno un Ingombro dimezzato rispetto a quanto segnato.}

Es. Tups ha indosso una Armatura ad Anelli (ingombro 2 essendo indossata), una spada lunga (arma media, incombro 2), una mazza chiodata (ing. 2), 18 oggetti leggeri (ing. 1), uno zaino (ing. 1), una tenda (ing. 2), una lanterna (ing 1). Totale Ingombro = 11.

Tups e' una creatura Media con Forza -1 e Costituzione -1 (è un po' gracile e debole..) questo gli concede una Capacità di Carico di 16-1-1=14.

Se il carico viene appoggiato su un carro puoi spingerlo a movimento pieno se entro il tuo CdC, a metà movimento se entro il doppio della CdC ed ad un quarto del movimento se entro il quadruplo della CdC.

In caso più creature spingano il carro considerate come CdC quella più alta ed aggiungete metà della seconda più alta e basta, ignorando le altre creature che spingono.

\subsubsection{Creature Più Grandi e Più Piccole}

Nella \textbf{Tabella: CdC trasportato in base alla taglia}\index{Ingombro trasportato in base alla taglia} viene riportata la Capacità di Carico in base alla taglia. Al valore dato dalla taglia vanno sommati i valori di Forza e Costituzione.

\medskip

\begin{tabularx}{0.45\textwidth}{ll|ll}
\textbf{Taglia}	& \textbf{Ing.}&\textbf{Taglia} & \textbf{Ing.}\\
\toprule
Piccolissima &1/4& Grande & 25\\
Minuta & 1 & Enorme& 36\\
Minuscola & 4& Mastodontica&49\\
Piccola & 9 & Colossale&64\\
Media & 16&&\\
\end{tabularx}

\medskip

\subsubsection{Creature con più zampe}

Creature con 4 zampe o più possono trasportare carichi maggiori. Consulta la Tabella sottostante ed eventualmente moltiplica i modificatori riportati con quelli dovuti alla taglia.

%\begin{center}
%\includegraphics[height=0.5\linewidth]{immagini/cavallo.png}
%\end{center}

\textbf{Tabella: modificatori trasporto per creature con più zampe}\index{Tabella modificatori trasporto per creature con più zampe}

\medskip

\begin{tabularx}{0.45\textwidth}{ll}
\textbf{Zampe Creatura}	&	\textbf{CdC}\\
\toprule
4 zampe & x2\\
6 zampe & x2.5\\
8 zampe & x3\\
12 zampe & x4\\
ogni altre 2 zampe & +0.5\\
\end{tabularx}

\medskip

Un cavallo essendo Grande e quadrupede può trasportare senza problemi fino ad un massimo di CdC 50.

\subsection{Altri Tipi di Movimento}

\label{altri-tipi-di-movimento}

\subsubsection{Nuotare}\index{Nuotare}\label{nuotare}

Una creatura con una velocità di Nuotare può muoversi attraverso l'acqua alla sua velocità indicata senza fare prove di Nuotare e ha 2 Bonus su tutte le Prove per nuotare.

Un incantatore si considera Distratto se lancia un incantesimo mentre nuota.

Se non si ha il tipo di movimento Nuotare \textbf{muoversi nell'acqua} si considera come \textbf{"terreno" difficile}, e quindi il movimento costa il doppio dei PA.

\subsubsection{Scalare}\index{Scalare}\label{scalare}

Una creatura con una velocità di Scalare ha 2 Bonus su tutte le Prove di Arrampicarsi. 

Una creatura non ha penalità alla PDAD e PDAA mentre si sta arrampicando.

Se non si ha il tipo di \textbf{movimento Scalare} si considera come \textbf{terreno molto difficile}, e quindi servono 3 PA per 1.5 metri.

\subsubsection{Scavare}\index{Scavare}\label{scavare}

Una creatura con una velocità di Scavare può scavare tunnel attraverso la terra, ma non attraverso la roccia a meno che il testo descrittivo non dica il contrario. Le creature non possono caricare o correre mentre scavano.

La maggior parte delle creature scavatrici non lascia tunnel che altre creature possono utilizzare (sia perché il materiale attraverso cui scavano riempie il tunnel dietro di loro o perché in realtà non spostano materiale quando scavano), vedere la descrizione della singola creatura per i dettagli.

\subsubsection{Volare}\index{Volare}\label{volare}

Volare per una creatura dotata di questa abilità è come camminare per una creatura "terrestre". Una creatura dotata di volo usa le sua azioni per muoversi ma difficilmente sarà influenzata dal terreno difficoltoso.

\end{multicols}

\pagebreak
\section{Veleni, Pozioni e Malattie}\index{Veleni}\index{Pozioni}\index{Malattie}

\label{veleni-e-pozioni}


\begin{changemargin}{0.3cm}{0.3cm}\begin{enfasi}{
			Un giorno, un uomo fu colpito da una freccia avvelenata. Gli amici e i parenti, in ansia, chiamarono un medico. Quando gli si avvicinarono per prendere la freccia, l'uomo disse loro: "Prima di farlo, vorrei sapere chi mi ha trafitto con questa freccia... Era uno schiavo, un re, o un bramino? Era grande? Piccolo? Di che colore era la sua pelle? Dove viveva? E la freccia com'è stata costruita? Quale veleno è stato impiegato? ..."
			
			Mentre si stava ponendo tutte queste domande... il veleno fece il suo effetto e l'uomo ferito finì per morire. (Budda)}\end{enfasi}\end{changemargin}\medskip

\begin{multicols}{2}
	
	\subsection{Tipo di Veleno e Pozione}\label{tipidiveleno}

tipi di veleno (contatto, ferimento, ingestione)
identificare

insorgenza ed effetto
	
I veleni e pozioni possono distinguersi in base a come si viene in contatto con loro.
	Non tutti i veleni sono tossici se ingeriti o inalati.
	
	Per identificare una pozione naturale è necessario una prova di Erboristeria a DC 12 + la rarità della pianta oppure in caso di Veleni la difficoltà è pari al Tiro Salvezza dello stesso. Costa 1 Azione ogni 10 di DC oppure con Erboristeria 6 o più costa 1 Azione ogni 15 DC e con 12 punti costa 1 Azione ogni 20 DC. Le pozioni se non descritto diversamente devono essere bevute (ingestione).
	
	\textbf{Contatto}: sono contratti nel momento in cui qualcuno tocca il veleno con la pelle nuda. I veleni a contatto hanno solitamente un tempo di insorgenza di 1 round. Un veleno a contatto può essere un unguento, balsamo, liquido di qualsiasi densità o anche polvere se specifica per contatto e non inalazione.
	
	\textbf{Ingestione}: si attivano quando una creatura li mangia o li beve. I veleni ad ingestione hanno solitamente un tempo di insorgenza di 10 minuti.
	
	\textbf{Ferimento}: vengono trasferiti soprattutto con gli attacchi di alcune creature e tramite armi cosparse di veleno. I veleni a ferimento hanno solitamente un tempo di insorgenza istantaneo.
	
	\textbf{Inalazione (R)}: si attivano nel momento in cui una creatura entra in un'area che contiene tali veleni. Molti veleni ad inalazione riempiono un volume pari ad un cubo con spigolo di 3x3x3 metri per dose. Le creature possono tentare di trattenere il fiato mentre si trovano all'interno dell'area per evitare di inalare la tossina.
	Una creatura può trattenere il fiato per 6 round per il suo punteggio di Costituzione, con un minimo di 3 round ed ogni Azione diminuisce il tempo rimanente di 1 round.
	Trascorso il tempo devono fare un Tiro Salvezza su Tempra a difficoltà 12 ogni round per non inalare il gas. Ogni round in cui si trattiene il fiato la prova di difficoltà aumenta di 1.
	Vedi anche le regole per trattenere il fiato e soffocare in \hyperlink{trattenereilfiato}{Ambiente}
	
	
	\subsection{Insorgenza ed Effetto}\index{Insorgenza veleno}\index{Tempo di attivazione veleno}\label{insorgenzaveleno}
	
	Per insorgenza si intende quanto tempo ci mette il veleno o pozione a fare effetto. Se il tempo di insorgenza è 1 Turno significa che per gli effetti del veleno/pozione ed il Tiro Salvezza lo si effettua dopo 10 minuti. Se nella tabella del veleno/pozione insorgenza non è specificata significa che l'effetto è immediato dopo l'entrata in contatto con il veleno.
	
	L'effetto di un veleno/pozione è immediato dopo l'insorgenza. Verificare la descrizione del veleno per capirne l'effetto. Se il Tiro Salvezza su Tempra riesce il veleno non ha fatto effetto e si può ritenere neutralizzato.
	
	Ci sono alcuni casi in cui è presente la voce Frequenza, in questi rare occasioni il Tiro Salvezza va ripetuto ogni volta che passa la Frequenza indicata, in caso di fallimento del Tiro Salvezza i danni indicati vengono riapplicati.
	
	%\begin{center}
	%\includegraphics[height=0.5\linewidth]{immagini/potion.png}
	%\end{center}
	
	\begin{changemargin}{0.3cm}{0.3cm}\begin{narratore}
			I veleni qui proposti sono alcuni dei tanti presenti e possibili. Usali come linee guida. Se per tua etica e stile non ti piacciono i veleni, specialmente quelli più cattivi, suggerisco di usare le Pozioni Generiche che trovi in fondo al capitolo. Sono veleni più blandi e meno personali, probabilmente più facilmente usabili anche dai giocatori.
	\end{narratore}\end{changemargin}
	
	\subsubsection{Avvelenati}\index{Avvelenati}\label{avvelenato}
	
	\textbf{Prima dose}: Quando si viene esposti a un veleno per la prima volta (durante la propria azione o quella di qualcun altro), è necessario effettuare un Tiro Salvezza entro l'insorgenza per evitare di venire avvelenati.
	
	\textbf{Successo}: Si resiste al veleno. Non si subiscono effetti negativi e non sono necessari ulteriori Tiri Salvezza.
	
	\textbf{Fallimento}: Siete stati avvelenati e si subisce subito l'effetto elencato.
	
	\textbf{Più dosi}: Se si vieni esposti a più dosi dello stesso veleno nello stesso round la difficoltà del Tiro Salvezza aumenta di 1 per dose aggiuntiva.\index{Veleno più dosi}
	
	\textbf{In tempi diversi}: se si viene esposti al veleno in tempi diversi, ogni volta ci sarà un nuovo Tiro Salvezza e si subiranno gli eventuali effetti nei tempi previsti.
	
	Se si viene esposti a veleni diversi è necessario effettuare un Tiro Salvezza per ogni tipo di veleno assunto.
	
	
	\begin{changemargin}{0.3cm}{0.3cm}\begin{tcolorbox}[title = Veleno ?]
			{Il veleno è un arma a doppio taglio. Finché la usi tu va bene ma se te la usano contro, magari lo stesso, diventa un problema. Ci sono anche degli aspetti etici nell'usare i veleni, valuta se i tuoi Tratti ti permettono di usare dei veleni e di che tipi.}\end{tcolorbox}\end{changemargin}
	
	\subsection{Applicare il Veleno}\index{Applicare il Veleno}\label{applicareveleno}
	
	Applicare il veleno ad un'arma o ad una munizione richiede 3 Azioni.
	
	Ogni volta che un personaggio applica o prepara un veleno per l'uso deve tirare 3d6+Intelligenza e se ottiene un fallimento critico nella prova è entrato in contatto con il veleno e deve effettuare un Tiro Salvezza contro il veleno come di regola. Ciò non consuma la dose di veleno.
	
	Ogni volta che un personaggio attacca con un'arma avvelenata, se esegue un fallimento critico con il tiro di dadi per il Tiro per Colpire, si espone agli effetti del veleno. Ciò consuma il veleno sull'arma.
	Un pozione di veleno è sufficiente per coprire di veleno un arma media oppure 3 frecce. Il veleno viene così consumato e rimane attivo sull'arma finché questa non colpisce.
	
	Una creatura sotto gli effetti di un veleno, che si siano già scatenati o meno, ha la condizione di Avvelenato.
	
	\subsection{Rimuovere il veleno}
	
	L'incantesimo \hyperlink{incrimuoviveleno}{Rimuovi Veleno} (pag. \pageref{incrimuoviveleno}) rimuove i veleni, e quindi la condizione avvelenato, che non hanno ancora fatto effetto purchè la DC del Veleno sia inferiore alla DC dell'incantesimo Rimuovi Veleno. Se la DC del veleno non è espressa allora si considera che basta il semplice lancio dell'incantesimo per annullarne gli effetti.
	
	Ogni Critico Magico ottenuto con la Prova della magia nel lancio dell'incantesimo  equivale a +4 nel calcolo della DC vedi (\hyperlink{magietirosalvezza}{Tiri Salvezza - Resistere all'incantesimo}, pag. \pageref{magietirosalvezza}) per superare quello del veleno.
	
	Una prova di Pronto Soccorso\index{Pronto Soccorso e Veleni}\index{Veleni e Pronto Soccorso}, che sia almeno la metà della DC del veleno entro il tempo dell'insorgenza, permette di effettuare un nuovo Tiro Salvezza. Una volta fatta la prova non è più possibile rifarla se non dopo l'insorgenza.
	Un trattamento continuativo di Pronto Soccorso per 8 ore permette di effettuare un nuovo Tiro Salvezza dopo l'attivazione del veleno.
	
	\medskip
	
	\subsection{Creare Veleni Naturali}\index{Creare Veleni Naturali}\label{crearevelenonaturale}
	
	I veleni naturali possono essere realizzati usando Erboristeria. La DC per preparare un veleno è uguale alla DC del Tiro Salvezza su Tempra che richiede -5. Se gli ingredienti si comprano il costo per preparare la pozione è metà del costo di vendita indicato, se si cercano in natura il costo per produzione scende ad un quarto. Il tempo per preparare queste pozioni/droghe è pari alla DC/2 in ore.
	
	Ottenendo un fallimento critico con la prova di Erboristeria ci si espone al veleno durante la sua preparazione. Se la prova di DC Erboristeria ha successo se ne preparano 1d2+1 dosi.
	
	Gli esempi seguenti rappresentano solo alcuni dei possibili veleni. Tutti i costi sono espressi in Monete d'Oro.
	
	I Veleni sono presentati, specialmente nel Mostruario con questa dizione: Nome Veleno, Uso (I/R/F/C), tempo Insorgenza, DC del Tiro Salvezza, Effetto. 
	

	\begin{changemargin}{0.3cm}{0.3cm}\begin{narratore}
			I veleni fanno parte della lunga tradizione dei problemi ed avversità nei giochi di ruolo. Quando volete usare un veleno pensate innanzitutto perché si trova li, per chi doveva essere usato, per quale scopo. Non è detto che tutti i veleni debbano uccidere, un abile ladro potrebbe anche usare veleni stordenti o che indeboliscono la volontà del suo obiettivo giusto quel tanto che basta a farsi aprire la cassaforte.
	\end{narratore}\end{changemargin}
	
\end{multicols}




\textbf{Tabella: Veleni}\index{Tabella Veleni}\label{tabellaveleni}

\medskip

\begin{tabularx}{1\textwidth}{m{4.5cm}lllm{6.5cm}l} %{XlllXl}
	\toprule
	\textbf{Nome Veleno} & \textbf{Uso} & \textbf{TS} & \textbf{Ins.} & \textbf{Effetto (danno)} & \textbf{MO}\\
	\toprule
	Bacca Viola di Barsar\index{Bacca Viola di Barsar}  & I  & 18  & 1 round  & Incapace di violenze per 3d8 ore  & 40 \\
	\toprule
	Bacche Azzurre di fosso \index{Bacche Azzurre di fosso}  & I  & 21  & 1 Turno  & -1d3 Intelligenza e Saggezza per 6 ore& 55\\
	\toprule
	Bava fermentata di Lucos \index{Bava fermentata di Lucos}& F  & 15  & - & 1d8 Punti Ferita  & 25\\
	\toprule
	Cenere di Corteccia Gialla \index{Cenere di Corteccia Gialla} & F  & 15  & 6 round  & Privo di sensi per 1d3 ore  & 25\\
	\toprule
	Concentrato Viola \index{Concentrato Viola} & F  & 15  &  & 2d6 Punti Ferita & 15\\
	\toprule
	Dita di Daraka\index{Dita di Daraka} & F  & 17  & - & -1d6 Forza, per 1 ora & 35\\
	\toprule
	Erba puntuta rosa \index{Erba puntuta rosa}  & I  & 22  & 1 round  & -1d6 Destrezza, per 1 ora  & 60\\
	\toprule
	Fegato di Toporagno Viola \index{Fegato di Toporagno Viola} & I  & 25  & 1 ora  & 2d6 di danno a Saggezza e Intelligenza. Permanente & 75 \\
	\toprule
	Fiocco bianco di Mucot \index{Fiocco bianco di Mucot}  & C  & 20  & - & Dorme per 2d12 ore  & 20\\
	\toprule
	Fumi di Curna\index{Fumi di Curna} & R  & 18  & - & -1d3 Saggezza & 40\\
	\toprule
	Gelo blu \index{Gelo blu} & F  & 18  &  & 3d6 Punti Ferita da freddo& 25\\
	\toprule
	Grasso di Toporagno Viola \index{Grasso di Toporagno Viola} & C  & 13  & 1 round  & 2d12 Punti Ferita & 15\\
	\toprule
	Lingua di Kreex \index{Lingua di Kreex} & F  & 20  & - & La ferita sanguina. +1 danno da sanguinamento. 1 uso nelle 24 ore. & 50 \\
	\toprule
	Mistura Rossa \index{Mistura Rossa} & F  & 13  & -  & -1d6 TC/TS per 10 minuti & 10\\
	\toprule
	Muschio Giallo \index{Muschio Giallo}& I  & 20  & 1 round  & la creatura guadagna una taglia. -2 Int e Sag. Durata 10 minuti  & 50\\
	\toprule
	Nocciolo di Dennar \index{Nocciolo di Dennar}  & I  & 13  & 1 round  & -1d2 Forza, per 3gg  & 15\\
	\toprule
	Olio di Nabar \index{Olio di Nabar}  & R-F& 20  & - & Confuso per 2d6 round  & 50\\
	\toprule
	Pelle di Rospo Azzurro \index{Pelle di Rospo Azzurro}  & C  & 22  & 1 minuto & Paralizzato per 1d6 turni& 60\\
	\toprule
	Polline di Rosa di Omro\index{Polline di Rosa di Omro} & I  & 15  & - & -1d3 Costituzione e Destrezza, per 1 ora & 25\\
	\toprule
	Profumo di Ragmor \index{Profumo di Ragmor}  & R  & 16  & - & -1d3 Carisma, per 1 giorno & 30\\
	\toprule
	Sangue di Thrun \index{Sangue di Thrun} & C  & 26  & - & -1d3 Costituzione & 80\\
	\toprule
	Succo di Ythis\index{Succo di Ythis} & I  & 14  & 1 Turno  & -1d2 Intelligenza, per 1g  & 20\\
	\toprule
	Veleno di Ottalm\index{Veleno di Ottalm}  & F  & 20  & - & Morte o -1d2 Costituzione permanente  & 50\\
	\toprule
	Veleno di Serpe del Sangue \index{Veleno di Serpe del Sangue} & F  & 25  & - & Paralisi per 1d6 ore -1d4 punti Forza per 7 giorni & 75 \\
\end{tabularx}

\medskip

\textbf{Applicazione}: \textbf{I}(ngestione), \textbf{F}(erimento), \textbf{C}(ontatto), \textbf{R}(espirazione).

Il Tiro Salvezza è sempre su Tempra se non specificato diversamente

I punti caratteristica persi si recuperano al ritmo di 1 al giorno se non permanenti o indicato diversamente.


\subsection{Pozioni naturali}\index{Pozioni}\label{pozioninaturali}

\begin{changemargin}{0.3cm}{0.3cm}\begin{enfasi}{
			Io credo che una foglia d'erba non sia meno di una giornata di lavoro compiuto dagli astri. (Walt Whitman)
}\end{enfasi}\end{changemargin}

\begin{multicols}{2}
	
	Il tempo per \textbf{preparare} queste pozioni/droghe è pari alla DC/2 in ore, mentre la difficoltà della prova di Erboristeria è pari alla DC -5. Se gli ingredienti si comprano il costo per preparare la pozione è metà del costo di vendita indicato, se si cercano in natura il costo per produzione scende ad un quarto.
	
	Se la prova di DC Erboristeria ha successo se ne preparano 1d2+1 pozioni (da 1 dose).
	
	Non si può beneficiare di più di una dose di pozioni naturali (per tipo) al giorno, a differenza di quelle magiche.
	
\end{multicols}

\medskip
{\small
	\begin{xltabular}{0.95\textwidth}{llllXlc}
		\textbf{Nome}  & \textbf{Uso} & \textbf{Ins.} & \textbf{DC} & \textbf{Effetto}& \textbf{Loc.} & \textbf{Costo} \\
		\toprule
		Arduuar\index{Arduuar} & I  & 1 round & 25& Rimuove Veleni  & SZ7 & 75 \\
		\toprule
		Arkasun\index{Arkasun} & C  & 1 Turno  & 25& Cura 1d6 Punti Ferita a Turno per 3 turni& MT7 & 75 \\
		\toprule
		Arlan\index{Arlan}  & C  & 5 round  & 15& Cura 1d6+3 Punti Ferita & TT5 & 50 \\
		\toprule
		Arlandas\index{Arlandas} & R  & 1 ora& 24& Rinsalda le fratture & CF5 & 200  \\
		\toprule
		Attarna\index{Attarna} & I  & 1 Turno & 20& Concede un nuovo Tiro Salvezza per Malattie con un +1d6  & TF7 & 50 \\
		\toprule
		Bacche di Ljust \index{Bacche di Ljust} & I  & 1 round & 16& Preso la sera recuperi il doppio dei Punti Ferita minimo 4) & AZ6 & 10 \\
		\toprule
		Bacio di Ljust\index{Bacio di Ljust}  & C  & 1 round & 35& Cura 100 Punti Ferita& HO8 & 500  \\
		\toprule
		Barannie\index{Barannie} & I  & 1 minuto  & 15& Rimuove nausea & MD6 & 3 \\
		\toprule
		Burthelas \index{Burthelas} & I  & 1 Turno & 32& Rigenera le mani& HD7 & 410  \\
		\toprule
		Corteccia Dagmathir\index{Polvere di corteccia di Dagmathir}  & R  & 1 round & 25& Rimuove un livello di Affaticamento  & SS5 & 15 \\
		\toprule
		Corteccia di Aklent\index{Corteccia di Aklent}  & I  & 1 Turno & 10& La corteccia masticata per almeno 10 round concede per le 24 ore successive un +1 Tiro Salvezza vs Veleno  & MT6 & 1  \\
		\toprule
		Culcoa\index{Culcoa}& C  & 1 round & 16& Recuperi 2d6 da danno da fuoco & TS7 & 15 \\
		Darsirion\index{Darsirion}  & C  & 1 round & 25& Cura 1d4 Punti Ferita& CM4 & 5 \\
		\toprule
		Delrean Plus\index{Delrean Plus} & I  & 1 round & 18& Allontana insetti per 3 giorni & CC6 & 5  \\
		\toprule
		Delrean\index{Delrean} & C  & 1 round & 15& Allontana insetti per 1 giorno & CC6 & 2  \\
		\toprule
		Draaf \index{Draaf} & C  & 1 round & 20& Cura 1d8 Punti Ferita& SO6 & 50 \\
		\toprule
		Eldrin'tail\index{Eldrin'tail}& I  & 1 round  & 15& Concede un nuovo Tiro Salvezza su Veleni  & FH7 & 18 \\
		\toprule
		Estratto di Bacca Illa\index{Estratto di Bacca Illa bruciata}& I  & 1 round & 15& +2 Iniziativa, +2 Destrezza, -1d6 Tiro Salvezza su Volontà, per 10 minuti  & MS6 & 5  \\
		\toprule
		Estratto radice Gisenosa\index{Estratto di radice Gisenosa}  & I  & 3 turni  & 15& Cura tosse e raffreddore  & MT6 & 3  \\
		\toprule
		Febfendi \index{Febfendi}& C  & 1 Turno & 25& Rigenera orecchie  & CF7 & 75 \\
		\toprule
		Garioe\index{Garioe}& I  & 1 round & 25& Cura 2d6 Punti Ferita& AZ7 & 95 \\
		\toprule
		Geffnull \index{Geffnull}& I  & 5 round & 28& Cura 3d8+3 Punti Ferita & EV8 & 150 \\
		\toprule
		Gusterbloon \index{Gusterbloon}  & C  & 1 round & 20& La pelle diventa più scura concedendo un +1d6 alla prove di Nascondersi & CM5 & 8  \\
		\toprule
		Gylvert\index{Gylvert} & I  & 1 minuto  & 25& Concede respirare sott'acqua per 4 ore & MO7 & 3 \\
		\toprule
		Harfy \index{Harfy} & C  & - I & 12& -1 al sanguinamento  & SS6 & 3 \\
		\toprule
		Harfindar\index{Harfindar}  & I  & 1 Turno & 15& Fa abortire& SS7 & 3 \\
		\toprule
		Jojopo\index{Jojopo}& C  & 1 round & 15& Recuperi 2d6 da danno da freddo & FM6 & 18 \\
		\toprule
		Kelventare\index{Kelventare}  & I  & 1d4 round & 28& Recuperi 2d6 Punti Ferita & TT7 & 100 \\
		\toprule
		Klagul\index{Klagul}& C  & 1 Turno & 20& Pulisce i denti & SS4 & 2 \\
		\toprule
		Klandor\index{Klandor} & I  & I  & 15& Rimuove paralisi. Aumenta di 1 il livello di affaticamento& HB6 & 18 \\
		\toprule
		Klynkyx\index{Klynkyx} & C  & 6 Turno & 15& Fa cadere tutti i capelli per 1d6+4 gg & MO6 & 4  \\
		\toprule
		Lievito Muschio Bianco \index{Lievito di Muschio Bianco} & I  & 1 minuto  & 12& I prodotti da forno che usano questo lievito causano meteorismo incontrollabile ed incredibilmente puzzolente per 12 ore & CA3 & 1  \\
		\toprule
		Lingua Rossa di Xabax\index{Lingua Rossa di Xabax}& C  & 1 Turno & 20& Cura 2d6 Punti Ferita ma se c'è malattia o veleno la rimuove causando 2d6 PF di danno & TA7 & 13 \\
		\toprule
		Melandrir\index{Melandrir}  & I  & 1 round & 15& Concede un nuovo Tiro Salvezza per Malattie con +5  & CF7 & 100 \\
		\toprule
		Mirenna\index{Mirenna} & I  & 1 round & 20& Cura 5 Punti Ferita  & CM6 & 30 \\
		\toprule
		Miscela 31\index{Miscela 31}& I  & 1 Turno & 20&La cavalcatura è estremamente resistente. +6 ore di galoppo al giorno & SM6 & 15  \\
		\toprule
		Muschio argentato\index{Muschio Argentato}& I  & I  & 25& Rimuove Malattie magiche & MU8 & 250  \\
		\toprule
		Musekiss\index{Musekiss} & C  & 1 ora& 30& Rigenera arti inferiori & TH9 & 550  \\
		\toprule
		Nazamuse \index{Nazamuse}& I  & I  & 30& Rimuove Veleni e Malattie naturali & EW9 & 175  \\
		\toprule
		Nelthalion \index{Nelthalion} & I  & I  & 15& Fa vomitare& SR3 & 1  \\
		\toprule
		Petali di Lisbeth \index{Petali di Lisbeth}  & I  & 1 Turno & 15&+2 Intelligenza, -2 Destrezza per 10 minuti & MC6 & 20 \\
		\toprule
		Polline di Rosa Verde\index{Polline di Rosa Verde}& R  & 3 turni & 25& Recuperi 2d4 danni Intelligenza e Saggezza  & FA8 & 35 \\
		\toprule
		Radice secca di Kathaus\index{Radice secca di Kathaus} & R  & 1 round  & 20& +2 Forza e Destrezza per 1 ora & FW6 & 50 \\
		\toprule
		Rewky\index{Rewky} 		 & I  & 1 Turno  & 25& Cura 2d8 Punti Ferita& TD6 & 20\\
		\toprule
		Siranmuse\index{Siranmuse}  & I  & 1 giorno  & 30& Rigenera organi interni & SS8 & 850  \\
		\toprule
		Ucsaboo \index{Ucsaboo}  & C  & 1 Turno & 30& Rigenera occhi  & MO8 & 400  \\
		\toprule
		Uovo di Urk\index{Uovo di Urk}& I  & 1 Turno & 12& 1 giorno di cibo& FH7 & 1  \\
		\toprule
		Uscaboo \index{Uscaboo}  & R  & 1 Turno & 25& Rimuove cecità  & MO7 & 125 \\
		\toprule
		Wickalim\index{Wickalim} & I  & 1 ora  & 15& Cura 2 Punti Ferita  & TD3 & 5 \\
		\toprule
		Yaveth\index{Yaveth}& I  & 1 Turno  & 20& Cura 2d8 Punti Ferita& MO5 & 100 \\
\end{xltabular}}


\subsubsection{Note sui Veleni e Pozioni}

\textbf{Fegato di Toporagno Viola}: avvelenamento riconoscibile dai tipici occhi iniettati di sangue

\textbf{Bava fermentata di Lucos}: Lucos e' una lucertola erbivora e pacifica. La bava raccolta va fatta fermentare al buoi per 1 settimana prima di essere utilizzabile.

\textbf{Toporagno Viola}: secondo molti il Toporagno è l'animaletto preferito di Cattalm. Aggressivo, violento, pericoloso in ogni sua fibra.

\textbf{Dita di Daraka}: le Dita di Daraka sono il frutto dell'albero di Daraka. Il baccello di forma allungata e nera ricorda le dita dell'antica dea dell'oscurità

\textbf{Olio di Nabar}: le piccole bacche di Nabar sono esclusivamente mangiate dai Toporagni, immuni ai loro malefici effetti. Bollito a lungo diviene un ottimo unguento per la pelle.

\textbf{Uovo di Urk}: Urk e' un grosso coleottero, l'uovo è poco piu' grande di una nocciola. Solitamente viene prima affumicato con legno di faggio, mangiato crudo il sapore è di muffa e terra.

\textbf{Miscela 31}: un insieme studiato di droghe per i cavalli. Terminato l'effetto la creatura deve fare un Tiro Salvezza DC 23 o cadere svenuto per 12 ore.

\textbf{Bacca Viola di Barsar}: curiosità il Toporagno viola ' schifato da queste bacche.

\textbf{Veleno di Ottalm}: l'Ottalm e' una variante di Toporagno viola dotato di un pungiglione velenoso

\textbf{Cenere di Corteccia Gialla}: la corteccia va prima macerata e battuta in acqua e sale. La poltiglia risultante va seccata e poi fatta scaldare senza bruciarla direttamente

\textbf{Lingua Rossa di Xabax}: è il petalo lungo della Xabax. Dei 7 petali solo quello lungo ha i sostanze necessarie a preparare l'unguento.

\textbf{Radice secca di Kathaus}: piccolo tubero nero estremamente duro e legnoso. Solitamente si lascia seccare al sole prima di macinarla

\textbf{Petali di Lisbeth}: estremamente profumati, assomigliano a quelli di rosa

\textbf{Fumi di Curna}: la Curna è l'inflorescenza del cardo mariano

\textbf{Corteccia di Aklent}: chiamato anche \textit{Cespuglio Puzzola} per il suo pungente e caratteristico odore.

\textbf{Estratto di radice Gisenosa}: pianta tipo cardo, estremamente spinosa. Tende a crescere circondata dal \textit{Tribulus terrestris} o "baciapiedi".

\textbf{Muschio Argentato}: molto simile, per un non esperto, al Muschio Bianco. Si raccolgono le bacche.

\subsection{Dove trovare le piante}

Es: Gusterbloon FT5. La prima Lettera indica il CLIMA, la Seconda indica l'AMBIENTE, la Terza indica la RARITA'. La rarità indica la possibilità, su un d10, di trovare l'erba/pianta ricercata. Tirare 1d10 e fare di più del numero indicato, chiaramente se c'è corrispondenza di clima ed ambiente.

\medskip

\textbf{Tabella: della corrispondenza Pozioni - Luoghi}\index{Tabella della corrispondenza Pozioni - Luoghi}

\medskip

\begin{tabular}{ll|ll|ll}
	\textbf{1' lett.} & \textbf{Clima} &  \textbf{2' lett.} & \textbf{Ambiente} & \textbf{2' lett.} & \textbf{Ambiente}\\
	\toprule
	A & Arido  & A & Alpino & B & Gole\\
	C & Freddo & C & Foresta di Conifere  & D & Foresta Decidua\\
	E & Ghiacci perenni & F & Argini fiumi e torrenti & G  & Campi ghiacciati\\
	F & Freddo severo  & H & Campi secchi &J & Giungla, Foreste piovose\\
	H & Umido e caldo & M & Montagna & N & Oceano, distese salate\\
	M & Temperato  & S & Erba bassa & T & Erba alta\\
	S & Semi arido & U & Caverne e sotterranei & V & Vulcanica\\
	T & Temperato fresco & W & Discariche / Rifiuti & Z & Deserto\\
	X & Sconosciuto  & X & Sconosciuto&&\\
\end{tabular}

\subsection{Pozioni generiche}\index{Pozioni generiche}\index{Pozioni}

Il Narratore è libero di usare tutte le pozioni e veleni indicate sopra oppure usare delle pozioni generiche pronte all'uso, comprabili in quasi tutti i negozi di erboristeria o di pozioni.

Nella tabella sono indicati i costi ed effetti di queste pozioni. L'insorgenza è sempre immediata, la durata per le cure è immediata, per le altre è 1 ora (quindi la pozione Rimuovi Veleno ti "immunizza" per 1 ora contro un veleno). Per le pozioni che causano danno il Tiro Salvezza è per annullarne gli effetti.

\textbf{Tabella: delle pozioni generiche}\index{Tabella delle pozioni generiche}\label{pozionigeneriche}

\medskip

\begin{tabularx}{0.95\textwidth}{lXcc}
	\textbf{Nome Pozione}&  \textbf{Effetto}&  \textbf{Costo (mo)}& \textbf{Applicazione}\\
	\toprule
	Cura& recuperi 1d8+1 Punti Ferita & 50 & Ingestione\\
	Cura potenziata& recuperi 3d8+3 Punti Ferita & 125  & Ingestione\\
	Indebolente& -1d6 TC. TS DC 15 Tempra & 34 & Ingestione\\
	Indebolente potenziata& -1d6 TC. TS DC 18 Tempra& 50 & Ferimento \\
	Veleno& subisci 2d4+2 di danno. TS DC 15 Tempra & 30 & Ingestione \\
	Veleno potenziata& subisci 2d4+2 di danno. TS DC 18 Tempra & 25 & Ferimento \\
	Rimuovi Veleno& annulla l'insorgenza di un veleno se presa entro l'attivazione, oppure concede un nuovo TS con +1d6 & 75 & Ingestione\\
\end{tabularx}

Queste pozioni generiche come le pozioni naturali hanno effetto solo la prima volta che sono prese nell'arco delle 24 ore.


\subsection{Opzionale - Droghe}\index{Droga}\index{Opzionale - Droghe}\hypertarget{droghe}{}\label{droghe}

\textbf{Tabella: Elenco Droghe}\index{Tabella Elenco Droghe}

\medskip
{\small 
	
	\begin{tabularx}{0.99\textwidth}{XlllXrr}
		\textbf{Nome}  & \textbf{Uso} & \textbf{Ins.} & \textbf{DC} & \textbf{Effetto}& \textbf{Loc.} & \textbf{Costo} \\
		\toprule
		Foglie fermentate di Luside\index{Foglie fermentate di Luside}  & I  & 1 Turno  & 17& Allucinazioni sensoriali per 2d4 ore. +2 Carisma ed Intelligenza & SF7 & 5  \\
		\toprule
		Ferpillon \index{Ferpillon}& I  & 1 round & 20& Fa dormire per 24 ore& SC5 & 50 \\
		\toprule
		Unto Grigio \index{Unto Grigio} & I  & 1 round & 24& Rimuove condizionamenti mentali causati da incantesimi di livello 5 o meno& AH9 & 80 \\
		\toprule
		Cenere di Arpasur \index{Cenere di Arpasur}  & R  & 1 round & 20& Rimuove 2 livelli di affaticato  & FT6 & 10  \\
		\toprule
		%Carne secca di Toporagno Viola \index{Carne secca di Ragno Viola} & I  & 1 round & 24& +4 Forza -4 Intelligenza (minimo -3) per 1 turno& SH7 & 30 \\
		%\toprule
		Estratto alcolico di Melzaa\index{Estratto alcolico di Melzaa}  & I  & 1 round & 20& +1d4 Forza, +1d4 Destrezza. -1d6 Tiro Salvezza su Volontà. Per 3 ore & AF6 & 25 \\
		\toprule
		Essenza profumata di Inut\index{Essenza profumata do Inut} & R  & I  & 15& +2 Intelligenza, per 1d8 ore& HB6 & 15 \\
		\toprule
		Polline di Julnnaus\index{Polline di Julnnaus} & R  & I  & 20& +3 Costituzione per 2 ore & FO6 & 25 \\
		\toprule
		Polline del fiore di Erain \index{Polline del fiore di Erain} & R  & 1 round  & 20& +2 Forza e Intelligenza e Destrezza. +3d6 Punti Ferita temporanei, per 1 ora  & FT7 & 75 \\
	\end{tabularx}
}
\begin{multicols}{2}
	
	\medskip
	
	\textbf{L'utilizzo delle droghe è completamente opzionale, è il Narratore a decidere la loro presenza e disponibilità anche in base alla sensibilità dei giocatori}.
	
	Le droghe danno dipendenza. Terminato l'effetto entro 24 ore effettuare un Tiro Salvezza su Volontà a difficoltà 15 o prenderne un altra dose, il successivo Tiro Salvezza avrà difficoltà +1 e così via.
	
	Ogni qual volta si prende una nuova dose entro 2 settimane dalla prima il Tiro Salvezza per non diventare dipendenti aumenta di 1. Non prendere una dose aumenta il livello di Affaticamento di uno.
	
	Sono necessari 7 Tiri Salvezza riusciti di seguito per terminare l'effetto di dipendenza.
	
	
	\subsection{Malattie}\index{Malattie}\hypertarget{malattie}{}\label{malattie}
	
	Le malattie in linea di principio si gesticono come dei veleni, si esegue un Tiro Salvezza per verificare se ci si è contagiati ed altri TS per guarire. 
	Solitamente il tempo di scatenamento di una malattia non è così immedidato come un veleno eppure quelle magiche possono essere dirompenti ed agire in pochi minuti.
	
	Una malattia richiede un Tiro Salvezza per evitare completamente di prendere il malanno e poi più Tiri Salvezza riusciti in successione per guarire una volta fallito il primo.
	
	Ogni malattia  deve aver indicato il tempo di insorgenza, il Tiro Salvezza iniziale, ogni quanto va rifatto il Tiro Salvezza, quanti successi al TS sono necessari per guarire, gli effetti che si subiscono.
	
	Es. Febbre Demoniaca minore: 1 minuto, TS Tempra DC 18, 6 ore, 3 successi, -1 Costituzione e Saggezza
	
	La Febbre Demoniaca minore costringe al Tiro Salvezza su Tempra a DC 18 dopo un solo minuto che la si è presa. Successivamente ogni 6 ore va rifatto il Tiro Salvezza e la malattia permane finchè non si sono fatti almeno 3 successi consecutivi al TS. Ogni 6 ore il malato perde 1 punti a Costituzione e Saggezza.
	
	Per guarire da una malattia, non naturale, come quelle afflitte dai mostri è necessario passare i Tiri Salvezza richiesti oppure avere a disposizione un incantesimo di \hyperlink{rimuovimalattie}{Rimuovi Malattie} (pag. \pageref{rimuovimalattie}).
	
	Una prova di \textbf{Pronto Soccorso}\index{Pronto Soccorso e Malattie}\index{Malattie e pronto soccorso}, con DC pari almeno alla metà della DC della malattia (o 15 se non indicata), effettuata tra un Tiro Salvezza e successivo, permette di avere un +1 al Tiro Salvezza per resistere agli effetti della malattia.
	
	L'incantesimo Rimuovi Malattie garantisce la guarigione dall'afflizione purchè la DC di lancio dell'incantesimo sia superiore alla DC della malattia.
	
	Ogni Critico Magico ottenuto con la Prova della magia nel lancio dell'incantesimo  equivale a +4 nel calcolo della DC vedi (\hyperlink{magietirosalvezza}{Tiri Salvezza - Resistere all'incantesimo}, pag. \pageref{magietirosalvezza}) per superare quella della malattia.
	
	Essere colpiti più volte dalla stessa malattia non ne aumenta la difficoltà di guarigione ne cambia i tempi ed effetti della stessa.
	
	Esempi di Malattie:\\
	
	\textbf{Influenza Demoniaca}:  1 minuto, TS Tempra DC 16, 12 ore, 2 successi, -1 Costituzione/12 ore\index{Influenza Demoniaca}
	
	\textbf{Corruzione di Rezh}: 1 giorno, TS Volontà DC 18, 1 ora, 2 successi, -1d6 Punti Ferita Massimi/6 ore\index{Corruzione di Rezh}
	
	\textbf{Piaga Funginea}\index{Piaga Funginea}: 8 ore, TS Tempra DC 24, 12 ore, 2 successi, -1 punto a Destrezza e Intelligenza/12 ore
	
	\textbf{Torpore Violento}\index{Torpore Violento}: 24 ore, TS Volontà DC 12, 12 ore, 1 successo, +1 al Danno con Armi da Mischia e -1 Saggezza
	
	\textbf{Febbre Demoniaca minore}\index{Febbre Demoniaca minore}: 1 minuto, TS Tempra DC 18, 6 ore, 3 successi, -1 Costituzione e Saggezza
	
	\textbf{Sangue Nero}\index{Sangue Nero}: 10 minuti, TS Tempra 28, 12 ore, 1 successo, perdita metà Punti Ferita rimasti
	
	\textbf{Peste T}\index{Peste T}: 1 minuto, TS Tempra 30, 2 ore, 3 successi, esegui 3 successi consecutivi altrimenti vieni trasformato in uno zombi
	
\end{multicols}

\pagebreak



\pagebreak
\pagebreak


APPUNTI
\pagebreak
\pagebreak



ispirarsi a nome verbo, ma fai te gli accoppiamenti e stabilisci te cosa succede per ogni critico 
i livelli (II, III..) aggiuntivi hanno prerequisiti di consocenza magica maggiore e statistica maggiore, causano una maggiore perdita di resistenza ma garantiscono maggiore risultato, il tempo di lancio (iniziativa) e' piu' lenta piu' e' potente l'incantesimo.

Creare Corpo  ???
Muovere Elemento I, II
Muovere Corpo (feather fall, levitate, volare..)
Muovere Elemento
Proteggere Corpo
Proteggere Elemento 
Proteggere Mente
Distruggo Elemento (piccole cose.. poi piu' grandi)
Distruggo Corpo (maledizione, penalita'..)
Ripara Corpo
Ripara Elemento
Conoscere Corpo
Conoscere elemento
Conoscere Mente
Ripara Mente
Alterare Mente (charmi, compulsion..)
Distruggo Mente (come attacco)
Riparare Spirito, da Mercanteggiare se serve, se ci sono mostri effetti contro lo spirito...


Creomente (bonus a prove mentali), elemento
Muovocorpo, elemento, mente (possessione ?)
Trasformo
Altero
Proteggocorpo, mente, elemento, spirito
Distruggocorpo, mente, elemento, spirito
Attaccoelemento, spirito
Ripararecorpo, mente, elemento, spirito
Conoscerecorpo, mente, elemento, spirito

Corpo
Mente
Elemento
Spirito, cio' che riguarda la non vita e l'anima






condizioni: prendere da obss e ridurre sensibilmente  e semplificare bonus e malus, aggiungere condizioni solo se necessario

incantesimi solitamente costano da 4 punti azione in su in base  al livello (II, III; IV..) ogni livello in piu' aumenta di 1 l'iniziativa, spostarsi costala meta' della in base alla distanza coperta
i mostri in base alla loro taglia e cosa fanno
per altre azioni vedi obss e traduci in p.a.

magia: deve essere lanciare piu' difficile lanciare incantesimi e consumare "risorse" (resistenza/energia/stamina..)

certi rami magici possono privilegiare certe scuole di magia. lavorare su liste e ridurre e tanto. l'idea di base e' per esempio crea fuoco puo' diventare a seconda del punteggio del tiro altre cose, il valore influenza la distanza, AoE, danno, se e' un raggio o esplosione e che raggio...  Pochi incantesimi ma che si evolvono

lanciare un incantesimo: tirare a secondo dall'incantesimo su capacita' magica e avere un punteggio minimo di  mente, corpo o volonta'.  Ogni punteggio -3 rispetto alla prova, es. capacita' magica 13 e tiro 8, potenzi un fattore dell'incantesimo (distanza, AoE, danno, tipo di effetto..). Gli incantesimi hanno un punteggio minimo di mente/corpo/volonta' per essere tirati ed anche di capacita' magica
Ogni incantesimo ha degli attributi, danno, distanza, aoe, durata ed un punteggio minimo di competenza magica e corpo/mente/volonta'. se il tiro riesce bene puoi potenziare un attributo presente ma non darlo/aggiungerlo se questo e' assente. se lancio l'incantesimo crea fiamma, inc. base difficolta' 1, se faccio un ottimo tiro potro' potenziare la durata e aoe (l'area di luce che fa) ed il danno, ma non posso aggiungere distanza perche' e' un attributo assente.
ci sara' poi l'incantesimo globo di fuoco, la versione base della palla di fuoco, questa ha piu' attributi ma ad esempio non ha durata, o meglio l'istantanea non puo' essere migliorata.
Mercanteggiare di aggiungere attributi assenti quando il tiro e' veramente buono, ovvero tiri veramente basso, direi almeno un -6 per aggiungere un attributo a livello base (3 metri)
si possono spendere risorse aggiuntive per potenziare l'incantesimo, ovvero abbassare il risultato del dado
scuola di magia: raggruppa gli incantesimi per tipo
ci si puo' specializzare in un "verbo" di magia: hai +3 alla prova, ma -3 a tutte le altre scuole

lanciare incantesimi mentre si combatte o si e' stato colpito: si puo' fare ma il primo critico si ignora

se l'incantesimo riesce nel lancio non c'e' TS. alcuni incantesimi possono avere una prova di difesa per essere evitati/dimezzati l'effetto

questo implica che non ci saranno mai incantesimi super potenti in tutto..

quanti incantesimi lanciare:  a piacere, ma ogni volta che lanci lo stesso il costo in vitalita' aumenta. 

quando lanci un incantesimo e fallisci la prova non succede nulla, ma l'incantesimo l'hai lanciato e quindi perdi la vitalita' 
quando lanci un incantesimo e fallisci con 19-20 la prova succedono cose  brutte
quando lanci l'incantesimo e fai veramente basso 2, non perdi vitalita'


mostri:
se il mostro tira la sua difesa c'e' il rischio che riesca sempre ad alti livelli. considerare abilita' che abbassano la difesa, l'idea di base e' che se un mostro ha difesa  o piu' deve essere di un livello tale da dover affrontare pg con rami che danno penalita' alla difesa
fare una lista minima di 20 mostri tipici e classici, verificare bx per compatibilita'


--------------------------------------------------------------------------------



- fare 2 o 20  successo critico o fallimento  critico

- il giocatore dichiara cio' che fa e' solo il master a stabilire se serve una prova. 

- il tempo e' un fattore, tabelle random per incontri basati su tempo trascorso, si computa il tempo reale.



demoni della bibbia

Abaddon: In alcune interpretazioni, è un angelo caduto o un demone dell'abisso.
Asmodeo: Menzionato nel Libro di Tobia, è un demone associato all'incesto e agli spiriti maligni. Un demone spesso associato alla lussuria e ai desideri sessuali.

Astaroth: Un demone associato all'occultismo e ai rituali magici.Astoret: Un'entità idolatrica spesso associata al male. (1 Re 11:5, 1 Samuele 7:3-4)
Astoret o Astarte: Divinità idolatrica spesso associata al male. (1 Re 11:5, Giudici 2:13)



Azazel: Menzionato nel Libro di Levitico come parte del rituale del capro espiatorio nel Giorno dell'Espiazione. In alcune tradizioni apocrife, è uno dei tre demoni che guidano i caduti.
Baal: Anche se il termine può riferirsi anche a divinità non demoniache, talvolta è associato a forze maligne o idolatria.
Belial: Un termine usato in diverse parti della Bibbia, spesso per riferirsi a persone malvagie o depravate.
Belzebù: Conosciuto come il "signore delle mosche", è menzionato nel Nuovo Testamento come un principe dei demoni.

Chemos o Chemosh: Un dio pagano associato a sacrifici. (1 Re 11:7)


Dagon: Un'idolo filisteo associato a un dio della fertilità. (Giudici 16:23, 1 Samuele 5:1-5)


Legione: Nel Nuovo Testamento, è il nome di un gruppo di demoni che possedeva un uomo.
Lilith: Una figura nel folclore ebraico, a volte considerata un demone, associata all'oscurità e alla sessualità.
Mammona: Un termine utilizzato da Gesù nei vangeli per riferirsi alla ricchezza o al denaro personificati come un idolo.
Moloch: Un idolo a cui venivano offerti sacrifici umani e che rappresentava il male. (Levitico 18:21, Geremia 32:35)


Mastema: Compare nel Libro di Giubileo, dove viene descritto come un demone associato all'oppressione.
Samael: Spesso associato all'angelo della morte o all'accusatore.
Satana o Lucifero: Descritto come il capo dei demoni, colui che si ribellò contro Dio e fu cacciato dal cielo. Presente in varie parti della Bibbia, come ad esempio nel libro di Isaia e nel Nuovo Testamento. Un angelo caduto che rappresenta il male e la ribellione contro Dio.

Serapide: Un'idolatria spesso associata alla cultura egizia. (Atti 19:23-41)


arcangeli


Arcangelo Michele: Spesso riferito come un capo degli angeli, associato alla protezione e alla lotta contro le forze del male. (Daniele 10:13, Giuda 1:9)

Gabriele: Annunciò la nascita di Giovanni Battista e di Gesù ai loro genitori. (Luca 1:11-20, 1:26-38)

Raffaele: Appare nell'Apocrifo di Tobia, aiutando Tobia in diverse avventure. (Tobia 3:16-17, 5:4-28, 12:15)


Michele: L'unico arcangelo esplicitamente menzionato nella Bibbia. Viene spesso descritto come un guerriero spirituale e un difensore del popolo di Dio. (Daniele 10:13, Giuda 1:9)

Gabriele: Pur non essendo sempre menzionato come arcangelo nella Bibbia, è spesso considerato un arcangelo per il suo ruolo di portatore di importanti messaggi divini, incluso l'annuncio dell'incarnazione a Maria. (Luca 1:19, 26)

Raffaele: Non menzionato esplicitamente come arcangelo nella Bibbia, è citato nel Libro di Tobia come l'angelo che accompagna il giovane Tobia in un viaggio. (Tobia 3:17, 12:15)

Uriele: Il nome non compare nella Bibbia, ma è menzionato in alcuni testi apocrifi e nelle tradizioni ebraiche e cristiane. Viene spesso associato a un angelo di luce, conoscenza o preghiera.

Uriel: Menzionato in alcuni testi apocrifi ed extra-biblici come un angelo della divina giustizia.

Raziel: Anche questo nome non compare nella Bibbia, ma è menzionato in alcuni testi apocrifi e nella tradizione ebraica. Raziel è associato a segreti e conoscenze arcane.


Saraqael: Ancora una volta, non menzionato esplicitamente nella Bibbia, è presente in alcune tradizioni apocrife ebraiche e cristiane.

angeli varie

Angeli Custodi: Anche se non menzionati con nomi specifici nelle Sacre Scritture, si crede che Dio assegni agli individui angeli custodi per proteggerli e guidarli.

Angeli dell'Apocalisse: Descritti nell'Apocalisse di Giovanni come figure che eseguono giudizi e compiti divini. (Apocalisse 7:1-2, 8:2)

Serafini: Descritti come esseri con sei ali che adorano Dio e che appaiono nel tempio celeste. (Isaia 6:1-7)

Cherubini: Esseri con ruoli di custodia e protezione, spesso associati alla presenza divina. (Genesi 3:24, Ezechiele 10:1-22)

Angeli dell'Annunciazione: Gli angeli che annunciavano eventi importanti, come l'annunciazione a Maria. (Luca 1:26-38)

Angeli che lodano Dio: Esseri che adorano e lodano Dio costantemente, come descritto in Apocalisse e altri passi biblici. (Apocalisse 4:8-11)

Angeli che eseguono giustizia: Esseri inviati da Dio per eseguire la sua volontà e giudizio, come nei racconti dell'Antico Testamento.

Eserciti angelici: Gruppi di angeli che servono Dio e sono pronti a eseguire i suoi comandi. (Luca 2:13, Apocalisse 19:14)

Angeli messaggeri: Angeli inviati per consegnare messaggi o guidare gli individui in situazioni specifiche. (Numeri 20:16, Atti 7:53)

Angeli della resurrezione: Descritti come annunciatori della risurrezione e della venuta di Cristo. (Matteo 28:2-7, 1 Corinzi 15:52)

bestie


Leviatano: Un mostro marino menzionato nella Bibbia, spesso interpretato come un simbolo delle forze del caos.
Behemoth: Un altro animale descritto nel Libro di Giobbe, spesso interpretato come un grande e potente animale terrestre. (Giobbe 40:15-24)

Scorpione: Nel contesto delle profezie apocalittiche, rappresenta forze maligne e giudizio. (Apocalisse 9:1-11)

Unicorno: Menzionato in alcune traduzioni della Bibbia, potrebbe riferirsi a un animale mitico o a un rinoceronte. (Deuteronomio 33:17, Numeri 23:22, 24:8)

Bestia dalla terra: Menzionata nell'Apocalisse, rappresenta un'altra potenza maligna. (Apocalisse 13:11-18)

test2

\end{document}

