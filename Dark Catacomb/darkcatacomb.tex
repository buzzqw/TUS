\documentclass[a4paper,twoside,openany]{book}

% ========== CONFIGURAZIONE GEOMETRIA ==========
\usepackage[a4paper,margin=2cm]{geometry}

% ========== PACCHETTI LINGUISTICI E CODIFICA ==========
\usepackage[italian]{babel}
\usepackage[italian=guillemets]{csquotes}
\usepackage{hyphenat}
\hyphenation{
	mate-mati-ca
	re-cu-pe-ra-re
	com-pe-ten-za
	intel-li-gen-za
	cara-tte-ri-sti-che
	abi-li-tà
	in-can-te-si-mo
	at-tac-co
	di-fe-sa
	per-so-nag-gio
	av-ven-tu-ra
	espe-rien-za
	equi-pag-gia-men-to
}

% ========== PACCHETTI GRAFICI E COLORI ==========
\usepackage{xcolor}

% Definizione colori del tema OBSS
\definecolor{OBSSnavyblue}{RGB}{0,0,128}
\definecolor{OBSSprussia}{RGB}{0, 49, 83}
\definecolor{OBSSblue}{RGB}{30,84,131}
\definecolor{OBSSgold}{RGB}{255,215,0}
\definecolor{OBSSpurple}{RGB}{88,57,131}
\definecolor{lightgray}{gray}{0.95}
\definecolor{LightCyan}{RGB}{0,191,255}
\definecolor{SlateGray}{RGB}{112,128,144}
\definecolor{Lavender}{RGB}{230,230,250}
\definecolor{Coral}{RGB}{255,127,80}
\definecolor{Tan}{RGB}{210,180,140}
\definecolor{LightGreen}{RGB}{34,139,34}
\definecolor{Mauve}{RGB}{221,160,221}

% ========== PACCHETTI TIPOGRAFICI ==========
\usepackage{fontspec}
%\setmainfont{AtkinsonHyperlegibleNext-Regular.ttf}[
%Path=./fonts/,
%BoldFont=AtkinsonHyperlegibleNext-Bold.ttf,
%ItalicFont=AtkinsonHyperlegibleNext-RegularItalic.ttf,
%BoldItalicFont=AtkinsonHyperlegibleNext-BoldItalic.ttf,
%Ligatures=TeX
%]

\usepackage{microtype}
% microtype migliora la qualità tipografica

% ========== PACCHETTI LAYOUT E SPAZIATURA ==========
\usepackage{setspace}
\usepackage{ragged2e}
\usepackage{quoting,enumitem}

% ========== PACCHETTI MATEMATICI E SCIENTIFICI ==========
\usepackage{amssymb,siunitx}

% ========== PACCHETTI TABELLE ==========
\usepackage{booktabs,multirow}
\usepackage{xltabular,tabularx}
\usepackage{colortbl}
\usepackage{hhline}
\usepackage{multicol,array}

% ========== PACCHETTI GRAFICI E FIGURE ==========
\usepackage{graphics,adjustbox,svg}
\usepackage{caption}
\usepackage{wrapfig}
\usepackage{tikz}
\usetikzlibrary{shadows,shapes.misc,calc}

% ========== PACCHETTI BOX E CORNICI ==========
\usepackage{tcolorbox}
\tcbuselibrary{skins,breakable,theorems}
\usepackage[framemethod=TikZ]{mdframed}

% ========== PACCHETTI BIBLIOGRAFIA E INDICI ==========
\usepackage[backend=bibtex]{biblatex}
\addbibresource{bibliografia.bib}
\usepackage{makeidx,imakeidx}

% ========== PACCHETTI VARI ==========
\usepackage{soul}
\usepackage{pdfpages}
\usepackage{etoolbox}
\usepackage[absolute,overlay]{textpos}
\usepackage{url}
\def\UrlBreaks{\do\/\do-\do_\do.\do:\do=\do&}
\usepackage{varioref}
\usepackage{fontawesome5}


% ========== CONFIGURAZIONI GENERALI ==========
\setcounter{secnumdepth}{-1}
\setcounter{tocdepth}{3}
\raggedbottom
\sloppy
\DeclareMathSizes{11}{11}{8}{6}

% Configurazione XeTeX per l'italiano
\XeTeXlinebreaklocale "it"
\XeTeXlinebreakskip = 0pt plus 1pt

% ========== CONFIGURAZIONE TCOLORBOX ==========
\tcbset{colback=brown!10, fonttitle=\scshape}

% ========== CONFIGURAZIONE INDICI ==========
% Macro per contare le occorrenze negli indici
\def\CountIndexOccurrences#1{%
	\expandafter\newcount\csname #1\endcsname%
	\def\indexentry##1##2{\expandafter\advance\csname #1\endcsname 1}%
	\IfFileExists{#1.idx}{\input{#1.idx}}{}%
}

% Inizializzazione contatori indici
\CountIndexOccurrences{OBSSv2}
\CountIndexOccurrences{Incantesimi}
\CountIndexOccurrences{Mostruario}
\CountIndexOccurrences{Tabelle}
\CountIndexOccurrences{OggettiMagici}
\CountIndexOccurrences{Abilita}

% Macro per visualizzare il totale delle voci
\def\TotalBox#1{\vfill%
	\fbox{Ci sono \expandafter\the\csname #1\endcsname\ voci in questo indice}\par}

% Creazione degli indici
\makeindex[columns=1, title=Indice Analitico, intoc=true]
\makeindex[columns=1, name=Tabelle, title=Elenco delle Tabelle, intoc=true]
\makeindex[columns=1, name=Incantesimi, title=Elenco degli Incantesimi, intoc=true]
\makeindex[columns=1, name=Mostruario, title=Elenco dei Mostri, intoc=true]
\makeindex[columns=1, name=OggettiMagici, title=Elenco degli Oggetti Magici, intoc=true]
\makeindex[columns=1, name=Abilita, title=Elenco delle Abilità, intoc=true]

% ========== CONFIGURAZIONE INTESTAZIONI ==========

\usepackage{fancyhdr}
\pagestyle{fancy}
\fancyhf{}
\fancyhead[LE,RO]{\leftmark}
\fancyfoot[C]{\thepage}
\renewcommand{\sectionmark}[1]{\markboth{#1}{}}

\fancypagestyle{plain}{%
	\fancyhf{}
	\fancyhead[RO]{%
		\rotatebox{90}{
			\begin{tikzpicture}[overlay,remember picture]
				\node[
				fill=lightgray,
				text=black,
				font=\footnotesize,
				inner ysep=12pt,
				inner xsep=20pt,
				rounded rectangle,
				anchor=east,
				minimum width=7cm,
				xshift=-60mm,
				yshift=-21mm,
				text height=0.4cm
				] at ($ (current page.north east) + (-1cm,-0cm) + (-4*\thesection cm,0cm) $)
				{\sffamily\itshape\small\nouppercase{\leftmark}};
			\end{tikzpicture}
		}
	}
	\fancyhead[LE]{%
		\rotatebox{90}{
			\begin{tikzpicture}[overlay,remember picture]
				\node[
				fill=lightgray,
				text=black,
				font=\footnotesize,
				inner ysep=12pt,
				inner xsep=20pt,
				rounded rectangle,
				anchor=east,
				minimum width=7cm,
				xshift=-60mm,
				yshift=-4mm,
				text height=0.4cm
				] at ($ (current page.north west) + (1cm,0cm) + (-4*\thesection cm,0cm) $)
				{\sffamily\itshape\small\nouppercase{\leftmark}};
			\end{tikzpicture}
		}
	}
	\renewcommand{\headrulewidth}{0pt}
	\renewcommand{\footrulewidth}{0pt}
	\fancyfoot[C]{\thepage}
}
\pagestyle{plain}

% ----- DEFINIZIONI PER MARGINI -----
\def\changemargin#1#2{\list{}{\rightmargin#2\leftmargin#1}\item[]}
\let\endchangemargin=\endlist

% ========== CONFIGURAZIONE TITOLI ==========
\usepackage{titlesec}

\titleformat{\chapter}[display]
{\normalfont\Huge\bfseries\sffamily\color{OBSSblue}}
{\textcolor{OBSSgold}{\chaptertitlename\ \thechapter}}
{2ex}
{\centering}

\titleformat{\section}
{\normalfont\Large\bfseries\sffamily\color{OBSSblue}}
{\thesection}
{1em}
{}
[\textcolor{OBSSgold}{\titlerule[0.8pt]}]

\titleformat{\subsection}
{\normalfont\large\bfseries\sffamily\color{OBSSblue}}
{\thesubsection}
{1em}
{}

\titlespacing{\chapter}{0pt}{0pt}{20pt}
\titlespacing{\section}{0pt}{10pt}{5pt}
\titlespacing{\subsection}{0pt}{8pt}{4pt}
\titlespacing{\subsubsection}{0pt}{6pt}{3pt}

% ========== AMBIENTI PERSONALIZZATI ==========
\newtcolorbox{narratore}[1][Narratore]{
	enhanced,
	colback=OBSSblue!8,
	colframe=OBSSblue,
	boxrule=1.2pt,
	arc=4mm,
	drop shadow={shadow xshift=1mm, shadow yshift=-1mm, opacity=0.4, shadow scale=1.05},
	title={\textbf{\textcolor{white}{\large \faBook} #1}},
	fonttitle=\sffamily\bfseries\large,
	colbacktitle=OBSSblue,
	coltitle=white,
	attach boxed title to top left={yshift=-2.5mm, xshift=3mm},
	boxed title style={boxrule=0pt, colframe=white, arc=4mm,
		drop shadow={shadow xshift=0.3mm, shadow yshift=-0.3mm, opacity=0.2}
	},
	left=3mm, right=3mm, top=4mm, bottom=2mm,
	before skip=\baselineskip, after skip=\baselineskip,
}

\newtcolorbox{giocatore}[1][Giocatore]{
	enhanced,
	breakable,
	colback=OBSSpurple!8,
	colframe=OBSSpurple,
	boxrule=1.2pt,
	arc=4mm,
	drop shadow={shadow xshift=1mm, shadow yshift=-1mm, opacity=0.4, shadow scale=1.05},
	title={\textbf{\textcolor{white}{\large $\dagger$} #1}},
	fonttitle=\sffamily\bfseries\large,
	colbacktitle=OBSSpurple,
	coltitle=white,
	attach boxed title to top left={yshift=-2.5mm, xshift=3mm},
	boxed title style={boxrule=0pt, colframe=white, arc=4mm,
		drop shadow={shadow xshift=0.3mm, shadow yshift=-0.3mm, opacity=0.2}
	},
	left=3mm, right=3mm, top=4mm, bottom=2mm,
	before skip=\baselineskip, after skip=\baselineskip,
}

\newtcolorbox{enfasi}{
	enhanced,
	colback=OBSSgold!8,
	colframe=OBSSgold,
	boxrule=1.2pt,
	arc=4mm,
	drop shadow={shadow xshift=0.3mm, shadow yshift=-0.3mm, opacity=0.2},
	left=3mm, right=3mm, top=2mm, bottom=2mm,
	before upper={\hyphenpenalty=10000\exhyphenpenalty=10000},
	before skip=\baselineskip, after skip=\baselineskip,
}

% ========== COMANDI PERSONALIZZATI ==========
\newcommand{\OBSSseparator}{%
	\begin{center}%
		\begin{tikzpicture}%
			\draw[OBSSgold, line width=1pt] (-2,0) -- (-0.5,0);%
			\node[circle, fill=OBSSgold, minimum size=4mm] at (0,0) {};%
			\draw[OBSSgold, line width=1pt] (0.5,0) -- (2,0);%
		\end{tikzpicture}%
	\end{center}%
}

\newcommand{\FatePoint}{\textcolor{OBSSgold}{$\bullet$}}
\newcommand{\FatePoints}[1]{%
	\textcolor{OBSSgold}{%
		\foreach \n in {1,...,#1}{$\bullet$\,}%
	}%
}

%\NewDocumentCommand{\feat}{m}{
	%	\noindent\rule{\linewidth}{2pt}
	%	\index[Abilita]{#1}\hypertarget{#1}{}\label{#1}
	%	\vspace{-5.5mm}
	%	\begin{tcolorbox}[
		%%		colback=OBSSgold!12, colframe=OBSSgold,
		%		boxrule=1.5pt, arc=3pt, boxsep=1pt,
		%		left=6pt, right=6pt, top=3pt, bottom=3pt
		%		]
		%		\centering\textbf{\textcolor{OBSSgold!80!black}{#1}}
		%	\end{tcolorbox}
	%	\pdfbookmark[3]{#1}{#1}
	%	\vspace{1mm}\noindent
	%}

\NewDocumentCommand{\feat}{m}{
	\noindent\rule{\linewidth}{2pt}
	\index[Abilita]{#1}\hypertarget{#1}{}\label{#1}
	\vspace{-5.5mm}
	\begin{center}\textbf{\textcolor{OBSSgold!90!black}{#1}}\end{center}
	\pdfbookmark[3]{#1}{#1}
	\vspace{-1mm}\noindent
}

\NewDocumentCommand{\incantesimo}{m}{
	\noindent\rule{\linewidth}{2pt}
	\index[Incantesimi]{#1}\hypertarget{#1}{}\label{#1}
	\vspace{-5.5mm}
	\begin{center}\textbf{\textcolor{OBSSgold!90!black}{#1}}\end{center}
	\pdfbookmark[3]{#1}{#1}
	\vspace{-1mm}\noindent
}

\NewDocumentCommand{\mostro}{m}{
	\noindent\rule{\linewidth}{2pt}
	\index[Mostruario]{#1}\hypertarget{#1}{}\label{#1}
	\vspace{-5.5mm}
	\begin{center}\textbf{\textcolor{OBSSgold!90!black}{#1}}\end{center}
	\pdfbookmark[3]{#1}{#1}
	\vspace{-1mm}\noindent
}

\NewDocumentCommand{\oggettomagico}{m}{
	\noindent\rule{\linewidth}{2pt}
	\index[OggettiMagici]{#1}\hypertarget{#1}{}\label{#1}
	\vspace{-5.5mm}
	\begin{center}\textbf{\textcolor{OBSSgold!80!black}{#1}}\end{center}
	\pdfbookmark[3]{#1}{#1}
	\vspace{-1mm}\noindent
}



% ========== SISTEMA TABELLE COLORATE ==========
\newcommand{\obsscurrentcolor}{Coral}
\newcommand{\obsssetcolor}[1]{\renewcommand{\obsscurrentcolor}{#1}}

% Comandi per cambiare colore
\newcommand{\obssgrey}{\obsssetcolor{SlateGray}}
\newcommand{\obsspurple}{\obsssetcolor{Mauve}}
\newcommand{\obssblue}{\obsssetcolor{LightCyan}}
\newcommand{\obssgreen}{\obsssetcolor{LightGreen}}
\newcommand{\obsscoral}{\obsssetcolor{Coral}}


% ========== HYPERREF E BOOKMARK ==========
% Importante: hyperref deve essere caricato per ultimo (quasi)
\usepackage[
unicode,
pdfencoding=auto,
bookmarks=true,
colorlinks=true,
linkcolor=OBSSblue,
citecolor=OBSSpurple,
urlcolor=OBSSblue,
filecolor=OBSSblue,
anchorcolor=OBSSblue,
pdftitle={OBSS - Old Bell School System},
pdfsubject={Gioco di Ruolo Fantasy},
pdfauthor={Andres Zanzani},
pdfkeywords={GDR, Fantasy, Gioco di Ruolo, OBSS},
pdfcreator={XeLaTeX},
pdfproducer={TexStudio on Debian},
pdfborder={0 0 0},
breaklinks=true,
bookmarksopenlevel=1,
bookmarksnumbered=true,
bookmarksopen=true
]{hyperref}


\usepackage{bookmark}
% bookmark deve essere caricato dopo hyperref

% ========== INIZIO DOCUMENTO ==========


\begin{document}

\def \versione {0.0.3} \fontsize{12}{14}\selectfont

\thispagestyle{empty}

{\Huge \begin{center}
		Dark Catacomb
\end{center}}

\vfill
\begin{center}
	\Large{\color{black} Survival Adventure Game}
\end{center}

\newpage~\thispagestyle{empty}%%\newpage~\thispagestyle{empty}

\newpage~\thispagestyle{empty}%%\newpage~\thispagestyle{empty}

\bigskip
Non temere l'ignoto, affrontalo con rispetto.

	\vspace{\fill}
\begin{center}\textbf{\versione} - \today\end{center}
\thispagestyle{empty}

\newpage~\thispagestyle{empty}%%\newpage~\thispagestyle{empty}

\newpage~\thispagestyle{empty}%%\newpage~\thispagestyle{empty}


\newcommand{\riga}{\rule{\textwidth}{0.4pt}}


{\Huge \begin{center} Dark Catacomb\end{center}}

\bigskip

\begin{center}{\LARGE Manuale per Giocatore ed il Demiurgo}\\ \end{center}

{\large \begin{center} Guida e Regole per il Gioco di Ruolo Apocalittico \end{center}}

\begin{center}di \end{center}

{\LARGE \begin{center} Andres Zanzani \end{center}}

\vspace{2cm}


\vfill

\begin{mdframed}[roundcorner=10pt]

\medskip

\textbf{Playtesting}: ...to be done...

\bigskip

\begin{flushleft}\textbf{Condizioni d'uso}: Dark Catacomb, DKC, è un marchio registrato di Andres Zanzani (azanzani@gmail.com).
\end{flushleft}

\vspace{0.5cm}


\medskip

\end{mdframed}


\pagebreak

\begin{center}
{\LARGE Io sono l'Alfa e l'Omega, dice il Signore Dio, Colui che è, che era e che viene, l'Onnipotente!} {\normalsize (Apocalisse 1:8)}
\end{center}


\vfill

Dark Catacomb e' un gioco di ruolo che tocca diversi temi religiosi. Se l'argomento ti da fastidio, mi dispiace. Cambia gioco.

\pagebreak

\setcounter{page}{1}

\begin{multicols}{2}
{\small \tableofcontents{}}

\end{multicols}

\vfill

\begin{changemargin}{0.3cm}{0.3cm}\begin{tcolorbox}
In principio Dio creò il cielo e la terra. La terra era informe e deserta e le tenebre ricoprivano l'abisso e lo spirito di Dio aleggiava sulle acque.

Dio disse: «Sia la luce!». E la luce fu. Dio vide che la luce era cosa buona e separò la luce dalle tenebre e chiamò la luce giorno e le tenebre notte. (Genesi 1:5)
\end{tcolorbox}\end{changemargin}

\pagebreak

%pietra bianca
%100 atti unici di altruismo disinteressato
%10 demoni uccisi

%10 angeli uccisi
%100 anime raccolte
%100 7 peccati, la grande meretrice

%superbia: radicata convinzione della propria superiorità, reale o presunta, che si traduce in atteggiamento di altezzoso distacco o anche di ostentato disprezzo verso gli altri, nonché di disprezzo di norme, leggi, rispetto altrui;

%avarizia, derivante più precisamente dall'etimologia latina avaritia, collegata all'avidità della fame:[9] cupidigia, avidità, costante senso di insoddisfazione per ciò che si ha già e bisogno sfrenato di ottenere sempre di più;

%lussuria: incontrollata sensualità, irrefrenabile desiderio del piacere sessuale fine a se stesso, concupiscenza, carnalità, divinizzazione del sesso sempre maggiore, che può andare dalla fornicazione sino all'adulterio, e agli atti più estremi e perversi;

%invidia: in relazione a un bene o una qualità posseduta da un altro, si prova dispiacere e astio per non avere noi quel bene e a volte un risentimento tale da desiderare il male di colui che ha quel bene o qualità;

%gola: nel suo senso concreto, è l'irrefrenabile bramosia di ingurgitare cibi o bevande senza fermarsi al limite della sazietà imposto dal corpo, ma proseguire nella consumazione per puro piacere e ingordigia. Nel suo senso astratto, "goloso" è chi abusa di una determinata cosa, andando al di là del limite imposto dalla natura umana.

%ira: alterazione dello stato emotivo che manifesta in modo violento un'avversione profonda e vendicativa verso qualcosa o qualcuno;

%accidia: torpore malinconico, inerzia nel vivere e nel compiere opere di bene, pigrizia, indolenza, infingardaggine, svogliatezza, abulia.



%https://thealexandrian.net/wordpress/17308/roleplaying-games/hexcrawl
%http://osrsimulacrum.blogspot.com/2020/05/making-wilderness-play-meaningful-system.html
%http://www.innomedimaria.it/apocalisse/apocalisse.htm
%https://www.maranatha.it/Bibbia/8-Apocalisse/73-ApocalissePage.htm


\section{Introduzione}\index{Introduzione}

e se per una volta la parola fosse Speranza ?

Dark Catacomb è un gioco di ruolo ambientato in una Terra martoriata e battuta, dove pochi sono sopravvissuti e la loro vita è una condanna e maledizione continua, eppure il \emph{fil rouge} che vuole legare i personaggi con l'ambiente e' la Speranza.

La Speranza di una vita migliore, la Speranza di una vita che possa portare gioia, la Speranza dell'Amore, la Speranza di una vita non fine a se stessa ma una vita legata alla gioia, nella vita come nella morte.

Leggerete fra poco che nulla aiuta i personaggi vivere meglio, ed anzi sembra che tutto convergi verso lo sterminio dei pochi sopravvissuti, ma non è così.

Qualcuno chiama la Speranza anche Fede, poco importa, è la forza che questa imbriglia a mandare avanti questo dannato mondo e fare in modo che il Sole sorga ogni mattino.

La battaglia dei personaggi non li porterà alla vittoria ma ad na nuova battaglia, non combatteranno per se stessi ma gli uni per gli altri.

Non si combatte perché il bicchiere è mezzo pieno o mezzo vuoto ma perché c'è ancora acqua dentro.

Le avventure tipiche in DKC sono di scoperta, ritrovamento, salvataggi, ricostruzione, speranza ed amore. Ma anche di battaglie cruenti e mortali, dove mai i personaggi dovranno perdere la fiducia nei compagni, compagne, se stessi e in un futuro migliore.

C'è chi farà il lavoro \emph{duro}, chi starà retrovie, chi aiuterà come potrà, ogni giorno la Bestia sfiderà i personaggi a perdere Speranza.

\pagebreak

\section{Dark Catacomb - L'Ambientazione}\index{Ambientazione}


%https://www.gliscritti.it/dchiesa/bibbia_cei08/nt73-libro_dell_apocalisse.htm

\begin{narratore}
L'Ambientazione di Dark Catacomb si inspira a piene mani dall'Apocalisse di San Giovanni e non vuole offendere nessun credente. Se quanto nell'ambientazione vi incuriosisce potete leggere per intero il Libro dell'Apocalisse di San Giovanni, non sono molte pagine!

\end{narratore}

\subsection{La Storia}\index{Ambientazione}

\begin{multicols}{2}

\subsubsection{Quella che già conosci}\index{La storia precedente}

\textit{Carissimi figlioli}

vi lascio queste poche pagine preziose, perché rare, affinché non dimentichiate chi eravate, da dove venivate e dove siete destinati ad andare.

Anche se tutti voi conoscete a grandi linee quello che è successo è opportuno chiarire e comprendere il perché noi siamo qui, perché non siamo andati oltre.

Circa 200 anni fa c'è stata l'Apocalisse. Non quella ecologica e climatica, non quella delle cryptovalute, non quella razziale ma quella che il Signore Dio nostro aveva profetizzato a San Giovanni. La vera Apocalisse.\index{Apocalisse}

I risultati li vediamo ancora e probabilmente per sempre.

Inizialmente grandine e fuoco mescolati a sostanze venefiche caddero sulla terra bruciando un terzo del pianeta, un terzo degli alberi e tutta l'erba verde.

Cadde poi una grande meteorite nell'oceano ed un terzo di tutti i mari divennero tossici, un terzo della vita marina morì ed un terzo di tutte le navi andarono distrutte.

Un altra grande stella fiammeggiante cadde dal cielo e colpì un terzo dei fiumi e delle sorgenti d'acqua. Anche queste acque divennero velenose e molti uomini morirono per averle bevute.

Alla quarta tromba angelica un terzo del sole, della luna e delle stelle fu colpito e la penombra divenne eterna.

Poi fu la volta del pozzo dell'Abisso. L'angelo aprì il pozzo e da questo uscirono prima un fumo nero denso come di fornace e poi \textit{cavallette} che dovevano causare molta sofferenza e dolore ma solo agli uomini e senza ucciderli.

Queste cavallette grandi come cavalli avevano capelli lunghi e denti da leone, il torace come corazze di ferro ed il rombo delle loro ali come quello degli aeroplani.

Il loro re era l’angelo dell’Abisso, che in ebraico si chiama Abaddon, in greco Sterminatore.\index{Abbadon}

Al sesto squillo di tromba vennero liberati i 4 Cavalieri, si proprio quelli, i Cavalieri dell'Apocalisse ed un altro terzo dell'umanità venne ucciso.

La civiltà, la società, la cultura, l'umanità stessa era ormai una vaga ombra di quello che era all'inizio dell'Apocalisse. E non erano passati che pochi giorni!

Ma il peggio doveva ancora venire.

A questo punto un \textbf{enorme drago rosso}, Satana\index{Satana}, con sette teste e dieci corna e sulle teste aveva sette diademi, apparì nel cielo. La sua coda trascinava un terzo delle stelle del cielo e le scagliava sulla Terra.

Dal mare salì una \textbf{Bestia}\index{Bestia} anch'essa con sette teste e dieci corna, sulle corna dieci diademi e su ciascuna testa un titolo blasfemo. Questa creatura immonda era simile a una pantera, con le zampe come quelle di un orso e la bocca come quella di un leone.

Satana diede la sua forza, il suo trono e il suo grande potere alla Bestia.

Gli stupidi umani presi per l'ammirazione e la bramosia del potere incominciarono ad adorare Satana perché aveva dato il potere alla Bestia ed alla Bestia perché proferiva parole blasfeme e d'orgoglio contro Dio che stava causando la distruzione della terra.

Chiunque adorasse Satana e la Bestia venne marchiato con il numero 666 affinché su di loro si scatenasse poi l'ira di Dio.

Mentre gli angeli avvelenavano e prosciugavano le acque e le sorgenti altre potenze facevano brillare il sole di una luce e calore così intenso che molti uomini morirono per il gran caldo. Ancora più furiosi gli uomini bestemmiavano Dio invece di rendergli gloria.

Dalla bocca di Satana e della Bestia continuavano ad uscire gli spiriti dei demoni. Molti di questi andarono per le nazioni del mondo a radunare i governanti nel luogo che si chiama Armaghedòn.\index{Armaghedòn}

E le nazioni fecero la guerra tra loro mentre la Bestia gioiva di come aveva offuscato i loro pensieri. E fu l'olocausto nucleare. Del poco che rimaneva rimase ancora di meno, qualsiasi cosa fosse.

Quando poi il settimo angelo versò la sua coppa nell’aria seguirono fulmini, tempeste ed un terremoto così forte che l'intero globo ne fu martoriato.

Solo i credenti, i puri di spirito e cuore, coloro che avevano ricevuto la \textbf{pietra bianca con inciso il loro nuovo nome} dalle mani di un angelo poterono andare alla \textbf{Città Eterna}.

Una città scesa dal cielo splendente come una gemma preziosissima, come pietra di diaspro cristallino. Una città cinta da grandi ed alte mura con dodici porte: con un angelo sopra ogni porta.

Le mura della città poggiano su dodici basamenti con inciso i nomi dei dodici apostoli dell’Agnello.\index{Città Eterna}

Gli altri, miseri, erano destinati alla morte per mano dei demoni e delle piaghe divine.


\begin{changemargin}{0.3cm}{0.3cm}\begin{enfasi}{
Per mezzo di questi prodigi, che le fu concesso di compiere in presenza della bestia, seduce gli abitanti della terra, dicendo loro di erigere una statua alla bestia, che era stata ferita dalla spada ma si era riavuta.

E le fu anche concesso di animare la statua della bestia, in modo che quella statua perfino parlasse e potesse far mettere a morte tutti coloro che non avessero adorato la statua della bestia.

Essa fa sì che tutti, piccoli e grandi, ricchi e poveri, liberi e schiavi, ricevano un marchio sulla mano destra o sulla fronte, e che nessuno possa comprare o vendere senza avere tale marchio, cioè il nome della bestia o il numero del suo nome.

Qui sta la sapienza. Chi ha intelligenza calcoli il numero della bestia: è infatti un numero di uomo, e il suo numero è seicentosessantasei. (Apocalisse 13:14)
}\end{enfasi}\end{changemargin}


\subsubsection{La storia nuova}\index{La nuova Storia}

Come avrai capito caro figlio c'è qualcosa che non torna.

Tutta l'umanità doveva essere giudicata e quindi salvata o uccisa. Salvata nella Città Eterna o uccisa dalla guerra, dalle piaghe o dalla Bestia.

Eppure, non troppo numerosi, ma siamo ancora qui. In una terra dal cielo tinto di rosso come fosse un perenne tramonto, con un sole stanco e debole ed un cielo pressoché privo di firmamento.

Le nostre navi solcano nuovamente i mari e raccolgono il pesce che ancora c'è.

Le nostre vecchie città non esistono più, non esiste più quella che una volta era chiamata industria o tecnologia.

L'inverno atomico è passato e le radiazioni, se mai potessimo misurarle, non sono più un problema.

La terra asciutta e brulla reclama lavoro e acqua. La civiltà è tornata indietro di più di mille anni ad un periodo di ignoranza e barbarie.

Quello che ci è rimasto è un mondo diverso dalla geografia riscritta, dalla natura distorta.

Perché ci siamo salvati ? perché non siamo stati giudicati?

Per un \textit{disguido} alcuni pensano.

Sappiamo che almeno uno degli angeli che portava le pietre bianche non giunse mai a destinazione e disperse le sue pietre della salvezza sul mondo.

100, 1000, 10000 ? forse un milione?  non lo sappiamo quanti dei nostri avi non ricevettero la pietra.

Non potevano essere uccisi dalla Bestia e dalle sue armate perché puri ma non potevano neanche accedere alla Città Eterna perché non avevo la pietra con il loro nuovo nome. Destinati a vivere in un incubo.

Noi tutti siamo i loro discendenti. Cerchiamo di sopravvivere in quella che per noi è una \textbf{oscura catacomba}\index{Dark Catacomb}.

Camminiamo sopra i resti dei nostri avi, abbiamo sottoterra intere città spogliate di vita e piene di creature che umane non sono più.

Quello che era il \textbf{nostro} mondo ora non lo è più. Demoni e angeli continuano a combattere e le loro discendenze portano guerra anche tra noi.

Non esiste più il concetto di nazione, di popolo. So che una volta tutti noi potevamo parlare insieme con una sola lingua, tutti potevano sapere qualsiasi cosa in qualsiasi momento.
Ecco.. non più. Consegnare una missiva a pochi giorni di cavallo è diventato un lavoro anche pericoloso.

Siamo rimasti l'ombra di ciò che eravamo, ma forse questo è anche la nostra salvezza.

Insieme si stanno ricostruendo villaggi, la terra con il duro lavoro della schiena e delle braccia produce ancora qualche frutto.
Molte nostre comunità cercano la pace e la democrazia, molte altre ubbidiscono al giogo di un demone o di un padrone umano.

Sono tornati i mestieri di una volta e non ci sono accorgimenti o tecnologie che possono lavorare al tuo posto. Ancora mi chiedo come si facesse una volta senza un bravo calzolaio.

Concetti come la razza purtroppo esistono ancora, l'ignoranza becera e meschina è insita in noi ed il seme di Satana attecchisce vivace tra i bruti e gli stupidi.

Mi piace pensare che il Signore Dio nostro ci abbia voluto dare un altra possibilità per essere salvati, la possibilità di creare una nuova civiltà ispirata alla sua grandezza.

\subsection{La nuova civilta'}\index{La nuova civilta'}

Città da milioni di abitanti sono state distrutte dalle piaghe, malattie, elementi tossici e dalle guerre dei popoli e nazioni.

Quello che sopravvive sono piccoli e pochi grandi insediamenti dove una economia di base o poco più permette alle persone di sopravvivere.

I paesi spesso non superano i 200 abitanti e le città più grandi i 20000.

Si è tornati a quello che era il medioevo.

Ogni tanto qualche reperto antico viene ancora trovato, molto spesso sono cumuli di ruggine o apparecchi che nessuno più sa usare.

Il denaro è stato sostituito da un bene di più pratico valore le gemme.

Gli insediamenti sono costruiti sfruttando i materiali e resti di antiche città. Si è tornati a costruire attorno alle risorgive d'acqua ed ai fiumi che permettono di coltivare i campi.

I terremoti hanno raso al suolo ogni manufatto umano o hanno portato nel sottosuolo intere città.

Immensi e profondissimi crepacci hanno inghiottito intere regioni. Orde di non morti abitano lussuosi edifici alla ricerca di qualcosa di vivo di cui nutrirsi.

La maggior parte del territorio rimane inesplorato ma non disabitato. Fortificazioni ed gruppi di risorti perlustrano e saccheggiano ciò che rimane.

La natura ha ripreso vita e si è diffusa ma è mutata. Mentre le piante si sono mostrate più resistenti ai cambiamenti gli animali sono regrediti ad uno stato più selvaggio ed aggressivo.

La Bestia si è divertita a creare e popolare il mondo dei mostri che sfidavamo in alcuni giochi da tavolo fatti da ragazzi.

\subsection{Quel che rimane}\index{Quel che rimane}

La Terra, se ancora così la si vuole ancora chiamare è diventata una desolata zona di guerra.

Non pensate però che angeli e demoni si diano battaglia ovunque e dappertutto. Satana ha reclamato la vittoria sulla Terra ma non per questo governa ovunque.

La Città Eterna, manifestazione del potere divino, è inespugnabile alle truppe della Bestia ma non per questo l'assedio si ferma.

I Demoni sono creature preziose e rare e Satana sa che deve moderarne l'uso. La Bestia vorrebbe dare a ferro e fuoco ogni angolo del pianeta ma non può usare le truppe del Drago.

Allo stesso modo gli angeli hanno lo scopo di proteggere la Città Eterna e cantare le lodi al Signore e non si immischiano in altre questioni.

Quello che la Bestia può è comandare i risorti e usarli per causare ancora più morti e rigenerare così le sue truppe.

Un avventuriero difficilmente incontrerà un Demone od un Angelo nella sua vita eppure non finirà di incontrare le truppe della Bestia e disgraziati che per qualche tozzo di pane sono disposti ad uccidere ed ingraziarsi le lodi di Satana.

Le zone forse più interessanti solo le profonde ed oscure catacombe che una volta chiamavamo città.
Città troppo tecnologiche per risultare ancora pratiche ed abitabili e raramente dotate di funzionali acquedotti o campi agricoli che possano sostenere la vita.

\subsubsection{I Nuovi abitanti}\index{Abitanti}

Per millenni abbiamo creduto di essere soli o addirittura qualcuno era convinto che discendessimo da creature venute dallo spazio.

Adesso sappiamo che non siamo soli, non lo siamo mai stati.

Nel corso dei secoli recenti demoni e angeli hanno giaciuto con le nostre donne e nostri uomini, ne è discesa una stirpe di creature dal destino segnato.

I nefilim sono creature figlie di due mondi, in parte umane in parte angeliche o demoniache.

Angeli e Demoni che non potevano procreare con la loro stirpe hanno scoperto il piacere della carne, o forse il dovere, di poter procreare con gli umani.

\subsubsection{Gli altri abitanti}\index{Risorti}

Gli altri abitanti sono i Risorti\index{Risorti}. Come nel peggiore degli incubi coloro che non sono stati richiamati al Signore nei giorni dell'Apocalisse sono diventati preda dei demoni.

I morti e risorti sono esseri dal potere non sempre uguale. Coloro che erano più affini a Satana ed alla Bestia hanno sviluppato poteri maggiori, capacità che definiremo ultraterrene se non magiche. La quasi totalità varia tra scheletri, zombi, spiriti e creature simili sempre affamate.

E mai, mai, mai farsi uccidere da un risorto! La tua anima, il tuo bene più prezioso, verrà altrimenti catturata ed offerta ai risorti di più alto grado se non ai demoni direttamente.

\subsubsection{La maledizione del 33}\index{33}

Per uno sciocco scherzo di Satana allo scoccare di ogni 33 anni di vita succede qualcosa di infausto.

Il primo accadimento è l'infertilità, maschile e femminile, nessuna oltre i 33 anni riesce a rimanere incinta e nessun maschio oltre i 33 anni ad essere più fertile.

Il secondo accadimento è che a 66 anni si muore, tutti. Semplicemente non ci si sveglia più. Il cuore cessa di battere e si muore per un pò.
E' scioccante vedere chi si ama, i propri amici morire così senza una causa apparente.

Ed è per questo che il compleanno dei 66 anni viene festeggiato anche più della nascita, con una festa che dura per giorni, con tutti propri cari, parenti ed amici, fino ad arrivare all'ultimo saluto quando il festeggiato va a riposare per un ultima volta come tutti noi lo conoscevamo.

Il problema è ai compimento dei 99 anni, od ai 33 anni dopo la morte, quando le forze demoniache reclamano il corpo ed il defunto risorge come un morto vivente.

Questi fatti sono risaputi ed in quasi tutte le famiglie si procede con la cremazione dei corpi dei defunti. Accade in remoti villaggi e quando il dolore per il distacco è troppo forte che i corpi vengano seppelliti in semplici sudari di stoffa. Quello succede dopo 33 anni potete ben immaginarlo.

Il trauma peggiore di tutti però è stato vedere gli avi, coloro che erano morti da tanti e tanti anni risorgere. L'Apocalisse ha fatto risorgere i morti e chi aveva la benedizione dell'Agnello è assunto in cielo, ma tutti gli altri, e vi assicuro che i loro numeri sono incalcolabili sono risorti come dannati, come spiriti affamati della poca vita rimasta.

Immense città finite sottoterra per gli immensi terremoti e fratture sono ora popolate di non morti e forse di qualche sopravvissuto, forse. Di certo tutti i loro tesori ed averi sono rimasti tutti li, pronti per il primo o forse secondo saccheggiatore.

I saggi sono convinti che sia una maledizione del grande Drago, di Satana, per beffeggiarsi dell'età in cui è morto il Figlio di Dio.
Satana ha il pieno potere delle creature rimaste sulla terra che non sono ascese nella Città Eterna.

Solo i Nefilim sono immuni a tutte e tre le maledizioni.

\subsection{Vivere e salvarsi}\index{Le Scelte}

Carissimi figlioli, dopo quando detto sembra ridicolo parlare di vivere e salvarsi, eppure una flebile speranza c'è sempre.

E' vero che a 33 anni non potrai avere più una discendenza, cosa pericolosissima in un mondo dove siamo rimasti veramente pochi a vivere, ed è altrettanto vero che a 66 anni morirai, ma c'è sempre una speranza. SEMPRE.

Grazie alle visioni indotte dagli angeli e purtroppo dai demoni abbiamo imparato che è possibile salvarsi, evitare l'infausta maledizione del 33 seppure dovendo scegliere di lasciare questo regno.

Ci sono stati lasciate dalle potenze angeliche queste possibilità di redenzione:

\begin{itemize}

\item compiere 100 distinte azioni di pura bontà
\item uccidere 1 demone maggiore
\item trovare una delle pietre bianche ed incidervi il proprio nuovo nome
\item portare la Speranza a 20 punti

\end{itemize}

chi adempie ad almeno una di queste missioni potrà ascendere alla Città Eterna e salvarsi.\\

Se invece vuoi seguire i dettami di Satana queste sono le opzioni che ti porteranno a diventare un Demone

\begin{itemize}

\item compiere 100 distinte azioni di pura malvagità
\item uccidere 100 anime non marchiate con il nome della Bestia o di Satana
\item uccidere 1 angelo maggiore
\item vivere l'intera vita ubbidendo solo ai sette peccati capitali
\item portare la Disperazione 20 punti

\end{itemize}

chi adempie ad almeno una di queste missioni potrà recarsi alla pozza di fuoco e zolfo dove Satana comanda le sue legioni e chiedere di diventare un Demone.

Si possono definire una salvezza queste scelte ? Non sta a me deciderlo ma al cuore delle persone che vorranno intraprenderlo, che vorranno lasciare questo inferno sulla terra per il Paradiso oppure per governarlo come empio Demone.

Molti altri intraprendono la vita dell'avventuriero alla ricerca dei tesori che giacciono incustoditi sopra e sotto il suolo sapendo che prima o poi Satana reclamerà la loro anima e la Bestia la loro vita.

\subsection{Le Stirpi di Dark Catacomb}\index{Le Stirpi di Dark Catacomb}\index{Stirpi}

Le stirpi presenti in Dark Catacomb sono 2, gli Umani ed i Nefilim.

I nefilim sono frutti degli incroci (\textit{proibiti}?) tra umani ed angeli o demoni.\index{Nefilim}

Solo per il fatto di avere sangue angelico o demoniaco non significa che siano buoni o malvagi a priori, a differenza dei loro progenitori ultraterreni i nefilim hanno un anima e come tale sono dotati di libero arbitrio.

I nefilim angelici sono solitamente di bell'aspetto, esadattili (sempre sei dita per mano), i maschi tendono ad avere la barba rossa e le donne una folta capigliatura nera come la pece.

I nefilim demoniaci sono solitamente alti oltre i due metri, con numerosa corna che amano ingioiellare ed ali da pipistrello (anche se molti preferiscono dire da Drago).

Un nefilim sente maggiormente il richiamo del sangue e della tradizione spirituale a cui appartiene ma non per questo seguirà forzatamente le orme del genitore ultraterreno.

A differenza degli umani un nefilim non è soggetto alla maledizione del 33, il suo percorso di vita può arrivare oltre i 250 anni. I nefilim possono procreare solo con umani, generando solo semplici umani.

Gli umani invece possono decidere di vivere come meglio credono i 66 anni che gli sono concessi.\index{Umani}


\section{I Tratti}\index{Tratti}\hypertarget{tratti}{}\label{tratti}

\begin{enfasi}{Chi dunque sa fare il bene e non lo compie, commette peccato. (Giacomo il Giusto 4.17, Lettera di Giacomo)
		\smallskip

		E' un diritto naturale saziarsi l'anima con la vendetta. (Attila)
		\smallskip

		Est Sularus Oth Mithas. ("Il mio onore è la mia vita", Giuramento dei Cavalieri di Solamnia)
}\end{enfasi}


	\index{Tratti}
	In DKC non c'è una netta distinzione tra bene e male, legge e caos, tra ciò che è giusto e ciò che è sbagliato.

	In DKC esistono i Tratti, aspetti e sfumature caratteriali che \textbf{contribuiscono} al background del personaggio, aiutano il giocatore a ruolare meglio e gli possono fornire quelle linee guida per interpretare in maniera più corretta il personaggio che si è voluto creare.

	Un Tratto è un dettaglio che aiuta meglio a inquadrare il personaggio, ne delinea i caratteri principali concedendogli sfumature diverse.

	\textbf{Ogni giocatore sceglie 5 Tratti per il proprio personaggio alla creazione del personaggio.} Questi che suggeriranno il personaggio nell'agire e nelle scelte.

	\begin{giocatore}[Scegliere i Tratti]
		I Tratti non sono il personaggio, non lo bloccano ne lo fissano eterno nel tempo. Un personaggio è sempre in costante evoluzione e così il suo carattere, morale, comportamento e desideri. Non essere rigido ma usa i Tratti per darti dei suggerimenti da cui farti ispirare.
	\end{giocatore}

	I Tratti non hanno accezione positiva o negativa, servono solo ad inquadrare il personaggio e capire quale Patrono è più interessato al personaggio. Non vogliono definire se sei buono o cattivo, ognuno ha la propria morale indipendentemente dai Tratti posseduti.

	\textbf{Al primo livello scegli un Tratto maggiormente caratteristico per il personaggio, questo avrà valore 1, gli altri 4 Tratti avranno valore 0.}

	Col passare del tempo e delle avventure i Tratti aumenteranno valore o potranno essere sostituiti, in concerto tra Narratore e giocatore in base a come giocato, da altri Tratti. \textbf{più è alto un valore di Tratto più questo è presente e permeante nelle scelte del personaggio}.

	Durante le avventure il Narratore a seguito di particolari scene e recitazione potrà fare aumentare di un punto un Tratto del personaggio.

	Ad esempio a seguito di una particolare situazione e climax di avventura il Narratore potrebbe concedere a tutti o qualcuno il Tratto Coraggio o dare un +1 a Coraggio a chi ha già questo Tratto. Per i Tratti non presi si considera il valore base in punti di -1, ovvero il primo punto serve per prendere il Tratto ed i successivi per enfatizzarli.

	Mentre è \emph{relativamente} facile acquisire nuovi Tratti è complicato cambiare quelli già presenti. Parlane con il Narratore, saprà preparare situazioni ed avventure che ti aiuteranno a comprendere come evolvere il personaggio ed eventualmente ad evolvere i Tratti scelti.

	Ogni azione particolarmente importante dove il personaggio abbia seguito un Tratto porta il personaggio ad avvicinarsi al \textbf{Patrono} competente per quel Tratto.

	All'aumentare del valore della somma dei Tratti comuni con il Patrono il personaggio potrà acquisire dei poteri, indipendentemente sia un credente o meno di quel Patrono.

	\noindent\begin{itemize}[leftmargin=*] \setlength{\itemsep}{0pt}
		\item A \textbf{'5'} punti si può incominciare a sentire la presenza di un Patrono

		\item A \textbf{'10'} punti si sente la vicinanza di un Patrono

		\item A \textbf{'15'} punti si è legati ad Patrono

		\item A \textbf{'20'} punti si è un campione del Patrono
	\end{itemize}

	Non è necessario credere in un Patrono per sentirne la vicinanza, esserne legati o campione, semplicemente è la propria natura (i propri Tratti) che è affine al Patrono, che lo si voglia o meno. I poteri si prendono solo dal Patrono che ha somma tratti più alta rispetto agli altri.

	Dato che lo scopo di un Patrono è fare che i propri Tratti siano dominanti sugli altri, avere persone di alto livello e potere che siano così affini a lui tornerà utile al momento di risorgere

	Per individuare il Patrono più affine, quello che vi darà i poteri, verificate il vostro Tratto a maggior valore sulla \hyperlink{tabellacollegamentopatronotratto}{Tabella Collegamento Patrono - Tratto} (pag. \pageref{tabellacollegamentopatronotratto}) ed individuate il Patrono che quel Tratto maggiormente caratterizzante, in caso il Tratto fosse condiviso tra più Patroni verificate gli altri Tratti ed in base alla somiglianza scegliete il Patrono.
	Verificate poi in \hyperlink{cosmologia}{Cosmologia} (pag. \pageref{patroni}) i poteri concessi dal Patrono. Questo controllo è opportuno farlo ad ogni aumento di valore di Tratto.

	\medskip

	\textbf{Lista dei Tratti}\index[Tabelle]{Tabella dei Tratti}

	Ogni Tratto è brevemente descritto nel suo significato generico. Il personaggio è libero di interpretare il Tratto come più lo sente proprio.

	\medskip

	\noindent\begin{itemize}[leftmargin=*] \setlength{\itemsep}{0pt}
		\item \textbf{Altruista}: Persona che mette gli altri al primo posto, anche sacrificando i propri bisogni.
		\item \textbf{Ambizioso}: Pensa solo ai propri interessi e bisogni, senza considerare quelli degli altri
		\item \textbf{Arrogante}: Ha un'opinione esagerata di sé stesso e tende a sminuire gli altri.
		\item \textbf{Avaro}: Eccessivamente attaccato ai propri beni materiali e riluttante a condividere.
		\item \textbf{Cinico}: Tende a vedere il peggio nelle persone e nelle situazioni, spesso con un atteggiamento sprezzante.
		\item \textbf{Codardo}: Che manca di coraggio e tende a evitare situazioni di pericolo.
		\item \textbf{Compassionevole}: Mostra empatia e comprensione verso le sofferenze degli altri.
		\item \textbf{Coraggioso}: Affronta le paure e le sfide con determinazione.
		\item \textbf{Crudele}: Senza pietà e compassione, provoca sofferenza intenzionalmente.
		\item \textbf{Curioso}: Ha un forte desiderio di conoscere e imparare cose nuove.
		\item \textbf{Disonesto}: Non dice la verità e inganna gli altri per il proprio vantaggio.
		\item \textbf{Dissoluto}: Vive in modo sregolato e senza considerare le conseguenze morali delle proprie azioni.
		\item \textbf{Entusiasta}: Mostra grande energia e passione per ciò che fa.
		\item \textbf{Estroverso}: Socievole e a proprio agio in situazioni sociali.
		\item \textbf{Gentile}: Tratta gli altri con rispetto e considerazione.
		\item \textbf{Impulsivo}: Tende ad agire e reagire senza pensare troppo alle conseguenze.
		\item \textbf{Indeciso}: Non riesce rapidamente a prendere decisione soffermandosi troppo nel ponderare le scelte.
		\item \textbf{Intransigente}: Non è disposto a scendere a compromessi o a considerare punti di vista diversi.
		\item \textbf{Invidioso}: Prova risentimento verso chi possiede qualcosa che lui desidera.
		\item \textbf{Leale}: Fedele e affidabile nei confronti degli amici e delle persone care.
		\item \textbf{Paziente}: Capace di aspettare senza irritarsi o perdere la calma.
		\item \textbf{Prudente}: Pondera attentamente le situazioni difficili o pericolose.
		\item \textbf{Sospettoso}: Sei convito che tutti abbiano interesse a danneggiarti.
		\item \textbf{Testardo}: Determinato e persistente nel raggiungere i propri obiettivi, nonostante le difficoltà.
		\item \textbf{Vanitoso}: Sei certo delle tue eccezionali qualità, capacità ed aspetto.
		\item \textbf{Vendicativo}: Cerca di punire chi gli ha fatto un torto, spesso in modo sproporzionato.

	\end{itemize}

	\medskip


\section{I Patroni}

I Patroni sono, collettivamente, come si fanno chiamare le potenze Angeliche e Demoniache presenti sulla Terra.

Satana ben sapendo che gli uomini hanno una intrinseca necessità nel credere in qualcosa, preferibilmente di tangibile e reale ha schierato 13 Archidiavoli perché fossero adorati e facessero da \emph{guida} ai mortali, per condurli alla sua salvezza.

Non è dato sapere il perché, si pensa che sia stato un atto di volontà del Supremo e per nostra grazie anche 13 Angeli sono presenti tra i Patroni, ovvero abbiamo la possibilità di credere in qualcosa di concreto e visibile, tangibile e reale.

\noindent\begin{tabularx}{\linewidth}{lX}
	\toprule
	Nome & Caratteristiche \\
	\midrule
Lucifero  & Arrogante, Ambizioso, \textit{Curioso} \\
Belzebù   & Crudele, Impulsivo, \textit{Estroverso} \\
Asmodeo   & Dissoluto, Cinico, \textit{Entusiasta} \\
Mammon    & Avaro, Disonesto, \textit{Paziente} \\
Leviatano & Sospettoso, Vendicativo, \textit{Coraggioso} \\
Baal      & Crudele, Vanitoso, \textit{Leale} \\
Astaroth  & Invidioso, Arrogante, \textit{Gentile} \\
Belial    & Disonesto, Dissoluto, \textit{Altruista} \\
Baphomet  & Intransigente, Cinico, \textit{Compassionevole} \\
Lilith    & Disonesto, Vanitoso, \textit{Prudente} \\
Abaddon   & Vendicativo, Sospettoso, \textit{Estroverso} \\
Valak     & Codardo, Impulsivo, \textit{Curioso} \\
Mefistofele & Crudele, Avaro, \textit{Leale} \\
\bottomrule
\end{tabularx}


\noindent\begin{tabularx}{\linewidth}{lX}
	\toprule
	Nome & Caratteristiche \\
	\midrule
Michele   & Coraggioso, Leale, \textit{Intransigente} \\
Gabriele  & Compassionevole, Gentile, \textit{Vanitoso} \\
Raffaele  & Paziente, Estroverso, \textit{Cinico} \\
Uriele    & Entusiasta, Curioso, \textit{Testardo} \\
Sealtiel  & Altruista, Paziente, \textit{Impulsivo} \\
Jehudiel  & Leale, Coraggioso, \textit{Sospettoso} \\
Barachiel & Compassionevole, Estroverso, \textit{Indeciso} \\
Metatron  & Prudente, Curioso, \textit{Arrogante} \\
Sandalphon& Estroverso, Gentile, \textit{Avaro} \\
Anael     & Entusiasta, Compassionevole, \textit{Crudele} \\
Zadkiel   & Curioso, Paziente, \textit{Invidioso} \\
Phanuel   & Altruista, Gentile, \textit{Codardo} \\
Cassiel   & Prudente, Leale, \textit{Ambizioso} \\
\bottomrule
\end{tabularx}


Ogni Patrono è caratterizzato da 3 Tratti, ovvero \emph{sovraintende} a quell'aspetto caratteriale. Si, avete capito bene, se il vostro Tratto principale e' \emph{Crudele} allora è probabile che \emph{Mefistofele} vi abbia già adocchiato.

Ma non temete, per tutti c'è possibilità di salvezza, nella loro bontà ultraterrena i Patroni Angelici possono fare da guida a chi si macchia di un peccato per condurli alla salvezza eterna.

Ognuno di questi Patroni, a seconda del punteggio del Tratto che il personaggio ha, concede dei benefici, per la propria gloria e testimonianza.


\section{Speranza e Disperazione}

La Speranza o Fede per chi crede in un Patrono rappresenta una forza reale, mistica, ultraterrena che appartiene a ogni creatura vivente.

I personaggi hanno tutti un valore di Speranza che può aumentare o diminuire a seconda delle azioni fatte dai personaggi stessi.

Il valore di Speranza e Disperazione si lega filo doppio con il valore dei Tratti che un personaggio ha.


\section{La Magia}

\begin{changemargin}{0.3cm}{0.3cm}\begin{enfasi}{
Non lascerai vivere colei che pratica la magia. (Libro dell'Esodo 22,17-18)\\

"Non si trovi in mezzo a te chi fa passare il figlio o la figlia nel fuoco, chi usa la divinazione, chi fa presagi, chi pratica la magia, chi fa incantesimi, chi consulta gli spiriti, chi evoca i morti. Infatti chiunque fa queste cose è in abominio al Signore; e a motivo di queste abominazioni il Signore tuo Dio sta per scacciare quelle nazioni davanti a te. (Deuteronomio 18:10-12)
} \end{enfasi}\end{changemargin}

La magia è sempre stata insieme a noi, occultata e protetta da sguardi curiosi.

La magia è il dono della Madre alle sue figlie per le loro figlie.\index{Matrilineare}
La discendenza magica è quasi esclusivamente matrilineare, ovvero la figlia di una strega sarà a sua volta una strega.

Ci sono rari casi in cui può saltare una o due generazioni ma mai di più.

Stregoni di altro genere sono estremamente rari.

La magia si manifesta in incantesimi grazie ad una sintesi di canto e gesti che canalizzano l'energia dalla Sorgente.

\subsection{Le caratteristiche degli incantesimi}\index{Le caratteristiche degli incantesimi}\label{caratteristicheincantesimi}

La descrizione di ciascun incantesimo inizia con un blocco di informazioni che comprende il nome dell'incantesimo, i prerequisiti, le Azione richiesti di lancio, il Costo, gittata e durata dell'incantesimo, rarità, la descrizione dell'incantesimo e gli effetti aggiuntivi che si possono avere in caso di successo nella Prova di Magia.

Quando un personaggio lancia qualsiasi incantesimo, si usano le seguenti regole base indipendentemente dall'effetto dell'incantesimo.

\subsubsection{Imparare un nuovo incantesimo}

Una Strega che trova un incantesimo che non conosce deve farsi insegnare a formularlo con una Prova di Incantamento.

\begin{narratore}
In Dark Catacomb non esistono i libri di magia o le pergamene di incantesimo, non c'è sistema che permetta di scrivere il canto di un incantesimo. Quando si trova una Strega che conosce un nuovo incantesimo lo si deve imparare.\\
La cosa più simile ad una \textit{pergamena} può essere una boccetta con dentro il canto dell'incantesimo.
\end{narratore}

\pagebreak

\section{Ambiente}\index{Ambiente}

\label{ambiente}
\begin{changemargin}{0.3cm}{0.3cm}\begin{enfasi}{
poiché per mezzo di lui\\
sono state create tutte le cose,\\
quelle nei cieli e quelle sulla terra,\\
quelle visibili e quelle invisibili... (Colossesi 1,16)\\\medskip

Le terre non si potranno vendere per sempre, perché la terra è mia e voi siete presso di me come forestieri e inquilini (Levitico 25:23)
}\end{enfasi}\end{changemargin}\medskip


\section{Elenco oggetti magici}

\subsection{Armi}

\textbf{Spada Corta Angelica}\index{Spada Corta Angelica}: questa spada corta, ma sono note anche rare spade lunghe, è estremamente affilata ed in grado di ferire qualsiasi Demone. Bonus +1\\

\textbf{Martello Divino}\index{Martello Divino}: questo maglio da guerra è di un metallo lucente, con rune ed il simbolo di un toro su una delle due teste. Il martello è avvolto da fiamme sacre che causano 1d6 di danno da fuoco. Bonus +1\\

\textbf{Gran Croce dei Santi}\index{Gran Croce dei Santi}: questo flagello pesante è costruito come una croce al quale sono attaccate diverse catene con al termine palle chiodate. Bonus +2\\

\textbf{Mazza del pentimento}\index{Mazza del pentimento}: questa mazza chiodata quando colpisce infligge 1d6 danni aggiuntivi a Demoni e creature con il marchio della Bestia o il numero di Satana. Bonus +1\\

\textbf{Spada in Corno demoniaco}\index{Spada in Corno demoniaco}: questa spada, lunga o spadone, è costruita da un corno di un demone maggiore. Tramite lungo lavoro il corno viene levigato fino a creare un arma nera con venature rosse che goccia un liquido acido. L'arma causa 1d8 danni aggiuntivi a Vigore da acido. Bonus +2\\

\textbf{Pugnale Assetato}\index{Pugnale Assetato}: questo pugnale quando colpisce una creatura vivente trasferisce 1d4 Vigore alla creatura che lo brandisce. Bonus +1\\

\textbf{Ascia taglia ali}\index{Ascia tagliaali}: questa ascia da battaglia ha la capacità di impedire alle creature di volare. Ogni volta che la creatura viene colpita deve fare una Prova di Corpo con un MS +6 o non poter volare per 10 round. Bonus +1\\

\textbf{Alabarda Infernale}\index{Alabarda Infernale}: questa arma, che può essere anche una lancia od un falcione ad asta, viene solitamente brandito dagli ufficiali demoniaci. Bonus +1

\subsection{Armature e Scudi}

\textbf{Scudo Gran Croce}\index{Scudo Gran Croce}: questo scudo pesante di metallo è decorato con la Croce. Bonus +1\\

\textbf{Armatura del Pio}\index{Armatura del Pio}: questa armatura di cuoio consumato rossiccio è di semplice fattura. Il Bonus aumenta di 1 quando il Vigore di chi la porta scendo sotto la metà. Bonus +1\\

\textbf{Armatura del fustigatore}\index{Armatura del fustigatore}: questa armatura di scaglie è schizzata del sangue di demone. L'arma brandita da chi indossa questa armatura guadagna un Bonus di +1 al danno. Bonus +1\\

\textbf{Armatura della Benedizione del Sacro Cuore}\index{Armatura della Bendizione del Sacro Cuore}: questo pettorale di armatura porta i fregi del Sacro Cuore in azzurro e rosso. All'alba chi indossa l'armatura recupera 1d6 punti Vigore. Bonus +2\\

\textit{Scudo Nero grondante}\index{Scudo Nero grondante}: questo scudo medio di metallo nero è picchiettato di gocce di sangue. Chi brandisce lo scudo può usare 6 PA per costringere l'avversario ad effettuare una Prova di Corpo con MS almeno +3 o 3d4 aghi sparati dallo Scudo lo colpiscono; ogni dado causa 1 punto di danno. L'abilità è usabile 1 volta al giorno. Bonus +1\\

\textbf{Scudo dell'occhio maledetto}\index{Scudo dell'occhio maledetto}: questo piccolo scudo di metallo ha dipinto un occhio al centro. Utilizzando 6 PA la creatura che fronteggi deve effettuare una Prova di Corpo oppure avere 3 Penalità alla successiva PDAA. Bonus +1\\

\textbf{Armatura del fumo infernale}\index{Armatura del fumo infernale}: un fumo denso e nero ti avvolge a guisa di armatura. Questa armatura completa ha una Penalità di 1, richiede Corpo 10, ed ha Ingombro 6. Ogni giorno che la si porta almeno 6 ore è necessario effettuare una Prova di Mente con MS +6 o essere segnati con il nome della Bestia. Bonus +1

\textbf{Armatura delle stelle cadute}\index{Armatura delle stelle cadute}: in questa armatura ogni anello raffigura una stella del firmamento. Ogni giorno puoi usare fino a 10 PA per fare si che un dardo di energia esca da un anello e colpisca una creatura entro 9 metri. Ogni Dardo costa 1 PA di attivazione. Ogni Armatura a 5d10 anelli ancora attivi quando si trova. Bonus +1\\

\textbf{Armatura dei Generali di ghiaccio}\index{Armatura dei Generali di ghiaccio}: questa mezza armatura dei generali del Cocito è estremamente resistente al fuoco. L'armatura concede di dimezzare tutto il danno da fuoco o Fiamme Sacre ed Infernali. Bonus +2

\subsection{Altri indossabili}
\textbf{Cintura degli attrezzi}\index{Cintura degli attrezzi}: questa umile cintura è fatta in pelle resistente e lavorata. Contiene fino a 4 (1d4) tasche ognuna delle quali può contenere 1 di Ingombro.

Contiene anche altre 3 tasche speciali:\\
- in una è sempre presente un martello, equivalente ad un martello da guerra\\
- in una ci sono 2d20 chiodi da carpentiere, viti e bulloni. \\
- in una c'è un rotolo di spago, un filo piombato, una livella\\
Ogni giorno gli oggetti delle tasche speciali vengono ripristinati se usati\\

\textbf{Sacco del sale}\index{Sacco del sale}: questo piccolo sacchetto contiene sale di origine divina. Gettato attorno ad un demone, richiede 1 minuto per preparare il circolo, impedisce allo stesso di uscire dal circolo.\\

\textbf{Stivali verdi}: questi stivali in cuoi leggero sono estremamente comodi e robusti. Concedono 1 Bonus alla Prova di Arrampicarsi e di Furtività per muoversi silenziosamente\\

\textbf{Guanto del Re Pescatore}\index{Guanto del Re Pescatore}: questo sacro guanto adornato con numerosi anelli che raffigurano santi ed alti prelati quando indossato rende l'arma dalla stessa mano brandita eccellente contro Demoni e risorti. Contro queste creature concede +2 Bonus a PDAA e l'arma è sempre in grado di ferire queste creature\\

\textbf{Borsetta chiara}\index{Borsetta chiara}: questa piccola borsetta di una stoffa leggera e molto chiara ha una Capacità di Carico di 16 pur non pesando nulla.\\

\textbf{Guanti insanguinati}\index{Guanti insanguinati}: questi guanti di pelle sono incrostati di sangue. Ogni ferita causata dall'arma impugnata provoca Sanguinamento 1.\\

\textbf{Spallaccio viola}\index{Spallaccio viola}: questo spallaccio brilla di una debole luce quando entro 50 metri c'è un Demone od un risorto.\\

\textbf{Stivali del mare}\index{Stivali del mare}: questi stivali ottenuti dalla pelle di qualche pesce marino concedono a chi li porta 3 Bonus alle prove di Nuotare.

\textbf{Sette croci dell'Agnello}\index{Sette croci dell'Agnello} ognuna di queste croci, di piccole dimensioni apparentemente d'argento, può essere trovata singolarmente.

Quando messa in una collana attorno al collo ognuna conferisce un potere diverso.

- Concede 1 Bonus alla prove di Corpo\\
- Concede 1 Bonus alla prove di Mente\\
- Concede 1 Bonus alla prove di Volontà\\
- Concede 1 Bonus alle prove di Conoscenza Biblica\\
- Ogni giorno all'alba recuperi 1 punto di Affaticamento\\
- Concede 2 Bonus per riconoscere Risorti e Demoni\\
- Permette di parlare una (a caso) antica lingua terrestre\\

\textbf{Benda della visione}\index{Benda della visione}: questa benda in stile pirata da mettere su un occhio concede all'occhio coperto di vedere nell'oscurità.

\subsection{Anelli, Bracciali ed indossabili preziosi}
Sigillo Supremo
Anello del Fuoco Eterno
Fede del Legame
Anello solare

Anello dell'Eterno Fuoco
Anello del Cuore nero
Sigillo Infernale
Bracciale della Bestia

\subsection{Altri oggetti}

\textbf{Ciotola della manna}\index{Ciotola della manna}: ogni giorno questa ciotola alle 12.00 si riempie di una zuppa nutriente ma insapore. Il contenuto è sufficiente a sfamare una persona per un giorno.\\

Rielaborando materiali già magici ed infondendoli di ulteriore magia dalla Sorgente è possibile andare a creare ulteriore oggetti magici.

E' lasciata alla fantasia dell'Demiurgo il ritrovamento di oggetti magici, il suggerimento che lascio è che siano oggetti che possano avere una utilità anche per i padroni originali.


\end{multicols}

\pagebreak

\section{Mostruario}

\subsection{Introduzione}

I Mostri in Dark Catacomb sono reali e presenti.
I personaggi sanno di doverseli aspettare ovunque ed ancora di più se si mettono a curiosare in zone inesplorate.

In DKC i mostri ti danno la caccia, ti inseguono e se un demone decide di nutrirsi delle vostre anime allora preparatevi a scappare!

Un bivacco attorno al fuoco in mezzo al deserto può essere sicuro quanto dormire in una città sotto l'assedio dei Risorti, in qualsiasi momento un demone potrebbe decidere di velocizzare il tutto ed intervenire trasportando le "sue truppe" da una parte all'altra.

In DKC le tabelle che permettono di capire che incontri si fanno hanno un ruolo importante.


\subsection{Tipo}

I mostri principali di DKC sono i Risorti, i Demoni, gli Angeli e le creature mostruose che la Bestia si è divertita a creare. A tutti questi va aggiunto qualsiasi persona che abbia o meno motivo per attaccarvi.


Il gioco comprende i seguenti tipi di mostri:

\smallskip\textbf{Aberrazioni}, creature totalmente aliene, figli dell'incubo di un demone.

\smallskip\textbf{Bestie},  alcune possiedono poteri magici, ma la maggior parte è priva di intelligenza e non ha alcuna forma di società o linguaggio. Esempi classici di bestie sono tutte le specie di animali comuni, i dinosauri e le versioni giganti degli animali.

\smallskip\textbf{Celestiali}, angeli e spiriti celesti.

\smallskip\textbf{Costrutti}, sono creati e non partoriti. Alcuni sono programmati dai loro creatori per seguire una semplice serie di istruzioni, mentre altri sono senzienti e capaci di pensare per proprio conto. I golem sono i costrutti più rappresentativi.

\smallskip\textbf{Draghi}, sono grandi creature rettili di antica origine ed enorme potere. I Draghi sono stati richiamati da Satana e tutti obbediscono a lui. In questa categoria si collocano anche creature lontanamente imparentate con i veri draghi, ma meno potenti, meno intelligenti e meno magiche, come le viverne e gli pseudodraghi.

\smallskip\textbf{Elementali}, sono creature fatte di pure elemento tenuti in vita dalla magia.
Alcune creature di questo tipo sono poco più che masse animate del rispettivo elemento, e includono le creature chiamate semplicemente elementali.

\smallskip\textbf{Giganti}, troneggiano sugli umani e i loro simili. Sono di forma umana, sebbene alcuni abbiano più teste (ettin) o deformità (fomori). Le sei varianti dei veri giganti sono gigante di collina, gigante di pietra, gigante del gelo, gigante del fuoco, gigante delle nuvole, gigante delle tempeste. Oltre questi, anche ogri e troll sono giganti.

\smallskip\textbf{Demoni}, angeli rinnegati, creature umane ascese all'odio più pure. Le schiere dei Demoni sono pressoché illimitate ma nulla possono contro il volere di Nostro Signore. I Demoni sono esseri malvagi, bugiardi, traditori. Per quanto il loro numero sia vasto Satana preferisce uccidere gli umani con gli umani, i Risorti, con il supporto di demoni di basso grado ed il comando dei suoi luogotenenti.

\smallskip\textbf{Melme}, sono creature gelatinose che difficilmente hanno una forma fissa. Vivono principalmente sottoterra, stabilendosi in grotte e sotterranei, nutrendosi di rifiuti, carcasse o creature tanto sfortunate da incapparvi. I protoplasmi neri e i cubi gelatinosi sono tra le melme più riconoscibili.

\smallskip\textbf{Mostruosità}, sono mostri nel senso più stretto del termine creature spaventose che non sono comuni, né davvero naturali, e quasi mai benigne.
Sfuggono a qualsiasi categorizzazione, e in qualche modo servono da categoria onnicomprensiva per quelle creature che non corrispondono a nessun altro tipo di mostro.

\smallskip\textbf{Risorti}, sono creature un tempo vive e poi morte eppure ritornate dopo il grande risveglio dei morti o perché morti da 33 e più anni. La maggior parte sono simili a zombi o scheletri ma più in vita la creatura si era comportata in maniera malvagia più ha attirato il male in se al momento del risveglio andando a creare tipologie di non morti ogni volta diverse.

\smallskip\textbf{Piante}, in questo contesto si tratta di creature vegetali, non della normale flora. La maggior parte di esse sono mobili, e alcune sono carnivore. Per quanto rari i demoni si sono divertiti a mutare e corrompere la natura generando esseri contorti ed alieni.

\smallskip\textbf{Umanoidi}, erano la popolazione principale della Terra ora assieme ai Nefilim sono la minoranza. E' possibile trovare umani mutati dalla magia corruttrice dei demoni che hanno gli aspetti più diversi.

\medskip

\pagebreak


demoni della bibbia

Abaddon: In alcune interpretazioni, è un angelo caduto o un demone dell'abisso.
Asmodeo: Menzionato nel Libro di Tobia, è un demone associato all'incesto e agli spiriti maligni. Un demone spesso associato alla lussuria e ai desideri sessuali.

Astaroth: Un demone associato all'occultismo e ai rituali magici.Astoret: Un'entità idolatrica spesso associata al male. (1 Re 11:5, 1 Samuele 7:3-4)
Astoret o Astarte: Divinità idolatrica spesso associata al male. (1 Re 11:5, Giudici 2:13)



Azazel: Menzionato nel Libro di Levitico come parte del rituale del capro espiatorio nel Giorno dell'Espiazione. In alcune tradizioni apocrife, è uno dei tre demoni che guidano i caduti.
Baal: Anche se il termine può riferirsi anche a divinità non demoniache, talvolta è associato a forze maligne o idolatria.
Belial: Un termine usato in diverse parti della Bibbia, spesso per riferirsi a persone malvagie o depravate.
Belzebù: Conosciuto come il "signore delle mosche", è menzionato nel Nuovo Testamento come un principe dei demoni.

Chemos o Chemosh: Un dio pagano associato a sacrifici. (1 Re 11:7)


Dagon: Un'idolo filisteo associato a un dio della fertilità. (Giudici 16:23, 1 Samuele 5:1-5)


Legione: Nel Nuovo Testamento, è il nome di un gruppo di demoni che possedeva un uomo.
Lilith: Una figura nel folclore ebraico, a volte considerata un demone, associata all'oscurità e alla sessualità.
Mammona: Un termine utilizzato da Gesù nei vangeli per riferirsi alla ricchezza o al denaro personificati come un idolo.
Moloch: Un idolo a cui venivano offerti sacrifici umani e che rappresentava il male. (Levitico 18:21, Geremia 32:35)


Mastema: Compare nel Libro di Giubileo, dove viene descritto come un demone associato all'oppressione.
Samael: Spesso associato all'angelo della morte o all'accusatore.
Satana o Lucifero: Descritto come il capo dei demoni, colui che si ribellò contro Dio e fu cacciato dal cielo. Presente in varie parti della Bibbia, come ad esempio nel libro di Isaia e nel Nuovo Testamento. Un angelo caduto che rappresenta il male e la ribellione contro Dio.

Serapide: Un'idolatria spesso associata alla cultura egizia. (Atti 19:23-41)


arcangeli


Arcangelo Michele: Spesso riferito come un capo degli angeli, associato alla protezione e alla lotta contro le forze del male. (Daniele 10:13, Giuda 1:9)

Gabriele: Annunciò la nascita di Giovanni Battista e di Gesù ai loro genitori. (Luca 1:11-20, 1:26-38)

Raffaele: Appare nell'Apocrifo di Tobia, aiutando Tobia in diverse avventure. (Tobia 3:16-17, 5:4-28, 12:15)


Michele: L'unico arcangelo esplicitamente menzionato nella Bibbia. Viene spesso descritto come un guerriero spirituale e un difensore del popolo di Dio. (Daniele 10:13, Giuda 1:9)

Gabriele: Pur non essendo sempre menzionato come arcangelo nella Bibbia, è spesso considerato un arcangelo per il suo ruolo di portatore di importanti messaggi divini, incluso l'annuncio dell'incarnazione a Maria. (Luca 1:19, 26)

Raffaele: Non menzionato esplicitamente come arcangelo nella Bibbia, è citato nel Libro di Tobia come l'angelo che accompagna il giovane Tobia in un viaggio. (Tobia 3:17, 12:15)

Uriele: Il nome non compare nella Bibbia, ma è menzionato in alcuni testi apocrifi e nelle tradizioni ebraiche e cristiane. Viene spesso associato a un angelo di luce, conoscenza o preghiera.

Uriel: Menzionato in alcuni testi apocrifi ed extra-biblici come un angelo della divina giustizia.

Raziel: Anche questo nome non compare nella Bibbia, ma è menzionato in alcuni testi apocrifi e nella tradizione ebraica. Raziel è associato a segreti e conoscenze arcane.


Saraqael: Ancora una volta, non menzionato esplicitamente nella Bibbia, è presente in alcune tradizioni apocrife ebraiche e cristiane.

angeli varie

Angeli Custodi: Anche se non menzionati con nomi specifici nelle Sacre Scritture, si crede che Dio assegni agli individui angeli custodi per proteggerli e guidarli.

Angeli dell'Apocalisse: Descritti nell'Apocalisse di Giovanni come figure che eseguono giudizi e compiti divini. (Apocalisse 7:1-2, 8:2)

Serafini: Descritti come esseri con sei ali che adorano Dio e che appaiono nel tempio celeste. (Isaia 6:1-7)

Cherubini: Esseri con ruoli di custodia e protezione, spesso associati alla presenza divina. (Genesi 3:24, Ezechiele 10:1-22)

Angeli dell'Annunciazione: Gli angeli che annunciavano eventi importanti, come l'annunciazione a Maria. (Luca 1:26-38)

Angeli che lodano Dio: Esseri che adorano e lodano Dio costantemente, come descritto in Apocalisse e altri passi biblici. (Apocalisse 4:8-11)

Angeli che eseguono giustizia: Esseri inviati da Dio per eseguire la sua volontà e giudizio, come nei racconti dell'Antico Testamento.

Eserciti angelici: Gruppi di angeli che servono Dio e sono pronti a eseguire i suoi comandi. (Luca 2:13, Apocalisse 19:14)

Angeli messaggeri: Angeli inviati per consegnare messaggi o guidare gli individui in situazioni specifiche. (Numeri 20:16, Atti 7:53)

Angeli della resurrezione: Descritti come annunciatori della risurrezione e della venuta di Cristo. (Matteo 28:2-7, 1 Corinzi 15:52)

bestie


Leviatano: Un mostro marino menzionato nella Bibbia, spesso interpretato come un simbolo delle forze del caos.
Behemoth: Un altro animale descritto nel Libro di Giobbe, spesso interpretato come un grande e potente animale terrestre. (Giobbe 40:15-24)

Scorpione: Nel contesto delle profezie apocalittiche, rappresenta forze maligne e giudizio. (Apocalisse 9:1-11)

Unicorno: Menzionato in alcune traduzioni della Bibbia, potrebbe riferirsi a un animale mitico o a un rinoceronte. (Deuteronomio 33:17, Numeri 23:22, 24:8)

Bestia dalla terra: Menzionata nell'Apocalisse, rappresenta un'altra potenza maligna. (Apocalisse 13:11-18)

test2

\begin{narratore}

	Personalmente ho trovato di grande interesse personale e formativo Le Confessione di Sant'Agostino e Gli esercizi spirituali di Sant'Ignazio di Loyola.

	Vi suggerisco di nutrire la vostra spiritualità perché, se ancora non ne avete percezione, ha sempre bisogno di ispirazione e illuminazione.
\end{narratore}

{\small \printindex}

\end{document}

