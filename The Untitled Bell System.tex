%%\documentclass[11pt,twoside]{book}
\documentclass[a4paper,11pt,twoside,openany]{dndbook}
\usepackage{multicol}
\usepackage{lmodern}
\usepackage[T1]{fontenc}
\usepackage[utf8]{inputenc}
%%\usepackage[a4paper]{geometry}
\geometry{verbose,tmargin=2cm,bmargin=2.5cm,lmargin=1.5cm,rmargin=2cm}
\setcounter{secnumdepth}{-1}
\usepackage{longtable}
\usepackage{booktabs}
\usepackage{textcomp}
\usepackage{url}
\usepackage[italian]{babel}
\usepackage{setspace}
\usepackage{graphicx}
\PassOptionsToPackage{normalem}{ulem}
\usepackage{ulem}
\usepackage{makeidx}
\usepackage[unicode=true,
bookmarks=true,
pdftitle={TUS - The Untitled System},pdfauthor={Andres Zanzani},
breaklinks=false,pdfborder={0 0 1},backref=section,colorlinks=false]
{hyperref}
\hypersetup{hidelinks,pdfcreator={LaTeX}}
\usepackage{bookmark}
\usepackage{supertabular}
%%\usepackage{DejaVuSans}
%%\usepackage{ebgaramond}
%%\usepackage{libertine}
\usepackage{palatino}

\usepackage{fancyhdr}

\fancyhf{} % clear all header and footers
\renewcommand{\headrulewidth}{0pt} % remove the header rule
\fancyfoot[LE,RO]{\thepage} % Left side on Even pages; Right side on Odd pages
\pagestyle{fancy}
\fancypagestyle{plain}{%
	\fancyhf{}%
	\renewcommand{\headrulewidth}{0pt}%
	\fancyhf[lef,rof]{\thepage}%
}

\usepackage{pdfpages}
\makeindex
\makeatletter
\makeatother
\raggedbottom

\usepackage{array}
\newcolumntype{L}[1]{>{\raggedright\let\newline\\\arraybackslash\hspace{0pt}}m{#1}}
\newcolumntype{k}[1]{>{\centering\let\newline\\\arraybackslash\hspace{0pt}}m{#1}}
\newcolumntype{R}[1]{>{\raggedleft\let\newline\\\arraybackslash\hspace{0pt}}m{#1}}
\newcolumntype{D}[1]{>{\centering}m{#1}}

\begin{document}

\title{The Untitled System (TUS)}
\date{\today\\\textbf{Gauss Edition} v3.0.1\\\includegraphics[width=6.85139in,height=3.79514in]{image1.png}}
\author{Andres Zanzani}
\maketitle
\thispagestyle{empty}
%%\includegraphics[width=6.85139in,height=3.79514in]{./media/image1}

\newpage~\newpage~


%%\normalsize

%%\linespread{1.5}

Dedicato all'unica Donna mai amata, colei che ogni giorno mi accompagnanei sogni

Mai rinunciare ai tuoi sogni.
\thispagestyle{empty}

\newpage~\thispagestyle{empty}%%\newpage~\thispagestyle{empty}

\setcounter{page}{1}

%%\begin{multicols}{2}
\tableofcontents{}
%%\end{multicols}

\pagebreak{}

\section{Introduzione}

\pagestyle{plain} 
\begin{quotebox}Si puo' scoprire di piu' su una persona in un'ora di gioco che in un anno di conversazione. (Platone)
\end{quotebox}

Per chi ha gia' esperienza con i giochi di ruolo: TUS e' un rpg ispirato a D20 OGL che usa 3d6 al posto del d20 per una distribuzione normalizzata. E' un sistema classless dove la magia e' freeform ed e' ispirata a Ars Magica e Talislanta. Il sistema e' basato su prove che possono esplodere (anche il danno) ed un sistema di svantaggi e vantaggi alla Gurps. 

Per tutti gli altri...

TUS e' un gioco di ruolo cooperativo e narrativo nel quale i giocatori creano personaggi che vivranno fantastiche e strabilianti avventure. Il Narratore si preoccupera' di dipanare la storia e fare partecipare i personaggi. Come in un gioco di narrazione ogni personaggio contribuira' attivamente alla storia con le sue scelte, le sue decisioni e le sue azioni.

Se sei un giocatore allora preparati a creare un personaggio che ti accompagnera' in terribili caverne e oscuri villaggi e fin sulle vette delle montagne piu' alte. Sarai tu a decidere tutto, dall'aspetto,al nome, alle sue capacita' e cio' che possiede. Sara' un pirata rubacuori o un cavaliere timido.. un barbaro delle steppe o uno stregone ? Oro o Gloria ? Onore o Tirannia ? La scelta dipende sempre e solo da te.

Se sei il Narratore invece tu governi il mondo, la storia, l'avventura. Il tuo ruolo e' di illustrare lo scenario in cui i giocatori si muovono e prendono decisioni. Li condurrai nelle profondita' della terra alla ricerca del Tomo dimenticato di Atmos oppure a sfidare i grandi Draghi per la corona dell'Onniscienza?

Il tuo compito non e' facile, usa fantasia, buon senso e la regola principale: divertititi. Quando sei in difficolta' non cercare la regola precisa, usa la tua piu' grande alleata: l'immaginazione, unisci un pizzico di assennatezza e cerca di stupire i giocatori. Lo scopo e' sempre e solo uno, divertirsi insieme e crescere, come giocatori, come personaggi, come amici.

Oltre questo manuale avrai bisogno anche di un po' di dadi, i classici usati nei giochi di ruolo.
Chiamati solitamente d4, d6, d8, d10, d12, d20 stanno ad indicare un dado a 4 facce, il dado a 6 facce (di questi devi averne 3 o 4 almeno!), il dato a 8 facce, quello a 10 (solitamente vengono venduti in coppia, per poter ottenere il d100), il solitario dado a 12 facce e il sempiterno dado a 20 facce.
Ogni qual volta ti verra' chiesto di tirare un dado questo sara' scritto con la notazione XdZ, ovvero tiro X volte un dado con Z facce. Es. 4d6 indica di tirare 4 volte il dado a 6 facce.

Anche qualche miniatura potrebbe essere necessaria, altrimenti anche gli omaggi di merendine o degli ovetti al cioccolato possono essere sufficienti.

All'interno di questo manuale troverai tutto il necessario, come regole, per giocare, a te (voi) servira' fantasia, amicizia, dadi, qualche foglio di carta e divertimento (sorry, patatine e bibite non sono incluse nel manuale!)

Disegna ed usa una mappa ogni qual volta la descrizione e la situazione necessitano di una descrizione accurata ed un posizionamento precisa. 

Crea e gioca il personaggio che piu' ti aggrada, che piu' senti tuo e ti fa divertire, non cercare le combinazioni di Abilita' e capacita' che ti danno piu' potere altrimenti prima o poi il personaggio ti verra' a noia.

Piu' giocherai, piu' il tuo personaggio guadagnera' esperienza ed anche tu lo interpreterai meglio. Il personaggio acquisira' oggetti meravigliosi, armature lucenti, armi volanti, oro, preziosi e gioielli e chissa' cos'altro.

Il Narratore si preoccupera' di dirti quanta esperienza il tuo personaggio ha maturato, in base a come hai giocato, come hai collaborato nel gruppo, quanto hai aiutato il gruppo di giocatori a divertirsi. Ti terra' impegnato in scontri pericolosi, forse mortali, mettera' a dura prova il tuo personaggio e come gruppo riuscirete, forse non sempre, a risolvere le intricate situazioni che il Narratore ha preparato. Ricorda che il Narratore ha sempre l'ultima parola in ogni discussione.

In questo libro troverai molte regole, eppure molte situazioni dovranno essere gestite utilizzando la prima regola, divertirsi. Buon senso, esperienza e fiducia nel Narratore risolveranno ogni situazione.

Sia che tu decida di essere il Narratore sia che tu decida di interpretare un personaggio e' necessario che tu legga con attenzione i capitoli che seguono.
E' importante che tu abbia una buona conoscenza delle regole base e che, soprattutto, sappia dove cercare cosa quando ti servira'!

Buona lettura e buon divertimento!

Nota riguardo ai generi: il TUS il giocatore viene indicato come maschile. Non e' una questione di sessismo, semplicemente in italiano la forma neutra (e quindi applicabile a giocatori e giocatrici) e' maschile, le giocatrici sanno giocare meglio dei compagni maschi!

\pagebreak{}

\subsection{Termini Comuni}

Ti elenco un po' di termini\index{termini comuni} che troverai ripetuti piu' volte nel libro.

\textbf{Arrotondamenti}: \index{Arrotondamenti}sempre per difetto se non esplicitato diversamente. Es. 7/2 = 3, 9/4=2.

\textbf{Abilita'}: \index{Abilita'}sono le capacita' particolari che il personaggio ha imparato ad usare. Sono spesso al limite del magico, permettono azioni particolari ed anche di sovvertire le regole. Sono rare e si prendono ai passaggi di livello.

\textbf{Azione}: \index{Azione}indica un intervallo di tempo. Puo' misurare l'intervallo di tempo necessario al lancio di Essenze, per il combattimento, il bere pozioni, lo spostarsi... in ogni round si possono fare 3 Azioni.

\textbf{Bonus}: \index{Bonus}qualsiasi modifica dovuta a fattori esterni, ambientali, magici, di circostanza o che decida il Narratore e' un bonus o malus da applicare al tiro di dado o difficolta' nella prova. Se ci sono piu' bonus dello stesso tipo si tiene solo il valore maggiore.

\textbf{Classe}: In TUS non ci sono classi. Ogni personaggio e' costruito in base a cio' che sa fare. Quindi non troverete la parola Classe nel manuale.

\textbf{Check/Prova}: \index{Check}\index{Prova}un check (o prova) e' il tiro di 3d6 piu' un punteggio indicato dalla Caratteristica e Competenza coinvolta.

\textbf{Prova di Concentrazione}:\index{Prova di Concentrazione} quando un incantatore vuole usare una Essenza ma e' disturbato, attaccato, ferito o comunque distratto durante il lancio deve effettuare una prova di concentrazione per capire se riesce a lanciare la magia. La prova da superare e' pari al danno +10 e si effettua con 3d6 + CM (Competenza Magica) + Caratteristica interessata dall'Essenza.

\textbf{Classe di Difficolta' (DC)}:\index{Classe di Difficolta'} \index{DC}indica quanto e' difficile riuscire in una prova. puo' essere usato per verificare le competenze (nuotare..) come le conoscenze (veleni..). Indica nelle magie la difficolta' a resistere ad una magia. Indica a che valore arrivare per superare e riuscire nel nella prova.

\textbf{Competenza} \index{Competenza}(skill)\index{skill}: la competenza indica il grado di conoscenza di una singola capacita'. Possa essere lo studio di una lingua, l'arrampicarsi, il notare piccole cose.

\textbf{Competenza con le Armi (da mischia o distanza) (CA):} \index{Competenza con le Armi}\index{CA}e' la tua capacita' di saper colpire l'avversario con armi da mischia (spade, mazze..) o da tiro/distanza (pugnali da lancio, archi, balestre..) 

\textbf{Competenza Magica (CM):} \index{Competenza Magica}\index{CM}e' la tua capacita' di attingere dalle Essenze, piu' e' alto questo valore piu' le tue magie saranno efficaci.

\textbf{Creatura}\index{Creatura}: qualsiasi essere presente e partecipante nell'avventura viene identificata come creatura.

\textbf{Dadi Vita}\index{Dadi Vita}: per dadi vita si intendono i livelli di una creatura. Di base servono ad indicare quanti punti ferita e livello ha. Se non indicato una creatura ha 6 PF per Dado Vita.

\textbf{Difesa:} \index{Difesa}per Difesa si intende il valore totale ottenuto da 10 + Scudo + Armatura + Agilita' + vari ed eventuali bonus.

\textbf{+1d6 oppure -1d6}: e' un bonus o malus ad una prova. Aggiungi o sottrai un tiro di dado a 6 alla prova, oppure un +4/-4

\textbf{Distanza}:\index{Distanza} la distanza, per quando riguarda il combattimento e' misurato in quadretti da 1 metro.

\textbf{Devoto}\index{Devoto}: un usufruitore di Essenze che si e’ legato ad un Patrono e’ ha almeno 3 tratti in comune.
Puo’ scegliere le Essenze del suo Patrono con i vantaggi e svantaggi che concedono. Vedi anche Seguace

\textbf{Esplosione del 6}:\index{Esplosione del 6} quando, esegui il Tiro per Colpire, Tiro Salvezza, prova di Magia (leggi le specifiche nel capitolo dedicato) o comunque ogni volta che viene indicato che vale l'esplosione del 6 significa che per ogni dado tirato che ha fatto 6 va segnato il risultato e ritirato il dado. Il risultato del nuovo tiro va anche lui sommato e se si fa un 6 nuovamente si somma e si continua a ritirare finche' si continua a fare 6.

\textbf{Essenza:} \index{Essenza}indica un potere magico, una magia o incantesimo. Le Essenze si suddividono in varie tipologie che governano aspetti diversi della magia.

\textbf{Iniziativa}: \index{Iniziativa}e' una prova di Agilita' oppure Intelletto. Stabilisce l'ordine delle azioni in combattimento. Chi ha il punteggio piu' alto agisce per primo.

\textbf{Livello}:\index{Livello} il livello indica la competenza e potere raggiunto dal personaggio. Puo' indicare quando e' forte il nemico/personaggio. 

\textbf{LP}: \index{LP}Livello Potere indica a che forza si manifesta una Essenza, possa essere tramite un oggetto o tramite un potere magico di un mostro.

\textbf{Incantatore:} \index{Incantatore}indica un qualsiasi usufruitore di magia/Essenze a qualsiasi titolo.

\textbf{Mischia}: \index{Mischia}con mischia si intende il combattimento di contatto,spada a spada, ovvero quando il tuo personaggio combatte con una spada o comunque con un'arma che non e' da tiro (arco/balestre..) contro un avversario.
Si considera in mischia qualsiasi creatura che il personaggio possa raggiungere con la sua arma non da tiro. Un nemico di grandi dimensioni (o con un arma lunga) potrebbe essere in mischia con il personaggio ma non viceversa.

\textbf{Movimento}: \index{Movimento}il movimento rappresenta la capacita' di spostarsi. Una Azione di movimento rappresenta lo spostarsi del personaggio. Piu' e' alto il valore di movimento piu' una creatura puo' muoversi lontano.

\textbf{Narratore:}\index{Narratore} il Narratore e' la persona che conduce l'avventura, stabilisce le regole e controlla gli elementi della storia. Il dovere di ogni Narratore e' fare divertire ed essere corretto.

\textbf{Patrono}:\index{Patrono} o divinita'. Il Patrono e' un essere superiore che puo' concedere poteri e garantire vantaggi. Il Patrono concede l'uso delle Essenze.

\textbf{Penalita'} \index{Penalita'}: come il bonus le penalita' o malus sono valori, numeri, che indicano le circostanze sfavorevoli, magie penalizzanti o quant'altro renda piu' difficile la prova. Purtroppo a differenza dei Bonus le penalita', se non specificato diversamente, si sommano sempre fra loro. 

\textbf{PG, Personaggio}: \index{Personaggio}e' la creatura che viene guidata, gestita, ruolata dal giocatore.

\textbf{PNG}: \index{PNG}personaggio non giocante. Sono personaggi particolari, importanti o meno che il Narratore tiene per condurre l'avventura.

\textbf{Punti Esperienza}: \index{Punti Esperienz} \index{PX} ogni qual volta si risolvano difficolta', mostri, indovinelli, si giochi bene il personaggio e ci si diverta si guadagna esperienza. Questi punti accumulati nel tempo stabiliscono il livello del personaggio.

\textbf{Punteggi caratteristica}: \index{Punteggi caratteristica}abbreviati anche in caratteristica o statistiche. Ogni personaggio ha 5 Caratteristiche: Potenza (POT), Agilita' (AGI), Intelletto (INT), Volonta' (VOL) e Magnetismo (MAG). Piu' e' alto il valore maggiore e' la valenza o capacita' del personaggio in quello specifico ambito.

\textbf{Punti Fato}:\index{Punti Fato} \index{Fortuna del Principiante}o Fortuna del Principiante sono dei punti a disposizione che il giocatore puo’ trasformare in d6 da aggiungere ai Tiri Salvezza o Tiri per Colpire o Tiri Competenze. Vengono chiamati Fortuna dei Principianti perche’ il loro numero e’ diminuisce all’aumentare di livello.

\textbf{Punti ferita (PF)}:\index{Punti ferita} \index{PF}indicano l’energia vitale della creatura. Finche’ la creatura ha 1 punto ferita combattera’ al suo meglio , senza problemi (certo.. potrebbe anche decidere di scappare piuttosto che morire..).
Ad ogni passaggio di livello si guadagna un dado vita, stabilito’ da certe regole. Ogni ferita si sottrae da questa somma di energie e quando si raggiungono gli 0 (zero) punti ferita si sviene, incapaci di agire. Se si viene ulteriormente feriti o comunque i propri punti ferita scendono fino 10+valore triplo della Potenza allora si e’ morti.

\textbf{Resistenza alla Magia (RM)}:\index{Resistenza alla Magia} \index{RM}Una creatura potrebbe avere una naturale resistenza alle Essenze, in ogni forma si presenti. Ogni qual volta la creature e' influenzata direttamente da una Essenza deve effettuare una prova di RM, ovvero tirare 3d6 sommare il valore di RM e se e' superiore alla prova di magia effettuata dall'incantatore l'Essenza non ha effetto.

\textbf{Riduzione del Danno (DR)}: \index{Riduzione del Danno} \index{DR} alcune creature hanno una resistenza innata al danno e ferite. Questa resistenza si denota come DR.
E’ solitamente indicata come Valore/Particolare, dove il valore indica quanto si e’ resistenti e il Particolare indica da che cosa si e’ danneggiati. Es “10/cold iron” indica che si ha una resistenza a tutti i danni di 10 tranne quelli causati da ferro freddo.
Se il particolare e’ indicato da un trattino “-” allora questa resistenza non e’ oltrepassabile e si applica a tutti i tipi di danno.

\textbf{Resistenza al Danno (RD)} \index{Resistenza al Danno}\index{RD} : una creatura potrebbe avere una resistenza ad una tipologia di danno. In questo caso si considera che dimezzi automaticamente il danno subito.
In caso di Essenza se il TS riesce il danno e' ridotto a 0.

\textbf{Round}:\index{Round} il combattimento o azioni sono divise in round. Un round rappresenta una unita' temporale di circa 6 secondi. Durante il round ogni creatura ha la possibilita' di agire in base alla sua iniziativa ed eseguire fino a 3 azioni.

\textbf{Tiro per colpire}:\index{Tiro per colpire} \index{TC}e' una prova di CA (Competenza Armi) contro Difesa (armatura + scudo + abilita' + magia...). Il Tiro per Colpire puo' essere da mischia (ovvero per le creature prossime alla tua arma, a distanza di mischia) oppure da distanza (per archi, balestre, ma anche pugnali..).. Leggi bene il capitolo del combattimento.

\textbf{Tiro Salvezza (TS)}:\index{Tiro Salvezza} \index{TS}quando una creatura e' sottoposta ad un effetto particolare spesso viene concesso un Tiro Salvezza per mitigare o annullare gli effetti. Il Tiro Salvezza e' un'azione passiva che non occupa tempo o azioni. 
I Tiro Salvezza riguardano i riflessi e lo schivare (Agilita'), resistere a veleni/malattie o cambiamenti del corpo (Potenza) oppure per resistere ad attacchi mentali ed effetti che agiscano sull'arbitrio (Volonta').

\textbf{Tratto}: \index{Tratto}indica una componente del carattere. Ogni personaggio sceglie 4 tratti per comporre e costruire la sua personalita'.

\textbf{Turno}: \index{Turno}sono 10 minuti, ovvero 100 round

\textbf{Uno porta male}: \index{Uno porta male}se tiri un 1 con il dato togli 1 dal risultato totale. Non per questo un 6 tirato diventa un 5, l’esplosione del 6 rimane.. solo che togli 1 al risultato finale. Detta diversamente 1 vale 0.

\pagebreak


\section{Razze}\index{razze}

\begin{quotebox}
Il vero viaggio di scoperta non consiste nel trovare nuovi territori, ma nel possedere altri occhi, vedere l'universo attraverso gli occhi di un altro, di centinaia d'altri: di osservare il centinaio di universi che ciascuno di loro osserva, che ciascuno di loro e'. (Marcel Proust)\linebreak
\linebreak
Nessuna specie e' un'isola. (Mary Midgley)
\end{quotebox}

\subsection{Umani}\index{Umani}

Gli uomini con il loro desiderio di scoperte, potere, gloria e violenza sono la razza dominatrice.

Le caratteristiche fisiche degli umani sono varie quanto i climi del mondo. Le origini degli umani sono anche indicate dai loro stili tradizionali di adornare i loro corpi, non solo nell'abbigliamento o nei gioielli che indossano, ma anche in elaborate capigliature, piercing, tatuaggi e persino scarificazioni.

Gli umani sono stati la razza creata da Ljust e Calicante insieme perche' con la loro spinta caotica, mutevole e vitale potessero fare e disfare ricominciando continuamente da capo e migliorando di continuo.

\textbf{Modificatori razziali:} +1 ad una caratteristica a piacere

\textbf{Caratteristiche fisiche}: altezza 150-185 cm, 50-130 kg, aspettativa di vita 70 anni

\textbf{Dimensioni:} Medie

\textbf{Velocita'}: 9

\textbf{Linguaggi}: comune

\textbf{Vantaggio}: +1 abilita' al primo livello

\subsection{Elfi}\index{Elfi}

\label{elfi}

Gli elfi sono la razza creata direttamente da Ljust perche' guidasse il mondo con l'eleganza, l'intelligenza, la lungimiranza tipica di una razza senza eta'.

Dopo millenni di pace e vita nell'intero mondo, dopo che bellezze naturali ed architettoniche si erano diffuse in armonia il mondo, la creazione delle nuove razze e la loro spinta espansionistica hanno portato gli elfi a diventare insofferenti, ad essere infastiditi dagli altri. Sono diventati progressivamente xenofobi ed hanno incominciato a stravolgere l'impianto originale del loro mandato.

Se erano stati creati come guida etica, morale e culturale di tutto il creato adesso molte fazioni vedono come necessaria una pulizia etnica per portare a compimento la purezza originaria del piano divino. 

Molti hanno preso a conquistare, soggiogare e sterminare le razze inferiori.. qualsiasi creatura che non sia elfica, un una spirale di violenza ed espansione senza eguali.

Altri hanno preso a ritirarsi sempre piu' lontano, sempre piu' all'interno del loro regno, rimanendo custodi solitari della purezza del creato. 

Gli elfi rappresentano la l'idea originale del creato e questo spesso li porta ad essere piu' affini agli gli dei originari e con Kyriel che con le successive divinita'.

Gli elfi sono generalmente piu' alti e snelli degli umani. Gli occhi sono sempre grigi, con riflessi metallici, le gambe agili.

Gli elfi apprezzano la parola scritta, la magia e la ricerca scrupolosa. Le loro menti e i loro sensi acuti, insieme alla loro innata pazienza, li rendono predisposti per la magia. Le ricerche e le scoperte arcane vengono viste sia come obiettivi pratici, allo stesso livello dell'essere un soldato o un architetto, che come impresa artistica di grandezza pari alla poesia e alla scultura.

\textbf{Modificatori razziali:} +1 Intelletto, +1 Agilita', -1 Potenza

\textbf{Caratteristiche fisiche}: altezza 165-195 cm, 50-110 kg, aspettativa
di vita 1000+ anni

\textbf{Dimensioni:} Medie

\textbf{Velocita'}: 9

\textbf{Linguaggi}: comune

\textbf{Vantaggio} visione crepuscolare di 18 metri

\subsection{Nani}\index{Nani}

\label{nani}

I nani sono una razza stoica ma severa abituata al comunismo piu' puro, senza un vero concetto di proprieta' ma di pura comunanza di beni secondo l'idea che ogni nano lavora per la comunita' e non per se stesso.

I nani sono una razza bassa e piazzato e raggiungono un'altezza massima di circa 140 cm con una corporatura robusta e compatta che da' loro un aspetto massiccio. Sia i maschi che le femmine portano orgogliosamente i capelli lunghi e gli uomini decorano spesso le barbe con vari generi di fermagli e trecce intricate, altresi' vero che nani pelati sono frequenti, ma non senza barba. Le donne nane non hanno barba ne peluria in eccesso. Il sesso e' libero e socialista.

I nani sono guidati da onore e tradizione e comunismo. Sono spesso visti come burberi, ma hanno un forte sentimento di amicizia e giustizia e rispetto per chi lavora sodo e si impegna per la comunita'.

I nani sono la razza creata da Erondil.

Giudicano gli Elfi con severita' perche' non hanno saputo portare a termine il dettato della Creazione Universale e quindi si sentono il compito, l'onere e l'onore di forgiare il creato e nel creato la bellezza e la maestosita' di Erondil ed Ljust. 

\textbf{Modificatori razziali:} +1 Potenza, +1 Volonta', -1 Magnetismo

\textbf{Caratteristiche fisiche}: altezza 100-140 cm, 45-90 kg, aspettativa
di vita 450 anni

\textbf{Dimensioni:} Medie

\textbf{Velocita'}: 6

\textbf{Linguaggi}: Comune

\textbf{Speciale:} Professione: Architetto o Fabbro ha un +1

\textbf{Vantaggio} visione crepuscolare di 18 metri

\textbf{Svantaggio:} Pessimo carattere

\subsection{Mezzelfo}\index{Mezzelfo}

\label{mezzelfo}

Per un elfo non c'e' nulla di piu' impuro di un mezz'elfo. Nessun mezz'elfo nasce per volonta' di un Elfo. Ogni mezz'elfo e' figlio di violenza. Questo e' almeno quello che continuano a dire gli elfi.

Ci sono anche rari mezz'elfi nati da rapporti romantici. Benché solitamente di breve durata, anche per gli standard umani, questi incontri segreti portano di solito alla nascita dei mezzelfi, una razza che discende da due culture, ma non e' erede di nessuna. I mezzelfi possono riprodursi tra loro, ma persino questi mezzelfi ``di sangue puro'' sono visti come bastardi dagli elfi.

I mezzelfi sono piu' alti degli umani ma piu' bassi degli elfi. Ereditano la corporatura slanciata e i lineamenti attraenti del loro lignaggio elfico, ma il colore della loro pelle e' normalmente dettato dalla loro parte umana. I loro occhi tendono ad essere simili a quelli degli umani nella forma, ma presentano un'esotica gamma di colori dall'ambra al viola fino al verde smeraldo e al blu scuro, sempre con riflessi metallici.

I mezzelfi comprendono la solitudine e sanno che il carattere spesso e' piu' un prodotto dell'esperienza di vita che della razza di appartenenza. Se in un gruppo c'e' gia' un elfo difficilmente un mezz'elfo andra' d'accordo.

\textbf{Modificatori razziali:} +1 ad una Caratteristica a propria
scelta

\textbf{Caratteristiche fisiche}: altezza 150-180 cm, 50-100 kg, aspettativa
di vita 210 anni

\textbf{Dimensioni:} Medie

\textbf{Velocita'}: 9

\textbf{Linguaggi}: Comune

\textbf{Vantaggio}: visione crepuscolare di 18 metri

\index{mezzorco}

\subsection{Mezzorco}

\label{mezzorco}

Agli occhi delle razze civilizzate, i mezzorchi sono delle mostruosita', il risultato di perversione e violenza e raramente sono il risultato di unioni amorose, come tali solitamente sono costretti a crescere velocemente e duramente, lottando continuamente per proteggersi o farsi un nome. Alcuni mezzorchi trascorrono le loro intere vite a dimostrare agli orchi purosangue che sono feroci quanto loro.

I mezzorchi sono alti in media 1.9 metri, con fisico potente e pelle verdastra o grigia. I loro canini crescono spesso piuttosto lunghi fino a sporgere dalle loro bocche e queste "zanne", unite ad una fronte ampia e le orecchie un po' a punta, danno loro quel noto aspetto "bestiale". Sebbene i mezzorchi possano risultare impressionanti, pochi li definirebbero belli. A dispetto di questi ovvi tratti orcheschi, i mezzorchi sono tanto variegati quanto i loro genitori umani.

Se all'interno delle tribu' orchesche devono guadagnarsi continuamente il rispetto dei ``purosangue'', nella societa' umana non va meglio. Derisi, sbeffeggiati, esclusi ed abbandonati i mezzorchi spesso trovano rifugio nella criminalita'.

Gli orchi sono stati creati direttamente da Cattalm e molto della tendenza chaotica e distruttrice del loro creatore rimane nella natura dei mezzorchi.

Spesso vittime di pregiudizi e' comune opportuno tenersi alla larga da questi animali.

\textbf{Modificatori razziali:} +2 Potenza -1 Magnetismo

\textbf{Caratteristiche fisiche}: altezza 160-210 cm, 60 - 140 kg,
aspettativa di vita 70 anni

\textbf{Dimensioni:} Medie

\textbf{Velocita'}: 9

\textbf{Linguaggi}: Comune

\textbf{Vantaggio} visione crepuscolare di 18 metri

\textbf{Svantaggio:} Seguire il Chaos

\subsection{Drow}\index{drow}

\label{drow}

La genesi e storia dei Drow si divide un due grandi tronconi storici, strettamente legati alla storia Elfica.

In principio Shayalia, gelosa degli Elfi plasmo i Drow a loro immagine e poi li rese cupi, oscuri, freddi come la notte perche' fossero l'ombra nera degli elfi.

Conosciuti anche come elfi scuri, essi dimorano nelle profondita' del sottosuolo, in complesse citta' plasmate nella roccia dalle Essenze.

I drow hanno una fisicita' simile a quella degli uomini, ma condividono i lineamenti e lo slancio degli elfi, comprese le caratteristiche lunghe orecchie a punta. Il colore della pelle dei drow varia dal nero carbone al viola scuro. I loro capelli sono solitamente bianchi o argentei, sebbene non siano insolite altre varianti.

La societa' drow e' per tradizione matriarcale e suddivisa in classi. I maschi drow solitamente adempiono ai ruoli militari, difendendo la specie dai pericoli esterni, mentre le femmine drow assumono ruoli di comando e autorita'.

A rafforzare questi ruoli, un drow ogni venti nasce con capacita' eccezionali e viene quindi considerato un nobile, e la maggioranza di questi drow speciali sono femmine. Le casate nobili determinano la politica drow, e ciascuna di esse e' governata da una nobile matriarca e composta di famiglie di rango inferiore, imprese commerciali e compagnie militari.

I drow sono fortemente motivatidall'interesse e dalla crescita personale, che plasmano la loro cultura con ribollenti intrighi e conflitti politici, mentre i drow comuni fanno del loro meglio per ottenere il favore della nobilta', e quest'ultima si eleva al potere per mezzo di una miscela di omicidi, seduzioni e tradimenti.

I drow hanno un forte senso di superiorita' razziale e suddividevano le altre razze in due gruppi distinti: gli schiavi e coloro che non sono ancora schiavi.

L'odio dei drow verso gli elfi separa questi esseri da tutte le altre razze, e gli elfi scuri non desiderano nulla di piu' al mondo che distruggere tutto quello che ha a che vedere con i loro cugini di superficie.

I drow danno grande importanza al potere e alla sopravvivenza, e non provano alcun rimorso a causa delle scelte spregevoli che potrebbero essere costretti a fare per assicurare la propria sopravvivenza. Non sanno cosa farsene della compassione e sono spietati nei confronti dei loro nemici, antichi o attuali che siano.

Poi gli Elfi diventarono piu' reclusi, indifferenti, xenofobi e nazisti, pari passo che Shayalia riusciva a manipolare la creazione di Ljust. E mentre Shayalia era distratta con gli Elfi, Sumkjr si faceva largo nei drow. Una volta conosciuti come l'anima nera del mondo adesso sono tra i maggiori portatori di speranza, vita, saggezza e cultura.

Riuscendo ad adattare il proprio regime sociale le matriarche Drow sono diventate le filantrope che si interessano dei poveri, degli emarginati, degli svantaggiati, dell'ambiente e cultura promuovendo una nuova consapevolezza universale.

Chiaro che, purtroppo, non tutti hanno accettato questa conversione e si possono trovare, a pari degli Elfi, soggetti che perseguono le vecchie abitudini.

\textbf{Modificatori razziali:} +1 Agilita', +1 Intelletto, -1 Potenza

\textbf{Caratteristiche fisiche}: altezza 140-190 cm, 40 - 100 kg,
aspettativa di vita 1000+ anni

\textbf{Dimensioni:} Medie

\textbf{Velocita'}: 9

\textbf{Linguaggi}: Comune

\textbf{Vantaggio} visione crepuscolare di 36 metri

\textbf{Svantaggio:} \sout{razzisti}

\subsection{Nibali}\index{Nibali}

\label{nibali}

I Nibali sono una razza creata magicamente per essere schiava ai grandi maghi del nord.

La leggenda dice che i terribili maghi del nord, partendo da una coppia di umani (dopo che a migliaia erano morte atrocemente nei precedenti esperimenti) riusci' a creare manipolando con la magia, un razza piu' robusta, piu' forte, piu' intelligente ed allo stesso tempo piu' docile e disciplinati con pregio che ogni figlio generato sarebbe stato assolutamente identico fisicamente al padre o alla madre.

Queste cose accadevano ormai piu' di 2000 anni or sono ed il regno del male eterno crollo' sotto la sua stessa incapacita' di evolversi e percepire i nuovi problemi.

I Nibali hanno continuato a prosperare ed usufruendo di quanto il regno del ghiaccio gli aveva lasciato hanno creato una tra le civilta' piu' moderne, democratiche e civili del mondo.

Per molti l'estrema efficienza e dedizione dei Nibali e' odiosa, un giogo che non lascia spazio alle liberta' personali, per i Nibali e' solo un modo efficiente di progredire.

Tutti i Nibali sono uguali tra loro a parita' di sesso ma il fatto che non possano avere figli con altre razze non li rende un popolo chiuso o razzista, anzi l'assorbire il meglio di ogni cultura li rende migliori ed anche ottimi diplomatici. 

Cio' che veramente distingue un Nibali da un altro e' l'acconciatura, i tatuaggi, il vestiario... L'estrema liberta' personale, legata indissolubilmente alla liberta' di gruppo, permette ad un nibali di esprimersi come meglio crede nell'aspetto esteriore.

\textbf{Modificatori razziali:} +1 Potenza, +1 Intelletto, - 1 Volonta'

\textbf{Caratteristiche fisiche}: altezza 183cm maschi, 172 cm femmine, 50 - 120 kg, aspettativa di vita 130 anni

\textbf{Dimensioni:} Medie

\textbf{Velocita'}: 9

\textbf{Linguaggi}: Comune

\textbf{Svantaggio}: Seguire la Legge

\subsection{Diversi}\index{Diversi}

\label{diversi}

Benedetti o maledetti i Diversi non sono come noi. Non sono gli amici che ti aspetti. Un Diverso e' frutto di una unione non voluta. Se i Patroni non possono agire direttamente nel mondo, o almeno questo e' quello che cerca di evitare Gradh, sovente invece usano i loro poteri per creare una stirpe a loro fedele.

Un Diverso e' fedele al suo Patrono e non puo' fare diversamente. Per fortuna sono sterili con gli umani, altrimenti avrebbero gia' dominato il mondo.

Un Diverso e' piu' forte e piu' intelligente e puo' meglio nell'oscurita. Purtroppo per loro la loro vita' frenetica e' segnata da una breve durata. Solitamente un Diverso non supera i 40 anni di vita.

Un Diverso e' segnato, da qualche parte sul suo corpo c'e' il simbolo del suo Patrono.

\textbf{Modificatori razziali:} +1 Potenza, +1 Intelletto

\textbf{Caratteristiche fisiche}: altezza 155-185 cm, 50-110 kg, aspettativa di vita 45 anni (40+1d10 anni)

\textbf{Dimensioni:} Medie

\textbf{Velocita'}: 9

\textbf{Linguaggi}: comune

\textbf{Speciale} visione crepuscolare di 18 metri, deve individuare un Patrono ed avere almeno 3 tratti comuni.

\subsection{Altri}\index{Altri}

\label{altri}

in un mondo dominato dal chaos chi ha provato a scappare nell'oscurita' delle caverne e della notte ha subito la punizione di Ljust per non aver tentato di migliorare il mondo.

Questi esseri insolitamente gracili hanno una forte intelligenza e agilita', la loro carnagione e' diventata chiara, quasi madreperlacea. Ormai sono passati duemila anni da quando il primo Altro nacque e a seguito ogni madre per generazioni partori' solo Altri, finche' non ci fu nessun umano, finche' tutti ebbero pagato il peccato di non volere migliorare il mondo.

La maggior parte degli Altri si e' votata a Calicante ed ai Patroni Oscuri, alle arti magiche piu' malvagie e corruttive. Pochi, reietti, sentono la colpa e abbracciano la Luce e vengono in superficie.

Un Altro puo' essere riconosciuto da una voglia naturale, come un tatuaggio, che disegna tre anelli dorati attorno al polso.

Trattati come mostri o malvagi senza neanche una domanda, un Altro non ha mai la vita facile, per fortuna la loro naturale agilita' e la capacita' innata di vedere nell'oscurita' gli permette di vivere, anche se spesso solo di notte, lontano dalle luci e dagli affetti che vorrebbero provare.

\textbf{Modificatori razziali:} +1 Intelletto, +2 Agilita' , -2 Potenza

\textbf{Caratteristiche fisiche}: altezza 155-185 cm, 50-110 kg, aspettativa
di vita 100 anni

\textbf{Dimensioni:} Medie

\textbf{Velocita'}: 12

\textbf{Linguaggi}: comune

\textbf{Vantaggio} visione crepuscolare di 36 metri

\bigskip

\textbf{Nota sugli Svantaggi}: il giocatore, in accordo con il Narratore, puo' scegliere uno svantaggio diverso da quello indicato purche' sia coerente con la storia del personaggio.


\pagebreak

\section{Caratteristiche Speciali}

\label{caratteristiche-speciali}
\begin{quotebox}Non basta avere gli occhi per vedere (anonimo)
\end{quotebox}

\subsection{Visione Crepuscolare}\index{Visione Crepuscolare}

Quello che per molti e’ oscurita’ per chi ha visione crepuscolare e’ vedere bene purche’ ci sia una fonte minima di luce.

La visione crepuscolare e' una visione a colori.
Un incantatore dotato di visione crepuscolare puo' leggere una Pergamena fino a quando ha accanto come fonte di luce anche la piu' smorta delle candele.

I personaggi dotati di visione crepuscolare possono vedere all’esterno nelle notti illuminate dalla luna come se si trovassero alla luce del giorno.

\subsection{Fiuto}\index{Fiuto}

Questa qualita’ speciale permette ad una creatura di sfruttare l'olfatto per individuare i nemici nascosti o in avvicinamento e di seguire le tracce. Le creature dotate di fiuto possono identificare con l'olfatto gli odori familiari come gli umani fanno con quello che vedono.

La creatura puo' individuare le creature entro 6 metri di distanza con l'olfatto. Se l'avversario e' sottovento, il raggio aumenta a 18 metri; se e' sopravento, il raggio diminuisce a distanza mischia.
Gli odori piu' forti, come il fumo, spazzatura o corpi in decomposizione, possono essere individuati al doppio del raggio sopra indicato.

Quando una creatura individua un odore, non viene rivelata l'esatta posizione della sua fonte, ma solo la sua presenza entro il raggio d'azione. La creatura puo' utilizzare un'Azione per individuare la direzione da cui proviene l'odore. Quando si trova a distanza di mischia dalla fonte, ne individua la posizione.

Una creatura dotata di fiuto puo' seguire tracce utilizzando l'olfatto, effettuando una prova di Sopravvivenza per trovare e seguire una traccia. La tipica DC di una traccia fresca e' 10 (a prescindere dalla superficie su cui si trova la traccia). La DC aumenta o diminuisce a seconda dell'intensita' della traccia, del numero di creature che la lasciano e del tempo trascorso da quando e' stata lasciata. Per ogni ora trascorsa la DC aumenta di 2.

Per il resto, questa capacita' segue le regole dell'abilita' Sopravvivenza. Le creature che seguono tracce con il fiuto ignorano gli effetti delle superfici su cui si trova la traccia e della scarsa visibilita'.

Una creatura con la capacita' Fiuto identifica gli odori familiari cosi' come un umano potrebbe identificare un luogo familiare. L'acqua, e in particolare l'acqua corrente, nega la capacita' di seguire tracce delle creature.

Alcuni forti odori possono facilmente mascherarne altri. La presenza di un odore simile rende impossibile individuare o identificare esattamente una creatura mediante il Fiuto; la DC base dell'abilita' Sopravvivenza per seguire tracce in presenza di odori coprenti passa da 10 a 20.


\subsection{Vista Cieca (aka “Daredevil”)}\index{Vista Cieca}

Utilizzando sensi diversi dalla vista, come la percezione delle vibrazioni, un fiuto sensibile, un udito acuto o un sonar, una creatura dotata di vista cieca si muove e combatte bene quanto una creatura dotata della vista. 

Invisibilita’, buio e la maggior parte delle forme di Occultamento sono inutili, anche se la creatura dotata di vista cieca deve avere una linea di effetto per notare una determinata creatura o oggetto. 

Il raggio della capacita' e' indicato nella descrizione della creatura. La creatura, in genere, non deve effettuare prove di Consapevolezza per notare creature entro il raggio della sua vista cieca.

A meno che non sia diversamente indicato, la vista cieca e' sempre attiva e la creatura non deve compiere azioni per attivarla. Alcune forme di vista cieca devono essere attivate come azione immediata. In questo caso, viene indicato nella descrizione della creatura.

Se una creatura deve attivare la vista cieca, ne ottiene i benefici solo durante il proprio turno.

\subsection{Tremorsense}\index{Tremorsense}
Una creatura dotata di Tremorsense e' sensibile alle vibrazioni del suolo, e puo' automaticamente individuare qualsiasi cosa sia in contatto con il terreno entro il raggio specificato dal tremorsense.

Le Creature Acquatiche dotate di tremorsense (ecolocalizzazione) possono percepire la posizione di creature in contatto con l’acqua.

Il raggio della capacita' e' specificato nel testo descrittivo della creatura

\pagebreak

\section{Le Caratteristiche}\index{Caratteristiche}

\label{le-caratteristiche}

\begin{quotebox}Vivere non e' respirare: e' agire, e' fare uso degli organi, dei sensi, delle facolta', di tutte quelle parti di noi stessi per cui abbiamo il sentimento di esistere. (Jean-Jacques Rousseau)
\end{quotebox}


Ogni personaggio ha 5 caratteristiche (chiamate anche statistiche) che rappresentano i suoi attributi base e costituiscono il suo potenziale talento e capacita' innata. 

Anche se raramente un personaggio effettua una prova usando soltanto una sua Caratteristica, i punteggi di Caratteristica influiscono praticamente su ogni aspetto delle capacita' e delle abilita' del personaggio.

Le 5 caratteristiche sono:

\textbf{Potenza}\index{Potenza}: indica la forza fisica ma anche la resistenza agli sforzi del personaggio, Un personaggio con un punteggio di Potenza pari a -5 e' morto.

\textbf{Agilita'}\index{Agilita}: indica la capacita' di coordinamento, riflessi ed agilita' del personaggio, Un personaggio con un punteggio di Agilita' pari a -5 e' incapace di muoversi ed e' completamente immobile (ma non privo di sensi).

\textbf{Intelletto}\index{Intelletto}: indica la componente razionale, logica, cognitiva del personaggio. Un personaggio con un punteggio di Intelletto pari a -5 e' in stato di coma.

\textbf{Volonta'}\index{Volonta'}: indica la forza di volonta', il buon senso, la perspicacia e l'intuito del personaggio. Un personaggio con un punteggio di Volonta' pari a -5 e' incapace di pensiero razionale ed e' privo di sensi.

\textbf{Magnetismo}\index{Magnetismo}: misura la forza della personalita', la capacita' di persuasione, il magnetismo personale, la predisposizione al comando e il fascino di un personaggio. Un personaggio con un punteggio di Magnetismo pari a -5 e' privo di sensi.

\smallskip

Ogni punteggio di Caratteristica in genere va da 0 a 3, anche se i bonus e le penalita' razziali possano alterarli; un punteggio di Caratteristica buona e' 1, 2 ottima, 0 e' ``normale'', 3 e' giudicato ``eccezionale''.

Un punteggio di -1 e giudicato debole, un -2 subnormale, un -3 severamente problematico, un -4 porta quasi ad un non utilizzo della caratteristica, un -5 e' opportuno che stia nel letto e basta (se non e' gia' in una bara)

Ogni giocatore distribuisce 6 punti tra le 5 Caratteristiche, ogni Caratteristica deve avere come minimo un punteggio di -1 e come massimo 2 prima dei modificatori razziali.

Ogni quattro livelli (4, 8, 12, 16, 20..) si puo' aumentare di un punto una caratteristica, fino a raggiungere un massimo di valore 5. Per aumentare oltre 5 sono necessario oggetti magici o Essenze.


\section{Punti Ferita}\index{Punti Ferita}

\begin{quotebox}Chi non stima la vita, non la merita. (Leonardo da Vinci)
\end{quotebox}


Punti ferita rappresentano l’energia vitale del personaggio e finche’ il personaggio/avversario ha almeno 1 punto ferita combattera’ e lottera’ al meglio delle sue capacita'’.

Ogni personaggio parte con 4 punti ferita al primo livello + il punteggio della Potenza.
Ad ogni livello, oltre il primo, guadagna 1d4 Punti Ferita + il punteggio della Potenza. 
Ogni punto preso in Competenza Armi aumenta i punti ferita presi di 3. Ulteriori abilita’ possono alzare questo punteggio.

Segna nella scheda i PF (Punti Ferita) totali che hai e indica il valore attuale di volta in volta che per vari motivi di gioco ne perdi o riprendi.

Segna sulla scheda sempre qual e’ il totale di punti ferita attuale, dopo ogni colpo o danno.
I punti ferita si recuperano in diversi modi:

\begin{itemize}
\item 
per ogni notte di riposo (almeno 8 ore) il proprio valore di Potenza+CA (con un minimo di 1) 
\item
 Tramite l'Essenza della Cura (magie , pozioni.. o altri oggetti magici) 
\item 
Competenza Sopravvivenza (Guarire), tramite trattamenti piu' o meno lunghi 
\end{itemize}


\section{Punti Fato (Fortuna del Principiante)}\index{Punti Fato}
\begin{quotebox}Se il destino e' contro di noi, peggio per lui. (motto del 1º Reggimento Carabinieri Paracadutisti "Tuscania")
\end{quotebox}

In un mondo non facile la Fortuna del Principiante aiuta’ chi non ha esperienza.
Ogni personaggio ha un numero di Punti Fato pari a (20 - Livello)/4, con un minimo di 1. I Punti Fato si conteggiano per Sessione di gioco. 

Ad ogni sessione si azzerano e si ricalcolano, ne consegue che non si accumulano Punti Fato tra una sessione e l’altra.

Esempio di calcolo:
Un personaggio di livello 5 ha: 20 - 5= 15/4 = 4 (arrotondi per eccesso) Punti Fato da usare nella sessione se non li userai tutti non potrai cumularli per la sessione successiva.

Un Punto Fato si usa come azione di reazione ed ogni Punto Fato utilizzato concede un bonus di +1d6 alla prova in corso. 

Il Punto Fato puo'’ essere utilizzato per avere un dado in piu' nel Tiro Salvezza oppure nei Tiri per Colpire, con potenziale esplosione del dado, oppure una prova di competenza o per aumentare la propria Difesa per quel round.

I Punti Fato si devono dichiarare prima del tiro o dopo il tiro ma prima di sapere se la prova (Tiro Salvezza, Tiro per Colpire... propria o dell’avversario) ha avuto successo. 
Prima di dirti se sei riuscito o meno il Narratore ti chiedera’ (oppure sarai tu a dichiararlo) se vuoi usare dei Punti Fato.

Una volta dichiarato l’ammontare di Punti Fato che si vogliono utilizzare non e’ possibile utilizzarne di piu' o di meno.

\section{Tratti}\index{Tratti}

\label{tratti}
\begin{quotebox}
Chi dunque sa fare il bene e non lo compie, commette peccato. (Giacomo il Giusto 4.17, Lettera di Giacomo. NdA riferendosi ai Tratti scelti)\linebreak\linebreak
E' un diritto naturale saziarsi l'anima con la vendetta. (Attila)\linebreak\linebreak
Est Sularus Oth Mithas. (“Il mio onore e' la mia vita”, Giuramento dei Cavalieri di Solamnia.)\end{quotebox}

In TUS non c'e' una netta distinzione tra bene e male, legge e caos, tra cio' che e' giusto e cio' che e' sbagliato.

In TUS esistono i Tratti, aspetti e sfumature caratteriali che contribuiscono al background del personaggio, aiutano il giocatore a ruolare meglio e gli possono fornire quelle linee guida per interpretare in maniera piu' corretta il personaggio che ha voluto creare.

Un Tratto e' un dettaglio che aiuta meglio a inquadrare il personaggio, ne delinea i 'tratti' principali concedendogli sfumature diverse.

\textbf{Ogni giocatore sceglie 4 Tratti per il proprio personaggio alla creazione del giocatore.} Questi saranno le ''bussole morali, etiche e comportamentali'' che guideranno il personaggio nell'agire e nelle scelte.

I Tratti non sono il personaggio ricorda, non lo bloccano ne lo fissano eterno nel tempo. Un personaggio e' sempre in costante evoluzione e cosi' il suo carattere, morale, comportamento e desideri. Non essere rigido ma usa i Tratti per darti delle linee guida entro cui muoverti.

\textbf{Dei Tratti scelti al primo livello individuane uno, questo partira', sempre al primo livello, con valore 1, gli altri partiranno a valore 0.}

Col passare del tempo e delle avventure potranno essere guadagnati o sostituiti (in concerto tra Narratore e giocatore in base a come giocato) da altri Tratti.

Potranno essere anche enfatizzati certi Tratti, ovvero il Narratore a seguito di particolari scene e ruolate potra' fare aumentare di un punto un Tratto del personaggio. Ad esempio a seguito di una particolare scelta e climax di avventura il Narratore potrebbe concedere a tutti o qualcuno Tratto Coraggioso o dare un +1 a Coraggioso a chi ha gia' questo Tratto. Per i Tratti non presi si considera il valore base in punti di -1. ovvero il primo punto serve per prendere il Tratto ed i successivi per enfatizzarli.

Ogni azione particolarmente importante dove il personaggio abbia seguito un Tratto porta il personaggio ad avvicinarsi al Patrono competente per quel tratto.

Nella scheda troverai dei check da mettere vicino ai tratti, questi vengono segnati a seguito di azioni idonee ad accrescere il valore del tratto; raggiunti i 10 punti il Tratto aumentera' di 1 punto e si ricominciera' a segnare una nuova decina.

Sara' il Narratore durante l'avventura a dirti quando segnare, o cancellare, dei punti parziali.

All'aumentare del valore del Tratto il personaggio potra' acquisire dei poteri, indipendentemente sia un credente di quella divinita'
(Patrono) o meno.

\begin{itemize}
	\item A \textbf{5} punti si puo' incominciare a sentire la presenza di un Patrono
legato ad un Tratto
	\item A \textbf{10} punti si sente la vicinanza di un Patrono legato ad un Tratto
	\item A \textbf{15} punti si e' legati ad Patrono da un Tratto
	\item A \textbf{20} punti si e' un Campione del Patrono legato ad un Tratto.
\end{itemize}

Chiunque voglia diventare un incantatore deve avere individuare un Patrono ed avere almeno due tratti in comune con questo ed altri due tratti a piacimento.

\smallskip

Il valore dei tratti solitamente aumenta con il passare del livello e dell'esperienza. Il Narratore puo' sempre decidere in base ad azioni particolarmente ispirate o comportamenti idonei di aumentare anche nel mezzo di un livello il valore di un tratto.

Il Narratore e' libero di inserire nuovi Tratti a suo piacere o richiesto dai giocatori, ci si deve pero' ricordare di attribuire questi tratti anche ai Patroni.

\bigskip

\textbf{Tabella dei Tratti}\index{Tratti}
\bigskip

\begin{tabular}[c]{@{}lllll@{}}
\toprule 
Accumulatore & Aggressivo & Allegro & Anarchico & Arrogante\tabularnewline
Caritatevole & Altruista & Altezzoso & Aperto & Avventato\tabularnewline
Combattivo & Attento & Avaro & Calmo & Calcolatore\tabularnewline
Crudele & Bugiardo & Buono & Caritatevole & Casto\tabularnewline
Disordinato & Cattivo & Clemente & Codardo & Controllato\tabularnewline
Egoista & Corretto & Cortese & Creativo & Coraggioso\tabularnewline
Freddo & Determinato & Diretto & Disciplinato & Curioso\tabularnewline
Impacciato & Distaccato & Distruttivo & Doppiogiochista & Disponibile\tabularnewline
Incostante & Equilibrato & Esuberante & Fiducioso & Empatico\tabularnewline
Indipendente & Gentile & Giusto & Immaturo & Generoso\tabularnewline
Introverso & Impetuoso & Implacabile & Incontentabile & Imparziale\tabularnewline
Istintivo & Ingenuo & Indomito & Indifferente & Indisciplinato\tabularnewline
Logorroico & Iracondo & Innovativo & Integerrimo & Industrioso\tabularnewline
Onesto & Morigerato & Ironico & Irrazionale & Leale\tabularnewline
Ordinato & Pessimista & Lussurioso & Libero & Meticoloso\tabularnewline
Perfezionista & Prudente & Narcisista & Negligente & Mite\tabularnewline
Pio & Riflessivo & Paranoico & Osservatore & Permaloso\tabularnewline
Saccente & Sadomasochista & Pratico & Passionale & Pianificatore\tabularnewline
Socievole & Serio & Rigido & Riservato & Protettivo\tabularnewline
Studioso & Solitario & Sarcastico & Semplice & Razionale\tabularnewline
Tenace & Superficiale & Scontroso & Sincero & Sadico\tabularnewline
Tradizionalista & Tranquillo & Sicuro & Sprovveduto & Semplice\tabularnewline
Valoroso & Truffatore & Silenzioso & Tollerante & Sognatore\tabularnewline
Vanitoso & Vendicativo & Sospettoso & Volubile & Superbo\tabularnewline
\bottomrule
\end{tabular}

\pagebreak

\section{Competenze}\index{Competenze}

\label{competenze}
\begin{quotebox}
Chi dice che una cosa e' impossibile, non dovrebbe disturbare chi la sta facendo.\linebreak
Non hai veramente capito qualcosa fino a quando non sei in grado di spiegarlo a tua nonna. Albert Einstein\end{quotebox}


Ogni personaggio ha conoscenze e/o sa fare qualcosa e questo qualcosa e' una competenza. A seconda dei background ed avventure giocate i personaggi valorizzano determinate competenze.

Alla creazione del personaggio attribuire 1 punto libero in Artigianato o Professione o Intrattenere o Cultura per giustificare competenze di background.

Se non specificato diversamente per tutte le prove di competenza (Base, Attive) valgono tre regole base (golden rules):\index{Golden Rules}

\begin{itemize}
\item 
I 6 esplodono, ovvero se nella prova dei 3d6 un dato fa sei, somma
il risultato e ritira, e se fa 6 nuovamente sommi il risultato e ritiri
ancora e ancora 
\item 
Gli 1 portano male, se fai 1 con il dado togli 1 alla somma dei dadi
tirati (e quindi il dado che ha fatto 1 conta zero) 
\item 
Affidarsi alla sorte. Per ogni 4 punti di competenza che rinunci a
sommare nella prova tira un dado a 6 in piu', che rispetta le regole
base. 
\end{itemize}

\textbf{Regola Bonus}: quando una penalita' e' indicata come ``-1d6'' ( o peggio..) significa che si toglie 1 dado alla prova, se non e' una prova si toglie 4 al risultato finale per ogni dado di penalita'. Il risultato minimo per il tiro dei dadi e' 0.


\subsection{Competenze di Base}\index{Competenze di Base}

\label{competenze-di-base}

\begin{quotebox}
Anche se indubbiamente il desiderio di conoscere è naturale per tutti gli uomini, la voglia di imparare non è cosa da tutti; la maggior parte, anzi, assaggiato quanto lo studio sia fatica e provata la stanchezza sulla propria pelle, butta alla leggera la noce ancor prima di aver rotto il guscio per gustarne il gheriglio. (Richard de Bury)\linebreak\linebreak
Lo studio è per i perdenti! (Lobo)
\end{quotebox}

Ogni personaggio al primo livello sceglie delle Competenze di Base, su queste distribuisce 4 punti, con un massimo di 2 punti per Competenza al primo livello.

Ad ogni livello successivo distribuisce un numero di punti pari la meta' del punteggio di Intelletto +2 ((int+2)/2)) con un minimo di 0 punti tra le competenze gia' conosciute perfezionate nell'avventura o apprese ex novo. Nessuna competenza di base puo' avere un numero un punteggio superiore al livello del personaggio+2.

\bigskip

Queste le competenze ed i loro ambiti di utilizzo:

\textbf{Acrobatica (}Agilita')\textbf{:}\index{Acrobatica} arrampicarsi, equilibrio, saltare, acrobazia

\textbf{Arcano (Intelletto)}:\index{Arcano} Conoscenza arcana, piani, occulta, riconoscere essenze

\textbf{Consapevolezza (Volonta')}:\index{Consapevolezza} percezione, percepire inganni, muoversi silenziosamente, nascondersi nelle ombre

\textbf{Cultura (Intelletto)}: \index{Cultura}geografia, natura, erboristeria, storia, religione, tradizioni e storia locale, dungeon, ingegneria, falsificare

\textbf{Criminalita'} (Agilita'):\index{Criminalita} travestirsi, disattivare congegni, artista della fuga, mani di fata

\textbf{Faccia tosta} (Magnetismo): \index{Faccia tosta}valutare, intimidire, diplomazia (arguzia), raggirare, persuadere

\textbf{Intrattenere} (Magnetismo): \index{Intrattenere}cantare, suonare, recitare, travestirsi, diplomazia

\textbf{Lavoro} (Volonta'):\index{Lavoro} artigianato (sarto, fabbro, apicoltore..), professione ( allevatore, architetto, azzeccagarbugli, barcaiolo..
cacciatore.. conciatore..mercante..)

\textbf{Resistenza} (Potenza): \index{Resistenza}nuotare, correre, saltare, scalare

\textbf{Sopravvivenza} (Volonta'):\index{Sopravvivenza} seguire tracce, sopravvivenza, gestire animali, cavalcare, usare una corda, curare/pronto soccorso

\bigskip

Il Narratore potrebbe quindi chiederti genericamente una prova di
Criminalita' oppure dirti di fare una prova da mani di fata (borseggiare),
il risultato non cambia effettui comunque una prova di Criminalita',
ovvero 3d6 + punteggio in Criminalita' + Agilita' + eventuali bonus
o malus.

Ricordati sempre delle 3 regole base: esplosione del 6, gli uno portano
male ed affidarsi alla sorte.

\bigskip

Potranno esserci situazioni od oggetti che concedono un bonus specifico ad una competenza ovvero non tanto un bonus a Criminalita' ma solo a mani di fata, in quel caso il bonus si applica non a tutte le prove di Criminalita' ma solo a quelle specifiche di mani di fata.


\pagebreak

\subsection{Competenze Attive}\index{Competenze Attive}

\label{competenze-attive}
\begin{quotebox}C'e' solo un modo per allenarsi: quello giusto. (Carl Lewis)\linebreak\linebreak
Wang Chi: Sei pronto?\linebreak
Jack Burton: Io sono nato pronto! (Grosso guaio a Chinatown, Film 1986)
\end{quotebox}

Ogni personaggio prende 3 punti da distribuire nelle Competenze Attive a livello. Si puo' assegnare massimo 1 punto a livello in una singola Competenza Attiva.

Le Competenze Attive sono: Competenza Magica\index{Competenza Magica}, Competenza Armi\index{Competenza Armi}, Tiro Salvezza su Tempra,\index{Tiro Salvezza su Tempra} Tiro Salvezza su Arbitrio \index{Tiro Salvezza su Arbitrio}e Tiro Salvezza su Agilita'\index{Tiro Salvezza su Agilita'}.

\textbf{Competenza Magica (CM)} \index{CM}(varie): indica la capacita' e competenza nel lanciare una Essenza.

\textbf{Competenza Armi (CA)} \index{CA}(Potenza o Agilita'): e' la capacita' e bravura di combattere con un'arma da mischia o da tiro/distanza

Il \textbf{Tiro Salvezza su Tempra} indica quanto si e' in grado di sopportare le sofferenze fisiche o attacchi contro la propria vitalita' e salute. Ai Tiri Salvezza su Tempra si aggiunge il valore della Potenza. 

Il \textbf{Tiro Salvezza su Arbitrio} indica la resistenza contro l'influenza mentale ed altri effetti magici, cio' che vuole modificare il tuo libero arbitrio nelle scelte e nell'agire. Ai Tiri Salvezza su Arbitrio si aggiunge il valore di Volonta'.

Il \textbf{Tiro Salvezza su Riflessi} indica quanto si e' agili e pronti per evitare ostacoli o magie. Ai Tiri Salvezza su Riflessi si aggiunge il valore di Agilita'

\bigskip

Esempio: un personaggio di quarto livello distribuisce le competenze
attive in questa maniera:

1 livello: +1 Competenza Armi , +1 Tiro Salvezza Tempra, +1 Tiro Salvezza Agilita'

2 livello: +1 Competenza Armi, +1 Competenza Magica, +1 Tiro Salvezza Agilita'

3 livello: +1 Competenza Magica, +1 Tiro Salvezza Tempra, +1 Tiro Salvezza Arbitrio

4 livello:+ 1 Competenza Armi, +1 Competenza Magica, +1 Tiro Salvezza Agilita'

per un totale di +3 CA, +3 CM, +3 TS Agilita', +2 TS Tempra, +1 TS Arbitrio

\bigskip

Ogni punto attribuito (CA, CM, TS) permette di usufruire di +1 nella prova relativa.

La \textbf{Competenza Armi} (abbreviata in \textbf{CA}) indica la capacita' e bravura nel colpire l'avversario.

Il \textbf{Tiro per colpire}\index{armi da mischia}, per le armi da mischia si risolve con una prova di Competenza Armi (\textbf{CA}) + Potenza + eventuali capacita' e bonus magici contrapposto alla Difesa dell'avversario (Agilita' + armatura/scudi/bonus).

Il \textbf{Tiro per Colpire con armi da distanza} \index{armi da distanza}(archi, pugnali da lancio, sassi..) si risolve con una prova di Competenza Armi (\textbf{CA}) + Agilita' + eventuali capacita' e bonus magici contrapposto alla Difesa dell'avversario (Agilita' (schivare) + armatura/scudi/bonus).

Quando si assegna un punto ad \textbf{CA} va precisata su quale gruppo di arma si prende. Controllare l'elenco Armi per Tipologia Omogenea.\index{Tipologia Omogenea}

Il personaggio puo' decidere di assegnare il suo punto ad una tipologia che gia' conosce, migliorando cosi' la sua capacita' e competenza nell'uso od apprendere un altra tipologia di arma. Il giocatore deve considerare che migliore e' la sua capacita' con una tipologia di arma piu' facilmente puo' usufruire di vantaggi nella stessa, ma conoscera' meno armi.

Se il giocatore non ha assegnato alcun punto nella \textbf{CA} puo' utilizzare, senza penalita' al colpire, solo le armi raggruppate come armi semplici.

Le \textbf{armi semplici} sono: Pugnale, Mazza Leggera, Randello, Morningstar,
Lancia corta da fante, Bastone, Balestra (Leggera), Giavellotto\index{armi semplici}

Per poter utilizzare \textbf{Armature Leggere} e' necessario avere almeno un punto in Competenza Armi.\index{Armature Leggere}

Per poter utilizzare \textbf{Armature Medie} e \textbf{Scudi Leggeri} o \textbf{Medi} e' necessario avere almeno 2 punto in Competenza Armi.\index{Armature Medie}

Con almeno 3 punti in CA si possono usare senza penalita' \textbf{Armature Pesanti} e \textbf{Scudi Pesanti}.\index{Armature Pesanti}\index{Scudi Pesanti}

Usare un'\textbf{Arma senza l'adeguata competenza} nel gruppo di appartenenza impone un -2d6 al Tiro per Colpire.\index{Arma senza l'adeguata competenza}

Usare un'\textbf{Armatura senza l'adeguata competenza} impedisce di usare il valore di Agilita' in Difesa ed il bonus conferito dall'armatura alla Difesa si riduce di 1.\index{Armatura senza l'adeguata competenza}

Usare uno \textbf{Scudo senza l'adeguata competenza} peggiora il Tiro per Colpire di 1 e lo scudo conferisce un bonus massimo a Difesa di 1.\index{Scudo senza l'adeguata competenza}

La \textbf{Competenza Magica} (abbreviata in \textbf{CM}) permette al personaggio di poter lanciare piu' magie , piu' potenti, piu' efficaci e piu' facilmente.

Un personaggio con alta \textbf{CM} sa manipolare piu' Essenze e con risultati migliori.

I tiri salvezza si eseguono confrontando la DC (prova di difficolta') che conosce il Narratore con la prova relativa.

Se ti chiedono un Tiro Salvezza Riflessi per evitare un fulmine, farai un Tiro Salvezza Riflessi (3d6 + valore TS Riflessi + bonus Agilita' +- varie ed eventuali) questo valore lo comunicherai al Narratore che lo confrontera' con la DC del Tiro Salvezza.

Il Narratore non ti dira' di fare un Tiro Salvezza a difficolta' 18, e' lui che confronta il tuo tiro con la difficolta', potra' dirti che la prova e' complessa, difficile o facile...


\pagebreak

\section{Costruiamo il Personaggio}\index{Personaggio}

\label{costruiamo-il-personaggio}
\begin{quotebox}
Mai dimenticare chi sei, perché di certo il mondo non lo dimentichera'.
Trasforma chi sei nella tua forza, cosi' non potra' mai essere la tua
debolezza. Fanne un'armatura, e non potra' mai essere usata contro
di te." (Tyrion Lannister)
\end{quotebox}
	
Come prima cosa prepara davanti a te la scheda ed un foglio dove prendere note ed appunti.

Parti immaginando, visualizzando l'aspetto del tuo personaggio. Come te lo immagini ? In possente barbaro delle steppe ghiacciate od un mago scavezzacollo alle prime esperienze ?

Individua il nome e pensa a cio' che conosce, quali esperienze ha avuto e quali lo hanno segnato.

Come si comporta con gli altri? E' un tipo ordinato, ha delle fisse, ha qualche tic o abitudine ?

E' cresciuto in famiglia, in un clan, vagabondo, per strada.. cosa l'ha portato e che scelte ha fatto per arrivare fino ad adesso ?

Quale e' il suo stile di combattimento e strategia tipo ? Magia, Spada, dalle retrovie.. incitare i compagni.. scappare...

Per incominciare leggi il capito sulle Razze ed individua quella del tuo personaggio.

Hai 6 punti caratteristica, 0 e' un valore medio, -1 debole, +1 buono, al massimo una caratteristica puo' avere 3 (eccezionale) come valore. Messi i punteggi applica i modificatori razziali se presenti.

Se hai Intelletto pari o superiore a 2 scegli un altra lingua parlata/scritta oltre al comune.

Competenze Attive, qui ha 3 punti da distribuire tra Competenza Armi, Competenza Magica e Tiri Salvezza.

La competenza armi ti aiuta nel colpire meglio. La competenza magica e' l'unica cosa che ti permette di usare la magia.

Ricorda anche che i punti in Competenze Armi vanno dichiarati a quale lista di armi.

Se non hai punti in Competenza Armi puoi usare solo le armi semplici senza penalita'.

Puoi assegnare ai Tiri Salvezza un punto per tipo (non puoi mettere due punti in un solo Tiro Salvezza al passaggio di un livello). Questi valori determinano la tua capacita' di sopravvivenza e di resistere a traumi e magie

Competenze di Base, al primo livello distribuisci 4 punti con un massimo di 2 punti per Competenza.

Assegna anche 1 punto libero a Professione / Artigianato / Intrattenere / Cultura giustificando il background del personaggio.

Ad ogni livello successivo distribuisci un numero di punti pari a meta' del punteggio di Intelletto +2 tra le competenze gia' conosciute e perfezionate nell'avventura o apprese ex novo.

A questo punto scegli i Tratti. Fallo con attenzione e cura, stai costruendo il tuo personaggio ed i tratti delineano a forti pennellate il carattere. Ricordati che saranno fondamentali per le Essenze.

Scegli lo svantaggio di ruolo e se vuoi anche svantaggi e vantaggi. Ricorda di giocarlo, altrimenti non e' divertente.

Se hai messo dei punti in Competenza Magica a questo punto devi scegliere (dopo aver parlato con il Narratore su quale sistema magico e' in uso) quali Essenze conosci. Ricorda che per ogni punto in Competenza Magica puoi decidere di apprendere due nuove Essenze o specializzarti in un Essenza gia' nota dandogli un bonus di +1 alle prove.

Passa alle Abilita', al primo livello ne scegli due, stai attento ai prerequisiti ed anche ad eventuali abilita' che ti concede la tua razza. Ogni livello dispari prenderai un altra abilita'.

Scegli l'equipaggiamento, armatura, armi, zaino, due torce, qualche razione di cibo.. un peluche.. quello che ti sembra indispensabile per l'avventura.

Aggiorna poi la parte di scheda relativa alla Difesa segnando che bonus ti da l'armatura e scudo indossata.



\pagebreak

\section{Regole per le competenze}\index{Regole per le competenze}\index{Competenze}

\label{regole-per-le-competenze}
\begin{quotebox}
Occorre che la legge sia breve, perché piu' facilmente i mal pratici la ricordino. (Seneca)
\end{quotebox}

\textbf{Le prove (i check) per le competenze si eseguono tirando 3d6, al risultato dei dadi si somma il punteggio della competenza (di base, attiva) e della caratteristica collegata ed eventuali bonus magici e di circostanza o Abilita', il risultato ottenuto deve essere comunicato al Narratore, il quale lo confrontera' con il DC della prova.}

Quando dovete stabilire una difficolta' partite pensando che la prova deve essere rapportata da una persona ``normale''. Non pensate ``se la dovessi fare io allora la prova sarebbe impossibile'', ``se la prova la fa Arsenio Lupin la prova e' facilissima'. Partite dal presupposto che la difficolta' deve racchiudere in se tutti gli elementi circostanziali.

Pensate se piove, c'e' poca luce, il personaggio sta correndo, e' ferito, fa le cose di fretta ed anche alla complessita' della cosa che deve fare, saltare un fosso di 4 metri non e' come uno di 2 metri al buio, senza scarpe, sotto la pioggia ed inseguiti e con le tasche strapiene di monete...

Decifrare uno scritto antico potra' essere una passeggiata per un linguista esperto, ma per una ``persona normale'' che non ha idea di cosa puo' avere davanti la prova e' semplicemente impossibile.

E non spaventatevi se i personaggi falliscono le prove, rendera' l'avventura piu' interessante, e vi permettera' a voi Narratore di introdurre fatti, suggerimenti ed indizi.

\bigskip

\textbf{Quando devi fare una prova per una competenza di base in cui
non sei preparato, ovvero non hai punti devi tirare solo 1d6 + punteggio
della caratteristica collegata.}

\textbf{Quando si scrive -1d6 significa che si tira un dado in meno
(o due se e' -2d6), parimente se c'e' scritto +1d6 si tira un dado
a 6 in piu' e si somma.}

\bigskip

La tabella qui sotto serve a rapportare la difficolta' alla abilita' minima necessaria per riuscire la prova con un tiro medio (un punteggio di 10 lanciando 3d6). Usate queste indicazione per avere una idea delle scale di difficolta'.

Il Narratore non ti dira' fammi una prova a difficolta' 10, ma dira' che la prova non presenta elementi di particolare difficolta'.

\bigskip

\textbf{Classe di Difficolta'}\index{Classe di Difficolta'}

\begin{tabular}[c]{@{}lll@{}}
\toprule 
Classe di Difficolta' (DC) & Descrizione difficolta' & Competenza necessaria\tabularnewline
DC 5 & Estremamente facile & Mediocre\tabularnewline
DC 10 & Facile & Normale\tabularnewline
DC 15 & Normale & Buona\tabularnewline
DC 20 & Difficile & Ottimo\tabularnewline
DC 25 & Molto difficile & Eccellente\tabularnewline
DC 30 & Estremamente difficile & Stupefacente\tabularnewline
DC 35 & Quasi impossibile & Fenomenale\tabularnewline
DC 40 & Inumana & Divina\tabularnewline
\bottomrule
\end{tabular}

\bigskip

Se devi fare una prova su una Caratteristica non correlata ai Tiri Salvezza, devi tirare 3d6 e sommare il punteggio della Caratteristica e altri punteggi inerenti. Confronta poi questo risultato con la difficolta' (DC) che ti dice il Narratore.

\subsection{Superare o Fallire la prova di tanto...}\index{Superare o Fallire la prova di tanto}

Ogni qual volta la prova e' superata brillantemente (di 10 o piu' rispetto alla difficolta' necessaria) il Narratore puo' decidere di dare maggiori informazioni, concedere bonus alle azioni successive.. qualsiasi cosa possa valorizzare quanto agevolmente la prova e' stata superata.

Viceversa se la prova fallisce di 10 o piu' rispetto al valore necessario il Narratore potrebbe descrivere come miseramente la prova e' fallita e come il risultato pessimo influenzi l'azione o quelle successive.

Questi modificatori aggiuntivi non si applicano alla prova del Tiro per Colpire.

\subsection{Prove opposte}\index{Prove opposte}

Ci sono situazioni in cui il personaggio deve effettuare una prova in contrapposizione con un avversario, ad esempio muoversi silenziosamente alle spalle di una guardia, rubare dalle tasche del mercante, intimidire l'orchetto per farsi dare indicazioni..

In questi casi il personaggio ed il Narratore effettuano una prova, chi ottiene il valore piu' alto vince, in caso di parita' vince chi ha il valore piu' alto nella competenza, poi nella caratteristica ed infine l'eventuale ``avversario''.

\bigskip

\textbf{Alcuni esempi di prova contrapposta}

\begin{itemize}
\item Ingannare qualcuno: Faccia Tosta \textgreater{} Consapevolezza
\item Travestirsi per sembrare qualcun'altro: Intrattenere \textgreater{} Consapevolezza
\item Creare una mappa falsa: Cultura \textgreater{} Cultura
\item Nascondersi: Consapevolezza \textgreater{} Consapevolezza
\item Indimidire: Faccia tosta \textgreater{} TS Volonta'
\item Rubare: Criminalita' \textgreater{} Consapevolezza
\item Slegarsi da delle corde: Criminalita' \textgreater{} Sopravvivenza
\end{itemize}

Ogni qual volta la prova opposta riguarda una \textbf{caratteristica correlata ai Tiri Salvezza} (Potenza, Agilita', Volonta') per tutti gli interessati, fate fare un Tiro Salvezza come valore contrapposto. Es. una prova di braccio di ferro e' un Tiro Salvezza Tempra contrapposto

\bigskip

\subsection{Vantaggi e Svantaggi}

Il Narratore a seconda delle circostanze puo' concedervi un bonus od un svantaggio.

\begin{tabular}[c]{@{}lll@{}}
\toprule 
Vantaggio/Svantaggio & Valore in prove dinamiche & Valore in prove fisse\tabularnewline
Bonus leggero & +1 & +1\tabularnewline
Bonus normale & +2 & +2\tabularnewline
Bonus forte & +1d6 & +4\tabularnewline
Bonus molto forte & +2d6 & +8\tabularnewline
Svantaggio leggero & -1 & -1\tabularnewline
Svantaggio normale & -2 & -2\tabularnewline
Svantaggio forte & -1d6 & -4\tabularnewline
Svantaggio molto forte & -2d6 & -8\tabularnewline
\bottomrule
\end{tabular}

\bigskip

Il valore nelle prove dinamiche e' da usarsi quando la prova viene fatta tirando i 3d6, in questo caso si potranno sommare bonus (+2) o addirittura tirare dadi in piu' (+2d6) o se in svantaggio dadi in meno, fino a tirare solo 1d6 (con 2d6 di penalita')

Si intendo prove a valore fisso quando non e' necessario tirare dei dadi (es. Difesa), in questo caso il punteggio si alza/abbassa del punteggio indicato.

Cercate di rimanere sempre tra questi valori di vantaggio e svantaggio, altrimenti potete dire che la prova e' direttamente riuscita o fallita'.
Il giocatore puo' comunque richiedere di effettuare la prova anche se il risultato e' certo.

\textbf{Se un personaggio non e' in difficolta' o pressione} nell'effettuare la prova puo' prendere il 10 (+ competenze + abilita..), ovvero non tirare i dadi e considerare che abbia tirato 10 con i dadi. L'azione impiega 10 round.

\textbf{Se il personaggio non ha impellenti limiti di tempo}, ovvero puo' dedicare almeno 10 minuti per lavorarci (100 round) puo' considerare di prendere 15. Ovvero come se avesse fatto la prova e tirato 15 con i 3d6.

\textbf{Se il tempo diventa un fattore da non considerare}, ovvero il personaggio ha almeno 1 ora per pensare e lavorare considerare di avere tirato 18.

E' il Narratore che ti permette o meno di usare questi punteggi, in base alla situazione, urgenza, pericolosita' di cio' che ti circonda. Mettersi a scassinare una porta in un dungeon chiedendo il 10 richiede un estremo sangue freddo o incoscienza.

Prendere il 18 e' fattibile solo se il personaggio non ha penalita'
nell'effettuare la prova.

\textbf{Aiutare un altro}:\index{Aiutare un altro} si puo' aiutare un amico in una prova dandogli supporto e suggerimenti. L'aiutante deve effettuare una prova a due gradini di difficolta' inferiore (-10) (esempio se il personaggio impegnato deve fare una prova a difficolta' 25 l'aiutante la fa a difficolta' 15, se ci riesce da un +2 alla prova del compagno. Piu' personaggi possono aiutare lo stesso amico; i bonus di questo tipo sono cumulabili fino ad un bonus pari alla meta' della difficolta' da battere (es +12 nel caso di difficolta' 25).

Il Narratore valutera' la possibilita' che piu' di un personaggio fornisca aiuto valutando spazi, modi e tempi (non e' facile aiutare qualcuno ad infilare un filo nella cruna di un ago).

\textbf{Esplosione del 6}: \index{Esplosione del 6}anche nelle prove di delle competenze base c'e' l'esplosione del 6. Se con un dado fai 6 lo sommi e ritiri e continui cosi' se fai ancora 6.

\textbf{Tirare un 1 nei check}: \index{Tirare un 1 nei check}porta male anche nei check competenze, ovvero non si somma. Tirare un 3 nella prova di competenza (tutti 1 nei 3d6) non e' un fallimento automatico e' solo un tiro molto basso (zero).

\subsubsection{Successo Parziale - Opzionale}\index{Successo Parziale}\index{Prova con Rischio}

Il Narratore puo' anche decidere di valutare una prova non riuscita come un successo parziale.

Se la prova fallisce di 1 potra' considerarsi riuscita anche se con un problema leggero, se e' fallita di 2 si porta dietro un problema serio se e' fallita di 4 e' riuscita con un problema critico.

Ad esempio se Tups vuole slegarsi i polsi, con un fallimento di 1 avra' fatto rumore forse svegliando la guardia, con un fallimento di 2 si e' slogato il pollice con un -2 alle prove di CA e prove con quella mano per un giorno.

Con un fallimento di 4 si e' slogato il polso ed ha bisogno di aiuto urgente!

In questo sistema puo' essere anche il giocatore a richiedere una ``Prova con Rischio'' in situazioni di particolare tensione ed urgenza.

\pagebreak

\subsection{Descrizione delle Competenze di Base}

\label{descrizione-delle-competenze-di-base}
\begin{quotebox}
E quando Alessandro vide l'ampiezza dei suoi domini pianse, perché non c'erano piu' mondi da conquistare. Sono i vantaggi di un'istruzione classica. (Hans, Trappola di Cristallo, 1988)
\end{quotebox}

\textbf{Criminalita}' (Agilita'):\index{Criminalita}

questa competenza riguarda molte delle capacita' spesso usate dai ladri. Ogni qual volta c'e' da scassinare un lucchetto, liberarsi da dei legacci o manette, rubare dalle tasche di qualcuno, oppure travestirsi, mascherarsi, truccarsi per sembrare qualcun'altro, e' necessario una prova di Criminalita'.

Solitamente ad una prova di Criminalita' e' contrapposta alla prova di Consapevolezza dell'avversario.

\textbf{Sopravvivenza} (Volonta'):\index{Sopravvivenza}

questa competenza riguarda tutta una serie di capacita' e conoscenze legate dall'attenzione e dalla vita selvaggia.

Ogni qual volta si debba legare un nemico, seguirne le tracce, sopravvivere nei boschi recuperando cibo ed un giaciglio, o si debba calmare ed addestrare animali entra in gioco la Sopravvivenza.
Anche l'abilita' di pronto soccorso, per determinare malattie, prestare le prime cure o anche a lungo termine, ricade in sopravvivenza.

\textbf{Faccia Tosta} (Magnetismo):\index{Faccia Tosta}

ogni qual volta il personaggio deve interagire con un altro convincendolo, persuadendolo anche in maniera aggressiva oppure per ingannarlo si usa la competenza Faccia Tosta.

Se il personaggio vuole ingannare un avversario e' una prova opposta di Faccia Tosta contro Consapevolezza (percepire inganni).

Intimidire l'avversario e' una prova contrapposta a difficolta' 3d6 + livello (dadi vita) dell'avversario + Magnetismo dell'avversario.

\textbf{Acrobatica} (Agilita'):\index{Acrobatica}

se devi trovare l'equilibrio su un sottile cornicione, atterrare sui piedi dopo un salto, saltare su un terrazzo, arrampicarsi su una parete sono tutte prove di Acrobatica.

Ogni qual volta agilita', precisione ed equilibrio del personaggio sono messe a prova viene richiesta una prova di Acrobatica.

\textbf{Consapevolezza (}Volonta'\textbf{)}:\index{Consapevolezza}

la Consapevolezza e' la capacita' di verificare le piccole cose, di accorgersi dei piccoli mutamenti. Consapevolezza e' la capacita' di calarsi nell'ambiente e farlo proprio.

Ogni qual volta dovete verificare, vedere, udire, percepire qualcosa, sentire nella voce dell'avversario una incrinatura perche' si sta raccontando menzogne oppure voi dovete fondervi nell'ambiente muovendosi silenziosamente o nascondendovi allora sono effettuate una prova di Consapevolezza.

\textbf{Cultura} (Intelletto):\index{Cultura}

tutto cio' che e' sapere non magico e' cultura. Possa essere la storia e gli avvenimenti di un continente, le tradizioni ed abitudini di una nazione. Sapere come si prepara un decotto o quale e' il nome di un animale, ogni ora passata su un libro e' conoscenza.

Sapere se un edificio e' stato costruito bene, e come e quando puo' essere una prova di ingegneria.

Conoscere le feste tipiche di una divinita' e' una prova di religione. Conoscere i mostri tipici di un dungeon e le loro abitudini alimentari e' una competenza di dungeon.

Falsificare uno scritto o mappa in maniera credibile richiede Cultura

\textbf{Arcano} (Intelletto):\index{Arcano}

ricordare i dettagli su un antico oggetto magico, riconoscere uno scritto magico in una pergamena, comprendere le Essenze lanciate da altri o riconoscerne gli effetti.

Sapere la geografia astrale dei piani, sapere usare un oggetto magico intuendone le funzioni, riconoscere creauture magiche (DC e' il CR della creatura modificato dalla rarita' della stessa). Riconoscere o sapere formulare rituali non standard.

Ogni qual volta si ha a che fare con la magia in qualsiasi sua forma e' una prova di Arcano.

\textbf{Lavoro} (Volonta'):\index{Lavoro}

qualsiasi mestiere che costruisca (Artigianato, come sarto, fabbro, apicoltore, barcaiolo.. cacciatore.. conciatore..) o fornisca servizi (Professione come architetto, avvocato.. mercante.. ) e' assimilabile ad un Lavoro.

La competenza in lavoro permette di guadagnarsi da vivere e conoscere un mestiere che potrebbe rendersi utile.

Lavoro e' spesso una competenza di background del personaggio, ma rimane sempre utile in tante e diverse situazioni.

\textbf{Intrattenere} (Magnetismo):\index{Intrattenere}

un personaggio che voglia cantare, suonare, recitare, travestirsi deve effettuare una prova di Intrattenere.

Maggiore e' la competenza di Intrattenere maggiore sara' noto il nome del personaggio, maggiore sara' l'affluenza alle sue esibizioni.

Una prova di diplomazia fatta su Intrattenere sara' basata piu' sulla fisicita' e sensualita' che sulla capacita' di argomentare.

\textbf{Resistenza} (Potenza):\index{Resistenza}

Nuotare in acque placide o burrascose, correre per chilometri , saltare un crepaccio o scalare una dirupo sono tutte prove di Resistenza.

Ogni qual volta si deve mettere a prova forza e resistenza fisica entra in campo la prova di resistenza.

Valutate oltre alla pura ``forza/resistenza' anche i fattori esterni in quanto possono influire facilmente con modificatori anche significativi.

\bigskip

Come sempre ogni volta che si stabilisce una difficolta' cercate sempre di essere obiettivi e lineari, valutate ogni interferenza e fattore esterno che possa essere applicato, considerate i bonus e malus della situazione e siate corretti.

\subsubsection{Esempi Prove Competenza}

\label{esempi-prove-competenza}

\begin{itemize}
	\item Una prova riuscita (base DC 15) in Sopravvivenza (pronto soccorso) puo' fare recuperare 1d4 PF dopo uno scontro o concedere un +2 ad un tiro salvezza su tempra per resistere ad un veleno.
\end{itemize}

\begin{itemize}
	\item Saltare	
\end{itemize}

\index{Saltare}\index{Salto}

\begin{tabular}[c]{@{}ll@{}}
\toprule 
\textbf{Salto in Lungo (Distanza)} & DC\tabularnewline
1.5 m & 5\tabularnewline
3 m & 10\tabularnewline
5 m & 15\tabularnewline
7 m & 20\tabularnewline
+1,5 m & +5\tabularnewline
\bottomrule
\end{tabular}
\bigskip

\begin{tabular}[c]{@{}ll@{}}
	\toprule 
\textbf{	Salto in Alto (Altezza)} & DC\tabularnewline
	0.02 m & 4\tabularnewline
	0.5 m & 8\tabularnewline
	1 m & 12\tabularnewline
	1.5 m & 16\tabularnewline
	+0.5 m & +4\tabularnewline
	\bottomrule
\end{tabular}

\bigskip

\subsubsection{Linguaggi}\index{Linguaggi}

\label{linguaggi}

In Yeru ogni razza e' custode di una propria lingua parlata e scritta. Ogni personaggio che abbia almeno Intelletto -1 parla il linguaggio della propria razza, con 1 lo scrive anche. 
Per ogni punto superiore o pari a 2 parla e scrive un altra lingua che sara' scelta alla creazione del personaggio.
Per ogni punto in Cultura che dedica specificatamente ai linguaggi, parla e scrive un altra lingua.

\bigskip

\textbf{Tabella delle Lingue}

\begin{tabular}[c]{@{}lll@{}}
\toprule 
Razza & Linguaggio Parlato & Scritto\tabularnewline
Nanica & Nanico, Comune & Nanico\tabularnewline
Umani & Comune & Comune\tabularnewline
Elfica & Elfico & Elfico\tabularnewline
Drow & Elfico, Comune & Elfico\tabularnewline
Orco, Goblin. Gnoll & Goblinoide & Goblinoide\tabularnewline
Giganti & Gigante & Nanico\tabularnewline
Uccelli senzienti & Auran & Auran\tabularnewline
Abitanti marini senzienti & Acquan & Acquan\tabularnewline
Abitanti dei boschi & Silvano & Silvano\tabularnewline
Draghi & Draconico & Draconico\tabularnewline
Celestiale & Celestiale & Celestiale\tabularnewline
Infernale & Infernale & Infernale\tabularnewline
Elementali Fuoco & Ignam & Ignam\tabularnewline
Elementali della Terra & Terran & Terran\tabularnewline
\bottomrule
\end{tabular}

\subsubsection{Volare}

\label{volare}

Volare e' una competenza che si impara con sofferenza ed impegno. Se si vuole imparare a governare il volo e' necessario dedicare specifici punti di competenza a Volare e sapere volare.

\pagebreak

\section{Combattimento}\index{Combattimento}

\label{combattimento}
\begin{quotebox}
Si vis pacem, para bellum (``se vuoi la pace, prepara la guerra', anonimo)\linebreak\linebreak
Non conta come cadi, ma se e come ti rialzi (anonimo)\linebreak\linebreak
Non sono un eroe. No e non lo saro' mai. Sono solo un cattivo che viene pagato per pestare tipi peggiori di lui. (Deadpool)
\end{quotebox}

Il combattimento e' tra le fasi principali di un avventura ed e' quando i prodi o codardi danno sfoggio della loro maestria con le armi o magie.

\bigskip

Il combattimento e' diviso in 2 fasi:\index{combattimento}
\begin{itemize}
\item verifica dell'iniziativa 
\item risoluzione delle azioni (movimento, attacco, azione varie..) 
\end{itemize}

\subsection{L'Iniziativa}\index{Iniziativa}

\label{liniziativa}

L'iniziativa e' una prova (3d6+) di Agilita' o Intelletto ed abilita' inerenti che potete avere.

Il giocatore scegli il valore che preferisce. Se viene scelta Agilita' saranno i riflessi a determinare la reazione del personaggio, mentre Intelletto guidera' la capacita' di cogliere le tattiche dell'avversario ed anticiparle.

Chi ha l'iniziativa tra giocatori e nemici piu' alta incomincia per primo e successivamente agiscono gli altri in ordine decrescente. 

In caso di Iniziativa di pari punteggio agisce per primo chi ha il punteggio caratteristica piu' alto, altrimenti lo scontro sara' in contemporanea.

L'iniziativa vale per l'intero scontro e si ritira al cambio dell'avversario.

\subsubsection{Risoluzione delle Azioni}\index{Risoluzione delle Azioni}

\label{risoluzione-delle-azioni}

Dal piu' veloce al piu' lento c'e' la risoluzione delle azioni.

Il Narratore chiedera' al piu' veloce di dichiarare la sua azione e agire, proseguira' poi a chiedere e fare agire gli altri giocatori e nemici.

In questo modo la scelta dell'azione avviene quando e' il turno del giocatore che potra' agire anche in base alle azioni e risoluzioni gia' avvenute.

E' possibile ritardare la propria azione per aspettare una determinata situazione. Il personaggio che ritarda la propria azione agisce per primo tra i soggetti che agiscono in quel valore o successivo di iniziativa .

Se un personaggio dichiara di fare una certa azione in conseguenza di un'altra vuol dire che ritarda la propria azione, cio' gli sara' possibile solo se ha ancora azioni da spendere nel round ed e' piu' veloce di colui che deve compiere l'azione che scatena la reazione.

\subsection{Azioni nel Round}\index{Azioni nel Round}\index{Azioni}

\label{azioni-nel-round}

Un personaggio puo' eseguire 3 Azioni per round.

Queste azione possono essere eseguite nell'ordine preferito.

Nella tabella sottostante sono indicate le Azioni principali che un personaggio puo' eseguire, sono linee guida da seguire. Nel capitolo dedicato al combattimento vengono elencate altre Azioni ed i loro costi, in azioni, relative.

Una Azione non puo' essere interrotta da un altra Azione, ma puo' essere seguita da una Reazione o da una Azione Immediata, se nel proprio round.

Se un personaggio vuole fare piu' attacchi spostandosi nel campo di battaglia puo', ad esempio, usare una Azione per eseguire un attacco, usare una Azione di movimento per spostarsi fino a tutto il suo movimento a disposizione, ed usare un ultima azione di attacco per eseguire un ultimo singolo attacco, questo secondo attacco conta comunque come attacco multiplo con le dovute penalita'.

\pagebreak

\textbf{Tabella Azioni per Round}

\bigskip

\begin{tabular}[c]{@{}ll@{}}
\toprule 
Cosa si fa & Costo Azioni\tabularnewline
Eseguire un unico attacco con armi in mischia & 1\tabularnewline
Eseguire due o piu' attacchi con armi in mischia & 2\tabularnewline
Scoccare una freccia/dardo & 1\tabularnewline
Scoccare due o piu' frecce/dardo & 2\tabularnewline
Lanciare una Essenza & 2\tabularnewline
Eseguire una Azione di Movimento (ci si sposta fino a tutto
il proprio movimento) & 1\tabularnewline
Scambiare un discorso con qualcuno & 1\tabularnewline
Cercare qualcosa nello zaino di pronto & 2\tabularnewline
Usare qualcosa di appena preso dallo zaino/cintura & 1\tabularnewline
Bere una pozione tenuta alla cintura & 1\tabularnewline
Estrarre l'arma (poi rimane estratta) & 1\tabularnewline
Imbracciare lo scudo (poi rimane imbracciato) & 1\tabularnewline
Usare un anello/bacchetta/verga/bastone magico & 1\tabularnewline
Scambiare poche battute con qualcuno & 0\tabularnewline
Eseguire una prova su una competenza & 2\tabularnewline
Azione Immediata & {*}\tabularnewline
Azione Reazione & {*}\tabularnewline
\bottomrule
\end{tabular}

\smallskip

Questo elenco non e' completo, prendetelo come linee guida per stabilire il peso delle decisioni dei giocatori.

\bigskip

L'ordine con cui si eseguono le Azioni non e' importante. L'Azione di Movimento puo' essere spezzata tra altre Azioni (movimento parziale, attacco/essenza altra azione, movimento parziale).

Un personaggio potrebbe attaccare, muoversi ed ancora attaccare, questo secondo attacco avrebbe le penalita' descritte negli attacchi multipli.
\smallskip

Una Azione ``\textbf{Reazione}'' \index{Azione Reazione}puo' essere eseguita liberamente anche fuori dal proprio round. Questa Azione e' solitamente dovuta ad Abilita' particolari. Se non indicato diversamente una Reazione costa zero Azioni e accade immediatamente dopo la causa che la scatena.
\smallskip

Una Azione ``\textbf{Immediata}'' \index{Azione Immediata}puo' essere eseguita liberamente nel proprio round, primo o dopo la propria Azione. Se non indicato diversamente una Reazione costa zero Azioni.

E' possibile se non descritto specificatamente nell'Abilita' eseguire solo una Azione Immediata ed una Azione di Reazione per round. Una Azione Immediata puo' essere eseguita anche interrompendo un altra Azione.

Una creatura che ha una distanza di mischia (portata) pari o superiore
ai 2 metri si considera che abbia un bonus di +4 all'iniziativa per
il primo round, ovvero come se usasse un'arma lunga. Il round successivo
la sua iniziativa in corpo a corpo non avra' questo vantaggio a meno
che abbia mantenuto la distanza non facendosi raggiungere.

\subsubsection{Il Tempo (Round, Minuti e Turni)}\index{Round}

\label{il-tempo-round-minuti-e-turni}
\begin{quotebox}
``L'esitazione e' la morte del vantaggio'' (Magic, V.E. Schwab)
\end{quotebox}

Un Round dura 6 secondi circa, e' un lasso di tempo normale per agire, correre, parlare.. combattere. Un Minuto sono quindi 10 round, ed un Turno dura 10 Minuti (o 100 round).

I round si usano nelle azioni di combattimento o dove la tensione deve rimanere costantemente alta ed a ogni azione corrisponde un evolversidella situazione.

\pagebreak

\subsection{Movimento}\index{Movimento}

\label{movimento}

\begin{quotebox}``Un mobile piu' lento non puo' essere raggiunto da uno piu' rapido; giacché quello che segue deve arrivare al punto che occupava quello che e' seguito e dove questo non e' piu' (quando il secondo arriva); in tal modo il primo conserva sempre un vantaggio sul secondo'' , Paradosso di Zenone
\end{quotebox}

Il movimento di un personaggio e' dato dalla sua taglia e razza e da cio' che porta, dai pesi, ingombri ma anche magie ed oggetti magici.

Il movimento indica quanti metri per Azione (di Movimento) il personaggio puo' fare.

Una creatura o personaggio potrebbe anche decidere di spostarsi piu' velocemente del solito ovvero correndo o andando veloci.

In caso di \textbf{Corsa} \index{Corsa}si raddoppiano i metri percorsi (2x9 metri) per Azione (di Movimento). Per un umano (Movimento 9) significa fare 18 metri in una Azione.

Chi attacca il personaggio che corre ha un bonus di 1d6 al Tiro per colpire. Il personaggio che va veloce ha un malus di 2d6 nel Tiro per Colpire nel round in cui si corre.

In caso di \textbf{Andare veloci} \index{Andare veloci}la distanza percorsa aumento della meta', quindi da 9 metri si passa a 14 metri, o da 6 metri a 9 metri per Azione di Movimento.

Chi attacca il personaggio che va veloce ha un bonus di 1d6 al Tiro per colpire. Il personaggio che va veloce ha un malus di 1d6 nel Tiro per Colpire nel round in cui si muove veloce.

Non e' possibile spostarsi anche solo di 1 metro se non si spendono Azioni di Movimento.

Queste precisazioni hanno senso e vanno usate quando si tratta di combattere ed il dislocamento e' fondamentale, durante gli spostamenti normali si usa la normale gestione del movimento orario.

Nel caso di spostamento diagonale si conta una distanza di 1,5 metri per quadretto.

Per \textbf{distanza di Tocco} \index{distanza di Tocco} \index{Tocco}si intende una distanza che permette il toccare l'avversario, quindi non piu' di un metro per creature di taglia media.

Se non indicata nell'avversario/mostro la distanza di tocco aumenta di 0.5 metri per ogni taglia oltre la media.

Per \textbf{distanza di Mischia} \index{distanza di Mischia} \index{Mischia}si intende una distanza che permette il combattimento corpo a corpo (1 metro attorno al personaggio). Nei mostri questa distanza e' indicata dalla portata, per le armi da lancio e' chiamata gittata.

Es. per un umano armato di lancia, la distanza di mischia e' 2 metri perche' l'arma e' lunga.

Per un nano armato di martello la distanza di mischia e' 1 metro. Per un gigante delle colline la distanza di mischia e' 2 metri. Per un grande drago la distanza di mischia per il soffio (portata) e' 18 metri dato che puo' attaccare da quella distanza facilmente, mentre per gli artigli e' 3 metri.

Quando si parla di ``\textbf{quadretto}'' \index{quadretto}per indicare una distanza od una influenza si intendo un quadretto di mappa di 1 metro x 1 metro.

L'Azione (di movimento) puo' avvenire prima e dopo l'Azione (di Attacco).

A distanza di mischia una creatura di dimensioni medie puo' avere al massimo 8 creature medie.


\subsubsection{Creature Grandi e Piccole in Combattimento {*} (Opzionale)}

\label{creature-grandi-e-piccole-in-combattimento-opzionale}

Le creature molto piccole possono stare piu' di una ad una distanza
di mischia, mentre creature grandi tenderanno ad occupare tutto lo
spazio di mischia.

\textbf{Tabella: Taglia e Scala delle Creature}
\bigskip

\begin{tabular}[c]{@{}ll@{}}
	\toprule 
	Taglia della Creatura & Creature in distanza di mischia\tabularnewline
	Piccolissima & 100\tabularnewline
	Minuta & 64\tabularnewline
	Minuscola & 32\tabularnewline
	Piccola & 16\tabularnewline
	Media & 8\tabularnewline
	\bottomrule
\end{tabular}

\smallskip
Questi sono i valori tipici delle creature per la taglia indicata.
Sono frequenti eccezioni.
\bigskip

\pagebreak

\subsection{Vita e Morte}

\label{vita-e-morte}
\begin{quotebox}Chi non conosce la morte, non conosce la vita. (Grand Hotel, film 1932)
\end{quotebox}

Quando un personaggio raggiunge i 0 (zero) Punti Ferita si considera svenuto, ovvero inabile a fare qualsiasi cosa. Una Essenza di Cura od una pozione di Cura lo portera’ cosciente ed ai punti ferita curati. Una prova di pronto soccorso (Sopravvivenza) a DC 15 lo portera’ ad 1 punto ferita.

Un personaggio morente ha Punti Ferita negativi (-1 o meno) ed e' privo di sensi e prossimo alla morte. Continuera’ a perdere un punto ferita a round fiche il valore non raggiungera’ il triplo della Potenza+10 ed il personaggio morira’, se non viene curato.
Una Essenza di Cura, di qualsiasi livello di potere lo portera’ a 1 Punti Ferita successive cure funzioneranno normalmente.
Una prova di Sopravvivenza (pronto soccorso, 3 Azioni) a difficolta’ 11 piu’il valore dei punti ferita negativi portera’ il personaggio a 0 punti ferita.

Es. Tups e’ gravemente ferito ed ha attualmente -6 punti ferita, Jade decide di provare a curarlo (dopo averlo spostato in un posto piu’ sicuro). Jade tenta una prova di pronto soccorso per almeno stabilizzare il compagno, la sua difficolta’ alla prova e’ 11+6 ovvero deve superare con Sopravvivenza DC 17 per riportarlo a 0 PF (svenuto)

\textbf{Quando un personaggio arriva a punti ferita negativi pari 10+triplo del suo punteggio di Potenza e’ morto}. Es. Se ha Potenza 2 morira’ a -10-6=-17 PF, se ha Potenza 0 morira’ a 10 PF, se ha Potenza -2 morira’ a -10+6=-4 PF. In caso di valori di Potenza pari od inferiore a -3 il personaggio muore a -3 punti ferita.

\bigskip

Un personaggio morto non puo'’ beneficiare delle cure normali o magiche, e non puo'’ essere riportato in vita da una Essenza. Solo un Patrono ha sufficiente potere per riportare l’anima nel corpo e riportare in vita la creatura. L’Essenza di Distruzione puo' rianimare un corpo, ma come non morto.

Ogni round successivo ad essere andato a 0 punti ferita, quindi svenuto, deve effettuare un Tiro Salvezza Tempra a difficolta’ 15, se riesce riprende coscienza e va ad 1 punto ferita.

Se fallisce la prova puo’ effettuarne un altra a DC +1 rispetto alla precedente il round successivo. Quanto la difficolta’ raggiunge 18 (ovvero 3 prove fallite di seguito) il personaggio incomincia a morire, va a -1 punti ferita e diventa morente.

La prova puo’ essere fatta per un massimo di 5 volte, se i successi sono superiori agli insuccessi il personaggio torna ad 1 PF, altrimenti va a -1 PF (morente).

Eventuali punti caratteristica persi si recuperano al ritmo di 1 punto totale al giorno, se non indicati come perdita permanente
\pagebreak

\subsection{Tiro per Colpire e Difesa}\index{Tiro per Colpire}\index{Difesa}

\label{tiro-per-colpire}
\begin{quotebox}Applica sempre la giusta forza, mai troppa mai troppo poca. (Kano Jigoro)
\end{quotebox}

Il Tiro per Colpire e' una prova contrapposta data da:

Se l'attaccante usa:

\begin{itemize}
	\item \textbf{Armi da Mischia o Contatto}: l'attaccante deve effettuare un \textbf{Tiro per colpire (TC)}= 3d6 + Competenza con Armi + Potenza ed eventuali bonus magici dell'arma o fattori circostanziali (ambiente, maledizioni..)
	\item
	\textbf{Armi da Distanza o Versatili}: l'attaccante deve effettuare un Tiro per Colpire (TC) = 3d6+ Competenza Armi + Agilita' (archi, balestre, pugnali, scimitarre...)
	\item	\textbf{Essenze}: l'attaccante deve effettuare un Tiro per Colpire (TC) = 3d6+ Agilita' 
\end{itemize}

Se la Distanza e' tocco usera' il valore della Difesa a tocco.

Chi si difende ha una \textbf{Difesa} pari a: 10 + Agilita' + Scudo + Armatura + eventuali bonus magici ed Abilita' e bonus circostanziali. 
Il giocatore puo' decidere di rinunciare a del bonus dato dalla Competenza con Armi per avere un migliore punteggio di Difesa. Questi punti non saranno a disposizione nell'attacco successivo.

Ogni qual volta di parla di Bonus Difesa si intende un valore da sommare al valore Difesa ottenuto con il calcolo di cui sopra.

Quando si ha un bonus alla Difesa o Agilita' questo si somma per ottenere il valore di Difesa

\subsection{La Difesa}

\label{la-difesa}
\begin{quotebox}
		La difesa e' sempre legittima (anonima vittima)
\end{quotebox}
Ogni Tiro per Colpire (3d6 + Competenza con Armi + Potenza o Agilita' + eventuali bonus/malus) si raffronta la Difesa ovvero un valore pari a 10 + Agilita' + Scudo + Armatura + eventuali bonus/malus.

Se il Tiro per Colpire e' pari o superiore al valore della Difesa l'avversario e' stato colpito e si stabilira' il grado di ferita, dato dall'arma + punteggio Potenza ed altri fattori quali bonus magici e di abilita' (se in mischia).

Se invece chi ha la difesa ottiene un risultato piu' alto avra' parato, schivato, evitato.. La scelta la si lascia al giocatore, evitato l'attacco non subiscono ferite.

Altre situazione possono avvantaggiare la Difesa quali coperture, nascondigli, come fosse, porte, compagni di taglia molto piu' grande della propria. Consultate i paragrafi relativi per capire il vantaggio che possono dare.

Ci sono occasioni in cui non e' importante penetrare la difesa e sferrare un colpo ma semplicemente basta toccare l'avversario.

Altre volte l'avversario e' sorpreso e non puo' difendersi completamente.

Se il personaggio usa una Essenza, in distanza di Mischia, se non opera tramite un arma, ovvero ``consegna' la magia con una spadata, si considera che sia un incantesimo a tocco.

Se e' \textbf{sufficiente toccare l'avversario} la Difesa sara' 10 + Agilita' + bonus magici, senza bonus Scudo e Armatura.

Se \textbf{l'avversario e' sorpreso} ovvero non si aspetta l'attacco la Difesa sara' 10 + bonus magici + Armatura, senza bonus di Scudo e Agilita'.

\textbf{Anche per il Tiro per Colpire valgono le Golden Rules. I d6 esplodono in caso si tiri 6 con il dado, fare 1 porta male ed affidarsi alla sorte.}

Se il Tiro per colpire (TC) e' superiore od uguale alla Difesa allora hai colpito, se e' inferiore hai mancato.

Se i modificatori e circostanze portano il danno inflitto ad essere negativo comunque farai 1 di danno.

Questa regola si applica ai modificatori del danno dell'arma che appunto non possono portare il danno totale ad essere inferiore a 1, se ci sono protezioni magiche o riduzioni del danno questo puo' diventare zero e quindi non ferirai l'avversario (ma se diventa negativo non lo curi!).

Come prima cosa, come spiegato poco prima, ricorda che per ogni 6 tirato (nei 3d6 del Tiro per Colpire) devi tirarne un altro e continuate a tirare finche' continui a fare 6 con il dado.

Se colpisci, \textbf{ogni due 6 tirati} (contando quelli del Tiro per Colpire e quelli scaturiti dal fatto di aver tirato 6), l'arma fa del danno in piu' ovvero un critico. Tira nuovamente il danno dell'arma + Potenza, senza magia o Abilita' particolari ogni due 6 tirati nel Tiro per Colpire.

Puoi \textbf{togliere 4} al tuo attacco per tirare un d6 in piu', l'azione e' da fare nelle situazioni piu' disperate dove solo la fortuna puo' risolvere il duello.

In caso si tiri un 1 nel Tiro per Colpire questo abbassa di 1 (quindi 1 non conta) il valore totale ma non influisce sul fatto di aver fatto critico o meno.

\textbf{Il fatto di tirare un critico non e' garanzia di aver colpito,
bisogna sempre superare la Difesa}.

Anche per il Tiro per Colpire valgono le regole base delle Competenze.
La Difesa e' un valore fisso e come tale non e' soggetta a modifiche
causate dalle regole base delle Competenze.

\subsection{Tirare 3 volte 1}\index{Tirare 3 volte 1}

Se nei primi 3 tiri per colpire fai tre volte 1 mancherai l'avversario (indipendentemente dal risultato finale del Tiro per Colpire) ed il Narratore potrebbe decidere brutte cose sul tuo attacco (ti cade l'arma, colpisci un amico, ti si rompe l'arma, ti ferisci, cadi, appare un demone delle fosse per caso...)

\subsection{Tiro Critico}\index{Tiro Critico}

Ogni qual volta hai colpito, tiri un danno aggiuntivo di arma (senza bonus magici o di Abilita', solo arma + Potenza) per ogni due volte che hai tirato 6 nel tiro per colpire.

Es tiro 6 4 5, tiro in aggiunta 6, tiro in aggiunta un 6, tiro in aggiunta 4: come danno tiri 2 volte il danno dell'arma ( e sommi due volta la Potenza), una perche' ho colpito una perche' hai tirato tre volte 6 (se avessi tirato un ulteriore 6 sarebbero stati Arma + Potenza + bonus/abilita + 2{*}(Arma+Potenza)).

L'esplosione del danno si ha con il primo dado di danno, eventuali danni critici non esplodono.

\subsection{Esplosione del Danno}\index{Esplosione del Danno}

Ogni qual volta dal tiro del dado del danno ottieni il valore massimo (nel classico d8 per la spada ad esempio fai 8 ed e' quindi il valore massimo del dado), ritiri il dado e sommi ancora il valore (del solo dado). In caso di armi con piu' dadi (esempio 2d4, il valore massimo deve essere ottenuto come somma dei due dadi, ovvero 8). Non c'e' esplosione del danno per le armi con danno inferiore o uguale a 6. Questa esplosione di danno si puo' fare una sola volta.

Alcune armi hanno una esplosione del danno diversa. Nella tabella delle armi dove e' segnato EDX (es ED9), il valore X sta per il valore minimo sufficiente per tirare un'altra volta il danno, quindi in caso di ED9 puoi fare il critico con 9 o 10.

Questa e' una caratteristica di alcune armi estremamente letali.

L'esplosione del danno non esplode a sua volta, anche se fai il massimo del dado con il dado aggiunto questo non esplode nuovamente.

Il danno aggiunto da critico, ottenuto lanciando almeno due 6, non ha il vantaggio dell'esplosione del danno.

\subsection{Attacchi multipli}\index{Attacchi multipli}

Con una Azione il personaggio puo' eseguire un singolo attacco in mischia.

Con due Azioni il personaggio puo' effettuare piu' tiri per colpire consecutivi.

Ogni Tiro per Colpire oltre al primo ha un malus cumulativo di 5, quindi anche eseguendo due Azioni di Attacco separate la prima azione di attacco non ha malus mentre la seconda azione di attacco ha -5 al colpire.

Se ho CA 5 e Potenza 2, il primo Tiro per Colpire ha un bonus di 7, il secondo di 2 (7-5) ed un potenziale terzo attacco ha un -3 al colpire.

Se il malus porta il Tiro per Colpire a -5 o meno allora non e' possibile effettuare ulteriori tiri per colpire.

Nel caso il personaggio voglia eseguire attacchi multipli, deve dichiarare se fare gli attacchi su un solo avversario o su piu'.

Se dopo il primo attacco il target muore (in caso di azione di multiattacco) non puoi dirigere gli attacchi rimanenti su altri bersagli tranne se ha l'abilita' Proseguire e il successivo avversario e' gia' in mischia con te.

Diversamente puo' dichiarare di effettuare il primo attacco su una creatura ed il secondo (o successivi) su altro, purche' in mischia con il personaggio.

\subsection{Armi da Tiro - Archi - Balestre (Arco / Balestre / Pugnali..)}

Il numero di attacchi massimi con armi da lancio e' 2 usando due Azioni.

Solo particolari Abilita' permettono di effettuare ulteriori attacchi.

Il bonus al danno dato da Potenza si applica in automatico per le fionde, Pugnali..ovvero con tutte le armi che vengono lanciate ``a mano'', gli archi applicano questo bonus solo se sono di tipo composito, le balestre non lo applicano mai.

\subsection{Attacchi con armi a spargimento (olio incendiato/acqua benedetta..)}\index{armi a spargimento}

In caso l'attacco manchi tirare un d8 e consultare questo schemino per capire dove la boccia e' caduta:

1 2 3

4 \textbf{X} 5

6 7 8

\textbf{X} si considera il bersaglio del tiro.

Se il tiro manca di 5 o piu' tirare un 2d6 per determinare lungo la direzione indicata dal d8 precedente a quanti metri e' caduto distante dal bersaglio, ovvero contate i metri dal target.

Ad esempio con il tiro del d8 faccio 5 e poi tirando 2d6 faccio 4, significa che la boccetta e' caduta a destra del bersaglio a 4 metri.

E' anche possibile che ci si sia tirati sui piedi la boccetta (es faccio 7 e poi 6.. potrei averla tirata addosso ad un compagno o dietro di me!).

\subsection{Impreparato -- Colti di Sorpresa}\index{Impreparato}\index{Sorpresa}

Se i personaggi vengono colti di sorpresa, ovvero non si aspettano di essere attaccati, si deve considerare questo primo round come round di sorpresa. Quando si e' sorpresi non si puo' usare la propria Agilita' in Difesa.

Per quel round e per quell'attacco ti difenderai solo con la Armatura (senza scudo), non potrai reagire; dal round successivo potrai dichiarare l'iniziativa ed agire normalmente. Le medesime considerazioni valgono per gli avversari.

Per valutare se un personaggio e' sorpreso effettuate un tiro salvezza su Riflessi, confrontandolo con la bravura nel nascondersi degli avversari, se la prova e' fallita il personaggio e' sorpreso. 

Quando personaggi e nemici sono colti entrambi di sorpresa per valutare chi effettivamente e' sorpreso effettuate un Tiro Salvezza su Riflessi, chi ottiene il risultato piu' basso e' sorpreso e per quel round non potra' agire.

\subsection{Modificatori di attacco o difesa per situazioni particolari} \index{situazioni particolari}

Il migliore suggerimento che si puo' dare nel gestire le situazioni di combattimento piu' caotiche e' pensare a queste come ad un film, valutate la cinematicita' della situazione.

Non e' una questione di miniature, spazi, quadretti.. e' una questione di divertimento e visualizzazione della scena. Soluzioni non ortodosse per situazioni non ortodosse.

Concedete un d6 di bonus o malus (ovvero togliete un d6 dal numero di d6 da tirare) ogni qual volta il giocatore abbia un vantaggio o svantaggio ed allo stesso modo all'avversario. Se il bonus o malus e' sulla Difesa allora usate +-2 (o +-4 se il bonus e' maggiore) di bonus al posto del d6.

\bigskip

\textbf{Esempi in situazione di Attacco (bonus o malus al Tiro per Colpire)}

\begin{itemize}
	\item Situazione con +2 bonus: piu' di uno ad attaccare un avversario

	\item Situazioni con 1d6 bonus: posizione sopraelevata, carica, invisibile

	\item Situazione con 1d6 di svantaggio: sei abbagliato, sei intralciato, sei prono, sei ristretto nei movimenti, sei spaventato o scosso, usare un arma da lancio contro un avversario in mischia, attaccare con arma lunga in distanza da mischia
\end{itemize}

\textbf{Esempi in situazione di Difesa:}

\begin{itemize}
	\item Situazioni con +2/+4 bonus (bonus alla Difesa): hai copertura (vedi sotto),

	\item Situazioni con -4 di svantaggio (malus alla Difesa): sei accecato, immobilizzato, sei in ginocchio o seduto, sei prono, sei ristretto in uno spazio, sei stordito, lanci una Essenza 
\end{itemize}

\textbf{Quando si scrive -1d6 significa che si tira un dado in meno (o due se e' -2d6), parimente se c'e' scritto +1d6 si tira un dado a 6 in piu' e si somma}.

\textbf{Quando il malus e' alla Difesa considerare ogni -1d6 come un -4 alla Difesa}.

Se non si vuole tirare il dado di bonus/malus considerare allora ogni d6 come valore +-4 (a seconda che sia un bonus od un malus).

\textbf{In linea di principio in combattimento un bonus leggero e' un +2, un bonus medio e' +1d6 (o +4), un bonus molto alto e' +2d6 (o +8), viceversa per per i malus}.

\bigskip
Ricordate sempre lo scopo e' divertirsi, a scapito (per il Narratore) di qualche mostro, non siate rigidi ma dinamici e adattatevi alle situazioni.

\pagebreak

\subsection{Azioni particolari in combattimento:}

\subsubsection{Attacco di Opportunita'} \index{Opportunita}se un avversario usando una Azione di movimento esce o attraversa, comunque termina il suo movimento fuori dalla zona di mischia del personaggio, al personaggio e' concesso un singolo attacco. Questo attacco e' una Reazione che costa una Azione. Stessa cosa vale per gli avversari.

\subsubsection{Carica} \index{Carica}l'avversario deve essere ad una distanza entro 1 Azione di movimento (9 o 6 metri). Si corre fino ad essere a distanza di mischia.

Si ottiene un +1d6 a Tiro per Colpire, -4 alla Difesa, l'attacco successivo al primo ha un -15 al colpire (questo per valutare se e' possibile farlo o meno). La carica e attacco costa 3 Azioni. Non si considerano altri malus per avere corso oltre quelli indicati.

\subsubsection{Controcarica}\index{Controcarica} un'arma con il talento controcarica se usata contro un avversario/cavalcatura in carica infligge il doppio del danno dell'arma e colpisce per prima, tranne in cui l'avversario abbia un arma lunga e sia in carica in questo caso l'attacco e' contemporaneo. Preparare un'arma per la controcarica e' una Reazione che costa una Azione.

Se l'avversario attacca con un arma lunga senza caricare non si usa
l'azione di controcarica, ma si tirano le rispettive iniziative.

\subsubsection{Carica con Arma da Controcarica} \index{Controcarica}se usi un arma con il talento controcarica per caricare un avversario la tua arma fa il doppio del danno dell'arma (bonus magici esclusi). Se il difensore non ha un arma lunga allora valgono anche le considerazioni di Arma Lunga.

\subsubsection{Aiutare un altro}\index{Aiutare} Si puo' aiutare un amico ad attaccare o a difendersi negli scontri in mischia, distraendo o interferendo con un avversario.

Si puo' portare un attacco in mischia (1 Azione) contro un avversarioche ha gia' ingaggiato battaglia con un proprio alleato.

Si effettua un Tiro per Colpire contro Difesa dell'avversario con 1d6 di vantaggio. Se l'attacco va a segno, non si fa danno, l'amico ottiene bonus di +1d6 al Tiro per Colpire con il prossimo attacco (entro la fine del successivo round) verso quell'avversario o un bonus di +4 alla Difesa contro il prossimo attacco di quell'avversario (a propria scelta). 

Piu' personaggi possono aiutare lo stesso alleato; i bonus di questo tipo sono cumulabili (massimo 4 su taglia media), purche' l'avversario sia circondato.

\subsubsection{Colpo di Grazia} \index{Colpo di Grazia}costa 3 Azioni, si puo' utilizzare un'arma da mischia per infliggere un colpo di grazia ad un target indifeso. Si puo' anche usare un arco o una balestra, l'importante e' che si sia adiacente al bersaglio.

L'attaccante colpisce automaticamente ed infligge due colpi critici (due volte in piu' il danno dell'arma e Potenza). Se il difensore sopravvive al danno, deve superare un Tiro Salvezza su Tempra DC pari ai danni inflitti o muore.

Le creature immuni ai colpi critici, non subiscono danni critici, né devono superare un Tiro Salvezza su Tempra per evitare di essere uccisi da un colpo di grazia.

\subsubsection{Danno non letale}\index{Danno non letale} il danno non letale e' una forma di danno causato da armi particolari o quando volutamente lo scopo e' fare svenire il nemico e non ucciderlo.

Il danno non letale si tratta come il danno normale ma va segnato a parte nella scheda.

\subsubsection{Senza Competenza}\index{Senza Competenza} (Arma) usare un'Arma senza l'adeguata competenza impone un -2d6 al Tiro per Colpire (quindi il TC diventa 1d6+Potenza+abilita'..) 

\subsubsection{Armi Leggere} \index{Armi Leggere}Il giocatore puo' usare queste armi come armi secondarie senza subire penalita'.

\subsubsection{Armi Versatili} \index{Armi Versatili}Il giocatore puo' liberamente usare la Agilita' invece della Potenza sui tiri per colpire e danno con armi versatili.

\subsubsection{Lanciare armi} \index{Lanciare armi}una spada o comunque un arma non fatta per essere lanciata puo' comunque essere scagliata contro l'avversario.
Il Tiro per Colpire prende un -1d6 e l'arma fa una categoria di danno inferiore (la spada lunga fa 1d6, una spada corta 1d4..). La gittata di lancio e' 3 metri.

\subsubsection{Colpi Potenti}\index{Colpi Potenti} il giocatore puo' liberamente aggiungere un +2 al danno togliendo 1 al Competenza Armi (requisito Competenza Armi +1). Non si puo' togliere piu' di CA/4 al Tiro per Colpire.

\subsubsection{Maestria del combattimento} \index{Maestria del combattimento}il giocatore puo' liberamente aggiungere +4 alla Difesa per ogni -1d6 al Competenza Armi. Il bonus e' applicabile solo per gli attacchi in mischia.

Viceversa puo' prendere un -4 Difesa per alzare di +1d6 il Tiro per Colpire e quindi migliorare l'attacco. Non e' possibile assegnare in questa maniera piu' di 2d6.

\subsubsection{Danno non letale con arma non idonea} \index{Danno non letale con arma non idonea}se vuoi fare danno non letale con un'arma non predisposta al danno non letale hai un -1d6 al Tiro per Colpire.

\subsubsection{Fiancheggiare} \index{Fiancheggiare}se due personaggi sono attorno allo stesso bersaglio ma non sono a fianco entrambi prendono +2 al Tiro per Colpire o alla Difesa (a loro scelta quale bonus prendere).

Al massimo ci possono essere 4 personaggi attorno ad una creatura di taglia media che prendono il bonus di fiancheggiare. Il tipo di bonus si sceglie round per round, se non dichiarato vale come +2 al Tiro per Colpire.

Se tirando una ipotetica riga che collega i due personaggi questa attraversa in pieno il quadretto dell'avversario allora c'e' la situazione di fiancheggiamento.

\bigskip

Esempio di fiancheggiamento

\begin{tabular}[c]{@{}lll@{}}
\toprule 
Personaggio A & Personaggio G & Personaggio D\tabularnewline
Personaggio B & Avversario & Personaggio E\tabularnewline
Personaggio C & Personaggio H & Personaggio F\tabularnewline
\bottomrule
\end{tabular}

\bigskip

In questo schema il fiancheggiamento e' preso dalle coppie: A-F, B-E, C-D, G-H

\bigskip

Se la creatura puo' fronteggiare ed attaccare piu' creature contemporaneamente queste non godranno del bonus di fiancheggiamento.

\subsubsection{Arma Doppia} \index{Arma Doppia}un'arma doppia e' un'arma che e' pericolosa da entrambe le estremita'. puo' essere usata come arma singola, oppure, incorrendo nelle penalita' del combattimento con due armi, come appunto due armi. Se non specificato un'arma doppia usata per combattimento con due armi equivale ad usare due armi medie.

\subsubsection{Arma Lunga} \index{Arma Lunga}l'arma lunga da diritto a colpire piu' lontano ovvero entro 2 metri. Concede un bonus all'iniziativa pari a +4. Questo bonus rimane valido finche' l'avversario non entra in distanza di mischia (ovvero entro 1 metro).

Nel caso in cui anche l'avversario abbia un arma lunga non considerare il bonus di 4 all'iniziativa (avendolo entrambi si annulla a vicenda).

Es. Tups armato di spada lunga affronta uno brigante armato di lancia lunga. Tups ha iniziativa 5, il brigante 2 ma ha un arma lunga e quindi la sua iniziativa e' 6.

Il brigante attacca Tups mentre questo si avvicina, il valore di iniziativa mi ``mostra' come il brigante sfruttando la sua arma lunga riesca ad agire prima di Tups. Una volta che Tups si e' avvicinato in mischia sara' piu' veloce del brigante.

Se il brigante avesse avuto iniziativa 0 Tups avrebbe attaccato per primo, praticamente il brigante non sarebbe riuscito a sfruttare il vantaggio dato dell'arma lunga (5 contro 4 di iniziativa).

Se il brigante avesse dichiarato di attaccare e poi allontanarsi avrebbe costretto Tups ad usare due azioni di movimento per raggiungerlo.

Il brigante avrebbe attaccato da distanza di 2 metri ed usato una Azione per andare in distanza di 11 metri (2+9 di velocita'). Tups avanzando normalmente non avrebbe raggiunto il brigante solo facendo una doppia Azione riesce ad andare in mischia

Tups potrebbe correre, usando quindi solo un'Azione (9mx3=27 metri) e poi attaccare ma avrebbe un -2d6 al Tiro per Colpire.

Il brigante una volta raggiunto da Tups butta l'arma lunga a terra per non avere -1d6 di penalita' al colpire ed estrae un pugnale.

\subsubsection{Arma lunga a breve distanza} \index{Arma lunga a breve distanza}e' possibile usare un'arma lunga a distanza di 1 metro (mischia) con un -1d6 al Tiro per Colpire.

\subsubsection{Magia in combattimento}\index{Magia in combattimento} l'incantatore che lancia una Essenza mentre e' in combattimento prende un -4 alla Difesa. Se viene colpito deve fare una prova di concentrazione per mantenere l'incantesimo.

\subsubsection{Prova di Concentrazione} \index{Prova di Concentrazione}quando un incantatore vuole usare una Essenza ma e' severamente disturbato o ferito durante il lancio deve effettuare una prova di concentrazione per capire se riesce a lanciare la magia. Vedi capitolo Magia.

\subsubsection{Uscire da una zona minacciata} \index{Uscire da una zona minacciata}Il nemico potrebbe colpirti mentre la attraversi se ha ancora azioni disponibili. Uscire dal combattimento ed entrare in zona di mischia costa un movimento.

\subsubsection{Preparare una arma lunga contro una carica} \index{Preparare una arma lunga contro una carica}e' una Reazione che costa una Azione.

\subsubsection{Prendere la Mira (cecchino)} \index{Prendere la Mira (cecchino)}per ogni round in cui prendi solo la mira, 2 Azioni, guadagni un +1 al Tiro per Colpire, fino ad un massimo di +3 alla fine del terzo round, quando puoi scagliare la freccia (o dardo o pugnale..), oppure all'inizio del round 4 tirando l'iniziativa con un bonus di +3.

\subsubsection{Alzarsi da prono}\index{Alzarsi da prono} costa due Azione. Prendi un -4 alla Difesa ed un -4 Iniziativa. Una prova di Acrobatica con difficolta' 15 ti permette di dimezzare questi malus e costa una sola Azione alzarsi. Con difficolta' 20 annulli i malus e costa 1 Azione alzarsi.
	
\subsubsection{Combattimento con due armi}\index{Combattimento con due armi} ll combattimento con due armi e' possibile solo se l'arma secondaria e' leggera.

Non serve usare Azioni per attaccare con l'arma secondaria, l'arma secondaria deve essere leggera e non applichi il danno dato dalla Potenza.

I tiri per colpire di entrambe le armi hanno un -4.

Con l'arma secondaria se non si hanno abilita' specifiche si fa solo 1 attacco.

\subsubsection{Usare un'arma da lancio sotto minaccia} \index{Usare un'arma da lancio sotto minaccia}usare un'arma da lancio come arco, balestra o pugnale (che si vuole lanciare) mentre si combatte in mischia impone la negazione del bonus della Agilita' alla Difesa ed il Tiro per Colpire ha un -4.

\subsubsection{Usare un'arma da lancio mirando ad un avversario impegnato
in combattimento}\index{Usare un'arma da lancio mirando ad un avversario impegnato in combattimento} non e' facile prendere la mira corretta, hai un -1d6 al Tiro per Colpire. Il bonus si annulla se c'e' una differenza di 2 o piu' taglie tra avversario e compagno.

\subsubsection{Usare un'arma con due mani} \index{Usare un'arma con due mani}un'arma non leggera se usata a due mani permette di applicare una volta e mezza il danno dovuto dalla Potenza.

\subsubsection{Difesa totale} \index{Difesa totale}costa 2 Azioni. Non fai nessun attacco, puoi fare solo una Azione e guadagni un +8 in Difesa. Non causi Attacchi di Opportunita' se attraversi la zona di mischia di un avversario.

\subsubsection{Disingaggiare} \index{Disingaggiare}costa 2 Azioni e ti sposti di 3 metri. Un'avversario ti puo' colpire se ha una iniziativa migliore della tua o ti insegue (ed e' veloce quanto te). Non causi Attacchi di Opportunita' se attraversi la zona di mischia di un avversario

\subsubsection{Mettersi sulla difensiva} \index{Mettersi sulla difensiva}prendi un bonus di +4 alla Difesa, il tuo Tiro per Colpire ha una penalita' di -1d6

\subsubsection{Disarmare*}\index{Disarmare} fai una prova contrapposta Competenza Armi + Agilita' (chi disarma) contro Competenza Armi + Potenza (chi viene disarmato)

Un'arma a due mani concede un bonus di +4, un'arma leggera un malus di -2 a chi deve essere disarmato. Se si fallisce di 5 o piu' hai disarmato te stesso e non l'avversario. Costa 1 Azione.

\subsubsection{Finta*} \index{Finta}fai una prova contrapposta di Competenza Armi + Faccia Tosta (chi fa la finta) contro Competenza Armi + Consapevolezza (chi subisce la finta). Se la prova riesce l'avversario perde il bonus della Agilita' alla Difesa fino alla fine del round successivo.

Se fallisci di 5 o piu' perdi tu il round prossimo il bonus di Agilita'.
Costa 1 Azione.

\subsubsection{Spingere un avversario*} \index{Spingere un avversario}e' una prova contrapposta di Potenza.

Se vinci spingi l'avversario fino a 0.5 metri nella direzioni che vuoi per successo nella prova, altrimenti l'avversario ti spinge nella direzione che vuole fino a 0.5 metri per successo ottenuto (se vinci la prova di 7 sposti l'avversario fino a 3.5 metri).
Costa due azioni.

\subsubsection{Afferrare un avversario*}\index{Afferrare un avversario} e' una prova contrapposta di Potenza. Chi ha una taglia maggiore guadagna un bonus di +2 per taglia di differenza.

Ogni round la prova va rifatta, se chi e' afferrato vince si libera dalla presa. Costa 2 Azioni fare e mantenere e liberarsi dalla presa. Si considera che chi afferra (o e' afferrato) abbia almeno una mano occupata nell'afferrare.

I due contendenti perdono il bonus di Agilita' alla Difesa. Se uno dei due vuole muoversi si esegui una prova di Spingere l'avversario.

\subsubsection{Fare cadere un avversario} \index{Fare cadere un avversari}e' una prova contrapposta di Potenza o Agilita', ogni contendente sceglie quella che preferisce. 

Ognuno fa un Tiro Salvezza su Tempra o Riflessi e si confrontano i risultati, se la prova fallisce di 5 o piu' e' chi voleva fare cadere che cade.

Per ogni gamba oltre la seconda che ha l'avversario questo ha un bonus di +2 alla prova. Costa 2 Azioni. L'avversario se fallisce la prova diventa prono.

\subsubsection{Modificare le proprie dimensioni*}\index{Modificare le proprie dimensioni}

Nel caso il personaggio modifichi le sue dimensioni la sua Difesa cambia di conseguenza

\bigskip

\begin{tabular}[c]{@{}ll@{}}
\toprule 
Nuova Taglia & Modificatore alla Difesa\tabularnewline
Piccolissima & +8\tabularnewline
Minuta & +4\tabularnewline
Minuscola & +2\tabularnewline
Piccola & +1\tabularnewline
Media & +0\tabularnewline
Grande & -1\tabularnewline
Enorme & -2\tabularnewline
Mastodontica & -4\tabularnewline
Colossale & -8\tabularnewline
\bottomrule
\end{tabular}

\bigskip

\textbf{{*} Le azioni marcate con {*} sono opzionali e concesse a
discrezione del Narratore}.

\pagebreak

\section{Nascondigli e coperture}\index{Nascondigli}\index{Coperture}

\label{nascondigli-e-coperture}
Un obiettivo puo' essere non sempre pienamente visibile ed anzi nascondersi
attivamente.

L'obiettivo in questo caso si dice che abbia copertura. Questa copertura puo' essere leggera, media e completa.

Se l'obiettivo ha piu' della meta' (ma non totale) della superficie ``esposta' allora la copertura si definisce \textbf{leggera}, ovvero ha +2 alla Difesa.

Se l'obiettivo ha meno della meta' (ma non completamente) della superficie ``esposta' allora la copertura si definisce \textbf{media}, ovvero ha +4 alla Difesa.

Se l'obiettivo si sa dove e' ma si nasconde completamente affacciandosi solo per controllare i personaggi o tirare una freccia ogni tanto, dietro ad un muro, finestra, porta, una creatura piu' grande di lui (almeno 2 taglie).. allora la copertura si definisce \textbf{completa}, ovvero ha +8 alla Difesa.

Meta' del Bonus di copertura si applica anche ai Tiri Salvezza contro
Essenze a raggio (es. Fuoco Palle che esplodono intorno..).

Se un avversario e' invisibile allora si seguono le regole della Invisibilita'.

\subsection{Invisibilita'}\index{Invisibilita'}

\label{invisibilita}

La capacita' di muoversi praticamente senza essere visti non e' a prova di fallimento. Anche se non possono essere viste, le creature invisibili possono essere percepite comunque da altri sensi, come l'olfatto, l'udito o il tatto.

L'invisibilita' rende una creatura non individuabile tramite la vista ma non rende di per sé una creatura immune ai Tiri Critici o Esplosioni del Danno.

Una creatura accecata, o che combatte contro una creatura invisibile, puo' effettuare una prova di Consapevolezza a difficolta' 20 (oppure 10 + prova di Consapevolezza dell'avversario se questo di nasconde attivamente) per individuare la creatura purche' questa sia entro raggio di 3 metri dal personaggio.

L'osservatore ha la sensazione che ``ci sia qualcosa' ma non puo' vederlo o prenderlo di mira in modo accurato con un attacco.

e' praticamente impossibile (DC 30) determinare la posizione esatta di una creatura invisibile con una prova di Consapevolezza.

Una creatura invisibile oggetto di un attacco specifico nel ``suo quadretto'' se non prima individuata ha un vantaggio alla Difesa come se avesse \textbf{Copertura completa} (+8 Difesa).

Una creatura invisibile ha un bonus di +1d6 al colpire contro creature che non lo vedono, e anche se la sua posizione viene determinata (prova di Consapevolezza DC 30 riuscita) ha comunque un bonus alla Difesa di +4.

Ci sono molti modificatori che possono essere applicati a questa DC se la creatura invisibile si sta muovendo o sta compiendo un'attivita' rumorosa.

\bigskip

\textbf{Tabella Modificatori Consapevolezza per Rilevare Creature Invisibili}

\bigskip

\begin{tabular}[c]{@{}ll@{}}
\toprule 
La Creatura Invisibile sta... & Consapevolezza\tabularnewline
Muovendosi a velocita' dimezzata & -5\tabularnewline
Muovendosi a piena velocita' & -10\tabularnewline
Correndo o caricando & -20\tabularnewline
Usando Muoversi Silenziosamente & Prova di Furtivita' (Consapevolezza) +10\tabularnewline
Ferma & +20\tabularnewline
A qualche metro di distanza (3 metri) & +1, +2 per ogni 3 metri oltre\tabularnewline
Dietro un ostacolo (porta) & +5\tabularnewline
Dietro un ostacolo (parete di pietra) & +20\tabularnewline
\bottomrule
\end{tabular}

\bigskip

Una creatura particolarmente grossa e lenta potrebbe godere di una probabilita' inferiore di essere mancata.

Se un personaggio invisibile raccoglie un oggetto visibile, l'oggetto resta visibile. Una creatura invisibile puo' raccogliere un piccolo oggetto visibile e nasconderselo addosso (mettendolo in una tasca o sotto il mantello, chiudendolo nel pugno) e renderlo effettivamente invisibile.

Uno potrebbe spargere su un oggetto invisibile della farina per tenere traccia almeno della sua posizione (finché la farina non cade del tutto o viene soffiata via).

Le creature invisibili lasciano impronte. Le loro tracce possono essere seguite senza problemi. Impronte su sabbia, fango o altre superfici soffici possono dare ai nemici indicazioni sulla posizione della creatura invisibile.

Una creatura invisibile nell'acqua muove il liquido, rivelando la propria posizione. La creatura invisibile rimane comunque difficile da vedere e gode dei benefici di una copertura media (+4 alla Difesa). Una torcia accesa invisibile emana comunque luce (cosi' come un oggetto invisibile soggetto ad una Essenza di Illusione di luce o un altra Essenza simile).

Le creature invisibili non possono utilizzare gli attacchi con lo sguardo. L'invisibilita' non influisce sulla Essenza di Rivelazione.

\pagebreak

\section{Lista Armi per Tipologia Omogenea}\index{Lista Armi}\index{Tipologia Omogenea}

\label{lista-armi-per-tipologia-omogenea}
\begin{quotebox}La forza non risiede in una Spada, ma nelle braccia di un valoroso. (The Legend of Zelda: Twilight Princess)
\end{quotebox}

Ogni qual volta si assegna un punto a Competenza Armi si puo' decidere se continuare a perfezionarsi in una Lista di Armi o prenderne una nuova.

Nella scheda segnatevi quando assegnate un punto in Competenza Armi come questo viene usato. Non e' possibile ``riassegnare'' i punti, una volta deciso la scelta non si cambia.

Ricordo che usare un arma senza l'adeguata competenza impone un -2d6 al Tiro per Colpire.

\subsection{Armi Leggere}\index{Armi Leggere}Pugnale, Spada Corta, Mazza leggera, Martello Leggero, Stocco, Scimitarra

Puoi usare Agilita' al posto di Potenza con queste armi per il Tiro
per Colpire.

\begin{itemize}
	\item +6 CA: aumenti di un grado il dado di danno dell'arma (d4 - d6 - d8 - d10 - 2d6 - 2d8 - 2d10 - 3d6..)
	
	\item +12 CA: +2 Tiro per Colpire
	
	\item +18 CA: la tua arma acquista EDX anche con 6 di danno massimo 
\end{itemize}

\subsection{Asce}\index{Asce} Ascia ad una mano, Ascia da battaglia, Ascia Martello, Urgrosh, Grande Ascia Doppia

\begin{itemize}

\item +6 CA: La furia dei tuoi attacchi e' tale che guadagni un +2 al danno

\item +12 CA: Le ferite che provochi sono cosi profonde che provochi sanguinamento. Il primo attacco del round se andato a segna causa 1d4 di danno extra da sanguinamento. Il danno non si applica il round successivo.

\item +18 CA: Le asce nelle tue mani abbattono nemici cosi come abbattono degli arbusti. Puoi sacrificare 5 al Tiro per Colpire ma aumentare il bonus alla forza a x2 per le armi a 1 mano e a x3 per quelle a due. Il bonus non si cumula con usare armi a due mani, e' alternativo.

\end{itemize}

\subsection{Rompi Cranio} \index{Rompi Cranio}Randello, Mazza Leggera, Mazza Pesante, Morningstar,
Martello Leggero, Flagello, Martello da guerra, Grosso randello, Flagello
Pesante

\begin{itemize}
	\item +6 CA: Sei diventato cosi abile che puoi controllare la forza dei tuo colpi,puoi fare danno non letale senza malus al colpire (altrimenti -4). Puoi scegliere di ridurre di 4 il Tiro per Colpire per aumentare il danno di 4 
	
	\item +12 CA: I tuo colpi frastornano il nemico. Se anche solo un tuo attacco va a segno l'avversario deve fare un Tiro Salvezza Tempra (DC 10+CA) se fallisce subira' -2 Iniziativa ed -2 Difesa per la prossima azione 
	
	\item +18 CA: aumenti di un grado il dado di danno dell'arma (d4 - d6 - d8 - d10 - 2d6 -2d8 - 2d10)
\end{itemize}

\subsection{Archi} \index{Archi}Fionda, Arco Lungo, Arco Corto, Arco Lungo Composito, Arco Corto Composito

\begin{itemize}

\item +6 CA: aggiungi il valore di Agilita' al danno

\item +12 CA: La tua maestria nell'utilizzo dell'arco in combattimento e' tale che non subisci nessuna penalita' nel lanciare frecce a nemici in mischia o con copertura pari o minore di leggera.

\item +18 CA: scagli una freccia in piu'

\end{itemize}

L'aggiungere il bonus di Agilita' al danno non si somma se si applica un danno da Potenza ( in caso di archi compositi), devi scegliere che bonus applicare

\subsection{Balestre}\index{Balestre}Balestra leggera, Balestra pesante, Balestra ad una mano, Balestra leggera a ripetizione, Balestra pesante a ripetizione

\begin{itemize}

\item +6 CA: guadagni l'abilita' Ricarica rapida

\item +12 CA: La tua maestria nell'utilizzo delle balestre in combattimento e' tale che non subisci nessuna penalita' nel lanciare frecce a nemici in mischia o con copertura pari o minore di leggera

\item +18 CA: La tua Potenza della tua arma unita alla tua esperienza e precisione sono armi mortali. Puoi decidere di prendere la mira su un nemico per un round, se quel nemico e' ancora colpibile nel turno successivo scagli la freccia che se colpisce il bersaglio fara' il cinque volte il danno.

\end{itemize}

\subsection{Armi doppie} \index{Armi doppie}Bastone, Grande Ascia Doppia, Flagello Doppio, Spada a due lame, Urgrosh

\begin{itemize}
\item +6 CA: La tua competenza nell'uso di queste armi ti rende estremamente versatile dandoti la possibilita' a inizio del tuo turno di scegliere se essere difensivo o offensivo aumentando di 2 il Tiro per Colpire o la Difesa. Non costa azioni.

\item +12 CA: La tua tecnica e' imprevedibile per l'avversario puoi scegliere se avere un +1d6 di danno con tutti i tuoi attacchi o eseguire un attacco extra questo turno

\item +18 CA: La tua maestria e' tale che l'avversario vede 3 lame. Per ogni attacco con la mano primaria puoi eseguire due attacchi extra con meta' Potenza senza modificatori magici.
\end{itemize}

\subsection{Armi da carceriere} \index{Armi da carceriere}Falcetto, Flagello, Flagello Pesante, Falcione in asta, Brandistocco, Falce, Flagello Doppio, Frusta

\begin{itemize}
\item +6 CA: Nati come attrezzi da contadino e strumenti di lavoro nelle tue mani dispensano morte e sofferenza, guadagni +2 danno

\item +12 CA: La tua capacita' di infliggere dolore con le tue armi e' terrificante. Dopo una tua azione di attacco se hai fatto un critico, tutti i nemici che ti possono vedere devono superare un Tiro Salvezza Arbitrio DC pari a 10+CA se falliscono subiscono -3 ai tiri per colpire per quel round.

\item +18 CA: i tuoi colpi sono letali. L'EDX dell'arma, se presente, diminuisce di 1 (ad esempio su un flagello che e' 8, su 1d8 diventa 7)
\end{itemize}

\subsection{Palle rotanti} Flagello, Flagello Pesante, Catena chiodata

\begin{itemize}
\item +6 CA: ignori il bonus di protezione dato dallo scudo.

\item +12 CA: L'impatto dei tuo colpi e' tale da frantumare le ossa. L'avversario deve fare un tiro salvezza su Tempra (DC 10+CA) su fallimento subisce -2 Agilita' per 1 minuto. Una creatura non puo' essere influenzata da questo effetto piu' di due volte

\item +18 CA: La velocita' e forza dei tuo colpi e' tale da distruggere le difese del nemico, l'avversario prende -5 alla Difesa contro di te 
\end{itemize}

\subsection{Armi aggraziate}\index{Armi aggraziate} Stocco, Scimitarra, Falcione

puoi decidere di usare l'Agilita' per determinare il bonus al colpire
ed al danno.

\begin{itemize}
	\item +6 CA: puoi eseguire un critico, anche su creature normalmente immuni
ai critici

\item +12 CA: il EDX dell'arma, se presente, diminuisce di 1

\item +18 CA: per ogni -1 al danno che prendi la tua iniziativa aumenta di 2, fino ad un massimo di 6
\end{itemize}

\subsection{Armi della morte}\index{Armi della morte} Picca Leggera, Picca Pesante, Falce, Falcetto

\begin{itemize}
	\item +6 CA: puoi eseguire un Colpo di Grazie con il costo di 1 Azione

\item +12 CA: aumenti di un grado il dado di danno (d4 - d6 - d8 - d10 - 2d6 - 2d8 - 2d10 - 3d6...)

\item +18 CA: aumenti di un grado il dado di danno (d6 - d8 - d10 - 2d6 - 2d8 - 2d10 - 3d6...)

\end{itemize}

\subsection{Armi da stordimento}\index{Armi da stordimento} Pugno nudo, Manganello, Guanto chiodato

+\begin{itemize}
\item +6 CA: un avversario inconsapevole se colpito con queste armi (durante il round di sorpresa) deve dare un Tiro Salvezza Tempra con DC pari al danno (del round causato da Armi da Stordimento) o rimanere stordito per 1d6 round

\item +12 CA: raddoppi il tuo bonus di danno dato dalla Potenza

\item +18 CA: la tua arma da stordimento fa 1d6 di danno in piu'
\end{itemize}

\subsection{Lance} \index{Lance}Alabarda, Tridente, Urgrosh, Lancia da fante, Naginata,Falcione in asta, Lancia, Brandistocco,Tridente

\begin{itemize}
\item +6 CA: puoi usarla anche contro avversari a distanza di mischia senza malus, perdi il bonus di controcarica mentre sei in mischia

\item +12 CA: usata contro una carica fai il quadruplo del danno

\item +18 CA: Se non sei in mischia con un avversario puoi usare la tecnica della Colpo Perforante (questa azione richiede tutto 3 Azioni) puoi caricare un avversario tra 6 e 18 metri: puoi sacrificare 1 punto CA e guadagnare 5 al danno (massimo 10 CA/50 danno) poi esegui un attacco solo col arma. Questo colpo ti porta in mischia con l'avversario e ti lascia scoperto per quel turno, hai un -4 alla Difesa

\end{itemize}

\subsection{Armi letali} Pugnale, Machete\index{Armi letali}

\begin{itemize}

\item +6 CA: contro avversari sorpresi aggiungi al danno il tuo CA

\textbf +12 CA: la tua arma fa piu' danno. Guadagni una categoria di danno (d4 - d6 - d8..)

\textbf +18 CA: guadagni EDX. Lo si applica solo facendo il danno massimo con il dado, se l'arma ha gia' un edx (perche' con il bonus precedente e' arrivata ad 1d8 di danno)
questo diminuisce di 1

\end{itemize}

\subsection{Aste} \index{Aste}Giavellotto, Lancia corta da fante, Lancia da fante, Tridente

\begin{itemize}

\item +6 CA: in caso di critico puoi lasciare l'arma nel corpo dell'avversario,
penalizzandolo con un -2 Agilita'. La lancia quando rimossa fa il suo dado di danno (senza
Potenza o bonus magici)

\item +12 CA: raddoppi la gittata

\item +18 CA: guadagni un +2 all'iniziativa
\end{itemize}

\subsection{Spade}\index{Spade} Spada Corta, Spada Lunga, Spadone a due mani, Spada bastarda, Spada a due lame, Katana

\begin{itemize}

\item +6 CA: La tua maestria nella tecnica della spada ti conferisce +1
a danno, Tiro per Colpire e Difesa

\item +12 CA: La tua abilita' con la spada ora ti permette di disarmare l'avversario. Questa abilita' consuma una azione e l'avversario deve superare un Tiro Salvezza su Agilita' (DC 10+CA) per evitare di essere disarmato. Valgono le considerazioni di Disarmare

\item +18 CA: Hai raggiunto l'apice della maestria con la spada i tuo colpi sono precisi e difficili da prevedere ottieni +5 a danno, Tiro per Colpire e Difesa, oltretutto hai rimosso ogni movimento superfluo dalla tua tecnica
\end{itemize}

\subsection{Scudi}\index{Scudi} Leggeri, Medi, Pesanti

sei un maestro nell'uso degli scudi, anche come arma.

Puoi usare lo scudo come arma, uno scudo piccolo fa 1d4 di danno (B/S), uno scudo medio fa 1d6 di danno (B/S), uno scudo pesante fa 1d8 di danno (B/S).

Attaccare con lo scudo e con l'arma e' considerato attacco a due mani.

\begin{itemize}
	\item +1 CA: sei competente in tutte le tipologie di scudo

\item +2 CA: il bonus di Difesa aumenta di 1 e ogni 4 volte che prendi la competenza

\item +3 CA: la penalita' CM diminuisce di 1 e di 1 ogni 4 volte che prendi la competenza

\item +4 CA: la penalita' CA diminuisce di 1 e di 1 ogni 4 volte che prendi la competenza

\item +5 CA: aumenta di 1 la categoria di danno dello scudo (1d4 - 1d6 - 1d8 - 1d10 - 2d6 - 2d8) ed ogni 6 punti ulteriori (5,11,17..)

\item +12 CA: Abituato a prevedere e parare gli attacchi nemici ora riesci a difendere anche gli alleati adiacenti a te, ogni alleato adiacente a te ha un +2 Difesa. Se desideri puoi subire il danno di un attacco diretto ad un alleato entro 1 metro (al tuo fianco). Usare questa abiltia' e' una Reazione che non costa Azioni.

\end{itemize}

\subsection{Bloccanti} Bolas, Net\index{Bloccanti}

\begin{itemize}
	\item +6 CA: una creature avvolta dalla tua rete o bolas e' intralciato
e non puo' muoversi

\end{itemize}

\subsection{Armi da tiro} Pugnale, Lancia corta da fante, Lancia da fante, Martello Leggero, Ascia ad una mano, Tridente\index{Armi da tiro}

Hai accesso a due abilita':

Tiro Devastante: puoi lanciare una delle tue armi con tale forza da triplicarne il danno (arma e Potenza) ma la precisione ne risente -8 al colpire. Costa 2 Azioni.

Ventaglio di lame: Puoi scagliare le tue armi per un massimo di 6 alla volta lancia un solo Tiro per Colpire con una penalita' di -6. Le tue armi colpiscono a caso (scelti dal Narratore) chiunque sia intorno a te nel raggio della tua portata di tiro. Ricordati di raccogliere tutte le armi dopo averle lanciate. Costa 3 Azioni.

\begin{itemize}
\item +6 CA: Se diventato estremamente preciso nel lancio della tua arma hai un +2 al colpire e un +1 ai danni

\item +12 CA: La tua abilita' ti permette di non avere tempi morti dopo il lancio di un arma puoi istantaneamente estrarne un altra senza consumare azioni.

\item +18 CA: raddoppi la Gittata dell'arma
\end{itemize}

\subsection{Pugno nudo} pugni/calci\index{Pugno nudo}

\begin{itemize}
	\item +1 CA:tuoi pugni fanno danno letale (1d4)
\end{itemize}

Ogni volta che prendi questa competenza, e CA +2 rispetto alla volta precedente, il danno aumenta seguendo questa progressione: 1d6 (CA 3), 1d8 (CA 5), 2d6 (CA 7), 2d8 (CA 9), 2d10 (CA 11), 3d6 (CA 13), 3d8 (CA 16), 3d10 (CA 19), 4d6 (CA 21)..

\subsection{Armi Semplici} Pugnale, Mazza Leggera, Randello, Morningstar, Lancia corta da fante, Bastone, Balestra (Leggera), Giavellotto\index{Armi Semplici}

Questa suddivisione e' sceglibile anche da chi ha Competenza Armi a zero 

\pagebreak

\section{Abilita'}\index{Abilita'}

\label{abilita}
\begin{quotebox}
Il martirio e' l'unica maniera per un uomo di diventare famoso se non ha abilita' (George Bernard Shaw, The Devil's Disciple)
\end{quotebox}

Le abilita' sono capacita' peculiari, frutto di allenamento o doti particolari. Le abilita' si ottengono piu' raramente ed hanno sempre un effetto pratico.

\textbf{Al primo livello si prendono due Abilita'.} Ogni 2 livelli (e quindi al 3, 5, 7, 9... ) si prende un'altra Abilita' che puo ` essere la stessa gia' presa oppure una nuova abilita' appresa durante le avventure.

E' possibile che siano indicati dei Prerequisiti sotto il nume dell'Abilita', in questo caso vanno rispettati per prendere l'Abilita' in questione.

Eventuali prerequisiti successivi vengono indicati volta per volta.

Non prendete le Abilita' in base al potere, forza, combinazione che hanno ma perche' il linea con la storia del personaggio.
Scegliere un accozzaglia di abilita' solo perche' forti non rende un personaggio forte ma sbilanciato. 

\medskip
Le Abilita' devono essere prese in base al percorso evolutivo del personaggio, in base a quanto vissuto ed appreso durante le avventure.
\medskip

E' possibile cambiare una Abilita' scelta, rispettando comunque i requisiti, al 5, 7, 11 e 17 livello.

\subsection{Animalia}\index{Animalia}

Prerequisiti: Essenza Trasformazione, Competenza Magica 2.

Si acquisisce la capacita' di trasformarsi in un animale. Costo 2 Azioni

La \textbf{prima volta} che si prende questa Abilita' ci si puo' trasformare in animali non magici di taglia piccola o media, per 10 minuti per punteggio in Competenza Magica. Ci si puo' trasformare 1 sola volta al giorno.

La \textbf{seconda volta}, Competenza Magica 6, che si prende questa abilita' si acquisisce la capacita' di trasformarsi in animali minuscoli o grandi e ci si puo' trasformare due volte in piu' al giorno. 

La \textbf{terza volta} che si prende questa Abilita', Competenza Magica 10, si acquisisce la capacita' di trasformarsi in animali di taglia minuta o enorme e ci si puo' trasformare altre tre volte al giorno. Il tempo minimo di trasformazione giornaliera e' di 16 ore

La \textbf{quarta volta}, Competenza Magica 16, si acquisisce di trasformarsi in animali di taglia piccolissima o mastodontica ed anche magici (sempre nel limite della taglia). Il tempo minimo di trasformazione e' di 24 ore al giorno, e puo' trasformarsi quante volte vuole al giorno.

\subsection{Armatura del Devoto}\index{Armatura del Devoto}

Requisito: Tratti in comune 1 (somma dei tratti in comune con il Patrono)

Il costante allenamento con la tua armatura ti permette di indossare armature leggere senza rischio di sbagliare il lancio di Essenze.

La \textbf{seconda volta} che si prende questa Abilita', Tratti in comune 6, puoi lanciare Essenze senza rischio di fallire con armature medie.

La \textbf{terza volta} che si prende l'abilita, Tratti in comune con il Patrono 10, puoi portare Armature Pesanti senza penalita'.

\subsection{Arciere a cavallo}\index{Arciere a cavallo}

Il malus di tirare frecce da cavallo diminuisce di 2 ogni volta che prendi questa Abilita'.

Le penalita' standard sono -4 e -6 a seconda che si trotti (movimento x2) o galoppi (movimento x3)

\subsection{Arma Focalizzata}\index{Arma Focalizzata}

ottieni un +1 a Iniziativa e Tiro per Colpire quando usi un arma specifica
di cui hai competenza.

\subsection{Attacco turbinante}\index{Attacco turbinante}

Requisito: Competenza Armi 12

Usando 3 Azioni puoi eseguire un singolo attacco (con un malus di
5 al Tiro per Colpire) contro tutti gli avversari in mischia attorno
a te.

\subsection{Cecchino}\index{Cecchino}

Requisito: Competenza Armi 3

La penalita' per i tiri oltre il primo incremento di range diminuisce
di 1d6.

La \textbf{seconda volta} che prendi questa Abilita', la penalita'
per i tiri fino al secondo incremento di range diminuisce di 2d6.

La \textbf{terza volta} che prendi questa abilita' sei in grado di
estendere ancora di piu' il tuo tiro e portarlo ad un quarto incremento
con un -1d6 di penalita' al colpire. Tirare nei primi due incrementi
non ha penalita'.

\subsection{Colpi poderosi}\index{Colpi poderosi}

Requisito: Competenza Armi 1

Il tuo stile enfatizza colpi poderosi.

Guadagni un +1 al danno con una lista di arma.

\subsection{Colpo furtivo (Attacco alle spalle)}\index{Attacco alle spalle}\index{Colpo furtivo}

Requisito: Competenza Armi 3

Ogni qual volta l'avversario viene attaccato in mischia di sorpresa,
per ogni volta che hai preso questa Abilita' (prerequisito Competenza
Armi +2 rispetto alla volta precedente) il danno aumenta di +2d6.

\subsection{Colpo Indebolente}\index{Colpo Indebolente}

Requisito: Colpo furtivo 6d6, Competenza Armi 12

Colpo Indebolente e' una forma avanzata di colpo furtivo. Ogni colpo
Indebolente abbassa o Potenza o Agilita' (scelta giocatore) di 1 punto.

All'avversario e' concesso un Tiro Salvezza Riflessi con DC 10 + 1/2CA + Potenza/Agilita' (a seconda dell'indebolimento scelto)

\subsection{Colpo Mortale}\index{Colpo Mortale}

Requisito: Competenza Armi 5

Esegui il Tiro per Colpire come sempre, ma ne dimezzi il risultato ottenuto. Se colpisci il danno causato dall'attacco viene raddoppiato ad esclusione dei bonus magici. Colpo mortale e' una Azione Immediata da dichiararsi prima che il Narratore ti informi se si e' colpito o meno.

\subsection{Colpo Paralizzante}\index{Colpo Paralizzante}

Requisito: Colpo Indebolente, Colpo furtivo 8d6, Competenza Armi 18

Il target dopo che e' stato studiato per 10 round con il prossimo tuo colpo andato a segno in mischia, entro 10 round dallo studio, deve effettuare un Tiro Salvezza Tempra con DC pari al doppio del danno inflitto (con un bonus di +1 per ogni round non studiato dei 10 round) o rimanere paralizzato per 3d6 round

\subsection{Combattere alla Cieca}\index{Combattere alla Cieca}

E' la capacita' di attaccare gli avversari che non sono chiaramente percepibili.

Requisito: Consapevolezza 2

Un avversario con copertura leggera non ottiene bonus alla Difesa, con copertura media ha un +2 alla Difesa, con copertura totale ha un +4 alla Difesa.

Un attaccante invisibile, non ottiene alcun vantaggio al colpire il personaggio in mischia. I bonus dell'attaccante Invisibile si applicano lo stesso solo per gli attacchi da distanza.

Non c'e' bisogno di effettuare prove di Acrobatica per muoversi a piena velocita' mentre si e' Accecati.

La \textbf{seconda volta} che prendi Abilita' (Consapevolezza a 10), gli attacchi in mischia riducono di ulteriore due il bonus alla Difesa.

\subsection{Combattimento con due armi}\index{Combattimento con due armi}\index{due armi}

Requisito: Agilita' 2, Potenza 2 , Competenza Armi 2

La \textbf{prima volta} che prendi questa abilita' puoi eseguire un attacco con l'arma secondaria, che deve essere leggera, e non applichi il danno dato dalla Potenza. Entrambi i tiri per colpire hanno un-2.

Requisito Agilita' 3, Competenza Armi 12

La \textbf{seconda volta} che prendi questa abilita' puoi fare con l'arma secondaria leggera fino a 2 attacchi e non applichi il danno dato dalla Potenza. Non hai penalita' al Tiro per Colpire.

\subsection{Creare Oggetti Magici}\index{Creare Oggetti Magici}

Requisito: Competenza Magica 6

tramite questa Abilita' l'incantatore e' in grado di infondere una Essenza fino al Livello di Potere 13 in un oggetto magico. Se infonde piu' Essenze deve possederle e la somma non puo' essere superiore a 26.

\subsection{Creare Oggetti Magici Superiori}\index{Creare Oggetti Magici Superiori}

Requisito: Creare Oggetti Magici, Competenza Magica 12

tramite questa Abilita' l'incantatore e' in grado di infondere un'Essenza fino al Livello di Potere 21 in un oggetto magico. Se infonde piu' Essenze deve possederle e la somma non puo' essere superiore a 42.

\subsection{Creare Oggetti Magici Meravigliosi}\index{Creare Oggetti Magici Meravigliosi}

Requisito: Creare Oggetti Magici Superiori, Competenza Magica 16

tramite questa Abilita' l'incantatore e' in grado di infondere un'Essenza fino al Livello di Potere 29 in un oggetto magico. Se infonde piu' Essenze deve possederle e la somma non puo' essere superiore a 56

\subsection{Decifrare scritti magici}\index{Decifrare scritti magici}

Requisito: Competenza Magica 1

Saper leggere le scritte magiche. Puoi leggere una pergamena contenente la descrizione di una Essenza con un bonus di +5

\subsection{Difendere Cavalcatura}\index{Difendere Cavalcatura}

Ogni qual volta la cavalcatura viene colpita, puoi effettuare una prova di cavalcare per negare il colpo. La tua prova di Sopravvivenza deve essere maggiore del Tiro per colpire dell'avversario

L'Abilita' e' utilizzabile solo una volta per round, per un solo attacco, costa 1 Azione.

\subsection{Distillare pozioni}\index{Distillare pozioni}

Requisito: Competenza Magica 1

Competenza nel distillare pozioni.

Acquisti un bonus di +4 su Conoscenze Erboristeria (Cultura)

\subsection{Doppia porzione}\index{Doppia porzione}

Requisito: Combattimento con due armi, Competenza Armi 4

il costante allenamento con due armi ti permette di applicare il bonus al danno dovuto alla Potenza in maniera piena anche all'arma secondaria. 

\subsection{Energia Psichica}\index{Energia Psichica}

Requisito: Potenza 1, Volonta' 2, Competenza Armi 1, Competenza Magia
1

Anni di allenamento, meditazione e ritiri a Nanda Parbat sei in grado di raccogliere la tua Energia Chi.

Ogni giorno dopo almeno 6 ore di riposo e 2 ore di meditazione/allenamento riempi il tuo corpo di energia Chi pari a (CA+CM)/2+Volonta'/2

\subsection{Colpo Psichico}\index{Colpo Psichico}

Requisito: Energia Psichica, Agilita' 2

Costo 1 Azione di Attacco, Tiro per Colpire di contatto, e fare 1d6 di danno per punto Chi speso. Non puoi usare un numero di punti Psichici superiore alla Volonta'

\subsection{Raggio Psichico}\index{Raggio Psichico}

Requisito: Colpo Psichico, Volonta' 3, Competenza Armi 5

Puoi effettuare un attacco a distanza entro 9 metri usando l'Energia Psichica. Il colpo, Tiro per Colpire a tocco, causa 1d6 di danno per punto Psichico speso. E' possibile usare uno o piu' punti Psichici per aumentare la distanza ogni volta di 9 metri. Non puoi usare un numero di punti Chi superiore alla Volonta'. Costo 1 Azione di Attacco

\subsection{Essenza Psichica}\index{Essenza Psichica}

Requisito Energia Psichica, Volonta' 3, Competenza Armi 8, Competenza
Magia 3

Puoi usare la tua Energia Psichica per eseguire prestazioni oltre
l'umano.

Sei in grado di simulare le Essenze di Cura e Movimento (Spostare e Concetto solo se stesso) e Alterazione (solo se stesso) usando l'energia Psichica. Costo 2 Azioni

\bigskip

\begin{tabular}[c]{@{}ll@{}}
\toprule 
Punti Chi consumati & Livello di Potere ottenuto\tabularnewline
1 & 11\tabularnewline
2 & 13\tabularnewline
3 & 16\tabularnewline
5 & 19\tabularnewline
8 & 22\tabularnewline
13 & 25\tabularnewline
\bottomrule
\end{tabular}

\bigskip

La durata di Alterazione e Movimento e' 1 round. E' possibile fare durare 10 round in piu per ogni punto Psichico speso.

Non e' possibile usare le Essenze con Competenza Magica per 1 ora dopo aver usato Essenza Psichica.

\subsection{Eludere}\index{Eludere}

La \textbf{prima volta} che si prende\emph{:} Requisito: Agilita'
2, Competenza Armi 4

Il costante allenamento ad evitare essenze e trappole ti permette, nel caso il Tiro Salvezza su Riflessi di annullare il danno qualora il TS riuscito permetta di dimezzare.

La \textbf{seconda volta} che si prende: Requisito Agilita' 3, Competenza
Armi 7

Il costante allenamento ad evitare essenze e trappole ti permette, nel caso il Tiro Salvezza su Riflessi permetta di dimezzare, di annullare totalmente il danno e se fallisci il Tiro Salvezza su Riflessi di dimezzare il danno.

\subsection{Esperto}\index{Esperto}

Prerequisito: Caratteristica collegata almeno a 1

sei un esperto in un argomento. Ogni qual volta prendi questa abilita' guadagni un +1 alle prove su una competenza a tua scelta.

\subsection{Animaletto / Famiglio}\index{Famiglio}

Guadagni un animale naturale. Questo animaletto ha al massimo un numero di dadi vita pari alla tua Volonta'. Puoi insegnare azioni di base al tuo animale e fargli fare dei compiti semplici.

Requisito: Competenza Magica 1, se prendi due volte questa Abilita' guadagni un Famiglio (vedi argomento specifico).

\subsection{Fare Infuriare}\index{Fare Infuriare}

Le tue abilita' dialettiche sono incredibili

Prerequisito: Competenza Armi 2 e Magnetismo 3

Impieghi 2 Azioni ad infamare ed inveire contro un avversario. Il target deve fare un Tiro Salvezza Arbitrio a DC 10+1/2CA + Magnetismo oppure perdere il bonus di Agilita' (al Tiro per Colpire e Difesa) per quel round ed il successivo.

L'avversario puo' non comprendere la tua lingua ma deve avere Intelletto maggiore di -2.

\subsection{Ferocia}\index{Fare Infuriare}

Prerequisito; Competenza Armi 1

La tua rabbia e' tale da sconfiggere, temporaneamente, la morte.

\subsection{Finta Morte}\index{Finta Morte}

Sei in grado di simulare la morte, rallentando il cuore.

Come Reazione sei in grado di cadere a terra (stramazzare!) morto. Solo una prova di Sopravvivenza (Cura) DC 20 puo' rivelare che sei vivo.

L'effetto dura al massimo 2 minuti. La capacita' non e' ripetibile in intervalli inferiori ai 10 minuti.

\subsection{Flagello Danzante}\index{Flagello Danzante}

Requisito: Competenza Armi 1

quando usi il tuo Flagello hai un bonus di +1 alla Difesa e +1 iniziativa.

\subsection{Forgiato nella furia}\index{Forgiato nella furia}

Requisito: Competenza Armi 5

Quando effettui un critico, ovvero hai tirato almeno 2 volte 6, si considera che tu abbia tirato un 6 in piu' per il conteggio totale del numero di critici

\subsection{Freccia chiamata, freccia consegnata}\index{Freccia chiamata, freccia consegnata}

Requisito: Competenza Armi 2

puoi tirare 2 frecce, una volta al giorno, come azione immediata.

\subsection{Furia}\index{Furia}

Requisito: Competenza Armi 1

il tuo stile di combattimento e' rappresentato dalla cieca furia omicida. Aggiungi +1d6 al danno ad ogni Tiro per Colpire che fai ed i tuoi avversare guadagnano +1d6 al colpire verso di te. 

\subsection{Giocoliere}\index{Giocoliere}

Requisito: Agilita' 2

hai un talento naturale per maneggiare gli oggetti.

Qualsiasi prova di Atletica che coinvolga il maneggiare oggetti o l'equilibrio ha un +2 di Bonus.

Puoi lanciare un secondo pugnale come azione immediata all'azione di attacco di lancio pugnale con un -3 al Tiro per Colpire. Un eventuale terzo pugnale lanciato ha il normale malus di -5 (e -10.. e cosi' via).

\subsection{Guerriero dell'Essenza}\index{Guerriero dell'Essenza}

Non segui solo la via della magie e neanche quella della spada, il tuo stile fonde entrambi in un fendente di pura magia

La \textbf{prima volta} che prendi questa abilita', Competenza Armi 2, Competenza Magia 2: sei in grado di scaricare un incantesimo a distanza Tocco con la tua arma. Effettui un Tiro per Colpire normale (CA+Potenza+...) e se colpisci oltre al danno dell'attacco scarichi anche l'Essenza. L'Azione di Attacco e' compresa nelle 2 Azioni usate per lanciare l'Essenza.

La \textbf{seconda volta} che prendi questa abilita' Competenza Armi 6, Competenza Magia 3: sei in grado di attaccare con l'arma e poi riattaccare scaricando l'Essenza con l'arma. Consumi 3 Azioni, fai un attacco, un altro attacco (con penalita' per attacchi multipli) dove scarichi l'Essenza.

\subsection{Ho detto CADI!}\index{Ho detto CADI!}

Requisito: Competenza Armi 4

Se colpisci 3 volte consecutivamente (fino a 3 round distinti ma consecutivi) un avversario questo deve fare una Tiro Salvezza su Tempra DC 10+1/2CA + Potenza o cadere prono.

\subsection{Incantare in Combattimento}\index{Incantare in Combattimento}

Ogni volta che prendi questa Abilita' il bonus alla prova di concentrazione aumenta di 3.

\subsection{Incantatore Prudente}\index{Incantatore Prudente}

La prima volta che prendi questa abilita' il malus alla Difesa mentre lanci una Essenza sotto minaccia diminuisce di 2 (da -4 a -2).

La \textbf{seconda volta} che prendi questa abilita', CA minimo 3, il malus alla Difesa diminuisce di 1.

La \textbf{terza volta} che prendi questa abilita', CA minimo 6, il malus alla Difesa diventa 0.

In ogni caso se si viene colpito bisogna fare la prova di concentrazione.

\subsection{Immunita' ai veleni}\index{Immunita' ai veleni}

Il corpo si abitua ai veleni, il personaggio guadagna un +2 TS.

La seconda volta che prendi l'Abilita' divieni immune ai veleni naturali.
Non riesci piu' ad ubriacarti normalmente.

La terza volta divieni immune ai veleni magici, e non ti puoi piu' ubriacare o subire gli effetti di fumi tossici (ma puoi sempre soffocare).

\subsection{Imposizione delle mani (energia negativa o positiva a seconda dei tratti)}\index{Imposizione delle mani}

Requisito: Competenza Magica 3, Tratti 3

Se i tuoi tratti sono in comune con un Patrono positivo puoi convogliare energia positiva (cura), se sono in comune con un Patrono neutrale o malvagio puoi convogliare energia negativa. Usabile un numero di volte pari al valore di Volonta'. Effetto 1d6+Volonta'

La \textbf{seconda volta}, requisito Competenza Magica 6, che prendi questa Abilita' aumenti di 2d6 l'effetto e di 1 volte l'uso.

La \textbf{terza volta}, requisito Competenza Magica 12, che prendi questa Abilita' aumenti di 3d6 l'effetto e di 1 volte l'uso.

La \textbf{quarta volta}, requisito Competenza Magica 18: che prendi questa Abilita' aumenti di 4d6 l'effetto e di 1 volte l'uso.

L'energia proviene dalle mani (non conta se ci sono guanti) e si applica solo a tocco. Usa 2 Azioni

\subsection{Incanalare energia (energia negativa o positiva a seconda dei tratti)}\index{Incanalare energia}

Competenza Magica 1, Tratti 3

sei in grado di incanalare l'energia magica.

Se i tuoi tratti sono in comune con un Patrono positivo puoi convogliare energia positiva (cura), se sono in comune con un Patrono neutrale o malvagio puoi convogliare energia negativa. Usabile un numero di volte pari punteggio di Volonta'. Effetto 1d6+Volonta'. Influenzi 1 creatura.

La \textbf{seconda volta}, Competenza Magica 6, che prendi questa Abilita' aumenti di 2d6 l'effetto e di 1 volta l'uso. Influenzi fino a 2 creature.

La \textbf{terza volta}, Competenza Magica 12, che prendi questa Abilita' aumenti di 3d6 l'effetto e di 1 volta l'uso. Influenzi fino a 4 creature.

La \textbf{quarta volta}. Competenza Magica 18, che prendi questa Abilita' aumenti di 4d6 l'effetto e di 1 volta l'uso. Influenzi fino a 6 creature.

L'energia proviene dalle mani (non conta se ci sono guanti) ed influenza una o piu' creature entro un metro da te. Usa 2 azioni.

\subsection{Incanalare energia a distanza}\index{Incanalare energia a distanza}

Requisito: Incanalare energia

puoi lanciare l'energia fino a 9 metri, influenza un raggio di 3 metri. Usa 2 azioni.

La \textbf{seconda volta} che prendi questa Abilita' l'energia arriva fino a 18 metri.

La \textbf{terza volta} che prendi questa Abilita' il l'energia arriva fino a 50 metri.

\subsection{Incanalare energia concentrata}\index{Incanalare energia concentrata}

Requisito: Incanalare energia

puoi lanciare l'energia fino a distanza 18 metri. Singolo obiettivo. Usa 2 azioni.

Ogni volta che prendi questa competenza aggiungi un obiettivo entro i 18 metri sul quale dividere a piacimento i dadi disponibili dell'incanalare energia.

L'abilita' non e' cumulabile con ``Incanalare energia a distanza'.

\subsection{Iniziativa migliorata}\index{Iniziativa migliorata}

Aumenti l'iniziativa di +1. L'Abilita' puo' essere presa piu' volte ed il bonus si cumula.

\subsection{Kensai}\index{Kensai}

per ogni -5 al Tiro per Colpire guadagni un +10 all'Iniziativa. Il bonus deve essere usato entro la fine del round successivo. La dichiarazione va eseguita ogni round che si intende usare al momento del controllo delle iniziative.

\subsection{La mia morte la tua morte}\index{La mia morte la tua morte}

Per ogni singolo avversario di combattimento puoi fare che il primo colpo a segno dello scontro causi un danno aggiuntivo pari al doppio di Competenza Armi. L'avversario guadagna un bonus al Tiro per Colpire ed al danno pari al valore della tua Competenza Armi e attacca prima di te quando dichiari di usare questa Abilita'.

\subsection{La mia Testa e' piu' Dura}\index{La mia Testa e' piu' Dura}

Requisiti: Competenza Armi 1

la tua Arma Rompi Cranio fa +2 danni

\subsection{Lo scudo e' mio amico}\index{Lo scudo e' mio amico}

Requisiti: Competenza Armi 1

La penalita' alla Competenza Magica diminuisce di 1

La \textbf{seconda volta} che si prende questa Abilita', Competenza Armi 3, la penalita' al CA diminusce di 1, la penalita' CM diminuisce ulteriormente 2

La \textbf{terza volta} che si prende questa Abilita', Competenza Armi 5, la penalita' al CA diminuisce di 3, la penalita' CM diminuisce di ulteriormente 2.

\subsection{Magie efficaci}\index{Magie efficaci}

Requisiti: Competenza Magica 5

Le tue magie sono straordinariamente efficaci.

Scegli una Essenza, i DC per resistere alle magie di questa Essenza aumentano di 1. L'Abilita' puo' essere presa piu' volte (ogni volta e' richiesto +5 in CM rispetto alla volta precedente) ed il bonus si somma o si applica ad altra Essenza.

\subsection{Montagna umana}\index{Montagna umana}

Forse una volta eri gracile e debole, adesso sei una montagna di muscoli.

Quando prendi questa Abilita' aumenti di 1 i punti ferita presi per livello.

La \textbf{seconda volta} che prendi questa Abilita' aumenti di 1 i punti ferita presi per livello. 

La \textbf{terza volta} che prendi questa Abilita' aumenti il dado per tirare i punti ferita (da d4 a d6).
I bonus sono cumulativi e retroattivi ai livelli precedenti, tranne che l’aumento di dado vita.

La \textbf{quarta volta} che prendi questa Abilita' aumenti di una taglia.

\subsection{Occhio Clinico}\index{Occhio Clinico}

Requisiti: Competenza Armi 2

sei in grado di fare critici a creature normalmente immuni ai critici.

\subsection{Opportunista}\index{Opportunista}

Requisiti: Competenza Armi 2

puoi tentare di colpire un avversario (un attacco di opportunita') che esce da un area che tu minacci. L'abilita' e' usabile una volta per round come Reazione a costo 0 Azioni.

\subsection{Passo sicuro}\index{Passo sicuro}

la capacita' di non essere rallentati in un ambiente ostile. E' necessario dichiarare su quale ambiente si prende l'abilita'. In questi ambienti il terreno non e' difficile per te.

\bigskip

\begin{tabular}[c]{@{}ll@{}}
\toprule 
Ambiente & Ambiente\tabularnewline
Acquatico (sopra e sotto la superficie dell'acqua) & Giungla\tabularnewline
Foresta (conifere e decidue) & Pianura\tabularnewline
Deserto (terre brulle e deserto sabbioso) & Montagna (compreso colline)\tabularnewline
Freddo (ghiacciai, ghiaccio, neve e tundra) & Palude\tabularnewline
Sotterraneo (caverne, dungeon) & Urbano (edifici, strade, fogle)\tabularnewline
\bottomrule
\end{tabular}

\bigskip

Ogni qual volta si prende nuovamente questa abilita' si sceglie un
ambiente diverso e si somma al precedente.

\subsection{Passo tattico}\index{Passo tattico}

Requisiti: Competenza Armi 3

Guadagni una Azione di Movimento per round. L'Abilita' puo' essere presa massimo 1 volta. Questa Azione puo' essere fatta solo nel tuo round e costa 0 Azione eseguirla.

\subsection{Percettivo}\index{Percettivo}

La tua Consapevolezza e attenzione ai particolari e' sopra la media.
Prendi un bonus di +2 alla prove di Consapevolezza. L'Abilita' puo'
essere presa piu' volte, il bonus oltre la prima volta diventa +1.

\subsection{Persona veramente malvagia}\index{Persona veramente malvagia}

Requisiti: Competenza Armi 1

Due volte al giorno aggiungi il tuo valore di CA al colpire ed al danno, in mischia ad un avversario che vedi. L'Abilita' puo' essere dichiarata, come Reazione a costa 1 Azione, dopo il Tiro per Colpire ma prima che il Narratore dica se il colpo e' andato a segno o meno.

\subsection{Piu' sono grossi piu' fanno rumore quando cadono}\index{Piu' sono grossi piu' fanno rumore quando cadono}

Requisiti: Competenza Armi 1

Quando attacchi una creatura di almeno 2 taglie piu' grosse di te fai +1 danno aggiuntivo ogni 2 punti CA. Se e' solo una taglia superiore aggiungi 1 danno in piu' ogni 3 punti CA.

\subsection{Proseguire}\index{Proseguire}

La \textbf{prima volta} che prendi questa abilita' requisiti: Competenza Armi 1

Se uccidi l'avversario con il tuo ultimo colpo, in mischia, puoi effettuare una reazione di attacco con 3d6 + Competenza Armi + Potenza + Abilita' (senza contare bonus dovuti alla magia dell'arma) ed attaccare l'avversario successivo entro 1.5m con un -2 al colpire -1 al danno, se uccidi questa creature con un colpo non puoi effettuare altri attacchi ad altre creature.

La \textbf{seconda volta} che prendi questa abilita' Requisiti: Proseguire, Competenza Armi 6

Se uccidi la creatura con il tuo ultimo colpo, in mischia, puoi effettuare una reazione di attacco con l'arma senza contare bonus dovuti alla magia dell'arma e attaccare la creatura successiva in distanza di 1 metro con un -2 al colpire -1 al danno, se la uccidi puoi proseguire con la reazione di attacco (e ti sposti entro 2 metri) con la creature successiva e cosi'' via, ogni volta hai un -2 al colpire ed un -1 al danno cumulativo.

Questa abilita' permette di usare piu' reazioni per round.

\subsection{Questo e' il mio pugnale}\index{Questo e' il mio pugnale}

Requisiti: Competenza Armi 1

Ogni qual volta fai un critico con il tuo pugnale sommi la tua CA al danno. L'Abilita' e' usabile 1 volta per avversario nelle 24 ore e si applica automaticamente al primo critico effettuato.

\subsection{Questa e' la mia arma!}\index{Questa e' la mia arma!}

Requisiti: Competenza Armi 1

Ogni volta che colpisci il medesimo avversario fai un danno aggiuntivo (Max +1 per round di combattimento, anche se lo colpisci piu' nel round). Fino ad un massimo +5. La prima volta che non colpisci nel round l'avversario il bonus torna a +0.

\subsection{Radici magiche}\index{Radici magiche}

Requisiti: Competenza Magica 1

Finche' sei influenzato da una Essenza, utilizzando una azione la tua arma guadagna un +1 al colpire/danno e si considera un'arma +1. Per ogni Essenza che ti influenza nel round, oltre la prima (non da oggetti magici) il bonus aumenta di +1/+1, max +3/+3

\subsection{Rappresaglia}\index{Rappresaglia}

Vedere i tuoi amici feriti ti riempie di rabbia.
Quanto un compagno (o te stesso) scende sotto meta'’ dei punti ferita guadagni un +1 a Difesa e
 Tiro per Colpire e Tiri Salvezza. La durata massima dell’effetto e’ 1 minuto (10 round) a giorno e
 deve essere consecutiva. Il giocatore sceglie se attivare o meno l’abilita’.
Puoi prendere questa Abilita' fino a 3 volte, ogni volta il bonus massimo sale di 1.

\subsection{Resistenza della pietra}\index{Resistenza della pietra}

Nel tempo hai allenato la tua Potenza a reggere gli urti, trasformazioni,
veleni e quant'altro volesse modificare il tuo corpo. Ogni qual volta
prendi questa Abilita' ottieni un bonus di +2 al Tiro Salvezza su
Tempra. Il bonus e' cumulativo, +2 la prima volta, +1 la seconda,
+1 la terza ed ultima volta possibile

\subsection{Rilevare il Magico}\index{Rilevare il Magico}

Competenza Magica 1

Se lo puoi vedere sai anche se e' magico. Costa una Azione attivare la vista magica

\subsection{Ricarica rapida (Balestra)}\index{Ricarica rapida}

Agilita' 2, Tiro preciso

come tiro rapido, solo per balestre

\subsection{Riflessi fulminei}\index{Riflessi fulminei}

Nel tempo hai allenato i tuoi riflessi a schivare e prevedere qualsiasi ostacolo. Ogni qual volta prendi questa Abilita' ottieni un bonus di +2 ai Tiri Salvezza su Riflessi. Il bonus e' cumulativo, +2 la prima volta, +1 la seconda, +1 la terza ed ultima volta possibile.

\subsection{Schivare trappole}\index{Schivare trappole}

prova di Agilita' superiore a DC trappola

La \textbf{prima volta} che prendi l'abilita' requisiti Agilita' 3

Se la prova ha successo dimezzi il danno della trappola. La prova e' attiva ovvero il personaggio non deve essere bloccato.

La \textbf{seconda volta} che prendi l'abilita' requisiti Schivare trappole, Competenza Armi 5

Se la prova di schivare trappole riesce puoi farne un altro per evitare completamente il danno. Se la prova di schivare trappole fallisce puoi comunque tentare un'altra volta.

E' anche possibile usare questa Abilita' per evitare Attacco furtivo (check Agilita' superiore a Tiro colpire avversario)

\subsection{Schivata prodigiosa}\index{Schivata prodigiosa}

come Reazione ad una Azione di attacco puoi aggiungere +2 alla tua Difesa. Puoi applicare il bonus dopo il Tiro per Colpire dell'avversario ma prima di sapere se ti ha colpito o meno.

\subsection{Seconda pelle}\index{Seconda pelle}

Requisito: Competenza Armi 1

Il costante allenamento con la tua armatura ti permette di indossarle senza grosse penalita'.

Il malus alle prove di Agilita' diminuisce di 1.

La \textbf{seconda volta} che si prende questa Abilita', Competenza Armi 6, il malus alle prove di Agilita' diminuisce di ulteriori 2. 
Il malus alle penalita' al movimento diminuisce di 1.

Puoi dormire in armature medie senza essere affaticato la mattina.

La \textbf{terza volta} che si prende questa Abilita', Competenza Armi 11, il malus alle prove di Agilita' diminuisce di ulteriori 2. Il malus alle penalita' al movimento diminuisce di un ulteriore 1. 

Puoi dormire in armature pesanti senza essere affaticato la mattina.

\subsection{Segugio}\index{Segugio}

Requisito: Intelletto 1, Volonta' 1, CA 1

Hai un talento naturale per seguire le persone

Con due Azioni ti focalizzi su un target che puoi vedere e finche' lo vedi rimani focalizzato. Tutte le tue Azioni che coinvolgono quel target hanno un +1 di bonus. Rimanere focalizzato costa 1 Azione per round.

La \textbf{seconda} volta che prendi questa abilita', Competenza Armi 6, il bonus sale a +2.

La \textbf{terza} volta che prendi questa abilita', Competenza Armi 12, il bonus sale a +3.

Il bonus puo' essere usato al TC, TS causati dall'avversatio, prove di competenza.. ma non al danno.

\subsection{Senso Trappola}\index{Senso Trappola}

Requisiti: Intelletto 2, Agilita' 3

hai un senso innato per trovare le trappole. Ti viene concesso una prova di consapevolezza (reazione) nel passare entro 1 metro da una trappola (che fara' il Narratore) .

La \textbf{seconda volta} che prendi l'Abilita' il raggio aumenta fino a 3 metri e prendi un +2 alla prova. La \textbf{terza volta} che prendi l'Abilita' il raggio aumenta a 9 metri.

\subsection{Senza Traccia}\index{Senza Traccia}

Requisiti: Passo sicuro

la capacita' di non lasciare impronte nell'ambiente scelto. Ogni volta che prendi questa Abilita' puoi scegliere un ambiente diverso (vedi Abilita' Passo Sicuro) di cui hai competenza. La prova di sopravvivenza per inseguirti ha una difficolta' aumentata di 10.

\subsection{Stai giu'!}\index{Stai giu'!}

Quando esegui un critico su un avversario la forza del tuo colpo e' tale da metterlo prono. L'avversario deve fare un Tiro Salvezza Tempra DC 10+1/2CA+Potenza o cadere prono. L'Abilita' funziona su creature di taglia pari o inferiore a quella del personaggio.

La \textbf{seconda volta} che prendi l'Abilita' puoi influenzare anche creature di una taglia superiore.

La \textbf{terza volta} che prendi l'Abilita' puoi influenzare anche creature di due taglie superiori.

\subsection{Tiro preciso}\index{Tiro preciso}

Requisiti: Agilita' 3, Competenza Armi 1

guadagni un +1 colpire e +1 al danno per i tiri, con armi da tiro o archi, entro 9 metri.

\subsection{Tiro rapido}\index{Tiro rapido}

Requisiti: Agilita' 3, Tiro Preciso, Competenza Armi 2

Puoi effettuare un tiro in piu' con Arco o Pugnale lanciato. Ogni
proiettile lanciato nel round prende un -4 al Tiro per Colpire. Per
poter usufruire dell'attacco in piu' devi usare l'Azione di attacco
multiplo.

\subsection{Toccata e fuga}\index{Toccata e fuga}

Prendendo -5 al Tiro per Colpire all'Azione di Attacco, guadagni un'azione di 1 movimento.

\subsection{Tocco pietoso}\index{Tocco pietoso}

Requisiti: Patrono buono, Imposizione delle mani, Competenza Magica 3

Il tuo tocco lenisce non solo le ferite ma anche le sofferenze e dolori. Ogni qual volta usi l'Abilita' Imposizione delle mani puoi aggiungere anche questa Abilita' come Azione Immediata.

Usando l'Imposizione delle mani puoi, rinunciando ad un numero di d6 curativi indicati, rimuovere le seguenti afflizioni.

\textbf{2d6} Tratti in comune 3:

Affaticato: il soggetto non e' piu' affaticato

Scosso: il soggetto non e' piu' scosso.

Infermo: il soggetto non e' piu' infermo.

Frastornato: il soggetto non e' piu' frastornato

\textbf{3d6} Tratti in comune 6:

Malato: funziona come l'Essenza di Cura, fatto da un incantatore di pari livello.

Stordito: il soggetto non e' piu' stordito

Confuso: il soggetto non e' piu' confuso

Nauseato: il soggetto non e' piu' nauseato

\textbf{4d6} Tratti in comune 9:

Maledetto: funziona come Essenza di Protezione rimuovere maledizione, usando il livello del incantatore come livello.

Impaurito: il soggetto non e' piu' impaurito

Avvelenato: funziona come Essenza di Cura rimuovi la condizione di avvelenato usando il livello del incantatore come livello di potere.

Ristorativo: il soggetto recupera 1d4 punti in una caratteristica

\textbf{5d6} Tratti in comune 11:

Rigenerante: il tocco dell'incantatore puo' fare rigenerare arti tagliati, se il soggetto e' ancora vivo.

Accecato: il soggetto non e' piu' cieco

Sordo: il soggetto non e' piu' sordo

Paralizzato: il soggetto non e' piu' paralizzato

Pietrificato: come Essenza di Trasformazione pietra in carne. Il soggetto non e' piu' pietrificato

\subsection{Vampiro}\index{Vampiro}

Requisiti: Odore del sangue

La tua sete di sangue diventa cura. Il bonus di sete di sangue puo' aumentare fino a +5.

Se il bonus aumenta da +3 a +4 o +5 puoi, ingurgitando il sangue avversario, curarti di 1d6 impiegando un 2 azioni

\subsection{Volonta' Ferrea}\index{Volonta' Ferrea}

Nel tempo hai allenato la tua volonta' per resistere a qualsiasi debolezza e paura. Ogni qual volta prendi questa Abilita' ottieni un bonus di +2 ai Tiri Salvezza su Arbitrio. Il bonus e' cumulativo, +2 la prima volta, +1 la seconda, +1 la terza ed ultima volta possibile

\pagebreak

\section{Famiglio}\index{Famiglio}

\label{famiglio}
\begin{quotebox}
Abbiamo imparato a volare come gli uccelli, a nuotare come i pesci, tuttavia non abbiamo imparato l'arte di vivere come fratelli. (Martin Luther King )
\end{quotebox}

I famigli sono animali scelti dal personaggio, tramite l'Abilita' Famiglio, perché gli siano d'aiuto nelle avventure e per compagnia. Un famiglio ha un legame speciale con il suo padrone.

Un famiglio e' un normale animale che mantiene aspetto, Dadi Vita, Competenza Armi, bonus ai Tiri Salvezza Base, Abilita' e Talenti del normale animale che era, ma viene trattato come bestia magica al fine di determinare qualsiasi effetto che dipenda dal suo tipo. 

Solo un normale animale, non modificato, puo' diventare un famiglio.

Un famiglio conferisce delle Capacita' Speciali al suo padrone, come indicato nella tabella sotto. Queste Capacita' Speciali si applicano solo quando il padrone e il famiglio sono entro 100 m l'uno dall'altro.

Se un famiglio viene congedato, perso oppure muore, puo' essere sostituito una settimana dopo con uno speciale rituale che costa 200 mo per livello del personaggio. Per {\small completare} il rituale occorrono 8 ore.

\bigskip

\begin{tabular}[c]{@{}ll@{}}
\toprule 
Famiglio & Capacita' speciale\tabularnewline
Lucertola / Capra & Il padrone guadagna bonus +2 alle prove di Sopravvivenza\tabularnewline
Corvo & Il padrone guadagna bonus +2 alle prove di Faccia Tosta\tabularnewline
Donnola / Volpe & Il padrone guadagna bonus +1 al Tiro Salvezza su Riflessi\tabularnewline
Falco & Il padrone guadagna bonus +2 su Consapevolezza sulla vista\tabularnewline
Gatto & Il padrone guadagna bonus +2 alle prove di Criminalita'\tabularnewline
Gufo & Il padrone guadagna bonus +2 su Consapevolezza sulla vista\tabularnewline
Lontra / Ornitorinco & Il padrone guadagna bonus +2 alle prove di Resistenza\tabularnewline
Pipistrello / Scoiattolo Volante & Il padrone guadagna bonus +2 alle prove di Acrobatica\tabularnewline
Riccio & Il padrone guadagna bonus +1 al Tiro Salvezza su Volonta'\tabularnewline
Scimmia & Il padrone guadagna bonus +2 alle prove di Criminalita'\tabularnewline
Topo & Il padrone guadagna bonus +1 al Tiro Salvezza su Tempra\tabularnewline
\bottomrule
\end{tabular}

\bigskip

Utilizzare le statistiche base di una creatura della specie del famiglio, apportando i seguenti cambiamenti.

\bigskip

\textbf{Dadi Vita}: Ai fini degli effetti legati al numero dei Dadi Vita, utilizzare il punteggio di CA del personaggio del padrone o il normale totale di DV del famiglio, quale dei due sia piu' alto.

\textbf{Attacchi}: Utilizzare la CA del padrone. Utilizzare il modificatore di Agilita' o Potenza del famiglio, quale dei due sia piu' alto, per calcolare il bonus di attacco in mischia del famiglio con gli Attacchi Naturali. Il danno e' uguale a quello di una normale creatura della specie del famiglio.

\textbf{Difesa}: il famiglio ha una Difesa pari al padrone o propria a seconda del valore piu' alto.

\textbf{Tiro Salvezza}: Per ogni Tiro Salvezza, utilizzare i bonus al Tiro Salvezza Base del famiglio (Tempra +2, Riflessi +2, Volonta' +0) o quelli del padrone quali siano i migliori. Il famiglio applica i suoi valori di caratteristica ai Tiri Salvezza, e non condivide nessuno dei bonus che il suo padrone potrebbe ricevere ai propri Tiri Salvezza.

\bigskip

\textbf{Descrizione delle Capacita' del Famiglio}

Tutti i famigli possiedono Capacita' Speciali (oppure le attribuiscono ai loro padroni) a seconda dei livelli combinati del padrone nelle classi che concedono i famigli, come indicato nella tabella seguente. Le capacita' elencate nella tabella sono cumulative.

\bigskip

\begin{tabular}[c]{@{}llll@{}}
\toprule 
CM del Padrone & Mod. Difesa Famiglio & intelletto famiglio & Speciale\tabularnewline
1-2 & +1 & -2 & Allerta, Condividere Essenza, \tabularnewline &&& Eludere Migliorato\\
&&&Legame Empatico\tabularnewline
3-4 & +2 & -1 & Trasmettere Essenza a contatto\tabularnewline
5-6 & +3 & 0 & Parlare con il Padrone\tabularnewline
7-8 & +4 & 0 & Parlare con gli Animali\\
&&&della Sua Specie\tabularnewline
9-10 & +5 & 1 & Vedere attraverso Famiglio\tabularnewline
11-12 & +6 & 1 & -\tabularnewline
13-14 & +7 & 2 & -\tabularnewline
15-16 & +8 & 2 & -\tabularnewline
17-18 & +9 & 3 & -\tabularnewline
19-20 & +10 & 3 & -\tabularnewline
\bottomrule
\end{tabular}
\bigskip

\textbf{Livello}: Il numero indicato qui e' il livello del padrone del famiglio, articolato in fasce.

\textbf{Modificatore armatura}: Il numero indicato qui e' in aggiunta al Bonus di Armatura Naturale esistente del famiglio.

\textbf{Intelletto}: Il punteggio di Intelletto del famiglio. Si tiene questo valore o quello del famiglio a seconda di quale sia piu' alto. 

\textbf{Speciale}: Le capacita' speciali acquisite dal famiglio (e/o dal padrone).

\textbf{Allerta}: Quando il famiglio e' a portata di braccio dal padrone, questi guadagna +2 alle prove di Consapevolezza

\textbf{Condividere Essenze}: A propria discrezione, il padrone puo' lanciare qualsiasi Essenza che abbia effetto su di ``sé'' sul suo famiglio (come una Essenza a contatto), al posto di se stesso.

Il padrone puo' lanciare sul suo famiglio Essenze anche se queste normalmente non hanno effetto su creature del tipo del famiglio (bestie magiche).

\textbf{Eludere Migliorato}: Se il famiglio e' soggetto a un attacco che normalmente permette un Tiro Salvezza su Riflessi per dimezzare i danni, il famiglio non subisce danni se supera il Tiro Salvezza e solo la meta' dei danni se fallisce il Tiro Salvezza.

\textbf{Legame Empatico}: Il padrone ha un legame empatico con il suo famiglio fino a una distanza di 1 km. Il padrone non puo' vedere attraverso gli occhi del famiglio, ma puo' comunicare telepaticamente con esso. A causa della natura limitata del legame, si possono comunicare solo emozioni generiche.

\textbf{Trasmettere Essenze a Contatto}: Se il padrone ha Competenza Magica 3 o superiore, il famiglio puo' trasmettere Essenza a contatto per lui. Se il padrone e il famiglio sono in contatto quando il padrone lancia un'Essenza a contatto, egli puo' designare il suo famiglio come ``colui che crea il contatto''.

Il famiglio puo' allora trasmettere l'Essenza a contatto proprio come il padrone. E' necessario che l'attacco del famiglio sia nello stesso round, ma successivamente come azione, del lancio dell'Essenza.

\textbf{Parlare col Padrone}: Se il padrone ha Competenza Magica 5 o superiore, il famiglio e il padrone possono comunicare verbalmente, come se utilizzassero un linguaggio comune. Le altre creature o animali non sono in grado di comprendere la loro conversazione, se non utilizzando ausili magici.

\textbf{Parlare con Animali della Sua Specie}: Se il padrone ha Competenza Magica 7 o superiore, il famiglio e' in grado di comunicare con animali della sua specie generica (incluse le varianti crudeli): pipistrelli con pipistrelli, topi con roditori, gatti con felini, falchi e gufi e corvi con uccelli, serpenti e lucertole con rettili, rospi con anfibi, scimmie con altri primati, donnole con ermellini e mustelidi... La comunicazione e' limitata dalla Intelligenza delle creature con cui il famiglio comunica.

\textbf{Vedere attraverso Famiglio}: Se il padrone ha Competenza Magica 9 o superiore, puo' vedere attraverso il famiglio. Attivare questa abilita' costa 1 azione immediata.

\pagebreak

\section{La Magia}\index{Magia}\index{Essenza}

\label{la-magia}
\begin{quotebox}
"Le parole sono, nella mia NON modesta opinione, la nostra massima ed inesauribile fonte di magia. In grado sia di infliggere dolore che di alleviarlo" (Albus Silente)\linebreak\linebreak
Non lascerai vivere colei che pratica la magia. (Libro dell'Esodo)
\end{quotebox}

Si intende incantatore o mago qualsiasi usufruitore di Essenze a qualsiasi titolo ed uso.

La Magia ci circonda ed e' accessibile, ma non tutti sanno dominarla e chi non la sa dominare ne viene dominato.

Sono considerazioni false, ma difficilmente contraddicibili al popolino.

Tra una levatrice che segue Atherim, un cavaliere di Sumkjr ed un negromante di Sixiser o una spia di Shayalia ci sono notevoli differenze di pensiero, comportamento ed azione.

Non sempre chi si dice di essere fedele ad un Patrono lo e' o peggio lo e' di qualcun' altro.

Bisogna sempre prestare attenzione ad un Devoto, i suoi comportamenti seguono interessi non sempre lineari od ovvi.

Solo essendo un Devoto si hanno i bonus concessi dal Patrono. \index{Devoto}

Le essenze scelte possono essere solo quelle offerte dal Patrono prescelto.

Se hai scelto di essere un Devoto,\index{Devoto} quindi hai almeno 3 tratti in comune, dovra' scegliere le Essenze del suo Patrono con le limitazioni e vantaggi indicati.

Se hai scelto di essere un Seguace \index{Seguace}allora hai almeno 2 tratti in comune, puoi scegliere le Essenze offerte dal Patrono ma non hai limitazioni nella scelta delle Essenze ne vantaggi offerti.

\subsection{Competenza Magica ed Essenza}\index{Competenza Magica}\index{Essenza}

\label{competenza-magica-ed-essenza}

Ogni qual volta il personaggio attribuisce un punto alla Competenza Magica puo' decidere se aggiungere due Essenze a quelle da lui conosciute, oppure attribuire un +1 ai check di magia ad una Essenza gia' conosciuta (bonus di specializzazione).

Quando deve fare una prova magia su una Essenza appresa ma non come specialista tirera' 3d6+ punteggio di Competenza Magica + caratteristica collegata + vari ed eventuali.

Quando deve fare una prova magia su una Essenza in cui ha dedicato una competenza aggiuntiva tirera' 3d6+ Competenza Magica + Bonus di specializzazione + caratteristica collegata + vari ed eventuali.

\bigskip

\textbf{Es. Un incantatore di 6 livello ha un punteggio di 6 in Competenza Magica} ed ha attribuito i suoi punti in questa maniera

Essenza Alterare

Essenza Attacco +1 +1 +1

Essenza Rivelazione

Essenza Cura +1

Se deve usare una Essenza di Alterare o Rivelazione potra' fare una prova di CM a +6 (piu' caratteristica collegata), se deve fare una prova magia su Attacco il suo punteggio di CM e' 6+1+1+1 (piu' la caratteristica collegata), mentre su Cura ha 6+1 (piu' la caratteristica collegata).

Questo bonus di specializzazione si somma anche nelle prove di Concentrazione che riguardino questa essenza.

\bigskip

\textbf{Un incantatore di 8 livello invece ha diviso 4 punti di Competenza
Magica in questa maniera}:

Essenza Cura +1

Essenza Creazione

Essenza Protezione

Essenza Difesa +1

L'Essenza di Creazione e Protezione usera' un CM a +4, per la Cura e Difesa a +5 e caratteristica collegata.

\bigskip

\textbf{Il punteggio di specializzazione che una Essenza puo' avere deve essere inferiore o pari a meta' del valore di Competenza Magica}. Es. se hai CM a 4 il bonus di specializzazione massimo ad una singola Essenza puo' essere +2

\subsection{Le regole delle Essenze}\index{regole delle Essenze}

\label{le-regole-delle-essenze}

Ci sono dei punti fermi, delle regole che sovrintendono la magia e queste sono:
\begin{itemize}
\item Non e' permesso riportare in vita i morti. Solo un Patrono puo' restituire l'anima ad un corpo. 

\item Non e' permesso creare vita 

\item Declama la tua magia o non funzionera' 

\end{itemize}

\subsection{Creature ed Elementi}\index{Creature ed Elementi}

\label{creature-ed-elementi}

Ogni Essenza che si va a formulare ha un ambito di applicazione che riguarda \textbf{Creature Naturali}, \textbf{Creature} \textbf{Magiche},
\textbf{Elementi}, \textbf{Energia, Concetto o Virtu'}.\index{Creature Naturali}
\index{Creature}\index{Magiche}\index{Elementi}\index{Energia}\index{Concetto}\index{Virtu}
\bigskip

Le \textbf{Creature Naturali} sono Insetti, Rettili, Animali, Umanoidi, Piante, Pesci.

Le \textbf{Creature Magiche} sono: Demoni, Fate, Spiriti, Non morti, Aberrazioni (tutto cio' che non naturale o di Yeru) e Draghi.

Se una Creatura Naturale ha poteri magici allora si e' una Creatura Magica.

Gli \textbf{Elementi} sono: acqua, terra, aria, metallo, legno. 

I \textbf{Concetti} sono: Spazio, Essenza e Vita

\textbf{Energia} comprende: Fuoco, Luce, Suono, Elettricita', Energia Positiva, Energia Negativa, Freddo.

Le \textbf{Virtu}' comprendono i Tratti

\bigskip

\textbf{Tabella Raggruppamenti Elementi e Creature}

\begin{dndtable}[LLLLLL]
%%\begin{tabular}[c]{@{}llllll@{}}
\textbf{Creature Naturali} & \textbf{Creature Magiche} & \textbf{Energia} & \textbf{Elementi} & \textbf{Concetto} & \textbf{Virtu}'\tabularnewline
Pesci & Demone & Fuoco & Acqua & Spazio & Tratti\tabularnewline
Piante & Fate & Suono & Aria & Essenza & \tabularnewline
Rettili & Spiriti & Elettricita' & Terra & Vita & \tabularnewline
Umanoidi & Non Morti & Energia Positiva & Legno & & \tabularnewline
Animali & Aberrazioni & Energia Negativa & Metallo & & \tabularnewline
 & Draghi & Luce & & & \tabularnewline
 & Spiriti & Freddo & & & \tabularnewline
 & & & & & \tabularnewline
\end{dndtable}
%%\bottomrule
%%\end{tabular}

\bigskip

Nelle specifiche delle Essenze troverete se queste lavorano su Elementi, Creature Naturali o Magiche, Energia, Concetti o Virtu' o solo specifiche componenti di queste.

\subsection{Caratteristiche base delle Essenze}\index{Caratteristiche base delle Essenze}

\label{caratteristiche-base-delle-essenze}

Ogni magia che si va a creare ha queste caratteristiche di base:

\smallskip

\textbf{Tempo di lancio}: due Azioni\index{Tempo di lancio}

\textbf{Durata}: istantanea\index{Durata}

\textbf{Distanza}: distanza di mischia (a tocco)\index{Distanza}

\textbf{Area di Effetto}: 1 creatura \index{Area di Effetto}

\textbf{Obiettivi}: quando lanci una Essenza determina se l'obiettivo e' una Creatura o Elemento oppure un punto nello spazio entro la distanza stabilita.\index{Obiettivi}

\textbf{Obiettivi Speciali}:\index{Obiettivi Speciali} puoi anche lanciare una Essenza su un oggetto e la prima creatura che tocchera' l'oggetto diventera' l'obiettivo della magia. La durata rimane limitata ad un minuto ed a costo 3.

\textbf{Potenziamenti}: l'incantatore decide di potenziare una Essenza come preferisce, aumentando la difficolta' di esecuzione della stessa. Consultare l'elenco per i dettagli\index{Potenziamenti}

\textbf{Sommando le varie caratteristiche di base della magia si determina la difficolta' totale da superare con una prova su CM + Punteggio Caratteristica correlata all'Essenza + Bonus.}

\textbf{Solo in caso di superamento si riesce a lanciare la magia e si verifica che livello di potere di potere si e' raggiunto (una volta sottratta la difficolta')}

Di base una formulazione magica sara': applico l'Essenza X alla/e Creatura o Elemento Z che si trova Y distante.

\subsection{Recitare l'Essenza}\index{Recitare l'Essenza}\index{Recitare}

\label{recitare-lessenza}

puo' sembrare sciocco o inutile ma se un giocatore non recita la sua Essenza questa non funzionera'.

In TUS la magia e' libera e freeform, ovvero non ci sono liste di incantesimi, ogni giocatore si inventa gli effetti che vuole, prendendo ispirazione (e limiti) dalle linee guida della Essenza.

Il giocatore ``declamera'\,'' la sua Essenza (Es. ``Possa questa piana ardere come il Deserto di Fiamma di Daruk-Yum'') ed in base alla prova effettuata vedra' se va a fuoco veramente oppure e' poco piu' di una candela.

Il giocatore come visto sopra, e dettagliato successivamente, stabilisce cosa influenzare con la sua Essenza e poi sara' la prova di magia a stabilire quanto viene influenzato (l'effetto, ovvero il Livello di Potere) l'obiettivo.

Il Narratore deve preoccuparsi di fare declamare sempre l'Essenza, questo perche' aiuta a comprendere cosa si vuole ottenere dall'Essenza, cosa che i fattori numerici (distanza, obiettivo, durata... ) non descrivono accuratamente.

\subsection{Potenziamenti delle Caratteristiche dell'Essenza}\index{Potenziamenti delle Caratteristiche dell'Essenza}

\label{potenziamenti-delle-caratteristiche-dellessenza}

I potenziamenti definiscono e migliorano la magia che si va a lanciare; questi possono riguardare Distanza, Area di effetto, Contingenza, Selezione, Durata.

\subsubsection{Distanza}\index{Distanza}

\label{distanza}

\textbf{Tocco} (+0): la magia si applica con un Tiro per Colpire a
Tocco

\textbf{Corto} (+2): la magia arriva ad una distanza entro 10 metri

\textbf{Medio} (+3) : la magia arriva ad una distanza di 50 metri

\textbf{Lungo} (+5): la magia arriva ad una distanza di 250 metri

\textbf{Lunghissimo} (+9): la magia arriva ad una distanza di 1000 metri

\textbf{Estremo} (+16): la magia arriva ad una distanza di 10 km

\textbf{Planetario} (+30): la magia arriva ovunque tu possa immaginare l'obiettivo (anche magicamente)

\subsubsection{Area di Effetto}\index{Area di Effetto}

\label{area-di-effetto}

\textbf{1 soggetto} (+1): per ogni soggetto influenzato. Se e' di taglia superiore alla media la difficolta' e' +2. I soggetti influenzati dalla medesima essenza devono essere entro 3 metri dal primo obiettivo oppure e' necessario operare tramite un area di effetto circolare (raggio)

\textbf{3 metri di raggio} (+2): la magia influenza 3 metri di raggio oppure una dimensione contigua non superiore a 14 quadretti. Per ogni +2 aggiuntivo di difficolta' aumenti il raggio di 3 metri (oppure 10 quadretti consecutivi)

\textbf{1.5 km di raggio} (+25): influenzi un ampia zona come un piccolo paese

\textbf{10 km} (+30): influenzi una grande citta'

\textbf{50 km} (+40): influenzi una regione

\textbf{200 km} (+50): influenzi una intera regione

\textbf{Planetaria} (+60): influenzi l'intero pianeta

\textbf{Deselezione} (+1): con questo potenziamento escludi una creatura od oggetto dall'area di effetto. Ogni +1 toglie una persona dagli effetti della magia (se Area di Effetto a Raggio).

Es. voglio tirare una Fuoco Palla toroidale attorno a me. Pago +1 di Deselezione (mi escludo dall'esplosione) oltre il +2 per il raggio corto.

Es. voglio tirare una Fuoco Palla ai miei nemici intorno a me. Pago +4 (perche' scelgo 4 soggetti) nella Area di Effetto e su ognuno di loro ``cadra' una Fuoco Palla di che interessera' solo loro singolarmente, i soggetti influenzati devono essere tutti nell'Area di Effetto.

\subsubsection{Contingenza}\index{Contingenza}

\label{contingenza}

Ogni volta che lanci una magia scegli tutti gli aspetti della stessa ma la l'effetto (durata) non incomincia finche' non accade una certa cosa. Questa cosa, contingenza, deve essere chiara, evidente, tale che un normale uomo possa notarla. La condizione deve manifestarsi entro 9 metri da dove arriva l'area di effetto. Eventuali condizioni particolari possono essere concordate con il Narratore con costi maggiori. 

Non e' facile attivare le contingenze. I costi sono molto elevati.

\textbf{Breve} (+8) se la contingenza non accade entro 10 minuti l'incantesimo scompare

\textbf{Medio} (+16) se la contingenza non accade entro 1 ora l'incantesimo scompare

\textbf{Lungo} (+24) se la contingenza non accade entro 1 giorno l'incantesimo scompare

\textbf{Settimana} (+32) se la contingenza non accade entro 7 giorni l'incantesimo scompare

\textbf{Mese} (+40) se la contingenza non accade entro 30 giorni l'incantesimo scompare

\textbf{Anno} (+48) se la contingenza non accade entro 1 anno l'incantesimo scompare

\subsubsection{Durata}\index{Durata}

\label{durata}

\textbf{Concentrazione} (+1): la magia dura finche' ti concentri, per un massimo di round pari al tuo valore di Competenza Magica + Volonta'. Non puoi lanciare altre Essenze finche' rimani concentrato.

\textbf{Istantanea} (+0): la durata e' istantanea e non perdura

\textbf{Breve} (+1) : la durata e' di 1 round per CM

\textbf{Corta} (+3): la durata e' di 10 minuti

\textbf{Media} (+5): la durata e' di 1 ora

\textbf{Lunga} (+8): la durata e' di 8 ore

\textbf{Giorno} (+12): la durata e' di 24 ore

\textbf{Settimana} (+20): la durata e' di 1 settimana

\textbf{Mese} (+35): la durata e' di 30 giorni

\textbf{Anno} (+40): la durata e' di 1 anno

\textbf{Permanente} (+50): la magia dura in maniera permanente o finche' vuoi tu

Per Durata di una Essenza si intende sia quanto dura l'effetto sia quanto lo si puo' trattenere prima che debba manifestarsi. Un incantatore puo' trattenere un numero di round pari al suo valore in Competenza Magica + Intelletto.

L'Essenza di Cura e Attacco hanno sempre durata Istantanea

La Distruzione di materia e' sempre permanente come durata ed immediato come effetto e ha costo 8.

\subsubsection{Aree di effetto diverse}\index{Aree di effetto diverse}

\label{aree-di-effetto-diverse}

\textbf{L'Area di Effetto puo' essere non solo sferica, ma anche una linea od un cono.}

L'usufruitore di magia potra' restringere l'area di effetto, fino ad essere uno spicchio (il cono) della circonferenza iniziale oppure una linea. La distanza raggiunta e' sempre pari al costo pagato per la Distanza.

Per ogni magia formulata l'incantatore deve avere una descrizione da parte del giocatore che non puo' essere il semplice elenco di Durata, Distanza, Essenza.. ma deve dare un nome alla magia, descriverne gli effetti, pena il non funzionamento della magia. Vedere gli esempi presentati nel capitolo sulla Magia.

Nella formulazione di una magia Il Narratore ha sempre e comunque l'ultima parola sui costi ed effetti.

Tenete a disposizione dei segnalini per ``disegnare'' l'area di effetto.

\subsection{Influenzati da piu' Essenze}\index{Influenzati da piu' Essenze}

\label{influenzati-da-piu-essenze}

Quando un personaggio e' influenzato da \textbf{due o piu' effetti temporanei creati da Essenze} che danno lo stesso tipo di bonus, malus o danno (protezione verso fuoco, bonus alla Difesa o TS... , multiple palle di acido), si tiene conto solo di quella dal livello di potere maggiore.

\subsection{Scegliere l'effetto dell'Essenza}\index{Scegliere l'effetto dell'Essenza}

\label{scegliere-leffetto-dellessenza}

Nella descrizione delle Essenze quando trovate per un livello di potere elencati piu' possibilita', dovete sceglierne uno solo.

Esempio:

\begin{tabular}[c]{@{}ll@{}}
\toprule 
<11 & - Rimuovi la condizione abbagliato\tabularnewline
& - Curi 1d6 pf
\end{tabular}

\subsection{Altre regole}

\label{altre-regole}

\subsubsection{Attacco con Essenze non di Attacco}\index{Attacco con Essenze non di Attacco}

Alcune Essenze implicano un danno anche se non sono Essenze di Attacco, come riportato negli esempi per Alterazione, ma concettualmente valido anche per altre Essenze

Se l'incantatore acquisisce la capacita' di un attacco (solitamente magico, e non naturale) tramite una Trasformazione o una Alterazione, ma anche Creazione (vedi pioggia di fuoco..) potra' usare questi poteri dal round successivo facendo un danno di una categoria di Livello Potere immediatamente inferiore a quello ottenuto se fosse stato nella Essenza Attacco, se questo e' di forma magica.

SI deve considerare che la manifestazione dell'Essenza si completi e sia usabile dal round successivo.

Se acquisisce una forma di attacco naturale potra' comunque usare l'attacco dal round successivo con un danno coerente alla forma di attacco acquisito (morso, artiglio..).

\subsubsection{Usare due Essenze}\index{Usare due Essenze}

Ci sono situazioni in cui diviene necessario usare due Essenze, in questo caso il tempo di lancio aumenta in modo significativo.

Partendo dal presupposto che si devono conteggiare le difficolta' base (distanza, target, durata..) per ogni Essenza usata si deve fare un solo check di competenza magica con la difficolta' piu' alta. Il tempo di lancio aumenta di 2 round.

Se quindi lanciare una magia di norma costa due Azioni, lanciare due Essenze porta il tempo di lancio a 3 round. Il Lancio di tre Essenze viene terminato alla fine dei 5 round. Il livello di potere raggiunto sara' il medesimo (essendo unica la difficolta', quella maggiore) per tutte le magie accodate.

\subsubsection{Alterare le Essenze}\index{Alterare le Essenze}

\label{alterare-le-essenze}

l'incantatore puo' modificare a piacimento, aumentando la difficolta' della magia o proprie energie, le magie che va a formulare.

\begin{itemize}
\item 
\textbf{Magia efficace} 
Magia efficace: sacrificando PF puo’ aumentare la difficolta’ a resistere alla magia
\begin{itemize}
	\item Sacrificando 4 punti ferita la difficolta’ del Tiro Salvezza aumenta di 1
	\item Sacrificando 8 punti ferita la difficolta’ del TS aumenta di 2

	\item Sacrificando 16 punti ferita la difficolta’ del TS aumenta di 3
\end{itemize}
\end{itemize}
%
\begin{itemize}
\item 
\textbf{Magia eterea}: aumentando di 2 la difficolta' di lancio (ovvero tolgo 2 al risultato della prova di competenza magica) le proprie magie hanno pieno effetto su creature eteree o incorporee 
\end{itemize}
%
\begin{itemize}
\item 
\textbf{Magia pietosa}: aumentando di 3 la difficolta' di lancio (ovvero tolgo 3 al risultato della prova di competenza magica) le magie infliggono danni temporanei. Le magie che infliggono danni di un tipo particolare (come da fuoco) infliggono danni temporanei dello stesso tipo. 
\end{itemize}

\subsubsection{Esecuzione di Essenze in maniera collaborativa}\index{Esecuzione di Essenze in maniera collaborativa}\index{collaborativa}

Nel caso in cui si voglia usare un Essenza con una difficolta' totale non raggiungibile e' possibile, sotto certi limiti, fare in modo che un gruppo di incantatori riesca nell'impresa.

Si divide la difficolta' totale (DC) della Essenza da lanciare tra i vari incantatori (non piu' di 7 incantatori possono partecipare) ed ogni incantatore deve superare una prova pari al doppio della difficolta' ottenuta.

Il tempo di lancio e' di 1 round per ogni incantatore impegnato nella formulazione.

Es. 5 incantatori vogliono lanciare una Protezione estesa e duratura, per un a difficolta' totale minima 32. In questa situazione ogni incantatore deve fare una prova di CM (32/5)x2= 14, ovvero ogni incantatore deve superare una prova di CM a difficolta' 14. Se anche un solo incantatore sbaglia la prova al termine del lancio dell'Essenza, dopo 5 round, questa fallira' e l'Essenza non sara' lanciata.

\subsubsection{Tentare Essenza con impedimenti}\index{Tentare Essenza con impedimenti}\index{impedimenti}

se mani e bocca sono bloccati l'incantatore non puo' formulare Essenze. Per usare una Essenza e' necessario avere entrambe le mani e la bocca liberi.

Aumentando di 5 la difficolta' puoi non usare le mani, se aumenta di 10 la difficolta' puoi non usare la bocca. Quindi se l'incantatore e' legato ed imbavagliato puo' lanciare una Essenza con la sola forza del pensiero con una difficolta' aumentata di 15, ovvero la difficolta' base aumenta di 15.

Una Essenza lanciata con impedimenti se non supera 11 come volare non ottiene l'effetto minimo dell'essenza (e si considera comunque formulata).

\subsection{Riuscire e Fallire nella prova di Magia}\index{Riuscire e Fallire nella prova di Magia}

\label{riuscire-e-fallire-nella-prova-di-magia}

Per capire se si riesce nella Magia si deve innanzitutto superare, con una prova di Competenza Magica (3d6 + CM + Punteggio Caratteristica correlata all'Essenza + Bonus) il valore dato dalla somma ottenuta da Tempo di Lancio, Durata, Distanza, Area di Effetto/Obiettivi, Potenziamenti.

Si sottrae al valore ottenuto nella prova di Competenza Magica la somma delle difficolta' base (Tempo di Lancio, Durata, Distanza, Area di Effetto/Obiettivi, Potenziamenti...) e si controlla il risultato nella colonna Livello di Potere dell'Essenza usata per verificarne l'effetto ottenuto.

Nella tabella delle Essenza il primo livello di potere e' indicato come ``\textless xx'', ovvero se si riesce a superare la difficolta' impostata dai fattori base ed il valore eccedente e' inferiore a xx, si usa quel effetto.

Se non si riesce a superare la difficolta' base l'Essenza non avra' effetto e manifestazione.

In sintesi l'incantatore deve superare, con il suo check su Competenza Magica, la difficolta' data dai parametri base (Area di Effetto, Distanza, Durata, impedimenti..) se supera questo valore ha di sicuro ottenuto il valore minimo di effetto, se invece lo supera di xx o piu' ottiene effetti maggiori.

\bigskip

 \textbf{Un incantatore puo' sempre scegliere un Livello di potere
inferiore rispetto a quello ottenuto.}

Ad esempio voglio shockare con l'elettricita' un avversario:

Distanza: nasce dal palmo della mano, ovvero ha come distanza massima mischia (tocco), difficolta' +0

Area di Effetto: un solo obiettivo, difficolta' +1

Durata: 0, istantanea, difficolta'+0

La difficolta' base e' quindi 1, Se con la prova di magia ottengo 8 (o comunque 1 o piu') avro' il minimo effetto, ovvero uno shock da 1d6 di danno.

\bigskip

\textbf{Un incantatore puo' formulare nel giorno un numero di Essenze pari a (CM/2)+3.}\index{formulare nel giorno}

\textbf{Se nel lancio di una Essenza ottiene almeno un critico (esplosione di magia) non si computa questa Essenza per il numero di Essenze lanciabili al giorno.}

Come si evince nessun incantatore ha il perfetto controllo delle Essenze dato che non puo' controllarne a pieno la forza.

Portare un armatura senza le dovute competenze ed Abilita' rende piu' difficile la prova di Competenza Magia. Vedere il capitolo armature per le penalita' relative.

\subsubsection{Resistenza alla Magia}\index{Resistenza alla Magia}

Una creatura potrebbe avere una naturale resistenza alle Essenze, in ogni forma si presenti.

Il valore di RM (Resistenza Magia) indica tale resistenza e piu' e' alta piu' la creatura e' immune alle Essenze, che lo voglia o meno.

Ogni qual volta la creatura e' influenzata direttamente da una Essenza deve effettuare una prova di RM, ovvero tirare 3d6 sommare il valore di RM e se e' superiore alla prova di magia effettuata dall'incantatore l'Essenza non ha effetto.

In caso di essenze scaturite da oggetti (anelli, bastoni, pozioni)
la prova di RM deve superare 6+LP (livello di potere) della Essenza
generata per annullarne gli effetti.

\subsubsection{L'esplosione del 6 nella Magia}\index{esplosione del 6 nella Magia}

\label{lesplosione-del-6-nella-magia}

Anche nella prova di Competenza Magica i 6 esplodono, ma in maniera diversa.

I 6 tirati nella prova di CM vengono ritirati, e ritirati ancora nel caso, ma i successivi 6 NON sono sommati alla prova.

Similarmente al Tiro per Colpire, ogni due 6 tirati si aumenta di uno il livello di potere ottenuto.

Es. Tups vuole incenerire l'orchetto che lo sta caricando. La sua prova di Competenza Magica e' data da 3d6 + 7. Tira con i dadi 6, 4, 3. Quindi la sua prova ha un totale di 20.

Ritira poi il 6 ed ottiene un altro 6, ritira anche questo e ottiene un altro 6! La situazione e' decisamente esplosiva!!! Ritira ancora e ottiene un 2.

Con la prova di CM a 20, data la distanza entro 10 metri (costo 2), la selezione (1 soggetto) il danno e' 5d6 ma avendo fatto ben tre 6 nel tiro il livello di potere aumenta di 1, arrivando il danno a ben 7d6. L'orchetto e' incenerito a dovere!

Per le prove di Competenza Magica l'uno non viene conteggiato conta 0.

\subsubsection{Tentare la sorte con la Magia}\index{Tentare la sorte con la Magia}

\label{tentare-la-sorte-con-la-magia}

Anche nella prova di competenza magica puoi Tentare la Sorte, ovvero rinunci ad un +4 di bonus (da CM, Intelletto...) e aggiungi un d6 in piu' nel tiro della prova.

\subsection{Resistere all'Essenza (Tiro Salvezza)}\index{Resistere all'Essenza}\index{Tiro Salvezza}

\label{resistere-allessenza-tiro-salvezza}

Una volta che la prova di magia e' superata e quindi l'Essenza liberata, anche in base alla descrizione e note dell'Essenza, e' possibile dimezzare o annullare l'effetto dell'Essenza.

Il tiro salvezza richiesto, in base a quanto indicato nell'Essenza, ha difficolta' pari alla stessa prova superata dal incantatore +3 per ogni due 6 ottenuti nella prova.

Se il Tiro Salvezza riescie o fallisce di piu' di 10 (\textbf{successo critico} o \textbf{fallimento critico}) il Narratore potra' decidere di applicare svantaggi o vantaggi al risultato finale.\index{piu' di 10}

Nella descrizione delle Essenze e' indicato cosa succede in caso di riuscita o fallimento del Tiro Salvezza.

\subsection{Piu' Essenze nello stesso round}\index{Piu' Essenze nello stesso round}

Ad alti livelli un incantatore puo' usare i Livelli di Poteri inferiore con estrema facilita' fino a poter usare piu' Essenze nello stesso round.

L'incantatore puo' lanciare piu' Essenze nello stesso round purche' la somma dei Livelli di Potere usati non superi il suo punteggio in CM+Intelletto.

Questa capacita' non e' usufruibile prima di avere CM a 22

\subsection{Check di Concentrazione}\index{Check di Concentrazione}

Se l'incantatore viene colpito, severamente distratto o ``impedito'' mentre effettua una magia deve fare una prova di concentrazione per capire se e' in grado di mantenere e finalizzare la magia oppure perderla.

La prova di \textbf{Concentrazione} ha difficolta' pari al danno subito+10.

Se la prova viene richiesta perche' ``\textbf{distratto}'' (non e' facile lanciare Essenze a cavallo o mentre si corre, sotto un temporale..) prova da superare (Competenza Magica + Caratteristica correlata alla Essenza in uso + eventuali punteggi da altre Abilita') per lanciare con efficacia l'Essenza ha difficolta' DC 15.

\subsection{Un ultimo suggerimento}

L'ultimo consiglio e' infine rivolto specificatamente ai Narratore, lasciate che i giocatori si esprimano inventando nuove magie e manifestazioni curiose e poco ortodosse. Cercate di valutarne la correttezza ricordando che le Essenza e come sono descritte vogliono essere degli esempi. Lo scopo finale e' sempre e solo divertirsi.

\subsection{Lista delle Essenze}\index{Lista delle Essenze}\index{Essenze}

\textbf{Alterare} (Intelletto): la capacita' di alterare il corpo per dargli capacita' o aspetto diverse o superiori.\index{Alterare}

\textbf{Attacco} (Intelletto): la capacita' di utilizzare la magia per attaccare e fare danno\index{Attacco}

 \textbf{Charme} (Magnetismo): la capacita' di controllare pensieri
ed emozioni di altre creature\index{Charme}

\textbf{Convocazione} (Intelletto): la capacita' di chiamare l'archetipo
della creatura.\index{Convocazione}

\textbf{Creazione} (Volonta): la capacita' di creare oggetti, materiali o elementi liberi\index{Creazione}

\textbf{Cura} (Volonta): la capacita' di curare esseri viventi o oggetti\index{Cura}

\textbf{Difesa} (Magnetismo): la capacita' di proteggersi contro il danno, magico o normale\index{Difesa}

\textbf{Distruzione} (Volonta): la capacita' di distruggere oggetti, materiali, elementi liberi o creature o anche equilibri organici\index{Distruzione}

\textbf{Illusione} (Magnetismo): la capacita' di produrre illusioni piu' o meno reali e complesse

\textbf{Movimento} (Agilita)': la capacita' di influenzare qualsiasi tipo di movimento quali il volo, levitazione, movimento del corpo o muovere oggetti.\index{Movimento}

\textbf{Protezione} (Potenza): la capacita' di proteggere da veleni, malattie, controllo del pensiero, elementi, dall'ambiente..\index{Protezione}

\textbf{Rivelazione} (Magnetismo): la capacita' di aumentare la consapevolezza nel proprio ambiente e utilizzare la magia per osservazione e divinazione\index{Rivelazione}

\textbf{Trasformazione} (Potenza): la capacita' di trasformare un elemento o creatura in un altro elemento e /o creature\index{Trasformazione}

\bigskip

Suggerisco di segnarsi nella scheda le Essenze e formulazioni piu' usate, quasi a creare un proprio libro di magia cosi' che sia piu' facile calcolare i costi degli incantesimi tipici.

\pagebreak

\subsection{Essenza Alterare -- Intelletto}\index{Essenza Alterare}

\label{essenza-alterare---intelletto}

\textbf{Alterare} e' la capacita' di \textbf{donare capacita' non possedute} ad una creatura naturale o magica, possano essere delle ali, delle branchie o una pelle resistente al fuoco, soffiare ghiaccio come un Drago o avere le ali di un Pegaso ed il corno di un Narvalo.

\bigskip

\textbf{Essenza Alterare}
\begin{itemize}
\item 
Al target viene concesso un Tiro Salvezza su Arbitrio per negare gli effetti 
\item 
Per modifiche minori si intende: aspetto (occhi, bocca, naso, capelli), respirazione, forme di attacco naturali 
\item 
Per modifica maggiori si intende: razza, sesso, movimento 
\item 
Per modifica superiori si intendono capacita' magiche (movimento/attacco..) 
\end{itemize}

\bigskip

\begin{tabular}{clc}
\toprule 
Livello di Potere & Creature Naturali\tabularnewline
<=11 & Concedi al target la Visione Crepuscolare a distanza di 18 metri \\
13 & - Concedi al target un +1 in una caratteristica\tabularnewline
& - Concedi al target la Visione Crepuscolare a distanza di 36 metri\tabularnewline
16 & Concedi al target un +2 in una caratteristica\tabularnewline
19 & - Respiri sott'acqua\tabularnewline
& - Concede al target una modifica minore del corpo.\tabularnewline
& - Concedi al target delle ali. +1 Azione Movimento, manovrabilita’ bassa\tabularnewline
22 & Concedi al target di potersi adattare ad un ambiente di fuoco.\tabularnewline
& - Concede al target due modifiche minore del corpo.\tabularnewline
& - Concedi al target delle ali. +2 Azioni Movimento , manovrabilita’ media\tabularnewline
25 & - Concedi al target di adattarsi ad un elemento di origine naturale\\
&dall’Essenza Attacco. \tabularnewline
& - Concede al target una modifica maggiore del corpo ed una minore\tabularnewline
& - Concedi al target delle ali. +4 Azioni Movimento , manovrabilita’ alta\tabularnewline
28 & Concede al target due modifiche maggiori del corpo e due minori\tabularnewline
31 & Concede al target una modifica superiore e due maggiori\tabularnewline
\bottomrule
\end{tabular}

\bigskip

Esempi:
\begin{itemize}
\item 
``Con il potere della natura. Questo ragno mi concedera' la sua tela' 
\item 
``Per il soffio del grande Gurthok. Possa io soffiare fiamme'' .
Solo dal round successivo potro' soffiare fiamme e potra' sfruttare
l'Alterazione come forma di Attacco di livello di difficolta' immediatamente
inferiore a quello ottenuto per l'Alterazione. 
\end{itemize}

\pagebreak 

\subsection{Essenza Attacco -- Intelletto}\index{Essenza Attacco}

\label{essenza-attacco---intelletto}
\begin{itemize}
\item 
Essenza \textbf{Attacco significa generare Energia come attacco contro
l'avversario.}\\
\textbf{}Va sempre specificato la forma di energia con cui si attacca
ed eventualmente verificato con gli elementi di Attacco del Patrono. 
\item 
Al target viene concesso un Tiro Salvezza su Riflessi per dimezzare
il danno. In caso di \textbf{successo critico} si dimezza ulteriormente. In
caso di fallimento critico si raddoppiano i danni. 
\item 
La Durata massima di un Essenza di Attacco e' sempre istantanea, non
puoi affogare un soggetto semplicemente riempiendo la stanza con un
attacco ad acqua, ne puoi creare una Fuocopalla Ardente che brucia
per un minuto 
\item 
Se si Attacca con energia negativa un non morto lo si cura, con energia
positiva lo si danneggia 
\item 
Se si Attacca con energia positiva un vivente non gli si fa nulla
(non lo si cura), con energia negativa lo si danneggia 
\item 
Se si attacca con una forma di Energia che ha il proprio Patrono o
energia neutrale (-) il danno e' quello riportato in tabella, altrimenti
si ottiene il Livello di Potere inferiore a quello determinato. 
\end{itemize}

\bigskip

\begin{tabular}[c]{@{}ll@{}}
\toprule 
Livello di Potere & Luce(P), Energia Positiva(P)\\
&Fuoco(-) Elettricita'(-) \\
&Suono(N), Energia Negativa(N)\tabularnewline
<=11 & 1d6\tabularnewline
13 & 2d6\tabularnewline
16 & 3d6\tabularnewline
19 & 5d6\tabularnewline
22 & 7d6\tabularnewline
25 & 10d6\tabularnewline
28 & 13d6\tabularnewline
31 & 15d6\tabularnewline
34 & 18d6\tabularnewline
37 & 20d6\tabularnewline
43 & 25d6\tabularnewline
\bottomrule
\end{tabular}

P = Energia Positiva, - = Energia neutra, N = Energia Negativa
\bigskip

Esempi:
\begin{itemize}
\item 
``Mani brucianti di Alac Zalzir'' 
\item 
``Invoco i demoni dei ghiacci che infilzino i miei avversari nelle loro gelide lance'' 
\item 
``Canto i segreti riti di Zungur e rompo il dito secco della pettegola perche' la mia voce tramortisca i miei avversari'' 
\item 
``Traccio nell'aria le antiche rune di Boz Dan Don e tre lame di acciaio trafiggano i nemici'' 
\item 
``Per tutte le battaglie: Palla di Fuoco!'' 
\item 
``IT'S OVER 9000!'' 
\end{itemize}
\bigskip

Esempio pratico:

Un attacco di Luce causa danno suddiviso equamente da calore (assimilabile a fuoco) e da energia magica.

\subsubsection{Mani brucianti di Alac Zalzir}

Distanza: esce dal palmo della mano, ovvero ha come distanza massima e' mischia, difficolta' +0

Area di Effetto: un solo obiettivo, difficolta' +1

Durata: 0, istantanea, difficolta'+0

Quindi fatte le somme (Distanza + AoE + Durata) questa versione di Mani Brucianti ha difficolta' 1.

Se con la prova faccio 15 ottengo un livello di potere pari a 14 (15-1) che significa che le mie Mani Brucianti fanno 2d6 di danno

\subsubsection{Fuocopalla Ardente}

Distanza: entro i 10 metri, difficolta +2

Area di Effetto: distanza 3 metri radius, difficolta +4

Durata: istantanea, difficolta' 0

I costi base sono 6, se con la prova faccio 24, avro' un livello di potere pari a 18, sufficiente per fare 3d6 di danno!

\pagebreak

\subsection{Essenza Charme -- Magnetismo}\index{Essenza Charme}

\label{essenza-charme---magnetismo}
\begin{itemize}
\item 
L'Essenza Charme \textbf{agisce sull'attitudine della Creatura Naturale o Magica}. Il soggetto deve essere senziente e con Volonta' ed Intelletto maggiori o uguali a -2 
\item 
Essenza di Charme permette anche di comunicare non verbalmente. Attenzione alla difficolta' data da Durata e Target e Distanza 
\item 
Non si puo' usare l'Essenza di Charme su creature con 3 CR superiori alla propria CM se l'obiettivo non vuole. 
\item 
I CR indicati si riferiscono alla sommatoria di creature influenzate. 
\item 
Al target viene concesso un Tiro Salvezza su Arbitrio per negare gli effetti. In caso di fallimento critico la durata viene raddoppiata. 
\end{itemize}

\bigskip

\begin{dndtable}[L{3.5cm} L{13cm}]
Livello di Potere & Creature Naturali o Magiche\tabularnewline
<=11 &Influenzi obiettivi fino a 1/3 CR
\\
13 & Influenzi obiettivi fino a 1 CR
\\
& - Comunichi con una creatura non piu' di 144 telepaticamente che capisca la tua lingua.
\\
16 & - Influenzi obiettivi fino a 2 CR
\\
& - Comunichi con una creatura telepaticamente che capisca la tua lingua.
\\
19 & - Influenzi obiettivi fino a 3 CR
\\
& - Comunichi con una creatura telepaticamente che non capisca la tua lingua ed abbia Intelletto 2 o piu'
\\
22 & - Influenzi obiettivi fino a 5 CR
\\
& - Comunichi con una creatura telepaticamente che non capisca la tua lingua \\
& e abbia Intelletto 1 o piu'.
\\
25 & Influenzi obiettivi fino a 7 CR
 \\
28 & - Influenzi obiettivi fino a 9 CR
\\
& - Comunichi con un obiettivo telepaticamente che non comunichi verbalmente. \\
31 & Influenzi obiettivi fino a 15 CR
\\
34 & Influenzi obiettivi fino a 11 CR
\\
37 & Influenzi obiettivi fino a 13 CR
\\
43 & Influenzi obiettivi fino a 15 CR\\
\bottomrule
\end{dndtable}

\bigskip

Un incantatore puo' utilizzare un Essenza di Charme per influenzare e quindi rendere Amichevoli od Impaurire, a seconda della differenza tra la CM dell'incantatore e CR dell'obiettivo si possono avere effetti diversi.

\bigskip

Se la creatura ha:

\begin{dndtable}[L{3.5cm} L{6.5cm} L{6.5cm}]
 & Rendere Amichevole & Impaurire\tabularnewline
CM-CR \\
e' 1 o piu' & fallimento di 2 o meno la creatura e' amichevole & se il TS fallisce di 2 o meno la creatura e' scossa\tabularnewline
& 3 la creatura e' affascinata & 3 la creatura e' spaventata\tabularnewline
& 4 la creatura e' charmata & 4 la creatura e' in preda al panico\tabularnewline
& se il TS di 5 o piu' la creatura e' dominata & \tabularnewline
CM-CR \\
tra 0 e -1 & 3 o meno la creatura e' amichevole & 3 o meno la creatura e' scossa\tabularnewline
& 4 la creatura e' affascinata & 4 la creatura e' spaventata\tabularnewline
& 5 la creatura e' charmata & 5 la creatura e' in preda al panico\tabularnewline
& se il TS di 6 o piu' la creatura e' dominata & \tabularnewline
CM-CR\\
tra -2 e -3 & 4 o meno la creatura e' amichevole & 4 o meno la creatura e' scossa\tabularnewline
 & 5 la creatura e' affascinata & 5 la creatura e' spaventata\tabularnewline
 & 6 la creatura e' charmata & 6 o piu' la creatura e' in panico\tabularnewline
 & se il TS fallisce di 7 o piu' la creatura e' dominata & \tabularnewline
\end{dndtable}

}
\bigskip

Esempi:
\begin{itemize}
\item 
``Per il potere di Gaya. Possa il mio tocco renderti docile'' 
\item 
``Racconto la storia della stupenda Aralda Hucnoss e fisso gli occhi dell'orco. Ora sei mio, dolce amore'' 
\item 
``Sembra talco ma non e', serve a darti l'allegria. Se lo mangi o lo respiri ti da subito l'allegria!'' 
\end{itemize}

\pagebreak

\subsection{Essenza Convocazione -- Intelletto}\index{Essenza Convocazione}


\label{essenza-convocazione---intelletto}

L'Essenza Convocazione e' la \textbf{capacita' di richiamare l'archetipo di una Creatura Naturale o Magica per farla agire al tuo fianco.}
\begin{itemize}
\item 
Il CR indicato si riferisce alla sommatoria dei CR totali convocati di creature naturali 
\item 
Un incantatore non puo' evocare creature naturali con piu' di 3 CR rispetto al suo valore di CM 
\item 
Creatura magiche convocate possono avere al massimo CR pari al CM-3, con un massimo di CR 7 
\item 
Evocare una creatura magica o Elementale costa a parita' di CR il livello di potere immediatamente successivo. Con LP 22 evochi una creatura magica con CR 3 
\item 
Evocare un Drago a parita' di CR e' piu' difficile di 2 livelli. Con LP 31 evochi un drago CR 7 
\item 
CR 10 e' il massimo livello di creature naturali evocabile, oltre sono sempre sommatorie di piu' creature evocate. 
\end{itemize}

\bigskip

\begin{tabular}[c]{@{}ll@{}}
\toprule 

Livello di Potere & Creature\tabularnewline
\textless=11 & Convochi fino a 1/3 CR\tabularnewline
13 & Convochi fino a 1/2 CR\tabularnewline
16 & Convochi fino a 1 CR\tabularnewline
19 & Convochi fino a 3 CR\tabularnewline
22 & Convochi fino a 5 CR\tabularnewline
25 & Convochi fino a 7 CR\tabularnewline
28 & Convochi fino a 9 CR\tabularnewline
31 & Convochi fino a 11 CR, max CR 9\tabularnewline
34 & Convochi fino a 13 CR, max CR 9\tabularnewline
37 & Convochi fino a 15 CR, max CR 10\tabularnewline
43 & Convochi fino a 17 CR, max CR 10\tabularnewline
\bottomrule
\end{tabular}

\bigskip

Esempi:
\begin{itemize}
\item 
``O sommi antenati concedetemi la sapienza di richiamare il mammuth lanoso'' 
\item 
``Dalle cime delle montagne piu' alte urlo il richiamo di Ferlin Caf. A me venga il Drago di bronzo'' 
\item 
``Lime, Rum, Ghiaccio, Menta e Zucchero grezzo. Agito e offro. Io convoco il Pirata Verdemarcio'' 
\item 
``Conchiglie, grasso di pecora e nessun nome. Qui voglio il Ciclope'' 
\end{itemize}

\pagebreak

\subsection{Essenza Creazione -- Volonta'}\index{Essenza Creazione}

\label{essenza-creazione---volonta}

Creare e' l'\textbf{atto di plasmare la magia per creare Elementi e manifestare dal nulla un elemento od oggetto}.

Non si possono creare Creature (naturali o magiche) secondo la regola che non si puo' creare Vita. Quando si usa l'Essenza Creare per richiamare un Elementale in realta' si deve usare l'Essenza della Convocazione.

\begin{itemize}
\item Creare un elemento non e' congiurare un Elementale. Crei una struttura entro le dimensioni e pesi limite indicati. 
\end{itemize}

\begin{itemize}
\item Se si vuole creare un Unione di Elementi (cibo, ottone, lava..) la quantita' e volumi prodotti sono inversamente proporzionali alla complessita' dell'elemento. Piu' e' complesso l'elemento creato meno ne puoi creare. In base alla complessita' e precisione dell'oggetto da creare diminuire quantita' e volumi. 
\item E' possibile creare piu' oggetti contemporaneamente. Calcolata la somma dei volumi si prende la difficolta' immediatamente superiore. 
\item Non e' possibile creare qualcosa all'interno di creature vive. 
\item Non e' possibile creare qualcosa dove non vedi. 
\item Un muro di ghiaccio (ad esempio) non potra' fare danno immediatamente ma solo dal round successivo. Si deve considerare che il muro si manifesti in accrescimento nel round che viene creato. Il danno e' pari all'Essenza di Attacco immediatamente inferiore al Livello di Potere utilizzata per l'Essenza di Creazione. 
\item In caso di oggetti che cadono, il danno e' quello immediatamente inferiore al Livello di Potere nell'Essenza di Attacco. L'Area di Effetto e' quella stabilita' dai costi base. 
\item La durata se non contata come costo e' di 1 round 
\item Se si crea qualcosa di intangibile (es. Luce) e che non fa danno, la durata ha costo dimezzato. 
\item Se si crea materia solida attorno al target ed entro distanza di 3 metri (o il doppio della sua portata se e' maggiore) dallo stesso, viene concesso un Tiro Salvezza su Riflessi per uscire dalla creazione prima che questa sia completa. 
\item La massa creata si distribuisce in blocchi minimi di 1 metro di lato (il cubo base) connessi tra loro. 
\item La massa creata obbedisce alle leggi delle fisica quando possibile. Un muro d'acqua cade il round successivo alla creazione. 
\item Un cubo base e' un cubo di lato 1 metro 
\end{itemize}

\begin{tabular}[c]{@{}ll@{}}
\toprule 
Livello di Potere & Assenza o Presenza\tabularnewline
\textless=11 & Un cubo di lato fino a 10 cm e 100 grammi (minuscola)\tabularnewline
13 & Un cubo di lato fino a 20 cm o 500gr (piu' piccola)\tabularnewline
16 & Un cubo di lato fino a 0.5 metro o 3kg (piccola)\tabularnewline
19 & Un cubo di lato fino a 1 metro (cubo base) o 25kg (piccola)\tabularnewline
22 & Fino a 2 cubi base e 100kg (media) di peso\tabularnewline
25 & Fino a 4 cubi base e 200kg (grande)\tabularnewline
28 & Fino a 8 cubi base e 400kg (enorme)\tabularnewline
31 & Fino a 16 cubi base e 800kg (enorme)\tabularnewline
34 & Fino a 32 cubi base e 1.6 tonnellate (mastodontica)\tabularnewline
37 & Fino a 64 cubi base e 3.2 tonnellate (mastodontica)\tabularnewline
43 & Fino a 128 cubi base e 6.4 tonnellate (colossale)\tabularnewline
\bottomrule
\end{tabular}

\bigskip

Esempi:
\begin{itemize}
\item `Evoco il grande spirito di Lunzac sommo artigiano reale perche' crei un comodino di perfetta fattura' 
\item ``Chiedo alle possenti maree del mare del sud di riempire d'acqua il campo di battaglia' 
\item ``Per tutti le braci infernali che questo fuoco risplenda nella notte'' 
\end{itemize}

\bigskip

Esempi pratici
\begin{itemize}
\item Per creare una Luce equivalente ad una torcia e' sufficiente la difficolta' 11 ( raggio 3 metri), con difficolta' 13 si crea l'equivalente di una lanterna (raggio 6 metri di luce) 
\item Mentre l'Essenza Distruzione crea oscurita' perche' distrugge la luce, tramite l'Essenza Creazione e' possibile creare oscurita' o luce magica. 
\item Con difficolta' 20 puoi creare Cibo per 1 persona per 1 giorno (comprensivo di costi base) 
\item Es. creo un muro di ghiaccio fatto di 4 cubi base (2m lunghezza{*}1m larghezza{*}2m altezza. Livello potere 25) fara' danno per chi lo attraversa (o se gli cade sopra) pari a livello di potere 22 in Essenza Attacco (7d6). 
\item Suggerisco di tenere a disposizione dei cubetti 2x2 Lego per costruire ``visivamente'', come cubi base, le proprio creazioni 
\end{itemize}

\pagebreak

\subsection{Essenza Cura -- Volonta'}\index{Essenza Cura}

\label{essenza-cura---volonta}

L'Essenza della Cura e' la \textbf{capacita' di riempire il vuoto causato da una Distruzione} o \textbf{sanare le ferite di un Attacco.} L'Essenza di Cura agisce su Vita. puo' essere la cura di energie vitali, degli organi, di stati biologici e mentali.


\begin{itemize}
\item 
La Durata di un Essenza di Cura e' sempre e solo istantanea, tranne se specificato diversamente 
\item 
Una Essenza di Cura usata su un non morto equivale a causargli un danno pari all'ammontare che l'Essenza di Attacco avrebbe causato. Un Tiro Salvezza su Tempra puo' dimezzare i danni. Un successo critico li dimezza ulteriormente. 
\item 
L'Essenza di Cura ha un Area di Effetto sempre a target e mai a raggio. Se sono indicati piu' target questi devono essere entro un metro l'uno dall'altro. 
\end{itemize}

\begin{tabular}[c]{@{}ll@{}}
	\toprule 
Livello di Potere & Presenza (solo Vita)\tabularnewline	
<=11 & - Rimuovi la condizione abbagliato.
\\
& - Curi 1d6 pf
\\
13 & - Rimuovi la condizione Frastornato
\\
& - Curi 2d6 pf
\\
& - Crei un link vitale tra te ed un target.\\
& Puoi condividere i tuoi punti ferita
con le creature collegate.\\
& Durata 10 minuti. Costa 1 azione mantenere ed
usare questa condivisione.
 \\
16 & - Rimuovi la condizione Affaticato / Scosso / Infermo / Nauseato.
\\
& - Curi 4d6 pf, puoi dividere la cura fino a 2 target
\\
& - Crei un link vitale tra te e fino a 3 target. Puoi condividere i tuoi punti ferita\\
& con le creature collegate. Durata 1 ora.\\
& Costa 1 azione mantenere ed usare questa condivisione.\\
19 & - Rimuovi la condizione di Malato / Accecato / Assordato / Confuso / Esausto\\
& - Ristori 2 punti ad una Caratteristica\\
& - Curi 6d6 pf, puoi dividere la cura fino a 3 target\\
& - Crei un link vitale tra te e fino a 4 target.\\
& Puoi condividere i tuoi punti ferita
con le creature collegate.\\
& Durata 1 ora. Costa 1 azione mantenere ed usare
questa condivisione.
\\
22 &- Rimuovi la condizione di Avvelenato
\\
& - Ristori 3 punti divisi su piu' Caratteristiche
\\
& - Curi 9d6 pf, puoi dividere la cura fino a 6 target
\\
&- Crei un link vitale tra te e fino a 5 target.\\
& Puoi condividere i tuoi punti ferita
con le creature collegate. Durata 1 ora.\\
& Costa 1 azione mantenere ed usare
questa condivisione.
\\
25 & - Rinsaldi fratture
\\
& - Recuperi tutti i punti caratteristica
\\
& - Curi 12d6 pf, puoi dividere la cura fino a 9 target
\\
& - Crei un link vitale tra te e fino a 7 target.\\
& Puoi condividere i tuoi punti ferita
con le creature collegate. Durata 1 ora. \\
& Costa 1 azione mantenere ed usare
questa condivisione.
\\
28 &- Curi fino a 60 pf e tutte le malattie.
\\
& - Ristori un livello temporaneo perso
\\
31 & - Rigeneri i tessuti ed arti
\\
& - Ristori un livello permanente perso
\\
& - Curi 16d6 pf, puoi dividere la cura fino a 10 target
\\
34 & - Ringiovanisci il target di 3d6 anni
\\
& - Curi 20d6 pf, puoi dividere la cura fino a 16 target
\\
& - Curi completamente tutte le ferite e condizioni di un target
\\
37 & Curi il target di tutte le condizioni, punti caratteristica, \\
& livelli e punti ferita,
 ringiovanisci il target di 3d6 anni\\
40 & Sacrifichi la tua vita per portare in vita un’altra creatura.\\

\bottomrule
\end{tabular}

\bigskip


Esempi:
\begin{itemize}
\item 
''Grande Ljust protettrice di cio' che e' vivo concedimi di lenire
le sofferenze di questo bravo uomo.'' 
\item 
``Che la mano del guaritore curi le tue malattie'' 
\item 
``Possa questo bacio purificarti'' 
\item 
``Con l'aiuto degli antichi sacerdoti il tuo spirito sia ripristinato'' 
\item 
``Possa la spada dei grandi guerrieri infonderti l'energia che queste empie creature ti hanno tolto'' 
\end{itemize}

\bigskip

Esempio Pratico

\subsubsection{Mano calda di Ljust}

Distanza: mischia, costo 0

Area di Effetto: 1 target, costo 0

Durata: Istantenea, costo 0

E’ necessario una prova di CM pari a 13 per ottenere una Mano calda di Ljust che curi su un target
2d6 punti ferita. La difficolta’ e’ 0+0+0+13 = 13

\subsubsection{Sfera curativa di Ljust}

Distanza: 10 metri, difficolta' +2

Area di Effetto: 5 creature da includere, +5

Ricordiamoci che la Cura non puo' essere usata come area di effetto sferica, il costo e' per poter selezionare le persone da curare.

Durata: istantanea, difficolta' +0

Essenza Cura: 19, cura 9d6, puoi dividere la cura fino a 6 target

Fatte tutte le somme una Sfera curativa di Ljust da 9d6 ha difficolta' +2+5+0+19 = 26

\pagebreak

\subsection{Essenza Difesa -- Magnetismo}\index{Essenza Difesa}

\label{essenza-difesa---magnetismo}

l'Essenza di Difesa \textbf{permette di creare delle barriere/scudi/armature che possono proteggere genericamente dal danno o da un elemento definito}. L'Essenza di Difesa si applica su Creature o Elementi. La durata deve essere calcolata a parte, qui viene indicato solo il massimo della protezione concessa o a round.
\begin{itemize}
\item 
Al target viene concesso un Tiro Salvezza su Tempra per negare gli effetti. 
\item 
Essenza di Difesa puo' essere usata come controincantesimo per l'Essenza di Attacco. E' necessario superare con una prova di competenza magica la prova di competenza magica dell'avversario. Si annulla solo l'effetto su un target (se stesso o altro). Si consuma un utilizzo di Essenze. 
\item 
Se ci sono piu' Essenze di Difesa attive non si sommano i tipi di bonus equivalenti, si tiene solo quello che fornisce il bonus piu' alto. 
\end{itemize}

\bigskip

\begin{tabular}[c]{@{}ll@{}}
\toprule 
Livello di Potere & Creature, Presenza\tabularnewline
<=11 & - +1 Difesa\\
& - +1 Tiri Salvezza \\
13 & Crei una difesa che protegge per 4 Punti Ferita in tutto \\
16 & - Crei una difesa da un elemento per 3 Punti Ferita a round \\
& - +2 ad un Tiro Salvezza\\
& - +3 Difesa\\
19 & - Crei una difesa da un elemento per 5 Punti Ferita a round\\
& - +2 a tutti i Tiri Salvezza\\
& - +4 Difesa, +2 ad un Tiro Salvezza\\
22 &- Crei una difesa per 6 Punti Ferita a round\\
& - Crei una difesa da un elemento specifico\\
&anche magico per 60 Punti Ferita in tutto\\
& - Barriera verso gli insetti normali\\
25 &- Crei una difesa per 8 Punti Ferita a round\\
&- Immune ad un elemento non magico\\
&- +6 Difesa\\
28 &- Resistenza ad un Energia da Essenza Attacco o naturale\\
&- Barriera verso le piante normali\\
&- +6 ad un Tiro Salvezza\\
&- Immune a livello di potenza 11 di Attacco\\
31 &- +6 a tutti i Tiro Salvezza\\
&- Barriera verso gli animali normali\\
&- Immune a livello di potenza 16 di Attacco / Trasformazione\\
&- Crei una barriera intorno a te, che ti scherma\\
& da tutti gli attacchi fisici non magici\\
34 &- Barriera verso gli animali magici\\
&- Crei una barriera intorno a te che ti scherma\\
&da tutti gli attacchi fisici anche magici\\
&- Immune a livello di potenza 22 di Attacco / Trasformazione / Distruzione\\
37 &Immune a livello di potenza 28 di Attacco / Trasformazione / Distruzione\\
43 &Immune a livello di potenza 34 di Attacco / Trasformazione / Distruzione\\
\bottomrule
\end{tabular}

\bigskip

Esempi:
\begin{itemize}
\item 
``Chiamo la magia del grande cristallo perche' mi protegga contro
i miei avversari'' 
\item 
``Canto le invocazioni di morte. Possano le osse dei miei antenati
proteggermi'' 
\item 
``Incido il mio petto con le sacre rune di Qizdo!'' 
\end{itemize}

\pagebreak

\subsection{Essenza Distruzione -- Volonta'}\index{Essenza Distruzione}

\label{essenza-distruzione---volonta}

\textbf{Essenza Distruzione -- Presenza}

La Distruzione di Elementi e' la \textbf{distruzione di Elementi}
\begin{itemize}
\item 
La distruzione di Elementi e' diretta, senza Tiro Salvezza. 
\item 
La distruzione si intente di singolo e specifico oggetto non di volumi di piu' oggetti, tranne se omogenei (es un tavolo di legno, la serratura di metallo..) 
\item 
La distruzione di materia e' sempre permanente come durata. La durata ha difficolta' 8 + eventuale contingenza. 
\end{itemize}

\bigskip

\begin{tabular}[c]{@{}ll@{}}
\toprule 
Livello di Potere & Presenza (tranne Vita)\tabularnewline
\textless=11 & Un cubo di lato fino a 10 cm e 100 grammi (minuscola)\tabularnewline
13 & Un cubo di lato fino a 20 cm o 500gr (piu' piccola)\tabularnewline
16 & Un cubo di lato fino a 0.5 metro o 3kg (piccola)\tabularnewline
19 & Un cubo di lato fino a 1 metro (cubo base) o 25kg (piccola)\tabularnewline
22 & Fino a 2 cubi base e 100kg (media) di peso\tabularnewline
25 & Fino a 4 cubi base e 200kg (grande)\tabularnewline
28 & Fino a 8 cubi base e 400kg (enorme)\tabularnewline
31 & Fino a 16 cubi base e 800kg (enorme)\tabularnewline
34 & Fino a 32 cubi base e 1.6 tonnellate (mastodontica)\tabularnewline
37 & Fino a 64 cubi base e 3.2 tonnellate (mastodontica)\tabularnewline
43 & Fino a 128 cubi base e 6.4 tonnellate (colossale)\tabularnewline
\bottomrule
\end{tabular}

\bigskip

Il livello di Potere richiesto cambia anche in funzione del materiale che si va a distruggere.

\bigskip

Consultare la \textbf{Tabella Modificatore Distruzione Elementi} per verificare il moltiplicatore al Livello di Potere necessario in base al materiale che si vuole distruggere.

\bigskip

\textbf{Tabella Modificatore Distruzione Elementi}

\begin{tabular}[c]{@{}lll@{}}
\toprule 
Modificatore & Durezza & Esempio\tabularnewline
0.5 & Estremamente facile & Sabbia\tabularnewline
0.7 & Facile & Vetro / Acqua\tabularnewline
1 & Normale & Legno / terriccio\tabularnewline
1.1 & Difficile & Ceramica / Pietra / terra dura\tabularnewline
1.3 & Molto difficile & Ferro / Mattone\tabularnewline
1.5 & Estremamente difficile & Acciaio/ Mithral\tabularnewline
1.7 & Quasi impossibile & Acciaio Nanico\tabularnewline
2 & Inumana & Adamantio\tabularnewline
2.5 & Ultraterrena & Acciaio Nanico Runico\tabularnewline
5 & Divina & Artefatti\tabularnewline
\bottomrule
\end{tabular}

\bigskip

Se quindi si vuole dimostrare la propria potenza distruggendo una serratura di metallo semplicemente toccandola la difficolta' e' 11 (livello di potere) {*} 1.3 (ferro) = 14.3 = 14 di difficolta' piu' i fattori di base (8 di durata, distanza, contingenza...)

L'arrotondamento si fa all'unita' piu' vicina.

Esempi:
\begin{itemize}
\item 
``Per tutte le torce consumate. Buio!'' 
\item 
``Chiamo a me i terremoti passati. Il tuo castello crolli come la sabbia' 
\item 
``Getto a terra il sangue di un Ragnroll. Che una fossa ti colga!'' 
\end{itemize}


\textbf{Essenza Distruzione -- Creature}
\begin{itemize}
\item 
La distruzione di Creature Naturali o Creature Magiche causa la distruzione dell'equilibrio metabolico e mentale. 
\item 
Una Essenza di Distruzione non puo' mai causare danno diretto, si deve usare l'Essenza Attacco. 
\item 
Al target viene concesso un Tiro Salvezza su Tempra per negare gli effetti 
\item 
Non e' possibile distruggere ``parti'' di esseri viventi 
\item 
Su molte creature magiche le condizioni indicate non hanno effetto, tenere sempre conto della durata degli effetti. 
\end{itemize}

\bigskip

\begin{tabular}[c]{@{}ll@{}}
\toprule 
Livello di Potere & Creature\tabularnewline
<=11 &Attribuisci la condizione abbagliato. \\
13 & Attribuisci la condizione: Frastornato\\
16 & Attribuisci la condizione: Affaticato / Scosso / Infermo\\
19 & - Attribuisci la condizione di: Malato / Accecato\\
& / Assordato / Esausto / Nauseato\\
& - Diminuisci di 1 punto una caratteristica\\
22 & - Attribuisci la condizione di Avvelenato\\
& - Diminuisci di 2 punti una caratteristica\\
25 & - Distruggi l’armonia spirituale della creatura \\
&(-2 al colpire, al danno, ai Tiri Salvezza)\\
& - 3 AGI oppure -3 POT\\
28 & - Distruggi le ossa. -3 POT e AGI\\
& - Distruggi temporaneamente le esperienze, il target perde\\
& un livello di esperienza\\
31& - Distruggi i tessuti ed arti, causi la distruzione del corpo (TS o morte).\\
& - Distruggi permanentemente le esperienze, il target perde\\
&un livello di esperienza\\
34 &Invecchi il target di 3d6 anni\\

\bottomrule
\end{tabular}

\bigskip

Esempi
\begin{itemize}
\item 
`Evoco lo stregone Adbul Aziz. Mi conceda di fare marcire le tue interiora!'' 
\item 
``Grande martello, Grande martello, affonda la tua testa nella sua' 
\item 
``Osserva la spirale di Oman Gur Tha. Non sei mai stato cosi'' stanco'' 
\item 
``Oh Padrone Shayalia ti offro lo spirito del mio nemico'' 
\end{itemize}

\bigskip

\textbf{Essenza Distruzione - animazione dei morti}

Molti incantatori seguaci di Sixiser utilizzano i non morti per portare caos e distruzione nel creato

\begin{itemize}
\item 
L'Essenza di Distruzione (animazione) su Creature Naturali o Magiche non puo' concedere piu' dadi vita di quanti posseduti dalla creatura originale. Le caratteristiche di Volonta' e Intelletto e Magnetismo vengono ridotte ad un terzo, a meno di aumentare il livello di potere a quello superiore. 
\item 
Il CR indicato si riferisce alla sommatoria dei CR totali animati di creature naturali 
\item 
Un incantatore non puo' animare creature naturali con piu' di 3 CR rispetto al suo valore di CM 
\item 
Se si vuole animare una creatura magica la difficolta' passa a quella successiva e si somma con il costo del mantenimento delle caratteristiche mentali 
\item 
La difficolta' della Durata dell'animazione e' 8. Ed e' permanente
finche' la creatura animata non viene distrutta o dismessa dall'incantatore. 
\end{itemize}

\bigskip

\begin{tabular}[c]{@{}ll@{}}
\toprule 
Livello di Potere & Creature\tabularnewline
\textless=11 & Animi fino a 1/3 CR\tabularnewline
13 & Animi fino a 1/2 CR\tabularnewline
16 & Animi fino a 1 CR\tabularnewline
19 & Animi fino a 3 CR\tabularnewline
22 & Animi fino a 5 CR\tabularnewline
25 & Animi fino a 7 CR\tabularnewline
28 & Animi fino a 9 CR\tabularnewline
31 & Animi fino a 11 CR, max CR 9\tabularnewline
34 & Animi fino a 13 CR, max CR 9\tabularnewline
37 & Animi fino a 15 CR, max CR 10\tabularnewline
43 & Animi fino a 17 CR, max CR 10\tabularnewline
\bottomrule
\end{tabular}

\bigskip

Esempi: d'uso:
Creare una zona di Oscurita' equivale ad usare l'Essenza Distruzione sulla Luce. Considerate la grandezza dell'ambiente e la durata che vi serve

\pagebreak

\subsection{Essenza Illusione -- Magnetismo}\index{Essenza Illusione}

\label{essenza-illusione---magnetismo}

l'Essenza Illusione e' la \textbf{capacita' di creare immagini, suoni,
odori, profumi di cose che non esistono}.
\begin{itemize}
\item 
Una Illusione non puo' mai essere usata per ferire direttamente un avversario. 
\item 
Una illusione non offre mai resistenza fisica. 
\item 
In caso di creazione di proprie immagine una Essenza che causi danno ad area le distruggera' tutte. 
\item 
Un Tiro salvezza su Arbitrio permette di discernere l'illusione. 
\item 
E' sempre e solo sensoriale (uditiva, visiva, olfattiva), e non influenza mai il tocco. 
\item 
In base alla complessita' e precisione dell'oggetto diminuire quantita' e volumi. 
\item 
Per Allarme si intende un trigger, un ``sensore'' che puo' essere usato per attivare ulteriori Essenze. Questo trigger evita di pagare il costo della contingenza ovvero usi due Essenze, la prima per creare l'Allarme (che deve anche durare) e la seconda che si attiva a seguito
dell'Allarma. 
\end{itemize}

\bigskip

\begin{tabular}[c]{@{}ll@{}}
\toprule 
Livello di Potere & Dimensione\tabularnewline
<=11 &Un cubo di lato fino a 20 cm\\
13 &- Un cubo di lato fino a 50 cm\\
&- Crei 1d4 immagini illusorie di te. \\
&Ogni attacco a segno elimina prima una immagine.\\
16& - Un cubo di lato fino a 1 metro \\
&- Crei una illusione che ti rende sfocato, +3 CA\\
&- Crei 2d4 immagini illusorie di te.\\
& Ogni attacco a segno elimina prima una immagine.\\
&- Crei un Allarme sonoro che si attiva al passaggio\\
19& - Un cubo di lato fino a 2 metri\\
&- Crei una illusione che ti rende invisibile. Attaccare rende visibile.\\
&- Crei un Allarme sonoro che si attiva quando visto\\
22& -Un cubo di lato fino a 3 metri\\
& - Crei una illusione di te a 2 metri, +4 Difesa\\
& - Crei un Allarme sonoro che si attiva al passaggio o suono o se visto.\\
& - Crei un Allarme sonoro e visivo che si attiva al passaggio o se visto\\
25 &- Un cubo di lato fino a 4 metri\\
&- Crei una illusione che rende invisibile te \\
&e altre 3 creature di taglia media. Attaccare rende visibile.\\
&- Crei un Allarme sonoro e visivo che si attiva \\
&se ci sono creature anche invisibili\\
28 & Un cubo di lato fino a 6 metri\tabularnewline
31 & Un cubo di lato fino a 10 metri\tabularnewline
34 & Un cubo di lato fino a 15 metri\tabularnewline
37 & Un cubo di lato fino a 20 metri\tabularnewline
43 & Un cubo di lato fino a 50 metri\tabularnewline
\bottomrule
\end{tabular}

\bigskip

Esempi:
\begin{itemize}
\item 
``Dal Deserto di Darnhub chiamo il miraggio di una bellissima donna' 
\item 
``Lasciami proteggere la porta. Creo questa campanellina perche' suoni ad ogni passaggio'' 
\item 
``Incido la runa di Abildan. Chiunque passi da qui fara' suonare le trombe di Torricelli'' 
\end{itemize}

\pagebreak

\subsection{Essenza Movimento -- Agilita'}\index{Essenza Movimento}

\label{essenza-movimento---agilita}

L'Essenza di Movimento significa \textbf{potersi teletrasportare e spostarsi di piani nonché' alterare la velocita' e spostare le cose}.

\textbf{Essenza Movimento (Spostare) -- Creature}

\begin{itemize}
\item 
La Creatura Naturale o Magica che viene influenzata dell'Essenza di Movimento deve avere taglia media o inferiore. 
\item 
Per ogni creatura oltre la prima ed anche per taglia superiore alla media la difficolta' aumenta di 2. In caso di taglia superiore alla media e creatura oltre la prima i costi si sommano. Quindi spostare 3 creature grandi a distanza di kilometri ha difficolta' 19+2{*}2 (2 creature oltre la prima) +2{*}3 (3 creature di taglia oltre la media)= 29. 
\item 
Al target viene concesso un TS su Potenza per resistere agli effetti 
\item 
La durata e' sempre istantanea 
\end{itemize}

\bigskip

\begin{tabular}[c]{@{}ll@{}}
\toprule 
Livello di Potere & Creature\tabularnewline
<=11 & - Ci si puo'’ spostare entro raggio di 3 metri\\
&- Caduta piuma\\
13& - Ci si puo'’ spostare entro raggio di 12 metri\\
&- Levitazione\\
16& - Ci si puo'’ spostare entro raggio di 100 metri\\
&- Si diventa eterei\\
&- Volare\\
19& - Ci si puo'’ spostare entro 5 km\\
&- Si puo'’ passare da un piano all’altro\\
22 & - Ci si puo' spostare entro 20 km\tabularnewline
25 & - Ci si puo' spostare entro 100 km\tabularnewline
28 & - Si puo' coprire una distanza di 200 km\tabularnewline
\bottomrule
\end{tabular}

\bigskip

Esempi:
\begin{itemize}
\item 
``Lucido la punta degli stivali e sbatto i tacchi. E sono dove voglio'' 
\item 
``Per i grandi rapaci possa io arrivare in cima alla montagna' 
\item 
``Banchetti e monete, specchietti e cadute. Portatemi a palazzo Dornean'' 
\end{itemize}

\bigskip

\textbf{Essenza Movimento - Concetto}

Alterare il Movimento e' alterare la velocita' di azione di una creatura.
Questo concede di muoversi piu' velocemente o attaccare piu' volte o riuscire ad agire piu' volte nel round.

Al target viene concesso un Tiro Salvezza su Arbitrio per negare gli effetti.

\bigskip

\begin{tabular}[c]{@{}ll@{}}
\toprule 
Livello di Potere & Concetto\tabularnewline
\textless=11 & Il target ottiene una Azione di movimento bonus\tabularnewline
13 & Il target ottiene due Azioni di movimento bonus\tabularnewline
16 & Il target ottiene un bonus di una Azione\tabularnewline
19 & Il target ottiene un bonus di una Azione ed una Azione di movimento\tabularnewline
22 & Il target ottiene due Azioni in piu'\tabularnewline
25 & Il target ottiene tre Azioni in piu'\tabularnewline
28 & Il target puo' usare due abilita' a disposizione (ma non puo' lanciare
2 Essenze)\tabularnewline
\bottomrule
\end{tabular}

\bigskip

Esempi:
\begin{itemize}
\item 
``Possa lo spirito dei grandi corridori guidarti i passi'' 
\item 
``Evoco a me i riti del mago Ratl Go Nw. Ora il tuo corpo e' veloce
come l'argento'' 
\item 
``Un passo a destra, uno a sinistra, incrocia le braccia. La velocita' del Lupo e' nei tuoi piedi'' 
\end{itemize}
Esempio pratico:
\begin{itemize}
\item 
con Essenza Movimento a difficolta' 28 posso usare Incanalare energia due volte oppure Imposizione delle mani ed Incanalare energia (non posso usare due Essenze in un round) 
\end{itemize}

\bigskip

\textbf{Essenza Movimento (Blocco) - Concetto/Creature}

Tramite l'Essenza Movimento e' possibile inibire lo spostamento di creature.

Al target viene concesso un Tiro Salvezza su Tempra per negare gli effetti.
\begin{itemize}
\item 
Il valore massimo e' riferito al numero massimo di CR influenzati. 
\item 
il target non puo' essere di 3 CR superiore al valore di CM dell'incantatore 
\item 
Il numero massimo di creature influenzabili e' pari alla meta' dei
dadi vita influenzati, con un minimo di 1 
\end{itemize}

\bigskip

\begin{tabular}[c]{@{}ll@{}}
\toprule 
Livello di Potere & Concetto\tabularnewline
\textless=11 & Il target inciampa, -1 azione\tabularnewline
13 & Il target e' rallentato, viene dimezzato la velocita' di movimento\tabularnewline
16 & Il target e' fermo nella sua posizione (immobilizzato), massimo 2
CR\tabularnewline
19 & I target rimangono fermi nella loro posizione, totale 5 CR\\
& non piu'di 3 CR a target\tabularnewline
22 & Il target e' fermo nella sua posizione, massimo 7 CR\tabularnewline
25 & I target rimangono fermi nella loro posizione, totale 12 CR\\
&non piu'di 4 CR a target\tabularnewline
28 & Il target e' fermo nella sua posizione, massimo 9 CR\tabularnewline
31 & I target rimangono fermi nella loro posizione, totale 16 CR\\
&non piu' di 5 CR a target\tabularnewline
34 & Il target e' fermo nella sua posizione, massimo 12 CR non puo'\\ &teletrasportarsi o cambiare di piano.\tabularnewline
37 & I target rimangono fermi nella loro posizione, totale 24 CR\\
&non piu' di 7 CR a target\tabularnewline
43 & Il target rimane fermo nella sua posizione, massimo 18 CR\tabularnewline
\bottomrule
\end{tabular}


\begin{tabular}[c]{@{}ll@{}}
\toprule 
Livello Potere & Presenza, Creature\tabularnewline
\textless=11 & Un cubo di lato fino a 10 cm e 100 grammi (minuscola). Area singolo
Target\tabularnewline
13 & Un cubo di lato fino a 20 cm o 500gr (piu' piccola). Area singolo Target\tabularnewline
16 & Un cubo di lato fino a 0.5 metro o 3kg (piccola). Area singolo Target\tabularnewline
19 & Un cubo di lato fino a 1 metro (cubo base) o 25kg (). Area singolo
Target\tabularnewline
22 & Fino a 2 cubi base e 100kg (media) di peso. Area 2x1x1 metro\tabularnewline
25 & Fino a 4 cubi base e 200kg (grande). Area influenzata 2x2x1 metro\tabularnewline
28 & Fino a 8 cubi base e 400kg (enorme). Area influenzata 4x2x1 metri\tabularnewline
31 & Fino a 16 cubi base e 800kg (enorme). Area influenzata 8x2x1 metro\tabularnewline
34 & Fino a 32 cubi base e 1.6 tonnellate (mastodontica).\\
&Area influenzata 16x2x1 metri\tabularnewline
37 & Fino a 64 cubi base e 3.2 tonnellate (mastodontica).\\
&Area influenzata 8x8x1 metri\tabularnewline
43 & Fino a 128 cubi base e 6.4 tonnellate (colossale).\\
&Area influenzata 32x4x1 metri\tabularnewline
\bottomrule
\end{tabular}

\bigskip


Esempi
\begin{itemize}
	\item 
	``Chiamo a me le ossa dei ladri. Fermate la creatura' 
	\item 
	``Potenze dell'aria, rallentate la fuga della creatura' 
	\textbf{Essenza Movimento -- Elementi, Creature}
	\item 
	Tramite l'Essenza Movimento e' possibile spostare un oggetto, sollevandolo e muovendolo. 
	\item 
	Se si muovono piu' target considerare la somma di volumi e pesi per determinare il costo. 
	\item 
	Il movimento ottenuto e' di 10 metri a round. Se serve piu' velocita' aumentare la difficolta' (+5 = raddoppio velocita') 
	\item 
	Un cubo base e' un cubo di lato 1 metro 
	\item 
	Al target se vivente e' concesso un Tiro Salvezza su Tempra per resistere e non farsi spostare 
	\item 
	Il target viene spostato, nel corso dei round, fino a Distanza calcolata nelle difficolta'. 
	\item 
	E' possibile usare l'Essenza di Movimento per scagliare proiettili. Usato per scagliare proiettili (sassi, dardi, frecce.. oggetti fino a taglia media) il danno causato e' pari a quello immediatamente inferiore causato dall'Essenza di Attacco. 
	\item 
	E' possibile usare la Essenza di Movimento per fare cadere una massa su singolo target o piu' target a seconda della dimensione della massa, vedi indicazione Area. Il danno causato e' di 1d6 per cubo base che colpisce l'avversario. Quindi una colonna alta 8 cubi e larga 1 fa 8d6 di danni. Massimo 20d6, Tiro Salvezza su Riflessi per dimezzare. In caso di fallimento critico i danni si raddoppiano, in caso di successo critico i danni si dimezzano ulteriormente. 
	\item 
	Le aree influenzate sono indicative, ricordarsi di distribuire i cubi secondo numero e forma volute. 
\end{itemize}
\bigskip


Esempio:
\begin{itemize}
\item 
``Chiamo i venti possenti che distrussero Orton Gal No. Sollevate questo carro!'' 
\item 
``Per il passo leggero degli Orunkes, io cammino nell'aria' 
\item 
``Spiriti delle tempeste, scagliate la vostra rabbia contro chi osa sfidarci!'' 
\end{itemize}

\pagebreak

\subsection{Essenza Protezione -- Potenza}\index{Essenza Protezione}

\label{essenza-protezione---potenza}

l'Essenza di Protezione si applica su Creature, Elementi, Concetto e \textbf{permette di schermare o annullare gli effetti magici e non che altererebbero il nostro corpo}.

\begin{itemize}
\item 
Al target viene concesso un Tiro Salvezza su Arbitrio per negare gli
effetti 
\item 
E' possibile usare l'Essenza di Protezione come controincantesimo verso l'Essenza di Trasformazione o Alterazione o Charme o Movimento o Rivelazione. E' necessario superare con il proprio check di magia il valore di difficolta' della prova che ha generato l'Essenza che si vuole controbattere. 
\end{itemize}

\bigskip

\begin{tabular}[c]{@{}ll@{}}
\toprule 
Livello di Potere & Creature, Presenza, Concetto\tabularnewline
\textless=11 & Proteggi dalla condizione di Abbagliato.\tabularnewline
13 & Proteggi dalla condizione di Frastornato / Scosso\tabularnewline
16 & - Proteggi dalla condizione di Affaticato / Infermo / Spaventato\tabularnewline
19 & - Proteggi dalla condizione di Malato / Esausto / Nauseato / Confuso\tabularnewline
22 & - Proteggi dalla condizione di Avvelenato / Charmato / In preda al panico\tabularnewline
25 & - Proteggi dalla condizione di Posseduto / Accecato / Assordato
\\
&- Proteggi dalla distruzione fino a 2 livello di esperienza
\\
28 &- Proteggi dalla condizione di Dominato / Maledetto
\\
&- Proteggi dalla distruzione fino a 4 livelli di esperienza
\\
31 &- Proteggi dalla distruzione di esperienza fino a durata
\\
&- Proteggi da tutti i condizionamenti mentali fino a durata
\\
34 & Proteggi dall'Essenza di Charm e Rivelazione fino al livello 22\tabularnewline
37 & Proteggi dall'Essenza di Charm e Rivelazione fino al livello 25\tabularnewline
43 & Proteggi dall'Essenza di Charm e Rivelazione fino al livello 28\tabularnewline
\bottomrule
\end{tabular}

\bigskip


Esempi:
\begin{itemize}
\item 
Possa la saggezza dei miei antenati proteggermi 
\item 
Spirito e Bonta'. Possa la Massima Ljust proteggermi dalle Ombre (protezione esperienza) 
\item 
``Traccio il circolo dei Sacerdoti Gurla. Nessuno potra' osservarmi'' 
\item 
``Sbatto il piede e pronuncio la parola di potere Shrak. Non mi trasformerai in un rospo!'' (controincantesimo )
\end{itemize}

\pagebreak

\subsection{Essenza Rivelazione -- Magnetismo}\index{Essenza Rivelazione}

\label{essenza-rivelazione---magnetismo}

La Rivelazione si applica a Creature (Naturali o Magiche), Elementi, Concetti e Virtu'. \textbf{Permette di capire i tratti fisici principali e lo stato di ``salute''. Permette di espandere la propria coscienza per accedere alla comprensione di eventi od oggetti del passato}.
\begin{itemize}
\item 
Alla Creatura oggetto della Rivelazione viene concesso un Tiro Salvezza
su Arbitrio per annullare gli effetti. 
\item 
Non e' possibile usare l'Essenza Rivelazione per determinare un evento futuro. 
\item 
L'Essenza Rivelazione serve per divinare o percepire in maniera piu' approfondita. Viene lasciato al giocatore ed al Narratore l'utilizzo creativo di questa Essenza. Quelli qui sotto proposti sono esempi di utilizzo, linee guida. 
\end{itemize}

\bigskip

\begin{tabular}[c]{@{}ll@{}}
\toprule 
Livello di Potere & Presenza, Concetto, Virtu', Creature\tabularnewline
<=11 &- Comprendi se un oggetto e’ magico
\\
&- Sei in grado di leggere il magico
\\
13& - Comprendi eta’, peso e dimensioni.
\\
&- Comprendi se e’ presente una protezione
\\
&- Sei in grado di leggere una pergamena magica senza difficolta’
\\
16& -
Comprendi lo stato del corpo se influenzato da qualche Essenza
\\
&- Comprendi quale tipo di protezione e’ presente
\\
&- Comprendi i tratti del soggetto.
\\
&- Conferisci ai tuoi occhi la visione della magia
\\
&- Conferisci ai tuoi occhi la visione crepuscolare 18m\\
19& - Comprendi lo stato originario del target.
\\
&- Comprendi la natura e specifiche di un oggetto magico
\\
&- Conferisci ai tuoi occhi di vedere nell’oscurita’
\\
22& - La tua Consapevolezza e’ oltre le illusioni di pari potenza.
\\
&- Conferisci ai tuoi occhi di vedere nell’oscurita’ magica
\\
25& - Puoi scrutare luoghi fino ad un 1 km di distanza
\\
28& - Puoi scrutare persone fino ad 5 km di distanza.
\\
&- Toccando un oggetto ti permette di conoscere la sua storia.
\\
&- La tua Consapevolezza di permette di vedere il vero delle cose e persone
\\
31& - Puoi scrutare luoghi e persone fino a 10 km di distanza.
\\
&- Concentrandoti puoi conoscere la storia di un logo/oggetto fino\\
& a 5 km di distanza\\
34& La tua comprensione ti permette di conoscere nei dettagli la storia e
leggende\\
& di qualsiasi manufatto purche’ tu ne abbia una vaga idea di come e’ fatto\\
37& La tua comprensione e’ leggendaria. Puoi conoscere fatti ed accadimenti \\
&\\di qualsiasi era
43& La tua comprensione e’ tale come se tu fossi stato presente e partecipe
\\
\bottomrule
\end{tabular}

\bigskip

Esempi:
\begin{itemize}
\item 
\begin{quotebox}
``Grande Atmos, invoco la tua benedizione. Aiutami a comprendere questa bacchetta''
\item 
``Per i tomi della biblioteca del tempo. Chi costrui' il castello di Hul Barton?''
\item 
``Benedetto servitore di Atmos concedimi di scrutare nella biblioteca segreta. Raccontami la storia di Rozanda Durand''
\end{quotebox}
\end{itemize}

\pagebreak


\subsection{Essenza Trasformazione -- Potenza}\index{Essenza Trasformazione}

\label{essenza-trasformazione---potenza}

l'Essenza della Trasformazione \textbf{altera la forma e sostanza di Creature, Elementi, Energia.}

\bigskip

\textbf{Essenza Trasformazione -- Presenza, Energia}

L'Essenza di Trasformazione prende la materia gia' pronta e concede all'incantatore di Trasformare le forme e sostanze come piu' gli aggrada. La forma trasformata obbedisce alle leggi della fisica es. l'acqua non puo' volare se non e' retta da qualcosa.

\begin{itemize}
\item 
L'Essenza di Trasformazione puo' essere applicata ad una Essenza di Attacco. Si fa una prova con una difficolta' di un livello maggiore a quello usato per l'Attacco e comunque se ne subisce la conseguenza (getti aria possono farti volare via o danneggiare lo stesso se non si e' protetti) 
\item 
E' anche possibile trasformare l'acqua in fuoco, ma per il principio che una Essenza deve fare una sola cosa, il danno da fuoco potra' esserci solo dal round successivo, se ancora esiste 
\item 
Se si vuole trasformare la materia in un insieme di materia (fango, lava, pane.. si devono considerare le difficola' di ogni singolo elemento che lo compone, acqua e terra, terra e fuoco) 
\item 
Se un soggetto e' influenzato da una Essenza di trasformazione viene concesso un Tiro Salvezza su Tempra per annullarne gli effetti. Il Narratore decidera' il tipo e gli effetti 
\end{itemize}

\bigskip

\begin{tabular}[c]{@{}ll@{}}
\toprule 
Livello di Potere & Presenza, Energia\tabularnewline
\textless=11 & Un cubo di lato fino a 10 cm e 100 grammi (minuscola)\tabularnewline
13 & Un cubo di lato fino a 20 cm o 500gr (piu' piccola)\tabularnewline
16 & Un cubo di lato fino a 0.5 metro o 3kg (piccola)\tabularnewline
19 & Un cubo di lato fino a 1 metro (cubo base) o 25kg (piccola)\tabularnewline
22 & Fino a 2 cubi base e 100kg (media) di peso\tabularnewline
25 & Fino a 4 cubi base e 200kg (grande)\tabularnewline
28 & Fino a 8 cubi base e 400kg (enorme)\tabularnewline
31 & Fino a 16 cubi base e 800kg (enorme)\tabularnewline
34 & Fino a 32 cubi base e 1.6 tonnellate (mastodontica)\tabularnewline
37 & Fino a 64 cubi base e 3.2 tonnellate (mastodontica)\tabularnewline
43 & Fino a 128 cubi base e 6.4 tonnellate (colossale)\tabularnewline
\bottomrule
\end{tabular}

Esempi:
\begin{itemize}
\item 
``Acqua e Terra, Fuoco e Acqua. Le mie mani non hanno fine'' 
\item 
``Per i riti degli antichi alchimisti cio' che era adesso non e'
piu' lui'' 
Esempio pratico:
\item 
Trasformando in acqua un cubo di 3{*}3{*}3 (27 cubi base) di terra si mette in seria difficolta' l'avversario (Tiro salvezza su riflessi concesso per evitare di cadere) 
\end{itemize}

\bigskip

\textbf{Essenza Trasformazione -- Creature}

\begin{itemize}
\item 
L'Essenza di Trasformazione puo' trasformare Creature Naturali, Creature Magiche ed Elementi tra loro. 
\item 
L'Essenza di Trasformazione se effettuata su una creatura senziente costa il livello superiore di potere. Per trasformare in pietra una creatura normale di 4 CR (in caso di PG il Livello e' il CR) la prova ha difficolta' 28 
\item 
Non si puo' influenzare o trasformare in una creatura con piu' di 3 CR superiore alla CM dell'incantatore 
\item 
Se si vuole trasformare in una creatura magica il livello di potere passa a quello successivo (e si somma con il costo della senziente) 
\item 
I CR indicati si riferiscono alla somma dei CR influenzati 
\item 
Al target viene concesso un Tiro Salvezza su Arbitrio per negare gli effetti 
\item 
Il target trasformato mantiene le caratteristiche mentali precedenti
ma prende quelle fisiche della creatura 
\end{itemize}

\bigskip

\begin{tabular}[c]{@{}ll@{}}
\toprule 
Livello di Potere & Creature\tabularnewline
<=11& - Un cubo di lato fino a 10 cm e 100 grammi\\
&- 1/3 CR\\
13& - Un cubo di lato fino a 20 cm o 500gr\\
&- 1 /2 CR\\
16& - Un cubo di lato fino a 0.5 metro o 3kg\\
&- 1 CR\\
19& - Un cubo di lato fino a 1 metro (cubo base) o 25kg\\
&- 2 CR\\
22& - Fino a 2 cubi base e 100kg di peso\\
&- 3 CR\\
25& - Fino a 4 cubi base e 200kg\\
&- 5 CR\\
28& - Fino a 8 cubi base e 400kg\\
&- 7 CR\\
31& - Fino a 16 cubi base e 800kg\\
&- 9 CR\\
34& - Fino a 32 cubi base e 1.6 tonnellate\\
&- 11 CR\\
37& - Fino a 64 cubi base e 3.2 tonnellate\\
&- Fino a 13 CR, max singolo CR 7\\
43& - Fino a 128 cubi base e 6.4 tonnellate\\
&- Fino a 15 CR, max singolo CR 9\\
\bottomrule
\end{tabular}

\bigskip


Esempi:
\begin{itemize}
\item 
``Chiedo l'aiuto di tutte le streghe. Trasformate il mio nemico in un rospo!'' 
\item 
``Bava di Lumaca e sterco di vacca. Rumina nel prato'' 
\end{itemize}

\pagebreak

\subsection{La Magia (Semplificata -- Opzionale)}\index{La Magia Semplificata}


Questo sistema di Magia e' piu' ``classico'' e riprende gli standard della 3ed e Pathfinder.

Per ogni punto in Competenza Magica l'usufruitore di magia sceglie due Incantesimi. Il livello massimo sceglibile e lanciabile e' indicato nella Tabella Punti Magia Posseduti.

Ogni volta che si prende un punto in CM e' possibile dimenticare un incantesimo e sostituirlo con un altro.

Un usufruitore di magia ha a disposizione un numero di punti magia dato dal suo punteggio di CM, vedi tabella Punti Magia Posseduti.

Un incantesimo ``costa' un numero di punti magia pari al doppio del suo livello.

Usate come lista di incantesimi quella di Pathfinder compresa nel Core Book, oppure online su http://aonprd.com/SpellsCustom.aspx\ e selezionare PFS Legal

La DC per resistere all'incantesimo e' 10+CM+Intelletto.

Ogni volta che l'incantesimo fa riferimento al livello dell'usufruitore di magia considerate invece il punteggio di CM/2, gli altri fattori dell'incantesimo rimangono inalterati.

\bigskip

\textbf{Tabella Punti Magia posseduti}

\bigskip

\begin{longtable}[c]{@{}lll@{}}
\toprule 
Valore Competenza Magia
& (CM) Punti Magia Posseduti & Max livello incantesimo lanciabile e sceglibile\\
1 & 3 & 1\tabularnewline
2 & 5 & 1\tabularnewline
3 & 7 & 2\tabularnewline
4 & 11 & 2\tabularnewline
5 & 13 & 3\tabularnewline
6 & 17 & 3\tabularnewline
7 & 19 & 3\tabularnewline
8 & 23 & 4\tabularnewline
9 & 29 & 4\tabularnewline
10 & 31 & 5\tabularnewline
11 & 37 & 5\tabularnewline
12 & 41 & 6\tabularnewline
13 & 43 & 6\tabularnewline
14 & 47 & 7\tabularnewline
15 & 53 & 7\tabularnewline
16 & 59 & 7\tabularnewline
17 & 61 & 8\tabularnewline
18 & 67 & 8\tabularnewline
19 & 71 & 9\tabularnewline
20 & 73 & 9\tabularnewline
\bottomrule
\end{longtable}

\pagebreak

\section{Vantaggi}\index{Vantaggi}

\label{vantaggi}
\begin{quotebox}
Adoro fare il supereroe! L'orario di lavoro e' pessimo, la paga e' inesistente... ma almeno non corro il rischio di venire licenziato! (PK)
\end{quotebox}

\bigskip

Ogni personaggio puo' avere, e non e' obbligatorio averne, dei vantaggi. Questi devono essere interessanti, piacevoli, divertenti e soprattutto giocabili.

Ogni vantaggio ha un costo, da pagare ad ogni livello. Come detto non deve essere obbligatorio prendere un vantaggio, né tanto meno si devono prendere vantaggi solo perche' fanno essere forti. Lo scopo di un vantaggio e' stupire e divertirsi. 

Avere un vantaggio significa essere diverso, essere un freak, avere quel particolare che ti rende diverso ed unico, ma non per questo sempre il piu' forte, potente o invincibile. Un vantaggio non e' solo una capacita', e' un'occasione di gioco di ruolo. Il giocatore e' invitato ad essere creativo nella scelta dei vantaggi ed anche nella creazione di nuovi, il costo poi si decide con il Narratore.

Diversi vantaggi non hanno un effetto pratico concreto ed immediato ma sono di arricchimento al background, alla storia del personaggio, introducono occasioni di gioco e divertimento. Quando si scelgono i vantaggi, e di conseguenza gli svantaggi, non e' come andare a fare scorta di poteri super e straordinarie abilita', ma di peculiarita', manie, specialita' che il personaggio possiede e che ancora una volta lo rendono diverso, unico, solo tuo. 

Pertanto vantaggi e svantaggi vanno anche e soprattutto giocati ed interpretati.

I Vantaggi con {*} e tutte quelle con costo 20 o superiore sono a discrezione del Narratore.

I vantaggi si scelgono al primo livello, ogni vantaggio preso a livelli successivi va concordato con il Narratore.

I punti di costo di un Vantaggio si pagano con i punti guadagnati dagli Svantaggi.

I bonus dati alle competenze si intendono specifiche sulla prova quando indicato tra parentesi.

\bigskip

\textbf{Ali della provvidenza} \index{Ali della provvidenza}20 : hai delle ali, a te la scelta di forma e colore, solitamente stanno sulle scapole e ti fanno volare (volare buono)

\textbf{Ambidestro}\label{Ambidestro}\index{Ambidestro} 10: puoi usare indifferentemente le mani. I malus alle prove dove si usano due mani diminuiscono di 2

\textbf{Amico degli animali}\index{Amico degli animali} 5: +2 alle prove per gestire gli animali (anche selvaggi)

\textbf{Anfibio}\index{Anfibio} 20: puoi respirare sia sott'acqua che l'aria

\textbf{Arcobaleno}\index{Arcobaleno} 10: sei un artista. Le tue dita spontaneamente
producono colore

\textbf{Aura di coraggio}\index{Aura di coraggio} 15: intorno a te, in distanza entro 3 metri infondi coraggio. +2 TS vs Essenza Charme

\textbf{Artigli}\index{Artigli} 5: ogni tanto ricordati di spuntare gli unghiotti. 1d4 di danno per attacco. Gli attacchi con la seconda mano non hanno le penalita' delle due armi.

\textbf{Bere fa bene}\index{Bere fa bene} 5: Prerequisito: Il fegato non conta. Il tuo corpo metabolizza l'alcool in maniera molto efficace. Un litro di birra ti fa recuperare 1d4 PF, un bottiglia di liquore 1d8 PF. Se di pessima qualita' no.. 

\textbf{Caduta gatto}\index{Caduta gatto} 5: +2 alle prove di Agilita' sulle cadute

\textbf{Camaleonte}\index{Camaleonte} 10-20: la tua pelle puo' cambiare colore. Tempo necessario 1 minuto/1round

\textbf{Cambiaforma}\index{Cambiaforma} 40: come Essenza Alterazione, Livello Potere 18

\textbf{Camminare sull'aria} \index{Camminare sull'aria}30: non troppo controllato. Qualsiasi cosa che non sia camminare richiede una prova di Agilita'

\textbf{Camminare sulle acque} \index{Camminare sulle acque} 30: ma non darti delle arie..

\textbf{Magnetico} \index{Magnetico}5-10: sprigioni luce quando vuoi. per fortuna non letteralmente. +2 alle prove al Magnetismo

\textbf{Consumi ridotti} \index{Consumi ridotti}5: bevi e mangi la meta' di un uomo normale. Sei sotto peso.

\textbf{Controllo del metabolismo} \index{Controllo del metabolismo} 10: solo il nome e' fantastico! Annulli il danno da Sanguinamento.

Recuperi i punti ferita come se avessi il doppio del punteggio di Potenza.

\textbf{Cure efficaci} \index{Cure efficaci}10: +1d6 ogni volta che una Essenza
ti cura

\textbf{Daredevil} \index{Daredevil}10: ti piace buttarti nelle mischia, specialmente se si corrono pericoli. +2 Tiri per Colpire / Difesa finche' sei circondato da tre o piu' avversari

\textbf{Denti} \index{Denti}5: il tuo morso fa male, 1d4, lavati i denti ogni tanto..

\textbf{Digestione universale} \index{Digestione universale}5: purche' non faccia male si mangia, +2 TS su Tempra vs Veleni

\textbf{Direzione Assoluta} \index{Direzione Assoluta}5: sai sempre dove e' il nord magnetico. Hai un +4 alle prove di orientamento (Sopravvivenza)

\textbf{Duro da soggiogare} \index{Duro da soggiogare}5: +2 TS su Arbitrio su Essenze Charme

\textbf{Duro da uccidere} \index{Duro da uccidere}5: non svieni a 0 PF, ma a -LV/2 in PF. Muori a 15+Potenza x 3 PF

\textbf{Empatia con le piante} \index{Empatia con le piante}10: io comprendo la sofferenza dell'erba pestata

\textbf{Empatia} 5: +2 alle prove di percepire inganni (Consapevolezza)

\textbf{Empatia Animale} \index{Empatia Animale}10: +4 alle prove per gestire gli animali (anche selvaggi)

\textbf{Empatia spirituale} \index{Empatia spirituale}5: non parli con gli spiriti, ma ne senti le emozioni

\textbf{Ermafrodito} \index{Ermafrodito}10: lgbtE!

\textbf{Forgiato nell'acciaio} \index{Forgiato nell'acciaio}5: Tramite dolorose operazioni la tua pelle e' stata rivestita con placche di metallo. +3 alla Difesa

\textbf{Forma d'ombra} \index{Forma d'ombra}30: considera il poterti trasformare in un ombra 1 ora per livello. Non puoi andare in spazi assolati e senza ombra.

\textbf{Fortunato} \index{Fortunato}5-10: 3 volte al giorno puoi ritirare un 1 sul dado a 6, da dichiarare prima che il Narratore ti abbia detto se la prova e' riuscita o meno.

\textbf{Guarigione accelerata}\index{Guarigione accelerata}: 5 ogni mattina recuperi il doppio dei Punti Ferita che normalmente recupereresti. Si cumula con Controllo Metabolismo. \index{Controllo Metabolismo.}

\textbf{Guaritore}\index{Guaritore} 5: sai dove mettere le mani. +4 alle prove di Sopravvivenza (pronto soccorso)

\textbf{Il fegato non si conta} \index{Il fegato non si conta}10: puoi bere tanto e non ti ubriachi

\textbf{Illuminato} \index{Illuminato}10-20: fai luce.. letteralmente. Emetti luce in un raggio di 3/6 metri. Puoi controllare (20) l'emissione o meno (10)

\textbf{Immune}\index{Immune} 5-20: a cosa ?

\textbf{Invisibile} \index{Invisibile}40: il tuo corpo e' invisibile. Sempre. E non e' magia...

\textbf{Ira} \index{Ira}5: sei capace di infuriarti. +2 POT -1 Difesa. Ogni altri 5 punti +2 POT -1 a Difesa Max 20 punti. Durata 4 (anche non consecutivi)round ogni 5 punti. Si attiva come Azione a costo 1.

\textbf{La mia ombra e' mia amica} \index{La mia ombra e' mia amica}10: Riesci a posizionare la tua ombra dove vuoi. Si considera tu possa lanciare Essenze a tocco tramite la tua ombra (che deve essere presente) entro raggio 3 metri.

\textbf{Legami di furia} 15\index{Legami di furia} : Puoi evocare lacci eterei che minacciano i tuoi nemici. Per 3 volte al giorno con il costo di 1 Azione tutti gli avversari in raggio entro 9 metri attorno a te sono intralciati per un round. TS vs Riflessi DC 10+½ lv + Magnetismo) per liberarsi.

\textbf{Lento e Fermo} 5: \index{Lento e Fermo}Sei eccezionalmente stabile sui tuoi piedi. Non puoi essere mosso se non da una creatura di 2 taglie superiori.

\textbf{Lingua universale} \index{Lingua universale}10. Le tue capacita' linguistiche sono impressionanti. Dopo due giorni a contatto con una nuova lingua sei in grado di parlarla correttamente. Dopo 3g di lontananza dall'ambiente dimentichi la lingua. Guadagni un +2 ai check basati sulla lingua

\textbf{Magia esplosiva} \index{Magia esplosiva}10: le tue Essenze di Attacco hanno un dado in piu' di danno (quando c'e' da tirare un dado..)

\textbf{Mani di Fata} \index{Mani di Fata}10: +4 prova di Criminalita' che coinvolgano le mani. Puoi prendere 16 come prendessi un 10 nelle prove relative. 

\textbf{Mano Piede palmata} \index{Mano Piede palmata}5: +4 alle prove di nuotare

\textbf{Mattiniero} \index{Mattiniero}5-10-15: ti basta dormire 6/5/4 ore per notte
per essere riposato completamente

\textbf{Medium} \index{Medium}10-20: alcune volte lo vuoi tu, altre volte ti cercano loro

\textbf{Memoria fotografica}\index{Memoria fotografica} 20-50: per fortuna non e' permanente (50). +8 alle prove per ricordare dettagli (Cultura e Consapevolezza)

\textbf{Naso peloso} \index{Naso peloso}5: le tue narici filtrano le tossine presenti nell'aria che respiri. +2 alle prove relative. Il tuo naso e' di dimensioni.. non piccole.

\textbf{Non dormi}\index{Non dormi} 20{*}: e non so come fai..

\textbf{Non invecchi}\index{Non invecchi} 20{*}: non invecchi (ma possono ucciderti lo stesso)

\textbf{Non mangi bevi} \index{Non mangi bevi}20: e non so come fai..

\textbf{Non respiri} \index{Non respiri}20: e non so come fai..

\textbf{L'Odore del sangue} \index{L'Odore del sangue}10: L'odore di sangue e' una droga potente
Prerequisiti: non puoi avere ``Il fegato non conta' . Guadagni un +1 a Tiro per Colpire ed un +1 al danno per ogni nemico che hai ucciso con la tua arma nel round. Questo bonus non puo' superare il +4/+4. Il bonus rimane attivo fino al round successivo all'ultima uccisione fatta. Creature con meno di 3 lv di te non contano.

\textbf{Oracolo} \index{Oracolo}20: per qualcuno e' una maledizione

\textbf{Ottima vista} \index{Ottima vista}5: hai un ottima vista (12/10). +2 alle prove relative che usano la vista.

\textbf{Ottimo olfatto e gusto} \index{Ottimo olfatto e gusto}5: hai un ottimo gusto ed olfatto. +2 alle prove relative che usano olfatto o gusto.

Con un prova su Intelletto a DC 15 puoi capire cosa e' una pozione.

\textbf{Ottimo tatto} \index{Ottimo tatto}5: hai un ottimo tatto. sai leggere con le dita. Sei in grado di trovare una porta nascosta toccando la parete.

\textbf{Ottimo udito}\index{Ottimo udito} 5: hai un ottimo udito. +2 alle prove che coinvolgono l'udito

\textbf{Parlare con gli animali}\index{Parlare con gli animali} 20: scegli una famiglia (ovini, marsupiali, caviette..)

\textbf{Parlare con le piante} \index{Parlare con le piante}20: ho sempre voluto parlare con le zucchine..

\textbf{Percezione Cieca}(vista cieca):\index{Percezione Cieca} \index{vista cieca}30: riesci a percepire qualsiasi cosa con i tuoi sensi, dall’odore, al calore. Riesci a “vedere” attraverso e fino 18 metri, 10 cm di pietra, 20 cm di legno, 0.5 cm di metallo

\textbf{Perfetto equilibrio} 5:\index{Perfetto equilibrio} +2 alle prove relative di Acrobatica

\textbf{Piedi veloci} 10: il tuo movimento aumenta di 3 metri

\textbf{Pollice verde} 5: +4 alle prove di Lavoro (Erboristeria, Professione Giardiniere..)

\textbf{Polmoni di ferro} 5: puoi trattenere il respiro 20*POT round (minimo 20 round)

\textbf{Precognizione} 30{*}: Puoi usare l'Essenza Rivelazione. In automatico conosci l'Essenza ed hai un valore in Competenza Magica pari al livello per questa Essenza (oppure un +6 al check se hai gia' l'Essenza)

\textbf{Recupero} 10: il tuo corpo produce spontaneamente caffeina.  Ignori la condizione affaticato

\textbf{Resistenza} 5-10: +1/+2 TS a Riflessi o Tempra o Arbitrio

\textbf{Resistenza al danno} 10: -1 danno. -1 danno aggiuntivo ogni 5 punti aggiuntivi

\textbf{Resistenza al magico} 20: Hai una RM 3, ogni qual volta sei influenzato da una Essenza tira 3d6+RM se e' superiore alla prova di Competenza Magica di chi ti lancia l'Essenza o 6+Livello Potere (in caso di oggetti/pozioni..) l'Essenza non ha effetto.

\textbf{Resistenza al fuoco/freddo/elettricita'} 5-10: ignori i primi 3/6 punti di danno per round

\textbf{Ricostruzione} 30: perdere una mano non e' mai stato un problema..

\textbf{Rigenerazione} 30: +1PF {*}T (non rigeneri arti)

\textbf{Rigenerazione} \textbf{veloce} 40: +1PF per round (non rigeneri arti). Muori se distruggono il tuo corpo (o non rimane che cenere).

\textbf{Rimpicciolimento} 30: puoi diminuire fino a due taglie. Durata fino a 8 ore

\textbf{Rinoceronte} 10 : La tua carica e' distruttiva. Si considera che niente sotto la robustezza di sbarre di ferro (durezza 15) possa fermare la tua carica. Dietro di te lasci una scia di distruzione. +2 ai Tiri per Colpire in Carica.

\textbf{Scudo Mentale} \index{Scudo Mentale}10: +2 TS su controlli ed influenze mentali

\textbf{Sensi protetti}\index{Sensi protetti} 5: +2 TS contro suoni/luci/vapori o Essenza di Distruzione che agisca sui tuoi sensi

\textbf{Senso comune} \index{Senso comune}5: se stai per fare una brutta figura un campanellino ti avvisa

\textbf{Senso della moda}\index{Senso della moda} 5: sai sempre come vestirti bene. anche solo con uno straccetto

\textbf{Senso delle vibrazioni} {Senso delle vibrazioni \index{tremorsense} (tremorsense) 30: tutto emette vibrazioni, o quasi, raggio di 18 metri intorno a te

\textbf{Senso del tempo} \index{Senso del tempo}5: sai sempre che ore sono, giorno o notte.

\textbf{Senso ragno}\index{Senso ragno} 15: no non ti ha morso un uomo radioattivo,ma sei estremamente sensibile ai pericoli. +2 iniziativa, non puoi essere sorpreso

\textbf{Senza paura} \index{Senza paura}10: sei immune alla paura, magica o meno.

\textbf{Silenzioso} \index{Silenzioso}5: +4 alle prove di Consapevolezza (muoversi silenziosamente)

\textbf{Spine} \index{Spine}5: e sei pure brutto. 1d4 di danno

\textbf{Super piastrine} \index{Super piastrine}5 Riduci il danno da Sanguinamento di 1 a fine di ogni round.

\textbf{Talento per le lingue}\index{Talento per le lingue} 5: impari due lingue investendo 1 punto in Conoscenza Linguistica

\textbf{Talento selvaggio}: \index{Talento selvaggio}parliamone

\textbf{Tocco gelido} \index{Tocco gelido}10: toccando un morto (entro 1 giorno per livello) puoi vedere e sentire cosa e' successo nel suo ultimo round di vita.

\textbf{Troll} \index{Troll}50: rigeneri 5 pf a round anche se i PF sono negativi. Rigeneri anche arti. Puoi essere ``ucciso'' solo da fuoco o acido. Una condizione potrebbe comunque tenerti a punti ferita negativi (es. immerso sott'acqua).

\textbf{Udito subsonico}\index{Udito subsonico} 10: senti le frequenze inudibili per gli umani (come un cane)

\textbf{Vedere l'invisibile} \index{Vedere l'invisibile}15: meglio la vista a raggiX.. sbav..

\textbf{Comprensione del vero}\index{Comprensione del vero} 10: la verita' ha un suono tutto suo. +4 alla prove di percepire inganni

\textbf{Visione oscura} \index{Visione oscura}15: vedi nell'oscurita' piu' totale, anche magica, fino a 18 metri

\textbf{Visione Perimetrale} \index{Visione Perimetrale}5: sogliola ? +2 alle prove di Consapevolezza da lato

\textbf{Visione Telescopica}\index{Visione Telescopica} 10: +4 alle prove di Consapevolezza e visione solo da lontano

\textbf{Voce suadente} \index{Voce suadente}5: +2 alle prove di Magnetismo che usano la voce

\textbf{Voce subsonica}\index{Voce subsonica} 10: emetti suoni non udibili dagli umani. I cani ti odiano

\pagebreak

\section{Svantaggi}\index{Svantaggi}

\label{svantaggi}
\begin{quotebox}
Se devi essere storpio, meglio essere uno storpio ricco. (Tyrion Lannister)\end{quotebox}

Uno svantaggio caratterizza il personaggio, ne definisce limiti e paure. Ogni personaggio deve avere almeno 1 svantaggio di ruolo e questo non gli da punti bonus.

I punti presi con gli Svantaggi psico/fisici servono a coprire i punti spesi con i Vantaggi. Ovviamente l'Evil Narratore gradisce anche piu' svantaggi...

\textbf{Ogni giocatore deve giocare i suoi svantaggi altrimenti non acquisisce punti esperienza e gli sara' negato l'uso dei Vantaggi.}

Uno svantaggio puo' essere ``annullato'' nel corso della storia del personaggio e deve esserci una avventura che giustifichi il tutto. Come sempre il Narratore ha l'ultima parola su ogni scelta di vantaggi e svantaggi.

\bigskip

Suggerimenti
\begin{itemize}
\item 
prendi degli svantaggi che siano divertenti da giocare, anche se ti metteranno nei guai. 
\item 
prendi degli svantaggi che siano interessanti da giocare con gli altri giocatori anche se metteranno loro nei guai 
\item 
prendi degli svantaggi che c'entrino con il personaggio 
\item 
prendi degli svantaggi di cui non andrai a pentirti 
\end{itemize}

\textbf{Fai attenzione}:

\begin{itemize}
\item 
evita gli svantaggi che sono difficili da giocare o perche' completamente avulsi dal sistema o totalmente inutili o severamente dannosi per gli altri. Se vuoi essere un pacifista estremo, valuta bene il personaggio ed il gruppo.. 
\item 
non prendere svantaggi che ti possa vergognare a recitare 
\item 
non prendere svantaggi che non c'entrano con il personaggio (in perfetta contraddizione con quanto gia' detto) 
\item 
non prendere svantaggi insulsi (tipo la paura di girare a destra , degli ascensori..) 
\item 
se prendi uno svantaggio severo, recitalo bene, il Narratore sapra' ricompensarti 
\end{itemize}

Gli svantaggi si dividono in due categorie, \textbf{Svantaggi di Ruolo} e \textbf{Svantaggi psico/fisici}. 

Gli \textbf{Svantaggi di Ruolo} sono dei piccoli difetti, tic, problemi grandi e piccoli che servono a dare uno spessore piu' ``umano'' al personaggio. Hanno una descrizione volutamente ambigua e scherzosa, sceglili con attenzione e discuti con il Narratore come intendi interpretare questo svantaggio.

Il giocatore e' invitato a creare nuovi svantaggi di ruolo. Questi svantaggi non concedono un bonus o malus ne danno punti per prendere vantaggi.

\bigskip

Gli \textbf{Svantaggi psico/fisici} sono invece piu' impattanti nel gioco, nella quotidianita' dando concreti svantaggi. Questi svantaggi forniscono i punti con i quali ``pagare'' i vantaggi. In fondo trovate un elenco di Fobie

\pagebreak

\subsubsection{Svantaggi di Ruolo}\index{Svantaggi di Ruolo}

\bigskip

\textbf{Alcolismo}:\index{Alcolismo} ti piace bere, e tanto.. ma quando smetti ?

\textbf{Alla moda}\index{Alla moda}: tua probabilmente, anche con vestiti nuovi non ti vesti mai bene. L'accostamento di colori e' sempre un pugno nell'occhio.

\textbf{Amico degli animali}:\index{Amico degli animali} intesi come pulci, zecche, pidocchi, cimici.. mosche. Hai uno zoo su di te.

\textbf{Attira animali}: \index{Attira animali}non sai il perche' ma sei sempre circondata da gatti, cani, coniglietti, coccatrici..

\textbf{Attira guai}\index{Attira guai}: non e' colpa mia se il drago ha deviato per venire a fare la popo' qui..

\textbf{Banana}: \index{Banana}quella che provi a farti nei capelli, ma non riesci. 
I tuoi capelli non vanno d'accordo con te

\textbf{Bassa soglia del dolore}: \index{Bassa soglia del dolore}mi ha graffiato, aiuto! sto morendo!!!

\textbf{Brufoli}: \index{Brufoli}pieno, hai la faccia butterata e continuano a formarsi questi disgustosi brufoli gialli

\textbf{Ciuccione:} \index{Ciuccione}non lo fai spesso, ma nei momenti in cui sei piu' nervoso tiri fuori il vecchio ciuccio di legno.. (o in mancanza va sempre bene il proprio pollice)

\textbf{Codardo}:\index{Codardo} e' meglio scappare, pardon, raccogliamo prima tutte le informazioni

\textbf{Cogito ergo sum}: \index{Cogito ergo sum}hai la tendenza a parlare tra te e te, ma ad alta voce anche se ci sono persone intorno e pure se non sono amichevoli

\textbf{Credulone}: \index{Credulone}ma dai ? davvero ? e a quale altezza volava l'asino ?

\textbf{Criceto}: \index{Criceto}intesa come memoria. Non riesci ad associare nomi a volti.

\textbf{Denti marci}: \index{Denti marci}probabilmente lo spazzolino che usi non ha setole di vero cinghiale...

\textbf{Dita nel naso}:\index{Dita nel naso} spero che siano almeno buone

\textbf{Diva}: \index{Diva}o almeno tu credi di esserlo. Non perdi occasione per dare sfoggio delle tue inesistenti capacita' canore, comiche, estetiche... con grosse risate di tutti

\textbf{Faccia comune}: \index{Faccia comune}come ti chiami ? mi sembra di averti gia' visto...

\textbf{Galante}: \index{Galante}al limite del maniacale, in ogni tuo gesto sei formale, appropriato e cordiale

\textbf{Killer}:\index{Killer} no, non sei un assassino. Hai pero' sempre le mani ed i piedi freddi.

\textbf{Impaurisci animali}: \index{Impaurisci animali}puo' essere anche comodo, se non fosse per i cavalli che scappano e gli orsi che attaccano...

\textbf{Incapace di divertirsi}: \index{Incapace di divertirsi}quindi ? E' un problema tuo, non mio

\textbf{Inglese:} \index{Inglese}inteso come umorismo. Nessuno mai capisce le tue battute

\textbf{Mangione}: \index{Mangione}CIOMP!. Mai lesinare, potrebbe essere l'ultimo pasto!

\textbf{Meteora}:\index{Meteora} soffri di meteorismo compulsivo e rumoroso, per non parlare dell'odore sgradevole

\textbf{Megalomane}:\index{Megalomane} coinvolgiamo gli eserciti dei sette regni e penetriamo nel dungeon!

\textbf{Mentina:} \index{Mentina}se mangiassi solo aglio e cipolla il tuo alito sarebbe meno puzzolente

\textbf{Musichiere}: \index{Musichiere}con la bocca. Fischi di continuo, in ogni occasione che sei sovrappensiero o molto teso.. ti metti a fischiettare

\textbf{Non empatico}: \index{Non empatico}perche' piange il bambino a cui ho appena dato a fuoco l'orsetto ?

\textbf{Ossessione}:\index{Ossessione} ancora, ancora, ancora. Un'altra crema per la pelle!!

\textbf{Pacco}: \index{Pacco}il tuo. Hai sempre una mano laggiu. Forse i pantaloni sono stretti ? e no, non ti stringo la mano.

\textbf{Pessimo carattere}: \index{Pessimo carattere}va bene essere burbero.. ma devi sempre renderlo palese ?

\textbf{Pezzata}: \index{Pezzata}no, non la mucca o la tua cavalla ma la tua ascella. Sudi copiosamente, che sia per caldo o freddo.. o nervoso.

\textbf{Rigidezza mentale}:\index{Rigidezza mentale} no, non capisco, la mappa dice di andare a destra. Non mi importa se non c'e' una destra.

\textbf{Saccente}\index{Saccente}: la risposta giusta e' solo la tua. Non c'e' dubbio.. per te.

\textbf{Sangue dal naso}: \index{Sangue dal naso}capita, e sempre appena vedi una donna/uomo (a seconda dei gusti) che ti piace

\textbf{Sciarpina}: \index{Sciarpina}devi sempre avere addosso e visibile un capo di un certo tipo, altrimenti non esci di caverna.

\textbf{Segreto}: \index{Segreto}ho un segreto, talmente tanto segreto che non so se lo so neanche io...

\textbf{Seguire il Chaos}: \index{Seguire il Chaos}e' piu' forte di te, non riesci mai ad ubbidire a qualsiasi legge o autorita' preposta.

\textbf{Seguire la Legge}: \index{Seguire la Legge}e' piu' forte di te, non importa che legge sia, tu non la violi.

\textbf{Tatuato:} \index{Tatuato}il tatuaggio e' il modo di vivere. Hai almeno il 30\% del corpo gia' tatuato e non perdi occasioni per farti nuovi tatuaggi.

\textbf{Topi}:\index{Topi} sei una TOPI!

\textbf{Unghie}:\index{Unghie} sei un divoratore compulsivo di unghie, la punta delle dita ti sanguina a volte

\textbf{Ultima parola}\index{Ultima parola}: e' piu' forte di te, devi avere l'ultima parola in ogni discorso,

\pagebreak

\subsection{Svantaggi psico/fisici}\index{Svantaggi psico/fisici}

\label{svantaggi-psicofisici}

\textbf{Albino}\index{Albino}

Bianco, quasi latte. Non ti abbronzi e non sopporti la luce, la tua pelle e' delicata.

\textbf{13}: Oltre ad essere estremamente riconoscibile hai i seguenti svantaggi: Miopia e Fotosensibilita' e Pelle Sensibile.

\textbf{Allergia}\index{Allergia}

Hai una qualche forma allergica. Spero non grave. Assicurati di avere sempre con te una pozione di Essenza rimuovi veleno.

\textbf{5:} In presenza di un allergene specifico il personaggio starnutisce sonoramente finche' l'allergene non viene allontanato, -1 a tutti i Check. (es. Allergico alla Birra)

\textbf{10}: Il personaggio soffre di attacchi di tosse, iperlacrimazione, giramenti di testa, -2 a tutte le prove. Tiro Salvezza su Tempra DC 10 per non soffocare. Il tiro va ripetuto ogni 20 round finche' non ti sei allontanato dall'allergene.

\textbf{15:} Il personaggio soffre di violenti attacchi di tosse, nausea, sudori freddi, palpitazione. -5 a tutte le prove, E' necessario un Tiro Salvezza su Tempra DC 15 o perdere i sensi. I tiri vanno ripetuti ogni 5 round finche' l'allergene non e' allontanato.

\textbf{20}: Il personaggio cade in preda di una crisi respiratoria, ed e' incapace di compiere qualsiasi azione che non sia vomitare , annaspare e tossire sangue. Fallendo un Tiro Salvezza su Tempra DC 25 il personaggio muore annegando nel suo stesso vomito. Il tiro va ripetuto ogni round fino a che l'allergene non e' allontanato.

\textbf{Allucinazioni}\index{Allucinazioni}

C'e' qualcosa che non va nella tua testa, ogni tanto si innesca una scintilla.

\textbf{10}: Il personaggio vede e sente cose che non ci sono. Ogni giorno tiri un dado a sei facce.
Se escono 1 o 2, non succede nulla.
Con 3,4 o 5 si verificheranno uno o due episodi allucinatori con modalita' e tempi a discrezione del Narratore.
Con 6 il personaggio sara' vittima di visioni orrende e disgustose con durata di 1d4 ore. 

\textbf{Amnesia}\index{Amnesia}

\textbf{10}: Hai dimenticato il tuo passato e con quello il ricordo di amici, nemici, obiettivi. Non c'e' modo di recuperare i ricordi perduti.

\textbf{Asceta}\index{Asceta}

10, lo dice la regola. Non porterai con te piu' di 10 oggetti.

\textbf{20}: non puoi avere piu' di 10 oggetti con te, magici o normali o monete o armi. Per fortuna i vestiti non contano.

\textbf{Balbuziente}\index{Balbuziente}

Sai parlare, ma male.

\textbf{5:} Hai una fastidiosa tendenza a balbettare proprio quando hai qualcosa da dire di importante. In queste situazioni critiche dalle tue labbra escono solo suoni abbozzati.

\textbf{Pessimo Carattere}\index{Pessimo Carattere}

Le buone maniere non sono mai una opzione.

\textbf{5}: Non hai mai imparato l'arte della diplomazia, e detesti essere contraddetto o insultato. Questo non significa che passi alle vie di fatto, ma che di fronte ad un insulto o ad una critica schietta tendi a zittire il proprio interlocutore con espressioni davvero poco simpatiche. Hai un -2 alle prove basate sul Magnetismo

\textbf{Spendaccione}\index{Spendaccione}

\textbf{10}: devi spendere meta' dei tuoi guadagni di missione in piaceri futili (mangiare cibi costosi, bere vino e liquori pregiati, vestiti lussuosi, no armi od oggetti magici)

\textbf{15}: devi spendere tutti i tuoi guadagni di missione in piaceri futili (mangiare cibi costosi, bere vino e liquori pregiati, vestiti lussuosi, no armi od oggetti magici)

\textbf{Caritatevole}\index{Caritatevole}

\textbf{10}: devi donare meta' dei tuoi guadagni di missione in beneficenza

\textbf{15}: non puo' tenere piu' di 10 mo in contanti

\textbf{Cecita'}\index{Cecita'}

\textbf{10}: Sei orbo, visione laterale compromessa, problemi nel capire la distanza delle cose.
Le competenze quali Sopravvivenza e i Check CA per colpire con armi da lancio hanno
un -4. La Difesa peggiora di 2

\textbf{20}: sei cieco. Non vedi. tutti i nemici sono Occultati.

\textbf{Cleptomania}\index{Cleptomania}

\textbf{5}: Senti il bisogno irresistibile di appropriarti di oggetti “interessanti”, di tanto in tanto. Se in un giorno non hai rubato almeno un oggetto non potrai usare Punti Fato per quel giorno.

\textbf{Codice Etico/Voto}\index{Codice Etico}\index{Voto}

Hai fatto un voto, una promessa, un giuramento che condiziona il tuo agire.

5-10 : stabilisci bene le regole, nero su bianco, e si chiaro con il Narratore

\textbf{Compulsivo}\index{Compulsivo}

Ci sono certi comportamenti, per te necessari, dei quali non puoi fare assolutamente a meno (es: camminare evitando le macchie sul terreno o passando solo su quelle, sfilare l'arma solo in un certo modo, ecc) 
Questi comportamenti vanno dichiarati ed esplicitati al momento della scelta dello svantaggio.

\textbf{5-10}: quando sei preda del comportamento compulsivo hai un -2 alle prove di Consapevolezza / sei sempre l'ultimo ad agire indipendentemente dall'iniziativa tirata o dall'ordine di marcia / altro

\textbf{Daltonismo}\index{Daltonismo}

Sei cieco ai colori, un tramonto sara' qualcosa di triste visto in grigio

\textbf{5}: non hai la consapevolezza dei colori (acromatopsia). Vedi tutto in scala di grigi.

\textbf{Deformita'}\index{Deformita'}

Non tutti nascono belli o dritti. C'e' anche chi nasce storto e brutto.

\textbf{5}: Malformazione minore, incide a scelta tra Potenza o Agilita'. Togli 1 punto a questa statistica.

\textbf{10}: Due caratteristiche a tua scelta non possono superare i 2 punti se non magicamente. Hai movimento dimezzato.

\textbf{20}: Malformazione grave. Tre caratteristiche a tua scelta non possono superare 1 punto. Hai movimento dimezzato

\textbf{Depressione}\index{Depressione}

Ogni giorno e' un pessimo giorno e nulla lo fara' migliorare

\textbf{8}: Adori il Blues ma purtroppo hai perso la gioia di vivere, l'entusiasmo, la speranza.

Nulla sembra avere importanza, non fai che trascinarti stancamente da un giorno all'altro. -2 ad ogni prova di competenza

\textbf{Dipendenza}\index{Dipendenza}

\textbf{10}: Hai una dipendenza, possa essere alcool, droga, donne...Se non ne consumi ogni giorno una congrua dose (il Narratore ti sapra' dire quanto basta) prendi un -2 a tutti i Tiri Salvezza. Dopo 3 giorni di astinenza divieni anche Depresso

\textbf{Dislessia}\index{Dislessia}

jk j0j zo mdbbdfd

\textbf{10}: Non sei in grado di leggere e scrivere. Non sei capace di associare correttamente suoni a lettere e forme a suoni

\textbf{Disonesta' Compulsiva}\index{Disonesta' Compulsiva}

Menti, e' piu' forte di te.

\textbf{5}: Il personaggio e' portato dalla propria insicurezza a mentire sempre e comunque. Ogni volta che il personaggio e' costretto ad ammettere le proprie responsabilita' o comunque a parlare contro il proprio interesse, o in qualunque situazione in cui si senta ``esaminato'', egli si inventera' storielle piuttosto fantasiose anche mettendo in pericolo amici e parenti.

\textbf{Dolore Cronico}\index{Dolore Cronico}

oh che male. Incatatore usi un'Essenza di cura su di me anche oggi ?

\textbf{10}: non recuperi Punti Ferita se non magicamente

\textbf{Emofilia}\index{Emofilia}

tendi a sanguinare sempre, anche nei momenti meno opportuni

\textbf{8}: CEROTTO!!! (ogni attacco che subisci automaticamente cumula Sanguinamento +1)

\textbf{Epilessia}\index{Epilessia}

sempre e solo nei momenti meno opportuni

\textbf{15}: ogni qual volta fai un 3 con un Tiro Salvezza o un Tiro per Colpire, cadi a terra per 1d6 round in preda alle convulsioni, si considera che il Tiro per Colpire o salvezza sia fallito. Sei considerato indifeso.

\textbf{Feticismo}\index{Feticismo}

Se non annusi un piede di donna diventi depresso.

\textbf{5}: Il personaggio e' irresistibilmente attratto da un oggetto, corpo, categoria... Ogni giorno in cui egli si trova lontano dalla sua fonte di piacere, si consideri caduto in Depressione.

\textbf{Ricordi}\index{Ricordi}

ehi.. ci sei? perche' ti sei paralizzato ? e queste cose quando le hai imparate ?

\textbf{5}: ad ogni check di competenza tira un d4. Con 1-2 fai la prova normale, con 3 fai la prova con un -2, con 4 fai la prova con un +2

\textbf{Fobie}\index{Fobie}

\textbf{Varie}: Il personaggio e' terrorizzato da un oggetto, da una categoria di persone o di esseri viventi, da una situazione. In presenza della causa scatenante, il personaggio cade in preda ad un attacco di panico: l'unico suo desiderio e' quello di fuggire il piu' possibile lontano dalla fonte del suo terrore, con ogni mezzo; chiunque gli sbarri il cammino e' da considerarsi un nemico. Se il personaggio si trova nell'impossibilita' di fuggire, egli cade in uno stato catatonico finche' la causa scatenante non viene eliminata. Vedere in fondo tabella possibili fobie

\textbf{Fotosensibilita'}\index{Fotosensibilita}

La luce anche se leggera ti da fastidio.

\textbf{5}: Il personaggio ha un -1 in ogni tiro in cui la luminosita' e' almeno quella diurna

\textbf{10}: Il personaggio ha un -2 in ogni tiro in cui la luminosita' e' almeno quella di una lanterna

\textbf{20}: Il personaggio ha un -3 in ogni tiro in cui la luminosita' e' almeno quella di una torcia. Il personaggio e' cosi' sensibile che e' per lui impossibile muoversi liberamente in luoghi direttamente o meno illuminati, preferira' muoversi e viaggiare di notte.

\textbf{Ghiro}\index{Ghiro}

ti piace dormire e tanto. Ronf

\textbf{5}: +2 per ogni 2 ore oltre le 8, altrimenti sei affaticato.

\textbf{Goffaggine}\index{Goffaggine}

\textbf{10}:Il punteggio della Agilita' non puo' superare 2. Hai un -2 a tutte le prove che richiedano Agilita' (disattivare congegni, svuotare tasche, arrampicarsi, iniziativa....) 

\textbf{Igenista}\index{Igenista}

ho finito il sapone. HO FINITO IL SAPONE! .. non tocco quella spada, anche se brilla di luce sacra e vola a mezz'aria finche' non sara' disinfettata!

\textbf{5}: hai l'impulso a pulirti di continuo e pulire tutto cio' che dovrai toccare.

\textbf{Incoscienza}\index{Incoscienza}

\textbf{5}: Non hai paura di nulla. Letteralmente. Se devi fare una cosa il piano piu' diretto ed immediato e' la scelta migliore. Non riesci a studiare piani che durino piu' di un minuto. Prendi un +1 all'Iniziativa ed un -2 al Competenza con Armi

\textbf{Indeciso}\index{Indeciso}

Non facciamolo, aspettiamo domani..magari e' meglio!

\textbf{10}: non agisci mai per primo. -4 alle prove di iniziativa

\textbf{Incubi Ricorrenti}\index{Incubi Ricorrenti}

\textbf{8}: Il personaggio non riesce a dormire bene. Ogni notte tira un 1d4. Con 1 il personaggio dorme normalmente, 2 o 3 il personaggio dorme un sonno agitato e si sveglia affaticato, con 4 ti svegli in piena notte urlando, la mattina sei esausto.

\textbf{Libro Aperto}\index{Libro Aperto}

si, lo so, posso stare zitto, tanto avete gia' capito tutto.

\textbf{5}: non e' che non sei in grado di mentire e' che hai un -4 alle prove di Ingannare

\textbf{Emicrania}\index{Emicrania}

Non e' mai un buon giorno. Soffri di continui e feroci mal di testa.

\textbf{15:} Il personaggio soffre di violenti mal di testa.. Ogni giorno il personaggio tira un d4: con 1 il personaggio non lamenta alcun effetto, con 2 o 3 subisce una penalita' di -1 a tutti lle prove, con 4 la penalita' diventa -2.

\textbf{Maledetto}\index{Maledetto}

Sei Maledetto. Un oscuro destino ha macchiato la tua anima

\textbf{5-10}: porti una maledizione. Discutine con il Narratore

\textbf{Miopia}\index{Miopia}

Spera di trovare degli occhiali

\textbf{5}: Ci vedi poco. Hai un -2 a tutti i check competenza con armi da colpire da lontano e prove di Consapevolezza oltre i 12 metri.

\textbf{15}: Ci vedi molto poco. Hai una prova -4 competenza con armi da colpire da lontano e prove di Consapevolezza oltre i 9 metri. 

\textbf{Muto}\index{Muto}

Non puoi parlare e cosa peggiore non riesci neanche ad infamare il
tizio che ti sta pestando il piede

\textbf{10}: Non sei in grado di emettere suoni. Non parli o meglio nessuno ti sente. Prendi un -4 ai check basati su Magnetismo e competenze orali

\textbf{Discalculia}\index{Discalculia}

1+1= ?

\textbf{10}: il personaggio ha un disturbo che gli impedisce di padroneggiare il concetto di numerazione. Non solo non e' in grado di svolgere le operazioni piu' semplici, non e' neanche in grado di comprendere i concetti di maggiore/minore, o informazioni quantitative di qualunque tipo. 
Attenzione al resto che ti danno...

\textbf{Obesita'}\index{Obesita'}

Sei decisamente fuori forma, e di tanto.

\textbf{10}: Agilita' non puo' essere sopra 2. Hai un -4 alla prove di Agilita' ed ai tiri salvezza su riflessi. Guadagni un +2 ai Tiri Salvezza su Tempra

\textbf{Olfatto/Gusto Difettoso}\index{Olfatto/Gusto Difettoso}

Naso, palato, lingua bruciata, abuso di peperoncino o wasabi.. possono essere tante le cause

\textbf{5}: -2 due alle prove che usano gusto od olfatto. Non senti sapori e odori se non estremi.

\textbf{Onesta' Compulsiva}\index{Onesta' Compulsiva}

\textbf{7}: Non sai mentire, la sola idea di dire una menzogna ti rende nervoso. Prendi un -4 a Faccia Tosta. Se messo alle strette, il personaggio confessera' tutto a prescindere dall'importanza delle informazioni in suo possesso.

\textbf{Ossa di Cristallo}\index{Ossa di Cristallo}

Si chiamerebbe osteogenesi imperfetta ma per te sono solo dolori continui.

\textbf{5}: Il personaggio ha le ossa fragili. Ogni danno causato da arma da botta causa 2 PF in piu' di danno

\textbf{10}: Il personaggio ha le ossa fragili. Ogni danno causato da arma da botta causa 5 PF in piu' di danno

\textbf{Paranoioso}\index{Paranoioso}

Sei paranoico e noioso.

\textbf{5}: Ti comporti sempre in modo furtivo, anche senza che ce ne sia effettivo bisogno, destando cosi' sospetti nelle persone che hai attorno.

Ogni prova di Consapevolezza ha una difficolta' di -5 aggiuntiva ed un fallimento indica che il target ha qualcosa di vitale da nascondere.

\textbf{Pelle Sensibile}\index{Pelle Sensibile}

Non ami il Sole, o almeno la tua pelle non lo ama.

\textbf{5}: Il tuo personaggio si scotta facilmente, un'esposizione prolungata senza le adeguate protezioni comporta dolorose e antiestetiche bruciature e disagi.

\textbf{10}: Sei oltremodo sensibile agli ultravioletti. Ogni danno da fuoco causa 2 danni aggiunti.

\textbf{Pigro}\index{Pigro}

sei lento e svogliato

\textbf{5}: -2 all'iniziativa

\textbf{Rumoroso}\index{Rumoroso}

Non lo fai apposta, ma c'e' sempre un qualche rumore intorno a te. Una spada che sbattocchia, uno sbadiglio, un rutto, una scarpa rumorosa..

\textbf{5}: hai un -4 alle prove di muoversi silenziosamente

\textbf{SANGUE DEBOLE}\index{SANGUE DEBOLE}

\textbf{10}: Il sistema immunitario del personaggio fa decisamente pena. -2 ai Tiri Salvezza su Tempra

\textbf{Sbadataggine}\index{Sbadataggine}

Ops..non me ne ero accorta!

\textbf{7}: Tendi a non fare caso a quello che succede intorno a te, meno che tu non abbia ottimi motivi per stare all'erta, o non stia cercando attivamente qualcosa prendi un -4 a Consapevolezza

\textbf{Schizofrenia}\index{Schizofrenia}

Non sono stato io, ma l'altro!

\textbf{4}: Hai piu' personalita', o forse ne e' convinto l'altro.

Il personaggio ha almeno una seconda personalita' (max 6).
Ogni Personalita' in piu' da gestire, oltre la prima, concede un +1 al costo.
Quindi avere 3 personalita' porta' lo svantaggio a 6 punti

Ogni giorno viene tirato 1d6. Con 6, durante il giorno la seconda personalita' viene alla luce.

\textbf{Sfortunato}\index{Sfortunato}

le cose non capitano e basta, bisogna saperle anche cercare

\textbf{5}: ignori il primo critico che fai (TC o TS) nella giornata

\textbf{7}: ignori i primi tre critici che fai (TC o TS) nella giornata

\textbf{Sindrome Maniaco Depressiva}\index{Sindrome Maniaco Depressiva}\index{Incoscienza}\index{Depressione}

Oggi e' venerdi'!!! e' Venerdi!!!

\textbf{7}: Il personaggio alterna stati di euforia a momenti di cupa disperazione. Ogni giorno viene tirato 1d4. Con 1 il personaggio ha un umore ``normale''. Con 2 o 3 si consideri in Depressione , con 4 e' in uno stato di gioiosa esaltazione (vedi Incoscienza ) e spavalderia.

\textbf{Soggezione}\index{Soggezione}

chiedo scusa

\textbf{10}: Il personaggio e' molto insicuro e tende a fidarsi ciecamente degli altri, specie se carismatici . Prendi un -2 alle prove di Criminalita' e Faccia Tosta.
Prendi un -2 ai Tiri Salvezza su Essenza Charme

\textbf{Sonno Leggero}\index{Sonno Leggero}\index{affaticato}

ogni rumore di disturba, non riesci mai a dormire bene

\textbf{5}: Se dormi in una zona con rumori naturali / umani (bosco/citta') non riesci a riposare bene. La mattina sei Affaticato. Puoi evitare il problema usando tappi per le orecchie, che ti impongono un -4 alle prove di Consapevolezza su udito per svegliarti.

\textbf{Sordita'}\index{Sordita'}

Il silenzio ha un suono tutto suo dice chi ci sente, per te e' solo uno straziante urlo muto.

\textbf{10}: Non ci senti. Non puoi fare prova di Consapevolezza che richiedano l'uso dell'udito. Non puoi ascoltare le persone che parlano. Ma puoi leggere le labbra se sai farlo.

\textbf{Vertigini}\index{Vertigini}

I disagi si manifestano nel momento in cui il personaggio e' conscio dell'altezza. Solo per il fatto di camminare in montagna non ha penalita'

\textbf{5}: Ad altezze superiori i 20 metri tendi a bloccarti. Prendi un -2 a tutti i check

\textbf{7}: Ad altezze superiori i 10 metri tendi a bloccarti. Prendi un -2 a tutti i check

\textbf{10}: Ad altezze superiori i 6 metri tendi a bloccarti. Prendi un -2 a tutti i check

\textbf{Visione notturna ridotta}\index{Visione notturna ridotta}

I tuoi occhi non lavorano bene con luminosita' ridotta.

\textbf{5}: Quando la luminosita' e' pari o inferiore a quella di una torcia il personaggio ha un -2 ai Tiro per Colpire.

\textbf{Timidezza}\index{Timidezza}

\textbf{5}: Sei timido e riservato.

Hai un -2 alle prove basate su Magnetismo

\textbf{Zoppo}\index{Zoppo}

sei claudicante

\textbf{5}: Hai movimento dimezzato

\textbf{10}: sei significativamente storpio. -2 alle prove che richiedono Agilita', il tuo movimento e' dimezzato

\bigskip

\textbf{Tabella Fobie (5-15 punti)}\index{Fobie}

\begin{tabular}[c]{@{}ll@{}}
\toprule 
\textbf{Nome Fobia} & \textbf{Descrizione}\tabularnewline
Blennofobia & Paura Delle Cose Viscide\tabularnewline
Keraunofobia & Paura Dei Tuoni\tabularnewline
Ipocondria & Paura Delle Malattie\tabularnewline
Claustrofobia & Paura Dei Luoghi Chiusi\tabularnewline
Coimetrofobia & Paura Del Cimitero\tabularnewline
Edonofobia & Paura Di Poter Provare Piacere Fisico\tabularnewline
Eisoptrofobia & Paura Degli Specchi\tabularnewline
Glossofobia & Paura Di Parlare In Pubblico\tabularnewline
Monofobia & Paura Di Rimanere Solo\tabularnewline
Necrofobia & Paura Dei Corpi Morti\tabularnewline
Nictofobia & Paura Del Buio\tabularnewline
Acrofobia & Paura Delle Altezze\tabularnewline
Agorafobia & Paura Degli Spazi Aperti\tabularnewline
Rupofobia & Paura Dello Sporco E Non Igienico. Devi pulire\tabularnewline
Afefobia & Paura Del Contatto E Di Essere Toccati\tabularnewline
Asimmetrofobia & Paura Delle Cose Non Simmetriche\tabularnewline
Gimnofobia & Paura Della Nudita'\tabularnewline
Emofobico & Paura Del Sangue\tabularnewline
Traumatofobia & Paura Di Ferirsi\tabularnewline
Sciofobia & Paura Delle Ombre\tabularnewline
\bottomrule
\end{tabular}

\pagebreak

\section{Cosmologia}\index{Cosmologia}

\label{cosmologia}
\begin{quotebox}
e' piu' facile dominare su chi non crede in niente (La Storia Infinita, Kmorf)\\
Tu credi che c'e' un Dio solo? Fai bene; anche i demo'ni lo credono e tremano! (Giacomo Il Giusto 2, 19. NdA Riferendosi al proprio Patrono...)
\end{quotebox}

In TUS le divinita' sono leggermente diverse dalle tradizionali divinita' dei giochi di ruolo.

In TUS le divinita' amano sporcarsi le mani, partecipare nelle faccende delle creature che le adorano, per loro e' una sfida continua ad avere piu' credenti, adepti e persone piu' simili, per tratti, a loro.

I Patroni sono stati creati come parossismo dell'animo umano, dove tutto e' un eccesso. Come spiriti liberati dal vaso di Pandora hanno il solo scopo di portare i loro Tratti al dominio rendendoli i piu' comuni e presenti tra le creature.

\noindent\rule{\textwidth}{1pt}

\bigskip

In principio era il nulla che in se' conteneva il tutto.

Energia derivante dalle piu' primordiali pulsioni esplodeva in tutta la sua potenza e senza alcun controllo.

Amore, odio, paura, dolore, gioia, serenita'...tutto era aggrovigliato in una fitta ed infinita matassa il cui bandolo era nascosto, attorcigliato, introvabile o...probabilmente ancora inesistente.

Nei millenni a seguire dette energie e pulsioni hanno iniziato a scindersi, trovando una propria personale connotazione e si sono venute a creare tre entita': Atmos, colui che per suo principio controllava l'andamento del tempo e dello spazio, il testimone, lo scriba; Ljust l'energia positiva, il calore, la luce e la vita; Calicante, energia negativa, gelido odio, distruzione e morte.

Atmos e' una sorta di spettatore imparziale mentre Ljust e Calicante rappresentano le due lingua di fiamma di un unica energia creatrice.

\textbf{Ljust} \index{Cosmologia}e' la rappresentazione di cio' che luce ed amore portano sempre con se'. Rappresenta la purezza del sentimento d'amore, la protezione della vita, il rispetto per l'altro, la curiosita' per il nuovo, la voglia di migliorarsi sempre, la forza di combattere con coraggio e valore per il proprio credo.

\textbf{Calicante}\index{Calicante} e' la rappresentazione del buio, dell'odio e della rabbia. Lui e' vendetta e fredda distruzione. Lui non protegge alcuna forma di vita, le usa, le sfrutta e solo in tali casi ne subisce la presenza. Lui ama sadicamente la sofferenza. 
 
\textbf{Atmos} \index{Atmos}e' lo storico, colui che segna il passaggio del tempo e trascrive ogni accadimento di Yeru.

E' il testimone del groviglio divino che sono Ljust e Calicante, due creature unite da un unica energia.

Insieme i due Patroni della Genesi hanno dato vita a tutto cio' che conosciamo. Calicante ha creato Tiya\index{Tiya} ed Ljust ha generato Curyan\index{Curyan}, i due regni che compongono il nostro mondo, Yeru. Hanno giocato con forme ed energie creando due regni fra loro speculari ma contrapposti e distinti. Tiya e Curyan, come Calicante ed Ljust, sono parte di un tutto ma, esattamente come i Patroni delle Genesi, sono anche profondamente diversi e magicamente divisi. Esiste infatti una barriera sia fisica formata da catene montuose quasi invalicabili, tempeste marine perenni e mortali ma anche magiche, che ne delimitano i confini e che li tiene nettamente divisi.

Ma proprio come i loro due creatori che divisi e distanti totalmente non possono stare, non possono esistere, cosi' Tiya e Curyan sono si' divisi ma anche in contatto fra loro tramite i Portali. Portali che si generano autonomamente, senza alcun controllo e previsione, a causa dell'entropia magica che preme, spinge e si autoalimenta nel ``non luogo'' al confine dei due regni e che e' generata dalle continue sfide fra i due Patroni. Sono queste magiche vie che consentono di spostarsi tra Tiya e Curyan e viaggiare nel ``non luogo'' ovvero cio' che e' al di fuori di Yeru.

Ljust e Calicante decisero, stranamente di comune accordo, di generare un Patrono che sovrintendesse a queste fratture, che fosse capace di percepire, aprire e bloccare questi Portali. Cosi' venne creato \textbf{Lynx}, il Guardiano dei Portali.

In molti cercano di passare da Tiya a Curyan per cercare la pace, la serenita'...altri cercano di valicare il confine inverso alla ricerca di avventura, potere\ldots .alcuni ci provano per vie normali, altri attraversando i Portali, molti si sono persi per sempre nel ``non luogo''.

Lynx \index{Lynx}sovrintende allo spazio, ai portali che con l'avvicendarsi di caos e ordine, di bene e male, di luce e tenebra stanno sempre piu' creando fratture al confine esistente fra i due regni. Lynx li percepisce, li ``sente'', sa dove si stanno generando o riassorbendo, con il passare del tempo infatti alcuni di questi Portali sono divenuti stabili e definitivi mentre altri sono stati riassorbiti dalla naturale entropia, altri invece si generano casualmente e sempre in modo totalmente casuale rimangono attivi o si esauriscono. Viaggiando di continuo nel non luogo Lynx chiude i portali piu' grandi ma per uno che ne chiude un altro si apre.

Proprio nello svolgimento di questo suo importante ruolo Lynx si scontro' con una strana creatura, rettiloide, gigantesca, alata, potente, forte, sapiente e magica.

Un Drago rosso,\index{Ta'hil} Ta'hil. Quest'ultimo si muoveva nel ``non luogo'' con la massima liberta', senza alcuna difficolta' e si avvicino' a Lynx. I diari di Atmos narrano di come Lynx cerco' di fermarlo e di parlarci, di come venne ferocemente attaccato, delle urla del Patrono Custode che si sentirono echeggiare in entrambi regni, del suono quasi simile ad un gutturale ruggito che squarcio' il silenzio nei regni di Tiya e Curyan. Dell'intervento di Ljust e Calicante. La prima a salvare Lynx ed il secondo a scoprire, conoscere questa nuova affascinante ``arma'.

Lynx si salvo'. Ljust gli infuse le sue essenze di cura e lo aiuto' a rigenerarsi. Si lascio' pero' sfregiato a memoria dell'incontro.

I Draghi avendo scoperto il nostro mondo e mossi dalla loro sete di conoscenza e di potere si sono avventurati nel ``non luogo'' e sono usciti dai Portali presenti su Tiya e Curyan.

Un'orda di draghi di tutti i colori ha oscurato i cieli. Da quel momento saccheggi, razzie e violenza furono perpetrati indifferentemente nei due regni. Erano molto intelligenti e furbe. Potenti oltre l'immaginabile, manipolavano una magia per alcuni versi diversa e arcana slegata dai singoli Regni magici.

Avevano una robustezza fuori dall'ordinario. Ma soprattutto, non temevano i Patroni. Non si sottomisero a loro.

Atmos incanalo' le energie primordiali e divine dei Patroni della Genesi andando a creare delle divinita' che potessero rivaleggiare con i draghi e potessero difendere Yeru.

Il primo creato da Atmos, con l'aiuto di Ljust fu \textbf{Gradh}\index{Gradh}, Patrono dell'Umanita' (e di tutte le razze senzienti), colui che avrebbe difeso il creato dai draghi e dagli altri Patroni.

Gradh racchiude in se' il dualismo dei due Patroni della Genesi, l'istinto innato alla protezione, alla difesa ed alla cura propri di Ljust e l'istinto di vendetta, violenza e furia di Calicante.

Si getta con coraggio nelle battaglie, attacca il nemico senza paura, protegge il piu' debole, difende la vita ma non teme di percorrere la strada della vendetta piu' distruttiva verso chi sfrutta e distrugge vite senza motivo.

Gradh ama ``calarsi'' fra la gente e vivere con loro, come loro. 
Non si sente totalmente a suo agio ne' nel Pantheon con gli altri Patroni ne' fra la gente comune, lui e' Umano tra i Patroni e Patrono tra gli Umani. Passionale e gentile e' il Patrono che maggiormente ha a cuore le sorti di Yeru e delle sue razze.

Le lingue di energie divine erano troppo intense, chaotiche e pure perche' Atmos potesse governarle per plasmare da solo gli ulteriori Patroni con la stessa naturalezza con cui aveva creato Gradh. Ed ecco che questi altri Patroni risultano meno perfetti e divini, piu' imperfetti ed ``umani'' in quanto originati da pulsioni violente e incontrollabili, vere e non filtrate in alcun modo.

Ogni Patrono ha un accesso limitato o negato ad alcune Essenze, non sono piu' liberi e totali manipolatori dell'energia arcana e divina legata ad un definito reame magico ma al contrario hanno un accesso limitato e vincolato alla magia. 
I Patroni plasmano le volonta', fondano regni, comandano nell'ombra come pedine le creature che osano chiedere i loro favori.

Gradh percepi' sin da subito che i Draghi rappresentavano un elemento di ulteriore chaos, di ulteriore sofferenze e guerra. Come Patrono di Yeru e delle sue creature sentiva i Draghi come creature aliene, non originarie, non facenti parte del piano della Genesi.

Diffidente per natura Gradh decise di proporre ai Patroni della Genesi di fare un patto con i Draghi.

Ecco che poco piu' di 19 anni fa, il 15 febbraio 2000, sull'isola di Atilantis che divide Tiya e Curyan si trovarono Atmos, la fiamma di Ljust e Calicante e Gradh da una parte mentre Ta'hil, il drago rosso malvagio e immortale e Dyenos\index{Dyenos}, il drago d'argento sapiente e buono dall'altra.

Gradh cerco' di imporre la cacciata dei Draghi e la chiusura dei Portali, Atmos rimase in silenzio a trascrivere la discussione. Ljust cerco' di mediare capendo che non tutti i Draghi erano malvagi e che avrebbero potuto dare tanto a Yeru.

Calicante finse di dare ragione a Ljust con il solo scopo di portare maggior caos e distruzione attraverso i Draghi.

Capito che l'esito dell'incontro era gia' deciso Gradh abbandono' la Piana della Solitudine lasciando ai Draghi ed ai Patroni della Genesi di formalizzare la spartizione di Yeru. Per lui era stata una sonora sconfitta e da allora fu ancora di piu' e' diffidente, se non prevenuto, verso tutti i draghi.

Calicante e Ta'hil rimasero strettamente in contatto e si stanziarono a Tiya mentre Dyenos giuro' fedelta' e fiducia a Ljust e decisero di governare insieme Curyan, grazie allo loro volonta' di calarsi tra gli umani.

Se un Patrono agisce in prima persona o in modo indiscriminato sa che scatenera' la reazione di Gradh o l'intervento di Atmos che gli impediranno un uso incontrollato e massivo dei suoi poteri direttamente sul mondo. Questo pero' non sempre li ferma e la stessa natura creature e piante, vengono spesso influenzate dal volere dei Patroni.

A Tiya, ma a volte anche a Curyan, nascono sempre piu' spesso aberrazioni, malattie sempre nuove, terre maledette dove non puo' crescere nulla, per non parlare di pazzie che spesso coinvolgono chi invece dovrebbe proteggere i comuni cittadini.

E' una dura vita quella dell'uomo comune che continuamente deve affrontare siccita' o alluvioni, morie di animali ed un meteo irregolare se non assurdo. Ad ogni passo deve guardarsi intorno perche' non puoi mai sapere chi ha venduto l'anima per vivere un giorno in piu'.

A Curyan si vede svilupparsi l'armonia e la quasi perfetta convivenza fra natura e razze superiori. Esiste il dolore, esiste la malattia e la morte ma il tutto come naturale ciclo della vita come parte della stessa che viene protetta, guidata, aiutata.

I nemici principali sono i Draghi che spesso fanno incursioni per portare distruzione e morte e seminare paura e pazzia.

Non e' sempre tutto idilliaco, vaste regioni di Curyan stanno diventando incubatrici di razze oscure e malvagie, legioni di non morti guidate da potenti negromanti si ammassano sui confini, i Draghi addestrano i loro adepti corrotti, e scure spire nere nel cielo promettono tempesta.

\pagebreak

\subsection{Patroni (Dei)}\index{Patroni}

\label{patroni-dei}
\begin{quotebox}
Conan: A quali dei preghi?\linebreak
Subotai: Io prego ai quattro venti e tu?\linebreak
Conan: Io prego Crom, ma solo raramente... lui non ascolta. (Conan il Barbaro, film 1982)\linebreak\linebreak
Infatti, come il corpo senza lo spirito e' morto, cosi' anche la fede senza le opere e' morta. (Giacomo Il Giusto 2, 26. NdA Riferendosi ai punteggi dei Tratti collegati al Patrono...)\end{quotebox}

Le creature tutte, anche chi non usa le essenze possono sentire l'influenza di questi Poteri, di questi Patroni.

Se un personaggio per il suo modo di essere (giocare) e comportarsi ha almeno un tratto in comune con un Patrono ed anzi matura e potenzia queste convinzioni, anche se non ha giurato fedelta' ad un Patrono potrebbe comunque sentire l'influenza del Patrono e ricevere dei doni da lui.

Un Patrono e' ben contento se qualcuno segue i suoi dettami, Tratti, senza che sia un usufruitore di magia e dona a coloro che lo fanno dei piccoli poteri come riconoscimento per la fedelta' a lui riservata, volutamente o meno. I poteri indicati sotto ``Tratti in Comune'' sono cumulativi.

Ad ogni Patrono troverete associate le Essenze (privilegiate, normali, limitate e negate), troverete i Tratti che li caratterizza ed anche due forme di Energia, che rappresentano la loro forma di attacco tipico. 

Ricordatevi di usare questi Elementi quando effettuate un'Essenza di Attacco, o un incanalare energia, altrimenti non raggiungerete il Livello di Potere prefissato, tranne se userete un elemento neutrale. 

Le forme di Energia vengono distinte tra fonti positive, neutrali e negative, vi servono anche per inquadrare meglio il vostro Padrone pardon il Patrono che servite.

Fate la somma degli elementi, se positiva il Patrono si puo' considerare buono, se a valore zero il Patrono e' neutrale, se a valore negativo il Patrono e' malvagio.

Nella descrizione del Patrono troverete anche la sua manifestazione, ovvero cosa accade quando un personaggio agisce in maniera particolarmente e significativamente consona ai tratti seguiti dal Patrono. L'effetto e' puramente ambientale e di circostanza ma lascia sempre colpito chiunque lo possa osservare.

Un incantatore che si affida ad un Patrono, almeno 3 Tratti in comune, diventando quindi un Devoto, \index{Devoto}segue le normali regole delle Essenza tenendo pero' conto dei vantaggi (essenze favorite) e svantaggi (essenze limitate e negate) che comporta.

\bigskip

\textbf{Tabella Elementi}

\begin{tabular}[c]{@{}lll@{}}
\toprule 
Positivi (+1) & Neutrali (0) & Negativi (-1)\tabularnewline
Energia Positiva & Fuoco & Energia Negativa\tabularnewline
Radiosa & Elettricita' & Suono\tabularnewline
\bottomrule
\end{tabular}
\bigskip

Ogni Patrono concede delle Essenza privilegiate, neutrali, limitate o negate.

Sulla \textbf{Essenza privilegiata} \index{Essenza privilegiata}l'incantatore prende un +4 alle prove di CM

Sulla \textbf{Essenza normale}\index{Essenza normale} non ha bonus/malus

Sulla \textbf{Essenza limitata}\index{Essenza limitata} ha un -2 alle prove di CM

Le \textbf{Essenze negate} \index{Essenze negate}non sono accessibili/sceglibili, ovvero non puo' investire punti competenza magica per prendere queste essenze e non potra' mai lanciare magie da pergamena da queste essenze.

Utilizzare i poteri concessi dai Tratti costa 2 Azioni se non specificato diversamente.

I Patroni sono:

\subsubsection{Ljust}

\label{ljust}\index{Ljust}

la Dama della Luce, colei che irradia calore e amore. Generatrice delle pulsioni d'amore, protezione, gentilezza, gioia e perdono. Racchiude in se' l'aspetto protettivo di una madre, la forza e l'audacia di una combattente, la passionalita' di una giovane amante e l'allegria, la ricerca del nuovo, la fantasia di una bambina. Ljust incarna la bellezza della vita ed ogni creatura che la contempla vede quella che per lei e' la massima armonia e cade prona al suo fascino.

Ljust puo' essere scelta solo da un personaggio con 4 tratti in comune con lei, fondamentalmente si nasce per essere Devoti di Ljust. Nel corso dei decenni Ljust decise di selezionare, scegliere e premiare le donne che piu' mostravano in modo innato e profondo amore per la vita, curiosita' per il nuovo, forza incrollabile, dedizione, fiducia, rispetto e cura degli altri donando loro i poteri e la possibilita' di studiare e crescere come Allieve della Luce. Queste Allieve devono seguire la regola degli 8 Passi.

\textbf{Simbolo:} una stella circondata da raggi solari

\textbf{Tratti}: Coraggioso, Generoso, Empatico, Protettivo

\textbf{Manifestazione}: luce dorata inonda l'incantatore.

\bigskip

Tratto in comune a 5 punti: un oggetto che tocchi diventa luminoso come una torcia (3 metri di raggio) per 1 ora. Due volte al giorno

Tratto in comune a 10 punti: guadagni un +2 ai Tiri Salvezza contro Distruzione

Tratto in comune a 15 punti: una armatura di luce di protegge, guadagni un +2 a tutti i Tiri Salvezza

Tratto in comune a 20 punti: puoi emettere una luce che causa 10d6 di danno. Distanza 54 metri, TS Riflessi DC 25 per dimezzare, un target

\bigskip

Elementi: Energia Positiva, Radiosa

\bigskip

Essenze Privilegiate: Cura, Protezione (+4 CM)\\
Essenze Normali: Difesa, Charme, Attacco (+0 CM)\\
Essenze Limitate: Creazione, Rivelazione, Alterazione, Movimento (-2 CM)
Essenze Negate: Convocazione, Illusione, Distruzione, Trasformazione (essenze non accessibili)

\paragraph{Gli 8 Passi delle Allieve}\index{8 Passi delle Allieve}\index{Allieve}

\label{gli-8-passi-delle-allieve}

Le Allieve della Luce sono una gruppo segreto di Devote che per totale affinita' con Ljust hanno intrapreso il duro percorso del bene e dell'amore. E' tra i gruppi piu' antichi fondati a Yeru.

Le Allieve, 99 come numero massimo, ma purtroppo spesso meno numerose, sono Devote di Ljust e devono seguire gli 8 Passi della Luce

\begin{enumerate}
	\item Ama e proteggi con tutta te stessa, con totale e sincera dedizione chi hai attorno a te.
	
	\item Non lasciare che la tua inazione generi sofferenza.

	\item Si un punto di paragone. Fai che la tua Luce elevi le persone che hai intorno e possano vedere in Tu sei speranza, serenita', calma, protezione e sicurezza.

	\item Usa l'intelligenza, la furbizia e l'arguzia. Si lungimirante e risoluta nell'azione.

	\item La tua opera e' per il bene comune. Fa che la tua Luce sia sempre alta ed intensa.

	\item Non cercare altra Luce se non la tua e quella delle tue sorelle.

	\item Sii luminosa ma non accecare chi e' intorno a te.

	\item Sii la differenza tra la disperazione e la speranza.
\end{enumerate}

Le Allieve hanno costruito un ballo armonioso trasformando in danza i passi della loro Regola.

\subsubsection{Calicante}\index{Calicante}

\label{calicante}

e' oscuro, gelido e arrabbiato. Racchiude in se' odio, violenza, distruzione, vendetta e perenne insoddisfazione. Raccoglie la personalita' capricciosa e scontenta di un bambino, la noia violenta e sadica di un giovane uomo e la forza distruttiva di un uragano e la rabbia di un combattente che non ha piu' nulla da perdere. Calicante solo con la presenza mette a disagio, ti fa sentire in pericolo, affascina ma con le armi della paura e della incostanza.

Calicante puo' essere scelto solo dai personaggi che hanno 4 tratti in comune con lui. I suoi Devoti sono i migliori assassini, sua professione piu' affine. Coloro che mostrano il maggiore sprezzo del pericolo e della vita altrui. I suoi prediletti sono coloro che sono temuti, odiati\ldots coloro che sono violenti e crudeli ma mortalmente efficienti e decisivi in ogni situazione di combattimento.

\textbf{Simbolo}: un turbine nero

\textbf{Tratti}: Egoista, Vendicativo, Superbo, Iracondo

\textbf{Manifestazione}: spada grondante sangue nero

\bigskip

Tratto in comune a 5 punti: Puoi creare una zona di oscurita'. Raggio 1 metro, entro 9 metri, durata 10 minuti. Una volta al giorno

Tratto in comune a 10 punti: La tua lama si ammanta di ombra. Guadagni un +2 al Tiro per Colpire e +1d4 di danno per 2d6 round

Tratto in comune a 15 punti: Crei 4 dardi di energia negativa. Ogni dardo fa 2d6 di danno, colpisce automaticamente entro 18 metri. Una volta al giorno

Tratto in comune a 20 punti: Crei una zona di energia negativa attorno a te nel raggio di 3 metri, dimezzi tutto il danno che ricevi, non puoi curarti nel mentre. Durata 10 minuti consecutivi, 1 volta al giorno.

\bigskip

Elementi: Energia Negativa, Fuoco

\bigskip

Essenze Privilegiate: Distruzione,Attacco

Essenze Normali: Illusione, Charme, Trasformazione

Essenze Limitate: Creazione, Difesa, Convocazione, Movimento

Essenze Negate: Protezione, Alterazione, Cura, Creazione

\subsubsection{Atmos}\index{Atmos}

\label{atmos}

il custode del Tempo e della Torre dell'Orologio, come ha avviato il tempo e la creazione dei nuovi Patroni cosi'' fermera' la sfida fra loro, i Patroni sopravvissuti saranno giudicati, le loro opere valutate e Ljust o Calicante ne trarranno giovamento. Come una sfida da una singola moneta di rame nuovi Patroni, nuovi ideali saranno creati e noi, piccole creature vedremo nascere nuove civilta' e regni fiorenti. La storia e' poco nota, solo i pochi Devoti di Atmos, scribi e studiosi della biblioteca del Tempo, conoscono il segreto e lo scorrere del tempo e della gara, gli altri, ignoranti, vivranno il loro tempo con un padrone sicuramente guidato da un Patrono.

Atmos, il Patrono del Tempo e' il custode della storia e del tempo, e' colui che tiene traccia dei mille e piu' mondi che sono stati creati.

Ha il compito di avviare ed interrompere il tempo. Atmos ha il potere unico e riservato solo a lui di poter bandire dal creato un Patrono qualora questo diventasse troppo forte e minacciasse Calicante e Ljust. Atmos ha gia' usato questo potere in passato. Atmos sia per la sua natura totalmente neutrale sia per il suo ruolo non si e' mai schierato. Tutti i Patroni temono Atmos per il suo potere, il piu' terribile per loro, ovvero il loro alienamento, l'oblio, la dimenticanza l'essere distolti dal tempo e dalla sfida.

Per essere un Devoto di Atmos al momento del rito e' necessario che il futuro Devoto possieda almeno quattro tratti in comune con lui, e amare la storia e la conoscenza.

Vestito di un morbido saio marrone e calzari di cuoio si muove tra gli infiniti scaffali della Biblioteca del Sapere con sempre uno strano misuratore del tempo appeso alla vita.

\textbf{Simbolo:} un libro bianco con un orologio da taschino appoggiato
sopra

\textbf{Tratti}: Osservatore, Distaccato, Studioso, Riflessivo

\textbf{Manifestazione}: l'Essenza si sviluppa come a rallentatore, in realta' e' solo un effetto illusorio

\bigskip

Tratto in comune a 5 punti: Conosci sempre la data esatta e l'ora.

Tratto in comune a 10 punti: Hai una intuizione innata per la conoscenza. Hai +1d6 alle prove di Cultura. Arcano prende un bonus +2

Tratto in comune a 15 punti: Puoi creare 8 tue immagini speculari per trarre in inganno i tuoi avversari. Una volta al giorno, durata 1 ora o finche' colpite.

Tratto in comune a 20 punti: Ogni qual volta che devi fare una provadi Cultura o Arcano puoi prendere il 16 come prendessi 10

\bigskip

Elementi: Suono, Radiosa

\bigskip

Essenze Privilegiate: Rivelazione, Creazione

Essenze Normali: Illusione, Cura, Attacco

Essenze Limitate: Charme, Difesa, Convocazione, Distruzione

Essenze Negate: Protezione, Alterazione, Trasformazione, Movimento

\subsubsection{Lynx}\index{Lynx}

\label{lynx}

Patrono dei Portali, e' sceglibile solo da personaggi che abbiano almeno 3 tratti in comune. E' il primo Patrono generato da Ljust e Calicante, creato per proteggere Yeru.

Serio, sguardo gelido di un azzurro chiarissimo e' il Custode dei Portali e di cio' che e' Oltre. Letale guardiano per chi cerca di passarli senza permesso, guida attenta per chi chiede il suo aiuto ed il suo permesso. Si fa scudo delle sue cicatrici per allontanare tutti. 
Solitario controllore del mondo.

I suoi Devoti sono i viaggiatori per eccellenza, coloro che presidiano e proteggono Yeru da cio' che e' alieno, da cio' che potrebbe disturbare la creazione.

\textbf{Simbolo}: un portale

\textbf{Tratti}: Solitario, Serio, Rigido, Controllato

\textbf{Manifestazione}: come se il panorama non avesse piu' orizzonte

\bigskip

Tratto in comune a 5 punti: Una volta al giorno puoi eseguire una Azione di movimento in piu'

Tratto in comune a 10 punti: Acquisisci una Azione di movimento in piu' a round

Tratto in comune a 15 punti: Puoi toccare una creatura extraplanare e costringerla a tornare sul suo piano. TS su Arbitrio DC 30. Una volta al giorno
 
Tratto in comune a 20 punti: Puoi teletrasportarti per 500km al giorno (anche piu' teletrasporti purche' la somma totale non superi 500km)

\bigskip

Elementi: Fuoco, Elettricita'

\bigskip

Essenze Privilegiate: Movimento

Essenze Normali: Protezione, Convocazione, Rivelazione

Essenze Limitate: Cura, Difesa, Attacco, Illusione

Essenze Negate: Charme, Alterazione, Trasformazione, Creazione, Distruzione

\subsubsection{Gradh}\index{Gradh}

\label{gradh}

Il primo Patrono creato da Atmos sotto la guida di Ljust e l'influenza di Calicante.

Gradh racchiude in se' l'istinto innato alla protezione, alla difesa ed alla cura propri di Ljust. Gradh e' quanto di piu' simile e profondamente legato a Ljust sia stato generato. Lui e' equilibrio, razionalita' ed empatia.

Dove vi e' difesa, cura e protezione e creazione vi e' Gradh.

Gradh non ama sfidare apertamente Cattalm perché sa che farebbe esattamente il suo gioco, ecco che con astuzia cerca di attirarlo nel suo terreno di gioco, dove nessuna vita sara' in pericolo e li' da sfoggio a della sua superiorita' strategica e di combattimento.

Ma Calicante non poteva permettere la creazione di un Patrono totalmente votato ad Ljust e cosi' infuse in Gradh la freddezza della vendetta e la furia della rabbia. Ecco che allora Gradh nell'atto di difendere l'umanita', spesso la deve in primis proteggere da sé stesso.

Passionale e freddo e' forse il Patrono piu' umano del Pantheon attuale. Il suo sguardo caldo e carismatico che quando ama e protegge e' di un rassicurante color cioccolato, puo' divenire freddo e tagliente con le sfumature della fredda terra ghiacciata quando e' preda della furia della battaglia o della vendetta. Gradh ama studiare il mondo attorno a se' e passare inosservato. Spesso si nasconde fra la gente e "vive" la sua vita umana. Ma non si lascia avvicinare veramente da nessuno.

Gradh attira a se' con la stessa facilita' con cui allontana da se'.

Il Devoto di Gradh e' fiero ed orgoglioso, indomito e protettivo, ed addolorato, perche' per quanto si sforzi di punire il male questo continua sempre a prosperare.

\textbf{Simbolo}: uno scudo con incise sopra due spirali intrecciate.

\textbf{Tratti}: Indomito, Protettivo, Vendicativo, Coraggioso

\textbf{Manifestazione}: due spire una nera come ombra ed una lucente come scintilla circondano la sua arma intrecciandosi

\bigskip

Tratto in comune a 5 punti: Il tuo tocco cura 3d6 PF, ma ti causa 1d6 di danno. 2 volte al giorno

Tratto in comune a 10 punti: Per 10 minuti hai un bonus di +4 Ts su Riflessi e Tempra. Una volta al giorno

Tratto in comune a 15 punti: Emani un aura che concede a tutti i tuoi compagni entro raggio 3 metri un +2 TS. Una volta al giorno, per 30 minuti consecutivi

Tratto in comune a 20 punti: Esplodi la tua ira in una palla di energia negativa. 10d6 di danno, raggio 6 metri entro 36 metri. Una volta al giorno

\bigskip

Elementi: Energia Positiva - Energia Negativa

\bigskip

Essenze Privilegiate: Protezione, Attacco

Essenze Normali: Cura, Difesa, Creazione

Essenze Limitate: Alterazione, Trasformazione, Movimento

Essenze Negate: Charme, Illusione, Distruzione,Rivelazione, Convocazione

\subsubsection{Atherim}\index{Atherim}

\label{atherim}

il Patrono custode. Molti vedono nel seno generoso di Atherim un segno di volutta' e passione. Si lasciano incantare dalla sua procace bellezza e non vedono gli occhi di cristallo che incutono timore a chi osa anche solo pensare di avvicinarla.

Atherim e' la custode dei sogni e delle speranze, colei alla quale affidare, come ad una madre, i desideri. E' il Patrono dei Bambini, dei Segreti e delle Levatrici.

Dal sorriso allegro e dall'animo buono sara' sempre pronta ad aiutarti a realizzare i tuoi sogni. E come una madre Atherim protegge e custodisce i segreti e le passioni. Atherim e' muta. E' colei che custodisce per sempre, dentro il suo animo i segreti di Yeru.

Il Devoto di Atherim si prende a cuore coloro che hanno fatto una promessa, punisce chi le infrange e chi svela i segreti. Molti Devoti di Atherim sono diplomatici, notai e levatrici.

\textbf{Simbolo:} una mano di donna guantata che tiene un'ampolla ricca di flussi

\textbf{Tratti}: Allegro, Calmo, Industrioso, Buono

\textbf{Manifestazione}: un silenzio sereno e tranquillizzante cala attorno all'incantatore

\bigskip

Tratto in comune a 5 punti: Puoi aggiungere 1d6 ad un Tiro salvezza dopo averlo tirato ma prima di sapere se ha avuto successo o meno. Una volta al giorno, come Reazione.

Tratto in comune a 10 punti: Guadagni 30 pf temporanei. Durata 1 ora, una volta al giorno, come azione immediata.

Tratto in comune a 20 punti: Ogni pozione che bevi fa il doppio di durata o effetto se immediata.

\bigskip

Elementi: Energia Positiva, Elettricita'

\bigskip

Essenze Privilegiate: Difesa

Essenze Normali: Cura, Movimento, Protezione

Essenze Limitate: Alterazione, Attacco, Illusione, Creazione

Essenze Negate: Distruzione, Rivelazione, Trasformazione, Convocazione, Charme

\subsubsection{Belevon}\index{Belevon}

\label{belevon}

e' il Patrono che meglio incarna la bugia e la finzione al fine di un proprio tornaconto. Lui ama solo se stesso. E' un narcisista che si circonda solo di persone che lo assecondano e lo adulano. Aborrisce la solitudine ma allo stesso tempo odia essere toccato da qualcuno.

E' sempre alla ricerca di nuove cose, di oggetti meravigliosi che scambia e ricambia con altri oggetti. Gli piace discutere e mercanteggiare, controbattere e portare fino allo stremo la vendita.

Dall'aspetto di un giovane ragazzo incarna perfettamente una pericolosa canaglia.

Il Devoto di Belevon e' ben descritto dal mercante ricco e curioso che mai si lascia perdere una occasione di trattare merci nuove. Non e' spinto dalla cupidigia o dall'accumulo bensi' dall'Arte del commercio e dello scambio.

\textbf{Simbolo}: una gabbia dorata

\textbf{Tratti}: Bugiardo, Narcisista, Casto, Doppiogiochista

\textbf{Manifestazione}: come se le sbarre dorate di una gabbia si intrecciassero attorno all'incantatore

\bigskip

Tratto in comune a 5 punti: Puoi creare un suono immaginario. Durata 10 secondi, entro 9 metri , 3 volte al giorno. Azione Reazione.

Tratto in comune a 10 punti: Acquisisci la capacita' di respirare sott'acqua per 10 minuti. Una volta al giorno. Azione Immediata.

Tratto in comune a 15 punti: La creatura che tocchi si placa e diventa indifferente a quello che succedo. TS Arbitrio DC 30. 3 volte al giorno. Costa 2 Azioni.

Tratto in comune a 20 punti: Toccando un oggetto vieni a conoscenza per sommi capi della storia di chi l'ha creato. Una volta al giorno. Costa 3 Azioni.

\bigskip

Elementi: Fuoco, Suono

\bigskip

Essenze Privilegiate: Illusione

Essenze Normali: Charme, Alterazione, Rivelazione

Essenze Limitate: Movimento, Protezione, Distruzione, Attacco

Essenze Negate: Creazione, Trasformazione, Cura, Convocazione, Difesa

\subsubsection{Cattalm}\index{Cattalm}

\label{cattalm}

Generato direttamente da Calicante, come risposta alla creazione di Gradh da parte di Ljust, e' pura distruzione, chaos ed entropia. Cattalm si prefigura il solo scopo di distruggere, portare chaos e malattie, terremoti ed alluvioni.

Cattalm e' tra i pochi Patroni che osa sfidare apertamente Gradh e lo fa con gioia perche' sa che la loro battaglia altro non fara' che portare ulteriore distruzione. Cattalm accetta ed invita ad essere suo Devoto ogni creatura capace di odio, capace di distruggere e ferire. Molti suoi Devoti sono creature mostruose o aberrazioni.

Cattalm invece e' tra i Patroni piu' meravigliosi, con una candida pelle lucente, ali di piuma argentee ed una leggera armatura argentata. Per quanto i lineamenti delicati ne facciano un essere bellissimo per quanto ambisca alla distruzione.

Cattalm adora il chaos che manifesta nei modi piu' violenti con terremoti, alluvioni, maremoti, malattie se non direttamente piogge infuocate. Non agisce quasi mai uccidendo le persone piuttosto infettandole e spargendo piaghe, carestie e piaghe per ottenere il massimo risultato.

Ljust non poteva non intervenire nella creazione di un Patrono cosi' esplicitamente malvagio e, di nascosto da Calicante, instillo' in Cattalm l'amore e protezione per i bambini. Cattalm distrugge, avvelena, indebolisce ma non i bambini, neanche indirettamente, piuttosto si attiva lui stesso per annullare i malefici causati dalla sua natura.
E' gia' capitato che interi villaggi venissero inondati e fossero trovati sui tetti in legno a modo di chiatte tutti i piccoli.

Ogni qual volta succede una calamita' si suole dire che ``Cattalm ha battuto il piede''

\textbf{Simbolo}: un'onda gigante che sovrasta la costa

\textbf{Tratti}: Distruttivo, Anarchico, Meticoloso, Sadico

\textbf{Manifestazione}: il rumore del tuono

\bigskip

Tratto in comune a 5 punti: Attraverso le tue armi indebolisci l'avversario designato. -2 Potenza per 1 minuto dopo un attacco andato a segno. Una volta al giorno.

Tratto in comune a 10 punti: Il tuo tocco imputridisce cibo (fino a 50kg) e acqua (50m/r). Una volta al giorno

Tratto in comune a 15 punti: Il tuo sguardo riempie di collera. TS Arbitrio DC 30 o il target attacca un soggetto a caso. Due volta al giorno

Tratto in comune a 20 punti: Generi un cono d'ombra che danneggia i tuoi avversari. Influenzi un cono che al termine e' largo 6 metri e lungo 27 metri, 10d6 di danno. Una volta al giorno

\bigskip

Elementi: Energia Negativa - Elettricita'

\bigskip

Essenze Privilegiate: Distruzione, Trasformazione

Essenze Normali:, Charme, Alterazione, Attacco

Essenze Limitate: Illusione, Convocazione, Movimento
 
Essenze Negate: Cura, Creazione, Protezione, Difesa, Rivelazione

\subsubsection{Efrem}\index{Efrem}

\label{efrem}

e' il Patrono di chi fa della natura la propria casa. Incarna in se' gli aspetti piu' puri della natura stessa, aggressivo come solo i felini piu' letali sanno essere; ma anche selvaggio come le radure piu' nascoste e rigorosa come solo la natura puo' essere.

Efrem si prefigge di difendere la Natura dalla contaminazione dell'uomo, da questa specie infestante che distrugge tutto cio' che incontra.

I Devoti di Efrem sono legati maggiormente all'elemento naturale. Manipolano le essenze principalmente elementali e si difendono o attaccano usando anche animali e creature naturali. In rari casi costringendo anche i Draghi alla ubbidienza.

I Devoti di Efrem hanno l'obiettivo supremo di proteggere gli animali e le piante, i luoghi e tutto cio' che e' naturale e non artificiale. Solitamente solitario e scontroso non riesce a capire il perche' dell'odio che, dal suo punto di vista, l'uomo scarica su Yeru.

Un Devoto di Efrem rispetta la vita come la morte, nel processo naturale che e' l'evoluzione ed il ciclo vitale. A volte decide di stabilirsi in un certo ambiente e lo elegge come suo territorio e come la sua casa lo protegge. Altre volte decide di essere ramingo ed intervenire in tutta Yeru per proteggere i suoi amati fiori e cuccioli.

\textbf{Simbolo}: una staffa con un rampicante attorcigliato attorno

\textbf{Tratti}: Indifferente, Leale, Fiducioso, Pratico

\textbf{Manifestazione}: spire di foglie avvolgono la spada

\bigskip

Tratto in comune a 5 punti: Il tuo tocco rende docili gli animali non magici. TS Arbitrio 20 per resistere. 3 volte al giorno. Costo 2 Azioni.

Tratto in comune a 10 punti: Guadagni un +4 a tutte le prove di Sopravvivenza che si effettuano in un all'ambiente naturale. Costo 2 Azioni.

Tratto in comune a 15 punti: Puoi benedire delle bacche affinche' queste siano nutrienti e curative. Puoi incantare 1d6 bacche al giorno. Ogni bacca, max 1 al giorno, cura 1d6 PF e rimuove le malattie o veleni. Costa 3 Azioni.

Tratto in comune a 20 punti: Il tuo tocco e' quello del padrone. Puoi ammansire creature anche magiche che tocchi. TS Arbitrio DC 30. Una volta al giorno. Costo 2 Azioni

\bigskip

Elementi: Radiosa, Suono

\bigskip

Essenze Privilegiate: Creazione

Essenze Normali: Convocazione, Trasformazione, Alterazione

Essenze Limitate: Cura, Distruzione, Attacco

Essenze Negate: Protezione, Difesa, Illusione, Charme, Rivelazione, Movimento

\subsubsection{Erondil}\index{Erondil}

\label{erondil}

Patrono di Terra e Aria, Erondil e' il Signore degli elementi piu' concreti e razionali. Colui che dotato di infinito potere e razionalita' dona ai suoi Devoti il potere della manipolazione della terra. Il dono di creare con semplice ``fango'' costruzioni gigantesche e di millenaria forza. Conclude le sue opere con attenzione e precisione. Pur con fatica perche' se il risultato finale non lo soddisfa scatena i suoi fulmini per distruggerlo all'istante. Perfezionista ed incontentabile, difficilmente qualcosa e' esattamente come lui se la immaginava. 

Ordinato ed esuberante e' il signore delle tempeste, dei tuoni e dei fulmini,dei terremoti e distruzioni. Ama circondarsi del fragore del tuono, del rombo della terra che si sgretola. Sa essere distruttivo verso coloro che non rispettano Yeru.
Ha braccia e petto ricoperti da tatuaggi quasi argentei che narrano le leggende di Terra e Aria. Erondil, il signore di tuoni e terremoti.


I Devoti di Erondil sono gli ingegneri dell'impossibile, ogni qual volta si deve sfidare la materia, la gravita' e la stessa ragione un Devoto di Erondil trovera' pane per i suoi denti, trovera' la sfida adatta ad un Costruttore dell'Impossibile.

\textbf{Simbolo:} un castello di sabbia con un fulmine sopra

\textbf{Tratti}: Perfezionista, Incontentabile, Sognatore, Esuberante

\textbf{Manifestazione}: suono di tempesta e rombo di frana

\bigskip

Tratto in u'omune a 5 punti: Non temi piu' le cadute. Ogni volta che cadi da piu' di 1 metro un soffio d'aria ti sostiene facendoti atterrare dolcemente.

Tratto in comune a 10 punti: Il tuo tocco plasma la pietra, Puoi aprire un passaggio (3m/h{*}1m/l{*}3m/p) in una parete di pietra. Una volta al giorno. Costo 2 Azioni.

Tratto in comune a 15 punti: Puoi scagliare un fulmine dalle tue mani. 10D6 di danno, fino a 3 target. TS Riflessi DC 30 per dimezzare. Costo 2 Azioni.

Tratto in comune a 20 punti: Sei in grado di creare una fossa profondissima (1km e oltre) sotto il tuo avversario (taglia fino a grande). TS Riflessi 35 o cadere. Una volta al giorno. Dopo 1 minuto la fossa si chiude con chi c'e' dentro. Costo 2 Azioni.

\bigskip

Elementi: Fuoco, Elettricita'

\bigskip

Essenze Privilegiate: Creazione

Essenze Normali: Trasformazione, Distruzione

Essenze Limitate: Difesa, Attacco, Protezione

Essenze Negate: Convocazione, Movimento, Charme, Protezione, Alterazione,Illusione, Cura, Rivelazione

\subsubsection{Gaya}\index{Gaya}

\label{gaya}

Patrono di Acqua e Fuoco, nelle profondita' della terra, della nebbia piu' fitta, Gaya si diverte a dipingere. Adora circondarsi dei flussi di fuoco e acqua quasi a creare una danza in mezzo a loro. Adora i suoni della natura, l'infrangersi delle onde sugli scogli, il cadere delle gocce di pioggia sull'acciottolato, il borbottare di un fuoco scoppiettante. Dipinge mescolando il caldo ed il freddo. L'acqua cristallina ed impetuosa al fuoco intrigante ed ardente. Gelosa del bello e delle arti tiene tutte le sue opere al sicuro in un ordine quasi maniacale e protette. Innovativa ma essenziale, utilizza elementi semplici per far risplendere le meraviglie della natura. Gaya e' la pittrice di tramonti e tempeste.

I Devoti di Gaya sono artisti volubili e sopra le righe. Sono coloro che ricreano la magia dell'alba o del tramonto o del mare in tempesta nelle loro opere. sono coloro che mettono poesia e follia nella normalita'. 

\textbf{Simbolo:} un pennello sul cielo

\textbf{Tratti}: Innovativo, Ordinato, Istintivo, Prudente

\textbf{Manifestazione}: spire di fuoco e acqua avvolgono all'incantatore

\bigskip

Tratto in comune a 5 punti: Puoi creare fino a 5 litri di acqua o 1 litro di liquore di buona qualita'. Una volta al giorno. Costo 2 Azioni.

Tratto in comune a 10 punti: Il tuo metabolismo non teme il freddo. Resisti al Danno magico da freddo e sei immune a quello naturale.

Tratto in comune a 15 punti: Puoi respirare sott'acqua come respiri l'aria.

Tratto in comune a 20 punti: Generi una pioggia di fuoco. In 6 metri di circonferenza, 10d6 di danno, intorno a te. Resisti il danno da fuoco, anche magico, per 5pf/round. Costo 2 Azioni.

\bigskip

Elementi: Elettricita', Fuoco

\bigskip

Essenze Privilegiate: Creazione

Essenze Normali: Trasformazione, Distruzione

Essenze Limitate: Difesa, Attacco, Protezione

Essenze Negate: Convocazione, Movimento, Charme, Protezione, Alterazione,Illusione, Cura, Rivelazione

\bigskip

\textbf{Gaia} ed \textbf{Erondil} sono come le due facce della stessa medaglia e sovraintendono agli elementi, Gaia acqua e fuoco e Erondil Aria e Terra; agiscono come espressione diretta dei Patroni maggiori, sono piccole manifestazione del loro immane potere.

\subsubsection{Krondal}\index{Krondal}

\label{krondal}

e' un Patrono potente ma schivo e riservato. Si tiene in disparte, fuori dai giochi finche' non percepisce la privazione della liberta'.

Non puo' vedere nel futuro, non puo' conoscere le persone ma il suo formidabile istinto lo fa diventare il combattente piu' temibile che si puo' incontrare. Coraggioso fin quasi all'avventato, agisce in battaglia senza paura. Dallo spirito buono, Krondal entra in campo nei momenti piu' importanti, quando non e' una situazione a decidersi ma il futuro della vita, della propria liberta' personale.

Krondal nutre un profondo rispetto per la liberta' ed e' profondamente contrario ad ogni schiavismo, razzismo o dittatura.

Un Devoto di Krondal e' tipicamente una guardia del corpo, un protettore, lo sceriffo che sa e deve decidere per il bene del suo paese, costi quello che costi. 

Un Devoto di Krondal non giudica le persone o i fatti bensi si attiene alla sua etica di protezione e liberta'.

Sotto vestiti dimessi e lisi, ma sempre puliti nasconde un fisico da combattente.

\textbf{Simbolo}: una spada tenuta verticalmente davanti a se

\textbf{Tratti}: Avventato, Pio, Corretto, Libero

\textbf{Manifestazione}: il mantello o veste del Devoto diventa sporca
di terra e sangue

\bigskip

Tratto in comune a 5 punti: Maledici il tuo avversario , dandogli un -2 TC e Difesa, per 1 minuto. TS Arbitrio DC 20 per resistere. Costo 2 Azione.

Tratto in comune a 10 punti: Non puoi essere legato o ammanettato. Ad un tuo gesto i nodi si sciolgono e le manette si aprono. Costo 1 Azione Immediata

Tratto in comune a 15 punti: La tua presenza porta speranza. Ogni compagno entro 3 metri da te guadagna 1d6 da usare entro 3 round. Una volta al giorno. Costo 1 Azione.

Tratto in comune a 20 punti: La tua arma e' piu' efficace contro i nemici. Ogni creatura colpita deve fare un Tiro Salvezza Arbitrio DC 25 o rimanere paralizzata per 3 round. Una volta che la creatura riesce nel TS non puo' essere piu' influenzata nelle successive 24 ore.

\bigskip

Elementi: Energia Positiva, Fuoco

\bigskip

Essenze Privilegiate: Distruzione

Essenze Normali: Cura, Difesa, Attacco

Essenze Limitate: Convocazione, Trasformazione, Charme, Movimento

Essenze Negate: Illusione, Rivelazione, Protezione, Creazione, Alterazione

\subsubsection{Ledyal}\index{Ledyal}

\label{ledyal}

e' il Patrono senza un volto preciso, senza una voce se non un canto. Mutevole di corpo e senza una definizione chiara del suo essere. Si manifesta con un lungo mantello color rosso fuoco dal tessuto fatto da mille farfalle. Il suo tocco e' vita e pace, protegge chi necessita dei suoi favori indipendentemente dal fatto che li chieda o meno. Desidera un mondo senza sofferenza, con solo pace ed armonia. Sospettoso/a e profondamente introverso non crede a coloro che gli danno ragione. Ha il cuore pieno di vita e di bonta' ma non puo'/non sa amare.

Ledyal ha anche una sorella gemella, o forse un'altra personalita'. O forse sono lo stesso Patrono, nessuno le ha mai viste insieme. La ``gemella' \textbf{Laydel} non tollera la sofferenza, disprezza chi causa dolore, uccide senza timore qualunque creatura abbia peccato contro un innocente, chiunque abbia causato sofferenza.

I Devoti di Ledyal al momento del Rito possono prendere Cura come Essenza privilegiata oppure Attacco.

\textbf{Simbolo}: una farfalla viola/rosso sangue che vola

\textbf{Tratti}: Caritatevole, Sospettoso. Introverso/Integerrimo, Clemente/Implacabile

\textbf{Manifestazione}: come se un mantello di farfalle avvolgesse il Devoto

\bigskip

Tratto in comune a 5 punti: Il tuo tocco e' vita/attacco. 3 volte al giorno puoi toccare una creatura vivente e curarla/causa di 1d6 PF. Costo 2 Azioni.

Tratto in comune a 10 punti: Il tuo tocco e' pace. La creatura toccata deve riuscire in un Tiro Salvezza Arbitrio DC 25 o cadere addormentata. Non puoi attaccare/danneggiare questa creatura. Una volta al giorno. Costo 2 Azioni (comprende anche l'Azione di tocco)

Tratto in comune a 15 punti: La tua aura protegge i tuoi compagni. Entro raggio 3 metri i tuoi compagni hanno un +4 alla Difesa ed un +2 ai Tiri Salvezza. Durata 10 minuti consecutivi, una volta al giorno. Costo 2 Azioni.

Tratto in comune a 20 punti: Irradi una sfera curativa/attacco intorno a te. Ogni creatura nel raggio di 6 metri viene curata di 60PF. Una volta al giorno. In caso di Laydel l'effetto e' opposto. Costo 2 Azioni.

\bigskip

Elementi: Energia Positiva, Radiosa

\bigskip

Essenze Privilegiate: Cura/Attacco

Essenze Normali: Alterazione, Protezione, Difesa

Essenze Limitate: Movimento, Convocazione, Illusione, Rivelazione

Essenze Negate: Distruzione, Attacco/Cura, Creazione, Trasformazione,Charm

\subsubsection{Nethergal}\index{Nethergal}

\label{nethergal}

Patrono Messaggero. Sulla piuma di un'oca vola la lettera di Nethergal. Rapida, impetuosa, diretta, Nethergal e' la messaggera, colei alla quale affidare pensieri e scritti. Sarcastica e logorroica curiosera' sui tuoi scopi, ti chiedera' informazioni sugli scritti affidatole con esplicita franchezza ed avra' sempre qualcosa da ridire sul messaggio da portare ma sara' anche altrettanto diretta e precisa nel consegnarlo. 

Nethergal non e' solo chiacchiere e pettegolezzi, qualsiasi testo venga scritto lei lo conosce, non esiste codice o segreto scritto che lei non conosca.

Il Devoto di Nethergal e' un fine linguista, un esperto di indovinelli e rebus, un Devoto che a differenza di Atmos non si limita a custodire gli scritti ma ne diffonde la conoscenza.

Un Devoto di Nethergal e' un maestro, un professore di lingue di un Collegio, un dotto esperto di mille argomenti.

\textbf{Simbolo:} una piuma bianca cangiante

\textbf{Tratti}: Sarcastico, Impetuoso, Immaturo, Logorroico

\textbf{Manifestazione}: cascata di piume, un oca in volo

\bigskip

Tratto in comune a 5 punti: Puoi inviare un messaggio di massimo 144 caratteri ad un soggetto entro 50 metri senza essere udito/visto. Una volta all'ora. Costo 1 Azione.

Tratto in comune a 10 punti: Mettendo la mano su un tomo ne apprendi il contenuto come se lo avessi letto. Un tomo a settimana. Perdi le conoscenze cosi' acquisite dopo una settimana. Costo 3 Azioni.

Tratto in comune a 15 punti: Puoi volare, 1 ora al giorno. manovrabilita' buona. Costo 1 Reazione.

Tratto in comune a 20 punti: Puoi costringere una creatura a rivelarti le informazioni che ha. TS su Arbitrio DC 30 per resistere. Una volta al giorno. Costo 2 Azioni.

\bigskip

Elementi: Elettricita', Energia Negativa

\bigskip

Essenze Privilegiate: Movimento

Essenze Normali: Illusione, Trasformazione, Charme

Essenze Limitate: Alterazione, Distruzione, Rivelazione, Attacco

Essenze Negate: Protezione, Convocazione, Cura, Creazione, Difesa

\subsubsection{Nedraf}\index{Nedraf}

\label{nedraf}

Il Patrono Sopravvissuto, il vecchio lupo ormai stanco che ha attraversato e combattuto mille battaglie. La sua carne e' ferita, il suo corpo ricoperto di cicatrici di guerra e lividi ma nulla lo fara' crollare. Tenacia, passione, esperienza e tanta rabbia rendono Nedraf non solo un combattente eccellente in qualsiasi occasione ma un conoscitore dell'ambiente attorno a se'. Grazie al suo impeccabile allenamento sa sfruttare al meglio le risorse a disposizione. Sa spronare con passione gli uomini a suoi ordini.
Nedraf rappresenta colui che vorresti sempre accanto in ogni battaglia. 

Molti capitani di ventura e ufficiali al comando sono Devoti di Nedraf. Il Devoto di Nedraf non si arrende, non rinuncia, non abbandona i compagni ma non per questo e' avventato o irrazionale nelle scelte.

\textbf{Simbolo:} una mano forte, avvolta in una benda sporca di sangue
che brandisce una spada

\textbf{Tratti}: Disciplinato, Combattivo, Tenace, Aggressivo

\textbf{Manifestazione}: si spande nell'aria odore di sangue e metallo

\bigskip

Tratto in comune a 5 punti: Puoi portare armature leggere senza penalita' alla CM

Tratto in comune a 10 punti: Acquisisci un punto bonus su una Lista armi. Puo' essere nota o meno

Tratto in comune a 15 punti: Puoi portare armature medie senza penalita' alla CM ed Agilita'

Tratto in comune a 20 punti: Puoi portare armature pesanti senza penalita' alla CM ed Agilita'

\bigskip

Elementi: Energia positiva, Suono

\bigskip

Essenze Privilegiate: Attacco

Essenze Normali: Convocazione, Alterazione, Protezione

Essenze Limitate: Difesa, Movimento, Creazione, Distruzione

Essenze Negate: Distruzione, Cura, Illusione, Charme, Trasformazione

\subsubsection{Nihar}

\label{nihar}

e' il Patrono degli eroi per caso. Ponderato e tranquillo e' anche amante del buon vino e del gozzovigliare. e' colui che non sceglieresti mai come compagno d'armi a causa del suo aspetto "comune" e del suo atteggiamento goliardico. Ma poi al momento di esserci, di combattere, di far la differenza ecco che strabilia tutti e "risolve" la partita.

Ha le sembianze di un piccolo uomo, dai vestiti sfarzosi e ricercati e dall'espressione guardinga ed allegra. Si protegge sempre e a qualunque costo, mostrando al mondo esattamente cio' che l mondo vuole vedere. Controlla attentamente la realta' attorno a se' e anche se e' sempre piu' facile vederlo con un calice in mano, se non ci si lascia ingannare dalle apparenze si notera' come i suoi occhi non perdano mai di vista il pericolo, il problema. Sta attento, non si fida di nulla e di nessuno. Ha fatto dei suoi difetti i suoi punti di forza.

\textbf{Simbolo}: una daga appoggiata vicino ad un calice di vino

\textbf{Tratti}: Altruista, Determinato, Cortese, Attento

\textbf{Manifestazione}: il suono di un brindisi

\bigskip

Tratto in comune a 5 punti: Puoi trasformare l'acqua in vino. Un litro al giorno. Costo 2 Azioni.

Tratto in comune a 10 punti: Costo una Azione immediata, ottieni un bonus di +2d6 ad una azione in quel round.

Tratto in comune a 15 punti: Il tuo pugnale causa 1d4 di danno aggiuntivo.

Tratto in comune a 20 punti: I manicaretti che prepari sono buonissimi. Chiunque si sazi con una pietanza da te preparata recupera 2d6 PF e viene curato dai veleni. Max 6 persone al giorno.

\bigskip

Elementi: Energia Positiva, Fuoco

\bigskip

Essenze Privilegiate: Trasformazione

Essenze Normali: Creazione, Protezione, Difesa

Essenze Limitate: Movimento, Alterazione, Cura, Convocazione

Essenze Negate: Illusione, Rivelazione, Charme, Distruzione, Attacco

\subsubsection{Orudjs}\index{Orudjs}

\label{orudjs}

ovvero il Patrono della illusione e della finzione. Colui che solo con il dono della parola, il gesticolare delle mani, la voce carismatica e lo sguardo intrigante riesce a vendere ogni sua parola come verita' assoluta. Adora il teatro per cio' che per lui e', la rappresentazione della falsita' umana, l'essere tante persone ed in realta' nessuna. Adora la politica ed i suoi intrighi. Finge di ascoltare chi gli sta vicino ma in realta' non e' interessato alle storie altrui perche' le sue sono sempre le migliori.

E' un codardo senza limiti e le poche verita' che dice, e sono veramente rare, sono da lui dette solo per salvarsi.

Dall'aspetto piuttosto ordinario e quasi scontato appena apre bocca ed inizia i suoi racconti riesce a calamitare l'attenzione dell'intera sala. Possiede infatti una voce calda e suadente che accompagnata alla buonissima dialettica da lui posseduta, incanta ogni orecchio in ascolto.

I suoi Devoti sono abili attori ed intrattenitori, spie sotto copertura, diplomatici o politicanti.

\textbf{Simbolo}: una maschera teatrale con solo la bocca aperta e
gli occhi

\textbf{Tratti}: Ironico, Codardo, Saccente, Socievole

\textbf{Manifestazione}: il suono di una risata profonda e contagiosa

\bigskip

Tratto in comune a 5 punti: Il tuo eloquio e' gia' leggendario. +2 alle prove di Intrattenere.

Tratto in comune a 10 punti: Sei in grado di creare fino a 4 suoni/rumori distanti 6 metri l'uno dall'altro. Durata 1 minuto. Tre volte al giorno. Costo 1 Azione Immediata

Tratto in comune a 15 punti: Il tuo eloquio e' gia' leggendario. +4 aggiuntivo alle prove di Intrattenere.

Tratto in comune a 20 punti: La tua voce e' suadente. Chiunque ti ascolti per piu' di un minuto deve fare un Tiro Salvezza Arbitrio DC 30 oppure considerarsi un tuo amico. Una volta al giorno

\bigskip

Elementi: Elettricita', Fuoco

\bigskip

Essenze Privilegiate: Charme

Essenze Normali: Illusione, Rivelazione, Alterazione

Essenze Limitate: Distruzione, Difesa, Cura, Movimento

Essenze Negate: Convocazione, Trasformazione, Protezione, Creazione, Attacco

\subsubsection{Orlaith}

\label{orlaith}

ovvero il Patrono della giustizia e della Vendetta. Lui segue pedissequamente le leggi e pretende che i suoi sottoposti eseguano senza alcuna discussione gli ordini impartiti. E' mosso da uno spirito gentile e buono che pero' tiene ben nascosto dietro le sue azioni dirette ed incisive, spudorate, vendicative e mortali. Orlaith e' vendetta che si fa legge. Agisce per senso di giustizia con i suoi metodi. Di lui attirano il portamento e lo sguardo fiero.

I Devoti di Orlaith spesso sono giudici e giustizieri, persone che hanno deciso di portare la giustizia ovunque, perche' Orlaith non puo' stare fermo, c'e' sempre qualcuno da giudicare e punire.

\textbf{Simbolo}: la bilancia

\textbf{Tratti}: Imparziale, Giusto, Gentile, Valoroso

\textbf{Manifestazione}: l'immagine di una stadera, sbilanciata.

\bigskip

Tratto in comune a 5 punti: Richiami a te 1 mastino (normale) che obbedisce ai tuoi comandi. Durata 1 minuto. Una volta al giorno. Costo 2 Azioni.

Tratto in comune a 10 punti: Un paio di manette si manifesta attorno ai polsi della creatura (massimo taglia grande). TS Riflessi DC 25 per annullare. Costo 2 Azioni.

Tratto in comune a 15 punti: La creatura toccata deve dire la verita' alle tue domande. Durata 10 minuti. Una volta al giorno. Costo 2 Azioni.

Tratto in comune a 20 punti: Crei raggio di luce lungo 27 metri e largo pochi centimetri. Ogni creatura attraversata subisce 5d6 di danno. Una volta al giorno. Costo 2 Azioni.

\bigskip

Elementi: Luce, Suono

\bigskip

Essenze Privilegiate: Convocazione

Essenze Normali: Alterazione, Attacco, Cura

Essenze Limitate: Distruzione, Trasformazione, Protezione, Difesa

Essenze Negate: Creazione, Rivelazione, Movimento, Illusione, Charme

\subsubsection{Rezh}\index{Rezh}

\label{rezh}

il Patrono che disprezza tutto. Rezh ama, vuole, tocca, rimira solo le sue monete lucide e brillanti. Non sono mai abbastanza, nessuna ricchezza e' mai abbastanza per lei. Rezh, l'avara tiene tutto per se', non conosce compassione, non conosce carita', non conosce condivisione. La sua fame di denaro, di ricchezze la rende prona a qualsiasi bassezza. Disprezza tutto e tutti e giudica tutto e tutti seguendo solo il suo personale metro di giudizio. In ogni moneta c'e' un po' di Rezh. Nella ossidatura di ogni moneta si puo' vedere l'impronta di Rezh.

Se il denaro compra la liberta' Rezh deve accumularne ancora e ancora se mai sara' abbastanza.

I Devoti di Rezh solitamente sono scelti da lei tra le fila dei piu' avidi e ricchi. Il loro scopo e' di accumulare ricchezze, sempre di piu'.

Spesso i Devoti di Rezh diventano esploratori, tombaroli, persone sempre alla ricerca di un tesoro e di una moneta in piu'.

\textbf{Simbolo:} una pila di monete con un ratto vicino

\textbf{Tratti}: Avaro, Arrogante, Cattivo, Freddo

\textbf{Manifestazione}: un rumore di monete che cadono avvolge l'incantatore

\bigskip

Tratto in comune a 5 punti: Sei un esperto di monete e gemme, nessun falsario puo' ingannarti. +4 alle prove di Consapevolezza e Cultura relative.

Tratto in comune a 10 punti: Puoi incantare una gemma (valore minimo 10mo) e usarla per proiettare una illusione fino a 20m{*}10m{*}10m. La gemma viene poi distrutta. Durata illusione 1 ora ogni 10 mo di valore della gemma. Costo 2 Azioni.

Tratto in comune a 15 punti: Puoi tirare fuori dalle tasche 1 moneta d'oro ogni volta che vuoi. Max 10 mo al giorno. Costo 1 Azione.

Tratto in comune a 20 punti: La tua armatura viene coperta da monete d'oro. Guadagni +4 alla Difesa e +4 TS Tempra per 1 ora. Costo 1 Reazione 

\bigskip

Elementi: Energia Negativa, Elettricita'

\bigskip

Essenze Privilegiate: Protezione

Essenze Normali: Charme, Illusione, Convocazione

Essenze Limitate: Movimento, Alterazione, Trasformazione, Difesa

Essenze Negate: Distruzione, Cura, Attacco, Creazione, Rivelazione

\subsubsection{Sumkjr}\index{Sumkjr}

\label{sumkjr}

Patrono dell'Arcano di Luce. Sumkir e' bonta', correttezza, lealta', giustizia, protezione.

Sumkir e' il cavaliere che protegge gli innocenti, e' la spada ``di Ljust'' nella battaglia finale. Difende i deboli e lenisce le ferite.

Sumkjr porta la Luce di Ljust ovunque, nessun pericolo potra' mai fermare Sumkjr dalla sua continua, infinita, cerca del bene.

Un Devoto di Sumkjr agisce lealmente e con onore, sempre perseguendo il bene ultimo, il suo essere non puo' essere piegato al male, all'ingiustizia, al disonore.

Con coraggio e determinazione il Devoto affronta ogni sfida ma non solo per senso del dovere, ma perche' profondamente votato al suo destino. Sumkjr sa che poche persone reggono tale standard perche' a differenza dei Devoti della Patrona delle Genesi i suoi Devoti non nascono per essere tali, ma lo diventano grazie alla loro profonda e determinata forza di volonta'. Per questo motivo Ljust interviene in loro favore con l'elaborato Rito del Rinnovo, grazie al quale ogni anno al Devoto meritevole e pentito di aver perso anche solo per poco la giusta direzione, la Luce, viene fatto recuperare ogni punto Tratto perso perche' agito fuori dalle 7 Regole Luminose.

Sumkjr e' un soldato valoroso, il migliore amico del giusto.

Calicante, preso dall'orrore alla vista di un Patrono cosi' fatto, lo privo' della capacita' di amare e provare veri sentimenti d'affetto. Portare il bene per un Devoto di Sumkjr e' un qualcosa di normale come e' normale non riuscire ad essere empatico con chi soffre. Il Devoto sa cosa deve fare e perche', ma non riesce a commuoversi od amare di fronte alle sofferenze od alle carezze di una donna/uomo.

\textbf{Simbolo:} tre gocce di sangue che cadono una dietro l'altra

\textbf{Tratti}: Giusto, Curioso, Buono, Valoroso

\textbf{Manifestazione}: il Devoto e' avvolto da un mantello di broccato dorato

\bigskip

Tratto in comune a 5 punti: Il tocco della tua spada e’ vita. Una creatura toccata con la tua arma recupera 3d6 punti ferita. Una volta al giorno. Costo 2 Azioni. 

Tratto in comune a 10 punti: La tua Volonta’ e’ piu’ forte del metallo. Guadagni un +2 ai Tiri Salvezza su Arbitrio

Tratto in comune a 15 punti: Concentri l’energia del tuo Patrono in un cono di assordante. Il cono e’ lungo 18 metri e largo al termine 3 metri, chiunque sia preso nell’area subisce 10d6 di danno. 
Una volta al giorno. Costo 2 Azioni.

Tratto in comune a 20 punti: Sacrifichi la tua vita per portare in vita una creatura morta da non piu’ di 1 giorno. Una volta. Costo 2 Azioni.

\bigskip

Elementi: Energia Positiva, Radiosa

\bigskip

Essenze Privilegiate: Cura

Essenze Normali: Protezione, Difesa, Creazione

Essenze Limitate: Convocazione, Trasformazione,Charme

Essenze Negate: Attacco, Movimento, Illusione, Alterazione, Rivelazione, Distruzione

\bigskip

\paragraph{Le 7 Regole Luminose}\index{7 Regole Luminose}

\label{le-7-regole-luminose}

Le Sette regole Luminose sono un insieme di norme e comportamenti tenuti, a vario titolo, dai Devoti che vogliono seguire la strada della Luce di Ljust.

I Devoti di Sumkjr devono seguirle tutte e 7 pena la perdita di potere (punti Tratto), altri Devoti di altri Patroni, sempre positivi od almeno neutrali, seguono solo alcune di questi dettami, come regola per non cadere nelle braccia di Calicante


\begin{enumerate}
	\item Proteggi i deboli e chi non sa difendersi dai soprusi
	\item Ama la vita e proteggila. L'Amore deve vincere sopra ogni cosa
	\item Combatti contro le ingiustizie e chi porta sofferenze e dolore
	\item Lenisci le ferite ed i dolori. Placa gli animi e favorisci la pace
	ed armonia
	\item Onesta' e Lealta' sono le tua fondamenta
	\item Sei un maestro di virtu'. Fa che gli altri possano prendere ispirazione
	dalle tue gesta
	\item Sii luminoso ma non accecare gli altri
\end{enumerate}

\subsubsection{Shayalia}\index{Shayalia}

\label{shayalia}

Patrono dell'Arcano di Tenebra. Shayalia e' l'anima oscura della perdizione, del tradimento, della lussuria piu' sordida e peccaminosa. Adora i bordelli. Le piace l'odore acre del sudore, la pelle lucida di oli e profumi. Le passioni, le vendette che li si consumano. La distruzione fisica e morale che in quei luoghi viene perpetrata.

Shayalia e' una donna che gode di se stessa, che vive dei piaceri piu' sordidi. Vive di vendette lungamente e ben dettagliatamente progettate. Vendicativa ed amorale, non giudica con metro di giudizio umano, il suo godere non e' neppure lontanamente comprensibile. Shayalia e' quanto di piu' vicino a Calicante sia stato creato. Sono le passioni, le pulsioni, i liquidi umorali che la fanno inebriare.

Shayalia e' la concubina che ti ammalia e ti distrugge, goccia dopo goccia. I veleni sono le sue armi, le debolezze umane il suo campo.

I Devoti di Shayalia sono spie, figli bastardi, amanti di potenti signori che agiscono all'ombra.

Ljust disgustata dalla visione di un Patrono del genere instillo' in Shayalia l'amore per la natura, piante ed animali. E cosi' molti dei piu' famosi botanici, erboristi e zoologi sono Devoti di Shayalia, forse le uniche cose che Shayalia veramente puo' amare.

\textbf{Simbolo:} un cuscino stropicciato e sporco di sangue

\textbf{Tratti}: Lussurioso, Volubile, Pessimista, Sadomasochista

\textbf{Manifestazione}: il Devoto e' avvolto da un mantello di velluto nero

\bigskip

Tratto in comune a 5 punti: I tempi per preparare una pozione sono dimezzati.

Tratto in comune a 10 punti: I costi per preparare una pozione sono dimezzati.

Tratto in comune a 15 punti: Dal tuo palmo secerni veleno. Il tuo tocco, o tramite arma in mischia veicola il veleno. TS Tempra DC 25 o -2 a Volonta' e Agilita' per 10 minuti. Costo 1 Azione.

Tratto in comune a 20 punti: Il tuo tocco e' vita per la natura. Puoi curare animali e piante anche magiche con la Essenza di Distruzione, gli effetti sono quelli della Cura.

\bigskip

Elementi: Energia Negativa, Elettricita'

\bigskip

Essenze Privilegiate: Distruzione

Essenze Normali: Charme, Alterazione, Illusione

Essenze Limitate: Attacco, Convocazione, Trasformazione

Essenze Negate: Movimento, Protezione, Creazione, Cura, Rivelazione,Difesa

\textbf{Sumkir} e \textbf{Shayalla} sono complementari nel tenere in mano le file sfuggenti e pericolose della magia. Agiscono come espressione diretta di dei Patroni maggiori

\subsubsection{Sixiser}

il Patrono che e' indifferente al presente in quanto totalmente, compulsivamente ossessionato dal futuro e dal suo destino. Negli angoli piu' remoti dei mondi conosciuti si narra che Sixiser accumuli di tutto, indifferente a tutto e tutti. Terrorizzato dal futuro che vede, da una ipotetica fine di se' e del tutto vive una vita di ritiro, spirituale e fisico. Si priva volontariamente di tutto il necessario. Ma allo stesso accumula qualunque oggetto incroci la sua strada nella speranza di un ritorno.

E' paranoico e non si fida di nessuno. Usa i suoi poteri di divinazione per conoscere e scrutare tutti.

I Devoti di Sixiser sono spesso negromanti circondati da non morti ed altre creature silenziose ed ubbidienti. Chi si rifugia alla ricerca della solitudine e dello studio chi invece mira ad espandere e governare intere citta' e nazioni al fine di sentirsi piu' sicuro.

\textbf{Simbolo}: un forziere straripante di ogni cosa che non si puo' chiudere

\textbf{Tratti}: Riservato, Morigerato, Accumulatore, Paranoico

\textbf{manifestazione}: due mani uncinate che circondano, come a nascondere, la testa dell'incantatore

\bigskip

Tratto in comune a 5 punti: acquisisci la fino 18 metri, o 36 metri se gia' presente.

Tratto in comune a 10 punti: vedi nell'oscurita' anche magica. Vedi le trappole nel raggio di mischia intorno a te.

Tratto in comune a 15 punti: Toccando un oggetto sei in grado di capirne tutte le proprieta' magiche e non.

Tratto in comune a 20 punti: Sei in grado di animare una creatura morta da non piu' di un giorno come non morto da 1 CR (tipo zombi/scheletro a secondo dello stato). Una volta al giorno. Costo 2 Azioni.

\bigskip

Elementi: Elettricita', Energia Negativa

\bigskip

Essenze Privilegiate: Rivelazione

Essenze Normali: Distruzione, Illusione, Difesa

Essenze Limitate: Movimento, Attacco, Convocazione, Charme

Essenze Negate: Creazione, Trasformazione, Difesa, Cura, Alterazione

\subsubsection{Tazher}\index{Tazher}

\label{tazher}

il Patrono delle Ombre; colui che silenzioso, ti uccide. Non saprai mai il perche'. Non conoscerai mai il suo aspetto ma, se improvvisamente hai una sensazione di gelo, Tazher e' dietro di te pronto a prendere la tua vita. Doppiogiochista dall'animo cattivo, chiedi il suo aiuto solo se sei disposto a pagarne il prezzo che lui e lui solo decidera'. Vive di notte, vive la notte. Le ombre sono le sue amiche e la tenebra il suo mantello. Profondamente individualista con un carattere scontroso e permaloso, non ha amici, non intrattiene relazioni di alcun tipo.

Il Devoto di Tazher e' il ladro, l'assassino, il bandito, chiunque viva per l'oscurita' ed il proprio tornaconto. Un Devoto di Tazher e' estremamente pericoloso in combattimento.

\textbf{Simbolo}: lo scintillio della lama nel buio

\textbf{Tratti}: Scontroso, Calcolatore, Perfezionista, Cattivo

\textbf{Manifestazione}: l'ombra del Devoto prende vita muovendo l'arma

\bigskip

Tratto in comune a 5 punti: Guadagni +4 alle prove di criminalita'.

Tratto in comune a 10 punti: Una volta al giorno fai un attacco in piu'. Una Azione Immediata.

Tratto in comune a 15 punti: Finche' cammini sopra delle ombre o al buio sei invisibile. Puoi essere comunque rilevato con la luce o Essenza di Rivelazione.

Tratto in comune a 20 punti: Una volta al giorno su tutti gli attacchi andati a segno in quel round fai il doppio del danno. Costo 1 Azione.

\bigskip

Elementi: Energia Negativa, Fuoco

\bigskip

Essenze Privilegiate: Attacco

Essenze Normali: Convocazione, Trasformazione, Charme

Essenze Limitate: Illusione, Alterazione, Distruzione, Movimento

Essenze Negate: Rivelazione, Protezione, Creazione, Cura, Difesa

\subsubsection{Thaft}\index{Thaft}

\label{thaft}

il Patrono che accompagna nella nascita e nella morte. Silenzioso, resta in disparte e osserva lo scorrere della vita degli uomini. Quasi umile nella sua semplicita', Thaft e' ovunque. Testimone silenzioso della vita umana; nel momento in cui una vita scivola via, Thaft assiste, nell'attimo in cui una vita nasce, Thaft e' presente.

Thaft sa anche che non si puo' essere sempre e solo osservatori. Attraverso il suo taccuino sacro e magico puo' decidere e giudicare della vita degli uomini. Perche' se una spada ferisce, e' solo Thaft che ne decide la morte.

I Devoti di Thaft sono i sacerdoti dell'ultimo viaggio, coloro che proteggono e vegliano sulle anime e corpi dei morti. Profondamente contrari all'utilizzo dei non-morti ne perseguono la distruzione.

Un Devoto di Thaft rispetta la vita come la morte e non teme di arrecare distruzione per un equilibrio maggiore.

Thaft e' stato plasmato da Atmos.

\textbf{Simbolo}: un libro aperto con un teschio sopra

\textbf{Tratti}: Semplice, Silenzioso, Mite, Sicuro

\textbf{Manifestazione}: si sente il pianto di un bambino appena nato o il sospiro della morte

\bigskip

Tratto in comune a 5 punti: il tuo tocco e' letale per i non morti. Un tuo tocco infligge 2d6 di danno ad un non morto. Costo 2 Azioni.

Tratto in comune a 10 punti: Il tuo tocco lenisce. Una volta al giorno puoi rimuovere Cecita' o Sordita'. Costo 2 Azioni.

Tratto in comune a 15 punti: Un non morto deve effettuare un Tiro Salvezza Tempra DC 30 o essere distrutto se toccato dalla tua mano. Costo 2 Azioni.

Tratto in comune a 20 punti: Uccidi la creatura toccata. TS Arbitrio DC 35 o morte. Una volta alla settimana. Costo 2 Azioni.

\bigskip

Elementi: Suono, Radiosa

\bigskip

Essenze Privilegiate: Distruzione

Essenze Normali: Cura, Movimento, Convocazione

Essenze Limitate: Protezione, Difesa, Trasformazione, Charme

Essenze Negate: Alterazione, Rivelazione, Illusione, Creazione, Attacco

\subsubsection{Torbiorn}\index{Torbiorn}

\label{torbiorn}

il Patrono che meglio incarna il concetto ``non e' mai abbastanza'.

Alto, bello come un quadro ma, proprio come quest'ultimo, senza calore e vita, Torbiorn rasenta la perfezione maniacale nel vestirsi, nell'atteggiarsi.

Nulla e' mai abbastanza per lui. Nessuno e' mai alla sua altezza. Ed eccolo che con arroganza e ironia va a modificare tutto il modificabile per poter placare questa profonda insoddisfazione. Qualora il risultato finale raggiunto non lo soddisfi, e accade molto spesso, ecco che prende il sopravvento il suo cinismo e distrugge tutto senza curarsi della sofferenza che sta arrecando a chi gli sta attorno.

Il Devoto di Torbiorn e' il tipico aristocratico ricco e svogliato,colui che cerca sempre la strada piu' facile e meno rischiosa.

Incurante degli altri si diverte nello sfruttare i lavori altrui e trarne giovamento.

\textbf{Simbolo}: uno specchio opaco

\textbf{Tratti}: Altezzoso, Indifferente, Vanitoso, Permaloso

\textbf{Manifestazione}: schegge di specchio rotto tutto intorno al Devoto come un turbine

\bigskip

Tratto in comune a 5 punti: Con un gesto puoi rinfrescare i tuoi vestiti rendendoli puliti e profumati. Costo 1 Azione.

Tratto in comune a 10 punti: Il tuo sputo e' velenoso. TS Tempra DC 20 oppure -2 Potenza. Durata 1 minuti. Tre volte al giorno. Costo 1 Azione.

Tratto in comune a 15 punti: Fissando l'obiettivo negli occhi lo costringi a fermarsi. Il target non puo' piu' muovere le gambe. TS Arbitrio DC 30. Una volta al giorno. Costo 2 Azioni.

Tratto in comune a 20 punti: Dalle tue dita partono dei viticci che pungono fino a 10 avversari.

Ogni viticcio causa 2d6 di danno, TS Tempra DC 25 per dimezzare. Costo 2 Azioni.

\bigskip

Elementi: Fuoco, Suono

\bigskip

Essenze Privilegiate: Trasformazione

Essenze Normali: Distruzione, Movimento, Attacco

Essenze Limitate: Charme, Rivelazione, Convocazione, Alterazione

Essenze Negate: Illusione, Protezione, Cura, Difesa, Creazione

\subsubsection{Tabella collegamento Patrono - Tratto}\index{Tabella collegamento Patrono - Tratto}

\label{tabella-collegamento-patrono---tratto}

\begin{tabular}[c]{@{}lllll@{}}
\toprule 
Nome Patrono & Tratto & Tratto & Tratto & Tratto\tabularnewline
Atherim & Allegro & Calmo & Industrioso & Buono\tabularnewline
Atmos & Osservatore & Distaccato & Studioso & Riflessivo\tabularnewline
Belevon & Bugiardo & Narcisista & Casto & Doppiogiochista\tabularnewline
Calicante & Egoista & Vendicativo & Superbo & Iracondo\tabularnewline
Cattalm & Distruttivo & Anarchico & Meticoloso & Sadico\tabularnewline
Efrem & Indifferente & Leale & Fiducioso & Pratico\tabularnewline
Erondil & Perfezionista & Incontentabile & Sognatore & Esuberante\tabularnewline
Gaya & Innovativo & Ordinato & Istintivo & Prudente\tabularnewline
Gradh & Indomito & Protettivo & Vendicativo & Coraggioso\tabularnewline
Krondal & Avventato & Pio & Corretto & Libero\tabularnewline
Ledyal & Caritatevole & Sospettoso & Introverso/integerrimo & Clemente/implacabile\tabularnewline
Ljust & Generoso & Empatico & Coraggioso & Protettivo\tabularnewline
Lynx & Solitario & Rigido & Serio & Controllato\tabularnewline
Nedraf & Disciplinato & Combattivo & Tenace & Aggressivo\tabularnewline
Nethergal & Sarcastico & Impetuoso & Immaturo & Logorroico\tabularnewline
Nihar & Altruista & Determinato & Cortese & Attento\tabularnewline
Orudjs & Ironico & Codardo & Saccente & Socievole\tabularnewline
Orlaith & Imparziale & Giusto & Gentile & Valoroso\tabularnewline
Rezh & Avaro & Arrogante & Cattivo & Freddo\tabularnewline
Shayalia & Lussurioso & Volubile & Pessimista & Sadomasochista\tabularnewline
Sixiser & Riservato & Morigerato & Accumulatore & Paranoico\tabularnewline
Sumkjr & Giusto & Curioso & Buono & Valoroso\tabularnewline
Tazher & Scontroso & Calcolatore & Perfezionista & Cattivo\tabularnewline
Thaft & Semplice & Silenzioso & Mite & Sicuro\tabularnewline
Torbiorn & Altezzoso & Indifferente & Vanitoso & Permaloso\tabularnewline
\bottomrule
\end{tabular}

\pagebreak

\section{Equipaggiamento}

\label{equipaggiamento}

\subsection{Ricchezza e Denaro}\index{Ricchezza e Denaro}


\begin{quotebox}
- Doc... c'e' soltanto bisogno di un pochino di plutonio.
- Ah, sono certo che nell'85 il plutonio si compra nella drogheria sotto casa, ma nel '55 la faccenda e' molto piu' complicata! (Ritorno al futuro)
\end{quotebox}


\label{ricchezza-e-denaro}

Un personaggio che inizia a giocare generalmente ha monete d'oro sufficienti per acquistare gli elementi di base: qualche arma, un'armatura adatta ed un po' di attrezzatura varia. Man mano che il personaggio intraprende avventure e accumula bottino puo' permettersi un equipaggiamento migliore ed oggetti magici. Al primo livello i personaggi hanno monete ed equipaggiamento per un totale di 100 mo.

Inoltre, ogni personaggio inizia il gioco con un abito del valore di 10 mo o meno. Per personaggi di livello superiore al 1°, vedi \textbf{Tabella: Ricchezza dei personaggio per Livello}.

\textbf{Vendere il Bottino}\index{Vendere il Bottino}

In generale, e' possibile vendere qualsiasi cosa alla meta' del prezzo indicato, comprese armi, armature, equipaggiamento, oggetti magici e oggetti creati dai personaggi. Le merci di scambio costituiscono l'eccezione alla regola del meta' prezzo.

Una merce di scambio, in questo senso, e' un bene di valore che puo' essere facilmente scambiato quasi come fosse equivalente ai contanti.

\textbf{Monete}\index{Monete}

La moneta piu' comune e' la moneta d'oro (mo). Una moneta d'oro vale 10 monete d'argento (ma). Ogni moneta d'argento vale 10 monete di rame (mr). Oltre a monete di rame, argento e oro ci sono anche le monete di platino (mp), che valgono ognuna 10 monete d'oro.

\begin{tabular}[c]{@{}lllll@{}}
\toprule 
\textbf{Valore di Cambio} & Moneta Rame (mr) & Moneta Argento (ma) & Moneta Oro (mo) & Moneta Platino (mp)\tabularnewline
Moneta Rame & 1 & 1/10 & 1/100 & 1/1000\tabularnewline
Moneta Argento & 10 & 1 & 1/10 & 1/100\tabularnewline
Moneta Oro & 100 & 10 & 1 & 1/10\tabularnewline
Moneta Platino & 1000 & 100 & 10 & 1\tabularnewline
\bottomrule
\end{tabular}

\bigskip

\textbf{Altre Ricchezze}\index{Altre Ricchezze}

I mercanti di solito scambiano merci anche senza l'uso di monete.
Per farsi un'idea delle transazioni commerciali, alcune merci di scambio
sono descritte nella tabella.

\begin{tabular}[c]{@{}ll@{}}
\toprule 
Costo & Oggetto\tabularnewline
1 mr & Frumento (0.5 kg)\tabularnewline
2 mr & Farina (0.5 kg) o pollo (1)\tabularnewline
1 ma & Ferro (0.5 kg)\tabularnewline
5 ma & Tabacco o rame (0.5 kg)\tabularnewline
1 mo & Cannella (0.5 kg) o capra (1)\tabularnewline
2 mo & Zenzero o pepe (0.5 kg) o pecora (1)\tabularnewline
3 mo & Maiale (1)\tabularnewline
4 mo & Lino (1 m\textsuperscript{2})\tabularnewline
5 mo & Sale o argento (0.5 kg)\tabularnewline
10 mo & Seta (1 m) o mucca (1)\tabularnewline
15 mo & Zafferano o chiodi di garofano (0.5 kg) o bue (1)\tabularnewline
\bottomrule
\end{tabular}

\pagebreak

\section{Equipaggiamento - Armi}\index{Equipaggiamento}\index{Armi}

\label{equipaggiamento---armi}
\begin{quotebox}
Questo e' il mio fucile. Ce ne sono tanti come lui, ma questo e' il mio. Il mio fucile e' il mio migliore amico, e' la mia vita. Io debbo dominarlo come domino la mia vita. Senza di me il mio fucile non e' niente; senza il mio fucile io sono niente. Debbo saper colpire il bersaglio, debbo sparare meglio del mio nemico che cerca di ammazzare me, debbo sparare io prima che lui spari a me e lo faro'. Al cospetto di Dio giuro su questo credo: il mio fucile e me stesso siamo i difensori della patria, siamo i dominatori dei nostri nemici, siamo i salvatori della nostra vita e cosi' sia, finché non ci sara' piu' nemico ma solo pace, amen.
\\
(Full Metal Jacket, Film, 1987)
\end{quotebox}

Vengono elencate le arme suddivise per categorie.

Ricordo che usare un'Arma senza l'adeguata competenza impone un -2d6 al colpire

\bigskip
\begin{longtable}{|L{3.5cm} |k{1cm} |k{1.5cm}| k{1cm} |L{4.5cm} |k{3cm}|}
	\textbf{Nome Lista Arma} & \textbf{Costo} (mo) & \textbf{Danno} & T/P/B & \textbf{Gittata} & \textbf{Speciale}\tabularnewline
	\textbf{Asce} & & & & & \tabularnewline
	Ascia ad una mano & 6 & 1d6 & T & 6 metri & versatile\tabularnewline
	Ascia da battaglia & 10 & 1d10 & T & - & -\tabularnewline
	Ascia Martello & 16 & 1d6/1d8 & T/B & & \tabularnewline
	Urgrosh & 18 & 1d6/1d8 & P/T & Controcarica, arma lunga & \tabularnewline
	Grande Ascia Doppia & 25 & 1d12 & T & arma doppia, arma lunga & \tabularnewline
	\textbf{Rompi Cranio} & & & & & \tabularnewline
	Randello & 1 & 1d6 & B & & \tabularnewline
	Mazza Leggera & 3 & 1d6 & B/T & & \tabularnewline
	Mazza Pesante & 5 & 1d8 & B/T & & \tabularnewline
	Morningstar & 6 & 1d8 & B/P & & \tabularnewline
	Martello Leggero & 3 & 1d6 & B/P & 6 metri & \tabularnewline
	Martello da guerra & 5 & 1d8 & B/P & 6 metri & \tabularnewline
	Grosso randello & 2 & 1d8 & B & & \tabularnewline
	Flagello Pesante & 15 & 1d10 & B & & \tabularnewline
	Flagello & 8 & 1d8 & B & & \tabularnewline
	\textbf{Archi} & & & & & \tabularnewline
	Fionda & - & 1d4 & B & 10 metri & da tiro\tabularnewline
	Arco Lungo & 75 & Frecce & P & 20 metri & da tiro\tabularnewline
	Arco Lungo Composito & 110 & Frecce & P & 36 metri & da tiro\tabularnewline
	Arco Corto Composito & 75 & 1d6 & P & 20 metri & da tiro\tabularnewline
	Arco Corto & 30 & 1d6 & P & 15 metri & da tiro\tabularnewline
	\textbf{Balestre} & & & & & \tabularnewline
	Balestra leggera & 35 & Dardi & P & 15 metri & da tiro\tabularnewline
	Balestra pesante & 50 & 1d10 & P & 20 metri & da tiro\tabularnewline
	Balestra ad una mano & 100 & 1d4 & P & 12 metri & da tiro\tabularnewline
	Balestra leggera a ripetizione & 250 & Dardi & P & 6 metri & da tiro, 6 cariche\tabularnewline
	Balestra pesante a ripetizione & 400 & 1d10 & P & 12 metri & da tiro, 8 cariche\tabularnewline
	\textbf{Armi doppie} & & & & & \tabularnewline
	Bastone & 3 & 1d6 & B & arma lunga, versatile & \tabularnewline
	Grande Ascia Doppia & 25 & 1d12 & T & arma lunga & \tabularnewline
	Flagello Doppio & 90 & 1d10 & B & arma lunga & \tabularnewline
	Spada a due lame & 100 & 1d8 & T & arma lunga & \tabularnewline
	Urgrosh & 18 & 1d6/1d8 & T/P & Controcarica, arma lunga & \tabularnewline
	\textbf{Armi Esperte} & & & & & \tabularnewline
	Pugno/Calcio nudo & 0 & 1d4 & B & versatile & \tabularnewline
	Falcetto & 6 & 1d6 & T & & \tabularnewline
	Flagello & 8 & 1d8 & B & & \tabularnewline
	Flagello Pesante & 15 & 1d10 & B & & \tabularnewline
	Alabarda & 10 & 1d10 & P/T & Controcarica, arma lunga, ED9 & \tabularnewline
	Falcione in asta & 12 & 1d10 & P/T & Controcarica, arma lunga, ED9 & \tabularnewline
	Brandistocco & 10 & 2d4 & P/T & Controcarica, arma lunga & \tabularnewline
	Falce & 18 & 2d4 & P/T & arma lunga & \tabularnewline
	Flagello Doppio & 90 & 1d10 & B & arma lunga & \tabularnewline
	Frusta & 1 & 1d3 & T & arma lunga & \tabularnewline
	\textbf{Palle rotanti} & & & & & \tabularnewline
	Flagello & 8 & 1d8 & B & & \tabularnewline
	Flagello Pesante & 15 & 1d10 & B & & \tabularnewline
	Frusta & 1 & 1d3 & T & arma lunga & \tabularnewline
	Catena chiodata & 25 & 2d4 & P & arma lunga & \tabularnewline
	\textbf{Armi Aggraziate} & & & & & \tabularnewline
	Stocco & 20 & 1d6 & P & & versatile\tabularnewline
	Scimitarra & 15 & 1d6 & T & & versatile\tabularnewline
	Falcione & 75 & 2d4 & T & & ED7\tabularnewline
	\textbf{Armi della morte} & & & & & \tabularnewline
	Picca Leggera & 4 & 1d4 & P & & \tabularnewline
	Picca Pesante & 8 & 1d6 & P & arma lunga & \tabularnewline
	Falce & 18 & 2d4 & P/T & arma lunga & \tabularnewline
	\textbf{Armi da stordimento} & & & & & \tabularnewline
	Pugno/Calcio nudo & 0 & 1d4 & B & & versatile\tabularnewline
	Guanto chiodato & 5 & 1d4 & P & & \tabularnewline
	Manganello & 1 & 1d6 & & non letale & \tabularnewline
	\textbf{Lance} & & & & & \tabularnewline
	Alabarda & 10 & 1d10 & P/T & Controcarica, arma lunga, ED9 & \tabularnewline
	Tridente & 15 & 1d8 & P/T & 3 metri & arma lunga, Controcarica \tabularnewline
	Urgrosh & 18 & 1d6/1d8 & P/T & Controcarica, arma lunga & \tabularnewline
	\textbf{Lance lunghe} & & & & & \tabularnewline
	Naginata & 8 & 1d10 & T & arma lunga, ED9 & \tabularnewline
	Falcione in asta & 12 & 1d10 & P/T & Controcarica, arma lunga, ED9 & \tabularnewline
	Brandistocco & 10 & 2d4 & P/T & Controcarica, arma lunga & \tabularnewline
	Lancia & 10 & 1d8 & P & arma lunga & \tabularnewline
	\textbf{Armi letali} & & & & & \tabularnewline
	Pugnale & 2 & 1d4 & P & 6 metri & versatile\tabularnewline
	Machete & 10 & 1d6 & T & & \tabularnewline
	\textbf{Aste} & & & & & \tabularnewline
	Giavellotto & 1 & 1d6 & P & 12 metri & \tabularnewline
	Lancia corta da fante & 1 & 1d6 & P & 6 metri & versatile\tabularnewline
	Lancia da fante & 2 & 1d8 & P & 6 metri &Arma lunga, Controcarica \tabularnewline
	Tridente & 15 & 1d8 & P/T & 3 metri & Arma lunga, Controcarica \tabularnewline
	\textbf{Spade} & & & & & \tabularnewline
	Spada Corta & 10 & 1d6 & P & & versatile\tabularnewline
	Spada Lunga & 15 & 1d8 & T & & \tabularnewline
	Spadone a due mani & 50 & 2d6 & T & & \tabularnewline
	Spada bastarda & 35 & 1d10 & T & & \tabularnewline
	Spada a due lame & 100 & 1d8 & T & arma lunga, arma doppia & \tabularnewline
	Katana & 300 & 1d10 & T & ED9, versatile & \tabularnewline
	\textbf{Spada e Scudo} & & & & & \tabularnewline
	Spada Corta & 10 & 1d6 & P & & versatile\tabularnewline
	Spada Lunga & 15 & 1d8 & T & & \tabularnewline
	\textbf{Bloccanti} & & & & & \tabularnewline
	Bolas & 4 & 1d3 & B & intralciato & \tabularnewline
	Rete & 8 & & & intralciato & \tabularnewline
	\textbf{Armi da tiro} & & & & & \tabularnewline
	Pugnale & 2 & 1d4 & P & 6 metri & versatile\tabularnewline
	Lancia corta da fante & 1 & 1d6 & P & 6 metri & \tabularnewline
	Lancia da fante & 2 & 1d8 & P & 6 metri &arma lunga, Controcarica\tabularnewline
	Martello Leggero & 3 & 1d6 & B/P & 6 metri & versatile\tabularnewline
	Ascia ad una mano & 6 & 1d6 & T & 6 metri & versatile\tabularnewline
	Tridente & 15 & 1d8 & P/T & 3 metri&arma lunga, Controcarica \tabularnewline
	\textbf{Pugno nudo} & B & Leggi sotto & & & \tabularnewline
	\textbf{Frecce e Dardi} & & & & & \tabularnewline
	Frecce da caccia & 20/ 1mo & 1d6 & P & & \tabularnewline
	Frecce da guerra & 10/1 mo & 1d8 & P & & \tabularnewline
	Dardi da balestra pesanti & 10/1 mo & 1d8 & P & & \tabularnewline
	Dardi da balestra leggeri & 15/1 mo & 1d6 & P & & \tabularnewline
	Biglie di Marmo (fionde) & 15/1 mo & 1d4 & B & & \tabularnewline
	Sasso & - & 1d2 & B & & \tabularnewline
	\textbf{Armi Semplici} & & & & & \tabularnewline
	Pugnale & 2 & 1d4 & P & 6 metri & versatile\tabularnewline
	Mazza Leggera & 3 & 1d6 & B/T & & versatile\tabularnewline
	Randello & 1 & 1d6 & B & & versatile\tabularnewline
	Morningstar & 6 & 1d8 & B/P & & \tabularnewline
	Lancia corta da fante & 1 & 1d6 & P & 6 metri & versatile\tabularnewline
	Giavellotto & 1 & 1d6 & P & 12 metri & \tabularnewline
	Bastone & 3 & 1d6 & B & arma lunga, versatile & \tabularnewline
	Balestra leggera & 35 & Dardi & P & 15 metri & \tabularnewline
\end{longtable}

\textbf{Pugno Nudo}: Ogni volta che prendi questa competenza, e CA +3 rispetto alla volta precedente, il danno aumenta seguendo questa progressione: 1d6 (4lv), 1d8 (7lv), 2d6 (10lv), 2d8 (13lv), 2d10 (14lv), 3d6 (17lv). 

\bigskip

\textbf{Gittata}\index{Gittata}
La distanza indicata e' quello a pieno Tiro per Colpire. Ogni arma a distanza puo' fino a due categorie di distanza oltre quando indicato. Se il target e' entro la distanza indicata non si hanno malus al colpire, se il target e' tra il primo e secondo incremento il malus al colpire e' -1d6. Se il target e' tra il secondo e' terzo incremento il malus al colpire e' di -2d6.

Un giavellotto tirato entro 12 metri non ha malus, ma tirato entro 24 metri ha un -1d6 al colpire, a distanza di 36 metri un -2d6 al colpire.

\medskip


Una \textbf{Freccia o Dardo che colpisce si considera distrutta}, se manca si considera che abbia un 50\% (4-5-6 su un d6) che sia ancora integra.

\medskip

Una \textbf{arma di taglia superiore} \index{arma di taglia superiore}aumenta di una categoria il suo dado di danno (1d4-1d6-1d8-1d10-2d6-2d8-2d10..)

\medskip

Attaccare con un'\textbf{Arma troppo grande} \index{Arma troppo grande}rispetto alla propria taglia impone un -1d6 al Tiro per Colpire per ogni taglia di differenza tra arma e personaggio.

\medskip

Le Armi hanno tutte segnato una \textbf{tipologia di danno}\index{tipologia di danno}, ovvero T/B/P. 	Queste lettere stanno ad indicare se il danno e' di tipo Taglio, Botta 	o da Penetrazione. Questa caratteristica puo' essere importante perche' 	determinate creature possono essere immuni o subire meno danno da 	un particolare tipo di ferita (es uno scheletro contro un'arma da 	punta o un cubo gelatinoso contro un arma da penetrazione..)

\medskip

\textbf{Armi Improvvisate}\index{Armi Improvvisate}
	
Talvolta oggetti che non sono stati creati per essere armi hanno comunque una certa efficacia in combattimento. Dal momento che non si tratta di oggetti pensati per questo utilizzo, la creatura che attacca con uno di essi subisce una penalita' -1d6 al Tiro per Colpire. Un'arma improvvisata di piccole dimensioni (bottiglia) fa 1d3 di danno, di medie dimensioni (la gamba di una sedia) da 1d6, di grandi dimensioni (la gamba di un tavolo) fa 1d8 di danno
	
Un'arma da lancio improvvisata ha una gittata 3 metri.

\medskip

\textbf{Lanciare armi}\index{Lanciare armi}
	
una spada o comunque un arma non fatta per essere lanciata puo' comunque essere scagliata contro l'avversario. Il Tiro per Colpire prende un -1d6 e l'arma fa una categoria di danno inferiore (la spada lunga fa 1d6, una spada corta 1d4..). La gittata e' 3 metri.

\medskip

\textbf{Armi Perfette}\index{Armi Perfette}
	
Un'arma perfetta e' una versione di ottima fattura di un'arma normale.
Impugnare un'arma perfetta permette di aggiungere +1 al Tiro per Colpire.
	
Un arma (o proiettile) perfetta costa il doppio di un arma normale. 
Tutte le armi magiche sono anche armi perfette, tenete conto dei bonus 	indicati nell'arma magica, non dovete sommare anche quelli di essere un arma perfetta.
		
Le versioni perfette di armature e scudi concedono un bonus di +1 alla Difesa.



\pagebreak

\subsection{Equipaggiamento - Armature e Scudi}\index{Armature}\index{Scudi}

\label{equipaggiamento---armature-e-scudi}

Le armature aiutano ad essere non colpiti (alzano la Difesa), riducono il danno subito{*} (Bonus protezione) e penalizzano la prova di check magia e le prove di competenza basate su Agilita'.

Allo stesso tempo le prove di Agilita' saranno fatte con una penalita' di 3 ed il movimento diminuira' di due metri per Azione, fino ad un minimo di 0.

Quando una armatura si danneggia (e' a 0 di resistenza), il suo Bonus di Protezione al colpo ed il Bonus di Difesa diminuiscono di 3 (con un minimo di 0), quando raggiunge un valore negativo pari a resistenza totale (es -20 per un armatura leggera) e' a brandelli e non puo' piu' proteggere (ne essere riparata).

Armature diverse, specifiche o magiche hanno punteggio diversi, questa tabella serve come linea guida per il Narratore.

Quasi tutte le armature, ad eccezzione della Imbottita penalizzano l'uso di certe competenze basate su Agilita'. 

La ``Prove Agilita' e' la penalita' che si applica alle prove di competenze di Agilita' mentre si indossa un quel tipo di armatura 

\subsubsection{Tabella Armature}

\label{tabella-armature}
{\small
\begin{tabular}[c]{@{}llllllllll@{}}
\toprule 
Armatura &Costo & Difesa & Bonus Prot.* & Resist.* & Prove Agilita ’ & Prove CM & Tipo & Mov.& Peso (kg)\tabularnewline
Imbottita & 5 & 1 & 0 & 10 & 0 & 0 & L & 0 & 2\tabularnewline
Cuoio & 10 & 2 & 1 & 20 & 0 & -1 & L & 0 & 3\tabularnewline
Cuoio rinforzato & 25 & 3 & 1 & 25 & 0 & -2 & L & 0 & 5\tabularnewline
Giaco di Maglia & 15 & 4 & 2 & 30 & -1 & -3 & M & 0 & 6\tabularnewline
Scaglie & 50 & 5 & 2 & 40 & -1 & -4 & M & 0 & 7\tabularnewline
Anelli & 150 & 6 & 3 & 35 & -1 & -5 & M & 0 & 9\tabularnewline
Pettorale & 200 & 6 & 3 & 40 & -2 & -5 & M & 0 & 8\tabularnewline
Bande & 250 & 7 & 4 & 50 & -2 & -6 & P & 0 & 9\tabularnewline
Mezza armatura & 1200 & 8 & 4 & 55 & -2 & -7 & P & 1 & 12\tabularnewline
da Campo & 1400 & 9 & 5 & 60 & -3 & -7 & P & 2 & 13\tabularnewline
Completa & 1500 & 10 & 6 & 70 & -4 & -8 & P & 3 & 15\tabularnewline
\bottomrule
\end{tabular}}

\textbf{{*} Opzionale}: il Narratore puo' decidere di non fare tenere traccia dei danni subiti dall'armatura o quando assorbe dei colpi subito

\bigskip

\textbf{Usare un'Arma senza l'adeguata competenza} impone un -2d6 al Tiro per Colpire.

\textbf{Usare un'Armatura senza l'adeguata competenza} impedisce di usare il bonus di Agilita'

\textbf{Usare uno Scudo senza l'adeguata competenza} peggiora il Tiro per Colpire di 1 e diminuisce di 1 il Bonus Difesa concesso.

\textbf{Dormire in Armatura}: se si dorme in un'armatura media o pesante, il giorno seguente si e' automaticamente Affaticati. Si subisce penalita' -1 a Potenza e Agilita' e non si puo' Caricare o Correre.

Dormire in un'armatura leggera non provoca Affaticamento.

La \textbf{capacita' di movimento} del personaggio rimarra' la medesima fino all'armatura a bande poi calera' progressivamente. Il valore indicato nella colonna Mov. sono i metri in meno che il personaggio fa per Azione di Movimento.

Ad Esempio un umano in armatura completa ha movimento 6 metri, un nano 3 metri. 

\textbf{Un armatura perfetta} concede un ulteriore +1 al Bonus di Protezione e Bonus Difesa, diminuisce la possibilita' di fallire magie di 3. Diminuisce di un metro la penalita' al movimento. 

Un armatura perfetta costa il doppio di una equivalente normale.

\textbf{Un armatura magica}, e' anche perfetta e concede un Bonus di Protezione e Difesa diverso, che non si somma con quella della perfetta, valgono solo i bonus magici indicati nell'oggetto.

\textbf{Peso}: il peso indicato si riferisce alla versione per personaggi di taglia Media. Le armature adattate per personaggi di taglia Piccola pesano la meta', mentre per quelli di taglia Grande pesano il doppio.

\pagebreak

Gli \textbf{Scudi} \index{Scudi}permettono di aumentare la propria Difesa, piu' lo scudo e' imponente e pesante piu' protegge, piu' aumentano le penalita' alle prove di competenza magica, meno rende facile combattere (penalita' Tiro per Colpire).

Gli Scudi possono essere di tipo Leggero, Medio, Pesante.

\subsubsection{Tabella Scudi}

\label{tabella-scudi}

\begin{tabular}[c]{@{}llllll@{}}
\toprule 
Scudi & Costo (MO) & Bonus Difesa & Penalita' TC & Penalita' CM & Peso (kg, Tipo)\tabularnewline
Brocchiero & 5 & 1 & 0 & 1 & 2 (L)\tabularnewline
Scudo leggero di legno & 3 & 2 & 0 & 2 & 3 (L)\tabularnewline
Scudo leggero di metallo & 9 & 2 & 0 & 3 & 3 (L)\tabularnewline
Scudo medio legno & 5 & 3 & -1 & 4 & 4 (M)\tabularnewline
Scudo medio metallo & 12 & 3 & -1 & 5 & 5 (M)\tabularnewline
Scudo pesante di legno & 7 & 4 & -2 & 6 & 5 (P)\tabularnewline
Scudo pesante di metallo & 20 & 4 & -2 & 8 & 6 (P)\tabularnewline
\bottomrule
\end{tabular}

Uno scudo ha Resistenza totale pari a 20 volte il suo bonus di Difesa (Opzionale).

\bigskip

Un Armatura o Scudo magico aggiunge un +20 alla resistenza totale
per ogni +1 magico posseduto, oltre ad eventuali bonus aggiuntivi
alla Difesa 

\bigskip
\textbf{Indossare e Togliere Armature}
\bigskip

Indossare e togliere armature e' una operazione che richiede tempo ed attenzione, farlo in fretta non aiuta ed anzi tende a peggiorare la protezione data dall'armatura.

\bigskip

\begin{tabular}[c]{@{}llll@{}}
\toprule 
Tipo di Armatura & Indossare & Indossare in fretta & Togliere\tabularnewline
Scudo & 1 azione & - & 1 azione\tabularnewline
Imbottita, Cuoio, Cuoio rinforzata\\
Giaco di Maglia & 1 minuto & 5 round & 5 round\tabularnewline
Scaglie, Anelli, Pettorale, Bande & 4 minuti & 1 minuto{*} & 1 minuto\tabularnewline
Mezza armatura, da Campo, Completa & 4 minuti{*}{*} & 4 minuti{*} & 1d4+1 minuti\tabularnewline
\bottomrule
\end{tabular}

\bigskip

{*} Se qualcuno aiuta, il tempo si dimezza. Un singolo personaggio che non sta facendo altro puo' aiutare uno o due personaggi adiacenti a lui. Due personaggi non possono aiutarsi l'un l'altro a indossare un'armatura contemporaneamente.

{*}{*} Bisogna essere aiutati per indossare questa armatura. Senza aiuto e' possibile indossarla solo in fretta.

Indossare un'armatura in fretta implica un malus di -1 al Bonus di protezione, Difesa ed un malus aggiuntivo di +1 alle prove di Agilita'

\pagebreak

\section{Merci e Servizi}\index{Merci}\index{Servizi}

\label{merci-e-servizi}

Oltre ad armi e armature, un personaggio puo' avere una notevole varieta' di attrezzature a disposizione, dalle razioni da viaggio, alle corde (che possono essere utili in molte circostanze). 


\section{Equipaggiamento d'Avventura}\label{Equipaggiamento}\index{Equipaggiamento}

\label{equipaggiamento-davventura}


\begin{supertabular}{|L{4cm} |k{1.5cm} |k{1.5cm}|L{5cm} |k{1.5cm} |k{1.5cm}|}
	\textbf{Oggetto}&	\textbf{Costo}	& \textbf{Peso}&\textbf{Oggetto}&	\textbf{Costo}	&\textbf{Peso}\\
Acciarino e pietra focaia	&1 mo&	— & 	Ago da cucito&	5 ma&	—\\
Amo da pesca&	1 ma&	—&	Ampolla (vuota)&	3 mr&	0,75 kg\\
Anello con sigillo&	5 mo&	—&	Ariete portatile&	10 mo	&10 kg\\
Asta (3 m)&	5 mr&	4 kg&	Barile (vuoto)&	2 mo&	15 kg\\
Boccale di ceramica	&2 mr&	0,5 kg&	Boccetta di inchiostro o pozione&	1 mo&	—\\
Borsa da cintura (vuota)&	1 mo&	0,25 kg1&	Bottiglia di vetro&	2 mo&	0,5 kg\\
Brocca di ceramica&	3 mr&	4,5 kg&	Campanella&	1 mo&	—\\
Candela&	1 mr&	—&	Cannocchiale&	1.000 mo&	0,5 kg\\
Caraffa di ceramica&	2 mr&	2,5 kg&	Carrucola e paranco&	5 mo&	2,5 kg\\
Carta (foglio)&	4 ma&	—&	Cassa (vuota)&	2 mo&	12,5 kg\\
Catena (3 m)&	30 mo&	1 kg&	Ceralacca&	1 mo&	0,5 kg\\
Cesto (vuoto)&	4 ma&	0,5 kg&	Chiodo da rocciatore&	1 ma&	0,25 kg\\
Clessidra&	25 mo	&0,5 kg&	Coperta invernale&	5 ma&	1,5 kg1\\
Corda di canapa (15 m)	&1 mo&	5 kg&	Corda di canapa (15 m)&1 mo&5 kg\\
Corda di seta (15 m)&10 mo&2,5 kg&	Cote per affilare&2 mr&0,5 kg\\
Custodia per mappe o pergamene&1 mo&0,25 kg&	Fischietto&8 ma&—\\
Gessetto, (1 pezzo)&1 mr&—&	Giaciglio&1 ma&2,5 kg1\\
Inchiostro (boccetta da 30 g)&8 mo&—&	Lampada comune&1 ma&0,5 kg\\
Lanterna a lente sporgente&12 mo&1,5 kg&	Lanterna schermabile&7 mo&1 kg\\
Legna da ardere (per giorno)&1 mr&10 kg&	Maglio&1 mo&5 kg\\
Manette&15 mo&1 kg&	Manette perfette&50 mo&1 kg\\
Martello&5 ma&1 kg&	Olio (ampolla da 0,5 l)&1 ma&0,5 kg\\
Orologio ad acqua&1.000 mo&100 kg&	Otre&1 mo&2 kg1\\
Pennino&1 ma&—&	Pentola di ferro&8 ma&2 kg\\
Pala o badile&2 mo&4 kg&	Pergamena (Foglio)&2 ma&—\\
Piccone da minatore&3 mo&5 kg&	Piede di porco&2 mo&2,5 kg\\
Rampino&1 mo&2 kg&	Razioni da viaggio (al giorno)&5 ma&0,5 kg1\\
Rete da pesca (2,25 m)&4 mo&2,5 kg&	Sacco (vuoto)&1 ma&0,25 kg1\\
Sapone (per 0,5 kg)&5 ma&0,5 kg&	Scala a pioli (3 m)&2 ma&10 kg\\
Secchio (vuoto)&5 ma&1 kg&	\textbf{Serratura o lucchetto}&&\\
Semplice&20 mo&0,5 kg&	Media&40 mo&0,5 kg\\
Buona&80 mo&0,5 kg&	Superiore&150 mo&0,5 kg\\
Specchio piccolo di metallo&10 mo&0,25 kg&	Tela (m2)&1 ma&0,5 kg\\
Tenda&10 mo&10 kg&	Torcia&1 mr&0,5 kg\\
Tribolo&1 mo&1 kg&	Zaino&2 mo&1 kg1\\
\textbf{Oggetti e sostanze speciali}&&&	\textbf{Oggetti e sostanze speciali}&&\\
Acido (ampolla)&10 mo&0,5 kg&	Acqua santa (ampolla)&25 mo&0,5 kg\\
Antitossina (boccetta)&50 mo&—&	Bastone di fumo&20 mo&0,25 kg\\
Borsa dell’impedimento&50 mo&2 kg&	Fuoco dell’alchimista (ampolla)&20 mo&0,5 kg\\
Pietra del tuono&30 mo&0,5 kg&	Tizzone ardente&1 mo&—\\
Torcia inestinguibile&110 mo&0,5 kg&	Verga del sole&2 mo&0,5 kg\\
\textbf{Attrezzi di classe e di Abilita'}&&&\textbf{Attrezzi di classe e di Abilita'}&&\\
Agrifoglio e vischio&—&—&	Arnesi da artigiano&5 mo&2,5 kg\\
Arnesi da artigiano perfetti&55 mo&2,5 kg&	Arnesi da scasso&30 mo&0,5 kg\\
Arnesi da scasso perfetti&100 mo&1 kg&	Attrezzi da scalatore&80 mo&2,5 kg\\
Attrezzi perfetti&50 mo&0,5 kg&	Bilancia da mercante&2 mo&0,5 kg\\
Borsa del guaritore&50 mo&0,5 kg&	Laboratorio da alchimista&200 mo&20 kg\\
Lente d’ingrandimento&100 mo&—&	Simbolo sacro d’argento&25 mo&0,5 kg\\
Simbolo sacro di legno&1 mo&—&	Strumento musicale comune&5 mo\\
Strumento musicale perfetto&100 mo&1,5 kg1&	Trucchi per il camuffamento&50 mo&4 kg1\\
\textbf{Vestiario}&&&\textbf{Vestiario}&&\\
Abito da artigiano&1 mo&2 kg1&	Abito da contadino&1 ma&1 kg1\\
Abito da cortigiano&30 mo&3 kg&	Abito da esploratore&10 mo&4 kg1\\
Abito da intrattenitore&3 mo&2 kg&	Abito da Monaco&5 mo&1 kg1\\
Abito da nobile&75 mo&5 kg&	Abito da studioso&5 mo&3 kg1\\
Abito da viaggiatore&1 mo&2,5 kg&	Abito invernale&8 mo&3,5 kg1\\
Abito reale&200 mo&7,5 kg&	Veste da Chierico&5 mo&3 kg1\\
\textbf{Vitto e alloggio}&&&	\textbf{Vitto e alloggio}&&\\
Banchetto (a persona)&10 mo&—&	Birra Boccale&4 mr&0,5 kg\\
Birra Caraffa&2 ma&4 kg&	Carne (1 pezzo)&3 ma&0,25 kg\\
Formaggio (1 pezzo)&1 ma&0,25 kg&\textbf{Locanda (al giorno)}&&\\
Buona&2 mo&—&	Normale&5 ma&—\\
Scadente&2 ma&—&	Pane (a pagnotta)&2 mr&0,25 kg\\
\textbf{Pasti (al giorno)}&&&\textbf{Pasti (al giorno)}&&\\
Buono&5 ma&—&	Normale&3 ma&—\\
Scadente&1 ma&—&	Vino&&\\
Comune (caraffa)&2 ma&3 kg&	Buono (bottiglia)&10 mo&0,25 kg\\
\textbf{Cavalcature e relativo Equipaggiamento}&&&\textbf{Cavalcature e relativo Equipaggiamento}&&\\
Asino o mulo&8 mo&—&	Bardatura&&\\
Creatura Media&×22&×12&	Creatura Grande&×42&×22\\
Cane da galoppo&150 mo&—&	Cane da guardia&25 mo&—\\
Cavallo&&&Cavallo&&\\
Cavallo leggero&75 mo&—&	Cavallo leggero (addestrato)&110 mo&—\\
Cavallo pesante&200 mo&—&	Cavallo pesante (addestrato)&300 mo&—\\
Pony&30 mo&—&	Pony (addestrato)&45 mo&—\\
Morso e briglie&2 mo&0,5kg&	Nutrimento (al giorno)&5 mr&5 kg\\
Sacche da sella&4 mo&4 kg&&&\\
\textbf{Sella}&&&\textbf{Sella}&&\\
Da carico&5 mo&7,5 kg&	Da galoppo&10 mo&12,5 kg\\
Militare&20 mo&15 kg&	Sella esotica\\
Da carico&15 mo&10 kg&	Da galoppo&30 mo&15 kg\\
Militare&60 mo&20 kg&Stallaggio (al giorno)&5 ma&—\\
\textbf{Trasporti}&&&\textbf{Trasporti}\\\\
Barca a remi&50 mo&50 kg&	Barcone&3.000 mo&—\\
Carretto&15 mo&100 kg&	Carro&35 mo&200 kg\\
Carrozza&100 mo&300 kg&	Galea&30k mo&—\\
Nave a vela&10k mo&—&	Nave da guerra&25k mo&—\\
Nave lunga&10k mo&—&	Remo&2 mo&5 kg\\
Slitta&20 mo&150 kg\\
\textbf{Servizio}&\textbf{Costo}&&\textbf{Servizio}&\textbf{Costo}\\
Diligenza pubblica&3 mr/1,5 Km&&	Messaggero&5 mr per 1,5 Km\\
Mercenario esperto&8 mo al giorno&	&Mercenario normale&1 mo al giorno\\
Pedaggio stradale o d’ingresso&1 ma&&	Passaggio in nave&1 ma per 1,5 Km\\
\end{supertabular}

\bigskip

\textbf{Acciarino e Pietra Focaia}: 1 mo, Accendere una torcia con acciarino e pietra focaia costa 3 Azioni e accendere qualsiasi altro fuoco in questo modo richiede almeno altrettanto tempo.

\textbf{Ago da Cucito}: 5 ma, -, arnese in acciaio usato per cucire, filiforme, appuntito ad un'estremita' e munito all'altra di un forellino ovale (cruna), nel quale si fa passare il filo

\textbf{Alveare da Viaggio}: 10 mo, 5 kg, questi cesti di paglia forniscono una casa portatile alle api. Sono a forma di cupola, con un buco sulla sommita' coperto da un cestino con maglie piu' strette, a mo' di tappo. Questo buco permette di raccogliere piccole quantita' di miele senza distruggere l'intero alveare. Alcuni agricoltori pagano gli apicoltori per viaggiare fino alle loro fattorie con delle api in modo che queste ultime possano impollinare le loro colture.

Distruggere un alveare da viaggio fa sciamare le api in una nube con raggio di mischia. Una creatura resta Accecata fintanto che rimane nella nube, e deve superare un Tiro Salvezza su Tempra con DC 12 o diventa Inferma per 1 minuto. La condizione Infermo e' un effetto di Veleno.

\textbf{Amo da Pesca}: 1 ma,-, piccolo uncino metallico con due punte divergenti, sul quale si infila l'esca per far abboccare il pesce alla lenza.

\textbf{Ampolla (vuota)}: 3 mr, 0.75 kg, piccola anfora in vetro o ceramica con una sola ansa e collo sottile terminante in un beccuccio.

\textbf{Anello con Sigillo:} 5 mo, cerchietto di metallo, generalmente pregiato, con un'incisione atta ad imprimere sigilli su ceralacca. 

\textbf{Anello per Veleno:} +20 mo rispetto a costo anello, questo anello ha un piccolo scompartimento sotto la gemma, di solito utilizzato per contenere del veleno. Aprirlo e chiuderlo richiede un'azione; farlo senza essere notati richiede una prova di Rapidita' di Mano con DC 20.

\textbf{Ariete Portatile:}10 mo, 10 kg, questa trave di legno rivestita di metallo fornisce bonus +2 alle prove di Potenza per sfondare porte, ma permette a una seconda persona di Aiutare senza dover effettuare alcun tiro, aggiungendo un altro +2 alla prova.

\textbf{Barile (vuoto):} 2 mo, 15 kg (contiene circa 115 lt), recipiente a forma di cilindro allargato al centro, fatto di doghe di legno tenute insieme da cerchi di ferro.

\textbf{Biglie}: 1 ma, 1 kg, come i triboli, le biglie possono rallentare gli avversari. Un sacco di biglie pesante 1 kg puo' coprire 1 metro quadro. Una creatura che entri nella zona piena di biglie deve effettuare un Tiro Salvezza su Riflessi con DC 10 o cade Prona. Una creatura che si muove a meta' della sua velocita' o piu' lentamente puo' attraversare una zona piena di biglie senza problemi.

\textbf{Boa Comune:} 5 ma, 8 kg, una boa viene utilizzata per segnare un determinato punto nell'acqua, permettendo di ritornarvi successivamente. e' formata da un galleggiante (una vescica piena d'aria o una zucca sigillata), una cima copre una lunghezza di 60 metri ed una pietra di 20 kg come ancora. Il galleggiante e' solitamente dipinto con colori sgargianti ed ha una bandierina per attirare l'attenzione. Anche se le boe resistono alle correnti ed al tempo, offrono ben poca opposizione alle creature intelligenti che desiderano sabotarle.

\textbf{Boa Superiore:} 10 mo, 15 kg, questa boa ha un galleggiante tondo o ovoidale, solitamente di rame, una catena invece che una cima ed un'ancora di metallo invece di un peso. Per il resto e' come una boa normale.

\textbf{Boccale di Ceramica}: 2 mr, 0.5 kg bicchiere alto e largo con manico e, in alcuni casi, con beccuccio.

\textbf{Boccetta o Fiala}, 1 mo, un contenitore di vetro o metallo che contiene 30 grammi di liquido.

\textbf{Borsa da Cintura (vuota):} 1 mo, 0.25 kg, custodia, a forma di sacchetto di varie fogge, in pelle, in stoffa ecc., in cui si trasportano denaro, cose personali, oggetti vari.

\textbf{Bottiglia di Vetro:} 2 mo, 0.5 kg, recipiente per liquidi in vetro con corpo generalmente cilindrico e collo di diametro notevolmente piu' piccolo, che puo' essere chiuso da un tappo.

\textbf{Brocca di Ceramica:} 3 mr, 4,5 kg, una semplice brocca di ceramica chiusa con un tappo che contiene 4,5 litri di liquido.

\textbf{Calumet:} 20 mo, un calumet e' una pipa cerimoniale in due pezzi con un fornello fatto di pietra o argilla e un cannello di legno con intricate incisioni decorato con feticci appesi. La pipa viene solitamente trasportata in una speciale sacca di cuoio abbellita con perle, decorazioni e ciondoli. La pipa viene usata per fumare varie misture a base di erbe richieste per certi rituali. Fumare collettivamente un calumet a volte rientra negli incontri diplomatici come segno di solidarieta' tra i diversi gruppi. Si ottiene Bonus +1 alle prove di Diplomazia (Faccia Tosta) effettuate contro chiunque con cui si abbia fumato assieme il calumet in questo modo.

\textbf{Campanella:} 1 mo, piccola campana con batacchio interno, suonata tirando una fune o scuotendola con la mano.

\textbf{Candela:} 1 mr, una candela Illumina con luce tenue un'area di mishcia. Una candela non puo' aumentare il livello della luce oltre quella normale. Una candela brucia per 1 ora.

\textbf{Cannocchiale:} 1.000 mo, 0.5 kg, gli oggetti visti attraverso un cannocchiale sono ingranditi al doppio della loro misura.

\textbf{Caraffa di Ceramica:} 2 mr, 2.5 kg vaso con corpo e bocca larghi, collo stretto e un solo manico.

\textbf{Carrucola e Paranco:} 5 mo, 2.5 kg dispositivo per il sollevamento manuale di pesi, formato da una staffa che sorregge una ruota scanalata in cui scorre una fune. Il paranco e' un sistema meccanico usato per il sollevamento di carichi pesanti, costituito da due o piu' carrucole collegate da un cavo.

\textbf{Carta (foglio):} 4 ma, un foglio di carta misura normalmente 20 per 15 centimetri e non e' adatto per le pergamene magiche. Ha durezza 0, 1 punto ferita ed una DC per romperlo di 5.

\textbf{Carta di Riso (foglio):} 5 mr, questa carta e' fatta di riso. Ha Durezza 0, 1 Punto Ferita ed una DC per romperla di 2.

\textbf{Cassa (vuota):} 2 mo, 12.5 kg, contenitore in legno o metallo a forma di parallelepipedo, usato per imballaggio e trasporto o come mobile per conservare oggetti, abiti. 

\textbf{Catena (3 m):}30 mo, 1 kg, la catena ha Durezza 10 e 5 Punti Ferita. Puo' essere spezzata con
una prova di Potenza con DC 26.

\textbf{Caviglia per Impiombature:} 8 ma, 2.5 kg, questi chiodi di metallo lucido possono aiutare a usare le corde in molti modi, incluso intrecciare e disfare nodi, districare corde e cordicelle, raccordarle e metterle in tensione. Di solito, un chiodo e' lungo da 10 a 30 centimetri, ha un corpo sottile quasi simile a un ago, ed e' smussato a entrambe le estremita'. I chiodi piu' piccoli vengono legati al collo tramite dei cordini, mentre quelli piu' grandi vengono tenuti in dei tubini. Una caviglia per impiombature fornisce Bonus +2 alle Prove di Competenza che coinvolgono l'uso di una corda. 

\textbf{Ceralacca:} 1 mo, 0.5 kg Miscuglio di resine naturali, minerali e coloranti che si rammollisce col calore per poi indurirsi nuovamente, adatto per sigillature.

\textbf{Cesto (vuoto):} 4 ma, 0.5 kg Contenitore quadrangolare o ovale, con sponde alte e manico o manici per afferrarlo.

\textbf{Chiodo da Rocciatore:} 1 ma, 0.25 kg, i chiodi da roccia sono degli ancoraggi artificiali utilizzati dagli arrampicatori e dagli alpinisti allo scopo di proteggersi, in caso di caduta, oppure per autoassicurarsi in caso di sosta. Possono anche essere utilizzati per fissare la corda per le calate o per la progressione in arrampicata artificiale. Si tratta, in genere, di lame o sottili cunei di metallo la cui forma ne consente l'infissione nelle fessure della roccia, grazie all'utilizzo di un apposito martello. La parte terminale del chiodo da roccia e' sempre costituita da un occhiello, o foro, che consente l'inserimento di un moschettone o di un cordino.

\textbf{Clessidra (1 minuto):} 20 mo, 0.25 kg, una normale clessidra richiede 1 ora per riempire di sabbia la camera inferiore; esistono clessidre piu' grandi e piu' piccole, che arrivano a durare appena 6
secondi.

\textbf{Clessidra (1 ora):} 25 mo, 0.5 kg

\textbf{Clessidra (6 secondi):} 15 mo, 0.1 kg

\textbf{Colonia di Scarafaggi Necrofagi}: 3 mo, 0.5 kg, questa giara di vetro contiene scarafaggi necrofagi carnivori. Gli scarafaggi devono essere nutriti con almeno 125 grammi di carne al giorno oppure muoiono. Quando rilasciati su un organismo morto, ne divorano le carni in 1d4 giorni, lasciando solo le ossa. Gli scarafaggi necrofagi mangiano soltanto la carne morta e non possono danneggiare le creature viventi. Una volta rilasciati, gli scarafaggi non possono essere rimessi nella giara.

\textbf{Coperta:} 2 ma, 0.5 kg, questa coperta calda ha delle cinghie che permettono di legarla una volta arrotolata.

\textbf{Coperta Invernale:} 5 ma, 1,5 kg, panno pesante che si stende sul letto o giaciglio. Di solito sono pellicce ricucite assieme.

\textbf{Corda di Canapa}: (15 m) 1 mo 5 kg Questa corda ha 2 Punti Ferita e puo' essere spezzata con un prova di Potenza con DC 23.

\textbf{Corda di Ragnatela} (15m),100 mo, 2 kg, tanto rara da essere sconosciuta in superficie, la corda di ragnatela viene tessuta nelle profondita’ a partire dalla tela secreta dai ragni giganti.
La corda di ragnatela ha 6 Punti Ferita e puo' essere rotta con una prova di Potenza con DC 25

\textbf{Corda di Seta} (15 m): 10 mo, 2.5 kg, questa corda ha 4 Punti Ferita e puo' essere spezzata con una prova di Potenza con DC 24

\textbf{Corda per Armi}: 1 ma, le corde per armi sono legacci di pelle lunghi 60 centimetri che si allacciano all’elsa dell’arma e al polso. Se si lascia andare l’arma o si viene disarmati si puo' recuperarla con un’azione veloce ed essa non si allontana dall’area di mischia. Non e' possibile pero' cambiare arma finché non si e' slegata la prima (3 Azioni) o si taglia la corda (Azione di attacco, Durezza 0, 0 Punti Ferita). A differenza di quanto avviene con un guanto d’arme con sicura, e' possibile utilizzare una mano cui e' legata una corda per armi, anche se l’arma penzolante puo' interferire con le azioni piu' delicate.

\textbf{Corno da Segnalazione:}1 mo, 1 kg, suonare un corno richiede una prova di Intrattenere (strumenti a fiato) con DC 10 e puo' comunicare concetti come ``All'attacco!'', ``Aiuto!'', ``Avanzare!'', ``Ritirata!'', ``Fuoco!'' e ``Allarme!''. Il suono di un corno puo' essere udito chiaramente a 500 metri di distanza. Per ogni successivo incremento di 250 metri, la prova di Consapevolezza per sentirlo subisce penalita' -1.

\textbf{Cote per Affilare:} 2 mr, 0.5 kg, pietra di forma tondeggiante di smeriglio o altro materiale abrasivo che, ruotando, serve ad affilare o a levigare.

\textbf{Custodia per Pergamene:} 1 mo, 0.25kg, una custodia di legno o pelle puo' tenere quattro Pergamene; e' possibile inserirne di piu', ma recuperarne una diviene difficile e costa 2 Azioni, mentre recuperarne la sola presente costa 1 Azione. Per danneggiarne il contenuto e' necessario distruggere il contenitore (Durezza 2 per la pelle e 5 per il legno, 2 Punti Ferita, Rompere DC 15). Una custodia per pergamene non e' impermeabile.

\textbf{Fischietto da Segnalazione:} (o silenzioso) 8 ma (9ma), con una prova di Intrattenere (strumenti a fiato) con DC 5 si riescono comunicare concetti come ``All'attacco!'', ``Aiuto!'', ``Avanzare!'', ``Ritirata!'', ``Fuoco!'' e ``Allarme!''. Il suono del fischietto si puo' udire chiaramente (Consapevolezza DC 0) fino a 250 metri di distanza. Per ogni 250 metri successivi la prova di Consapevolezza subisce penalita' -2. I fischietti silenziosi possono essere uditi solo dagli animali o dalle creature con l'udito particolarmente fino.

\textbf{Fondina da Manica:} 100 mo, 0.5 kg, quando indossata all'interno di maniche voluminose, questa fondina di cuoio permette di estrarre una balestra a mano o una pistola da giacca nascosta come Azione. L'arma e' posta su delle corsie e viene estratta direttamente nella propria mano. A differenza di un fodero da polso, una fondina da manica e' abbastanza voluminosa da essere evidente a un'ispezione ravvicinata, sebbene se portata al di sotto di abiti sufficientemente larghi potrebbe non suscitare una prova di Consapevolezza per notarla. Una singola fondina da manica puo' contenere una balestra a mano o una pistola da giacca, non entrambe.

\textbf{Forziere piccolo:}

\textbf{Forziere medio} 5 mo 25 kg

\textbf{Forziere grande} 10 mo 50 kg

\textbf{Forziere enorme} 25 mo 125 kg

\textbf{Gessetto} (1 pezzo) 1 mr, Barretta di gesso bianco o colorato utilizzata per scrivere

\textbf{Gesso per Impronte:} 2 mo, 0.5 kg, Questo gesso a presa rapida e' perfetto per preservare una serie di impronte al fine di esaminarle in un secondo momento. Spendendo 1 minuto a disporre il gesso e aspettare che si asciughi, si possono copiare le impronte, permettendo ad altri di esaminarle senza viaggiare fino al luogo in cui sono state rilevate, ed evitando che la DC della prova di Sopravvivenza per analizzarle aumenti per via del tempo trascorso o delle condizioni atmosferiche. 

\textbf{Giaciglio:} 1 ma, 2.5 kg, letto misero, perlopiu' fatto di paglia o di cenci.

\textbf{Giardino da Viaggio:} 200 mo, 250 kg, questo kit per carri pesanti comprende scatole e vasi predispo­sti per la crescita di un'ampia varieta' di vegetali, in aggiunta allo spazio per una coppia di animali, come delle capre, e il loro nutrimento. Un giardino da viaggio fornisce cibo ed erbe curative. Funziona in modo simile a una Borsa del Guaritore, fornendo fino a 5 usi al giorno, e non si consuma mai. Inoltre, coloro che ingeriscono quotidianamente una certa varieta' di ­erbe e verdure fresche del giardino ottengono bonus +1 ai Tiri Salvezza contro Malattia. 

\textbf{Gile' di Sughero:} 25 mo, 0.5 kg, questo gile' di tessuto contiene tasche piene di sughero, che forniscono a chi lo indossa una maggiore capacita' di galleggiamento. Inizialmente indossato da pescatori e marinai, protegge dall'annegamento. Mentre si indossa un gile' di sughero si subisce penalita' -2 alle prove di Agilita' e Nuotare, ma anziché finire sott'acqua dopo aver fallito la prova di 5 o piu', questo accade solo fallendo di 10 o piu'. Inoltre, si ottiene bonus +4 alle prove di Nuotare (Resistenza) per evitare danni da Affaticamento. Un gile' di sughero puo' essere indossato sotto l'armatura.

\textbf{Inchiostro} (boccetta da 30 g): 8 mo, un inchiostro diverso da nero costa il doppio.

\textbf{Kit da Cacciatore:} 15 mo 21,75 kg il kit comprende acciarino e pietra focaia, una borsa da cin­tura, una borsa per componenti di incantesimi, una corda, un giaciglio, un kit da rancio, un otre, una pentola di ferro, razioni da viaggio (5 giorni) e uno zaino.

\textbf{Kit da Cortigiana:} 10 mo 2.5 kg il kit comprende oggetti che aiutano una cortigiana a dare sollievo al corpo e allo spirito. Per il corpo, il kit contiene un rasoio, oli profumati e balsami, profumi, una pentola per tenere in caldo e una certa varieta' di vestiti accattivanti. Libri di poesia, letteratura e teatro, spesso riguardanti argomenti salaci e pieni di doppi sensi, servono invece per intrattenere la mente.

\textbf{Kit da Intrepido:} 9 mo, 21 kg, il kit comprende acciarino e pietra focaia, una borsa da cin­tura, una corda, un giaciglio, un kit da rancio, un otre, una pentola di ferro, razioni da viaggio (5 giorni), sapone, torce (10) e uno zaino.

\textbf{Kit da Investigatore:} 40 mo, 18,5 kg, il kit comprende acciarino e pietra focaia, una borsa da cin­tura, una corda, un giaciglio, inchiostro, un kit da rancio, un kit per creazioni alchemiche, un otre, un pennino, una pentola di ferro, razioni da viaggio (5 giorni), sapone, torce (10) e uno zaino.

\textbf{Laccio per Libri:} 3 ma, 0.25 kg, questa cordicella intrecciata di metallo possiede un fermaglio che si fissa al lucchetto di un normale libro. L'altra estremita' della cordicella viene attaccata a una cintura o a un anello da cintura. La cordicella e' lunga 3 metri e ritraibile. Se si fa cadere il proprio libro mentre e' attaccato al laccio, lo si puo' recuperare con il costo di 1 Azione. Mentre e' attaccato al personaggio, il libro non puo' mai trovarsi oltre a 1 metro di distanza dal personaggio. Slacciare il libro richiede un'Azione; in alternativa, la cordicella (Durezza 5, 10 Punti Ferita) puo' essere tagliata per liberare il libro.

\textbf{Lampada Comune:} 3 ma, 0.5 kg, una lampada Illumina con luce normale un'area piccola di 4,5 metri di raggio. Una lampada non puo' aumentare il livello della luce oltre quella normale o intensa. Una lampada brucia per 6 ore con 0.5 litri d'olio. E' dotata di una piccola schermatura per impedire alla luce di uscire. e' possibile trasportarla con una mano.

\textbf{Lavagna:} 1 mo, 1 kg, questa piastra di pietra nera (ardesia) levigata grande come un libro e' circondata da una cornice di legno. Strofinandone la superficie con un panno bagnato si cancellano le scritte lasciate con il gesso.

\textbf{Legna da Ardere} (per giorno): 1 mr, 10 kg, insieme di pezzi di rami o di tronchi d'albero da ardere.

\textbf{Lente del Cacciatore:} 100 mo, questa complessa lente viene posta su un occhio e occupa lo slot occhi quando e' in uso. Quando la si utilizza con un attacco a distanza, si riduce qualsiasi penalita' di gittata ai propri attacchi di 2. Tuttavia, gli oggetti entro 9 metri diventano difficili da vedere, e si subisce penalita' -2 alle prove di Consapevolezza basate sulla vista mentre si porta una lente del cacciatore.

\textbf{Magnete:} 5 ma, 0.25 kg, i magneti piu' piccoli sono piuttosto deboli ed utilizzati principalmente per individuare o attirare a brevi distanze ferro, mithral o adamantio. Questo magnete a ferro di cavallo puo' sollevare fino a 1 kg di metallo. 

\textbf{Manette:} Molte manette hanno serrature; aggiungere il costo della serratura desiderata al costo delle manette.

Allo stesso prezzo si possono comprare manette per creature di taglia Piccola. Per creature di taglia Grande le manette costano 10 volte il costo indicato e per creature di taglia Enorme 100 volte il costo indicato. Le creature di taglia Mastodontica, Colossale, Minuscola, Minuta e Piccolissima possono essere trattenute solo con manette costruite appositamente che costano 100 volte il costo indicato.

\textbf{Manette Perfette:} 50 mo, 1 kg

\textbf{Martello:} 5 ma 1 kg Se viene usato in combattimento, il martello viene considerato un'arma improvvisata ad una mano che infligge danni contundenti come fosse un guanto d'arme chiodato della stessa taglia.

\textbf{Olio:} (ampolla da 0.5 l) 1 ma 0.5 kg in una lanterna 0.5 litri d'olio bruciano per 6 ore. e' possibile utilizzare un'ampolla d'olio come arma a spargimento, e occorrono 3 Azioni per preparare un'ampolla con una miccia. Una volta lanciata, c'e' solo una probabilita' del 75\% che l'ampolla prenda fuoco. Si consideri l'attacco come un attacco da contatto a distanza con portata di 6 metri. Il colpo diretto provoca 1d6 danni da fuoco. Tutte le creature entro raggio di mischia dal punto in cui e' caduta l'ampolla subiscono 1 danno da fuoco come effetto dello spargimento. Nel round successivo al colpo diretto la vittima subisce 1d6 danni aggiuntivi. La vittima puo' sfruttare 2 Azioni per tentare di spegnere le fiamme prima di subire questi danni aggiuntivi. Occorre superare un Tiro Salvezza su Riflessi con DC 15 per spegnere le fiamme. Rotolarsi per terra da' al personaggio bonus +2 al Tiro Salvezza. Tuffarsi in un lago o smorzare le fiamme con mezzi magici spegne automaticamente le fiamme.

\textbf{Orologio ad Acqua:} 1.000 mo, 100 kg, questo grande congegno ingombrante fornisce l'ora esatta con lo scarto di mezz'ora per giorno da quando e' stato regolato l'ultima volta. Richiede una fonte d'acqua e deve essere tenuto immobile poiché segna il tempo con il flusso regolare delle gocce d'acqua.

\textbf{Ospedale Mobile:} 1.000 mo, 250 kg, questo kit per carri include tutto l'equipaggiamento necessario a prendersi cura di un massimo di 10 malati o feriti contemporaneamente. Comprende due tende grandi, 10 giacigli con coperte, un tavolo robusto, un kit da cerusico e cinque borse del guaritore. Fornisce a chiunque lo usi bonus +2 alle prove di Guarire per prestare pronto soccorso, puo' essere usato per trattare ferite mortali con un singolo uso della borsa del guaritore anziché due, e raddoppia il ritmo di recupero dei pazienti durante le cure a lungo termine.

\textbf{Otre:} 1 mo, 2 kg recipiente di pelle animale, utilizzato per trasportare e conservare i liquidi.

\textbf{Pala o Badile:} 2 mo, 4 kg, se una pala e' usata in combattimento, viene considerata come un'arma improvvisata ad una mano che infligge danni contundenti pari a quelli di un randello della stessa taglia. 

\textbf{Pennino;} 1 ma, piccola lamina in acciaio innestata sul cannello della penna per scrivere; l'inchiostro la raggiunge attraverso l'immersione nel calamaio, oppure direttamente da un serbatoio inserito nel cannello.

\textbf{Pentola di Ferro:} 8 ma 2 kg recipiente da cucina munito di due manici, utilizzato per cuocere le vivande.

\textbf{Pergamena} (Foglio), 2 ma, pelle ovina o caprina, lavata, depilata e sbiancata, usata per scriverci.

\textbf{Piccone da Minatore:} 3 mo 5 kg se un piccone da minatore e' usato in combattimento, viene considerato come un'arma improvvisata ad una mano che infligge danni perforanti pari a quelli di un piccone pesante della stessa taglia.

\textbf{Piede di Porco:} 2 mo, 2.5 kg Un piede di porco fornisce Bonus +2 alle prove di Potenza effettuate per forzare una porta o uno scrigno. Se usato in combattimento, il piede di porco viene considerato un'arma improvvisata ad una mano che infligge danni contundenti pari a quelli di un randello della stessa taglia.

\textbf{Polveri:} 1 mr 0.25 kg, il gesso polverizzato, la farina ed altri materiali simili sono popolari fra gli avventurieri per notare le creature invisibili. Lanciare un sacco di polveri in una zona di mischia richiede un Tiro per Colpire verso Difesa 5 e rivela momentaneamente se vi si trova una creatura Invisibile. Un metodo piu' efficace e' quello di spargere le polveri su una superficie (3 Azioni) e cercare le tracce.

\textbf{Pompa Antincendio} 200 mo 250 kg Questo kit per carri pesanti comprende un serbatoio d'acqua, una pompa e un erogatore rotante. Se l'operatore supera una prova di Potenza con DC 20, la pompa antincendio rilascia un flusso d'acqua che raggiunge fino i 18 metri di distanza. Ciascuna persona che aiuta con la pompa fa diminuire a DC di 5. Operare o aiutare costa 3 Azioni. La pompa estingue un area di mischia di fuoco non magico per round. Il serbatoio contiene acqua a sufficienza per 10 round di pompaggio e ci vogliono 10 minuti per ricaricarlo da un corso d'acqua, uno stagno, un lago o un altro corpo idrico.

\textbf{Prigione Portatile} 200 mo 150 kg Questo kit per carri comprende una gabbia di sbarre di metallo con una porta su un lato. Sebbene le prigioni portatili siano state originariamente ideate dai viaggiatori per contenere animali feroci, le guardie cittadine le usano abitualmente per radunare i criminali, e alcuni cacciatori di taglie le impiegano per trasportare grandi gruppi di prigionieri. La maggior parte di queste prigioni e' dotata di lucchetti: aggiungere il costo del lucchetto desiderato al costo della prigione portatile. Una gabbia pensata per le persone include panche e corrimano a cui vengono attaccate delle manette. Una prigione portatile pensata per gli animali include un trogolo per l'acqua e una porta piu' piccola per fornire cibo.

\textbf{Punta di Metallo} 5 mr 0.5 kg Questa punta di metallo lunga 30 centimetri si usa per tenere le porte aperte o per assicurarvi corde per scalare. Sentire una punta di metallo che viene martellata in posizione richiede una prova di Consapevolezza con DC 5.

\textbf{Rampino} 1 mo 2 kg Lanciare efficacemente il rampino da scalata richiede un attacco a distanza, considerandolo un'arma da lancio con gittata di 3 metri. Gli oggetti con ampio spazio per ricevere l'aggancio di un rampino hanno Difesa 5.

\textbf{Razioni da Viaggio} (al giorno) 5 ma 0.5 kg Quantita' di cibo, bevande ecc. che consuma giornalmente un'avventuriero o un viaggiatore.

\textbf{Rete da Pesca} (2,25 m) 4 mo 2.5 kg Attrezzo costituito da un intreccio, a maglie piu' o meno fitte, di fili di fibre naturali o artificiali, usato per pescare.

\textbf{Rete per Farfalle} 5 mo 1 kg1 Una delle estremita' di quest'asta di 2 metri ha una rete sottile. e' possibile utilizzarla per setacciare materiali abbastanza sottili da passare attraverso la stretta retina, come la sabbia o l'acqua. e' possibile anche utilizzarla per catturare creature Minute o Piccolissime come si trattasse di una rete (l'arma), anche se non e' necessario ripiegare la rete se manca il bersaglio, ed il manico della retina si utilizza come la corda di una rete.

\textbf{Richiamo per Animali} 1 ma - Questi fischietti di canna o bambu' imitano i versi di vari animali selvatici. Ciascun fischietto e' legato a uno specifico tipo di animale e un verso specifico (che solitamente segnala la disponibilita' di cibo o di un compagno per attirare l'animale). Un richiamo fornisce bonus +2 alle prove di Sopravvivenza per seguire le tracce di animali del tipo a cui e' legato o per cavarsela in territori selvaggi.

\textbf{Sacco (vuoto)} 1 ma 0.25 kg Involucro di tela ruvida, carta, tela o altri materiali che si prestano all'uso, di forma allungata e aperto in alto, in cui si conservano o si trasportano materiali o oggetti

\textbf{Sapone} (per 0.5 kg) 5 ma 0.5 kg Prodotto comunemente usato per detergere persone, abiti, oggetti.

\textbf{Scala a Pioli} (3 m) 2 ma 10 kg Struttura fissa a gradini che permette di salire o di scendere da un livello all'altro in edifici o in luoghi aperti. Una semplice scala a pioli di legno.

\textbf{Scendicorde} 50 mo 1,5 kg Si puo' collocare questo macchinario di metallo su una sezione di una corda tesa connessa da un punto alto a uno piu' basso, permettendo cosi' di scendere lungo la corda verso il basso con facilita'. Usare un scendi corde richiede una sola mano, lasciando l'altra libera durante la discesa. Fissare lo scendi corde a una corda e' un Azione. Iniziare la discesa e' un'azione veloce. La corda viene discesa al ritmo di 18 metri per round (2 movimenti per round). Farlo non richiede alcun'azione, ma ci si deve muovere lungo la corda verso il basso. Recuperare lo scendi corde una volta che si e' terminata la discesa e raggiunta l'estremita' della corda e' un'Azione.
Si puo' lasciar andare lo scendi corde come azione immediata.

\textbf{Secchio} (vuoto) 5 ma 1 kg Recipiente piuttosto capace, di forma cilindrica o troncoconica, in legno o in metallo, dotato di manico semicircolare, usato per contenere liquidi o altri materiali

\textbf{Serratura o Lucchetto} La DC per aprire una serratura (o un lucchetto) con Disattivare Congegni (Criminalita') dipende dalla qualita' della serratura (o del lucchetto); molto semplice (DC 20), media (DC 25), buona (DC 30) e superiore (DC 40).

\textbf{Semplice} 20 mo 0.5 kg

\textbf{Media} 40 mo 0.5 kg

\textbf{Buona} 80 mo 0.5 kg

\textbf{Superiore} 150 mo 0.5 kg

\textbf{Spago} (15 m) 1 mr 0.25 kg Venduto in gomitoli di 15 metri, spaghi e lana sono utili per creare gli interruttori delle trappole e sono necessari per le frecce e gli uncini da scalata. Spaghi e lana hanno Durezza 0, 1 Punto Ferita e una DC per romperli di 14.

\textbf{Specchio} \textbf{Piccolo} \textbf{di} \textbf{Metallo} 10 mo 0.25 kg Lastra levigata di vetro, metallizzata su una faccia, che riflette la luce e le immagini.

\textbf{Tabacchiera} (Stagno o Legno) 5 mo - Il coperchio a cerniera di questa piccolissima scatoletta ornata aderisce alla guarnizione formando una tenuta stagna. La scatoletta e' usata per contenere vari tabacchi, polveri e sostanze simili. La scatoletta puo' essere fatta di qualsiasi tipo di materiale, dal legno all'avorio fino ai metalli preziosi incastonati di gemme.

\textbf{Tabacchiera} (Osso o Guscio di Tartaruga) 25 mo

\textbf{Tabacchiera} (Avorio o Metallo Prezioso) 300 mo

\textbf{Tappi per Orecchie} 3 mr - Fatti di cotone o sughero cerato, i tappi per orecchie concedono Bonus +2 al Tiro Salvezza contro gli effetti che richiedono l'udito ma infliggono penalita' -5 alle prove di Consapevolezza basate sull'udito.

\textbf{Tatuaggio} 1 mr - 20 mo - Il costo di un tatuaggio dipende dalla sua qualita', dalla dimensione e dal numero di colori utilizzati. Un tatuaggio grande come una moneta, di colore blu che scolorira' nel giro di dieci anni puo' costare 1 mr, uno grande come una mano in inchiostro nero che non scolorisce 1 ma ed uno che copre l'intera schiena e che richiede piu' sessioni costa 10 mo. Ogni colore aggiuntivo costa come un singolo tatuaggio della stessa taglia.

\textbf{Tela} (m2) 1 ma 0.5 kg Uno dei tre tipi, con la saia e il raso, di armatura dei tessuti, in cui i fili della trama passano alternativamente sopra e sotto i fili dell'ordito, costituendo un tessuto compatto senza rovescio.

\textbf{Tenda} \textbf{piccola} 10 mo 10 kg Le tende hanno diverse dimensioni e possono ospitare tra 1 e 10 persone. Un tenda piccola ospita una creatura Media e richiede 20 minuti per essere montata, una tenda media ospita 2 creature e richiede 30 minuti per essere montata, una grande ospita 4 creature e richiede 45 minuti ed un padiglione ospita 10 creature e richiede 90 minuti (due creature Piccole contano come una Media ed una Grande conta come due Medie). Le tende a padiglione sono abbastanza grandi da consentire di accendere un piccolo fuoco al centro. Smontare una tenda richiede la meta' del suo tempo di montaggio.

\textbf{Tenda media} 15 mo 20 kg

\textbf{Tenda grande} 30 mo 20 kg

\textbf{Tenda enorme} (padiglione) 100 mo 25 kg

\textbf{Tomo} \textbf{delle} \textbf{Epopee} 50 mo 1,5 kg Questo corposo libro e' rilegato in tela cerata e decorato con scene di gloriosi combattimenti tra antichi eroi e feroci mostri. Contiene svariati racconti di valore, sconfitta e vittoria, tutti accompagnati da illustrazioni dai colori vivaci. Dopo aver consultato il libro per 1 ora, per le 24 ore successive si guadagna Bonus +2 alle prove di Intrattenere (canto) e Intrattenere (oratoria) e Bonus +2 alle prove di Conoscenze (nobilta') riguardanti i lignaggi eroici.

\textbf{Torcia} 1 mr 0.5 kg Una torcia brucia per 1 ora ed Illumina con luce normale un'area di 3 metri. Se usata in combattimento, la torcia viene considerata un'arma improvvisata ad una mano che infligge danni contundenti 1d4 piu' 1 punto ferita da fuoco. 

\textbf{Trampolino} \textbf{Pieghevole} 50 mo 5 kg Questo compatto trampolino si smonta e rimonta come una tenda, permettendo un agevole trasporto. Montare o smontare il trampolino richiede 1 minuto. Quando utilizzato da due creature, un trampolino pieghevole fornisce bonus +5 a tutte le prove di Acrobatica effettuate per Saltare. Se una creatura in caduta cade sul trampolino, ignora i primi danni dati dalla distanza di caduta.

\textbf{Tribolo} 1 mo 1 kg I triboli sono chiodi di ferro a quattro punte costruiti in modo da avere sempre una punta rivolta verso l'alto. Si spargono sul terreno nella speranza che i nemici ci camminino sopra o almeno rallentino per evitarli. Una borsa contenente 1 kg di triboli copre un quadretto.

Ogni volta che ci si muove in un'area coperta con i triboli (o si passa un round combattendo mentre ci si trova nell'area), si rischia di pestarne uno. I triboli effettuano un Tiro per Colpire senza alcun bonus contro la creatura. Per questo attacco lo scudo, l'armatura della creatura non contano. Se la creatura indossa le scarpe o qualche altra copertura per i piedi, ha un bonus alla Difesa di +2. Se i triboli riescono a colpire, la creatura ne ha pestato uno. Il tribolo infligge 1 punto ferita e la velocita' della creatura e' dimezzata a causa del piede ferito. Questa penalita' al movimento dura 24 ore, fino a quando la creatura non viene curata con successo con una prova di Sopravvivenza con DC 15 oppure fino a quando non riceve almeno 1 punto di cure magiche.
Una creatura alla Carica o che sta correndo deve fermarsi immediatamente se pesta un tribolo. Qualsiasi creatura che si muove a velocita' dimezzata o piu' lentamente puo' camminare attraverso una distesa di triboli senza problemi.

I triboli potrebbero essere inefficaci contro avversari insoliti.

\textbf{Veste da Apicoltore} 20 mo 5 kg1 Questi pesanti strati di vestiti, uniti ad un cappello ampio e dotato di una rete, rendono impossibile a creature Minute e Piccolissime di entrare in contatto con il corpo. Indossare una veste da apicoltore dimezza la velocita' ma concede RD 10/- contro gli sciami di creature Piccolissime e RD 5/--- contro sciami di creature Minute.

\textbf{Veste Uncinata} 10 mo 2 kg Piccole coperture di pelle impediscono alle centinaia di piccoli aghi uncinati che ricoprono la superficie di questo abito di ferire chi lo indossa. Qualsiasi creatura che ferisca chi lo porta con un attacco naturale o senz'armi deve superare un Tiro Salvezza su Riflessi con DC 15 o subisce 1 danno. Se una creatura ingoia chi lo indossa, subisce 1 danno per round finché non lo sputa o chi lo indossa non fugge o muore (in questo caso la veste ha subito troppi danni per essere una minaccia). La veste puo' essere indossata soltanto se non si indossano armature o se ne indossa una di tipo leggero.

\textbf{Zaino} 2 mo 1 kg Sacco di grossa tela o di altro materiale molto resistente, che si porta appeso alle spalle, puo' contenere 0,05 metri cubi (50 litri) di materiali nella tasca principale. 

\textbf{Zaino} \textbf{Perfetto} 50 mo 2 kg Questo zaino ha numerose tasche, utili per conservare gli oggetti necessari per andare in avventura. Ci sono ganci per attaccare oggetti come borracce, borse e coperte arrotolate. Ha fasce imbottite che si tirano sul petto e sulle spalle per distribuire meglio il peso. Come un normale zaino puo' contenere 0,05 metri cubi (50 litri) di materiali nella tasca principale. Indossando uno zaino perfetto, il punteggio di Potenza ai fini di determinare la Capacita' di Carico e' considerato maggiore di +1.

\subsection{Oggetti e Sostanze Speciali}\index{Sostanze Speciali}

\label{oggetti-e-sostanze-speciali}

Tutti gli oggetti inclusi nella lista, fatta eccezione per la torcia inestinguibile e l'acquasanta, possono essere fabbricati da un personaggio con la competenza Lavoro (alchimia). La DC per creare gli aggetti e' indicata come: Creazione DC XX

\textbf{Acido} (ampolla) 10 mo 0.5 kg e' possibile lanciare un'ampolla d'acido come arma a spargimento. Si consideri l'attacco come un attacco di contatto con gittata 3 metri. Il colpo diretto provoca 1d6 danni da acido. Tutte le creature entro 1,5 metri dal punto in cui e' caduta l'ampolla subiscono 1 danno da acido come effetto dello spargimento.
Creazione DC 15

\textbf{Acquasanta} (ampolla) 25mo 0.5 kg L'acquasanta infligge danni ai Non Morti e agli Esterni malvagi quasi come se fosse acido. Un'ampolla di acquasanta puo' essere lanciata come arma a spargimento.

Si consideri l'attacco come un attacco di contatto con gittata di 3 metri. Un'ampolla si rompe se scagliata contro il corpo di una creatura corporea, ma contro una creatura incorporea l'ampolla deve essere aperta e l'acquasanta versata sulla creatura. Di conseguenza, si puo' spruzzare una creatura incorporea con l'acquasanta solo se si e' adiacenti ad essa.

Il colpo diretto di un'ampolla di acquasanta provoca 2d4 danni ai Non Morti e agli Esterni malvagi. Tutte le creature di questo tipo entro raggio di mischia da dove e' caduta l'ampolla subiscono 1 danno come effetto dello spargimento.

I templi dei Dei buoni vendono acquasanta a prezzo di costo (senza guadagno). L'acquasanta si ottiene usando l'Essenza di Creazione, acqua, a livello potere 18 per 5 fialette.

\textbf{Antiemetico} 25 mo 0.25 kg Questo liquido verde dolce e saporito crea un senso di calore e conforto. Lo sciroppo copre lo stomaco e lo rende piu' resistente. Per 1 ora dopo averlo bevuto si ottiene Bonus +5 ai Tiri Salvezza per resistere agli effetti che rendono Nauseati o Infermi. Monodose. Creazione DC 18

\textbf{Antiepidemico} (fiala) 50 mo - Bevendo una fiala di questo liquido bianco latte dal pessimo sapore si ottiene Bonus +5 ai Tiri Salvezza contro le Malattie, effettuati nell'ora successiva. Se gia' infetti, si possono effettuare due Tiro Salvezza per resistere alla Malattia in quella determinata giornata (senza il bonus +5) e tenere il risultato migliore. Monodose. Creazione DC 18

\textbf{Antitossina} (boccetta) 50 mo - Se si beve l'antitossina, si ottiene Bonus +5 a tutti i Tiri Salvezza su Tempra contro Veleni per 1 ora. Monodose. Creazione DC 18

\textbf{Bastone} \textbf{del} \textbf{Fumo} 20 mo 0.25 kg Questo bastone di legno trattato con procedimento alchemico crea istantaneamente un denso fumo opaco quando viene infiammato. Il fumo riempie un cubo con spigolo di 3 metri (distanza di mischia) (come per l'Essenza Creazione), tranne che il fumo viene dissipato in 1 round da un vento moderato o piu' intenso. Il bastone si consuma in 1 round e il fumo si dissolve
poi naturalmente. Creazione DC 18

\textbf{Benedizione dell'Alchimista} 1 mo - Molto amata dai giovani libertini, si tratta di una polvere cristallina simile al sale. Mischiata con l'acqua crea una bevanda frizzante che cura gli effetti della sbornia. Monodose. Creazione DC 15

\textbf{Borsa dell'Impedimento} 50 mo 2 kg Questa borsa di cuoio rotonda e' piena di melassa, resina o altra sostanza appiccicosa. Quando si scaglia la borsa contro una creatura (come attacco di contatto a distanza con gittata 3 metri), la borsa si apre e la sostanza contenuta invischia ed intralcia la vittima, diventando resistente ed elastica con l'esposizione all'aria. Una creatura Intralciata subisce penalita' -2 al Tiro per Colpire e penalita' -2 alla Agilita', e inoltre deve effettuare un Tiro Salvezza su Riflessi con DC 15 o resta appiccicata al suolo, incapace di muoversi. Anche con un Tiro Salvezza riuscito, puo' solo muoversi con una penalita' dl movimento di 1.
La sostanza non agisce su creature di taglia Enorme o superiore. Una creatura volante non viene appiccicata al suolo, ma deve effettuare un Tiro Salvezza su Riflessi con DC 15 o perde la capacita' di Volare (sempre che usi le ali per farlo), cadendo a terra. La borsa dell'impedimento non funziona sott'acqua.

Una creatura appiccicata al suolo (o impossibilitata a Volare) puo' liberarsi con una prova di Potenza riuscita con DC 17 oppure infliggendo 15 danni alla sostanza con un'arma tagliente. Una creatura che tenta di sfregare via la sostanza da sé o da un'altra creatura che assiste non ha bisogno di effettuare un Tiro per Colpire; colpire la sostanza e' automatico, poi la creatura che colpisce effettua un tiro per i danni per vedere quanta sostanza e' riuscita a sfregare via. Una volta libera, la creatura si muove a velocita' dimezzata, anche volando. Una creatura invischiata dalla sostanza puo' lanciare Essenze ma deve superare una prova di Concentrazione con DC 20. La sostanza diventa fragile dopo 2d4 round, staccandosi da sola e perdendo ogni effetto. Un'applicazione di solvente universale su una creatura appiccicata dissolve la sostanza alchemica immediatamente. Creazione DC 18

\textbf{Fermasangue} 25 mo - Questa sostanza rosa e appiccicosa aiuta a curare le ferite. Utilizzarne una dose concede Bonus +4 alle prove di Guarire quando si effettua pronto soccorso, si guariscono le ferite da tribolo ed oggetti simili o si trattano ferite mortali. Una dose di fermasangue pone termine ad un effetto di Sanguinamento come se si fosse superata una prova di Sopravvivenza con DC 15. Quando si trattano le ferite mortali, utilizzare una dose di fermasangue conta come un utilizzo della borsa del guaritore (e si ottiene bonus +4). La confezione contiene 3 dosi. Creazione DC 18

\textbf{Fiasco Alcalino} 15 mo 0.5 kg Questo fiasco di liquidi caustici reagisce con gli acidi naturali delle melme. e' possibile lanciare un fiasco alcalino come arma a spargimento con gittata 3 metri. Contro le creature non melme un fiasco alcalino funziona come un'Ampolla d'acido. Contro le melme e altre creature acide il fiasco alcalino infligge i danni raddoppiati indicati da Ampolla d'Acido. Creazione DC 18

\textbf{Fumogeno} 25 mo - Questa piccola sfera di argilla contiene due sostanze alchemiche separate da una sottile barriera. Quando si rompe la sfera, le sostanze si uniscono e riempiono un area di mischia con una nuvola di fumo nerastro e innocuo. Il fumogeno funziona come un bastone del fumo, ma il fumo rimane per 1 round prima di disperdersi. e' possibile lanciare un fumogeno come attacco di contatto con gittata 3 metri. Creazione DC 18

\textbf{Fuoco dell'Alchimista} 20 mo 0.5 kg Si puo' lanciare un'ampolla di fuoco dell'alchimista come arma a spargimento. Si consideri l'attacco come un attacco di contatto a distanza, con gittata 3 metri.

Il colpo diretto provoca 1d6 danni da fuoco. Tutte le creature entro raggio di mischia dal punto in cui e' caduta l'ampolla subiscono 1 danno da fuoco come effetto dello spargimento. Nel round successivo al colpo diretto la vittima subisce 1d6 danni da fuoco aggiuntivi. La vittima puo' sfruttare 2 Azioni per tentare di spegnere le fiamme prima di subire questi danni aggiuntivi. Occorre superare un Tiro Salvezza su Riflessi con DC 15 per spegnere le fiamme. Rotolarsi per terra (1 Azione) da' al personaggio bonus +2 al Tiro Salvezza. Tuffarsi in un lago o smorzare le fiamme con mezzi magici spegne automaticamente le fiamme. Creazione DC 18

\textbf{Gesso per Calchi:} 5 ma 2.5 kg Questa polvere bianca e secca, mischiata con l’acqua, si addensa nel giro di un’ora per creare un materiale solido. Puo' essere utilizzato per creare un calco di un’orma o di un bassorilievo, riempire buchi o crepe nei muri o (se applicato ad una copertura di stoffa) per fermare un osso rotto. Il gesso indurito ha Durezza 1 e 5 Punti Ferita ogni 2.5 centimetri di spessore. Un vaso di 2 kg di gesso puo' coprire un raggio di mischia per la profondita' di 2.5 centimetri, creare cinque ingessature per l’avambraccio o il polpaccio di una creatura di taglia Media o due ingessature complete per braccio o gamba. Monodose. Creazione DC 18

\textbf{Ghiaccio Liquido} (fiala) 40 mo 1 kg Detto anche "ghiaccio dell'alchimista", questo fluido blu cristallino inizia ad evaporare appena tolto dal contenitore. Nei successivi 1d6 round e' possibile utilizzarlo per congelare un liquido o coprire un oggetto con un sottile strato di ghiaccio. e' possibile anche lanciare il ghiaccio liquido come arma a spargimento. Un colpo diretto infligge 1d6 danni da freddo, mentre le creature entro raggio di mischia subiscono 1 danno da freddo per lo spargimento. La confezione contiene 3 dosi. Creazione DC 18

\textbf{Grasso Alchemico} 5 mo 0.5 kg Ogni vaso di questa sostanza nerastra puo' coprire una creatura Media o due Piccole. Coprendosi di grasso alchemico si ottiene Bonus +5 alle prove di Criminalita' e per sfuggire alle prese. L'effetto dura 4 ore o finché si lava via il grasso. Creazione DC 18

\textbf{Individua Luce} 1 mo - Questa piastra di metallo grande quanto una mano e' coperta da una crema trasparente sensibile alla luce. Se esposta alla luce, la crema si scurisce e diviene opaca a seconda di quanta luce sia presente. La luce intensa la fa scurire in 1 round, quella normale in 3 round, quella fioca in 10 round. Viene spesso utilizzata da creature dotate di Visione Crepuscolare per capire se sono passate di recente creature che per vedere utilizzano la luce. La piastra viene venduta avvolta in un panno pesante per evitare esposizioni accidentali. Creazione DC 18

\textbf{Pietra del Tuono} 30 mo 0.5 kg Si puo' scagliare questa pietra con un attacco a distanza con gittata 6 metri. Quando colpisce una superficie dura (o e' colpita con Potenza), crea un rumore assordante che equivale a un attacco sonoro. Le creature presenti entro una distanza di 3 metri devono effettuare un Tiro Salvezza su Tempra con DC 15 o restano Assordate per 1 ora. Le creature Assordate, oltre alle ovvie conseguenze, subiscono penalita' -4 all'Iniziativa e una probabilita' del 20\% di sbagliare a lanciare e perdere qualsiasi Essenza con una componente verbale che cercano di lanciare. Monouso. Creazione DC 18

Dal momento che non e' necessario colpire uno specifico bersaglio, si puo' mirare su un determinato area di mischia. Si consideri la zona come se avesse Difesa 5. Creazione DC 18

\textbf{Polvere Lampo} 50 mo - Questa polvere grigia brucia ed esplode quasi istantaneamente se esposta al fuoco, frizionandola o lanciandola con Potenza contro una superficie (1 Azione). Le creature entro raggio 3 metri sono Accecate per 1 round (Tempra DC 13 nega). La confezione contiene 3 dosi. Creazione DC 18

\textbf{Polvere per Starnuti} (borsa) 60 mo 1 kg Questa polvere giallo-rossa e' un'arma a spargimento che causa starnuti incontrollabili per 1d4+1 round. Chiunque si trovi nella zona di mischia dell'impatto deve superare un Tiro Salvezza su Tempra con DC 12 per resistere alla polvere, mentre per chi si trova nella zona di 3 metri adiacente la DC e' 8. Le creature che lo falliscono devono superare un Tiro Salvezza con DC 10 in ogni round di effetto o sono affaticate fino al loro turno successivo. La confezione contiene 3 dosi. Creazione DC 18

\textbf{Proteggilama} 40 mo - Questa resina trasparente protegge un'arma dagli attacchi di Melme, Rugginofagi ed effetti che corrodono o sciolgono le armi, rendendola immune a tali attacchi per 24 ore. Un vasetto puo' coprire un'arma a due mani, due armi ad una mano o leggere o 50 munizioni. Applicarla richiede 2 Azioni. La confezione contiene 3 dosi. Creazione DC 18

\textbf{Sali} 25 mo - Questi cristalli grigi dall'odore pungente fanno riprendere conoscenza a chi li inala. I sali concedono un nuovo Tiro Salvezza per resistere ad Essenze o effetti che rendono Privi di Sensi.
Un contenitore di sali puo' essere usato una dozzina di volte se tappato dopo ogni utilizzo, ma si dissolve in poche ore se lasciato aperto. Creazione DC 18

\textbf{Solvente Universale} (fiala) 20 mo 0.25 kg Questa gelatina viola ribollente divora gli adesivi. Ogni fiala puo' coprire un raggio di mischia. Distrugge i normali adesivi (come la pece, la resina o la colla) in 1 round, ma richiede 1d4+1 round per dissolvere adesivi piu' potenti (borse dell'impedimento, ragnatele, ecc.). Non ha effetti sugli adesivi magici. Creazione DC 18

\textbf{Tizzone Ardente} 1 mo - La sostanza alchemica sulla punta di questo piccolo bastone di legno si infiamma quando viene sfregata contro una superficie ruvida. Creare una fiamma con un tizzone ardente e' molto piu' rapido che crearla con acciarino, pietra focaia (o lente d'ingrandimento) e esca. Accendere una torcia con un tizzone ardente costa 2 Azione (invece che 3 Azioni) e per accendere qualsiasi altro fuoco occorre almeno 2 Azioni. Creazione DC 18

\textbf{Verga del Sole} 2 mo 0.5 kg Questa verga di ferro lunga 30 cm e con la punta dorata risplende vivacemente quando viene percossa (2 Azione). Illumina con luce normale un'area di 3 metri di raggio. Una verga del sole brilla per 6 ore dopodiché la punta dorata si consuma e diventa inutile. Creazione DC 18

\pagebreak

\subsection{Armi Alchemiche}\index{Armi Alchemiche}

\label{armi-alchemiche}

Le armi alchemiche sono ideate per ferire gli altri, sebbene possano avere anche altri utilizzi. Ciascuna di queste sostanze puo' essere prodotta superando una prova di abilita' di Lavoro (alchimia).

\textbf{Fiala di Polvere di Diamante} (1 dose) 25 mo - Ciascuna di queste fiale e' riempita di cristalli minerali finemente macinati. Quando si infrange una fiala con un proprio pugno, il bersaglio colpito deve superare un Tiro Salvezza su Riflessi con DC 20 per proteggersi gli occhi o resta Accecato per 1 round. Creazione DC 18

\textbf{Tirapugni Spargi Polveri} 50 mo 0.25 kg Questo guanto di cuoio senza dita include quattro piccole borsette lungo le nocche in cui si possono inserire minuscole fiale di vetro. Si possono riempire le fiale di veleno o minerali macinati. Quando si sferra un pugno a qualcuno, le fiale si infrangono, rilasciando il loro contenuto sulla faccia e sugli occhi del bersaglio. Insieme, le quattro fiale contengono una dose di veleno o minerali macinati; non hanno alcun effetto a meno che tutte e quattro non siano piene. Creazione DC 18

\textbf{Unguento dell'Arma Sacra} 30 mo 0.25 kg Quest' unguento violetto e' conservato in un piccolo vasetto di ceramica. Quando applicato su un'arma (2 Azioni), forma un rivestimento trasparente. Le armi ricoperte da questo unguento infliggono 2d4 danni addizionali ai Non Morti e agli Esterni malvagi. Una creatura influenzata dal balsamo deve superare un Tiro Salvezza su Riflessi con DC 10 o subisce 1d4 danni addizionali il round successivo. Qualsiasi arma non magica ricoperta con questo unguento influenza i Non Morti o gli Esterni malvagi come se fosse un'arma magica. Qualsiasi arma magica ricoperta con questo unguento influenza i Non Morti o gli Esterni malvagi come se l'arma avesse la capacita' speciale Tocco Fantasma. L'unguento rimane attivo finché non si mette a segno un attacco con l'arma o passa 1 minuto, quale dei due eventi si verifichi prima. Ciascuna dose di unguento puo' ricoprire un'arma o 10 munizioni. Creazione DC 18

\subsection{Attrezzature Alchemiche}

\label{attrezzature-alchemiche}

Le attrezzature alchemiche sono oggetti da avventurieri che possono rivelarsi estremamente utili in varie situazioni, compresa la battaglia, l'esplorazione di dungeon o la fabbricazione di altri oggetti alchemici. Queste attrezzature possono essere realizzate da chi possiede l'abilita' Artigianato (alchimia).

\textbf{Capsula del Vomito} 12 mo - Queste piccole capsule sono fatte da un mix concentrato di erbe che provocano la nausea. Per usare una capsula, la si morde e se ne ingeriscono i contenuti, che causano quasi immediatamente il vomito. L'attacco di vomito dura per tutto il round durante il quale non si possono compiere altre azioni. I round seguenti si recupera pienamente, e non si soffrono altri effetti negativi. Queste capsule vengono molto spesso usate dai Ladri che lavorano in squadra per creare diversivi e distrazioni in modo da attirare o sviare l'attenzione della gente dalle loro attivita', cosi' come da coloro che sono interessati a fingersi malati, come i pugili che truccano gli incontri o i criminali che cercano di seminare il caos durante un arresto. Creazione DC 18

\textbf{Carta Reagente} 1 mo - Questo pezzo di carta puo' aiutare a identificare i liquidi. Il suo colore cambia a seconda di tratti come acidita', salinita' e magia. Consumare un foglio conferisce Bonus +2 alle prove di Lavoro (alchimia) o Arcano per identificare Pozioni o altri liquidi. Creazione DC 18

\textbf{Corda di Liana di Sangue}

\textbf{Flagranza Mascherante} (Animale) 25 mo - Quest'oggetto e' disponibile in una varieta' di fragranze (che corrispondono a qualsiasi singolo Animale, Umanoide o Bestia Magica). Una fiala applicata su una creatura Media ne cambia l'odore rendendolo uguale a quello della creatura della fragranza mascherante per 8 ore. Creazione DC 18

\textbf{Flagranza Mascherante} (Umanoide) 50 mo - Creazione DC 21

\textbf{Flagranza Mascherante} (Bestia Magica) 100 mo - Creazione DC 24

\textbf{Inchiostro Luce di Fuoco} (fiala) 40 mo - Questo inchiostro infuso alchemicamente aiuta ad assicurarsi che un messaggio segreto venga distrutto dopo essere stato letto. Se la luce colpisce l'inchiostro dopo che quest'ultimo si e' asciugato, le sostanze chimiche lo fanno bruciare spontaneamente nel giro di 1 minuto. Questa combustione e' di piccole dimensioni: non e' abbastanza significativa da dar fuoco ad altro che alla carta. L'inchiostro usato su altri materiali come pietra o legno semplicemente svanisce, non lasciando alcuna traccia della scrittura. Una fiala di questo inchiostro ne contiene abbastanza da scrivere 10 brevi messaggi di non piu' di 50 parole ciascuno. Creazione DC 18

\textbf{Liquido dell'Aderenza} 20 mo 0.5 kg Questa bottiglia di vetro e' piena di una sostanza appiccicosa apprezzata dai marinai per l'aderenza che fornisce sui ponti delle navi. Quando applicato sulle suole delle calzature e fatto asciugare per 1 ora, il liquido dell'aderenza fornisce Bonus +2 alle prove di Acrobatica per mantenere l'equilibrio. Il liquido dell'aderenza non ha alcun effetto quando viene in contatto con superfici scivolose o molto scivolose come ghiaccio o Unto. Creazione DC 18

\textbf{Olio dei Maestri} 50 mo 0.25 kg Quest'olio dorato profuma di truciolato di legno. Quando lo si applica sulle corde di uno strumento a corda o sul corpo di uno strumento di legno, ne migliora la qualita' del suono. Per 1 ora, chiunque suoni lo strumento ottiene Bonus +2 alla prova di Intrattenere appropriata. Creazione DC 18

\textbf{Pastiglia dell'Usignolo} 50 mo - Questa caramella ricoperta di miele e' fatta di reagenti calmanti. Se mangiata, ha bisogno di 1 round per iniziare ad avere effetto, dopodiché conferisce Bonus +2 alle prove di Intrattenere (canto) per 1 ora. Creazione DC 18

\textbf{Pietre di Via} 50 mo 0.5 kg Questi piccoli sassolini bianchi sono trattati alchemicamente in modo che emanino una luce soffusa quando attivati sfregandoli gli uni contro gli altri. La luminescenza e' fioca, appena sufficiente a illuminare la pietra. Sebbene non siano abbastanza luminose da fungere da effettiva fonte di illuminazione, possono essere disposte secondo degli schemi in modo da creare messaggi o disposte su un sentiero, segnalandolo in modo che altri possano seguirlo. Creazione DC 18

\textbf{Polvere Tracciante} 30 mo - Quando sparsa per terra, questa sottilissima polvere blu chiaro rivela le tracce di qualsiasi creatura o individuo che sia passato nell'area nelle ultime 48 ore. La polvere fornisce anche Bonus +10 alle prove di Sopravvivenza per seguire tracce o, se non si ha addestramento in Sopravvivenza, permette invece di seguire le tracce delle creature le cui impronte sono state rivelate fino a 1,5 chilometri di distanza usando Consapevolezza anziché Sopravvivenza. Una singola applicazione puo' coprire un'area di 3 metri. La polvere tracciante viene venduta in piccole borse di cuoio che contengono 10 applicazioni ciascuna. Creazione DC 18

\textbf{Tabacco del Battipista} 200 mo - Quando inalato, questo tabacco finemente macinato e trattato alchemicamente potenzia significativamente i propri sensi, specialmente l'olfatto. Fornisce la capacita' Fiuto e Bonus +2 alle prove di Consapevolezza per 1 ora. Una volta che l'effetto svanisce, il proprio corpo e' scosso da terribili dolori mentre le proprie articolazioni si irrigidiscono e si bloccano, e si subiscono 1d2 danni a Agilita'. Creazione DC 18

\textbf{Tonico Rauco} 50 mo - Questo tonico e' fangoso, e il suo odore assomiglia al sentore di trucioli di ferro. Bere un tonico rauco rende la voce piu' profonda e roca per 1 ora, fornendo Bonus +5 alle prove di Intimidire. Creazione DC 18

\subsection{Rimedi Alchemici}\index{Rimedi Alchemici}

\label{rimedi-alchemici}

I rimedi alchemici sono sostanze usate per superare condizioni avverse o proteggersi da tipi specifici di attacchi. La maggior parte dei rimedi si utilizza per ingestione o applicandoli sulla propria pelle o sui vestiti. Queste sostanze possono essere realizzate da chi possiede l'abilita' Lavoro (alchimia).

\textbf{Aiuto Gassato} 25 mo - Questo pacchetto e' pieno di foglie dai bordi spinosi e ha un odore pungente quasi abbastanza forte da far lacrimare gli occhi. Mentre si masticano le foglie, si ignorano gli effetti della fatica. Le foglie durano per 10 round, dopodiché ne rimane solo un mucchietto di poltiglia. Quando l'effetto dell'aiuto dell'iracondo si esaurisce, si diventa invece Esausti. Un pacchetto basta per 1 sola volta. Creazione DC 18

\textbf{Balsamo Anti-veleno} 15 mo - Questo balsamo alle erbe puo' essere applicato direttamente sulla pelle per prevenire gli effetti dei Veleni a contatto. Se una creatura tocca un veleno a contatto, ma applica su di sé il balsamo entro 1 round dal contatto, effettua il Tiro Salvezza due volte e tiene il risultato migliore. Monouso. Creazione DC 18

\textbf{Balsamo Coagulante} 30 mo - Applicare questo balsamo alle erbe su una ferita sanguinante cura 1 danno e impedisce ulteriori danni da Sanguinamento per 1 ora per applicazione. Dopo un'ora, se l'effetto di Sanguinamento non e' stato appropriatamente trattato, la ferita riprende a sanguinare e deve essere applicato altro balsamo. Benché il balsamo coagulante possa essere applicato successivamente alla stessa ferita, applicarne piu' dosi non guarisce danni addizionali. La confezione e' per 3 usi. Creazione DC 18

\textbf{Intruglio Fortificante} 20 mo 0.5 kg Questo liquido genera una piacevole sensazione di calore quando ingerito. Per l'ora successiva, si ottiene Bonus Morale +2 ai Tiri Salvezza contro Paura. Usare piu' dosi nell'arco delle stesse 24 ore rende Nauseati per 1 ora. La confezione e' per 3 usi. Creazione DC 18

\textbf{Tabacco Antiemetico}: 50 mo - Questo tabacco da fiuto puo' essere usato per liberarsi dagli effetti della Nausea. Se lo si assume prima di entrare in contatto con un effetto che renderebbe Nauseati e che permetterebbe un Tiro Salvezza, si effettuano due Tiro Salvezza contro quell'effetto e si tiene il risultato migliore. Una singola dose fornisce questo beneficio per 1 ora. La confezione e’ per 3 usi. Creazione DC 18

\pagebreak

\subsection{Attrezzi per professioni ed artigiani}\index{Attrezzi}\index{professioni}\index{artigiani}

\label{attrezzi-per-professioni-ed-artigiani}

Questo Equipaggiamento e' particolarmente utile se si possiedono certe competenze ed abilita'.

\textbf{Abaco} 2 mo 1 kg Questo oggetto aiuta nei calcoli matematici.

\textbf{Arnesi da Artigiano} 5 mo 2.5 kg Questo e' un set di arnesi speciali necessari per qualsiasi lavoro artigianale. Senza questi arnesi bisogna usare attrezzi improvvisati (penalita' -2 alle prove di Artigianato) se si e' costretti a realizzare comunque il lavoro.

\textbf{Arnesi da Artigiano Perfetti} 55 mo 2.5 kg Come gli arnesi da artigiano questi sono gli arnesi perfetti per il lavoro, quindi forniscono Bonus +2 alle prove di Artigianato effettuate usandoli.

\textbf{Arnesi da Scasso} 30 mo 0.5 kg Il set include grimaldelli e altri attrezzi da impiegare quando si usa Disattivare Congegni. Senza questi arnesi si devono utilizzare attrezzi improvvisati e si subisce penalita' di circostanza -2 alle prove di Disattivare Congegni.

\textbf{Arnesi da Scasso Perfetti} 100 mo 1 kg Questo set contiene arnesi di fattura migliore che conferiscono Bonus +2 alle prove di Disattivare Congegni (Criminalita')

\textbf{Asta da Equilibrista} 8 ma 6 kg Queste aste flessibili sono lunghe da 4,5 a 9 metri e, se usate in modo appropriato, aiutano a restare in equilibrio quando si attraversa una superficie stretta. Utilizzare un'asta da equilibrista concede Bonus +2 alle prove di Acrobatica per attraversare una superficie stretta.

\textbf{Astrolabio} 100 mo 3 kg Questo oggetto e' un disco piatto su cui sono montati altri due dischi. I dischi possono ruotare su un'asse centrale, che permette loro di muoversi con il passare dei giorni. Il disco piatto rappresenta la latitudine di chi lo utilizza, il disco superiore il cielo, pieno di indicazioni astronomiche. Chiunque puo' imparare ad utilizzare l'astrolabio per conoscere data ed ora durante la notte (in 1 minuto). Un astrolabio concede Bonus +2 alle prove di Conoscenze (geografia) e Sopravvivenza per muoversi nelle zone selvagge (e alle prove di Professione (marinaio) effettuate in navigazione).

\textbf{Attrezzi da Alchimista} 25 mo 2.5 kg Un Alchimista con gli attrezzi da alchimista ha tutte le componenti materiali necessarie a creare i suoi estratti, i Mutageni e le Bombe, eccetto per quelle componenti materiali dal costo specifico. Gli attrezzi da alchimista non concedono bonus alle prove di Artigianato (alchimia).

\textbf{Attrezzi da Armaiolo} 15 mo 1 kg Questo piccolo kit contiene tutti gli strumenti di cui una persona ha bisogno per creare, riparare e ripristinare le Armi da Fuoco, fatta eccezione per le materie prime necessarie. In mancanza di tale kit, non si puo' correttamente costruire o manutenzionare le Armi da Fuoco.

\textbf{Attrezzi da Cartografo} 10 mo 1 kg Al suo interno si trovano una piccola lavagnetta con una griglia incisa sopra e diversi gessi colorati. Utilizzandoli per disegnare una mappa in viaggio si ottiene Bonus +2 alle prove di Sopravvivenza per evitare di perdersi.

\textbf{Attrezzi da Scalatore} 80 mo 2.5 kg Questi chiodi, corde e ramponi conferiscono Bonus +2 alle prove di Scalare.

\textbf{Attrezzo Perfetto} 50 mo 0.5 kg Questo oggetto di ottima fattura e' l'attrezzo ideale per il lavoro richiesto e aggiunge Bonus +2 alla relativa Prova di Competenza (se necessaria). I bonus forniti da molteplici oggetti perfetti utilizzati per la stessa Prova di Competenza non si sommano.

\textbf{Bilancia da Mercante} 2 mo 0.5 kg Una bilancia da mercante conferisce Bonus +2 alle prove di Valutare per gli oggetti la cui stima avviene in base al peso, compresa qualsiasi cosa fatta di metallo prezioso.

\textbf{Borsa del Guaritore} 50 mo 0.5 kg Questa borsa piena di erbe, pomate e bende conferisce Bonus +2 alle prove di curare. Viene consumata dopo dieci utilizzi.

\textbf{Bussola} 10 mo 0.25 kg Una normale bussola che punta al nord concede Bonus +2 alle prove di Sopravvivenza per evitare di perdersi. Puo' essere utilizzata sottoterra allo stesso scopo con le prove di Conoscenze (dungeon).

\textbf{Calderone} 1 mo 2.5 kg Questo pentolone di metallo ha un uncino per appenderlo sul fuoco. Quelli da viaggio hanno tre o quattro piedi che li tengono sollevati. Puo' contenere circa 3,5 litri e puo' essere utilizzato per cucinare, creare Pozioni e cosi' via.

\textbf{Carrucola} 2 mo 5 kg Questa semplice puleggia, quando fissata, aggiunge Bonus +5 alle prove di Potenza per sollevare oggetti pesanti. assicurare una puleggia richiede 1 minuto.

\textbf{Finti Sintomi} 25 mo 2.5 kg Questa piccola scatola di legno ha diversi piccoli scompartimenti che ospitano oggetti utili per fingere una malattia, oltre ad un manuale che descrive i sintomi delle malattie piu' gravi. La scatola include false pustole, pillole che creano la schiuma alla bocca e misture di erbe che causano febbre e vomito. Utilizzare i finti sintomi concede Bonus +5 alle prove di Camuffare per fingersi malati. Vengono consumati dopo 10 utilizzi.

\textbf{Incudine} 5 mo 5-50 kg Anche se la taglia delle incudini varia a seconda della fucina dove viene usata, tutte hanno la stessa forma e costruzione. Le incudini da fabbro sono di solito piu' grandi e pesanti (45 kg) delle incudini da maniscalco (4,5 kg). Senza un'incudine, la maggior parte dei lavori di metallurgia e' impossibile.

\textbf{Laboratorio da Alchimista} 200 mo 20 kg Questa e' l'attrezzatura perfetta per creare oggetti alchemici e conferisce Bonus +2 a qualsiasi prova di Artigianato (alchimia), ma non ha peso sui costi legati alla competenza Artigianato (alchimia). Senza questo laboratorio, un personaggio con la competenza Artigianato (alchimia) ha comunque abbastanza attrezzi per utilizzare l'abilita', ma non per avere il bonus +2 fornito dal laboratorio

\textbf{Laboratorio da Alchimista Portatile} 75 mo 10 kg Questa versione compatta di un laboratorio da alchimista concede Bonus +1 alle prove di Artigianato (alchimia).

\textbf{Lente d'Ingrandimento} 100 mo --- Questa semplice lente consente di osservare oggetti piccoli. e' utile come sostituto di acciarino, pietra focaia ed esca quando si accendono fuochi. Per appiccare un fuoco con una lente d'ingrandimento ci vuole una fonte forte di luce, come la luce del sole diretta per focalizzare, esca da infiammare e tutto il round. Conferisce Bonus +2 alle prove di Valutare qualsiasi oggetto che sia piccolo o molto dettagliato, come una pietra preziosa 

\textbf{Libro delle Impronte} 50 mo 1,5 kg Questo libro di 50 pagine contiene disegni accurati di tutte le impronte di animali, umanoidi e mostri, oltre che informazioni sulla lunghezza del passo, la profondita' dell'impronta e altre informazioni simili. Il libro concede Bonus +2 alle prove per identificare una creatura dalle sue tracce, anche se l'uso di scarpe rende difficile o impossibile identificare gli umanoidi. Anche se il libro non permette di riconoscere gli individui, permette di distinguere un'impronta di Troll da quella di un Ogre, o quella di un Orso da quella di un Orsogufo. Libri venduti in regioni diverse possono contenere impronte diverse, a seconda delle creature piu' comuni nella zona.

\textbf{Libro per Ritratti} 10 mo 1,5 kg Questo libro di 100 pagine contiene disegni di tutte le razze presenti. Scegliendo il disegno appropriato ed aggiungendo capelli, barba ed altre caratteristiche come nei e cicatrici e' possibile, anche per un pessimo disegnatore, ricostruire l'aspetto di una persona.

\textbf{Mantice} 1 mo 1,5 kg I mantici sono utili per accendere un fuoco, e concedono Bonus +1 a simili prove di Sopravvivenza.

\textbf{Mazzo da Cartomante Comune} 1 mo 0.25 kg Questo mazzo di carte illustrate e' utilizzato da chi e' in sintonia con il mondo degli spiriti e predice il futuro o dai ciarlatani che truffano le persone ingenue o disperate. Un mazzo comune ha semplici disegni su pergamene o semplici tavolette di legno. Un mazzo da cartomante di qualita' e' di legno con immagini raffinate; puo' fare da focus per l'Essenza Rivelazione e concede bonus +1 alle prove di Professione (cartomante), Professione (medium) e altre prove simili. Un mazzo da cartomante perfetto puo' essere di legno, avorio o anche metallo, con immagini dipinte o incise e spesso abbellite da intarsi d'oro e gemme incastonate; hatutti i benefici di un mazzo di qualita', ma concede Bonus +2 alle prove sopra menzionate.

\textbf{Mazzo da Cartomante di Qualita'} 25 mo 0.5 kg

\textbf{Mazzo da Cartomante Perfetto} 50 m 0.5 kg

\textbf{Sega} 4 mr 1 kg e' possibile inserire una sega fra una porta ed il suo telaio per tagliare barre o chiavistelli di legno, infliggendo 5 danni piu' il bonus di Potenza, ed impiegando tutto il round. Per sentire una sega che viene usata e' necessaria una prova di Consapevolezza con DC 10. Le seghe utilizzate per tagliare il ghiaccio sui fiumi hanno una punta per spaccarlo prima di segare.

\textbf{Sestante} 500 mo 1 kg Un sestante serve a misurare la latitudine. Concede Bonus +4 alle prove di Sopravvivenza per orientarsi in superficie.

\textbf{Strumenti per Forgiare Armi da Fuoco} 15 mo 1 kg Questa piccola serie di strumenti contiene tutto il necessario per creare, riparare e rimettere in funzione le armi da fuoco, tranne le materie prime necessarie. Senza, non e' possibile costruire o provvedere adeguatamente alla manutenzione di armi da fuoco.

\textbf{Strumento Musicale Comune} 5 mo 1,5 kg Uno strumento perfetto conferisce Bonus +2 alle prove di Intrattenere in cui viene utilizzato

\textbf{Strumento Musicale Perfetto} 100 mo 1,5 kg

\textbf{Trappola per Orsi} 2 mo 5 kg Anche se sono create per intrappolare grandi animali, queste trappole funzionano bene anche su umanoidi o mostri. Le fauci taglienti di queste trappole sono agganciate ad una catena, di solito assicurata al suolo cosi' che la vittima non possa trascinarsi via. Aprire le fauci della trappola o staccarla da suolo richiede una prova di Potenza con DC 20.

\textbf{TAGLIOLA CR 1}

Tipo meccanico; Consapevolezza DC 15; Disattivare Congegni (Criminalita') DC 20

Funzionamento

Attivatore posizione; Ripristino manuale

Effetti Tiro per Colpire +10 mischia , danni 2d6+3; fauci si chiudono attorno alla caviglia della creatura e dimezzano la velocita' base della creatura (o tengono immobile la creatura se la trappola e' legata ad un oggetto solido); la creatura puo' fuggire con una prova di Criminalita' con DC 22 o una prova di Potenza con DC 26.

\textbf{Trapano} 5 ma 0.5 kg Un trapano puo' creare un buco di 2.5 centimetri di diametro nella roccia, nel legno e nel metallo (2 Azioni). Il materiale piu' resistente usura o rompe il trapano piu' in fretta. Sentire il rumore di un trapano richiede una prova di Consapevolezza con DC 15.

\textbf{Trucchi per il Camuffamento} 50 mo 4 kg Questa e' l'attrezzatura perfetta per camuffarsi e conferisce Bonus +2 alle prove di Camuffare. Viene consumata dopo dieci utilizzi.

\textbf{Vaso di sanguisughe} 5 mo 2.5 kg Questo resistente vaso di ceramica ha un coperchio forato che permette il passaggio dell'aria. Di norma e' pieno a meta' di acqua e contiene quattro sanguisughe adulte, lunghe circa 9 centimetri. Un vaso di sanguisughe concede Bonus +2 alle prove di Guarire per trattare i Veleni. Utilizzate per i salassi medici, le sanguisughe sopravvivono per sei mesi fra un pasto e l'altro.

\pagebreak

\subsection{Cavalcature e Relativo Equipaggiamento}\index{Cavalcature}

\label{cavalcature-e-relativo-equipaggiamento}

Queste sono le cavalcature comuni che si possono trovare nelle citta'. Alcune citta' potrebbero avere delle cavalcature in piu', come cammelli o perfino grifoni, in base alla zona in cui si trovano. Queste scelte addizionali sono a discrezione del Narratore.

\subsubsection{Accessori e Varie}

\label{accessori-e-varie}

\textbf{Bardatura per Creatura Media} 30 mo La bardatura e' semplicemente un tipo di armatura che copre la testa, il collo, l'addome, il corpo e possibilmente le zampe di un cavallo o di un'altra cavalcatura. Piu' pesante e' la bardatura, migliore e' la protezione e minore la velocita'. Le bardature sono realizzate con ogni tipo di Armatura.

Come per qualsiasi creatura non umanoide di taglia Grande, un'armatura per un Cavallo costa quattro volte il costo di quella di un umano (cioe' di una creatura umanoide di taglia Media) e pesa anche il doppio. Se la bardatura e' per un Pony, o per un'altra creatura di taglia Media, il costo e' solo il doppio e il peso e' lo stesso di un'armatura Media indossata da un umanoide. Le bardature medie o pesanti rallentano le cavalcature come mostrato nella tabella sotto.

Le cavalcature volanti non possono Volare con bardature medie o pesanti.

Per mettere e togliere la bardatura occorre cinque volte il tempo indicato per una normale armatura. Gli animali bardati non possono essere usati per trasportare carichi che non siano il cavaliere e le normali sacche da sella.

Un cavallo bardato perde il 30\% della sua velocita'.

\textbf{Finimenti per Animali} 2 mo 1 kg Queste imbracature in pelle o canapa permettono di bloccare e controllare gli animali domestici. Finimenti preconfezionati per gli animali addomesticati piu' comuni, come cani, gatti, cavalli e buoi si trovano in tutti i mercati, ma possono essere creati per qualsiasi animale.

\textbf{Gabbia, Piccolissima o Minuta} 10 mo 1 kg Queste gabbie portatili e sicure servono a contenere creature, in genere animali, ma quelle piu' grandi possono contenere di tutto. Le gabbie sono fatte di ferro, legno o bambu', a seconda del luogo e del mercante che le vende. 

\textbf{Gabbia, Minuscola} 2 mo 2.5 kg

\textbf{Gabbia, Piccola} o Media 15 mo 30 kg

\textbf{Gabbia, Grande} 30 mo 120 kg

\textbf{Gabbia, Enorme} 60 mo 480 kg

\textbf{Morso e Briglie} 2 mo 0.5 kg Una briglia e' parte dell'attrezzatura usata per guidare una cavalcatura. La briglia include la testiera e il morso, che va collocato nella bocca del cavallo. A quest'ultimo sono attaccate le redini.

\textbf{Nutrimento} (al giorno) 5 mr 5 kg Cavalli, asini, muli e pony possono pascolare per nutrirsi, ma e' molto meglio procurare loro il cibo. Se si possiede un cane da galoppo, bisogna nutrirlo almeno con un po' di carne.

\textbf{Sacche da Sella} 4 mo 4 kg Queste robuste borse a tenuta stagna sono appese ad una sella per incrementare la capacita' di trasporto. Ogni lato delle sacche da sella puo' in genere trasportare 10 kg di oggetti che possono essere contenuti in una borsa. Tali sacche non aumentano l'ammontare di peso che una cavalcatura puo' trasportare, offrono semplicemente un luogo dove stivare dell'attrezzatura.

\textbf{Sella da Carico} 5 mo 7,5 kg Una sella da carico porta equipaggiamento e provviste, non un cavaliere. Una sella da carico tiene tanto equipaggiamento quanto la cavalcatura puo' trasportare.

\textbf{Sella da Galoppo} 10 mo 12.5 kg Se si viene colpiti e si perdono i sensi mentre si e' su una sella da galoppo, si ha una probabilita' del 50\% di rimanere in sella.

\textbf{Sella Militare} 20 mo 15 kg Una sella militare cinge il cavaliere aggiungendo Bonus +2 alle prove di Cavalcare per rimanere in sella. Se si viene colpiti e si perdono i sensi mentre si e' su una sella militare, si ha una probabilita' del 75\% di rimanere in sella.

\textbf{Slitta per Cani} 20 mo 150 kg Questa slitta e' lunga un paio di metri ed e' creata per essere trascinata sulla neve da una muta di cani. La maggior parte delle slitte ha una piattaforma sul fondo su cui si appoggia il cocchiere. Una slitta per cani ha una capacita' di trasporto pari a quella sommata di tutti i cani che la tirano.

\textbf{Stallaggio} (al giorno) 5 ma 

\textbf{Asino o Mulo} 8 mo --- L'asino e il mulo sono imperturbabili di fronte al pericolo, coraggiosi, dal piede fermo e capaci di trasportare carichi pesanti per grandi distanze. Diversamente dai cavalli, sono disposti (ma non sono impazienti) ad entrare nei dungeon o in altri posti strani o minacciosi.

\textbf{Cane da Galoppo} 150 mo --- Questo cane di taglia Media e' addestrato in modo particolare per trasportare un cavaliere umanoide Piccolo. e' coraggioso in combattimento come un cavallo da guerra. Data la statura, non si subiscono danni quando si cade da un cane da galoppo.

\textbf{Cane da Guardia} 25 mo --- Questo cane di taglia Piccola e' stato addestrato alla battaglia. Ha una buona Potenza, un corpo spesso e un basso centro di massa. I cani da guardia sono venduti presso molte grandi citta', ed in alcune culture sono utilizzati come combattenti per sport o impiegati in speciali unita' di fanteria.

\textbf{Cavallo Leggero} 75 mo --- Un cavallo e' adatto come cavalcatura per Umani. Un pony e' piu' piccolo di un cavallo standard ed e' una cavalcatura adatta per umani piccoli. I cavalli da guerra e i pony da guerra possono essere cavalcati facilmente incombattimento.
Vedi Addestrare Animali per una lista di comandi che cavalli e pony possono conoscere se addestrati per il combattimento.

\textbf{Cavallo leggero Addestrato al Combattimento} 110 mo

\textbf{Cavallo Pesante} 200 mo

\textbf{Cavallo Pesante Addestrato al Combattimento} 300 mo

\textbf{Pony} 30 mo

\textbf{Pony Addestrato al Combattimento} 45 mo

\pagebreak

\subsection{Vestiario}\index{Vestiario}

\label{vestiario}

Si presuppone che un personaggio inizi il gioco con un abito del valore di 10 mo o meno. Abiti addizionali possono essere comprati normalmente.

\textbf{Abito da Artigiano} 1 mo 2 kg Una camicia con bottoni, una gonna o pantaloni con i lacci, scarpe e forse un cappello o un berretto. Quest' abito puo' includere anche una cintura o un grembiule di pelle o di stoffa per tenere gli attrezzi.

\textbf{Abito da Contadino} 1 ma 1 kg Un'ampia camicia e calzoni sformati di stoffa oppure un'ampia camicia e una gonna o sopravveste. Fasce di stoffa usate come scarpe.

\textbf{Abito da Cortigiano} 30 mo 3 kg Eleganti abiti di sartoria in qualsiasi moda o qualunque sia lo stile diffuso nelle corti dei nobili. Chiunque tenti di influenzare nobili o cortigiani, mentre indossa abiti da strada, incontrera' notevoli difficolta' (penalita' -2 alle prove basate sul Carisma per esercitare influenza su queste persone). Senza gioielli (che costano circa 50 mo aggiuntive) si ha l'apparenza di una persona comune fuori posto.

\textbf{Abito da Esploratore} 10 mo 4 kg Questo e' un corredo completo di abiti per qualcuno che non sa mai cosa lo aspetta. Comprende stivali robusti, calzoni o gonna di pelle, una cintura, una camicia (magari con un panciotto o una giubba), guanti e un mantello. Piuttosto che una gonna di pelle, si puo' indossare una sopravveste di pelle sopra la gonna di stoffa. Gli abiti hanno parecchie tasche (soprattutto il mantello). Il corredo include anche qualsiasi accessorio che possa essere utile, come una sciarpa o un cappello a tesa larga.

\textbf{Abito da Intrattenitore} 3 mo 2 kg Un corredo di abiti vistosi e forse anche appariscenti per fare spettacolo. Anche se gli abiti sembrano stravaganti, il loro taglio decisamente pratico permette di compiere acrobazie, ballare, camminare sulla corda o anche solo Correre (se il pubblico diventa minaccioso).

\textbf{Abito da Monaco} 5 mo 1 kg Questi semplici abiti comprendono sandali, calzoni larghi e una camicia ampia, tenuti insieme da fasce. Questi abiti sono ideati per dare massima mobilita' e sono fatti con stoffa di alta qualita'. Si possono nascondere piccole armi nelle tasche celate nelle pieghe e le fasce sono abbastanza resistenti da servire come corde corte.

\textbf{Abito da Nobile} 75 mo 5 kg Questo corredo di abiti e' disegnato specificamente per essere costoso e per essere esibito. Metalli e pietre preziose sono lavorati nella stoffa. Per inserirsi in un ambiente nobiliare, ogni aspirante nobile ha bisogno anche di un anello con sigillo e di gioielli (del valore di almeno 100 mo).

\textbf{Abito da Studioso} 5 mo 3 kg Un abito lungo, una cintura, un cappello, scarpe morbide e possibilmente un mantello, sono adatti perfettamente per chi studia

\textbf{Abito da Viaggiatore} 1 mo 2.5 kg Stivali, una gonna o pantaloni di lana, una robusta cintura, una camicia (magari con un panciotto o una giubba) e un ampio mantello con cappuccio

\textbf{Abito Invernale} 8 mo 3,5 kg Un soprabito di lana, camicia di lino, cappello di lana, mantello pesante, pantaloni o gonna pesanti e stivali. Quando si indossano abiti invernali, si aggiunge Bonus +5 ai Tiri Salvezza su Tempra contro l'esposizione al freddo.

\textbf{Abito Regale} 200 mo 7,5 kg Questi sono solo gli abiti, non lo scettro, la corona, l'anello e altri oggetti regali. Gli abiti regali sono ostentati, con pietre preziose, oro, seta e pelliccia in abbondanza.

\textbf{Pelliccia} 12 mo 2.5 kg La forma piu' basilare di difesa dal freddo, le pellicce tengono caldo chi le indossa. Coprirsi con una pelliccia concede bonus +2 ai Tiri Salvezza su Tempra per resistere agli ambienti freddi e ai loro effetti. Non si somma ai bonus ottenuti dall'abilita' Sopravvivenza.

\textbf{Racchette da Neve} 5 mo 2 kg Reti di corda o tendini in tensione all'interno di cornici di legno permettono di distribuire meglio il peso sulla neve, permettendo di camminarvi con maggiore facilita'. Riducono le penalita' dovute a camminare sulla neve, muoversi sulla neve costa 1 movimento e con le racchette costa 0 azioni movimento di penalita'

\textbf{Ramponi} 5 mo 1 kg Utili sui terreni dove e' difficile avere trazione, i ramponi sono punte o uncini che si aggiungono alla suola della scarpa. Riducono le penalita' dovute al camminare su una superficie liscia, camminare sul ghiaccio e' terreno difficile, ma con i ramponi no. I ramponi causano danni alle superfici delicate.

\textbf{Vesti per Ambienti Caldi} 8 mo 2 kg Vestirsi con questi abiti leggeri e traspiranti tiene molto piu' fresco di quanto non accada restando nudi. Di solito comprendono una veste ampia di lino ed un turbante o velo. Questi vestiti concedono bonus +2 ai Tiri Salvezza su Tempra per resistere al caldo ed ai suoi effetti. 

Questi oggetti pesano un quarto del valore se vengono fatti per personaggi
di taglia Piccola, ma costano la stessa cifra.

\pagebreak

\subsection{Vitto e Alloggio}\index{Alloggio}\index{Vitto}

\label{vitto-e-alloggio}

Questi prezzi sono per vitto e alloggio nei locali commerciali di una citta' di media grandezza.

\textbf{Banchetto} (a persona) 10 mo --- Grande pranzo con molti invitati.

\textbf{Birra Boccale} 4 mr 0.5 kg Bevanda alcolica ottenuta dalla fermentazione del malto, dell'orzo o di altri cereali, con aggiunta aromatizzante di luppolo

\textbf{Birra Caraffa} 2 ma 4 kg Bevanda alcolica ottenuta dalla fermentazione del malto, dell'orzo o di altri cereali, con aggiunta aromatizzante di luppolo

\textbf{Carne} (1 pezzo) 3 ma 0.25 kg Alimento costituito dalla parte commestibile degli animali macellati.

\textbf{Formaggio} (1 pezzo) 1 ma 0.25 kg Prodotto che si ricava dal latte per coagulazione

\textbf{Locanda Buona} (al giorno) 2 mo Un alloggio scadente in una locanda consta di un posto sul pavimento vicino al camino. Un alloggio normale e' un posto su un pavimento sollevato e riscaldato, con una coperta e un cuscino. Un buon alloggio e' una piccola stanza privata con un letto, qualche comodita' e un vaso da notte coperto in un angolo.

\textbf{Locanda Normale} (al giorno) 5 ma

\textbf{Locanda Scadente} (al giorno) 2 ma 

\textbf{Pane} (a pagnotta) 2 mr 0.25 kg

\textbf{Pasti Buono} (al giorno) 5 ma --- Un pasto scadente puo' essere composto da pane, rape cotte, cipolle e acqua. Un pasto normale puo' comprendere pane, stufato di pollo, carote e birra o vino annacquati. Un buon pasto puo' essere composto da pane e dolci, manzo, piselli e birra o vino.

\textbf{Pasti Normale} (al giorno) 3 ma ---

\textbf{Pasti Scadente} (al giorno) 1 ma ---

\textbf{Vino Comune} (caraffa) 2 ma 1 lt Bevanda alcolica ottenuta dal mosto d'uva fatto fermentare.

\textbf{Vino Buono} (bottiglia) 10 mo 1 lt Bevanda alcolica ottenuta dal mosto d'uva fatto fermentare.

\pagebreak

\subsection{Trasporti}\index{Trasporti}

\label{trasporti}

I prezzi indicati sono per comprare il veicolo, escluso ciurme o animali.

\textbf{Barca a remi} 50 mo 50 kg Una barca lunga tra i 2,4 e i 3,6 metri, a due remi, per due o tre persone di taglia Media. Si muove alla velocita' di 2,25 km/h.

\textbf{Barcone} 3.000 mo --- Una barca lunga tra i 15 e i 22.5 metri e larga tra i 4,5 e i 6 metri. Dotata di pochi remi per integrare il suo unico albero con vela quadrata, ha un equipaggio variabile dalle 8 alle 15 unita'. Puo' trasportare dalle 40 alle 50 tonnellate di carico oppure 100 soldati. Puo' sia compiere traversate per mare che viaggiare lungo il corso dei fiumi (ha la chiglia piatta). Viaggia a una velocita' di 1,5 km/h.

\textbf{Carretto} 15 mo 100 kg Un veicolo a due ruote trainato da un solo cavallo (o altro animale da soma). Comprende anche i finimenti.

\textbf{Carro} 35 mo 200 kg Questo e' un veicolo aperto a quattro ruote per trasportare carichi pesanti. In genere, lo tirano due cavalli (o altre bestie da soma). Comprende anche i finimenti necessari per tirarlo.

\textbf{Carrozza} 100 mo 300 kg Questo veicolo a quattro ruote puo' trasportare fino a quattro persone in una cabina chiusa piu' i due conducenti. In genere, lo tirano due cavalli (o altre bestie da soma). Comprende anche i finimenti necessari per tirarlo.

\textbf{Galea} 30.000 mo --- Una nave a tre alberi con 70 remi su ciascun lato e un equipaggio totale di 200 unita'. Ha una lunghezza di 39 metri e una larghezza di 6 metri. Puo' trasportare fino a 150 tonnellate di carico o 250 soldati. Con l'aggiunta di 8.000 mo puo' essere dotata di sperone e castelli con piattaforme di lancio a prua, a poppa e a meta' scafo. Questa nave non puo' affrontare viaggi in mare aperto e si mantiene vicina alla costa. Viaggia alla velocita' di circa 6 km/h, se si impiegano i remi o le vele.

\textbf{Nave a vela} 10.000 mo --- Questa nave ha una lunghezza compresa tra i 22.5 e i 27 metri e una larghezza di 6 metri, un equipaggio di 20 unita' e capacita' di trasporto fino a 150 tonnellate. Dotata di due alberi con vele quadrate, puo' compiere traversate per mare. Raggiunge una velocita' di circa 3 km/h. 

\textbf{Nave da guerra} 25.000 mo --- Una nave della lunghezza di 30 metri, ad albero unico, e possibilita' di usare i remi. Ha un equipaggio variabile dai 60 agli 80 rematori. Puo' trasportare fino 160 soldati, ma non per lunghe distanze, poiché non vi e' lo spazio sufficiente a stivare le provviste che sarebbero necessarie a un tal numero di soldati. Non puo' compiere traversate per mare e resta vicina alla costa. Non viene usata per il trasporto merci. Viaggia alla velocita' di 3,75 km/h se si usano i remi o la vela.

\textbf{Nave Lunga} 10.000 mo --- Una nave della lunghezza di 22.5 metri, con 50 remi e un equipaggio totale di 50 unita'. Ha un unico albero con una vela quadrata. Puo' trasportare fino a 50 tonnellate di carico oppure 120 soldati. Puo' compiere traversate in mare aperto. Viaggia alla velocita' di 4,5 km/h se si impiegano i remi o la vela.

\textbf{Remo} 2 mo 5 kg \textbf{Slitta} 20 mo 150 kg Si tratta di un carro su pattini adatto per muoversi sulla neve e sul ghiaccio. Di solito, la tirano due cavalli (o altre bestie da soma). Comprende anche dei finimenti necessari per trascinarla.

\pagebreak

\subsection{Magie e Servizi}\index{Magie e Servizi}\index{Servizi}

\label{magie-e-servizi}

Talvolta la migliore soluzione a un problema e' affidarsi a qualcun altro che lo possa risolvere.

\textbf{Diligenza Pubblica} 3 mr per 1,5 Km Il prezzo indicato vale per una corsa su una diligenza che trasporta persone (e bagagli leggeri) tra due citta'. Su una diligenza che trasporti persone entro la medesima citta', una corsa costa 1 mr, e permette solitamente di arrivare ovunque si voglia.

\textbf{Livello di Potere} \texttimes 50 mo (per lP) Questo e' il costo per avere un incantatore che manipola le Essenze. Questo costo presuppone che si possa andare dall'incantatore e chiedergli di manipolare una certa Essenza a proprio piacimento (solitamente gli servono almeno 8 ore per prepararsi). Se si vuole portare da qualche parte l'incantatore per fargli usare l'Essenza e' necessario negoziare con lui, e la risposta di base e' ``no''.

Se l'Essenza ha conseguenze pericolose, l'incantatore deve ricevere delle prove certe che il personaggio ha la possibilita' di pagare e che non manchera' di farlo nel caso queste conseguenze si verifichino (sempre che accetti di manipolare l'Essenza richiesta, cosa nient'affatto sicura). Quando si tratta di Essenze che trasportano il personaggio e l'incantatore lungo una distanza, e' necessario pagare l'Essenza due volte anche se il personaggio non desidera tornare indietro con l'incantatore.

Non tutti i villaggi e i paesi hanno un incantatore abbastanza capace a manipolare essenze. Come regola generale, e' necessario spostarsi almeno in un piccolo paese per essere abbastanza sicuri di trovare un incantatore in grado di lanciare il Livello di Potere richiesto. In un piccolo paese si potrebbe trovare un incantatore in grado di lanciare Essenze di Punti Potere 11, in un grande paese per quelli di Punti Potere 13, in una piccola citta' per quelli di Punti Potere 15, in una grande citta' per quelli di livello potere 15-24, in una metropoli per quelli di livello potere 25+. Nemmeno in una metropoli si e' certi di trovare un incantatore capaci di lanciare magie con Punti Potere 34+. 

\textbf{Mercenario Esperto} 3 ma al giorno Il prezzo indicato e' la paga giornaliera di artigiani, carrettieri, muratori, scrivani, soldati di ventura e altri aiutanti abili in un mestiere. Il valore rappresenta la paga minima, perché alcuni aiutanti addestrati chiedono molto di piu'.

\textbf{Mercenario Normale} 1 ma al giorno Il prezzo indicato e' la paga giornaliera di camerieri, cuochi, facchini, operai e altri semplici lavoratori.

\textbf{Messaggero} 2 mr per 1,5 Km Questo termine include sia i messaggeri a cavallo che quelli a piedi. Se accettano di consegnare un messaggio perché il destinatario si trova in un luogo dove erano comunque diretti, potrebbero chiedere meta' della somma indicata.

\textbf{Pedaggio Stradale o d'Ingresso} 5 mr Puo' capitare di dover pagare una tassa per transitare su una strada molto frequentata, ben sorvegliata e tenuta in buone condizioni, per il pattugliamento e la manutenzione. Occasionalmente le grandi citta' fortificate richiedono il pagamento di un pedaggio all'ingresso o all'uscita della citta' stessa (a volte solo all'ingresso).

\textbf{Passaggio in Nave} 1 ma per 1,5 Km La maggioranza delle navi non e' attrezzata per il trasporto di passeggeri, ma molte hanno la capienza per imbarcarne alcuni a bordo mentre trasportano le merci. Raddoppiare il costo indicato per creature di taglia superiore a quella Media o che sono difficili da stivare nella nave.

\subsection{Oggetti da Intrattenimento}\index{Intrattenimento}

\label{oggetti-da-intrattenimento}

\textbf{Carte Segnate} 1 mo 0.5 kg Che siano piegate, colorate o graffiate, le carte segnate permettono a chi ne fa uso di riconoscere la carta a seconda dei segni fatti sul suo retro. Accorgersi di carte segnate richiede una prova di Consapevolezza con DC 25.

\textbf{Dadi Truccati Normali} 10 mo --- La maggior parte dei dadi truccati e' appesantita da una sostanza piu' pesante situata all'opposto del numero che si desidera. e' possibile accorgersi di un dado truccato con una prova di Consapevolezza o Valutare con DC 15. I dadi di qualita' superiore (ad esempio, dadi di legno intagliati attorno ad una occlusione piu' pesante) hanno DC maggiori che possono andare da 20 a 30.

\pagebreak

\subsection{Materiali Speciali}\index{Materiali Speciali}

Le armature e le armi si possono costruire con materiali che possiedono delle innate qualita' speciali. Se si costruisce un'armatura o arma con piu' di un materiale speciale, si ricevono i benefici solo del materiale prevalente. Si puo' pero' costruire un'arma doppia con ogni testa fatta di un materiale speciale diverso.

\subsubsection{Acciaio Forgiato a Caldo}\index{Acciaio Forgiato a Caldo}

\label{acciaio-forgiato-a-caldo}

\begin{tabular}[c]{@{}ll@{}}
\toprule 
Tipo di oggetto in Acciaio Forgiato a Caldo & Modificatore al costo\tabularnewline
Munizione & +15 mo per munizione\tabularnewline
Arma & +600 mo\tabularnewline
Armatura leggera & +1000 mo\tabularnewline
Armatura media & +2500 mo\tabularnewline
Armatura pesante & +3000 mo\tabularnewline
\bottomrule
\end{tabular}

I grandi fabbri si sono imbattuti nel segreto della lavorazione dell'acciaio
forgiato a caldo, nel tentativo di creare strumenti facilmente utilizzabili
in fucina. Non ci volle molto tempo per adattare le sue proprieta'
uniche ad armi e armature. L'acciaio forgiato a caldo incanala il
calore in una sola direzione per proteggere chi lo indossa o chi lo
impugna.

\subsubsection{Acciaio Forgiato a Freddo}\index{Acciaio Forgiato a Freddo}

\label{acciaio-forgiato-a-freddo}

\begin{tabular}[c]{@{}ll@{}}
\toprule 
Tipo di oggetto in Acciaio Forgiato a Freddo & Modificatore al costo\tabularnewline
Munizione & +15 mo per munizione\tabularnewline
Arma & +600 mo\tabularnewline
Armatura leggera & +1000 mo\tabularnewline
Armatura media & +2500 mo\tabularnewline
Armatura pesante & +3000 mo\tabularnewline
\bottomrule
\end{tabular}

\subsubsection{Acciaio Vivente}\index{Acciaio Vivente}

\label{acciaio-vivente}

\begin{tabular}[c]{@{}ll@{}}
\toprule 
Tipo di oggetto in Acciaio Vivente & Modificatore al costo\tabularnewline
Munizione & +10 mo per munizione\tabularnewline
Arma & +600 mo\tabularnewline
Armatura leggera & +600 mo\tabularnewline
Armatura media & +1000 mo\tabularnewline
Armatura pesante & +1500 mo\tabularnewline
Scudo & +100 mo\tabularnewline
Altri oggetti & 500 mo/kg\tabularnewline
\bottomrule
\end{tabular}

\subsubsection{Adamantio}\index{Adamantio}

\label{adamantio}

\begin{tabular}[c]{@{}ll@{}}
\toprule 
Tipo di oggetto in Adamantio & Modificatore al costo\tabularnewline
Munizione & +60 mo per munizione\tabularnewline
Arma & +1500 mo\tabularnewline
Armatura leggera & +5000 mo\tabularnewline
Armatura media & +10000 mo\tabularnewline
Armatura pesante & +15000 mo\tabularnewline
Scudo & +1000 mo\tabularnewline
Altri oggetti & 5000 mo/kg\tabularnewline
\bottomrule
\end{tabular}

Questo metallo durissimo si trova solo nei meteoriti e contribuisce alla qualita' di un'arma o di un'armatura. Le armi in adamantio sono estremamente resistenti ed hanno Durezza 25

L'adamantio e' tanto costoso che le armi e le armature fatte in questo materiale sono sempre perfette; il costo della qualita' perfetta e' incluso nei prezzi.

Quindi le armi e le munizioni in adamantio hanno Bonus di +1 ai Tiri per Colpire, e la penalita' date dall'armatura (agilita' e CM) viene diminuita di 1 rispetto ad una normale armatura del suo stesso tipo. Gli oggetti senza parti metalliche non possono essere costruiti con l'adamantio. Una freccia puo' essere in adamantio, ma un bastone ferrato
no.

\subsubsection{Argento Alchemico}\index{Argento Alchemico}

\label{argento-alchemico}

\begin{tabular}[c]{@{}ll@{}}
\toprule 
Tipo di Oggetto in Argento Alchemico & Modificatore al costo\tabularnewline
Munizione & +2 mo per munizione\tabularnewline
Arma leggera & +20 mo\tabularnewline
Arma media & +90 mo\tabularnewline
Arma pesante & +180 mo\tabularnewline
Scudo & +100 mo\tabularnewline
\bottomrule
\end{tabular}

Il processo di argentatura alchemica non puo' essere applicato alle armi non metalliche, e non funziona sui metalli speciali come l'adamantio, il ferro freddo e il mithral.

\subsubsection{Ferro Freddo}\index{Ferro Freddo}

\label{ferro-freddo}

Questo ferro viene estratto nelle profondita' del sottosuolo ed e' noto per la sua efficacia contro demoni e folletti. Viene forgiato ad una temperatura inferiore per conservare le sue delicate proprieta'. Costruire armi fatte di ferro freddo costa il doppio rispetto alle loro normali controparti. Inoltre qualsiasi potenziamento magico costa 2.000 mo addizionali. Questo aumento viene applicato la prima volta che l'oggetto viene potenziato, non una volta per qualita' aggiunta. 

Gli oggetti senza parti di metallo non possono essere costruiti in ferro freddo. Una freccia potrebbe essere fatta di ferro freddo ma un randello no. Un'arma doppia che e' fatta solo per meta' di ferro freddo aumenta il suo costo del 50\%.

\subsubsection{Mithral}\index{Mithral}

\label{mithral}

\begin{tabular}[c]{@{}ll@{}}
\toprule 
Tipo di Oggetto in Mithral & Modificatore al costo\tabularnewline
Armatura leggera & +1000 mo\tabularnewline
Armatura media & +4000 mo\tabularnewline
Armatura pesante & +9000 mo\tabularnewline
Scudo & +1000 mo\tabularnewline
Altri oggetti & +1000 mo/kg\tabularnewline
\bottomrule
\end{tabular}

\bigskip

Il mithral e' un metallo molto raro, luccicante, simile all'argento, piu' leggero del ferro ma altrettanto duro. Quando viene lavorato come l'acciaio, diventa un meraviglioso materiale con cui creare armature, e occasionalmente viene usato anche per altri oggetti. La maggior parte delle armature in mithral e' piu' leggera di una categoria del normale, ed e' piu' agevole per il movimento e le altre limitazioni. Le armature pesanti sono trattate come armature medie, e le armature medie sono trattate come leggere, ma le armature leggere restano leggere.

Questa diminuzione non si applica alla competenza necessaria per indossare l'armatura in questione (per indossare un'armatura pesante di mithral occorre avere CA 3, anche se questa viene considerata come media per altri fattori). Occorre essere competenti nel tipo di armatura appropriato, altrimenti si incorre nelle relative penalita' come di norma.

Le probabilita' di fallimento di una Essenza per armature e scudi in mithral diminuiscono di 5 punti e la penalita' ad agilita' diminuiscono di 3 (fino a un minimo di 0), le penalita' al movimento diminuiscono di 1 metro.

Le armi o le armature fatte di mithral vanno trattate come oggetti perfetti; il costo della qualita' perfetta e' gia' compreso nei prezzi riportati sopra.

\subsubsection{Pelle di Drago}\index{Pelle di Drago}

\label{pelle-di-drago}

I fabbricanti di armature possono lavorare le pelli dei draghi per produrre armature o scudi di qualita' perfetta.
Un drago fornisce pelle sufficiente per una singola armatura di pelle perfetta per una creatura di una taglia piu' piccola del drago. Selezionando solo le scaglie e le parti di pelle migliori, un fabbricante di armature puo' produrre una corazza di bande perfetta per una creatura di due taglie piu' piccola, una mezza armatura per una creatura tre taglie piu' piccola e una corazza di piastre perfetta o un'armatura completa per una creatura di quattro taglie piu' piccola. In ogni caso, c'e' sempre pelle sufficiente per produrre uno scudo leggero o pesante perfetto in aggiunta all'armatura, purché il drago sia Grande o maggiore. 
e la pelle di drago proviene da un Drago che ha immunita' ad un tipo di energia, anche l'armatura e' immune a quel tipo di energia, sebbene non conferisca alcuna protezione a chi la indossa. Se allo scudo o all'armatura viene conferita in seguito la capacita' di proteggere chi la indossa da un tipo di energia specifico, il costo di questo potenziamento viene ridotto del 25\%.

Le armature di pelle di drago costano il doppio di un'armatura perfetta di quel tipo, ma non richiedono piu' tempo per essere costruite (si raddoppino tutti i risultati di Artigianato).

\pagebreak

\section{Sfondare ed Entrare}\index{Sfondare}\index{Entrare}

\label{sfondare-ed-entrare}

Quando si tenta di spaccare un oggetto le scelte sono due: colpirlo con un'arma o romperlo con la forza bruta.

\textbf{Colpire un Oggetto}

\begin{tabular}[c]{@{}ll@{}}
\toprule 
Tabella: Taglia e Difesa degli Oggetti Taglia & Modificatore Difesa\tabularnewline
Colossale & -8\tabularnewline
Mastodontica & -6\tabularnewline
Enorme & -4\tabularnewline
Grande & -2\tabularnewline
Media & +0\tabularnewline
Piccola & +2\tabularnewline
Minuscola & +4\tabularnewline
Minuta & +6\tabularnewline
Piccolissima & +8\tabularnewline
\bottomrule
\end{tabular}

\bigskip

\textbf{Modificatore Difesa}: Gli oggetti sono piu' facili da colpire delle creature poiché di solito non si muovono, ma molti sono abbastanza resistenti da scrollarsi di dosso qualche danno ad ogni colpo. La Difesa di un oggetto e' pari a 10 + il suo modificatore di Taglia (vedi Tabella: Colpire un Oggetto) + il suo modificatore di Agilita' (caso mai ne avesse uno). Se si usano 3 Azioni per prendere la mira, si colpisce automaticamente con un'arma da mischia e si ottiene bonus +1d6 al colpire con un'arma a distanza.

\textbf{Durezza}

Questi conteggio non vengono applicati ad Armature e Scudi che seguono le loro tabelle di resistenza e durezza.

\bigskip

\begin{tabular}[c]{@{}lll@{}}
\toprule 
Sostanza & Durezza & \tabularnewline
Vetro & 1 & 1 ogni 2,5 cm di spessore\tabularnewline
Carta o stoffa & 0 & 2 ogni 2,5 cm di spessore\tabularnewline
Corda & 0 & 2 ogni 2,5 cm di spessore\tabularnewline
Ghiaccio & 0 & 3 ogni 2,5 cm di spessore\tabularnewline
Cuoio o pelle & 2 & 5 ogni 2,5 cm di spessore\tabularnewline
Legno & 5 & 10 ogni 2,5 cm di spessore\tabularnewline
Pietra & 15 & 15 ogni 2,5 cm di spessore\tabularnewline
Ferro o acciaio & 10 & 10 ogni 2,5 cm di spessore\tabularnewline
Mithral & 15 & 30 ogni 2,5 cm di spessore\tabularnewline
Adamantio & 20 & 40 ogni 2,5 cm di spessore\tabularnewline
\bottomrule
\end{tabular}

\bigskip

\textbf{Attacchi di Energia}: Gli attacchi di energia (fuoco, elettricita'..) infliggono meta' danno alla maggior parte degli oggetti; dividere per 2 i danni prima di applicare la Durezza. Alcuni tipi di energia possono essere particolarmente efficaci contro certi oggetti, a discrezione del Narratore. Per esempio, il fuoco potrebbe infliggere danno pieno a pergamene, stoffa e altri oggetti che bruciano facilmente. Un attacco sonoro potrebbe causare danno pieno ad oggetti di vetro e cristallo.

\textbf{Danni da Armi a Distanza}: Gli oggetti subiscono la meta' dei danni da un'arma a distanza (tranne che per le Macchine d'Assedio e simili). Dividere per 2 i danni prima di applicare la Durezza dell'oggetto.

\textbf{Armi Inefficaci}: Certe armi semplicemente non possono infliggere danni a certi oggetti. Per esempio, un'arma contundente non e' in grado di tagliare una corda. Allo stesso modo e' decisamente difficile abbattere una porta o un muro di pietra con la maggior parte delle armi da mischia, a meno che non siano specificamente ideate per farlo, come picconi e martelli.

\textbf{Immunita'}: Gli oggetti inanimati sono immuni ai Danni Non Letali e ai Colpi Critici. Anche gli oggetti animati, se non considerati come delle creature, hanno queste immunita'.

\textbf{Armi, Armature e Scudi Magici}: Ogni +1 di Bonus aggiunge anche 2 alla Durezza ad armi, armature e scudi, e +20 alla resistenza dell'oggetto.

\textbf{Vulnerabilita' a Certi Attacchi}: Certi attacchi possono essere particolarmente efficaci contro alcuni oggetti. In questi casi gli attacchi infliggono danni raddoppiati e possono ignorare la Durezza dell'oggetto.

\textbf{Oggetti Danneggiati}: Un oggetto danneggiato rimane pienamente funzionale con la condizione Rotto fino a quando i Punti Ferita non arrivano a 0, e a quel punto e' considerato distrutto. Gli oggetti danneggiati (ma non quelli distrutti) possono essere riparati con la Competenza Artigianato e alcune Essenze (vedi la condizione Rotto per maggiori dettagli).

\textbf{Tiro Salvezza}: Gli oggetti non magici incustoditi non effettuano
mai Tiro Salvezza. Si considera che abbiano fallito i loro Tiro Salvezza,
e siano quindi sempre soggetti ad Essenze ed altri attacchi che ammettono
un Tiro Salvezza per resistere o negare l'effetto. Un oggetto custodito
da un personaggio (che lo tenga in mano, lo tocchi o lo indossi) ottiene
un Tiro Salvezza proprio come se lo stesse effettuando il personaggio
(cioe' usando il suo bonus al Tiro Salvezza).

\textbf{Gli Oggetti Magici hanno sempre Tiro Salvezza}. Il bonus ai
Tiri Salvezza su Tempra, Riflessi o Volonta' di un Oggetti Magico sono
pari a 2 + meta' Livello dell'incantatore che li ha creati. Gli Oggetti
Magici custoditi effettuano i Tiri Salvezza come il loro possessore
oppure usano i loro Tiri Salvezza, quali che siano i migliori.

\bigskip

\begin{tabular}[c]{@{}llll@{}}
\toprule 
Oggetto & Durezza & & DC per Romperlo\tabularnewline
Corda (2,5 cm di diametro) & 0 & 2 & 23\tabularnewline
Porta di legno semplice & 5 & 10 & 13\tabularnewline
Cassa piccola & 5 & 1 & 17\tabularnewline
Porta di legno buona & 5 & 15 & 18\tabularnewline
Cassa del tesoro & 5 & 15 & 23\tabularnewline
Porta di legno robusta & 5 & 20 & 23\tabularnewline
Muro di pietra (spesso 30 cm) & 8 & 90 & 35\tabularnewline
Pietra tagliata (spessa 90 cm) & 8 & 540 & 50\tabularnewline
Catena & 10 & 5 & 26\tabularnewline
Manette & 10 & 10 & 26\tabularnewline
Manette perfette & 10 & 10 & 28\tabularnewline
Porta di ferro (spessa 5 cm) & 10 & 60 & 28\tabularnewline
\bottomrule
\end{tabular}

\bigskip

Oggetti animati: Gli oggetti animati contano come creature per determinarne
la Difesa (non sono considerati oggetti inanimati).

\subsection{Rompere Oggetti}\index{Rompere Oggetti}

\label{rompere-oggetti}

Quando si tenta di rompere qualcosa con Potenza improvvisa piuttosto che infliggendo danni regolari, bisogna effettuare una prova di Potenza (invece di un Tiro per Colpire e per il danno, come per Spezzare) per vedere se ci si riesce. Poiché la Durezza non influisce sulla DC per rompere l'oggetto, questo valore dipende piu' dal modo in cui e' costruito l'oggetto che non dal materiale. Vedi Tabella: DC per Rompere o forzare Oggetti per una lista delle DC piu' comuni relative al rompere gli oggetti.

Creature di Taglia superiore o inferiore a quella Media hanno bonus o penalita' di taglia sulla prova di Potenza per sfondare una porta:

\bigskip

\begin{tabular}[c]{@{}ll@{}}
\toprule 
Taglia & Modificatore per sfondare porta\tabularnewline
Piccolissima & -16\tabularnewline
Minuta & -12\tabularnewline
Minuscola & -8\tabularnewline
Piccola & -4\tabularnewline
Normale & +0\tabularnewline
Grande & +4\tabularnewline
Enorme & +8\tabularnewline
Mastodontica & +12\tabularnewline
Colossale & +16\tabularnewline
\bottomrule
\end{tabular}

\bigskip

Un piede di porco o un ariete portatile aumentano la probabilita' del personaggio di sfondare una porta.

\subsubsection{Tabella: DC per Rompere o Forzare oggetti - Prova di Potenza}

\label{tabella-dc-per-rompere-o-forzare-oggetti---prova-di-potenza}
\bigskip
\begin{tabular}[c]{@{}ll@{}}
\toprule 
Cosa abbatti & DC\tabularnewline
Abbattere una porta semplice & 13\tabularnewline
Abbattere una porta buona & 15\tabularnewline
Abbattere una porta robusta & 18\tabularnewline
Forzare corde legate & 23\tabularnewline
Piegare sbarre di ferro & 24\tabularnewline
Abbattere una porta robusta & 25\tabularnewline
Forzare catene legate & 26\tabularnewline
Abbattere una porta di ferro & 28\tabularnewline
\bottomrule
\end{tabular}
\bigskip

\pagebreak

\section{Ambiente}\index{Ambiente}

\label{ambiente}
\begin{quotebox}
La natura non e' crudele, e' solo spietatamente indifferente. Questa e' una delle piu' dure lezioni che un essere umano debba imparare. (Richard Dawkins)
\end{quotebox}

Dai deserti senza vita ai dungeon zeppi di trappole, l'ambiente aiuta a definire il mondo. Renderlo vivo, dinamico e ricco consente di creare un'esperienza di gioco emozionante e coinvolgente. Questo capitolo contiene le regole per aiutare il Narratore a definire il mondo di gioco, come dungeon, trappole, terreni e pericoli ambientali.

\subsection{Regole Ambientali}

\label{regole-ambientali}

I pericoli relativi a un tipo di terreno specifico sono descritti in Avventure nelle Terre Selvagge. I rischi ambientali comuni a piu' di un terreno sono invece descritti di seguito.

\subsubsection{Visione e Luce}\index{Visione}\index{Luce}

\label{visione-e-luce}

In un ambiente naturale l'illuminazione puo' assumere diverse gradazioni e queste gradazioni aiutano a comprendere fino a che distanza una creatura puo' vedere.

Le gradazione di luce possono essere:
\begin{itemize}
\item 
\textbf{Oscurita}': buio pesto, puo' essere naturale o magico 
\item 
\textbf{Penombra}: la poca illuminazione permette di riconoscere le
sagome 
\item 
\textbf{Luce}: una luce brillante, coprente, assolata. 
\end{itemize}
\textbf{Tabella delle fonti di luce}
\index{Luce estesa}\index{Penombra}
\begin{tabular}[c]{@{}llll@{}}
\toprule 
Oggetto & Luce Normale (raggio) & Luce estesa/Penombra & Durata\tabularnewline
Candela & 1 metro & - & 1 ora\tabularnewline
Torcia & 3 metri & 6 metri & 1 ora\tabularnewline
Lanterna & 6 metri & 12 metri & 6 ore/boccetta\tabularnewline
\bottomrule
\end{tabular}

\bigskip

In linea di massima una fonte di luce crea luce estesa in un raggio doppio rispetto alla luce normale.

Stare alla luce delle stelle con luna quasi piena e' essere in penombra (+2 Difesa)

Una creatura con \textbf{Visione Normale} \index{Visione Normale}vede fino alla distanza, come raggio circolare intorno alla fonte di luce, indicato in Luce Normale. Vedere in luce estesa e' penombra.

Una creatura con \textbf{Visione Crepuscolare} \index{Visione Crepuscolare}vede fino alla distanza, come raggio circolare intorno alla fonte di luce, indicato in Luce estesa, o indicato dalla razza se minore, oltre e' oscurita'.

\textbf{Oscurita'}\index{Oscurita'}: e' il buio piu' completo senza alcuna fonte di luce.

Per creature con visione normale l'oscurita' e' cio' che c'e' oltre la Luce Estesa.

Il \textbf{personaggio cieco}\index{cieco} ha -4 alla Consapevolezza visiva e tutti gli avversari hanno occultamento.

\textbf{Penombra}: puo' essere una notte stellata oppure luce estesa.\index{Penombra}

Per una creatura con visione normale la penombra concede agli avversari
un Copertura parziale, ovvero +2 alla Difesa.

\textbf{Luce}

Per una creatura con visione normale e' come essere al centro del raggio di illuminazione, oppure sotto il sole.

In un'area ad illuminazione diurna, sotto il sole o in piena luce, e' piu' difficile effettuare prova di criminalita'.

\subsubsection{Buio}\index{Buio}

\label{buio}

Le torce e le lanterne possono essere spente all'improvviso da una folata di vento, le fonti di luce magiche possono essere dissolte o contrastate ed alcune trappole magiche possono creare aree di buio impenetrabile.

In certi casi, alcuni personaggi o mostri potrebbero essere in grado di vedere mentre gli altri sono Accecati. Ai fini delle regole che seguono, una creatura Accecata e' semplicemente una creatura che non e' in grado di vedere nelle tenebre circostanti.

\subsubsection{Accecato}\index{Accecato}

\label{accecato}

Le creature Accecate perdono la loro capacita' di infliggere danni extra causati ad esempio dall'Abilita' di pugnalare alle spalle.

Le creature accecate devono effettuare una prova di Acrobatica con DC 10 per muoversi piu' velocemente della loro velocita' dimezzata. Se la prova fallisce cadono a terra prone. Le creature accecate non possono correre o caricare.

Tutti gli avversari di una creatura accecata godono di Occultamento nei suoi confronti ovvero hanno un +8 alla Difesa.

Una creatura accecata, o che combatte contro una creatura invisibile,\index{invisibile} puo' effettuare una prova di Consapevolezza a difficolta' 20 (oppure 15+Criminalita' dell'avversario se questo non vuole farsi trovare) per individuare la creatura purche' questa sia entro area di mischia dal personaggio.

Se la prova riesce e' possibile tentare l'attacco a -1d6 Tiro per colpire, la creatura gode sono di un +4 alla Difesa, tranne l'attacco avviene con Essenze/attacco ad area.

Una creatura Accecata \index{Accecata}subisce penalita' -4 alle prove di Consapevolezza e alla maggior parte delle prove basate su Potenza e Agilita' e fallisce automaticamente qualsiasi check di Competenze dipendenti dalla vista. 

Inoltre, una creatura accecata dal buio non puo' usare Essenze che prevedano l'uso dello sguardo ed e' immune alle Essenze che prevedono lo sguardo.

Se una creatura Accecata viene colpita da un nemico non visto, riesce a individuare la posizione attuale della creatura che lo ha colpito (finché la creatura non si muove, naturalmente). L'unica eccezione avviene se la creatura usa un attacco a distanza (nel qual caso il personaggio Accecato sa la direzione generica del nemico, ma non la sua posizione precisa).

Se viene quindi colpito in mischia si considera come se la prova di Consapevolezza per determinare l'avversario sia riuscito (-1d6 al Tiro per Colpire, +4 Difesa)

\subsubsection{Cadute}\index{Cadute}

\label{cadute}

Le creature che cadono subiscono 1d6 danno per cadute da altezze di 3 metri, piu' 1d6 ogni 3 metri oltre i 3. Dividi l'altezza in metri per 3, arrotonda per difetto, il numero che risulta sono i d6 di danno subiti. Es 16 metri di caduta sono 16/3=5d6 di danno. I danni da caduta non possono superare i 20d6 di danno (anche se la caduta e' da kilometri di altezza).

Le creature che subiscono danni letali da una caduta, atterrano in posizione prona.

Una prova di Acrobatica riuscita con DC 15 permette al personaggio di dimezzare il danno quando cade da meno di 20 metri.

Cadute su superfici morbide (terreno morbido, fango ecc.) convertono i primi 1d6 danni in Danni Non Letali. Questa riduzione e' cumulativa con la diminuzione del danno per l'uso della competenza Acrobatica.

Un personaggio non puo' utilizzare Essenze magiche mentre cade, a meno che la caduta non sia superiore a 150 metri o l'Essenza magica sia rapida. Utilizzare un'Essenza magica mentre si cade richiede una prova di Concentrazione con DC pari a 20.

\textbf{Cadere in Acqua}\index{Cadere in Acqua}

Le cadute in acqua sono gestite in modo leggermente diverso. Fino a quando l'acqua ha una profondita' di almeno di 3 metri ed il tuffo e' da una altezza entro 12 metri non si subiscono danni.

Si subiscono 2d6 di danni da una caduta oltre i 15 metri e 5d6 per cadute oltre i 15 metri.

I personaggi che si tuffano volontariamente in acqua non subiscono danni se superano una prova di Acrobatica o di Nuotare (Resistenza) con DC 15 se l'acqua e' profonda almeno 6 metri. La DC della prova aumenta di 5 ogni 5 metri oltre i 15.

\subsubsection{Effetti dell'Acido}\index{Acido}

\label{effetti-dellacido}

Gli acidi corrosivi infliggono 1d6 danni per round di esposizione, tranne nel caso di totale immersione (come in una vasca d'acido), che infligge 10d6 danni per round. Un attacco con l'acido, come quello di una boccetta lanciata o la saliva/soffio di un mostro, deve essere considerato come un round di esposizione.

I vapori prodotti dalla maggior parte degli acidi sono equivalenti a veleni inalati. Coloro che si avvicinano molto ad un grosso ammasso di acido devono effettuare un Tiro Salvezza su Tempra con DC 13 o subiranno 1 danno alla Potenza a round. Questo veleno non ha frequenza, pertanto una creatura e' salva se si allontana dall'acido.

Le creature immuni alle proprieta' caustiche dell'acido potrebbero comunque annegare se vi vengono totalmente immerse (vedi Annegamento).

\subsubsection{Effetti del Fumo}\index{Fumo}

\label{effetti-del-fumo}

Un personaggio costretto a respirare del fumo denso deve superare un Tiro Salvezza su Tempra ogni round (DC 15, +1 per ogni prova precedente) oppure passa il round a tossire e soffocare. Un personaggio che continua a soffocare per 2 round consecutivi subisce 1d6 Danni Non Letali. Il fumo oscura la vista, fornendo Copertura (+2 Difesa) ai personaggi che si trovano al suo interno.

\subsubsection{Fame e Sete}\index{Fame}\index{Sete}

\label{fame-e-sete}

I personaggi potrebbero trovarsi senz'acqua o cibo e privi dei mezzi per procurarsene. Nei climi normali, i personaggi di taglia Media hanno bisogno di almeno 2 litri di liquidi e 0.5 kg di cibo decente al giorno per evitare la fame. (I personaggi di taglia Piccola necessitano della meta'). Nei climi molto caldi, i personaggi possono aver bisogno di due o tre volte quella quantita' d'acqua per evitare la disidratazione.

\subsubsection{Oggetti Cadenti}\index{Oggetti Cadenti}

\label{oggetti-cadenti}

Proprio come i personaggi subiscono danni dalle cadute superiori a distanze di mischia, allo stesso modo subiscono danni se vengono colpiti da oggetti cadenti.

Gli oggetti che cadono addosso ai personaggi infliggono danni a seconda del loro peso e della distanza da cui sono caduti.

La \textbf{Tabella: Danno da Oggetti Cadenti} determina la quantita' di danni inflitti da un oggetto in base alla sua taglia. Si presume che l'oggetto sia fattodi un materiale denso e pesante, come la pietra. 
Gli oggetti fatti di materiali piu' leggeri potrebbero infliggere la meta' o meno del danno indicato, a discrezione del Narratore. Per esempio un masso Enorme che colpisce un personaggio infligge 6d6 danni, mentre un carro di legno potrebbe infliggerne solo 3d6.

Inoltre, se l'oggetto cade da una distanza inferiore ai 3 metri, infligge la meta' dei danni indicati. Se un oggetto cade da una distanza superiore ai 20 metri, infligge danni raddoppiati. L'oggetto che cade subisce la stessa quantita' di danni che infligge.

\bigskip

\textbf{Tabella: Danno da Oggetti Cadenti}

\begin{tabular}[c]{@{}ll@{}}
\toprule 
Taglia dell'Oggetto & Danno\tabularnewline
Minuscola o Piu' Piccola & 1d6\tabularnewline
Piccola & 2d6\tabularnewline
Media & 3d6\tabularnewline
Grande & 4d6\tabularnewline
Enorme & 6d6\tabularnewline
Mastodontica & 8d6\tabularnewline
Colossale & 10d6\tabularnewline
\bottomrule
\end{tabular}

\bigskip

Lasciar cadere addosso ad una creatura un oggetto richiede un attacco di contatto a distanza (Difesa a tocco contro attacco basato su Agilita'). Questi attacchi hanno di solito una gittata di 3 metri. Se un oggetto cade su una creatura (invece di venire lanciato), quella creatura deve effettuare un Tiro Salvezza su Riflessi con DC 15 per dimezzare il danno se e' consapevole dell'oggetto che sta cadendo. Gli oggetti cadenti che sono parte di una trappola usano le regole relative alle trappole invece che quelle qui descritte.

\subsubsection{Pericoli dell'Acqua}\index{Pericoli dell'Acqua}\index{Acqua}

\label{pericoli-dellacqua}

Qualsiasi personaggio puo' attraversare acque relativamente calme che non abbiano una profondita' maggiore alla sua altezza, senza bisogno di prove. Allo stesso modo, Nuotare (Resistenza) in acque calme richiede una prova di Resistenza con DC 10. I nuotatori addestrati possono prendere 10. Si ricordi che l'armatura o l'equipaggiamento pesante rendono qualsiasi tentativo di nuotare piu' difficile

D'altra parte, le acque rapide sono molto piu' pericolose.

Con una prova riuscita di Resistenza o una prova di Potenza con DC 15, i personaggi non rischiano di finire sott'acqua. Se falliscono, subiscono 1d3 Danni Non Letali per round (1d6 danni letali se le acque scorrono sopra rocce e avvallamenti).

L'acqua molto profonda non e' solo nera come la pece, rendendo molto pericolosa la navigazione, ma infligge danni ancora peggiori a causa della pressione dell'acqua nell'ordine di 1d6 danni al minuto ogni 30 metri che separano il personaggio dalla superficie. Un Tiro Salvezza su Tempra superato con successo (DC 15, +1 per ogni prova precedente) indica che il personaggio immerso non subisce danni in quel minuto. L'acqua molto fredda infligge 1d6 Danni Non Letali per minuto di esposizione a causa dell'ipotermia.

\textbf{Annegamento}\index{Annegamento}

Qualsiasi personaggio puo' trattenere il fiato per un numero di round pari 8 volte il suo punteggio di Potenza, con un minimo di 6 round. Se il personaggio compie almeno 2 Azioni, la durata restante per cui puo' trattenere il fiato e' ridotta di 1 round. Trascorso questo periodo di tempo, il personaggio deve effettuare una prova di Potenza con DC 10 ad ogni round per continuare a trattenere il fiato. Ogni round, la DC aumenta di 1.

Si puo' annegare in sostanze diverse dall'acqua, come la sabbia, le sabbie mobili, la polvere molto fine o un silos pieno di farro o semplicemente trattenendo il respiro.

\subsubsection{Pericoli del Caldo}\index{Caldo}

\label{pericoli-del-caldo}

Una creatura sottoposta a temperature molto elevate (sopra i 40° C) deve superare un Tiro Salvezza su Tempra ogni ora (DC 15, +1 per ogni prova precedente) oppure subisce 1d4 Danni Non Letali. Se indossa abiti pesanti o qualsiasi tipo di armatura, subisce penalita' -4 a questi Tiri Salvezza. Un personaggio con la competenza Sopravvivenza puo' ricevere un bonus a questo Tiri Salvezza ed essere in grado di applicarlo anche ad altri personaggi (vedi la descrizione dell'Abilita'). I personaggi Privi di Sensi iniziano a subire danni letali (1d4 danni all'ora).

In situazioni di caldo estremo (sopra i 40° C), un personaggio deve effettuare un Tiro Salvezza su Tempra ogni 10 minuti (DC 15, +1 per ogni prova precedente) oppure subisce 1d4 danni non letali. I personaggi che indossano abiti pesanti o qualsiasi tipo di armatura, subiscono penalita' -4 a questi Tiri Salvezza. Un personaggio con la competenza Sopravvivenza puo' ricevere un bonus a questo Tiro Salvezza ed essere in grado di applicarlo anche ad altri personaggi. I personaggi Privi di Sensi iniziano a subire danni letali (1d4 danni ogni 10 minuti).

Un personaggio che subisce Danni Non Letali a causa dell'esposizione al caldo, e' soggetto ad un colpo di calore ed e' Affaticato. Queste penalita' terminano quando il personaggio recupera dai Danni Non Letali subiti a causa del caldo.

Il caldo infernale (temperatura dell'aria sopra i 60° C, fuoco, acqua che bolle, lava) infligge danni letali. Respirare l'aria con queste temperature infligge 1d6 danni da fuoco al minuto (senza Tiro Salvezza). 
Inoltre, il personaggio deve superare un Tiro Salvezza su Tempra ogni 5 minuti (DC 15, +1 per ogni prova precedente) oppure subisce 1d4 Danni Non Letali. Coloro che indossano abiti pesanti o qualsiasi tipo di armatura, subiscono penalita' -4 a questi Tiro Salvezza.

L'acqua bollente infligge 1d6 danni da scottatura, a meno che il personaggio non vi venga completamente immerso, nel qual caso subirebbe 10d6 danni per round di esposizione.

\subsubsection{Prendere Fuoco}\index{Prendere Fuoco}\index{Fuoco}

\label{prendere-fuoco}

I personaggi esposti ad olio bollente, fuochi da campo, e fuochi magici non istantanei possono vedere i loro abiti, capelli o equipaggiamento prendere fuoco. Le Essenze magiche con durata istantanea non sono in grado di appiccare il fuoco, in quanto il calore e la fiammata appaiono e scompaiono in un attimo.

I personaggi che rischiano di prendere fuoco possono effettuare un Tiro Salvezza su Riflessi con DC 15 per evitare questo destino. Se i vestiti o i capelli di un personaggio prendono fuoco, egli subisce immediatamente 1d6 danni. Per ogni round successivo il personaggio in fiamme deve effettuare un altro Tiro Salvezza su Riflessi. Il fallimento indica che subisce altri 1d6 danni in quel round. Il successo indica che il fuoco si e' estinto (ovvero, una volta che supera il Tiro Salvezza, non sta piu' andando a fuoco).

Un personaggio che va a fuoco puo' estinguere automaticamente le fiamme saltando dentro a dell'acqua sufficiente a spegnerle. Se non ci sono grosse quantita' d'acqua a disposizione, rotolarsi sul terreno o smorzare la fiamma con mantelli o simili puo' concedere al personaggio un altro Tiro Salvezza con bonus +4.

Coloro che sono talmente sfortunati dal vedere il loro Equipaggiamento o vestiti prendere fuoco devono superare un Tiro Salvezza su Riflessi (DC 15) per ogni oggetto. Gli oggetti infiammabili che falliscono il tiro, subiscono la stessa quantita' di danni del personaggio.

\textbf{Effetti della Lava}\index{Lava}

La lava o il magma infliggono 2d6 danni per round di esposizione, tranne in caso di totale immersione (come quando un personaggio cade nel cratere di un vulcano attivo), che infligge 20d6 danni per round (piu' eventuali danni da caduta).

I danni provocati dal magma continuano per 1d3 round dopo il termine dell'esposizione, ma questi danni addizionali sono solo la meta' di quelli inflitti durante l'effettivo contatto (cioe', 1d6 o 10d6 per round). Un'Immunita' o una Resistenza al fuoco servono anche come resistenza o resistenza alla lava o al magma. Tuttavia, le creature Immuni o Resistenti al Fuoco potrebbero annegare se immerse nella lava (vedi Annegamento).


\subsubsection{Pericoli del Freddo}\index{Freddo}

\label{pericoli-del-freddo}

I personaggi non ben vestiti in climi freddi (sotto i 5° C) devono superare un Tiro Salvezza su Tempra ogni ora (DC 15, +1 per ogni prova precedente) oppure subiscono 1d6 Danni Non Letali. Un personaggio con la Competenza Sopravvivenza puo' ricevere un bonus a questo Tiro Salvezza ed essere in grado di applicarlo anche ad altri personaggi.

In condizioni di freddo estremo o di esposizione sotto i -17° C, un personaggio scoperto deve effettuare un Tiro Salvezza su Tempra ogni 10 minuti (DC 15, +1 per ogni prova precedente), subendo 1d6 Danni Letali per ogni Tiro Salvezza fallito. Un personaggio con la Competenza Sopravvivenza puo' ricevere un bonus a questo Tiro Salvezza ed essere in grado di applicarlo anche ad altri personaggi. I personaggi che indossano abiti invernali hanno bisogno di effettuare la prova per il freddo e l'esposizione solo una volta all'ora.

Un personaggio che subisce Danni Non Letali a causa del freddo o dell'esposizione, e' soggetto ai geloni o all'ipotermia (considerarlo come Affaticato). Queste penalita' terminano quando il personaggio recupera dai Danni Non Letali subiti a causa del freddo e dell'esposizione. 

Le condizioni di freddo intollerabile o di esposizione (sotto i -28° C) infliggono ai personaggi 1d6 danni letali per minuto (senza alcun Tiro Salvezza) se non specificatamente protetti.

\subsubsection{Effetti del Ghiaccio}\index{Ghiaccio}

I personaggi che camminano sul ghiaccio e' come se fossero su terreno difficile. Il movimento e' dimezzato, eventuali prove di Acrobatica hanno un aumento di difficolta' +5. I personaggi che sono per lungo tempo a contatto con il ghiaccio potrebbero subire dei danni da freddo estremo.

\subsubsection{Soffocamento Lento}\index{Soffocamento}

Un personaggio di taglia Media puo' respirare tranquillamente per circa 6 ore in una camera sigillata che misura 3 metri di lato. Dopo questo tempo, subisce 1d6 Danni Non Letali ogni 15 minuti. Ogni ulteriore personaggio di taglia Media oppure ogni fuoco significativo (una torcia, per esempio) riducono proporzionalmente la durata dell'aria respirabile. Una volta privi di sensi per l'accumulo di Danni Non Letali, i personaggi iniziano a subire Danni Letali allo stesso ritmo. I personaggi di taglia Piccola consumano meta' dell'aria dei personaggi di taglia Media.

\pagebreak

\subsection{Tempo Atmosferico - Meteo}\index{Meteo}

\label{tempo-atmosferico---meteo}

A volte il tempo atmosferico puo' giocare un ruolo importante nel corso di un'avventura. La Tabella: Tempo Atmosferico Casuale e' una tabella generica che puo' essere utilizzata per stabilire le condizioni atmosferiche locali. I termini della tabella sono definiti qui di seguito:

\bigskip

\textbf{Tabella: Tempo Atmosferico Casuale}

{\small
\begin{tabular}[c]{@{}lllll@{}}
\toprule 
d\% & Tempo Atmosferico & Clima Freddo & Clima Temperato {*} & Deserto\tabularnewline
01-70 & Normale & Freddo, calmo & Normale per la stagione {*}{*} & Torrido,calmo\tabularnewline
71-80 & Anormale & Ondata di Caldo (01-30) \\
&& Ondata di Freddo (31-100) & Ondata di Caldo (01-50)\\
&&& Ondata di Freddo (51-100) & Torrido,ventilato\tabularnewline
81-90 & Inclemente & Precipitazioni (neve) & Precipitazioni \\
&&&(normali per la stagione) & Torrido, ventilato\tabularnewline
91-99 & Tempesta & Tempesta di neve & Tempesta di fulmini\\
&&& tempesta di neve & Tempesta di polvere\tabularnewline
100 & Tempesta violenta & Tormenta & Bufera,tormenta\\
&&&uragano,tornado & Acquazzone\tabularnewline
\bottomrule
\end{tabular}}
{*} Temperato comprende foreste, colline, paludi, montagne, pianure
e zone marine calde.

{*}{*} L'inverno e' freddo, l'estate e' calda, l'autunno e la primavera sono moderati. Le paludi sono sempre leggermente piu' calde d'inverno.

\bigskip

\textbf{Acquazzone}: Considerarlo come pioggia (vedi Precipitazioni sotto), ma offre Occultamento come la nebbia. Puo' provocare inondazioni e dura di solito 2d4 ore.

\textbf{Caldo}: La temperatura e' tra 15° e 30° C di giorno, e tra 6 e 11 gradi in meno di notte.

\textbf{Calmo}: Vento leggero (tra 0 e 15 km/h).

\textbf{Freddo}: Temperatura tra -17° e 5° C durante il giorno, e tra 6 a 11 gradi in meno di notte.

\textbf{Moderato}: Temperatura tra i 5° e i 15° C durante il giorno, e tra 6 e 11 gradi in meno di notte.

\textbf{Ondata Caldo}: Fa aumentare la temperatura di 6° C.

\textbf{Ondata Freddo}: Abbassa la temperatura di 6° C.

\textbf{Precipitazioni}: Tirare un d100 per determinare se la precipitazione e' nebbia (01-30), pioggia/neve (31-90), o nevischio/ grandine (91-00). La neve e il nevischio si verificano solo quando la temperatura e' di 0° C o inferiore. La maggior parte delle precipitazioni dura 2d4 ore. La grandine, invece, dura solo 1d20 minuti ma di solito e' accompagnata da 1d4 ore di pioggia.

\textbf{Tempesta} (di Fulmini/di Neve/di Polvere): Il vento e' molto forte (da 45 a 75 km/h) e la visibilita' e' ridotta di tre quarti. Le tempeste durano 2d4-1 ore. Vedi Tempeste, sotto, per ulteriori dettagli.

\textbf{Tempesta} (Bufera/Tormenta/Uragano/Tornado): La velocita' del vento e' superiore ai 75 km/h (vedi Tabella: Effetti del Vento). Inoltre, le tormente sono accompagnate da pesanti nevicate (1d3 \texttimes{} 30 cm), e gli uragani sono accompagnati da acquazzoni. Le bufere durano 1d6 ore, le tormente 1d3 giorni. Gli uragani possono durare fino a una settimana, ma l'impatto maggiore per i personaggi avverra' in un periodo di tempo tra le 24 e le 48 ore, mentre il centro della tempesta si sposta nella loro zona. I tornado durano molto poco (1d6 \texttimes{} 10 minuti), e di solito si formano come parte di una tempesta di fulmini.

\textbf{Torrido}: Temperatura tra i 30° e i 43° C durante il giorno e tra 6 e 11 gradi in meno di notte.

\textbf{Ventilato}: La velocita' del vento va da moderata a forte (da 15 a 45 km/h); vedi Tabella: Effetti del Vento.

\textbf{Pioggia, Neve, Nevischio e Grandine}

La brutta stagione frequentemente rallenta o blocca i trasporti via terra e rende praticamente impossibile la navigazione. acquazzoni torrenziali e bufere oscurano la visuale tanto quanto lo farebbe una nebbia densa.

La maggior parte delle precipitazioni si manifesta come pioggia, ma nei climi freddi possono manifestarsi anche come neve, nevischio o grandine. Le precipitazioni di qualsiasi tipo, seguite da un calo della temperatura da sopra a sotto gli 0° C possono produrre ghiaccio.

\textbf{Pioggia}\index{Pioggia}

La pioggia dimezza la visibilita', e impone penalita' -4 alle prove di Consapevolezza. Ha lo stesso effetto di un vento molto forte sulle fiamme, sugli attacchi con armi a distanza e sulle prove di Consapevolezza come vento molto forte.

\textbf{Neve}\index{Neve}

Mentre cade, la neve ha gli stessi effetti della pioggia su visibilita', attacchi con armi a distanza e prove di Consapevolezza ed il movimento costa 1 azione in piu'. Una nevicata della durata di un giorno lascia al suolo 1d6 \texttimes{} 2.5 centimetri di neve.

\textbf{Neve Fitta}

Una fitta nevicata ha gli stessi effetti di una nevicata normale, ma oscura la visibilita' come la nebbia (vedi Nebbia). Un giorno di neve fitta lascia sul terreno 1d4 x 30 centimetri di neve ed entrare in zona di mischia cosi' alta costa 2 azioni di movimento. Una fitta nevicata accompagnata da venti forti o molto forti puo' dare origine a cumuli di neve profondi 1d4 x 1,5 metri, specialmente sopra e intorno ad oggetti abbastanza grandi da deflettere il vento (una capanna o una grande tenda, per esempio).
C'e' una probabilita' del 10\% che una nevicata fitta sia accompagnata da fulmini (vedi Tempesta di Fulmini). La neve ha gli stessi effetti del vento moderato sulle fiamme. La neve rende il terreno difficile.

\textbf{Nevischio}

Si tratta fondamentalmente di pioggia congelata, che ha gli stessi effetti della pioggia quando cade (eccetto che la probabilita' di estinguere fiamme protette e' del 75\%) e quelli della neve una volta depositatasi.

\textbf{Grandine}

La grandine non riduce la visibilita', ma il suono della grandine che cade rende piu' difficili le prove di Consapevolezza basate sull'udito (penalita' -4). A volte (probabilita' del 5\%) la grandine puo' essere talmente grossa da infliggere 1 danno letale (per tempesta) a qualsiasi cosa si trovi all'aperto. Una volta depositata, la grandine ha lo stesso effetto della neve sul movimento.

\subsubsection{Tempeste}\index{Tempeste}

\label{tempeste}

Gli effetti combinati delle precipitazioni (o della polvere) e del vento, che accompagnano tutte le tempeste, riducono la visibilita' di tre quarti, imponendo penalita' -8 a tutte le prove di Consapevolezza. Le tempeste rendono impossibili gli attacchi con le armi a distanza, tranne che con le armi da assedio, che subiscono penalita' -4 i Tiri per Colpire.
Estinguono automaticamente le candele, le torce o simili fiamme non protette. Le fiamme protette, come quelle delle lanterne, vengono agitate violentemente e hanno una probabilita' del 50\% di estinguersi. Vedi Tabella: Effetti del Vento per le possibili conseguenze sulle creature sorprese all'esterno senza ripari. 

Le tempeste sono di tre tipi.

\textbf{Tempesta di Polvere (CR 3)}

queste tempeste desertiche si differenziano dalle altre tempeste in quanto non hanno precipitazioni. Al contrario, le tempeste di polvere trasportano granelli di sabbia che oscurano la vista, soffocano le fiamme non protette e possono addirittura spegnere quelle protette (probabilita' del 50\%). Molte tempeste di polvere sono accompagnate da venti molto forti e si lasciano alle spalle un deposito di 1d6 \texttimes{} 2.5 centimetri di sabbia.
Esiste anche una probabilita' del 10\% di incontrare grandi tempeste di polvere con bufere di vento (vedi Tabella: Effetti del Vento). Queste violente tempeste di polvere infliggono 1d3 danni non letali per round a chiunque venga sorpreso all'aperto senza riparo e pongono anche il rischio del soffocamento (vedi Annegamento, eccetto che un personaggio con una sciarpa o simile protezione sulla bocca e il naso, non inizia a soffocare se non dopo un numero di round pari 10 \texttimes{} il suo punteggio di Potenza). Le grandi tempeste di polvere si depositano alle spalle (2d3-1) x 30 centimetri di sabbia.

\textbf{Tempesta di Neve}

oltre ai venti e alle precipitazioni comuni alle altre tempeste, le tempeste di neve depositano 1d6 \texttimes{} 2.5 centimetri di neve sul terreno.

\textbf{Tempesta di Fulmini}

oltre ai venti e alle precipitazioni (di solito pioggia, ma a volte anche grandine), le tempeste di fulmini sono accompagnate da scariche elettriche che rappresentano un pericolo per i personaggi che si trovano all'aperto senza riparo (specialmente se indossano armature metalliche). Come regola generale, si puo' considerare un fulmine al minuto per un periodo di un'ora nel cuore della tempesta. Ogni fulmine infligge danni da elettricita' tra 4d8 e 10d8. Una tempesta di fulmini su dieci viene accompagnata da un tornado.

\textbf{Tempeste Violente}

Venti molto forti e precipitazioni torrenziali riducono la visibilita' a zero, e rendono impossibile effettuare prove di Consapevolezza e compiere attacchi con armi a distanza. Le fiamme non protette vengono automaticamente spente, e c'e' una probabilita' del 75\% che cio' si verifichi anche per quelle protette. Le creature sorprese in queste zone devono effettuare un Tiro Salvezza su Tempra o devono affrontare effetti a seconda della propria taglia (vedi Tabella: Effetti del Vento). Le tempeste violente sono suddivise nei seguenti quattro tipi.

\textbf{Bufera}: Sebbene abbiano poche o nessuna precipitazione, le bufere possono provocare danni ingenti a causa della Potenza del vento.

\textbf{Tormenta}: La combinazione di forti venti, neve fitta (di solito 1d3 \texttimes{} 30 cm) e freddo intenso rende le tormente letali per chiunque non vi sia preparato. 

\textbf{Uragano}: Oltre ai venti molto forti e alla pioggia intensa, gli uragani sono seguiti da inondazioni. Molte attivita' in un'avventura sono impossibili in queste condizioni.

\textbf{Tornado}: Oltre ai venti molto forti, i tornado possono ferire gravemente ed uccidere quelli che vengono catturati al suo interno.

\subsubsection{Nebbia}\index{Nebbia}

\label{nebbia}

Sia nella forma di una nube a bassa altitudine che di una foschia che sale dal terreno, la nebbia ostacola la visuale oltre la distanza di mischia. Le creature piu' lontane di mischia godono di Copertura leggera (+2 Difesa).

La nebbia rende il terreno difficile.


\subsubsection{Venti}\index{Venti}

\label{venti}

I venti possono creare turbini di sabbia o polvere, alimentare grossi incendi, rovesciare piccole imbarcazioni e disperdere gas o vapori. Se sono forti a sufficienza possono addirittura buttare a terra i personaggi (vedi Tabella: Effetti del Vento), interferire con gli attacchi a distanza, o imporre penalita' ad alcune Prove di Competenze.

\textbf{Tabella: Effetti del Vento Potenza del Vento}

\begin{tabular}[c]{@{}lll@{}}
\toprule 
Potenza del Vento & Velocita' del Vento & Attacchi a Distanza\\
Leggero & 0-15km \tabularnewline
Moderato & 16,5-30 km/h  \tabularnewline
Forte & 31.5-45 & -2 \tabularnewline
Molto forte & 45.5-75km/h & -4 \tabularnewline
Bufera & 76.5-111km/h & impossibile  \tabularnewline
Uragano & 12-261km/h & impossibile \tabularnewline
Tornado & 262-450km/h & impossibile\tabularnewline
\bottomrule
\end{tabular}

\bigskip

\textbf{Vento Leggero}

Una brezza gentile, che non ha effetti pratici sul gioco.

\textbf{Vento Moderato}

Un vento sostenuto, che ha una probabilita' del 50\% di estinguere qualsiasi piccola fiamma non protetta, come quella di una candela.

\textbf{Vento forte:} Folate che spengono automaticamente le fiamme non protette (candele, torce e simili). Queste folate impongono penalita' -2 ai tiri per colpire a distanza ed alle prove di Consapevolezza.

\textbf{Vento Molto Forte}

Oltre a spegnere automaticamente le fiamme non protette, i venti di questa intensita' agitano violentemente le fiamme protette (come quelle di una lanterna) e hanno una probabilita' del 50\% di estinguerle. Gli attacchi con le armi a distanza e le prove di Consapevolezza subiscono penalita' -4. Questa e' la stessa velocita' del vento prodotta da un Essenza di Creazione Aria a livello 5.

\textbf{Bufera}\index{Bufera}

Abbastanza forti da abbattere i rami o addirittura interi alberi, le bufere estinguono automaticamente le fiamme non protette e hanno una probabilita' del 75\% di estinguere quelle protette, come quelle delle lanterne. Gli attacchi con le armi a distanza sono impossibili, e anche le armi da assedio subiscono penalita' -4 ai Tiri per Colpire. Le prove di Consapevolezza basate sull'udito subiscono penalita' -8 per l'ululare del vento.

\textbf{Uragano}\index{Uragano}

Estingue tutte le fiamme. Gli attacchi a distanza sono impossibili (eccetto con le armi da assedio che subiscono penalita' -8 ai tiri per colpire). Anche le prove di Consapevolezza basate sull'udito sono impossibili e tutto cio' che i personaggi possono udire e' l'ululare del vento. Gli uragani spesso sono in grado di abbattere gli alberi.

\textbf{Tornado (CR 10)}\index{Tornado}

Estingue tutte le fiamme. Tutti gli attacchi a distanza sono impossibili (compresi quelli con le armi da assedio), cosi' come le prove di Consapevolezza basate sull'udito. Invece di essere portati via (vedi Tabella: Effetti del Vento), i personaggi che si trovano nelle immediate vicinanze di un tornado e che falliscono un Tiro Salvezza su Tempra vengono risucchiati dentro il tornado.

Coloro che entrano in contatto con la nube conica vengono sollevati da terra e sbatacchiati per 1d10 round, subendo 6d6 danni per round, prima di venirne espulsi violentemente (con l'applicazione dei danni da caduta).

Sebbene la velocita' rotatoria di un tornado possa raggiungere i 450 km/h, il cono stesso si muove in avanti ad una media di 45 km/h (circa 75 metri per ogni round). Un tornado e' in grado di sradicare alberi, distruggere edifici e provocare altre forme di simile devastazione.

\pagebreak

\section{Avventure in Acqua}\index{Avventure in Acqua}

\label{avventure-in-acqua}
\begin{quotebox}
Guardo' il mare e capi' fino a che punto era solo, adesso. (Il vecchio e il mare, Ernest Hemingway)
\end{quotebox}

L'acqua permette alle societa' di esistere, ma puo' anche distruggerle. La vita non potrebbe esistere senza di essa. Il commercio ed il viaggio sono agevolati dalla sua presenza. Eppure, l'acqua puo' anche uccidere, sia annegando le persone, sia generando alluvioni e tsunami su larga scala. La vita terrestre e' dipendente dall'acqua ma allo stesso tempo la teme.

\textbf{Avventure Acquatiche}

Un'avventura acquatica puo' aver luogo ovunque l'acqua rappresenti l'elemento principale del territorio: come paludi, fiumi, laghi, stagni, oceani, il Piano dell'Acqua e simili. Le avventure Acquatiche, comunque, non richiedono che i personaggi abbiano la capacita' di respirare sott'acqua; l'introduzione di sfide Acquatiche per avventurieri di basso livello apportano ad un'avventura un bel po' di tensione e sensazione di pericolo.

\textbf{Adattarsi agli Ambienti Acquatici}

Le regole per il combattimento sott'acqua si applicano alle creature che non sono native di questo pericoloso ambiente, come la maggior parte dei PG. Per avventure Acquatiche prolungate ed esplorazioni particolarmente in profondita', i personaggi necessiteranno dell'uso della magia per proseguire le proprie avventure. Essenze di Alterazione per respirare Sott'Acqua e' di ovvia utilita', mentre Essenze di Difesa per Resistere all'Energia puo' aiutare con la temperatura.

Il danno da pressione puo' essere totalmente evitato tramite effetti di Essenze di Movimento. Le Essenze di Trasformazione sono forse le piu' utili in acqua, purché la forma assunta sia di natura acquatica.

\textbf{Adattamento Naturale:} Qualsiasi creatura del Sottotipo Acquatico puo' respirare sott'acqua facilmente e non viene influenzata dalle temperature estreme riscontrabili nel proprio ambiente tipico. Le creature Acquatiche e quelle con la capacita' di trattenere il respiro sono molto piu' resistenti al danno da pressione: non subiscono questo tipo di danno a meno che non vengano spostate istantaneamente da una profondita' ad un'altra (in tal caso si adattano al cambio di pressione dopo aver superato con successo cinque Tiro Salvezza su Tempra contro gli effetti della pressione).

\textbf{Avventure Nautiche}

L'acqua puo' fornire l'ambientazione per un'esperienza di gioco diversa ed unica: l'avventura nautica. In un simile scenario, gli effetti e i pericoli delle avventure subacquee sono sostituiti dalle sfide di superficie, dal momento che i personaggi e i loro avversari utilizzano navi e barche per spostarsi in tale ambiente. Di solito, le avventure nautiche si risolvono normalmente, con un combattimento a bordo di una nave simile ad uno terrestre. Se il combattimento avviene durante una tempesta o in mari agitati, considerate il ponte della nave come terreno difficile. Ricordatevi di considerare gli effetti sulle prove di Concentrazione per il tempo atmosferico o il rollio.

\textbf{Combattimento Rapido in Mare}

Quando sono le navi a combattere, le cose cambiano un po'. Le regole seguenti non hanno lo scopo di simulare accuratamente tutti gli aspetti di un combattimento navale, ma solo fornirvi rapide e semplici regole per sbrogliare tali situazioni quando si tramutano in un'avventura nautica, che sia una battaglia tra due navi o tra una nave ed un mostro marino.

{Preparazione:} Stabilite quali tipi di navi sono coinvolte nel combattimento (vedi Tabella: Statistiche delle Navi). Utilizzate una griglia da battaglia ampia e vuota per rappresentare le acque in cui ha luogo la battaglia. Un singolo quadretto corrisponde a 9 metri di distanza. Raffigurate ogni nave piazzando dei segnalini che occupino l'appropriato numero di quadretti (le navi giocattolo sono ottimi segnalini e potete reperirle nei negozi di modellismo).{}

{Cominciare il Combattimento:} Quando il combattimento inizia, lasciate che i personaggi (ed importanti PNG alleati) tirino l'Iniziativa normalmente; la nave si muove ed attacca sulla base del risultato di iniziativa del capitano. Se una delle navi in battaglia usa le vele per spostarsi, determinate casualmente in quale direzione sta soffiando il vento tirando 1d8 e seguendo le linee guida per le Armi a Spargimento che mancano il bersaglio.{}

{Movimento:} Sulla base del punteggio di Iniziativa del capitano, la nave puo' muoversi alla propria velocita' base in un singolo round come se l'Azione corrispondesse a quella del capitano stesso (o al doppio della sua velocita' come unica azione del round), finché ha il proprio equipaggio minimo al completo. La nave puo' incrementare o diminuire la propria velocita' di 9 metri per round, fino al raggiungimento della velocita' massima. In alternativa, il capitano puo' cambiare direzione (al massimo un lato di un quadretto alla volta) (2 Azioni). Una nave puo' cambiare direzione solo all'inizio del turno.{}

{Attacchi:} I membri in eccesso rispetto al requisito minimo di equipaggio di una nave possono essere collocati a manovrare le Macchine d'Assedio. Le Macchine d'Assedio attaccano sulla base del punteggio di Iniziativa del capitano.{}

Una nave puo' anche tentare di speronare un bersaglio se ospita l'equipaggio minimo. Per speronare un bersaglio, la nave deve muoversi di almeno 9 metri e finire con la prua in un quadretto adiacente ad esso. 
Quindi, il capitano della nave effettua una prova di Professione (marinaio): se il risultato e' pari o superiore alla Difesa del bersaglio, la nave colpisce il suo obiettivo, infliggendogli danni come indicato nella Tabella: Statistiche delle Navi e allo stesso tempo subendo il danno minimo. Una nave equipaggiata con uno sperone infligge al bersaglio 3d6 danni addizionali (l'imbarcazione attaccante non subisce danno addizionale).

\textbf{Affondamento}\index{Affondamento}

Una nave ottiene la condizione in affondamento quando i suoi Punti Ferita scendono a 0 o meno. Una nave in affondamento non puo' muoversi o attaccare e dopo 10 round si considera affondata. Ogni 25 danni subiti da una nave che affonda si riduce l'affondamento di 1 round. L’Essenza di Creazione consente di riparare una nave che affonda se i Punti Ferita della stessa sono riportati sopra lo 0, caso in cui la nave perde la condizione in affondamento. In genere, le riparazioni non magiche richiedono troppo tempo per salvare una nave dall'affondamento una volta che questa inizia a sprofondare.

\textbf{Statistiche di una Nave}

Nel mondo reale esiste una grande varieta' di barche e navi, dalle piccole zattere agli imponenti galeoni. A rappresentanza di cio', la Tabella: Statistiche delle Navi classifica sette dimensioni standard di nave e le rispettive statistiche. Cosi' come le culture del mondo reale hanno creato ed adattato differenti tipi di imbarcazioni, cosi' le razze di mondi fantasy potrebbero creare le proprie bizzarre navi.
I Narratore potrebbero utilizzare o modificare queste statistiche per soddisfare le esigenze delle loro creazioni e, comunque, descrivere tali mezzo di trasporto a proprio piacimento. Tutte le navi presentano i seguenti tratti.

\textbf{Tipo}: Si tratta di una categoria generale che elenca la tipologia base di nave.

\textbf{Difesa}: La Difesa della nave. Per calcolare la Difesa effettiva di una nave, aggiungete il punteggio di Professione (marinaio) del capitano alla Difesa base della stessa. Gli attacchi di contatto contro una nave ignorano il modificatore del capitano. Una nave non e' mai Impreparata.

\textbf{TS Base}: Il modificatore dei Tiri Salvezza Base di una nave (Tempra, Riflessi e Volonta') hanno lo stesso valore. Per determinare gli effettivi modificatori dei Tiro Salvezza di una nave, aggiungete il modificatore di Professione (marinaio) del capitano a questo valore.

Velocita' Massima: La velocita' massima di una nave in combattimento. Un asterisco indica che la nave ha delle vele e puo' spostarsi a velocita' raddoppiata se si muove nella stessa direzione del vento. Una nave che abbia solo delle vele puo' spostarsi solo in presenza di vento.

\textbf{Armamenti}: Il numero di Macchine d'Assedio che possono essere equipaggiate sulla nave. Uno sperone utilizza uno di questi slot e una nave puo' essere equipaggiata soltanto con uno sperone.

\textbf{Speronamento}: L'ammontare di danni che infligge una nave con un attacco di speronamento riuscito (senza uno sperone).

\textbf{Quadretti}: Il numero di quadretti che la nave occupa sulla griglia di combattimento. Una nave si considera sempre della larghezza di un quadretto.

\textbf{Equipaggio}: Il primo numero indica l'equipaggio minimo di cui la nave ha bisogno per funzionare normalmente, ad esclusione degli addetti alle armi. Il secondo indica il numero massimo della ciurma piu' i soldati o passeggeri aggiuntivi. Una nave senza il suo equipaggio minimo puo' soltanto muoversi, cambiare velocita', cambiare direzione, o speronare se il suo capitano supera una prova di Professione (marinaio) con DC 20.
Un equipaggio che eccede il numero minimo non influenza il movimento, ma i suoi componenti possono sostituire i membri caduti o manovrare armi aggiuntive.

\bigskip

\textbf{Tabella: Statistiche delle Navi}

{\small
\begin{tabular}[c]{@{}lllllllll@{}}
\toprule 
Tipo & Difesa & PF & TS base & Vel. (m/s) & Arma & Speronamento & Quad. & Equipaggio\tabularnewline
Zattera & 9 & 10 & +0 & 4.5 & 0 & 1d6 & 1 & 1/4\tabularnewline
Barca a Remi & 9 & 20 & +2 & 9 & 0 & 2d6+6 & 1 & 1/3\tabularnewline
Battello & 8 & 60 & +4 & 9 & 1 & 2d6+6 & 2 & 4/15+100\tabularnewline
Nave Lunga & 6 & 75 & +5 & 18 & 1 & 4d6+18 & 3 & 50/75+100\tabularnewline
Barca a Vela & 6 & 125 & +6 & 18 & 2 & 3d6+12 & 3 & 20/50+120\tabularnewline
Nave da Guerra & 2 & 175 & +7 & 18 & 3 & 3d6+12 & +4 & 60/80+160\tabularnewline
Galea & 2 & 200 & +8 & 27 & +4 & 6d6+24 & 4 & 200/250+200\tabularnewline
\bottomrule
\end{tabular}}

\pagebreak

\section{Avventure in Citta'}\index{Citta'}

\label{avventure-in-citta}
\begin{quotebox}
Dio creo' la campagna, e l'uomo creo' la citta'. (William Cowper)
\end{quotebox}

A prima vista, una citta' e' molto simile a un dungeon, in quanto e' composta da pareti, porte, stanze e corridoi. Le avventure ambientate in citta' differiscono da quelle ambientate nei dungeon per due motivi principali. I personaggi hanno accesso a un maggior numero di risorse e devono tenere conto della presenza delle forze dell'ordine.

\textbf{Accesso alle Risorse}: A differenza dei dungeon e delle terre selvagge, i personaggi possono comprare e vendere Equipaggiamento molto rapidamente in citta'. Una citta' grande o una metropoli probabilmente dispongono di PNG ed esperti di alto livello specializzati nei settori piu' oscuri della conoscenza, in grado di offrire aiuto e di interpretare gli indizi. E quando i personaggi sono malconci e contusi, possono sempre fare ritorno alle comodita' delle loro camere nella locanda.

La liberta' di effettuare una ritirata e l'accesso alle merci del mercato significa che i giocatori dispongono di un maggior controllo sui ritmi di gioco di un'avventura in citta'.

\textbf{Forze dell'Ordine}: L'altro elemento di distinzione tra andare all'avventura in una citta' ed esplorare un dungeon sta nel fatto che il dungeon e', quasi per definizione, un luogo senza regole dove la sola legge e' quella dellagiungla: uccidere o essere uccisi.

Una citta', d'altro canto, e' sorretta da un codice di leggi, molte delle quali sono state ideate esplicitamente per prevenire quel genere di comportamento nel quale gli avventurieri indulgono spesso e volentieri: uccidere e saccheggiare. Tuttavia, le leggi cittadine riconoscono la gravita' della minaccia che i mostri costituiscono alla stabilita' cittadina, ed e' assai raro che la proibizione di uccidere valga anche per mostri come le aberrazioni o gli esterni malvagi.

La maggior parte degli umanoidi malvagi, tuttavia, solitamente gode della stessa protezione riservata a tutti gli altri cittadini. Appartenere a un allineamento malvagio non e' un crimine (tranne forse in quelle citta' dove vige una severa teocrazia, col potere magico necessario per far valere la legge); soltanto gli atti malvagi vengono considerati un'infrazione alla legge. 

Anche quando gli avventurieri incontrano un malfattore impegnato a commettere i crimini piu' orribili nei confronti della popolazione cittadina, la legge vede comunque di cattivo occhio chi si fa giustizia da solo uccidendo il malfattore o impedendo in altri modi che venga condotto davanti a un tribunale per essere processato.

\textbf{Limitazioni alle Armi e alle Essenze}

Ogni citta' ha le sue leggi riguardo alle armi che e' possibile portare con sé circolando in pubblico e alle limitazioni alle Essenze.

Le leggi cittadine potrebbero non influenzare tutti i personaggi in egual modo. Un uomo di fede che si muove con un'arma al seguito non viene ostacolato in alcun modo dalla legge che impone di legare con un laccio le armi, ma un incantatore subisce una riduzione considerevole del suo potere se il suo Scrigno viene confiscato alle porte della citta'.

\textbf{Elementi Urbani}

Pareti, porte, illuminazione scarsa e terreno sconnesso: sotto molti aspetti, una citta' e' simile a un dungeon. Di seguito vengono descritti nuovi elementi adatti a un'ambientazione cittadina.

\textbf{Mura e Cancelli}

Molte citta' sono difese da un cerchio di mura. Delle normali mura cittadine sono in pietra rinforzata, spesse 1,5 metri e alte 6 metri. Un muro simile e' piuttosto liscio ed e' necessaria una prova di Scalare (Resistenza) con DC 30 per potervisi arrampicare. Le mura dispongono di piccoli merli su un lato per fornire un parapetto alle guardie in cima, e lo spazio per camminare sulle mura e' a malapena sufficiente per una guardia.

\textbf{Le mura}

A differenza delle citta' piu' piccole, le metropoli spesso sono dotate anche di mura interne, a volte delle vecchie mura erette quando la citta' era piu' piccola, oppure mura che separano i vari quartieri gli uni dagli altri. A volte queste mura sono alte e larghe come quelle esterne, ma molto piu' spesso hanno le dimensioni di quelle di una citta' grande o piccola.

\textbf{Torri di Guardia}: Alcune mura cittadine sono dotate di torri di guardia che spuntano a intervalli regolari. Sono poche le citta' che hanno guardie a sufficienza da collocare su ogni torre di guardia, a meno che la citta' non si aspetti un attacco dall'esterno. Le torri offrono una visuale elevata della campagna circostante oltre a un baluardo di difesa contro gli invasori nemici.

Le torri di guardia solitamente sono piu' alte di 3 metri rispetto al muro di cui fanno parte, e il loro diametro e' pari a 5 volte lo spessore delle mura. Delle feritoie per gli arcieri si aprono ai piani alti della torre, e la cima e' merlata allo stesso modo delle mura circostanti. Nelle torri piu' piccole (del diametro di circa 7,5 metri, lungo un muro spesso 1,5 metri) una semplice scala a pioli collegal'interno della torre al tetto. Nelle torri piu' grandi si trovano vere e proprie scale.

L'accesso alla torre e' protetto da pesanti porte in legno, con rinforzi in ferro e serrature buone (Disattivare Congegni (Criminalita') DC 30). Normalmente e' il capitano delle guardie a custodire la chiave d'accesso alla torre, e una seconda copia viene conservata nella fortezza interna o nella caserma cittadina.

\textbf{Cancelli}: Un tipico cancello d'accesso alla citta' e' composto da una guardiola con due saracinesche e delle feritoie nello spazio tra di esse. Nei paesi e nelle citta' piccole, l'entrata principale e' protetta da doppie porte di ferro incastrate nelle mura cittadine.

I cancelli rimangono solitamente aperti durante il giorno e chiusi a chiave o sbarrati di notte. Generalmente, soltanto un cancello lascia entrare i viaggiatori dopo il tramonto, ed e' sorvegliato da guardie che apriranno le porte solo per qualcuno che abbia un aspetto onesto, presenti i documenti appropriati, o le corrompa con una cifra sufficiente (in base al tipo di citta' e di guardie).

\textbf{Guardie e Soldati}

Una citta' solitamente e' dotata di personale militare di servizio a tempo pieno pari all'1\% della sua popolazione adulta, in aggiunta ai soldati di turno o di leva pari al 5\% della popolazione. I soldati a tempo pieno sono guardie cittadine responsabili del mantenimento dell'ordine in citta', con un ruolo simile a quello della polizia moderna, e (in misura assai minore) della difesa della citta' dagli assalti esterni. I soldati in leva forzata vengono chiamati alle armi in caso di un attacco in citta'.

Un tipico schieramento di guardie cittadine si distribuisce in tre turni diservizio da otto ore ciascuno, col 30\% delle sue forze in servizio di giorno (dalle 8 alle 16), il 35\% in servizio di sera (dalla 16 alle 24) e il 35\% di servizio nel turno di notte (dalle 24 alle 8). In qualsiasi momento, l'80\% delleguardie in servizio e' di pattuglia per le strade, mentre il 20\% rimanente e' assegnato a varie postazioni per la citta', pronti a reagire ad eventuali allarmi. Una postazione di guardia simile e' presente almeno in ogni vicinato cittadino (un vicinato e' composto da vari quartieri).

La maggioranza delle guardie cittadine e' composta da combattenti, quasi tutti di 1° livello. Gli ufficiali sono combattenti di livello piu' alto, e forse anche qualche incantatore.

\textbf{Macchine d'Assedio}\index{Macchine d'Assedio}

Le macchine d'assedio sono grosse armi, strutture temporanee o meccanismi tradizionalmente usati per assediare un castello o una fortezza.

\textbf{Catapulta Pesante}: \index{Catapulta}Una catapulta pesante e' una gigantesca macchina d'assedio in grado di scagliare macigni o altri oggetti pesanti con grande forza.Dal momento che l'arco di lancio della catapulta e' molto alto, il marchingegno e' in grado di colpire anche aree al di fuori della sua linea di visuale. Per fare fuoco con una catapulta pesante, il capo degli operatori del macchinario effettua una prova speciale con DC 15 usando solo il suo valore di Competenza di Difesa, il suo modificatore di Intelletto, la penalita' per la gittata e il modificatore relativo alla sezione inferiore della Tabella: Macchine d'Assedio. 

Se la prova ha successo, il macigno della catapulta colpisce la zona di mischia a cui la catapulta aveva mirato, infliggendo i danni indicati a qualsiasi oggetto o personaggio nella zona. I personaggi che superano con successo un Tiro Salvezza su Riflessi con DC 15 subiscono danni dimezzati. Una volta che il macigno ha colpito la zona, i tiri successivi colpiranno la stessa zona, a meno che la catapulta non venga ridirezione o il vento non cambi direzione o velocita'.

Se il macigno di una catapulta manca il bersaglio, si tira 1d8 per determinare dove atterra. Il risultato indica la direzione in cui il colpo devia, dove 1 indica verso la catapulta stessa e i valori da 2 a 8 le direzioni successive in senso orario attorno alla zona bersaglio. La distanza coperta e' pari a 1d4x10 metri.

Per caricare una catapulta e' necessaria una serie di azioni che portano via tutto il round. Occorre una prova di Potenza con DC 15 per abbassare il braccio della catapulta; la maggior parte delle catapulte hanno ruote che permettono fino a due operatori di usare l'azione di Aiutare un Altro per assistere l'operatore principale della carrucola.

Una prova di Professione (ingegnere d'assedio) con DC 15 consente di agganciare il braccio in posizione, e poi un'altra prova di Professione (ingegnere d'assedio) con DC 15 servira' per caricare il proiettile sulla catapulta. Sono necessarie quattro round per ricaricare una catapulta pesante (vari operatori della catapulta possono compiere queste azioni nello stesso round, quindi quattro persone possono ricaricare una catapulta nel giro di 1 solo round).

Una catapulta pesante occupa uno spazio di 4,5 metri.

\textbf{Catapulta Leggera}: Questa e' una versione piu' piccola e piu' leggera della catapulta pesante. Funziona essenzialmente come una catapulta pesante, con la differenza che e' necessaria una prova di Potenza con DC 10 per agganciare il braccio al suo posto, e soltanto 2 round per ridirezionare la catapulta.

Una catapulta leggera occupa uno spazio di 3 metri.

\textbf{Balista}: \index{Balista}Una balista e' in pratica una balestra pesante enorme fissa. La sua taglia rende difficile il suo utilizzo per la maggior parte delle creature.Quindi, una creatura media subisce penalita' --4 ai Tiri per Colpire quando usa una balista, e una creatura piccola subisce penalita' --6. Per una creatura di taglia inferiore alla grande sono necessari 2 round per ricaricare la balista dopo aver fatto fuoco.

Una balista occupa uno spazio di 1,5 metri.

\textbf{Ariete}:\index{Ariete} Questo tronco massiccio a volte e' legato e sospeso a un traliccio mobile che consente a coloro che lo manovrano di farlo oscillare con Potenza sempre crescente contro un bersaglio. Come unica azione del round, il personaggiopiu' vicino alla punta dell'ariete effettua un CA per Colpire contro la Difesa della costruzione, applicando penalita' --4 per la mancanza di competenza (non e' possibile avere competenza nell'uso di questo macchinario). Oltre ai danni indicati nella Tabella: Macchine d'Assedio, fino a nove altri personaggi possono spingere l'ariete e aggiungere i loro modificatori di Potenza al danno dell'ariete, se riservano un'azione di attacco per farlo. e' necessaria almeno una creatura Enorme o di taglia superiore, 2 creature Grandi, 4 creature Medie oppure 8 creature Piccole per manovrare un ariete (le creature Minuscole o di taglia inferiore non possono usare un ariete).

Un ariete solitamente e' lungo 9 metri. In una battaglia, le creature che manovrano un ariete devono disporsi in due file adiacenti di eguale lunghezza con l'ariete sorretto tra le due file. 

\textbf{Torre da Assedio}\index{Torre da Assedio}: Questo macchinario e' un'enorme torre di legno montata su ruote o cilindri che puo' essere spinta contro un muro per consentire agliassedianti di scalare la torre e quindi arrivare in cima alle mura beneficiando di Copertura. Le pareti in legno della torre di solito sono spesse circa 30 cm.

Una torre da assedio tipica occupa uno spazio di 4,5 metri. Le creature al suo interno la spingono a una velocita' di 3 metri (una torre da assedio non puo' correre). Le otto creature che spingono la torre al pian terreno godono di Copertura totale, quelle ai piani superiori godono di Copertura migliorata e possono tirare attraverso le feritoie per gli arcieri.

\textbf{Tabella: Modificatori di Attacco delle Catapulte}

\begin{tabular}[c]{@{}ll@{}}
\toprule 
Circostanza & Modificatore\tabularnewline
La linea di visuale non giunge fino alla zona bersaglio & -6\tabularnewline
Tiro consecutivo (gli operatori riescono a vedere dove sono\\
caduti i colpi andati a vuoto piu' recenti ) & +2 cumulativo per per colpo\\
&mancato precedente (max +10)\tabularnewline
Tiro consecutivo (gli operatori non riescono a vedere\\
dove sono caduti i colpi andati a vuoto piu' recenti\\
ma un osservatore fornisce indicazioni) & +1 cumulativo per per colpo\\
&mancato precedente (max +5))\tabularnewline
\bottomrule
\end{tabular}

\textbf{Tabella: Macchine d'Assedio}

\begin{tabular}[c]{@{}lllll@{}}
\toprule 
Macchina & Costo (mo) & Danno & Gittata (metri) & Operatori\tabularnewline
Catapulta pesante & 800 & 6d6 & 60 & 4\tabularnewline
Catapulta leggera & 5500 & 4d6 & 45 & 2\tabularnewline
Ballista & 500 & 3d8 & 36 & 1\tabularnewline
Ariete & 1000 & 3d6 & - & 10\tabularnewline
Torri da Assedio & 2000 & - & - & 20\tabularnewline
\bottomrule
\end{tabular}

\textbf{Strade Cittadine}\index{Strade Cittadine}

Le tipiche strade di una citta' sono strette e tortuose. La maggior parte delle vie cittadine e' larga dai 4,5 ai 6 metri, mentre i vicoli vanno da una larghezza di 3 metri a una di soltanto 1,5 metri. Se il pavimento lastricato e' in buone condizioni, e' possibile muoversi normalmente, mentre le strade in brutte condizioni e gravemente rovinate vengono considerate equivalenti a detriti sparsi, e aumentano la DC delle prove di Acrobatica di 2.

Alcune citta' non hanno grandi viali d'accesso, specialmente quelle che sono cresciute gradualmente partendo come piccoli insediamenti. Le citta' che sono state progettate a tavolino, o che forse sono state consumate da un grave incendio che ha consentito alle autorita' di costruire nuove strade su quelle che un tempo erano aree abitate, potrebbero disporre di alcune strade piu' grandi che le attraversano. Queste strade principali sono ampie 7,5 metri, e consentono ai carri di passare l'uno di fianco all'altro, con marciapiedi di 1,5 metri su entrambi i lati.

\textbf{Folla}: Le strade cittadine sono gremite di gente che va e viene, impegnata nelle varie faccende giornaliere. Nella maggior parte dei casi non e' necessario includere ogni popolano di 1° livello sulla mappa quando si giunge a un combattimento sul viale principale della citta'.

e' sufficiente invece indicare quali zone sulla mappa sono occupati dalla folla. Se la folla vede qualcosa di pericoloso, si allontanera' alla velocita' di 9 metri per round a conteggio di Iniziativa 0. Per entrare in contatto con la folla bisogna avere una distanza di mischia. La folla fornisce Copertura a chiunque riesca a entrarvi, consentendo unaprova di Furtivita' (Consapevolezza) e fornendo un bonus alla Difesa e ai Tiri Salvezza su Riflessi.

\textbf{Dirigere la Folla}: e' necessaria una prova di Diplomazia (Faccia Tosta) con DC 15 o di Intimidire (Faccia Tosta) con DC 20 per convincere una folla aspostarsi in una certa direzione, e la folla deve essere in grado di sentire o vedere il personaggio che effettua il tentativo. e' necessaria tutto un round per effettuare la prova di Diplomazia, mentre serve solo un'Azione per effettuare la prova di Intimidire.

Se due o piu' personaggi tentano di spingere la folla in due direzioni diverse, effettuano prove di Diplomazia (Faccia Tosta) o di Intimidire contrapposte per determinare a chi la folla dara' ascolto. La folla ignorera' entrambi, se tutti e due i risultati delle prove dovessero essere inferiori alle DC sopra indicate.

\textbf{Sopra e Sotto le Strade}

\textbf{Tetti}: Per arrampicarsi su un tetto di solito e' necessario scalare un muro (vedi la sezione Pareti), a meno che un personaggio non possa raggiungere un tetto saltando giu' da una finestra, un balcone o un ponte piu' alto. I tetti piatti sono comuni solo nelle zone a clima caldo (la neve, accumulandosi, puo' far crollare un tetto piatto) e sono facili da percorrere correndo. Spostarsi sulla cima di un tetto richiede una prova di Acrobatica con DC 20. Spostarsi orizzontalmente su un tetto inclinato (muovendosi in parallelo alla sua cima, in pratica) richiede una prova di Acrobatica con DC 15. Spostarsi su e giu' lungo un tetto inclinato richiede una prova di Acrobatica con DC 10.

Prima o poi un personaggio giungera' alla fine del tetto, e dovra' effettuare un lungo salto per passare al tetto successivo o per scendere a terra. La distanza che separa un tetto dal successivo di solito e' di 3 metri, ma il tetto dall'altra parte potrebbe essere piu' alto o piu' basso di 1,5 metri, o alla stessa altezza. Si usano le indicazioni date per Acrobatica (il picco d'altezza in un salto in lungo e' pari ad un quarto della distanza orizzontale) per determinare se il personaggio e' in grado di effettuare un salto.

\textbf{Fognature}: Per entrare nelle fognature, i personaggi solitamente devono aprire una grata (1 round) e saltare in basso per 3 metri. Le fognature sono costruite esattamente come dei dungeon, con la differenza che il pavimento e' scivoloso o ricoperto d'acqua. Le fognature sono anche simili ai dungeon per quello che riguarda le creature che e' possibile incontrare al loro interno. Alcune citta' sono state costruite sulle rovine di civilta' piu' antiche, quindi le fognature potrebbero anche condurre a tesori e pericoli appartenenti a un'era passata.

\textbf{Edifici Cittadini}

La maggior parte degli edifici cittadini e' divisa in tre categorie. Molti edifici in una citta' sono alti da due a cinque piani e sono costruiti l'uno di fianco all'altro per formare lunghe file, interrotte soltanto dalle vie principali o secondarie. Questi edifici a schiera solitamente ospitano un negozio a pianterreno, con uffici o appartamenti ai piani superiori.

Le locande, le imprese commerciali piu' ricche e i magazzini piu' grandi (oltre a eventuali mulini, concerie e altre attivita' che richiedano molto spazio) in generesono grossi edifici indipendenti alti fino a cinque piani.

Infine, le abitazioni, i negozi, i magazzini e i depositi piu' piccoli sono dei semplici edifici di legno a un piano, specialmente nei quartieri piu' poveri.

\textbf{Illuminazione Cittadina}

Se una citta' possiede grandi viali d'accesso, questi saranno illuminati da lanterne appese a un'altezza di circa 2 metri sui lati degli edifici. Queste lanterne sono poste a una distanza di 9 metri l'una dall'altra, quindi l'illuminazione in queste strade e' praticamente continua. Le strade secondarie e i vicoli non sono illuminati; e' consuetudine per i cittadini pagare un lanternaio che li accompagni, se devono uscire di notte.

I vicoli possono essere luoghi bui anche di giorno, grazie alle ombre degli edifici piu' alti circostanti. Un vicolo buio di giorno non e' buio a sufficienza da poter conferire Occultamento vero e proprio, ma puo' fornire Bonus +2 alle prove di Muoversi Silenziosamente.

\pagebreak

\section{Avventure e Disastri}\index{Avventure}\index{Disastri}

\label{avventure-e-disastri}
\begin{quotebox}
Per prima cosa, nessuno rimane indietro. (anonimo)
\end{quotebox}
I disastri naturali sono pericoli ambientali terrificanti che portano morte e devastazione. Quelli soprannaturali posso­no essere anche piu' distruttivi, poiché possono sfigurare per sempre un mondo. Un disastro e' piu' simile ad un'avventura che ad un incontro, e non ha uno specifico Grado di Sfida. Piuttosto, ogni parte del disastro dovrebbe essere trattata come un incontro separato ideato con un CR adeguato ai PG.

Sotto vengono presentate le regole per gestire gli effetti di tre diversi tipi di disastri, sia naturali che soprannaturali. Alcuni disastri si verificano rapidamente, come terremoti e tsunami, mentre altri procedono attraverso numerose fasi, come gli incendi forestali, i vulcani e le sollevazioni di non morti. Aggiustate lo schema dell'avventura per adattarlo al disastro, per permettere agli eventi di svolgersi nel corso di pochi minuti o molti giorni a seconda di quello che vi serve.

\textbf{Vulcani}\index{Vulcani}

Quando la crosta terrestre si rompe ed espelle il suo cuore fuso ha luogo uno dei disastri naturali piu' drammatici: l'eruzione di un vulcano. Le eruzioni vulcaniche offrono una vasta gamma di opzioni al Narratore, inclusi lava, bombe laviche, gas venefici e colate piroclastiche. I Narratore potrebbero anche considerare l'idea di far presagire una drammatica eruzione vulcanica (o draghi vulcanici) con pericoli preesistenti, come valanghe e terremoti minori.

\textbf{Lava}\index{Lava}

I flussi lavici generalmente sono associati alle eruzioni non esplosive e possono essere un elemento permanente dei vulcani attivi. Le colate laviche sono per lo piu' lente e si muovono a 4,5 metri per round (penalita' azione movimento 1), ma quelle piu' calde sono rapide, e raggiungono i 12 metri per round (nessuna penalita' azione movimento). La lava incanalata, come in un tubo lavico, e' molto pericolosa, poiché si muove alla velocita' di 36 metri per round (4 azioni di movimento a round) (un pericolo con CR 6). Le creature raggiunte da una colata lavica devono superare un Tiro Salvezza su Riflessi con DC 20 o sono sommerse dalla lava. Il successo indica che sono a contatto con la Lava ma non Immerse.

\textbf{Bombe Laviche} (CR 2 o 8)\index{Bombe Laviche}

Agglomerati di pietra fusa possono essere scagliati a molti chilometri da un vulcano che erutta, raffreddandosi in solida pietra prima di raggiungere il terreno. Una tipica bomba lavica colpisce un punto designato dal Narratore ed esplode in un raggio medio. Tutte le creature nell'area devono superare un Tiro Salvezza su Riflessi con DC 15 o subiscono 4d6 danni. Le creature che hanno Copertura o sono in grado di coprirsi (come con uno­scudo) ottengono bonus +2 a questo tiro. A volte si formano bombe laviche molto grandi che infliggono 12d6 danni. Le bombe laviche normali hanno CR 2, quelle grandi CR 5.

\textbf{Gas Venefici} (CR 5)\index{Gas Venefici}

Una delle minacce piu' insidiose di un vulcano e' il gas tossico, spesso non notato tra il fuoco e la distruzione. Diversi tipi di vapori venefici scaturiscono da un'eruzione vulcanica, alcuni visibili, altri no. I gas venefici infliggono 1d6 danni alla Potenza per round se inalati (Tempra DC 15 nega, la DC aumenta di 1 per ogni Tiro Salvezza precedente), e quelli visibili funzionano anche come Fumo Denso. Le nubi di gas venefici fluiscono verso il basso, e generalmente arrivano ad una altezza di 6 metri. Forti venti possono deviare le nubi di gas, cosi' come alte barriere, a condizione che il gas abbia un altro posto dove andare.

\textbf{Colate Piroplastiche} (CR 10)

Alcune eruzioni vulcaniche creano una devastante ondata di cenere ardente, gas bollenti e detriti vulcanici chiamata colata piroclastica che puo' viaggiare per chilometri. Una colata piroclastica viene trattata come una Valanga che viaggia a 150 metri per round, combinata con gli effetti dei gas venefici indicati sopra. Il contatto con i detriti roventi della colata infligge 2d6 danni da fuoco per round, mentre qualsiasi creatura seppellita dalla colata subisce 10d6 danni per round.

\textbf{Tsunami}\index{Tsunami}

Gli tsunami, talvolta attribuiti ad onde di marea, sono tremende ondate d'acqua causate da terremoti sottomarini, esplosioni vulcaniche, smottamenti o impatti di asteroidi. Gli tsunami non si possono individuare finché non raggiungono l'acqua poco profonda, quando la massa d'acqua forma una grande onda. A seconda dalle dimensioni dello tsunami e del­la pendenza della costa, l'onda puo' coprire qualsiasi distanza, dal centinaio di metri fino ad oltre un chilometro sulla terra ferma, lasciandosi dietro una scia di distruzione. L'acqua poi si ritira, trascinando via ogni sorta di detriti e creature fino in alto mare.

L'esatta devastazione causata e' soggetta alla discrezione del Narratore, ma un tipico tsunami abbatte o sradica tutte le strutture temporanee o mal costruite sul suo percorso, distrugge circa il 25\% degli edifici ben costruiti (causando danni significativi a quelli che restano) e lascia le fortificazioni solide leggermente danneggiate. Almeno 1/4 della popolazione che vive nell'area (inclusi animali e mostri) muore nel disastro, trascinato in mare, affogato sulla spiaggia o seppellito sotto le macerie.

Una creatura puo' evitare di essere portata via dal mare con una prova di Nuotare (Resistenza) con DC 25; altrimenti viene trascinata a 6d6 x 3 metri dalla riva. Le acque dopo uno tsunami sono sempre considerate agitate o tempestose, salvo influenze magiche. Una creatura coinvolta nel cedimento di un edificio subisce 6d6 danni (TS su Riflessi DC 15 dimezza), o la meta' se la struttura e' particolarmente piccola. C'e' una probabilita' del 50\% che la creatura venga sepolta (come per un Crollo), o che lo tsunami possa distruggere l'edificio, liberando la creatura dalle macerie.

\textbf{Sollevazione di Non Morti}\index{Non Morti}

Frutto di un'antica maledizione o di atti necromantici, uno dei disastri soprannaturali piu' terrificanti e' la sollevazione di Non Morti: il morto che emerge dal­la tomba per reclamare il vivo. Questo disastro puo' colpire qualsiasi area dove sono stati sepolti dei morti, non solo paesi e citta'. Piu' di un campo di battaglia ha visto sorgere una legione di rinsecchiti combattenti Non Morti. Le sollevazioni di Non Morti si svolgono ad ondate, con la tempistica che varia secondo le forze principali in gioco. Gli eventi possono succedersi nel corso di pochi giorni, con la devastazione di una citta', o protrarsi per settimane con la popolazione terrorizzata che si rannicchia dietro porte sprangate e lotta per sopravvivere. Durante il giorno, spesso la vita ritorna ad una parvenza di normalita', poiché la luce del giorno sopprime temporaneamente il potere della non morte.

\textbf{I Morti Inquieti}

Nelle prime notti di una sollevazione di Non Morti, i morti recenti si rianimano come zombi. Quelli sepolti in terra consacrata non si rianimano, ma i corpi lasciati insepolti o in fosse comuni barcollano fuori per le strade, portando scompiglio. Inizialmente, solo alcuni cadaveri sono capaci di liberarsi dal­le loro bare e tombe, ma ogni sera, il numero di cadaveri vivi aumenta. Quando giunge l'alba, i morti cercano la sicurezza nelle loro tombe o di altri luoghi nascosti. Chiunque venga colto dalla luce del giorno si agita Confuso finché non viene distrutto o raggiunge un rifugio. A discrezione del Narratore, cadaveri di non umanoidi possono risorgere come Non Morti nelle notti seguenti.

\textbf{Il Risveglio degli Scheletri}

Con l'avanzare della sollevazione, cadaveri sempre piu' vecchi si uniscono alle schiere dei Non Morti. Scheletri che recano tracce di vesti funebri marcite da tempo scavano con gli artigli una via d'uscita da cimiteri e cripte, ed agiscono con una malevolenza ed organizzazione raramente riscontrate tra i loro simili. I Non Morti rimangono privi di Intelletto, ma il potere magico dietro all'incursione dona loro l'efficienza e l'acume tattico di un esercito di viventi. Gli Scheletri scovano armi e corazze con cui equipaggiarsi per la battaglia. L'élite degli Scheletri campioni guida le truppe, utilizzando Oggetti Magici trafugati da tombe abbandonate. Infine, anche Ghoul e Wight vagano in cerca di preda per le strade durante il buio, insieme ad altri Non Morti minori dotati di libero arbitrio

\textbf{Anime Perse}

Mentre la sollevazione raduna le forze, si risvegliano anche le anime inquiete di cadaveri da tempo ridotti in polvere. Fantasmi, Ombre, Wraith e persino Spettri sorgono per dare la caccia ai vivi. Alcuni Fantasmi potrebbero liberarsi dalla malevola influenza della sollevazione e dei personaggi intraprendenti potrebbero raccogliere preziose informazioni da questi spiriti inquieti.

L'infusione di energia negativa fortifica i Non Morti all'interno dell'area dell'incursione, concedendo i benefici di una Essenza di Protezione (+2 Difesa/+2 Tiri Salvezza). Le aree una volta consacrate sono ora trattate come terreno normale, e possono fungere da nuove fonti di cadaveri per le armate Non Morte; il terreno santificato rimane inviolato.

Quando i Non Morti diventano piu' forti, l'ondata crescente di energia negativa avvicina il Piano delle Ombre, stingendo o ingrigendo i colori tranne durante le ore piu' brillanti del giorno. Anche i Non Morti piu' vulnerabili alla luce possonomuoversi impunemente dal tardo pomeriggio alla mezza mattinata.

\textbf{Necropoli}

Il flusso di energia negativa e' irreversibile, l'oscurita' infine reclama l'area, coprendola con un'ombra perpetua.Il terreno santificato resta un raro santuario, ma solo finché non viene distrutto dalle forze malevoli forze esterne.

Gli eroi morti negli scontri ritornano come spaventosi generali Non Morti. I pochi superstiti viventi vengono assoggettati come schiavi. L'area diviene una citta' della morte o ne viene cominciata la costruzione se non esisteva o non e' sopravvissuta alcuna citta'. I Non Morti dotati di libero arbitrio si radunano in questo nuovo santuario e solo gli eroi piu' grandi riescono a tornare da quest'area ormai avvizzita al mondo dei vivi.

\pagebreak

\section{Avventure nei Dungeon}\index{Dungeon}

\label{avventure-nei-dungeon}
\begin{quotebox}
Il dungeon e' inclinato. Le creature sono infuriate perché non riescono a giocare a biglie (Dungeon Keeper 2,Videogioco, 1999)
\end{quotebox}

Di tutti i luoghi strani che un avventuriero puo' esplorare, nessuno e' piu' letale di un dungeon. Questi labirinti, pieni di trappole mortali, mostri affamati etesori meravigliosi, provano ogni Abilita' dei personaggi. Queste regole si possono applicare a qualsiasi tipo di dungeon, dal relitto di una nave ad un vasto complesso di grotte sotterranee.

\textbf{Tipi di Dungeon}

I quattro tipi base di dungeon sono definiti dal loro stato attuale. Molti dungeon sono varianti di questi tipi base o combinazioni di piu' tipi. Occasionalmente, antichi dungeon vengono usati ripetutamente da nuovi abitanti per scopi diversi.

\textbf{Struttura in Rovina}: Un tempo abitato, questo luogo e' ora abbandonato (completamente o in parte) dai suoi creatori originari ed e' occupato da altre creature. Molte creature sotterranee vanno alla ricerca di costruzioni sotterraneeabbandonate in cui stabilire le loro tane. Qualsiasi trappola che possa essere esistita e' stata probabilmente gia' rimossa o attivata, ma e' possibile trovare bestie erranti.

\textbf{Struttura Occupata}: Questo dungeon viene ancora utilizzato. Delle creature (di solito intelligenti) ancora lo abitano, anche se potrebbero non essere i creatori del dungeon. Una struttura occupata potrebbe essere una casa, una fortezza, un tempio, una miniera attiva, una prigione, un quartier generale. 

Questo tipo di dungeon e' meno probabile che abbia trappole o bestie erranti, e piu' probabilmentedispone di guardie organizzate, sia stazionarie che di pattuglia. Le trappole e le bestie erranti che si possono incontrare sono spesso sotto il controllo degli occupanti. Le strutture occupate dispongono di arredo adatto agli abitanti, cosi' come decorazioni, riserve di cibo, e la possibilita' per gli abitanti di muoversi.

Gli abitanti possono disporre anche di un sistema di comunicazione, e quasi sempre
controllano almeno un accesso verso l'esterno.

Alcuni dungeon sono parzialmente occupati e parzialmente vuoti o in rovina. In questi casi, gli occupanti di solito non sono gli originari costruttori del luogo, ma bensi' un gruppo di creature intelligenti che hanno stabilito la loro base, tana o fortificazione all'interno del dungeon abbandonato.

\textbf{Riparo Sicuro}: Quando qualcuno vuole proteggere una cosa, spesso la seppellisce sottoterra. Che l'oggetto che vuole proteggere sia un favoloso tesoro, un artefatto proibito o il cadavere di un uomo importante, questi oggetti di valore vengono posti all'interno di un dungeon e circondati da barriere, trappole e guardiani.

Il dungeon del tipo riparo sicuro e' quello che avra' piu' trappole e meno bestie erranti. e' normalmente costruito in base alla funzionalita' piuttosto che all'aspetto, anche se a volte viene decorato con statue e pareti dipinte, specie per le tombe di personaggi importanti.

A volte, pero', una sala del tesoro o una cripta vengono costruite in modo da ospitare guardiani viventi. Il problema con questa strategia e' che occorre tenere in vita le creature tra un tentativo di intrusione e un altro. La magia e' di solito la soluzione migliore per rifornire di cibo e acqua queste creature. I costruttori di tombe e sepolcri, di solito, pongono non morti e costrutti, che non hanno bisogno di sostentamento o di riposo, a protezione dei loro dungeon. Le trappole magiche possono attaccare gli intrusi convocando mostri nel dungeon che scompaiono quando terminano il loro compito.

\textbf{Complesso di Caverne Naturali}: Le caverne sotterranee offrono riparo a qualsiasi tipo di creatura delle profondita'. Create naturalmente e collegate da un sistema di passaggi labirintici, queste caverne mancano di qualsiasi parvenza di ordine, logica o decorazioni. Senza alcuna potenza intelligente che lo abbia costruito, questo tipo di dungeon e' quello che ha minori probabilita' di presentare trappole o porte.

Molteplici varieta' di funghi vivono nelle caverne, a volte crescendo fino a formare enormi foreste di funghi e vesce, dove si aggirano predatori sotterranei si aggirano a caccia di chi si nutre di questi vegetali. Alcune varieta' di funghi producono un bagliore fosforescente in grado di fornire al complesso di caverne naturali una propria limitata fonte di illuminazione. In altre zone, l'uso di Essenza di Creazione puo' garantire luce sufficiente per la crescita di piante verdi.

Spesso, un complesso di caverne naturali e' collegato ad altri tipi di dungeon, essendo stato scoperto quando e' stato costruito il dungeon artificiale. Un complesso di caverne puo' collegare due dungeon indipendenti, producendo a volte uno strano ambiente misto. Un complesso di caverne naturali unito a un altro dungeon, spesso, offre un percorso che le creature sotterranee possono usare per raggiungere un dungeon artificiale e popolarlo.

\textbf{Terreno del Dungeon}

Le regole seguenti riguardano i terreni di base che si possono trovare in un dungeon.

\textbf{Pareti}

A volte, pareti in mattoni (pietre accatastate una sopra l'altra solitamente, ma non sempre, tenute insieme con la calce) dividono i dungeon in corridoi e stanze. Le pareti dei dungeon possono anche essere scolpite nella nuda roccia, ottenendo cosi' un aspetto scalpellato, oppure possono essere composte di pietra liscia e semplice come si trova nelle caverne naturali. Le pareti dei dungeon sono difficili da danneggiare o da sfondare, ma di solito sono facilmente scalabili.

\bigskip

\textbf{Tabella: Pareti}

\begin{tabular}{@{}llllll@{}}
\toprule 
Tipo di Parete & Spessore Tipico & DC per Sfondare & Durezza & & DC per Scalare\tabularnewline
Mattoni & 30 cm & 35 & 8 & 90 & 20\tabularnewline
Mattoni superiori & 30 cm & 35 & 8 & 120 & 25\tabularnewline
Mattoni rinforzati & 30 & 45 & 8 & 180 & 20\tabularnewline
Pietra Scolpita & 90 & 50 & 8 & 540 & 25\tabularnewline
Pietra grezza & 150 cm & 65 & 8 & 900 & 25\tabularnewline
Ferro & 7.5 cm & 30 & 10 & 90 & 25\tabularnewline
Carta & variabile & 1 & -- & 1 & 30\tabularnewline
Legno & 15 cm & 20 & 5 & 60 & 21\tabularnewline
\bottomrule
\end{tabular}

\bigskip
 
\textbf{Pareti in Mattoni}: Il tipo piu' comune di parete per un dungeon, le pareti in mattoni di solito sono spesse almeno 30 centimetri. Spesso queste antiche pareti presentano fori e fessure, all'interno dei quali possono annidarsi fanghiglie e piccole creature, che aspettano li' le loro prede. Le pareti di mattone sono in grado di bloccare tutti i rumori, tranne quelli piu' forti. e' necessaria una prova di Scalare (Resistenza) con DC 20 per muoversi lungo una parete in mattoni.

\textbf{Pareti in Mattoni di Qualita' Superiore}: A volte le pareti in mattoni sono costruite meglio (piu' lisce, con pietre meglio incastrate e meno danneggiate) e occasionalmente queste pareti di qualita' superiore sono coperte da calcina o stucco. Queste pareti sono spesso abbellite da dipinti, bassorilievi o altre decorazioni. Le pareti in mattoni di qualita' superiore non sono piu' difficili da danneggiare delle normali pareti in mattoni, ma sono piu' difficili da Scalare (Resistenza) (DC 25).

\textbf{Pareti rinforzate} Queste sono pareti in mattoni con sbarre di ferro su uno o entrambi i lati, o inserite all’interno della parete stessa per rinforzarla. La Durezza della parete rinforzata resta la stessa, ma i Punti Ferita vengono raddoppiati e la DC per la prova di Potenza per sfondarla viene incrementata di 10.

\textbf{Pareti di Pietra Scolpita}: Queste pareti generalmente si trovano in stanze o passaggi scavati nella nuda roccia. La ruvida superficie di una parete scolpita presenta minuscole sporgenze su cui possono crescere funghi e crepe all'interno delle quali possono vivere parassiti, pipistrelli o serpi sotterranee. 

Quando una parete di questo tipo ha un ``altro lato'' (la parete separa due stanze in un dungeon), la parete e' spessa almeno 90 centimetri; se fosse piu' sottile rischierebbe di far crollare tutto perché non sarebbe in grado di sostenere il peso della volta di pietra. e' necessaria una prova di Scalare (Resistenza) con DC 25 per scalare una parete di pietra scolpita.

\textbf{Pareti di Pietra Grezza}: Queste superfici sono irregolari e raramente piatte. Sono lisce al tocco ma piene di minuscoli buchi, alcove nascoste e sporgenze a varie altezze. Di solito sono bagnate o perlomeno umide, in quanto le caverne naturali sono in genere il prodotto di infiltrazioni d'acqua. Quando una parete di questo tipo da un ``altro lato'', la parete e' di solito spessa almeno 150 centimetri.

e' necessaria una prova di Scalare con DC 15 per muoversi lungo una parete di pietra grezza.

\textbf{Pareti di Ferro}: Queste pareti sono poste all'interno dei dungeon intorno a luoghi importanti come le sale del tesoro.

\textbf{Pareti di Carta}: Le pareti di carta sono l'opposto di quelle di ferro, utilizzate come schermi per impedire la vista ma nulla piu'.

\textbf{Pareti di Legno}: Le pareti di legno si trovano spesso come recenti aggiunte a dungeon piu' antichi, utilizzate per creare recinti per animali, depositi, o anche solo per dividere in una serie di stanze piu' piccole una piu' grande.

\textbf{Pareti Trattate Magicamente}: Queste pareti sono piu' forti della media, con una Durezza maggiore, con piu' Punti Ferita e per sfondarle bisogna superare una DC maggiore. La magia puo' di solito raddoppiare la Durezza e i Punti Ferita della parete e aggiungere fino a +20 alla sua DC per sfondarla. Una parete trattata magicamente ottiene anche un Tiro Salvezza contro Essenze che potrebbero avere effetto su di essa, con il bonus al Tiro Salvezza pari a 2 + meta' del livello dell'incantatore della magia che rinforza la parete. Creare una parete magica richiede il talento Creare Oggetti Meravigliosi e la spesa di 1.500 mo per ogni sezione di 3 per 3 metri.

\textbf{Pareti con Feritoie}: Le pareti con feritoie possono essere costruite con qualsiasi materiale resistente, ma sono di solito fatte in mattoni, pietra scolpita o legno. Permettono ai difensori di scagliare frecce o quadrelli da balestra contro gli intrusi restando dietro la relativa protezione di un muro. Gli arcieri dietro alle feritoie godono di una Copertura superiore che fornisce loro bonus +8 alla Difesa, bonus +4 ai Tiri Salvezza su Riflessi.

\textbf{Pavimenti}

Cosi' come per le pareti, esistono molti tipi di pavimenti per dungeon.

\textbf{Lastricato}: Come le pareti in mattoni, i pavimenti possono essere composti da pietre incastrate tra loro. Sono di solito piene di fessure e solitamente appena livellate. Fanghiglie e muffe crescono all'interno di queste fessure. In certi casi l'acqua scorre in piccoli scoli attraverso le pietre o forma pozze stagnanti. Il lastricato e' il tipo di pavimento piu' comune nei dungeon.

\textbf{Lastricato Irregolare}: Col passare del tempo, alcuni pavimenti possono diventare talmente irregolari da richiedere una prova di Acrobatica con DC 10 per correre o Caricare sulla loro superficie. Coloro che falliscono la prova non possono muoversi durante quel round. Pavimenti cosi' pericolosi dovrebbero essere in realta' l'eccezione e non la regola.

\textbf{Pavimento di Pietra Scolpita}: Ruvidi e irregolari, i pavimenti scolpiti nella pietra sono di solito coperti da pietre smosse, ghiaia, polvere e altri detriti. Una prova di Acrobatica con DC 10 e' necessaria per correre o Caricare su un simile pavimento. Un fallimento significa che il personaggio puo' ancora agire, ma non puo' correre o Caricare in quel round.

\textbf{Pietrisco Scarso}: Piccoli e sparuti detriti sono presenti a terra. Un pavimento su cui sia presente del pietrisco scarso aggiunge 2 alla DC delle prove di Acrobatica.

\textbf{Pietrisco Denso}: Il terreno e' ricoperto di detriti di tutte le dimensioni. Entrare in una zona di mischia ricoperto di pietrisco denso costa 2 azioni di movimento. Un pavimento cosparso di pietrisco denso aggiunge 5 alla DC delle prove di Acrobatica, e aggiunge 2 alla DC delle prove di Consapevolezza (Muoversi Silenziosamente)

\textbf{Pavimento di Pietra Liscia}: Pavimenti lisci, perfetti e a volte anche levigati si trovano solo nei dungeon creati da costruttori capaci e attenti.

\textbf{Pavimento di Pietra Naturale}: Il pavimento di una caverna naturale e' irregolare quanto le pareti. e' difficile che queste caverne presentino ampie superfici piane; e' piu' probabile che i loro pavimenti siano disposti su piu' livelli. 

Alcune superfici adiacenti potrebbero variare in elevazione di appena 30 centimetri, cosicché lo spostamento da un punto all'altro non sia piu' difficile del salire un gradino di una scala, ma in certi punti il pavimento potrebbe scendere o salire di diverse decine di centimetri, obbligando il personaggio a una prova di Resistenza (Scalare) per spostarsi da una superficie a un'altra.

A meno che non ci sia un percorso scavato dal tempo o ben battuto il terreno e' considerato difficile e quindi il movimento e' dimezzato, la DC delle prove diAcrobatica e' aumentata di 5. La Carica e la corsa in questi ambienti sono impossibili, tranne che sui percorsi in questione.

\textbf{Scivoloso}: Acqua, ghiaccio, melma o sangue possono rendere qualunque pavimento descritto in questa sezione piu' insidioso. I pavimenti scivolosi aumentano la DC delle prove di Acrobatica di 5.

\textbf{Grata}: Una grata spesso copre una fossa o una zona al di sotto del pavimento principale. Le grate sono di solito costruite in ferro, ma quelle piu' grosse potrebbero essere anche fatte di tronchi d’albero rinforzati. Molte grate hanno cardini che permettono l’accesso alla zona sottostante (queste grate possono essere chiuse a chiave come una porta), mentre altre sono fisse e create per non poter essere spostate. Una tipica grata di ferro spessa 2,5 centimetri ha 25 Punti Ferita, Durezza 10, e DC 27 per sfondarla o smuoverla.

\textbf{Sporgenze}: Le sporgenze permettono alle creature di camminare al di sopra di un'area sottostante. Spesso sono disposte intorno a fosse, lungo il corso di fiumi sotterranei, come balconate che circondano un'ampia stanza oppure forniscono una posizione dalla quale gli arcieri possono appostarsi per attaccare i nemici dall'alto. 

Le sporgenze strette (di ampiezza inferiore a 30 centimetri) richiedono a coloro che vi si muovono sopra delle prove di Acrobatica. Un fallimento implica che il personaggio che si stava muovendo cade dalla sporgenza.

A volte le sporgenze hanno una ringhiera. In questi casi i personaggi ottengono Bonus +5 alle prove di Acrobatica per muoversi lungo la sporgenza. Un personaggio vicino alla ringhiera ha Bonus +2 alla propria prova contrapposta di Potenza per evitare di essere spinto giu' dalla sporgenza. 

Le sporgenze a volte possono anche essere delimitate da balaustre alte 60-90 centimetri. Simili muri forniscono Copertura da aggressori entro distanza 3 metri dall'altro lato del muro, ammesso che il bersaglio sia piu' vicino alla balaustra di chi attacca.

Pavimenti Trasparenti: I pavimenti trasparenti, fatti di vetro rinforzato o di materiali magici (o addirittura dall'Essenza di Creazione, Muro), permettono di osservare un ambiente pericoloso dall'alto. I pavimenti trasparenti sono di solito posti al di sopra di pozze di lava, arene, tane di mostri e stanze di tortura.Possono essere usati dai difensori 
per sorvegliare un'area.

\textbf{Pavimenti Scorrevoli}: Un pavimento scorrevole e' un tipo di botola, creato per essere spostato e rivelare qualcosa che si trova al di sotto. In genere un pavimento scorrevole si muove tanto lentamente che chiunque vi si trovi sopra puo' evitare di cadere nell'apertura, purché abbia spazio per spostarsi. Se un pavimento di questo tipo scorre tanto velocemente che c'e' la possibilita' che un personaggio cada in quello che si trova sotto di esso (lance acuminate, una vasca con olio bollente, o una pozza infestata da squali) allora e' una trappola.

\textbf{Pavimenti Trappola}: Questi pavimenti sono stati progettati per diventare di colpo pericolosi. Con l'applicazione della giusta quantita' di peso o l'azionamento di una leva nelle vicinanze, spuntoni sbucano dal pavimento, fiammate o sbuffi di vapore partono da fori nascosti, o l'intero pavimento si muove. Questi strani pavimenti si trovano di solito dentro alle arene, progettati per rendere i combattimenti piu' appassionanti e letali. Questo tipo di pavimento e' costruito nello stesso modo di una trappola.

\textbf{Porte} \index{Porte}Le porte all'interno dei dungeon sono ben piu' che semplici entrate o uscite. Spesso possono essere dei veri e propri incontri. Le porte dei dungeon si presentano in tre tipi basilari: di legno, di pietra e di ferro.

\bigskip

\textbf{Tabella: Porte}

\begin{tabular}[c]{@{}llllll@{}}
\toprule 
Tipo di porta & Spessore tipico (cm) & Durezza & & DC per sfondare & \tabularnewline
 & & & & Bloccata & Chiusa a chiave\tabularnewline
Legno semplice & 2.5 & 5 & 10 & 13 & 15\tabularnewline
Legno buono & 3.75 & 5 & 15 & 16 & 18\tabularnewline
Legno robusto & 5 & 5 & 20 & 23 & 25\tabularnewline
Pietra & 10 & 8 & 60 & 28 & 28\tabularnewline
Ferro & 5 & 10 & 60 & 28 & 28\tabularnewline
Saracinesca di legno & 7.5 & 5 & 30 & 25{*} & 25{*}\tabularnewline
Saracinesca di ferro & 5 & 10 & 60 & 25{*} & 25{*}\tabularnewline
Serratura & - & 15 & 30 & - & -\tabularnewline
Cardini & - & 10 & 30 & - & -\tabularnewline
\bottomrule
\end{tabular}

{*} DC per sollevare. Usate la voce appropriata di porta per sfondare.

\bigskip

\textbf{Porte di Legno}: Costruite con spesse assi inchiodate, a volte rinforzate con sbarre di ferro (poste anche per impedire le deformazioni prodotte dall'umidita' dei dungeon), quelle di legno sono il tipo piu' comune di porta. Le porte di legno variano per durezza: possono essere semplici, buone o robuste. Le porte semplici (DC 15 per sfondarle) non sono progettate per tenere alla larga assalitori motivati.

Le porte di buona fattura (DC 18 per sfondarle), sebbene forti e resistenti, non sono comunque progettate per subire una grande quantita' di danni. Le porte robuste (DC 25 per sfondarle) sono rivestite in ferro e sono delle barriere discretamente resistenti contro coloro che cerchino di oltrepassarle. Cardini di ferro sorreggono la porta, e di solito un anello circolare posto al centro serve ad aprirla.A volte, al posto di un anello, una porta dispone di una sbarra di ferro su uno o entrambi i lati che funziona come maniglia.

Nei dungeon abitati queste porte sono di solito ben tenute (non bloccate) e non chiuse a chiave, anche se le zone importanti probabilmente saranno chiuse a chiave.

\textbf{Porte di Pietra}: Costruite da blocchi di pietra solida, queste porte pesanti e poco maneggevoli sono spesso pensate in modo da ruotare su se stesse quando vengono aperte, anche se i nani e altri abili artigiani sono in grado di costruire cardini forti abbastanza da sostenere il peso di una porta di pietra.

Le porte segrete nascoste lungo una parete di pietra sono solitamente di pietra. Altrimenti, le porte di questo tipo sono studiate per diventare resistenti barriere che proteggono qualsiasi cosa si trovi al di la' di esse. Di conseguenza si trovano spesso chiuse a chiave o sbarrate.

\textbf{Porte di Ferro}: Arrugginite ma resistenti, le porte di ferro in un dungeon sono dotate di cardini come quelle di legno. Queste porte sono le porte piu' resistenti del tipo non magico. Sono di solito chiuse a chiave o sbarrate.

\textbf{Sfondare}: Le porte dei dungeon possono essere chiuse a chiave, munite di trappole, rinforzate, sbarrate, sigillate magicamente o, a volte, semplicemente bloccate.

Tutti, ad eccezione dei personaggi piu' deboli, riusciranno a buttar giu' una porta con un pesante attrezzo come un maglio, e numerosi Essenze ed oggetti magici possono offrire ai personaggi un modo facile per superare una porta chiusa.

\textbf{DC 10 o inferiore}: Una porta che chiunque puo' sfondare.

\textbf{DC 11--15}: Una porta che una persona forte dovrebbe sfondare con un solo tentativo, e che una persona di potenza media potrebbe avere qualche speranza di abbattere in un solo colpo.

\textbf{DC 16--20}: Una porta che praticamente chiunque potrebbe sfondare, avendo a disposizione il tempo necessario. 

\textbf{DC 21--25}: Una porta che solo una persona forte o molto forte ha una speranza di sfondare, e probabilmente non al primo tentativo. 

\textbf{DC 26 o superiore}: Una porta che solo una persona dotata di una potenza eccezionale puo' avere una qualche speranza di sfondare.

\textbf{Serrature}: Le porte dei dungeon sono spesso chiuse a chiave e cosi' torna utile l'Abilita' Disattivare Congegni. Le serrature sono di solito costruite sulle porte, sul bordo opposto ai cardini o dritte nel centro della porta. Le serrature costruite dentro le porte di solito controllano una sbarra di ferro che si estende dalla porta dentro il muro che la sostiene, o una sbarra di ferro o di legno massiccio scorrevole che si prolunga dietro tutta la porta. 

Al contrario, i lucchetti non sono costruiti dentro la porta ma di solito scorrono tra due anelli, uno sulla porta e uno sul muro. Serrature piu' complesse, come quelle a combinazione o quelle ad enigma, sono di solito costruite dentro la porta stessa. 

Siccome queste serrature senza chiave sono piu' grandi e complesse, di solito si trovano solo sulle porte resistenti (robuste porte di legno, di pietra o di ferro). 
La DC per scassinare una serratura con una prova di Disattivare Congegni (Criminalita') spesso ricade tra 20 e 30, anche se esistono serrature con DC maggiori o inferiori. Una porta puo' disporre di piu' di una serratura, ognuna delle quali da aprire separatamente.

Le serrature sono spesso dotate di trappole, di solito aghi avvelenati che scattano all'infuori per pungere le dita del ladro.

\textbf{Spaccare una serratura}

Una porta speciale potrebbe avere una serratura senza chiave, ma che richiede che venga indovinata la giusta combinazione delle leve vicine o vengano premuti nell'ordine corretto i simboli su un pannello per riuscire ad aprirla.

\textbf{Porte Bloccate}: I dungeon sono spesso luoghi umidi, e in alcuni casi le porte rimangono bloccate, in modo particolare se sono fatte di legno. Di solito si suppone che all'incirca il 10\% delle porte di legno e il 5\% delle altre porte siano bloccate. Questi valori possono essere raddoppiati (al 20\% e 10\% rispettivamente) nel caso di dungeon da tempo abbandonati o trascurati.

\textbf{Porte Sbarrate}: Quando un personaggio cerca di sfondare una porta sbarrata, e' la qualita' della sbarra che fa la differenza, non il materiale della porta in sé. Sfondare una porta chiusa da una sbarra di legno richiede una prova di Potenza con DC 25, e la DC sale a 30 nel caso di una sbarra metallica.

I personaggi possono attaccare la porta e distruggerla, lasciando la sbarra appesa nel passaggio sgombro.

\textbf{Sigilli Magici}: Essenze di Attacco messe su una porta possono rendere ostico l'attraversamento di una porta.

Una porta su cui e' stato lanciato un blocco magico si considera chiusa anche se non ha fisicamente una serratura. e' necessario una Essenza che scassina o Distruggi magie oppure una prova riuscita di Potenza per oltrepassare una porta chiusa in questo modo.

\textbf{Cardini}: La maggior parte delle porte e' dotata di cardini. Ovviamente le porte scorrevoli non lo sono (queste sono piuttosto dotate di solchi sul pavimento, che permettono loro di scorrere a lato con facilita').

\textbf{Cardini Standard}: Questi cardini sono di metallo e tengono unita la porta al suo sostegno o alla parete. Ricordarsi che la porta si apre verso il lato dove si trovano i cardini (quindi se i cardini sono dal lato dei PG, la porta si aprira' verso di loro; altrimenti si aprira' verso l’altra direzione). 

Gli avventurieri possono rimuovere i cardini uno alla volta superando varie prove di Disattivare Congegni (Criminalita') (solo se, naturalmente, sono davanti al lato della porta su cui si trovano i cardini). Una simile azione ha una DC di 20, in quanto molti dei cardini sono arrugginiti o bloccati. 

Spaccare un cardine e' difficile. La maggior parte ha Durezza 10 e 30 Punti Ferita. La DC per spaccare un cardine e' la stessa che serve per abbattere la porta

\textbf{Cardini a Inserimento}: Questi cardini sono molto piu' complessi e si trovano solo in zone di eccellente costruzione. Questi cardini sono costruiti dentro la parete e permettono alla porta di aprirsi in entrambe le direzioni. I personaggi non possono raggiungere i cardini per rimuoverli a meno che non sfondino il sostegno della porta o la parete. I cardini a inserimento si trovano di solito sulle porte di pietra, ma a volte si vedono anche su porte di legno o di ferro.

\textbf{Perni}: I perni non sono veri cardini, ma semplici pioli che si protendono dal lato superiore e inferiore della porta e si infilano dentro i buchi nel suo sostegno, permettendole di girare. I vantaggi dei perni e' che non possono essere rimossi come i cardini e che sono facili da realizzare. Lo svantaggio e' che siccome la porta gira sul suo centro di gravita' (di solito nel mezzo), nulla piu' grosso di meta' dell'ampiezza della porta vi puo' passare attraverso.

Le porte dotate di perni sono di solito di pietra e spesso anche abbastanza larghe per ovviare allo svantaggio. Un'altra soluzione e' quella di piazzare il perno verso un'estremita' e fare la porta piu' spessa da quella parte e piu' sottile dall'altra, in modo che si apra piu' o meno come una porta normale.

Le porte segrete all'interno di muri spesso ruotano, in quanto la mancanza di cardini rende piu' facile occultare la presenza della porta. I perni permettono anche a oggetti come una libreria di essere usati come porte segrete.

\textbf{Porte Segrete}: Camuffata da comune porzione di muro (o di pavimento o di soffitto), da libreria, da focolare, da fontana, una porta segreta porta ad un passaggio segreto oppure ad una stanza.

Qualcuno che stia esaminando la zona puo' trovare una porta segreta (se ne esiste una) con una prova riuscita di Consapevolezza (con DC 20 per una porta segreta comune e DC 30 per una porta molto ben nascosta).

Molte porte segrete richiedono un metodo speciale per essere aperte, come un bottone nascosto o una piastra a pressione. Le porte segrete possono aprirsi come porte comuni, girare su un perno, scorrere, sprofondare, sollevarsi o anche calare come un ponte levatoio.

Un costruttore potrebbe piazzare una porta segreta molto bassa vicino al pavimento oppure molto in alto su un muro, in modo da rendere piu' difficile sia il rinvenimento che l'utilizzo della porta.

\textbf{Porte Magiche} Incantata dal costruttore originario, una porta puo' apostrofare gli esploratori invitandoli a non proseguire. Potrebbe essere protetta dai danni, con una Durezza maggiore o un numero maggiore di Punti Ferita, oltre che un bonus al Tiro Salvezza migliorato contro Essenze di Distruzione ed effetti simili. Una porta magica potrebbe non condurre allo spazio che si trova dietro di essa, ma essere in realta' un portale verso un luogo molto distante o addirittura verso un altro piano di esistenza. Altre porte magiche potrebbero aver bisogno di una parola d'ordine o di chiavi speciali per aprirsi.

\textbf{Saracinesche}: Queste porte speciali sono fatte con aste di ferro o di spesso legno rinforzato che calano da un recesso nella parte superiore di un arco. A volte una saracinesca dispone di barre orizzontali a formare una griglia, altre volte no. Sollevate di solito con un argano o simile macchinario, le saracinesche possono esser fatte scendere in fretta, e le sbarre terminano in punte per scoraggiare chiunque dal passarci sotto (o dal tentare di attraversarle in corsa mentre calano). Una volta scesa, una saracinesca si chiude, a meno che non sia cosi' grande che nessuna persona normale sarebbe in grado di sollevarla. In ogni caso, sollevare una tipica saracinesca richiede una prova di Potenza con DC 25.

\textbf{Pareti, Porte ed azioni di Individuazione}

Le pareti di pietra, di ferro e le porte di ferro sono generalmente sufficientemente spessi da bloccare la maggior parte delle Essenze di Rivelazione. Le pareti di legno, le porte di legno e di pietra in genere non sono sufficientemente spesse da fare altrettanto. Tuttavia, una porta segreta di pietra costruita in un muro e spessa come il muro stesso (almeno 30 centimetri) blocchera' la maggior parte di queste Azioni.

\textbf{Scale} Il metodo piu' tradizionale per collegare differenti livelli di un dungeon e' attraverso le scale. Un personaggio puo' salire o scendere una scala come parte del suo movimento senza penalita' ma non puo' correre. Aumentate la DC di qualsiasi prova di Acrobatica effettuate su una scala di 4. Alcune scale, particolarmente ripide, vengono trattate come terreno difficile.

\subsubsection{Pericoli nei Dungeon}

Nei dungeon e nelle caverne oltre ai mostri ci sono anche altri pericoli tra crolli, muffe, funghi e altro.

\textbf{Crolli e Cedimenti (CR 8)}

I crolli e i cedimenti nei tunnel sono estremamente pericolosi. Non solo gli esploratori di dungeon corrono il rischio di essere schiacciati da tonnellate di pietra, ma anche, qualora dovessero sopravvivere, di rimanere bloccati sotto un mucchio di detriti o di essere impossibilitati a raggiungere un'uscita.

Un crollo seppellisce chiunque si trovi nel mezzo della zona sepolta, e quindi i detriti che rotolano via infliggeranno danni a tutti coloro che si trovano nelle zone periferiche alla zona sepolta. Un tipico corridoio soggetto a un crollo potrebbe disporre di una zona sepolta con raggio 3 metri e una zona di scorrimento con raggio di mischia all'estremita' di quella sepolta.

Un soffitto pericolante puo' essere identificato con una prova di Conoscenze (ingegneria) con DC 20 o Artigianato (lavori in muratura) con DC 20. Da non dimenticare che le prove di Artigianato possono essere effettuate senza addestramento come prove di Intelligenza. Un Nano puo' effettuare questa prova semplicemente passando entro 3 metri di distanza da un soffitto pericolante. 

Un soffitto pericolante puo' crollare sotto l'impatto di una grossa potenza. Un personaggio puo' provocare un crollo distruggendo la meta' dei pilastri che reggono il soffitto.

I personaggi che si trovano nella zona sepolta subiscono 8d6 danni, o danni dimezzati se superano un Tiro Salvezza su Riflessi con DC 15. A quel punto sono sepolti. I personaggi nella zona di scorrimento subiscono 3d6 danni, o nessun danno se superano un Tiro Salvezza su Riflessi con DC 15. I personaggi che si trovano nella zona di scorrimento, sono anch'essi sepolti, se falliscono il Tiro Salvezza.

I personaggi sepolti subiscono 1d6 danni non letali per ogni minuto che rimangono sotto le macerie. Se un personaggio in queste condizioni cade privo di sensi, deve effettuare una prova di Potenza con DC 15. Se il personaggio fallisce la prova, inizia a subire 1d6 danni letali al minuto fino a quando non viene liberato o muore.

I personaggi che non sono stati sepolti possono estrarre i loro compagni da sotto le macerie. In 1 minuto, usando solo le mani, un personaggio puo' spostare una quantita' di roccia e detriti pari a cinque volte il proprio limite di carico pesante. La quantita' di roccia smossa che riempie un'area di mischia pesa all'incirca 1 tonnellata (1.000 kg). Equipaggiato con gli strumenti adatti, come un piccone, un piede di porco, o una pala, uno scavatore puo' impiegare la meta' del tempo che impiegherebbe facendolo a mano. Si potrebbe anche concedere a un personaggio sepolto di liberarsi da solo superando una prova di Potenza con DC 25.

\textbf{Fanghiglie, Muffe e Funghi}

Negli umidi e oscuri recessi dei dungeon, le muffe e i funghi prosperano. Per quanto riguarda Essenze e altri effetti speciali, tutte le fanghiglie, le muffe e i funghi sono considerati vegetali. Come le trappole, le fanghiglie e le muffe pericolose sono dotate di un CR, e i personaggi guadagnano Punti Esperienza per averle incontrate.

Una lucida melma organica ricopre qualsiasi cosa che rimanga per troppo tempo immersa nell'oscurita' e nell'umidita' dei dungeon. Questo tipo di fanghiglia, benché possa essere repellente, non e' pericoloso. Le muffe e i funghi abbondano nei luoghi bui, freddi e umidi. Sebbene alcuni siano innocui quanto le normali fanghiglie dei dungeon, altri sono alquanto pericolosi. Funghi commestibili, vesce, lieviti, muffe e altri tipi di funghi fibrosi, bulbosi o intere distese di spore fungine possono essere rinvenuti nella maggior parte dei dungeon. Di solito sono innocui e spesso sono anche commestibili (anche se la maggior parte e' poco invitante o ha uno strano sapore).

\textbf{Boleto Stridente}\index{Boleto Stridente}: Questi funghi viola di grandezza umana emettono un suono penetrante che dura 1d3 round ogni volta che c'e' un movimento o una sorgente di luce entro raggio 3 metri. Questo grido rende impossibile sentire altri suoni o rumori entro raggio di mischia. Il suono attira le creature nelle vicinanze che sono disposte ad investigare. Alcune creature che vivono vicino ai boleti stridenti hanno imparato che il rumore significa molto spesso cibo.

\textbf{Fanghiglia Verde}\index{Fanghiglia Verde} (CR 4): Questo pericolo dei dungeon e' una varieta' insidiosa della normale fanghiglia. La fanghiglia verde divora la carne e i materiali organici che vi entrano in contatto, ed e' addirittura capace di dissolvere i metalli. Di un verde splendente, bagnata e appiccicosa, si distribuisce a chiazze su pareti, pavimenti e soffitti e si riproduce consumando materiale organico. Si lascia cadere dalle pareti e dai soffitti quando individua del movimento (e possibile nutrimento) sotto di sé.

La fanghiglia verde infligge 1d3 danni alla Potenza per ogni round in cui divora la carne. Al primo round di contatto, la fanghiglia puo' essere asportata da una creatura (con la probabile distruzione dell'oggetto utilizzato per asportarla), ma dopo il primo round deve essere congelata, bruciata o tagliata (infliggendo danni anche alla sua vittima) per essere rimossa. Tutto cio' che infligge danni da fuoco o da freddo, la luce solare o una Essenza di Cura rimuovi malattia distruggono una chiazza di fanghiglia verde. Nel caso di legno o metallo, la fanghiglia verde infligge 2d6 danni per round, ignorando la Durezza del metallo ma non quella del legno. Non danneggia la pietra.

\textbf{Fungo Fosforescente}\index{Fungo Fosforescente}: Questo strano fungo sotterraneo emana una debole luminescenza violacea che illumina le caverne e i passaggi sotterranei come una candela. Rare macchie di questo fungo illuminano come una torcia.

\textbf{Muffa Gialla} \index{Muffa Gialla}(CR 6): Se disturbata, nel raggio di 3 metri rilascia una nube di spore velenose. Tutti coloro entro raggio di 3 metri dalla muffa devono superare un Tiro Salvezza su Tempra con DC 15 o subiscono 1d3 danni a Potenza. Un altro Tiro Salvezza su Tempra con DC 15 e' necessario una volta per round per i successivi 5 round o per evitare di subire altri 1d3 danni a Potenza. Un Tiro Salvezza riuscito blocca questo effetto. Il fuoco distrugge la muffa gialla, mentre la luce solare la rende inerte.

\textbf{Muffa Marrone} \index{Muffa Marrone}(CR 2): La muffa marrone si nutre di calore, estraendolo da tutto cio' che la circonda. Di solito si presenta in chiazze con diametro di dimensione mischia e la temperatura attorno alla muffa risulta sempre fredda in un raggio di 3 metri. Le creature viventi entro una distanza di mischia da essa subiscono 3d6 danni non letali da freddo. Se viene portata una fonte di fuoco entro mischia dalla muffa questa raddoppia immediatamente le proprie dimensioni. I danni da freddo, come quelli inflitti da un cono di freddo, la distruggono all'istante.

\pagebreak

\section{Pericoli in Avventura}\index{Pericoli in Avventura}


\begin{quotebox}
		{Un'avventura e' un risultato ragionevole. Due sono meglio, tre meritano di essere tramandate, e quattro... nessuno potra' mai contestare quattro avventure. (John Steinbeck)}
\end{quotebox}


\label{pericoli-in-avventura}
\begin{quotebox}Corre meno pericoli colui che, anche se e' al sicuro, sta in guardia. (Publilio Siro)
\end{quotebox}
Il mondo e' pieno di' pericoli oltre che di draghi ed immondi famelici. I pericoli sono minacce basate sulle peculiarita' della zona che hanno molto in comune con le trappole, ma che di solito fanno parte del posto anziché venir costruite. I pericoli si dividono in tre categorie principali: ambientali, viventi e magici.

I pericoli ambientali includono frane, incendi e simili. I pericoli viventi includono creature che pur non essendo considerate mostri, rappresentano una minaccia per gli av­venturieri incauti, come fanghiglie, funghi e muschi. I pericoli magici sono i piu' imprevedibili e possono essere residui di esperimenti arcani, strane radiazioni sotterraneo o antichi Essenze fallite.

\textbf{Antidweomer (CR 6)}\index{Antidweomer}

Zona di entropia magica che distrugge le Essenze, gli antidweomer si formano sui siti di grandi duelli magici, attraverso la distruzione di potenti artefatti o da vortici di energia mistica ai margini delle zone di antimagia. Le dimensioni variano da piccole bolle di appena pochi metri fino a grandi aree delle dimensioni di una citta'. 

Una prova riuscita di Sapienza Magica con DC 20 rivela la vicinanza di un antidweomer con un formicolio nell'aria. Una magia attiva portata in un antidweomer potrebbe venir dissolta, e qualsiasi Essenza lanciata al suo interno e' soggetta ad un controincantesimo immediato (l'incantatore deve fare un Tiro Salvezza su Arbitrio a difficolta' 20). Il conseguente rilascio di energia magica infligge 1d6 danni Comptenza Magia in un'esplosione a raggio mischia centrata su chi ha portato la magia nell'area o chi ne ha lanciato una nuova al suo interno (TS su Riflessi con DC 15 dimezza).

Se piu' scoppi sovrapposti colpiscono lo stesso bersaglio, si applica solo quello piu' dannoso. Una magia che ha resistito ad un tentativo di dissoluzione, non viene influenzato nuovamente a meno che non esca e rientri nell'antidweomer.

Gli antidweomer piu' potenti sono ancora piu' distruttivi. Ogni +1 di incremento del CR aumenta il LI delle prove di dissoluzione di 2 e la DC del Tiro Salvezza per i danni dell'esplosione di 1.

\textbf{Aria Viziata (CR 1 o 4)}\index{Aria Viziata}

Un pericolo invisibile, le sacche di gas sono un rischio per minatori, speleologi e avventurieri che investigano nelle caverne. I gas ininfiammabili come il diossido di carbonio o l'azoto hanno CR 1 e richiedono una prova di Sopravvivenza con DC 25 per essere notati. 

Le creature che respirano quell'aria devono superare un Tiro Salvezza su Tempra (DC 15 +1 per ogni tiro precedente) ogni ora o diventano Affaticate. Una volta Affaticate, iniziano a Soffocare Lentamente. Le creature che trattengono il fiato possono evitare questi effetti.

I vapori infiammabili come il gas di carbone sono molto piu' pericolosi (CR 4). Questo gas sostituisce l'aria respirabile nei polmoni, provocando affaticamento: inoltre, qualsiasi fiamma aperta o scintilla causa un'esplosione che infligge 6d6 danni (TS su Riflessi con DC 15 dimezza) a chi e' nella caverna o entro distanza mischia da un ingresso. Il fuoco brucia l'ossigeno nell'aria, rendendola irrespirabile per 2d4 minuti. Dopo un'esplosione,il gas infiammabile generalmente impiega molti giorni per ritornare a livelli pericolosi.

\textbf{Parassiti}\index{Parassiti}

Parassiti come cercaorecchie o larve necrofaghe provocano parassitosi, un tipo di Afflizione simile alle Malattie. Le parassitosi possono essere guarite solo attraverso trattamenti specifici; indipendentemente da quanti Tiri Salvezza si effettuano, la parassitosi continua ad affliggere il bersaglio. Anche se una Essenza di Cura per Rimuovi Malattia (o un effetto simile) blocca immediatamente una parassitosi, l'immunita' alle Malattie non offre protezione, dato che e' causata da parassiti.

\textbf{Cercaorecchie (CR 5)}\index{Cercaorecchie}

I cercaorecchie sono minuscoli vermi bianchi che vivono nel legno marcio o altri detriti organici. Si possono notare con una prova di Consapevolezza (DC 15). Altrimenti, una creatura vivente che frughi nella loro tana si trasferisce inavvertitamente addosso uno o piu' cercaorecchie, i quali poi cercano una zona calda sul corpo della creatura, prediligendo il condotto uditivo, e li depongono 2d8 uova prima di morire. 

Le uova si schiudono 4d6 ore dopo e le larve divorano la carne intorno. Alla morte del loro ospite, i vermetti strisciano fuori e ne cercano uno nuovo.

Rimuovi Malattia (Cura LP 19) uccide tutti i cercaorecchie o le uova non ancora schiuse su un ospite. Alcuni cercaorecchie preferiscono vivere nel legno corrotto, spesso nascondendosi nelle porte dei sotterranei. I piccoli fori lasciati da questa variante sono molto difficili da notare (Consapevolezza DC 20).

\textbf{Cercaorecchie}

Tipo: Parassitosi

TS: Tempra DC 15

Insorgenza: 4d6 ore

Frequenza: 1/ora

Effetti: 1d3 a Potenza

\textbf{Cristalli Mnemonici (CR 3)}\index{Cristalli Mnemonici}

I cristalli mnemonici sono grandi (3-12 metri d'altezza) grappoli di cristalli di quarzo viola che irradiano un'aura di Distruzione forte. Per identificarli occorre una prova di Conoscenze (arcane) con DC 25.

I cristalli mnemonici cumulano energia magica per crescere e difendersi, risucchiando gli incantesimi preparati degli incantatori che devono effettuare un Tiro Salvezza su Volonta' con DC 22 ogni round mentre sono entro raggio di 3 metri dai cristalli.

Se il tiro fallisce, perdono il 10\% dei Punti Potere a disposizione. Danneggiando o rompendo i cristalli, le magie assorbite vengono espulsi con un'esplosione di energia mentale che infligge 1d3 danni alla Saggezza a tutti coloro che si trovano entro raggio di mischia.

I cristalli mnemonici sono molto fragili (Durezza 0, 1 Punto Ferita).
In aree ricche di cristalli, le creature che vi passano attraverso devono superare una prova di Acrobatica con DC 10 per evitare di camminarci sopra o sfiorarli rompendoli.

\textbf{Larve Necrofaghe (CR 4)}\index{Larve Necrofaghe}

Una volta occupato un corpo vivente, le larve scavano verso il cuore, il cervello e altri organi interni chiave dell'ospite, provocandone infine la morte.

Nel primo round di parassitosi, applicando del fuoco nel foro di ingresso si possono uccidere le larve e salvare l'ospite, ma questo subisce 1d6 danni da fuoco. 

Anche estrarle funziona, ma piu' a lungo le larve restano nell'ospite, piu' danni provoca questo metodo. Per estrarre le larve occorre un'arma tagliente ed una prova di Guarire con DC 20, infliggendo 1d6 danni per ogni round che l'ospite e' stato afflitto da parassitosi. Se la prova di Guarire riesce una larva viene rimossa. Rimuovi Malattia (Cura LP 19) uccide tutte le larve necrofaghe presenti in un ospite.

\textbf{Larve Necrofaghe}

Tipo: Parassitosi

TS: Tempra DC 17

Insorgenza: immediata

Frequenza: 1/round

Effetti: 1d2 danni a Potenza per larva

\textbf{Minerale Magnetizzato (CR 2)}\index{Minerale Magnetizzato}

Le strane energie del mondo sotterraneo possono caricare pietre e vene di minerali con potenti campi magnetici, creando un pericolo per chi porta o indossa metalli ferrosi. Tutte le cose di ferro o acciaio portate entro raggio di 3 metri dal minerale sono trascinate verso di esso.

Le creature Piccole vengono trascinate anche con 7,5 kg di metallo, quelle Grandi solo con 30 kg. Per creature di altre taglie, il peso cambia in base alle regole della Capacita' di Trasporto. Le creature con indosso armature metalliche subiscono una penalita', chi e' colpito e' trascinato fino a 9 metri, subisce 2d6 danni per l'impatto con la roccia ed e' considerato afferrato. Liberare un oggetto colpito richiede una prova di Potenza DC 20-25

\textbf{Polla Maledetta (CR 3)}\index{Polla Maledetta}

I prolungati effetti di antiche maledizioni o l'energia nociva che si propaga da un oggetto magico maledetto sommerso possono trasformare una semplice polla d'acqua in un rischioso pericolo magico. Una polla maledetta attira i passanti nelle sue profondita' attraverso l'illusione (TS su Volonta' con DC 16 per dubitare) di uno sfavillante tesoro sul fondo profondo 3 metri. Qualsiasi creatura che giunga al tesoro attiva la maledizione.

Una creatura all'interno della polladeve superare un Tiro Salvezza su Volonta' con DC 16 o e' colpita dalla maledizione, che distorce la sua percezione della polla. L'acqua sembra addensarsi in un viscoso sapropelite, mentre la polla sembra raggiungere una profondita' di 12 metri.

Le prove di Nuotare (Resistenza) nella polla subiscono penalita' -10, la velocita' viene ridotta alla meta' del normale a causa di questi effetti e lanciare Essenze al suo interno richiede una prova di Concentrazione con DC pari a 20.

Una polla maledetta irradia una forte magia, e puo' essere distrutta da Distruzione Magie o da Protezione Rimuovi Maledizione (prova di livello dell'incantatore
con LP 15).

\textbf{Quercia Velenosa (CR 1 o 3)}\index{Quercia Velenosa}

Il contatto con una quercia velenosa (CR 1) causa una dolorosa eruzione cutanea pruriginosa che rende la vittima Inferma finché i danni non guariscono. Un pieno contatto col corpo o l'inalazione del fumo di una quercia velenosa che brucia potrebbero essere fatali (CR 3). Una prova di Conoscenze (natura) con DC 15 rivela i pericoli insiti nella pianta all'apparenza innocua. Questo pericolo puo' essereusato anche per piante nocive simili (edera velenosa, sommaco velenoso od ortiche pungenti, ma quest'ultime non sono pericolose quando bruciano).

\textbf{Quercia Velenosa}

Tipo: Veleno, contatto

TS: Tempra DC 13

Insorgenza: 1 ora

Effetti: 1d4 danni a Agilita', la creatura e' Inferma finché i danni
non guariscono

Cura: 1 TS

\pagebreak

\subsection{Avventure e Trappole}\index{Trappole}

\label{avventure-e-trappole}
\begin{quotebox}
Chi pone la trappola sempre allo stesso posto non prendera' alcun'iguana. (proverbio africano)
\end{quotebox}

Le trappole sono un pericolo comune nei dungeon. Da sbuffi di vapore bollente a raffiche di dardi avvelenati, le trappole possono servire a proteggere tesori o ad impedire agli intrusi di procedere. 

\textbf{Elementi di una Trappola}

Tutte le trappole, meccaniche o magiche, sono definite da queste peculiarita': CR, tipo, DC di Consapevolezza, DC di Disattivare Congegni, attivatore, ripristino ed effetti. Alcune trappole potrebbero anche includere elementi opzionali, quali i veleni o un tipo di aggiramento. Queste caratteristiche sono descritte sotto.

\textbf{Tipo}

Una trappola puo' essere di natura meccanica o magica.

\textbf{Meccaniche}: I dungeon sono spesso dotati di letali trappole meccaniche (non magiche). Una trappola viene di solito definita dalla sua posizione e dal meccanismo di attivazione, quanto e' difficile notarla prima che venga attivata, quanti danni e' in grado di infliggere, e dal fatto che gli eroi possano compiere o meno un Tiro Salvezza per mitigarne gli effetti. Le trappole che utilizzano frecce, lame affilate e altre armi, effettuano normali Tiri per Colpire, con un bonus di attacco specifico che dipende dal tipo di trappola. Si puo' costruire una trappola meccanica utilizzando con successo l'abilita' Artigianato (costruire trappole). (Vedi Progettare una Trappola piu' avanti e la descrizione della Competenza).

Le creature che superano una prova di Consapevolezza possono individuare una trappola meccanica prima che venga attivata. La DC della prova dipende dalla trappola stessa. In genere il successo indica che la creatura ha individuato il meccanismo di attivazione della trappola, come piastre a pressione, meccanismi collegati a porte e altri tipi di attivazioni insolite. Superare la prova di 5 punti o piu' fornisce anche alcune indicazioni su quello che la trappola e' predisposta a fare.

\textbf{Magica}: Ci sono molte Essenze che possono essere utilizzate per realizzare trappole pericolose. A meno che la descrizione dell'Essenza o dell'oggetto non specifichi altrimenti, e' consigliabile tenere conto dei seguenti punti.

Una prova riuscita di Consapevolezza (DC 28) permette di individuare una trappola magica prima che scatti.

Le trappole magiche concedono un Tiro Salvezza per evitarne gli effetti (DC 15).

Le trappole magiche possono essere disarmate da un personaggio con Scoprire Trappole che superi una prova di Disattivare Congegni (Criminalita') (DC 28). Gli altri personaggi non hanno possibilita' di disarmare una trappola magica.

Le trappole magiche sono a loro volta suddivise in trappole a Essenza e trappole a congegno magico. Le trappole a congegno magico sprigionano degli effetti magici una volta attivate, proprio come le bacchette, le verghe, gli anelli e gli altri oggetti magici. Per creare una trappola a congegno magico e' necessario il talento Creare Oggetti Meravigliosi. 

Le trappole a Essenza non sono altro che Essenze utilizzate come trappole. Per creare una trappola a Essenza sono necessari i servigi di un personaggio che sia in grado di lanciare l'Essenza richiesta, che normalmente e' il personaggio stesso che crea la trappola, oppure un PNG incantatore assunto a tale scopo.

\textbf{DC di Consapevolezza e Disattivare Congegni (Criminalita')}

Il costruttore stabilisce le DC delle prove di Consapevolezza e di Disattivare Congegni (Criminalita') per le trappole meccaniche. Per le trappole magiche, i valori delle DC dipendono dall'incantatore che ha creato la trappola.

\textbf{Trappola meccanica}: \index{Trappola meccanica}Tutte le prove di Consapevolezza e di Disattivare Congegni (Criminalita') hanno una DC base di 20. Aumentare o diminuire una o entrambe le DC modifica il CR della trappola (Tabella: Modificatori al CR delle Trappole Meccaniche). 

\textbf{Trappola Magica}:\index{Trappola Magica} Tutte le prove di Consapevolezza e di Disattivare Congegni (Criminalita') hanno una DC base di 28. Soltanto i personaggi addestrati su Senso Trappola possono effettuare una prova di Disattivare Congegni (Criminalita') su una trappola magica.

\pagebreak

\textbf{Tabella: Modificatori al CR delle Trappole Meccaniche}


\begin{tabular}[c]{@{}ll@{}}
	\toprule 
	Elemento & Modifica al CR\tabularnewline
\textbf{DC Consapevolezza}
& \\
15 o meno & -1
\\
16-20 &-
\\
21-25 &+1
\\
26-29 &+2
\\
30+ &+3
\\
\textbf{DC Disattivare congegni}
&\\
15 o meno &-1
\\
16-20 &-
\\
21-25 &+1
\\
26-29 &+2
\\
30+ &+3
\\
\textbf{Modificatori Tiri Salvezza}
&\\
15 o meno &-1
\\
16-20 &-
\\
21-25 &+1
\\
26-29 &+2
\\
30+ &+3
\\
\textbf{Competenza Armi}
&\\
+0 &-2
\\
+1/+5 &-1\\
+6/+10 &-
\\
+11/+15 &+1
\\
+16/+20 &+2
\\
\bottomrule
Ogni 10 punti di danno medio &+1\\
Ripristino automatico &+1
\\
Attivatore Visivo o di Prossimita’ &+1\\
Veleno &da +1/+10\\
\bottomrule
\end{tabular}

\bigskip

\textbf{Attivatore}

L'attivatore e' il meccanismo che definisce le condizioni che fanno scattare la trappola.

\textbf{Posizione}: Un meccanismo basato sulla posizione fa scattare la trappola quando qualcuno si trova in una zona di mischia predefinita.

\textbf{Prossimita'}: Questo meccanismo fa scattare la trappola quando una creatura si avvicina ad una distanza prestabilita. L'attivatore di prossimita' si differenzia da quello di posizione poiché non e' necessario che la creatura si trovi nella zona di mischia predefinita. Le creature in volo possono far scattare una trappola di prossimita' ma non quelle con un meccanismo di posizione. Gli attivatori di prossimita' meccanici sono estremamente sensibili al minimo spostamento d'aria. Pertanto, le trappole di prossimita' sono particolarmente indicate in quei luoghi come le cripte, dove l'aria e' solitamente stagnante.

L'attivatore di prossimita' usato piu' spesso nelle trappole a congegno magico e' l'Essenza Rivelazione.

\textbf{Sonoro}: Questo attivatore magico fa scattare la trappola quando viene individuato un suono. L'attivatore sonoro funziona come un orecchio dotato di bonus +15 alle prove di Consapevolezza. e' bene notare che questo tipo di attivatore viene ingannato da prove riuscite di Muoversi Silenziosamente, o lanciando un'Essenza di Illusione o di distruzione per ricreare una sorta di silenzio magico o altri effetti che bloccano l'udito. Una trappola con attivatore sonoro richiede l'uso dell'Essenza di Rivelazione costruzione.

\textbf{Visivo}: Questo attivatore magico funziona come un occhio, facendo scattare la trappola quando ``vede'' qualcosa. Il raggio visivo e il bonus alle prove di Consapevolezza dipendono dal potere della Rivelazione usata.

\textbf{Contatto}: In genere, l'attivatore a contatto, che fa scattare la trappola quando viene toccata, e' quello piu' facile da costruire. Questo attivatore puo' essere o meno integrato con il dispositivo che infligge il danno. Si puo' creare un attivatore a contatto magico aggiungendo un'Essenza di Illusione che attivi una sorta di allarme alla trappola e riducendo l'area di effetto fino a selezionare solo il punto di attivazione.

\textbf{A Tempo}: Questo attivatore fa scattare la trappola ad intervalli di tempo prestabiliti.

\textbf{Magia}: Tutte le trappole ad Essenza sono dotate di questo tipo di attivatore. La descrizione delle Essenze spiegano le modalita' di attivazione delle trappole ad attivazione di Essenza. 

\textbf{Durata}
A meno che non sia indicato diversamente, la maggior parte delle trappole ha durata istantanea; una volta attivate, non ci sono altri effetti e terminano di funzionare. Alcune trappole hanno una durata misurata in round. Alcune trappole continuano ad avere gli effetti indicati ad ogni round all'inizio dell'ordine di Iniziativa (o quando sono state attivate, se questo e' avvenuto durante un combattimento).

\textbf{Ripristino}
Il ripristino di una trappola e' l'insieme di condizioni per cui una trappola viene riattivata, pronta a scattare di nuovo. Solitamente per ripristinare una trappolaoccorre un minuto. Per una trappola con un metodo di ripristino piu' complicato, il tempo ed il lavoro richiesti potrebbero aumentare.

\textbf{Irripristinabile}: A meno di ricostruire la trappola, non c'e' modo di farla scattare piu' di una volta. Le trappole ad Essenza non permettono alcun tipo di ripristino.

\textbf{Riparabile:} La trappola puo' funzionare di nuovo, ma deve essere riparata. Riparare una trappola meccanica richiede una prova di Artigianato (costruire trappole) con una DC pari a quella necessaria per costruirla. Il costo deimateriali grezzi e' un quinto del prezzo di mercato della trappola. Per calcolare il tempo necessario a riparare una trappola si deve calcolare il tempo necessario per costruirla, ma utilizzare il costo delle materie prime invece del prezzo di mercato della trappola.

\textbf{Manuale}: Per risistemare la trappola e' necessario che qualcuno rimetta le parti al loro posto. e' il meccanismo di ripristino piu' comune tra le trappole meccaniche.

\textbf{Automatico:} La trappola si ripristina da sé dopo essere scattata ad un intervallo di tempo prestabilito.

\textbf{Aggiramento} (Elemento Opzionale)

Se un personaggio prevede di dover passare nei pressi della trappola che ha costruito o piazzato, e' buona norma costruire un meccanismo di aggiramento che permetta di disarmare temporaneamente la trappola. Gli aggiramenti, in genere, sono abbinati alle trappole meccaniche; le trappole ad Essenza, invece, consentono di specificare delle condizioni intrinseche che permettono all'incantatore diaggirarle. 

\textbf{Serratura}: Una serratura di aggiramento puo' essere aperta con una prova di Disattivare Congegni (Criminalita') con DC 30.

\textbf{Leva Nascosta}: Una leva nascosta puo' essere trovata con una prova di Consapevolezza con DC 25.

\textbf{Serratura Nascosta}: Una serratura di aggiramento nascosta combina le peculiarita' dei precedenti elementi: puo' essere trovata con una prova diConsapevolezza con DC 25 e aperta con una prova di Disattivare Congegni (Criminalita') con DC 30.

\textbf{Effetto}
Gli effetti di una trappola sono cio' che accade a chi la fa scattare. In genere, la trappola infligge danni o sprigiona gli effetti di una Essenza, ma alcune trappole hanno effetti speciali. Una trappola, di norma, effettua un Tiro per Colpire o da' diritto ad un Tiro Salvezza per essere evitata. A volte una trappola utilizza entrambe queste opzioni, altre volte nessuna (vedi Infallibile).

\textbf{Fosse}: Le fosse (coperte o scoperte) sono delle buche all'interno delle quali possono cadere i personaggi e subire danni da caduta. Una fossa non deve effettuare un Tiro per Colpire, ma superare un Tiro Salvezza su Riflessi (DC prestabilita dal costruttore) consente di non caderci dentro. Anche le altre trappole meccaniche che danno diritto ad un Tiro Salvezza rientrano in questa categoria. Le creature che cadono subiscono 1d6 danno per cadute entro 3 metri +1d6 ogni 3 metri di caduta. Le creature che subiscono danni letali da una caduta,
atterranno in posizione prona.
 
Una prova di Acrobatica riuscita con DC 15 permette al personaggio di dimezzare il danno se si cade da meno di 20 metri.

Cadute su superfici morbide (terreno morbido, fango ecc.) convertono i primi 1d6 danni in Danni Non Letali. Questa riduzione e' cumulativa con la diminuzione del danno per l'uso della competenza Acrobatica.

Le fosse presenti nei dungeon possono essere ripartite in tre categorie diverse: scoperte, coperte e baratri. Si possono oltrepassare fosse e baratri con un uso attento di Acrobatica o attraverso vari metodi magici.

Le fosse scoperte servono principalmente a impedire agli intrusi di avanzare verso una direzione, anche se possono provocare guai seri a quegli avventurieri che avanzano al buio, e possono rendere un combattimento in mischia nelle vicinanze assai piu' complicato.

Le fosse coperte sono assai piu' pericolose. Possono essere individuate con una prova di Consapevolezza con DC 20, ma soltanto se i personaggi esaminano attentamente l'area prima di attraversarla. Un personaggio che non riesce a individuare una fossa coperta ha diritto a un Tiro Salvezza su Riflessi con DC 20 per evitare di caderci dentro. Tuttavia, se stava correndo o se camminava senzaguardare, non ha diritto ad alcun Tiro Salvezza, e cade nella fossa automaticamente.

Una trappola puo' essere coperta semplicemente da un cumulo di oggetti (paglia, foglie, rametti, detriti), da un tappeto, oppure da una botola vera e propria costruita per apparire come una normale parte del pavimento.Tale botola solitamente si apre quando su di essa viene esercitato un peso sufficiente a farla scattare (di solito tra i 25 e i 40 kg). I costruttori di trappole piu' infidi a volte costruiscono botole che si richiudono subito dopo essere state aperte, per essere pronte a scattare su una nuova vittima. La botola potrebbe richiudersi achiave una volta scattata, lasciando il personaggio intrappolato incolume, maprigioniero a tutti gli effetti. Aprire una botola simile ha una difficolta' simile a quella richiesta per aprire una porta normale (sempre che il personaggio in questione riesca a raggiungerla) ed e' necessaria una prova di Potenza con DC 13 per tenere aperta una porta che si chiude a scatto.

Le fosse spesso contengono qualcosa di piu' pericoloso del duro pavimento sul fondo. Un costruttore di trappole potrebbe collocarvi spuntoni, mostri, pozze d'acido o di lava, o perfino dell'acqua (considerato che anche una vittima in grado di Nuotare prima o poi si stanchera' e affoghera', se intrappolata a lungo). Per spuntoni e altri elementi vedi Altre Peculiarita' delle Trappole.

A volte nelle fosse vivono dei mostri. Qualsiasi mostro in grado di entrare nella fossa potrebbe essere stato collocato la' dentro dall'ideatore del dungeon, o potrebbe semplicemente esservi caduto per caso senza riuscire ad arrampicarsi fuori.

Una trappola secondaria, meccanica o magica, all'interno di una fossa, puo' rivelarsi particolarmente letale. Se attivata da una vittima caduta nella fossa, la trappola secondaria attacca il personaggio gia' ferito quando meno se lo aspetta.

\textbf{Trappole con Attacco a Distanza}: Queste trappole scagliano dardi, frecce, lance e armi simili contro chiunque le abbia fatte scattare. Il costruttore prestabilisce il Bonus di Attacco della trappola. Una trappola con attacco a distanza puo' essere preparata per simulare gli effetti di un arco composito con un alto valore di potenza; che fornisce alla trappola un bonus ai danni pari al suo punteggio di Potenza. Queste trappole infliggono il danno a seconda del tipo di munizione impiegata. Se una trappola e' costruita con un alto punteggio di Potenza, avra' il corrispondente bonus ai danni.

\textbf{Trappole con Attacco in Mischia}: Queste trappole comprendono lame falcianti che spuntano dalle pareti e blocchi di pietra in caduta dal soffitto. Anche in questo caso, il costruttore prestabilisce il bonus di attacco della trappola. Queste trappole infliggono gli stessi danni delle armi da mischia ``impiegate''. Nel caso di un blocco di pietra in caduta, il Narratore puo' prestabilire un danno contundente a piacere; tuttavia, e' bene ricordare che perrimettere il blocco al suo posto, qualcuno dovra' essere in grado di sollevarlo.

Si puo' costruire una trappola con attacco in mischia con incorporato un bonus ai tiri per i danni, come se la trappola stessa disponesse di un alto punteggio di Potenza.

\textbf{Trappole ad Essenza}: Le trappole ad Essenza producono gli effetti dell'Essenza caricata. Come tutte le Essenze, per ogni trappola ad Essenza che consente un Tiro Salvezza, la DC e' pari a 5+LP caricata nella trappola.

\textbf{Trappole a Congegno Magico}: Queste trappole producono gli effetti ditutte le Essenze che sono state lanciate su di esse, secondo le rispettive descrizioni. Se l'Essenza lanciata su un congegno magico consente un Tiro Salvezza, la DC (in base al tipo di Essenza) del tiro e' 20.

\textbf{Speciale}: Alcune trappole sono dotate di peculiarita' che producono effetti speciali, quali l'annegamento in una fossa piena d'acqua o i danni alle caratteristiche dei veleni. A seconda dei casi, i Tiro Salvezza e i danni dipendono dal tipo di veleno o vengono prestabiliti dal costruttore.

\subsubsection{Altre Peculiarita' delle Trappole}

Alcune trappole sono dotate di peculiarita' opzionali che le rendono decisamente piu' letali. Le peculiarita' piu' comuni sono descritte di seguito:

\textbf{Attacco di Contatto}: Questa peculiarita' si applica alle trappole che colpiscono con un semplice attacco di contatto (in mischia o a distanza) riuscito.

\textbf{Bersagli Multipli}: Le trappole con questa peculiarita' possono aver effetto contemporaneamente su piu' bersagli.

\textbf{Composto alchemico}: Le trappole meccaniche possono incorporare alcuni composti alchemici o altre sostanze e oggetti speciali, quali Borse dell'Impedimento, Fuoco dell'Alchimista, pietre del tuono, e cosi' via. Alcuni di questi oggetti imitano gli effetti di una Essenza. Se l'oggetto riproduce l'effetto di una Essenza, il CR viene modificato come indicato nella Tabella: Modificatori al CR delle Trappole Meccaniche.

\textbf{Danni Ritardat}i: I danni ritardati sono quei danni che vengono inflitti solo dopo che e' trascorso un certo lasso di tempo da quando la trappola e' scattata. Una trappola infallibile infligge danni ritardati. 

\textbf{Fondo della Fossa}: Se in fondo alla fossa c'e' qualcosa di diverso dagli spuntoni, e' piu' semplice trattare questa insidia come una trappola separata (vedi Trappole Multiple) con un attivatore di posizione ad impatto, come nel caso di un personaggio in caduta. 

\textbf{Gas}: I Veleni ad inalazione sono il principale pericolo di una trappola a gas. Le trappole a gas, in genere, hanno le peculiarita' infallibile e danni ritardati.

\textbf{Infallibile}: Quando l'intero dungeon crolla sui personaggi e li seppellisce, neanche i riflessi piu' rapidi possono servire a qualcosa, poiché la mira delle pareti e' infallibile. Una trappola di questo tipo non ha un Bonus di Attacco né da' diritto ad un Tiro Salvezza per essere evitata, ma puo' infliggere danni ritardati. Anche molte trappole di liquido o gas sono infallibili.

\textbf{Liquido}: Tutte le trappole che prevedono un pericolo di annegamento ricadono in questa categoria. Le trappole che sfruttano un elemento liquido di solito sono infallibili e infliggono danni ritardati. 

\textbf{Spuntoni}: Gli spuntoni sul fondo di una fossa sono considerati pugnali, ciascuno con bonus di attacco +10. Il bonus ai danni per ogni spuntone e' +1 ogni per caduta da 1 metro, +2 entro 3 metri, +5 per cadute entro 9 metri, +7 per cadute entro i 12 metri , +10 per cadute oltre i 12 metri. Per cadute oltre i 3 metri considerare anche il danno da caduta.

Ogni personaggio che cade nella fossa e' attaccato da 1d4 spuntoni. Questo danno va aggiunto a quello inferto dalla caduta stessa, e le statistiche presentate sopra sono solo indicative delle trappole piu' comuni: alcune infatti potrebbero avere degli spuntoni piu' pericolosi sul fondo. Gli spuntoni non vengono sommati al danno medio della trappola (vedi Danno Medio, piu' avanti).

\textbf{Veleno}: Le trappole che impiegano Veleno sono molto piu' letali delle rispettive versioni senza veleno, pertanto hanno CR superiori. Per calcolare il modificatore di CR di un Veleno, vedi la Tabella: Modificatori al CR delle Trappole Meccaniche. Soltanto i Veleni che agiscono per contatto, ferimento e inalazione possono essere impiegati per una trappola; quelli ad ingestione no. Alcune trappole infliggono solo i danni da avvelenamento. Altre infliggono anche danni con attacchi a distanza o in mischia.

\subsubsection{Progettare una Trappola}

Progettare una trappola e' semplice. Iniziate col decidere che tipo di trappola volete creare.

\textbf{Trappole Meccaniche}: Selezionate gli elementi di cui si vuole dotare la trappola e aggiungete i modificatori al CR della trappola che tali elementi comportano (vedi Tabella: Modificatori al CR delle Trappole Meccaniche) per ottenere il CR finale di una trappola. Dal CR deriva la DC della prova di Artigianato (costruire trappole) per costruire la trappola (vedi piu' avanti).

\textbf{Trappole Magiche}: Come nel caso delle trappole meccaniche non serve altro che sapere quali elementi andranno a determinare il CR della trappola risultante. Se un personaggio vuole progettare e costruire una trappola magica, deve avere il talento Creare Oggetti Meravigliosi. Inoltre, deve essere in grado di lanciare l'Essenza o Essenze richieste dalla trappola (o, nel caso non sia in grado di farlo, di assoldare un PNG incantatore che lanci l'Essenza per lui).

\textbf{Danno Medio}: Se una trappola (meccanica o magica che sia) infligge danni in punti ferita, si calcola il danno medio di un colpo andato a segno e si arrotonda quel valore al multiplo di 10 piu' vicino. Se la trappola e' ideata per colpire piu' di un bersaglio, si deve moltiplicare questo valore per 2. Se la trappola e' ideata per infliggere danni nel corso di piu' round, si deve moltiplicare questo valore per il numero di round in cui la trappola resta attivata (o la media di essi, se il numero di round e' variabile). Si usa tale valore per modificare il CR della trappola, come indicato nella Tabella: Modificatori al CR delle Trappole Meccaniche. Eventuali danni dai veleni non contano ai fini di determinare tale valore, mentre i danni inferti da spuntoni e attacchi multipli vengono calcolati.

Nel caso di una trappola magica, viene applicato solo un modificatore al CR.

\textbf{Trappole Multiple}: Se una trappola in realta' e' composta da due o piu' trappole collegate tra loro che agiscono piu' o meno sulla stessa area, si determina il CR di ogni trappola separatamente.

\textbf{Trappole Multiple Dipendent}i: Se una trappola dipende dal successo di un'altra (cioe' un personaggio evita direttamente la seconda rappola se riesce asfuggire alla prima), allora i personaggi guadagnano PX per entrambe le trappole superando solo la prima, anche se fanno scattare la seconda.

\textbf{Trappole Multiple Indipendenti}: Se due o piu' trappole agiscono indipendentemente (cioe' nessuna dipende dal successo di un'altra per essere attivata), allora i personaggi guadagnano PX solo per le trappole
che superano.

\textbf{Costo delle Trappole Meccaniche}

Il costo base delle trappole meccaniche e' 1.000 mo \texttimes{} il CR della trappola. Se la trappola usa Essenze per il suo attivatore o ripristino, occorre calcolare questi costi separatamente. Se la trappola non puo' essere ripristinata, bisogna dimezzare il costo. Se ha un ripristino automatico, si aumenta il costo della meta' (+50\%). Le trappole molto semplici, come le fosse, potrebbero costare molto meno, a discrezione del Narratore. Tali trappole non dovrebbero costare piu' di 150 mo \texttimes{} il CR della trappola. 

Dopo aver determinato il costo base per il Grado di Sfida, viene aggiunto il prezzo di eventuali composti alchemici o veleni incorporati nella trappola. Se la trappola utilizza uno di questi elementi e dispone di ripristino automatico, il costo del veleno o del composto alchemico viene moltiplicato per 20 al fine di fornire un numero adeguato di dosi.

\textbf{Trappole multiple}: Se una trappola e' composta in realta' da due o piu' trappole collegate, va determinato il costo finale di ogni trappola separatamente, e poi vengono sommati i valori. Questo vale sia per le trappole multiple dipendenti che per quelle indipendenti.

Mediamente il costo di una trappola e' di 50mo per CR

\pagebreak

\textbf{Esempi di Trappole}

Le seguenti trappole sono solo alcuni esempi delle possibilita' offerte
dalle trappole per sfidare i personaggi.

\begin{multicols}{2}
	
\textbf{Dardo Avvelenato}\\
CR: 1 \\
Tipo: meccanico \\
DC Consapevolezza: 20 \\
DC Disattivare Congegni: 20 \\
Attivatore: contatto \\
Ripristino: nessuno \\
Effetto: Attacco a distanza 12 metri +10 (1d3 piu' Bava fermentata di Lucos)\\

\textbf{Freccia}\\
CR: 1 \\
Tipo: meccanico \\
DC Consapevolezza: 20 \\
DC Disattivare Congegni: 20 \\
Attivatore: contatto \\
Ripristino: nessuno \\
Effetto: Attacco a distanza 12 metri +15 (1d8+1/×3)\\

\textbf{Fossa}\\
CR: 1 \\
Tipo: meccanico \\
DC Consapevolezza: 20 \\
DC Disattivare Congegni: 20 \\
Attivatore: posizione \\
Ripristino: manuale \\
Effetto: fossa profonda 3 metri (2d6 danni da caduta) \\
TS: Riflessi DC 20 evita \\
Bersaglio: bersagli multipli (tutti i bersagli raggio di 3 metri)\\

\textbf{Lama Falciant}e\\
CR: 1 \\
Tipo: meccanico \\
DC Consapevolezza: 20 \\
DC Disattivare Congegni: 20 \\
Attivatore: posizione \\
Ripristino: manuale \\
Effetto: Attacco in mischia +10 (1d8+1/×3) \\
Bersaglio: bersagli multipli (tutti i bersagli in una linea entro 3 metri)\\

\textbf{Fossa con Spuntoni}\\
CR: 2 \\
Tipo: meccanico \\
DC Consapevolezza: 20 \\
DC Disattivare Congegni: 20 \\
Attivatore: posizione \\
Ripristino: manuale \\
Effetto: fossa profonda 3 metri m (1d6 danni da caduta) + spuntoni (Attacco in mischia +10, 1d4 spuntoni per bersaglio per 1d4+2 danni ciascuno) \\
TS: Riflessi DC 20 evita \\
Bersaglio: bersagli multipli (tutti i bersagli in un quadrato di 3 metri di lato)\\

\textbf{Mani Brucianti}\\
CR: 2 \\
Tipo: magico \\
DC Consapevolezza: 26 \\
DC Disattivare Congegni: 26 \\
Attivatore: prossimita' (Allarme) \\
Ripristino: nessuno \\
Effetto: Essenza Attacco (2d4 danni da fuoco) \\
TS: Riflessi DC 11 dimezza \\
Bersaglio: bersagli multipli (tutti i bersagli in un cono di 6 metri di lunghezza e 3 metri di finale)\\

\textbf{Giavellotto}\\
CR: 2 \\
Tipo: meccanico \\
DC Consapevolezza: 20 \\
DC Disattivare Congegni: 20 \\
Attivatore: posizione \\
Ripristino: nessuno \\
Effetto: Attacco a distanza 12 metri +15 (1d6+6), entro raggio 6 metri\\

\textbf{Freccia Acida}\\
CR: 3 \\
Tipo: magico \\
DC Consapevolezza: 27 \\
DC Disattivare Congegni: 27 \\
Attivatore: prossimita' (Allarme) \\
Ripristino: nessuno \\
Effetto: Essenza Attacco a distanza di 16 metri (2d4 danni da acido per 4 round)\\

\textbf{Fossa Celata}\\
CR: 3 \\
Tipo: meccanico \\
DC Consapevolezza: 25 \\
DC Disattivare Congegni: 20 
Attivatore: posizione \\
Ripristino: manuale \\
Effetto: fossa profonda media (3d6 danni da caduta) \\
TS: Riflessi DC 20 evita \\
Bersaglio: bersagli multipli (tutti i bersagli in un quadrato di 3 metri di lato)\\

\textbf{Arco Elettrico}\\
CR: 4 \\
Tipo: meccanico \\
DC Consapevolezza: 25 \\
DC Disattivare Congegni: 20 \\
Attivatore: contatto \\
Ripristino: nessuno \\
Effetto: Essenza Attacco (Arco elettrico, 4d6 danni da elettricita' )\\
TS: Riflessi DC 20 dimezza \\
Bersaglio: bersagli multipli (tutti i bersagli in una linea a distanza 6 metri)\\

\textbf{Falce a Parete}\\
CR: 4 \\
Tipo: meccanico \\
DC Consapevolezza: 20 \\
DC Disattivare Congegni: 20 \\
Attivatore: posizione \\
Ripristino: automatico \\
Effetto: Attacco in mischia +20 (2d4+6/×4)\\

\textbf{Blocco in Caduta}\\
CR: 5 \\
Tipo: meccanico \\
DC Consapevolezza: 20 \\
DC Disattivare Congegni: 20 \\
Attivatore: posizione \\
Ripristino: manuale \\
Effetto: Attacco in mischia +15 (6d6) \\
Bersaglio: bersagli multipli (tutti i bersagli in un quadrato di 3 metri di lato)\\

\textbf{Aria infuocata}\\
CR: 5 \\
Tipo: magico \\
DC Consapevolezza: 28 \\
DC Disattivare Congegni: 28 \\
Attivatore: prossimita' (Allarme) \\
Ripristino: nessuno \\
Effetto: Essenza Attacco (6d6 danni da fuoco, distanza media)\\
TS: Riflessi DC 14 dimezza \\
Bersaglio: bersagli multipli (tutti i bersagli in un’esplosione di raggio 3 metri)\\

\textbf{Colpo Infuocato}\\
CR: 6 \\
Tipo: magico \\
DC Consapevolezza: 30 \\
DC Disattivare Congegni: 30 \\
Attivatore: prossimita' (Allarme) \\
Ripristino: nessuno \\
Effetto: Essenza Attacco (8d6 danni da fuoco, distanza media)\\
TS: Riflessi DC 17 dimezza \\
Bersaglio: bersagli multipli (tutti i bersagli in un cilindro di raggio 3 metri)\\

\textbf{Freccia Avvelenata}\\
CR: 6 \\
Tipo: meccanico \\
DC Consapevolezza: 20 \\
DC Disattivare Congegni: 20 \\
Attivatore: posizione \\
Ripristino: nessuno \\
Effetto: Attacco a distanza 18 metri +15 (1d6 piu' Veleno ×3)\\

\textbf{Zanne Gelide}\\
CR: 7 \\
Tipo: meccanico \\
DC Consapevolezza: 25 \\
DC Disattivare Congegni: 20 \\
Attivatore: posizione \\
Durata: 3 round \\
Ripristino: nessuno \\
Effetto: Essenza Attacco distanza 3 metri (spruzzo di acqua gelata, 3d6 danni da freddo) \\
TS: Riflessi DC 20 dimezza \\
Bersaglio: bersagli multipli (tutti i bersagli in una stanza di 3x3x3 metri)\\

\textbf{Trappola a Gas}\\
CR: 8 \\
Tipo: meccanico \\
DC Consapevolezza: 25 \\
DC Disattivare Congegni: 20 \\
Attivatore: posizione \\
Ripristino: riparabile \\
Effetto: Gas velenoso \\
Bersaglio: bersagli multipli (tutti i bersagli che si trovano in una stanza 3x3x3 metri)\\

\textbf{Raffica di Frecce}\\
CR: 9 \\
Tipo: meccanico \\
DC Consapevolezza: 25 \\
DC Disattivare Congegni: 25 \\
Attivatore: visivo ( Occhio Arcano) \\
Ripristino: riparabile \\
Effetto: Attacco a distanza +20 (6d6) \\
Bersaglio: bersagli multipli (tutti i bersagli in una linea di 6 metri)\\

\textbf{Fossa Celata con Spuntoni}\\
CR: 8 \\
Tipo: meccanico \\
DC Consapevolezza: 25 \\
DC Disattivare Congegni: 20 \\
Attivatore: posizione \\
Ripristino: manuale \\
Effetto: Fossa profonda 15 m (5d6 danni da caduta) + spuntoni (Attacco in mischia +15, 1d4 spuntoni per bersaglio per 1d6+5 danni ciascuno) \\
TS: Riflessi DC 20 evita \\
Bersaglio: bersagli multipli (tutti i bersagli in un cubo con lato 3x3x3 metri)\\

\textbf{Pavimento Folgorante}\\
CR: 9 \\
Tipo: magico \\
DC Consapevolezza: 26 \\
DC Disattivare Congegni: 26 \\
Attivatore: prossimita' (Allarme) \\
Durata: 1d6 round \\
Ripristino: nessuno \\
Effetto: Essenza Attacco (Attacco di contatto in mischia +9, 4d6 danni da elettricita') 
Bersaglio: bersagli multipli (tutti i bersagli in una stanza di 6x6x3 metri)\\

\textbf{Risucchio di Energia}\\
CR: 10 \\
Tipo: magico \\
DC Consapevolezza: 34 \\
DC Disattivare Congegni: 34 \\
Attivatore: visivo (Visione del Vero) \\
Ripristino: nessuno \\
Effetto: Essenza Distruzione (Attacco di contatto a distanza 18 metri +10, 2d4 Livelli Negativi Temporanei) \\
TS: Tempra DC 23 nega dopo 24 ore\\

\textbf{Stanza di Lame}\\
CR: 10 \\
Tipo: meccanico \\
DC Consapevolezza: 25 \\
DC Disattivare Congegni: 20 \\
Attivatore: posizione \\
Durata: 1d4 round \\
Ripristino: riparabile \\
Effetto: Attacco in mischia +20 (3d8+3) \\
Bersaglio: bersagli multipli (tutti i bersagli che si trovano in una stanza di 3x3x3 metri)\\

\textbf{Cono di Schegge di Ghiaccio}\\
CR: 11 \\
Tipo: magico \\
DC Consapevolezza: 30 \\
DC Disattivare Congegni: 30 \\
Attivatore: prossimita' (Allarme) \\
Ripristino: nessuno \\
Effetto: Essenza Attacco (cono di lance di ghiaccio, 15d6 danni da freddo) \\
TS: Riflessi DC 17 dimezza \\
Bersaglio: bersagli multipli (tutti i bersagli in un cono di 18 metri di lunghezza e 6 metri finali)\\

\textbf{Lancia Mortale}\\
CR: 18 \\
Tipo: meccanico \\
DC Consapevolezza: 30 \\
DC Disattivare Congegni: 30 \\
Attivatore: visivo\\
Ripristino: manuale \\
Effetto: Attacco a distanza 36 metri +20 (1d8+6 piu' veleno)\\

\textbf{Inferno di fuoco}\\
CR: 13 \\
Tipo: magico \\
DC Consapevolezza: 31 \\
DC Disattivare Congegni: 31 \\
Attivatore: prossimita' (Allarme) \\
Ripristino: nessuno \\
Effetto: Essenza Attacco (60 danni da fuoco) \\
TS: Riflessi DC 14 dimezza \\
Bersaglio: bersagli multipli (tutti i bersagli in un’esplosione di 6 metri di raggio)\\

\textbf{Masso Schiacciante}\\
CR: 15 \\
Tipo: meccanico \\
DC Consapevolezza: 30 \\
DC Disattivare Congegni: 20 \\
Attivatore: posizione \\
Ripristino: manuale \\
Effetto: Attacco in mischia +15 (16d6) \\
Bersaglio: bersagli multipli (tutti i bersagli in un quadrato di 3 metri di lato)\\

\textbf{Attacco Potenziato}\\
CR: 16 \\
Tipo: magico \\
DC Consapevolezza: 33 \\
DC Disattivare Congegni: 33 \\
Attivatore: visivo (Visione del Vero) \\
Ripristino: nessuno \\
Effetto: Essenza Attacco (+9 contatto a distanza 18 metri, 30d6 danni, TS: Tempra DC 19 riduce a 5d6 danni)\\

\textbf{Ferimento}\\
CR: 14 \\
Tipo: magico \\
DC Consapevolezza: 31 \\
DC Disattivare Congegni: 31 \\
Attivatore: contatto \\
Ripristino: nessuno \\
Effetto: Essenza Distruzione (Sanguinamento 6, attacco di contatto in mischia +6)\\
TS: Volonta' DC 19 annulla\\

\textbf{Galleria di Fulmini}\\
CR: 17 \\
Tipo: magico \\
DC Consapevolezza: 29 \\
DC Disattivare Congegni: 29 \\
Attivatore: prossimita' (Allarme) \\
Durata: 1d6 round \\
Ripristino: nessuno \\
Effetto: Essenza Attacco (8d6 danni da elettricita') \\
TS: Riflessi DC 16 dimezza \\
Bersaglio: tutti i bersagli in una stanza di 12x3x3 metri\\

\textbf{Fossa Avvelenata}\\
CR: 12 \\
Tipo: meccanico \\
DC Consapevolezza: 25 \\
DC Disattivare Congegni: 20 \\
Attivatore: posizione \\
Ripristino: manuale \\
Effetto: Fossa profonda 15 m (5d6 danni da caduta) + spuntoni (attacco in mischia +15, 1d4 spuntoni per bersaglio per 1d6+5 danni ciascuno piu' veleno)\\
TS: Riflessi DC 25 evita \\
Bersaglio: bersagli multipli (tutti i bersagli in un quadrato di 3x3 metri)\\

\textbf{Sciame di Meteore}\\
CR: 19 \\
Tipo: magico \\
DC Consapevolezza: 34 \\
DC Disattivare Congegni: 34 \\
Attivatore: visivo\\
Ripristino: nessuno \\
Effetto: Essenza Attacco (4 meteore a bersagli separati, +9 contatto a distanza 27 metri, 2d6 da impatto piu' 6d6 danni da fuoco)\\
TS: Riflessi DC 23 dimezza danni da fuoco\\
Bersaglio: bersagli multipli (quattro bersagli, due dei quali non possono trovarsi ad una distanza superiore ai 12m l’uno dall’altro)\\

\textbf{Distruzione}\\
CR: 20 \\
Tipo magico \\
DC Consapevolezza: 34 \\
DC Disattivare Congegni: 34 \\
Attivatore: prossimita' (Allarme) \\
Ripristino: nessuno \\
Effetto: Essenza Distruzione (TS Morte)\\
TS: Tempra DC 23 riduce a 5d12 danni altrimenti 10d12\\

\end{multicols}

\pagebreak


\section{Veleni e Pozioni}\index{Veleni}\index{Pozioni}

\label{veleni-e-pozioni}


\begin{quotebox}
Un giorno, un uomo fu colpito da una freccia avvelenata. Gli amici e i parenti, in ansia, chiamarono un medico. Quando gli si avvicinarono per prendere la freccia, l'uomo disse loro: "Prima di farlo, vorrei sapere chi mi ha trafitto con questa freccia... Era uno schiavo, un re, o un bramino? Era grande? Piccolo? Di che colore era la sua pelle? Dove viveva? E la freccia com'e' stata costruita? Quale veleno e' stato impiegato? ..." Mentre si stava ponendo tutte queste domande... il veleno fece il suo effetto e l'uomo ferito fini' per morire. (Budda)
\end{quotebox}


Dal morso di una vipera alla lama avvelenata di un assassino, il veleno e' una costante minaccia. I veleni possono essere curati con Tiro Salvezza su Tempra ed Essenze di Cura

\bigskip

\textbf{Tipo di Veleno e Pozione}

\textbf{Contatto}: sono contratti nel momento in cui qualcuno tocca il veleno con la pelle nuda. Tali veleni possono essere usati come veleni da ferimento. I veleni a contatto hanno solitamente un tempo di insorgenza di 1 round. Un veleno a contatto puo' essere un unguento, balsamo, liquido di qualsiasi densita' o anche polvere se specifica per contatto e non inalazione.

\textbf{Ingestione}: si attivano quando una creatura li mangia o li beve. I veleni ad ingestione hanno solitamente un tempo di insorgenza di 10 minuti.

\textbf{Ferimento}: vengono trasferiti soprattutto con gli attacchi di alcune creature e tramite armi cosparse di veleno. I veleni a ferimento non hanno solitamente un tempo di insorgenza. 

\textbf{Inalazione}: si attivano nel momento in cui una creatura entra in un'area che contiene tali veleni. Molti veleni ad inalazione riempiono un volume pari ad un cubo con spigolo di 3x3x3 metri per dose. Le creature possono tentare di trattenere il fiato mentre si trovano all'interno dell'area per evitare di inalare la tossina. Le creature che trattengono il fiato hanno devono fare una prova su Potenza (3d6+bonus Potenza) a difficolta' 10 ogni round per non inalare il gas. Ogni round in cui si trattiene il fiato la prova di difficolta' aumenta di 1.

Vedi anche le regole per trattenere il fiato e soffocare in Ambiente.

\textbf{Insorgenza ed Effetto}

Per insorgenza si intende quanto tempo ci mette il veleno o pozione a fare effetto. Se il tempo di insorgenza e' 1 turno significa che per gli effetti del veleno/pozione il Tiro Salvezza lo si effettua dopo 10 minuti. Se nella tabella del veleno/pozione insorgenza non e' specificata significa che l'effetto e' immediato dopo l'entrata in contatto con il veleno.

L'effetto di un veleno/pozione e' immediato dopo l'insorgenza. Verificare la descrizione del veleno per capirne l'effetto. Se il Tiro Salvezza riesce il veleno non ha fatto effetto e si puo' ritenere neutralizzato. 

\bigskip

\textbf{Avvelenati}\index{Avvelenati}

\textbf{Prima dose}: Quando si viene esposti a un veleno per la prima volta (durante la propria azione o quella di qualcun altro), e' necessario effettuare un Tiro Salvezza per evitare di venire avvelenati. 

\textbf{Successo}: Si resiste al veleno. Non si subiscono effetti negativi e non sono necessari ulteriori Tiri Salvezza.

\textbf{Fallimento}: Siete stati avvelenati e si subisce subito l'effetto elencato.

\textbf{Piu' dosi}: Se si vieni esposti a piu' dosi dello stesso veleno nello stesso round la difficolta' del TS aumenta di 1 per dose aggiuntiva.

\textbf{In tempi diversi}: se si viene esposti al veleno in tempi diversi, ogni volta ci sara' un nuovo Tiro Salvezza e si subiranno gli eventuali effetti nei tempi previsti.

\bigskip

\textbf{Applicare il Veleno}\index{Applicare il Veleno}

Applicare il veleno ad un'arma o ad una munizione richiede 3 azioni.

Ogni volta che un personaggio applica o prepara un veleno per l'uso deve tirare 3d6+Intelletto e se ottiene come somma 3 o 4 e' entrato in contatto con il veleno e deve effettuare un Tiro Salvezza contro il veleno come di norma. Cio' non consumala dose di veleno. 

Ogni volta che un personaggio attacca con un'arma avvelenata, se ottiene un 3 o 4 naturale col Tiro per Colpire, si espone agli effetti del veleno. Cio' consuma il veleno sull'arma.

Un pozione di veleno e' sufficiente per coprire di veleno un arma media oppure 3 frecce. Il veleno viene cosi' consumato e rimane attivo sull'arma finche' questa non colpisce.

\textbf{Creare Veleni}\index{Creare Veleni}

I veleni possono essere realizzati usando Artigianato (alchimia) o Conoscenze (Erboristeria). La DC per preparare un veleno e' uguale alla DC del Tiro Salvezza su Tempra che richiede -5. Il costo per preparare un veleno e' pari alla meta' del costo di vendita.

Ottenendo un 3 o 4 naturale con la prova di Artigianato o Erboristeria ci si espone al veleno durante la sua preparazione. Il tempo necessario per preparare i veleni e' pari alla DC in ore. 

Gli esempi seguenti rappresentano solo alcuni dei possibili veleni.

\bigskip

\begin{longtable}{|k{4cm} |k{1cm} |k{1.5cm}| k{1.5cm} |k{5.5cm} |k{1.5cm}|}
	Nome Veleno & Uso & Tempra & Ins. & Effetto (danno) & Costo (mo)\tabularnewline
	\textbf{Nocciolo di Dennar} \index{Nocciolo di Dennar} & I & 13 & 1 turno & -1d2 Potenza, per 3gg & 15\tabularnewline
	\textbf{Succo di Ythis}\index{Succo di Ythis} & I & 14 & 1 turno & -1d2 Intelletto, per 1g & 20\tabularnewline
	\textbf{Sangue di Thrun} \index{Sangue di Thrun} & C & 26 & - & -1d3 Potenza & 80\tabularnewline
	\textbf{Erba puntuta rosa} \index{Erba puntuta rosa} & I & 22 & 1 turno & -1d6 Agilita' & 60\tabularnewline
	\textbf{Dita di Daraka}\index{Dita di Daraka} & F & 17 & - & -1d6 Potenza, per 10 minuti & 35\tabularnewline
	\textbf{Polline di Rosa di Omro}\index{Polline di Rosa di Omro} & I & 15 & - & -1d3 Potenza e Agilita',per 1 ora & 25\tabularnewline
	\textbf{Fumi di Curna} \index{Fumi di Curna}& R & 18 & - & -1d3 Volonta' & 40\tabularnewline
	\textbf{Olio di Nabar} \index{Olio di Nabar} & R-F & 20 & - & Confuso per 2d6 round & 50\tabularnewline
	\textbf{Bacche Azzurre di fosso} \index{Bacche Azzurre di fosso} & I & 21 & 1 turno & -1d3 Intelletto e Volonta' per 6 ore & 55\tabularnewline
	\textbf{Pelle di Rospo Azzurro} \index{Pelle di Rospo Azzurro} & C & 22 & 1 minuto & Paralizzato per 1d6 turni & 60\tabularnewline
	\textbf{Cenere di Corteccia Gialla} \index{Cenere di Corteccia Gialla} & F & 15 & 6 round & Privo di sensi per 1d3 ore & 25\tabularnewline
	\textbf{Fiocco bianco di Mucot} \index{Fiocco bianco di Mucot}& C & 20 & - & Dorme per 2d12 ore & 20\tabularnewline
	\textbf{Bava fermentata di Lucos} \index{Bava fermentata di Lucos} & F & 15 & - & 1d8 PF & 25\tabularnewline
	\textbf{Bacca Viola di Barsar}\index{Bacca Viola di Barsar} & I & 18 & 1 turno & Incapace di eseguire azioni violente per 3d8 ore & 40\\
	\textbf{Lingua di Kreex} \index{Lingua di Kreex} & F & 20 & - & La ferita sanguina. +1 danno da sanguinamento per round per 2 minuti Max +5 sanguinamento & 50 \\
	\textbf{Fegato di Toporagno Viola} \index{Fegato di Toporagno Viola}& I & 25 & 1 ora & 2d6 di danno a Volonta' e Intelletto. Permanente & 75\\
	\textbf{Muschio Giallo} \index{Muschio Giallo} & I & 20 & 1 round & la creatura guadagna una taglia. Sovradosaggi sono possibili. Durata 10 minuti & 50\tabularnewline
	\textbf{Veleno di Serpe del Sangue} \index{Veleno di Serpe del Sangue} & F & 25 & - & Paralisi per 1d6 ore -1d4 punti Potenza per 7 giorni & 75\\
	\textbf{Profumo di Ragmor} \index{Profumo di Ragmor}& R & 16 & - & -1d3 Magnetismo, per 1 giorno & 30\tabularnewline
	\textbf{Grasso di Toporagno Viola} \index{Grasso di Toporagno Viola} & C & 13 & 1 round & 2d12 PF & 15\tabularnewline
	\textbf{Veleno di Ottalm}\index{Veleno di Ottalm} & F & 20 & - & Morte o -1d2 Potenza permanente. & 50\tabularnewline
\end{longtable}


Applicazione: \textbf{I}(ngestione), \textbf{F}(erimento),\textbf{C}(ontatto),\textbf{R}(espirazione)
\bigskip

I punti caratteristica persi si recuperano al ritmo di 1 al giorno se non indicato diversamente.

\pagebreak

\subsection{Pozioni naturali}\index{Pozioni}

Il tempo per preparare queste pozioni/droghe e' pari alla DC/2 in ore, mentre la difficolta' e' pari alla DC -5. Se gli ingredienti si comprano (e non si trovano spontaneamente in natura) il costo per preparare la pozione e' meta' del costo di vendita indicato.

Se la prova di DC (Cultura, Erboristeria) ha successo se ne preparano 1d3 dosi.

\bigskip

\begin{longtable}{|L{3.2cm} |k{0.7cm} |k{1.7cm}| D{1cm} |L{5cm} |k{1.5cm} |k{1cm}|}
	Nome & Uso & Insorgenza & DC & Effetto & Localita' & Costo\\
	\textbf{Arlandas}\index{Arlandas} & R & 1 ora & 24 & Rinsalda le fratture & CF5 & 100\\
	\textbf{Burthelas} \index{Burthelas} & I & 1 turno & 32 & Rigenera le mani & HD7 & 410\\
	\textbf{Musekiss} \index{Musekiss}& C & 1 ora & 30 & Rigenera arti inferiori & TH9 & 550\\
	\textbf{Bacche di Ljust} \index{Bacche di Ljust} & I & 1 round & 16 & Preso la sera recuperi il doppio dei PF (minimo 4)& AZ6 & 30\\
	\textbf{Culcoa} \index{Culcoa}& C & 1 round & 16 & Recuperi 2d6 da danno da fuoco & TS7 & 30\\
	\textbf{Jojopo} \index{Jojopo} & C & 1 round & 15 & Recuperi 2d6 da danno da ghiaccio & FM6 & 25\\
	\textbf{Kelventare}\index{Kelventare} & I & 1d4 round & 28 & Recuperi 2d6 & TT7 & 90\\
	\textbf{Harfy} \index{Harfy} & C & - & 12 & Interrompe il sanguinamento & SS6 & 10\\
	\textbf{Arlan}\index{Arlan} & C & - & 15 & Cura 1d6+3 PF & TT5 & 25\\
	\textbf{Darsurion}\index{Darsurion} & C & 1 round & 25 & Cura 1d4 PF & CM4 & 75\\
	\textbf{Draaf} \index{Draaf} & C & 1 round & 20 & Cura 1d8 PF & SO6 & 50\\
	\textbf{Garioe}\index{Garioe} & I & 1 round & 25 & Cura 2d6 PF & AZ7 & 75\\
	\textbf{Geffnull} \index{Geffnull}& I & 5 round & 28 & Cura 3d8+3 PF & EV8 & 90\\
	\textbf{Mirenna}\index{Mirenna} & I & 1 round & 20 & Cura 5 PF & CM6 & 50\\
	\textbf{Rewky}\index{Rewky} & I & - & 25 & Cura 2d8 PF & TD6 & 75\\
	\textbf{Wickalim} \index{Wickalim} & I & - & 15 & Cura 2 PF & TD4 & 25\\
	\textbf{Lingua Rossa di Xabax}\index{Lingua Rossa di Xabax}& C & 1 turno & 20 & Cura 2d6 PF ma se c'e' malattia o veleno la rimuove ma causando 2d6 di danno& TA7 & 50\\
	\textbf{Yaveth} \index{Yaveth} & I & - & 20 & Cura 2d8 PF & MO5 & 50\\
	\textbf{Bacio di Ljust} \index{Bacio di Ljust} & C & 1 round & 35 & Cura 100 PF & HO8 & 125\\
	\textbf{Polline di Rosa Verde} \index{Polline di Rosa Verde} & R & 3 turni & 25 & Recuperi 2d4 danni Intelletto e Volonta' & FA8 & 75\\
	\textbf{Arkasun}\index{Arkasun} & C & - & 25 & Cura 1d6 PF a turno per 3 turni & MT7 & 75\\
	\textbf{Attarna} \index{Attarna} & I & 1 turno & 20 & Concede un nuovo TS per Malattie con un +4& TF7 & 50\\
	\textbf{Delrean} \index{Delrean} & C & 1 round & 15 & Allontana insetti per 1 giorno & CC6 & 2\\
	\textbf{Delrean Plus}\index{Delrean Plus} & I & 1 round & 18 & Allontana insetti per 3 giorni & CC6 & 5\\
	\textbf{Melandrir} \index{Melandrir} & I & 1 round & 15 & Concede un nuovo TS per Malattie con +5 & CF7 & 25\\
	\textbf{Uovo di Urk} \index{Uovo di Urk} & I & 1 turno & 12 & 1 giorno di cibo & FH7 & 2\\
	\textbf{Barannie} \index{Barannie} & I & - & 15 & Rimuove nausea & MD6 & 10\\
	\textbf{Eldrin'tail} \index{Eldrin'tail} & I & - & 15 & Concede un nuovo TS su Veleni & FH7 & 25\\
	\textbf{Harlindar}\index{Harlindar} & I & 1 turno & 15 & Fa abortire & SS7 & 25\\
	\textbf{Klandor} \index{Klandor} & I & - & 15 & Rimuove paralisi & HB6 & 25\\
	\textbf{Klynkyx} \index{Klynkyx} & C & 1 turno & 15 & Fa cadere tutti i capelli per 1d6+4 gg & MO6 & 8\\
	\textbf{Arduuar} \index{Arduuar} & I & 1 round & 25 & Rimuove Veleni & SZ7 & 75\\
	\textbf{Nazamuse} \index{Nazamuse}& I & - & 30 & Rimuove Veleni e Malattie & EW9 & 100\\
	\textbf{Nelthalion} \index{Nelthalion}& I & - & 15 & Fa vomitare & SR3 & 1\\
	\textbf{Ferpillon}{*} \index{Ferpillon}& I & 1 round & 30 & Fa dormire per 24 ore & SC5 & 50\\
	\textbf{Uscaboo} \index{Uscaboo}& R & 1 turno & 25 & Rimuove cecita' & MO7 & 75\\
	\textbf{Ucsaboo} \index{Ucsaboo} & C & 1 turno & 30 & Rigenera occhi & MO8 & 200\\
	\textbf{Febfendi} \index{Febfendi} & C & 1 turno & 25 & Rigenera orecchie & CF7 & 75\\
	\textbf{Siranmuse}\index{Siranmuse} & I & 1 giorno & 30 & Rigenera organi interni & SS8 & 350\\
	\textbf{Klagul} \index{Klagul} & C & 1 turno & 20 & & SS4 & 30\\
	\textbf{Corteccia di Aklent}\index{Corteccia di Aklent} & I & 1 turno & 10 & La corteccia masticata per almeno 10 round concede per le 24 ore successive un +1 TS vs Veleno& MT6 & 5\\
	\textbf{Petali di Lisbeth} \index{Petali di Lisbeth} & I & 1 turno & 15 & Cura tosse e raffreddore & MC6 & 20\\
	\textbf{Estratto di radice Gisenosa}\index{Estratto di radice Gisenosa} & I & - & 15 & +2 Intelletto, -2 Agilita' per 10 minuti & MT6 & 5\\
	\textbf{Gylvert} \index{Gylvert}& I & - & 25 & Concede respirare sott'acqua per 4 ore & MO7 & 75\\
	\textbf{Gusterbloon} \index{Gusterbloon} & C & 1 round & 20 & & CM5 & 40\\
	\textbf{Unto Grigio}{*} \index{Unto Grigio} & I & 1 round & 24 & Rimuove condizionamenti mentali fino a LP 21 & AH9 & 80\\
	\textbf{Lievito di Muschio Bianco} \index{Lievito di Muschio Bianco}& I & - & 12 & I prodotti da forno che usano questo lievito causano meteorismo incontrollabile ed incredibilmente puzzolente per 12 ore & CA3 & 1\\
	\textbf{Foglie fermentate di Luside*}\index{Foglie fermentate di Luside}& I & 10 round & 17 & Allucinazioni sensoriali per 2d4 ore. +2 Magnetismo ed Intelletto & SF7 & 4\\
	\textbf{Estratto di Bacca Illa bruciata}\index{Estratto di Bacca Illa bruciata}& I & - & 15 & +2 Iniziativa, +2 Agilita' , -4 TS Arbitrio, per 10 minuti& MS6 & 5\\
	\textbf{Cenere di Arpasur}{*} \index{Cenere di Arpasur} & R & 1 round & 20 & Rimuove condizione di affaticato & FT6 & 1\\
	\textbf{Corteccia polverizzata di Dagmather} \index{Polvere di corteccia di Dagmather}& R & 1 round & 25 & Rimuove condizione esausto e affaticato & SS5 & 2\\
	\textbf{Radice secca di Kathaus}\index{Radice secca di Kathaus} & R & - & 20 & +2 Potenza e Agilita' per 1 ora & FW6 & 25\\
	\textbf{Carne secca di Ragno Viola*} \index{Carne secca di Ragno Viola} & I & 1 round & 24 & +4 Potenza -4 Intelletto per 1 turno & SH7 & 30\\
	\textbf{Estratto alcolico di Melzaa*}\index{Estratto alcolico di Melzaa}& I & - & 20 & +1d4 Potenza , +1d4 Agilita' . -4 TS su Arbitrio. Per 3 ore& AF6 & 25\\
	\textbf{Essenza profumata di Inut*}\index{Essenza profumata do Inut} & R & - & 15 & +2 Intelletto, per 1d8 ore & HB6 & 15\\
	\textbf{Polline di Julnnaus}\index{Polline di Julnnaus}{*} & R & - & 20 & +3 Potenza per 2 ore & FO6 & 25\\
	\textbf{Miele polverizzato del fiore di Erain*} \index{Miele polverizzato del fiore di Erain}& R & - & 20 & +2 Potenza e Intelletto e Agilita'. +3d6 PF temporanei, per 1 ora& FT7 & 25 \\
\end{longtable}

Le pozioni con {*} danno dipendenza. Terminato l'effetto effettuare
un Tiro Salvezza su Arbitrio a difficolta' 15 o prenderne un altra
dose, il successivo Tiro Salvezza avra' difficolta' +1 e cosi'' via.

Ogni qual volta si prende una nuova dose entro 2 settimane dalla prima
il Tiro Salvezza per non diventare dipendenti aumenta di 1.

\bigskip

\textbf{Tabella decodifica codici localita'}
\smallskip

Es: Gusterbloon FT5

La prima Lettera indica il CLIMA, la Seconda indica l'AMBIENTE, la
Terza indica la RARITA'

La rarita' indica la possibilita', su un d10, di trovare l'erba/pianta
ricercata. Tirare 1d10 e fare meno del numero indicato, chiaramente
se c'e' corrispondenza di clima ed ambiente.
\bigskip

\begin{tabular}[c]{@{}llll@{}}
\toprule 
\textbf{Prima Lettera} & \textbf{Clima}&\textbf{Seconda Lettera} & \textbf{Ambiente}\tabularnewline
A & Arido & A & Alpino\tabularnewline
C & Freddo&B & Gole\tabularnewline
E & Ghiacci perenni&C & Foresta di Conifere\tabularnewline
F & Freddo severo&D & Foresta Decidua\tabularnewline
H & Umido e caldo&F & Argini fiumi e torrenti\tabularnewline
M & Temperato&G & Campi ghiacciati\tabularnewline
S & Semi arido&H & Campi secchi\tabularnewline
T & Temperato fresco&J & Giungla, Foreste piovose\tabularnewline
X & Sconosciuto&M & Montagna\tabularnewline
&&N & Oceano, distese salate\tabularnewline
&&S & Erba bassa\tabularnewline
&&T & Erba alta\tabularnewline
&&U & Caverne e underground\tabularnewline
&&V & Vulcanica\tabularnewline
&&W & Discariche/Rifiui\tabularnewline
&&Z & Deserto\tabularnewline
&&X & Sconosciuto\tabularnewline
\bottomrule
\end{tabular}

\pagebreak

\section{Movimento e Trasporto}\index{Trasporto}\index{Movimento}

\label{movimento-e-trasporto}



\begin{quotebox}
- E ti puoi trovare un'altra moglie!\linebreak
- Ah, questo si'. ma il guaio e' che mi ha portato via il fucile e il cavallo! Peccato, era cosi' bella, io mi ci ero affezionato. Le davo qualche frustata, ma lei non ci faceva caso.\linebreak
- Chi, tua moglie?\linebreak
- No, la mia cavalla. A trovare un'altra moglie si fa presto, ma una cavalla come quella non la ritrovo piu'. (Ombre rosse)
\end{quotebox}


Molte delle regole qui riportate sono opzionali, il Narratore puo' tenere conto solo di cio' che ritiene piu' opportuno.

Vi sono tre scale di movimento nel gioco:
\begin{itemize}
\item Tattico, per il combattimento, si usano le distanze Mischia e i quadretti
di 1 metro di lato 
\item Locale, per esplorare una zona, misurato in metri al minuto. 
\item Via Terra, per muoversi da un posto all'altro, misurato in km all'ora o al giorno. 
\end{itemize}
\textbf{Tipi di Movimento}

Quando si muovono nelle differenti scale di movimento, le creature generalmente camminano o vanno veloci o corrono.

\textbf{Camminare}:\index{Camminare} Camminare rappresenta un movimento non affrettato ma deciso di circa 4 km all'ora per un umano senza Ingombro.

\textbf{Andare Veloci}\index{Andare Veloci} Andare veloci e' un'andatura che corrisponde a un movimento di 8 km all'ora per un umano senza ingombro.

Chi attacca il personaggio che va veloce ha un bonus di 1d6 al Tiro per colpire. Il personaggio che va veloce ha un malus di 1d6 nel Tiro per Colpire nel round in cui si muove veloce.

\textbf{Correre}\index{Correre}: Significa muoversi di circa 13 km all'ora per un
umano in armatura completa.

Chi attacca il personaggio che va veloce ha un bonus di 1d6 al Tiro per colpire. Il personaggio che va veloce ha un malus di 2d6 nel Tiro per Colpire nel round in cui si corre.

Correre come azione di movimento raddoppia la velocita' di movimento e non la triplica. Solo in situazioni di non combattimento la corsa triplica il movimento.

\subsection{Tabella: Movimento e Distanza e Velocita' : a Piedi}\index{a Piedi}

\bigskip

\begin{tabular}[l]{@{}llll@{}}
	\toprule 
	Tipo di movimento & Movimento Velocita' (metri) \tabularnewline
&6m &9m &12m
\\
\textbf{In un round (tattico)*} \\
Camminare &6m &9m &12m
\\
Andare Veloci (x1.5) &9m& 12m& 18m
\\
Correre (x3)&18m &18m &24m
\\
\textbf{Un minuto (locale)}\\
Camminare & 60m &90m &120m
\\
Andare Veloci (x1.5) &90m &120m &180m
\\
Correre (x3)& 180m &270m &360m
\\
\textbf{Un’ora (via terra)}
\\
Camminare & 3km &4km& 6km
\\
Andare Veloci (x1.5) &4.5km &6km& 9km\\
Correre (x3) &9km &12km &18km
\\
\textbf{Un giorno (via terra})\\
Camminare 24km &32km &54km
\\
\bottomrule
\end{tabular}
\bigskip


\subsection{Movimento Tattico}\index{Movimento Tattico}

Durante un combattimento si utilizza la velocita' tattica e la distanza.

\textbf{Movimento Ostacolato}

Terreno difficile, ostacoli o scarsa visibilita' possono impedire i movimenti. Quando il movimento e' ostacolato si va a meta' della velocita'. Quindi sono necessari 2 Azioni per coprire la propria distanza di 9 metri (se si e' umano senza ingombro..) 

Se esiste piu' di una condizione particolare, aggiungere tra loro tutti i costi aggiuntivi applicabili.

In alcune situazioni il movimento e' talmente ostacolato che la distanza percorribile per Azione e' minima.. in tal caso si possono utilizzare tutte e 3 le Azioni per muoversi di una Azione di movimento (9/6 metri) in qualsiasi direzione.

Non applicare questa regola per attraversare terreni impraticabili o per muoversi quando non e' possibile farlo in alcun modo. 

Non si puo' Correre o Caricare agevolmente (Atletica DC 20) attraverso un percorso che ostacola il movimento.

\subsubsection{Movimento Locale}\index{Movimento Locale}

I personaggi che esplorano una zona usano il movimento locale, misurato in metri al minuto.
\begin{itemize}
\item 
Camminare: Un personaggio puo' camminare senza problemi in scala locale. 
\item 
Andare Veloci: Un personaggio puo' andare veloce senza problemi in scala locale. Vedi Movimento via terra, sotto, per il movimento in km all'ora. 
\item 
Correre: Un personaggio puo' Correre per un numero di round pari al triplo del proprio punteggio di Potenza su scala locale senza bisogno di riposarsi (minimo un round). Vedi la relativa sezione in Combattimento per le regole riguardanti la Corsa per Periodi Piu' Prolungati. 
\end{itemize}

\subsubsection{Movimento Via Terra}\index{Movimento Via Terra}

I personaggi che percorrono lunghe distanze usano il movimento via terra. Il movimento via terra e' misurato in ore o giorni. Un giorno rappresenta 8 ore di tempo di viaggio reale. Per imbarcazioni a remi, un giorno significa remare per 10 ore. Per navi a vela, rappresenta 24 ore.

\textbf{Camminare}\index{Camminare}

Si puo' camminare per 8 ore in un giorno di viaggio senza problemi.

Camminare piu' a lungo puo' sfinire (vedi Marcia forzata, sotto).

\textbf{Andare Veloci}\index{Andare Veloci}

Si puo' andare veloci per 1 ora senza problemi. Andare veloci per una seconda ora compresa tra due cicli di sonno provoca 1 Danno Non Letale, e ogni ora aggiuntiva provoca il doppio dei danni subiti nell'ora precedente. Un personaggio che subisce Danni Non Letali da andatura veloce e' considerato Affaticato.

Un personaggio Affaticato non puo' Correre o Caricare e subisce penalita' -1 a Potenza e Agilita

\textbf{Correre}\index{Correre}

Non e' possibile Correre per un lungo periodo di tempo. Tentativi di Correre e riposarsi a cicli funzionano come andare veloci.

\textbf{Terreno}\index{Terreno}

Il terreno su cui si viaggia influenza quale distanza viene percorsa in un'ora o in un giorno (vedi Tabella: Terreno e Movimento Via Terra). Una strada maestra e' una strada principale, dritta e lastricata. Una strada comune e' solitamente un cammino impervio. Un sentiero e' come una strada comune tranne per il fatto che permette di viaggiare solo in fila indiana e non avvantaggia un gruppo che viaggia con veicoli. Un terreno libero e' una zona selvaggia senza sentieri segnati.

\bigskip

\textbf{Tabella: Terreno e Movimento Via Terra (Opzionale)}

Nella tabella sono indicati i moltiplicatori per la distanza percorsa.

\begin{tabular}[c]{@{}llll@{}}
\toprule 
Terreno & Strada maestra & Strada comune & Sentiero non battuto\tabularnewline
Brughiera & x1 & x1 & x3/4\tabularnewline
Collina & x1 & x3/4 & x1/2\tabularnewline
Deserto Sabbioso & x1 & x1/2 & x1/2\tabularnewline
Foresta & x1 & x3/4 & x1/2\tabularnewline
Giungla & x1 & x3/4 & x1/4\tabularnewline
Montagna & x3/4 & x3/4 & x1/2\tabularnewline
Palude & x1 & \texttimes 3/4 & \texttimes 1/2\tabularnewline
Pianura & x1 & \texttimes 3/4 & \texttimes 1/2\tabularnewline
Tundra Ghiacciata & x1 & \texttimes 3/4 & \texttimes 3/4\tabularnewline
\bottomrule
\end{tabular}

\bigskip

\textbf{Marcia Forzata}\index{Marcia Forzata}

In un giorno di cammino normale, si puo' camminare per 8 ore. Il resto del giorno viene sfruttato per fare e disfare il campo, riposarsi e mangiare.

\textbf{Movimento in sella}\index{Movimento in sella}

Una cavalcatura che porta un cavaliere puo' muoversi con andatura veloce. Tuttavia, i danni che subisce sono danni normali invece che non letali. Puo' anche essere costretta a una marcia forzata, ma le sue prove di Potenza falliscono automaticamente e di nuovo i danni che subisce sono danni normali. Anche le cavalcature sono considerate Affaticate quando subiscono danni da andatura veloce o marcia forzata.

\subsection{Tabella: Cavalcature e Veicoli}\index{Cavalcature}\index{Veicoli}

\label{tabella-cavalcature-e-veicoli}\index{Cane}\index{Pony}\index{Carretto}\index{Zattera}\index{Barca}\index{Nave}

\begin{tabular}[c]{@{}lll@{}}
\toprule 
\textbf{Cavalcatura o Veicolo (carico trasportato)} & All'ora & Al giorno\tabularnewline
Cane da Galoppo &6km &48km
\\
Cane da Galoppo (50.5-150 kg)* &4.5km& 36km
\\
Cavallo Leggero& 7.5km &60km
\\
Cavallo Leggero (115,5-345 kg)* &5.25km &42km
\\
Cavallo Pesante &7.5km &60km
\\
Cavallo Pesante (150.5-450 kg)* &5.52km& 42km
\\
Pony &6km &48km
\\
Pony (75,5-225 kg)* &4.5km &36km
\\
Carretto o Carro &3km &24km
\\
\textbf{Imbarcazione} &&\\
Zattera o Chiatta (pertica o rimorchio) & 0.75km &7.5km
\\
Barcone (a Remi)** &1.5km &15km
\\
Barca a Remi** &2.25km& 22.5km
\\
Nave a Vela (vele) &3km &72km
\\
Nave da Guerra (vele e remi) &3.75km& 90km
\\
Nave Lunga (vele e remi) &4.5km& 108km
\\
Galea (remi e vele) &6km& 144km
\\
\bottomrule
\end{tabular}

*I quadrupedi, come i cavalli, possono portare carichi superiori rispetto ai personaggi. Vedi Capacita' di Trasporto per maggiori informazioni.

**Zattere, chiatte e barconi sono usati su laghi e fiumi. Se seguono la corrente, sommare la velocita' della corrente (di solito 4,5 km/h) alla velocita' dell'imbarcazione. Oltre a essere spinta con i remi per 10 ore, l'imbarcazione puo' anche essere trasportata dalla corrente per altre 14 ore, se qualcuno e' in grado di guidarla, e quindi si aggiungono altri 63 km alla distanza giornaliera percorsa. Queste imbarcazioni non possono essere spinte a remi contro una corrente molto forte, ma possono essere tirate controcorrente da animali da soma sulla riva.

\subsection{Fuga e Inseguimento}\index{Fuga}\index{Inseguimento}

Nel movimento round per round e' impossibile per un personaggio lento sfuggire ad un personaggio veloce senza qualche tipo di aiuto. Allo stesso modo, non e' un problema per un personaggio veloce sfuggire ad uno piu' lento.

Quando la velocita' dei due personaggi coinvolti e' uguale, c'e' un metodo abbastanza semplice per risolvere un inseguimento: se una creatura sta inseguendo un'altra ed entrambe si muovono alla stessa velocita', e l'inseguimento prosegue almeno per alcuni round, occorre effettuare prove contrapposte di Agilita' per vedere chi si muove piu' in fretta in questi round.

Se la creatura inseguita vince si allontana di 9 metri se questa distanza diventa superiore ai 100 metri ha seminato l'inseguitore. Se e' l'inseguitore a vincere, accorcia di 9 metri la distanza e quando la distanza e' mischia ha catturato il fuggitivo.

A volte un inseguimento che si svolge via terra potrebbe durare per lungo tempo (10 prove di Agilita' non risolutive) , con entrambe le parti che riescono solo a scorgersi a distanza.

Nel caso di un lungo inseguimento, una prova contrapposta di Resistenza determina quale delle due parti puo' mantenere piu' a lungo il ritmo. Se la creatura inseguita ottiene il risultato piu' alto, riesce a fuggire, altrimenti e' l'inseguitore che riesce a raggiungerla.

\subsection{Capacita' di Carico e Trasporto: Ingombro}\index{Capacita' di Carico}\index{Ingombro}

\label{capacituxe0-di-carico-e-trasporto-ingombro}

Le regole sull'ingombro determinano quanto l'Equipaggiamento trasportato da un personaggio possa rallentarlo. L'ingombro si divide in due parti: ingombro dell'armatura e ingombro del peso totale.

\textbf{Ingombro dell'Armatura}

L'armatura di un personaggio definisce il suo valore di Agilita' per il calcolo della Difesa, la sua penalita' di armatura alla prova, la sua velocita' e quanto si muove velocemente quando corre. A meno che il personaggio non sia troppo debole o non stia trasportando molta attrezzatura, questo e' tutto quello che si deve sapere. L'attrezzatura extra che trasporta non lo rallentera' piu' di quanto non faccia gia' la sua armatura.

Tuttavia, se il personaggio e' debole o sta trasportando molta attrezzatura, allora bisogna calcolare l'ingombro del peso. Farlo e' molto importante quando il personaggio sta tentando di trasportare qualche oggettopesante. 

\textbf{Ingombro del Peso}

Nel caso si intenda determinare se l'attrezzatura di un personaggio sia abbastanza pesante da rallentarlo (piu' di quanto non faccia gia' la sua armatura), calcolare il peso totale di armatura, armi e attrezzatura. Confrontare questo totale con la Potenza del personaggio con la Tabella Capacita' di Trasporto. In base a come il peso si confronta con la capacita' di trasporto, il personaggio sta trasportando un carico leggero, medio o pesante.

Come l'armatura, anche il carico fornisce al personaggio un bonus di Agilita' massimo alla Difesa, una penalita' alla prova (che funziona come una penalita' diarmatura alla prova), una velocita' e un fattore di corsa, come indicato nella Tabella: Effetti dell'Ingombro. Un carico medio o pesante influisce sulle Competenze del personaggio come quando indossa un'armatura media o pesante. Trasportare un peso leggero, invece, non lo ingombra affatto.

Se il personaggio indossa un'armatura, bisogna usare il valore peggiore (dell'armatura o del peso) per ogni categoria. Non bisogna sommare le penalita'.

\bigskip

\textbf{Tabella: Effetti dell'Ingombro}

\begin{tabular}[c]{@{}lllll@{}}
\toprule 
Carico & Agilita' max & Penalita' alla prova & Penalita' al Movimento (metri) & Correre\tabularnewline
Medio & +3 & -3 & 3 & x3\tabularnewline
Pesante & +1 & -6 & 6 & x2\tabularnewline
\bottomrule
\end{tabular}

\subsection{Sollevare e Trascinare}\index{Sollevare}\index{Trascinare}

Un personaggio puo' sollevare sopra la testa un peso fino al suo carico massimo. Il carico massimo del personaggio e' il valore piu' alto indicato per la Potenza del personaggio nella colonna carico pesante, alla Tabella Capacita' di Trasporto.

Un personaggio puo' sollevare dal terreno un peso fino al doppio del carico massimo, ma mentre e' sovraccaricato in questo modo, il personaggio perde qualsiasibonus di Agilita' alla Difesa, puo' muoversi per una distanza 3 metri per round e non puo' fare altro.

Un personaggio in genere puo' spingere o trascinare sul terreno un peso fino a cinque volte il suo carico massimo. Circostanze favorevoli possono raddoppiare queste cifre, e circostanze sfavorevoli possono ridurle a una volta e mezza o meno.

\textbf{Creature Piu' Grandi e Piu' Piccole}

I valori nella Tabella: Capacita' di Trasporto sono per creature bipedi di taglia Media.

Creature bipedi piu' grandi possono trasportare piu' peso in base alla categoria di taglia: Grande \texttimes 2, Enorme \texttimes 4, Mastodontica \texttimes 8 e Colossale \texttimes 16.

Creature bipedi piu' piccole possono trasportare meno peso in base alla categoria di taglia: Piccola \texttimes 3/4, Minuscola \texttimes 1/2, Minuta \texttimes 1/4 e Piccolissima \texttimes 1/8.

Le creature quadrupedi possono trasportare pesi superiori. Al posto dei precedenti modificatori, moltiplicare il valore corrispondente al punteggio di Potenza sulla Tabella come segue: Piccolissima x 1/4, Minuta x 1/2, Minuscola x 3/4, Piccola x 1,Media x 1,5, Grande x 3, Enorme x 6, Mastodontica \texttimes 12 e Colossale x 24.


\bigskip

\textbf{Tabella Capacita' di Trasporto}

\begin{tabular}[c]{@{}lll@{}}
\toprule 
Carico Leggero & Carico Medio & Carico Pesante\tabularnewline
Potenza {*} 10 in kg & Potenza {*} 20 in kg & Potenza {*} 30 in kg\tabularnewline
\bottomrule
\end{tabular}



\subsection{Altri Tipi di Movimento}

\label{altri-tipi-di-movimento}

Le informazioni qui di seguito sono raccolte da varie sezioni e messe
qui per vostra comodita'.

\textbf{Nuotare}\index{Nuotare}

Una creatura con una velocita' di Nuotare puo' muoversi attraverso l'acqua alla sua velocita' indicata senza fare prove di Resistenza. Si guadagna un bonus di +8 su qualsiasi prova di Resistenza per eseguire un'azione particolare o evitare un pericolo. La creatura puo' sempre scegliere di prendere 10 su una prova di Nuotare, anche se distratti o in pericolo quando si nuota. Una tale creatura puo' utilizzare l'azione di correre mentre nuota, a condizione che nuoti in linea retta.

\textbf{Scalare}\index{Scalare}

Una creatura con una velocita' di Scalare ha un bonus di +8 su tutti le prove di Resistenza. La creatura deve fare una prova di Resistenza per arrampicarsi su qualsiasi parete o pendenza con una DC superiore a 0, ma puo' sempre scegliere di prendere 10, anche se di fretta o minacciata durante la salita.

Se una creatura con una velocita' di Scalare tenta una scalata rapida (vedi sopra), guadagna un 2 punti movimento e fa una singola prova di Scalare (Resistenza) con una penalita' di -5. 

Una creatura mantiene il suo bonus di Agilita' alla Difesa (se presente) durante la salita, e gli avversari non ottengono bonus speciale per i loro attacchi contro di esso. Non puo', tuttavia, utilizzare l'azione correre mentre si arrampica.

\textbf{Scavare}\index{Scavare}

Una creatura con una velocita' di Scavare puo' scavare tunnel attraverso la terra, ma non attraverso la roccia a meno che il testo descrittivo non dica il contrario. Le creature non possono caricare o correre mentre scavano.

La maggior parte delle creature scavatrici non lascia tunnel che altre creature possono utilizzare (sia perché il materiale attraverso cui scavano riempie il tunnel dietro di loro o perché in realta' non spostano materiale quando scavano), vedere la descrizione della singola creatura per i dettagli.

\textbf{Velocita' Su Terreno}

La Velocita' Su Terreno é la normale velocita' per personaggi che non scalano, nuotano o volano.

\textbf{Volare}\index{Volare}

Una creatura con una velocita' di Volare riceve gratuitamente l'abilita' Volare come competenza.

\textbf{Volo e Manovrabilita'}

Una creatura con una velocita' di volare naturale riceve bonus (o penalita') alle prove di Volare in base alla propria manovrabilita':

Maldestra -8,

Scarsa -4,

Media +0,

Buona +4,

Perfetta +8.

Le creature che non hanno una specifica manovrabilita' (es. un umano), si presume abbiano manovrabilita' media e non hanno penalita' alle
prove.

\textbf{Volo e Taglia}

Una creatura piu' grande o piu' piccola della Taglia Media ha bonus o penalita' di taglia alle prove di Volare in base alla sua categoria di taglia:

Piccolissima +8,

Minuta +6,

Minuscola +4,

Piccola +2,

Grande -2,

Enorme -4,

Mastodontica -6,

Colossale -8.

Se nella creatura e' indicata una classe di manovrabilita' si intende gia' compresa di questi fattori. Vedere l'abilita' Volare per ulteriori dettagli.

\pagebreak

\section{Masterizzare}\index{Masterizzare}\index{Narratore}

\label{masterizzare}


\subsection{Il Narratore}

\begin{quotebox}
{Chi comanda al racconto non e' la voce: e' l'orecchio. Italo Calvino}
\end{quotebox}


\label{il-narratore}

Mentre il giocatore interpreta un personaggio in un'avventura, Il Narratore e' colui che la gestisce. Ha certamente molto piu' lavoro, ma ricreare un mondo intero affinché i propri amici lo esplorino, puo' dare molte soddisfazioni.

Il ruolo del Narratore non e' facile ma concede enormi privilegi. Vedere i propri amici giocare, divertirsi, ``ammattirsi'' dietro dubbi, indovinelli e situazioni da te create da tantissima soddisfazione e divertimento.

Il tuo ruolo e' quello del grande orchestratore, pianificatore o anche paesaggista se preferisci, con poche semplici pennellate delinei la struttura e saranno poi i giocatori ad aggiungere dettagli e situazioni.



\subsection{Punti Esperienza}\index{Punti Esperienza}\index{PX}

\label{punti-esperienza}

In TUS il passaggio di livello non e' vincolato da un numero di mostri affrontati o dai tesori ottenuti, bensi' dal fattore di difficolta' degli incontri e da come i giocatori hanno giocato.

Il consiglio e suggerimento primario e' ``passate di livello ogni qualvolta lo ritenete necessario al buon gioco ed all'avventura'.

Se questo consiglio puo' sembrare un po' troppo scarno propongo un altro approccio, semplice ma ancora piu' efficace e stimolante.

Prendete questa tabella dei punti esperienza

\subsubsection{Tabella punti Esperienza / Livello}

\label{tabella-punti-esperienza-livello}

\begin{tabular}[c]{@{}ll@{}}
\toprule 
Livello & Punti Esperienza\tabularnewline
2 & 15 xp\tabularnewline
3 & 25 xp\tabularnewline
4 - 10 & 35 xp per livello\tabularnewline
11 - 20 & 25 xp per livello\tabularnewline
\bottomrule
\end{tabular}

\bigskip

Ovvero sono necessari 15 punti esperienza per passare dal primo al secondo livello ed altrettanti per passare dal secondo al terzo livello.

Per passare dal terzo al quarto servono 25 punti esperienza, e per passare ogni livello dal 4 al 10 livello ne servono 35 di punti esperienza.

Dall'undicesimo al ventesimo servono 25 punti esperienza ad ogni livello.

\textbf{Per ogni incontro designato per sfidare il gruppo in maniera media o difficile assegnate 1 punto esperienza.}

\textbf{Per ogni incontro designato per essere potenzialmente mortale assegnate 2 punti esperienza.}

\textbf{Per ogni incontro finale, il climax dell'avventura, assegnate 3 punti esperienza, questi punti piu' che per lo scontro ``con il Boss finale'' vanno assegnati come merito per aver portato a termine una lunga avventura.}

Questi punti saranno assegnati al gruppo e quindi a tutti i giocatori, purche' abbiano almeno cercato di partecipare agli scontri/sfide.

Se il gruppo per propria ``incapacita'' o per ``sfortuna' trasforma un incontro facile (da 0 punti esperienza) in un incontro mortale, non dovete dare 2 punti esperienza. Cercate al piu' di premiare lo spirito di gruppo, le energie spese e se possibile la creativita' nell'uscirne vivi  nonostante tutto.

Quando dico ``incontro'' non pensate al semplice scontro con i mostri, per incontro si intende qualsiasi evento di ruolo che sfidi e metta alla prova i giocatori. Questa sfida puo' essere una arguta discussione con il nobile che non li vuole pagare al termine di una missione, alla sfida di un indovinello, rebus, delle trappole ben piazzate.

\bigskip

Ogni qual volta il giocatore o il gruppo:\index{Bonus PX}\index{Bonus Esperienza}
\begin{itemize}
\item 
\textbf{raggiunga gli obiettivi prefissati}; 
\item 
\textbf{faccia un ottimo gioco di ruolo}; 
\item 
\textbf{ottimizzi l'uso delle proprie Abilita' e capacita' (senza cadere nel powerplayer)}; 
\item 
\textbf{risolva i problemi in maniera creativa e fantasiosa e funzionale}; 
\item 
\textbf{abbia buona collaborazione ed interpretazione dei diversi ruoli all'interno del gruppo e come gruppo verso i ``terzi''}; 
\item 
\textbf{scopra o avvii indizi di avventura e creazione di nuovi plot};
\item
\textbf{raccolga 10000 monete d'oro (o tesoro equivalente)};
\end{itemize}

\bigskip

Premiate il giocatore/giocatori con 1 punto esperienza. Questi punti vanno dati per sessione di gioco.

In questo sistema sono necessarie circa 10 sessioni per passare di livello, potenzialmente anche molte meno se i giocatori si dimostrano bravi ed interpretano personaggi e situazioni in maniera brillante. 

Fate in modo che ogni sessione possa assegnare 1-3 punti almeno. Costruite la sessione perche' tutti i giocatori possano essere partecipanti e nessuno si senta escuso.

Include situazioni stimolanti che possano coinvolgere e sfidare il gruppo.
Nel limite del possibile ogni sessione dovrebbe includere una parte di ruolo, una parte di esplorazione, una parte di combattimento, una parte di riposo.

\bigskip

Una nota riguarda i \textbf{premi per monete d'oro}: puo' sembrare anacronistico quando c'e' gia' in sviluppo la sesta edizione del piu' famoso gioco di ruolo tornare a premiare i giocatori in base all'oro preso ai mostri. Posso pero' garantirvi che qualora il vostro gruppo sia particolarmente "povero" di giocate di ruolo o semplicemente preferisca uno stile piu' combattivo, sapere che l'oro raccolto equivale ad esperienza puo' rendere molto piu' dinamico ed avvincente l'andare in avventura.

\subsection{Incontri}\index{Incontri}


\begin{quotebox}
{Che e' la vita senza speranza? Una gittata di dadi fra le tenebre, fra i deliri. (Ambrogio Bazzero)}
\end{quotebox}

\label{incontri}

Un incontro e' un momento di tensione e speranza, paura e sfida. E' l'occasione di mostrare e manifestare le proprie capacita' e di lavorare come gruppo.

Un incontro non e' l'occasione per fare sfoggio del proprio potere assoluto, sia come Narratore, che come Giocatore. Il Narratore sapra' punire il giocatore che vuole essere oltre il gruppo e non parte di esso.

Troverete nelle pagine seguenti le istruzioni per creare delle sfide facili (0 punti esperienza), medie e alte (1 punto esperienza), straordinarie (2 punti esperienza) ed epiche (3 punti esperienza).

Sarete comunque sempre voi, il Narratore, a stabilire e sapere se una sfida e' provante o meno, se e' sfidante e critica per i giocatori e quindi volutarne sia l'impatto come punti esperienza che come difficolta'.

Un incontro e' un evento che mette i personaggi di fronte ad un problema specifico che devono risolvere. Molti sono combattimenti con i mostri o i PNG ostili, ma ce ne sono altri tipi: un corridoio irto di trappole, un'interazione politica con un re sospettoso, un passaggio pericoloso sopra un ponticello di corda traballante, un argomento scomodo con un PNG amichevole che ritiene che un personaggio lo abbia tradito, o qualsiasi cosa che aggiunga un po' di drammaticita' al gioco.

Rompicapi, sfide interpretative e prove di abilita' sono i metodi classici per la risoluzione degli incontri, ma gli incontri piu' complessi da costruire sono i piu' comuni incontri di combattimento.

Nel progettare un incontro di combattimento, in primo luogo decidete che livello di sfida volete far fronteggiare ai PG, quindi seguite i punti descritti qui di seguito.

\textbf{Determinare APL}: \index{APL}Determinate il livello medio dei personaggi: questo e' il Livello Medio del Gruppo (APL in breve, Average Party Level). Dovreste arrotondate questo valore al numero intero piu' vicino (questa e' una delle poche eccezioni alla regola dell'arrotondamento per difetto).

Si noti che questa guida di riferimento della creazione di un incontro presuppone un gruppo di quattro o cinque PG. Se il vostro gruppo ha sei o piu' giocatori, aggiungete uno al loro livello medio. Se il vostro gruppo contiene tre o meno giocatori, sottraete uno dal loro livello medio. Per esempio, se il vostro gruppo consiste di sei giocatori, due di 4° livello e quattro di 5° livello, il APL e' il 6° (28 livelli totali, diviso per sei giocatori, arrotondando per eccesso e aggiungendo uno al risultato finale).

\textbf{Determinare il CR}: Il Grado di Sfida (o CR) e' un numero di convenienza usato per indicare i rischi relativi presentati da un mostro, una trappola, un pericolo o un altro incontro: piu' il CR e' alto, piu' pericoloso e' l'incontro. Riferitevi alla Tabella: Determinare gli Incontri per determinare il Grado di Sfida che il vostro gruppo dovrebbe affrontare, in base alla difficolta' della sfida che volete e al APL.

\bigskip

\textbf{Tabella: Determinare gli Incontri}

\begin{tabular}[c]{@{}ll@{}}
\toprule 
Difficolta' & Grado di Sfida (CR)\tabularnewline
Facile & APL -1\tabularnewline
Media & APL\tabularnewline
Alta & APL +1\tabularnewline
straordinaria & APL +2\tabularnewline
Epica & APL +3\tabularnewline
\bottomrule
\end{tabular}

\bigskip

\textbf{Costruire l'Incontro}: per costruire un incontro come prima cosa calcolate il valore dell' APL.

Per sviluppare il vostro incontro, aggiungete le creature, le trappole ed i pericoli finche' non arrivate a vostro APL programmato.

Parti calcolando le sfide con CR piu' alto dell'incontro, completando il resto con sfide minori.

Per esempio, volete che il vostro gruppo di sei personaggi di 8° livello affronti alcuni Gargoyle (CR 4 ciascuno) e il loro capo, un Gigante delle Rocce (CR 8). I personaggi hanno APL 9 e la Tabella: Determinare gli Incontri stabilisce che unasfida Alta per un APL 9 e' un incontro di CR 10.

Partendo da un CR stabilito (10) seguite questa tabella per stabilire quanti ``mostri inserire nello scontro''.

\bigskip

\begin{tabular}[c]{@{}lll@{}}
\toprule 
CR obiettivo & CR creatura rispetto ad obiettivo APL & ``Peso'' per singola creatura\tabularnewline
CR & -6 & 10\tabularnewline
 & -5 & 15\tabularnewline
 & -4 & 20\tabularnewline
 & -3 & 30\tabularnewline
 & -2 & 50\tabularnewline
 & -1 & 75\tabularnewline
 & 0 & 100\tabularnewline
\bottomrule
\end{tabular}

\bigskip

\textbf{Per raggiungere l'obiettivo dobbiamo sommare ``i pesi''
fino a raggiungere 100, ovvero il 100\% della sfida.}

Nel nostro esempio un Gigante delle Rocce ha CR 8, ovvero un CR -2 rispetto al nostro obiettivo di difficolta' CR 10, ed i Gargoyle hanno CR 4 ovvero -6 rispetto al CR 10.

Un nemico con CR -2 ha peso 50, un CR -6 ha un peso di 10, per raggiungere l'obiettivo di un CR 10 mettero' 1 CR -2 ( ovvero UNO gigante delle rocce) e 5 CR -6 (ovvero CINQUE gargoyle) perche' 50 + 10{*}5 = 100

Se avessi avuto come obiettivo un CR 8, avrei potuto mettere 3 CR 5 (CR -3 = 30) ed un CR 2 (CR -6 = 10), oppure 5 CR 4 (CR +4 = 20)

Avversari con CR inferiore a 7 rispetto al APL si contano, ``pesano'', solo se sono superiori a 20 come unita'.

\textbf{Aggiungere i PNG}

Una creatura che possiede livelli, Abilita' , competenze, che potrebbe essere un personaggio si considera un PNG. Queste creature possono svolgere un ruolo molto importante e non vanno usate come semplici ``mostri''. Dategli uno spessore e creerete dei personaggi indimenticabili.

\textbf{Modifiche ad Hoc del CR}

Mentre potete modificare il CR specifico del mostro avanzandolo, applicando modifiche o livelli, potete anche aggiustare la difficolta' dell'incontro applicando modifiche ad hoc all'incontro o alla creatura in sé. 

Qui descritti ci sono tre modi aggiuntivi con cui potete alterare la difficolta' dell'incontro.

\textbf{Terreno Favorevole ai PG}

Un incontro contro un mostro che non e' nel suo elemento preferito (come uno Yeti incontrato in una caverna piena di lava, o un Drago enorme incontrato in una stanza molto piccola) da ai personaggi un vantaggio. Sviluppate l'incontronormalmente considerate l'incontro come se avesse un CR piu' basso del suo CR reale.

\textbf{Terreno Sfavorevole ai PG}

I mostri sono progettati con il presupposto che siano incontrati nel loro terreno preferito: incontrare un Aboleth sott’acqua non aumenta il CR dell'incontro, anche se nessun personaggio e' in grado di respirare sott'acqua. 

Se, d’altra parte, il terreno ha un impatto piu' significativo sull'incontro (come un incontro contro una creatura con Vista Cieca in una zona che sopprime ogni fonte di luce), si possono, nel caso, aumentare il CR dell'incontro fosse di un grado piu' alto.


\textbf{Modifiche all'Equipaggiamento dei PNG}

Potete aumentare o diminuire la difficolta' data dai PNG modificandone l'Equipaggiamento. Il valore combinato dell'Equipaggiamento di un PNG e' dato in Creare i PNG alla Tabella: Equipaggiamento dei PNG. Un PNG incontrato senza Equipaggiamento dovrebbe avere un CR ridotto di 1 (a condizione che la perdita di Equipaggiamento sia realmente controproducente per il PNG), mentre un PNG che ha un Equipaggiamento equivalente a quello di un personaggio (come indicato sulla Tabella: Ricchezza dei PNG per Livello) ha un CR superiore di 1 al suo CR reale.

Occorre prestare attenzione ad assegnare ai PNG questo equipaggiamento supplementare, specie ai livelli piu' alti, in cui potete consumare l'intero tesoro della vostra avventura in un colpo solo!

\textbf{Assegnare i PX}

I personaggi avanzano di livello sconfiggendo mostri, superando sfide e completando l'avventura: nel farlo guadagnano i Punti Esperienza (PX in breve). Potete assegnare Punti Esperienza non appena una sfida viene superata, ma cio' potrebbero interrompere il flusso del gioco. e' piu' facile assegnare i punti esperienza alla fine di una sessione di gioco che permetta ai personaggi di riflettere su quanto accaduto. Puo' usare il tempo a disposizione fra le sessioni di gioco per aggiornare la scheda.

\textbf{Disporre Tesori}

Mentre i personaggi avanzano di livello, anche la quantita' di tesori che trasportano ed usano aumenta. In TUS si suppone che tutti i personaggi di pari livello abbiano piu' o meno la stessa quantita' di tesoro e oggetti magici. Poiché il reddito primario per un personaggio deriva dai tesori e dai bottini ricavati dalle avventure, e' importante moderare la ricchezza e i tesori nelle vostre avventure. Per aiutarvi nel disporre i tesori, la quantita' di oggetti magici e di bottino che i personaggi ricevono per le loro avventure e' legata al CR degliincontri che affrontano: piu' e' alto il CR dell'incontro, maggiore sara' il tesoro assegnato.

\textbf{La Tabella: Ricchezza dei PNG} per Livello indica la quantita' di tesoro che ogni personaggio dovrebbe avere ad un livello specifico. Si noti che questa tabella si basa su un modello standard di gioco.

I giochi con magia rara potrebbero assegnare soltanto la meta' di questo valore, mentre i giochi piu' epici potrebbero raddoppiarlo. Si presume che parte del tesoro sia consumato nel corso di un'avventura (come pozioni e pergamene) e che alcuni degli oggetti meno utilizzati siano venduti per meta' del loro valore per acquistare un Equipaggiamento piu' utile.

La Tabella: Ricchezza dei PNG per Livello puo' anche essere usata per stanziare l'Equipaggiamento per i personaggi che cominciano dopo il 1° livello, come un nuovo personaggio creato per sostituirne uno morto. I personaggi non dovrebbero spendere piu' della meta' della loro ricchezza totale su un singolo oggetto.

Per un metodo equilibrato, i personaggi che vengono creati dopo il 1° livello dovrebbero spendere il 25\% della loro ricchezza per le armi, il 25\% per armatura e oggetti di protezione, il 25\% per altri oggetti magici, il 15\% per oggetti che si consumano come bacchette, pergamene e pozioni e il 10\% per un Equipaggiamento normale e monete. Tipi di personaggio differenti potrebbero spendere diversamente la loro ricchezza rispetto a come suggerito; ad esempio, gli incantatori arcani potrebbero spendere di piu' per oggetti magici e a consumo che per le armi.

\textbf{La Tabella: Valori del Tesoro per Incontro} elenca la quantita' di tesoro che ogni incontro dovrebbe assegnare in base al livello medio dei personaggi e alla velocita' di progressione di PX della campagna. Gli incontri facili dovrebbero assegnare un tesoro di un livello piu' basso rispetto al livello medio dei PG. Gli incontri piu' pericolosi, difficili ed eroici dovrebbero assegnare rispettivamente un tesoro di uno, due o tre livelli superiore al livello medio dei PG. Se nel gioco la magia e' rara, dimezzate questi valori. Se il gioco e' piu' epico, raddoppiate questi valori.

\bigskip

\textbf{Tabella: Valori del Tesoro per Incontro}\index{Tesoro}

\begin{tabular}[c]{@{}llll@{}}
\toprule 
Livello Medio Gruppo & per Incontro (mo) & Livello Medio Gruppo & per Incontro (mo)\tabularnewline
1 & 130 & 11 & 3500\tabularnewline
2 & 250 & 12 & 4500\tabularnewline
3 & 400 & 13 & 5500\tabularnewline
4 & 5500 & 14 & 7500\tabularnewline
5 & 750 & 15 & 9000\tabularnewline
6 & 1000 & 16 & 12000\tabularnewline
7 & 1300 & 17 & 16000\tabularnewline
8 & 1700 & 18 & 20000\tabularnewline
9 & 2100 & 19 & 25000\tabularnewline
10 & 2500 & 20 & 32000\tabularnewline
\bottomrule
\end{tabular}
\bigskip

\textbf{Tabella: Ricchezza dei Personaggi per Livello}

\bigskip

\begin{tabular}[c]{@{}llll@{}}
\toprule 
Livello Personaggio & Richezza & Livello Personaggio & Richezza\tabularnewline
1 & 100 & 11 & 82000\tabularnewline
2 & 500 & 12 & 55000\tabularnewline
3 & 1500 & 13 & 75000\tabularnewline
4 & 3000 & 14 & 100000\tabularnewline
5 & 5000 & 15 & 120000\tabularnewline
6 & 8000 & 16 & 160000\tabularnewline
7 & 12000 & 17 & 210000\tabularnewline
8 & 17000 & 18 & 270000\tabularnewline
9 & 25000 & 19 & 350000\tabularnewline
10 & 35000 & 20 & 500000\tabularnewline
\bottomrule
\end{tabular}

\bigskip

Gli incontri contro dei PNG solitamente ricompensano con un tesoro
tre volte superiore a quello con un mostro, grazie all'Equipaggiamento
del PNG. Per compensare, assicuratevi che i personaggi affrontino
un paio di incontri supplementari che assegnano poco in fatto di tesori.

Animali, Vegetali, Costrutti, Non Morti non intelligenti, Melme e trappole sono ottimi "incontri con poco tesoro". In alternativa, se i personaggi affrontano un certo numero di creature con poco o nessun tesoro, dovrebbero avere l'occasione di ottenere un certo numero di oggetti di valore piu' significativo nell'immediato futuro per compensare lo squilibrio. Come regola generale, i personaggi non dovrebbero possedere alcun oggetto magico di valore superiore alla meta' della ricchezza totale del personaggio, pertanto controllate bene prima di ricompensare i personaggi con oggetti molti costosi.

\subsubsection{Costruire un Bottino}\index{Costruire un Bottino}

Spesso e' sufficiente dire ai vostri giocatori che hanno trovato 5.000
mo in gemme e 10.000 mo in gioielli. Ma a volte, e' piu' interessante
fornire dei particolari. Dare a un tesoro una personalita' puo' non
solo aiutare la verosimiglianza del gioco, ma puo' a volte innescare
nuove avventure.

Le informazioni nelle pagine seguenti possono aiutarvi per determinare tipi di tesori in modo casuale: per molti degli oggetti sono stati dati dei valori, ma potete assegnarli come ritenete meglio. e' piu' facile collocare gli oggetti piu' costosi prima: se volete potete anche determinare gli oggetti magici in modo casuale usando le tabelle in Oggetti Magici, per stabilire quali oggetti siano presenti nel tesoro.

Una volta che avete consumato una parte considerevole del valore del tesoro, il resto puo' semplicemente essere composto da monete sparse e oggetti non magici con valori definiti in base alle vostre esigenze.

\textbf{Monete}: Le monete in un tesoro possono essere di rame, argento, oro e platino: quelle d'argento e d'oro sono le piu' comuni, ma potete decidere diversamente. Per le monete ed il loro valore di cambio andate all'Equipaggiamento.

\textbf{Gemme}: Anche se potete assegnare qualsiasi valore ad una gemma, alcune possono valere di piu' delle altre. Utilizzate le categorie di valore qui sotto (e le pietre preziose associate) come guida di riferimento quando assegnate i valori alle pietre preziose.

\textbf{Gemme di Bassa Qualita'} (10 mo): agata; azzurrite; quarzo blu; ematite; lapislazzuli; malachite; ossidiana; rodocrosite; occhio di tigre; turchese; perla di fiume (irregolare).

\textbf{Gemme Semi Preziose} (50 mo): eliotropio, corniola; calcedonio; crisoprasio; citrino; diaspro; lunaria; onice; crisolito; cristallo di roccia (quarzo chiaro); sardonica; sardonice; quarzo rosato, affumicato o rosa di stella; zircone.

\textbf{Pietre Preziose di Media Qualita'} (100 mo): ambra; ametista;
crisoberillo; corallo; granato rosso o verde-marrone; giada; giaietto;
perla bianca, dorata, rosa o argentata; spinello rosso, marrone-rosso
o verde scuro; tormalina.

\textbf{Pietre Preziose di Alta Qualita'} (500 mo): alessandrite; acquamarina; granato viola; perla nera; spinello blu scuro; topazio giallo oro.

\textbf{Gioielli} (1.000 mo): smeraldo; opale bianco, nero, o di fuoco; zaffiro blu; corindone giallo fuoco o vermiglio; zaffiro a stella blu o nero.

\textbf{Gioielli Eccezional}i (5.000 mo o piu'): smeraldo verde brillante cristallino, diamante, giacinto, rubino.

\textbf{Tesori non Magici} Questa categoria include monili, abiti raffinati, merci, oggetti alchemici, oggetti perfetti e altri.

Diversamente delle gemme, molti di questi oggetti hanno valori stabiliti, ma potete sempre aumentare il valore dell'oggetto decorandolo con pietre preziose o con fatture particolarmente artistiche.

Questo aumento di costo non conferisce capacita' aggiuntive: una scimitarra perfetta di Ferro Freddo impreziosita da gemme del valore di 40.000 mo funziona come una normale scimitarra di Ferro Freddo perfetta da 330 mo. Qui di seguito trovate numerosi esempi di tesori non magici, con i valori tipici.

\textbf{Oggetti d'Arte Raffinati} (100 mo o piu'): Anche se alcuni oggetti d'arte sono composti di materiali preziosi, il valore della maggior parte di pitture, sculture, opere letterarie, abiti raffinati, e simili consiste nella fattura con cui sono realizzati e nella bravura di chi li ha realizzati. Gli oggetti d'arte sono spesso ingombranti o difficili da spostare, e fragili, rendendone il recupero ed il trasporto un'avventura a sé.

\textbf{Monili Minori} (50 mo): Questa categoria comprende monili realizzati con materiali come ottone, bronzo, rame, avorio, o legni esotici, a volte impreziositi con gemme di bassa qualita' molto piccole o difettate. I monili minori includono anelli, braccialetti e orecchini.

\textbf{Monili Normali} (100--500 mo): La maggior parte dei monili e' realizzata con argento, oro, giada, o corallo, e decorata spesso con gemme semi preziose o pietre preziose di qualita' media. I monili normali comprendono tutti i tipi di monili minori piu' bracciali, collane e spille.

\textbf{Monili Preziosi} (500 mo o piu'): I monili preziosi sono realizzati in oro, mithral, platino, o simili metalli rari. Tali oggetti comprendono i tipi di monili normali piu' scettri, pendenti ed altri grandi oggetti. 

\textbf{Attrezzi perfetti} (100--300 mo): Questa categoria include le versioni perfette di armi, armature e attrezzi d'Abilita': vedi Equipaggiamento per i dettagli e i costi di questi oggetti.

\textbf{Oggetti Comun}i (fino a 1.000 mo): Ci sono molti oggetti di valore di natura alchemica o comune che possono essere utilizzati come tesoro. La maggior parte degli oggetti alchemici sono oggetti portabili e stimabili, ma anche altri come serrature, simboli sacri, cannocchiali, vini prelibati o abiti raffinati possono costituire parti interessanti di un tesoro. Anche le merci commerciali possono servire da tesoro: 5 kg di zafferano, per esempio, valgono 150 mo.

\textbf{Mappe del Tesoro e Oggetti d'Informazione} (variabili): Gli oggetti come mappe del tesoro, documenti legali di navi e case, liste di informatori o dei turni di guardia, parole d'accesso, e simili possono essere divertenti oggetti da trovare in un tesoro: potete stabilire il valore di questi oggetti come volete e possono essere di doppia utilita' in quanto possono generare idee per nuove avventure.

\textbf{Oggetti Magici}

Naturalmente, la scoperta di un Oggetto Magico e' il vero premio per qualsiasi avventuriero. Fate attenzione a collocare gli Oggetti Magici in un tesoro: e' molto piu' soddisfacente per molti giocatori trovare un oggetto magico piuttosto che comprarlo, cosi' non e' sbagliato mettere degli oggetti che poi verranno usati dai giocatori!

Anche se in genere dovreste collocare gli oggetti con attenta riflessione sui loro probabili effetti sulla vostra campagna, puo' essere divertente generare gli oggetti magici in un tesoro a caso. Fate attenzione, comunque! e' facile, con un po' di fortuna (o sfortuna) dei dadi gonfiare il vostro gioco con troppo tesoro o privarlo dello stesso. Il collocamento di oggetti magici casuali dovrebbe essere temperato sempre dal buon senso del Narratore.

\subsection{Recitare}\index{Recitare}

\label{recitare}

Un gioco di ruolo non e' un semplice tirare dadi, e' un incontro di pensieri, opinioni, sfide, lotte. E' un gioco catartico, liberatorio.

E' giusto che ci sia combattimento, lotta, sangue paura ed azione, allo stesso modo deve esserci la possibilita' di giocare i propri personaggi con i loro svantaggi storie.

Il giocatore deve sempre impersonare il personaggio, immedesimarsi e partecipare attivamente.

Ci possono essere situazioni di contorno, gestite velocemente, che vengono fatte in terza persona, eppure ogni volta che si rende necessario giocare questo deve essere vero, fatto dal giocatore calandosi appieno nel personaggio.

Quando un giocatore interpreta bene e descrive l'azione che va a svolgere in maniera partecipativa, coinvolgente, ispirata, dategli un premio, concedete un bonus di +1 all'azione che sta svolgendo.

Fatelo presente al giocatore che grazie alla sua interpretazione ha quel bonus.

\pagebreak

\section{Altre Abilita' speciali}

\label{altre-abilita-speciali}

\subsection{Etereo}\index{Etereo}

\label{etereo}

Una creatura diventata Eterea e' situata nel Piano Etereo che e' sovrapposto a quello Materiale.

Una creatura eterea e' Invisibile, senza sostanza e capace di muoversi in qualsiasi direzione, persino su e giu', ma solo a velocita' dimezzata. Una creatura eterea puo' muoversi attraverso oggetti solidi, incluse altre creature viventi. Unacreatura eterea puo' vedere e udire cio' che accade sul Piano Materiale, ma ogni cosa appare grigia ed effimera. La vista e l'udito di una creatura eterea che si trova sul Piano Materiale sono limitati a una distanza di 9 metri.

Essenze che agiscono sulla Potenza o protezione influenzano normalmente le creature eteree; questi effetti si propagano dal Piano Materiale verso il Piano Etereo, anche se non accade il contrario.

Una creatura eterea non puo' attaccare una creatura materiale ed Essenze lanciate mentre ci si trova in condizione di etereo possono influenzare solo elementi eterei. Alcune creature o oggetti materiali hanno attacchi o effetti speciali che funzionano anche sul Piano Etereo. Una creatura eterea considera tutte le altre creature eteree come se tutti fossero materiali.


Determinate creature o Abilita' conferiscono la capacita' soprannaturale di resistere al danno di certe tipologie di armi o fino ad un certo ammontare (per attacco).

Solitamente assume il valore di XX/ZZ ovvero quanto danno (XX) e' ignorato se non si e' attaccati con (ZZ). Ignorare il danno significa anche che effetti connessi all'attacco non funzionano, come veleni sull'arma.

E' applicabile un'unica DR in caso ce ne siano di piu' di una contemporanea, la scelta va fatta ad inizio scontro e rimane la stessa finche non e' finito lo scontro.

Determinate armi, particolarmente magiche possono comunque ignorare la DR \index{ignorare la DR}

\bigskip

\begin{tabular}[c]{@{}lll@{}}
\toprule 
DR da superare & Incantamento sull'arma & Attacco Naturale\tabularnewline
Incantamento +1 & +1 & Livello 3\tabularnewline
Incantamento +2 & +2 & Livello 6\tabularnewline
Ferro Freddo / Argento & +3 & Livello 9\tabularnewline
Adamantio & +4 & Livello 12\tabularnewline
\bottomrule
\end{tabular}

Proiettili (frecce, dardi, sassi) tirati da armi magiche sono considerate
magiche al pari dell'arma che li lancia.



\subsection{Paura}\index{Paura}

\label{paura}

Essenze, Oggetti Magici e certe creature possono influenzare i personaggi con paura. In molti casi, il personaggio deve effettuare un Tiro Salvezza su Volonta' per resistere agli effetti, e un tiro fallito indica che il personaggio e' scosso, spaventato o in preda al panico.

\textbf{Scosso}\index{Scosso}

I personaggi che sono scossi subiscono penalita' di -2 ai Tiri per Colpire, ai Tiri Salvezza e alle prove.

\textbf{Spaventato}\index{Spaventato}

I personaggi spaventati sono anche scossi, e inoltre fuggono dalla fonte della loro paura il piu' velocemente possibile, anche se possono scegliere la direzione di fuga. A parte cio', una volta che sono fuori vista (o udito) dalla fonte della loro paura, possono agire normalmente. Se la durata della paura non e' ancora arrivata al termine, qualora dovessero incontrare di nuovo la fonte della loro paura, cercherebbero nuovamente di fuggire. I personaggi che non sono in grado di fuggire possono combattere (anche se continuano ad essere scossi).

\textbf{In Preda al Panico}\index{In Preda al Panico}

I personaggi in preda al panico sono scossi e, inoltre, hanno una probabilita' del 50\% di far cadere a terra qualsiasi cosa stanno tenendo in mano, e di fuggire dalla fonte del loro terrore il piu' in fretta possibile seguendo un percorso di fuga completamente casuale. I personaggi in preda al panico fuggono davanti a qualsiasi altro pericolo che possano trovarsi di fronte. 

A parte cio', una volta che sono fuori vista (o udito) dalla fonte della loro paura, possono agire normalmente. I personaggi in preda al panico prendono anche la condizione Accovacciato se non possono fuggire.

\textbf{Terrore Crescente}\index{Terrore Crescente}

Gli effetti della paura sono cumulativi. Un personaggio scosso che viene nuovamente scosso diventa spaventato, mentre invece un personaggio scosso cheviene spaventato cade in preda al panico. Un personaggio spaventato che viene scosso o spaventato cade in preda al panico.


\subsection{Paralizzato}\index{Paralizzato}

\label{paralizzato}

Un personaggio paralizzato e' bloccato sul posto ed e' incapace di muoversi od agire

Ha punteggi effettivi di Potenza e Agilita' pari a -4 (-4 alla Difesa oltre ad non avere bonus di Agilita'), e' Indifeso e puo' compiere azioni esclusivamente mentali. Una creatura alata in volo, nel momento in cui viene paralizzata non puo' piu' battere le ali e precipita. Un nuotatore paralizzato non puo' piu' Nuotare e potrebbe annegare.

\pagebreak

\section{Creare Oggetti Magici}\index{Creare Oggetti Magic}

\label{creare-oggetti-magici}

Per Creare Oggetti Magici e' necessario avere le Abilta' Creazione oggetti magici.

Si possono Creare Oggetti Magici di vario tipo come:

\bigskip

\textbf{Creare Anelli Magici}\index{Anelli Magici}

Per creare un anello magico, un personaggio ha bisogno di una fonte di calore. Ha anche bisogno di una provvista di materiali, di cui il piu' ovvio e' un anello o pezzi di anello da assemblare. Il costo dei materiali e' compreso nel costo della creazione dell'anello.

\bigskip

\begin{tabular}[c]{@{}ll@{}}
\toprule 
Livello di Potere inserito nell'anello & Costo dell'anello (mo)\tabularnewline
\textless=11 & 1000\tabularnewline
13 & 2000\tabularnewline
16 & 3700\tabularnewline
19 & 25000\tabularnewline
22 & 50000\tabularnewline
\bottomrule
\end{tabular}

\bigskip

Un anello permette di fissare un Essenza in un anello per rendere l'effetto sempre attivo.

E' anche possibile inserire un Essenza ad attivazione, in questo caso consultare i costi delle Verghe.

Il livello massimo di potere di un anello magico preparabile da un incantatore e' 12+CM+bonus all'Essenza+caratteristica correlata all'Essenza.

Forgiare un anello richiede 1 giorno per ogni 1.000 mo del prezzo base.

Talento di creazione oggetto richiesto: Creare Oggetti Magici Superiori

Competenza usata nella creazione: Cultura

\subsection{Creare Armature Magiche}\index{Creare Armature Magiche}

Per creare un'armatura magica, un personaggio ha bisogno di una fonte di calore e di alcuni attrezzi per lavorare il ferro, il legno o il cuoio. Ha anche bisogno di una provvista di materiali, di cui il piu' ovvio e' l'armatura stessa o i pezzi di armatura da assemblare. Un'armatura che va incantata deve essere un'armatura perfetta e il suo costo e' aggiunto al costo totale di incantamento per determinare il valore finale di mercato.

Se i prerequisiti per la creazione dell'armatura comprendono delle Essenze, l'incantatore deve conoscere dette Essenze. 

Creare armature magiche richiede un giorno per ogni 1.000 mo del valore
del prezzo base.

Talento di creazione oggetto richiesto: Creare Oggetti Magici

Competenza usata nella creazione: Sapienza Magica o Artigianato (fabbricare armature).

\subsection{Creare Armi Magiche}\index{Creare Armi Magiche}

Per creare un'arma magica, un personaggio ha bisogno di una fonte di calore e alcuni attrezzi per lavorare il ferro, il legno o il cuoio. Ha anche bisogno di una provvista di materiali, di cui il piu' ovvio e' l'arma stessa o i pezzi di arma da assemblare. Solo un'arma perfetta puo' essere incantata per diventare un'armamagica, e il suo costo va aggiunto al costo totale di incantamento per determinare il valore finale di mercato.

Un'arma magica deve avere almeno bonus di potenziamento +1 per avere una qualsiasi delle Capacita' Speciali delle armi da mischia o da distanza.

Se i prerequisiti per la creazione dell'arma comprendono delle Essenze, l'incantatore deve conoscere dette Essenze.

Nel momento della creazione, l'incantatore deve decidere se l'arma emana luce o meno, come effetto secondario della magia infusa nell'arma. Questa decisione non influenza il prezzo o il tempo di creazione, ma una volta che l'oggetto e' completato, la decisione e' definitiva.

Creare armi doppie viene considerato analogo a creare due armi per quanto riguarda il costo, il tempo e le Capacita' Speciali.

Creare un'arma magica richiede una giornata per ogni 1.000 mo del valore del prezzo base.

Talento di creazione oggetto richiesto: Creare Oggetti Magici

Competenza usata nella creazione: Arcana, Artigianato (costruire archi) (per archi e frecce magici) o Artigianato (fabbricare armi) (per tutte le altre armi).

\subsection{Creare Bacchette}\index{Creare Bacchette}

\bigskip

\textbf{Costi Base delle Bacchette}

\begin{tabular}[c]{@{}ll@{}}
\toprule 
Livello di Potere inserito nella bacchetta & Costo della bacchetta (mo)\tabularnewline
\textless=11 & 375\tabularnewline
13 & 750\tabularnewline
16 & 4500\tabularnewline
19 & 11250\tabularnewline
22 & 21000\tabularnewline
\bottomrule
\end{tabular}

\bigskip

Una bacchetta e' un oggetto magico che conserva in se una Essenza caricata in precedenza.

Per ricaricare una bacchetta un incantatore deve infondere parte della sua magia nello stesso, la bacchetta recupera una carica ma l'incantatore oltre ad avere usato una magia ha per le successive 24 ore un -4 a tutti i CM.

Per creare una bacchetta, un personaggio ha bisogno di una provvista di materiali, di cui il piu' ovvio e' una bacchetta o i pezzi di una bacchetta da assemblare. Le bacchette sono sempre pienamente cariche (25 cariche) all'atto della creazione.

L'incantatore deve conoscere l'Essenza che inserisce nella Bacchetta.

Il livello massimo di potere di una bacchetta preparabile da un incantatore e' 14+CM+bonus all'Essenza+caratteristica correlata all'Essenza.

Creare una bacchetta richiede 1 giorno per ogni 1.000 mo del valore del prezzo base.

Talento di creazione oggetto richiesto: Creare Oggetti Magici.

Competenza usata nella creazione: Sapienza Magica, Artigianato (oreficeria),
Artigianato (scultura) o Professione (taglialegna).

\subsection{Creare Bastoni}\index{Creare Bastoni}

\textbf{Costi Base dei Bastoni}

\bigskip

\begin{tabular}[c]{@{}ll@{}}
\toprule 
Livello di Potere inserito nel bastone & Costo del bastone (mo)\tabularnewline
\textless=11 & 375\tabularnewline
13 & 750\tabularnewline
16 & 4500\tabularnewline
19 & 11250\tabularnewline
22 & 21000\tabularnewline
25 & 40000\tabularnewline
28 & 65000\tabularnewline
31 & 120000\tabularnewline
\bottomrule
\end{tabular}

\bigskip

Un Bastone e' un oggetto magico dove si carica una o piu' Essenze il cui livello totale non puo' essere superiore a quello indicato come Livello di Potere inserito nel bastone.

Quando un bastone viene attivato e' possibile usare un'Essenza alla volta.

Per creare un bastone, un personaggio ha bisogno di una provvista di materiali, di cui il piu' ovvio e' un bastone o i pezzi di un bastone da assemblare.

I bastoni sono sempre pienamente carichi (10 volte il Livello di Potere indicato) all'atto della creazione.

Il livello massimo di potere di un bastone preparabile da un incantatore e' 14+CM+bonus all'Essenza+caratteristica correlata all'Essenza. 

Per ricaricare un Bastone un incantatore deve infondere parte della sua magia nello stesso, il bastone recupera una carica ma l'incantatore oltre ad avere usato una magia ha per le successive 24 ore un -4 a tutti i CM.

l'incantatore deve conoscere le Essenze che inserisce nel bastone.

Creare un bastone richiede 1 giorno per ogni 1.000 mo del prezzo base.

Talento di creazione oggetto richiesto: Creare Oggetti Magici Superiori

Competenza usata nella creazione: Arcana, Artigianato (oreficeria), Artigianato (scultura) o Professione (taglialegna).

\pagebreak

\subsection{Creare Oggetti Magici}\index{Creare Oggetti Magici}

Per creare un oggetto meraviglioso, un personaggio di solito ha bisogno di un determinato Equipaggiamento o attrezzi particolari. Inoltre ha bisogno di una provvista di materiali, di cui il piu' ovvio e' l'oggetto stesso o i pezzi dell'oggetto da assemblare. 

Il costo dei materiali e' compreso nel costo della creazione dell'oggetto. I costi degli oggetti meravigliosi sono difficili da calcolare. Riferitevi alla Tabella: Calcolare il Valore dell'Oggetto Magico in Monete d'Oro e usate i prezzi deglioggetti nelle descrizioni come punto di riferimento. Creare un oggetto costa la meta' del prezzo di mercato indicato.

l'incantatore deve conoscere le Essenze che inserisce nell'oggetto.

Creare un oggetto meraviglioso richiede 1 giorno per ogni 1.000 mo del valore del prezzo base.

Talento di creazione oggetto richiesto: Creare Oggetti superiori.

Competenza usata nella creazione: Sapienza Magica o un'Abilita' adeguata
di Artigianato o Professione.

\subsection{Creare Pergamene}\index{Creare Pergamene}\index{Pergamene}

\begin{tabular}[c]{@{}ll@{}}
\toprule 
Livello di Potere della pergamena & Costo della pergamena (mo)\tabularnewline
\textless=11 & 100\tabularnewline
13 & 200\tabularnewline
16 & 400\tabularnewline
19 & 800\tabularnewline
22 & 1600\tabularnewline
25 & 3200\tabularnewline
28 & 6400\tabularnewline
31 & 12800\tabularnewline
34 & 25600\tabularnewline
37 & 52000\tabularnewline
\bottomrule
\end{tabular}

\bigskip

Se una pergamena include piu' Essenze il costo e' pari alla somma dei livelli di potere delle varie Essenze.

l'incantatore deve conoscere le Essenze che inserisce nella pergamena.

Per creare una pergamena c'e' un costo fisso di materiali pari a 100 mo per livello di potere da caricare. Per preparare una pergamena e' necessario 1 ora di lavoro per livello di potere.

Il livello massimo di potere una pergamena preparabile da un incantatore e' 16+CM+bonus all'Essenza+caratteristica correlata all'Essenza.

Talento di creazione dell'oggetto richiesto: Creare Oggetti Magici

Competenza usata nella creazione: Arcana, Artigianato (calligrafia) o Professione (scrivano).

\subsection{Creare Pozioni}\index{Creare Pozioni}\index{Pozioni}

Una pozione contiene l'infuso di una Essenza, ogni pozione e' quindi monouso

\bigskip

\begin{tabular}[c]{@{}ll@{}}
\toprule 
Livello di potere & Costo\tabularnewline
\textless=11 & 50\tabularnewline
13 & 100\tabularnewline
16 & 200\tabularnewline
19 & 400\tabularnewline
22 & 800\tabularnewline
\bottomrule
\end{tabular}

\bigskip

Il costo per creare la pozione e' meta' del prezzo base. Per creare una pozione, un personaggio ha bisogno di un piano di lavoro orizzontale e alcuni contenitori per mescolare i liquidi, oltre a una fonte di calore per bollire l'infuso.

Il livello massimo di potere di una pozione preparabile da un incantatore e' 14+CM+bonus all'Essenza+caratteristica correlata all'Essenza.

Tutti gli ingredienti e i materiali per mescere una pozione devono essere freschi e mai usati.

l'incantatore deve conoscere l'Essenza che inserisce nella pozione.

Talento di creazione oggetto richiesto: Distillare Pozioni

Competenza usata nella creazione: Sapienza Magica o Artigianato (alchimia).

\subsection{Creare Verghe}\index{Creare Verghe}\index{Verghe}

Una verga e' una bacchetta speciale che e' capace di rigenerare le proprie cariche. Sono oggetti preziosi e molto costosi.

Per creare una verga, un personaggio ha bisogno di una provvista di materiali, di cui il piu' ovvio e' una verga o i pezzi di una verga da assemblare.

\begin{tabular}[c]{@{}ll@{}}
\toprule 
Livello di Potere inserito nella verga & Costo della verga (mo)\tabularnewline
\textless=11 & 750\tabularnewline
13 & 1500\tabularnewline
16 & 3000\tabularnewline
19 & 20000\tabularnewline
22 & 40000\tabularnewline
\bottomrule
\end{tabular}

\bigskip

Una verga e' in grado di lanciare 1 volta al giorno la propria Essenza. Moltiplicare il costo per 4 se e' in grado di lanciarla 2 volte, moltiplicare per 8 se e' in grado di lanciarla 3 volte al giorno. 

Si puo' anche lanciare una volta in piu' nel giorno l'Essenza contenuta nella verga, dopo di che la verga si distrugge.

Il livello massimo di potere di una verga preparabile da un incantatore e' 10+CM+bonus all'Essenza+caratteristica correlata all'Essenza.

l'incantatore deve conoscere le Essenze che inserisce nella pozione.

Creare una verga richiede 1 giorno per ogni 1.000 mo del prezzo base.

Talento di creazione oggetto richiesto: Creare Oggetti Magici Superiori.

Competenza usata nella creazione: Arcana, Artigianato (fabbricare armi), Artigianato (oreficeria) o Artigianato (scultura).

\subsection{Aggiungere Nuove Capacita'}\index{Aggiungere Nuove Capacita}

A volte, la mancanza di fondi o tempo rende impossibile realizzare l'oggetto magico voluto, ma fortunatamente e' possibile potenziare o modificare un oggetto magico creato. Solo il tempo, l'oro e i vari prerequisiti richiesti dalla nuova capacita' che si vuole aggiungere all'oggetto magico pongono delle restrizione sul tipo di poteri addizionali che uno puo' infondere.

Il costo per aggiungere capacita' addizionali ad un oggetto e' lo stesso che se l'oggetto non fosse magico, meno il valore dell'oggetto originale. Quindi, una spada lunga +1 puo' diventare una spada lunga vorpal +2, e il costo della creazione e' uguale a quello di una spada lunga vorpal +2 meno il costo di una spada lunga +1.

Quando si determina il prezzo di un oggetto magico inventato bisogna considerare molti fattori. Il modo piu' semplice per decidere il prezzo e' confrontare il nuovo oggetto a un oggetto in questo capitolo che ha gia' un prezzo, e usare tale prezzo come guida.

Altrimenti, si possono usare le indicazioni riassunte nella Tabella: Calcolare il Valore dell'Oggetto Magico in Monete d'Oro.

\bigskip

\textbf{Calcolare il Valore dell'Oggetto Magico in Monete d'Oro}\index{Valore}

\begin{tabular}[c]{@{}lll@{}}
\toprule 
Effetto & Prezzo base & Esempio\tabularnewline
Bonus di caratteristica & Bonus al quadrato x 1.000 mo & Cintura dell'agilita' +2\tabularnewline
Bonus di armatura & Bonus al quadrato x 1.000 mo & Cotta di maglia+1\tabularnewline
Essenza & Vedi costi pergamena {*} 8 & Perla del potere\tabularnewline
Bonus alla CA & Bonus al quadrato x 2.000 mo & Anello di protezione+3\tabularnewline
Bonus ai Tiri Salvezza & Bonus al quadrato x 1.000 mo & Mantello della resistenza +5\tabularnewline
Bonus di competenza & Bonus al quadrato x 100 mo & Mantello del passo gatto\tabularnewline
Bonus dell'arma & Bonus al quadrato x 2.000 mo & Spada lunga+1\tabularnewline
\bottomrule
\end{tabular}

\pagebreak

\section{Oggetti Magici}\index{Oggetti Magici}

\label{oggetti-magici}
\begin{itemize}
\item 
Un personaggio puo' portare innumerevoli oggetti magici su di sé ma per determinare il bonus alla Difesa non si possono sommare piu' di 2 oggetti (es. 1 anello magico ed un braccialetto).L'armatura non si considera in questo conteggio, e deve essere una sola.
\item 
Lo stesso principio vale per il bonus ai Tiri Salvezza, puoi sommare solo i bonus provenienti da due oggetti.
\item 
Se il bonus e' alle caratteristiche si conta solo quello con il bonus maggiore.
\item 
Un personaggio non puo' portare piu' di due anelli magici altrimenti
entrano in risonanza causando 1d6 di danno a round per ogni anello oltre il secondo.
\end{itemize}



\subsection{Tabella: Generazione Casuale degli Oggetti Magici}\index{Generazione Casuale}

\label{tabella-generazione-casuale-degli-oggetti-magici}

\begin{tabular}[c]{@{}llll@{}}
\toprule 
Oggetto magico Minore & Oggetto magico& Oggetto magico & Oggetto\\
&Medio & Maggiore\\
01--04 & 01--10 & 01--10 & Armature e scudi\tabularnewline
05--09 & 11--20 & 11--20 & Armi\tabularnewline
10--44 & 21--30 & 21--25 & Pozioni\tabularnewline
45--46 & 31--40 & 26--35 & Anelli\tabularnewline
--- & 41--50 & 36--45 & Verghe\tabularnewline
47--81 & 51--65 & 46--55 & Pergamene\tabularnewline
--- & 66--68 & 56--75 & Bastoni\tabularnewline
82--91 & 69--83 & 76--80 & Bacchette\tabularnewline
92--100 & 84--100 & 81--100 & Oggetti meravigliosi\tabularnewline
\bottomrule
\end{tabular}


\subsection{Taglia e Oggetti Magici}

\label{taglia-e-oggetti-magici}

Quando un capo di vestiario o un gioiello magici vengono scoperti, il piu' delle volte la taglia non e' un problema: molti vestiti magici sono di facile utilizzo per tutti oppure si adattano magicamente a chi li indossa. Di regola, la taglia non dovrebbe impedire ai personaggi di varia tipologia fisica l'utilizzo di un oggetto magico. 

Ci possono essere delle rare eccezioni, specie con gli oggetti realizzati per una razza specifica.

Le armi e le armature rinvenute casualmente hanno una probabilita' del 30\% di essere Piccole (01--30), del 60\% di essere Medie (31--90), e del 10\% di essere di un'altra taglia.

\subsection{Oggetti Magici sul Corpo}\index{Oggetti Magici sul Corpo}

\label{oggetti-magici-sul-corpo}

Molti oggetti magici devono essere indossati da un personaggio che voglia usarli o beneficiare delle loro capacita'. Per una creatura di forma umanoide e' possibile indossare fino a 15 oggetti magici alla volta. Ognuno di questi oggetti deve essere indossato sopra una parte specifica del corpo denominata ``slot''.

Un corpo di forma umanoide puo' portare addosso Equipaggiamento magico consistente di un oggetto per ognuno dei gruppi seguenti, legato alla parte del corpo sulla quale viene indossato l'oggetto.

\textbf{Anello} (due al massimo): anelli.

\textbf{Armatura}: corazze e armature.

\textbf{Cintura}: cinture.

\textbf{Collo}: amuleti, collane, medaglioni, scarabei, spille e talismani.

\textbf{Corpo}: tuniche e vesti.

\textbf{Fronte}: corone, fasce e filatteri.

\textbf{Mani}: guanti e guanti d'arme.

\textbf{Occhi}: occhi, occhiali e lenti.

\textbf{Piedi}: scarpe, stivali e pantofole.

\textbf{Polso}: braccialetti e bracciali.

\textbf{Scudo}: scudi.

\textbf{Spalle}: cappe e mantelli.

\textbf{Testa}: cappelli, diademi, elmi e maschere.

\textbf{Torace}: camicie, giubbe, maglie e manti.

Naturalmente, un personaggio puo' possedere quanti oggetti vuole di uno stesso tipo. Ma oggetti magici dello stesso tipo addizionali, oltre a quelli previsti negli slot, non funzioneranno. 

Alcuni oggetti possono essere indossati o trasportati senza occupare spazio sul corpo del personaggio. La descrizione di un oggetto indica quando l'oggetto possiede questa proprieta'.


\subsection{Tiri Salvezza Contro i Poteri degli Oggetti Magici}\index{Tiri Salvezza}

\label{tiri-salvezza-contro-i-poteri-degli-oggetti-magici}

Gli oggetti magici normalmente riproducono Essenze o altri effetti magici. Per un Tiro Salvezza contro la magia o un effetto magico generato da un oggetto magico, la DC e' sempre il livello potere dell'Essenza generata.

Le descrizioni di molti oggetti riportano le DC dei Tiri Salvezza relativi ai vari effetti, in modo particolare quando l'effetto non e' descritto da un incantesimo equivalente (che rende difficile determinare rapidamente il suo livello di potere).


\subsection{Danneggiare gli Oggetti Magici}\index{Danneggiare gli Oggetti Magici}

\label{danneggiare-gli-oggetti-magici}

Un oggetto magico non deve compiere un Tiro Salvezza a meno che non sia incustodito, sia il bersaglio specifico dell'effetto, o il suo possessore ottenga un 3 naturale al suo Tiro Salvezza. 

Gli oggetti magici hanno sempre diritto a un tiro salvezza contro qualcosa che potrebbe danneggiarli, anche quando un oggetto normale dello stesso tipo non avrebbe alcuna possibilita' di effettuare un tiro salvezza. Gli oggetti magici usano sempre lo stesso bonus ai Tiri Salvezza, indipendentemente dal tipo (Tempra, Riflessi o Arbitrio). Il bonus ai Tiri Salvezza di un oggetto magico e' pari a 2 + 1/2 del livello potere massimo usabile. Le sole eccezioni a questa regola sono gli oggetti magici intelligenti, che effettuano i tiri salvezza su Arbitrio basandosi sul loro punteggio di Volonta'.


\subsection{Riparare gli Oggetti Magici}\index{Riparare gli Oggetti Magici}
\label{riparare-gli-oggetti-magici}

Per riparare un oggetto magico occorrono materiali e tempo, pari alla meta' del tempo e del costo per crearlo. L'Essenza di Creazione con livello di potere pari al livello di potere massimo dell'oggetto magico ripara gli oggetti magici danneggiati.


\subsection{Cariche, Dosi e Usi Multipli}\index{Cariche}\index{Dosi}\index{Usi Multipli}

\label{cariche-dosi-e-usi-multipli}

Molti oggetti, e in modo particolare le bacchette e i bastoni, hanno un potere limitato al numero di cariche che contengono. Normalmente gli oggetti dotati di cariche non superano mai il massimo di 25 cariche (10 per i bastoni). Se oggetti simili vengono trovati come parte casuale di un tesoro, si tira un 5d6 e si divide per 2 per determinare il numero delle cariche rimaste (arrotondando per difetto, minimo 1). Se un oggetto ha un numero massimo di cariche diverso da 25, si tira casualmente per stabilire quante cariche sono rimaste.

I prezzi dati si riferiscono sempre agli oggetti al massimo delle loro cariche (quando un oggetto viene creato, ha sempre il massimo delle cariche). Se un oggetto perde di valore perché non ha piu' cariche (il che e' valido per quasi tutti gli oggetti a cariche), il valore dell'oggetto parzialmente usato e' pari al numero di cariche rimaste. Nel caso di oggetti che invece potrebbero avere un'utilita'anche se privi di cariche, soltanto parte del valore dell'oggetto sara' basato sul numero di cariche rimaste.


\subsection{Acquisire Oggetti Magici}\index{Acquisire Oggetti Magici}

\label{acquisire-oggetti-magici}

\bigskip

\begin{tabular}[c]{@{}lllll@{}}
\toprule 
Dimensioni Comunita' & Valore Base & Minore & Medio & Maggiore\tabularnewline
Insediamento & 50mo & 1d4 oggetti & & \tabularnewline
Borgo & 200mo & 1d6 oggetti & & \tabularnewline
Villaggio & 500mo & 2d4 oggetti & 1d4 oggetti & \tabularnewline
Piccolo paese & 1000mo & 3d4 oggetti & 1d6 oggetti & \tabularnewline
Grande paese & 2000mo & 3d4 oggetti & 2d4 oggetti & 1d4 oggetti\tabularnewline
Piccola citta' & 4000mo & 4d4 oggetti & 3d4 oggetti & 1d6 oggetti\tabularnewline
Grande citta' & 8000mo & d4 oggetti & 3d4 oggetti & 2d4 oggetti\tabularnewline
Metropoli & 16000mo & {*} & 4d4 oggetti & 3d4 oggetti\tabularnewline
\bottomrule
\end{tabular}

{*} In una metropoli si trovano quasi tutti gli oggetti magici minori.

Gli oggetti magici sono preziosi e la maggior parte delle grandi citta' ha almeno uno o due fornitori di oggetti magici, dal semplice venditore di pozioni ad un fabbro specializzato nel forgiare spade magiche. Naturalmente, non ogni oggetto in questo manuale e' disponibile in ogni citta'.

Le linee guida seguenti aiutano i Narratori a determinare quali oggetti sono disponibili in una specifica comunita'. Esse presuppongono una campagna con un livello medio di magia. Alcune citta' potrebbero deviare di molto da questa linea di base a discrezione del Narratore. Il Narratore dovrebbe tenere una lista deglioggetti disponibili da ogni mercante e dovrebbe rimpinguare occasionalmente le scorte con nuove acquisizioni.

Il numero ed i tipi di oggetti magici disponibili in una comunita' dipendono dalla sua dimensione. Ogni comunita' ha un valore base legato ad essa (vedi Tabella: Oggetti Magici Disponibili).

C'e' una probabilita' del 75\% che qualsiasi oggetto di quel valore o inferiore si possa trovare in vendita facilmente in quella comunita'. Inoltre, la comunita' ha un certo numero di altri oggetti in vendita. Questi oggetti sono determinati a caso e sono ripartiti in categorie (minore, medio o maggiore).

Dopo aver determinato il numero di oggetti disponibili in ogni categoria, consultate la Tabella: Generazione Casuale degli Oggetti Magici per determinare il tipo di ogni oggetto (pozione, pergamena, anello, arma,ecc.) prima di passare alle tabelle specifiche per stabilire l'oggetto esatto. Ritirate ogni volta che gli oggetti non si adeguano al valore base della comunita'.

Se l'uso della magia nella campagna in cui si gioca e' raro, occorre dimezzare il valore base e il numero di oggetti in ogni comunita'. Nelle campagne con magia estremamente rara o senza magia potrebbero non esserci affatto oggetti magici invendita. I Narratori che conducono questo tipo di campagne dovrebbe prevedere delle modifiche alle sfideaffrontate dai personaggi data la mancanza di oggetti magici.

Le campagne con abbondanti oggetti magici potrebbero avere comunita' con il doppio del valore base stabilito e degli oggetti magici casuali disponibili. In alternativa, si potrebbe stabilire che tutte le comunita' siano di una categoria di dimensione maggiore allo scopo di stabilire gli oggetti magici disponibili. In una campagna con magia molto comune, tutti gli oggetti magici si possonoacquistare in una metropoli.

Oggetti e attrezzi non magici sono in genere disponibili in una comunita' di qualsiasi dimensione a meno che l'oggetto non sia molto costoso, come un'armatura completa, o fatto di un materiale insolito, come una spada lunga in adamantio. Questi oggetti dovrebbero seguire la linea guida del valore base per determinare la loro disponibilita', a discrezione del Narratore.

\pagebreak

\section{Armature Magiche}\index{Armature Magiche}

\label{armature-magiche}

Normalmente le armature magiche proteggono chi le indossa meglio di quanto potrebbe fare una qualsiasi armatura normale, pari modo gli scudi.

Una armatura magica ha un valore di Resistenza al colpo piu' alto e una Resistenza Totale migliore, ed e' molto probabile che abbia pure una penalita' alle prove di Agilita' e CM piu' bassa.

Una armatura puo' avere un valore bonus da +1 a +5 piu' eventuali altre capacita' magiche. Ogni +1 di una armatura alza di 1 il valore di Resistenza al colpo e raddoppia (o triplica, quadruplica\ldots ) il valore di Resistenza Totale.

Ogni +2 si abbassa di 1 la penalita' di Prove Agilita', ed ogni +1 si abbassa di 1 il malus alle prove di Competenza Magica.

Un'armatura viene sempre costruita in modo che, anche se dotata di stivali, elmo oguanti d'arme, questi pezzi possano essere sostituiti con altri stivali, elmo o guanti d'arme magici.


\subsection{Tabella: Generazione casuale di Armature e Scudi Magici}

\label{tabella-generazione-casuale-di-armature-e-scudi-magici}

\begin{tabular}[c]{@{}lllll@{}}
\toprule 
Minore & Medio & Maggiore & Capacita' Speciale & Prezzo{*}\tabularnewline
01--60 & 01--05 & --- & scudo+1 & 1.000 mo\tabularnewline
61--80 & 06--10 & --- & armatura+1 & 1.000 mo\tabularnewline
81--85 & 11--20 & --- & scudo+2 & 4.000 mo\tabularnewline
86--87 & 21--30 & --- & armatura+2 & 4.000 mo\tabularnewline
--- & 31--40 & 01--08 & scudo+3 & 9.000 mo\tabularnewline
--- & 41--50 & 09--16 & armatura+3 & 9.000 mo\tabularnewline
--- & 51--55 & 17--27 & scudo+4 & 16.000 mo\tabularnewline
--- & 56--57 & 28--38 & armatura+4 & 16.000 mo\tabularnewline
--- & --- & 39--49 & scudo+5 & 25.000 mo\tabularnewline
--- & --- & 50--57 & armatura+5 & 25.000 mo\tabularnewline
--- & --- & --- & armatura/scudo+6{*} & 36.000 mo\tabularnewline
--- & --- & --- & armatura/scudo+7{*} & 49.000 mo\tabularnewline
--- & --- & --- & armatura/scudo+8{*} & 64.000 mo\tabularnewline
--- & --- & --- & armatura/scudo+9{*} & 81.000 mo\tabularnewline
--- & --- & --- & armatura/scudo+10{*} & 100.000 mo\tabularnewline
88--89 & 58--60 & 58--60 & Armatura specifica{*}{*} & -\tabularnewline
90--91 & 61--63 & 61--63 & Scudo specifico{*}{*}{*} & -\tabularnewline
92--100 & 64--100 & 64--100 & Capacita' speciale e tirate ancora{*}{*},{*}{*}{*} & -\tabularnewline
\bottomrule
\end{tabular}

{*} Armature e scudi non possono avere bonus di potenziamento superiori a +5. Usate queste indicazioni per determinarne il prezzo quando vengono aggiunte delle Capacita' Speciali.

{*}{*} Tirate sulla Tabella: Armature Specifiche.

{*}{*} Tirate sulla Tabella: Scudi Specifici.



\subsection{Tabella Generazione Capacita' Speciali delle Armature}

\label{tabella-generazione-capacita-speciali-delle-armature}

\begin{tabular}[c]{@{}lllll@{}}
\toprule 
Minore & Medio & Maggiore & Capacita' Speciale & Modificatore Prezzo Base1\tabularnewline
01--25 & 01--05 & 01--03 & Felpa & +2.700 mo\tabularnewline
26--32 & 06--08 & 04 & Fortificazione Leggera & bonus +1{*}\tabularnewline
33--52 & 09--11 & --- & Scivolosa & +3.750 mo\tabularnewline
53--92 & 12--17 & --- & Ombra & +3.750 mo\tabularnewline
93--96 & 18--19 & --- & Mimetica & bonus +1\tabularnewline
97 & 20--29 & 05--07 & Scivolosa Migliorata & +15.000 mo\tabularnewline
98--99 & 30--49 & 08--13 & Ombra Migliorata & +15.000 mo\tabularnewline
--- & 50--74 & 14--28 & Resistenza all'Energia & +18.000 mo\tabularnewline
--- & 75--79 & 29--33 & Tocco Fantasma & bonus +3\tabularnewline
--- & 80--84 & 34--35 & Invulnerabilita' & bonus +3\tabularnewline
--- & 85--89 & 36--40 & Fortificazione Moderata & bonus +3\tabularnewline
--- & 90--94 & 41--42 & Della forma animale & bonus +3\tabularnewline
--- & --- & 44--48 & Scivolosa Superiore & +33.750 mo\tabularnewline
--- & --- & 49--58 & Ombra Superiore & +33.750 mo\tabularnewline
--- & --- & 59--83 & Resistenza all'Energia Migliorata & +42.000 mo\tabularnewline
--- & --- & 84--88 & Ritira o scegli & bonus +4{*}\tabularnewline
--- & --- & 89 & Forma Eterea & +49.000 mo\tabularnewline
--- & --- & 90 & Controllo dei Non Morti & +49.000 mo\tabularnewline
--- & --- & 91--92 & Fortificazione Pesante & bonus +5{*}\tabularnewline
--- & --- & 93--94 & Ritira o scegli & bonus +5{*}\tabularnewline
--- & --- & 95--99 & Resistenza all'Energia Superiore & +66.000 mo\tabularnewline
100 & 100 & 100 & Tirare ancora due volte{*}{*} & ---\tabularnewline
\bottomrule
\end{tabular}



\subsection{Tabella Generazione Capacita' Speciali degli Scudi}

\label{tabella-generazione-capacita-speciali-degli-scudi}

\begin{tabular}[c]{@{}lllll@{}}
\toprule 
Minore & Medio & Maggiore & Capacita' Speciale & Modificatore Prezzo Base{*}\tabularnewline
01--20 & 01--10 & 01--05 & Attirare Frecce & Bonus +1{*}\tabularnewline
21--40 & 11--20 & 06--08 & Sfondamento & bonus +1{*}\tabularnewline
41--50 & 21--25 & 09--10 & Accecante & bonus +1{*}\tabularnewline
51--75 & 26--40 & 11--15 & Fortificazione Leggera & bonus +1{*}\tabularnewline
76--92 & 41--50 & 16--20 & Deviazione delle Frecce & bonus +2{*}\tabularnewline
93--97 & 51--57 & 21--25 & Animato & bonus +2{*}\tabularnewline
98--99 & 58--59 & - & Della forma animale & bonus +2\tabularnewline
--- & 60--79 & 26--41 & Resistenza all'Energia & +18.000 mo\tabularnewline
--- & 80--85 & 42--46 & Tocco Fantasma & bonus +3{*}\tabularnewline
--- & 86--95 & 47--56 & Fortificazione Moderata & bonus +3{*}\tabularnewline
--- & 96--98 & 57--58 & Ritira o scegli & bonus +3{*}\tabularnewline
--- & 99 & 59 & Selvatica & bonus +3{*}\tabularnewline
--- & --- & 60--84 & Resistenza all'Energia Migliorata & +42.000 mo\tabularnewline
--- & --- & 85--86 & Ritira o scegli & bonus +4{*}\tabularnewline
--- & --- & 87 & Controllo dei Non Morti & +49.000 mo\tabularnewline
--- & --- & 88--91 & Fortificazione Pesante & bonus +5{*}\tabularnewline
--- & --- & 92--93 & Riflettente & bonus +5{*}\tabularnewline
--- & --- & 94 & Ritira o scegli & bonus +5{*}\tabularnewline
--- & --- & 95--99 & Resistenza all'Energia Superiore & +66.000 mo\tabularnewline
100 & 100 & 100 & Tirare ancora due volte2 & ---\tabularnewline
\bottomrule
\end{tabular}

{*} Da aggiungere al Bonus di Potenziamento sulla Tabella: Armature
e Scudi per determinare il prezzo di mercato totale.


\subsection{Tabella: Armature e Scudi Magici: Capacita' Speciale}

\label{tabella-armature-e-scudi-magici-capacita-speciale}

\begin{tabular}[c]{@{}ll@{}}
\toprule 
Armatura / Scudo & Costo\tabularnewline
Scudo+1 / Armatura +1 & 1000\tabularnewline
Scudo +2 / Armatura+2 & 4000\tabularnewline
Scudo +4 / Armatura+4 & 16000\tabularnewline
Scudo +5 / Armatura+5 & 25000\tabularnewline
Scudo +6 / Armatura+6 {*} & 36000\tabularnewline
Scudo +7 / Armatura+7 {*} & 49000\tabularnewline
Scudo +8 / Armatura+8 {*} & 64000\tabularnewline
Scudo +9 / Armatura+9 {*} & 81000\tabularnewline
Scudo +10 / Armatura+10 {*} & 100000\tabularnewline
\bottomrule
\end{tabular}

{*} Armature e scudi non possono avere bonus alla Difesa superiore
a +5. Usate queste indicazioni per determinarne il prezzo quando vengono
aggiunte delle Capacita' Speciali.

\subsection{Scudi Magici}\index{Scudi Magici}

\label{scudi-magici}

I bonus di difesa degli scudi sono cumulativi con l'Agilita' per determinare la difesa. I bonus magici degli scudi non vengono calcolati come bonus di attacco o ai danni quando uno scudo viene usato per attaccare. La capacita' specialeSfondamento, tuttavia, conferisce bonus +1 agli attacchi (CA) e ai danni.

Si puo' costruire uno scudo che funzioni anche come un'arma magica, ma il costo del bonus magico offensivo deve essere sommato al costo del bonus difensivo alla Difesa dello scudo.

\textbf{Attivazione}

Normalmente un personaggio trae beneficio da un'armatura magica o da uno scudo magico esattamente allo stesso modo in cui lo trae da un'armatura o da uno scudo normale: indossandoli. Se un'armatura o uno scudo sono dotati di una capacita' speciale che necessita di essere attivata da chi li indossa, allora chi li utilizza di solito deve pronunciare la parola di comando (2 Azioni).

\textbf{Armature per Creature Insolite}

Il costo dell'armatura per le creature non umanoidi, cosi' come per le creature che non sono né Medie né Piccole, varia. Il costo della qualita' perfetta e di qualsiasi potenziamento magico rimane lo stesso.



\subsection{Capacita' Speciali delle Armature Magiche e degli Scudi Magici}

\label{capacita-speciali-delle-armature-magiche-e-degli-scudi-magici}

Oltre alla resistenza od alla difesa l'armatura o lo scudo potrebbe avere delle capacita' speciali. Le capacita' speciali contano come bonus aggiuntivi per determinare il prezzo di mercato di un oggetto. Un'armatura o uno scudo non puo' avere un bonus effettivo (bonus di Difesa, bonus di resistenza piu' i bonus equivalenti delle capacita' speciali, inclusi quelli derivanti da capacita' ed Essenze) superiore a +10. Un'armatura o uno scudo dotata di una capacita' speciale deve avere almeno bonus di +1.

\bigskip

\textbf{Capacita' Speciali degli Scudi}

\bigskip

\begin{tabular}[c]{@{}ll@{}}
\toprule 
Capacita' Scudo & Capacita' Scudo\tabularnewline
Attirare Frecce: bonus +1{*} & Sfondamento: bonus +1{*}\tabularnewline
Accecante: bonus +1{*} & Fortificazione Leggera: bonus +1{*}\tabularnewline
Guardia: bonus +1{*} & Determinazione: +30.000 mo\tabularnewline
Guardia Superiore: bonus +2{*} & Animato: bonus +2{*}\tabularnewline
Resistenza all'Energia: +18.000 mo & Tocco Fantasma: bonus +3{*}\tabularnewline
Fortificazione Moderata: bonus +3{*} & Riflettente: bonus +5{*}\tabularnewline
Controllo dei Non Morti: +49.000 mo & Selvatica: bonus +3{*}\tabularnewline
Resistenza all'Energia Migliorata: +42.000 mo & Resistenza all'Energia Superiore: +66.000 mo\tabularnewline
\bottomrule
\end{tabular}

\bigskip

\textbf{Capacita' Speciali delle Armature}

\begin{tabular}[c]{@{}ll@{}}
\toprule 
Capacita' Armatura & Capacita' Armatura\tabularnewline
\textbf{Amorfa}:\index{Amorfa} +4500 mo & \textbf{Invulnerabilita'}\index{Invulnerabilita'}: bonus +3{*}\tabularnewline
\textbf{Giusto/Oscuro:}\index{Giusto/Oscuro:} +27000 mo & \textbf{Determinazione}\index{Determinazione}: +30000 mo\tabularnewline
\textbf{Mimetica}: \index{Mimetica}+2.700 mo & \textbf{Felpa}\index{Felpa}: +2\tabularnewline
\textbf{Denegante}:\index{Denegante} bonus +4{*} & \textbf{Fortificazione Leggera}\index{Fortificazione Leggera}: bonus +1{*}\tabularnewline
\textbf{Irrintracciabile}:\index{Irrintracciabile} + 7500 mo & \textbf{Scivolosa}\index{Scivolosa}: +3750 mo\tabularnewline
\textbf{Ombra}: \index{Ombra}+3750 mo & \textbf{Scivolosa Migliorata}\index{Scivolosa Migliorata}: +15000 mo\tabularnewline
\textbf{Ombra Migliorata}:\index{Ombra Migliorata} +15000 mo & \textbf{Resistenza all'Energia}: \index{Resistenza all'Energia}+18000 mo\tabularnewline
\textbf{Fortificazione Moderata}: \index{Fortificazione Moderata}bonus +3{*} & \textbf{Della Forma Animale}\index{Della Forma Animale}: bonus +3{*}\tabularnewline
\textbf{Scivolosa Superiore}:\index{Scivolosa Superiore} +33750 mo & \textbf{Ombra Superiore}: \index{Ombra Superiore}+33750 mo\tabularnewline
\textbf{Resistenza all'Energia Migliorata:} \index{Resistenza all'Energia Migliorata:}+42000 mo & \textbf{Forma Eterea}\index{Forma Eterea}: +49000 mo\tabularnewline
\textbf{Controllo dei Non Morti}:\index{Controllo dei Non Morti} +49000 mo & \textbf{Fortificazione Pesante}:\index{Fortificazione Pesante} bonus +5{*}\tabularnewline
\textbf{Resistenza all'Energia Superiore}\index{Resistenza all'Energia Superiore}: +66000 mo & \textbf{Nube Esplosiva}\index{Nube Esplosiva}: +5000 mo\tabularnewline
\bottomrule
\end{tabular}

{*} Da aggiungere al Bonus di Potenziamento sulla Tabella: Armature
e Scudi per determinare il prezzo di mercato totale.



\subsection{Descrizione delle Capacita' Speciali delle Armature e Scudi Magici}

\label{descrizione-delle-capacita-speciali-delle-armature-e-scudi-magici}

Gran parte delle armature e degli scudi magici hanno solo bonus di potenziamento, ma certi possiedono alcune delle capacita' speciali descritte qui sotto. Un'armatura o uno scudo dotati di Capacita' Speciali devono avere almeno bonus di potenziamento +1.

Il valore di CM indicato e' il livello minimo di competenza magica che si deve avere per creare l'oggetto.

\textbf{Accecante}\index{Accecante}

Uno scudo dotato di questo incantamento emana una luce accecante per un massimo di due volte al giorno su comando di chi lo impugna. Tutti coloro che si trovano entro 3 metri dallo scudo, eccetto chi lo impugna, devono effettuare un TiroSalvezza su Riflessi con DC 14 o restano Accecati per 1d4 round.

Essenza Creazione, Creare Oggetti Magici, CM 7, Prezzo bonus +1.

\textbf{Amorfa}\index{Amorfa}

Le armature con questa capacita' speciale forniscono a chi le indossa bonus di +4 alle prove di Artista della Fuga. In aggiunta, una volta al giorno a comando, chi indossa l'armatura (insieme a qualsiasi equipaggiamento indossi) puo' assumere la forma di un liquido viscoso che e' in grado di passare attraverso qualsiasi spazio nel quale potrebbe ragionevolmente scorrere del fango denso.

Mentre si usa questa capacita', la propria velocita' viene ridotta della meta' e si possono effettuare solo azioni di movimento.

Si puo' assumere questa forma per 1 minuto o finché non si spende un'azione per tornare alla propria forma naturale.

Un'armatura amorfa deve essere fatta principalmente di cuoio, stoffa o altro materiale organico e flessibile.

Essenza Trasformazione; CM 8; Creare Oggetti Magici Superiori, Prezzo
+4.500 mo.

\textbf{Animato}\index{Animato}

Come Azione, uno scudo animato puo' essere lasciato da solo a difendere il suo possessore. Per i 4 round successivi, lo scudo conferisce il suo bonus a chi lo ha lasciato e poi cade. 

Mentre e' animato, lo scudo conferisce il suo bonus di scudo e i bonus dati da qualsiasi altra capacita' speciale abbia, ma non puo' intraprendere azioni di sua volonta', come quelle conferite dalle capacita' accecante e sfondamento. 

Mentre e' animato, lo scudo condivide lo stesso spazio con il personaggio che lo ha attivato e lo accompagna, anche se il personaggio si muove tramite mezzi magici. Un personaggio con uno scudo animato continua a subire le penalita' associate all'uso dello scudo sulla competenza magica.

Se chi lo lascia ha una mano libera, puo' afferrarlo quando gli effetti magici svaniscono come Azione reazione. Una volta che uno scudo e' stato ripreso, non puo' essere animato nuovamente se non dopo almeno 4 round.

Essenza Movimento; CM 12, Creare Oggetti Magici Superiori, Prezzo bonus +2.

\textbf{Attirare Frecce}\index{Attirare Frecce}

Uno scudo dotato di questa capacita' attrae su di sé le armi a distanza. e' dotato di un bonus di Difesa +1 contro le armi a distanza in quanto le armi da lancio o da tiro virano verso di esso. Inoltre ogni arma da lancio o da tiro diretta verso un bersaglio che si trova entro distanza di mischia da chi impugna lo scudo devia dal suo bersaglio per dirigersi invece verso il portatore dello scudo.

Se chi impugna lo scudo gode di copertura totale rispetto all'attaccante, l'arma da lancio o da tiro non viene deviata. Inoltre, coloro che attaccano chi impugna lo scudo con armi a distanza ignorano le probabilita' di mancare che verrebbero normalmente applicate.

Le armi da lancio e da tiro che hanno un bonus magico superiore al bonus magico dello scudo non vengono deviate verso chi impugna lo scudo. Chi impugna lo scudo attiva e disattiva questa capacita' con una parola di comando.

Essenza Movimento, CM 8, Creare Oggetti Magici, Prezzo bonus +1.

\textbf{Controllo dei Non Morti}\index{Controllo dei Non Morti}

Le armature e gli scudi del controllo dei Non Morti sono decorati da macabri ornamenti e orpelli. Chi indossa un'armatura o uno scudo con questa proprieta' puo' controllare fino a 13 CR di non morti al giorno, come creati dall'Essenza Distruzione. All'alba di ogni giorno, colui che indossa l'armatura perde il controllo su qualsiasi non morto ancora ai suoi ordini. Le armature e gli scudi con questa capacita' sembrano fatti d'ossa; questa peculiarita' e' puramente decorativa e non ha nessun altro effetto sull'armatura.

Essenza Distruzione, CM 13, Creare Oggetti Magici Superiori, Prezzo +49.000 mo.

\textbf{Denegante}\index{Denegante}

Una volta al giorno, quando colui che indossa l'armatura e' bersaglio di un colpo critico o di un Attacco alle spalle effettuato con un'arma da mischia, puo' automaticamente negare questo critico o questo Attacco alle spalle e renderlo un attacco normale. Questa capacita' puo' essere applicata solo alle armature pesanti.

Essenza Difesa; Creare Oggetti Magici Meravigliosi, CM 13, Prezzo bonus +4.

\textbf{Determinazione}\index{Determinazione}

Essenza Cura, Creare Oggetti Magici Superiori, CM 10, Prezzo +30.000
mo.

\textbf{Felpa}\index{Felpa}

Le armature dotate della capacita' Felpa sono solitamente armature medie o pesanti. Una armatura Felpa non ha malus all'Agilita' e riduce di 5 le penalita' alla Competenza Magica.

Essenza Alterazione, Creare Oggetti Magici Superiori, CM 6, Prezzo
+2

\textbf{Forma Eterea}\index{Forma Eterea}

A comando, questo incantamento permette a chi indossa l'armatura di diventare Etereo (come per l'Essenza di Movimento) una volta al giorno. Il personaggio puo' rimanere Etereo per quanto tempo desidera ma, una volta tornato alla normalita', per quel giorno non puo' piu' diventare Etereo.

Essenza Movimento; CM 13; Creare Oggetti Magici Meravigliosi, Prezzo +49.000 mo.

\textbf{Fortificazione}\index{Fortificazione}

Questo scudo o armatura genera una forza magica che protegge con piu' efficacia le parti vitali di chi li indossa. Quando un colpo critico o un Attacco alle spalle vanno a segno su chi li indossa, c'e' una probabilita' che vengano negati e che il danno venga invece tirato normalmente.

Fortificazione Leggera: 25\%: bonus +1

Fortificazione Moderata: 50\%: bonus +3

Fortificazione Pesante: 75\%: bonus +5

Essenza Difesa, CM 13; Creare Oggetti Magici Superiori, Prezzo variabile (vedi sopra)

\textbf{Guardia}\index{Guardia}

Un scudo da guardia consente a chi lo brandisce di trasferire, in parte o per intero, il Bonus di Difesa ad una creatura adiacente (questo si somma a qualsiasi altro bonus). 

Come azione immediata, all'inizio del suo turno e prima di utilizzare una qualsiasi delle altre capacita' dello scudo, chi indossa lo scudo puo' scegliere un bersaglio adiacente e decidere in che misura conferire un bonus di Difesa andra' allocato su di esso all'inizio del suo turno.

Il suo bonus alla difesa del bersaglio dura fino al turno successivo di colui che indossa lo scudo, oppure finché quest'ultimo e il bersaglio non si trovano a piu' di distanza di mischia tra loro, a quel punto il bonus sul bersaglio termina e il Bonus di Difesa dello scudo riprende a funzionare normalmente per il suo portatore.

Questa capacita' ha effetto solamente sul Bonus di Difesa conferito dallo scudo, e non al suo Bonus ai Tiri per Colpire (se presente) né su qualsiasi altra capacita' dello scudo.

Essenza Difesa, CM 8, Creare Oggetti Magici, Prezzo bonus +1.

\textbf{Guardia Superiore}\index{Guardia Superiore}

Identica alla proprieta' da guardia, eccetto che, come Azione immediata, colui che indossa lo scudo puo' scegliere un qualsiasi numero di alleati a lui adiacenti perché ricevano del bonus dello scudo. 

Tutti gli alleati selezionati ricevono il medesimo bonus. Se un bersaglio degli effetti dello scudo si trova a piu' distanza di mischia dal portatore, gli effetti terminano per quello specifico bersaglio, ma non per gli altri eventuali bersagli.

Essenza Difesa, Livello 12; Creare Oggetti Magici Meravigliosi, Prezzo bonus +2.

\textbf{Giusto}\index{Giusto}

Un'armatura dotata di questa capacita' spesso reca dei simboli arcani di Ljust o Simkjr istoriati o smaltati su di essa. A comando, una volta al giorno chi la indossa puo' invocare gli effetti dell'Essenza Cura ed Attacco. Questa capacita' conferisce una competenza magica di 5 usabili tra le Essenze Cura ed Attacco al giorno. Non e' possibile invocare il potere piu' di 3 volte al giorno.

Una armatura del giusto e' sempre allineata verso il bene (energia positiva), al fine di determinare gli effetti dell'Essenza. Un'armatura del giusto fornisce un livello negativo permanente a qualsiasi creatura malvagia che tenti di indossarla. Questo livello negativo permane fintanto che l'armatura e' indossata e svanisce non appena questa viene rimossa.

Questo livello negativo non puo' essere eliminato in alcun modo (nemmeno per effetto dell'Essenza Cura) fintanto che la creatura indossa l'armatura.

Essenza Cura ed Attacco, CM 10, CreareOggetti Magici Superiori, Costo +27.000 mo.

\textbf{Oscuro}\index{Oscuro}

Questa armatura reca spesso cesellati su di essa simboli arcani di Calicante o Cattalm. A comando, una volta al giorno chi la indossa puo' invocare gli effetti dell'Essenza Distruzione ed Attacco. Questa capacita' conferisce una competenza magica di 5 usabili tra le Essenze Distruzione ed Attacco al giorno. Non e' possibile invocare il potere piu' di 3 volte al giorno.

Una armatura dell'Oscuro e' sempre allineata verso il male (energia negativa), al fine di determinare gli effetti dell'Essenza. Un'armatura dell'oscuro fornisce un livello negativo permanente a qualsiasi creatura buona che tenti di indossarla. Questo livello negativo permane fintanto che l'armatura e' indossata e svanisce non appena questa viene rimossa. Questo livello negativo non puo' essere eliminato inalcun modo (nemmeno per effetto dell'Essenza Cura) fintanto che la creatura indossa l'armatura. 

Essenza Distruzione ed Attacco, CM 10, Creare Oggetti Magici Superiori,
Costo +27.000 mo.

\textbf{Invulnerabilita'}\index{Invulnerabilita'}

Essenza Difesa, CM 18, Creare Oggetti Magici Meravigliosi, Prezzo bonus +3.

\textbf{Irrintracciabile}\index{Irrintracciabile}

Un'armatura irrintracciabile alleggerisce i passi di chi la indossa e ne camuffa l'aspetto. Le prove di Sopravvivenza per seguire le tracce del portatore subiscono penalita' -5, e chi indossa l'armatura ottiene Bonus di Competenza +5 alle prove di Muoversi Silenziosamente. Soltanto le armature di cuoio o di pelle possono essere irrintracciabili. 

Essenza Trasformazione, CM 5, Creare Oggetti Magici, Prezzo +7.500
mo.

\textbf{Mimetica}\index{Mimetica}

A comando, un'armatura di questo tipo muta la sua forma e appare come un normale set di vestiti. L'armatura conserva tutte le sue proprieta' (compreso il peso) anche quando e' mascherata. Solo Essenza di Rivelazione di livello potere 13 o piu' rivelano la reale natura dell'armatura trasformata.

Essenza Illusione, CM 10, Creare Oggetti Magici Superiori, Prezzo +2.700 mo.

\textbf{Nube Esplosiva}\index{Nube Esplosiva}

Questa armatura e' abitualmente decorata con incisioni in rilievo di nubi tempestose e fulmini. Se l'avversario colpisce chi la indossa e infligge almeno 10 danni da elettricita', l'armatura diventa visibilmente caricata di energia per 1 round. Come azione immediata nel suo turno successivo chi la indossa puo' utilizzare un'Essenza di Attacco a tocco infliggendo 1d6 danni da elettricita' per ogni 10 danni subiti dal portatore dal termine del suo turno precedente (massimo 5d6 per 50 o piu' danni da elettricita' subiti).

Essenza Protezione e Attacco, CM 5, Creare Oggetti Magici, Prezzo
+ 5.000 mo.

\textbf{Ombra}\index{Ombra}

Questo tipo di armatura rende chi la indossa sfuocato ogni volta che tenta di nascondersi, fornendo un bonus di competenza +1d6 alle sue prove di Criminalita' (Muoversi Silenziosamente). La penalita' di armatura alla prova basate su Agilita' si applica normalmente.

Essenza Illusione, CM 5, Creare Oggetti Magici, Prezzo +3.750 mo.

\textbf{Ombra Migliorata}\index{Ombra Migliorata}

Questo tipo di armatura rende chi la indossa sfuocato ogni volta che tenta di nascondersi, fornendo un bonus di competenza +2d6 alle sue prove di Muoversi Silenziosamente. La penalita' di armatura alla prova basate su Agilita' si applica normalmente.

Essenza Illusione, CM 10, Creare Oggetti Magici superiore, Prezzo
+15.000 mo.

\textbf{Ombra Superiore}\index{Ombra Superiore}

Questo tipo di armatura rende chi la indossa sfuocato ogni volta che
tenta di nascondersi, fornendo un bonus di competenza +3d6 alle sue
prove di Muoversi Silenziosamente. La penalita' di armatura alla prova
basate su Agilita' si applica normalmente.

Essenza Illusione, CM 15, Creare Oggetti Magici Meravigliosi, Prezzo
+33.750 mo.

\textbf{Resistenza all'Energia}\index{Resistenza all'Energia}

Questo tipo di armatura o scudo protegge contro un tipo di energia (acido, freddo, elettricita', fuoco o suono) ed e' decorata da disegni che raffigurano l'elemento dal quale protegge. L'armatura o lo scudo assorbono i primi 10 danni di energia per attacco che verrebbero subiti normalmente da chi li indossa.

Essenza Protezione, CM 3, Creare Oggetti Magici, Prezzo +18.000 mo.

\textbf{Resistenza all'Energia Migliorata}\index{Resistenza all'Energia Migliorata}

Questo tipo di armatura o scudo protegge contro un tipo di energia
(acido, freddo, elettricita', fuoco o suono) ed e' decorata da disegni
che raffigurano l'elemento dal quale protegge. L'armatura o lo scudo
assorbono i primi 20 danni di energia per attacco che verrebbero subiti
normalmente da chi li indossa

Essenza Protezione, CM 7, Creare Oggetti Magici superiore , Prezzo
+42.000 mo.

\textbf{Resistenza all'Energia Superiore}\index{Resistenza all'Energia Superiore}

Questo tipo di armatura o scudo protegge contro un tipo di energia (acido, freddo, elettricita', fuoco o suono) ed e' decorata da disegni che raffigurano l'elemento dal quale protegge. L'armatura o lo scudo assorbono i primi 30 danni di energiaper attacco che verrebbero subiti normalmente da chi li indossa (come per l'Essenza di Protezione).

Essenza Protezione, CM 11, Creare Oggetti Magici Meravigliosi, Prezzo +66.000 mo.

\textbf{Riflettente}\index{Riflettente}

Questo scudo e' simile a uno specchio. La sua superficie riflette perfettamente le immagini. Una volta al giorno puo' essere usato per riflettere una magia contro l'incantatore che l'ha lanciato. La prova di magia dell'Essenza riflessa non puo' essere superiore a 18.

Essenza Protezione, CM 14, Creare Oggetti Magici Meravigliosi, Prezzo 
bonus +5.

\textbf{Scivolosa}\index{Scivolosa}

Un'armatura scivolosa sembra perennemente rivestita da una sottile patina di grasso. Fornisce un bonus di competenza +1d6 alle prove di liberarsi da prese e manette di chi la indossa. La penalita' di armatura alla prova si applica normalmente.

Essenza Alterazione, CM 4, Creare Oggetti Magici, Prezzo +3.750 mo.

\textbf{Scivolosa Migliorata}

Come scivolosa, ma fornisce un bonus di competenza +2d6 alle prove di Criminalita'-

Essenza Alterazione, CM 10, Creare Oggetti Magici Superiori, Prezzo +10000 mo.

\textbf{Scivolosa Perfetta}

Come scivolosa, ma fornisce un bonus di competenza +3d6 alle prove di Criminalita'.

Essenza Alterazione, CM 15, Creare Oggetti Magici meravigliosa, Prezzo +15750 mo.

\textbf{Della Forma Animale}\index{Della Forma Animale}

Chi indossa un'armatura con questa capacita' conserva il bonus di protezione anche sotto un'Essenza di Trasformazione in animale. Le armature con questa capacita' sembrano di solito ricoperti di foglie. Mentre sei trasformato l'armatura non e' visibile.

Essenza Difesa e Trasformazione, CM 9, Creare Oggetti Magici Superiori, Prezzo bonus +3.

\textbf{Selvatica}\index{Selvatica}

Chi indossa uno scudo con questa capacita' conserva il bonus di protezione anche sotto un'Essenza di Trasformazione in animale. Gli scudi con questa capacita' sembrano di solito ricoperti di foglie. Mentre sei trasformato l'armatura non e' visibile.

Essenza Difesa e Trasformazione, CM 9, Creare Oggetti Magici Superiori, Prezzo bonus +3.

\textbf{Sfondamento}\index{Sfondamento}

Questo scudo e' fatto per effettuare un attacco con lo scudo. Uno scudo da sfondamento infligge danni come se fosse un'arma di due categorie di taglia piu' grande (uno scudo leggero quindi infliggerebbe 1d6 danni e uno scudo pesante infliggerebbe 1d8 danni). Lo scudo agisce come un'arma +1 quando viene usato per compiere attacchi con lo scudo. (Solo gli scudi leggeri e pesanti possono acquisire questa capacita').

Essenza Attacco, CM 8, Creare Oggetti Magici superiore, Prezzo bonus +1.

\textbf{Tocco Fantasma}\index{Tocco Fantasma}

Questa armatura o scudo sembrano quasi trasparenti. Il loro bonus magico si applica a pieno contro creature incorporee. L'armatura o lo scudo possono essere raccolti, spostati e indossati in qualsiasi momento dalle creature corporee ed incorporee. Le creature incorporee ottengono il bonus di difesa contro attacchi corporei ed incorporei, e mantengono comunque la capacita' di passare attraverso gli oggetti solidi.

Essenza Movimento, CM 14, Creare Oggetti Magici Superiore, Prezzo bonus +2

\pagebreak

\section{Armi Magiche e Speciali}\index{Armi Magiche e Speciali}

\label{armi-magiche-e-speciali}

Le armi magiche sono armi potenziate per colpire piu' facilmente ed infliggere danni maggiori. Le armi magiche hanno dei bonus di potenziamento che variano da +1 a +5 e che si applicano sia ai tiri per colpire che ai tiri per i danni quando vengono usate in combattimento. Tutte le armi magiche sono anche delle armi perfette, ma il loro bonus di perfezione all'attacco non si somma al loro bonus di potenziamento all'attacco.

Le armi magiche si suddividono in due categorie principali: da mischia e a distanza. Alcune di quelle elencate come armi da mischia (come il pugnale) possono venire usate anche come armi a distanza. In questo caso, il loro bonus di potenziamento viene applicato ad entrambi i tipi di attacco.

Alcune armi magiche possono essere dotate di capacita' speciali. Le capacita' speciali contano come bonus addizionali per determinare il prezzo di mercato dell'oggetto, ma non modificano i bonus di attacco o ai danni (tranne quando specificamente indicato).

Una sola arma non puo' possedere un bonus effettivo (il bonus di potenziamento piu' i bonus equivalenti delle capacita' speciali, inclusi quelli derivanti da capacita' ed Essenze del personaggio) superiore a +10. 

Un'arma con una capacita' speciale deve avere almeno bonus di potenziamento +1. Le armi non possono avere la stessa capacita' speciale piu' di una volta.



\subsection{Tabella: Armi Magiche}\index{Armi Magiche}

\label{tabella-armi-magiche}

\begin{tabular}[c]{@{}ll@{}}
\toprule 
Bonus Magico & Costo (mo)\tabularnewline
+1{*} & 2000\tabularnewline
+2{*} & 8000\tabularnewline
+3{*} & 18000\tabularnewline
+4{*} & 32000\tabularnewline
+5{*} & 50000\tabularnewline
+6{*}{*} & 72000\tabularnewline
+7{*}{*} & 98000\tabularnewline
+8{*}{*} & 128000\tabularnewline
+9{*}{*} & 162000\tabularnewline
+10{*}{*} & 200000\tabularnewline
\bottomrule
\end{tabular}

{*} Per le munizioni, questo prezzo vale per 50 frecce, quadrelli o proiettili per fionda.

{*}{*} Un'arma non puo' avere bonus di potenziamento maggiore di +5. Usate queste linee guida per determinarne il prezzo quando vengono aggiunte delle Capacita' Speciali.

\bigskip

\textbf{Armi a Distanza e Munizioni}

I bonus di delle armi a distanza e i bonus di delle munizioni non si cumulano tra loro. Si applica solo il piu' alto dei due bonus.

\textbf{Munizioni Magiche e Rottura}

Quando una freccia, un quadrello da balestra o un proiettile da fionda magici mancano il bersaglio, c'e' una probabilita' del 50\% che si rompano o che siano resi inutilizzabili. Una freccia, un quadrello o un proiettile magici che vanno a segno si distruggono automaticamente dopo aver inflitto il danno.

\textbf{Emanazione di Luce}

Il 30\% delle armi magiche emanano una luce intensa quanto quella emanata dall'Essenza Creazione Luce Livello Potere 11 (torcia). Queste armi luminose sono visibilmente magiche, non possono essere occultate quando vengono estratte e la loro luce non puo' essere spenta. Alcune delle armi descritte piu' avanti brillano sempre o mai, come specificato nella descrizione.

\textbf{Danneggiare le Armi Magiche}

un'arma magica puo' essere danneggiata solo da un'arma magica di pari o superiore grado.

\textbf{Attivazione}

Normalmente un personaggio sfrutta un'arma magica nella stessa maniera in cui sfrutta un'arma comune, vale a dire usandola per attaccare. Se un'arma ha una capacita' speciale che necessita di essere attivata da chi la usa, allora e' necessario che questi pronunci l'apposita parola di comando (2 Azioni). Un personaggio puo' attivare le Capacita' Speciali di 50 munizioni nello stesso momento, sempre che ogni munizione abbia la medesima capacita'.

\textbf{Armi Magiche e Colpi Critici}

Alcune qualita' delle armi, e alcune armi specifiche hanno un ulteriore effetto sui colpi critici. Questo effetto speciale agisce anche contro creature che ignorano i colpi critici. Con un tiro critico riuscito, applicate l'effetto speciale ma non il danno aggiuntivo.

\textbf{Armi per Creature Insolite}

Il costo delle armi per le creature che non sono né Medie né Piccole varia. Il costo della qualita' perfetta e di qualsiasi potenziamento
magico rimane lo stesso.

\textbf{Capacita' Speciali}

\bigskip

\begin{tabular}[c]{@{}ll@{}}
\toprule 
Capacita' speciale & Costo (mo)\tabularnewline
\textbf{Conduttiva} \index{Conduttiva} & bonus +1\tabularnewline
\textbf{Corrosiva}\index{Corrosiva} & bonus +1\tabularnewline
\textbf{Astuzia} \index{Astuzia}& bonus +1\tabularnewline
\textbf{Furiosa} \index{Furiosa} & bonus +1\tabularnewline
\textbf{Fiammagrigia} \index{Fiammagrigia}& bonus +1\tabularnewline
\textbf{Cacciatore} \index{Cacciatore} & bonus +1\tabularnewline
\textbf{Giurista} \index{Giurista} & bonus +1\tabularnewline
\textbf{Trasformante} \index{Trasformante} & +10000\tabularnewline
\textbf{Munizione Fantasma} \index{Munizione Fantasma} & bonus +2000 mo\tabularnewline
\textbf{Prensile} \index{Prensile}& bonus +2.000 mo\tabularnewline
\textbf{Volante} \index{Volante}& bonus +5\tabularnewline
\textbf{Anatema}\index{Anatema} & bonus +1\tabularnewline
\textbf{Difensiva}\index{Difensiva} & bonus +1\tabularnewline
\textbf{Infuocata} \index{Infuocata}& bonus +1\tabularnewline
\textbf{Gelida} \index{Gelida} & bonus +1\tabularnewline
\textbf{Folgorante} \index{Folgorante} & bonus +1\tabularnewline
\textbf{Tocco Fantasma} \index{Tocco Fantasma} & bonus +1\tabularnewline
\textbf{Pietosa} \index{Pietosa} & bonus +1\tabularnewline
\textbf{Accumula Magie} \index{Accumula Magie} & bonus +1\tabularnewline
\textbf{Tonante} \index{Tonante} & bonus +1\tabularnewline
\textbf{Distruzione} \index{Distruzione}& bonus +3\tabularnewline
\textbf{Ferimento} \index{Ferimento} & bonus +2\tabularnewline
\textbf{Velocita'} \index{Velocita'} & bonus +3\tabularnewline
\textbf{Energia Luminosa} \index{Energia Luminosa}& bonus +4\tabularnewline
\textbf{Danzante} \index{Danzante} & bonus +4\tabularnewline
\textbf{Vorpal}{*}{*} \index{Vorpal} & bonus +5\tabularnewline
\textbf{Incendiaria} \index{Incendiaria} & bonus +2\tabularnewline
\textbf{Nata dalla Furia} \index{Nata dalla Furia} & bonus +2\tabularnewline
\bottomrule
\end{tabular}

\subsection{Tabella: Capacita' Speciali delle Armi a Distanza}

\label{tabella-capacita-speciali-delle-armi-a-distanza}

\begin{tabular}[c]{@{}ll@{}}
\toprule 
Capacita' & Costo (mo)\tabularnewline
\textbf{Conduttiva} \index{Conduttiva}& bonus +1\tabularnewline
\textbf{Corrosiva}\index{Corrosiva} & bonus +1\tabularnewline
\textbf{Astuzia} \index{Astuzia} & bonus +1\tabularnewline
\textbf{Giurista}\index{Giurista} & bonus +1\tabularnewline
\textbf{Anatema} \index{Anatema} & bonus +1\tabularnewline
\textbf{Distanza}\index{Distanza} & bonus +1\tabularnewline
\textbf{Infuocata}\index{Infuocata} & bonus +1\tabularnewline
\textbf{Gelida} \index{Gelida}& bonus +1\tabularnewline
\textbf{Pietosa} \index{Pietosa}& bonus +1\tabularnewline
\textbf{Ritornante} \index{Ritornante}& bonus +1\tabularnewline
\textbf{Folgorante}\index{Folgorante} & bonus +1\tabularnewline
\textbf{Ricercante} \index{Ricercante} & bonus +1\tabularnewline
\textbf{Tonante} \index{Tonante}& bonus +1\tabularnewline
\textbf{Velocita'} \index{Velocita'} & bonus +3\tabularnewline
\textbf{Energia Luminosa}\index{Energia Luminosa} & bonus +4\tabularnewline
\bottomrule
\end{tabular}

\bigskip

Da aggiungere al bonus di potenziamento della Tabella: Armi, per determinare
il prezzo di mercato totale. Il bonus di potenziamento di un'arma
e i bonus equivalenti delle capacita' speciali non possono superare
il totale di +10.

\subsection{Descrizione delle Capacita' Speciali delle Armi Magiche}

\label{descrizione-delle-capacita-speciali-delle-armi-magiche}

Un'arma con una capacita' speciale deve avere almeno bonus di potenziamento +1.

\textbf{Pericolosa}\index{Pericolosa}

Questa capacita' migliora l'EDX dell'arma di 1. Se un'arma prima esplodeva il suo danno con 8 adesso migliora la possibilita' portandolo a 7.

Essenza Attacco, CM 10, Creare Oggetti Magici Superiori, Prezzo bonus +2.

\textbf{Anatema}\index{Anatema}

Un'arma anatema eccelle nell'attaccare un determinato tipo o sottotipo di creatura. Contro il nemico giurato, il suo bonus di potenziamento effettivo aumenta di +2 rispetto al normale. L'arma, inoltre, infligge +2d6 danni addizionali contro questo tipo di nemico. 

Essenza Attacco, CM 8, Creare Oggetti Magici superiore, Prezzo bonus +1.

\textbf{Conduttiva}\index{Conduttiva}

Un'arma conduttiva e' in grado di trasferire una Essenza spontanea posseduta dal personaggio attraverso la sua lama (esempio Incanalare Energia). Questa capacita' speciale dell'arma puo' essere utilizzata solamente una volta per round.

Essenza Movimento, CM 8, Creare Oggetti Magici, Prezzo bonus +1.

\textbf{Corrosiva}\index{Corrosiva}

A comando, un'arma corrosiva si ricopre di uno strato di acido che infligge 1d6 danni aggiuntivi da acido quando colpisce il bersaglio. L'acido non danneggia chi la impugna. L'effetto permane fino a quando non viene impartito un nuovo comando.

Essenza Attacco, CM 10, Creare Oggetti Magici, Prezzo bonus +1.

\textbf{Danzante}\index{Danzante}

Con 2 Azioni, un'arma danzante puo' essere lasciata libera in modo che combatta da sola. L'arma combatte per 4 round usando il CA di colui che l'ha lasciata libera e poi cade a terra. 

Mentre danza la persona che l'ha rilasciata non e' considerata armata con quell'arma, ma in tutti gli altri casi viene considerata impugnata o custodita dalla creatura per determinare tutte le manovre e gli effetti mirati contro un oggetto. 

Mentre danza, occupa lo stesso spazio del personaggio che l'ha attivata e puo' attaccare i nemici adiacenti (le armi con portata possono attaccare gli avversari fino a distanza 3 metri). 

Rimane sempre accanto alla persona che l'ha liberata, anche se si sposta con mezzi fisici o magici. Se colui che l'ha lasciata libera ha una mano libera puo' riprendere l'arma che sta attaccando da sola, come Azione reazione, ma una volta ripresa, la spada non potra' piu' danzare (attaccare da sola) prima di 4 round.

Essenza Movimento, Livello 15, Creare Oggetti Magici superiore, Prezzobonus +4.

\textbf{Difensiva}\index{Difensiva}

Un'arma difensiva permette a chi la impugna di trasferire una parte o tutto il bonus magico alla Difesa come un bonus cumulabile ad eventuali altri bonus. Come Azione reazione, chi la impugna puo' scegliere come disporre del bonus di potenziamento dell'arma all'inizio del round.

Essenza Difesa, CM 8, Creare Oggetti Magici, Prezzo bonus +1.

\textbf{Distanza}\index{Distanza}

Questa capacita' speciale puo' essere messa solo su armi a distanza, aumentando di 6 metri l'incremento di gittata.

Essenza Movimento, CM 6, Creare Oggetti Magici, Prezzo bonus +1.

\textbf{Distruzione}\index{Distruzione}

Un'arma della distruzione e' la rovina di tutti i Non Morti. Ogni creatura non morta colpita in combattimento deve superare un Tiro Salvezza su Volonta' con DC 14 o viene distrutta. Se il Tiro Salvezza riesce l'arma fa doppio danno (arma + bonus magici dell'arma). Un'arma della distruzione deve essere un'arma da mischia contundente (tipo B).

Essenza Cura o Attacco, CM 14, Creare Oggetti Magici superiore, Prezzo bonus +3

\textbf{Energia Luminosa}\index{Energia Luminosa}

Un'arma di energia luminosa si trasforma per la maggior parte in luce, anche se cio' non influisce sul suo peso. Fornisce sempre luce come una torcia (distanza 3 metri). 

Un'arma di energia luminosa ignora la materia non vivente. Le armature e scudi contano solo per il bonus magico che hanno perche' contro di loro l'arma passa attraverso (Agilita', deviazione, schivare e altri bonus simili si applicano normalmente.) U

n'arma di energia luminosa non puo' ferire Costrutti ed oggetti. Un non-morto subisce danno massimo dal dado dell'arma. 

Questa proprieta' si puo' applicare sono ad armi da mischia, da lancio e munizioni. 

Essenza Trasformazione, CM 16, Creare Oggetti Magici Meravigliosi, Prezzo bonus +4.

\textbf{Ferimento}\index{Ferimento}

Un'arma da ferimento infligge 1 danno da Sanguinamento quando colpisce una creatura. Danni multipli di quest'arma aumentano il danno da Sanguinamento. Le creature sanguinanti subiscono il danno da sanguinamento all'inizio del loro turno.

Il Sanguinamento puo' essere fermato con una prova di Guarire con DC 15 o con un'Essenza di Cura qualsiasi che curi le ferite. Un colpo critico non aggiunge ulteriore danno da Sanguinamento.

Le creature immuni ai colpi critici sono immuni ai danni da Sanguinamento inflitti da quest'arma.

Essenza Distruzione, CM 10, Creare Oggetti Magici, Prezzo bonus +2.

\textbf{Fiammagrigia}\index{Fiammagrigia}

Quest’arma risponde all'energia positiva o negativa incanalata. Quando chi la impugna usa incanalare energia nell'arma, questa si accende di una strana fiamma grigia che illumina come una torcia, aumenta di +1 il bonus di potenziamento dell'arma, ed infligge +1d6 danni alle creature da essa colpite.

Questa fiamma dura 1 round per ogni d6 danni o cure che incanalare normalmente fornisce.

Quando viene caricata con energia positiva, la fiamma e' di colore grigio argenteo, le creature buone sono immuni al danno addizionale inflitto dall'arma e l’arma viene considerata come buona e di argento ai fini di superare la riduzione del danno.

Quando viene caricata con energia negativa, la fiamma e' di color grigio cenere, le creature malvagie sono immuni al danno addizionale inflitto dall'arma, e l'arma viene considerata come malvagia e di ferro freddo ai fini di superare la riduzione del danno.

Essenza Cura o Distruzione, CM 6, Creare Oggetti Magici, Prezzo bonus +1.

\textbf{Folgorante}\index{Folgorante}

A comando, un'arma folgorante si ricopre di energia elettrica, che non danneggia chi la impugna ed ogni suo colpo andato a segno infligge 1d6 danni addizionali da elettricita'. L'effetto rimane attivo finché non viene disattivato con un altro comando.

Essenza Attacco, CM 8, Creare Oggetti Magici superiore, Prezzo bonus +1

\textbf{Gelida}\index{Gelida}

A comando, un'arma gelida emana un freddo glaciale. Il freddo non danneggia chi impugna l'arma e ogni colpo andato a segno infligge 1d6 danni addizionali da freddo. L'effetto rimane attivo finché non viene disattivato con un altro comando.

Essenza Attacco, CM 8, Creare Oggetti Magici superiore, Prezzo bonus +1

\textbf{Incendiaria}\index{Incendiaria}

Un'arma incendiaria funziona come arma Infuocata che fa anche Prendere Fuoco al bersaglio colpito da un colpo critico. Il bersaglio non ottiene un Tiro Salvezza su Agilita' per evitare di Prendere Fuoco, ma puo' effettuare un Tiro Salvezza ogni round nel suo turno per spegnere le fiamme. La capacita' Infuocata deve essere attiva affinché l'arma incendi i nemici.

Essenza Attacco, CM 12, Creare Oggetti Magici superiore, Prezzo bonus +2.

\textbf{Infuocata}\index{Infuocata}

A comando un'arma infuocata prende fuoco. Le fiamme non danneggiano chi impugna l'arma e infliggono 1d6 danni addizionali da fuoco per ogni colpo andato a segno. L'effetto rimane attivo finché non viene disattivato con un altro comando.

Essenza Attacco, CM 8, Creare Oggetti Magici, Prezzo bonus +1.

\textbf{Lancio}\index{Lancio}

Questa capacita' puo' essere posta solo su armi da mischia. Un'arma da mischia incantata con questa capacita' acquisisce una gittata di lancio di 3 metri e puo' essere lanciata senza malus se competente nel suo uso normale.

Essenza Movimento, CM 5, Creare Oggetti Magici, Prezzo bonus +1.

\textbf{Munizione Fantasma}\index{Munizione Fantasma}

Questa capacita' puo' essere conferita solo alle munizioni (frecce o dardi). Una munizione con questa capacita' speciale delle armi si dissolve 1 round dopo essere stata scagliata. In aggiunta, se il proiettile colpisce un bersaglio, la ferita causata si richiude non appena la munizione si disintegra. Il proiettile infligge danni normalmente, ma non lascia alcuna traccia visibile di violenza.

Il prezzo si riferisce a 50 munizioni fantasma.

Essenza Trasformazione, CM 7, Creare Oggetti Magici, Prezzo +1.000.

\textbf{Nata dalla Furia}\index{Nata dalla Furia}

Un'arma nata dalla furia attinge potere dalla rabbia e dalla frustrazione provate da colui chi la impugna quando si batte contro un avversario che si rifiuta di morire.

Ogni volta che chi la impugna infligge danni ad un avversario con quest'arma, il suo Bonus aumenta di +1 quando effettua attacchi contro quel nemico (Bonus totale massimo di +5).

Questo Bonus addizionale svanisce se l'avversario muore, se colui che impugna l'arma la utilizza per colpire un altro avversario (e ricomincia il ciclo), oppure quando e' trascorsa 1 ora. Solo le armi da mischia possono essere dotate della capacita' nata dalla furia.

Essenza Charme e Attacco, Livello 7, Creare oggetti magici superiore, Prezzo bonus +2.

\textbf{Pietosa}\index{Pietosa}

L'arma infligge 1d6 danni addizionali ma tutto il danno e' non letale. A comando, l'arma disattiva questa capacita' fino a quando non le viene ordinato di riattivarla (permettendole di infliggere danni letali, ma senza i danni addizionali derivanti dalla capacita').

Essenza Attacco, CM 5, Creare Oggetti Magici, Prezzo bonus +1.

\textbf{Prensile}\index{Prensile}

Questa capacita' puo' essere conferita solo alle fruste. Una frusta prensile puo', come azione veloce, aggrapparsi a un oggetto come se fosse un rampino. La frusta puo' poi essere usata per scalare superfici o dondolare attraverso una stanza o qualsiasi area all'aperto.

Essenza Movimento, CM 7, Creare Oggetti Magici, Prezzo +2.500.

\textbf{Ricercante}\index{Ricercante}

Solo le armi a distanza possiedono la capacita' ricercante. L'arma vira verso il suo bersaglio, negando qualsiasi malus che si potrebbe applicare, come quella dovuta all'Occultamento. Il possessore deve comunque mirare l'arma nella zona di mischia giusta. Le frecce sparate per errore in uno spazio vuoto, per esempio, non virano per colpire gli avversari Invisibili, se ce n'e' qualcuno nelle vicinanze.

Essenza Rivelazione, CM 12, Creazione oggetti magici superiore, Prezzo bonus +1.

\textbf{Ritornante}\index{Ritornante}

Questo incantamento puo' essere posto solo su armi che possono essere
lanciate. 

Un'arma ritornante con questa capacita' puo' ritornare indietro a chi l'ha lanciata fluttuando nell'aria. Ritorna appena prima che inizi il turno successivo di chi l'ha lanciata, e in questo modo e' pronta per essere usata di nuovo in quel turno. 

Riprendere un'arma ritornante mentre torna indietro e' un'Azione immediata.

Se il personaggio non puo' afferrarla, o se il personaggio si e' spostato dopo averla lanciata, l'arma cade a terra nella zona di mischia dalla quale e' stata lanciata.

Essenza Movimento, CM 7, Creare Oggetti Magici, Prezzo bonus +1.

\textbf{Tocco Fantasma}\index{Tocco Fantasma}

Un arma dotata del talento Tocco Fantasma e' in grado di colpire creature eteree infliggendo pieno danno.

\textbf{Tonante}\index{Tonante}

Un'arma tonante crea un tremendo frastuono simile a quello di un tuono, quando mette a segno un colpo critico. L'energia sonora non danneggia chi tiene in mano l'arma e infligge 1d8 danni sonori addizionali in caso di critico. Chi e' soggetto ad un colpo critico da un'arma tonante deve effettuare un Tiro Salvezza su Tempra con DC 14 o resta sordo in modo permanente.

Essenza Distruzione, CM 5, Creare Oggetti Magici, Prezzo bonus +1.

\textbf{Trasformante}\index{Trasformante}

Questa capacita' puo' essere conferita solo ad un'arma da mischia. Un'arma trasformante altera la sua forma a comando di chi la impugna, diventando una qualsiasi altra arma da mischia dotata della medesima forma generica e durezza dell'originale. 

Ad esempio, una spada lunga trasformante Media puo' assumere la forma di una qualsiasi altra arma di mischia ad una mano Media, come una scimitarra, un mazzafrusto od un tridente, ma non un'arma da mischia leggera o a due mani Media. 

Se lasciata incustodita, l'arma ritorna alla sua forma originaria.

Essenza Trasformazione, CM 10, Creare Oggetti Magici superiore, Prezzo +10.000 mo.

\textbf{Velocita'}\index{Velocita'}

Quando compie usa due azioni per l'attacco, il possessore di un'arma di velocita' puo' compiere un attacco addizionale con l'arma. L'attacco usa il CA pieno di chi la impugna, piu' qualsiasi modificatore appropriato alla situazione. Questo beneficio non e' cumulabile con Essenze di Movimento simili

Essenza Movimento, CM 7, Creare Oggetti Magici, Prezzo bonus +3.

\textbf{Volante}\index{Volante}

Questa capacita' speciale puo' essere conferita solo alle armi da mischia. Un'arma volante funziona come un'arma danzante, ma mentre danza puo' essere direzionata in modo che attacchi nemici a distanza di 3 metri. 

In aggiunta, in qualsiasi momento (persino quando l'arma non sta danzando), l'ultimo ad aver estratto l'arma puo' farla ritornare a sé come azione veloce. 

L'arma vola fino a un massimo di 150 metri a round per tornare dal suo proprietario, compiendo un tentativo di Spezzare a round per penetrare qualsiasi barriera che non possa aggirare o contro qualsiasi creatura che provi a trattenerla o bloccarla (liberandosi in caso di successo). 

Quando ritorna dal suo proprietario, l'arma vola in una mano libera o, se non ce l'ha, cade davanti ai suoi piedi. Se l'arma non riesce a tornare entro 4 round, cade inerte.

Essenza Movimento, CM 16, Creare Oggetti Magici Meravigliosi, Prezzo bonus +5.

\textbf{Vorpal}\index{Vorpal}

Questa temuta e potente capacita' permette all'arma di tagliare la testa di coloro che colpisce. Dopo aver ottenuto un 17 o piu' naturale con i primi 3d6 del check di arma, l'arma stacca la testa dell'avversario (se ne ha una) dal corpo. Alcune creature, come molte Aberrazioni o tutte le Melme, non hanno testa. Altre, come i Costrutti o i Non Morti (a parte i Vampiri), non sono influenzate dalla perdita della testa. La maggior parte delle altre creature, invece, muore quando la testa viene tagliata. Un'arma vorpal deve essere un'arma da mischia tagliente.

Essenza Attacco, CM 18, Creazione oggetti magici meravigliosi, Prezzo bonus +5.

\pagebreak

\section{Anelli Magici Speciali}\index{Anelli Magici Speciali}

\label{anelli-magici-speciali}

Un anello concede poteri magici, pochi usano delle cariche e molto spesso il potere e' permanente.

Gli anelli si adattano alla grandezza delle dita da piccolissimi a giganteschi, ma non per questo hanno meno o piu' poteri. Tutti possono portare fino a 2 anelli, oltre i due anelli si subiscono 1d6 di danno per anello oltre il secondo a round e l'anello aggiunto non funziona.

Solitamente un anello e' un oggetto senza un particolare peso o aspetto tranne quando descritto diversamente.

Un anello ha LP 25 o piu'.

Un anello viene attivato tramite un comando vocale se non descritto diversamente.

Per gli anelli che hanno delle Essenze troverete direttamente il livello di potere massimo usabile, che vale anche come TS per resistere agli effetti.

Se devi generare casualmente un anello tira un d100, confronta il risultato con la ``Tabella livello Anelli'' in base al tipo di anello che trovi ritira 1d100 sulla Tabella degli Anelli.

\bigskip

\textbf{Tabelle livello di Anelli}

\begin{tabular}[c]{@{}ll@{}}
\toprule 
d100 & Risultato\tabularnewline
1 & Speciale\tabularnewline
2-85 & Normale\tabularnewline
86-95 & Superiore\tabularnewline
96-99 & Maggiore\tabularnewline
100 & Maledetto\tabularnewline
\bottomrule
\end{tabular}

\bigskip

Un anello Speciale e' intelligente e tira 1 volta su Maggiore ed 1 volta su Superiore Maggiore.

Un anello con cariche non puo' essere speciale

\bigskip

\textbf{Tabella degli Anelli}

\begin{tabular}[c]{@{}lllll@{}}
\toprule 
Normale & Superiore & Maggiore & Nome Anello & Costo\tabularnewline
0-18 & - & - & Anello protezione \index{Anello protezione}+1 & 2000\tabularnewline
19-28 & - & - & Anello protezione \index{Anello protezione} & 2500\tabularnewline
29-36 & - & - & Anello del sostentamento\index{Anello del sostentamento} & 2500\tabularnewline
37-44 & - & - & Anello dello Scalare \index{Anello dello Scalare}& 2500\tabularnewline
45-52 & - & - & Anello del Saltare \index{Anello del Saltare}& 2500\tabularnewline
53-60 & - & - & Anello del Nuotare \index{Anello del Nuotare} & 2500\tabularnewline
61-70 & 01-05 & - & Anello dello scudo mentale \index{Anello dello scudo mentale} & 8000\tabularnewline
71-75 & 06-08 & - & Anello protezione +2 & 8000\tabularnewline
81-85 & 19-23 & - & Anello dello scudo di forza\index{Anello dello scudo di forza} & 8500\tabularnewline
86-90 & 24-28 & - & Anello dell'Ariete & 8600\tabularnewline
- & 29-34 & - & Anello dello scalare superiore & 10000\tabularnewline
- & 35-40 & - & Anello del saltare migliorato & 10000\tabularnewline
- & 41-46 & - & Anello del nuotare migliorato & 10000\tabularnewline
91-93 & 47-50 & - & Anello dell'amicizia con gli animali \index{Anello dell'amicizia con gli animali} & 10800\tabularnewline
94-96 & 51-56 & 01-02 & Anello resistenza energia minore \index{Anello resistenza energia} & 12000\tabularnewline
97-98 & 57-61 & - & Anello del potere del camaleonte \index{Anello del potere del camaleonte} & 12700\tabularnewline
99-100 & 62-66 & - & Anello di camminare sull'acqua\index{nello di camminare sull'acqua} & 18000\tabularnewline
- & 67-71 & 03-07 & Anello protezione +3 & 18000\tabularnewline
- & 72-76 & 08-10 & A scelta del Narratore & 18000\tabularnewline
- & 77-81 & 11-15 & Anello dell'Invisibilita' \index{nello dell'Invisibilita'} & 20000\tabularnewline
- & 82-85 & 16-19 & Anello della Essenza I \index{Anello della Essenza} & 20000\tabularnewline
- & 86-90 & 20-25 & Anello dell'Eludere \index{Anello dell'Eludere} & 25000\tabularnewline
- & 91-93 & 26-28 & Anello visione raggi x\index{Anello visione raggi x} & 25000\tabularnewline
- & 94-97 & 29-32 & Anello dell'intermittenza \index{Anello dell'intermittenza} & 27000\tabularnewline
- & 98-100 & 33-39 & Anello della resistenza all'energia maggiore & 28000\tabularnewline
- & - & 40-49 & Anello protezione +4 & 32000\tabularnewline
- & - & 50-55 & Anello della Essenza II & 40000\tabularnewline
- & - & 56-60 & Anello della liberta' di movimento \index{Anello della liberta' di movimento} & 40000\tabularnewline
- & - & 61-63 & Abello di resistenza all'energia superiore & 44000\tabularnewline
- & - & 64-65 & Anello di protezione +5 & 50000\tabularnewline
- & - & 66-70 & Anello delle stelle cadenti \index{nello delle stelle cadenti} & 50000\tabularnewline
- & - & 71-74 & Tira due volte su Superiore & 50000\tabularnewline
- & - & 75-79 & Anello della Essenza III & 70000\tabularnewline
 & - & 80-83 & Anello della Telecinesi \index{Anello della Telecinesi} & 75000\tabularnewline
- & - & 84-86 & Anello della rigenerazione \index{Anello della rigenerazione}& 90000\tabularnewline
 & & 87-88 & Anello di Riflettere Essenza\index{Anello di Riflettere Essenza} & 100000\tabularnewline
 & & 89-91 & Anello della Essenza IV & 100000\tabularnewline
 & & 92-93 & Anello dei tre desideri \index{Anello dei tre desideri} & 120000\tabularnewline
 & & 94 & Anello richiama del Djinni \index{Anello richiama del Djinni} & 125000\tabularnewline
 & & 95 & Anello del Comando degli Elementari della Terra \index{Anello del Comando degli Elementari della Terra}& 200000\tabularnewline
 & & 96 & Anello del Comando degli Elementari della Aria\index{Anello del Comando degli Elementari della Aria} & 200000\tabularnewline
 & & 98 & Anello del Comando degli Elementari della Fuoco \index{Anello del Comando degli Elementari della Fuoco} & 200000\tabularnewline
 & & 99 & Anello del Comando degli Elementari della Acqua \index{Anello del Comando degli Elementari della Acqua} & 200000\tabularnewline
 & & 100 & Tira 2 volte su normale e 2 volte su superiore & 250000\tabularnewline
\bottomrule
\end{tabular}

\bigskip

\textbf{Anello Anti Vomito}\index{Anello Anti Vomito}

Essenza Cura 13, costo 200 mo

L'anello concede a chi lo indossa +1d6 sulle prove contro vomito e offre una straordinaria resistenza all'alcool.

\textbf{Anello di Protezione}\index{Anello di Protezione}

Essenza Difesa 18, 2000 mo

Prezzo: 2.000 mo (+1), 8.000 mo (+2), 18.000 mo (+3), 32.000 mo (+4),
50.000 mo (+5)

Questo Anello offre costantemente una protezione Magica sotto forma di bonus alla difesa che varia da +1 a +5.

\textbf{Anello di Caduta Morbida}\index{Anello di Caduta Morbida}

Essenza Movimento 10, 2200 mo

Questo anello e' decorato sul suo bordo con una serie di piume. Conferisce gli stessi effetti di un incantesimo di Caduta Piuma (Movimento LP11), che si attivano immediatamente se chi lo indossa cade per piu' di 1,5 m. 

\textbf{Anello del Sostentamento}\index{Anello del Sostentamento}

Essenza Creazione 15, 2500 mo

Questo anello fornisce costantemente a chi lo indossa il necessario nutrimento per vivere. L'anello e' anche in grado di rinfrancare il suo corpo e la sua mente in modo che a chi lo indossa siano necessarie solo 2 ore di sonno al giorno per ottenere i benefici che normalmente otterrebbe con 8 ore di riposo. L'anello dev'essere indossato per un'intera settimana prima che cominci a funzionare. Se viene rimosso per qualsiasi motivo, il possessore deve di nuovo indossarlo per una settimana prima che ricominci a funzionare.

\textbf{Anello di Scalare}\index{Anello di Scalare}

Essenza Alterazione 15, 2500 mo

Questo e' un semplice anello di corda che si lega al dito. Conferisce a chi lo indossa bonus di competenza +1d6 alle prove di Scalare.

\textbf{Anello del Saltare}\index{Anello del Saltare}

Essenza Alterazione 15, 2500 mo

Questo anello di gomma permette a chi lo indossa di saltare meglio, concedendo bonus di competenza +1d6 a tutte le prove di Acrobazia effettuate per saltare in alto e in lungo.

\textbf{Anello di Nuotare}\index{Anello di Nuotare}

Essenza Alterazione 15, 2500 mo

Questo anello d'argento e' decorato con i disegni di creature marine lungo tutto il bordo. Conferisce continuamente a chi lo indossa bonus di competenza +1d6 alle prove di Nuotare.

\textbf{Anello di Scudo Mentale}\index{Anello di Scudo Mentale}

Essenza Protezion 18, 8000 mo

Questo anello,di solito finemente lavorato e realizzato in oro puro, rende chi lo indossa costantemente immune ad incantesimi di Charme di livello potere 13 o meno

\textbf{Anello dello Scudo di Forza}\index{Anello dello Scudo di Forza}

Essenza Creazione 18, 8500 mo

Questo anello, forgiato come una semplice banda di ferro, genera uno scudo di forza delle dimensioni e della forma di uno scudo, che rimane legato all'anello e puo' essere impugnato da chi lo indossa come se fosse uno scudo pesante (Difesa +3). Questa speciale creazione, dato che non pesa e non ingombra, non ha penalita' di armatura alla prova, né probabilita' di fallimento degli incantesimi. Puo' essere attivata e disattivata a piacere come Azione reazione.

\textbf{Anello dell'Ariete}\index{Anello dell'Ariete}

Essenza Attacco 21, 8600 mo

L'anello dell'ariete e' un anello decorato e forgiato in ferro o in una lega di ferro che porta come decorazione la piccola testa di un ariete. Chi lo indossa puo' ordinare all'anello di far partire una forza pari alla carica di un ariete, che si manifesta in una forma a malapena discernibile simile a una testa d'ariete o di capra.

Questa forza colpisce un unico bersaglio, infliggendo 1d6 danni se viene consumata una carica, 2d6 danni se ne vengono consumate 2 o 3d6 se ne vengono consumate 3 (il massimo).

L'attacco va considerato come un attacco a distanza, con una gittata massima di 9 metri e nessuna penalita' per la distanza. La forza del colpo e' notevole, e i soggetti colpiti dall'anello vengono considerati colpiti da una spinta (TS su Tempra DC 22 o indietreggiare a distanza di 3 metri), se si trovano a distanza media o meno da chi lo indossa. L'ariete e' di taglia Grande e ha Potenza 8.

Oltre alla sua modalita' di attacco, l'anello dell'ariete ha anche il potere di aprire porte come se fosse un personaggio con Potenza 1. Se si spendono 2 cariche, l'effetto e' equivalente a quello di un personaggio di Potenza 2, e se se ne spendono 3, a un personaggio di Potenza 3. Un anello appena creato e' dotato di 50 cariche. Una volta esaurite tutte e 50, l'anello diventa un normale oggetto non magico.

\textbf{Anello di Scalare, Nuotare, Saltare Migliorato}

Essenza Alterazione 18, 10000 mo

Dall'aspetto assolutamente identico agli anelli con potere base, queste
versioni concedono un bonus di 2d6 alle rispettive prove.

\textbf{Anello di Amicizia con gli Animali}\index{Anello di Amicizia con gli Animali}

Essenza Charme 15, 10800 mo

Un anello dell'amicizia con gli animali e' forgiato con decori di tipo animale. A comando, questo anello influenza un animale come se chi lo indossa avesse lanciato su di lui Essenza Charme LP 15.

\textbf{Anello di Resistenza all'Energia}\index{Anello di Resistenza all'Energia}

Essenza Protezione 18-21-24, 12.000 mo (minore), 28.000 mo (maggiore),
44.000 mo (superiore)

Questo anello protegge costantemente chi lo indossa dai danni di un tipo specifico di energia: acido, elettricita', freddo, fuoco o suono (a scelta del creatore dell'oggetto; determinarlo a caso se fa parte di un tesoro ritrovato). Quando chi lo indossa subirebbe tali danni, bisogna sottrarre il valore di resistenza dell'anello ai danni inflitti.

Un anello di resistenza all'energia minore conferisce 10 punti di resistenza. Un anello di resistenza all'energia maggiore conferisce 20 punti di resistenza. Un anello di resistenza all'energia superiore conferisce 30 punti di resistenza.

\textbf{Anello di Potere del Camaleonte}\index{Anello di Potere del Camaleonte}

Essenza Illusione 15, 12700 mo

Come Azione immediata, chi indossa questo anello ottiene la capacita' di confondersi magicamente con l'ambiente circostante, guadagnando un bonus di competenza +2d6 alle prove di Furtivita'. Con 2 Azioni inoltre, puo' usare l'Essenza di Illusione per cambiare il suo aspetto rimanendo nella taglia)

\textbf{Anello di Camminare sull'Acqua}\index{Anello di Camminare sull'Acqua}

Essenza Movimento 18, 15000 mo

Questo anello e' spesso ricavato da un corallo o da un metallo blu ornato da motivi marini. Permette a chi lo indossa puo' camminare sull'acqua.

\textbf{Anello Accumula Essenza}\index{Anello Accumula Essenza}

Essenza Varie 18-24-28, 18000 mo

Un anello accumula incantesimi contiene 3 essenze con LP fino a 13, 18, 21 a secondo del potere.

L'anello informa magicamente chi lo indossa su quali incantesimi vi sono attualmente contenuti.

\textbf{Anello di Invisibilita'}\index{Anello di Invisibilita'}

Essenza Illusione 21, 20000 mo

Attivando questo semplice anello d'argento, chi lo indossa puo' beneficiare degli effetti dell'Essenza di Illusione rendendosi invisibile, appena attacca diviene visibile.

\textbf{Anello delle Essenze}\index{Anello delle Essenze}

Essenze Protezione e Attacco, 18,24,30,34 20.000 mo (I), 40.000 mo
(II), 70.000 mo (III), 100.000 mo (IV)

Questo anello speciale puo' essere di quattro tipi diversi (anello delle Essenze I, anello delle Essenz II, anello delle Essenz III, e anello delle Essenz IV), tutti destinati all'utilizzo da parte di incantatore. Oltre al bonus indicato concedono +1d6 alle prove di concentrazione.

un anello della stregoneria I concede +1d6 alla prova di Competenza Magica per lanciare una Essenza specifica

un anello della stregoneria II concede +2d6 alla prova di Competenza Magica per lanciare una Essenza specifica

un anello della stregoneria III concede +3d6 alla prova di Competenza Magica per lanciare una Essenza specifica

un anello della stregoneria IV concede +4d6 alla prova di Competenza Magica per lanciare una Essenza specifica

\textbf{Anello di Eludere}\index{Anello di Eludere}

Essenza Difesa 24 , 25.000 mo

Questo anello conferisce costantemente a chi lo indossa l'agilita' di evitare i danni come se fosse dotato della abilita' di Eludere. Ogni volta che chi lo indossa supera un Tiro Salvezza su Riflessi per dimezzare i danni di un attacco, non subisce alcun danno. 

\textbf{Anello di Visione a Raggi X}\index{Anello di Visione a Raggi X}

Essenza Rivelazione 24, 25.000 mo

A comando, questo anello conferisce a chi lo indossa la capacita' di vedere all'interno e attraverso la materia solida. La portata della vista e' di 3 metri, entro i quali chi lo indossa e' in grado di vedere tutto come se si trovasse esposto alla normale luce esterna, anche senza illuminazione. La vista a raggi X e' in grado di oltrepassare 30 cm di pietra, 2,5 cm di metalli comuni o fino a 90 cm di legno o terra.

Sostanze piu' spesse o una sottile lamina di piombo impediscono la vista. Usare questo anello e' fisicamente stancante, e provoca a chi lo indossa 1 danno a Potenza per ogni minuto successivo ai primi 10 minuti di utilizzo in un giorno. L'anello deve essere utilizzato in incrementi di 1 minuto.

\textbf{Anello dell'Intermittenza}\index{Anello dell'Intermittenza}

Essenza movimento 24, 27000

A comando, questo anello rende chi lo indossa intermittente. La creature ottiene +4 in difesa e se colpito puo' effettuare un Tiro Salvezza su Riflessi sul Tiro per colpire dell'avversario, se lo passa non si viene colpiti.

\textbf{Anello di Liberta' di Movimento}\index{Anello di Liberta' di Movimento}

Essenza Movimento 28, 40000 mo

Questo anello d'oro permette a chi lo indossa di muoversi senza difficolta' date da movimenti difficili, non puo' essere ammanettato o legato, non puo' essere spostato o fermato magicamente,

\textbf{Anello di Stelle Cadenti}\index{Anello di Stelle Cadenti}

Essenza Attacco, Rivelazioni 28, 50000 mo

Questo anello puo' funzionare in due modi: se si trova in zone con luce scarsa o all'esterno durante la notte; quando chi lo indossa si trova sottoterra o in interni durante la notte. Durante la notte, a cielo aperto o in zone con luce scarsa o buio, l'anello delle stelle cadenti puo' effettuare a comando le seguenti funzioni:

\textbf{Luci Danzanti} (una volta ogni ora)

\textbf{Luce} (due volte per notte)

\textbf{Globo di Fulmini} (una volta per notte, all'aperto)

\textbf{Stelle Cadenti} (tre volte a settimana, all'aperto)

La prima funzione crea 4 luci (che illuminano in raggio di 3 metri) che puoi comandare singolarmente e spostare dove vuoi, entro distanza di 18 metri. Durata 1 ora.

La seconda funzione crea una luce di raggio 6 metri, durata 10 minuti.

Globo di fulmini genera delle sfere luminose sono simili al del luci danzanti, e chi lo indossa le controlla nel movimento. Le sfere hanno una gittata di media lunghezza e una durata di 4 round, e possono essere mosse a una velocita' di 36 metri (2 azioni) per round. Ogni sfera ha un diametro di circa 90 cm, e qualsiasi creatura che arrivi a distanza di mischia da una sfera ne scarica l'energia e ne subisce i relativi danni da elettricita', in base al numero delle sfere create.

Numero di Sfere Danno per Sfera

1 sfera 4d6 danni da elettricita'

2 sfere 3d6 danni da elettricita' ciascuna

3 sfere 2d6 danni da elettricita' ciascuna

4 sfere 1d6 danni da elettricita' ciascuna

Una volta attivata la funzione globo di fulmini, le sfere possono essere lanciate in qualsiasi momento prima del sorgere del sole (piu' sfere possono essere lanciate nello stesso round).

La funzione Stelle Cadenti produce tre stelle cadenti, che possono essere generate dall'anello ogni settimana, simultaneamente o una alla volta. Il loro impatto infligge 12 danni e si propaga (come fossero palle di fuoco) in una sfera del raggio di mischia infliggendo 24 danni da fuoco.

Qualsiasi creatura colpita da una stella cadente subisce pieni danni dall'impatto e pieni danni dalla propagazione a meno che non superi un tiro salvezza su Riflessi con DC 13.

Le creature non colpite dalla stella ma all'interno dell'area di propagazione ignorano i danni da impatto e subiscono solo meta' dei danni da fuoco se superano un tiro salvezza su Riflessi con DC 13. La gittata e' di media lunghezza, al limite dei quali la stella cadente esplode a meno che non colpisca prima una creatura o un oggetto. Una stella cadente segue sempre un percorso in linea retta, e qualsiasi creatura sulla sua traiettoria deve superare un tiro salvezza per non essere colpita. 

Di notte al chiuso o sottoterra, l'anello delle stelle cadenti ha le seguenti proprieta':

\textbf{Luminescenza} (due volte al giorno)

Pioggia di scintille (speciale, una volta al giorno)

Luminescenza crea una luce pari ad una torcia, 3 m raggio, 1 ora per
uso.

La pioggia di scintille e' una nuvola volante di scintille violacee scoppiettanti che si sprigionano dall'anello fino a una distanza di 3 metri in un arco di dimensioni di 3 metri. Le creature all'interno dell'area subiscono 2d8 danni ognuna se non indossano armature di metallo o non trasportano armi di metallo, nel caso contrario i danni diventano 4d8.

\textbf{Anello di Telecinesi}\index{Anello di Telecinesi}

Essenza Movimento 26, 75000 mo

Questo anello permette a chi lo indossa permette di spostare fino a 100Kg di materia per 30 minuti al giorno. Se usato su una creatura viene concesso un Tiro Salvezza su Arbitrio DC 18 per resistere, se il Tiro Salvezza e' riuscito quell'anello non funzionara' piu' per 24 ore su quel soggetto.

La capacita' va usata in incrementi di 1 minuto alla volta

\textbf{Anello di Rigenerazione}\index{Anello di Rigenerazione}

Essenza Cura 28, 90000 mo

Questo anello d'oro bianco con incastonato un grosso zaffiro verde conferisce continuamente ad una creatura vivente che lo indossa la capacita' di guarire 1 pf a turno (ogni 10 minuti) e l'immunita' ai danni da Sanguinamento. Se chi lo indossa perde un arto, un organo o qualsiasi altra parte del corpo mentre indossa questo anello, l'anello lo rigenera con gli stessi effetti dell'Essenza Cura per Rigenerazione.

\textbf{Anello di Riflettere Essenza}\index{Anello di Riflettere Essenza}

Essenza Protezione 28, 100000 mo

A comando, una volta al giorno, questo semplice anello di platino riflette automaticamente fino a LP 24.

\textbf{Anello dei Tre Desideri}\index{Anello dei Tre Desideri}

Essenze tutte 28, 120.000 mo

Su questo anello sono incastonati tre rubini: ogni rubino contiene la capacita' di esaudire un desiderio. L'anello fara' di tutto per fuorviare la richiesta.

\textbf{Anello del Richiamo del Djinni}\index{Anello del Richiamo del Djinni}

Essenza Convocazione 28,

Questo anello dei geni, uno dei molti che vengono menzionati nelle favole, e' un oggetto estremamente versatile. Agisce come un Portale speciale attraverso il quale uno specifico Djinni puo' essere evocato dal Piano Elementale dell'Aria. Quando l'anello viene strofinato (2 Azioni), il richiamo parte e il djinni compare al round seguente. Il djinni obbedisce fedelmente come servitore a chi lo indossa, ma mai per piu' di 1 ora al giorno. Se il djinni dell'anello dovesse essere ucciso, l'anello perderebbe ogni caratteristica magica e ogni valore.

\textbf{Anello di Comando degli Elementi}\index{Anello di Comando degli Elementi}

Essenza Convocazione, Charme 31, 200.000 mo

Tutti e quattro i tipi di anello del comando degli elementali sono molto potenti. Ognuno ha l'aspetto di un comune anello magico minore finché non viene attivato completamente (al verificarsi di certe condizioni come uccidere un elementale del tipo appropriato con un solo colpo o esporsi a materiale sacro dell'elemento appropriato), nel qual caso rivela alcuni poteri specifici oltre alle proprieta' comuni sotto descritte.

Gli elementali del piano a cui l'anello e' legato non possono attaccare chi lo indossa, e nemmeno avvicinarsi a piu' di distanza di mischia da lui. Se chi lo indossa lo desidera, puo' rinunciare a questa protezione e tentare invece di rendere l'elementale soggetto all'Essenza Charme, Tiro Salvezza su Volonta' con DC 17 nega, ma se questo fallisce, tuttavia, la protezione viene perduta e non si possono fare ulteriori tentativi di charme.

Chi indossa l'anello e' in grado di parlare con le creature del piano a cui l'anello e' legato. Le creature in questione capiscono che il personaggio sta indossando quel genere di anello, e mostrano un profondo rispetto per lui se i loro allineamenti sono affini. Se gli allineamenti sono opposti, le creature avranno paura di lui, qualora si dimostri forte con loro. Se e' debole, tuttavia, lo odieranno e tenteranno di ucciderlo.

Chi indossa un anello del comando degli elementali subisce le seguenti
penalita' ai tiri salvezza:

\begin{tabular}[c]{@{}ll@{}}
\toprule 
Elemento & Penalita' ai tiri salvezza\tabularnewline
Acqua & -2 tiri salvezza basati sul fuoco\tabularnewline
Aria & -2 tiri salvezza basati sul terra\tabularnewline
Fuoco & -2 tiri salvezza basati sul freddo o acqua\tabularnewline
Terra & -2 tiri salvezza basati sul aria o elettricita'\tabularnewline
\bottomrule
\end{tabular}

Oltre ai poteri sopra descritti, ogni specifico anello conferisce
a chi lo indossa le seguenti capacita' relative al suo elemento.

\textbf{Comando degli Elementi} (Acqua)\index{Comando degli Elementi Acqua}

Camminare sull'Acqua (uso illimitato)

Creare Acqua (uso illimitato)

Respirare Sott'Acqua (uso illimitato)

Muro di Ghiaccio (una volta al giorno), come Essenza Creazione LP 16

Tempesta di Ghiaccio (due volte alla settimana), come Essenza Attacco LP 16

L'anello ha l'aspetto di un Anello di Camminare sull'Acqua finché non si verificano certe condizioni.

\textbf{Comando degli Elementi} (Aria)\index{Comando degli Elementi Aria}

Caduta Morbida (uso illimitato, solo per chi lo indossa), annulli il danno da caduta

Resistere all'Energia (elettricita') (uso illimitato, solo per chi lo indossa), protegge da 5 di danno

Folata di Vento (due volte al giorno), Essenza Creazione LP 11

Muro di Vento (uso illimitato), Essenza Creazione 1LP 1

Camminare nell'Aria (una volta al giorno, solo per chi lo indossa, 1 ora)

Catena di Fulmini (una volta alla settimana), Essenza Attacco LP 16

L'anello ha l'aspetto di un Anello di Caduta Morbida finché non si verificano le condizioni per attivarne il pieno potenziale. Deve essere riattivato ogni volta che viene indossato da una nuova creatura. 

\textbf{Comando degli elementi (Fuoco)}\index{Comando degli elementi (Fuoco)}

Resistere all'Energia (fuoco) , protegge da 12 di danno a round

Mani Brucianti (uso illimitato), Essenza Attacco LP 11

Sfera Infuocata (due volte al giorno), Essenza Attacco 1LP 1

Pirotecnica (due volte al giorno), Essenza Creazione LP 13

Muro di Fuoco (una volta al giorno), Essenza Creazione LP 11

Colpo Infuocato (due volte alla settimana), Essenza Attacco LP 16

L'anello ha l'aspetto di un Anello di Resistenza all'Energia maggiore (fuoco) finché non si verificano le condizioni stabilite.

\textbf{Comando degli Elementi (Terra)}\index{Comando degli Elementi (Terra)}

Fondersi nella Pietra (uso illimitato, solo per chi lo indossa), Essenza Trasformazione LP 11

Ammorbidire Terra e Pietra (uso illimitato), Essenza Trasformazione LP 13

Scolpire Pietra (due volte al giorno), Essenza Trasformazione LP 16

Pelle di Pietra (una volta alla settimana, solo per chi lo indossa), Essenza Difesa LP 16

Passapareti (due volte alla settimana), Essenza Movimento LP 13

Muro di Pietra (una volta al giorno), Essenza Creazione LP 16

L'anello ha l'aspetto di un Anello di Fondersi nella Pietra (permettendo chi lo indossa di Fondersi Nella Pietra finché non si verificano le condizioni stabilite.



\section{Bastoni Magici Speciali}\index{Bastoni Magici}

\label{bastoni-magici-speciali}

Un bastone e' un lungo pezzo di legno in grado di contenere diverse Essenze. Al momento della sua creazione ha 10 cariche a disposizione.

Lanciare un incantesimo tramite un bastone costa in genere 2 Azioni. Per essere in grado di attivare un bastone, un personaggio deve tenerlo almeno con una mano (o quello che svolge le funzioni delle mani, nel caso di creature non umanoidi).

Tirate un d100. Un risultato di 01--30 indica che qualcosa (un disegno, un'iscrizione, ecc.) fornisce qualche indizio sulla sua funzione, mentre un risultato di 31--100 indica che l'oggetto non ha rune particolari.

I bastoni utilizzano i punteggi di statistica e le Abilita' relativi di chi li impugna per determinare la DC dei Tiri Salvezza contro i loro incantesimi. A differenza di altri oggetti magici, chi lo impugna puo' utilizzare la propria CM al posto di quello del bastone. Cio' vuol dire che i bastoni risultano molto piu' efficaci nelle mani di potenti incantatori.

I bastoni contengono un massimo di 10 cariche. Ogni Essenza lanciata
da un bastone consuma una o piu' cariche.

Quando termina le sue cariche, il bastone non puo' piu' essere usato finché non viene ricaricato. Ogni mattina l'incantatore puo' infondere una parte del suo potere nel bastone sempre che uno o piu' essenze conteniti nel bastone siano uguali alle sue.

Per infondere l'Essenza nel bastone e' sufficiente superare una prova di competenza magica pari al livello di potere dell'Essenza indicata nel bastone +4. Infondere potere nel bastone in questo modo ristora al bastone una carica e l'incantatore avra' un -4 a tutti i CM del giorno (ma non risultera' avere consumato una delle magie giornaliere)

Un bastone non puo' guadagnare piu' di una carica al giorno ed un incantatore non puo' infondere cariche in piu' di un bastone al giorno.

\bigskip

Tabella Bastoni

\begin{tabular}[c]{@{}llll@{}}
\toprule 
Medio & Maggiore & Nome & Prezzo\tabularnewline
01-15 & 01-03 & Bastone dello Tuono \index{Bastone dello Tuono} & 17600\tabularnewline
16-30 & 04-09 & Bastone del Fuoco \index{Bastone del Fuoco} & 18950\tabularnewline
31-40 & 10-11 & Bastone del Ghiaccio \index{Bastone del Ghiaccio} & 22800\tabularnewline
41-55 & 12-13 & Bastone dell'Alterazione della Taglia \index{Bastone dell'Alterazione della Taglia}& 26150\tabularnewline
56-75 & 14-19 & Bastone della Cura \index{Bastone della Cura}& 29600\tabularnewline
76-90 & 20-24 & Bastone del Fortunato \index{Bastone del Fortunato} & 41400\tabularnewline
91-95 & 25-31 & Bastone d'Illuminazione \index{Bastone d'Illuminazione} & 51500\tabularnewline
96-99 & 31-37 & Bastone della Difesa \index{Bastone della Difesa} & 82000\tabularnewline
100 & 38-45 & Bastone della Attacco \index{Bastone della Attacco}& 82000\tabularnewline
- & 46-52 & Bastone della Rivelazione \index{Bastone della Rivelazione} & 82000\tabularnewline
- & 53-60 & Bastone dello Charme \index{Bastone dello Charme} & 82000\tabularnewline
- & 61-68 & Bastone della Protezione \index{Bastone della Protezione} & 82000\tabularnewline
- & 69-76 & Bastone della Trasformazione \index{Bastone della Trasformazione} & 82000\tabularnewline
- & 77-85 & Bastone dell'Alterazione \index{Bastone dell'Alterazione} & 82000\tabularnewline
- & 86-93 & Bastone del Movimento \index{Bastone del Movimento} & 82000\tabularnewline
- & 94-98 & Bastone Nero (Distruzione) \index{Bastone Nero (Distruzione)} & 82000\tabularnewline
- & 99 & Ritira 1 volta su Medio e 1 volta su Maggiore & 130000\tabularnewline
- & 100 & Ritira 2 volta su Medio e 1 volta su Maggiore & 180000\tabularnewline
\bottomrule
\end{tabular}

\textbf{Bastone del Tuono}\index{Bastone del Tuono}

Essenza Illusione, Attacco LP 15

Il bastone quando usato emette un forte boato e genera luce in raggio di 3 metri e puo' lanciare per singola carica un piccolo fulmine da 4d6 entro distanza di 18 metri (solo dopo aver tuonato), DC 25 TS Riflessi dimezza

\textbf{Bastone del Fuoco}\index{Bastone del Fuoco}

Essenza Attacco, Trasformazione LP 15

Il bastone quando usato si trasforma in una lingua di fuoco (+1d6 di danno se usato come arma) e puo' lanciare per singola carica un globo di fuoco da 3d6 di danno entro 18 metri, DC 25 TS Riflessi dimezza

\textbf{Bastone del Ghiaccio}\index{Bastone del Ghiaccio}

Essenza Attacco, Trasformazione LP 15

Il bastone quando usato puo' congelare 9 metri cubi di acqua (un cubo di lato lungo mischia) puo' lanciare per singola carica un globo di ghiaccio da 3d6 di danno entro 18 metri, DC 25 TS Riflessi dimezza 

\textbf{Bastone dell'Alterazione della Taglia}\index{Bastone dell'Alterazione della Taglia}

Essenza Alterazione, LP 18

Il bastone permette per singola carica di aumentare di una taglia di dimensione, compreso cio' che si sta portando su di se. Durata 1 ora.

\textbf{Bastone del Fortunato}\index{Bastone del Fortunato}

Essenza Movimento, LP 18

Questo ``bastone'' non piu' grande di uno stuzzica denti, deve essere a contatto con il proprietario per poter essere attivato. Come azione immediata, anche dopo aver saputo il risultato di un tiro di dado e saputo se si supera o meno al prova (Tiro per Colpire, tiro salvezza, check\ldots ) puo' usare una carica per tirare 1d6 ulteriore.

\textbf{Bastone d'Illuminazione}\index{Bastone d'Illuminazione}

Essenza Creazione, LP 15

Questo bastone puo' illumiare in un raggio entro 6 metri. Usando una carica e' possibile creare delle zone di luce di raggio 12 metri, durata 8 ore.

\textbf{Bastone dell'Essenza...}\index{Bastone dell'Essenza...}

Essenza Vari, LP18

Il bastone concede un bonus di +2 alle prove di CM per l'Essenza del Bastone. Usando una carica del bastone si aumenta il singolo check di magia (CM) di 2d6.

\pagebreak

\section{Oggetti Maledetti}\index{Oggetti Maledetti}

\label{oggetti-maledetti}

Gli oggetti maledetti sono oggetti magici dotati di un'influenza potenzialmente negativa sul personaggio. A volte tendono a confondere il male con il bene, costringendo il loro possessore a fare scelte difficili. 

Gli oggetti maledetti non sono mai realizzati intenzionalmente, ma piuttosto sono il risultato di un lavoro mal riuscito, di artigiani con poca esperienza o della mancanza di componenti adeguate.

La maggior parte di questi oggetti funziona, ma non nel senso che si voleva e il loro uso produce inconvenienti dannosi.

Quando una prova di creazione di un oggetto magico fallisce di 5 o piu', tirate sulla tabella per determinare il tipo di maledizione che l'oggetto possiede.

\textbf{Tabella: Maledizioni Comuni degli Oggetti}

\begin{tabular}[c]{@{}ll@{}}
\toprule 
\% & Maledizione\tabularnewline
01-15 & Inganno\tabularnewline
16-40 & Effetto o Bersaglio Opposto\tabularnewline
41-50 & Funzionamento Discontinuo\tabularnewline
51-65 & Requisito\tabularnewline
66-90 & Inconveniente\tabularnewline
91-100 & Effetto completamente diverso\tabularnewline
\bottomrule
\end{tabular}

Gli oggetti maledetti sono identificati come qualsiasi altro oggetto magico con una sola eccezione: a meno che la prova effettuata per identificare l'oggetto non ecceda la DC di 10 (successo critico) o piu', la maledizione non viene individuata. Se la prova non eccede 10 o piu', ma riesce comunque, tutto quello che viene rivelato e' l'originale scopo dell'oggetto magico.

Se si sa che l'oggetto e' maledetto, la natura della maledizione puo' essere determinata usando la DC standard per identificare l'oggetto. 

\textbf{Rimuovere Oggetti Maledetti}\index{Rimuovere Oggetti Maledetti}

Mentre alcuni oggetti maledetti possono essere semplicemente posati, altri esercitano una forte compulsione sul possessore a tenerli con sé, a qualsiasi costo. Altri riappaiono anche se abbandonati o e' impossibile gettarli via.

Questi oggetti possono essere rimossi solo dopo che sul personaggio o l'oggetto viene lanciato una Essenza di Protezione Rimuovi Maledizione. La DC della prova di livello dell'incantatore per rimuovere la maledizione e' pari a 10 + CM dell'incantatore che ha creato l'oggetto.

Se la prova ha successo, l'oggetto puo' essere rimosso nel round successivo, ma la maledizione rimane e colpisce nuovamente se l'oggetto viene usato un'altra volta.

\textbf{Effetti Comuni degli Oggetti Maledetti}

Gli effetti piu' comuni degli oggetti maledetti sono i seguenti. I GM possono inventare nuovi effetti particolari per specifici oggetti maledetti.

\textbf{Inganno}

Chi utilizza l'oggetto continua a credere che sia cio' che sembra a prima vista, ma in realta' non ha alcun potere, a parte quello di ingannare. Chi lo usa e' mentalmente spinto a credere che funzioni, e non puo' essere convinto del contrario se non con l'uso d'Essenza di Protezione rimuovi maledizione

\textbf{Effetto o Bersaglio Opposto}

Questi oggetti maledetti tendono ad avere dei difetti di funzionamento che in alcuni casi generano effetti diametralmente opposti a quelli desiderati dal loro creatore, mentre in altri casi tendono a colpire chi li utilizza invece di qualcun altro.

Ma la cosa piu' interessante e' che questi oggetti potrebbero anche non essere uno svantaggio per chi li possiede. La categoria degli oggetti magici dagli effetti opposti include anche le armi che infliggono penalita' ai tiri per colpire e per i danni, invece che bonus.

Visto che un personaggio non dovrebbe sapere immediatamente quale sia il bonus di un oggetto magico, non dovrebbe venire a conoscenza nemmeno della natura della sua maledizione. Una volta che lo verra' a sapere, comunque, l'oggetto potra' essere abbandonato a meno che su di esso non vi sia qualche effetto magico che costringa il suo possessore a tenerlo e ad usarlo.

In questi casi, per liberarsi dall'oggetto sara' necessario l'Essenza di Protezione rimuovi maledizione

\textbf{Funzionamento Discontinuo}

Gli oggetti discontinui funzionano esattamente come dovrebbero, quando funzionano. Le tre tipologie a cui possono appartenere sono:

\textbf{Inaffidabile}: Ogni volta che l'oggetto viene attivato, c'e' una probabilita' del 5\% che non funzioni.

\textbf{Condizionato}: Questo oggetto funziona solo in determinate situazioni. Per determinare quali siano, scegliete una condizione di attivazione o consultato la tabella poco sotto.

\textbf{Incontrollabile}: Un oggetto incontrollabile tende ad attivarsi casualmente. Tirare un d\% ogni giorno. Con un risultato di 01--05 l'oggetto si attiva spontaneamente in un certo momento del giorno.

\begin{tabular}[c]{@{}ll@{}}
\toprule 
\% & Situazione\tabularnewline
01-03 & Temperatura sotto lo zero\tabularnewline
04-05 & Temperatura sopra lo zero\tabularnewline
06-10 & Durante il giorno\tabularnewline
11-15 & Durante la notte\tabularnewline
16-20 & Esposto alla luce solare\tabularnewline
21-25 & In assenza di luce solare\tabularnewline
26-34 & Sott'acqua\tabularnewline
35-37 & Fuori dall'acqua\tabularnewline
38-45 & Sottoterra\tabularnewline
46-55 & In superficie\tabularnewline
56-60 & Entro 3 metri da un tipo di creatura casuale\tabularnewline
61-64 & Entro 3 metri da una razza o tipo di creatura casuale\tabularnewline
65-72 & Entro 3 metri da un incantatore\tabularnewline
73-80 & Entro 3 metri da un incantatore di un Patrono specifico\tabularnewline
81-85 & Nelle mani di un personaggio non incantatore\tabularnewline
86-90 & Nelle mani di un personaggio incantatore\tabularnewline
91-95 & Nelle mani di una creatura con particolare tratto\tabularnewline
96 & Nelle mani di una creatura di un particolare sesso\tabularnewline
97-99 & Nei giorni non sacri o durante particolari ricorrenze astrologiche\tabularnewline
100 & A piu' di 150 km da un determinato luogo\tabularnewline
\bottomrule
\end{tabular}

\textbf{Requisito}

Alcuni oggetti hanno requisiti molto piu' difficili da soddisfare perché
funzionino. Per far funzionare l'oggetto in questione, potrebbe essere
necessario soddisfare una delle seguenti condizioni:
\begin{itemize}
\item Il personaggio deve mangiare il doppio del normale. 
\item Il personaggio deve dormire il doppio del normale. 
\item Il personaggio deve compiere una missione specifica (solo una volta,
poi l'oggetto funziona normalmente). 
\item Il personaggio deve sacrificare (distruggere) un valore pari a 100
mo di oggetti o materiali preziosi al giorno. 
\item Il personaggio deve sacrificare (distruggere) un valore pari a 2000
mo di oggetti magici ogni settimana. 
\item Il personaggio deve giurare lealta' ad un nobile in particolare, o
alla sua famiglia. 
\item Il personaggio deve abbandonare tutti gli altri oggetti magici. 
\item Il personaggio deve venerare una particolare Dio 
\item Il personaggio deve cambiare il suo nome in un altro. L'oggetto funziona
solo per i personaggi con un certo nome. 
\item Il personaggio deve avere un numero minimo di gradi in una particolare
competenza. 
\item Il personaggio deve sacrificare parte della propria energia vitale
(1 punto di Potenza) la prima volta che usa l'oggetto. Se il personaggio
trova un modo di recuperare i punti di Potenza persi, l'oggetto smette
immediatamente di funzionare. L'oggetto non smette di funzionare se
il personaggio guadagna punti di Potenza in seguito all'avanzamento
di livello o di un altro oggetto magico. 
\item L'oggetto deve essere purificato con l'acqua santa ogni giorno. 
\item L'oggetto deve essere usato per uccidere una creatura vivente al giorno. 
\item L'oggetto deve essere immerso nella lava vulcanica una volta al mese. 
\item L'oggetto deve essere usato almeno una volta al giorno, o smette di
funzionare per il suo attuale possessore. 
\item Quando viene brandito, l'oggetto deve spillare sangue (solo armi).
Non puo' essere messo da parte o cambiato con un altro oggetto finché
non ha messo a segno un colpo. 
\end{itemize}
I requisiti dipendono cosi' tanto dalla convenienza dell'oggetto che
non dovrebbero mai essere determinati a caso. Un oggetto intelligente
con un requisito spesso impone il proprio requisito grazie alla sua
personalita'. 

Se il requisito non viene soddisfatto, l'oggetto smette
di funzionare. Se invece viene soddisfatto, di solito l'oggetto funziona
per un giorno intero prima di dover di nuovo soddisfare il requisito
(anche se alcuni requisiti vanno soddisfatti una volta sola, altri
una volta al mese e altri ancora in continuazione).

\textbf{Inconveniente}

Gli oggetti che hanno degli inconvenienti hanno solitamente degli effetti positivi su chi li usa, ma hanno anche degli aspetti negativi. Anche se a volte gli inconvenienti vengono alla luce solo quando gli oggetti sono utilizzati (o tenuti in mano, nel caso di oggetti come le armi), di solito rimangono presenti fino a quando il personaggio non si libera dell'oggetto in questione.

A meno che non sia indicato diversamente, gli inconvenienti rimangono attivi per tutto il tempo in cui l'oggetto rimane in possesso del personaggio. La DC dei Tiro Salvezza per evitare questi effetti e' pari a 10 + il livello dell'incantatore dell'oggetto.

\bigskip

\begin{tabular}[c]{@{}ll@{}}
\toprule 
\% & Inconveniente\tabularnewline
01-04 & I capelli del PG crescono di 2,5 cm all'ora.\tabularnewline
05-09 & L'altezza del PG diminuisce di 30 cm (risultato di 01--50
su un d\%) oppure \\
&aumenta della stessa misura (un risultato di 51--100).\\
&Accade solo una volta.\tabularnewline
10-13 & La temperatura intorno all'oggetto e' di 5° C piu' fredda del normale.\tabularnewline
14-17 & La temperatura intorno all'oggetto e' di 5° C piu' calda del normale.\tabularnewline
18-21 & Il colore dei capelli del PG cambia.\tabularnewline
22-25 & II colore della pelle del PG cambia.\tabularnewline
26-29 & II PG ora porta un segno distintivo (un tatuaggio, una strana
luminescenza ecc.).\tabularnewline
30-32 & II sesso del PG cambia.\tabularnewline
33-34 & La razza o la specie del PG cambiano.\tabularnewline
35 & II PG viene colpito da una Malattia determinata casualmente,
che non puo' essere curata.\tabularnewline
36-39 & L'oggetto emette costantemente suoni sgradevoli (lamenti, maledizioni, insulti...).\tabularnewline
40 & L'oggetto ha un aspetto ridicolo (colori sgargianti, forma,brilla di un alone rosa ecc.).\tabularnewline
41-45 & II PG diventa estremamente possessivo nei confronti dell'oggetto.\tabularnewline
46-49 & II PG ha una paura incontrollabile di perdere l'oggetto o
che venga danneggiato.\tabularnewline
50-51 & Un tratto viene cambiato\tabularnewline
52-54 & II PG deve attaccare la creatura a lui piu' vicina (probabilita'
del 5\% ogni giorno).\tabularnewline
55-57 & II PG rimane Stordito per 1d4 round una volta che l'oggetto
e' servito al suo scopo\\
&(o casualmente 1 volta al giorno).\tabularnewline
58-60 & La vista del PG e' sfocata (penalita' --2 agli attacchi,
ai Tiri Salvezza e alle prove di Abilita'\\
&che richiedono la vista).\tabularnewline
61-64 & II PG guadagna un livello negativo.\tabularnewline
65 & II PG guadagna due livelli negativi.\tabularnewline
66-70 & II PG deve effettuare un TS su Volonta' ogni giorno
o subisce 1 danno a Intelletto.\tabularnewline
71-75 & Il PG deve effettuare un TS su Volonta' ogni giorno
o subisce 1 danno a volonta'.\tabularnewline
76-80 & II PG deve effettuare un TS su Volonta' ogni giorno
o subisce 1 danno a Magnetismo.\tabularnewline
81-85 & II PG deve effettuare un TS su Tempra ogni giorno
o subisce 1 danno a Potenza.\tabularnewline
86-90 & II PG deve effettuare un TS su Tempra ogni giorno
o subisce 1 danno a Agilita'.\tabularnewline
91-95 & II PG deve effettuare un TS su Tempra ogni giorno
o subisce 1 danno a Potenza ed Agilita'.\tabularnewline
96 & II PG viene trasformato in una creatura specifica (probabilita'
del 5\% ogni giorno).\tabularnewline
97 & II PG non puo' piu' usare Essenze con difficolta' oltre 18\tabularnewline
98 & II PG non puo' piu' usare Essenze con Livello di Potere oltre
15\tabularnewline
99 & II PG non puo' piu' usare Essenze\tabularnewline
100 & Tira due volte\tabularnewline
\bottomrule
\end{tabular}

\pagebreak

\section{Yeru}\index{Yeru}\index{Atilantis}

\label{yeru}

Yeru e' il pianeta di riferimento di TUS. Un pianeta spaccato sia fisicamente che magicamente.

Intorno a Yeru ruotano due stelle Sparka e Andhakara.\index{key}\index{Andhakara}

Sparka e' di un caldo colore dorato e' colei che porta calore e luce, attorno a lei Yeru fa un giro completo in 336 giorni da 24 ore l'uno. 

Sparka illumina sempre e solo l'emisfero nord di Yeru, chiamato Curyan.

Andhakara illumina sempre e solo l'emisfero sud di Yeru, Tiya, ed e' invece una stella azzurra e fredda, priva di vita, e' colei che porta tempeste energetiche e strani accadimenti naturali. Porta una fredda penombra.

Yeru compie il suo giro completo attorno a lui in 336 giorni da 24 ore l'uno.

Se le 14 (06-20) ore diurne vedono Sparka e Andhakara protagoniste in questa loro danza nel cielo; le 10 ore notturne vedono come totali protagoniste le due lune di Yeru di nome Idam e Kevatu. Gli abitanti di Yeru le chiamano le loro lune anche se in realta' non sono propriamente solo lune ma veri e propri pianeti abitati.

Le due lune sono grandi ed imponenti sul cielo notturno, Idam di un colore grigio rossastro e Kevatu di un caldo grigio madreperlato comandano le maree e influenzano con la loro presenza la navigazione. 

Yeru ha una distribuzione delle terre peculiare ed unica forse frutto del capriccio degli Dei della Genesi (Ljust e Calicante), potete immaginarlo come un sistema a spirale che partendo dal polo nord avvolge il pianeta fino all'equatore. 

Specularmente dal polo sud parte una spirale che arriva fino all'equatore. Le terre non si uniscono all'equatore, lasciando circa 50 km di mare aperto.

Le terre presenti sono grandi (e molto!) e piccole isole, non un sistema continentale unico.

Le terre emerse che compongono emisfero nord ed emisfero sud sono fra loro quasi simmetriche e simbiotiche. Forma e suddivisione delle grandi isole sono fra loro molto similari. Ma dal punto di vista morfologico e climatico ci sono profonde disparita'.

La zona di mare aperto di confine e' selvaggia ed imperscrutabile. Le piu' profonde e potenti tempeste scaricano di continuo la loro forza ed anche la magianon riesce a penetrare. Nell'occhio di questo perenne e gigantesco maelstrom c'e' la civilizzata e potentissima Atilantis, da molti ritenuta una isola leggendaria e culla della civilta'.

Curyan e' governata dalla forza della vita, questa regione vive una sorta di perenne calda stagione con gradazioni di temperatura e fenomeni atmosferici che cambiano dal picco di Yeru al sua massima estensione. Si incrociano zone dal clima torrido e umido ad altre con un caldo secco e senza precipitazioni; esistono fenomeni quali tempeste di sabbia nelle zone desertiche e tempeste tropicali forti e devastanti nelle lussureggianti baie centrali.

Ci sono territori piacevolmente caldi, rinfrescati da brezze fresche provenienti dai ghiacciai del nord.

Tiya invece e' un emisfero semi avvizzito, la luce che arriva basta appena a permettere l'agricoltura e gli animali hanno tutti un aspetto pallido ed emaciato.

E' l'emisfero dove vige la legge del piu' forte, dove si lotta per vivere e pochi sono gli stati che hanno un sistema di protezione efficace.

Il mare che abbraccia l'equatoriale e' forte e tumultuoso, pochissime barche si avventurano da un continente all'altro anche solo nel viaggio di 4 settimane nel punto piu' prossimo (escludendo la follia di attraversare il maelstrom), questo porta che scambi tra Tiya e Curyan siano estremamente ridotti.

\pagebreak

\section{I Portali}\index{Portali}

\label{i-portali}

In un mondo dove i trasferimenti marittimi non funzionano se non tra isola e isola dello stesso emisfero, l'Essenza di Movimento ed in particolare la capacita' di usare dei portali per trasferire merci e persone ha preso piede in maniera significativa.

Questo proliferare di piccoli, grandi, duraturi o istantanei tunnel ha causato uno squarcio nel tessuto dimensionale di Yeru generando a sua volta un proliferare di tunnel spontanei piu' o meno grandi e duraturi.

E questi Portali sono la causa di tantissimi problemi sia a Tiya che a Curyan in quanto non solo legano i due emisferi ma collegano tutta Yeru ad altri mondi (o almeno cosi' si pensa dato che pochi sono tornati per riferirlo..).

Ci sono portali conosciuti e stabili, fino ad ora, che collegano Tiya a Curyan, quasi tutto sotto il controllo, per non dire dentro il castello, di reali o potenti.

Ci sono zone dove piu' frequentemente si aprono portali ma la destinazione non e' sempre certa.

Poi ci sono i portali dei draghi. I draghi non sono nativi di Yeru ma sono stati attirati da queste porte magiche, causando scompiglio e terrore a Tiya e Curyan.

I draghi hanno ben compreso la natura di Yeru e con la loro fine intelligenza e innata capacita' di plasmare la magia hanno costruito i loro portali facendo venire centinaia di draghi. Tutti malvagi.

Si, su Yeru non ci sono draghi ``buoni'' se non con poche eccezioni.

Si e' sempre cercato di distruggere i portali dei draghi, con sacrificio e sangue. Molti sono stati distrutti, altri sono stati generati. E' una guerra senza fine, l'unica che puo' unire le persone dei due emisferi. 

\pagebreak

\section{Il Calendario}\index{Calendario}

\label{il-calendario}

Basato sul ciclo lunare presenta 12 mesi da 28 giorni.

Questi i nomi dei mesi a partire da quello che si definisce inizio
anno
\bigskip

1°) Ianas

2°) Prineva

3°) Marc

4°) Epral

5°) Meea

6°) Vernam

7°) Ilai

8°) Arkast

9°) Cester

10°) Koper

11°) Narava

12°) Kartan

\bigskip
La settimana e' a sua volta diviso in 7 giorni di nome

Kalint

Iratam

Munrat

Arai

Venran

Kittam

Viltar

Il giorno e' diviso in 24 ore

\pagebreak

\section{Condizioni}\index{Condizioni}

\label{condizioni}

\textbf{Abbagliato}:\index{Abbagliato} La creatura e' incapace di vedere bene a causa di un'eccessiva stimolazione degli occhi. Una creatura abbagliata subisce penalita' -1 al Tiro per Colpire e alle prove di Consapevolezza basate sulla vista.

\textbf{Accecato:}\index{Accecato} Il personaggio non riesce a vedere nulla. Subisce penalita' -2 alla Difesa, perde il suo bonus di Agilita' alla Difesa (se presente), subisce penalita' -2 alla maggior parte delle Competenze basate su Potenza e Agilita' e alle prove contrapposte di Consapevolezza.

Tutte le prove o le attivita' basate sulla visione (come ad esempio leggere, o eventuali prove di Consapevolezza basate sulla vista) falliscono automaticamente. Tutti gli avversari vengono considerati dotati di Occultamento nei confronti del personaggio accecato. 

I personaggi accecati devono effettuare una prova di Acrobatica con DC 10 per muoversi piu' veloci della propria velocita' dimezzata. Le creature che falliscono questa prova cadono a terra Prone. I personaggi che rimangono per lungo tempo accecati possono abituarsi ad alcune di queste penalita' e iniziare a superarne alcune, a discrezione del Narratore.

\textbf{Accovacciato}:\index{Accovacciato} Un personaggio accovacciato subisce penalita' -2 alla Difesa e -1 al Tiro per Colpire, perde due punti di bonus di Agilita' (se posseduti).

\textbf{Affascinato}:\index{Affascinato} Una creatura affascinata e' soggiogata da un effetto soprannaturale o di una Essenza. La creatura rimane in piedi o si siede tranquilla, senza effettuare alcuna azione se non prestare attenzione alla fonte del fascino, fintanto che dura l'effetto. L'effetto provoca penalita' -4 alle Prove di Competenza richieste come reazione, come ad esempio le prove di Consapevolezza. 

Qualsiasi potenziale minaccia, come ad esempio una creatura ostile in avvicinamento, consente alla creatura affascinata un nuovo Tiro Salvezza contro l'effetto del fascino. Qualsiasi minaccia palese, come ad esempio qualcuno che estrae un'arma, lancia un'Essenza o punta un'arma a distanza verso la creatura affascinata, interrompe automaticamente l'effetto. 

Un alleato della creatura affascinata puo' scuoterla per liberarla dall'effetto spendendo 2 Azioni.

\textbf{Affaticato}\index{Affaticato}: Un personaggio affaticato non puo' correre o Caricare e subisce una penalita' -1 a Potenza e Agilita'. Se compie qualsiasi cosa normalmente affaticante diventa Esausto. 

Ci vogliono 8 ore di riposo totale per rimuovere la condizione di affaticato o Cura a LP16 . Se un personaggio non dorme almeno 8 ore alla mattina e' affaticato.

\textbf{Amichevole}:\index{Amichevole} Una creatura amichevole non attacchera' il personaggio se non minacciata esplicitamente.

\textbf{Assordato}:\index{Assordato} Un personaggio assordato non puo' ascoltare. Subisce penalita' -2 alle prove di Iniziativa, fallisce automaticamente tutte le prove di Consapevolezza basate sul suono e ha una probabilita' del 20\% di fallire l'uso delle Essenze, presupponiamo che tutte le essenze abbiano componenti verbali e somatiche.

I personaggi che rimangono assordati per lunghi periodi di tempo, possono abituarsi a queste penalita' e superarne alcune, a discrezione del Narratore.

\textbf{Charmato}:\index{Charmato} una creatura charmata tratta il giocatore con un fidato amico ed alleato. Se la creatura viene minacciata o attaccata puo' fare un nuovo Tiro Salvezza su Arbitrio con un +2.

L'effetto di charme non permetto il controllo del target ma questo percepisce le tue parole nel modo piu' favorevole. Puoi anche dare ordini ma devi riuscire in una prova di Faccia Tosta contro un Tiro Salvezza su Arbitrio.

Un target influenzato da charme non fara' nulla di pericoloso per se stesso (tranne se convinto) o per altri soggetti che reputa amici.

\textbf{Confuso}: \index{Confuso}Una creatura confusa e' mentalmente ottenebrata e non puo' agire normalmente. Una creatura confusa non riesce a distinguere un alleato da un nemico e considera tutti come nemici.

Gli alleati che vogliono utilizzare un'Essenza a vantaggio della creatura confusa devono comunque toccarla con un attacco di contatto in mischia riuscito.

Se una creatura confusa e' attaccata, attacca sempre l'ultima creatura che la ha attaccata, finché quella creatura non muore o esce dalla sua visuale.

Tirate un dado sulla tabella seguente all'inizio di ogni turno della
creatura confusa ad ogni round per vedere quello che la creatura fa
in quel round.

\textbf{d100 Comportamento:}

01-25 Agisce normalmente

26-50 Non fa altro che balbettare in modo incoerente

51-75 Si infligge 1d8 + modificatore di Potenza con l'arma che tiene in mano

76-100 Attacca la creatura piu' vicina (a tale scopo, un Famiglio conta come parte del soggetto stesso)

Una creatura confusa che non e' in grado di eseguire l'azione indicata non fara' altro che balbettare in modo incoerente. Gli aggressori non hanno alcun vantaggio speciale quando attaccano una creatura confusa. Qualsiasi creatura confusa che venga attaccata, attacca automaticamente a sua volta il suo aggressore al suo turno successivo, fintanto che rimane confusa quando giunge il suo turno.

\textbf{Dominato}:\index{Dominato} si e' in grado di controllare le azioni di una qualsiasi creatura Umanoide mediante un legame telepatico con la mente del soggetto.

Se si ha un linguaggio in comune, si puo' generalmente costringere il soggetto ad eseguire i comandi entro i limiti delle sue capacita'. Se non si condivide nessun linguaggio, si possono impartire solo comandi di base come ``vieni qui'', ``vai li''', ``combatti'' o ``stai fermo''. Si e' a conoscenza di cio' che il soggetto sta provando ma non si ricevono percezioni sensoriali dirette da lui, né si puo' comunicare con lui telepaticamente.

Una volta impartito un ordine alla creatura dominata, questa continua a tentare di eseguirlo con l'esclusione di tutte le altre attivita' ad eccezione di quelle necessarie per la sopravvivenza quotidiana (come mangiare, dormire e cosi' via). Grazie a questo limitato spettro di attivita', una prova di Consapevolezza con DC 15 (invece che DC 25) puo' determinare se il comportamento del soggetto e' stato influenzato da un effetto di ammaliamento.

Concentrandosi completamente sull'Essenza (2 Azioni), si possono ricevere percezioni sensoriali come vengono interpretate dalla mente del soggetto, anche se questo non puo' comunque comunicarle. Non si puo' in realta' vedere attraverso gli occhi del soggetto, quindi non e' come se si fosse presenti, ma ci si puo' rendere conto di cosa sta succedendo. 

Ovviamente ordini palesemente autodistruttivi non vengono eseguiti. Una volta stabilito il controllo, il raggio di azione entro il quale puo' essere mantenuto e' illimitato purché entrambi i soggetti rimangano sullo stesso piano. Non c'e' bisogno di vedere il soggetto per controllarlo. Se ogni giorno non si trascorre almeno 1 minuto a concentrarsi sull'Essenza, il soggetto riceve un nuovo Tiro Salvezza per liberarsi dal controllo.

\textbf{Esausto}:\index{Esausto} Un personaggio esausto si muove a velocita' dimezzata e subisce penalita' -3 a Potenza e Agilita'. Dopo 1 ora di completo riposo (o Cura LP19), un personaggio esausto diventa solo Affaticato. Un personaggio Affaticato diventa esausto compiendo un'azione che normalmente lo affaticherebbe.

\textbf{Frastornato}:\index{Frastornato} La creatura e' incapace di agire normalmente.
Una creatura frastornata non puo' eseguire azioni, ma non subisce penalita' alla CA o Difesa. La condizione frastornato dura solitamente 1 round.

\textbf{Immobilizzato}:\index{Immobilizzato} Una creatura immobilizzata e' strettamente limitata nei movimenti e puo' compiere solo alcune azioni.

Una creatura immobilizzata non puo' muoversi ed e' Impreparata. Inoltre subisce una ulteriore penalita' -4 alla sua Difesa. Una creatura immobilizzata e' limitata nelle azioni che puo' compiere. Una creatura immobilizzata puo' tentare sempre di liberarsi, solitamente attraverso una prova di Artista della Fuga o un Tiro Salvezza su riflessi.

Puo' compiere azioni verbali e mentali, ma non puo' utilizzare, di norma, le Essenze. Un personaggio immobilizzato che tenta di utilizzare le Essenze o usare una Capacita' Magica deve effettuare una prova di Concentrazione. 

Se il soggetto e' legato la prova e' contro la prova di Criminalita' di chi ha legato.

\textbf{Impreparato}:\index{Impreparato} Un personaggio che non ha ancora agito in combattimento e' impreparato, non potendo ancora reagire alla situazione. Un personaggio impreparato perde il suo bonus di Agilita' alla Difesa (se presente)

\textbf{In Lotta}:\index{Lotta} Una creatura in lotta e' trattenuta da una creatura,
da una trappola o da un effetto. Le creature in lotta non possono
muoversi e subiscono penalita' -2 a Agilita'. Una creatura in lotta
subisce penalita' -2 a Tiro per Colpire e Difesa. Inoltre, le creature
in lotta non possono compiere azioni che richiedano due mani per essere
effettuate.

\textbf{Incorporeo}:\index{Incorporeo} Le creature di questo tipo non possiedono un corpo fisico. Le creature incorporee possono essere colpite solo da armi magiche con un bonus di +2 o superiore. Dimezzano gli effetti di essenze con DC inferiore o pari a 18. Le creature incorporee subiscono danno pieno da altri soggetti ed effetti incorporei, cosi' come tutti gli effetti di forza.

Una creatura incorporea puo' entrare in un oggetto corporeo o passarvi attraverso,

Gli attacchi di una creatura incorporea passano attraverso (ignorano) armature non magiche e scudi, solo la naturale Agilita' e appunto armature/scudi magici offrono resistenza.

Le creature incorporee possono muoversi ed agire normalmente nell'acqua come nell'aria. Le creature incorporee non possono cadere e subire danni da caduta.

Le creature incorporee non possono effettuare attacchi per Sbilanciare o Lottare, né possono essere sbilanciate o afferrate.

Le creature incorporee non hanno peso, e non fanno scattare trappole attivate dal peso.

Una creatura incorporea si muove sempre silenziosamente e non puo' essere sentita con Consapevolezza a meno che non lo desideri. Non ha punteggio di Potenza, e si applica il suo bonus di Agilita' agli attacchi in mischia ed a distanza

\textbf{Indifeso}:\index{Indifeso} Un personaggio addormentato, bloccato, legato, Paralizzato, Privo di Sensi o per qualche altro motivo completamente alla merce' dei suoi avversari, e' considerato indifeso.

Un personaggio indifeso viene considerato come se avesse Agilita' -5 (modificatore di -5 alla Difesa). Gli attacchi in mischia contro un personaggio indifeso ottengono bonus +1d6 (equivalente ad attaccare un personaggio Prono).

Gli attacchi a distanza, non ottengono alcun bonus particolare contro i bersagli indifesi.

\textbf{Colpo di Grazia}:\index{Colpo di Grazia} Come unica azione nel round, un nemico puo' utilizzare un'arma da mischia per infliggere un colpo di grazia ad un personaggio indifeso. Un nemico puo' anche usare un arco o una balestra, l'importante e' che sia adiacente al bersaglio.

L'attaccante colpisce automaticamente ed infligge un colpo critico. Se il difensore sopravvive, deve superare un Tiro Salvezza su Tempra (DC 10 + danni inflitti) o muore.

Le creature immuni ai colpi critici, non subiscono danni critici, né devono superare un Tiro Salvezza su Tempra per evitare di essere uccisi da un colpo di grazia.

\textbf{Infermo}:\index{Infermo} Un personaggio infermo subisce una penalita' -2 a tutti i Tiri per Colpire e per i danni delle armi, ai Tiri Salvezza, alle Prove di Competenza.

\textbf{Intralciato}:\index{Intralciato} Il personaggio rimane intrappolato. Un personaggio intralciato ha difficolta' di movimento, ma puo' comunque provare a muoversi, a meno che i legami che lo intralciano non siano ancorati a un oggetto immobile o impugnati da una forza contrapposta.

Una creatura intralciata puo' muoversi a velocita' dimezzata ma non puo' Correre o Caricare, e subisce penalita' -2 ai Tiri per colpire e penalita' -1 alla Agilita'. 

Un personaggio intralciato che cerca di lanciare una Essenza deve superare una prova di Concentrazione (DC 15) o perde la magia.

\textbf{Invisibile}:\index{Invisibile} Le creature invisibili non sono percepibili dalla vista, ricevono bonus +1d6 al Tiro per Colpire contro avversari visibili e negano il bonus di Agilita' alla Difesa dei loro avversari (se posseduto).

\textbf{Livelli Negativi}:\index{Livelli Negativi} Ci sono creature non morte con capacita' soprannaturali ed Essenze con effetti magici che possono risucchiare la vita e l'energia vitale; questo terrificante attacco e' noto come Risucchio di Energia e comporta dei Livelli Negativi che infliggono delle penalita' alle creature.

Se i livelli negativi di una creatura sono uguali o superiori ai suoi Dadi Vita totali, la creatura muore.

\textbf{Livelli Negativi Temporanei:} Una creatura con livelli negativi temporanei ha diritto ogni giorno ad un nuovo Tiro Salvezza per rimuovere il livello negativo. La DC di questo Tiro Salvezza e' la stessa dell'effetto che ha causato i livelli negativi.

\textbf{Livelli Negativi Permanenti:} Alcune capacita' e l'Essenza Distruzione comportano un risucchio di livello permanente ad una creatura. Questi sono trattati come livelli negativi temporanei, ma non permettono di effettuare ogni giorno un nuovo Tiro Salvezza per rimuoverli. 

\textbf{Ristorare Livelli Negativi}: Solo l'Essenza di Cura permette di recuperare i livelli persi, vuoi causati da mostri o da Distruzione. Una creatura portata a livelli negativi, ovvero sotto zero, e' morta e non si puo' riportare in vita o recuperare i livelli mancanti.

\textbf{Morente} \index{Morente}Un personaggio morente ha -1 Punti Ferita ed e' Privo di Sensi e prossimo alla morte.

\textbf{Morto}:\index{Morto} L'anima del personaggio abbandona permanentemente il suo corpo. I personaggi morti non possono beneficiare delle cure normali o magiche, e non possono essere riportati in vita da una Essenza. Solo un Dio ha sufficiente potere per riportare l'anima nel corpo e riportare in vita la creatura. L'Essenza di Distruzione puo' rianimare un corpo, ma come non morto.

\textbf{Nauseato}:\index{Nauseato} Le creature nauseate soffrono di disturbi di stomaco.
Le creature nauseate non sono in grado di attaccare, utilizzare Essenze, concentrarsi sulle Essenze o fare qualsiasi altra cosa che richieda attenzione. La sola azione che un tale personaggio puo' compiere e' una singola Azione di movimento per turno.

\textbf{Paralizzato}: \index{Paralizzato}Un personaggio paralizzato e' bloccato sul posto ed e' incapace di muoversi od agire. Ha punteggi effettivi di Potenza e Agilita' pari a 0, e' Indifeso e puo' compiere azioni esclusivamente mentali.

Una creatura alata in volo, nel momento in cui viene paralizzata non puo' piu' battere le ali e precipita. 
Un nuotatore paralizzato non puo' piu' Nuotare e potrebbe annegare. 

Una creatura puo' attraversare una zona occupata da una creatura paralizzata (o morta), che sia un alleato o meno e si considera come terreno difficile.

\textbf{Pietrificato}: \index{Pietrificato}Un personaggio pietrificato e' stato trasformato in pietra ed e' privo di sensi. Se un personaggio pietrificato si incrina o si rompe, ma i pezzi rotti sono uniti al corpo quando ritorna di carne, il personaggio non viene ferito o danneggiato. Se il corpo pietrificato del personaggio e' incompleto quando viene ritrasformato in carne, il corpo rimane incompleto e potrebbe avere una qualche perdita permanente di punti ferita e/o altre menomazioni.

\textbf{Paura}:\index{Paura} Essenze, Oggetti Magici e certe creature possono influenzare i personaggi con paura. In molti casi, il personaggio deve effettuare un Tiro Salvezza su Arbitrio per resistere agli effetti, e un tiro fallito indica che il personaggio e' scosso, spaventato o in preda al panico.

\textbf{Prono}\index{Prono}: chi e' prono ha un -1d6 ad attaccare ed un -4 alla Difesa. Alzarsi da prono costa 2 Azioni.

\textbf{Scosso}:\index{Scosso}I personaggi che sono scossi subiscono penalita' morale -2 ai Tiri per Colpire, ai Tiri Salvezza e alle prove.

\textbf{Spaventato}:\index{Spaventato} I personaggi spaventati sono anche scossi, e inoltre fuggono dalla fonte della loro paura il piu' velocemente possibile, anche se possono scegliere la direzione di fuga. 

A parte cio', una volta che sono fuori vista (o udito) dalla fonte della loro paura, possono agire normalmente. Se la durata della paura non e' ancora arrivata al termine, qualora dovessero incontrare di nuovo la fonte della loro paura, cercherebbero nuovamente di fuggire.

I personaggi che non sono in grado di fuggire possono combattere (anche se continuano ad essere scossi).

\textbf{Stordito/Svenuto}:\index{Stordito}\index{Svenuto} Agilita' alla Difesa e subisce una penalita' di -2 ulteriore alla Difesa.

\textbf{In Preda al Panico}:\index{In Preda al Panico} I personaggi in preda al panico sono scossi e, inoltre, hanno una probabilita' del 50\% di far cadere a terra qualsiasi cosa stanno tenendo in mano e di fuggire dalla fonte del loro terrore il piu' in fretta possibile seguendo un percorso di fuga completamente casuale.

I personaggi in preda al panico fuggono davanti a qualsiasi altro pericolo che possano trovarsi di fronte. A parte cio', una volta che sono fuori vista (o udito) dalla fonte della loro paura, possono agire normalmente.

I personaggi in preda al panico prendono anche la condizione Accovacciato se non possono fuggire.

\textbf{Terrore Crescente}:\index{Terrore Crescente} Gli effetti della paura sono cumulativi.

Un personaggio scosso che viene nuovamente scosso diventa spaventato, mentre invece un personaggio scosso che viene spaventato cade in preda al panico. Un personaggio spaventato che viene scosso o spaventato cade in preda al panico.

\textbf{Rotto}\index{Rotto}

La condizione rotto ha i seguenti effetti, a seconda dell'oggetto:

- Se l'oggetto e' un'arma, tutti gli attacchi effettuati con l'oggetto
subiscono penalita' -2 al CA per colpire e per i danni. Tali armi
ottengono un Colpo Critico soltanto con un 4 volte 6 naturale ed infliggono
solo 1 volta il danno in aggiunta.

- Se l'oggetto e' un'armatura o uno scudo, il bonus che concede alla
Difesa e' dimezzato, arrotondando per difetto. L'armatura rotta raddoppia
la penalita' di armatura alla Prova sulle Competenze.

- Se l'oggetto e' un attrezzo necessario per una Competenza, tutte
le prove di Competenza effettuate con esso subiscono penalita' -2.

- Se l'oggetto e' una Bacchetta o un Bastone, utilizzate il doppio
delle cariche necessarie ogni volta che viene usato.

- Se l'oggetto non rientra in nessuna delle precedenti categorie,
la condizione rotto non ha effetto sul suo uso. Gli oggetti con condizione
rotto, a prescindere dal tipo, valgono il 75\% del loro costo normale.
Se l'oggetto e' magico, puo' essere riparato soltanto con l'Essenza
di Creazione utilizzata da un incantatore di livello uguale o superiore
a quello che ha creato dell'oggetto.

\textbf{Sanguinante}\index{Sanguinante} Una creatura che sta subendo danni da sanguinamento subisce la quantita' di danno indicata all'inizio del suo turno. Il sanguinamento puo' essere interrotto superando una prova di sopravvivenza (pronto soccorso) con DC 15 o con l'uso di Essenza di Cura.

Alcuni effetti di sanguinamento causano un danno di caratteristica o persino un risucchio di caratteristica. Gli effetti di sanguinamento non si cumulano a meno che non causino tipi differenti di danno.

Quando due o piu' effetti di sanguinamento causano lo stesso tipo di danno, si tenga l'effetto peggiore. In questo caso, il risucchio di caratteristica e' peggiore del danno di caratteristica.

\pagebreak

\section{Scheda e Manuale}\index{Scheda}

\label{scheda-e-manuale}

Come TUS anche la scheda e' in continua evoluzione. Qui sotto trovate una delle ultime versioni della scheda pronta per essere stampata e del manuale di TUS. \url https://github.com/buzzqw/TUS

\begin{figure}
%%	\centering
	\includegraphics[width=1.13\linewidth]{TUS-Scheda-1.png}
\end{figure}

\begin{figure}
%%	\centering
	\includegraphics[width=1.13\linewidth]{TUS-Scheda-2.png}
\end{figure}


\pagebreak

\section*{Conversione Mostri}\index{Conversione Mostri}

TUS e' di base un sistema D20 fortemente modificato nelle dinamiche ma non nelle fondamenta dei valori numerici. Prendiamo ad esempio l'Orco comune da https://www.d20pfsrd.com/bestiary/monster-listings/humanoids/orcs/orc/ , tralasciamo le parti descrittive e concentriamoci sui numeri e valori.

\bigskip

\textbf{Orc (CR 1/3)} \textgreater{} questo valore rimane il medesimo in TUS

\textbf{XP 135} \textgreater{} questo valore non ha piu' senso 

\textbf{Orc warrior 1} \textgreater{} non ci interessa

\textbf{CE Medium humanoid} \textgreater{} ci indica che la creatura e' di taglia media, umanoide e malvagio, ai fini dei tratti la creatura non e' di livello tale da aver attirato l'attenzione di un Patrono.

\textbf{Init +0} \textgreater{} e' l'iniziativa

\textbf{Senses} darkvision 60 ft.; Perception --1 \textgreater{} rimane uguale, si tengono gli stessi valori ed abilita'. In questo caso 60 piedi indica che la distanza e' di 20 metri

\textbf{Weakness} light sensitivity \textgreater{} si cerca dove possibile l'equivalente in TUS, in questo caso fotofobia leggera oppure si applicano direttamente le penalita' indicate.

\textbf{AC} 13, touch 10, flat-footed 13 (+3 armor) \textgreater{} la AC deve essere aumentata di 2 in ogni componente

\textbf{Competenza Armi}: e' pari al BAB indicato

\textbf{Competenza Magia}: di base e' meta' del CR. Utile solo se la creatura ha poteri magici.

\textbf{hp} 6 (1d10+1) \textgreater{} rimane uguale

\textbf{Fort} +3, Ref +0, Will --1 \textgreater{} si traducono in Tempra, Agilita' e Arbitrio. Il punteggio rimane uguale

\textbf{Speed} 30 ft. \textgreater{} e' il movimento, in questo
caso e' 9 metri

\textbf{Melee} falchion +5 (2d4+4/18--20) \textgreater{} E' il mio Tiro per Colpire e danno. Rimane uguale

\textbf{Ranged} javelin +1 (1d6+3) \textgreater{} E' il Tiro per Colpire. Rimane uguale

\textbf{Str} 17, Dex 11, Con 12, Int 7, Wis 8, Cha 6 \textgreater{} togli 10 dalla somma di Forza e Costituzione e poi fai la media arrotondata per eccesso per determinare la Potenza, in questo caso ((17+12)/2-10)/2=3. Devi prendere solo la parte bonus.

\textbf{Base Atk} +1; CMB +4; CMD 14 \textgreater{} il primo valore determina la CA. Suggerisco di usare direttamente i bonus al colpire indicati nel melee. 

\textbf{Feats} Weapon Focus (falchion) \textgreater{} Arma Focalizzata. Il bonus dell'abilita' e' gia' calcolata nei valori di Melee

\textbf{Skills} Intimidate +2 \textgreater{} rimane uguale. In questo caso e' Faccia Tosta

\textbf{Ferocity} (Ex): An orc remains conscious and can continue fighting even if its hit point total is below 0. It is still staggered and loses 1 hit point each round. A creature with ferocity still dies when its hit point total reaches a negative amount equal to its Constitution score. \textgreater{} si prende la abilita' cosi' come e'.

\pagebreak

\twocolumn

\printindex{}

\onecolumn

\label{indice-analitico}

\pagebreak

\section{Licenza}\index{Licenza}

Autore ed Ideatore: Andres Zanzani azanzani@gmail.com

\bigskip
Coautore: Roberta Giorgini madgiorgini@yahoo.it

\bigskip

Playtesting: Fabrizio Bonetti, Emanuele Pezzi, Nicola Ricottone, Marco
Valmori, Edoardo Zanzani, Isotta Zanzani, Federica Angeli
\bigskip

Un ringraziamento speciale a tutta la mia famiglia che mi ha sopportato
e supportato in questi anni disperati!

\bigskip

Licenza: Creative Commons Licenses,Attribution-NonCommercial-ShareAlike 4.0 International , \url https://creativecommons.org/licenses/by-nc-sa/4.0

questo il link in italiano \url https://creativecommons.org/licenses/by-nc-sa/4.0/deed.it

Gioca, condividi, ristampa, fa volantinaggio, pubblica recensioni fantastiche, tappezza la Piazza di copie cartacee e digitali di TUS!

Puoi anche modificare, abbellire, stravolgere e colorare con gessetti TUS. Prendi ispirazione per i tuoi moduli ed avventure!.

Ricorda pero' che devi riconoscere ed attribuire all'Autore originale la paternita' di TUS e fornire link all'edizione originale nonche' indicare che modifiche hai fatto e renderle disponibili con la stessa licenza di TUS

\bigskip
TUS e' un prodotto non commerciale: non puoi utilizzare il materiale per scopi commerciali e farci soldi. Avvisami prima.

\bigskip
Non puoi aggiungere restrizioni aggiuntive: non puoi applicare termini legali o misure tecnologiche che impongano ad altri soggetti dei vincoli giuridici su quanto la licenza consente loro di fare. Non sono fornite garanzie. La licenza puo' non conferirti tutte le autorizzazioni necessarie per l'utilizzo che ti prefiggi. Ad esempio, diritti di terzi come i diritti all'immagine, alla riservatezza e i diritti morali potrebbero restringere gli usi che ti prefiggi sul materiale (very legalese!).

\bigskip

Andres Zanzani

\pagebreak

{\small 
\textbf{OPEN GAME LICENSE Version 1.0a}\index{OPEN GAME LICENSE}

The following text is the property of Wizards of the Coast, Inc. and is Copyright 2000 Wizards of the Coast, Inc ("Wizards"). All RighTS Reserved.

1. Definitions: (a)"Contributors" means the copyright and/or trademark owners who have contributed Open Game Content; (b)"Derivative Material" means copyrighted material including derivative works and translations (including into other computer languages), potation, modification, correction, addition, extension, upgrade, improvement, compilation, abridNarratoreent or other form in which an existing work may be recast, transformed or adapted; (c) "Distribute" means to reproduce, license, rent, lease, sell, broadcast, publicly display, transmit or otherwise distribute; (d)"Open Game Content" means the game mechanic and includes the methods, procedures, processes and routines to the extent such content does not embody the Product Identity and is an enhancement over the prior art and any additional content clearly identified as Open Game Content by the Contributor, and means any work covered by this License, including translations and derivative works under copyright law, but specifically excludes Product Identity. (e) "Product Identity" means product and product line names, logos and identifying marks including trade dress; artifacts; creatures characters; stories, storylines, plots, thematic elements, dialogue, incidents, language, artwork, symbols, designs, depictions, likenesses, formats, poses, concepts, themes and graphic, photographic and other visual or audio representations; names and descriptions of characters, spells, enchantments, personalities, teams, personas, likenesses and special abilities; places, locations, environments, creatures, equipment, magical or supernatural abilities or effects, logos, symbols, or graphic designs; and any other trademark or registered trademark clearly identified as Product identity by the owner of the Product Identity, and which specifically excludes the Open Game Content; (f) "Trademark" means the logos, names, mark, sign, motto, designs that are used by a Contributor to identify itself or its products or the associated products contributed to the Open Game License by the Contributor (g) "Use", "Used" or "Using" means to use, Distribute, copy, edit, format, modify, translate and otherwise create Derivative Material of Open Game Content. (h) "You" or "Your" means the licensee in terms of this agreement.
\textbf{2}. The License: This License applies to any Open Game Content that contains a notice indicating that the Open Game Content may only be Used under and in terms of this License. You must affix such a notice to any Open Game Content that you Use. No terms may be added to or subtracted from this License except as described by the License itself. No other terms or conditions may be applied to any Open Game Content distributed using this License.
\textbf{3}.Offer and Acceptance: By Using the Open Game Content You indicate Your acceptance of the terms of this License.
\textbf{4}. Grant and Consideration: In consideration for agreeing to use this License, the Contributors grant You a perpetual, worldwide, royalty-free, non-exclusive license with the exact terms of this License to Use, the Open Game Content.
\textbf{5}.Representation of Authority to Contribute: If You are contributing original material as Open Game Content, You represent that Your Contributions are Your original creation and/or You have sufficient righCS to grant the righCS conveyed by this License.
\textbf{6}.Notice of License Copyright: You must update the COPYRIGHT NOTICE portion of this License to include the exact text of the COPYRIGHT NOTICE of any Open Game Content You are copying, modifying or distributing, and You must add the title, the copyright date, and the copyright holder's name to the COPYRIGHT NOTICE of any original Open Game Content you Distribute.
\textbf{7}. Use of Product Identity: You agree not to Use any Product Identity, including as an indication as to compatibility, except as expressly licensed in another, independent Agreement with the owner of each element of that Product Identity. You agree not to indicate compatibility or co-adaptability with any Trademark or Registered Trademark in conjunction with a work containing Open Game Content except as expressly licensed in another, independent Agreement with the owner ofsuch Trademark or Registered Trademark. The use of any Product Identity in Open Game Content does not constitute a challenge to the ownership of that Product Identity. The owner of any Product Identity used in Open Game Content shall retainall rights, title and interest in and to that Product 
Identity.
\textbf{8}. Identification: If you distribute Open Game Content You must clearly indicate which portions of the work that you are distributing are Open Game Content.
\textbf{9}. Updating the License: Wizards or iCS designated AgenCS may publish updated versions of this License. You may use any authorized version of this License to copy, modify and distribute any Open Game Content originally distributed under any version of this License.
\textbf{10} Copy of this License: You MUST include a copy of this License with every copy of the Open Game Content You Distribute.
\textbf{11}. Use of Contributor CrediCS: You may not market or advertise the Open Game Content using the name of any Contributor unless You have written permission from the Contributor to do so.
\textbf{12} Inability to Comply: If it is impossible for You to comply with any of the terms of this License with respect to some or all of the Open Game Content due to statute, judicial order, or governmental regulation then You may not Use any Open Game Material so affected.
\textbf{13} Termination: This License will terminate automatically if You fail to comply with all terms herein and fail to cure such breach within 30 days of becoming aware of the breach. All sublicenses shall survive the termination of this License.
\textbf{14} Reformation: If any provision of this License is held to be unenforceable, such provision shall be reformed only to the extent necessary to make it enforceable.
\textbf{15} COPYRIGHT NOTICE - Open Game License v 1.0 Copyright 2000, Wizards of the Coast, Inc.} 

\pagebreak

{\small 
\section{Changelog}\index{Changelog}

1.0.1 aggiunto changelog, aggiunti ambiti agli dei con vantaggi e svantaggi, modificato da +2 a +1 bonus da riselezionare essenza

1.0.2 layout, layout, layout, prime correzioni ad ambiente, prime correzioni a masterizzare

1.0.3 perfezionati e sistemati dei, aggiornate razze,

1.0.4 rivisti costi base magia, aggiornate descrizione magia

1.0.5 aggiornati dei

1.0.6 vari errori di scrittura, aggiunti kender (al posto di halfling), aggiunto sgambetto, modificato combattimenti a due mani, chiarimenti, dettagli su incanalare energia ed imposizioni delle mani, Fare infuriare, riviste divinata', armature 

1.0.7 aggiornati dettagli divinita', modificati penalita' al CM per Armature

07/06/2018 STAMPA

1.0.8 correzioni divinita', aggiornati Drow, aggiunto CRP, gestione attacchi multipli,

1.0.9 rifiniture, precisazione su armature, iniziativa carte 

1.0.10 modificati bonus e gestione armature, layout, riordinati e sistemati termini di base, aggiunti scudi alla tipologia di arma, modificati costi essenza distruzione su elementi, chiarimenti su magia, layout, piccole correzioni e chiarimenti

15/10/2018 STAMPA

2.0.10 nell'iniziativa con carte si pescano carte in base al valore di Intelletto se si lanciano Essenze, chiarimenti, piu' magie nello stesso round, rimosso danno da sanguinamento in distruzione e corretti riferimenti a potenze ed Agilita', precisazione su movimento

2.0.11 sistemato elenco armi semplici, aggiunti nuovi svantaggi, aggiornato incanalare energia, +1 caratteristiche ogni 4 livelli, reso piu' chiaro combattimento a due mani

21/11/2018 STAMPA

2.0.12 modificata e semplificata magia, ridotti costi essenza, rimosso residui di usare oggetti magici, modificato check concentrazione, piccole correzioni, aggiunto link a scheda online, modificato valori base delle caratteristiche, 0 e' normale, 1 buono, 2 ottimo , 3 eccezionale, precisazioni sul linguaggio, avviato Monster Manual, aggiunta levitazione e volo, aggiunta abilita' ferocia, aggiunta breve descrizioni di yeru e portali, modificata iniziativa

2.0.13 aggiunti tratti, ripristinate e modificati divinita' di Codex, modificato recupero pf dopo nottata, modificato il sistema di modificatore in base +1d6 o -1d6 a seconda di un bonus o malus con lo scopo di semplificare e ridurre tanti modifiche ad un vantaggio o svantaggio di dado, migliorata la terminologia dimagus e incatatore e devoto, aggiornate tabelle abilita' armi specifiche, aggiunte tabelle anelli magici e bastoni magici

24/12/2018 STAMPA

2.0.15 rilettura e correzioni varie, aggiunta versione semplificata della magia (scelta 2 spell), aggiunta parte come creare il personaggio, le abilita' si prendono ogni livello dispari, precisazione sulle razze ed patroni collegati, piccoli aggiornamenti su tridente e sassi, aggiunte frasi di inizio capitolo (wip), rimosso vincolo dei due tratti per essere un incantatore, corretto livello di potere nei bonus dati dall'affinita' di tratto 1=10, 2=13, 3=15, 4=17, corretta abilita' trattenere il respiro

2.0.16 correzioni varie, sostituito competenza armi e competenza difesa dove usata genericamente con Tiro per colpire e Difesa, aggiornata scheda, aggiornata scheda su manuale, aggiunto svantaggio seguire la legge, aggiornato svantaggio seguire il chaos, precisato e modificato collegamento tratti, Patrono e magia adesso e' vincolante scegliere un Patrono per avere la magia

2.0.17 piccole correzioni, ricalibrato livelli di potere, modificato esplosione del 6 nella magia, rimossi i kender, semplificata l'iniziativa, aggiunti dettagli in cappello cosmologia, aggiunta abilita' lo scudo e' mio amico, rivisti i tratti, rivisti assegnazione tratti a dei, rivisto costo contingenze, sostituito la dizione Tiro salvezza su volonta' in Tiro salvezza su Arbitrio per evitare di avere una volonta' come caratteristica e come nome del tiro salvezza, aggiornata la scheda con tiro salvezza su arbitrio, piccola correzione su anello dell'ariete, modificato Colpi Poderosi limitandone l'esplosione, chiarimenti e correzioni sulla magia, modificate alcune capacita' dati dai gruppi d'arma (in particolare spade, spade e scudo) , aggiornate descrizioni e poteri abilita', aggiornata e dettagliata tabella creazione e distruzione introdotto concetto di cubo base

2.0.18 semplificato combattimento a due mani, aggiunte Abilita' collegate al Energia Psichica, sostituito termine congiurazione con convocazione, adesso la convocazione convoca in base ai CR non agli HD, anche charme influenza sui CR

01/02/2019 STAMPA

2.0.19 modificata come l'essenza di difesa puo' essere usata come controincantesimo, dettaglio durata su essenza creazione, modficata iniziativa adesso e' Agilita' o Intelletto + CD, modificata Essenza Attacco: adesso e' basata su Intelletto, modificata iniziativa: Dal piu' veloce al piu' lento c'e' la risoluzione delle dichiarazioni e delle azioni, Fiancheggiare: corretto bonus a +2 al compire o difesa. default al colpire, Combattimento a due armi. Senza competenza hai un -4 al colpire su ogni arma, con la competenza il -2 rimane solo sulla secondaria, la Difesa adesso e' di base 11, modificato energiachi in Energia Interiore, un incantatore puo' formulare nel giorno un numero di Essenze pari a CM+3. Per ogni critico (esplosione di magia) che ottiene il numero di Essenze lanciabili nel giorno diminuisce di 1 a causa del grande stress fisico sostenuto, modificato il livello di potere, si parte da 11 ed aumento di 3 per ogni livello di potere (\textless11, 13, 16, 19, 22, 25, 28\ldots ), Essenza Illusione adesso e' basata su magnetismo, corretto Consapevolezza su volonta'

2.0.20 aggiunta Resistenza alla Magia, piccole correzioni, aggiornata tabella energia psichica

13/02/2019 STAMPA
2.1.0 cambiata gestione del movimento adesso basata su raggi di effetto, aggiornata la scheda, layout, correzioni, rifacimento indice, aggiornata scheda, avanzamento indice, correzione doppi spazi, correzioni, modificato metodo di ricarica dei bastoni magici, modificato livello di punti ferita per morte 10+3*pot, aggiunto chiarimento su applicazione veleno ad armi, aggiornati vantaggi, corretto sistema magia opzionale, abbassato base difesa a 10, eliminati alcuni residui di distanza in metri, corrette residui citazione quadretti 2.1.1. aggiornata magia con gestione differenziata dei target influenzabili, piccole correzioni su essenza attacco, creata tabella collegamento valore tratto bonus ricevuti, layout su patroni , aggiornata scheda, aggiornata parte di conversione mostri, correzioni, dettagli su check contrapposti, aggiunta magia basata su carte, specificazione su movimento, ulteriori chiarimenti su movimento, aggiornate armatura per nuova gestione movimento, correzioni su morte, chiarimenti su Seguace e Devoto, dettagli su neve e nebbia connessi a movimento

07/03/2019 STAMPA

2.1.2 modificato essenza cura per livello potere \textless11, gestione sorpresa reciproca, precisazioni sulla magia, in caso di critico magicp si puo' lanciare un incantesimo in piu' al giorno, correzioni, modificato numero massimo di essenze al giorno pari a cm/2+3, modificati parametri di conversione mostri, abbassato ilbonus di difesa delle armature 

2.1.3 correzioni, aggiunta abilita' furia, passare da distanza corta a mischia costa 1 azione

2.2.0 tolta competenza difesa, correzioni, precisazione su bonus dovuti al valore del tratto, senza competenza il check e’ 1d6, rimossa scurovisione adesso rimane solo la visione crepuscolare, se una essenza si risove con un tiro per compire in mischia si usa il valore di CM al posto di CA, aggiunta categoria armi versatili dove si puo’ usare agilita’ al posto della forza per tc e danno, le capacita’ del famiglio si basano non sul livello del padrone ma sulla sua competenza magica, aggiunto dettaglio a prova di concentrazione in caso di distrazione, sostituito riferimento da bonus di circostanza a bonus, aggiornata procedura calcolo distanze e movimenti, modificate azioni per round rese moltopiu’ snelle ed immediate nella scelta.

25/07/2019 STAMPA

2.3.0 aggiungo gestione afferrare, aggiunto gestione fare cadere, dettagli su prono, corretti numerosi riferimenti ad azione standard e di round completo, rimosso distanza generica e ripristinati movimenti a metri e quadretti,correzioni al layout, modificata gestione carica adesso e' possibile ad alti livelli fare piu' attacchi, dettaglio su bonus e malus quando usare dadi e quando valori, modifiche su prendere la mira adesso hai un bonus all'iniziativa del 4' round, modifiche a maestria del combattimento, modificata finta l'effetto e' fino alla fine del round, fare cadere specificato che e' un prova contrapposta di tiri salvezza potenza/agilita', dettagli su lista d'armi, aggiornate e riviste abilita' per lista armi, rimossi ultimi riferimenti a CD, aggiornate quasi tutte le abilita', chiarimenti su assegnazione cm a essenze, rimosso Assenza come suddivisione, aggiornato cubo base a cubo di 1m di lato, riviste essenze fino a distruzione .essenza movimento dettagli su tiro salvezza, correzione su esempio movimento, precisazione su illusione ed allarmi e contigenze, essenza movimento precisazioni importanti su caduta oggetti, modifiche importanti su essenza protezione rimossa la possibilta' di rimuovere stati, rivista essenza trasformazione su creature, aggiustamenti a vantaggi, aggiustamenti a svantaggi, correzioni in cosmologia, correzioni su poteri per valore tratto comune, chiarimenti su armature, aggiornata copertura, rifatta parte relativa a invisibilita' adesso e' piu' organico, introdotto concetto di reazione, impostata versione a 2.3.0, cecchino reso omogeneo le penalita' al colpire, modificate qualche abilita' perche' usino reazione o immediata, aggiornamenti layout, modificata penalita' per portare armatura senza competenza, adeguato penalita' movimento armatura a nuova gestione, aggiornate tabelle movimento, modificati valori capacita' di trasporto, aggiornata procedura creazione oggetti magici, corrett riparare oggetti magici, aggiornati triboli, modificato armiin argento alchemico, scritto meglio armi in adamantio, aggiunta spiegazione su importanza declamazione essenza, rimosso riferimenti residui a classe armatura, rimossi riferimenti ad azione gratuita (adesso e' immediata o di reazione), aggiunto abilita' Seguigio, aggiunta abilita' Esperto, correzioni varie, inserite regole per attacco di opportunita', adeguate Abilita' alla gestione delle Azioni, aggiornate essenze, specificato costo in azioni per abilita' legate ai tratti, aggiornati poteri dati dai tratti per trattare con le azioni, corretti valori difesa delle armature, aggiunto dettagli se prova fallisce di 10 o piu', gestione bonus recitazione, aggiornato sommario, corretti vari riferimenti ad azione generica, correzione nei termini comuni su Difesa,chiarimenti su attacchi multipli usando piu' azioni singole di attacco, chiarimenti su riuscita o fallimento tiro salvezza, indicazioni su fallimento o riuscita critica di un tiro salvezza, indicazioni di successo e fallimento critico nella descrizione delle essenze, aggiornata struttura indice analitico, chiarimenti su attacchi multipli usando piu' azioni singole di attacco, chiarimenti su riuscita o fallimento tiro salvezza, indicazioni su fallimento o riuscita critica di un tiro salvezza, indicazioni di successo e fallimento critico nella descrizione delle essenze, aggiornata struttura indice analitico

19/09/2019 STAMPA

2.3.1 layout, piccole correzioni, aggiornamento indice, correzioni in termini comuni, pagine al centro,aggiunta informazione su gestione personalizzata dei tratti,chiarmenti su prendere 10, chiarimenti su fornire aiuto, sistemazione layout, chiarimenti su scelta linguaggi, corretta lingua scritta da Terran, precisazione su piu' attacchi eseguiti con singoli attacchi, specificato quando azioni di reazioni/immediate possono intervenire, modificato bonus per differenza di taglia da +4 a +2, chiarimenti su prono, in caso di svenuto dopo 5 tiri si prende il risultato con piu' successi,aggiuna morte per danni temporanei, aggiunta abilita' Incantatore Combattente, chiarimenti su linguaggio per Fare Infuriare, chiarimenti su Tentare Essenza con impedimenti, chiarimenti su prova magia, chiarimenti su Eludere, chiarimenti su Forgiato nella furia, chiarimenti su kensai, chiarimenti su proseguire, correzioni, ridotto il bonus alla prova di senso trappola, chiarimenti su segugio, reso piu' italiano il manuale (tolti alcuni termini inglesi), reso piu' chiaro il recupero della essenza a seguito di critico, aggiornata decrizione tipi di essenze, chiarimenti su penalita' armature alla prove di agilita', corretti pesi armatura, dettagli su caduta, chiarimenti su caduta in vulcano, corrette ed aggiornate pozioni, chiarimento su dipendenza, aggiornata e corretta descrizione veleno da contatto, modificate probabilita' di auto avvelenarsi applicando il veleno, correzioni varie

14/01/2020 STAMPA

2.3.2 chiarimenti su fumble (tirare 3 volte 1 al TC) , modificato limite per attacco multipli (-4 invece che -5), colpo potente modificat da dove si puo' togliere il valore al colpire, allineato bonus/malus in maestria del combattimento a -4, dettaglio su critici su creature invisibile, chiarimenti su termini comuni, cambiere svantaggi di razza, dettagli su scelta tratti, la dc dei ts richiesti e' segreta, con intelletto 2+ una lingua in piu', chiarimenti su quando e' possibile prendere il 18, chiarimenti su quando eseguire reazioni ed azioni immediate, chiarimenti su attacchi con essenze, chiarimenti su considerare un 1d6 con un +-4, chiarimenti su disarmo, chiarimenti su spingere l'avversario, piccoli aggiornamenti su lista armi, aggiunta abilita' Guerriero dell'Essenza, effetti minimi di lanciare un essenza con impedimento, rinominata abilita' Incantatore Combattente in Incantatore Prudente, migliorata descrzione abilita' opportunista, scritto meglio passo tattico, specificato che percettivo si puo' prendere piu' volte ma il bonus aumenta solo di +1, persona veramente malvagia costa 1 azione, chiarito meglio quando si puo' usare rappresaglia, rinominato salto e schivo in toccata e fuga, dettagliato meglio schivata prodigiosa, piccole correzioni nelle abilita', in famiglio corretto Livello del padrone con CM del patrone nella tabella delle capacita' del famiglio, modificato scrutare su famiglio, corretto modificatore di intelletto del famiglio

3.0.0 passaggio a Latex, varie correzioni

3.0.1 aggiunta possibilita' di cambiare una Abilita' scelta, cambiato font, specificato che la specializzazione magica si usa anche nelle prove di concentrazione, aggiunti punti esperienza per oro guadagnato, aggiunti suggerimenti per il Narratore, aggiornati vantaggi, 
}

\end{document}

