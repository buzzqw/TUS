\subsection{La Magia (Semplificata v2)}\index{La Magia Semplificata v2}

\begin{multicols}{2}

Questa versione della Magia Semplificata prende la lista degli Incantesimi OGL dells 5ed del famoso Gioco di Ruolo (mi spiace, non e' possibile nominarlo per motivi di copyright!) e la adatta e modifica per TUS.


%controllare presenza parole: divinit , saggezza, intelligenza .., radiosa radiante

%a fine lavoro sostituire \medskip\textbf{Deflagrazione Occulta}\index{Deflagrazione Occulta}\\
%mettendo  }\index{Incantesimo - 

Rimangono in essere tutte le regole di TUS gia' espresse tranne per la Sezione della Magia, questo capitolo sostituisce in toto il capitolo.

Queste le regole da applicare:\\
\begin{itemize}

\item 
Ogni volta che il mago acquisisce un punto in Competenza Magica sceglie due incantesimi dalla lista. Questi possono avere qualsiasi Difficolta'.
\item 
In caso di personaggio Devoto la prova di magia ha un bonus di +2, se sei un Seguace +1.
Se non vengono usati i Tratti questi bonus non sono applicabili.
\item
Ogni volta che il mago acquisisce un punto in Competenza Magica è possibile dimenticare un incantesimo e sostituirlo con un altro dalla lista
\item 
Per poter lanciare una magia il mago deve superare la Difficolta' dell'incantesimo con la prova di Competenza Magica.\\
La prova si effettua con 3d6+CM+Intelletto o Volonta' (una volta fatta la scelta non si cambia più).
\item 
Un incantatore può formulare nel giorno un numero di magie pari a (CM/2)+3.
\item 
Gli incantesmi con difficolta' 11 non contano nel numero delle magie lanciate nel giorno. Va comunque fatta la prova di competenze.
\item
Se nel lancio di un incantesimp ottiene almeno un critico (due volte sei nel lancio dei dadi) non si computa quest magia per il numero di incantesimi lanciabili al giorno.
\end{itemize}

Portare un armatura senza le dovute competenze ed Abilità rende più difficile la prova di Competenza Magia. Vedere il capitolo armature per le penalità relative.

\subsubsection{Recitare la Magia}\index{Recitare l'Essenza}\index{Recitare}

Può sembrare sciocco o inutile ma se un giocatore non recita la sua magia questa non funzionerà.

Anche in questa versione di Magia Semplificata lasciate al giocatore declamare la sua potenza!
Aggiunge fervore alla scena e pathos al gioco.

Il giocatore declamera' la sua magia "Possa questa piana ardere come il Deserto di Fiamma di Daruk-Yum" e dira' che incantesimo usa.

\subsubsection{Tiro per Colpire ed Magie}\index{Tiro per Colpire ed Essenze}

Quando l'incantesimo ti dice di fare un attacco a distanza devi effettuare un TC a distanza contro la Difesa a Tocco dell'avversario (tranne se sorpreso che allora perde anche il bonus di Agilità). Questo Tiro per Colpire e' effettuato con 3d6+CA+Agilita' più Abilita' e modificatori vari.

Quando la magia e' ad area non e' necessario effettuare un TC se non per difficili e specificate aree, ovvero si mira in una area ben circoscritta.

\subsubsection{L'esplosione del 6 nella Magia}\index{esplosione del 6 nella Magia}

\label{lesplosione-del-6-nella-magia}

Anche nella prova di Competenza Magica i 6 esplodono, i 6 tirati nella prova di CM vengono ritirati, e ritirati ancora nel caso, e sommati alla prova di magia.

Per le prove di Competenza Magica l'uno non viene conteggiato, conta 0.

Tenete traccia di quanti critici (due 6 tirati) fate, potrebbero servire per avere effetti "speciali" nell'incantesimo!

\subsubsection{Tentare la sorte con la Magia}\index{Tentare la sorte con la Magia}

\label{tentare-la-sorte-con-la-magia}

Anche nella prova di competenza magica puoi Tentare la Sorte, ovvero rinunci ad un +4 di bonus (da CM, Intelletto, non da oggetti magici...) e aggiungi un d6 in più nel tiro della prova.

\subsubsection{Riuscire e Fallire nella prova di Magia}\index{Riuscire e Fallire nella prova di Magia}

Per capire se si riesce nella formulazione dell'incantesimo si deve innanzitutto superare, con una prova di Competenza Magica (3d6 + CM + Intelletto) la Difficoltà indicata nell'incantesimo stesso.

Se non si riesce a superare la Difficoltà l'incantesimo non si attiva e non sortisce effetto (anche se il Narratore potrebbe descrivere l'eventuale incantesimo scaturito tirando tre volte uno con i dadi...).

Anche se la prova fallisce l'incantesimo si conta tra quelli lanciati e i componenti eventuali sono consumati.

\subsubsection{Resistere all'Essenza (Tiro Salvezza)}\index{Resistere all'Essenza}\index{Tiro Salvezza}

\label{resistere-allessenza-tiro-salvezza}

Una volta che la prova di magia è superata e quindi la magia liberata, anche in base alla descrizione e note dell'incantesimo, è possibile resistere all'effetto della magia.

Il Tiro Salvezza in base a quanto richiesto dalla magia e' pari alla prova di magia effettuata dal mago. Una prova di magia particolarmente efficace rendera' altrettanto difficile resistergli e in base all'incantesimo usato ed eventuali critici ottenuti potranno manifestarsi risultati ancora migliori.

Nella descrizione dell'incantesimo e' descritto cosa succede in caso di critico nella prova di magia.

Se il Tiro Salvezza riesce o fallisce di più di 10 (\textbf{successo critico}\index{Successo Critico} o \textbf{fallimento critico}\index{Fallimento Critico}) il Narratore potrà decidere di applicare svantaggi o vantaggi al risultato finale.\index{Più di 10}.

\subsubsection{Resistenza alla Magia}\index{Resistenza alla Magia}

Una creatura potrebbe avere una naturale resistenza alle Essenze.

Il valore di RM (Resistenza Magia) indica tale resistenza e più è alta più la creatura è immune alla magia, che lo voglia o meno.

Ogni qual volta la creatura è influenzata direttamente da una magia deve effettuare una prova di RM, ovvero tirare 3d6 sommare il valore di RM e se è superiore al TS dell'incantesimo.

In caso di essenze scaturite da oggetti (anelli, bastoni, pozioni) la prova di RM deve superare la difficolta' dell'incantesimo.

\subsection{Check di Concentrazione}\index{Check di Concentrazione}

Se il mago viene è severamente distratto, impedito, disturbato, sotto attacco, mentre effettua una magia la prova di magia questa deve riuscire di almeno di 5 rispetto alla Difficoltà dell'incantesimo, altrimenti la "distrazione" e' stata tale da impedire il buon esito della magia.

Se il mago e' colpito prima di lanciare un incantesimo deve effettuare una prova di competenza magica (3d6+CM+Intelletto) che deve essere superiore a 10 + danno subito altrimenti l'incantesimo non riesce.

\subsection{Mantenere la Concentrazione}\index{Mantenere di Concentrazione}\index{Concentrazione}

Alcuni incantesimi hanno come Durata l'indicazione Concentrazione, il mago puo' quindi mantenere attivo l'incantesimo un tempo pari a 1 round per CM oppure quanto scritto nell'incantesimo.
Il mago non puo' pero' formulare altri incantesimi finche' mantiene la concentrazione attiva.

\subsection{Influenzati da più Magie}\index{Influenzati da più Magie}

Quando un personaggio è influenzato da \textbf{due o più effetti magici} che danno lo stesso tipo di bonus, malus o danno nello stesso round (protezione verso fuoco, bonus alla Difesa o TS... , multiple palle di acido), si tiene conto solo di quella che ha la Difficolta' maggiore.

\subsubsection{Alterare le Magie}\index{Alterare le Magie}

Il mago può modificare la difficoltà dell'incantesimo che va a formulare tramite le proprie energie vitali.

\begin{itemize}
	\item
	\textbf{Magia efficace}: sacrificando PF puo’ aumentare la difficolta’ a resistere alla magia
		\begin{itemize}
		\item Ogni 4 punti ferita la difficolta’ del Tiro Salvezza aumenta di 1
		\end{itemize}
	\item
	\textbf{Magia eterea}: aumentando di 3 la Difficoltà dell'incantesimo le proprie magie hanno pieno effetto su creature eteree o incorporee
	\item
	\textbf{Magia pietosa}: aumentando di 3 la Difficoltà di lancio le magie infliggono danni temporanei. 
	Le magie che infliggono danni di un tipo particolare (come da fuoco) infliggono danni temporanei dello stesso tipo.
\end{itemize}

\subsubsection{Tentare Incantesimi con impedimenti}\index{Tentare Incantesimi con impedimenti} \index{impedimenti}

Se mani e bocca sono bloccati l'incantatore non può formulare magie. Per lanciare un incantesimo è necessario avere entrambe le mani e la bocca liberi.

Aumentando di 5 la Difficoltà puoi non usare le mani, se aumenta di 10 la Difficoltà puoi non usare la bocca.

Quindi se l'incantatore è legato ed imbavagliato può lanciare una magia con la sola forza del pensiero con una difficoltà aumentata di 15, ovvero la Difficoltà dell'incantesimo aumenta di 15.

\pagebreak

%1	\textbf{Difficolta'}: 13\\
%2	\textbf{Difficolta'}: 15\\
%3	\textbf{Difficolta'}: 18\\
%4	\textbf{Difficolta'}: 20\\
%5	\textbf{Difficolta'}: 23\\
%6	\textbf{Difficolta'}: 25\\
%7	\textbf{Difficolta'}: 28\\
%8	\textbf{Difficolta'}: 30\\
%9	\textbf{Difficolta'}: 33\\



\textbf{Aiuto}\\\index{Magia - Aiuto}
\textbf{Difficolta'}: 2\\
\textbf{Tempo di Lancio}: 2 Azioni\\
\textbf{Gittata}: 9 metri\\
\textbf{Componenti}: V, S, M (una sottile striscia di tessuto bianco)\\
\textbf{Durata}: 8 ore\\
Il tuo incantesimo aumenta la robustezza e risolutezza dei tuoi alleati. Scegli fino a tre creature a gittata. Per la durata, i punti ferita massimi e i punti ferita attuali di ciascun bersaglio aumentano di 5.\\
\textbf{Per ogni Critico ottenuto} nella prova di magia i punti ferita del bersaglio aumentano di ulteriori 5 punti

\medskip\textbf{Allarme}\\\index{Magia - Allarme}
\textbf{Difficolta'}: 13\\
\textbf{Tempo di Lancio}: 1 minuto\\
\textbf{Gittata}: 9 metri\\
\textbf{Componenti}: V, S, M (una campanella e un pezzo di pregiato filo d’argento)\\
\textbf{Durata}: 8 ore\\
Predisponi un allarme contro intrusioni indesiderate. Scegli una porta, una finestra o un’area a gittata che non sia più grande di un cubo di 6 metri di spigolo. Fino al termine dell’incantesimo, sarai avvertito da un allarme ogni volta che una creatura di taglia Minuscola o superiore entri in contatto o acceda all’area protetta. Quando lanci l’incantesimo, puoi indicare delle creature che non faranno scattare l’allarme. Scegli anche se l’allarme è udibile o solo mentale. Un allarme mentale, qualora ti trovi entro 1,5 chilometri dall’area protetta, ti avverte con un rumore nella tua mente. Il rumore è in grado di svegliarti se stai dormendo. Un allarme udibile produce il suono di una campanella per 10 secondi, udibile entro 18 metri.

\medskip\textbf{Alleato Planare}\\\index{Magia - Alleato Planare}
\textbf{Difficolta'}: 25\\
\textbf{Tempo di Lancio}: 10 minuti\\
\textbf{Gittata}: 18 metri\\
\textbf{Componenti}: V, S\\
\textbf{Durata}: Istantanea\\
Supplichi un’entità ultraterrena perché ti conceda aiuto. L’essere ti deve essere noto: un dio, un primordiale, un principe dei demoni, o qualche altra creatura di grande potere. Quell’entità invia un celestiale, elementale o immondo a essa leale perché ti aiuti, facendo comparire la creatura in uno spazio non occupato a gittata. Se conosci il nome di una specifica creatura, puoi pronunciarne il nome quando lanci questo incantesimo per richiedere l’aiuto di quella creatura, sebbene tu possa comunque riceverne un’altra (a discrezione del Narratore).\\
Quando la creatura appare, non è sotto l’obbligo di agire in alcun modo particolare. Puoi chiedere alla creatura di svolgere un servizio in cambio di una ricompensa, ma essa non è obbligata a soddisfarti. Il compito richiesto potrebbe essere facile (“portaci in volo oltre il baratro” o “aiutaci a combattere questa battaglia”) o complesso (“spia i nostri nemici” o “proteggici durante la nostra esplorazione del sotterraneo”). Devi essere in grado di comunicare con la creatura per patteggiare i suoi servigi. La ricompensa può assumere diverse forme. Un celestiale potrebbe chiedere una considerevole donazione di oro od oggetti magici a un tempio alleato, mentre un immondo potrebbe richiedere un sacrificio umano o il dono di un tesoro. Alcune creature potrebbero scambiare i loro servigi per una missione che dovrai intraprendere per conto loro. Come regola generale, un compito che può essere misurato in minuti richiede una ricompensa di 100 mo al minuto. Un compito misurato in ore, richiede 1.000 mo all’ora. Un compito misurato in giorni (massimo 10 giorni) richiede 10.000 mo al giorno. Il Narratore può modificare queste ricompense in base alle circostanze nelle quali si è lanciato l’incantesimo. Se il compito è allineato alla morale della creatura, la richiesta di pagamento potrebbe essere dimezzata o addirittura annullata. I compiti non pericolosi di solito chiedono solo la metà di quanto suggerito come pagamento, mentre i compiti molto pericolosi possono richiedere donazioni superiori. È raro che queste creature accettino compiti che sembrino suicida.\\
Dopo che la creatura ha completato il compito, o quando il periodo di servizio concordato è terminato, la creatura tornerà al suo piano natio dopo averti fatto rapporto, se appropriato al compito svolto e se possibile. Se non sei in grado di concordare un prezzo per i servigi della creatura, la creatura tornerà immediatamente al suo piano natio. Una creatura arruolata per unirsi al tuo gruppo è considerata come un suo membro, e riceve una quota piena delle ricompense in punti esperienza.

\medskip\textbf{Allucinazione Mortale}\\\index{Magia - Allucinazione Mortale}
\textbf{Difficolta'}: 20\\
\textbf{Tempo di Lancio}: 2 Azioni\\
\textbf{Gittata}: 36 metri\\
\textbf{Componenti}: V, S\\
\textbf{Durata}: Concentrazione, massimo 1 minuto\\
Attingi agli incubi di una creatura a gittata e che puoi vedere, e crei una manifestazione illusoria delle sue più insite paure, visibile solo per quella creatura. Il bersaglio deve effettuare un tiro salvezza su Arbitrio.\\
Se fallisce il tiro salvezza, il bersaglio è spaventato per la durata. Alla fine di ciascun turno del bersaglio, prima del termine dell’incantesimo, il bersaglio deve superare un tiro salvezza su Arbitrio o subire 4d10 danni. Se supera il tiro salvezza, l’incantesimo termina.\\
\textbf{Per ogni Critico ottenuto} nella prova di magia il danno aumenta di 1d10

\medskip\textbf{Alterare Sé Stesso}\\\index{Magia - Alterare Sé Stesso}
\textbf{Difficolta'}: 15\\
\textbf{Tempo di Lancio}: 2 Azioni\\
\textbf{Gittata}: Personale\\
\textbf{Componenti}: V, S\\
\textbf{Durata}: Concentrazione, massimo 1 ora\\
Assumi una forma diversa. Quando lanci questo incantesimo, scegli una della seguenti opzioni, l’effetto della quale permane per la durata dell’incantesimo. Per  la durata dell’incantesimo puoi terminare un’opzione per ottenere i benefici di un’altra.\\
Adattamento Acquatico. Adatti il tuo corpo a un ambiente acquatico, sviluppando branchie e dita palmate. Puoi respirare sott’acqua e ottieni velocità di nuoto pari alla tua velocità di passeggio.\\
\textit{Armi Naturali}. Sviluppi degli artigli, zanne, spuntoni, corna o una diversa arma naturale a tua scelta. I tuoi colpi senz’armi infliggono 1d6 danni contundenti, perforanti o taglienti, come appropriato all’arma naturale scelta, con la quale sei competente. Infine, l’arma naturale è magica e ricevi un bonus di +1 ai tiri per colpire e danno effettuati quando la usi.\\
\textit{Cambio di Aspetto}. Trasformi il tuo aspetto. Decidi il tuo aspetto esteriore, compresa l’altezza, il peso, i lineamenti facciali, il suono della tua voce, la lunghezza dei capelli, il colorito e qualsiasi peculiarità tu desideri. Puoi apparire come membro di un’altra razza, sebbene nessuna delle tue statistiche cambi. Inoltre non puoi apparire come una creatura di taglia diversa dalla tua, e la tua forma base resta la medesima; se sei bipede, non puoi usare questo incantesimo per diventare quadrupede, per esempio.\\
In qualsiasi momento della durata dell’incantesimo, puoi usare due Azioni per cambiare nuovamente di aspetto in questo modo.\\

\medskip\textbf{Amicizia con gli Animali}\\\index{Magia - Amicizia con gli Animali}
\textbf{Difficolta'}: 13\\
\textbf{Tempo di Lancio}: 2 Azioni\\
\textbf{Gittata}: 9 metri\\
\textbf{Componenti}: V, S, M (un po’ di cibo)\\
\textbf{Durata}: 24 ore\\
Questo incantesimo ti permette di convincere una bestia che non vuoi arrecargli danno. Scegli una bestia a gittata che puoi vedere. Questa deve vederti e udirti. Se l’Intelletto della bestia è 4 o più, l’incantesimo fallisce. Altrimenti, la bestia deve superare un tiro salvezza su Arbitrio o restare affascinata da te per la durata dell’incantesimo. Se tu o uno dei tuoi compagni danneggiate il bersaglio, l’incantesimo ha termine. Ai Livelli Più Alti.\\
\textbf{Per ogni critico ottenuto} nella prova di magia puoi agire su di una bestia aggiuntiva. 

\medskip\textbf{Anatema}\\\index{Magia - Anatema}
\textbf{Difficolta'}: 13\\
\textbf{Tempo di Lancio}: 1 minuto\\
\textbf{Gittata}: 9 metri\\
\textbf{Componenti}: V, S, M (un goccio di sangue)\\
\textbf{Durata}: Concentrazione, massimo 1 minuto\\
Fino a tre creature di tua scelta che puoi vedere, e che sono a gittata, devono effettuare un tiro salvezza su Arbitrio. Ogni bersaglio che fallisca questo tiro salvezza ed effettua un tiro per colpire o un tiro salvezza prima del termine dell’incantesimo, deve tirare un d4 e sottrarre il numero così ottenuto dal tiro per colpire o tiro salvezza.\\
\textbf{Per ogni Critico ottenuto} nella prova di magia puoi prendere come bersaglio una creatura aggiuntiva.

\medskip\textbf{Animale Messaggero}\\\index{Magia - Animale Messaggero}
\textbf{Difficolta'}: 15\\
\textbf{Tempo di Lancio}: 2 Azioni\\
\textbf{Gittata}: 9 metri\\
\textbf{Componenti}: V, S, M (un tocco di cibo)\\
\textbf{Durata}: 24 ore\\
Tramite questo incantesimo, usi un animale per consegnare un messaggio. Scegli una bestia Minuscola a gittata e che puoi vedere, come uno scoiattolo, una ghiandaia o un pipistrello. Specifichi un luogo, che devi aver visitato in passato, e un destinatario che corrisponda a una descrizione generica, come “un uomo o una donna che vesta l’uniforme della guardia cittadina” o “un nano dai capelli rossi che indossa un cappello a punta”. Pronuncia anche un messaggio di massimo venticinque parole. La bestia bersaglio viaggia per la durata dell’incantesimo verso il luogo specificato, coprendo circa 75 chilometri in 24 ore per un messaggero volante, o 40 chilometri per gli altri animali. Quando il messaggero arriva a destinazione, consegna il messaggio alla creatura da te descritta, replicando il suono della tua voce. Il messaggero parla solo a una creatura corrispondente alla descrizione da te fornita. Se il messaggero non riesce a raggiungere la destinazione prima del termine dell’incantesimo, il messaggio è perduto, e la bestia ritorna verso il punto in cui hai lanciato l’incantesimo.
\textbf{Per ogni Critico ottenuto} nella prova di magia la durata dell’incantesimo aumenta di 48 ore

\medskip\textbf{Animare Morti}\\\index{Magia - Animare Morti}
\textbf{Difficolta'}: 18\\
\textbf{Tempo di Lancio}: 1 minuto\\
\textbf{Gittata}: 3 metri\\
\textbf{Componenti}: V, S, M (una goccia di sangue, un pezzo di carne e un pizzico di polvere d’ossa)\\
\textbf{Durata}: Istantanea\\
Questo incantesimo crea un servitore non morto. Scegli una pila di ossa o un cadavere di un umanoide Medio o Piccolo a gittata. Il tuo incantesimo imbeve il bersaglio di una nefanda parvenza di vita, rianimandolo come creatura non morta. Il bersaglio diventa uno scheletro se scegli le ossa o uno zombi se scegli un cadavere. Durante ciascun tuo turno, puoi usare un’azione bonus per comandare mentalmente qualsiasi creatura da te creata con questo incantesimo che si trovi entro 18 metri da te (se controlli più creature, puoi comandarle tutte o solo alcune di loro allo stesso momento, inviando lo stesso comando a tutte). Decidi quale azione la creatura svolgerà e dove si muoverà durante il suo prossimo turno, oppure inviale un comando generale, come quello di stare di guardia a una particolare stanza o corridoio. Se non invii alcun comando, la creatura si limita a difendersi dalle creature ostili. Una volta ricevuto un ordine, la creatura continuerà a svolgerlo fino al suo compimento. La creatura è sotto il tuo controllo per 24 ore, dopodiché smetterà di eseguire i comandi che le impartirai. Per mantenere il controllo sulla creatura per altre 24 ore, devi lanciare di nuovo questo incantesimo su di essa prima del termine dell’attuale periodo di 24 ore. Questo impiego dell’incantesimo riafferma il tuo controllo su di  un massimo di quattro creature che hai animato con questo incantesimo, piuttosto che animarne una nuova.\\
\textbf{Per ogni Critico ottenuto} nella prova di magia animi o riaffermi il controllo su due creature non morte. Ciascuna di queste creature deve provenire da un cadavere o pila di ossa differenti.

\medskip\textbf{Animare Oggetti}\\\index{Magia - Animare Oggetti}
\textbf{Difficolta'}: 23\\
\textbf{Tempo di Lancio}: 1 minuto\\
\textbf{Gittata}: 36 metri\\
\textbf{Componenti}: V, S\\
\textbf{Durata}: Concentrazione, massimo 1 minuto\\
Gli oggetti prendono vita al tuo comando. Scegli fino a dieci oggetti non magici a gittata e che non siano indossati o trasportati. I bersagli Medi contano come due oggetti, i bersagli Grandi contano come quattro oggetti, i bersagli Enormi contano come otto oggetti. Non puoi animare oggetti di taglia più grossa di Enorme. Ogni bersaglio si anima e diventa una creatura sotto il tuo controllo fino al termine dell’incantesimo o finché non viene ridotto a 0 punti ferita.\\
Con un’azione bonus, puoi comandare mentalmente qualsiasi creatura che hai generato con questo incantesimo e che si trovi entro 150 metri da te (se controlli più creature, puoi comandarne solo alcune o tutte allo stesso tempo, impartendo lo stesso comando a ciascuna). Decidi tu quale azione intraprenderà la creatura e dove si muoverà durante il suo turno successivo, o puoi emettere un comando generico,come quello di fare la guardia a una particolare stanza o corridoio. Se non impartisci comandi, la creatura si limiterà a difendersi dalle creature ostili. Una volta dato un ordine, la creatura continuerà a seguirlo finché non avrà completato il suo compito.
\bigskip

\end{multicols}

\textbf{Statistiche degli Oggetti Animati}
\bigskip

\begin{tabular}{llllll}
Taglia		&Punti Ferita	&Difesa	&CA, danni					&Potenza	&Agilità\\ 
\toprule
Minuscola 	&20 			&18		&8, {1d4+4} 	&-3 		&4\\
Piccola 	&25 			&16 	&6, {1d8+2} 	&-2 		&2\\
Media 		&40 			&13 	&5, {2d6+1} 	&0 			&1\\
Grande 		&50 			&10 	&6, {2d10+2}	&2 			&0\\
Enorme 		&80 			&10 	&8, {2d12+4}	&4 			&-2\\
\end{tabular}

\bigskip

\begin{multicols}{2}

Un oggetto animato è un costrutto con Difesa, punti ferita, attacchi, Potenza e Agilità in base alla sua taglia. Il suo punteggio di Intelletto e Volonta' è -3, mentre Magnetismo e' -4.\\
Ha movimento 9 metri; se l’oggetto è privo di gambe o altre appendici che può usare per muoversi ha movimento 0, ha invece movimento volare 9 metri e può fluttuare. 
\\Se l’oggetto è ancorato a una superficie o un oggetto più grosso, come una catena attaccata al muro, la sua velocità è 0.\\
Possiede vista cieca con un raggio di 9 metri ed è cieco oltre questa distanza.\\
Quando l’oggetto animato scende a 0 punti ferita, ritorna alla sua normale forma di oggetto, e tutti i danni in eccesso vengono inflitti alla sua forma originale.\\
Se comandi a un oggetto di attaccare, questo può effettuare un singolo attacco da mischia contro una creatura entro 1 metri da esso. Effettua un attacco on CA e danni determinati dalla taglia (vedi tabella). Il Narratore potrebbe determinare che a seconda della sua forma, un oggetto potrebbe invece infliggere danni taglienti o perforanti.\\
\textbf{Per ogni Critico ottenuto} nella prova di magia puoi animare due oggetti aggiuntivi.

\medskip\textbf{Anti-Individuazione}\index{Anti-Individuazione}\\
\textbf{Difficolta'}: 18\\
\textbf{Tempo di Lancio}: 2 Azioni\\
\textbf{Gittata}: Contatto\\
\textbf{Componenti}: V, S, M (un pizzico di polvere di diamante del valore di 25 mo sparsa sul bersaglio, che l’incantesimo consuma)\\
\textbf{Durata}: 8 ore\\
Per la durata, nascondi il bersaglio con cui sei stato in contatto dalla magia di divinazione. Il bersaglio può essere una creatura consenziente o un luogo o un oggetto che occupi uno spazio equivalente a un cubo non superiore ai 3 metri di spigolo. Il bersaglio non può divenire bersaglio di alcuna magia di divinazione o essere percepito tramite sensi di scrutamento magici.

\medskip\textbf{Antipatia/Simpatia}\index{Antipatia/Simpatia}\\
\textbf{Difficolta'}: 30\\
\textbf{Tempo di Lancio}: 1 ora\\
\textbf{Gittata}: 18 metri\\
\textbf{Componenti}: V, S, M (o un pezzo di allume immerso nell’aceto per l’effetto antipatia o un goccio di miele per l’effetto simpatia)\\
\textbf{Durata}: 10 giorni\\
Questo incantesimo attrae o repelle delle creature di tua scelta. Prendi un bersaglio a gittata, che sia un oggetto Enorme o più piccolo o una creatura o un’area non più grande di un cubo di 60 metri di spigolo. Poi specifica una specie di creature intelligenti, come i draghi rossi, i goblin o i vampiri. Investi il bersaglio di un’aura che attrae o respinge le creature specificate per la durata. Scegli antipatia o simpatia come effetto dell’aura.\\
Antipatia. L’ammaliamento fa sì che le creature del tipo da te indicato provino un forte impulso a lasciare l’area ed evitare il bersaglio. Quando una creatura del genere può vedere il bersaglio o si avvicina entro 18 metri da esso, la creatura deve superare un tiro salvezza su Arbitrio o diventare spaventata. La creatura rimane spaventata finché può vedere il bersaglio o resta entro 18 metri da esso. Mentre è spaventata dal bersaglio, la creatura deve impiegare il suo movimento per muoversi verso il posto sicuro più vicino dal quale non possa più vedere il bersaglio. Se la creatura si muove più di 18 metri lontano dal bersaglio e non può vederlo, la creatura non è più spaventata, ma torna a essere spaventata se torna a vedere il bersaglio o si muove entro 18 metri da esso.\\
Simpatia. L’ammaliamento fa sì che le creature specificate provino un forte impulso ad avvicinarsi al bersaglio se si trovano entro 18 metri da esso o possono vederlo. Quando una simile creatura può vedere il bersaglio o si avvicina entro 18 metri da esso, la creatura deve superare un tiro salvezza su Arbitrio o usare il suo movimento durante ciascun turno per
entrare nell’area, o muoversi a portata del bersaglio. Quando la creatura l’avrà fatto, non potrà più volontariamente muoversi lontano dal bersaglio. Se il bersaglio danneggia o altrimenti nuoce alla creatura soggetta, questa può effettuare un tiro  salvezza su Saggezza per terminare l’effetto, come descritto di seguito.\\
Terminare l’Effetto. Se una creatura soggetta termina il suo turno mentre si trova più lontana di 18 metri dal bersaglio o non può vederlo, la creatura effettua un tiro salvezza su Arbitrio. Se supera il tiro salvezza, la creatura non è più soggetta al bersaglio e riconosce la sensazione di ripugnanza o attrazione come magica. Inoltre, una creatura soggetta all’incantesimo, ha diritto a un altro tiro salvezza su Arbitrio ogni 24 ore di durata dell’incantesimo. Una creatura che supera il tiro salvezza contro questo effetto è immune a esso per 1 minuto, dopodiché può subirlo nuovamente.

\medskip\textbf{Arma Magica}\index{Arma Magica}\\
\textbf{Difficolta'}: 15\\
\textbf{Tempo di Lancio}: 1 Azione Immediata\\
\textbf{Gittata}: Contatto\\
\textbf{Componenti}: V, S\\
\textbf{Durata}: Concentrazione, massimo 1 ora\\
Lanci l’incantesimo a contatto di un’arma non magica. Fino al termine dell’incantesimo, l’arma diventa un’arma magica con un bonus di +1 ai tiri per colpire e di danno.\\
\textbf{Per ogni Critico ottenuto} nella prova di magia puoi il bonus aumenta a +1.

\medskip\textbf{Arma Spirituale}\index{Arma Spirituale}\\
\textbf{Difficolta'}: 15\\
\textbf{Tempo di Lancio}: 1 Azione Immediata\\
\textbf{Gittata}: 18 metri\\
\textbf{Componenti}: V, S\\
\textbf{Durata}: 1 minuto\\
In un punto nella gittata, crei un’arma spettrale fluttuante, che resta per la durata o finché non lanci di nuovo questo incantesimo. Quando lanci l’incantesimo, puoi effettuare un attacco da mischia con incantesimo contro una creatura entro 1 metro dall’arma. Se colpisci, il bersaglio subisce danni da forza pari a 1d8 + il tuo modificatore di caratteristica da incantatore. Durante il tuo turno, con un’azione bonus, puoi spostare l’arma di 6 metri e ripetere l’attacco contro una creatura entro 1 metro dall’arma. L’arma può assumere qualsiasi forma tu voglia, magari affine al Patrono.\\
\textbf{Per ogni Critico ottenuto} nella prova di magia il danno aumenta di 1d6

\medskip\textbf{Armatura Magica}\index{Arma Magica}\\
\textbf{Difficolta'}: 13\\
\textbf{Tempo di Lancio}: 2 Azioni\\
\textbf{Gittata}: Contatto\\
\textbf{Componenti}: V, S, M (un pezzo di cuoio lavorato)\\
\textbf{Durata}: 8 ore
Lanci l’incantesimo a contatto di una creatura consenziente che non indossa un’armatura. Una forza magica protettiva circonda il bersaglio fino al termine dell’incantesimo. La Difesa del bersaglio diventa 13 + Agilità. L’incantesimo termina se il bersaglio indossa un’armatura o interrompe l’incantesimo con un’azione.


\medskip\textbf{Artificio Druidico}\index{Artificio Druidico}\\
\textbf{Difficolta'}: 11\\
\textbf{Tempo di Lancio}: 2 Azioni\\
\textbf{Gittata}: 9 metri\\
\textbf{Componenti}: V, S\\
\textbf{Durata}: Istantanea\\
Sussurrando agli spiriti della natura, crei, a gittata, uno dei seguenti effetti:
\begin{itemize}
\item
Crei un minuscolo e innocuo effetto sensoriale che predice quale clima ci sarà nel luogo in cui ti trovi per le prossime 24 ore. L’effetto potrebbe manifestarsi come una sfera dorata per i cieli limpidi, una nube per la pioggia, fiocchi di neve per la neve, e così via. L’effetto persiste per 1 round.
\item 
Fai immediatamente sbocciare un fiore, un seme o simile pianta.
\item 
Crei un istantaneo e innocuo effetto sensoriale, come foglie che cadono, uno sbuffo di vento, il suono di un piccolo animale, o il lieve tanfo di una puzzola. L’effetto deve entrare in un cubo di 1 metro.
\item Accendi o spegni istantaneamente una candela, una torcia o un piccolo falò.
\end{itemize}

\medskip\textbf{Aura Magica dell’Arcanista}\index{Magia - Aura Magica dell’Arcanista}\\
\textbf{Difficolta'}: 15\\
\textbf{Tempo di Lancio}: 2 Azioni\\
\textbf{Gittata}: Contatto\\
\textbf{Componenti}: V, S, M (un piccolo quadretto di seta)\\
\textbf{Durata}: 24 ore\\
Poni un’illusione su di una creatura od oggetto con cui sei in contatto, così che gli incantesimi di divinazione rivelino false informazioni su di esso. Il bersaglio può essere una creatura consenziente o un oggetto che non sia trasportato o indossato da un’altra creatura. Quando lanci questo incantesimo, scegli uno o entrambi i seguenti effetti. L’effetto permane per la durata. Se esegui questo incantesimo sulla stessa creatura od oggetto ogni giorno per 30 giorni, piazzando ogni volta lo stesso effetto, l’illusione permarrà finché non viene dissolta.\\
\textit{Aura Falsa}. Cambi il modo in cui il bersaglio risulta a incantesimi ed effetti magici, come individuazione del magico, che individuano le aure magiche. Puoi far apparire magico un oggetto normale, non magico un oggetto magico, o cambiare l’aura magica dell’oggetto così che sembri appartenere a una scuola di magia di tua scelta. Quando impieghi questo effetto su di un oggetto, puoi far sì che la falsa magia sia apparente a qualsiasi creatura che lo manipoli.\\ \textit{Mascherare}. Cambi il modo in cui il bersaglio risulta a incantesimi ed effetti magici che individuano il tipo di creatura o allineamento, come l’attivazione dell’incantesimo simbolo. Scegli un tipo di creatura o allineamento, e gli altri incantesimi ed effetti magici considereranno il bersaglio come fosse una creatura di quel tipo o di quell’allineamento, e non più di quello originale.

\medskip\textbf{Aura Sacra}\index{Aura Sacra}\\
\textbf{Difficolta'}: 30\\
\textbf{Tempo di Lancio}: 2 Azioni\\
\textbf{Gittata}: Personale\\
\textbf{Componenti}: V, S, M (un minuscolo reliquario del valore di almeno 1.000 mo contenente una reliquia sacra, come un pezzo di tessuto dell’abito di un santo o un frammento di pergamena di un testo religioso)\\
\textbf{Durata}: Concentrazione, massimo 1 minuto\\
Irradi da te luce divina che si raccoglie in una debole luminosità con raggio di 9 metri intorno a te. Quando lanci l’incantesimo, le creature da te scelte in questo raggio emanano luce fioca con un raggio di 1 metro e hanno {+2d6} a tutti i tiri salvezza, mentre le altre creature hanno {-2d6} sui tiri per colpire contro di loro fino al termine dell’incantesimo. Inoltre, quando un immondo o non morto colpisce una creatura bersaglio con un attacco in mischia, l’aura risplende di una luce intensa. L’attaccante deve superare un tiro salvezza su Tempra o restare accecato fino al termine dell’incantesimo.

\medskip\textbf{Bacche Benefiche}\index{Bacche Benefiche}\\
\textbf{Difficolta'}: 15\\
\textbf{Tempo di Lancio}: 2 Azioni\\
\textbf{Gittata}: Contatto\\
\textbf{Componenti}: V, S, M (un rametto di vischio)\\
\textbf{Durata}: Istantanea\\
Fino a dieci bacche compaiono nella tua mano e vengono infuse di magia per la durata. Una creatura può usare la sua azione per mangiare una bacca. Mangiare una bacca ripristina 1 punto ferita, e la bacca inoltre provvede nutrimento sufficiente per alimentare una creatura per un giorno.\\
Le bacche perdono la loro efficacia se non vengono consumate entro 24 ore dal lancio dell’incantesimo. 

\medskip\textbf{Bagliore Lunare}\index{Bagliore Lunare}\\
\textbf{Difficolta'}: 15\\
\textbf{Tempo di Lancio}: 2 Azioni\\
\textbf{Gittata}: 36 metri\\
\textbf{Componenti}: V, S, M (diversi semi di bella di notte e un pezzo di felpato opalescente)
\textbf{Durata}: Concentrazione, massimo 1 minuto\\
Un fascio argenteo di luce pallida risplende in un cilindro di raggio 1 metro, alto 12 metri centrato in un punto a gittata. Fino al termine dell’incantesimo, una luce fioca riempie il cilindro. \\
Quando una creatura entra nell’area dell’incantesimo per la prima volta durante un turno o inizia qui il suo turno, è avvolta da fiamme spettrali che provocano un dolore terribile, e deve effettuare un tiro salvezza su Potenza. Se fallisce il tiro salvezza subisce 2d10 danni radianti, o la metà di questi danni se lo supera. Un mutaforma effettua il tiro salvezza con svantaggio. Se lo fallisce ritorna immediatamente alla sua forma originale e non può assumere una forma diversa finché non esce dalla luce dell’incantesimo.\\
Durante ciascun tuo turno dopo  aver lanciato l’incantesimo, puoi usare un’azione per muovere il
fascio di 18 metri in qualsiasi direzione. 
\textbf{Per ogni Critico ottenuto} nella prova di magia il danno aumenta di 1d10


\medskip\textbf{Bagliore Solare}\index{Bagliore Solare}\\
\textbf{Difficolta'}: 25\\
\textbf{Tempo di Lancio}: 2 Azioni\\
\textbf{Gittata}: Personale (linea di 18 metri)\\
\textbf{Componenti}: V, S, M (una lente di ingrandimento)\\
\textbf{Durata}: Concentrazione, massimo 1 minuto\\
Una fascio di luce brillante esplode dalla tua mano in una linea larga 1 metri e lunga 18 metri. Ogni creatura sulla linea deve effettuare un tiro salvezza su Tempra. Se fallisce il tiro salvezza, la creatura subisce 6d8 danni radianti e rimane accecata fino al tuo prossimo turno. Se supera il tiro salvezza, subisce la metà dei danni e non è accecata. I non morti e le melme hanno svantaggio su questo tiro salvezza. Puoi creare una nuova linea di luminosità con un’azione durante qualsiasi tuo turno fino al termine dell’incantesimo.\\
Per la durata, una particella di luce brillante risplende nella tua mano. Produce luce in un raggio di 9 metri e penombra per ulteriori 9 metri. Questa luce è considerata luce solare.


\medskip\textbf{Banchetto degli Eroi}\index{Banchetto degli Eroi}\\
\textbf{Difficolta'}: 25\\
\textbf{Tempo di Lancio}: 10 minuti\\
\textbf{Gittata}: 9 metri\\
\textbf{Componenti}: V, S, M (una ciotola incrostata di gemme del valore di almeno 1.000 mo, che l’incantesimo consuma)\\
\textbf{Durata}: Istantanea\\
Crei un magnifico banchetto, comprensivo di cibi e bevande prelibate. Il banchetto viene consumato in 1 ora e scompare al termine di questo periodo, ma gli effetti benefici non si faranno sentire fino al termine dell’ora. Fino ad altre dodici creature possono
partecipare al banchetto. Una creatura che partecipi al banchetto ottiene diversi benefici. La creatura viene guarita da tutte le malattie e i veleni, diventa immune al veleno e all’essere
spaventata, e ha +2d6 su tutti i tiri salvezza su Arbitrio. I suoi punti ferita massimi aumentano di 2d10, e guarisce lo stesso quantitativo di punti ferita attuali. Questi benefici durano 24 ore. 

\medskip\textbf{Barriera di Lame}\index{Barriera di Lame}\\
\textbf{Difficolta'}: 25\\
\textbf{Tempo di Lancio}: 2 Azioni\\
\textbf{Gittata}: 18 metri\\
\textbf{Componenti}: V, S\\
\textbf{Durata}: Concentrazione, massimo 10 minuti Crei un muro verticale di lame rotanti fatte di energia magica, affilate come rasoi. Il muro compare a gittata e resta per la durata. Puoi creare un muro diritto lungo fino a 30 metri, alto 6 metri e spesso 1 metro, o un muro circolare di 18 metri massimo di diametro, alto 6 metri e spesso 1 metro. Il muro fornisce tre quarti di copertura alle creature dietro di esso, e il suo spazio è terreno difficile. \\
Quando una creatura entra per la prima volta in un turno nell’area del muro o comincia il suo turno lì, la creatura deve effettuare un tiro salvezza su Riflessi. Se la creatura fallisce il tiro salvezza subisce 6d10 danni taglienti, o la metà se lo supera.

\medskip\textbf{Beffa Crudele}\index{Beffa Crudele}\\
\textbf{Difficolta'}: 11\\
\textbf{Tempo di Lancio}: 2 Azioni\\
\textbf{Gittata}: 18 metri\\
\textbf{Componenti}: V\\
\textbf{Durata}: Istantanea\\
Scateni una serie di insulti avvolti da una subdola malia contro una creatura a gittata e che puoi vedere. Se il bersaglio ti può udire (sebbene non è necessario che ti comprenda), deve superare un tiro salvezza su Arbitrio o subire 1d4 danni e avere -1d6 al prossimo tiro per colpire che effettuerà prima del termine del suo prossimo turno.\\
Il danno dell’incantesimo aumenta di 1d8 quando raggiungi CM 5, CM 11 e CM 17.

\medskip\textbf{Benedizione}\index{Benedizione}\\
\textbf{Difficolta'}: 13\\
\textbf{Tempo di Lancio}: 2 Azioni
\textbf{Gittata}: 9 metri
\textbf{Componenti}: V, S, M (uno spruzzo di acqua sacra)
\textbf{Durata}: Concentrazione, massimo 1 minuto
Benedici fino a tre creature a gittata, scelte da te. Ogni
qualvolta un bersaglio effettua un tiro per colpire o un
tiro salvezza prima del termine dell’incantesimo, può
tirare un d4 aggiuntivo e sommare il risultato ottenuto al
tiro per colpire o al tiro salvezza.
Ai Livelli Più Alti. Quando lanci questo incantesimo
usando uno slot incantesimo di 2° livello o più alto, puoi
aggiungere una creatura come bersaglio per ogni livello
dello slot sopra il 1°.

\medskip\textbf{Blocca Mostri}\index{Blocca Mostri}\\
\textbf{Difficolta'}: 23\\
\textbf{Tempo di Lancio}: 2 Azioni\\
\textbf{Gittata}: 27 metri\\
\textbf{Componenti}: V, S, M (un piccolo pezzo dritto di ferro)\\
\textbf{Durata}: Concentrazione, massimo 1 minuto\\
Scegli una creatura a gittata e che puoi vedere. Il bersaglio deve superare un tiro salvezza su Arbitrio, o restare paralizzato per la durata. Questo incantesimo non ha effetto su non morti o costrutti. Al termine di ciascun suo turno, il bersaglio può effettuare un altro tiro salvezza su Arbitrio. Se lo supera, per quel bersaglio l’incantesimo ha termine.\\
\textbf{Per ogni Critico ottenuto} nella prova di magia puoi aggiungere una creatura come bersaglio purché siano entro 9 metri l’una dall’altra.


\medskip\textbf{Blocca Persone}\index{Blocca Persone}\\
\textbf{Difficolta'}: 15\\
\textbf{Tempo di Lancio}: 2 Azioni\\
\textbf{Gittata}: 18 metri\\
\textbf{Componenti}: V, S, M (un piccolo pezzo dritto di ferro)\\
\textbf{Durata}: Concentrazione, massimo 1 minuto\\
Scegli un umanoide a gittata e che puoi vedere. Il bersaglio deve superare un tiro salvezza su Arbitrio o restare paralizzato per la durata. Al termine di ciascun suo turno, il bersaglio può effettuare un altro tiro salvezza su Arbitrio. Se lo supera, per quel bersaglio l’incantesimo ha termine.\\
\textbf{Per ogni Critico ottenuto} nella prova di magia puoi aggiungere una creatura come bersaglio purché siano entro 9 metri l’una dall’altra.


\medskip\textbf{Bocca Magica}\index{Bocca Magica}\\
\textbf{Difficolta'}: 15\\
\textbf{Tempo di Lancio}: 1 minuto\\
\textbf{Gittata}: 9 metri\\
\textbf{Componenti}: V, S, M (un piccolo pezzo di favo e polvere di giada del valore di almeno 10 mo, che l’incantesimo consuma)\\
\textbf{Durata}: Fino a che dissolto\\
Impianti un messaggio in un oggetto a gittata, messaggio che viene pronunciato quando si soddisfa la condizione di attivazione. Scegli un oggetto che puoi vedere e che non sia indossato o trasportato da un’altra creatura. Poi pronuncia il messaggio, che deve essere di 25 parole o meno, ma può essere distribuito in un periodo di massimo 10 minuti. Infine, determina la circostanza che attiverà l’incantesimo, affinché questo trasmetta il tuo messaggio.\\
Quando la circostanza si manifesta, una bocca magica appare sull’oggetto e recita il messaggio con la tua voce e allo stesso volume con cui l’hai pronunciato. Se l’oggetto da te scelto ha una bocca o qualcosa che assomiglia a una bocca (per esempio, la bocca di una statua), la bocca magica appare così che le parole sembrino provenire dalla bocca dell’oggetto. Quando lanci questo incantesimo, puoi far sì che l’incantesimo termini dopo aver trasmesso il suo messaggio, o che perduri e ripeta il messaggio ogni volta che la condizione si attiva.\\
La circostanza di attivazione può essere generica o dettagliata quanto desideri, ma deve essere basata su condizioni visibili o udibili che avvengono entro 9 metri dall’oggetto. Per esempio, potresti istruire la bocca di parlare quando una qualsiasi creatura si avvicina entro  9 metri dall’oggetto o quando una campanella d’argento suona entro 9 metri da esso.


\medskip\textbf{Caduta Morbida}\index{Caduta Morbida}\\
\textbf{Difficolta'}: 13\\
\textbf{Tempo di Lancio}: 1 reazione, che effettui quando tu o una creatura entro 18 metri da te cadete\\
\textbf{Gittata}: 18 metri\\
\textbf{Componenti}: V, M (una piccola piuma o un pezzo di piuma)\\
\textbf{Durata}: 1 minuto\\
Scegli fino a cinque creature a gittata. La velocità di discesa di una creatura che cade diminuisce a 18 metri per round fino al termine dell’incantesimo. Se la creatura atterra prima del termine dell’incantesimo, non subisce danni da caduta e può atterrare sui suoi piedi; per quella creatura l’incantesimo ha termine.\\


\medskip\textbf{Calmare Emozioni}\index{Calmare Emozioni}\\
\textbf{Difficolta'}: 15\\
\textbf{Tempo di Lancio}: 2 Azioni\\
\textbf{Gittata}: 18 metri\\
\textbf{Componenti}: V, S\\
\textbf{Durata}: Concentrazione, massimo 1 minuto\\
Tenti di sopprimere le forti emozioni in un gruppo di persone. Ogni umanoide in una sfera di 6 metri di raggio centrata su di un punto a gittata da te scelto, deve effettuare un tiro salvezza su Arbitrio; se lo  desidera, una creatura può scegliere di fallire questo tiro salvezza. Se una creatura fallisce il tiro salvezza, scegli uno di questi due effetti. \\
Placare. Puoi sopprimere qualsiasi effetto che renda il bersaglio affascinato o spaventato. Quando questo incantesimo termina, gli effetti soppressi riprendono, purché la loro durata non sia nel frattempo esaurita.\\
Indifferenza. Puoi rendere un bersaglio indifferente nei confronti di una creatura di tua scelta, verso la quale è ostile. Questa indifferenza termina se il bersaglio viene attaccato o danneggiato da un incantesimo o se vede uno dei suoi amici venir danneggiato. Quando l’incantesimo termina, la creatura diventa di nuovo ostile, a meno che il Narratore non determini diversamente.

\medskip\textbf{Camminare sull’Acqua}\index{Camminare sull’Acqua}\\
\textbf{Difficolta'}: 18\\
\textbf{Tempo di Lancio}: 2 Azioni\\
\textbf{Gittata}: 9 metri\\
\textbf{Componenti}: V, S, M (un pezzo di sughero)\\
\textbf{Durata}: 1 ora\\
Questo incantesimo conferisce la capacità di muoversi attraverso superfici liquide (come acqua, acido, fango, neve, sabbie mobili o lava) come se fossero innocuo terreno solido (le creature che attraversano la lava fusa possono comunque subire danni dal calore). Fino a dieci creature consenzienti a gittata, e che puoi vedere, ricevono questa capacità per tutta la durata. Se il tuo bersaglio è immerso in un liquido, l’incantesimo riporta il bersaglio in superficie del liquido a una velocità di 18 metri per round. 

\medskip\textbf{Camminare nel Vento}\index{Camminare nel Vento}\\
\textbf{Difficolta'}: 25\\
\textbf{Tempo di Lancio}: 1 minuto\\
\textbf{Gittata}: 9 metri\\
\textbf{Componenti}: V, S, M (fuoco e acqua sacra)\\
\textbf{Durata}: 8 ore\\
Per la durata, tu e fino ad altre dieci creature consenzienti a gittata, che puoi vedere, assumete forma gassosa, diventando nubi. Mentre è in forma di nube, una creatura ha velocità di volo 90 metri e ha resistenza ai danni dalle armi non magiche. Ritornare alla forma normale richiede 1 minuto, durante il quale la creatura è inabile e non può muoversi. Fino al termine dell’incantesimo, una creatura può tornare alla forma di nube, che richiede una trasformazione di un minuto. Se una creatura è in forma di nube e sta volando quando l’effetto ha termine, la creatura scende 18 metri per round al minuto finché non atterra, al sicuro. Se non riesce ad atterrare dopo 1 minuto, la creatura cadrà per la distanza rimanente.

\medskip\textbf{Charme su Persone}\index{Charme su Persone}\\
\textbf{Difficolta'}: 13\\
\textbf{Tempo di Lancio}: 2 Azioni\\
\textbf{Gittata}: 9 metri\\
\textbf{Componenti}: V, S\\
\textbf{Durata}: 1 ora\\
Cerchi di affascinare un umanoide a gittata e che puoi vedere. Egli deve effettuare un tiro salvezza su Arbitrio, e avrà vantaggio se sta combattendo con te o i tuoi alleati. Se fallisce il tiro salvezza, è affascinato da te fino al termine dell’incantesimo o finché tu o i tuoi alleati non gli facciate qualcosa di nocivo. La creatura affascinata ti considera un amichevole conoscente. Quando l’incantesimo termina, la creatura è consapevole di essere stata affascinata da te. 
\textbf{Per ogni Critico ottenuto} nella prova di magia puoi puoi aggiungere una creatura come bersaglio per ogni livello. Quando lanci l’incantesimo, le creature bersaglio devono trovarsi entro 9 metri l’una dall’altra.

\medskip\textbf{Campo Anti-Magia}\index{Campo Anti-Magia}\\
\textbf{Difficolta'}: 30\\
\textbf{Tempo di Lancio}: 2 Azioni\\
\textbf{Gittata}: Personale (sfera di 3 metri di raggio)\\
\textbf{Componenti}: V, S, M (un pizzico di ferro in polvere o lime di ferro)\\
\textbf{Durata}: Concentrazione, massimo 1 ora\\
Vieni circondato da una sfera invisibile di anti-magia di 3 metri di raggio. Quest’area è separata dall’energia magica che permea il multiverso. All’interno della sfera non si possono lanciare incantesimi, le creature richiamate scompaiono e anche gli oggetti magici diventano normali. Fino al termine dell’incantesimo, la sfera si muove con te, centrata su di te. Gli incantesimi e altri effetti magici, eccetto quelli creati da un artefatto o divinità, sono soppressi all’interno della sfera né vi possono penetrare. Uno slot speso per lanciare un incantesimo soppresso è consumato. Mentre un effetto è soppresso, non funziona, ma il tempo che trascorre soppresso è conteggiato per la sua durata. 
\\Effetti con Bersaglio. Incantesimi e altri effetti magici, come dardo incantato e charme su persone, che prendono come bersaglio una creatura o un oggetto all’interno della sfera non hanno effetto su quel bersaglio.
\\Aree di Magia. L’area di un altro incantesimo o effetto magico, come palla di fuoco, non può estendersi all’interno della sfera. Se la sfera si sovrappone a un’area di magia, la parte di quell’area coperta dalla sfera viene soppressa. Per esempio, le fiamme generate da un muro di fuoco vengono soppresse all’interno della sfera, creando un buco nel muro se la sovrapposizione è sufficientemente grande. Incantesimi. Qualsiasi incantesimo o altro effetto magico attivo su di una creatura od oggetto all’interno della sfera viene soppresso finché la creatura o l’oggetto si trovano all’interno della sfera.\\
Oggetti Magici. Le proprietà e poteri degli oggetti magici vengono soppressi dalla sfera. Per esempio, una spada lunga +1 all’interno della sfera funziona come una spada lunga non magica. Le proprietà e i poteri delle armi magiche vengono soppressi se sono usati contro un bersaglio all’interno  della sfera o impugnate da un attaccante dentro la sfera. Se un’arma magica o munizione magica lascia interamente la sfera (per esempio, se tiri una freccia magica o scagli una lancia magica a un bersaglio al di fuori della sfera), la magia dell’oggetto non è più soppressa non appena esce dalla sfera.
\\Magia di Viaggio. Il teletrasporto e il viaggio planare non funzionano all’interno della sfera, che la sfera sia il punto di destinazione o di partenza di questo viaggio magico. All’interno della sfera, un portale verso un altro luogo, mondo, o piano di esistenza, così come uno spazio extradimensionale come quello creato dall’incantesimo trucco della corda, resta chiuso.
\\Creature e Oggetti. All’interno della sfera, una creatura o oggetto evocati o creati dalla magia svaniscono temporaneamente dall’esistenza. La creatura od oggetto riappare istantaneamente una volta che lo spazio occupato da essa non si trova più all’interno della sfera.
\\Dissolvi magie. Gli incantesimi e gli effetti magici come dissolvi magie non hanno effetto sulla sfera. Allo stesso modo, le sfere create da altri incantesimi campo antimagia non si annullano vicendevolmente. 

\medskip\textbf{Camuffare Sé Stesso}\index{Camuffare Sé Stesso}\\
\textbf{Difficolta'}: 13\\
\textbf{Tempo di Lancio}: 2 Azioni\\
\textbf{Gittata}: Personale\\
\textbf{Componenti}: V, S\\
\textbf{Durata}: 1 ora\\
Cambi il tuo aspetto, assieme a quello dei tuoi abiti, armatura, armi e altri oggetti che indossi, fino al termine dell’incantesimo o finché non impieghi un’azione per interrompere l’incantesimo. Puoi apparire 30 centimetri più basso o più alto, magro, grasso o una via di mezzo. Non puoi modificare la tua conformazione fisica, quindi devi adottare una forma che abbia la medesima distribuzione di arti. Per tutto il resto, l’illusione è limitata solo dalla tua fantasia.\\
I cambi apportati da questo incantesimo non sono in grado di sostenere un’ispezione fisica. Per esempio, se usi questo incantesimo per aggiungere un cappello al tuo abbigliamento, gli oggetti attraversano il cappello, e chiunque lo tocchi non avvertirebbe nulla e finirebbe per toccarti la testa e i capelli. Se usi questo incantesimo per apparire più magro di quello che sei, la mano di una persona che provasse a toccarti rimbalzerebbe su di te, mentre alla vista sembrerebbe fermarsi a mezz’aria. Per distinguere il tuo camuffamento, una creatura può usare la sua azione per ispezionare il tuo aspetto e deve superare una prova di Consapevolezza contro la DC del tiro salvezza dell’incantesimo. 

\medskip\textbf{Capanna}\index{Capanna}\\
\textbf{Difficolta'}: 18\\
\textbf{Tempo di Lancio}: 1 minuto\\
\textbf{Gittata}: Personale (semisfera di 3 metri di raggio)\\
\textbf{Componenti}: V, S, M (una piccola biglia di cristallo)\\
\textbf{Durata}: 8 ore\\
Una cupola di forza immobile del raggio di 3 metri si forma intorno e sopra di te, restando stazionaria per la durata. L’incantesimo termina se lasci l’area. Nove creature di taglia Media o inferiore possono entrare nella cupola insieme a te. L’incantesimo fallisce se l’area include una creatura più grande o più di dieci creature. Le creature e gli oggetti all’interno della cupola, quando lanci questo incantesimo, la possono attraversare liberamente. Tutte le altre creature e oggetti sono proibiti dall’attraversarla. Incantesimi e altri effetti magici non possono estendersi oltre la cupola o attraversarla. L’atmosfera all’interno dello spazio è confortevole e asciutta, quale che sia il clima all’esterno.\\
Fino al termine dell’incantesimo, puoi comandare che l’interno diventi illuminato fioco o buio. La cupola è opaca dall’esterno, di qualsiasi colore tu scelga, ma è trasparente dall’interno. 


\medskip\textbf{Caratteristica Potenziata}\index{Caratteristica Potenziata}\\
\textbf{Difficolta'}: 15\\
\textbf{Tempo di Lancio}: 2 Azioni\\
\textbf{Gittata}: Contatto\\
\textbf{Componenti}: V, S, M (pelo o piuma di una bestia)\\
\textbf{Durata}: Concentrazione, massimo 1 ora\\
Conferisci un potenziamento magico a una creatura con cui sei in contatto. Scegli uno degli effetti seguenti; il bersaglio ottiene quell’effetto fino al termine dell’incantesimo.\\
\textit{Astuzia della Volpe}. Il bersaglio ha +1d6 alle prove di Intelletto e Potenza\\
\textit{Forza del Toro}. Il bersaglio ha +1d6 alle prove di Potenza, e la sua capacità di Ingombro raddoppia.\\
\textit{Grazia del Gatto}. Il bersaglio ha +1d6 alle prove di Agilità. Inoltre, qualora non sia inabile, non subisce danni dalle cadute di 6 metri o meno.\\
\textit{Resistenza dell’Orso}. Il bersaglio ha +1d6 alle prove di Potenza. Ottiene anche 2d6 punti ferita temporanei, che vengono persi alla fine dell’incantesimo.\\
\textit{Saggezza del Gufo}. Il bersaglio ha +1d6 alle prove di Volonta'. \\
\textit{Splendore dell’Aquila}. Il bersaglio ha +1d6 alle prove di Magnetismo.\\
\textbf{Per ogni Critico ottenuto} nella prova di magia puoi prendere come bersaglio un’ulteriore creatura

\medskip\textbf{Carne in Pietra}\index{Carne in Pietra}\\
\textbf{Difficolta'}: 25\\
\textbf{Tempo di Lancio}: 2 Azioni\\
\textbf{Gittata}: 18 metri\\
\textbf{Componenti}: V, S, M (un pizzico di lime, acqua e terra)\\
\textbf{Durata}: Concentrazione, massimo 1 minuto\\
Cerchi di trasformare in pietra una creatura a gittata che puoi vedere. Se il corpo del bersaglio è fatto di carne, la creatura deve effettuare un tiro salvezza su Potenza. Se fallisce il tiro salvezza, è intralciata e la sua carne comincia a indurirsi. Se supera il tiro salvezza, la creatura non subisce l’incantesimo. Una creatura intralciata da questo incantesimo deve effettuare un altro tiro salvezza su Potenza al termine di ciascun suo turno. Se supera il tiro salvezza con successo per tre volte, l’incantesimo termina. Se fallisce il tiro salvezza per tre volte, viene trasformata in pietra e resta vittima della condizione pietrificato per la durata. I successi e i fallimenti non devono essere continuativi; tenere traccia di entrambi finché il bersaglio non ne ottiene tre di un tipo.\\
Se la creatura viene danneggiata fisicamente mentre è pietrificata, soffre di deformità simili ai danni arrecati alla pietra, se ritorna al suo stato originale. Se mantieni la tua concentrazione su questo incantesimo per la sua intera possibile durata, la creatura è trasformata in pietra finché l’effetto non viene rimosso.


\medskip\textbf{Catena di Fulmini}\index{Catena di Fulmini}\\
\textbf{Difficolta'}: 25\\
\textbf{Tempo di Lancio}: 2 Azioni\\
\textbf{Gittata}: 45 metri\\
\textbf{Componenti}: V, S, M (un po’ di pelliccia; un pezzo d’ambra, vetro o una verga di cristallo; e tre spille d’argento)\\
\textbf{Durata}: Istantanea\\
Crei una saetta di luce che colpisce un bersaglio a gittata che puoi vedere, scelto da te. Tre saette colpiscono il bersaglio e fino ad altri tre bersagli, ciascuno dei quali deve trovarsi entro 9 metri dal primo bersaglio. Un bersaglio può essere una creatura o oggetto e può essere bersaglio di una sola saetta. Un bersaglio deve effettuare un tiro salvezza su Agilita'. Il bersaglio subisce 10d8 danni da fulmine se fallisce il tiro salvezza, o la metà di questi danni se lo supera.
\textbf{Per ogni Critico ottenuto} nella prova di magia un'ulteriore saetta parte dal primo bersaglio verso un altro bersaglio.

\medskip\textbf{Cecità/Sordità}\index{Cecità/Sordità}\\
\textbf{Difficolta'}: 15\\
\textbf{Tempo di Lancio}: 2 Azioni\\
\textbf{Gittata}: 9 metri\\
\textbf{Componenti}: V\\
\textbf{Durata}: 1 minuto\\
Puoi accecare o assordare un nemico. Scegli una creatura a gittata e che puoi vedere. Il bersaglio deve effettuare un tiro salvezza su Potenza. Se lo fallisce, il bersaglio è accecato o assordato (a tua scelta) per la durata. Al termine di ciascun suo turno, il bersaglio può effettuare un tiro salvezza su Potenza. Se lo supera, l’incantesimo termina.\\
\textbf{Per ogni Critico ottenuto} nella prova di magia puoi prendere come bersaglio una creatura aggiuntiva.

\medskip\textbf{Celare}\index{Celare}
\textbf{Difficolta'}: 28\\
\textbf{Tempo di Lancio}: 2 Azioni\\
\textbf{Gittata}: Contatto\\
\textbf{Componenti}: V, S, M (una polvere composta da polvere di diamante, smeraldo, rubino e zaffiro del valore di almeno 50.000 mo, che l’incantesimo consuma)\\
\textbf{Durata}: Fino a che dissolto Tramite questo incantesimo, una creatura consenziente o un oggetto può essere nascosto, impossibile da individuare per la durata. Eseguendo questo incantesimo ed entrando in contatto con un bersaglio, questo diventa invisibile e non può essere preso come bersaglio dagli incantesimi di divinazione, né percepito da sensori di scrutamento creati da incantesimi di divinazione.\\
Se il bersaglio è una creatura, cade in uno stato di animazione sospesa. Per lui il tempo cessa di scorrere, e non invecchia. \\
Puoi predisporre una condizione per cui l’incantesimo termini anticipatamente. La condizione può essere qualsiasi cosa tu voglia, ma deve avvenire o essere visibile entro 1,5 chilometri dal bersaglio. Esempi includono “dopo 1.000 anni” o “quando il tarrasque si risveglia”. Questo incantesimo termina anche qualora il bersaglio subisca danni.


\medskip\textbf{Cerchio Magico}\index{Cerchio Magico}\\
\textbf{Difficolta'}: 18\\
\textbf{Tempo di Lancio}: 1 minuto\\
\textbf{Gittata}: 3 metri\\
\textbf{Componenti}: V, S, M (acqua sacra o argento e ferro in polvere del valore di almeno 100 mo, che l’incantesimo consuma)\\
\textbf{Durata}: 1 ora\\
Crei un cilindro di energia magica di 3 metri di raggio e alto 6 metri, centrato su di un punto del terreno a gittata e che puoi vedere. Rune luminose compaiono dovunque il cilindro si intersechi con il pavimento o altra superficie.\\
Scegli uno o più dei seguenti tipi di creature: celestiali, elementali, fatati, immondi o non morti. Il circolo influisce su di una creatura del tipo scelto nei modi seguenti:\\
\begin{itemize}
	\item 
La creatura non può entrare consapevolmente nel cilindro tramite alcun mezzo non magico. Se la creatura prova a usare il teletrasporto o il viaggio tra i piani per farlo, deve prima superare un tiro salvezza Arbitrio.
	\item 
La creatura ha svantaggio ai tiri per colpire contro i bersagli all’interno del cilindro.
	\item 
I bersagli all’interno del cilindro non possono essere affascinati, spaventati o posseduti dalla creatura. Quando lanci questo incantesimo, puoi decidere che la magia operi in direzione opposta, impedendo a una creatura del tipo specificato di lasciare il cilindro e proteggendo i bersagli all’esterno.
\end{itemize}
\textbf{Per ogni Critico ottenuto} nella prova di magia puoi aumentare la durata di 1 ora.

\medskip\textbf{Cerchio di Morte}\index{Cerchio di Morte\\}
\textbf{Difficolta'}: 25\\
\textbf{Tempo di Lancio}: 2 Azioni\\
\textbf{Gittata}: 45 metri\\
\textbf{Componenti}: V, S, M (una perla nera ridotta in polvere del valore di almeno 500 mo)\\
\textbf{Durata}: Istantanea\\
Una sfera di energia negativa del raggio di 18 metri, erutta in un punto a gittata. Ogni creatura in quell’area deve effettuare un tiro salvezza su Potenza. Un  ersaglio subisce 8d6 danni necrotici se fallisce il tiro salvezza, o la metà di questi danni se lo supera. \\
\textbf{Per ogni Critico ottenuto} nella prova il danno aumenta di 2d6.

\medskip\textbf{Cerchio di Teletrasporto}\index{Cerchio di Teletrasporto}\\
\textbf{Difficolta'}: 23\\
\textbf{Tempo di Lancio}: 1 minuto\\
\textbf{Gittata}: 3 metri\\
\textbf{Componenti}: V, M (gessi e inchiostri rari infusi di gemme preziose del valore di almeno 50 mo, che l’incantesimo consuma)\\
\textbf{Durata}: 1 round\\
Mentre lanci l’incantesimo, tracci un cerchio di 3 metri di diametro sul pavimento, inscritto con sigilli che collegano il posto in cui ti trovi a un cerchio di teletrasporto permanente di tua scelta, di cui conosci la sequenza dei sigilli e che si trovi sullo stesso piano di esistenza in cui ti trovi tu. Un portale luminoso si apre all’interno del cerchio tracciato da te e resta aperto fino al termine del tuo prossimo turno. Qualsiasi creatura che entri nel portale, riappare istantaneamente entro 1 metro dal cerchio di destinazione o nello spazio non
occupato più vicino, se non può comparire entro 1 metro da esso.\\
Molti grandi templi, gilde, e altri luoghi importanti possiedono dei cerchi di teletrasporto permanenti, incisi da qualche parte nelle loro prossimità. Ciascuno di questi cerchi possiede una sequenza di sigilli unica: una serie di rune magiche disposte seguendo una trama precisa.\\ Quando ottieni la capacità di lanciare questo incantesimo, apprendi le sequenze di sigilli di
due destinazioni sul Piano Materiale, determinate dal Narratore. Nel corso delle tue avventure puoi imparare nuove sequenze di sigilli. Puoi mandare a memoria una sequenza di sigilli dopo averla studiata per almeno 1 minuto.\\
Puoi creare un cerchio di teletrasporto permanente eseguendo questo incantesimo nello stesso luogo ogni giorno per un anno. Non devi usare il cerchio di teletrasporto quando lanci l’incantesimo in questo modo.

\medskip\textbf{Chiaroveggenza}\index{Chiaroveggenza}
\textbf{Difficolta'}: 18\\
\textbf{Tempo di Lancio}: 10 minuti\\
\textbf{Gittata}: 1,5 chilometri\\
\textbf{Componenti}: V, S, M (un focus del valore di almeno 100 mo, che sia un corno ingioiellato per udire o un occhio di vetro per guardare)\\
\textbf{Durata}: Concentrazione, massimo 10 minuti\\
Crei un sensore invisibile in un luogo a te familiare e che sia a gittata (un luogo che hai già visitato o visto precedentemente) o in un luogo ovvio ma che non ti è familiare (come dietro una porta o un angolo, o in mezzo un boschetto di alberi). Il sensore rimane sul posto per la durata, e non può essere attaccato né altrimenti vi si può interagire. Quando lanci questo incantesimo, scegli se vedere o udire. Puoi usare il senso scelto tramite il sensore, come ti trovassi nel suo spazio. Con un’azione, puoi passare da udire a sentire e viceversa. Una creatura che può vedere il sensore (una creatura munita di vedere invisibilità o di visione del vero) lo percepisce come un orbe intangibile e luminoso delle dimensioni del tuo pugno.

\medskip\textbf{Clone}\index{Clone}\\
\textbf{Difficolta'}: 30\\
\textbf{Gittata}: Contatto
\textbf{Componenti}: V, S, M (un diamante del valore di almeno 1.000 mo e almeno 16 centimetri cubi di carne della creatura che deve essere clonata, che l’incantesimo consuma, e un recipiente da almeno 2.000 mo di valore che abbia un coperchio sigillabile e sia grande a sufficienza da contenere una creatura Media, come una grossa urna, una bara, una fossa piena di fango nel terreno o un contenitore di cristallo pieno di acqua salata)\\
\textbf{Durata}: Istantanea\\
Questo incantesimo produce il duplicato inerte di una creatura vivente come salvaguardia dalla morte. Questo clone si forma all’interno di un recipiente sigillato e raggiunge la massima dimensione e maturità dopo 120 giorni; puoi anche decidere che il clone sia una versione più giovane della stessa creatura. Rimane inerte e sopravvive all’infinito, purché il recipiente resti indisturbato.\\
In qualsiasi momento dopo che il clone è maturato, se la creatura originale muore, la sua anima si trasferisce nel clone, purché l’anima sia libera e consenziente a tornare. Il clone è fisicamente identico all’originale e ha la stessa personalità, ricordi e caratteristiche, ma nulla dell’equipaggiamento dell’originale. I resti fisici della creatura originale, se esistono ancora, divengono inerti e non possono essere riportati alla vita, dato che l’anima della creatura è altrove. 

\medskip\textbf{Colpo Accurato}\index{Colpo Accurato}\\
\textbf{Difficolta'}: 11\\
\textbf{Tempo di Lancio}: 2 Azioni\\
\textbf{Gittata}: 9 metri\\
\textbf{Componenti}: S\\
\textbf{Durata}: Concentrazione, massimo 1 round\\
Allunghi la mano e punti il dito verso un bersaglio a gittata. La tua magia ti conferisce una breve comprensione delle difese del bersaglio. Durante il tuo prossimo turno, purché questo incantesimo non sia terminato, ottieni +1d6 al primo tiro per colpire contro quel bersaglio.

\medskip\textbf{Colpo Infuocato}\index{Colpo Infuocato}\\
\textbf{Difficolta'}: 23\\
\textbf{Tempo di Lancio}: 2 Azioni\\
\textbf{Gittata}: 18 metri\\
\textbf{Componenti}: V, S, M (pizzico di zolfo)\\
\textbf{Durata}: Istantanea\\
Una colonna verticale di fuoco divino scende dal cielo e si abbatte sul luogo da te specificato. Ogni creatura in un cilindro di 3 metri di raggio e alto 12 metri centrato su di un punto a gittata deve effettuare un tiro salvezza su Agilità. Una creatura subisce 8d6 danni da Luce se fallisce il tiro salvezza, o la metà di questi danni se lo supera.\\
\textbf{Per ogni Critico ottenuto} nella prova di magia il danno Luce aumenta di 2d6.

\medskip\textbf{Comando}\index{Comando}\\
\textbf{Difficolta'}: 13\\
\textbf{Tempo di Lancio}: 2 Azioni\\
\textbf{Gittata}: 18 metri\\
\textbf{Componenti}: V\\
\textbf{Durata}: 1 round\\
Pronunci un comando di una parola verso una creatura a gittata e che puoi vedere. Il bersaglio deve superare un tiro salvezza su Arbitrio o eseguire il comando durante il suo prossimo turno. L’incantesimo non ha effetto se il bersaglio è non morto, se non capisce la tua lingua, o se il tuo comando gli recherebbe danni. Seguono alcuni tipici comandi e i loro effetti. Puoi dare comandi diversi da quelli descritti qui, e in quel caso il Narratore determinerà il comportamento del bersaglio. Se il bersaglio non può eseguire il tuo comando, l’incantesimo ha fine.
\begin{itemize}
	\item 
Avvicinati. Il bersaglio si muove verso di te per il tragitto più breve e diretto, terminando il suo turno se si avvicina a 1 metri da te.
	\item 
Fermo. Il bersaglio non si muove e poi termina il suo turno. Una creatura volante resta sul posto, purché le sia possibile. Se deve muoversi per restare in aria, vola la distanza minima necessaria per farlo.
	\item 
	Getta. Il bersaglio getta qualsiasi cosa stia tenendo in mano e poi termina il suo turno. 	\item 
	Scappa. Il bersaglio spende il suo turno a muoversi lontano da te con il mezzo più veloce a sua disposizione.
	\item  Striscia. Il bersaglio si getta prono e poi termina il suo round.
\end{itemize}

\textbf{Per ogni Critico ottenuto} nella prova di magia puoi agire su di un’ulteriore creatura. Nel momento in cui lanci l’incantesimo, le creature bersaglio devono trovarsi entro 9 metri l’una da l’altra.


\medskip\textbf{Comprensione dei Linguaggi}\index{Comprensione dei Linguaggi}\\
\textbf{Difficolta'}: 13\\
\textbf{Tempo di Lancio}: 2 Azioni\\
\textbf{Gittata}: Personale\\
\textbf{Componenti}: V, S, M (un pizzico di sale e fuliggine)\\
\textbf{Durata}: 1 ora\\
Per la durata, capisci il significato letterale di qualsiasi linguaggio parlato che ascolti. Capisci anche qualsiasi linguaggio scritto che vedi, ma devi essere a contatto con la superficie su cui le parole sono scritte. Per leggere una pagina di testo impieghi 1 minuto. Questo incantesimo non decodifica i messaggi segreti in un testo o glifo, come un sigillo arcano, che non faccia parte di un linguaggio scritto.


\medskip\textbf{Compulsione}\index{Compulsione}\\
\textbf{Difficolta'}: 20\\
\textbf{Tempo di Lancio}: 2 Azioni\\
\textbf{Gittata}: 9 metri\\
\textbf{Componenti}: V, S\\
\textbf{Durata}: Concentrazione, massimo 1 minuto\\
Le creature di tua scelta entro la gittata, che puoi vedere e che ti possono sentire, devono effettuare un tiro salvezza su Arbitrio. Un bersaglio supera automaticamente il tiro salvezza se non può essere affascinato. Fino al termine dell’incantesimo, puoi usare un’azione bonus durante ciascun tuo turno per indicare una direzione orizzontale rispetto a te. Ogni bersaglio soggetto all’incantesimo deve usare quanto più possibile del suo movimento, durante il suo prossimo turno, per muoversi in quella direzione. Il bersaglio non può effettuare nessuna azione prima di muoversi. Dopo essersi mosso in questo modo, il bersaglio può effettuare un altro tiro salvezza su Arbitrio per tentare di terminare l’effetto.\\
Un bersaglio non può essere obbligato a muoversi dentro un pericolo palesemente letale, come fiamme o pozzi, ma per muoversi nella direzione indicata potrà provocare attacchi di opportunità.

\medskip\textbf{Comunione}\index{Comunione}\\
\textbf{Difficolta'}: 23\\
\textbf{Tempo di Lancio}: 1 minuto\\
\textbf{Gittata}: Personale\\
\textbf{Componenti}: V, S, M (incenso e una fiala di acqua sacra o blasfema)\\
\textbf{Durata}: 1 minuto\\
Comunichi con il tuo Patrono e gli poni fino a tre domande a cui si può dare risposta con un sì o un no. Devi porre le domande prima della fine dell’incantesimo. Riceverai la risposta corretta a ciascuna domanda. Le creature divine non sono necessariamente onniscienti, quindi potresti ricevere “non è chiaro” come risposta a una domanda che riguarda informazioni non pertinenti alle conoscenze della divinità. Nel caso in cui una risposta di una parola potrebbe essere fuorviante o contraria agli interessi della divinità, il Narratore potrebbe invece dare una breve frase come risposta.\\
Se lanci l’incantesimo due o più volte prima di aver terminato il tuo prossimo riposo lungo, c’è una probabilità cumulativa del 25\% che per ogni lancio dopo il primo tu non ottenga alcuna risposta. Il Narratore effettua questo tiro in segreto.

\medskip\textbf{Comunione con la Natura}\index{Comunione con la Natura}\\
\textbf{Difficolta'}: 23\\
\textbf{Tempo di Lancio}: 1 minuto\\
\textbf{Gittata}: Personale\\
\textbf{Componenti}: V, S\\
\textbf{Durata}: Istantanea\\
Per un istante diventi tutt’uno con la natura e ottieni informazioni sul territorio circostante. In ambienti esterni, l’incantesimo ti fornisce informazioni sul territorio entro 5 chilometri da te. In grotte e altri  ambienti naturali sotterranei, il raggio è limitato a 100 metri. L’incantesimo non funziona nei luoghi in cui la natura è stata soppiantata da costruzioni, come in sotterranei e paesi.\\
Apprendi immediatamente informazioni su un massimo di tre argomenti a tua scelta su uno dei seguenti soggetti, in relazione all’area:
\begin{itemize}
	\item 
	terreno e corpi d’acqua
	\item 
	piante, minerali, animali e popolazioni prevalenti
	\item 
    potenti celestiali, elementali, fatati, immondi o non morti
	\item 
    influenze da altri piani di esistenza
	\item
    edifici
\end{itemize}

\medskip\textbf{Confusione}\index{Confusione}
\textbf{Difficolta'}: 20\\
\textbf{Tempo di Lancio}: 2 Azioni\\
\textbf{Gittata}: 27 metri\\
\textbf{Componenti}: V, S, M (tre gusci di noce)\\
\textbf{Durata}: Concentrazione, massimo 1 minuto\\
Questo incantesimo assale e piega la mente delle creature, generando illusioni e provocando azioni incontrollate. Quando lanci questo incantesimo ogni creatura, in una sfera di 3 metri di raggio centrata su di un punto da te scelto entro la gittata, deve superare un tiro salvezza su Arbitrio o subirne gli effetti. Un bersaglio soggetto all’incantesimo non può effettuare reazioni e deve tirare un d10 all’inizio di ciascun suo turno per determinare il proprio comportamento per quel turno. 

\medskip

\begin{tabularx}{0.45\textwidth}{lX}
	\hline 
d10 & Comportamento\\ 
1 & La creatura usa tutto il suo movimento per muoversi in una direzione casuale. Per determinare la direzione, tira un d8 assegnando a ciascuna faccia un punto cardinale. La
creatura non effettuerà nessuna azione in questo turno. \\
2-6 & La creatura non può muoversi né attaccare in questo turno.\\
7-8 & La creatura usa la sua azione per effettuare un attacco da mischia contro una creatura determinata a caso entro la sua portata. Se non c’è nessuna creatura a portata, per questo turno la creatura non farà nulla.\\
9-10 & La creatura può agire e muoversi normalmente.\\
\end{tabularx} 

\medskip

Al termine di ciascun suo turno, un bersaglio soggetto all’incantesimo può effettuare un tiro salvezza su Arbitrio. Se lo supera, per lui l’effetto ha termine. \\
\textbf{Per ogni Critico ottenuto} nella prova di magia il raggio della sfera aumenta di 1 metro.

\medskip\textbf{Cono di Freddo}\index{Cono di Freddo}\\
\textbf{Difficolta'}: 23\\
\textbf{Tempo di Lancio}: 2 Azioni\\
\textbf{Gittata}: Personale (cono di 18 metri)\\
\textbf{Componenti}: V, S, M (un piccolo cristallo o cono di vetro)\\
\textbf{Durata}: Istantanea\\
Un’esplosione di aria fredda erutta dalle tue mani. Ogni creatura in un cono di 18 metri deve effettuare un tiro salvezza su Potenza. Una creatura subisce 8d8 danni da freddo se fallisce il tiro salvezza, o la metà di questi danni se lo supera. Una creatura uccisa da questo incantesimo diventa una statua di ghiaccio fino a quando disgela.\\
\textbf{Per ogni Critico ottenuto} nella prova di magia il danno aumenta di 1d8

\medskip\textbf{Conoscenza delle Leggende}\index{Conoscenza delle Leggende}
\textbf{Difficolta'}: 23\\
\textbf{Tempo di Lancio}: 10 minuti\\
\textbf{Gittata}: Personale\\
\textbf{Componenti}: V, S, M (incenso del valore di almeno 250 mo, che l’incantesimo consuma, e quattro strisce d’avorio del valore di almeno 50 mo)\\
\textbf{Durata}: Istantanea\\
Nomina o descrivi una persona, luogo od oggetto. L’incantesimo ti porta alla mente un breve riassunto delle conoscenze più importanti sull’argomento da te nominato. Se la cosa da te nominata non ha alcuna rilevanza leggendaria, non ottieni alcuna informazione. Maggiori informazioni hai sull’argomento, più precise e dettagliate saranno le informazioni che riceverai. L’informazione che riceverai sarà accurata, ma celata magari in linguaggio metaforico.

\medskip\textbf{Contagio}\index{Contagio}\\
\textbf{Difficolta'}: 23\\
\textbf{Tempo di Lancio}: 2 Azioni\\
\textbf{Gittata}: Contatto\\
\textbf{Componenti}: V, S\\
\textbf{Durata}: 7 giorni\\
Tramite il contatto puoi infliggere malattie. Effettua un attacco da mischia con incantesimo contro una creatura a portata. Se colpisci, infetti la creatura con una malattia a tua scelta tra quelle descritte di seguito. Al termine di ciascun turno del bersaglio, esso deve effettuare un tiro salvezza su Potenza. Dopo aver fallito tre di questi tiri salvezza, gli effetti della malattia permangono per la durata, e la creatura non effettua più tiri salvezza. Dopo aver superato tre di questi tiri salvezza, la creatura recupera dalla malattia, e l’incantesimo ha termine. \\
Dato che questo incantesimo induce nel suo bersaglio una malattia naturale, qualsiasi effetto che rimuova le malattie o migliori gli effetti delle malattie si applica a essa.\\
\begin{itemize}
	\item 
	\textit{Carne Putrida}. La pelle della creatura marcisce. La creatura ha -1d6 alle prove di Magnetismo e ogni danno e' raddoppiato.
\item 
	\textit{Debolezza Accecante}. Il dolore attanaglia la mente della creatura mentre i suoi occhi diventano bianco latte. La creatura ha -1d6  alle prove di Arbitrio e ai tiri salvezza su Arbitrio, ed è accecata.
\item 
   \textit{Febbre Lurida}. Una febbre devastante sconvolge il corpo della creatura. La creatura ha -1d6 alle prove di Potenza e ai tiri salvezza su Potenza, e ai tiri per
colpire che usano la Potenza.
\item 
\textit{Fitte}. La creatura è sopraffatta dai tremiti. La creatura ha -1d6 alle prove di Agilità e ai tiri salvezza su Agilità, e ai tiri per colpire che usano l'Agilita'.
\item 
\textit{Fuoco Mentale}. La mente della creatura è preda della febbre. La creatura ha -1d6 alle prove di Intelletto e ai tiri salvezza su Intelletto, e si comporta come se in combattimento fosse sotto l’effetto dell’incantesimo confusione.
\item 
\textit{Morte Melmosa}. La creatura inizia a sanguinare incessantemente. La creatura ha -1d6 alle prove di Potenza e ai tiri salvezza su Tempra. Inoltre, ogni qualvolta la creatura subisce danni, è stordita fino alla fine del suo prossimo turno.
\end{itemize}

\medskip\textbf{Contattare Altri Piani}\index{Contattare Altri Piani}\\
\textbf{Difficolta'}: 23\\
\textbf{Tempo di Lancio}: 1 minuto\\
\textbf{Gittata}: Personale\\
\textbf{Componenti}: V\\
\textbf{Durata}: 1 minuto\\
Contatti mentalmente un semidio, lo spirito di un saggio da tempo defunto, o qualche altra misteriosa entità di un altro piano. Contattare l’Intelletto extraplanare può affaticare o addirittura spezzare la tua mente. Quando lanci questo incantesimo, effettua un tiro salvezza su Intelletto con DC 15. Se lo fallisci, subisci 6d6 danni e resti demente fino al termine di un riposo lungo. Mentre sei demente, non puoi effettuare azioni, non puoi capire quello che dicono le altre creature, non puoi leggere, e parli solo farneticando. L’incantesimo ristorare superiore può porre fine a questo effetto. Se superi il tiro salvezza, puoi porre all’entità fino a cinque domande. Devi porre le domande prima del termine dell’incantesimo. Il Narratore risponderà a ciascuna domanda con una parola: “sì”, “no”, “forse”, “mai”, “irrilevante” o “confuso” (se l’entità non conosce la risposta alla domanda). Se una risposta di una parola
potrebbe risultare fuorviante, il Narratore potrebbe invece dare come risposta una breve frase.


\medskip\textbf{Contingenza}\index{Contingenza}\\
\textbf{Difficolta'}: 25\\
\textbf{Tempo di Lancio}: 10 minuti\\
\textbf{Gittata}: Personale\\
\textbf{Componenti}: V, S, M (una statuetta raffigurante te stesso scolpita in avorio e decorata con gemme del valore di almeno 1.500 mo)\\
\textbf{Durata}: 10 giorni\\
Scegli un incantesimo di difficolta' 23 o più basso che puoi lanciare, che abbia il tempo di lancio di 2 azioni, e che può avere te come bersaglio. Lanci quell’incantesimo (detto incantesimo contingente) come parte del lancio di contingenza, spendendo gli slot incantesimo di entrambi, ma senza che l’incantesimo contingente abbia effetto. Avrà invece effetto quando si avvererà una determinata circostanza. Descrivi questa circostanza mentre lanci i due incantesimi. Per esempio, contingenza lanciato assieme a respirare sott’acqua potrebbe stipulare che respirare sott’acqua entra in azione quando sei immerso nell’acqua o simile liquido.\\
L’incantesimo contingente ha effetto immediatamente dopo che la circostanza si verifica per la prima volta, che tu lo voglia o no, e poi contingenza termina. L’incantesimo contingente agisce solo su di te, anche se normalmente può prendere come bersaglio anche altri. Puoi usare un solo incantesimo contingenza alla volta. Se lanci di nuovo questo incantesimo, l’effetto di un altro incantesimo contingenza su di te avrà termine. Inoltre, contingenza per te ha termine se la componente materiale non dovesse più trovarsi sulla tua persona.


\medskip\textbf{Controincantesimo}\index{Controincantesimo}\\
\textbf{Difficolta'}: 18\\
\textbf{Tempo di Lancio}: 1 reazione, che effettui quando vedi una creatura entro 18 metri da te lanciare un incantesimo\\
\textbf{Gittata}: 18 metri\\
\textbf{Componenti}: S \\
\textbf{Durata}: Istantanea\\
Cerchi di interrompere una creatura nell’atto di lanciare un incantesimo. Se la creatura sta lanciando un incantesimo di difficolta' 18 o più basso, l’incantesimo fallisce e non ha effetto. Se sta eseguendo un incantesimo di difficolta' 20 o più alto, effettui una prova di magia.
Se la superi la prova di magia dell'altro mago l’incantesimo della creatura fallisce e non ha effetto. 


\medskip\textbf{Controllare Acqua}\index{Controllare Acqua}
\textbf{Difficolta'}: 20\\
\textbf{Tempo di Lancio}: 2 Azioni\\
\textbf{Gittata}: 90 metri\\
\textbf{Componenti}: V, S, M (un goccio d’acqua e un pizzico di polvere)\\
\textbf{Durata}: Concentrazione, massimo 10 minuti\\
Fino al termine dell’incantesimo, controlli qualsiasi acqua libera all’interno dell’area che hai scelto fino a un cubo di 30 metri di spigolo. Quando lanci questo incantesimo puoi scegliere qualsiasi tra i seguenti effetti. Come azione, durante il tuo turno, puoi ripetere lo stesso effetto o sceglierne uno diverso.\\
\begin{itemize}
	\item 
\textit{Allagamento}. Fai sì che il livello di tutta l’acqua nell’area aumenti fino a 6 metri. Se l’area include una costa, l’acqua inonda la terraferma. Se scegli un’area all’interno di un grosso corpo d’acqua, crei invece un’onda alta 6 metri che viaggia da un lato all’altro dell’area prima di infrangersi. Qualsiasi veicolo di taglia Enorme o inferiore sul percorso dell’onda viene trasportato dall’altro lato. Qualsiasi veicolo di taglia Enorme o inferiore colpito dall’acqua ha una percentuale del 25\% di cappottarsi.\\
Il livello dell’acqua resta elevato fino al termine dell’incantesimo o finché non scegli un effetto diverso. Se questo effetto ha prodotto un’onda, l’onda si ripete all’inizio del tuo turno successivo, finché perdura l’effetto di allagamento.\\
\item 
\textit{Dividere le Acque}. Fai sì che l’acqua nell’area si sposti a lato per creare un varco. Il varco si estende per l’area dell’incantesimo, e l’acqua divisa forma un muro su entrambi i lati del varco. Il varco resta fino al termine dell’incantesimo o finché non scegli un effetto diverso. L’acqua tornerà poi lentamente a riempire il varco nel corso del round successivo, fino a che non sarà risalita al suo normale livello.
\item 
\textit{Ridirigere il Flusso}. Fai sì che l’acqua corrente nell’area si muova in una direzione a tua scelta, anche se l’acqua deve superare degli ostacoli, risalire muri o dirigersi verso altre direzioni improbabili. L’acqua nell’area si muove secondo le tue indicazioni, ma una volta giunta oltre l’area dell’incantesimo, riprende il suo  flusso in base alle condizioni del terreno. L’acqua continua a muoversi nella direzione da te scelta fino al termine dell’incantesimo o finché non scegli un effetto diverso.
\item 
\textit{Turbine}. Questo effetto richiede un corpo d’acqua che copra un quadrato di 15 metri di lato e abbia una profondità di 7,5 metri. Fai sì che si formi un turbine al centro dell’area. Il turbine produce un vortice largo 1 metro alla base, largo fino a 15 metri in cima e alto 7,5
metri. Qualsiasi creatura od oggetto nell’acqua e che si trovi entro 7,5 metri dal vortice viene trascinato 3 metri verso di esso. Una creatura può nuotare per allontanarsi dal vortice effettuando una prova di Potenza (Atletica) contro la DC del tiro salvezza dell’incantesimo.
Quando una creatura entra nel vortice per la prima volta durante un turno o inizia lì il suo turno, deve effettuare un tiro salvezza su Potenza. Se lo fallisce, la creatura subisce 2d8 danni contundenti e viene catturata dal vortice fino al termine dell’incantesimo. Se supera il tiro salvezza, la creatura subisce la metà di questi danni, e non è catturata dal vortice. Una creatura catturata dal vortice può usare la sua azione per cercare di nuotare via dal vortice come descritto sopra, ma ha svantaggio alle prove di Potenza (Atletica) per farlo. La prima volta durante ciascun turno in un cui un oggetto entra nel vortice, l’oggetto subisce 2d8 danni contundenti; questo danno viene subito ogni round in cui l’oggetto rimane nel vortice.
\end{itemize}


\medskip\textbf{Controllare Tempo Atmosferico}\index{Controllare Tempo Atmosferico}
\textbf{Difficolta'}: 30\\
\textbf{Tempo di Lancio}: 10 minuti\\
\textbf{Gittata}: Personale (raggio di 1,5 chilometri)\\
\textbf{Componenti}: V, S, M (incenso bruciato e po’ di terra e legno mescolati nell’acqua)\\
\textbf{Durata}: Concentrazione, massimo 8 ore \\
Per la durata, assumi il controllo del clima entro 7,5 chilometri da te. Per lanciare questo incantesimo devi essere all’esterno. Muoversi in un posto dove non hai la visuale aperta verso il cielo, termina l’incantesimo anticipatamente. Quando lanci questo incantesimo, cambia le attuali condizioni climatiche, determinate dal Narratore in base alla stagione e la latitudine. Puoi modificare le precipitazioni, la temperatura e il vento. Ci vogliono 1d4 x 10 minuti perché la nuova condizione prenda effetto. Una volta che la condizione avrà preso effetto, potrai cambiarla di nuovo. Quando l’incantesimo termina, il clima tornerà gradualmente alla norma.\\
Quando cambi le condizioni climatiche, trova l’attuale condizione sulla seguente tabella e cambiala di uno stadio, verso l’alto o il basso. Quando cambi il vento, puoi cambiarne anche la direzione. 
\medskip
\textit{Precipitazione}
\begin{itemize}
	\item 
1 Limpido
	\item 
2 Qualche nuvola
	\item 
3 Coperto o foschia a terra
	\item 
4 Pioggia, grandine o neve
	\item 
5 Pioggia torrenziale, grandinata pesante, tormenta
\end{itemize}

\textit{Temperatura}

\begin{itemize}
 \item 
1 Caldo insopportabile
	\item 
2 Caldo
	\item 
3 Tiepido
	\item 
4 Fresco
	\item 
5 Freddo
	\item 
6 Freddo polare
	\item 
\end{itemize}

\textit{Vento}
\begin{itemize}
	\item 
1 Calmo
	\item 
2 Vento moderato
	\item 
3 Vento moderato
	\item 
4 Fortunale
	\item 
5 Tempesta
\end{itemize}

\medskip\textbf{Costrizione}\index{Costrizione}\\
\textbf{Difficolta'}: 23\\
\textbf{Tempo di Lancio}: 1 minuto\\
\textbf{Gittata}: 18 metri\\
\textbf{Componenti}: V\\
\textbf{Durata}: 30 giorni\\
Imponi un comando magico a una creatura a gittata che puoi vedere, obbligandolo ad adempiere un determinato compito o vietandole di svolgere un’azione o corso d’attività deciso da te. Se la creatura ti può capire, deve superare un tiro salvezza su Arbitrio o restare affascinata da te per la durata. Mentre la creatura è affascinata da te, subisce 5d10 danni ogni volta che agisce in maniera direttamente contraria alle tue istruzioni, ma non più di una volta al giorno. Una creatura che non ti può capire ignora gli effetti di questo incantesimo. Puoi dare qualsiasi comando di tua scelta, tranne un’attività che provocherebbe morte certa. Dovessi tu pronunciare un comando suicida, l’incantesimo avrebbe termine.\\
Puoi terminare l’incantesimo usando un’azione. Anche rimuovi maledizione, ristorare superiore o desiderio vi pongono termine.\\
\textbf{Se ottini almeno due Critici} nella prova di magia la durata è 1 anno. Se ottieni 3 Critici l’incantesimo dura finché non viene terminato da uno degli incantesimi sopra menzionati.

\medskip\textbf{Creare Cibo e Acqua}\index{Creare Cibo e Acqua}\\
\textbf{Difficolta'}: 18\\
\textbf{Tempo di Lancio}: 2 Azioni\\
\textbf{Gittata}: 9 metri\\
\textbf{Componenti}: V, S\\
\textbf{Durata}: Istantanea\\
Crei 22,5 chili di cibo e 120 litri d’acqua sul terreno o in contenitori a gittata, sufficienti a sostenere fino a quindici umanoidi o cinque cavalcature per 24 ore. Il cibo è blando ma nutriente, e marcisce dopo 24 ore se non viene consumato. L’acqua è pulita e non va a male. 

\medskip\textbf{Creare o Distruggere Acqua}\index{Creare o Distruggere Acqua}
\textbf{Difficolta'}: 13\\
\textbf{Tempo di Lancio}: 2 Azioni\\
\textbf{Gittata}: 9 metri\\
\textbf{Componenti}: V, S, M (un goccio d’acqua per creare acqua o qualche granello di sale per distruggerla)\\
\textbf{Durata}: Istantanea\\
Crei o distruggi l’acqua.\\
\textit{Creare Acqua}. Crei fino a 40 litri di acqua limpida in un contenitore aperto a gittata. In alternativa, l’acqua cade come pioggia in un cubo di 9 metri di spigolo che si trovi entro la gittata, estinguendo le fiamme esposte nell’area.\\
\textit{Distruggere Acqua}. Distruggi fino a 40 litri di acqua in un contenitore aperto a gittata. In alternativa, puoi distruggere la nebbia in un cubo di 9 metri di spigolo entro la gittata.\\
\textbf{Per ogni Critico ottenuto} nella prova di magia crei o distruggi ulteriori 40 litri d’acqua, o le dimensioni del cubo aumentano di 1 metro di spigolo

\medskip\textbf{Creare Non Morti}\index{Creare Non Morti}\\
\textbf{Difficolta'}: 25\\
\textbf{Tempo di Lancio}: 2 Azioni\\
\textbf{Gittata}: 3 metri\\
\textbf{Componenti}: V, S, M (un vaso di terracotta pieno di terra di cimitero, un vaso di terracotta pieno di acqua salmastra, e un onice nero del valore di 50 mo per ogni cadavere)\\
\textbf{Durata}: Istantanea\\
Puoi lanciare questo incantesimo solo di notte. Scegli fino a tre cadaveri di umanoidi Medi o Piccoli a gittata. Ogni cadavere diventa un ghoul sotto il tuo controllo (il Narratore possiede le statistiche di gioco di queste creature). Durante il tuo turno, con un’azione bonus, puoi comandare mentalmente qualsiasi creatura da te animata con questo incantesimo, se la creatura si trova entro 36 metri da te (se controlli più creature, puoi comandarle tutte o solo alcune nello stesso momento, impartendo lo stesso comando a ciascuna). Decidi tu quale azione effettuerà la creatura e dove si muoverà durante il suo prossimo turno, oppure puoi impartire un comando generico, come quello di fare la guardia a una specifica stanza o corridoio. Se non impartisci comandi, le creature si limiteranno a difendersi dalle creature ostili. Una volta ricevuto un comando, la creatura continuerà a eseguirlo finche il compito sarà completo. La creatura è sotto il tuo controllo per 24 ore, dopodiché smetterà di rispondere ai comandi che gli impartisci. Per mantenere il controllo della creatura per altre 24 ore, devi lanciare questo incantesimo sulla creatura prima che l’attuale periodo di 24 ore abbia termine. Questo impiego dell’incantesimo riasserisce il tuo controllo su di un massimo di tre creature che hai animato con questo incantesimo, anziché animarne di nuove.\\
\textbf{Se ottieni un Critico} nella prova di magia puoi rianimare o riasserire il controllo su quattro ghoul. Con due Critici puoi animare o riasserire il controllo su cinque
ghoul o due ghast o wight. Con tre Critici puoi animare o riasserire il controllo su sei ghoul, tre ghast o wight, o due mummie. 

\medskip\textbf{Creazione}\index{Creazione}\\
\textbf{Difficolta'}: 23\\
\textbf{Tempo di Lancio}: 1 minuto\\
\textbf{Gittata}: 9 metri\\
\textbf{Componenti}: V, S, M (un minuscolo pezzo di materiale dello stesso tipo di oggetto che intendi creare) \\
\textbf{Durata}: Speciale\\
Afferri pezzi di materia d’ombra dal Mondo delle Ombre per creare, a gittata, oggetti non viventi di materia vegetale: beni morbidi, corda, legno o qualcosa di simile. Puoi usare questo incantesimo anche per creare oggetti minerali come pietra, cristallo o metallo. L’oggetto creato non può essere più grosso di un cubo di 1 metro di spigolo, e l’oggetto deve essere di una forma e materiale che hai già visto in passato.\\
La durata dipende dal materiale dell’oggetto. Se l’oggetto è composto da più materiali, usa la durata più breve.
\medskip
Tabella Materiale - Durato
\medskip

\begin{tabularx}{0.45\textwidth}{lX}
	\hline 
Materia vegetale &1 giorno\\
Pietra o cristallo &12 ore\\
Metalli preziosi &1 ora\\
Gemme &10 minuti\\
Adamantio o mithril &1 minuto\\
\end{tabularx} 
\medskip

Usare qualsiasi materiale creato da questo incantesimo come componente materiale di un altro incantesimo farà fallire il nuovo incantesimo.\\
\textbf{Per ogni Critico ottenuto} nella prova di magia il cubo aumenta di 1 metro di spigolo

\medskip\textbf{Crescita di Spuntoni}\index{Crescita di Spuntoni}\\
\textbf{Difficolta'}: 15\\
\textbf{Tempo di Lancio}: 2 Azioni\\
\textbf{Gittata}: 45 metri\\
\textbf{Componenti}: V, S, M (sette spine affilate o sette ramoscelli, ciascuno di esse appuntito ad un’estremità)\\
\textbf{Durata}: Concentrazione, massimo 10 minuti\\
Il terreno in un raggio di 6 metri centrato su di un punto a gittata si contorce e genera spuntoni e spine molto acuminate. Per la durata, l’area diventa terreno difficile. Quando una creatura entra o si muove all’interno dell’area, subisce 2d4 danni per ogni 1 metro percorsi.
La trasformazione del terreno è talmente ben camuffata da sembrare naturale. Qualsiasi creatura che non abbia visto l’area al momento del lancio dell’incantesimo deve effettuare una prova di Saggezza (Percezione) contro la DC del tiro salvezza dell’incantesimo, per riconoscere il pericolo rappresentato dal terreno prima di entrarvi. 

\medskip\textbf{Crescita Vegetale}\index{Crescita Vegetale}\\
\textbf{Difficolta'}: 18\\
\textbf{Tempo di Lancio}: 2 Azioni o 8 ore\\
\textbf{Gittata}: 45 metri\\
\textbf{Componenti}: V, S\\
\textbf{Durata}: Istantanea\\
Questo incantesimo incanala vitalità nei vegetali entro una specifica area. Esistono due usi possibili per questo incantesimo, che conferiscono benefici immediati o a lungo termine. Se lanci questo incantesimo impiegando 1 azione, scegli un punto a gittata. Tutte i vegetali normali in un  raggio di 30 metri centrato su quel punto diventano densi e folti. Una creaturache attraversa l’area quadruplica il costo del suo movimento.\\
Puoi escludere dai suoi effetti una o più aree di qualsiasi dimensione all’interno dell’area dell’incantesimo.\\
Se lanci questo incantesimo nel corso di 8 ore, nutri la terra. Tutti i vegetali in un raggio di 750 metri centrato su di un punto a gittata diventano super produttivi per 1 anno. I vegetali producono il doppio del normale ammontare di cibo al momento del raccolto.

\medskip\textbf{Cura Ferite Leggere}\index{Cura Ferite Leggere}\\
\textbf{Difficolta'}: 13\\
\textbf{Tempo di Lancio}: 2 Azioni\\
\textbf{Gittata}: Contatto\\
\textbf{Componenti}: V, S\\
\textbf{Durata}: Istantanea\\
Una creatura a contatto con te recupera un numero di punti ferita uguale a 1d8 + Volontà. Questo incantesimo non ha effetto su non morti e costrutti.\\
\textbf{Per ogni Critico ottenuto} nella prova di magia curi 1d6 PF in più.

\medskip\textbf{Cura Ferite Medie}\index{Cura Ferite Medie}\\
\textbf{Difficolta'}: 18\\
\textbf{Tempo di Lancio}: 2 Azioni\\
\textbf{Gittata}: Contatto\\
\textbf{Componenti}: V, S\\
\textbf{Durata}: Istantanea\\
Una creatura a contatto con te recupera un numero di punti ferita uguale a 2d8 + 2*Volontà. Questo incantesimo non ha effetto su non morti e costrutti.\\
\textbf{Per ogni Critico ottenuto} nella prova di magia curi 1d6 PF in più.

\medskip\textbf{Cura Ferite Serie}\index{Cura Ferite Serie }\\
\textbf{Difficolta'}: 23 \\
\textbf{Tempo di Lancio}: 2 Azioni\\
\textbf{Gittata}: Contatto\\
\textbf{Componenti}: V, S\\
\textbf{Durata}: Istantanea\\
Una creatura a contatto con te recupera un numero di punti ferita uguale a 3d8 + 3*Volontà. Questo incantesimo non ha effetto su non morti e costrutti.\\
\textbf{Per ogni Critico ottenuto} nella prova di magia curi 1d6 PF in più.

\medskip\textbf{Cura Ferite di Massa}\index{Cura Ferite di Massa}

Come le Cura Ferite normali ma curi fino a 4 creature.
La Difficolta' aumenta di 6.\\
\textbf{Per ogni Critico ottenuto} nella prova curi una creatura in piu'.

\medskip\textbf{Dardo di Fuoco}\index{Dardo di Fuoco}\\
\textbf{Difficolta'}: 11\
\textbf{Tempo di Lancio}: 2 Azioni\\
\textbf{Gittata}: 36 metri\\
\textbf{Componenti}: V, S\\
\textbf{Durata}: Istantanea\\
Scagli una scintilla infuocata a una creatura od oggetto a gittata. Effettua un attacco a distanza con incantesimo contro il bersaglio. Se colpisci, il bersaglio subisce 1d10 danni da fuoco. Un oggetto infiammabile colpito da questo incantesimo prende fuoco, se non è indossato o trasportato.\\
Il danno dell’incantesimo aumenta di 1d8 quando raggiungi CM 5, CM 11 e CM 17.

\medskip\textbf{Dardo Tracciante}\index{Dardo Tracciante}\\
\textbf{Difficolta'}: 13\\
\textbf{Tempo di Lancio}: 2 Azioni\\
\textbf{Gittata}: 36 metri\\
\textbf{Componenti}: V, S\\
\textbf{Durata}: 1 round\\
Un lampo di luce viaggia verso una creatura a gittata, scelta da te. Effettua un attacco a distanza con incantesimo contro il bersaglio. Se colpisci, il bersaglio subisce 4d6 danni radianti, e il prossimo tiro per colpire effettuato contro di lui prima del termine del tuo
prossimo turno ha +1d6 al TC, grazie alla mistica luce fioca che continuerà a brillare intorno al bersaglio fino ad allora.\\
\textbf{Per ogni Critico ottenuto} nella prova di magia il danno aumenta di 1d6

\medskip\textbf{Danza Irresistibile}\index{Danza Irresistibile}\\
\textbf{Difficolta'}: 30\\
\textbf{Tempo di Lancio}: 2 Azioni\\
\textbf{Gittata}: 9 metri\\
\textbf{Componenti}: V\\
\textbf{Durata}: Concentrazione, massimo 1 minuto\\
Scegli una creatura a gittata e che puoi vedere. Il bersaglio comincia un comico balletto sul posto: agitando le gambe, battendo i piedi e saltellando per la durata. Le creature che non possono essere affascinate sono immuni a questo incantesimo.\\
Una creatura che balla deve usare 2 Azioni di Movimento per ballare senza lasciare il suo spazio e ha -1d6 ai tiri salvezza su Agilità e i tiri per colpire. Mentre il bersaglio è soggetto a questo incantesimo, le altre creature hanno +1d6 ai tiri per colpire contro di lui. Spendendo 2 Azioni la creatura che balla puo' affettuare un nuovo tiro salvezza su Arbitrio per
recuperare il controllo di se stessa. Se lo supera, l’incantesimo ha fine.

\medskip\textbf{Dardo Incantato}\index{Dardo Incantato}\\
\textbf{Difficolta'}: 13\\
\textbf{Tempo di Lancio}: 2 Azioni\\
\textbf{Gittata}: 36 metri\\
\textbf{Componenti}: V, S\\
\textbf{Durata}: Istantanea\\
Crei tre dardi luminosi di forza magica. Ciascun dardo colpisce una creatura a gittata che puoi vedere, scelta da te. Un dardo infligge 1d4 + 1 danni da forza al suo bersaglio. I dardi colpiscono tutti simultaneamente, e li puoi dirigere perché colpiscano una o più creature.
\textbf{Per ogni Critico ottenuto} nella prova di magia l’incantesimo crea un dardo aggiuntivo

\medskip\textbf{Deflagrazione Occulta}\index{Deflagrazione Occulta}\\
\textbf{Difficolta'}: 11\\
\textbf{Tempo di Lancio}: 2 Azioni\\
\textbf{Gittata}: 36 metri\\
\textbf{Componenti}: V, S\\
\textbf{Durata}: Istantanea\\
Un fascio di energia crepitante si dirige verso una creatura a gittata. Effettua un attacco a distanza con  incantesimo contro il bersaglio. Se colpisci, il bersaglio subisce 1d10 danni da forza.\\
Il danno dell’incantesimo aumenta di 1d18 quando raggiungi CM 5, CM 11 e CM 17.

\medskip\textbf{Desiderio}\index{Desiderio}\\
\textbf{Difficolta'}: 33\\
\textbf{Tempo di Lancio}: 2 Azioni\\
\textbf{Gittata}: Personale\\
\textbf{Componenti}: V\\
\textbf{Durata}: Istantanea\\
Desiderio è il più potente incantesimo che una creatura mortale possa lanciare. Semplicemente parlando ad alta voce, puoi modificare le stesse fondamenta della realtà a seconda dei tuoi bisogni. \\
L’uso basilare di questo incantesimo è quello di riprodurre l’effetto di qualsiasi altro incantesimo di 8° livello o più basso. Non devi soddisfare nessuno dei requisiti dell’incantesimo, comprese le componenti materiali costose. L’incantesimo ha semplicemente
effetto.\\
In alternativa, puoi creare uno dei seguenti effetti a tua
scelta:
\begin{itemize}
	\item 
Crei un oggetto del valore massimo di 25.000 mo, che non sia un oggetto magico. L’oggetto non può avere dimensioni superiori ai 90 metri in qualsiasi dimensione, e compare in uno spazio non occupato sul terreno.
	\item 
Permetti fino a venti creature che puoi vedere di recuperare tutti i punti ferita, e termini tutti gli effetti su di loro descritti dall’incantesimo ristorare superiore. 
	\item 
Conferisci a un massimo di dieci creature che puoi vedere la resistenza a un tipo di danno a tua scelta.
	\item 
Conferisci a un massimo di dieci creature che puoi vedere l’immunità a un singolo incantesimo o altro effetto magico per 8 ore. Per esempio, potresti rendere te e tutti tuoi compagni immuni all’attacco risucchia vita del lich.
	\item 
Annulli un qualsiasi evento recente obbligando a ritirare qualsiasi tiro effettuato nell’ultimo round (compreso il tuo ultimo turno). La realtà si rimodella per assecondare il nuovo risultato. Puoi far sì che il nuovo tiro abbia +2d6 o -2d6, puoi scegliere se usare il tiro originale o il nuovo tiro. Potresti anche riuscire a ottenere altro, oltre gli obiettivi negli esempi di cui sopra.\\
\end{itemize}
\medskip
Definisci i tuoi desideri quanto più possibile al Narratore. Il Narratore ha grande spazio di
manovra nel decidere cosa accada in questi casi; maggiore il desiderio, più grosse le probabilità che qualcosa vada storto. L’incantesimo potrebbe semplicemente fallire, l’effetto desiderato manifestarsi solo in parte, oppure potresti subire delle conseguenze impreviste, in base a come hai proferito il desiderio. Lo stress del lanciare questo incantesimo per creare qualsiasi effetto che non sia riprodurre un altro incantesimo ti indebolisce.\\
Dopo averne retto lo stress, ogni volta che lancerai un incantesimo, fino a che non avrai terminato un riposo lungo, subirai 1d10 danni necrotici per livello dell’incantesimo. Questo danno non può essere ridotto o diminuito in alcun modo. Inoltre, la tua Potenza scende a -3, se non è già a -3 o meno, per 2d4 giorni.\\
Per ciascun giorno che trascorri a riposare e non svolgere altro che un’attività leggera, il tuo tempo di recupero rimanente diminuisce di 2 giorni. Infine, c’è una probabilità del 33 percento che tu non sia mai più in grado di lanciare desiderio a causa dello stress sofferto per il lancio dell’incantesimo.

\medskip\textbf{Destriero Fantomatico}\index{Destriero Fantomatico}\\
\textbf{Difficolta'}: 18\\
\textbf{Tempo di Lancio}: 1 minuto\\
\textbf{Gittata}: 9 metri\\
\textbf{Componenti}: V, S\\
\textbf{Durata}: 1 ora\\
Una creatura quasi reale simile a un cavallo di taglia Grande, appare sul terreno in uno spazio non occupato di tua scelta e a gittata. Decidi tu l’aspetto della creatura, e questa compare equipaggiata di sella, morso e briglia. Qualsiasi equipaggiamento creato dall’incantesimo svanisce in una nuvola di fumo se viene portato a più di 3 metri di distanza dal destriero. Per la durata, tu o una creatura di tua scelta potete cavalcare il destriero. La creatura usa le statistiche del cavallo da corsa, eccetto che ha velocità 30 metri e può percorrere 15 chilometri in un’ora, o 20 chilometri ad andatura veloce. Quando l’incantesimo termina, il destriero inizia gradualmente a svanire, dando al cavallerizzo 1 minuto per smontare di sella. L’incantesimo termina se usi un’azione per interromperlo o se il destriero subisce danni.

\medskip\textbf{Disco Fluttuante}\index{Disco Fluttuante}\\
\textbf{Difficolta'}: 13\\
\textbf{Tempo di Lancio}: 2 Azioni\\
\textbf{Gittata}: 9 metri\\
\textbf{Componenti}: V, S, M (una goccia di mercurio)\\
\textbf{Durata}: 1 ora\\
Questo incantesimo crea un piano di forza orizzontale, perfettamente circolare, di 1 metro di diametro e 2,5 centimetri di spessore che fluttua a 1 metro da terra, in uno spazio non occupato di tua scelta a gittata e che puoi vedere. Il disco rimane attivo per la durata, e può sostenere 250 chili. Se gli viene poggiato sopra un peso superiore, l’incantesimo termina e tutto quello che vi si trova sopra cade a terra. Finché ti trovi entro 6 metri da esso, il disco è immobile. Se ti muovi più di 6 metri lontano da esso, il disco ti segue in modo da rimanere sempre a 6 metri da te. Può muoversi attraverso terreno disomogeneo, su e giù per le scale, pendenze e simili, ma non può superare cambi di altitudine di 3 o più metri. Per esempio, il disco non può attraversare un fossato profondo 3 metri, né potrebbe lasciare il fossato se fosse creato in fondo a esso. Se ti allontani più di 30 metri dal disco (di solito perché non riesce ad aggirare un ostacolo nel seguirti), l’incantesimo termina.

\medskip\textbf{Disintegrazione}\index{Disintegrazione}\\
\textbf{Difficolta'}: 25\\
\textbf{Tempo di Lancio}: 2 Azioni\\
\textbf{Gittata}: 18 metri\\
\textbf{Componenti}: V, S, M (una calamita e un pizzico di polvere)\\
\textbf{Durata}: Istantanea\\
Un sottile raggio verde parte dal tuo dito puntato verso un bersaglio a gittata e che puoi vedere. Il bersaglio può essere una creatura, un oggetto o una creazione di forza magica, come un muro creato da muro di forza. Una creatura bersaglio di questo incantesimo deve effettuare un tiro salvezza su Agilità. Il bersaglio subisce 10d6 + 40 danni da forza se fallisce il tiro salvezza. Se questo danno riduce il bersaglio a 0 punti ferita, questi è disintegrato. Una creatura disintegrata e tutto quello che indossa e trasporta, eccetto gli oggetti magici, viene ridotta a una pila di sottile polvere grigia. La creatura può essereriportata in vita solo  tramite un incantesimo resurrezione pura o desiderio.\\
Questo incantesimo disintegra automaticamente gli oggetti non magici o una creazione di forza magica di taglia Grande o più piccola. Se il bersaglio è un oggetto non magico o una creazione di forza di taglia Enorme o più grossa, questo incantesimo disintegra una porzione di essa pari a un cubo di 3 metri di spigolo. Gli oggetti magici ignorano questo incantesimo.\\
\textbf{Per ogni Critico ottenuto} nella prova di magia danno aumenta di 3d6.

\medskip\textbf{Dissolvi il Bene e il Male}\index{Dissolvi il Bene e il Male}\\
\textbf{Difficolta'}: 23\\
\textbf{Tempo di Lancio}: 2 Azioni\\
\textbf{Gittata}: Personale\\
\textbf{Componenti}: V, S, M (acqua sacra o argento e ferro in polvere)\\
\textbf{Durata}: Concentrazione, massimo 1 minuto \\
Un’energia luminosa ti circonda e ti protegge da fatati, non morti e creature originarie di luoghi al di là del Piano Materiale. Per la durata, i celestiali, elementali, fatati, demoni e non morti hanno svantaggio ai tiri per colpire contro di te. Puoi terminare l’incantesimo anticipatamente usando una delle seguenti funzioni speciali.\\
\textit{Spezzare Ammaliamento}. Con un’azione, puoi entrare in contatto con una creatura affascinata, spaventata o posseduta da un celestiale, elementale, fatato, demoni o non morto. La creatura con cui sei in contatto non è più affascinata, spaventata o posseduta da queste creature.\\
\textit{Congedo}. Con un’azione, effettua un attacco da mischia con incantesimo contro un celestiale, elementale, fatato, immondo o non morto nella tua portata. Se lo colpisci, puoi cercare di rimandare la creatura al suo piano di origine. La creatura deve superare un tiro salvezza su Arbitrio o venire rispedita sul suo piano nativo (se non vi si trova già). Se non si trovano sul loro piano nativo, i non morti vengono rispediti nel Mondo delle Ombre e i fatati nel Primo Mondo.

\medskip\textbf{Dissolvi Magie}\index{Dissolvi Magie}\\
\textbf{Difficolta'}: 18\\
\textbf{Tempo di Lancio}: 2 Azioni\\
\textbf{Gittata}: 36 metri\\
\textbf{Componenti}: V, S\\
\textbf{Durata}: Istantanea\\
Scegli una creatura, oggetto o effetto magico a gittata. Qualsiasi incantesimo di difficolta' 18 o più basso sul bersaglio ha fine. \\
\textbf{Per ogni Critico ottenuto} nella prova di magia la Difficolta' dispellabile aumenta di 3.

\medskip\textbf{Dito della Morte}\index{Dito della Morte}\\
\textbf{Difficolta'}: 25\\
\textbf{Tempo di Lancio}: 2 Azioni\\
\textbf{Gittata}: 18 metri\\
\textbf{Componenti}: V, S\\
\textbf{Durata}: Istantanea\\
Invii una scarica di energia negativa a una creatura a gittata e che puoi vedere, provocandole profondo dolore. Il bersaglio deve effettuare un tiro salvezza su Tempra. Il bersaglio subisce 7d8 + 30 danni necrotici se fallisce il tiro salvezza, o la metà di questi danni se lo supera.\\
Un umanoide ucciso da questo incantesimo si rianima come zombi sotto il tuo comando permanente all’inizio del tuo prossimo turno, e seguirà i tuoi ordini verbali al meglio delle sue capacità.


\medskip\textbf{Divinazione}\index{Divinazione}\\
\textbf{Difficolta'}: 25\\
\textbf{Tempo di Lancio}: 2 Azioni\\
\textbf{Gittata}: Personale\\
\textbf{Componenti}: V, S, M (incenso e un’offerta sacrificale appropriata alla tua religione, il cui valore complessivo sia di 25 mo, che saranno consumati dall’incantesimo)\\
\textbf{Durata}: Istantanea\\
La tua magia e un’offerta votiva ti mettono in comunicazione con un dio o il servitore di un Patrono. Gli puoi porre una singola domanda in merito a uno specifico obiettivo, evento o attività che debba verificarsi entro 7 giorni. Il Narratore dà una risposta veritiera. La replica potrebbe essere una breve frase, una rima criptica o un presagio. \\
L’incantesimo non tiene conto di ogni possibile circostanza che possa modificare il risultato, come il lancio di ulteriori incantesimi o la perdita o l’arrivo di un alleato.\\
Se lanci l’incantesimo due o più volte prima di aver terminato il giorno lungo, c’è una probabilità cumulativa del 25\% che per ogni lancio dopo il primo tu ottenga una lettura erronea. Il Narratore effettua questo tiro in segreto. 

\medskip\textbf{Dominare Bestie}\index{Dominare Bestie}\\
\textbf{Difficolta'}: 20\\
\textbf{Tempo di Lancio}: 2 Azioni\\
\textbf{Gittata}: 18 metri\\
\textbf{Componenti}: V, S\\
\textbf{Durata}: Concentrazione, massimo 1 minuto\\
Cerchi di affascinare una bestia a gittata che puoi vedere. Essa deve superare un tiro salvezza su Arbitrio o restare affascinata per la durata, ricevendo vantaggio al tiro se tu o i tuoi alleati la state combattendo.\\
Mentre la bestia è affascinata, finché voi due vi trovate sullo stesso piano di esistenza mantieni un collegamento telepatico con essa. Puoi usare questo collegamento telepatico per inviare comandi alla creatura mentre sei cosciente (non richiede un’azione),
a cui essa obbedirà al suo meglio. Puoi specificare un corso d’azione semplice e generico, come “Attacca quella creatura”, “Corri laggiù”, o “Prendi quell’oggetto”. Se la creatura completa l’ordine e non riceve ulteriori indicazioni da te, si difenderà e preserverà al meglio delle sue capacità.\\
Puoi impiegare 2 tue azioni per assumere il totale e preciso controllo del bersaglio. Fino al termine del tuo prossimo turno, il bersaglio effettuerà solo le azioni decise da te, e non farà nulla che tu non gli permetta di fare. Durante questo periodo, puoi anche far usare una reazione al bersaglio, ma ciò richiede l’uso della tua reazione.\\
Ogni volta che il bersaglio subisce danni, effettua un nuovo tiro salvezza su Arbitrio contro l’incantesimo. Se supera il tiro salvezza, l’incantesimo termina.\\
\textbf{Per ogni Critico ottenuto} nella prova di magia la durata raddoppia fino ad un massimo di 8 ore.

\medskip\textbf{Dominare Mostri}\index{Dominare Mostri}\\
\textbf{Difficolta'}: 30\\
\textbf{Tempo di Lancio}: 2 Azioni\\
\textbf{Gittata}: 18 metri\\
\textbf{Componenti}: V, S\\
\textbf{Durata}: Concentrazione, massimo 1 ora\\
Cerchi di affascinare una creatura a gittata che puoi vedere. Essa deve superare un tiro salvezza su
Arbitrio o restare affascinata per la durata, ricevendo vantaggio al tiro se tu o i tuoi alleati la state combattendo.\\
Mentre la creatura è affascinata, finché voi due vi trovate sullo stesso piano di esistenza mantieni un collegamento telepatico con essa. Puoi usare questo collegamento telepatico per inviare comandi alla creatura mentre sei cosciente (non richiede un’azione), a cui essa obbedirà al suo meglio. Puoi specificare un corso d’azione semplice e generico, come “Attacca quella creatura”, “Corri laggiù”, o “Prendi quell’oggetto”. Se la creatura completa l’ordine e non riceve ulteriori indicazioni da te, si difenderà e preserverà al meglio delle sue capacità.\\
Puoi impiegare due tua Azioni per assumere il totale e preciso controllo del bersaglio. Fino al termine del tuo prossimo turno la creatura effettuerà solo le azioni decise da te, e non farà nulla che tu non le permetta di fare. Durante questo periodo, puoi anche far usare una reazione alla creatura, ma ciò richiede l’uso della tua reazione. Ogni volta che il bersaglio subisce danni, effettua un nuovo tiro salvezza su Arbitrio contro l’incantesimo. Se supera il tiro salvezza, l’incantesimo termina.\\
\textbf{Per ogni Critico ottenuto} nella prova di magia la durata raddoppia fino ad un massimo di 8 ore.

\medskip\textbf{Dominare Persone}\index{Dominare Persone}\\
\textbf{Difficolta'}: 23\\
\textbf{Tempo di Lancio}: 2 Azioni\\
\textbf{Gittata}: 18 metri\\
\textbf{Componenti}: V, S\\
\textbf{Durata}: Concentrazione, massimo 1 minuto\\
Cerchi di affascinare un umanoide a gittata che puoi vedere. Esso deve superare un tiro salvezza su Arbitrio o restare affascinato per la durata, ricevendo vantaggio al tiro se tu o i tuoi alleati lo state combattendo.\\
Mentre il bersaglio è affascinato, finché voi due vi trovate sullo stesso piano di esistenza mantieni un collegamento telepatico con esso. Puoi usare questo collegamento telepatico per inviare comandi al bersaglio mentre sei cosciente (non richiede un’azione), a cui esso obbedirà al suo meglio. Puoi specificare un corso d’azione semplice e generico, come “Attacca quella creatura”, “Corri laggiù”, o “Prendi quell’oggetto”. Se il bersaglio completa l’ordine e non riceve ulteriori indicazioni da te, si difenderà e preserverà al meglio delle sue capacità.\\
Puoi impiegare la tua azione per assumere il totale e preciso controllo del bersaglio. Fino al termine del tuo prossimo turno, il bersaglio effettuerà solo le azioni decise da te, e non farà nulla che tu non gli permetta di fare. Durante questo periodo, puoi anche far usare una reazione al bersaglio, ma ciò richiede l’uso della tua reazione. Ogni volta che il bersaglio subisce danni, effettua un nuovo tiro salvezza su Arbitrio contro l’incantesimo. Se supera il tiro salvezza, l’incantesimo termina.\\
\textbf{Per ogni Critico ottenuto} nella prova di magia la durata raddoppia fino ad un massimo di 8 ore.

\medskip\textbf{Eroismo}\index{Eroismo}\\
\textbf{Difficolta'}: 13\\
\textbf{Tempo di Lancio}: 2 Azioni\\
\textbf{Gittata}: Contatto\\
\textbf{Componenti}: V, S\\
\textbf{Durata}: Concentrazione, massimo 1 minuto\\
Una creatura consenziente con cui sei in contatto vene infusa di coraggio. Fino al termine dell’incantesimo, la creatura è immune all’essere spaventata e, all’inizio di ciascun suo turno, ottiene punti ferita temporanei pari al tuo valore di Intelletto o Volontà. Quando l’incantesimo ha termine, il bersaglio perde tutti  i punti ferita temporanei rimanenti derivati da questo
incantesimo.

\medskip\textbf{Esilio}\index{Esilio}\\
\textbf{Difficolta'}: 20\\
\textbf{Tempo di Lancio}: 2 Azioni\\
\textbf{Gittata}: 18 metri\\
\textbf{Componenti}: V, S, M (un oggetto disprezzato dal bersaglio)\\
\textbf{Durata}: Concentrazione, massimo 1 minuto\\
Cerchi di spedire una creatura a gittata e che puoi vedere in un altro piano di esistenza. Il bersaglio deve superare un tiro salvezza su Arbitrio o venire esiliato. Se il bersaglio è natio del piano di esistenza in cui ti trovi, esili il bersaglio in un semipiano innocuo. Mentre è lì, il bersaglio è inabile. Il bersaglio rimane lì fino al termine dell’incantesimo, quando riapparirà nello spazio che aveva lasciato o nello spazio non occupato più vicino, se il suo spazio originale adesso è occupato. Se il bersaglio è natio di un diverso piano di esistenza da quello in cui ti trovi, il bersaglio svanisce emettendo un lieve scoppio, tornando al suo piano natio. Se l’incantesimo termina prima che sia trascorso 1 minuto, il bersaglio riappare nello spazio che aveva lasciato o nello spazio non occupato più vicino, se il suo spazio originale è occupato.\\
\textbf{Per ogni Critico ottenuto} nella prova di magia puoi influenzare un altra creatura

\medskip\textbf{Esplosione Solare}\index{Esplosione Solare}\\
\textbf{Difficolta'}: 30\\
\textbf{Tempo di Lancio}: 2 Azioni\\
\textbf{Gittata}: 45 metri\\
\textbf{Componenti}: V, S, M (fuoco e un pezzo di pietra di sole)\\
\textbf{Durata}: Istantanea\\
Un’intensa luce solare illumina in un raggio di 18 metri centrato su di un punto a gittata, scelto da te. Tutte le creature all’interno della luce devono effettuare un tiro salvezza su Tempra. Se fallisce il tiro salvezza, una creatura subisce 12d6 danni da Luce e resta accecata per 1 minuto. Se lo supera, subisce la metà dei danni e non resta accecata dall’incantesimo. Non morti e melme hanno -2d6 a questo tiro salvezza. Una creatura accecata da questo incantesimo effettua un altro tiro salvezza su Potenza alla fine di ciascun suo round. Se supera il tiro salvezza, non è più accecata.\\
Nella sua area, questo incantesimo dissolve qualsiasi oscurità generata da un incantesimo. 

\medskip\textbf{Estasiare}\index{Estasiare}
\textbf{Difficolta'}: 15\\
\textbf{Tempo di Lancio}: 2 Azioni\\
\textbf{Gittata}: Personale\\
\textbf{Componenti}: V, S\\
\textbf{Durata}: 1 minuto\\
Intessi una serie di parole svianti, facendo sì che delle creature di tua scelta entro la gittata, che puoi vedere e possano sentirti, effettuino un tiro salvezza su Arbitrio. Qualsiasi creatura che non può restare affascinata supera il tiro salvezza automaticamente, e se tu o i tuoi compagni state combattendo una creatura, questa ha vantaggio al tiro salvezza. Se fallisce il tiro salvezza, il bersaglio ha -1d6 sulle prove di Consapevolezza effettuate per percepire una qualsiasi creatura diversa da te fino al termine dell’incantesimo o finché il bersaglio non può più sentirti.
L’incantesimo termina se sei reso inabile o non puoi più parlare.

\medskip\textbf{Evoca Animali}\index{Evoca Animali}\\
\textbf{Difficolta'}: 18\\
\textbf{Tempo di Lancio}: 2 Azioni\\
\textbf{Gittata}: 18 metri\\
\textbf{Componenti}: V, S\\
\textbf{Durata}: Concentrazione, massimo 1 ora\\
Evochi spiriti fatati che assumono l’aspetto di bestie e compaiono in spazi non occupati a gittata e che puoi vedere. Scegli una delle seguenti opzioni per determinare ciò che appare:
\begin{itemize}
\item
- Una bestia di grado di sfida 2 o inferiore
\item
- Due bestie di grado di sfida 1 o inferiore
\item
- Quattro bestie di grado di sfida 1/2 o inferiore
\item
- Otto bestie di grado di sfida 1/4 o inferiore
\end{itemize}
\medskip
Ogni bestia è considerata anche un fatato, e sparisce quando scende a 0 punti ferita o quando l’incantesimo termina. \\
Le creature evocate sono amichevoli verso di te e i tuoi compagni. Tirare l’iniziativa per le creature evocate come gruppo, che agisce durante il proprio turno. Esse obbediscono a qualsiasi comando verbale che gli viene dato (senza bisogno che tu compia azioni). Se non dai comandi alle bestie, si difenderanno dalle creature ostili, ma non compiranno altre azioni.\\
\textbf{Per ogni Critico ottenuto} nella prova di magia appariranno due bestie in piu'

\medskip\textbf{Evoca Celestiali}\index{Evoca Celestiali}\\
\textbf{Difficolta'}: 28\\
\textbf{Tempo di Lancio}: 1 minuto\\
\textbf{Gittata}: 27 metri\\
\textbf{Componenti}: V, S\\
\textbf{Durata}: Concentrazione, massimo 1 ora\\
Evochi un celestiale di grado di sfida 4 o inferiore, che appare in uno spazio non occupato a gittata e che puoi vedere. Il celestiale sparisce quando scende a 0 punti ferita o l’incantesimo termina. Il celestiale è amichevole verso di te e i tuoi compagni per la durata dell’incantesimo. Tira l’iniziativa per il celestiale, che agisce durante il proprio turno. Obbedisce a qualsiasi comando verbale che gli viene dato (senza bisogno che tu compia azioni), purché non violi il suo allineamento. Se non dai comandi al celestiale, si difenderà dalle creature ostili, ma non compirà altre azioni.\\
\textbf{Per ogni Critico ottenuto} nella prova di magia aumenti di uno il CR della creatura evocata.

\medskip\textbf{Evoca Creature Boschive}\index{Evoca Creature Boschive}\\
\textbf{Difficolta'}: 20\\
\textbf{Tempo di Lancio}: 2 Azioni\\
\textbf{Gittata}: 18 metri\\
\textbf{Componenti}: V, S, M (una bacca di agrifoglio per creature convocata)\\
\textbf{Durata}: Concentrazione, massimo 1 ora \\
Evochi spiriti fatati che compaiono in spazi non occupati a gittata e che puoi vedere. Scegli una delle seguenti opzioni per determinare ciò che appare:
\begin{itemize}
\item Un fatato di grado di sfida 2 o inferiore
\item Due fatati di grado di sfida 1 o inferiore
\item Quattro fatati di grado di sfida 1/2 o inferiore
\item Otto fatati di grado di sfida 1/4 o inferiore
\end{itemize}
\medskip
Una creatura evocata sparisce quando scende a 0 punti ferita o quando l’incantesimo termina. Le creature evocate sono amichevoli verso di te e i tuoi compagni. Tirare l’iniziativa per le creature evocate come gruppo, che agisce durante il proprio turno. Esse obbediscono a qualsiasi comando verbale che gli viene dato (senza bisogno che tu compia azioni). Se non dai comandi ai fatati, si difenderanno dalle creature ostili, ma non compiranno altre azioni.\\
\textbf{Per ogni Critico ottenuto} nella prova di magia appariranno due creature in piu'

\medskip\textbf{Evoca Elementale}\index{Evoca Elementale}\\
\textbf{Difficolta'}: 23\\
\textbf{Tempo di Lancio}: 1 minuto\\
\textbf{Gittata}: 27 metri\\
\textbf{Componenti}: V, S, M (incenso bruciato per l’aria, argilla malleabile per la terra, zolfo e fosforo per il fuoco, o acqua e sabbia per l’acqua) \\
\textbf{Durata}: Concentrazione, massimo 1 ora\\
Evochi un servitore elementale. Scegli un’area a gittata composta di acqua, aria, fuoco o terra e che riempia un cubo di 3 metri di spigolo. Un elementale di grado di sfida 5 o minore appropriato all’area da te scelta compare in uno spazio non occupato entro 3 metri da essa. L’elementale sparisce quando scende a 0 punti ferita o l’incantesimo termina.\\
L’elementale è amichevole verso di te e i tuoi compagni per la durata dell’incantesimo. Tira l’iniziativa per l’elementale, che agisce durante il proprio turno. Obbedisce a qualsiasi comando verbale che gli viene dato (senza bisogno che tu compia azioni). Se non dai comandi all’elementale, si difenderà dalle creature ostili, ma non compirà altre azioni.\\
Se la tua concentrazione viene infranta, l’elementale non sparisce. Invece, perderai il controllo dell’elementale, che diventerà ostile verso di te e i tuoi compagni, e potrebbe attaccarvi. Un elementale fuori controllo non può essere congedato da te, e sparisce 1 ora dopo che lo hai convocato.\\
\textbf{Per ogni Critico ottenuto} nella prova di magia il grado di sfida aumenta di 1

\medskip\textbf{Evoca Elementali Minori}\index{Evoca Elementali Minori}\\
\textbf{Difficolta'}: 20\\
\textbf{Tempo di Lancio}: 1 minuto\\
\textbf{Gittata}: 27 metri\\
\textbf{Componenti}: V, S\\
\textbf{Durata}: Concentrazione, massimo 1 ora\\
Evochi degli elementali che compariranno in spazi non occupati a gittata e che puoi vedere. Scegli una della seguenti opzioni per decidere cosa appare:
\begin{itemize}
\item Un elementale di grado di sfida 2 o meno
\item Due elementali di grado di sfida 1 o meno
\item Quattro elementali di grado di sfida 1/2 o meno
\item Otto elementali di grado di sfida 1/4 o meno
\end{itemize}
\medskip
Un elementale evocato sparisce quando scende a 0 punti ferita o l’incantesimo termina. Un elementale evocato è amichevole verso di te e i tuoi compagni. Tirare l’iniziativa per gli elementali evocati come gruppo, che agisce durante il proprio turno. Essi obbediscono a qualsiasi comando verbale che gli viene dato (senza bisogno che tu compia azioni). Se non dai comandi agli elementali, si difenderanno dalle creature ostili, ma non compiranno altre azioni.\\
\textbf{Per ogni Critico ottenuto} nella prova di magia appariranno due Elementali

\medskip\textbf{Evoca Folletto}\index{Evoca Folletto}\\
\textbf{Difficolta'}: 25\\
\textbf{Tempo di Lancio}: 1 minuto\\
\textbf{Gittata}: 27 metri\\
\textbf{Componenti}: V, S\\
\textbf{Durata}: Concentrazione, massimo 1 ora \\
Evochi uno spirito fatato di grado di sfida 6 o inferiore, o uno spirito fatato che assuma la forma di una bestia di grado di sfida 6 o inferiore. Esso compare in uno spazio non occupato a gittata e che puoi vedere. La creatura fatata sparisce quando scende a 0 punti ferita o quando l’incantesimo termina.\\
La creatura fatata è amichevole verso di te e i tuoi compagni. Tirare l’iniziativa per la creatura fatata, che agisce durante i propri turni. Essa obbedisce a qualsiasi comando verbale che gli viene dato (senza bisogno che tu compia azioni), purché non violi il suo allineamento. Se non dai comandi, si difenderà dalle creature ostili, ma non compirà altre azioni.\\
Se la tua concentrazione viene infranta, la creatura fatata non sparisce. Invece, perderai il controllo della creatura fatata, che diventerà ostile verso di te e i tuoi compagni, e potrebbe attaccarvi. Una creatura fatata fuori controllo non può essere congedata da te, e sparisce 1 ora dopo che l’hai evocata.\\
\textbf{Per ogni Critico ottenuto} nella prova di magia aumenti di 1 il CR della creatura evocata.

\medskip\textbf{Evocazioni Istantanee}\index{Evocazioni Istantanee}\\
\textbf{Difficolta'}: 25\\
\textbf{Tempo di Lancio}: 1 minuto\\
\textbf{Gittata}: Contatto\\
\textbf{Componenti}: V, S, M (uno zaffiro del valore di 1.000 mo)\\
\textbf{Durata}: Fino a che dissolto \\
Entri a contatto con un oggetto del peso di 5 chili o meno e la cui dimensione più grossa non superi i 180 centimetri. L’incantesimo lascia un marchio sulla superficie dell’oggetto e ne incide invisibilmente il nome sullo zaffiro usato come componente materiale. Ogni volta che lanci questo incantesimo, devi usare uno zaffiro diverso.\\
In qualsiasi momento successivo, puoi usare la tua azione per pronunciare il nome dell’oggetto e frantumare lo zaffiro. L’oggetto appare istantaneamente nella tua mano quale che sia la distanza fisica o planare che vi separa, e l’incantesimo ha termine.\\
Se un’altra creatura sta impugnando o trasportando l’oggetto, frantumare lo zaffiro non trasporterà l’oggetto da te, ma invece apprenderai chi sia la creatura che ne è in possesso e indicativamente dove si trovi in questo momento.\\
Dissolvi magie, o un effetto simile applicato con successo allo zaffiro, termina l’effetto dell’incantesimo. 

\medskip\textbf{Fabbricare}\index{Fabbricare}\\
\textbf{Difficolta'}: 20\\
\textbf{Tempo di Lancio}: 10 minuti\\
\textbf{Gittata}: 36 metri\\
\textbf{Componenti}: V, S\\
\textbf{Durata}: Istantanea\\
Converti le materie prime in prodotti finiti dello stesso materiale. Per esempio, puoi fabbricare un ponte di legno da un cumulo di alberi, una corda da un mucchio di canapa, e abiti dal lino o la lana. Scegli le materie prima che puoi vedere a gittata. Puoi fabbricare un oggetto di taglia Grande o inferiore (contenuto in un cubo di 3 metri di spigolo, o otto cubi connessi di 1 metro di spigolo) data una sufficiente quantità di materie prime. Se stai lavorando con il metallo, la pietra o altre sostanze minerali, l’oggetto fabbricato non può essere più grande di taglia Media (contenuto in un singolo cubo di 1 metro di spigolo). La qualità degli oggetti creati da questo incantesimo è commisurata alla qualità delle materie prime.\\
Tramite questo incantesimo non si possono creare o trasmutare creature od oggetti magici. Inoltre non puoi usarlo per creare oggetti che normalmente richiedono un alto livello di lavorazione, come i gioielli, le armi, il vetro o le armature, a meno che tu non abbia la competenza con il tipo di strumenti da artigiano utilizzati per costruire questi oggetti.

\medskip\textbf{Faro di Speranza}\index{Faro di Speranza}\\
\textbf{Difficolta'}: 18\\
\textbf{Tempo di Lancio}: 2 Azioni\\
\textbf{Gittata}: 9 metri\\
\textbf{Componenti}: V, S\\
\textbf{Durata}: Concentrazione, massimo 1 minuto\\
Questo incantesimo conferisce speranza e vitalità. Scegli un qualsiasi numero di creature a gittata. Per la durata, ciascun bersaglio ha vantaggio ai tiri salvezza su Saggezza e ai tiri salvezza da morte, e da ogni cura ottiene il massimo numero di punti ferita possibili.

\medskip\textbf{Fatale}\index{Fatale}
\textbf{Difficolta'}: 33\\
\textbf{Tempo di Lancio}: 2 Azioni\\
\textbf{Gittata}: 36 metri\\
\textbf{Componenti}: V, S\\
\textbf{Durata}: Concentrazione, massimo 1 minuto\\
Attingendo alle paure più intime di un gruppo di creature, crei delle creature illusorie nella loro mente, visibili solo a loro. Ogni creatura in una sfera di 9 metri di raggio centrata su di un punto a tua scelta nella gittata, deve effettuare un tiro salvezza su Arbitrio. Se fallisce il tiro salvezza, la creatura diventa spaventata per la durata. L’illusione affonda nelle paure più intime della creatura, manifestando i suoi incubi peggiori come una implacabile minaccia. Alla fine di ciascun turno della creatura spaventata, questa deve superare un tiro salvezza su Arbitrio o subire 4d10 danni. Se supera il tiro salvezza, per quella creatura l’incantesimo ha termine.

\medskip\textbf{Favore Divino}\index{Favore Divino}\\
\textbf{Difficolta'}: 13\\
\textbf{Tempo di Lancio}: 1 Azione Immediata\\
\textbf{Gittata}: Personale\\
\textbf{Componenti}: V, S\\
\textbf{Durata}: Concentrazione, massimo 1 minuto\\
Le tue preghiere potenziano te e la tua arma. Fino al termine dell’incantesimo, quando colpisce, la tua arma infligge 1d4 danni da Luce aggiuntivi.

\medskip\textbf{Ferire}\index{Ferire}\\
\textbf{Difficolta'}: 25\\
\textbf{Tempo di Lancio}: 2 Azioni\\
\textbf{Gittata}: 18 metri\\
\textbf{Componenti}: V, S\\
\textbf{Durata}: Istantanea\\
Scateni una malattia virulenta su di una creatura a gittata che puoi vedere. Il bersaglio deve effettuare un tiro salvezza su Potenza. Il bersaglio subisce 14d6 danni necrotici se fallisce il tiro salvezza, o la metà di questi danni se lo supera. il danno non può ridurre i punti ferita del bersaglio sotto l’1. Se il bersaglio fallisce il tiro salvezza, i suoi punti ferita massimi sono ridotti per 1 ora di un ammontare uguale al danno necrotico subito. Qualsiasi effetto che rimuova una malattia permette ai punti ferita massimi del personaggio di poter tornare al valore normale prima che trascorra quel tempo.

\medskip\textbf{Fermare il Tempo}\index{Fermare il Tempo}\\
\textbf{Difficolta'}: 33\\
\textbf{Tempo di Lancio}: 2 Azioni\\
\textbf{Gittata}: Personale\\
\textbf{Componenti}: V\\
\textbf{Durata}: Istantanea\\
Fermi brevemente il flusso del tempo per tutti, tranne che per te. Il tempo non scorre per le altre creature,  mentre tu effettui 1d4 + 1 round  di fila, durante i quali puoi effettuare azioni e muoverti come sempre. Questo incantesimo termina se una delle azioni che usi durante questo periodo, o qualsiasi effetto che crei durante questo periodo, ha effetto su di una creatura diversa da te o su di un oggetto indossato o trasportato da qualcuno che non sia tu. Inoltre, l’incantesimo termina se ti muovi in un posto lontano più di 300 metri da quello in cui lo hai lanciato.

\medskip\textbf{Fiamma Perenne}\index{Fiamma Perenne}\\
\textbf{Difficolta'}: 15\\
\textbf{Tempo di Lancio}: 2 Azioni\\
\textbf{Gittata}: Contatto\\
\textbf{Componenti}: V, S, M (polvere di rubino del valore di 50 mo, che l’incantesimo consuma\\ \textbf{Durata}: Fino a che dissolto\\
Una luminosità simile a quella prodotta da una fiaccola si sprigiona da un oggetto con cui sei in contatto. L’effetto sembra quello di una normale fiamma, ma non produce calore né necessita ossigeno. Una fiamma perpetua può essere celata o nascosta ma non può essere smorzata né spenta.

\medskip\textbf{Fiamma Sacra}\index{Fiamma Sacra}\\
\textbf{Difficolta'}: 11\\
\textbf{Tempo di Lancio}: 2 Azioni\\
\textbf{Gittata}: 18 metri\\
\textbf{Componenti}: V, S\\
\textbf{Durata}: Istantanea\\
Una luminosità simile a quella prodotta da una fiaccola discende su di una creatura a gittata che puoi vedere. Il bersaglio deve superare un tiro salvezza su Agilità o subire 1d8 danni radianti. Il bersaglio non riceve il beneficio della copertura per questo tiro salvezza.\\
Il danno dell’incantesimo aumenta di 1d8 quando raggiungi CM 5, CM 11 e CM 17.

\medskip\textbf{Fiotto Acido}\index{Fiotto Acido}\\
\textbf{Difficolta'}: 11\\
\textbf{Tempo di Lancio}: 2 Azioni\\
\textbf{Gittata}: 18 metri\\
\textbf{Componenti}: V, S\\
\textbf{Durata}: Istantanea\\
Scagli una bolla di acido. Scegli una creatura a gittata o due creature a gittata che siano entro 1 metro l’una dall’altra. Il bersaglio deve superare un tiro salvezza su Agilità o subire 1d6 danni da acido.\\
Il danno dell’incantesimo aumenta di 1d8 quando raggiungi CM 5, CM 11 e CM 17.

\medskip\textbf{Folata di Vento}\index{Folata di Vento}\\
\textbf{Difficolta'}: 15\\
\textbf{Tempo di Lancio}: 2 Azioni\\
\textbf{Gittata}: Personale (linea di 18 metri)\\
\textbf{Componenti}: V, S, M (un seme di legume)\\
\textbf{Durata}: Concentrazione, massimo 1 minuto\\
Una linea di forte vento lunga 18 metri e larga 3 metri esplode partendo da te in una direzione a tua scelta, per la durata dell’incantesimo. Ogni creatura che inizia il suo turno dentro la linea deve superare un tiro salvezza su Forza o venire spinta lontano da te di 4,5 metri, seguendo la direzione della linea.\\
Qualsiasi creatura sulla linea deve spendere il doppio del movimento per avvicinarsi a te.\\
La folata disperde gas o vapori, estingue candele, torce e simili fiamme non protette nell’area. Le fiamme protette, come quelle della lanterne, si agitano, e hanno una probabilità del 50\% di estinguersi. Come azione bonus durante ciascun tuo turno, prima del termine dell’incantesimo, puoi cambiare la direzione in cui la linea si proietta da te.

\medskip\textbf{Fondersi nella Pietra}\index{Fondersi nella Pietra}\\
\textbf{Difficolta'}: 18\\
\textbf{Tempo di Lancio}: 2 Azioni\\
\textbf{Gittata}: Contatto\\
\textbf{Componenti}: V, S\\
\textbf{Durata}: 8 ore\\
Entri in un oggetto o superficie di pietra grossi abbastanza da contenere tutto il tuo corpo, fondendoti con la pietra assieme a tutto l’equipaggiamento che trasporti per la durata. Usando il tuo movimento, entri nella pietra in un punto con cui sei in contatto. Non resta nulla della tua presenza che rimanga visibile o altrimenti possa essere individuato da sensi non magici. Mentre sei fuso con la pietra, non puoi vedere ciò che avviene all’esterno, e qualsiasi prova di Saggezza (Percezione) che effettui per ascoltare i suoni prodotti fuori da essa è fatta con svantaggio. Resti consapevole del passare del tempo e puoi lanciare incantesimi su di te mentre sei fuso con la pietra. Puoi usare il tuo movimento per lasciare la pietra e ricomparire nel punto in cui vi sei entrato, terminando così l’incantesimo. Altrimenti non puoi muoverti.\\
I danni minori alla pietra non ti danneggiano, ma la sua parziale distruzione o cambio di forma (di modo che tu non vi entri più) ti espellono da essa e ti infliggono 6d6 danni contundenti. La completa distruzione della pietra (o la sua trasmutazione in un’altra sostanza) ti fa espellere e ti infligge 50 danni contundenti. Se vieni espulso, cadi prono in uno spazio non occupato, nel punto più vicino a quello in cui sei entrato nella pietra.

\medskip\textbf{Forma Eterea}\index{Forma Eterea}
\textbf{Difficolta'}: 28\\
\textbf{Tempo di Lancio}: 2 Azioni\\
\textbf{Gittata}: Personale\\
\textbf{Componenti}: V, S\\
\textbf{Durata}: Massimo 8 ore\\
Entri nelle regioni di confine del Piano Etereo, nell’area che si sovrappone al tuo piano attuale. Resti sul Confine Etereo per la durata o finché non usi un’azione per interrompere l’incantesimo. Se ti muovi verso l’alto o il basso, il costo del movimento è raddoppiato. Puoi vedere e udire il piano da cui provieni, ma tutto quello che si trova lì ti appare grigio, e non puoi vedere a più di 18 metri di distanza.\\
Mentre sei sul Piano Etereo, può interagire solo con altre creature su quel piano. Le creature che non sono sul Piano Etereo non ti possono percepire né interagire con te, a meno che una capacità speciale o la magia gli fornisca la possibilità di farlo.\\
Ignori tutti gli oggetti e gli effetti che non sono sul Piano etereo, potendo così attraversare gli  oggetti che percepisci sul piano da cui provieni. Quando l’incantesimo termina, ritorni immediatamente al piano da cui provieni nel punto che occupi attualmente. Se quando accade occupi lo stesso spazio di un oggetto solido o di una creatura, vieni immediatamente spostato nel più vicino spazio non occupato che puoi occupare e subisci 6 danni da forza per ogni metro di cui vieni spostato (o sua frazione). Questo incantesimo non ha effetto se lo esegui mentre sei già nel Piano Etereo o su di un piano che non vi confina, come uno dei Piani Esterni.\\
\textbf{Per ogni Critico ottenuto} nella prova di magia puoi portare con te un altra creatura.

\medskip\textbf{Forma Gassosa}\index{Forma Gassosa}\\
\textbf{Difficolta'}: 18\\
\textbf{Tempo di Lancio}: 2 Azioni\\
\textbf{Gittata}: Contatto\\
\textbf{Componenti}: V, S, M (un pezzo di garza e un filo di fumo)\\
\textbf{Durata}: Concentrazione, massimo 1 ora\\
Trasformi una creatura consenziente con cui sei in contatto, insieme a tutto ciò che sta indossando e trasportando, in una nube vaporosa per la durata. L’incantesimo termina se la creatura scende a 0 punti ferita. Le creature incorporee ignorano questo effetto. Mentre è in questa forma, l’unico metodo di movimento del bersaglio è una velocità di volo 3 metri. Il bersaglio può entrare e occupare lo spazio di un’altra creatura. Il bersaglio ha resistenza ai danni non magici, e ha vantaggio ai tiri salvezza su Potenza, Agilita' e Costituzione. Il bersaglio può attraversare piccoli buchi, strettoie, e anche semplici fori, sebbene consideri i liquidi come superfici solide. Il bersaglio non può cadere e resta fluttuante nell’aria anche se stordito o altrimenti reso inabile.\\
Mentre è nella forma di una nube vaporosa, il bersaglio non può parlare né manipolare oggetti, e qualsiasi oggetto stesse indossando o trasportando non può essere gettato, usato o altrimenti impiegato. Il bersaglio non può attaccare né lanciare incantesimi. 

\medskip\textbf{Forme Animali}\index{Forme Animali}\\
\textbf{Difficolta'}: 30\\
\textbf{Tempo di Lancio}: 2 Azioni\\
\textbf{Gittata}: 9 metri\\
\textbf{Componenti}: V, S\\
\textbf{Durata}: Concentrazione, massimo 24 ore\\
Trasformi magicamente altre creature in bestie. Scegli un qualsiasi numero di creature consenzienti a gittata e che puoi vedere. Trasformi ciascun bersaglio nella forma di una bestia di taglia Grande o minore con un grado di sfida 4 o inferiore. Nei turni successivi, puoi usare la tua azione per trasformare le creature soggette in nuove forme.\\
La trasformazione permane per ciascun bersaglio per la durata dell’incantesimo, o finché quel bersaglio scende a 0 punti ferita o muore. Puoi scegliere una forma diversa per ciascun bersaglio. Le  statistiche di gioco del bersaglio sono rimpiazzate dalle statistiche della bestia scelta, a eccezione dell’allineamento e dei punteggi di Intelletto, Volontà e Magnetismo che restano quelli del
bersaglio. Il bersaglio assume i punti ferita della sua nuova forma e, quando ritorna alla sua forma normale, ritorna al numero di punti ferita che aveva prima di trasformarsi. Se si ritrasforma perché è sceso a 0 punti ferita, il danno in eccesso viene applicato alla forma originale. Purché il danno in eccesso non riduca la forma normale della creatura a 0 punti ferita, essa non è priva di sensi. La creatura è limitata nelle azioni che può svolgere dalla natura della sua nuova forma, e non può parlare né lanciare incantesimi.\\
L’equipaggiamento del bersaglio si fonde nella nuova forma. Il bersaglio non può attivare, impugnare o in altro modo beneficiare del suo equipaggiamento.

\medskip\textbf{Frantumare}\index{Frantumare}\\
\textbf{Difficolta'}: 15\\
\textbf{Tempo di Lancio}: 2 Azioni\\
\textbf{Gittata}: 18 metri\\
\textbf{Componenti}: V, S, M (un frammento di metallo)\\
\textbf{Durata}: Istantanea\\
Un forte squillo, molto intenso, erutta da un punto a gittata di tua scelta. Ogni creatura in una sfera di 3 metri di raggio centrata su quel punto deve effettuare un tiro salvezza su Potenza. Una creatura subisce 3d8 danni da tuono se fallisce il tiro salvezza, o la metà di questi danni se lo supera. Una creatura composta di materiale inorganico, come pietra, cristallo o metallo, ha svantaggio sul tiro salvezza. Un oggetto non magico che non è indossato né trasportato subisce anch’esso danni se si trova nell’area dell’incantesimo.\\
\textbf{Per ogni Critico ottenuto} nella prova di magia il danno aumenta di 1d8.

\medskip\textbf{Freccia Acida}\index{Freccia Acida}\\
\textbf{Difficolta'}: 15\\
\textbf{Tempo di Lancio}: 2 Azioni\\
\textbf{Gittata}: 27 metri\\
\textbf{Componenti}: V, S, M (una foglia di rabarbaro in polvere e uno stomaco di pitone)\\
\textbf{Durata}: Istantanea\\
Una freccia verde luminosa saetta verso un bersaglio a gittata ed esplode con uno spruzzo d’acido. Effettua un attacco a distanza con incantesimo contro il bersaglio. Se colpisci, il bersaglio subisce immediatamente 4d4 danni da acido e 2d4 danni da acido al termine del suo prossimo turno. Se manchi, la freccia spruzza il bersaglio di acido infliggendo la metà dei danni iniziali e non arrecando danni al termine del prossimo turno del bersaglio.\\
\textbf{Per ogni Critico ottenuto} nella prova di magia il danno aumento di 1d4.

\medskip\textbf{Fulmine}\index{entry}\\
\textbf{Difficolta'}: 18\\
\textbf{Tempo di Lancio}: 2 Azioni\\
\textbf{Gittata}: Personale (linea di 30 metri)\\
\textbf{Componenti}: V, S, M (un pezzo di pelliccia e una verga d’ambra, cristallo o vetro)\\
\textbf{Durata}: Istantanea\\
Esplodi un fulmine che forma una linea lunga 30 metri e larga 1 metro che parte da dove ti trovi in una direzione scelta da te. Ogni creatura sulla linea deve superare un tiro salvezza su Agilita'. La creatura subisce 8d6 danni da fulmine se fallisce il tiro salvezza, o la metà di questi danni se lo supera.\\
Il fulmine incendia gli oggetti infiammabili nell’area che non sono indossati o trasportati. \\
\textbf{Per ogni Critico ottenuto} nella prova di magia il danno aumenta di 1d6.

\medskip\textbf{Fuorviare}\index{Fuorviare}\\
\textbf{Difficolta'}: 23\\
\textbf{Tempo di Lancio}: 2 Azioni\\
\textbf{Gittata}: Personale\\
\textbf{Componenti}: S\\
\textbf{Durata}: Concentrazione, massimo 1 ora\\
Diventi invisibile nello stesso momento in cui un tuo doppione illusorio compare nel posto in cui ti trovi. Il doppione resta per la durata dell’incantesimo, ma l’invisibilità termina se attacchi o lanci un incantesimo. Puoi usare la tua azione per far muovere il doppione illusorio fino al doppio della tua velocità e fargli compiere un gesto, parlare e comportarsi in qualsiasi maniera tu voglia.\\
Puoi vedere attraverso i suoi occhi e udire tramite le sue orecchie come se fossi nello spazio in cui si trova lui. Durante ciascun tuo turno, con un’azione bonus, puoi passare dall’usare i suoi sensi all’usare i tuoi, o viceversa. Mentre stai usando i suoi sensi, sei accecato e assordato riguardo i tuoi dintorni. 

\medskip\textbf{Gabbia di Forza}\index{Gabbia di Forza}\\
\textbf{Difficolta'}: 28\
\textbf{Tempo di Lancio}: 2 Azioni\\
\textbf{Gittata}: 30 metri\\
\textbf{Componenti}: V, S, M (polvere di rubino del valore di 1.500 mo)\\
\textbf{Durata}: 1 ora\\
Una prigione cubica, immobile e invisibile, composta di forza magica compare intorno a un’area a gittata da te scelta. La prigione può essere una gabbia o una scatola solida, a tua scelta. Una prigione nella forma di una gabbia può avere 6 metri di lato ed essere composta da sbarre di 1,5 centimetri separate di 1,5 centimetri tra di loro. Una prigione a forma di scatola può avere 3 metri di lato, creando una barriera solida che impedisce a qualsiasi materia di attraversarla e bloccando qualsiasi incantesimo lanciato dall’interno o l’esterno dell’area. Quando lanci questo incantesimo, qualsiasi creatura che è completamente all’interno della gabbia, è intrappolata. Le creature solo parzialmente nell’area della gabbia, o quelle troppo grosse per entrarvi, vengono spinte via dal centro dell’area finché non ne sono completamente fuori.\\
Una creatura all’interno della gabbia non può lasciarla tramite mezzi non magici. Se la creatura prova a usare il teletrasporto o il viaggio interplanare per lasciare la gabbia, deve prima effettuare un tiro salvezza su Carisma. Se lo supera, la creatura può usare quella magia per sfuggire alla gabbia. Se lo fallisce, la creatura non può uscire dalla gabbia e spreca l’uso dell’incantesimo o dell’effetto. La gabbia si estende anche sul Piano Etereo, bloccando così il viaggio etereo.\\
Questo incantesimo non può essere dissolto da dissolvi magie.

\medskip\textbf{Giara Magica}\index{Giara Magica}\\
\textbf{Difficolta'}: 25\\
\textbf{Tempo di Lancio}: 1 minuto\\
\textbf{Gittata}: Personale\\
\textbf{Componenti}: V, S, M (una gemma, cristallo, reliquario o qualche altro contenitore ornamentale del valore di almeno 500 mo)\\
\textbf{Durata}: Finché a che dissolto\\
Il tuo corpo entra in uno stato catatonico mentre la tua anima lo abbandona ed entra nel contenitore da te usato come componente materiale. Mentre la tua anima occupa il contenitore, sei consapevole dei tuoi dintorni come se fossi nello spazio del contenitore. Non puoi muoverti né usare reazioni. L’unica azione che puoi effettuare è quella di proiettare la tua anima fino a 30 metri di distanza, fuori dal contenitore, ritornando al tuo corpo vivente (e terminando l’incantesimo) o cercando di possedere un corpo umanoide.\\
Puoi tentare di possedere qualsiasi umanoide entro 30 metri da te e che tu possa vedere (le creature protette dagli incantesimi protezione dal bene e dal male o cerchio magico non possono essere possedute). Il bersaglio deve effettuare un tiro salvezza su Arbitrio e, se lo fallisce, la tua anima entra nel corpo del bersaglio, mentre l’anima del bersaglio resta intrappolata nel contenitore. Se lo supera, il bersaglio resiste ai tuoi tentativi di possederlo, e non puoi tentare di possederlo nuovamente prima che siano trascorse 24 ore.\\
Una volta che possiedi il corpo di una creatura, lo puoi controllare. Le tue statistiche di gioco sono rimpiazzate dalle statistiche della creatura, a eccezione del tuo allineamento e dei tuoi punteggi di Intelletto, Saggezza e Magnetismo. Mantieni i benefici forniti dai tuoi privilegi di classe. Se il bersaglio possiede dei livelli di classe, non puoi usare nessuno dei suoi privilegi di classe.\\
Nel frattempo, l’anima della creatura posseduta può percepire i dintorni del contenitore usando i propri sensi, ma non può muoversi né effettuare alcuna azione.\\
Mentre possiedi un corpo, puoi usare la tua azione per ritornare dal corpo ospite al contenitore, se ti trovi entro 30 metri da esso, riportando l’anima della creatura ospite nel suo corpo. Se il corpo ospite muore mentre sei al suo interno, la creatura muore, e tu devi effettuare un tiro salvezza su Arbitrio contro la tua DC dei tiri salvezza degli incantesimi. Se lo superi, ritorni al contenitore, se si trova entro 30 metri da te. Altrimenti, morirai.\\
Se il contenitore viene distrutto o l’incantesimo termina, la tua anima ritorna immediatamente al tuo corpo. Se il tuo corpo è più di 30 metri lontano o se è morto mentre cerchi di farvi ritorno, morirà anche la tua anima. Se l’anima di un’altra creatura è nel contenitore quando viene distrutto, l’anima della creatura ritorna al suo corpo, se il corpo è vivo e si trova entro 30 metri. Altrimenti, la creatura muore. Quando l’incantesimo termina, il contenitore viene distrutto.

\medskip\textbf{Glifo di Interdizione}\index{Glifo di Interdizione}\\
\textbf{Difficolta'}: 18\\
\textbf{Tempo di Lancio}: 2 Azioni\\
\textbf{Gittata}: Contatto\\
\textbf{Componenti}: V. S, M (incenso e diamante in polvere del valore di almeno 200 mo, che l’incantesimo consuma)\\
\textbf{Durata}: Fino a che dissolto o attivato \\
Quando lanci questo incantesimo, inscrivi un glifo che danneggia altre creature su di una superficie (come un tavolo o una sezione di pavimento o muro) o all’internodi un oggetto che può essere chiuso (come un libro, una pergamena o un forziere) per celare il glifo. Se scegli una superficie, il glifo può coprire un’area di superficie non maggiore di 3 metri di diametro. Se scegli un oggetto, quell’oggetto deve restare al suo posto; se l’oggetto viene spostato più di 3 metri dal punto in cui è stato lanciato l’incantesimo, il glifo è spezzato, e l’incantesimo termina senza essere stato attivato.\\
Il glifo è quasi invisibile e può essere trovato con una prova di Intelletto (Indagare) contro la DC del tiro salvezza dei tuoi incantesimi. Decidi tu cosa attivi il glifo al momento del lancio dell’incantesimo.\\
Per i glifi inscritti su di una superficie, l’attivazione tipica comprende entrare in contatto o stare sopra il glifo, rimuovere un altro oggetto che copra il glifo, avvicinarsi a una certa distanza dal glifo, o manipolare l’oggetto su cui è inscritto il glifo. Per i glifi inscritti su di un oggetto, l’attivazione tipica comprende aprire l’oggetto, avvicinarsi a una certa distanza dall’oggetto, o vedere o leggere il glifo. Una volta che il glifo è stato attivato, l’incantesimo ha termine.\\
Puoi definire meglio l’attivazione così che l’incantesimo si attivi solo in determinate circostanze o secondo certe peculiarità fisiche (come l’altezza o il peso), specie di creatura (per esempio, l’interdizione potrebbe agire contro le aberrazioni o gli elfi oscuri), o l’allineamento. Puoi anchepredisporre condizioni per evitare che il glifo  venga attivato, come la pronuncia di una parola d’ordine.\\
Quando inscrivi il glifo scegli rune esplosive o glifo incantesimo.
\medskip
\begin{itemize}
\item
\textit{Glifo Incantesimo}. Puoi inserire un incantesimo preparato di difficolta' 18 o inferiore nel glifo lanciandolo come parte della creazione del glifo. L’incantesimo deve prendere come bersaglio una singola creatura o un’area. L’incantesimo che viene inserito non ha effetto immediato se lanciato in questo modo. Quando il glifo è attivato, l’incantesimo inserito viene lanciato. Se l’incantesimo ha un bersaglio, prende come bersaglio la creatura che ha attivato il glifo. Se l’incantesimo agisce su di un’area, l’area è incentrata su quella creatura. Se l’incantesimo evoca creature ostili o crea oggetti o trappole nocive, questi appaiono quanto più vicino possibile all’intruso e lo attaccano. Se l’incantesimo richiede concentrazione, questa è mantenuta fino al termine della sua normale durata.
\item
\textit{Rune Esplosive}. Quando attivato, il glifo erutta energia magica in una sfera di raggio 6 metri centrata sul glifo. La sfera si propaga intorno agli angoli. Ogni creatura nell’area deve effettuare un tiro salvezza su Agilità. Una creatura subisce 5d8 danni da acido, fulmine, fuoco, freddo o tuono se fallisce il tiro salvezza (a tua scelta quando crei il glifo), o la metà di questi danni se supera il tiro salvezza.
\end{itemize}
\medskip
\textbf{Per ogni Critico ottenuto} nella prova di il danno del glifo rune esplosive aumenta di 1d8 per ogni.

\medskip\textbf{Globo di Invulnerabilità}\index{Globo di Invulnerabilità}\\
\textbf{Difficolta'}: 25\\
\textbf{Tempo di Lancio}: 2 Azioni\\
\textbf{Gittata}: Personale (raggio di 3 metri)\\
\textbf{Componenti}: V. S, M (una pallina di vetro o di cristallo che si frantuma quando l’incantesimo termina) \\
\textbf{Durata}: Concentrazione, massimo 1 minuto\\
Una barriera immobile e lievemente scintillante si erge in un raggio di 3 metri intorno a te e vi rimane per la durata.\\
Qualsiasi incantesimo di Difficolta' 23 o più basso lanciato dall’esterno della barriera non può agire sulle creature o gli oggetti al suo interno, anche se l’incantesimo viene lanciato usando uno slot incantesimo di livello più alto. Questi incantesimi possono prendere come bersaglio creature e oggetti all’interno della barriera, ma non avranno effetto su di essi. Allo stesso modo, l’area all’interno della barriera viene esclusa dalle aree di effetto di questi incantesimi.\\
\textbf{Per ogni Critico ottenuto} nella prova di magia puoi bloccare un livello superiore di Difficolta'.

\medskip\textbf{Guardiani Spirituali}\index{Guardiani Spirituali}\\
\textbf{Difficolta'}: 18\\
\textbf{Tempo di Lancio}: 2 Azioni\\
\textbf{Gittata}: Personale (raggio di 4,5 metri)\\
\textbf{Componenti}: V, S, M (un simbolo sacro)\\
\textbf{Durata}: Concentrazione, massimo 10 minuti\\
Richiami degli spiriti che ti proteggano. Per la durata dell’incantesimo, essi fluttueranno intorno a te a una distanza di 4,5 metri. Se sei buono o neutrale, la forma spettrale sarà angelica o fatata (a tua scelta). Se sei malvagio, avranno un aspetto immondo. Quando lanci questo incantesimo, puoi designare un qualsiasi numero di creature che ne siano immuni. La velocità di una creatura soggetta viene dimezzata all’interno dell’area, e quando una creatura entra nell’area per la prima volta durante un turno o inizia il suo turno lì, deve effettuare un tiro salvezza su Saggezza. Se fallisce il tiro salvezza subisce 3d8 danni radianti (se sei buono o neutrale) o 3d8 danni necrotici (se sei malvagio), o la metà di questi danni se lo supera.\\
\textbf{Per ogni Critico ottenuto} nella prova di magia  il danno aumenta di 1d8 

\medskip\textbf{Guardiano della Fede}\index{Guardiano della Fede}\\
\textbf{Difficolta'}: 20\\
\textbf{Tempo di Lancio}: 2 Azioni\\
\textbf{Gittata}: 9 metri\\
\textbf{Componenti}: V\\
\textbf{Durata}: 8 ore\\
Un guardiano spettrale Grande compare per la durata e fluttua in uno spazio non occupato a gittata e che puoi vedere, scelto da te. Il guardiano occupa quello spazio ed è indistinguibile eccetto per una spada luminosa e uno scudo con il simbolo del tuo Patrono.\\
Qualsiasi creatura a te ostile che entri in uno spazio entro 3 metri dal guardiano per la prima volta durante un turno, deve effettuare un tiro salvezza su Agilità. La creatura subisce 20 danni radianti se fallisce il tiro salvezza, o la metà di questi danni se lo supera. Il guardiano svanisce dopo aver inflitto un totale di 60 danni.

\medskip\textbf{Guarigione}\index{Guarigione}\\
\textbf{Difficolta'}: 25\\
\textbf{Tempo di Lancio}: 2 Azioni\\
\textbf{Gittata}: 18 metri\\
\textbf{Componenti}: V, S\\
\textbf{Durata}: Istantanea\\
Scegli una creatura a gittata e che puoi vedere. Un’ondata di energia positiva travolge la creatura, facendole recuperare 70 punti ferita. L’incantesimo pone anche termine a qualsiasi cecità, sordità e malattia che affligga il bersaglio. Questo incantesimo non ha effetto su costrutti o non morti.\\
\textbf{Per ogni Critico ottenuto} nella prova di magia l’ammontare guarito aumenta di 10.

\medskip\textbf{Guarigione di Massa}\index{Guarigione di Massa}\\
\textbf{Difficolta'}: 33\\
\textbf{Tempo di Lancio}: 2 Azioni\\
\textbf{Gittata}: 18 metri\\
\textbf{Componenti}: V, S\\
\textbf{Durata}: Istantanea\\
Un effluvio di energia guaritrice scorre da te verso le creature ferite che ti circondano. Ripristini fino a 700 punti ferita, divisi come preferisci tra qualsiasi creatura a gittata e che puoi vedere. Le creature guarite da questo incantesimo sono curate anche di tutte le malattie e da qualsiasi effetto che le renda accecate o assordate. Questo incantesimo non ha effetto su costrutti o non morti. 

\medskip\textbf{Guida}\index{Guida}\\
\textbf{Difficolta'}: 11\
\textbf{Tempo di Lancio}: 2 Azioni\\
\textbf{Gittata}: Contatto\\
\textbf{Componenti}: V, S\\
\textbf{Durata}: Concentrazione, massimo 1 minuto\\
Lanci l’incantesimo a contatto di una creatura consenziente. Una volta, prima che l’incantesimo termini, il bersaglio può tirare un d4 e sommare il risultato tirato a una prova di caratteristica a sua scelta. Può tirare il dado prima o dopo aver effettuato la prova di caratteristica. L’incantesimo ha poi termine. 

\medskip\textbf{Guscio Anti-Vita}\index{Guscio Anti-Vita}\\
\textbf{Difficolta'}: 23\\
\textbf{Tempo di Lancio}: 2 Azioni\\
\textbf{Gittata}: Personale (raggio di 3 metri)\\
\textbf{Componenti}: V, S\\
\textbf{Durata}: Concentrazione, massimo 1 ora\\
Una barriera luminosa si estende fino a un raggio di 3 metri intorno a te, muovendosi con te e rimanendo centrata su di te, tenendo distanti le creature che non siano non morti o costrutti. La barriera permane per la durata. \\
La barriera impedisce a una creatura soggetta di attraversarla in alcun modo. Una creatura soggetta può lanciare incantesimi o effettuare attacchi con armi a distanza o con portata attraverso la barriera. Se ti muovi in modo che una creatura soggetta venga forzata ad attraversare la barriera, l’incantesimo termina.

\medskip\textbf{Identificare}\index{Identificare}\\
\textbf{Difficolta'}: 13\\
\textbf{Tempo di Lancio}: 1 minuto\\
\textbf{Gittata}: Contatto\\
\textbf{Componenti}: V, S, M (una perla del valore di almeno 100 mo e una piuma di gufo)\\ \textbf{Durata}: Istantanea\\
Scegli un oggetto con cui devi restare a contatto per tutto il lancio dell’incantesimo. Se si tratta di un oggetto magico o altro oggetto imbevuto di magia, ne apprendi le proprietà e come usarle, se richiede sintonia per l’uso, e quante cariche abbia, se ne ha. Apprendi se degli incantesimi stiano agendo sull’oggetto e cosa siano. Se l’oggetto è stato creato da un incantesimo, apprendi quale incantesimo lo abbia creato. Se invece durante l’esecuzione resti a contatto con una creatura, apprendi se degli incantesimi stiano agendo su di essa e quali siano.

\medskip\textbf{Illusione Minore}\index{Illusione Minore}\\
\textbf{Difficolta'}: 11\
\textbf{Tempo di Lancio}: 2 Azioni\\
\textbf{Gittata}: 9 metri\\
\textbf{Componenti}: S, M (un pezzo di vello)\\
\textbf{Durata}: 1 minuto\\
Crei l’immagine di un oggetto o un suono a gittata per la durata dell’incantesimo. L’illusione ha termine se la interrompi con un’azione o lanci di nuovo questo incantesimo.\\
Se crei un suono, il suo volume può variare da quello di un bisbiglio a un urlo. Può essere la tua voce, la voce di qualcun altro, il ruggito di un leone, un battito di tamburi, o qualsiasi altro suono tu scelga. Il suono continua incessante per tutta la durata, oppure puoi produrre suoni diversi in momenti diversi prima del termine dell’incantesimo.\\
Se crei l’immagine di un oggetto (come una sedia, un’impronta fangosa o un piccolo forziere) non può essere più grande di un cubo di 1 metro di spigolo. L’immagine non può produrre suoni, luci, odori o qualsiasi altro effetto sensoriale. L’interazione fisica con l’oggetto lo rivela come illusione, perché le cose lo possono attraversare.\\
Una creatura che usa la sua azione per esaminare il suono o l’immagine può determinare che si tratta di un’illusione con una prova riuscita di Intelletto (Indagare) contro la DC del tiro salvezza del tuo incantesimo. Se una creatura riconosce l’illusione per quello che è, per lei l’illusione sbiadisce. 

\medskip\textbf{Illusione Programmata}\index{Illusione Programmata}\\
\textbf{Difficolta'}: 25\\
\textbf{Tempo di Lancio}: 2 Azioni\\
\textbf{Gittata}: 36 metri\\
\textbf{Componenti}: V, S, M (un pezzo di vello e polvere di giada del valore di almeno 25 mo)\\
\textbf{Durata}: Fino a che dissolto\\
Crei, a gittata, l’illusione di un oggetto, creatura o qualche altro fenomeno visibile che si attiva quando viene soddisfatta una specifica condizione. Fino ad allora l’illusione è impercettibile. Non può essere più grande di un cubo di 9 metri di spigolo, e decidi tu quando lanci l’incantesimo, come si comporti l’illusione e che suoni produca. L’esibizione programmata può durare fino a 5 minuti. Quando occorrono le condizioni da te specificate, l’illusione si manifesta e si comporta nel modo da te descritto. Una volta che l’illusione ha terminato la sua esibizione, scompare e rimane dormiente per 10 minuti. Dopo questo periodo, l’illusione può essere attivata di nuovo.\\
La condizione di attivazione può essere generica o dettagliata quanto vuoi, sebbene debba essere basata su condizioni visibili o udibili che avvengano entro 9 metri dall’area. Per esempio, potresti creare un’illusione di te stesso che appare e avverta chi tenti di aprire una porta munita di trappola, oppure potresti predisporre l’illusione perché si attivi solo quando una creatura pronunci la parola o la frase giusta.\\
L’interazione fisica con l’immagine la rivela come illusione, dato che le cose le passano attraverso. Una creatura che usi la sua azione per esaminare l’immagine può determinare che è un’illusione con una prova riuscita di Intelletto (Indagare) contro la DC del tiro salvezza dell’incantesimo. Se una creatura riconosce l’illusione per quello che è, essa può vedere attraverso l’immagine, e qualsiasi suono prodotto dall’immagine le suona artefatto.

\medskip\textbf{Immagine Maggiore}\index{Immagine Maggiore}\\
\textbf{Difficolta'}: 18\\
\textbf{Tempo di Lancio}: 2 Azioni\\
\textbf{Gittata}: 36 metri\\
\textbf{Componenti}: V, S, M (un pezzo di vello)\\
\textbf{Durata}: Concentrazione, massimo 10 minuti\\
Crei l’immagine di un oggetto, una creatura o qualche altro fenomeno visibile non più grande di un cubo di 6 metri di spigolo. L’immagine appare in un punto a gittata che puoi vedere e vi rimane per la durata dell’incantesimo. L’immagine sembra completamente reale, e comprende suoni, odori e la temperatura appropriata alla cosa raffigurata. Non puoi generare calore o freddo sufficiente a provocare danni, né un suono abbastanza forte da infliggere danno da tuono o assordare una creatura, o un odore che possa far star male una creatura (come il fetore di un troglodita). Finché resti a gittata dell’illusione, puoi usare un’azione per far muovere l’immagine in qualsiasi altro punto a 
gittata. Quando l’immagine cambia posizione, puoi alterarne l’aspetto così che i suoi movimenti appaiano naturali. Per esempio, se crei l’immagine di una creatura e la muovi, puoi alterare l’immagine in modo che sembri camminare. Allo stesso modo, puoi impiegare l’illusione per produrre suoni diversi in momenti diversi, fino a farle portare avanti una conversazione.\\
L’interazione fisica con l’immagine la rivela come illusione, dato che le cose vi passano attraverso. Una creatura che usa la sua azione per esaminare l’immagine può determinare che si tratta di un’illusione con una prova riuscita di Intelletto (Indagare) contro la DC del tiro salvezza del tuo incantesimo. Se una creatura riconosce l’illusione per quello che è, la creatura può vedervi attraverso, e per quella creatura tutte le altre qualità sensoriali svaniscono.\\
\textbf{Se ottieni un critico} l’incantesimo dura finché non viene dissolto, senza richiedere la tua concentrazione.
	
\medskip\textbf{Immagine Proiettata}\index{Immagine Proiettata}\\
\textbf{Difficolta'}: 28\\
\textbf{Tempo di Lancio}: 2 Azioni\\
\textbf{Gittata}: 750 chilometri\\
\textbf{Componenti}: V, S, M (una tua piccola riproduzione fatta di materiali del valore almeno di 5 mo)\\
\textbf{Durata}: Concentrazione, massimo 1 giorno\\
Crei una copia illusoria di te stesso che permane per la durata. La copia può apparire in qualsiasi luogo entro la gittata che tu abbia già visto, ignorando qualsiasi ostacolo frapposto. L’illusione riproduce il tuo aspetto e i tuoi rumori ma è intangibile. Se l’illusione subisce danni, scompare, e l’incantesimo ha termine.\\
Puoi usare la tua azione per far muovere questa illusione fino al doppio della tua velocità e farle compiere un gesto, parlare e comportarsi in qualsiasi maniera tu voglia. Imita alla perfezione i tuoi comportamenti.\\
Puoi vedere attraverso i suoi occhi e udire tramite le sue orecchie come se fossi nello spazio in cui essa si trova. Durante ciascun tuo turno, con un’azione bonus, puoi passare dall’usare i suoi sensi all’usare i tuoi, o viceversa. Mentre stai usando i suoi sensi, sei accecato e assordato riguardo i tuoi dintorni.\\
L’interazione fisica con l’immagine la rivela come illusione, dato che le cose le passano attraverso. Una creatura che usi la sua azione per esaminare l’immagine può determinare che è un’illusione con una prova riuscita di Consapevolezza contro la DC del tiro salvezza dell’incantesimo. Se una creatura riconosce l’illusione per quello che è, essa può vedere attraverso l’immagine, e qualsiasi suono prodotto dall’immagine le suona artefatto.

\medskip\textbf{Immagine Silenziosa}\index{Immagine Silenziosa}\\
\textbf{Difficolta'}: 13\\
\textbf{Tempo di Lancio}: 2 Azioni\\
\textbf{Gittata}: 36 metri\\
\textbf{Componenti}: V, S, M (un pezzo di vello)\\
\textbf{Durata}: Concentrazione, massimo 10 minuti\\
Crei l’immagine di un oggetto, una creatura o qualche altro fenomeno visibile non più grande di un cubo di 4,5 metri di spigolo. L’immagine appare in un punto a gittata che puoi vedere e resta per la durata dell’incantesimo. L’immagine è puramente visiva; non è accompagnata da suoni, odori o altri effetti sensoriali. Puoi usare un’azione per far muovere l’immagine in qualsiasi altro punto a gittata. Quando l’immagine cambia posizione, puoi alterarne l’aspetto così che i suoi movimenti appaiano naturali. Per esempio, se crei l’immagine di una creatura e la muovi, puoi alterare l’immagine in modo che sembri camminare.\\
L’interazione fisica con l’immagine la rivela come illusione, dato che le cose vi passano attraverso. Una creatura che usa la sua azione per esaminare l’immagine può determinare che si tratta di un’illusione con una prova riuscita di Consapevolezza contro la DC del tiro salvezza del tuo incantesimo. Se una creatura riconosce l’illusione per quello che è, la creatura può vedervi attraverso.

\medskip\textbf{Immagine Speculare}\index{Immagine Speculare}\\
\textbf{Difficolta'}: 15\\
\textbf{Tempo di Lancio}: 2 Azioni\\
\textbf{Gittata}: Personale\\
\textbf{Componenti}: V, S\\
\textbf{Durata}: 1 minuto\\
Nel tuo spazio compaiono tre duplicati illusori di te stesso. Fino al termine dell’incantesimo, i duplicati si muovono con te e imitano le tue azioni, scambiandosi di posto in modo da rendere impossibile determinare quale sia l’immagine reale. Puoi usare la tua azione per congedare i duplicati illusori.\\
Ogni volta che una creatura ti prenda come bersaglio di un attacco nella durata dell’incantesimo, tira un d20 per determinare se l’attacco colpisca invece uno dei tuoi duplicati.\\
Se hai tre duplicati, devi tirare 6 o più per spostare l’attacco della creatura contro un duplicato. Con due duplicati, devi tirare 8 o più. Con un duplicato, devi tirare 11 o più.\\
La Difesa del duplicato è pari a 10 + il tuo modificatore di Agilita'. Se un attacco colpisce un duplicato, questo è distrutto. Un duplicato può essere distrutto solo da un attacco che lo colpisce. Ignora invece tutti gli altri danni o effetti. L’incantesimo ha termine quando tutti e tre i duplicati sono stati distrutti.\\
Una creatura che non può vedere, o si affida a sensi diversi dalla vista (come la vista cieca), o che può  distinguere le illusioni come false (come la visione del vero), ignora gli effetti di questo incantesimo. 

\medskip\textbf{Imprigionare}\index{Imprigionare}\\
\textbf{Difficolta'}: 33\\
\textbf{Tempo di Lancio}: 2 Azioni\\
\textbf{Gittata}: 9 metri\\
\textbf{Componenti}: V, S, M (una raffigurazione su vello o una statuetta incisa con le fattezze del bersaglio, e una componente speciale che varia a seconda della versione che scegli dell’incantesimo, del valore di almeno 500 mo per Dado Ferita del bersaglio)\\
\textbf{Durata}: Fino a dissolvimento\\
Crei dei vincoli magici per bloccare una creatura a gittata e che puoi vedere. Il bersaglio deve superare un tiro salvezza su Arbitrio o essere avvinto dall’incantesimo; se lo supera, è immune all’incantesimo qualora lo lanci di nuovo. Mentre è soggetta a questo incantesimo, la creatura non ha bisogno di respirare, mangiare o bere e non invecchia. Gli incantesimi di divinazione non possono localizzare né percepire il bersaglio.\\
Quando lanci questo incantesimo, scegli una delle seguenti forme di prigionia.
\medskip
\begin{itemize}
\item
\textit{Incatenamento}. Catene pesanti, ben saldate al terreno, tengono il bersaglio ancorato. Il bersaglio è intralciato fino al termine dell’incantesimo, e non può muoversi né essere mosso in alcun modo fino ad allora. La componente speciale per questa versione dell’incantesimo è una catenella di metallo prezioso. 
\item
\textit{Isolamento Minimo}. Il bersaglio rimpicciolisce fino a 2,5 centimetri di altezza ed è imprigionato in una gemma o simile oggetto. La luce può attraversare normalmente la gemma (permettendo al bersaglio di vedere all’esterno e ad altre creature di vedere dentro), ma null’altro può attraversarla, neppure tramite teletrasporto o viaggio planare. La gemma non può essere tagliata né infranta finché l’incantesimo rimane in atto. La componente speciale per questa versione dell’incantesimo è una grande gemma trasparente, come il corindone, il diamante o il rubino.
\item
\textit{Prigione Confinata}. L’incantesimo trasporta il bersaglio in un minuscolo semipiano interdetto al teletrasporto e al viaggio planare. Il semipiano può essere un labirinto, una gabbia, una torre, o qualsiasi altra struttura chiusa scelta da te. La componente speciale per questa versione dell’incantesimo è una rappresentazione in miniatura della prigione fatta di giada.
\item
\textit{Sepoltura}. Il bersaglio viene sepolto nelle profondità della terra in una sfera di forza magica grande a sufficienza da contenere il bersaglio. Nulla può attraversare la sfera, né alcuna creatura può teletrasportarsi o usare il viaggio planare per entrarvi o uscire. La componente speciale per questa versione dell’incantesimo è una piccola sfera di mithril. 
\item
\textit{Sopore}. Il bersaglio cade addormentato e non può essere risvegliato. La componente speciale per questa versione dell’incantesimo consiste di rare erbe soporifere.
\end{itemize}
\medskip
\textit{Terminare l’Incantesimo}. Durante il lancio dell’incantesimo, in qualsiasi delle sue versioni, puoi specificare una condizione che possa porre fine all’incantesimo e liberare il bersaglio. La condizione può essere tanto specifica o elaborata quanto desideri, ma il Narratore deve concordare che la condizione sia ragionevole e possa avverarsi. Le condizioni possono essere basate sul nome, l’identità o la divinità di una creatura, ma comunque basate su azioni o qualità percepibili e non su cose intangibili come il livello, la classe o i punti ferita.\\
Un incantesimo dissolvi magie può porre fine all’incantesimo solo se lanciato come incantesimo di 9° livello, che prenda come bersaglio la prigione o la componente materiale usata per crearla.\\
Puoi usare una particolare componente speciale per creare solo una prigione alla volta. Se lanci l’incantesimo di nuovo usando la stessa componente, ilbersaglio del primo lancio dell’incantesimo  viene immediatamente liberato dal suo vincolo.

\medskip\textbf{Inaridire}\index{Inaridire}\\
\textbf{Difficolta'}: 20\\
\textbf{Tempo di Lancio}: 2 Azioni\\
\textbf{Gittata}: 9 metri\\
\textbf{Componenti}: V, S\\
\textbf{Durata}: Istantanea\\
Energia necromantica avvolge una creatura di tua scelta a gittata e che puoi vedere, deprivandola di linfa e vitalità. Il bersaglio deve effettuare un tiro salvezza su Tempra. Se fallisce il tiro  salvezza, il bersaglio subisce 8d8 danni necrotici, o la metà di questi danni se supera il tiro salvezza. L’incantesimo non ha effetto su non morti o costrutti.\\
Se il bersaglio è un vegetale non magico che non sia anche una creatura, come un albero o un cespuglio, non effettua alcun tiro salvezza, avvizzisce e muore all’istante.\\
\textbf{Per ogni Critico ottenuto} nella prova di magia il danno aumenta di 1d8.

\medskip\textbf{Individuazione del Bene e del Male}\index{Individuazione del Bene e del Male}\\
\textbf{Difficolta'}: 13\\
\textbf{Tempo di Lancio}: 2 Azioni\\
\textbf{Gittata}: Personale\\
\textbf{Componenti}: V, S\\
\textbf{Durata}: Concentrazione, massimo 10 minuti\\
Per la durata, apprendi se entro 9 metri da te si trova un’aberrazione, celestiale, elementale, fatato, demone o non morto, e la sua posizione. Allo stesso modo, apprendi se entro 9 metri da te si trovi un luogo o oggetto che sia stato consacrato o dissacrato magicamente.\\
L’incantesimo può penetrare la maggior parte delle barriere, ma è bloccato da 30 centimetri di pietra, 2,5 centimetri di metallo comune, un sottile foglio di piombo o 1 metro di legno o terra. 
\textbf{Nota}: questo incantesimo non ha effetto sulle creature che seguono i Tratti.

\medskip\textbf{Individuazione del Magico}\index{Individuazione del Magico}\\
\textbf{Difficolta'}: 13\\
\textbf{Tempo di Lancio}: 2 Azioni\\
\textbf{Gittata}: Personale\\
\textbf{Componenti}: V, S\\
\textbf{Durata}: Concentrazione, massimo 10 minuti\\
Per la durata, percepisci la presenza della magia entro 9 metri da te. Se percepisci la magia in questo modo, puoi usare la tua azione per vedere una flebile aura che  si estende intorno a qualsiasi creatura o oggetto visibile nell’area che rechi magia, e ne apprendi anche la scuola di magia, se ce l’ha.\\
L’incantesimo può penetrare la maggior parte delle barriere, ma è bloccato da 30 centimetri di pietra, 2,5 centimetri di metallo comune, un sottile foglio di piombo o 1 metro di legno o terra.

\medskip\textbf{Individuazione delle Malattie e dei Veleni}\index{Individuazione delle Malattie e dei Veleni}\\
\textbf{Difficolta'}: 13\
\textbf{Tempo di Lancio}: 2 Azioni\\
\textbf{Gittata}: Personale\\
\textbf{Componenti}: V, S, M (una foglia di tasso)\\
\textbf{Durata}: Concentrazione, massimo 10 minuti\\
Per la durata, percepisci la presenza e posizione di veleni, creature velenose e malattie entro 9 metri da te. Inoltre riesci a identificare il tipo di veleno, creatura velenosa o malattia. L’incantesimo può penetrare la maggior parte delle barriere, ma è bloccato da 30 centimetri di pietra, 2,5 centimetri di metallo comune, un sottile foglio di piombo o 1 metro di legno o terra.

\medskip\textbf{Individuazione dei Pensieri}\index{Individuazione dei Pensieri}\\
\textbf{Difficolta'}: 15\\
\textbf{Tempo di Lancio}: 2 Azioni\\
\textbf{Gittata}: Personale\\
\textbf{Componenti}: V, S, M (un pezzo di rame)\\
\textbf{Durata}: Concentrazione, massimo 1 minuto\\
Per la durata, puoi leggere i pensieri di certe creature. Quando lanci questo incantesimo e con altre due Azioni in ciascun round successivo sino al termine dell’incantesimo, puoi concentrare la tua mente su qualsiasi creatura che tu possa vedere e si trovi entro 9 metri da te. Se la creatura che hai scelto ha un punteggio di Intelletto -3 o meno o non parla nessun linguaggio, la creatura ignora l’effetto.\\
Inizialmente, apprendi solo i pensieri di superficie della creatura: quelli più ricorrenti. Con un’azione, puoi o spostare la tua attenzione sui pensieri di un’altra creatura o tentare di sondare più a fondo la mente della stessa creatura. Se sondi più a fondo, il bersaglio deve effettuare un tiro salvezza su Arbitrio. Se lo fallisce, ottieni una percezione dei suoi ragionamenti (se ve ne sono), del suo stato emotivo, e di ogni cosa abbia prevalenza nei suoi pensieri (come una preoccupazione, l’amore, o l’odio). Se supera il tiro salvezza, l’incantesimo termina. A ogni modo, il bersaglio sa che stai sondando la sua mente e, a meno che non sposti la tua attenzione verso la mente di un’altra creatura, nel suo turno la creatura può usare la propria azione per effettuare una prova di Intelletto contesa dalla tua prova di Intelletto; se la vince, l’incantesimo termina.\\
Le domande poste verbalmente alla creatura bersaglio, ovviamente, modellano il corso dei suoi pensieri, cosicché questo incantesimo risulta particolarmente efficace negli interrogatori.\\
Puoi anche usare questo incantesimo per individuare la presenza di creature pensanti che non puoi vedere. Quando lanci questo incantesimo o con 2 Azioni nella sua durata, puoi cercare pensieri entro 9 metri da te. L’incantesimo può penetrare le barriere, ma è bloccato da 60 centimetri di pietra, 5 centimetri di metallo che non sia il piombo, o un sottile foglio di piombo. Non puoi individuare una creatura con Intelletto 3 o meno, o una creatura che non parla alcun linguaggio. Una volta individuata in questo modo la presenza di una creatura, puoi leggerne i pensieri per la durata dell’incantesimo finché resta nella gittata, come descritto sopra, anche se non puoi vederla.

\medskip\textbf{Infliggi Ferite}\index{Infliggi Ferite}\\
\textbf{Difficolta'}: 13 \\
\textbf{Tempo di Lancio}: 2 Azioni\\
\textbf{Gittata}: Contatto\\
\textbf{Componenti}: V, S\\
\textbf{Durata}: Istantanea\\
Effettua un attacco in mischia con incantesimo contro una creatura a portata. Se colpisci, il bersaglio subisce 3d10 danni necrotici.
\textbf{Per ogni Critico ottenuto} nella prova di magia il danno aumenta di 1d8.

\medskip\textbf{Ingrandire/Ridurre}\index{Ingrandire/Ridurre}\\
\textbf{Difficolta'}: 15\\
\textbf{Tempo di Lancio}: 2 Azioni\\
\textbf{Gittata}: 9 metri\\
\textbf{Componenti}: V, S, M (un pizzico di ferro in polvere)\\
\textbf{Durata}: Concentrazione, massimo 1 minuto\\
Fai sì che una creatura od oggetto a gittata e che puoi vedere ingrandisca o rimpicciolisca per la durata dell’incantesimo. Scegli una creatura o un oggetto che non sia né indossato né trasportato. Se il bersaglio non è consenziente, può effettuare un tiro salvezza su Costituzione. Se lo supera, l’incantesimo non ha effetto. Se il bersaglio è una creatura, tutto ciò che sta indossando e trasportando cambia taglia assieme a essa. Qualsiasi oggetto lasciato cadere da una creatura soggetta a questo incantesimo ritorna subito alla sua taglia normale.\\
\medskip
\begin{itemize}
\item
\textit{Ingrandire}. La taglia del bersaglio raddoppia in tutte le dimensioni, e il suo peso è moltiplicato per otto. Questa crescita aumenta la sua taglia di una categoria: da Media a Grande, per esempio. Senon c’è spazio  sufficiente perché il bersaglio raddoppi la sua taglia, la creatura od oggetto assume la taglia più grossa possibile permessagli dallo spazio disponibile. Fino al termine dell’incantesimo, il bersaglio ha vantaggio alle prove di Potenza e ai tiri salvezza su Potenza. Le armi del bersaglio crescono per raggiungere la nuova taglia. Mentre queste armi sono ingrandite, gli attacchi del bersaglio con esse infliggeranno 1d4 danni aggiuntivi. 
\item
\textit{Ridurre}. La taglia del bersaglio si dimezza in tutte le dimensioni, e il suo peso è ridotto a un ottavo. Questa crescita diminuisce la sua taglia di una categoria: da Media a Piccola, per esempio. Fino al termine dell’incantesimo, il bersaglio ha -1d6 alle prove di Forza e ai tiri salvezza su Potenza. Le armi del bersaglio rimpiccioliscono per raggiungere la nuova taglia. Mentre queste armi sono rimpicciolite, gli attacchi del bersaglio con esse infliggeranno 1d4 danni di meno (ma senza ridurre il danno dell’arma a meno di 1).
\end{itemize}

\medskip\textbf{Insetto Gigante}\index{Insetto Gigante}\\
\textbf{Difficolta'}: 20\\
\textbf{Tempo di Lancio}: 2 Azioni\\
\textbf{Gittata}: 9 metri\\
\textbf{Componenti}: V, S\\
\textbf{Durata}: Concentrazione, massimo 10 minuti\\
Per la durata dell’incantesimo, trasformi fino a dieci centopiedi, tre ragni, cinque vespe o uno scorpione a gittata, in versioni giganti della loro forma naturale. Un centopiedi diventa un centopiedi gigante, un ragno diventa un ragno gigante, una vespa diventa una vespa gigante e uno scorpione diventa uno scorpione gigante. Ogni creatura obbedisce ai tuoi comandi vocali e, in combattimento, agisce in ciascun round durante il tuo turno. Il Narratore possiede le statistiche di queste creature, e sarà sempre Il Narratore a risolvere le loro azioni e i loro movimenti. Una creatura resta nella sua forma gigante per la durata, finché non scende a 0 punti ferita, o finché non usi un’azione per interrompere l’effetto su di essa.\\
Il Narratore può permetterti di scegliere bersagli differenti. Per esempio, se trasformi un’ape, la sua versione gigante potrebbe avere le stesse statistiche della vespa gigante.

\medskip\textbf{Interdizione alla Morte}\index{Interdizione alla Morte}\\
\textbf{Difficolta'}: 20\\
\textbf{Tempo di Lancio}: 2 Azioni\\
\textbf{Gittata}: Contatto\\
\textbf{Componenti}: V, S\\
\textbf{Durata}: 8 ore\\
Lanci l’incantesimo a contatto con una creatura. Conferisci al bersaglio protezione dalla morte. La prima volta che il bersaglio dovesse scendere a 0 punti ferita in seguito al danno subito, il bersaglio scende invece a 1 punto ferita e l’incantesimo ha fine. Se l’incantesimo è ancora attivo quando il bersaglio è vittima di un effetto che lo ucciderebbe all’istante senza infliggere danni,quell’effetto viene invece negato sul bersaglio e l’incantesimo ha fine.

\medskip\textbf{Intermittenza}\index{Intermittenza}\\
\textbf{Difficolta'}: 18\\
\textbf{Tempo di Lancio}: 2 Azioni\\
\textbf{Gittata}: Personale\\
\textbf{Componenti}: V, S\\
\textbf{Durata}: 1 minuto\\
Tira un d20 alla fine di ciascun tuo turno per la durata di questo incantesimo. Se ottieni 11 o più, svanisci dal tuo attuale piano di esistenza e riappari sul Piano Etereo (l’incantesimo fallisce e il lancio è sprecato qualora tu fossi già su quel piano). All’inizio del tuo prossimo turno, e quando l’incantesimo termina, qualora tu fossi sul Piano Etereo, ritorni in uno spazio non occupato di tua scelta e che puoi vedere, entro 3 metri dallo spazio da cui sei svanito. Se nessuno spazio non occupato è disponibile entro questa gittata, compari nello spazio non occupato più vicino (determinato casualmente se è disponibile più di uno spazio). Puoi interrompere l’incantesimo con un’azione.\\
Mentre sei sul Piano Etereo, puoi vedere e udire il piano da cui provieni, che percepisci in sfumature di grigio, ma non puoi comunque percepire nulla che si trovi a più di 18 metri di distanza. Puoi interagire solo con creature che si trovano sul Piano Etereo. Le creature che non si trovano lì non possono né percepirti né interagire con te, a meno che non siano provviste della capacità di farlo.

\medskip\textbf{Intimorire Infernale}\index{Intimorire Infernale}\\
\textbf{Difficolta'}: 13\\
\textbf{Tempo di Lancio}: 1 reazione, che puoi effettuare in risposta al danno arrecatoti da una creatura entro 18 metri da te che puoi vedere\\
\textbf{Gittata}: 18 metri\\
\textbf{Componenti}: V, S\\
\textbf{Durata}: Istantanea\\
Punti il dito, e la creatura che ti ha danneggiato viene momentaneamente avvolta da fiamme diaboliche. La creatura deve effettuare un tiro salvezza su Agilità.Subisce 2d10 danni da fuoco se  fallisce il tiro salvezza, o la metà di questi danni se lo supera.\\
\textbf{Per ogni Critico ottenuto} nella prova di magia i danno aumenta di 1d8

\medskip\textbf{Intralciare}\index{Intralciare}\\
\textbf{Difficolta'}: 13\\
\textbf{Tempo di Lancio}: 2 Azioni\\
\textbf{Gittata}: 27 metri\\
\textbf{Componenti}: V, S\\
\textbf{Durata}: Concentrazione, massimo 1 minuto\\
Rampicanti e rami stritolanti spuntano dal terreno in un quadrato di 6 metri di lato a partire da un punto a gittata. Per la durata, questi vegetali trasformano il terreno nell’area in terreno difficile.\\
Una creatura nell’area nel momento in cui lanci questo incantesimo deve superare un tiro salvezza su Potenza o restare intralciata da questi vegetali fino al termine dell’incantesimo. Una creatura intralciata dai vegetali può usare le sue azioni per effettuare una prova di Forza contro la DC del tiro salvezza dell’incantesimo. Se la supera, si libera. Quando l’incantesimo ha termine, i vegetali evocati svaniscono.

\medskip\textbf{Inversione della Gravità}\index{Inversione della Gravità}\\
\textbf{Difficolta'}: 28\\
\textbf{Tempo di Lancio}: 2 Azioni\\
\textbf{Gittata}: 30 metri\\
\textbf{Componenti}: V, S, M (una calamita e un fil di ferro)\\
\textbf{Durata}: Concentrazione, massimo 1 minuto 
Questo incantesimo inverte la gravità in un cilindro di raggio 15 metri, alto 30 metri, centrato in un punto a gittata. Quando lanci questo incantesimo, tutte le creature e gli oggetti che non sono in qualche modo ancorati al terreno cadono verso l’alto e raggiungono la cima dell’area. Una creatura può tentare un tiro salvezza su Agilita' per afferrare un oggetto fisso a portata, per evitare di cadere in questo modo, in caso lo superi.\\
Se lungo questa caduta si incontra un oggetto solido (il soffitto), gli oggetti e le creature che cadono vi impattano come accadrebbe durante una normale caduta. Se un oggetto o creatura raggiunge la cima dell’area senza colpire nulla, rimane lì, oscillando lievemente, per la durata.\\
Al termine della durata, gli oggetti e le creature colpite ricadono verso il basso.

\medskip\textbf{Inviare}\index{Inviare}\\
\textbf{Difficolta'}: 18\\
\textbf{Tempo di Lancio}: 2 Azioni\\
\textbf{Gittata}: Illimitata\\
\textbf{Componenti}: V, S, M (un piccolo pezzo di cavo di rame)\\
\textbf{Durata}: 1 round\\
Invii un breve messaggio di 25 parole o meno a una creatura con cui sei familiare. La creatura sente il messaggio nella sua mente, ti riconosce come mittente, e può risponderti in modo simile. L’incantesimo permette a creature con un punteggio di Intelletto almeno di 1 di comprendere il significato del tuo messaggio.\\
Puoi inviare il messaggio attraverso qualsiasi distanza e anche su altri piani di esistenza, ma se il bersaglio è su di un piano diverso dal tuo, c’è una probabilità del 5\% che il messaggio non arrivi.

\medskip\textbf{Invisibilità}\index{Invisibilità}\\
\textbf{Difficolta'}: 15\\
\textbf{Tempo di Lancio}: 2 Azioni\\
\textbf{Gittata}: Contatto\\
\textbf{Componenti}: V, S, M (un ciglio avvolto nella gomma arabica)\\
\textbf{Durata}: Concentrazione, massimo 1 ora \\
Lanci l’incantesimo a contatto di una creatura. Il bersaglio diventa invisibile fino alla fine dell’incantesimo. Qualsiasi cosa il bersaglio stia indossando o trasportando diventa invisibile finché resta sul bersaglio. L’incantesimo ha fine per il bersaglio che attacca o esegue un incantesimo.\\
\textbf{Per ogni Critico ottenuto} nella prova di magia puoi scegliere un’ulteriore creatura bersaglio.

\medskip\textbf{Invisibilità Superiore}\index{Invisibilità Superiore}
\textbf{Difficolta'}: 20\\
\textbf{Tempo di Lancio}: 2 Azioni\\
\textbf{Gittata}: Contatto\\
\textbf{Componenti}: V, S\\
\textbf{Durata}: Concentrazione, massimo 1 minuto\\
Lanci l’incantesimo a contatto di una creatura. Il bersaglio diventa invisibile fino alla fine dell’incantesimo. Qualsiasi cosa indossata o trasportata dal bersaglio diventa invisibile finché resta addosso al bersaglio.

\medskip\textbf{Invocare il Fulmine}\index{Invocare il Fulmine}\\
\textbf{Difficolta'}: 18\\
\textbf{Tempo di Lancio}: 2 Azioni\\
\textbf{Gittata}: 36 metri\\
\textbf{Componenti}: V, S\\
\textbf{Durata}: Concentrazione, massimo 10 minuti\\
Una nube di tempesta compare nella forma di un cilindro alto 3 metri con un raggio di 18 metri, centrato su di un punto che puoi vedere, 30 metri sopra di te. L’incantesimo fallisce automaticamente se non puoi vedere il punto nell’aria dove apparirà la nube di tempesta (per esempio, se sei in una stanza che non può accogliere la nube). Quando lanci l’incantesimo, scegli un punto che puoi vedere entro la gittata. Un fulmine si abbatterà dalla nuvola su quel punto. Ogni creatura entro 1 metro da  quel punto deve effettuare un tiro salvezza su Agilità. Una creatura  subisce 3d10 danni da fulmine se fallisce il tiro salvezza, o la metà di questi danni se lo supera. Durante ciascun tuo round fino al termine dell’incantesimo, puoi usare due Azioni per richiamare un altro fulmine in questo modo, prendendo come bersaglio lo stesso punto o uno diverso.\\
Se quando lanci questo incantesimo ti trovi all’esterno in condizioni di tempesta, l’incantesimo ti fornisce il controllo della tempesta esistente invece di crearne una nuova. Sotto queste condizioni, il danno dell’incantesimo aumenta di 1d10. 
\textbf{Per ogni Critico ottenuto} nella prova di magia il danno aumenta di 1d8

\medskip\textbf{Labirinto}\index{Labirinto}\\
\textbf{Difficolta'}: 30\\
\textbf{Tempo di Lancio}: 2 Azioni\\
\textbf{Gittata}: 18 metri\\
\textbf{Componenti}: V, S\\
\textbf{Durata}: Concentrazione, massimo 10 minuti\\
Bandisci una creatura a gittata e che puoi vedere in un semipiano labirintico. Il bersaglio rimane lì per la durata dell’incantesimo o finché non fugge dal labirinto. Il bersaglio può impiegare la sua azione per tentare di fuggire. Quando lo fa, effettua una prova di Intelletto DC 25. Se la supera, fugge, e l’incantesimo termina (un minotauro o un demone goristro riescono automaticamente).\\
Quando l’incantesimo termina, il bersaglio riappare nello spazio che aveva lasciato o, se quello spazio è occupato, nel più vicino spazio non occupato. 

\medskip\textbf{Lama Infuocata}\index{Lama Infuocata}\\
\textbf{Difficolta'}: 15\\
\textbf{Tempo di Lancio}: 1 Azione Immediata\\
\textbf{Gittata}: Personale\\
\textbf{Componenti}: V, S, M (una foglia di sommacco)\\
\textbf{Durata}: Concentrazione, massimo 10 minuti \\
Crei nella tua mano una lama infuocata. La lama è simile in dimensioni e forma a una scimitarra, e rimane per la durata. Se lasci andare la lama, questa sparisce, ma ne puoi creare un’altra con un’azione bonus. Puoi usare la tua azione per effettuare un attacco in mischia con incantesimo usando la lama infuocata. Se colpisci, il bersaglio subisce 3d6 danni da fuoco. La lama infuocata emana luce intensa in un raggio di 3 metri e luce fioca per ulteriori 3 metri.\\
\textbf{Per ogni due Critici ottenuti} nella prova di magia il danno aumenta di 1d6.

\medskip\textbf{Legame Planare}\index{Legame Planare}\\
\textbf{Difficolta'}: 23\\
\textbf{Tempo di Lancio}: 1 ora\\
\textbf{Gittata}: 18 metri\\
\textbf{Componenti}: V, S, M (un gioiello del valore di almeno 1.000 mo, che l’incantesimo consuma)\\
\textbf{Durata}: 24 ore\\
Con questo incantesimo, cerchi di vincolare un celestiale, elementale, fatato o immondo al tuo servizio. La creatura deve restare nella gittata per l’intero lancio dell’incantesimo. (Di solito, la creatura viene prima evocata al centro di un cerchio magico invertito per tenerla intrappolata mentre questo incantesimo viene lanciato). Al completamento del lancio, il bersaglio deve effettuare un tiro salvezza su Arbitrio. Se fallisce il tiro salvezza, è vincolato al tuo servizio per la durata. Se la creatura è stata evocata o creata da un altro incantesimo, la durata di quell’incantesimo viene estesa per corrispondere alla durata di questo incantesimo. Una creatura vincolata deve eseguire le tue istruzioni al meglio delle sue capacità. Potresti comandare la creatura di accompagnarti nel corso di un’avventura, di proteggere un luogo o di consegnare un messaggio. La creatura obbedisce le tue istruzioni alla lettera, ma se ti è ostile, cercherà di distorcere le tue parole ai suoi fini. Se la creatura adempie completamente alle tue istruzioni prima del termine dell’incantesimo, qualora vi troviate sullo stesso piano di esistenza ritornerà da te per comunicarti l’avvenuto. Se vi trovate su piani di esistenza diversi, ritornerà nel luogo dove l’hai vincolata e rimarrà lì fino al termine dell’incantesimo.\\
\textbf{Per ogni Critico ottenuto} nella prova di magia raddoppi la permanenza della creatura

\medskip\textbf{Legame Telepatico}\index{Legame Telepatico}\\
\textbf{Difficolta'}: 23\\
\textbf{Tempo di Lancio}: 2 Azioni\\
\textbf{Gittata}: 9 metri\\
\textbf{Componenti}: V, S, M (pezzi di gusci d’uovo da due differenti specie di creature)\\
\textbf{Durata}: 1 ora\\
Stabilisci un collegamento telepatico tra un massimo di otto creature consenzienti a gittata di tua scelta, collegando psichicamente ciascuna creatura alle altre per la durata dell’incantesimo. Le creature con punteggio di Intelletto 2 o meno ignorano questo incantesimo. Fino al termine dell’incantesimo, i bersagli possono comunicare telepaticamente tramite questo legame, che condividano o meno un linguaggio comune. La comunicazione è possibile a qualsiasi distanza, ma non può estendersi su differenti piani di esistenza.

\medskip\textbf{Lentezza}\index{Lentezza}\\
\textbf{Difficolta'}: 18\\
\textbf{Tempo di Lancio}: 2 Azioni\\
\textbf{Gittata}: 36 metri\\
\textbf{Componenti}: V, S, M (una goccia di melassa) \\
\textbf{Durata}: Concentrazione, massimo 1 minuto\\
Modifichi lo scorrere del tempo intorno a un massimo di sei creature di tua scelta in un cubo di 12 metri di spigolo a gittata. Ciascun bersaglio deve superare un tiro salvezza su Arbitrio o subire gli effetti dell’incantesimo per la sua durata.\\
La velocità di un bersaglio soggetto all’incantesimo è dimezzata, questi subisce una penalità di -2 alla Difesa e ai tiri salvezza su Agilità, e non può usare reazioni. Durante il suo turno, può usare un’azione o un’azione bonus, ma non entrambe. Quali che siano le capacità o gli oggetti magici della creatura, durante il suo turno questa non può effettuare più di un attacco in mischia o a distanza.\\
Se la creatura tenta di lanciare un incantesimo con tempo di lancio 1 azione, tira un d20. Con 11 o più, l’incantesimo non avrà effetto fino al prossimo turno della creatura, e la creatura dovrà usare la sua azione in quel turno per completare l’incantesimo. Se non potrà farlo, l’incantesimo viene sprecato.\\
Una creatura sotto l’effetto di questo incantesimo effettua un altro tiro salvezza su Arbitrio al termine del suo turno. Se supera questo tiro salvezza, l’effetto ha termine.\\

\medskip\textbf{Levitazione}\index{Levitazione}\\
\textbf{Difficolta'}: 15\\
\textbf{Tempo di Lancio}: 2 Azioni\\
\textbf{Gittata}: 18 metri\\
\textbf{Componenti}: V, S, M (o un piccolo laccio di cuoio oppure un pezzo di cavo d’oro piegato a forma di tazza con un lungo stelo alla fine)\\
\textbf{Durata}: Concentrazione, massimo 10 minuti \\
Una creatura o oggetto a gittata che puoi vedere, scelto da te, si alza verticalmente fino a 6 metri e rimane sospeso per la durata dell’incantesimo. L’incantesimo può levitare un bersaglio pesante fino a 250 chili. Una creatura non consenziente che superi un tiro salvezza su Costituzione ignora l’effetto.\\
Il bersaglio può muoversi solo spingendo o tirando verso un oggetto fisso o superficie a portata (per esempio un muro o un soffitto). Durante il tuo turno puoi cambiare l’altitudine del bersaglio fino a 6 metri in entrambe le direzioni. Se sei tu il bersaglio, ti puoi muovere verso l’alto o il basso come parte del tuo movimento. Altrimenti puoi usare la tua azione per muovere il bersaglio, che deve rimanere nella gittata dell’incantesimo. Quando l’incantesimo termina, qualora sia ancora in aria, il bersaglio fluttua dolcemente a terra. 

\medskip\textbf{Libertà di Movimento}\index{Libertà di Movimento}\\
\textbf{Difficolta'}: 20\\
\textbf{Tempo di Lancio}: 2 Azioni\\
\textbf{Gittata}: Contatto\\
\textbf{Componenti}: V, S, M (una striscia di cuoio, avvolta intorno a un braccio o simile appendice)\\
\textbf{Durata}: 1 ora\\
Lanci l’incantesimo a contatto di una creatura consenziente. Per la sua durata, il movimento del bersaglio ignora il terreno difficile, mentre gli incantesimi o altri effetti magici non possono ridurre la sua velocità né far sì che il bersaglio sia paralizzato o intralciato.\\
Il bersaglio può spendere 1 metro di movimento per liberarsi automaticamente da qualsiasi restrizione non magica, come manette o una creatura da cui è afferrato. Infine, trovarsi sott’acqua non comporta penalità al movimento o gli attacchi del bersaglio. 

\medskip\textbf{Lingue}\index{Lingue}\\
\textbf{Difficolta'}: 18\\
\textbf{Tempo di Lancio}: 2 Azioni\\
\textbf{Gittata}: Contatto\\
\textbf{Componenti}: V, M (un piccolo modello di argilla di uno ziggurat)\\
\textbf{Durata}: 1 ora\\
Questo incantesimo conferisce alla creatura con cui sei stato in contatto al momento del lancio dell’incantesimo la capacità di comprendere qualsiasi linguaggio parlata che ode. Inoltre, quando il bersaglio parla, qualsiasi creatura che conosca almeno un linguaggio e può udire il bersaglio, comprende ciò che dice.

\medskip\textbf{Localizza Animali e Piante}\index{Localizza Animali e Piante}
\textbf{Difficolta'}: 15\\
\textbf{Tempo di Lancio}: 2 Azioni\\
\textbf{Gittata}: Personale\\
\textbf{Componenti}: V, S, M (un pezzo di pelo di un segugio) \\
\textbf{Durata}: Istantanea\\
Descrivi o nomina uno specifico tipo di bestia o vegetale. Concentrandoti sulla voce della natura nei tuoi dintorni, apprendi la direzione e la distanza dalla più vicina creatura o vegetale di quella specie, se ce ne sono entro 7,5 chilometri.

\medskip\textbf{Localizza Creatura}\index{Localizza Creatura}\\
\textbf{Difficolta'}: 20\\
\textbf{Tempo di Lancio}: 2 Azioni\\
\textbf{Gittata}: Personale\\
\textbf{Componenti}: V, S, M (un pezzo di pelliccia di segugio)\\
\textbf{Durata}: Concentrazione, massimo 1 ora\\
Descrivi o nomina una creatura che ti è familiare. Percepisci la direzione della posizione della creatura, purché quella creatura si trovi entro 300 metri da te. Se la creatura si muove, conosci anche la direzione del suo movimento.\\
L’incantesimo può localizzare una specifica creatura a te nota, o la più vicina creatura di una specie (come umano o unicorno), purché tu abbia visto una simile creatura da vicino (entro 9 metri) almeno una volta. Se la creatura che descrivi o nomini ha una forma diversa, per esempio è sotto gli effetti dell’incantesimo metamorfosi, questo incantesimo non sarà in grado di localizzare la creatura.\\
Questo incantesimo non può localizzare una creatura se un flusso di acqua corrente largo almeno 3 metri blocca un percorso diretto tra te e la creatura.

\medskip\textbf{Localizza Oggetto}\index{Localizza Oggetto}\\
\textbf{Difficolta'}: 15\\
\textbf{Tempo di Lancio}: 2 Azioni\\
\textbf{Gittata}: Personale\\
\textbf{Componenti}: V, S, M (un ramoscello biforcuto)\\
\textbf{Durata}: Concentrazione, massimo 10 minuti \\
Descrivi o nomina un oggetto che ti è familiare. Percepisci la direzione della posizione dell’oggetto, purché quell’oggetto si trovi entro 300 metri da te. Se l’oggetto si muove, conosci anche la direzione del suo movimento.\\
L’incantesimo può localizzare uno specifico oggetto a te noto, purché tu lo abbia visto da vicino (entro 9 metri) almeno una volta. In alternativa, l’incantesimo può localizzare l’oggetto più vicino di un particolare tipo, come certi tipi di abbigliamento, gioielleria, mobili, attrezzi o armi.\\
Questo incantesimo non può localizzare un oggetto se qualsiasi spessore di piombo, anche un foglio sottile, blocca un percorso diretto tra di te e l’oggetto. 

\medskip\textbf{Loquacità}\index{Loquacità}\\
\textbf{Difficolta'}: 30\\
\textbf{Tempo di Lancio}: 2 Azioni\\
\textbf{Gittata}: Personale\\
\textbf{Componenti}: V\\
\textbf{Durata}: 1 ora\\
Fino al termine dell’incantesimo, quando effettui una prova di Magnetismo puoi rimpiazzare il numero tirato con 15. Inoltre, non importa quello che dici, la magia o l'analisi che determina se stai dicendo la verità indicherà sempre che sei onesto.

\medskip\textbf{Luce}\index{Luce}\\
\textbf{Difficolta'}: 11\\
\textbf{Tempo di Lancio}: 2 Azioni\\
\textbf{Gittata}: Contatto\\
\textbf{Componenti}: V, M (una lucciola o del muschio fosforescente)\\
\textbf{Durata}: 1 ora\\
Lanci l’incantesimo a contatto di un oggetto che non sia più grosso di 3 metri in qualsiasi direzione. Fino al termine dell’incantesimo, l’oggetto irradia una luce intensa in un raggio di 6 metri e una luce fioca per ulteriori 6 metri. La luce può essere di qualsiasi colore tu voglia. Coprire completamente un oggetto con qualcosa di opaco blocca la luce. L’incantesimo termina se lo lanci di nuovo o lo interrompi con un’azione. Se un oggetto bersaglio è tenuto o indossato da una creatura ostile, quella creatura deve superare un tiro salvezza su Agilita' per evitare l’incantesimo. 

\medskip\textbf{Luce Diurna}\index{Luce Diurna}\\
\textbf{Difficolta'}: 18\\
\textbf{Tempo di Lancio}: 2 Azioni\\
\textbf{Gittata}: 18 metri\\
\textbf{Componenti}: V, S\\
\textbf{Durata}: 1 ora\\
Una sfera di luce con raggio 18 metri si espande da un punto a tua scelta entro la gittata. La sfera irradia luce intensa e luce fioca per ulteriori 18 metri. Se scegli un punto su di un oggetto che stai reggendo o che non è indossato o trasportato, la luce si irradia dall’oggetto e si muove con esso. Coprire completamente un oggetto con qualcosa di opaco, come un vaso o un elmo, blocca la luce. Se qualsiasi parte dell’area di questo incantesimo si sovrappone con l’area di oscurità creata da un incantesimo di difficolta' 18 o più basso, l’incantesimo che ha creato l’oscurità viene dissolto.

\medskip\textbf{Luci Danzanti}\index{Luci Danzanti}\\
\textbf{Difficolta'}: 11\\
\textbf{Tempo di Lancio}: 2 Azioni\\
\textbf{Gittata}: 36 metri\\
\textbf{Componenti}: V, S, M (un pezzo di fosforo o legno stregato, o un lombrico)\\
\textbf{Durata}: Concentrazione, massimo 1 minuto\\
Crei, a gittata, fino a quattro luci delle dimensioni di una torcia, facendole apparire come torce, lanterne o sfere luminose che fluttuano nell’aria per la durata dell’incantesimo. Puoi anche combinare le quattro luci in un’unica forma luminosa vagamente umanoide ditaglia Media. Qualsiasi  forma scegli, ciascuna luce emette una luce fioca in un raggio di 3 metri. Come azione bonus durante il tuo turno, puoi spostare le luci fino a 18 metri in un nuovo punto a gittata.\\
Una  luce deve trovarsi entro 6 metri da un’altra luce creata con questo incantesimo, e le luci svaniscono se eccedono la gittata dell’incantesimo. 

\medskip\textbf{Luminescenza}\index{Luminescenza}\\
\textbf{Difficolta'}: 13\\
\textbf{Tempo di Lancio}: 2 Azioni\\
\textbf{Gittata}: 18 metri\\
\textbf{Componenti}: V\\
\textbf{Durata}: Concentrazione, massimo 1 minuto \\
Tutti gli oggetti in un cubo di 6 metri di spigolo a gittata vengono circondati da una luce blu,  verde o viola (a tua scelta). Qualsiasi creatura nell’area quando l’incantesimo viene lanciato, viene anch’essa circondata dalla luce se fallisce un tiro salvezza su Agilità. Per la durata dell’incantesimo, gli oggetti e le creature soggette emettono una luce fioca con raggio di 3 metri. Qualsiasi tiro per colpire contro una creatura od oggetto soggetto ha vantaggio se l’attaccante può vederlo, e la creatura od oggetto non può beneficiare dell’invisibilità.

>>>

Mani Brucianti
[burning hands\textbf{Difficolta'}:
1° livello, invocazione
\textbf{Tempo di Lancio}: 2 Azioni
\textbf{Gittata}: Personale (cono di 4,5 metri)
\textbf{Componenti}: V, S
\textbf{Durata}: Istantanea
Mentre tieni le mani con i pollici che si toccano e le dita
tese, un sottile fiotto di fiamme parte da ciascuna delle
punta delle tue dita. Ogni creatura in un cono di 4,5
metri deve effettuare un tiro salvezza su Agilità.
Una creatura subisce 3d6 danni da fuoco se fallisce il
tiro salvezza, o la metà se lo supera.
Il fuoco incendia gli oggetti infiammabili nell’area che
non siano indossati o trasportati.
Ai Livelli Più Alti. Quando lanci questo incantesimo
usando uno slot incantesimo di 2° livello o più alto, il
danno aumenta di 1d6 per ogni livello dello slot sopra il
1°.
Mano Arcana
[arcane hand\textbf{Difficolta'}:
5° livello, invocazione
\textbf{Tempo di Lancio}: 2 Azioni
\textbf{Gittata}: 36 metri
\textbf{Componenti}: V, S, M (un guscio d’uovo e un guanto di
pelle di serpente)
\textbf{Durata}: Concentrazione, massimo 1 minuto
Crei una mano Grande, composta di energia
trasparente e luminosa, in uno spazio non occupato a
gittata e che puoi vedere. La mano permane per la
durata dell’incantesimo, e si muove al tuo comando,
imitando i movimenti della tua mano.
La mano è un oggetto che ha CA 20 e punti ferita uguali
ai tuoi punti ferita massimi. Ha Forza 26 (+8) e
Agilita' 10 (+0). La mano non riempie il suo spazio.
Quando lanci l’incantesimo e come azione bonus
durante i tuoi turni successivi, puoi muovere la mano
fino a 18 metri e poi generare uno dei seguenti effetti.
Mano Afferrante. La mano cerca di afferrare una
creatura di taglia Enorme o più piccola che si trovi entro 
1 metro da essa. Per risolvere l’azione di lottare usi la
Forza della mano. Se il bersaglio è di taglia Media o
inferiore, hai vantaggio alla prova. Mentre la mano tiene
afferrato il bersaglio, puoi usare un’azione bonus per
fare stritolare il bersaglio dalla mano. Quando lo fai, il
bersaglio subisce danni contundenti pari a 2d6 + il tuo
modificatore di caratteristica da incantatore.
Mano di Forza. La mano cerca di spingere una
creatura di 1 metro in una direzione a tua scelta.
Effettua una prova di Forza della mano contesa dalla
prova di Forza (Atletica) del bersaglio. Se il bersaglio è
di taglia Media o inferiore, hai vantaggio alla prova. Se
vinci la contesa, la mano spinge il bersaglio di 1 metro
più 1 metro moltiplicato il tuo modificatore di
caratteristica da incantatore (minimo 1 metro). La
mano si muove assieme al bersaglio per restare entro
1 metro da lui.
Mano Frapposta. La mano si frappone tra di te e una
creatura di tua scelta finché non le dai un comando
diverso. La mano si muove di modo da restare tra di te
e il bersaglio, fornendoti metà copertura contro il
bersaglio. Il bersaglio non può muoversi attraverso lo
spazio della mano se il suo punteggio di Forza è uguale
o inferiore al punteggio di Forza della mano. Se il suo
punteggio di Forza è superiore al punteggio di Forza
della mano, il bersaglio può muoversi attraverso lo
spazio della mano, ma considera quello spazio come
fosse terreno difficile.
Pugno Serrato. La mano colpisce una creatura o un
oggetto entro 1 metro da essa. Effettua un attacco in
mischia con incantesimo usando la mano e le tue
statistiche di gioco. Se colpisci, il bersaglio subisce 4d8
danni da forza.
Ai Livelli Più Alti. Quando lanci questo incantesimo
usando uno slot incantesimo di 6° livello o più alto, il
danno dell’opzione pugno serrato aumenta di 2d8 e il
danno dell’opzione mano afferrante aumenta di 2d6 per
ogni livello dello slot sopra il 5°.
Mano Magica
[mage hand\textbf{Difficolta'}:
Trucchetto, evocazione
\textbf{Tempo di Lancio}: 2 Azioni
\textbf{Gittata}: 9 metri
\textbf{Componenti}: V, S
\textbf{Durata}: 1 minuto
Una mano spettrale fluttuante compare in un punto a
gittata, scelto da te. La mano resta per la durata
dell’incantesimo o finché non viene interrotta con
un’azione. La mano svanisce se si dovesse trovare a
più di 9 metri da te o se lanci nuovamente
l’incantesimo.
Puoi usare la tua azione per controllare la mano. Puoi
usare la mano per manipolare un oggetto, aprire una
porta o un contenitore non chiusi a chiave, inserire o
recuperare un oggetto da un contenitore aperto, o
versare fuori i contenuti di una fiala. Puoi muovere la
mano di 9 metri ogni volta che la usi.
La mano non può attaccare, attivare oggetti magici o
trasportare più di 5 chili.
Marchio del Cacciatore
[hunter’s mark\textbf{Difficolta'}:
1° livello, divinazione
\textbf{Tempo di Lancio}: 1 Azione Immediata
\textbf{Gittata}: 27 metri
\textbf{Componenti}: V
\textbf{Durata}: Concentrazione, massimo 1 ora
Scegli una creatura a gittata che puoi vedere. La
creatura è misticamente marchiata come tua preda.
Fino al termine dell’incantesimo, infliggi 1d6 danni
aggiuntivi al bersaglio ogni volta che lo colpisci con un
attacco con arma, e hai vantaggio alle prove di
Saggezza (Percezione) o Saggezza (Sopravvivenza)
per trovarlo. Se il bersaglio scende a 0 punti ferita prima
del termine dell’incantesimo, puoi usare un’azione
bonus durante il tuo prossimo turno per marchiare una
nuova creatura.
Ai Livelli Più Alti. Quando lanci questo incantesimo
usando uno slot incantesimo di 3° o 4° livello, puoi
mantenere la concentrazione sull’incantesimo per un
massimo di 8 ore. Quando usi uno slot incantesimo di
5° livello o superiore, puoi mantenere la concentrazione
sull’incantesimo per un massimo di 24 ore.
Messaggio
[message\textbf{Difficolta'}:
Trucchetto, trasmutazione
\textbf{Tempo di Lancio}: 2 Azioni
\textbf{Gittata}: 36 metri
\textbf{Componenti}: V, S, M (un piccolo pezzo di cavo di
rame)
\textbf{Durata}: 1 round
Punti il dito verso una creatura a gittata e sussurri un
messaggio. Il bersaglio (e solo il bersaglio) ode il
messaggio e può replicare con un sussurro che solo tu
puoi udire.
Puoi lanciare questo incantesimo anche attraverso
oggetti solidi, se sei familiare col bersaglio e sai che
questi si trova dietro la barriera. Il silenzio magico, 30
centimetri di pietra, 2,5 centimetri di metallo normale,
un sottile foglio di piombo o 1 metro di legno bloccano
l’incantesimo. L’incantesimo non deve seguire una linea
retta, e può liberamente aggirare gli angoli o
attraversare gli spiragli.
Metamorfosi
[polymorph\textbf{Difficolta'}:
4° livello, trasmutazione
\textbf{Tempo di Lancio}: 2 Azioni
\textbf{Gittata}: 18 metri
\textbf{Componenti}: V, S, M (un bozzolo di bruco)
\textbf{Durata}: Concentrazione, massimo 1 ora
Questo incantesimo trasforma una creatura a gittata,
che puoi vedere, in una nuova forma. Una creatura non
consenziente deve superare un tiro salvezza su
Saggezza per evitare l’effetto. I mutaforma superano
automaticamente il tiro salvezza. L’incantesimo non ha
effetto su di un bersaglio con 0 punti ferita.
La trasformazione permane per la durata
dell’incantesimo o finché il bersaglio non scende a 0
punti ferita o muore. La nuova forma può essere quella
di qualsiasi bestia il cui grado di sfida sia uguale o più
basso di quello del bersaglio (o del livello del bersaglio,
se questi non ha un grado di sfida). Le statistiche di
gioco del bersaglio, compresi i punteggi delle
caratteristiche mentali, vengono rimpiazzate dalle 
statistiche della bestia scelta. Egli mantiene però il suo
allineamento e personalità.
Il bersaglio assume i punti ferita della sua nuova forma.
Quando ritorna alla sua forma normale, la creatura
ritorna al numero di punti ferita che aveva prima di
trasformarsi. Se però si ritrasforma perché ridotto a 0
punti ferita, qualsiasi danno in eccesso si ripercuote
sulla sua normale forma. Purché il danno in eccesso
non riduca la forma normale della creatura a 0 punti
ferita, ella non cade priva di sensi.
La creatura è limitata nelle azioni che può svolgere
dalla natura della sua nuova forma, e non può
dialogare, lanciare incantesimi, o effettuare qualsiasi
altra azione che richieda mani o di parlare.
L’equipaggiamento del bersaglio si fonde nella nuova
forma. La creatura non può attivare, usare, impugnare o
beneficiare in alcun modo del suo equipaggiamento.
Metamorfosi Pura
[true polymorph\textbf{Difficolta'}:
9° livello, trasmutazione
\textbf{Tempo di Lancio}: 2 Azioni
\textbf{Gittata}: 9 metri
\textbf{Componenti}: V, S, M (un goccio di mercurio, un
mucchietto di gomma arabica, e uno sbuffo di fumo)
\textbf{Durata}: Concentrazione, massimo 1 ora
Scegli una creatura od oggetto non magico a gittata e
che puoi vedere. L’incantesimo non ha effetto su di un
bersaglio con 0 punti ferita. Trasformi la creatura in una
creatura diversa, la creatura in un oggetto, o l’oggetto in
una creatura (l’oggetto non deve essere indossato né
trasportato da un’altra creatura). La trasformazione
permane per la durata dell’incantesimo o finché il
bersaglio non scende a 0 punti ferita o muore. Se ti
concentri su questo incantesimo per l’intera durata, la
trasformazione diventa permanente.
I mutaforma ignorano questo incantesimo. Una creatura
non consenziente può effettuare un tiro salvezza su
Saggezza e, se lo supera, ignora l’effetto di questo
incantesimo.
Creatura in Creatura. Se trasformi una creatura in
un’altra specie di creatura, la nuova forma può essere
quella di qualsiasi specie tu voglia, il cui grado di sfida
sia pari o inferiore a quello del bersaglio (o del suo
livello, se il bersaglio non ha un grado di sfida). Le
statistiche di gioco del bersaglio, compresi i punteggi
delle caratteristiche mentali, vengono rimpiazzate dalle
statistiche della nuova forma. Egli mantiene però il suo
allineamento e personalità.
Il bersaglio assume i punti ferita della sua nuova forma.
Quando ritorna alla sua forma normale, la creatura
ritorna al numero di punti ferita che aveva prima di
trasformarsi. Se però si ritrasforma perché ridotta a 0
punti ferita, qualsiasi danno in eccesso si ripercuote
sulla sua normale forma. Purché il danno in eccesso
non riduca la forma normale della creatura a 0 punti
ferita, ella non cade priva di sensi.
La creatura è limitata nelle azioni che può svolgere
dalla natura della sua nuova forma, e non può
dialogare, lanciare incantesimi, o effettuare qualsiasi
altra azione che richieda mani o di parlare, a meno che
la nuova forma non sia capace di svolgere queste
azioni.
L’equipaggiamento del bersaglio si fonde nella nuova
forma. La creatura non può attivare, usare, impugnare o
beneficiare in alcun modo del suo equipaggiamento.
Oggetto in Creatura. Puoi trasformare un oggetto in un
qualsiasi tipo di creatura, purché la taglia della creatura
non sia maggiore della taglia dell’oggetto e il grado di
sfida della creatura sia 9 o meno. La creatura è
amichevole verso di te e i tuoi compagni. Essa agisce
durante i tuoi turni. Decidi tu quali azioni essa compirà e
come si muove. Il Narratore possiede le statistiche della
creatura e risolverà tutte le sue azioni e i suoi
movimenti.
Se l’incantesimo diventa permanente, perdi il controllo
della creatura. A seconda di come l’hai trattata,
potrebbe restare amichevole nei tuoi confronti.
Creatura in Oggetto. Se trasformi una creatura in un
oggetto, essa si trasforma assieme a qualsiasi cosa stia
indossando o trasportando. Le statistiche della creatura
diventano quelle dell’oggetto, e, dopo che l’incantesimo
termina e la creatura ritorna alla sua forma normale,
questa non ha più ricordi del tempo trascorso in forma
di oggetto.
Miraggio Arcano
[mirage arcana\textbf{Difficolta'}:
7° livello, illusione
\textbf{Tempo di Lancio}: 10 minuti
\textbf{Gittata}: Vista
\textbf{Componenti}: V, S
\textbf{Durata}: 10 giorni
Fai sì che un pezzo di terreno a gittata, in un’area
quadrata fino a 1,5 chilometri, appaia, risuoni e odori
come qualche altro tipo di terreno. La conformazione
generale del terreno rimane tuttavia la stessa. Campi
aperti o una strada possono essere trasformati in un
acquitrino, colline, un crepaccio o qualche altro tipo di
terreno difficile o invalicabile. Un laghetto può essere
trasformato in una radura erbosa, un precipizio in una
lieve pendenza, un burrone cosparso di rocce in una
strada ampia e liscia.
Allo stesso modo, puoi modificare l’aspetto delle
strutture, o aggiungerne dove non ve ne sono.
L’incantesimo non camuffa, occulta né aggiunge
creature.
L’illusione comprende elementi uditivi, visivi, tattili e
olfattivi, così da poter trasformare un terreno sgombro
in terreno difficile (o viceversa) o impedire altrimenti il
movimento nell’area. Qualsiasi pezzo di terreno illusorio
(come una pietra o un bastone), che venga rimosso
dall’area dell’incantesimo, svanisce immediatamente.
Le creature con visione del vero possono vedere oltre
l’illusione e distinguere la vera forma del terreno;
tuttavia, gli altri elementi dell’illusione rimangono, così,
sebbene la creatura sia consapevole della presenza
dell’illusione, vi può comunque interagire fisicamente.
Modificare Memoria
[Modify Memory\textbf{Difficolta'}:
5° livello, ammaliamento
\textbf{Tempo di Lancio}: 2 Azioni
\textbf{Gittata}: 9 metri
\textbf{Componenti}: V, S
\textbf{Durata}: Concentrazione, massimo 1 minuto
Tenti di rimodellare i ricordi di un’altra creatura. Una
creatura che puoi vedere deve effettuare un tiro
salvezza su Saggezza. Se la stai combattendo, la
creatura ha vantaggio sul tiro salvezza. Se fallisce il tiro
salvezza, il bersaglio diventa affascinato da te per la
durata dell’incantesimo. Il bersaglio affascinato è inabile
e inconsapevole dell’ambiente circostante, sebbene sia
ancora in grado di udirti. Se subisce danni o diviene
bersaglio di un altro incantesimo, questo incantesimo 
termina, e nessuno dei ricordi del bersaglio viene
modificato.
Mentre il bersaglio resta affascinato da questo
incantesimo, puoi agire sui ricordi del bersaglio in
merito a un evento che abbia vissuto nelle ultime 24 ore
e che non sia durato più di 10 minuti. Puoi eliminare
permanentemente tutti i ricordi dell’evento, permettere
al bersaglio di ricordare l’evento con perfetta chiarezza
e dettagli particolareggiati, modificare il ricordo dei
dettagli dell’evento, o creare il ricordo di un altro evento.
Devi poter parlare al bersaglio per descrivere il modo in
cui i suoi ricordi saranno colpiti, e questi deve essere in
grado di comprendere il tuo linguaggio, affinché i ricordi
modificati si instaurino nella sua memoria. Se
l’incantesimo termina prima che tu abbia finito di
descrivere i ricordi modificati, la memoria della creatura
non viene alterata. Altrimenti, i ricordi modificati si
instaurano al termine dell’incantesimo.
Una memoria modificata non influisce necessariamente
sul comportamento della creatura, in particolare se i
suoi ricordi contraddicono le inclinazioni naturali,
l’allineamento o la fede della creatura. Una memoria
modificata in modo illogico, come impiantare il ricordo di
quanto la creatura ami immergersi nell’acido, viene
rimossa, come fosse un brutto sogno. Il Narratore può
giudicare un ricordo modificato troppo insensato perché
abbia alcun effetto su di una creatura.
Un incantesimo rimuovi maledizione o ristorare
superiore lanciato sul bersaglio ne ripristina i veri
ricordi.
Ai Livelli Più Alti. Quando lanci questo incantesimo
usando uno slot incantesimo di 6° livello o più alto, puoi
alterare i ricordi di un bersaglio riguardo un evento
svoltosi fino a 7 giorni prima (6° livello), 30 giorni prima
(7° livello), 1 anno prima (8° livello) o qualsiasi punto
nel passato della creatura (9° livello).
Movimenti del Ragno
[spider climb\textbf{Difficolta'}:
2° livello, trasmutazione
\textbf{Tempo di Lancio}: 2 Azioni
\textbf{Gittata}: Contatto
\textbf{Componenti}: V, S, M (una goccia di bitume e un
ragno)
\textbf{Durata}: Concentrazione, massimo 1 ora
Lanci l’incantesimo a contatto di una creatura
consenziente. Fino al termine dell’incantesimo, la
creatura ottiene la capacità di spostarsi verso l’alto, il
basso e lungo superfici verticali o stando a testa in giù
sul soffitto, tenendo le mani libere. Il bersaglio ottiene
anche velocità di scalata pari alla sua velocità di
passeggio.
Muovere il Terreno
[Move Earth\textbf{Difficolta'}:
6° livello, trasmutazione
\textbf{Tempo di Lancio}: 2 Azioni
\textbf{Gittata}: 36 metri
\textbf{Componenti}: V, S, M (un badile di ferro e un piccola
borsa contenente un misto di tipi di terreno – argilla,
concime e sabbia)
\textbf{Durata}: Concentrazione, massimo 2 ore
Scegli un’area sul terreno a gittata, non più grande di
12 metri di lato. Per la durata, puoi rimodellare terriccio,
sabbia o argilla nell’area in qualsiasi modo tu voglia.
Puoi innalzare o abbassare l’altitudine dell’area, creare
o riempire un fossato, erigere o abbassare un muro, o
formare un pilastro. La portata di questi cambiamenti
non può eccedere metà della dimensione più grossa
dell’area. Così, se operi su di un quadrato di 12 metri di
lato, puoi creare un pilastro alto 6 metri, innalzare o
abbassare l’altitudine del terreno di 6 metri, scavare un
fossato profondo 6 metri, e così via. Ci vogliono 10
minuti per completare questi mutamenti.
Al termine di ogni 10 minuti trascorsi a concentrarsi
sull’incantesimo, puoi scegliere una nuova area di
terreno su cui operare.
Dato che la trasformazione del terreno avviene
lentamente, le creature nell’area di solito non possono
restare intrappolate o ferite dal movimento del terreno.
L’incantesimo non può manipolare la pietra naturale o
le costruzioni in pietra. Le rocce e le strutture si
muovono per adattarsi al nuovo terreno. Se il modo in
cui modelli il terreno renderebbe una struttura instabile,
questa potrebbe crollare.
Allo stesso modo, questo incantesimo non influenza
direttamente la crescita dei vegetali. La terra smossa
trasporta con sé qualsiasi vegetale presente.
Muro di Forza
[wall of force\textbf{Difficolta'}:
5° livello, invocazione
\textbf{Tempo di Lancio}: 2 Azioni
\textbf{Gittata}: 36 metri
\textbf{Componenti}: V, S, M (un pizzico di polvere prodotta
frantumando una gemma trasparente)
\textbf{Durata}: Concentrazione, massimo 10 minuti
Un invisibile muro di forza si forma in un punto a gittata
scelto da te. Il muro appare in qualsiasi orientamento
da te desiderato, come una barriera orizzontale o
verticale oppure angolata. Può fluttuare nell’aria o
appoggiarsi su di una superficie solida. Puoi darle la
forma di una cupola semisferica o di una sfera con un
raggio massimo di 3 metri, oppure darle l’aspetto di una
superficie piana composta da un massimo di dieci
pannelli di 3 metri per 3 metri. Ogni pannello deve
essere contiguo a un altro pannello. In qualsiasi forma,
il muro ha uno spessore di 75 centimetri e resta per
tutta la durata dell’incantesimo. Se il muro taglia uno
spazio di una creatura, quando compare, la creatura
viene spinta da un lato del muro (a tua discrezione).
Nulla può attraversare fisicamente il muro. È immune a
tutti i danni e non può essere dissolto da dissolvi magie.
Tuttavia, il muro è distrutto all’istante dall’incantesimo
disintegrazione. Il muro si estende anche sul Piano
Etereo, impedendo ai viaggiatori eterei di attraversarlo.
Muro di Fuoco
[wall of fire\textbf{Difficolta'}:
4° livello, invocazione
\textbf{Tempo di Lancio}: 2 Azioni
\textbf{Gittata}: 36 metri
\textbf{Componenti}: V, S, M (un piccolo pezzo di fosforo)
\textbf{Durata}: Concentrazione, massimo 1 minuto
Crei un muro di fuoco su di una superficie solida a
gittata. Puoi creare un muro lungo fino a 18 metri, alto
fino a 6 metri e spesso 30 centimetri, o un muro
circolare di 6 metri di diametro, 6 metri di altezza e 30
centimetri di spessore. Il muro è opaco e rimane per la
durata dell’incantesimo.
Quando il muro appare, ogni creatura nella sua area
deve effettuare un tiro salvezza su Agilità. Una
creatura subisce 5d8 danni da fuoco se fallisce il tiro
salvezza, o la metà se lo supera.
Un lato del muro, selezionato da te quando lanci questo
incantesimo, infligge 5d8 danni da fuoco a ciascuna
creatura che termini il suo turno entro 3 metri da quel
lato o all’interno del muro. Una creatura subisce lo
stesso danno quando entra nel muro per la prima volta
durante un turno. L’altro lato del muro non infligge
danni.
Ai Livelli Più Alti. Quando lanci questo incantesimo
usando uno slot incantesimo di 5° livello o più alto, il
danno aumenta di 1d8 per ogni livello dello slot sopra il
4°.
Muro di Ghiaccio
[wall of ice\textbf{Difficolta'}:
6° livello, invocazione
\textbf{Tempo di Lancio}: 2 Azioni
\textbf{Gittata}: 36 metri
\textbf{Componenti}: V, S, M (un piccolo pezzo di quarzo)
\textbf{Durata}: Concentrazione, massimo 10 minuti
Crei un muro di ghiaccio su di una superficie solida a
gittata. Puoi creare una cupola semisferica o una sfera
con un raggio massimo di 3 metri, o puoi creare una
superficie piana composta di un massimo di dieci
panelli quadrati di 3 metri di lato. Ogni pannello deve
essere contiguo ad almeno un altro pannello. In ogni
forma, il muro è spesso 30 centimetri e rimane per la
durata dell’incantesimo.
Se, quando compare, il muro attraversa lo spazio di una
creatura, la creatura viene spinta da una parte del muro
(a tua scelta) e deve effettuare un tiro salvezza su
Agilita'. Se fallisce il tiro salvezza, la creatura
subisce 10d6 danni da freddo, o la metà di questi danni
se lo supera.
Il muro è un oggetto che può essere danneggiato e
sfondato. Ogni sezione di 3 metri ha CA 12 e 30 punti
ferita, ed è vulnerabile al danno da fuoco. Ridurre una
sezione di 3 metri a 0 punti ferita la distrugge e lascia
nello spazio che era occupato dal muro una brezza di
vento gelido. Una creatura che si muova attraverso
questa brezza di vento gelido per la prima volta in un
turno, deve effettuare un tiro salvezza su Potenza.
Se lo fallisce, la creatura subisce 5d6 danni da freddo,
o la metà di questi danni se lo supera.
Ai Livelli Più Alti. Quando lanci questo incantesimo
usando uno slot incantesimo di 7° livello o più alto,
entrambi i tipi di danno aumentano di 1d8 per ogni
livello dello slot sopra il 6°.
Muro di Pietra
[wall of stone\textbf{Difficolta'}:
5° livello, invocazione
\textbf{Tempo di Lancio}: 2 Azioni
\textbf{Gittata}: 36 metri
\textbf{Componenti}: V, S, M (un piccolo blocco di granito)
\textbf{Durata}: Concentrazione, massimo 10 minuti
Un muro di pietra solida non magico si forma in un
punto a gittata, scelto da te. Il muro è spesso 15
centimetri ed è composto di pannelli di 3 per 3 metri.
Ogni pannello deve essere contiguo ad almeno un altro
pannello. In alternativa, puoi creare pannelli 3 x 6 metri
di soli 7,5 centimetri di spessore.
Se, quando compare, il muro attraversa lo spazio di una
creatura, la creatura viene spinta da una parte del muro
(a tua scelta). Se la creatura fosse circondata da tutte le
parti dal muro (o dal muro e un’altra superficie solida),
la creatura può effettuare un tiro salvezza su Agilità.
Se lo supera, può usare la sua reazione per muoversi
della sua velocità in modo da non essere più
intrappolata nel muro.
Il muro può aver qualsiasi forma tu desideri, sebbene
non possa occupare lo stesso spazio di una creatura od
oggetto. Il muro può anche non essere verticale o
poggiare su di un piano. Deve, tuttavia, fondersi con ed
essere sostenuto da pietra già esistente. Quindi, puoi
usare questo incantesimo per creare un ponte su di un
baratro o creare un rampa.
Se crei un muro non verticale del genere, più lungo di 6
metri, devi dimezzare le dimensioni di ciascun pannello
per creare dei supporti. Puoi modellare rozzamente la
pietra per creare merlature, spalti e così via.
Il muro è un oggetto fatto di pietra che può essere
danneggiato e sfondato. Ogni pannello ha CA 15 e 30
punti ferita ogni 2,5 centimetri di spessore. Ridurre un
pannello a 0 punti ferita lo distrugge e potrebbe far
crollare i pannelli connessi, a discrezione del Narratore.
Se mantieni la concentrazione su questo incantesimo
per la sua intera durata, il muro diventa permanente e
non può essere dissolto. Altrimenti, il muro sparisce al
termine dell’incantesimo.
Muro Prismatico
[prismatic wall\textbf{Difficolta'}:
9° livello, abiurazione
\textbf{Tempo di Lancio}: 2 Azioni
\textbf{Gittata}: 18 metri
\textbf{Componenti}: V, S
\textbf{Durata}: 10 minuti
Un piano di luci brillanti e multicolore forma un muro
verticale opaco, largo fino a 27 metri, alto 9 metri e
spesso 2,5 centimetri, centrato su di un punto a gittata
e che puoi vedere. In alternativa, puoi modellare il muro
in una sfera, fino a 9 metri di diametro, centrata su di un
punto a gittata di tua scelta. Il muro resta fisso sul posto
per la durata dell’incantesimo. Se posizioni il muro in
modo che attraversi lo spazio occupato da una
creatura, l’incantesimo fallisce, e la tua azione e lo slot
incantesimo sono sprecati.
Il muro irradia luce intensa fino a una gittata di 30 metri
e luce fioca per ulteriori 30 metri. Tu e le creature
indicate da te al momento del lancio dell’incantesimo
potete attraversare e restare vicini al muro senza
pericolo. Se un’altra creatura che può vedere il muro si
muove entro 6 metri da esso o inizia lì il suo turno, deve
superare un tiro salvezza su Potenza o restare
accecata per 1 minuto.
Il muro consiste di sette strati, ognuno di un diverso
colore. Quando una creatura cerca di immergersi o
attraversare il muro, lo fa uno strato alla volta,
attraverso tutti gli strati del muro. Mentre si immerge o
attraversa ciascuno strato, la creatura deve superare un
tiro salvezza su Agilità o subire le proprietà di
ciascuno strato, uno alla volta, come descritto di
seguito.
Il muro può essere distrutto, uno strato alla volta, in
ordine dal rosso al violetto, in un modo specifico per
ogni strato. Una volta che uno strato è distrutto, lo sarà
per la durata dell’incantesimo. Una verga di
cancellazione distrugge un muro prismatico, ma un
campo anti-magia non ha effetto su di esso.
1. Rosso. Il bersaglio subisce 10d6 danni da fuoco se
fallisce il tiro salvezza, o la metà di questi danni se lo
supera. Finché questo strato esiste, gli attacchi a
distanza non magici non possono attraversare il muro. 
Lo strato può essere distrutto infliggendogli 25 danni da
freddo.
2. Arancio. Il bersaglio subisce 10d6 danni da acido se
fallisce il tiro salvezza, o la metà di questi danni se lo
supera. Finché questo strato esiste, gli attacchi a
distanza magici non possono attraversare il muro. Lo
strato può essere distrutto da un forte vento.
3. Giallo. Il bersaglio subisce 10d6 danni da fulmine se
fallisce il tiro salvezza, o la metà di questi danni se lo
supera. Questo strato può essere distrutto infliggendogli
60 danni di forza.
4. Verde. Il bersaglio subisce 10d6 danni da veleno se
fallisce il tiro salvezza, o la metà di questi danni se lo
supera. Un incantesimo passapareti, o un altro
incantesimo di pari livello o più alto che può aprire un
portale su di una superficie solida, distrugge questo
strato.
5. Blu. Il bersaglio subisce 10d6 danni da freddo se
fallisce il tiro salvezza, o la metà di questi danni se lo
supera. Lo strato può essere distrutto infliggendogli
almeno 25 danni da fuoco.
6. Indaco. Se fallisce il tiro salvezza, il bersaglio è
intralciato. Deve poi effettuare un tiro salvezza su
Costituzione all’inizio di ciascun suo turno. Se supera il
tiro salvezza tre volte, l’incantesimo termina. Se fallisce
il tiro salvezza tre volte, viene permanentemente
trasformato in pietra e diventa vittima della condizione
pietrificato. I successi e i fallimenti non devono essere
consecutivi; tieni traccia di entrambi finché il bersaglio
non ne ha ottenuti tre dello stesso tipo.
Finché questo strato esiste, non si possono lanciare
incantesimi attraverso il muro. Lo strato viene distrutto
dalla luce intensa emanata dall’incantesimo luce diurna
o da un simile incantesimo di pari livello o più alto.
7. Violetto. Se fallisce il tiro salvezza, il bersaglio è
accecato. Deve poi effettuare un tiro salvezza su
Saggezza all’inizio del tuo prossimo turno. Se supera il
tiro salvezza, la cecità termina. Se fallisce il tiro
salvezza, la creatura viene trasportata su di un altro
piano di esistenza a scelta del Narratore e non è più accecata
(di solito, una creatura che non è sul suo piano natio,
viene esiliata su di esso, mentre le altre creature sono
di solito gettate nei piani Astrale o Etereo). Questo
strato è distrutto dall’incantesimo dissolvi magie o da un
incantesimo simile di pari livello o più alto che possa
porre fine a incantesimi ed effetti magici.
Muro di Spine
[wall of thorns\textbf{Difficolta'}:
6° livello, evocazione
\textbf{Tempo di Lancio}: 2 Azioni
\textbf{Gittata}: 36 metri
\textbf{Componenti}: V, S, M (una manciata di spine)
\textbf{Durata}: Concentrazione, massimo 10 minuti
Crei un muro di cespugli robusti, malleabili e impigliati,
ricolmi di spine appuntite. Il muro compare a gittata su
di una superficie solida e rimane per la durata
dell’incantesimo. Il muro che puoi creare può essere
lungo fino a 18 metri, alto fino a 3 metri, e spesso fino a
1 metro o un circolo che abbia un diametro di 6 metri e
sia alto fino a 6 metri e spesso 1 metro. Il muro blocca
la linea di visuale.
Quando il muro compare, ogni creatura nella sua area
deve effettuare un tiro salvezza su Agilità. Se
fallisce il tiro salvezza, una creatura subisce 7d8 danni
perforanti, o la metà di questi danni se lo supera.
Una creatura può muoversi attraverso il muro, seppure
in maniera lenta e dolorosa. Per ogni 1 metro che la
creatura si muove attraverso il muro, deve spendere 6
metri di movimento. Inoltre, la prima volta che una
creatura entra nel muro durante un turno o vi termina il
suo turno dentro, la creatura deve effettuare un tiro
salvezza su Agilita'. Subisce 7d8 danni taglienti se
fallisce il tiro salvezza, o la metà di questi danni se lo
supera.
Ai Livelli Più Alti. Quando lanci questo incantesimo
usando uno slot incantesimo di 7° livello o più alto, il
danno aumenta di 1d8 per ogni livello dello slot sopra il
6°.
Muro di Vento
[wind wall\textbf{Difficolta'}:
3° livello, invocazione
\textbf{Tempo di Lancio}: 2 Azioni
\textbf{Gittata}: 36 metri
\textbf{Componenti}: V, S, M (un minuscolo ventaglio e una
piuma di origini esotiche)
\textbf{Durata}: Concentrazione, massimo 1 minuto
Un muro di forte vento si leva dal terreno in un punto a
gittata di tua scelta. Puoi creare un muro lungo fino a 15
metri, alto 4,5 metri e spesso 30 centimetri. Puoi
modellare il muro in qualsiasi maniera desideri purché
componga un percorso continuo sul terreno. Il muro
rimane per la durata dell’incantesimo.
Quando il muro appare, ogni creatura all’interno della
sua area deve effettuare un tiro salvezza su Potenza. Una
creatura subisce 3d8 danni contundenti se fallisce il tiro
salvezza, o la metà di questi danni se lo supera.
Il forte vento tiene lontana foschia, fumo e altri gas. Le
creature volanti di taglia Piccola o minore non possono
attraversare il muro. I materiali leggeri trascinati nel
muro volano verso l’alto. Frecce, quadrelli e altre
munizioni normali vengono deviati e mancano
automaticamente il bersaglio (i macigni scagliati dai
giganti e dalle macchine d’assedio, e munizioni simili,
ne ignorano invece gli effetti). Le creature in forma
gassosa non possono attraversarlo.
Nube Incendiaria
[incendiary cloud\textbf{Difficolta'}:
8° livello, evocazione
\textbf{Tempo di Lancio}: 2 Azioni
\textbf{Gittata}: 45 metri
\textbf{Componenti}: V, S
\textbf{Durata}: Concentrazione, massimo 1 minuto
Una nube di fumo turbinante attraversata da lapilli
incandescenti si forma in una sfera di 6 metri di raggio
centrata su di un punto a gittata. La nube si propaga
intorno agli angoli ed è oscurata pesantemente. Rimane
per la durata dell’incantesimo o finché un vento di
velocità moderata o superiore (almeno 15 chilometri
all’ora) non la disperde.
Quando la nube appare, ogni creatura al suo interno
deve effettuare un tiro salvezza su Agilità. Una
creatura subisce 10d8 danni da fuoco se fallisce il tiro
salvezza, e la metà di questi danni se lo supera. Una
creatura deve effettuare il tiro salvezza anche quando
entra per la prima volta nell’area o termina lì il suo
turno.
All’inizio di ciascun tuo turno, la nube si muove di 3
metri lontano da te in una direzione a tua scelta.
Nube Maleodorante
[stinking cloud\textbf{Difficolta'}:
3° livello, evocazione
\textbf{Tempo di Lancio}: 2 Azioni
\textbf{Gittata}: 27 metri
\textbf{Componenti}: V, S, M (un uovo marcio o foglie di
cavolo puzzolente)
\textbf{Durata}: Concentrazione, massimo 1 ora
Crei, in un punto a gittata, una sfera di 6 metri di raggio
composta di un gas giallo e nauseabondo. La nube si
propaga dietro gli angoli e la sua area è oscurata
pesantemente. La nube permane nell’aria per la durata.
Ogni creatura che si trovi completamente all’interno
della nube all’inizio del proprio turno, deve effettuare un
tiro salvezza su Potenza contro il veleno. Se il tiro
salvezza fallisce, la creatura spende la sua azione di
quel turno a vomitare e barcollare. Le creature che non
hanno bisogno di respirare o che sono immuni al veleno
superano automaticamente il tiro salvezza.
Un vento moderato (almeno 15 chilometri all’ora)
disperde la nube dopo 4 round. Un vento forte (almeno
30 chilometri all’ora) lo disperde dopo 1 round.
Nube Mortale
[cloudkill\textbf{Difficolta'}:
5° livello, evocazione
\textbf{Tempo di Lancio}: 2 Azioni
\textbf{Gittata}: 36 metri
\textbf{Componenti}: V, S
\textbf{Durata}: Concentrazione, massimo 10 minuti
Crei una sfera di 6 metri di raggio formata da una
nebbia velenosa giallo-verde centrata in un punto a
gittata di tua scelta. La nebbia si propaga dietro gli
angoli. Rimane per la durata dell’incantesimo o finché
un forte vento non disperde la nebbia, terminando
l’incantesimo. La sua area è oscurata pesantemente.
Quando una creatura entra nell’area dell’incantesimo
per la prima volta in un turno o inizia lì il suo turno,
quella creatura deve effettuare un tiro salvezza su
Costituzione. La creatura subisce 5d8 danni da veleno
se fallisce il tiro salvezza, o la metà di questi danni se lo
supera. Le creature ne sono soggette anche se
trattengono il respiro o non hanno bisogno di respirare.
La nebbia si allontana di 3 metri da te all’inizio di ogni
tuo turno, spostandosi lungo la superficie del terreno. I
vapori, essendo più pesanti dell’aria, tendono a
scendere verso il basso, arrivando addirittura a
insinuarsi nelle aperture.
Ai Livelli Più Alti. Quando lanci questo incantesimo
usando uno slot incantesimo di 6° livello o più alto, il
danno aumenta di 1d8 metri per ogni livello dello slot
sopra il 5°.
Nube di Nebbia
[fog cloud\textbf{Difficolta'}:
1° livello, evocazione
\textbf{Tempo di Lancio}: 2 Azioni
\textbf{Gittata}: 36 metri
\textbf{Componenti}: V, S
\textbf{Durata}: Concentrazione, massimo 1 ora
Crei una sfera di foschia del raggio di 6 metri centrata
su di un punto a gittata. La sfera si propaga intorno agli
angoli, e la sua area è oscurata pesantemente. Rimane
per la durata dell’incantesimo o finché un vento di
velocità moderata o superiore (almeno 15 chilometri
all’ora) non la disperde.
Ai Livelli Più Alti. Quando lanci questo incantesimo
usando uno slot incantesimo di 2° livello o più alto, il
raggio della foschia aumenta di 6 metri per ogni livello
dello slot sopra il 1°.
Occhio Arcano
[arcane eye\textbf{Difficolta'}:
4° livello, divinazione
\textbf{Tempo di Lancio}: 2 Azioni
\textbf{Gittata}: 9 metri
\textbf{Componenti}: V, S, M (un pezzo di manto di pipistrello)
\textbf{Durata}: Concentrazione, massimo 1 ora
Crei a gittata un occhio magico e invisibile, che fluttua
nell’aria per la durata dell’incantesimo.
Ricevi mentalmente le informazioni visive dall’occhio,
che ha vista normale e scurovisione fino a 9 metri.
L’occhio può guardare in tutte le direzioni.
Con un’azione, puoi spostare l’occhio di 9 metri in
qualsiasi direzione. Non c’è limite a quanto lontano
possa spostarsi l’occhio, ma non può entrare in un altro
piano di esistenza. Una barriera solida blocca il
movimento dell’occhio, ma questo può attraversare
un’apertura di una grandezza minima di 2,5 centimetri
di diametro.
Onda Tonante
[thunderwave\textbf{Difficolta'}:
1° livello, invocazione
\textbf{Tempo di Lancio}: 2 Azioni
\textbf{Gittata}: Personale (cubo di 4,5 metri di spigolo)
\textbf{Componenti}: V, S
\textbf{Durata}: Istantanea
Un’onda di forza tonante si proietta da te. Ogni creatura
in un cubo di 4,5 metri di spigolo che origina da te deve
effettuare un tiro salvezza su Potenza. Se fallisce il
tiro salvezza, una creatura subisce 2d8 danni da tuono
e viene allontana 3 metri da te. Se supera il tiro
salvezza, la creatura subisce la metà dei danni e non
viene allontanata.
Inoltre, gli oggetti non ancorati che sono totalmente
all’interno dell’area vengono spinti 3 metri lontano da te
dall’effetto dell’incantesimo, e l’incantesimo produce un
rimbombo tonante udibile fino a 90 metri.
Ai Livelli Più Alti. Quando lanci questo incantesimo
usando uno slot incantesimo di 2° livello o più alto, il
danno aumenta di 1d8 per ogni livello dello slot sopra il
1°.
Oscurità
[darkness\textbf{Difficolta'}:
2° livello, invocazione
\textbf{Tempo di Lancio}: 2 Azioni
\textbf{Gittata}: 18 metri
\textbf{Componenti}: V, M (pelo di pipistrello e un pizzico di
bitume o un pezzo di carbone)
\textbf{Durata}: Concentrazione massimo 10 minuti
L’oscurità magica si propaga da un punto a gittata,
scelto da te, per riempire una sfera di 4,5 metri di raggio
per la durata dell’incantesimo. L’oscurità si propaga
intorno agli angoli. Una creatura con scurovisione non
può vedere in questa oscurità, e la luce non magica non
può illuminarla.
Se il punto che hai scelto è su di un oggetto che stai
trasportando o uno che non è indossato o trasportato,
l’oscurità emana dall’oggetto e si muove con esso.
Coprire completamente la fonte dell’oscurità con un
oggetto opaco, come un vaso o un elmo, blocca
l’oscurità.
Se qualsiasi parte dell’area di questo incantesimo si
sovrappone con l’area di luce creata da un incantesimo
di 2° livello o più basso, l’incantesimo che ha creato la
luce viene dissolto.
Palla di Fuoco
[fireball\textbf{Difficolta'}:
3° livello, invocazione
\textbf{Tempo di Lancio}: 2 Azioni
\textbf{Gittata}: 45 metri
\textbf{Componenti}: V, S, M (una minuscola palla di guano di
pipistrello e zolfo)
\textbf{Durata}: Istantanea
Un fascio di luce gialla parte dal tuo dito puntato verso
un punto a gittata scelto da te, e poi esplode con un
boato sommesso e si trasforma in un getto di fiamme.
Ogni creatura in una sfera di 6 metri di raggio centrata
in quel punto deve effettuare un tiro salvezza su
Agilita'. Una creatura subisce 8d6 danni da fuoco se
fallisce il tiro salvezza, o la metà di questi danni se lo
supera.
Il fuoco si propaga intorno agli angoli. Il fuoco incendia
gli oggetti infiammabili nell’area che non sono indossati
o trasportati.
Ai Livelli Più Alti. Quando lanci questo incantesimo
usando uno slot incantesimo di 4° livello o più alto, il
danno base aumenta di 1d6 per ogni livello dello slot
sopra il 3°.
Palla di Fuoco Ritardata
[delayed blast fireball\textbf{Difficolta'}:
7° livello, invocazione
\textbf{Tempo di Lancio}: 2 Azioni
\textbf{Gittata}: 45 metri
\textbf{Componenti}: V, S, M (una minuscola palla di guano di
pipistrello e zolfo)
\textbf{Durata}: Concentrazione, massimo 1 minuto
Un fascio di luce gialla parte dal tuo dito puntato, per
condensarsi per la durata dell’incantesimo nella forma
di una pallina luminosa in un punto a gittata, scelto da
te. Quando l’incantesimo termina, o perché la tua
concentrazione è spezzata o perché decidi tu di porgli
fine, la pallina esplode con un boato sommesso e si
trasforma in un getto di fiamme che si propaga dietro gli
angoli. Ogni creatura in una sfera di 6 metri di raggio
centrata in quel punto deve effettuare un tiro salvezza
su Agilita'. Una creatura subisce danni da fuoco pari
al danno totale accumulato se fallisce il tiro salvezza, o
la metà di questi danni se lo supera.
Il danno base dell’incantesimo è 12d6. Se al termine del
tuo turno la pallina non è ancora detonata, il danno
aumenta di 1d6.
Se la pallina luminosa viene toccata prima che
l’incantesimo abbia avuto fine, la creatura che la tocca
deve effettuare un tiro salvezza su Agilità. Se
fallisce il tiro salvezza, l’incantesimo termina
immediatamente, facendo eruttare fiamme dalla pallina.
Se supera il tiro salvezza, la creatura può lanciare la
pallina fino a 12 metri di distanza. Quando colpisce una
creatura od oggetto solido, l’incantesimo ha fine e la
pallina esplode.
Il fuoco danneggia gli oggetti nell’area e incendia gli
oggetti infiammabili che non sono indossati o
trasportati.
Ai Livelli Più Alti. Quando lanci questo incantesimo
usando uno slot incantesimo di 8° livello o più alto, il
danno base aumenta di 1d6 per ogni livello dello slot
sopra il 7°.
Parlare con gli Animali
[speak with animals\textbf{Difficolta'}:
1° livello, divinazione (rituale)
\textbf{Tempo di Lancio}: 2 Azioni
\textbf{Gittata}: Personale
\textbf{Componenti}: V, S
\textbf{Durata}: 10 minuti
Per la durata dell’incantesimo, ottieni la capacità di
comprendere e comunicare verbalmente con le bestie.
Il sapere e la consapevolezza di molte bestie sono
limitati dal loro intelletto ma, come minimo, le bestie
possono fornirti informazioni riguardo luoghi e mostri
nelle vicinanze, compresi quelli che possono percepire
o hanno percepito nei giorni passati. A discrezione del
Narratore potresti riuscire a convincere una bestia a farti un
piccolo favore.
Parlare con i Morti
[speak with dead\textbf{Difficolta'}:
3° livello, negromanzia
\textbf{Tempo di Lancio}: 2 Azioni
\textbf{Gittata}: 3 metri
\textbf{Componenti}: V, S, M (incenso acceso)
\textbf{Durata}: 10 minuti
Conferisci un’apparenza di vita e Intelletto a un
cadavere a gittata, scelto da te, permettendogli di
rispondere alle domande che gli poni. Il cadavere deve
avere ancora una bocca e non può essere non morto.
L’incantesimo fallisce se il cadavere è già stato
bersaglio di questo incantesimo negli ultimi 10 giorni.
Fino al termine dell’incantesimo, puoi porre al cadavere
fino a cinque domande. Il cadavere conosce solo quello
che già sapeva in vita, compresi i linguaggi parlati. Le
risposte sono di solito brevi, criptiche o ripetitive, e il
cadavere non è sotto nessun obbligo a darti risposte
veritiere se gli sei ostile o ti riconosce come suo
nemico. Questo incantesimo non riporta l’anima della
creatura nel corpo, ma solo lo spirito che lo muove. Di
conseguenza, il cadavere non può apprendere nuove
informazioni, non capisce nulla di quello che è
successo da quando è morto, e non può fare
valutazioni su eventi futuri.
Parlare con le Piante
[speak with plants\textbf{Difficolta'}:
3° livello, trasmutazione
\textbf{Tempo di Lancio}: 2 Azioni
\textbf{Gittata}: Personale (raggio di 9 metri)
\textbf{Componenti}: V, S
\textbf{Durata}: 10 minuti
Infondi i vegetali entro 9 metri da te di capacità
senziente e di limitata mobilità, dandole la capacità di
comunicare con te ed eseguire dei semplici comandi.
Puoi interrogare i vegetali in merito a eventi avvenuti
nell’ultimo giorno nell’area dell’incantesimo, ottenendo
informazioni sulle creature di passaggio, il clima e altro.
Puoi anche trasformare il terreno difficile prodotto dalla
crescita dei vegetali (come cespugli e fitto sottobosco)
in terreno ordinario per la durata dell’incantesimo.
Oppure puoi trasformare del terreno normale in cui
siano presenti dei vegetali in terreno difficile, che
rimane per la durata dell’incantesimo facendo sì, per
esempio, che rampicanti e rami rallentino gli inseguitori.
A discrezione del Narratore i vegetali potrebbero svolgere
anche altri compiti per tuo conto. L’incantesimo non
permette ai vegetali di sradicarsi e muoversi, ma
possono muovere liberamente rami, steli e gambi.
Se una creatura vegetale si trova nell’area, puoi
comunicare con essa come se parlaste lo stesso
linguaggio, ma non ottieni alcuna capacità magica per
influenzarla.
Questo incantesimo può far sì che i vegetali creati
dall’incantesimo intralciare rilascino una creatura
intralciata.
Parola Divina
[divine word\textbf{Difficolta'}:
7° livello, invocazione
\textbf{Tempo di Lancio}: 1 Azione Immediata
\textbf{Gittata}: 9 metri
\textbf{Componenti}: V
\textbf{Durata}: Istantanea
Pronunci una parola divina, infusa del potere che ha
modellato il mondo all’alba della creazione. Scegli un
qualsiasi numero di creature a gittata e che puoi
vedere. Ogni creatura che può udirti deve effettuare un
tiro salvezza su Arbitrio. Se fallisce il tiro salvezza, la
creatura subisce un effetto in base ai suoi attuali punti
ferita:
- 0 punti ferita o meno: assordata per 1 minuto
- 40 punti ferita o meno: assordata e accecata per 10
minuti
- 30 punti ferita o meno: accecata, assordata e
stordita per 1 ora
- 20 punti ferita o meno: uccisa all’istante
Quali che siano i suoi attuali punti ferita, un celestiale,
elementale, fatato o immondo che fallisca il tiro
salvezza è obbligato a tornare al suo piano di origine
(se non vi si trova già) e non può tornare sul tuo attuale
piano prima che siano passate 24 ore, a meno dell’uso
dell’incantesimo desiderio.
Parola Guaritrice
[healing word\textbf{Difficolta'}:
1° livello, invocazione
\textbf{Tempo di Lancio}: 1 Azione Immediata
\textbf{Gittata}: 18 metri
\textbf{Componenti}: V
\textbf{Durata}: Istantanea
Una creatura a gittata che puoi vedere, scelta da te,
recupera punti ferita pari a 1d4 + il tuo modificatore di
caratteristica da incantatore. Questo incantesimo non
ha effetto su non morti o costrutti.
Ai Livelli Più Alti. Quando lanci questo incantesimo
usando uno slot incantesimo di 2° livello o più alto, la
cura aumenta di 1d4 per ogni livello dello slot sopra il
1°.
Parola Guaritrice di Massa
[mass healing word\textbf{Difficolta'}:
3° livello, invocazione
\textbf{Tempo di Lancio}: 1 Azione Immediata
\textbf{Gittata}: 18 metri
\textbf{Componenti}: V
\textbf{Durata}: Istantanea
Mentre pronunci parole di cura, fino a sei creature a
gittata che puoi vedere, scelte da te, recuperano punti
ferita pari a 1d4 + il tuo modificatore di caratteristica da
incantatore. Questo incantesimo non ha effetto su non
morti o costrutti.
Ai Livelli Più Alti. Quando lanci questo incantesimo
usando uno slot incantesimo di 4° livello o più alto, i
punti ferita recuperati aumentano di 1d4 per ogni livello
dello slot sopra il 3°.
Parola del Potere Stordire
[power word stun\textbf{Difficolta'}:
8° livello, ammaliamento
\textbf{Tempo di Lancio}: 1 Azione Immediata
\textbf{Gittata}: 18 metri
\textbf{Componenti}: V
\textbf{Durata}: Istantanea
Pronunci una parola di potere che può travolgere la
mente di una creatura a gittata e che puoi vedere,
lasciandola confusa. Se il bersaglio ha 150 punti ferita o
meno, è stordito. Altrimenti, l’incantesimo non ha
effetto.
Il bersaglio stordito deve effettuare un tiro salvezza su
Costituzione al termine di ciascun suo turno. Se lo
supera, l’effetto di stordimento ha fine.
Parola del Potere Uccidere
[power word kill\textbf{Difficolta'}:
9° livello, ammaliamento
\textbf{Tempo di Lancio}: 1 Azione Immediata
\textbf{Gittata}: 18 metri
\textbf{Componenti}: V
\textbf{Durata}: Istantanea
Pronunci una parola di potere che costringe a morire
all’istante una creatura a gittata che puoi vedere. Se la
creatura che scegli ha 100 punti ferita o meno, muore.
Altrimenti, l’incantesimo non ha effetto.
Parola del Ritiro
[word of recall\textbf{Difficolta'}:
6° livello, evocazione
\textbf{Tempo di Lancio}: 2 Azioni
\textbf{Gittata}: 1 metro
\textbf{Componenti}: V
\textbf{Durata}: Istantanea
Te e fino a cinque creature consenzienti entro 1 metro
da te vi teletrasportate istantaneamente in un luogo
sicuro indicato precedentemente, detto santuario. Tu e
tutte le creature teletrasportate con te, riapparite nello
spazio non occupato più vicino al punto indicato quando
hai preparato questo santuario (vedi sotto). Se lanci
questo incantesimo senza aver prima preparato un
santuario, l’incantesimo non ha effetto.
Devi indicare un santuario, lanciando l’incantesimo
all’interno del luogo, per esempio un tempio, che sia
dedicato o fortemente collegato alla tua divinità. Se tenti
di lanciare l’incantesimo in questa maniera in un’area
che non sia dedicata alla tua divinità, l’incantesimo non
ha effetto.
Passapareti
[passwall\textbf{Difficolta'}:
5° livello, trasmutazione
\textbf{Tempo di Lancio}: 2 Azioni
\textbf{Gittata}: 9 metri
\textbf{Componenti}: V, S, M (un pizzico di semi di sesamo)
\textbf{Durata}: 1 ora
Per la durata dell’incantesimo, compare un passaggio
in un punto a gittata che puoi vedere, su di una
superficie di legno, calcina o pietra (come una parete,
un soffitto o un pavimento) scelta da te. Scegli le
dimensioni dell’apertura: al massimo larga 1 metro, 
alta 2,4 metri e profonda 6 metri. Il passaggio non crea
instabilità nella struttura che lo circonda.
Quando l’apertura sparisce, qualsiasi creatura od
oggetto ancora nel passaggio creato dall’incantesimo
viene espulso al sicuro nello spazio non occupato più
vicino alla superficie su cui hai lanciato l’incantesimo.
Passare Senza Tracce
[pass without trace\textbf{Difficolta'}:
2° livello, abiurazione
\textbf{Tempo di Lancio}: 2 Azioni
\textbf{Gittata}: Personale
\textbf{Componenti}: V, S, M (ceneri di una foglia di vischio
bruciata e un ramoscello di abete rosso)
\textbf{Durata}: Concentrazione, massimo 1 ora
Un velo d’ombra e silenzio si irradia da te, proteggendo
te e i tuoi compagni dall’essere individuati. Per la durata
dell’incantesimo, ogni creatura a tua scelta entro 9 metri
da te (te compreso) riceve un bonus di +10 alle prove di
Agilita' (Furtività) e le sue tracce non possono
essere seguite eccetto che da mezzi magici. Una
creatura che riceve questo bonus non lascia tracce né
altri segni del suo passaggio.
Passo Velato
[misty step\textbf{Difficolta'}:
2° livello, evocazione
\textbf{Tempo di Lancio}: 1 Azione Immediata
\textbf{Gittata}: Personale
\textbf{Componenti}: V
\textbf{Durata}: Istantanea
Avvolto rapidamente da una foschia argentata, ti
teletrasporti di massimo 9 metri in uno spazio non
occupato che puoi vedere.
Passo Veloce
[longstrider\textbf{Difficolta'}:
1° livello, trasmutazione
\textbf{Tempo di Lancio}: 2 Azioni
\textbf{Gittata}: Contatto
\textbf{Componenti}: V, S, M (un pizzico di terra)
\textbf{Durata}: 1 ora
La velocità di una creatura con cui sei in contatto
aumenta di 3 metri fino al termine dell’incantesimo.
Ai Livelli Più Alti. Quando lanci questo incantesimo
usando uno slot incantesimo di 2° livello o più alto, puoi
prendere come bersaglio un’ulteriore creatura per ogni
livello dello slot sopra il 1°.
Paura
[fear\textbf{Difficolta'}:
3° livello, illusione
\textbf{Tempo di Lancio}: 2 Azioni
\textbf{Gittata}: Personale (cono di 9 metri)
\textbf{Componenti}: V, S, M (una piuma bianca o il cuore di
una gallina)
\textbf{Durata}: Concentrazione, massimo 1 minuto
Proietti un’immagine illusoria delle peggiori paure di una
creatura. Ogni creatura in un cono di 9 metri deve
superare un tiro salvezza su Arbitrio o far cadere
qualsiasi cosa stia impugnando e restare spaventata
per la durata dell’incantesimo.
Mentre è spaventata da questo incantesimo, una
creatura deve, durante ciascun suo turno, effettuare
l’azione Scattare e muoversi lontano da te tramite il
tragitto più sicuro, a meno che non abbia spazio per
muoversi. Se la creatura termina il suo turno in un posto
dove non ha linea di visuale su di te, può effettuare un
tiro salvezza su Arbitrio. Se lo supera, l’incantesimo,
per quella creatura, ha termine.
Pelle di Corteccia
[barkskin\textbf{Difficolta'}:
2° livello, trasmutazione
\textbf{Tempo di Lancio}: 2 Azioni
\textbf{Gittata}: Contatto
\textbf{Componenti}: V, S, M (una manciata di corteccia di
quercia)
\textbf{Durata}: Concentrazione, massimo 1 ora
La pelle del bersaglio con cui sei in contatto al
momento del lancio dell’incantesimo diventa ruvida e
dall’aspetto simile alla corteccia fino al termine
dell’incantesimo, e la CA del bersaglio non può essere
inferiore a 16, quale che sia l’armatura che sta
indossando.
Pelle di Pietra
[stoneskin\textbf{Difficolta'}:
4° livello, abiurazione
\textbf{Tempo di Lancio}: 2 Azioni
\textbf{Gittata}: Contatto
\textbf{Componenti}: V, S, M (polvere di diamante del valore di
100 mo, che l’incantesimo consuma)
\textbf{Durata}: Concentrazione, massimo 1 ora
Lanci l’incantesimo a contatto di una creatura
consenziente, la cui pelle si tramuta in una sostanza
dura come la pietra. Fino alla fine dell’incantesimo il
bersaglio ha resistenza ai danni contundenti, perforanti
e taglienti di natura non magica.
Piaga degli Insetti
[insect plague\textbf{Difficolta'}:
5° livello, evocazione
\textbf{Tempo di Lancio}: 2 Azioni
\textbf{Gittata}: 90 metri
\textbf{Componenti}: V, S, M (qualche granello di zucchero,
qualche chicco di grano, e una passata di lardo)
\textbf{Durata}: Concentrazione, massimo 10 minuti
Uno sciame di locuste affamate riempie una sfera di 6
metri di raggio centrata in un punto a gittata scelto da
te. La sfera si propaga intorno agli angoli. La sfera
rimane per la durata dell’incantesimo, e la sua area è
oscurata leggermente. L’area della sfera è terreno
difficile.
Quando l’area appare, ogni creatura al suo interno deve
effettuare un tiro salvezza su Potenza. Una
creatura subisce 4d10 danni se fallisce il tiro
salvezza, o la metà di questi danni se lo supera. Una
creatura deve effettuare questo tiro salvezza anche
quando entra per la prima volta nell’area
dell’incantesimo durante un turno o se termina il proprio
turno al suo interno.
Ai Livelli Più Alti. Quando lanci questo incantesimo
usando uno slot incantesimo di 6° livello o più alto, il
danno aumenta di 1d10 per ogni livello dello slot sopra
il 5°.
Porta Dimensionale
[dimension door\textbf{Difficolta'}:
4° livello, evocazione
\textbf{Tempo di Lancio}: 2 Azioni
\textbf{Gittata}: 150 metri
\textbf{Componenti}: V
\textbf{Durata}: Istantanea
Ti teletrasporti dalla tua attuale posizione in qualsiasi
altro posto a gittata. Arrivi esattamente nel posto
desiderato. Può essere un luogo che puoi vedere, uno
che puoi visualizzare, o uno che puoi descrivere
indicando distanza e direzione, come “30 metri verso il
basso” o “90 metri in alto a nordovest con un angolo di
45 gradi.”
Puoi portare con te oggetti il cui peso non ecceda la tua
capacità di carico. Puoi portare con te anche una
creatura consenziente della tua taglia o più piccola con
equipaggiamento fino al limite della sua capacità di
carico. La creatura deve essere entro 1 metro da te
quando lanci questo incantesimo.
Se dovessi arrivare in un posto già occupato da un
oggetto o creatura, tu e la creatura che viaggia con te
subite ciascuno 4d6 danni da forza, e l’incantesimo non
riesce a teletrasportarvi.
Portale
[gate\textbf{Difficolta'}:
9° livello, evocazione
\textbf{Tempo di Lancio}: 2 Azioni
\textbf{Gittata}: 18 metri
\textbf{Componenti}: V, S, M (un diamante del valore di
almeno 5.000 mo)
\textbf{Durata}: Concentrazione, massimo 1 minuto
Evochi in uno spazio non occupato a gittata che puoi
vedere un portale collegato a un posto preciso su di un
diverso piano di esistenza. Il portale è un’apertura
circolare creata da te, da 1,5 a 6 metri di diametro. Puoi
orientare il portale in qualsiasi direzione desideri. Il
portale resta per la durata.
Il portale ha un fronte e un dietro su entrambi i piani in
cui compare. Il viaggio attraverso il portale è possibile
solo muovendosi dal fronte. Qualsiasi cosa lo faccia
viene istantaneamente trasportata nell’altro piano,
comparendo nello spazio non occupato più vicino al
portale.
Divinità e altri sovrani planari possono impedire ai
portali creati da incantesimi di aprirsi in loro presenza o
in qualsiasi punto dei loro domini.
Quando lanci questo incantesimo, puoi pronunciare il
nome di una specifica creatura (lo pseudonimo, titolo o
soprannome non funzionano). Se quella creatura si
trova su di un piano diverso dal tuo, il portale si apre in
prossimità della creatura nominata e attira la creatura
attraverso di sé, verso lo spazio non occupato più vicino
dal tuo lato del portale. Non detieni alcun potere
speciale sulla creatura, ed essa è libera di agire come il
Narratore ritiene appropriato. Potrebbe andarsene, attaccarti
o aiutarti.
Preghiera di Guarigione
[prayer of healing\textbf{Difficolta'}:
2° livello, invocazione
\textbf{Tempo di Lancio}: 10 minuti
\textbf{Gittata}: 9 metri
\textbf{Componenti}: V
\textbf{Durata}: Istantanea
Fino a sei creature a gittata che puoi vedere, scelte da
te, recuperano ciascuna punti ferita pari a 2d8 + il tuo
modificatore di caratteristica da incantatore. Questo
incantesimo non ha effetto su non morti o costrutti.
Ai Livelli Più Alti. Quando lanci questo incantesimo
usando uno slot incantesimo di 3° livello o più alto, la
cura aumenta di 1d8 per ogni livello dello slot sopra il
2°.
Presagio
[augury\textbf{Difficolta'}:
2° livello, divinazione (rituale)
\textbf{Tempo di Lancio}: 1 minto
\textbf{Gittata}: Personale
\textbf{Componenti}: V, S, M (dei bastoncini, ossa o simili
oggetti marchiati appositamente e del valore di almeno
25 mo)
\textbf{Durata}: Istantanea
Gettando bastoncini intarsiati con gemme, facendo
rotolare ossa di drago, impilando carte elaborate o
impiegando qualche altro strumento di divinazione,
ricevi un presagio da un’entità ultraterrena riguardo il
risultato di uno specifico corso di azione che intendi
intraprendere nei prossimi 30 minuti. Il Narratore sceglie tra i
seguenti presagi:
- Prosperità, per i risultati positivi
- Calamità, per i risultati negativi
- Prosperità e calamità, per i risultati sia positivi che
negativi
- Nulla, per i risultati che non sono né particolarmente
positivi né negativi
L’incantesimo non tiene conto di ogni possibile
circostanza che possa modificare il risultato, come il
lancio di ulteriori incantesimi o la perdita o l’arrivo di un
alleato.
Se lanci l’incantesimo due o più volte prima di aver
terminato il tuo prossimo riposo lungo, c’è una
probabilità cumulativa del 25% che per ogni lancio dopo
il primo tu ottenga una lettura erronea. Il Narratore effettua
questo tiro in segreto.
Prestidigitazione
[Prestidigitation\textbf{Difficolta'}:
Trucchetto, trasmutazione
\textbf{Tempo di Lancio}: 2 Azioni
\textbf{Gittata}: 3 metri
\textbf{Componenti}: V, S
\textbf{Durata}: Massimo 1 ora
Questo incantesimo è un trucco magico minore che gli
incantatori novizi impiegano per fare pratica. Crei a
gittata uno dei seguenti effetti magici:
- Crei un effetto sensoriale innocuo e istantaneo
come una pioggia di scintille, un soffio di vento, una
debole nota musicale o uno strano odore.
- Illumini o spegni istantaneamente una candela, una
torcia o piccolo fuoco da campo.
- Ripulisci o insozzi istantaneamente un oggetto non
più grosso di 0,03 metri cubi.
- Raffreddi, riscaldi o insapori per 1 ora fino a 0,03
metri cubi di materiale non vivente.
- Fai comparire per 1 ora un colore, un piccolo segno
o un simbolo su di un oggetto o una superficie.
- Crei un ninnolo non magico o un’immagine illusoria
che entri nella tua mano e che resta fino al termine
del tuo prossimo turno.
Se lanci questo incantesimo più volte, puoi tenere attivi
fino a tre effetti non istantanei alla volta, e puoi
interrompere uno di questi effetti con un’azione.
Previsione
[foresight\textbf{Difficolta'}:
9° livello, divinazione
\textbf{Tempo di Lancio}: 1 minuto
\textbf{Gittata}: Contatto
\textbf{Componenti}: V, S, M (una piuma di colibrì)
\textbf{Durata}: 8 ore
Lanci l’incantesimo a contatto di una creatura
consenziente per conferirle una limitata capacità di
vedere nell’immediato futuro. Per la durata, il bersaglio
non può essere sorpreso e ha vantaggio sui tiri per
colpire, prove di caratteristica e tiri salvezza. Inoltre,
sempre per la durata, le altre creature hanno
svantaggio sui tiri per colpire contro il bersaglio.
L’incantesimo ha immediatamente termine se lo lanci di
nuovo prima che la sua durata abbia fine.
Produrre Fiamma
[produce flame\textbf{Difficolta'}:
Trucchetto, evocazione
\textbf{Tempo di Lancio}: 2 Azioni
\textbf{Gittata}: Personale
\textbf{Componenti}: V, S
\textbf{Durata}: 10 minuti
Una fiammella compare nella tua mano. La fiammella
resta lì per la durata dell’incantesimo e non danneggia
né te né il tuo equipaggiamento. La fiamma produce
luce intensa nel raggio di 3 metri e luce fioca per
ulteriori 3 metri. L’incantesimo termina se lo interrompi
con un’azione o se lo lanci di nuovo.
Puoi usare la fiamma anche per attaccare, sebbene
farlo ponga termine all’incantesimo. Quando lanci
questo incantesimo, o con un’azione in un turno
successivo, puoi scagliare la fiamma a una creatura
entro 9 metri da te. Effettua un attacco a distanza con
incantesimo. Se colpisci, il bersaglio subisce 1d8 danni
da fuoco.
Il danno dell’incantesimo aumenta di 1d8 quando arrivi
al 5° livello (2d8), 11° livello (3d8) e 17° livello (4d8).
Proibizione
[forbiddance\textbf{Difficolta'}:
6° livello, abiurazione (rituale)
\textbf{Tempo di Lancio}: 10 minuti
\textbf{Gittata}: Contatto
\textbf{Componenti}: V, S, M (uno spruzzo di acqua sacra,
incensi rari, e un rubino in polvere del valore di 1.000
mo)
\textbf{Durata}: 1 giorno
Crei una interdizione al viaggio magico che protegge
fino a 4.000 metri quadri di pavimento, fino a un’altezza
di 9 metri dal suolo. Per la durata dell’incantesimo, le
creature non possono teletrasportarsi nell’area o usare
passaggi, come quello creato dall’incantesimo portale,
per entrare nell’area. L’incantesimo protegge l’area dal
viaggio planare, e quindi impedisce alle creature di
accedere all’area tramite il Piano Astrale, il Piano
Etereo, le Lande Fatate o il Mondo delle Ombre, o
l’incantesimo spostamento planare.
Inoltre, l’incantesimo danneggia i tipi di creatura scelti
da te durante il lancio. Scegli uno o più dei seguenti:
celestiali, elementali, fatati, immondi e non morti.
Quando una creatura selezionata entra nell’area
dell’incantesimo per la prima volta in un turno o inizia
qui il suo turno, la creatura subisce 5d10 danni radianti
o necrotici (a tua scelta, quando lanci l’incantesimo).
Quando lanci questo incantesimo, puoi stabilire una
parola d’ordine. Una creatura che pronuncia la parola
d’ordine mentre entra nell’area dell’incantesimo, non
subisce danni da esso.
L’area dell’incantesimo non può sovrapporsi all’area di
un altro incantesimo proibizione. Se esegui proibizione
ogni giorno per 30 giorni nello stesso posto,
l’incantesimo durerà finché non viene dissolto, e le
componenti materiali saranno consumate durante
l’ultimo lancio.
Proiezione Astrale
[astral projection\textbf{Difficolta'}:
9° livello, negromanzia
\textbf{Tempo di Lancio}: 2 Azioni
\textbf{Gittata}: 3 metri
\textbf{Componenti}: V, S, M (per ogni creatura soggetta a
questo incantesimo, devi fornire un giacinto del valore
di almeno 1.000 mo e un lingotto d’argento
elegantemente scolpito del valore di almeno 100 mo,
tutti i quali sono consumati dall’incantesimo)
\textbf{Durata}: Speciale
Tu e fino ad altre otto creature consenzienti a gittata
proiettate i vostri corpi astrali nel Piano Astrale
(l’incantesimo fallisce e il lancio è sprecato qualora vi
trovaste già in quel piano). Il corpo materiale che ti lasci
alle spalle è privo di sensi e in uno stato di animazione
sospesa; non ha bisogno di cibo né di acqua e non
invecchia.
Il tuo corpo astrale assomiglia in tutto e per tutto alla tua
forma mortale, replicando le tue statistiche di gioco e i
tuoi oggetti. La principale differenza è l’aggiunta di un
cordone argenteo che si estende dalle scapole per 30
centimetri dietro di te, divenendo poi invisibile. Il
cordone è la tua connessione al tuo corpo materiale.
Finché questa connessione resterà intatta, potrai
tornare a casa. Se il cordone viene tagliato (un
avvenimento che accade solo quando uno specifico
effetto lo indica) la tua anima e corpo vengono separati,
uccidendoti all’istante.
La tua forma astrale può viaggiare liberamente per il
Piano Astrale e attraversare i portali che da lì
conducono ad altri piani. Se entri in un nuovo piano o
ritorni al piano su cui eri al momento del lancio
dell’incantesimo, il tuo corpo e i tuoi oggetti vengono
trasportati lungo il cordone argenteo, permettendoti di
rientrare nel tuo corpo al momento dell’ingresso nel
nuovo piano. La tua forma astrale è una incarnazione
separata. Qualsiasi danno o altro effetto che si applica
a essa, non ha effetto sul tuo corpo fisico, né vi
compare al tuo ritorno.
L’incantesimo ha termine per te e i tuoi compagni
quando userai un’azione per interromperlo. Quando
l’incantesimo termina, la creatura su cui agisce torna al
proprio corpo fisico, e si risveglia.
L’incantesimo potrebbe anche avere una fine anticipata
per te o uno dei tuoi compagni. Un incantesimo dissolvi
magie usato con successo sul corpo astrale o fisico
termina l’incantesimo per quella creatura. Se il corpo
originale della creatura o la sua forma astrale scende a
0 punti ferita, per quella creatura l’incantesimo ha
termine. Se l’incantesimo ha termine e il cordone
argenteo è intatto, il cordone trascina indietro al suo
corpo la forma astrale della creatura, ponendo fine al
suo stato di animazione sospesa.
Se vieni riportato al tuo corpo prematuramente, i tuoi
compagni devono restare nella loro forma astrale e
trovare per proprio conto la via di ritorno ai loro corpi, di
solito scendendo a 0 punti ferita.
Protezione dal Bene e dal Male
[protection from evil and good\textbf{Difficolta'}:
1° livello, abiurazione
\textbf{Tempo di Lancio}: 2 Azioni
\textbf{Gittata}: Contatto
\textbf{Componenti}: V, S, M (acqua sacra o argento e ferro in
polvere, che l’incantesimo consuma)
\textbf{Durata}: Concentrazione, massimo 10 minuti
Fino al termine dell’incantesimo, una creatura
consenziente in contatto con te al momento
dell’esecuzione è protetta da certi tipi di creature:
aberrazioni, celestiali, elementali, fatati, immondi e non
morti.
La protezione conferisce diversi benefici. Le creature di
quei tipi hanno svantaggio ai tiri per colpire contro il
bersaglio. Il bersaglio non può essere affascinato,
spaventato o posseduto da loro. Se il bersaglio è già
affascinato, spaventato o posseduto da una simile
creatura, il bersaglio ha vantaggio su qualsiasi nuovo
tiro salvezza contro l’effetto in questione.
Protezione dall’Energia
[protection from energy\textbf{Difficolta'}:
3° livello, abiurazione
\textbf{Tempo di Lancio}: 2 Azioni
\textbf{Gittata}: Contatto
\textbf{Componenti}: V, S
\textbf{Durata}: Concentrazione, massimo 1 ora
Lanci l’incantesimo a contatto di una creatura
consenziente. Per la durata dell’incantesimo, il
bersaglio ha resistenza a un tipo di danno scelto da te:
acido, freddo, fuoco, fulmine o tuono.
Protezione dai Veleni
[protection from poison\textbf{Difficolta'}:
2° livello, abiurazione
\textbf{Tempo di Lancio}: 2 Azioni
\textbf{Gittata}: Contatto
\textbf{Componenti}: V, S
\textbf{Durata}: 1 ora
Neutralizzi il veleno che agisce su di una creatura
avvelenata con cui sei in contatto. Se più di un veleno
affligge il bersaglio, neutralizzi il veleno che sai essere
presente, o ne neutralizzi uno a caso.
Per la durata dell’incantesimo, il bersaglio ha vantaggio
ai tiri salvezza contro l’essere avvelenato, e ha
resistenza al danno da veleno.
Punizione Marchiante
[branding smite\textbf{Difficolta'}:
2° livello, invocazione
\textbf{Tempo di Lancio}: 1 Azione Immediata
\textbf{Gittata}: Personale
\textbf{Componenti}: V
\textbf{Durata}: Concentrazione, massimo 1 minuto
La prossima volta che colpisci una creatura con un
attacco in mischia con arma nella durata
dell’incantesimo, l’arma riluce di un bagliore astrale
mentre colpisci. L’attacco infligge 2d6 danni radianti
aggiuntivi al bersaglio, che diventa visibile qualora sia
invisibile ed emette luce fioca in un raggio di 1 metro.
Inoltre il bersaglio non può diventare invisibile fino al
termine dell’incantesimo.
Ai Livelli Più Alti. Quando lanci questo incantesimo
usando uno slot incantesimo di 3° livello o più alto, il
danno aggiuntivo aumenta di 1d6 per ogni livello dello
slot sopra il 2°.
Purificare Cibo e Bevande
[purify food and water\textbf{Difficolta'}:
1° livello, trasmutazione (rituale)
\textbf{Tempo di Lancio}: 2 Azioni
\textbf{Gittata}: 3 metri
\textbf{Componenti}: V, S
\textbf{Durata}: Istantanea
Tutti i cibi e le bevande non magiche in una sfera di 1,5
metri di raggio, centrata in un punto a gittata di tua
scelta, vengono purificati e liberati da veleni e malattie.
Raggio di Gelo
[ray of frost\textbf{Difficolta'}:
Trucchetto, invocazione
\textbf{Tempo di Lancio}: 2 Azioni
\textbf{Gittata}: 18 metri
\textbf{Componenti}: V, S
\textbf{Durata}: Istantanea
Un fascio gelato di luce azzurra colpisce una creatura a
gittata. Effettua un attacco a distanza con incantesimo
contro il bersaglio. Se colpisci, egli subisce 1d8 danni
da freddo, e la sua velocità è ridotta di 3 metri fino
all’inizio del tuo prossimo turno.
Il danno dell’incantesimo aumenta di 1d8 quando
raggiungi il 5° livello (2d8), l’11° livello (3d8) e il 17°
livello (4d8).
Raggio di Affaticamento
[ray of enfeeblement\textbf{Difficolta'}:
2° livello, negromanzia
\textbf{Tempo di Lancio}: 2 Azioni
\textbf{Gittata}: 18 metri
\textbf{Componenti}: V, S
\textbf{Durata}: Concentrazione, massimo 1 minuto
Un fascio nero di energia debilitante parte dal tuo dito
diretto contro una creatura a gittata. Effettua un attacco
a distanza con incantesimo contro il bersaglio. Se
colpisci, il bersaglio infliggerà la metà dei danni con gli
attacchi con arma che usano la Forza fino al termine
dell’incantesimo.
Al termine di ciascun turno del bersaglio, questi può
effettuare un tiro salvezza su Potenza contro
l’incantesimo. Se lo supera, l’incantesimo ha termine.
Raggio Rovente
[scorching ray\textbf{Difficolta'}:
2° livello, invocazione
\textbf{Tempo di Lancio}: 2 Azioni
\textbf{Gittata}: 36 metri
\textbf{Componenti}: V, S
\textbf{Durata}: Istantanea
Crei tre raggi di fuoco e li proietti verso tre bersagli a
gittata. Puoi proiettarli contro lo stesso bersaglio o
bersagli diversi.
Effettua un attacco a distanza con incantesimo per
ciascun raggio. Se colpisci, il bersaglio subisce 2d6
danni da fuoco.
Ai Livelli Più Alti. Quando lanci questo incantesimo
usando uno slot incantesimo di 3° livello o più alto, crei
un raggio aggiuntivo per ogni livello dello slot sopra il
2°.
Ragnatela
[web\textbf{Difficolta'}:
2° livello, evocazione
\textbf{Tempo di Lancio}: 2 Azioni
\textbf{Gittata}: 18 metri
\textbf{Componenti}: V, S, M (un pezzo di tela di ragno)
\textbf{Durata}: Concentrazione, massimo 1 ora
Evochi una spessa massa di tela densa e appiccicosa
in un punto a gittata, scelto da te. Per la durata, la
ragnatela riempie un cubo di 6 metri di spigolo da quel
punto. La ragnatela è terreno difficile e rende quell’aera
oscurata leggermente.
Se la tela non è ancorate tra due masse solide (come
pareti o alberi) o stesa lungo un pavimento, parete o
soffitto, la ragnatela evocata crolla su se stessa, e
l’incantesimo termina all’inizio del tuo prossimo turno.
Le tele distese su di una superficie piatta hanno una
profondità di 1 metro.
Ogni creatura che inizia il suo turno nella ragnatela o
che vi entra durante il proprio turno deve effettuare un
tiro salvezza su Agilità. Se lo fallisce, la creatura è
intralciata finché rimane nella ragnatela o finché non si
libera.
Una creatura intralciata dalle ragnatele può usare la
sua azione per effettuare una prova di Forza contro la
DC del tiro salvezza dell’incantesimo. Se la supera, non
è più intralciata.
Le ragnatele sono infiammabili. Qualsiasi cubo di 1,5
metri di spigolo di ragnatela che venga esposto al
fuoco, brucia in 1 round, infliggendo 2d4 danni da fuoco
a qualsiasi creatura che inizi il suo turno in mezzo al
fuoco.
Randello Incantato
[shillelagh\textbf{Difficolta'}:
Trucchetto, trasmutazione
\textbf{Tempo di Lancio}: 1 Azione Immediata
\textbf{Gittata}: Contatto
\textbf{Componenti}: V, S, M (vischio, una foglia di
quadrifoglio, e una randello o bastone da
combattimento)
\textbf{Durata}: 1 minuto
Il legno di un randello o bastone da combattimento che
stai impugnando viene infuso del potere della natura.
Per la durata dell’incantesimo, usando quell’arma puoi
usare la tua caratteristica da incantatore al posto della
Forza per i tiri per colpire e danno da mischia, e il dado
di danno dell’arma diventa un d8. L’arma diventa anche
magica, se già non lo è. L’incantesimo ha termine se lo
lanci di nuovo o se lasci l’arma.
Reggia Meravigliosa
[magnificient mansion\textbf{Difficolta'}:
7° livello, evocazione
\textbf{Tempo di Lancio}: 1 minuto
\textbf{Gittata}: 90 metri
\textbf{Componenti}: V, S, M (un portale in miniatura scolpito
in avorio, un piccolo pezzo di marmo lucido, e un
minuscolo cucchiaio d’argento, ciascuno di questi
oggetti deve essere almeno del valore di 5 mo)
\textbf{Durata}: 24 ore
Entro la gittata, evochi un’abitazione extradimensionale
che rimane per la durata dell’incantesimo. Scegli dove è
posizionato il suo portone d’ingresso. Il portone
d’ingresso emette una lieve luminosità ed è largo 1,5
metri per 3 metri di altezza. Tu e tutte le creature da te
indicate quando hai lanciato l’incantesimo potete
entrare nell’abitazione extradimensionale, fino a quando
il portone resta aperto. Puoi aprire o chiudere il portone
se ti trovi entro 9 metri da esso. Mentre è chiuso, il
portone è invisibile.
Oltre il portone si trova un magnifico ingresso, oltre il
quale si dipanano numerose stanze. L’atmosfera è
pulita, fresca e accogliente.
Puoi creare quanti piani desideri, ma lo spazio non può
eccedere 50 cubi ognuno di 3 metri di spigolo. Il luogo è
ammobiliato e decorato come preferisci. Contiene cibo
sufficiente a soddisfare un banchetto di 9 portate per
100 persone. Uno staff di 100 servitori quasi trasparenti
è al servizio di chiunque vi faccia ingresso. Sta a te
decidere l’aspetto visivo di questi servitori e il loro
abbigliamento. Essi obbediscono assolutamente ai tuoi
ordini. Ogni servitore può svolgere qualsiasi compito un
normale servitore umano possa svolgere, ma non
possono attaccare o effettuare alcuna azione che
potrebbe arrecare direttamente danno a un’altra
creatura. I servitori possono quindi raccogliere oggetti,
pulire, riparare, ripiegare vestiti, accendere fuochi,
servire cibi, versare vini e così via. I servitori possono
recarsi in qualsiasi punto della dimora, ma non possono
uscirne. I mobili e gli altri oggetti creati da questo
incantesimo diventano fumo quando vengono portati
fuori dalla dimora. Quando l’incantesimo termina,
qualsiasi creatura all’interno dello spazio
extradimensionale viene espulsa nello spazio aperto più
vicino all’uscita.
Regressione Mentale
[feeblemind\textbf{Difficolta'}:
8° livello, ammaliamento
\textbf{Tempo di Lancio}: 2 Azioni
\textbf{Gittata}: 45 metri
\textbf{Componenti}: V, S, M (una manciata di sfere di argilla,
cristallo, vetro o minerali)
\textbf{Durata}: Istantanea
Assalti la mente di una creatura a gittata e che puoi
vedere, cercando di frammentarne l’intelletto e la
personalità. Il bersaglio subisce 4d6 danni e
deve effettuare un tiro salvezza su Arbitrio.
Se fallisce il tiro salvezza, i punteggi di Intelletto e
Carisma della creatura scendono a 1. La creatura non
può lanciare incantesimi, attivare oggetti magici,
comprendere linguaggi, o comunicare in alcun modo
comprensibile. La creatura può, tuttavia, identificare i
suoi amici, seguirli e anche proteggerli.
Ogni 30 giorni, la creatura può ripetere il tiro salvezza
contro l’incantesimo. Se lo supera, l’incantesimo ha
termine.
L’incantesimo può essere terminato da ristorare
superiore, guarigione o desiderio.
Reincarnazione
[reincarnate\textbf{Difficolta'}:
5° livello, trasmutazione
\textbf{Tempo di Lancio}: 1 ora
\textbf{Gittata}: Contatto
\textbf{Componenti}: V, S, M (oli e unguenti rari del valore di
almeno 1.000 mo, che l’incantesimo consuma)
\textbf{Durata}: Istantanea
Entri a contatto con un umanoide morto o un frammento
di umanoide morto. Purché la creatura non sia morta da
più di 10 giorni, l’incantesimo gli forma un nuovo corpo
adulto e poi ne richiama l’anima affinché entri nel corpo.
Se l’anima del bersaglio non è libera o consenziente a
farlo, l’incantesimo fallisce.
La magia modella un nuovo corpo, che probabilmente
provocherà un cambio di razza alla creatura. Il Narratore tira
un d100 e consulta la seguente tabella per determinare
quale forma assuma la creatura una volta riportata in
vita, oppure sarà Il Narratore a scegliere la forma.
d100 Razza
01-04 Dragonide
05-13 Nano di collina
14-21 Nano di montagna
22-25 Elfo oscuro
26-34 Elfo alto
35-42 Elfo dei boschi
43-46 Gnomo della foresta
47-52 Gnomo delle rocce
53-56 Mezzelfo
57-60 Mezzorco
61-68 Halfling piede lesto
69-76 Halfling tozzo
77-96 Umano
97-00 Tiefling
La creatura reincarnata ricorda la sua vita e le sue
esperienze passate. Mantiene le capacità che aveva
nella sua forma originale, eccetto il cambiamento della
sua razza originale per la nuova e la conseguente
modifica dei tratti razziali.
Resistenza
[resistance\textbf{Difficolta'}:
Trucchetto, abiurazione
\textbf{Tempo di Lancio}: 2 Azioni
\textbf{Gittata}: Contatto
\textbf{Componenti}: V, S, M (un mantello in miniatura)
\textbf{Durata}: Concentrazione, massimo 1 minuto
Lanci l’incantesimo a contatto con una creatura
consenziente. Una volta prima del termine
dell’incantesimo, il bersaglio può tirare un d4 e
sommare il risultato ottenuto a un tiro salvezza a sua
scelta. Può tirare il dado prima o dopo aver effettuato il
tiro salvezza. Poi l’incantesimo termina.
Respirare Sott’Acqua
[water breathing\textbf{Difficolta'}:
3° livello, trasmutazione
\textbf{Tempo di Lancio}: 2 Azioni (rituale)
\textbf{Gittata}: 9 metri
\textbf{Componenti}: V, S, M (una cannuccia o una pagliuzza)
\textbf{Durata}: 24 ore
Questo incantesimo consente a un massimo di dieci
creature consenzienti a gittata e che puoi vedere, di
respirare sott’acqua fino al termine dell’incantesimo. Le
creature soggette mantengono anche il loro normale
metodo di respirazione.
Resurrezione
[resurrection\textbf{Difficolta'}:
7° livello, negromanzia
\textbf{Tempo di Lancio}: 1 ora
\textbf{Gittata}: Contatto
\textbf{Componenti}: V, S, M (un diamante del valore di
almeno 1.000 mo, che l’incantesimo consuma)
\textbf{Durata}: Istantanea
Lanci l’incantesimo a contatto di una creatura morta da
non più di un secolo, che non è morta di vecchiaia e
che non sia non morta. Se la sua anima è libera e
consenziente, il bersaglio ritornerà in vita con tutti i suoi
punti ferita.
Questo incantesimo neutralizza tutti i veleni e cura le
normali malattie che affliggevano la creatura quando è
morta. Tuttavia non rimuove malattie magiche,
maledizioni e simili; se questi effetti non sono rimossi
prima del lancio dell’incantesimo, affliggeranno il
bersaglio al suo ritorno in vita.
Questo incantesimo chiude tutte le ferite mortali e
ripristina qualsiasi parte del corpo mancante.
Tornare dalla morte è un’ordalia. Il bersaglio subisce
una penalità di -4 a tutti i tiri per colpire, tiri salvezza e
prove di caratteristica. Ogni volta che il bersaglio
termina un riposo lungo, la penalità viene ridotta di 1
finché non scompare.
Lanciare questo incantesimo per riportare in vita una
creatura che è morta da un anno o più ti sfianca. Fino al
termine di un riposo lungo, non potrai più lanciare
incantesimi e avrai svantaggio su tutti i tiri per colpire,
prove di caratteristica e tiri salvezza.
Resurrezione Pura
[true resurrection\textbf{Difficolta'}:
9° livello, trasmutazione
\textbf{Tempo di Lancio}: 1 ora
\textbf{Gittata}: Contatto
\textbf{Componenti}: V, S, M (un po’ di acqua sacra e diamanti
del valore di 25.000 mo, che l’incantesimo consuma)
\textbf{Durata}: Istantanea
Lanci l’incantesimo a contatto di una creatura morta da
non più di 200 anni e che sia morta per qualsiasi motivo
ma non di vecchiaia. Se la sua anima è libera e
consenziente, la creatura ritornerà in vita con tutti i suoi
punti ferita.
Questo incantesimo chiude tutte le ferite, neutralizza
qualsiasi veleno, cura tutte le malattie e rimuove
qualsiasi maledizione che affliggeva la creatura quando
è morta. L’incantesimo rimpiazza gli organi e gli arti
danneggiati.
L’incantesimo può fornire anche un nuovo corpo se
l’originale non esiste più, in qual caso devi pronunciare
il nome della creatura. La creatura riapparirà poi in uno
spazio non occupato di tua scelta, entro 3 metri da te.
Rianimare Morti
[raise dead\textbf{Difficolta'}:
5° livello, negromanzia
\textbf{Tempo di Lancio}: 1 ora
\textbf{Gittata}: Contatto
\textbf{Componenti}: V, S, M (una diamante del valore di
almeno 500 mo, che l’incantesimo consuma)
\textbf{Durata}: Istantanea
Riporti in vita una creatura morta, purché questa non
sia morta da più di 10 giorni. Se l’anima della creatura è
sia consenziente che libera di riunirsi al corpo, la
creatura torna in vita con 1 punto ferita.
Questo incantesimo neutralizza anche qualsiasi veleno
e cura le malattie non magiche che affliggevano la
creatura al momento della morte. Questo incantesimo,
tuttavia, non rimuove le malattie magiche, maledizioni o
simili effetti; se questi non vengono rimossi prima del
lancio dell’incantesimo, riprenderanno a manifestarsi
quando la creatura torna in vita. L’incantesimo non può
riportare in vita una creatura non morta.
Questo incantesimo richiude tutte le ferite mortali, ma
non ripristina le parti del corpo mancanti. Se la creatura
è priva di parti del corpo o organi fondamentali per la
sopravvivenza (la testa, per esempio) l’incantesimo
fallisce automaticamente.
Tornare dalla morte è un’ordalia. Il bersaglio subisce
una penalità di -4 a tutti i tiri per colpire, tiri salvezza e
prove di caratteristica. Ogni volta che il bersaglio
termina un riposo lungo, la penalità viene ridotta di 1
finché non scompare.
Rigenerazione
[regenerate\textbf{Difficolta'}:
7° livello, trasmutazione
\textbf{Tempo di Lancio}: 1 minuto
\textbf{Gittata}: Contatto
\textbf{Componenti}: V, S, M (un rosario e acqua sacra)
\textbf{Durata}: 1 ora
Lanci l’incantesimo a contatto di una creatura per
stimolare la sua capacità di guarigione naturale. Il
bersaglio recupera 4d8 + 15 punti ferita. Per la durata
dell’incantesimo, il bersaglio recupera 1 punto ferita
all’inizio di ciascun suo turno (10 punti ferita al minuto).
Le membra recise del corpo del bersaglio (dita, gambe,
code e così via), se ne ha, vengono ripristinate in 2
minuti. Se hai la parte recisa e la tieni appoggiata al
moncherino, l’incantesimo fa sì che l’arto si ricucia
istantaneamente col moncherino.
Rimuovi Maledizione
[remove curse\textbf{Difficolta'}:
3° livello, abiurazione
\textbf{Tempo di Lancio}: 2 Azioni
\textbf{Gittata}: Contatto
\textbf{Componenti}: V, S
\textbf{Durata}: Istantanea
Tutte le maledizioni che affliggono una creatura o
oggetto a contatto con te, terminano. Se l’oggetto è un
oggetto magico maledetto, la maledizione resta, ma
l’incantesimo infrange la sintonia del proprietario con
l’oggetto così che lo possa rimuovere o gettare.
Rinascita
[revivify\textbf{Difficolta'}:
3° livello, negromanzia
\textbf{Tempo di Lancio}: 2 Azioni
\textbf{Gittata}: Contatto
\textbf{Componenti}: V, S, M (diamante del valore di 300 mo,
che l’incantesimo consuma)
\textbf{Durata}: Istantanea
Una creatura morta nell’ultimo minuto e con cui sei in
contatto, ritorna in vita con 1 punto ferita. Questo
incantesimo non può riportare in vita le persone morte
di vecchiaia, né può ripristinare le parti del corpo
mancanti.
Riparare
[mending\textbf{Difficolta'}:
Trucchetto, trasmutazione
\textbf{Tempo di Lancio}: 1 minuto
\textbf{Gittata}: Contatto
\textbf{Componenti}: V, S, M (due calamite)
\textbf{Durata}: Istantanea
Questo incantesimo ripara una singola rottura o
spaccatura in un oggetto con cui sei a contatto, come
una catenella spezzata, due metà di una chiave rotta,
un mantello lacerato, o un otre che perde. Purché la
rottura o la spaccatura non sia più grande di 30
centimetri in qualsiasi dimensione, sei in grado di
ripararle, senza lasciare traccia dei danni subiti.
Questo incantesimo può riparare fisicamente un
oggetto magico o un costrutto, ma non è in grado di
ripristinare le funzioni magiche di questi oggetti.
Riposo Inviolato
[gentle repose\textbf{Difficolta'}:
2° livello, negromanzia (rituale)
\textbf{Tempo di Lancio}: 2 Azioni
\textbf{Gittata}: Contatto
\textbf{Componenti}: V, S, M (un pizzico di sale e un pezzo di
rame posto su ciascun occhio del cadavere, che
devono restare lì per la durata)
\textbf{Durata}: 10 giorni
Entri a contatto con un cadavere o altri resti. Per la
durata, il bersaglio è protetto dalla putrefazione e non
può diventare non morto.
L’incantesimo, inoltre, estende il limite di tempo per
rianimare il bersaglio dalla morte, dato che i giorni
trascorsi sotto l’influenza di questo incantesimo non 
sono conteggiati nel limite di tempo di incantesimi come
rianimare morti.
Risata Incontenibile
[hideous laughter\textbf{Difficolta'}:
1° livello, ammaliamento
\textbf{Tempo di Lancio}: 2 Azioni
\textbf{Gittata}: 9 metri
\textbf{Componenti}: V, S, M (piccole torte e una piuma che
viene agitata nell’aria)
\textbf{Durata}: Concentrazione, massimo 1 minuto
Una creatura a gittata di tua scelta e che puoi vedere
percepisce tutto come tremendamente ilare e
divertente, scoppiando in fragorose risate finché è
soggetta a questo incantesimo. Il bersaglio deve
superare un tiro salvezza su Arbitrio o cadere prono,
restando inabile e incapace di rialzarsi per la durata. Le
creature con un punteggio di Intelletto 4 o meno,
ignorano l’effetto.
Al termine di ciascun suo turno, e ogni volta che
subisce danni, il bersaglio può effettuare un altro tiro
salvezza su Saggezza. Il bersaglio ha vantaggio al tiro
salvezza se questo è stato provocato dai danni. Se lo
supera, l’incantesimo termina.
Riscaldare il Metallo
[heat metal\textbf{Difficolta'}:
2° livello, trasmutazione
\textbf{Tempo di Lancio}: 2 Azioni
\textbf{Gittata}: 18 metri
\textbf{Componenti}: V, S, M (un pezzo di ferro e una fiamma)
\textbf{Durata}: Concentrazione, massimo 1 minuto
Scegli un manufatto di metallo, come un’arma di
metallo o un’armatura di metallo media o pesante, a
gittata e che puoi vedere. Fai sì che l’oggetto risplenda
di rosso per il calore. Qualsiasi creatura in contatto
fisico con l’oggetto subisce 2d8 danni da fuoco quando
lanci questo incantesimo. Fino al termine
dell’incantesimo, puoi usare un’azione bonus per
infliggere di nuovo questo danno nei tuoi turni
successivi.
Se una creatura sta impugnando o indossando l’oggetto
e subisce danno da esso, la creatura deve superare un
tiro salvezza su Potenza o gettare l’oggetto se ne è
in grado. Se non getta l’oggetto, ha svantaggio ai tiri per
colpire e le prove di caratteristica fino all’inizio del suo
prossimo turno.
Ai Livelli Più Alti. Quando lanci questo incantesimo
usando uno slot incantesimo di 3° livello o più alto, il
danno aumenta di 1d8 per ogni livello dello slot sopra il
2°.
Ristorare Inferiore
[lesser restoration\textbf{Difficolta'}:
2° livello, abiurazione
\textbf{Tempo di Lancio}: 2 Azioni
\textbf{Gittata}: Contatto
\textbf{Componenti}: V, S
\textbf{Durata}: Istantanea
Puoi porre fine a una malattia o condizione che affligge
una creatura con cui sei a contatto. La condizione può
essere accecato, assordato, avvelenato o paralizzato.
Ristorare Superiore
[greater restoration\textbf{Difficolta'}:
5° livello, abiurazione
\textbf{Tempo di Lancio}: 2 Azioni
\textbf{Gittata}: Contatto
\textbf{Componenti}: V, S, M (polvere di diamante del valore di
almeno 100 mo, che l’incantesimo consuma)
\textbf{Durata}: Istantanea
Imbevi una creatura a contatto di energia positiva per
annullare un effetto debilitante. Puoi ridurre il livello di
sfinimento del bersaglio di uno, o terminare uno dei
seguenti effetti che affliggono il bersaglio:
- Un effetto che ha affascinato o pietrificato il
bersaglio
- Una maledizione, compresa la sintonia di un
bersaglio con un oggetto maledetto
- Qualsiasi riduzione di uno dei punteggi di
caratteristica del bersaglio
- Un effetto che riduce i punti ferita massimi del
bersaglio
Risveglio
[awaken\textbf{Difficolta'}:
5° livello, trasmutazione
\textbf{Tempo di Lancio}: 8 ore
\textbf{Gittata}: Contatto
\textbf{Componenti}: V, S, M (un’agata del valore di almeno
1.000 mo, che l’incantesimo consuma)
\textbf{Durata}: Istantanea
Dopo aver trascorso il tempo di lancio a disegnare
tracciati magici con una gemma preziosa, entri a
contatto con una bestia o vegetale Enorme o di taglia
inferiore. Il bersaglio deve essere privo di punteggio di
Intelletto o avere Intelletto 3 o meno. Il bersaglio
ottiene Intelletto 10. Il bersaglio ottiene anche la
capacità di parlare un linguaggio che conosci. Se il
bersaglio è un vegetale, ottiene la capacità di muovere i
suoi arti, radici, liane, rampicanti e così via, e ottiene
sensi simili a quelli di un umano. Il Narratore sceglierà le
statistiche appropriate al tipo di vegetale risvegliato,
come le statistiche per il cespuglio risvegliato o l’albero
risvegliato.
La bestia o vegetale risvegliato è affascinato da te per
30 giorni o finché tu o i tuoi compagni non gli
arrecherete danno. Quando la condizione affascinato
termina, la creatura risvegliata sceglie se rimanerti
amichevole, in base a come l’hai trattata mentre era
affascinata.
Ritirata Rapida
[expeditious retreat\textbf{Difficolta'}:
1° livello, trasmutazione
\textbf{Tempo di Lancio}: 1 Azione Immediata
\textbf{Gittata}: Personale
\textbf{Componenti}: V, S
\textbf{Durata}: Concentrazione, massimo 10 minuti
Questo incantesimo ti permette di muoverti a
un’andatura incredibile. Quando lanci questo
incantesimo, e poi con un’azione bonus durante
ciascun tuo turno fino al termine dell’incantesimo, puoi
effettuare l’azione Scattare.
Saltare
[jump\textbf{Difficolta'}:
1° livello, abiurazione
\textbf{Tempo di Lancio}: 2 Azioni
\textbf{Gittata}: Contatto
\textbf{Componenti}: V, S, M (la zampa posteriore di una
cavalletta)
\textbf{Durata}: 1 minuto
La distanza di salto della creatura con cui sei in contatto
al momento del lancio è triplicata fino al termine
dell’incantesimo.
Salvare i Morenti
[spare the dying\textbf{Difficolta'}:
Trucchetto, necromazia
\textbf{Tempo di Lancio}: 2 Azioni
\textbf{Gittata}: Contatto
\textbf{Componenti}: V, S
\textbf{Durata}: Istantanea
Una creatura a 0 punti ferita, con cui sei a contatto,
diventa stabile. L’incantesimo non ha effetto su non
morti o costrutti.
Santificare
[hallow\textbf{Difficolta'}:
5° livello, invocazione
\textbf{Tempo di Lancio}: 24 ore
\textbf{Gittata}: Contatto
\textbf{Componenti}: V, S, M (erbe, oli e incensi del valore di
almeno 1.000 mo, che l’incantesimo consuma)
\textbf{Durata}: Fino a che dissolto
Infondi l’area circostante a un punto con cui sei in
contatto di potere sacro (o blasfemo). L’area può avere
un raggio massimo di 18 metri, e l’incantesimo fallisce
se include un’area già sotto l’effetto di un incantesimo
santificare. L’area soggetta all’incantesimo genera i
seguenti effetti.
Per prima cosa, celestiali, elementali, fatati, immondi e
non morti non possono entrare nell’area, né una simile
creatura può affascinare, spaventare o possederne
altre al suo interno. Qualsiasi creatura affascinata,
spaventata o posseduta da una creatura simile non è
più affascinata, spaventata o posseduta dal momento in
cui entra in quest’area. Puoi escludere uno o più tipi di
queste creature da questo effetto.
Seconda cosa, puoi vincolare un effetto ulteriore
all’area. Scegli l’effetto dalla lista seguente, o scegline
uno presentatoti dal Narratore. Alcuni di questi effetti si
applicano alle creature nell’area; puoi decidere se gli
effetti si applichino a tutte le creature, le creature che
seguono una specifica divinità o capo, o le creature di
un tipo specifico, come orchi o troll. Quando una
creatura soggetta all’incantesimo entra in quest’area
per la prima volta durante un turno o inizia il suo turno
qui, deve effettuare un tiro salvezza su Arbitrio. Se lo
supera, la creatura ignora l’effetto aggiuntivo finché non
lascia l’area.
Coraggio. Le creature soggette non possono essere
spaventate mentre restano in quest’area.
Interferenza Extradimensionale. Le creature soggette
non possono muoversi o viaggiare usando il
teletrasporto o altri mezzi extradimensionali o
interplanari.
Lingue. Le creature soggette possono comunicare con
qualsiasi altra creatura nell’area, anche se non
condividono un linguaggio comune.
Luce Diurna. Luce intensa riempie l’area. L’oscurità
magica creata da incantesimi di livello più basso dello
slot usato per lanciare questo incantesimo non possono
estinguere la luce.
Oscurità. L’oscurità riempie l’area. La luce normale, e
anche la luce magica creata da incantesimi di livello più
basso dello slot usato per lanciare questo incantesimo,
non possono illuminare l’area.
Paura. Le creature soggette sono spaventate mentre
restano in quest’area.
Protezione dall’Energia. Le creature soggette
ricevono resistenza a un tipo di danno a tua scelta (a
eccezione dei danni contundenti, perforanti o taglienti),
finché restano nell’area.
Riposo Inviolato. I corpi morti seppelliti nell’area non
possono essere trasformati in non morti.
Silenzio. Nessun suono può emanare dall’interno
dell’area, e nessun suono può entrarvi.
Vulnerabilità all’Energia. Le creature soggette
ricevono vulnerabilità a un tipo di danno a tua scelta (a
eccezione dei danni contundenti, perforanti o taglienti),
finché restano nell’area.
Santuario
[sanctuary\textbf{Difficolta'}:
1° livello, abiurazione
\textbf{Tempo di Lancio}: 1 Azione Immediata
\textbf{Gittata}: 9 metri
\textbf{Componenti}: V, S, M (un piccolo specchio d’argento)
\textbf{Durata}: 1 minuto
Proteggi una creatura a gittata dagli attacchi. Fino al
termine dell’incantesimo, qualsiasi creatura che prenda
come bersaglio la creatura protetta con un attacco o
incantesimo dannoso deve prima effettuare un tiro
salvezza su Saggezza. Se fallisce il tiro salvezza,
l’attaccante deve scegliere un nuovo bersaglio o
perdere l’attacco o l’incantesimo. Questo incantesimo
non protegge la creatura protetta dagli effetti ad area,
come lo scoppio di una palla di fuoco.
Se la creatura protetta effettua un attacco o lancia un
incantesimo che agisce su creature nemiche,
l’incantesimo termina.
Santuario Privato
[private sanctum\textbf{Difficolta'}:
4° livello, abiurazione
\textbf{Tempo di Lancio}: 10 minuti
\textbf{Gittata}: 36 metri
\textbf{Componenti}: V, S, M (un sottile foglio di piombo, un
pezzo di vetro opaco, un batuffolo di cotone o tessuto, e
crisolito in polvere)
\textbf{Durata}: 24 ore
Proteggi con la magia un’area. L’area è un cubo che
può essere piccolo fino a 1 metro di spigolo o grande
fino a 30 metri di spigolo. L’incantesimo agisce fino al
termine della durata o finché non usi un’azione per
interromperlo.
Quando lanci l’incantesimo, decidi che tipo di
protezione questo fornisce, scegliendo una o più delle
seguenti proprietà:
- Il suono non può attraversare il perimetro dell’area
protetta.
- Il perimetro dell’area protetta appare buio e
nebbioso, impedendo di vedervi attraverso (anche
alla scurovisione).
- Sensori creati da incantesimi di divinazione non
possono apparire all’interno dell’area protetta o
attraversare la sua barriera perimetrale.
- Le creature nell’area non possono essere bersaglio
di incantesimi di divinazione.
- Nulla può teletrasportarsi dentro o fuori dell’area
protetta.
- All’interno dell’area protetta, il viaggio planare è
interdetto.
Lanciare questo incantesimo sullo stesso punto ogni
giorno per un anno, rende l’effetto permanente.
Ai Livelli Più Alti. Quando lanci questo incantesimo
usando uno slot incantesimo di 5° livello o più alto, puoi
aumentare le dimensioni del cubo di 30 metri di spigolo
per ogni livello dello slot sopra il 4°. Quindi, usando uno
slot incantesimo di 5° livello, potresti proteggere un
cubo fino a 60 metri di spigolo.
Scagliare Maledizione
[bestow curse\textbf{Difficolta'}:
3° livello, negromanzia
\textbf{Tempo di Lancio}: 2 Azioni
\textbf{Gittata}: Contatto
\textbf{Componenti}: V, S
\textbf{Durata}: Concentrazione, massimo 1 minuto
Una creatura con cui sei a contatto deve superare un
tiro salvezza su Arbitrio o restare maledetta per la
durata dell’incantesimo. Quando lanci questo
incantesimo, scegli la natura della maledizione tra le
seguenti opzioni:
- Scegli un punteggio di caratteristica. Mentre è
maledetto, il bersaglio ha -1d6 alle prove di
caratteristica e i tiri salvezza basati su quel
punteggio di caratteristica.
- Mentre è maledetto, il bersaglio ha svantaggio sui
tiri per colpire contro di te.
- Mentre è maledetto, il bersaglio deve effettuare un
tiro salvezza su Arbitrio all’inizio di ciascun suo
turno. Se lo fallisce, spreca l’azione di quel suo
turno senza fare nulla.
- Mentre il bersaglio è maledetto, i tuoi attacchi e
incantesimi infliggono 1d8 danni necrotici aggiuntivi
contro di lui.
L’incantesimo rimuovi maledizione termina questo
effetto. A discrezione del Narratore, puoi scegliere una
maledizione dall’effetto diverso, ma non dovrebbe
essere comunque più potente di quelle descritte qui
sopra. Il Narratore detiene il giudizio finale sull’effetto di una
maledizione.
Ai Livelli Più Alti. Quando lanci questo incantesimo
usando uno slot incantesimo di 4° livello, la durata
diventa concentrazione, massimo 10 minuti. Se usi uno
slot incantesimo di 5° livello o più alto, la durata diventa
8 ore. Se usi uno slot incantesimo di 7° livello o più alto,
la durata diventa 24 ore. Se usi uno slot incantesimo di
9° livello, la durata diventa permanente finché non
dissolta. Utilizzare uno slot incantesimo di 5° livello o
più alto fornisce una durata che non richiede
concentrazione.
Scassinare
[knock\textbf{Difficolta'}:
2° livello, trasmutazione
\textbf{Tempo di Lancio}: 2 Azioni
\textbf{Gittata}: 18 metri
\textbf{Componenti}: V
\textbf{Durata}: Istantanea
Scegli un oggetto a gittata e che puoi vedere. L’oggetto
può essere una porta, scatola, delle manette, una
serratura o un altro oggetto che possieda un metodo
comune o magico per prevenirne l’accesso.
Un bersaglio che è chiuso da una serratura comune o
che è bloccato o sbarrato viene aperto, sbloccato o
liberato. Se l’oggetto ha più serrature, solo una di
queste viene aperta.
Se scegli un bersaglio che è tenuto chiuso con
serratura arcana, quell’incantesimo resta soppresso per
10 minuti, durante i quali il bersaglio può essere aperto
come di norma.
Quando lanci questo incantesimo, un sonoro bussare,
udibile fino a 90 metri di distanza, emana dall’oggetto
bersaglio.
Sciame di Meteore
[meteor swarm\textbf{Difficolta'}:
9° livello, invocazione
\textbf{Tempo di Lancio}: 2 Azioni
\textbf{Gittata}: 1,5 chilometri
\textbf{Componenti}: V, S
\textbf{Durata}: Istantanea
Sfere incandescenti di fuoco si schiantano a terra in
quattro punti differenti a gittata e che puoi vedere. Ogni
creatura, in una sfera di 2 metri di raggio centrata sul
punto scelto da te, deve effettuare un tiro salvezza su
Agilita'. La sfera si propaga intorno agli angoli. Una
creatura subisce 20d6 danni da fuoco e 20d6 danni
contundenti se fallisce il tiro salvezza, o la metà di
questi danni se lo supera. Una creatura nell’area di più
di uno scoppio infuocato ne subisce gli effetti solo una
volta.
L’incantesimo danneggia gli oggetti nell’area e incendia
gli oggetti infiammabili che non sono indossati o
trasportati.
Scolpire Pietra
[stone shape\textbf{Difficolta'}:
4° livello, trasmutazione
\textbf{Tempo di Lancio}: 2 Azioni
\textbf{Gittata}: Contatto
\textbf{Componenti}: V, S, M (argilla malleabile, che deve
essere lavorata per ottenere una vaga forma
dell’oggetto di pietra)
\textbf{Durata}: Istantanea
Scolpisci in qualsiasi forma che si presti ai tuoi scopi un
oggetto di pietra di taglia Media o inferiore o una
sezione di pietra non più grossa di 1 metro in qualsiasi
direzione, con cui sei in contatto.
Così, per esempio, potresti scolpire una grossa pietra in
un’arma, idolo o feretro, o creare un piccolo passaggio
attraverso il muro, purché il muro sia spesso meno di
1 metro. Potresti anche modellare una porta di pietra o
la sua cornice per sigillare la porta. L’oggetto che crei
può avere fino a due cardini e un chiavistello, ma è
impossibile creare meccanismi più complessi.
Scopri il Percorso
[find the path\textbf{Difficolta'}:
6° livello, divinazione
\textbf{Tempo di Lancio}: 1 minuto
\textbf{Gittata}: Personale
\textbf{Componenti}: V, S, M (degli attrezzi da divinazione -
dei bastoncini d’avorio, ossa, carte, denti o rune incise -
del valore di almeno 100 mo e un oggetto dal luogo che
desideri trovare)
\textbf{Durata}: Concentrazione, massimo 1 giorno
Questo incantesimo ti permette di trovare la rotta fisica
più breve e diretta verso uno specifico luogo fisso con
cui hai familiarità ed è sullo stesso piano di esistenza.
Se indichi una destinazione su di un altro piano di
esistenza, una destinazione che si muove (come una 
fortezza mobile) o una destinazione non specifica
(come “la tana di un drago verde”), l’incantesimo
fallisce.
Per la durata dell’incantesimo, finché sei nello stesso
piano di esistenza della destinazione, saprai quanto è
distante e in che direzione si trovi. Mentre sei in viaggio
verso di essa, ogni volta che ti si presenterà la
possibilità di scegliere tra percorsi diversi, determinerai
automaticamente qual è la via più breve e la rotta più
diretta (ma non necessariamente la più sicura) per
raggiungere la destinazione.
Scopri Trappole
[find traps\textbf{Difficolta'}:
2° livello, divinazione
\textbf{Tempo di Lancio}: 2 Azioni
\textbf{Gittata}: 36 metri
\textbf{Componenti}: V, S
\textbf{Durata}: Istantanea
Avverti la presenza di qualsiasi trappola a gittata che
sia nella tua linea di visuale. Una trappola, ai fini di
questo incantesimo, comprende qualsiasi cosa che sia
in grado di infliggere un effetto improvviso o inaspettato
che tu possa considerare dannoso o indesiderabile, e
che è stato espressamente inteso come tale dal suo
creatore. Di conseguenza, l’incantesimo percepirebbe
un’area sotto l’incantesimo allarme, un glifo di
interdizione, o una botola meccanica, ma non
rivelerebbe una debolezza naturale del pavimento, un
soffitto instabile o una buca nascosta.
L’incantesimo si limita a rivelare la presenza delle
trappole. Non apprendi la posizione delle trappole, ma
apprendi la natura generica del pericolo posto dalle
trappole che hai percepito.
Scrigno Segreto
[secret chest\textbf{Difficolta'}:
4° livello, evocazione
\textbf{Tempo di Lancio}: 2 Azioni
\textbf{Gittata}: Contatto
\textbf{Componenti}: V, S, M (un forziere lavorato, 1 metro x
50 cm x 50 cm, costruito con rari materiali del valore di
almeno 5.000 mo, e una sua replica Minuscola fatta
degli stessi materiali e del valore di almeno 50)
\textbf{Durata}: Istantanea
Nascondi un forziere e tutti i suoi contenuti sul Piano
Etereo. Quando lanci questo incantesimo devi essere in
contatto con il forziere e la replica in miniatura che
serve da componente materiale. Il forziere può
contenere fino a 0,25 metri cubi di materiale non
vivente (1 x metro x 50 centimetri x 50 centimetri).
Mentre il forziere rimane sul Piano Etereo, puoi usare
un’azione per entrare in contatto con la replica e
richiamare il forziere. Esso riapparirà in uno spazio non
occupato sul terreno entro 1 metro da te. Puoi
rispedire il forziere nel Piano Etereo, usando un’azione
ed entrando in contatto sia col forziere che con la
replica.
Dopo 60 giorni, c’è una percentuale cumulativa del 5%
al giorno che l’effetto dell’incantesimo abbia termine.
L’effetto termina se l’incantesimo viene lanciato
nuovamente, se la replica del forziere viene distrutta, o
se decidi di terminare l’incantesimo con un’azione. Se
l’incantesimo termina e il forziere si trova sul Piano
Etereo, viene irrimediabilmente perduto.
Scritto Illusorio
[illusory script\textbf{Difficolta'}:
1° livello, illusione (rituale)
\textbf{Tempo di Lancio}: 1 minuto
\textbf{Gittata}: Contatto
\textbf{Componenti}: S, M (un inchiostro a base di piombo del
valore di almeno 10 mo, che l’incantesimo consuma)
\textbf{Durata}: 10 giorni
Scrivi su di una pergamena, un pezzo di carta o
qualche altro materiale adatto a scrivere e lo infondi di
una potente illusione che permane per la durata
dell’incantesimo.
Per te e qualsiasi creatura da te indicata al lancio
dell’incantesimo, la scritta appare normale, con la tua
grafia, e trasmette qualsiasi significato volevi
comunicare quando hai vergato il testo. Per tutti gli altri,
la scritta appare come se fosse redatta in una scrittura
ignota o magica, che risulta incomprensibile. In
alternativa, puoi far sì che la scritta sembri un
messaggio totalmente diverso, in una grafia e
linguaggio differente, sebbene debba essere un
linguaggio a te conosciuto.
In caso l’incantesimo venisse dissolto, sia la scritta
originale che l’illusione svaniscono.
Una creatura con visione del vero può leggere il
messaggio nascosto.
Scrutare
[scrying\textbf{Difficolta'}:
5° livello, divinazione
\textbf{Tempo di Lancio}: 10 minuti
\textbf{Gittata}: Personale
\textbf{Componenti}: V, S, M (un focus del valore di almeno
1.000 mo, come una sfera di cristallo, un specchio
d’argento o una fonte ricolma di acqua sacra)
\textbf{Durata}: Concentrazione, massimo 10 minuti
Puoi vedere e udire una particolare creatura a tua
scelta che si trovi sul tuo stesso piano di esistenza. Il
bersaglio deve effettuare un tiro salvezza su Arbitrio,
modificato da quanto bene conosci il bersaglio e la tua
connessione fisica a esso. Se il bersaglio sa che stai
lanciando l’incantesimo, può fallire volontariamente il
tiro salvezza, in caso desiderasse essere osservato da
te.
Modificatore al
Conoscenza Tiro Salvezza
Di seconda mano (ne hai sentito parlare) +5
Di prima mano (hai incontrato il bersaglio) +0
Familiare (conosci bene il bersaglio) -5
Modificatore al
Connessione Tiro Salvezza
Descrizione o immagine -2
Proprietà o indumento -4
Parte del corpo, ciocca di capelli, pezzo d’unghia, -10
o simile
Se supera il tiro salvezza, il bersaglio ignora gli effetti
dell’incantesimo, e non potrai usare di nuovo questo
incantesimo contro di lui prima che siano passate 24
ore.
Se il tiro salvezza fallisce, l’incantesimo crea un
sensore invisibile entro 3 metri dal bersaglio. Tramite il
sensore puoi udire e vedere come se fossi sul posto. Il
sensore si muove assieme al bersaglio, rimanendo
entro 3 metri da lui per la durata dell’incantesimo. Una
creatura che può vedere oggetti invisibili vede il 
sensore come una sfera luminosa delle dimensioni
all’incirca di un pugno.
Invece di prendere come bersaglio una creatura, puoi
scegliere come bersaglio dell’incantesimo un luogo che
hai già visto in passato. Quando scegli questa opzione,
il sensore compare in quel luogo ma non si muove.
Scudo
[shield\textbf{Difficolta'}:
1° livello, abiurazione
\textbf{Tempo di Lancio}: 1 reazione, che effettui quando sei
colpito da un attacco o bersaglio dell’incantesimo dardo
incantato
\textbf{Gittata}: Personale
\textbf{Componenti}: V, S
\textbf{Durata}: 1 round
Compare una barriera di forza magica invisibile a
proteggerti. Fino all’inizio del tuo prossimo turno hai un
bonus di +5 alla CA, compreso l’attacco innescante, e
non subisci danni da dardo incantato.
Scudo della Fede
[shield of faith\textbf{Difficolta'}:
1° livello, abiurazione
\textbf{Tempo di Lancio}: 1 Azione Immediata
\textbf{Gittata}: 18 metri
\textbf{Componenti}: V, S, M (una piccola pergamena con su
scritto un frammento di testo sacro)
\textbf{Durata}: Concentrazione, massimo 10 minuti
Compare un campo scintillante che circonda una
creatura a gittata, scelta da te, conferendole un bonus
di +2 alla CA per la durata dell’incantesimo.
Scudo di Fuoco
[fire shield\textbf{Difficolta'}:
4° livello, invocazione
\textbf{Tempo di Lancio}: 2 Azioni
\textbf{Gittata}: Personale
\textbf{Componenti}: V, S, M (un po’ di fosforo o una lucciola)
\textbf{Durata}: 10 minuti
Fiamme sottili e vaporose avvolgono il tuo corpo per la
durata dell’incantesimo, emettendo luce intensa in un
raggio di 3 metri e luce fioca per ulteriori 3 metri. Puoi
terminare l’incantesimo in anticipo, usando un’azione
per interromperlo.
Le fiamme ti forniscono uno scudo caldo o uno scudo
freddo, a tua scelta. Lo scudo caldo ti conferisce
resistenza al danno da freddo, mentre lo scudo freddo ti
fornisce resistenza al danno da caldo.
Inoltre, ogni qualvolta una creatura entro 1 metro da te
ti colpisce con un attacco in mischia, lo scudo erutta
fiamme. L’attaccante subisce 2d8 danni da fuoco da
uno scudo caldo, o 2d8 danni da freddo da uno scudo
freddo.
Scurovisione
[darkvision\textbf{Difficolta'}:
2° livello, trasmutazione
\textbf{Tempo di Lancio}: 2 Azioni
\textbf{Gittata}: Contatto
\textbf{Componenti}: V, S, M (o un pizzico di carota o di agata
secca)
\textbf{Durata}: 8 ore
Una creatura consenziente con cui sei in contatto
ottiene la capacità di vedere al buio. Per la durata
dell’incantesimo, quella creatura ha scurovisione fino a
una gittata di 18 metri.
Segugio Fedele
[faithful hound\textbf{Difficolta'}:
4° livello, evocazione
\textbf{Tempo di Lancio}: 2 Azioni
\textbf{Gittata}: 9 metri
\textbf{Componenti}: V, S, M (un minuscolo fischietto
d’argento, e un pezzo d’osso, e un filo)
\textbf{Durata}: 8 ore
Puoi evocare un cane da guardia fantasma in uno
spazio non occupato a gittata e che puoi vedere, dove
rimarrà per la durata dell’incantesimo, finché non viene
congedato con un’azione, o finché non si allontanerà
più di 30 metri da te.
Il segugio è invisibile a tutte le creature eccetto che a te
e non può essere danneggiato. Quando una creatura di
taglia Piccola o superiore si avvicina entro 9 metri da
esso senza aver prima pronunciato la parola d’ordine
da te specificata quando hai lanciato l’incantesimo, il
segugio inizia ad abbaiare a grande volume. Il segugio
vede le creature invisibili e può vedere nel Piano
Etereo. Esso ignora le illusioni.
All’inizio di ciascun tuo turno, il segugio tenta di
mordere una creatura entro 1 metro da esso e che ti
sia ostile. Il bonus di attacco del segugio è uguale al tuo
modificatore di caratteristica da incantatore + il tuo
bonus di competenza. Se colpisce, infligge 4d8 danni
perforanti.
Sembrare
[seeming\textbf{Difficolta'}:
5° livello, illusione
\textbf{Tempo di Lancio}: 2 Azioni
\textbf{Gittata}: 9 metri
\textbf{Componenti}: V, S
\textbf{Durata}: 8 ore
Questo incantesimo ti permette di cambiare l’aspetto di
un qualsiasi numero di creature a gittata e che puoi
vedere. Fornisci a ciascun bersaglio un nuovo aspetto
illusorio. Una creatura non consenziente può effettuare
un tiro salvezza su Arbitrio e, se lo supera, ignora
l’incantesimo.
L’incantesimo camuffa l’aspetto fisico oltre che gli abiti,
le armature, le armi e l’equipaggiamento. Puoi far
sembrare ciascuna creatura 30 centimetri più bassa o
più alta, sembrare magra, grassa o una via di mezzo.
Non puoi cambiare la conformazione del corpo del
bersaglio, e quindi devi scegliere una forma che abbia
la stessa distribuzione basilare di arti. Per tutto il resto,
l’illusione è limitata solo dalla tua fantasia.
L’incantesimo permane per la sua durata, a meno che
tu non usi una tua azione per interromperlo prima.
I cambi apportati da questo incantesimo non sono in
grado di sostenere un’ispezione fisica. Per esempio, se
usi questo incantesimo per aggiungere un cappello
all’abbigliamento di una creatura, gli oggetti
attraversano il cappello, e chiunque lo tocchi non
avvertirebbe nulla e finirebbe per toccare la testa e i
capelli della creatura. Se usi questo incantesimo per
apparire più magro di quello che sei, la mano di una
persona che provasse a toccarti rimbalzerebbe su di te,
mentre alla vista sembrerebbe fermarsi a mezz’aria.
Una creatura può usare la sua azione per ispezionare
un bersaglio ed effettuare una prova di Intelletto
(Indagare) contro la DC del tiro salvezza 
dell’incantesimo. Se la riesce, capisce che il bersaglio è
camuffato.
Semipiano
[demiplane\textbf{Difficolta'}:
8° livello, evocazione
\textbf{Tempo di Lancio}: 2 Azioni
\textbf{Gittata}: 18 metri
\textbf{Componenti}: S
\textbf{Durata}: 1 ora
Crei una porta d’ombra su di una superficie piana a
gittata e che puoi vedere. La porta è grande
abbastanza da permettere il passaggio senza problemi
a una creatura Media. Quando viene aperta, la porta
conduce a un semipiano che appare come una stanza
vuota di 9 metri in ciascuna dimensione, fatta di legno e
pietra. Quando l’incantesimo termina, la porta
scompare, e qualsiasi creatura od oggetto all’interno del
semipiano rimane intrappolato lì, mentre la porta
scompare anche dall’altro lato.
Ogni volta che esegui questo incantesimo, crei un
nuovo semipiano, oppure permetti alla porta d’ombra di
connettersi a un semipiano creato da un precedente
lancio dell’incantesimo. Inoltre, se conosci la natura e i
contenuti di un semipiano creato dal lancio di questo
incantesimo da parte di un’altra creatura, puoi far sì che
la porta d’ombra si colleghi invece a quel semipiano.
Serratura Arcana
[arcane lock\textbf{Difficolta'}:
2° livello, abiurazione
\textbf{Tempo di Lancio}: 2 Azioni
\textbf{Gittata}: Contatto
\textbf{Componenti}: V, S, M (polvere d’oro del valore di
almeno 25 mo, che viene consumata dall’incantesimo)
\textbf{Durata}: Fino a che dissolto
Lanci l’incantesimo a contatto di una porta, finestra,
portale, forziere o altro ingresso chiuso, e questo
diventa chiuso a chiave per la durata. Tu e le creature
che hai indicato, quando hai lanciato questo
incantesimo, potete aprire l’oggetto normalmente. Puoi
anche predisporre una parola d’ordine che, quando
pronunciata entro 1 metro dall’oggetto, sopprime
l’incantesimo per 1 minuto. Altrimenti l’apertura è
invalicabile fino a che non viene distrutta o
l’incantesimo è dissolto o soppresso. Lanciare
scassinare sull’oggetto sopprime serratura arcana per
10 minuti.
Mentre è soggetto a questo incantesimo, l’oggetto è più
difficile da distruggere o aprire a forza; la DC per
romperlo o scassinare una serratura su di esso
aumenta di 10.
Servitore Inosservato
[unseen servant\textbf{Difficolta'}:
1° livello, evocazione (rituale)
\textbf{Tempo di Lancio}: 2 Azioni
\textbf{Gittata}: 18 metri
\textbf{Componenti}: V, S, M (un pezzo di corda e un pezzo di
legno)
\textbf{Durata}: 1 ora
Questo incantesimo crea una forza invisibile, senza
cervello e informe, che svolge dei semplici compiti al
tuo comando, fino al termine dell’incantesimo. Il
servitore si forma in uno spazio sul terreno non
occupato, entro la gittata. Ha CA 10, 1 punto ferita,
Forza 2 e non può attaccare. Se scende a 0 punti ferita,
l’incantesimo ha termine.
Come azione bonus, durante ciascun tuo turno, puoi
comandare mentalmente il servitore di muoversi fino a
4,5 metri e interagire con un oggetto. Il servitore può
svolgere dei semplici compiti alla stregua di un servitore
umano, come raccogliere cose, pulire, riparare, piegare
abiti, accendere fuochi, servire il cibo e versare il vino.
Una volta impartito il comando, il servitore svolgerà il
compito al meglio delle sue capacità finché non l’avrà
completato, e poi aspetterà il tuo prossimo comando.
Se comandi al servitore di svolgere un compito che lo
farà muovere a più di 18 metri da te, l’incantesimo
termina.
Sfera Congelante
[freezing sphere\textbf{Difficolta'}:
6° livello, invocazione
\textbf{Tempo di Lancio}: 2 Azioni
\textbf{Gittata}: 90 metri
\textbf{Componenti}: V, S, M (una piccola sfera di cristallo)
\textbf{Durata}: Istantanea
Un globo gelido di energia fredda parte dalla punta
delle tue dita verso un punto di tua scelta a gittata, dove
esplode in una sfera di 18 metri di raggio. Ogni creatura
nell’area deve effettuare un tiro salvezza su
Costituzione. Se fallisce il tiro salvezza, una creatura
subisce 10d6 danni da freddo. Se lo supera, subisce la
metà di questi danni.
Se il globo colpisce un corpo d’acqua o un liquido
composto principalmente d’acqua (escluse però le
creature a base d’acqua), congela il liquido fino a una
profondità di 15 centimetri in un’area quadrata di 9 metri
di lato. Il ghiaccio dura 1 minuto. Le creature che
stavano nuotando sulla superficie dell’acqua congelata
restano intrappolate nel ghiaccio. Una creatura
intrappolata può usare un’azione per effettuare una
prova di Forza contro la DC del tiro salvezza
dell’incantesimo, al fine di liberarsi.
Se lo desideri, dopo aver completato l’incantesimo, puoi
trattenerti dallo sparare il globo. Un piccolo globo, circa
delle dimensioni di una pietra da fionda, freddo al
contatto, appare nella tua mano. In qualsiasi momento,
tu, o una creatura a cui hai dato il globo, potete lanciare
il globo (fino a una gittata di 12 metri). Questo si
frantumerà all’impatto, con lo stesso effetto del normale
lancio dell’incantesimo. Puoi anche poggiare il globo a
terra senza che si frantumi. Dopo 1 minuto, se il globo
non è già stato frantumato, esploderà.
Ai Livelli Più Alti. Quando lanci questo incantesimo
usando uno slot incantesimo di 7° livello o più alto, il
danno aumenta di 1d6 per ogni livello dello slot sopra il
6°.
Sfera Elastica
[resilient sphere\textbf{Difficolta'}:
4° livello, invocazione
\textbf{Tempo di Lancio}: 2 Azioni
\textbf{Gittata}: 90 metri
\textbf{Componenti}: V, S, M (un pezzo semisferico di cristallo
trasparente e un pezzo semisferico corrispondente di
gomma arabica)
\textbf{Durata}: Concentrazione, massimo 1 minuto
Una sfera di energia luminosa avvolge una creatura od
oggetto di taglia Grande o inferiore a gittata. Una
creatura non consenziente deve effettuare un tiro 
salvezza su Agilita'. Se lo fallisce, la creatura è
avvolta dall’incantesimo per la sua durata.
Nulla (né oggetti fisici, né energia, né altri effetti di
incantesimi) può attraversare questa barriera, in entrata
o uscita, sebbene una creatura all’interno della sfera
possa respirare senza problemi. La sfera è immune a
tutti i danni, e una creatura al suo interno non può
essere danneggiata da attacchi o effetti originanti
dall’esterno, né una creatura all’interno della sfera può
danneggiare nulla che si trovi all’esterno.
La sfera è priva di peso e grande giusto a sufficienza
per contenere la creatura o l’oggetto al suo interno. Una
creatura avvolta può usare la sua azione per spingere
contro le pareti della sfera e quindi farla rotolare fino
alla metà della velocità della creatura. Allo stesso
modo, il globo può essere raccolto e mosso da altre
creature.
Un incantesimo disintegrazione che prenda come
bersaglio il globo lo distrugge senza danneggiare nulla
al suo interno.
Sfera Infuocata
[flaming sphere\textbf{Difficolta'}:
2° livello, evocazione
\textbf{Tempo di Lancio}: 2 Azioni
\textbf{Gittata}: 18 metri
\textbf{Componenti}: V, S, M (un po’ di sego, un pizzico di
zolfo, e una manciata di ferro in polvere)
\textbf{Durata}: Concentrazione, massimo 1 minuto
Per la durata dell’incantesimo compare una sfera di 1,5
metri di diametro in uno spazio a gittata, scelto da te.
Qualsiasi creatura che termini il suo turno entro 1,5
metri dalla sfera deve effettuare un tiro salvezza su
Agilita'. La creatura subisce 2d6 danni da fuoco se
fallisce il tiro salvezza, o la metà di questi danni se lo
supera.
Con un’azione bonus, puoi spostare la sfera di 9 metri.
Se fai schiantare la sfera contro una creatura, la
creatura deve effettuare un tiro salvezza contro il danno
della sfera, e la sfera smetterà di muoversi per quel
turno.
Quando muovi la sfera, la puoi spostare oltre barriere
alte fino a 1 metro, e farle saltare spazi larghi fino a 3
metri. La sfera incendia gli oggetti infiammabili non
indossati o trasportati, e irradia una luce intensa in un
raggio di 6 metri e una luce fioca per ulteriori 6 metri.
Ai Livelli Più Alti. Quando lanci questo incantesimo
usando uno slot incantesimo di 3° livello o più alto, il
danno aumenta di 1d6 per ogni livello dello slot sopra il
2°.
Sfocatura
[blur\textbf{Difficolta'}:
2° livello, illusione
\textbf{Tempo di Lancio}: 2 Azioni
\textbf{Gittata}: Personale
\textbf{Componenti}: V
\textbf{Durata}: Concentrazione, massimo 1 minuto
Il tuo corpo diventa sfocato, indistinto e tremolante a
chiunque ti veda. Per la durata dell’incantesimo, tutte le
creature hanno svantaggio sui tiri per colpire contro di
te. Gli attaccanti che non si affidano alla vista sono
immuni a questo effetto, per esempio se hanno vista
cieca o sono in grado di distinguere le illusioni, come
per visione del vero.
Sguardo Penetrante
[eyebite\textbf{Difficolta'}:
6° livello, negromanzia
\textbf{Tempo di Lancio}: 2 Azioni
\textbf{Gittata}: Personale
\textbf{Componenti}: V, S
\textbf{Durata}: Concentrazione, massimo 1 minuto
Per la durata dell’incantesimo, i tuoi occhi si tramutano
in un vuoto nero infuso di terribile potere. Una creatura
a tua scelta entro 18 metri da te e che puoi vedere,
deve superare un tiro salvezza su Arbitrio o, per la
durata, subire uno dei seguenti effetti di tua scelta.
Durante ciascun tuo turno, fino al termine
dell’incantesimo, puoi usare la tua azione per prendere
come bersaglio un’altra creatura, ma non puoi prendere
di nuovo come bersaglio una creatura che abbia
superato un tiro salvezza contro questo lancio di
sguardo penetrante.
Addormentato. Il bersaglio cade privo di sensi. Si
risveglia qualora subisca qualsiasi ammontare di danno
o se un’altra creatura usa la sua azione per scuoterlo
dal sonno.
Ammalato. Il bersaglio ha svantaggio sui tiri per colpire
e le prove di caratteristica. Al termine di ciascun suo
turno, può effettuare un altro tiro salvezza su Arbitrio.
Se lo supera, l’effetto ha termine.
Impanicato. Il bersaglio è spaventato da te. Durante
ciascun suo turno, la creatura spaventata deve
effettuare l’azione Scattare e muoversi lontano da te
tramite il tragitto più breve e sicuro possibile, a meno
che non abbia spazio per muoversi. Se il bersaglio si
muove in un luogo lontano almeno 18 metri da te, dove
non ti possa vedere, questo effetto ha termine.
Silenzio
[silence\textbf{Difficolta'}:
2° livello, illusione (rituale)
\textbf{Tempo di Lancio}: 2 Azioni
\textbf{Gittata}: 36 metri
\textbf{Componenti}: V, S
\textbf{Durata}: Concentrazione, massimo 10 minuti
Per la durata dell’incantesimo, nessun suono può
essere creato all’interno o attraversare una sfera di 6
metri di raggio centrata su di un punto a gittata, scelto
da te. Qualsiasi creatura o oggetto che si trovi
completamente all’interno della sfera è immune al
danno da tuono, e le creature che sono completamente
al suo interno sono assordate. È impossibile lanciare un
incantesimo che comprende una componente verbale
mentre si è al suo interno.
Simbolo
[symbol\textbf{Difficolta'}:
7° livello, abiurazione
\textbf{Tempo di Lancio}: 2 Azioni
\textbf{Gittata}: Contatto
\textbf{Componenti}: V, S, M (mercurio, fosforo e diamante e
opale in polvere con un valore totale di almeno 1.000
mo, che l’incantesimo consuma)
\textbf{Durata}: Fino a che dissolto o attivato
Quando lanci questo incantesimo, inscrivi un glifo
dannoso su di una superficie (come una sezione di
pavimento, muro o un tavolo) o all’interno di un oggetto
che può essere chiuso per nascondere il glifo (come un
libro, una pergamena o un forziere). Se scegli una
superficie, il glifo può coprire un’area di superficie non
maggiore di 3 metri di diametro. Se scegli un oggetto,
quell’oggetto deve restare al suo posto; se l’oggetto 
viene spostato più di 3 metri dal punto in cui è stato
lanciato l’incantesimo, il glifo è spezzato, e
l’incantesimo termina senza essere stato attivato.
Il glifo è quasi invisibile e può essere trovato con una
prova di Intelletto (Indagare) contro la DC del tiro
salvezza dei tuoi incantesimi.
Decidi tu cosa attivi il glifo al momento del lancio
dell’incantesimo.
Per i glifi inscritti su di una superficie, l’attivazione tipica
comprende entrare in contatto o stare sopra il glifo,
rimuovere un altro oggetto che copra il glifo, avvicinarsi
a una certa distanza dal glifo, o manipolare l’oggetto su
cui è inscritto il glifo.
Per i glifi inscritti su di un oggetto, l’attivazione tipica
comprende aprire l’oggetto, avvicinarsi a una certa
distanza dall’oggetto, o vedere o leggere il glifo.
Puoi definire meglio l’attivazione così che l’incantesimo
si attivi solo in determinate circostanze o secondo certe
peculiarità fisiche (come l’altezza o il peso) o specie di
creatura (per esempio, la protezione potrebbe agire
contro le megere o i mutaforma). Puoi anche
predisporre condizioni per evitare che il glifo venga
attivato, come la pronuncia di una parola d’ordine.
Quando inscrivi il glifo scegli una delle opzioni seguenti
come suo effetto. Una volta attivato, il glifo riluce,
riempiendo una sfera di 18 metri di raggio di luce fioca
per 10 minuti, dopo i quali l’incantesimo termina. Ogni
creatura nella sfera quando il glifo si attiva diventa
bersaglio del suo effetto, così come una creatura che
entri per la prima volta nella sfera durante un turno o
termine lì il suo turno.
Demenza. Ogni bersaglio deve effettuare un tiro
salvezza su Intelletto. Se fallisce il tiro salvezza, il
bersaglio diventa demente per 1 minuto. Una creatura
demente non può effettuare azioni, non comprende
quello che gli altri le dicono, non può leggere, e parla
solo farfugliando. Il Narratore ne controlla i movimenti, che
risultano erratici.
Discordia. Ogni bersaglio deve effettuare un tiro
salvezza su Costituzione. Se lo fallisce, il bersaglio
inizia a bisticciare e discutere con un’altra creatura per
1 minuto. In questo periodo, è incapace di effettuare
qualsiasi comunicazione significativa e ha svantaggio ai
tiri per colpire e le prove di caratteristica.
Dolore. Ogni bersaglio deve effettuare un tiro salvezza
su Costituzione. Se lo fallisce, il bersaglio diventa
inabile a causa del dolore lacerante.
Morte. Ogni bersaglio deve effettuare un tiro salvezza
su Costituzione, subendo 10d10 danni necrotici se lo
fallisce, o la metà di questi danni se lo supera.
Paura. Ogni bersaglio deve effettuare un tiro salvezza
su Saggezza e, se lo fallisce, restare spaventato per 1
minuto. Mentre è spaventato, il bersaglio getta qualsiasi
cosa stesse tenendo e deve muoversi almeno 9 metri
lontano dal glifo durante ciascuno suo turno, se in
grado.
Sfiducia. Ogni bersaglio deve effettuare un tiro
salvezza su Carisma. Se fallisce il tiro salvezza, il
bersaglio è sopraffatto dalla disperazione per 1 minuto.
Durante questo periodo, non può attaccare o prendere
come bersaglio nessuna creatura con capacità,
incantesimi o altri effetti magici nocivi.
Sonno. Ogni bersaglio deve effettuare un tiro salvezza
su Saggezza, e cadere privo di sensi per 10 minuti se lo
fallisce. Una creatura si risveglia se subisce danni o se
qualcuno usa un’azione per risvegliarla.
Stordimento. Ogni bersaglio deve effettuare un tiro
salvezza su Saggezza, e restare stordito per 1 minuto
se lo fallisce.
Simulacro
[simulacrum\textbf{Difficolta'}:
7° livello, illusione
\textbf{Tempo di Lancio}: 12 ore
\textbf{Gittata}: Contatto
\textbf{Componenti}: V, S, M (neve o ghiaccio in quantità per
creare una copia a dimensioni reali della creatura
duplicata; un po’ di capelli, unghie o altro pezzo del
corpo di quella creatura da piazzare in mezzo alla neve
o al ghiaccio; e un rubino in polvere del valore di 1.500
mo, sparso sopra il duplicato e consumato
dall’incantesimo)
\textbf{Durata}: Fino a che dissolto
Modelli un duplicato illusorio di una bestia o umanoide
che resti a gittata per l’intero tempo di lancio
dell’incantesimo. Il duplicato è una creatura, in parte
reale e formata di ghiaccio o neve, che può effettuare
azioni e interagire come una normale creatura. Sembra
essere identica all’originale, ma ha la metà dei punti
ferita massimi di quella creatura e si presenta priva di
equipaggiamento. Altrimenti, l’illusione usa tutte le
statistiche della creatura che duplica.
Il simulacro è amichevole verso di te e le creature da te
indicate. Obbedisce ai comandi da te pronunciati,
muovendosi e agendo in accordo ai tuoi desideri e
agendo durante il tuo turno in combattimento. Il
simulacro è privo della capacità di apprendere o
diventare più potente, e quindi non accresce mai di
livello o nelle caratteristiche, né può recuperare gli slot
incantesimi spesi.
Se il simulacro è danneggiato, puoi ripararlo in un
laboratorio alchemico, usando erbe rare e minerali del
valore di 100 mo per punto ferita recuperato. Il
simulacro rimane finché non scende a 0 punti ferita, a
quel punto si ritrasforma in neve e si scioglie all’istante.
Se lanci di nuovo questo incantesimo, qualsiasi
duplicato da te creato con questo incantesimo
attualmente attivo viene immediatamente distrutto.
Sogno
[dream\textbf{Difficolta'}:
5° livello, illusione
\textbf{Tempo di Lancio}: 2 Azioni
\textbf{Gittata}: Speciale
\textbf{Componenti}: V, S, M (una manciata di sabbia, una
punta di inchiostro, e una penna per scrivere presa da
un volatile addormentato)
\textbf{Durata}: 8 ore
Questo incantesimo modella i sogni di una creatura.
Scegli una creatura a te nota come bersaglio
dell’incantesimo. Il bersaglio deve trovarsi sul tuo
stesso piano di esistenza. Le creature che non
dormono, come gli elfi, non possono essere soggette a
questo incantesimo. Tu o una creatura consenziente
con cui sei a contatto entrate in uno stato di trance,
agendo da messaggero. Mentre è in trance, il
messaggero è consapevole di ciò che lo circonda, ma
non può effettuare azioni o muoversi.
Per la durata dell’incantesimo, se il bersaglio è
addormentato, il messaggero appare nei sogni del
bersaglio e può conversare con lui finché questi rimane
addormentato. Il messaggero può anche modellare
l’ambiente del sogno, creando terreni, oggetti e altre
immagini. Il messaggero può emergere dalla trance in
qualsiasi momento, terminando anticipatamente l’effetto
dell’incantesimo. Al risveglio, il bersaglio ricorda
perfettamente il suo sogno. Se il bersaglio è sveglio
quando lanci l’incantesimo, il messaggero ne viene a
conoscenza e può porre fine alla trance (e
all’incantesimo) o aspettare che il bersaglio si
addormenti. A quel punto il messaggero potrà
comparire nei sogni del bersaglio.
Puoi fare apparire il messaggero al bersaglio con un
aspetto mostruoso e terrificante. Se lo fai, il
messaggero può consegnare un messaggio di al
massimo dieci parole e poi il bersaglio deve effettuare
un tiro salvezza su Arbitrio. Se fallisce il tiro
salvezza, gli echi della spaventosa mostruosità
generano un incubo per la durata del sonno del
bersaglio che gli impedisce di ottenere qualsiasi
beneficio da quel riposo. Inoltre, quando il bersaglio si
sveglia, subisce 3d6 danni.
Se possiedi una ciocca di capelli, delle unghie tagliate,
o simile porzione del corpo del bersaglio, egli effettuerà
il suo tiro salvezza con svantaggio.
Sonno
[sleep\textbf{Difficolta'}:
1° livello, ammaliamento
\textbf{Tempo di Lancio}: 2 Azioni
\textbf{Gittata}: 27 metri
\textbf{Componenti}: V, S, M (un pizzico di sabbia, petali di
rosa o un grillo)
\textbf{Durata}: 1 minuto
Questo incantesimo pone le creature in un torpore
magico. Tira 5d8; il totale è il numero di punti ferita di
creature su cui l’incantesimo può agire. Le creature,
entro 6 metri dal punto a gittata scelto da te, sono
influenzate in ordine ascendente di punti ferita
(ignorando le creature svenute).
A partire dalla creatura con il numero più basso di punti
ferita attuali, ogni creatura soggetta a questo
incantesimo perde i sensi fino al termine
dell’incantesimo, chi dorme subisce danni, o qualcuno
usa un’azione per scuotere o prendere a schiaffi
l’addormentato. Sottrarre i punti ferita di ciascuna
creatura dal totale prima di considerare la creatura con
il valore di punti ferita più basso successiva. I punti
ferita di una creatura devono essere uguali o inferiori al
totale rimanente perché l’effetto agisca su di essa.
Non morti e creature che non possono essere
affascinate non sono influenzate da questo
incantesimo.
Ai Livelli Più Alti. Quando lanci questo incantesimo
usando uno slot incantesimo di 2° livello o più alto, tira
2d8 aggiuntivi per ogni slot incantesimo sopra il 1°.
Spada Arcana
[arcane sword\textbf{Difficolta'}:
7° livello, invocazione
\textbf{Tempo di Lancio}: 2 Azioni
\textbf{Gittata}: 18 metri
\textbf{Componenti}: V, S, M (una spada di platino in miniatura
con l’impugnatura e il pomello di rame e zinco, del
valore di 250 mo)
\textbf{Durata}: Concentrazione, massimo 1 minuto
Per la durata dell’incantesimo, crei a gittata un piano di
forza a forma di spada fluttuante.
Quando la spada appare, effettui un attacco in mischia
con incantesimo contro un bersaglio scelto da te entro
1 metro dalla spada. Se colpisci, il bersaglio subisce
3d10 danni da forza. Fino al termine dell’incantesimo,
puoi usare un’azione bonus ogni tuo turno per muovere
la spada di 6 metri in un punto che puoi vedere e
ripetere questo attacco contro lo stesso bersaglio o uno
differente.
Spostamento Planare
[plane shift\textbf{Difficolta'}:
7° livello, evocazione
\textbf{Tempo di Lancio}: 2 Azioni
\textbf{Gittata}: Contatto
\textbf{Componenti}: V, S, M (una verga di metallo biforcuta
del valore di almeno 250 mo, sintonizzata verso uno
specifico piano di esistenza)
\textbf{Durata}: Istantanea
Tu e un massimo di altre otto creature consenzienti,
che si stringono le mani per formare un cerchio, venite
trasportati su di un diverso piano di esistenza. Puoi
specificare una destinazione bersaglio in termini
generici, e riapparirai all’interno o in prossimità di quella
destinazione, a discrezione del Narratore.
In alternativa, se conosci la sequenza di sigilli di un
cerchio di teletrasporto verso un altro piano di
esistenza, l’incantesimo può condurti a quel cerchio. Se
il cerchio di teletrasporto è troppo piccolo per contenere
tutte le creature che trasporti con te, esse riappariranno
nello spazio non occupato più vicino possibile al
cerchio.
Puoi usare questo incantesimo per bandire una
creatura non consenziente in un altro piano. Scegli una
creatura a portata ed effettua un attacco in mischia con
incantesimo contro di essa. Se colpisci, la creatura
deve effettuare un tiro salvezza su Arbitrio. Se la
creatura fallisce il tiro salvezza, viene trasportata in un
luogo casuale sul piano di esistenza da te specificato.
Una creatura così trasportata dovrà trovare per proprio
conto il modo di tornare sul tuo attuale piano di
esistenza.
Spruzzo Colorato
[color spray\textbf{Difficolta'}:
1° livello, illusione
\textbf{Tempo di Lancio}: 2 Azioni
\textbf{Gittata}: Personale (cono di 4,5 metri)
\textbf{Componenti}: V, S, M (un pizzico di polvere o sabbia
che sia colorata di rosso, giallo e blu)
\textbf{Durata}: 1 round
Dalla tua mano si sprigiona una raffica di luci colorate e
abbaglianti. Tira 6d10; il totale è l’ammontare di punti
ferita di creature su cui questo incantesimo agisce. Le
creature, in un cono di 4,5 metri che origina da te, sono
soggette in ordine ascendente dei loro attuali punti
ferita (ignorando le creature prive di sensi e le creature
che non possono vedere).
A partire dalla creatura che ha il minor numero di punti
ferita attuali, ciascuna creatura soggetta a questo
incantesimo resta accecata fino al termine
dell’incantesimo. Sottrarre i punti ferita di ciascuna
creatura dal totale prima di passare alla creatura col
totale più basso di punti ferita successiva. I punti ferita
di una creatura devono essere uguali o minori del totale
rimanente perché l’incantesimo agisca su di essa.
Ai Livelli Più Alti. Quando lanci questo incantesimo
usando uno slot incantesimo di 2° livello o più alto, tira
2d10 aggiuntivi per ogni livello dello slot sopra il 1°.
Spruzzo Prismatico
[prismatic spray\textbf{Difficolta'}:
7° livello, invocazione
\textbf{Tempo di Lancio}: 2 Azioni
\textbf{Gittata}: Personale (cono di 18 metri)
\textbf{Componenti}: V, S
\textbf{Durata}: Istantanea
Otto raggi di luce multicolore partono dalla tua mano.
Ogni raggio è di un diverso colore e ha un potere e uno
scopo diverso. Ogni creatura in un cono di 18 metri
deve effettuare un tiro salvezza su Agilità. Per ogni
bersaglio, tirare un d8 per determinare quale sia il
colore del raggio che lo ha colpito.
1. Rosso. Il bersaglio subisce 10d6 danni da fuoco se
fallisce il tiro salvezza, o la metà di questi danni se lo
supera.
2. Arancio. Il bersaglio subisce 10d6 danni da acido se
fallisce il tiro salvezza, o la metà di questi danni se lo
supera.
3. Giallo. Il bersaglio subisce 10d6 danni da fulmine se
fallisce il tiro salvezza, o la metà di questi danni se lo
supera.
4. Verde. Il bersaglio subisce 10d6 danni da veleno se
fallisce il tiro salvezza, o la metà di questi danni se lo
supera.
5. Blu. Il bersaglio subisce 10d6 danni da freddo se
fallisce il tiro salvezza, o la metà di questi danni se lo
supera.
6. Indaco. Se fallisce il tiro salvezza, il bersaglio è
intralciato. Deve poi effettuare un tiro salvezza su
Costituzione all’inizio di ciascun suo turno. Se supera il
tiro salvezza tre volte, l’incantesimo termina. Se fallisce
il tiro salvezza tre volte, viene permanentemente
trasformato in pietra e diventa vittima della condizione
pietrificato. I successi e i fallimenti non devono essere
consecutivi; tieni traccia di entrambi finché il bersaglio
non ne ha ottenuti tre dello stesso tipo.
7. Violetto. Se fallisce il tiro salvezza, il bersaglio è
accecato. Deve poi effettuare un tiro salvezza su
Saggezza all’inizio del tuo prossimo turno. Se supera il
tiro salvezza, la cecità termina. Se fallisce il tiro
salvezza, la creatura viene trasportata su di un altro
piano di esistenza a scelta del Narratore e non è più accecata
(di solito, una creatura che non è sul suo piano natio,
viene esiliata su di esso, mentre le altre creature sono
di solito portate nei piani Astrale o Etereo).
8. Speciale. Il bersaglio è colpito da due raggi. Tira
altre due volte, ritirando gli 8.
Spruzzo Velenoso
[poison spray\textbf{Difficolta'}:
Trucchetto, evocazione
\textbf{Tempo di Lancio}: 2 Azioni
\textbf{Gittata}: 3 metri
\textbf{Componenti}: V, S
\textbf{Durata}: Istantanea
Stendi la mano verso una creatura a gittata e che puoi
vedere, e proietti una nube di gas velenoso dal tuo
palmo. La creatura deve superare un tiro salvezza su
Costituzione o subire 1d12 danni da veleno.
Il danno dell’incantesimo aumenta di 1d12 quando
raggiungi il 5° livello (2d12), 11° livello (3d12) e 17°
livello (4d12).
Stretta Folgorante
[shocking grasp\textbf{Difficolta'}:
Trucchetto, invocazione
\textbf{Tempo di Lancio}: 2 Azioni
\textbf{Gittata}: Contatto
\textbf{Componenti}: V, S
\textbf{Durata}: Istantanea
Dalle tue mani saettano lampi che infliggono una
scossa a una creatura con cui provi a entrare in
contatto. Effettua un attacco in mischia con incantesimo
contro il bersaglio. Hai vantaggio sul tiro per colpire se il
bersaglio sta indossando un’armatura fatta di metallo.
Se colpisci, il bersaglio subisce 1d8 danni da fulmine, e
non può effettuare reazioni fino all’inizio del suo
prossimo turno.
Il danno dell’incantesimo aumenta di 1d8 quando
raggiungi il 5° livello (2d8), l’11° livello (3d8) e il 17°
livello (4d8).
Suggestione
[suggestion\textbf{Difficolta'}:
2° livello, ammaliamento
\textbf{Tempo di Lancio}: 2 Azioni
\textbf{Gittata}: 9 metri
\textbf{Componenti}: V, M (la lingua di un serpente e un pezzo
di favo o un goccio di olio dolce)
\textbf{Durata}: Concentrazione, massimo 8 ore
Suggerisci un corso di attività (limitato a una o due
frasi) e influenzi magicamente una creatura a gittata e
che puoi vedere e udire e ti possa capire, scelta da te.
Le creature che non possono essere affascinate sono
immuni a questo effetto. La suggestione deve essere
pronunciata in modo che il corso d’azione suoni
ragionevole. Chiedere a una creatura di pugnalarsi,
gettarsi su di una lancia, darsi fuoco, o fare qualche
altro atto palesemente dannoso nega automaticamente
gli effetti dell’incantesimo.
Il bersaglio deve effettuare un tiro salvezza su
Saggezza. Se fallisce il tiro salvezza, esso segue il
corso d’azione da te descritto al meglio delle sue
capacità. Il corso d’azione suggerito può proseguire per
l’intera durata dell’incantesimo. Se l’attività suggerita
può essere completata in un tempo più breve,
l’incantesimo ha termine quando il soggetto termina di
fare ciò che gli è stato chiesto.
Puoi anche specificare condizioni che attiveranno
un’attività speciale per la durata dell’incantesimo. Per
esempio, potresti suggerire a un cavaliere di cedere il
suo cavallo da guerra al primo mendicante che incontri.
Se la condizione non viene soddisfatta prima del
termine dell’incantesimo, l’attività non verrà svolta.
Se tu o uno qualsiasi dei tuoi compagni danneggia il
bersaglio, l’incantesimo ha termine.
Suggestione di Massa
[mass suggestion\textbf{Difficolta'}:
6° livello, ammaliamento
\textbf{Tempo di Lancio}: 2 Azioni
\textbf{Gittata}: 18 metri
\textbf{Componenti}: V, M (la lingua di un serpente e un pezzo
di favo o un goccio di olio dolce)
\textbf{Durata}: 24 ore
Suggerisci un corso di attività (limitato a una o due
frasi) e influenzi magicamente fino a dodici creature a
gittata che puoi vedere e udire e ti possano capire,
scelte da te. Le creature che non possono essere
affascinate sono immuni a questo effetto. La
suggestione deve essere pronunciata in modo che il
corso d’azione suoni ragionevole. Chiedere a una
creatura di pugnalarsi, gettarsi su di una lancia, darsi
fuoco, o fare qualche altro atto palesemente dannoso
nega automaticamente gli effetti dell’incantesimo.
Ogni bersaglio deve effettuare un tiro salvezza su
Saggezza. Se fallisce il tiro salvezza, esso segue il
corso d’azione da te descritto al meglio delle sue
capacità. Il corso d’azione suggerito può proseguire per
l’intera durata dell’incantesimo. Se l’attività suggerita
può essere completata in un tempo più breve,
l’incantesimo ha termine quando il soggetto termina di
fare ciò che gli è stato chiesto.
Puoi anche specificare condizioni che attiveranno
un’attività speciale per la durata dell’incantesimo. Per
esempio, potresti suggerire a un gruppo di soldati di
cedere tutti i loro soldi al primo mendicante che
incontrino. Se la condizione non viene soddisfatta prima
del termine dell’incantesimo, l’attività non verrà svolta.
Se tu o uno qualsiasi dei tuoi compagni danneggia una
creatura soggetta a questo incantesimo, per quella
creatura l’incantesimo ha termine.
Ai Livelli Più Alti. Quando lanci questo incantesimo
usando uno slot incantesimo di 7° livello o più alto, la
durata è 10 giorni. Quando usi uno slot incantesimo di
9° livello, la durata è un anno e un giorno.
Taumaturgia
[thaumaturgy\textbf{Difficolta'}:
Trucchetto, trasmutazione
\textbf{Tempo di Lancio}: 2 Azioni
\textbf{Gittata}: 9 metri
\textbf{Componenti}: V
\textbf{Durata}: Massimo 1 minuto
Manifesti a gittata una trucco minore, un segno di
potere soprannaturale. Crei a gittata uno dei seguenti
effetti magici:
- La tua voce risuona tre volte più forte del normale
per 1 minuto.
- Fai sì che le fiamme tremolino, si intensifichino,
affievoliscano o cambino colore per 1 minuto.
- Provochi innocui tremori sul terreno per 1 minuto.
- Crei un rumore istantaneo, come un rombo di tuono,
il verso di un corvo, o un sussurro inquietante, che
origina da un punto a gittata scelto da te.
- Fai sì che una porta o una finestra non chiusa a
chiave si spalanchi o si chiuda di colpo.
- Modifichi l’aspetto dei tuoi occhi per 1 minuto.
Se lanci questo incantesimo più volte, puoi tenere attivi
fino a tre effetti da un minuto alla volta, e puoi
interrompere questi effetti con un’azione.
Telecinesi
[telekinesis\textbf{Difficolta'}:
5° livello, trasmutazione
\textbf{Tempo di Lancio}: 2 Azioni
\textbf{Gittata}: 18 metri
\textbf{Componenti}: V, S
\textbf{Durata}: Concentrazione, massimo 10 minuti
Ottieni la capacità di muovere o manipolare creature o
oggetti tramite il pensiero. Quando lanci questo
incantesimo, e come tua azione durante ciascun round,
puoi esercitare la tua volontà su di una creatura od
oggetto a gittata e che puoi vedere, provocando l’effetto
appropriato tra quelli seguenti. Puoi agire round dopo
round sempre sullo stesso bersaglio, o sceglierne uno
nuovo ogni volta. Se cambi bersaglio, il bersaglio
precedente non è più soggetto all’incantesimo.
Creatura. Puoi tentare di muovere una creatura di
taglia Enorme o più piccola. Effettua una prova di
caratteristica usando la tua caratteristica da incantatore
contesa da una prova di Forza della creatura. Se vinci
la contesa, muovi la creatura di 9 metri in qualsiasi
direzione, compreso verso l’alto, ma senza eccedere la
gittata dell’incantesimo. Fino al termine del tuo
prossimo turno, la creatura è intralciata dalla tua presa
telecinetica. Una creatura sollevata in alta, resta
sospesa a mezz’aria.
Nei round successivi, puoi usare la tua azione per
tentare di mantenere la tua presa telecinetica sulla
creatura ripetendo la contesa.
Oggetto. Puoi tentare di muovere un oggetto che pesa
fino a 500 chili. Se l’oggetto non è indossato o
trasportato, lo sposti automaticamente di 9 metri in
qualsiasi direzione, ma senza superare la gittata
dell’incantesimo.
Se l’oggetto è indossato o trasportato da una creatura,
devi effettuare una prova di caratteristica con la tua
caratteristica da incantatore contesa dalla prova di
Forza della creatura. Se vinci la contesa, trascini via
l’oggetto da quella creatura e lo muovi di 9 metri in una
qualsiasi direzione, senza però superare la gittata
dell’incantesimo.
Puoi esercitare un controllo preciso sugli oggetti tramite
la tua presa telecinetica, riuscendo così a manipolare
un attrezzo semplice, aprire una porta o un contenitore,
inserire o recuperare un oggetto da un contenitore
aperto, o versare del materiale in una fiala.
Teletrasporto
[teleport\textbf{Difficolta'}:
7° livello, evocazione
\textbf{Tempo di Lancio}: 2 Azioni
\textbf{Gittata}: 3 metri
\textbf{Componenti}: V
\textbf{Durata}: Istantanea
Questo incantesimo teletrasporta istantaneamente te e
altre otto creature consenzienti (oppure un singolo
oggetto) a gittata e che puoi vedere, scelte da te, in una
destinazione di tua scelta. Se il bersaglio è un oggetto,
deve poter entrare in un cubo di 3 metri di spigolo, e
non può essere tenuto o trasportato da una creatura
non consenziente.
La destinazione che scegli ti deve essere nota, e deve
essere sullo stesso piano di esistenza in cui ti trovi. La
tua familiarità con la destinazione determina se vi riesce
ad arrivare. Il Narratore tira il d100 e consulta la tabella.
Familiarità Errore Area
Simile
Fuori
Bersaglio
Sul
Bersaglio
Cerchio
permanente
- - - 01-100
Oggetto
associato
- - - 01-100
Molto
familiare
01-05 06-13 14-24 25-100
Visto
casualmente
01-33 34-43 44-53 54-100
Visto una
volta
01-43 44-53 54-73 74-100
Descrizione 01-43 44-53 54-73 74-100
Falsa
destinazione
01-50 51-100 - -
Familiarità. “Cerchio permanente” indica un cerchio di
teletrasporto permanente di cui conosci la sequenza dei
sigilli. “Oggetto associato” indica che possiedi un
oggetto preso negli ultimi sei mesi dalla destinazione
desiderata, come il libro della biblioteca di un mago,
biancheria della suite reale, o un pezzo di marmo della
tomba segreta di un lich.
“Molto familiare” è un luogo in cui sei stato molto
spesso, un posto che hai studiato attentamente, o un
posto che puoi vedere quando lanci l’incantesimo.
“Visto casualmente” è un posto che hai visto più di una
volta ma con cui non sei molto familiare. “Visto una
volta” è un posto che hai visto una volta sola, magari
tramite la magia. “Descrizione” è un luogo la cui
posizione e aspetto conosci solo tramite la descrizione
di qualcun altro, magari una mappa.
“Falsa destinazione” è un posto che non esiste. Magari
hai cercato di scrutare il nascondiglio di un nemico ma
hai invece visto un’illusione, oppure stai cercando di
teletrasportarti in un posto familiare che non esiste più.
Sul Bersaglio. Tu e il tuo gruppo (o l’oggetto bersaglio)
apparite dove desideri.
Fuori Bersaglio. Tu e il tuo gruppo (o l’oggetto
bersaglio) apparite a una distanza casuale dalla
destinazione in una direzione casuale. La distanza fuori
bersaglio è 1d10 x 1d10 percento della distanza
viaggiata. Per esempio, se hai provato a viaggiare per
180 chilometri, atterri fuori bersaglio e tiri 5 e 3 su due
d10, allora saresti fuori bersaglio del 15%, ovvero 27
chilometri. Il Narratore determina la direzione fuori bersaglio
casualmente, tirando un d8 e indicando l’1 come nord, il
2 come nordest, il 3 come est e così via seguendo le
direzioni della bussola. Se ti stai teletrasportando in una
città costiera e finisci 27 chilometri al largo in mare,
potresti essere nei guai!
Area Simile. Tu e il tuo gruppo (o l’oggetto bersaglio)
finite in un’area diversa che è visualmente o
tematicamente simile all’area bersaglio. Per esempio,
se sei diretto al tuo laboratorio personale, potresti finire
nel laboratorio di un altro mago o in un negozio di
oggetti alchemici che possiede molti degli attrezzi e
strumenti del tuo laboratorio. In genere, compari nel
luogo simile più vicino, ma dato che l’incantesimo non
ha limiti di gittata, potresti finire praticamente dovunque
sullo stesso piano.
Errore. L’imprevedibile magia dell’incantesimo provoca
un viaggio difficile. Ogni creatura teletrasportata (o
l’oggetto bersaglio) subisce 3d10 danni da forza, e il
Narratore ritira sulla tabella per vedere dove finiscano
(possono capitare più errori, che infliggono danni ogni
volta).
Tempesta di Fuoco
[fire storm\textbf{Difficolta'}:
7° livello, invocazione
\textbf{Tempo di Lancio}: 2 Azioni
\textbf{Gittata}: 45 metri
\textbf{Componenti}: V, S
\textbf{Durata}: Istantanea
Una tempesta composta di fiamme ruggenti compare in
un punto a gittata, scelto da te. L’area della tempesta
consiste di un massimo di dieci cubi di 3 metri di
spigolo, che puoi disporre come preferisci. Ogni cubo
deve avere almeno una faccia adiacente a quella di un
altro cubo. Ogni creatura nell’area deve effettuare un
tiro salvezza su Agilità. Se lo fallisce subisce 7d10
danni da fuoco, o la metà di questi danni se lo supera.
Il fuoco danneggia gli oggetti nell’area e incendia gli
oggetti infiammabili che non sono indossati o
trasportati. Se lo desideri, la vita vegetale nell’area
resta illesa dagli effetti di questo incantesimo.
Tempesta di Ghiaccio
[ice storm\textbf{Difficolta'}:
4° livello, invocazione
\textbf{Tempo di Lancio}: 2 Azioni
\textbf{Gittata}: 90 metri
\textbf{Componenti}: V, S, M (un pizzico di polvere e alcune
gocce d’acqua)
\textbf{Durata}: Istantanea
Una grandinata di ghiaccio si abbatte a terra in un
cilindro di 6 metri di raggio e 12 metri di altezza centrato
su di un punto a gittata. Ogni creatura nel cilindro deve
effettuare un tiro salvezza su Agilità. La creatura
subisce 2d8 danni contundenti e 4d6 danni da freddo
se fallisce il tiro salvezza, o la metà se lo supera.
La grandine trasforma l’area di effetto della tempesta in
terreno difficile fino al termine del tuo prossimo turno.
Ai Livelli Più Alti. Quando lanci questo incantesimo
usando uno slot incantesimo di 5° livello o più alto, il
danno contundente aumenta di 1d8 per ogni livello dello
slot sopra il 4°.
Tempesta di Nevischio
[sleet storm\textbf{Difficolta'}:
3° livello, evocazione
\textbf{Tempo di Lancio}: 2 Azioni
\textbf{Gittata}: 45 metri
\textbf{Componenti}: V, S, M (un pizzico di polvere e qualche
goccia d’acqua)
\textbf{Durata}: Concentrazione, massimo 1 minuto
Fino al termine dell’incantesimo, pioggia gelida e
nevischio si abbattono in un cilindro alto 6 metri e del
raggio di 12 metri centrato in un punto da te scelto a
gittata. L’area è oscurata pesantemente, mentre le
fiamme esposte vengono spente.
Il terreno nell’area è coperto di ghiaccio scivoloso,
rendendolo terreno difficile. Quando una creatura entra
nell’area dell’incantesimo per la prima volta durante un
turno o inizia il suo turno lì, deve effettuare un tiro
salvezza su Agilita'. Se lo fallisce, cade prona.
Se una creatura nell’area dell’incantesimo si sta
concentrando, deve superare un tiro salvezza su
Costituzione contro la DC del tiro salvezza
dell’incantesimo o perdere la concentrazione.
Tempesta di Vendetta
[storm of vengeance\textbf{Difficolta'}:
9° livello, evocazione
\textbf{Tempo di Lancio}: 2 Azioni
\textbf{Gittata}: Vista
\textbf{Componenti}: V, S
\textbf{Durata}: Concentrazione, massimo 1 minuto
Si forma una ribollente nube di tempesta, centrata in un
punto che puoi vedere e che si propaga in un raggio di
110 metri. L’area è illuminata da fulmini, vi riecheggiano
tuoni e venti forti la spazzano. Quando la nube
compare, ogni creatura sotto di essa (ovvero non più di
1.500 metri sotto la nube) deve effettuare un tiro
salvezza su Costituzione. Se fallisce il tiro salvezza, la
creatura subisce 2d6 danni da tuono e resta assordata
per 5 minuti.
Ogni round in cui mantieni la concentrazione su questo
incantesimo, la tempesta, durante il tuo turno, produce
ulteriori effetti.
Round 2. Pioggia acida cade dalla nube. Ogni creatura
e oggetto sotto la nube subiscono 1d6 danni da acido.
Round 3. Richiami sei fulmini dalla nube per colpire sei
creature o oggetti di tua scelta, che si trovino sotto la
nube. Una specifica creatura od oggetto non può
essere colpita da più di un fulmine. Una creatura colpita
deve effettuare un tiro salvezza su Agilità. La
creatura subisce 10d6 danni da fulmine se fallisce il tiro
salvezza, o la metà di questi danni se lo supera.
Round 4. La nube produce una fitta grandinata. Ogni
creatura sotto la nube subisce 2d6 danni contundenti.
Round 5-10. Folate di vento e pioggia gelida si
abbattono sull’area sotto la nube. L’area diventa terreno
difficile ed è oscurata pesantemente. Ogni creatura
nell’area subisce 1d6 danni da freddo. Nell’area diventa
impossibile effettuare attacchi con armi a distanza. Il
vento e la pioggia sono considerati una distrazione
grave ai fini del mantenere la concentrazione sugli
incantesimi. Infine, folate di forte vento (che va dai 30 ai
75 chilometri all’ora) disperdono automaticamente
nebbia, foschia e simili fenomeni nell’area, che siano
naturali o magici.
Tentacoli Neri
[black tentacles\textbf{Difficolta'}:
4° livello, evocazione
\textbf{Tempo di Lancio}: 2 Azioni
\textbf{Gittata}: 27 metri
\textbf{Componenti}: V, S, M (un pezzo di tentacolo di una
piovra gigante o di un calamaro gigante)
\textbf{Durata}: Concentrazione, massimo 1 minuto
Viscidi tentacoli d’ebano riempiono un quadrato di 6
metri di lato sul terreno, a gittata e che puoi vedere. Per
la durata dell’incantesimo, questi tentacoli trasformano
l’area in terreno difficile.
Quando una creatura entra nell’area soggetta per la
prima volta in un turno o comincia qui il suo turno, la
creatura deve superare un tiro salvezza su Agilità o
subire 3d6 danni contundenti e rimanere intralciata dai
tentacoli fino al termine dell’incantesimo. Una creatura
che inizia il suo turno nell’area ed è già intralciata dai
tentacoli, subisce 3d6 danni contundenti.
Una creatura intralciata dai tentacoli può usare la sua
azione per effettuare una prova di Forza o Agilita' (a
sua scelta) contro la DC del tiro salvezza
dell’incantesimo. Se la supera, si libera.
Terremoto
[earthquake\textbf{Difficolta'}:
8° livello, invocazione
\textbf{Tempo di Lancio}: 2 Azioni
\textbf{Gittata}: 150 metri
\textbf{Componenti}: V, S, M (un pizzico di terriccio, un pezzo
di pietra e un grumo di argilla)
\textbf{Durata}: Concentrazione, massimo 1 minuto
Provochi un disturbo sismico in un punto sul terreno a
gittata e che puoi vedere. Per la durata, un intenso
tremore scuote il terreno in un cerchio di 30 metri di
raggio centrato su quel punto e scuote le creature e le
strutture in quell’area che sono a contatto del terreno.
Il terreno nell’area diventa terreno difficile. Ogni
creatura a terra che si sta concentrando deve effettuare
un tiro salvezza su Potenza. Se lo fallisce, la sua
concentrazione è infranta.
Quando lanci questo incantesimo e alla fine di ogni
turno che hai speso a concentrarti su di esso, ogni
creatura nell’area che si trovi a terra deve effettuare un
tiro salvezza su Agilità. Se lo fallisce, la creatura
cade prona.
Questo incantesimo ha effetti aggiuntivi a seconda del
tipo di terreno nell’area, a discrezione del Narratore.
Fenditure. All’inizio del turno successivo a quello in cui
hai lanciato l’incantesimo si aprono delle fenditure per
tutta l’area dell’incantesimo. Un totale di 1d6 fenditure si
aprono in punti scelti dal Narratore. Ognuna di esse è
profonda 1d10 x 3 metri, larga 3 metri e si estende da
un lato dell’area dell’incantesimo all’altro. Una creatura
che si trova sul punto in cui si apre una fenditura deve
superare un tiro salvezza su Agilità o cadervi
dentro. Una creatura che riesca il tiro salvezza si sposta
sul bordo della fenditura, nel momento in cui questa si
apre.
Una fenditura che si apre sotto una struttura la fa
crollare immediatamente (vedi sotto).
Strutture. Il tremore infligge 50 danni contundenti a
qualsiasi struttura in contatto col terreno nell’area
quando lanci l’incantesimo e alla fine di ciascuno dei
tuoi turni fino al termine dell’incantesimo. Se una
struttura scende a 0 punti ferita, crolla e potrebbe
danneggia le creature vicine. Una creatura distante
dalla struttura metà della altezza o meno della struttura,
deve effettuare un tiro salvezza su Agilità. Se lo
fallisce, la creatura subisce 5d6 danni contundenti,
cade prona ed è sommersa dalle macerie. Dovrà poi
impiegare un’azione riuscendo una prova di Forza
(Atletica) DC 20 per liberarsi. Il Narratore può modificare
verso l’alto o il basso la DC, a seconda della natura
delle macerie. Se supera il tiro salvezza, la creatura
subisce solo la metà dei danni e non cade né resta
sepolta.
Terreno Illusorio
[hallucinatory terrain\textbf{Difficolta'}:
4° livello, illusione
\textbf{Tempo di Lancio}: 10 minuti
\textbf{Gittata}: 90 metri
\textbf{Componenti}: V, S, M (una pietra, un rametto e un
pezzo di pianta verde)
\textbf{Durata}: 24 ore
Fai sì che un pezzo di terreno naturale a gittata, in un
cubo di 45 metri di spigolo, appaia, risuoni e odori come
qualche altro tipo di terreno naturale. Di conseguenza,
campi aperti o una strada possono essere trasformati in
un acquitrino, colline, un crepaccio o qualche altro tipo
di terreno difficile o invalicabile. Un laghetto può essere
trasformato in una radura erbosa, un precipizio in una
lieve pendenza, un burrone cosparso di rocce in una
strada ampia e liscia. Le strutture edificate,
l’equipaggiamento e le creature all’interno dell’area non
mutano d’aspetto.
Le peculiarità tattili del terreno sono immutate, così che
le creature che entrano nell’area è probabile che
svelino l’illusione. Se al contatto la differenza non è
ovvia, una creatura che esamina con cautela l’illusione
può tentare una prova di Intelletto (Indagare) contro
la DC del tiro salvezza dei tuoi incantesimi per dubitare
di essa. Una creatura che riconosca l’illusione per
quello che è, la percepisce come una vaga immagine
sovrapposta al terreno.
Tocco Gelido
[chill touch\textbf{Difficolta'}:
Trucchetto, negromanzia
\textbf{Tempo di Lancio}: 2 Azioni
\textbf{Gittata}: 36 metri
\textbf{Componenti}: V, S
\textbf{Durata}: 1 round
Crei una scheletrica mano spettrale nello spazio di una
creatura a gittata. Effettua un attacco a distanza con
incantesimo contro la creatura, per aggredirla con il
gelo della morte. Se colpisci, il bersaglio subisce 1d8
danni necrotici, e non può recuperare punti ferita fino
all’inizio del tuo prossimo turno. Fino ad allora, la mano
resterà serrata sul bersaglio.
Se colpisci un bersaglio non morto, esso avrà anche
svantaggio ai tiri per colpire contro di te fino alla fine del
suo prossimo turno.
Il danno dell’incantesimo aumenta di 1d8 quando
raggiungi il 5° livello (2d8), l’11° livello (3d8) e il 17°
livello (4d8).
Tocco Vampirico
[vampiric touch\textbf{Difficolta'}:
3° livello, negromanzia
\textbf{Tempo di Lancio}: 2 Azioni
\textbf{Gittata}: Personale
\textbf{Componenti}: V, S
\textbf{Durata}: Concentrazione, massimo 1 minuto
Il contatto con la tua mano avvolta dall’ombra può
risucchiare la forza vitale altrui per curare le tue ferite.
Effettua un attacco in mischia con incantesimo contro
una creatura a portata. Se colpisci, il bersaglio subisce
3d6 danni necrotici, e tu recuperi un numero di punti
ferita pari alla metà del danno necrotico che hai inflitto.
Fino al termine dell’incantesimo, puoi effettuare ogni
turno di nuovo questo attacco come tua azione.
Ai Livelli Più Alti. Quando lanci questo incantesimo
usando uno slot incantesimo di 4° livello o più alto, il
danno aumenta di 1d6 per ogni livello dello slot sopra il
3°.
Trama Ipnotica
[hypnotic pattern\textbf{Difficolta'}:
3° livello, illusione
\textbf{Tempo di Lancio}: 2 Azioni
\textbf{Gittata}: 36 metri
\textbf{Componenti}: S, M (un bastoncino luminoso di incenso
o una fiala di cristallo piena di materiale fosforescente)
\textbf{Durata}: Concentrazione, massimo 1 minuto
Crei a gittata una trama contorta di colori che si muove
nell’aria all’interno di un cubo di 9 metri di spigolo. La
trama appare per un momento e poi svanisce. Ogni
creatura nell’area che veda la trama deve effettuare un
tiro salvezza su Arbitrio. Se fallisce il tiro salvezza,
una creatura rimane affascinata per la durata. Mentre è
affascinata da questo incantesimo, la creatura è inabile
e ha velocità 0.
L’incantesimo termina per la creatura soggetta, qualora
questa subisca danni o se qualcuno usa un’azione per
scuoterla dal suo stato confusionale.
Trasformazione
[shapechange\textbf{Difficolta'}:
9° livello, trasmutazione
\textbf{Tempo di Lancio}: 2 Azioni
\textbf{Gittata}: Personale
\textbf{Componenti}: V, S, M (un cerchietto di giada del valore
di almeno 1.500 mo, che devi poggiare sulla tua testa
prima di lanciare l’incantesimo)
\textbf{Durata}: Concentrazione, massimo 1 ora
Per la durata assumi la forma di una creatura differente.
La nuova forma può essere quella di qualsiasi creatura
il cui grado di sfida sia pari o inferiore al tuo livello. La 
creatura non può essere un costrutto o un non morto, e
devi averla vista almeno una volta. Ti trasformi in un
esemplare medio di quella creatura, uno senza livelli di
classe o il tratto Incantesimi.
Puoi restare nella forma assunta fino al termine
dell’incantesimo. Ti ritrasformi automaticamente se cadi
privo di sensi, scendi a 0 punti ferita o muori.
Le tue statistiche di gioco sono rimpiazzate dalle
statistiche della creatura scelta, fatta accezione del tuo
allineamento, e dei tuoi punteggi di Intelletto,
Saggezza e Carisma. Mantieni tutte le tue competenze
nelle abilità e i tiri salvezza, oltre a ottenere quelle della
creatura. Se la creatura possiede le tue stesse
competenze e il bonus indicato nelle sue statistiche è
più alto del tuo, usa il bonus della creatura al posto del
tuo. Non puoi usare nessuna azione leggendaria o
azione da tana della nuova forma.
Quando ti trasformi, assumi i punti ferita e i Dadi Vita
della creatura. Quando ritorni alla tua forma normale,
ritorni al numero di punti ferita che avevi prima di
trasformarti. Tuttavia, se ti ritrasformi perché sei stato
ridotto a 0 punti ferita, tutto il danno in eccesso viene
riportato alla tua forma originale. A meno che il danno in
eccesso non riduca la tua forma normale a 0 punti
ferita, non cadrai privo di sensi.
Mantieni tutti i benefici di qualsiasi privilegio di classe,
razza, o altra fonte e puoi usarli se la nuova forma è
fisicamente capace di farne uso. Tuttavia, non puoi
usare nessuno dei tuoi sensi speciali, come la
scurovisione, a meno che la nuova forma non possieda
anch’essa lo stesso senso. Puoi parlare solo se la
creatura è normalmente in grado di parlare.
Quando ti trasformi scegli se il tuo equipaggiamento
cade a terra nel tuo spazio, si fonde con la nuova forma
o sia indossato da essa. L’equipaggiamento indossato
funziona come di norma, ma sta al Narratore decidere se sia
comodo per la nuova forma indossare un simile pezzo
di equipaggiamento, in base alla taglia e le dimensioni
della creatura. Il tuo equipaggiamento non cambia
dimensioni né si adatta alla nuova forma, e qualsiasi
equipaggiamento che la nuova forma non può
indossare deve essere fatto cadere a terra o fondersi
con la nuova forma. L’equipaggiamento che si fonde è
inefficace.
Nella durata dell’incantesimo, puoi usare la tua azione
per assumere una forma diversa seguendo le stesse
restrizioni e regole della forma originale, con una
eccezione: se la tua nuova forma ha più punti ferita
della forma attuale, i tuoi punti ferita restano al livello
attuale.
Traslazione Arborea
[tree stride\textbf{Difficolta'}:
5° livello, evocazione
\textbf{Tempo di Lancio}: 2 Azioni
\textbf{Gittata}: Personale
\textbf{Componenti}: V, S
\textbf{Durata}: Concentrazione, massimo 1 minuto
Ottieni la capacità di entrare in un albero e muoverti dal
suo interno all’interno di un altro albero della stessa
specie entro 150 metri. Entrambi gli alberi devono
essere vivi e almeno della tua stessa taglia. Devi usare
1 metro di movimento per entrare nell’albero. Apprendi
istantaneamente la posizione di tutti gli altri alberi della
stessa specie entro 150 metri e, come parte del
movimento impiegato per entrare nell’albero, puoi
passare in uno degli altri alberi o uscire dall’albero in cui
sei entrato. Riappari in un punto a tua scelta entro 1,5
metri dall’albero di destinazione, utilizzando altri 1,5
metri di movimento. Se non ti rimane movimento da
usare, riappari entro 1 metro dall’albero in cui sei
entrato.
Per la durata dell’incantesimo puoi usare questa
capacità di trasporto una volta per round. Devi
terminare ogni turno al di fuori di un albero.
Trasporto Vegetale
[transport via plants\textbf{Difficolta'}:
6° livello, evocazione
\textbf{Tempo di Lancio}: 2 Azioni
\textbf{Gittata}: 3 metri
\textbf{Componenti}: V, S
\textbf{Durata}: 1 round
Questo incantesimo crea un legame magico tra un
vegetale inanimato di taglia Grande o maggiore a
gittata e un altro vegetale, a qualsiasi distanza, sullo
stesso piano di esistenza. Devi aver visto o essere
entrato in contatto almeno una volta con il vegetale di
destinazione. Per la durata dell’incantesimo, qualsiasi
creatura può entrare nel vegetale bersaglio e uscire dal
vegetale di destinazione usando 1 metro di
movimento.
Trova Cavalcatura
[find steed\textbf{Difficolta'}:
2° livello, evocazione
\textbf{Tempo di Lancio}: 10 minuti
\textbf{Gittata}: 9 metri
\textbf{Componenti}: V, S
\textbf{Durata}: Istantanea
Evochi uno spirito che assume la forma di una
cavalcatura insolitamente intelligente, forte e leale,
stabilendo un legame duraturo con esso. Apparendo in
uno spazio a gittata, non occupato, il destriero assume
la forma di tua scelta, come quella di un cavallo da
guerra, un pony, un cammello, un alce o un mastino (il
Narratore potrebbe darti la possibilità di evocare come
destrieri anche altri tipi di animali). Il destriero ha le
statistiche della forma scelta, sebbene sia di tipo
celestiale, fatato o immondo (a tua scelta) invece del
suo normale tipo. Inoltre, se il tuo destriero ha
Intelletto 5 o meno, la sua Intelletto diventa 6, e
ottiene la capacità di comprendere un linguaggio a tua
scelta tra quelli che sei in grado di parlare.
Il tuo destriero serve da cavalcatura, sia in
combattimento che fuori da esso, e possiedi un legame
istintivo con esso, che vi permette di combattere come
foste un unico insieme. Mentre sei in groppa alla tua
cavalcatura, puoi far sì che qualsiasi incantesimo che
lanci e che prenda come bersaglio solo te, prenda
come bersaglio anche il tuo destriero.
Quando il destriero scende a 0 punti ferita, scompare,
non lasciandosi dietro alcuna forma fisica. Puoi
congedare il destriero in qualsiasi momento con
un’azione, facendolo sparire. In entrambi i casi, lanciare
di nuovo questo incantesimo evoca lo stesso destriero,
ripristinato al massimo dei suoi punti ferita.
Mentre il tuo destriero si trova entro 1,5 chilometri da te,
puoi comunicare telepaticamente con esso.
Non puoi avere più di un destriero legato da questo
incantesimo alla volta. Con un’azione, puoi liberare il
destriero da questo legame in qualsiasi momento,
facendolo sparire.
Trova Famiglio
[find familiar\textbf{Difficolta'}:
1° livello, evocazione (rituale)
\textbf{Tempo di Lancio}: 1 ora
\textbf{Gittata}: 3 metri
\textbf{Componenti}: V, S, M (10 mo di carbone, incenso e
erbe che devono essere consumate dal fuoco in un
braciere d’ottone)
\textbf{Durata}: Istantanea
Ottieni il servizio di un famiglio, uno spirito che assume
una forma animale di tua scelta: cavalluccio marino,
corvo, donnola, falco, gatto, granchio, gufo, lucertola,
pesce (frizzo), piovra, pipistrello, ragno, rana (rospo),
ratto o serpente velenoso. Apparendo in uno spazio a
gittata, non occupato, il famiglio ha le statistiche della
forma scelta, sebbene sia di tipo celestiale, fatato o
immondo (a tua scelta) invece di una bestia.
Il tuo famiglio agisce in maniera indipendente da te, ma
ubbidisce sempre ai tuoi comandi. In combattimento,
tira la propria iniziativa e agisce durante il proprio turno.
Un famiglio non può attaccare, ma può svolgere le altre
azioni come di norma.
Quando il famiglio scende a 0 punti ferita, scompare,
non lasciandosi dietro alcuna forma fisica. Riappare
quando lanci di nuovo questo incantesimo.
Mentre il tuo famiglio si trova entro 30 metri da te, puoi
comunicare telepaticamente con esso. Inoltre, con
un’azione puoi vedere attraverso gli occhi del famiglio e
ascoltare quello che sente fino all’inizio del tuo
prossimo turno, ottenendo i benefici di qualsiasi senso
speciale di cui il famiglio sia in possesso. Durante
questo periodo, sei cieco e sordo per quel che
concerne i tuoi sensi.
Con un’azione, puoi congedare temporaneamente il
famiglio. Questo scompare in una dimensione tascabile
dove aspetta di essere richiamato. In alternativa, lo puoi
congedare per sempre. Con un’azione, mentre è
temporaneamente congedato, lo puoi far riapparire in
uno spazio non occupato entro 9 metri da te.
Non puoi avere più di un famiglio alla volta. Se esegui
questo incantesimo mentre hai già un famiglio, fai
adottare a quello che hai già una nuova forma. Scegli
una delle forme dalla lista precedente. Il tuo famiglio si
trasforma nella creatura prescelta.
Infine, quando lanci un incantesimo con gittata contatto,
il tuo famiglio può trasmettere l’incantesimo come se
l’avesse lanciato lui. Il famiglio deve trovarsi entro 30
metri da te, e deve usare la propria reazione per
trasmettere l’incantesimo nel momento in cui lo lanci.
Se l’incantesimo richiede un tiro per colpire, per questo
tiro usi il tuo modificatore di attacco.
Trucco della Corda
[rope trick\textbf{Difficolta'}:
2° livello, trasmutazione
\textbf{Tempo di Lancio}: 1 minuto
\textbf{Gittata}: Contatto
\textbf{Componenti}: V, S, M (estratto di grano in polvere e un
laccio di pergamena)
\textbf{Durata}: 1 ora
Entri a contatto con un pezzo di corda lungo fino a 18
metri. Un’estremità della corda si leva nell’aria finché la
corda non pende perpendicolare al terreno.
All’estremità opposta della corda, un’entrata invisibile si
apre su di uno spazio extradimensionale che resta fino
al termine dell’incantesimo.
Lo spazio extradimensionale può essere raggiunto
arrampicandosi fino alla cima della corda. Lo spazio
può contenere fino a otto creature di taglia Media o
inferiore. La corda può essere trascinata nello spazio,
facendola sparire dalla vista di chi è fuori di esso.
Attacchi e incantesimi non possono attraversare
l’ingresso in entrata o uscita dallo spazio
extradimensionale, ma chi si trova al suo interno può
vedere fuori come se vedesse attraverso una finestra di
1 x 1 metro centrata sulla corda.
Qualsiasi cosa si trovi nello spazio extradimensionale
ne cade fuori al termine dell’incantesimo.
Unto
[grease\textbf{Difficolta'}:
1° livello, evocazione
\textbf{Tempo di Lancio}: 2 Azioni
\textbf{Gittata}: 18 metri
\textbf{Componenti}: V, S, M (un pezzo di cotenna di maiale o
burro)
\textbf{Durata}: 1 minuto
Grasso scivoloso ricopre il terreno in un quadrato di 3
metri di lato, centrato su di un punto a gittata, e lo
trasforma in terreno difficile per la durata
dell’incantesimo.
Quando compare il grasso, ciascun bersaglio che si
trova in piedi nell’area deve superare un tiro salvezza
su Agilita' o cadere prono. Una creatura che entra
nell’area o termina il suo turno lì, deve superare un tiro
salvezza su Agilita' o cadere prona.
Vedere Invisibilità
[see invisibility\textbf{Difficolta'}:
2° livello, divinazione
\textbf{Tempo di Lancio}: 2 Azioni
\textbf{Gittata}: Personale
\textbf{Componenti}: V, S, M (un pizzico di talco e una
manciata di polvere d’argento)
\textbf{Durata}: 1 ora
Per la durata dell’incantesimo, vedi le creature e gli
oggetti invisibili come se fossero visibili, e inoltre puoi
vedere nel Piano Etereo. Le creature e gli oggetti eterei
ti appaiono spettrali e trasparenti.
Velocità
[haste\textbf{Difficolta'}:
3° livello, trasmutazione
\textbf{Tempo di Lancio}: 2 Azioni
\textbf{Gittata}: 9 metri
\textbf{Componenti}: V, S, M (una grattata di radice di
liquirizia)
\textbf{Durata}: Concentrazione, massimo 1 minuto
Scegli una creatura consenziente a gittata e che puoi
vedere. Fino al termine dell’incantesimo, la velocità del
bersaglio è raddoppiata, ottiene un bonus di +2 alla CA,
ha vantaggio ai tiri salvezza su Agilità, e ottiene
un’azione aggiuntiva durante ciascun suo turno.
Quest’azione può essere impiegata solo per effettuare
le azioni Attaccare (un solo attacco con un’arma),
Disimpegnarsi, Nascondersi, Scattare o Usare un
Oggetto.
Quando l’incantesimo termina, il bersaglio non può
muoversi o effettuare azioni fino al suo prossimo turno,
mentre è pervaso da un’improvvisa sonnolenza.
Vigilanza e Interdizione
[guards and wards\textbf{Difficolta'}:
6° livello, abiurazione
\textbf{Tempo di Lancio}: 10 minuti
\textbf{Gittata}: Contatto
\textbf{Componenti}: V, S, M (incenso bruciato, un piccolo
misurino di zolfo e olio, un laccio legato, un piccolo
ammontare di sangue di colosso di terra, e una piccola
verga d’argento del valore di almeno 10 mo)
\textbf{Durata}: 24 ore
Crei una interdizione che protegge fino a 225 metri
quadri di pavimento (un’area quadrata di 15 metri di
lato, o cento quadrati di 1 metro di lato o venticinque
quadrati di 3 metri di lato). L’area interdetta può essere
alta fino a 6 metri, e modellata come preferisci. Puoi
interdire diversi piani di una roccaforte dividendo l’area
tra di essi, purché tu possa camminare
ininterrottamente in ogni area adiacente, mentre lanci
l’incantesimo.
Quando lanci questo incantesimo, puoi specificare gli
individui che ignorano qualcuno o tutti gli effetti di
questo incantesimo. Puoi anche specificare una parola
d’ordine che, pronunciata ad alta voce, rende chi la
proferisce immune a questi effetti.
Vigilanza e interdizione crea i seguenti effetti all’interno
dell’area interdetta.
Corridoi. La nebbia riempie tutti i corridoi interdetti,
rendendoli oscurati pesantemente. Inoltre, a ogni
intersezione o biforcazione del passaggio che offre una
scelta di direzione, c’è una probabilità del 50% che una
creatura, escluso te, creda di stare andando nella
direzione opposta a quella che ha scelto.
Porte. Tutte le porte nell’area interdetta sono chiuse
magicamente, come se fossero sigillate
dall’incantesimo serratura arcana. Inoltre, puoi coprire
fino a dieci porte con un’illusione (equivalente della
funzione oggetto illusorio dell’incantesimo illusione
minore) per farle sembrare delle semplici sezioni di
muro.
Scale. Ragnatele ricoprono da cima a fondo tutte le
scale nell’area interdetta, come per l’incantesimo
ragnatela. Questi fili ricrescono in 10 minuti se vengono
bruciati o strappati mentre vigilanza e interdizione resta
attivo.
Altri Incantesimi in Effetto. Puoi piazzare uno dei
seguenti effetti magici di tua scelta all’interno dell’area
interdetta dell’edificio.
- Piazza luci danzanti in quattro corridoi. Puoi indicare
un semplice programma che le luci ripeteranno per
la durata di vigilanza e interdizione.
- Piazza bocca magica in due posti.
- Piazza nube maleodorante in due posti. I vapori
appaiono nel posto da te indicato; ritornano entro 10
minuti se dispersi dal vento mentre vigilanza e
interdizione è ancora attivo.
- Piazza una folata di vento costante in un corridoio o
stanza.
- Piazza una suggestione in un luogo. Seleziona
un’area quadrata di 1 metro di lato, e qualsiasi
creatura che entra o passa attraverso quell’area
riceve mentalmente la suggestione.
L’intera area interdetta irradia magia. Un incantesimo
dissolvi magie lanciato contro uno specifico effetto, se
riesce, rimuove solo quell’effetto.
Puoi creare una struttura perennemente vigilata e
interdetta lanciandovi questo incantesimo ogni giorno
per un anno.
Vincolo di Interdizione
[warding bond\textbf{Difficolta'}:
2° livello, abiurazione
\textbf{Tempo di Lancio}: 2 Azioni
\textbf{Gittata}: Contatto
\textbf{Componenti}: V, S, M (una coppia di anelli di platino del
valore di 50 mo l’uno, che tu e il bersaglio dovete
indossare per la durata)
\textbf{Durata}: 1 ora
Lanci l’incantesimo a contatto di una creatura che vuoi
proteggere. Crei una connessione mistica tra di te e il
bersaglio fino al termine dell’incantesimo. Finché il
bersaglio è entro 18 metri da te, ottiene un bonus di +1
alla CA e ai tiri salvezza e ha resistenza a tutti i danni.
Inoltre, ogni volta che il bersaglio subisce danni, tu ne
subisci la stessa quantità.
L’incantesimo ha fine se scendi a 0 punti ferita o tu e il
bersaglio vi allontanate più di 18 metri. Ha fine anche
se lo lanci di nuovo sulla stessa creatura su cui è già in
atto. Puoi interrompere l’incantesimo con un’azione.
Visione del Vero
[true sight\textbf{Difficolta'}:
6° livello, divinazione
\textbf{Tempo di Lancio}: 2 Azioni
\textbf{Gittata}: Contatto
\textbf{Componenti}: V, S, M (un unguento per gli occhi che
costa 25 mo; fatto di funghi in polvere, zafferano e
grasso; viene consumato dall’incantesimo)
\textbf{Durata}: 1 ora
Lanci l’incantesimo a contatto di una creatura
consenziente. Il bersaglio riceve la capacità di vedere le
cose come sono realmente. Per la durata
dell’incantesimo, la creatura ha visione del vero, nota
porte segrete nascoste dalla magia, e può vedere nel
Piano Etereo, fino a una gittata di 36 metri.
Vita Falsata
[false life\textbf{Difficolta'}:
1° livello, negromanzia
\textbf{Tempo di Lancio}: 2 Azioni
\textbf{Gittata}: Personale
\textbf{Componenti}: V, S, M (un piccolo ammontare di alcool
o spirito distillato)
\textbf{Durata}: 1 ora
Potenziandoti con una parvenza necromantica di
vitalità, ottieni 1d4 + 4 punti ferita temporanei per la
durata.
Ai Livelli Più Alti. Quando lanci questo incantesimo
usando uno slot incantesimo di 2° livello o più alto,
ottieni 5 punti ferita temporanei aggiuntivi per ogni
livello dello slot sopra il 1°.
Volare
[fly\textbf{Difficolta'}:
3° livello, trasmutazione
\textbf{Tempo di Lancio}: 2 Azioni
\textbf{Gittata}: Contatto
\textbf{Componenti}: V, S, M (una piuma dell’ala di qualsiasi
volatile)
\textbf{Durata}: Concentrazione, massimo 10 minuti
Lanci l’incantesimo a contatto di una creatura
consenziente. Per la durata dell’incantesimo, il
bersaglio ottiene velocità di volo 18 metri. Quando
l’incantesimo ha fine, qualora sia ancora in aria, il
bersaglio cade, a meno che non riesca a frenare la
discesa.
Ai Livelli Più Alti. Quando lanci questo incantesimo
usando uno slot incantesimo di 4° livello o più alto, puoi
prendere come bersaglio un’ulteriore creatura per ogni
livello dello slot sopra il 3°.
Vuoto Mentale
[mind blank\textbf{Difficolta'}:
8° livello, abiurazione
\textbf{Tempo di Lancio}: 2 Azioni
\textbf{Gittata}: Contatto
\textbf{Componenti}: V, S
\textbf{Durata}: 24 ore
Fino al termine dell’incantesimo, una creatura
consenziente con cui sei in contatto durante il lancio è
immune al danno psichico, qualsiasi effetto che ne
percepirebbe le emozioni o leggerebbe i pensieri,
incantesimi di divinazione e la condizione affascinato.
L’incantesimo nega anche gli incantesimi desiderio e
altri incantesimi o effetti di simili potenza impiegati per
influenzare la mente del bersaglio o per ottenere
informazioni su di esso.
Zona di Verità
[zone of truth\textbf{Difficolta'}:
2° livello, ammaliamento
\textbf{Tempo di Lancio}: 2 Azioni
\textbf{Gittata}: 18 metri
\textbf{Componenti}: V, S
\textbf{Durata}: 10 minuti
Crei una zona magica che protegge contro i raggiri in
una sfera di 4,5 metri di raggio centrata su di un punto a
gittata di tua scelta. Fino al termine dell’incantesimo,
una creatura che entra nell’area dell’incantesimo per la
prima volta durante un turno, o inizia il suo turno al suo
interno, deve effettuare un tiro salvezza su Arbitrio. Se
fallisce il tiro salvezza, la creatura non può pronunciare
bugie deliberatamente mentre è nel raggio
dell’incantesimo. Sei a conoscenza se una creatura ha
superato o fallito il tiro salvezza.
Una creatura soggetta all’incantesimo ne è
consapevole e può quindi evitare di rispondere a
domande a cui risponderebbe normalmente con una
bugia. Questa creatura può dare risposte elusive
purché rimanga entro i confini della verità.

\end{multicols}